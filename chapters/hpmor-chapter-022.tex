\chapter{La méthode scientifique}

\lettrine{U}{ne} petite chambre d'étude, proche, mais pas dans le dortoir de Serdaigle, l'une des nombreuses, nombreuses pièces inutilisées de Poudlard. Sol de pierre grise, murs de brique rouge, plafond de bois sombre taché, quatre globes de verre luminescents incrustés dans les quatre murs de la pièce. Une table circulaire qui ressemblait à une large plaque de marbre noir placée sur d'épais pieds de marbre noir en colonne, mais qui s'était révélée être très légère (de poids comme de masse) et n'était pas difficile à prendre et à déplacer si nécessaire. Deux confortables chaises rembourrées qui avaient d'abord semblé être rivées au sol à des emplacements peu pratiques mais qui, comme ils avaient fini par s'en rendre compte, filaient vers l'endroit où vous vous teniez dès que vous vous penchiez en arrière dans une posture laissant penser que vous étiez sur le point de vous asseoir.

Il semblait aussi y avoir un certain nombre de chauve-souris volant autour de la pièce.

C'était là, noteraient les futurs historiens -- \emph{si} tout ce projet devait un jour produire quelque chose de valeur -- que l'étude scientifique de la magie avait commencé, avec deux jeunes élèves de Poudlard en première année.

Harry James Potter-Evans-Verres, théoricien.

Et Hermione Jean Granger, expérimentatrice et cobaye.

Harry se débrouillait maintenant mieux en classe, du moins dans les cours qu'il considérait être intéressants. Il avait lu plus de livres, et pas seulement des livres pour enfants de onze ans. Il avait pratiqué la métamorphose encore et encore tous les jours pendant l'une de ses heures supplémentaires, utilisant l'autre heure pour commencer l'Occlumancie. Il prenait les classes qui le méritaient \emph{au sérieux}, ne se contentant pas de rendre ses devoirs tous les jours, mais utilisant son temps libre pour apprendre plus qu'il n'était requis, pour lire d'autres livres en plus des manuels fournis, pas seulement avec l'espoir de mémoriser quelques réponses de contrôles, mais avec celui de maîtriser le sujet et d'exceller. On ne voyait pas beaucoup ça en dehors de Serdaigle. Et même \emph{à} Serdaigle, ses seuls véritables concurrents étaient Padma Patil (dont les parents venaient d'une culture non anglophone et qui avait par conséquent été élevée dans une véritable éthique du travail), Anthony Goldstein (venu d'un groupe ethnique particulier qui gagnait 25~\% des prix Nobel), et bien sûr, marchant à grandes enjambées au-dessus de tout le monde, telle un Titan qui flânerait à travers une meute de bébés chiens, Hermione Granger.

Pour réaliser cette expérience particulière, il fallait que le sujet apprenne seize nouveaux sorts, seul, sans aide ni correction. Ce qui voulait dire que le sujet était Hermione. Point.

Il faudrait maintenant mentionner que les chauves-souris volant autour de la pièce ne luisaient \emph{pas}.

Harry avait du mal à accepter les implications de ce fait.

<<~\emph{Oogely boogely~!}~>> dit à nouveau Hermione.

À nouveau, au bout de la baguette de Hermione, il y eut l'apparition abrupte et sans transition d'une chauve-souris. Un instant, air. L'instant suivant, chauve-souris. Ses ailes semblaient déjà bouger au moment où elle apparut.

Et elle ne \emph{luisait toujours pas}.

<<~Je peux m'arrêter maintenant~? dit Hermione.

--- Es-tu certaine~>>, dit Harry à travers ce qui semblait être un pavé coincé dans sa gorge, <<~que tu ne pourrais pas les faire luire, avec un peu plus de pratique~?~>> Il violait la procédure expérimentale qu'il avait écrite à l'avance, ce qui était un péché, et il la violait parce qu'il n'aimait pas les résultats qu'il obtenait, ce qui était un péché \emph{mortel}, vous pouviez aller en Enfer de la Science pour ça, mais de toute façon ça ne semblait pas avoir d'importance.

<<~Tu as changé quoi cette fois-ci~? dit Hermione d'un ton un peu las.

--- La durée des sons \emph{oo}, \emph{eh} et \emph{ee}. C'est censé être 3 pour 2 pour 2, pas 3 pour 1 pour 1.

--- \emph{Oogely boogely}~!~>> dit Hermione.

La chauve-souris se matérialisa avec une seule aile et tourbillonna de façon pathétique jusqu'au sol, puis s'effondra sur la pierre grise en décrivant un arc de cercle.

<<~Et en vrai c'est quoi~? dit Hermione.

--- 3 pour 2 pour 1.

--- \emph{Oogely boogely~!}~>>

Cette fois la chauve-souris n'avait pas d'ailes du tout et tomba comme une souris morte en faisant un “plop”.

<<~3 pour 1 pour 2.~>>

Et oh la chauve-souris se matérialisa bien, et elle vola immédiatement vers le plafond, en bonne santé et luisant d'un vert vif.

Hermione hocha la tête avec satisfaction. <<~D'accord, quoi ensuite~?~>>

Il y eut une longue pause.

<<~\emph{Vraiment}~? On doit \emph{vraiment} dire \emph{Oogely boogely}, avec les durées des sons \emph{oo}, \emph{eh} et \emph{ee} ayant un rapport de 3 pour 1 pour 2, ou la chauve-souris ne luira pas~? \emph{Pourquoi}~? \emph{Pourquoi} \emph{? Pour l'amour de tout ce qui est sacré, pourquoi~?}

--- Pourquoi pas~?

--- \emph{AAAAAAAAARRRRRRGHHHH!}~>>

\emph{Bam. Bam. Bam.}

Harry avait réfléchi à la nature de la magie pendant un moment, puis il avait conçu une série d'expériences basées sur la théorie que quasiment tout ce que les sorciers croyaient au sujet de la magie était faux.

On ne pouvait pas \emph{vraiment} avoir besoin de dire “Wingardium Leviosa” exactement comme il le fallait pour pouvoir faire léviter quelque chose, parce que, franchement, “Wingardium Leviosa”~? L'univers allait vérifier que vous aviez dit “Wingardium Leviosa” exactement comme il le fallait et sinon il ne ferait pas voler la plume~?

Non. Évidemment que non, une fois que vous y réfléchissiez sérieusement. Quelqu'un, très probablement un enfant en maternelle, mais en tout cas un utilisateur anglophone de la magie qui pensait que “Wingardium Leviosa” sonnait voletant et flottillant, avait initialement prononcé ces mots en jetant le sort pour la première fois. Puis il ou elle avait dit à tout le monde que c'était nécessaire.

Mais (s'était dit Harry) ça ne devait pas \emph{forcément} être fait comme ça, ce n'était pas propre à l'univers, c'était propre à \emph{nous}.

Il y avait une vieille histoire transmise de scientifique à scientifique, un conte moral, l'histoire de Blondlot et des rayons N.

Peu après la découverte des rayons X, un éminent physicien nommé Prosper-René Blondlot -- qui avait été le premier à mesurer la vitesse des ondes radio et à montrer qu'elles se propageaient à la vitesse de la lumière -- avait annoncé la découverte d'un phénomène nouveau et incroyable, les rayons N, qui provoqueraient le léger éclaircissement d'un écran. Il fallait se concentrer pour le voir, mais c'était là. Les rayons N avaient plein de propriétés intéressantes. Ils étaient courbés par l'aluminium et pouvaient être concentrés par un prisme en aluminium pour venir frapper un fil de sulfure de cadmium, qui luirait alors légèrement dans le noir…

Bientôt, des dizaines de scientifiques avaient confirmé les résultats de Blondlot, en particulier en France.

Mais il y avait encore d'autres scientifiques, en Angleterre et en Allemagne, qui disaient qu'ils n'étaient pas tout à fait certains d'avoir vu cette légère lueur.

Blondlot avait dit qu'ils avaient probablement mal mis en place la machinerie.

Un jour Blondlot avait donné une démonstration des rayons N. Les lumières avaient été éteintes, et son assistant avait annoncé les éclaircissements et les assombrissements tandis que Blondlot faisait ses manipulations.

Ça avait été une démonstration normale, tous les résultats étant ceux attendus.

Même si un scientifique américain nommé Robert Wood avait discrètement volé le prisme en aluminium du centre du mécanisme de Blondlot.

Et ça avait été la fin des rayons N.

\emph{La réalité}, avait un jour dit Philip K. Dick, \emph{c'est ce qui refuse de disparaître quand on cesse d'y croire.}

Rétrospectivement, le péché de Blondlot avait été évident. Il n'aurait pas dû dire à son assistant ce qu'il faisait. Avant de lui demander de décrire la clarté de l'écran, Blondlot aurait dû s'assurer que l'assistant ne savait \emph{ni} ce qu'il essayait ni quand il l'essayait. Ça aurait pu être aussi simple que ça.

De nos jours on appelait ça <<~en aveugle~>>, et c'était une des choses que les scientifiques modernes considéraient comme acquises. Si vous faisiez une expérience de psychologie pour voir si les gens se mettaient plus en colère quand on les frappait sur la tête avec des matraques rouges qu'avec des matraques vertes, vous n'aviez pas le droit de regarder les sujets vous-même et de décider à quel point ils étaient <<~en colère~>>. Vous prendriez des photos d'eux après qu'ils auront été frappés par la matraque, et vous enverriez les photos à un panel d'évaluateurs, qui noteraient sur une échelle de 1 à 10 à quel point chaque personne paraissait être en colère, évidemment \emph{sans} savoir avec quelle couleur de matraque ils s'étaient fait frapper. Il n'y avait en effet absolument aucune bonne raison de dire sur quoi portait l'expérience. Vous ne diriez \emph{certainement} pas aux sujets de l'expérience que vous \emph{pensiez} qu'ils devraient être plus en colère s'ils étaient frappés par des matraques rouges. Vous leur offririez juste 20 livres sterling, les attireriez dans une pièce, les frapperiez avec une matraque d'une couleur bien sûr aléatoirement choisie et vous prendriez une photo. En fait, le frappage-par-matraque et la prise de photo seraient faits par un assistant qui n'avait pas entendu parler de l'hypothèse, comme ça, il n'aurait pas l'air de s'attendre à quelque chose, il ne frapperait pas plus fort, et il ne prendrait pas la photo pile au bon moment.

Blondlot avait détruit sa réputation en faisant le genre d'erreur qui vous aurait valu une note exécrable et probablement un rire moqueur de la part de votre chargé de cours dans une classe de première année sur la conception de procédures expérimentales… en 1991.

Mais ça avait été il y a un peu plus longtemps, en 1904, et ça avait donc pris des mois avant que Robert Wood ne formule l'hypothèse alternative évidente et ne trouve un moyen de la tester, et des dizaines de scientifiques s'étaient fait avoir.

Plus de deux siècles après que la science eut débuté. Si tard dans l'histoire de la science, ça n'avait toujours pas été évident.

Ce qui rendait \emph{totalement} plausible le fait que, dans le petit monde magique, où la science semblait à peine connue, personne n'ait jamais essayé le premier test, le plus simple de tous, la chose la plus évidente qu'un scientifique moderne penserait à vérifier.

Les livres étaient pleins d'instructions compliquées sur les choses que vous deviez faire \emph{exactement comme il fallait} afin de jeter un sort. Et Harry avait émis l'hypothèse qu'obéir à ces instructions, que vérifier que vous les suiviez correctement, \emph{avait} probablement un effet. Ça vous forçait \emph{à vous concentrer sur le sort}. Vous dire de juste agiter votre baguette et de faire un vœu ne marcherait probablement \emph{pas} aussi bien. Et une fois que vous croyiez que le sort était censé fonctionner d'une certaine façon, une fois que vous l'aviez pratiqué de cette manière, vous pourriez ne pas être capable de vous convaincre que ça pourrait marcher d'une \emph{autre} manière…

… si vous essayiez la méthode simple mais erronée, et essayiez de tester des formes alternatives \emph{vous-même}.

Mais si vous ne \emph{saviez pas} ce à quoi avait ressemblé le sort original~?

Et si vous donniez à Hermione une liste de sorts qu'elle n'avait pas encore étudiés, trouvés dans un livre de sorts de farce stupides de la bibliothèque de Poudlard, et donniez à certains de ces sorts les instructions correctes et originales, tandis que d'autres auraient un geste modifié, un mot inversé~? Et si vous gardiez les mêmes instructions, mais lui disiez qu'un sort censé créer un ver rouge était censé créer un ver bleu~?

Eh bien dans ce cas, il s'était avéré que…

… là, Harry avait du mal à croire à ses résultats…

… si vous disiez à Hermione de dire <<~Oogely boogely~>> avec un rapport dans la durée des voyelles de 3 pour 1 pour 1, au lieu du rapport correct de 3 pour 1 pour 2, vous obteniez toujours la chauve-souris, mais elle ne luisait plus.

Non pas que les croyances soient ici \emph{sans importance}. Non pas que \emph{seuls} les mots et les mouvements de baguette importent.

Si vous donniez une information totalement incorrecte à Hermione sur ce qu'un sort était censé faire, il cesserait de fonctionner.

Si vous ne lui disiez pas du tout ce qu'un sort était censé faire, il cesserait de fonctionner.

Si elle savait vaguement ce qu'un sort était censé faire, ou si elle était partiellement dans l'erreur, alors le sort fonctionnerait comme tel que décrit dans le livre original, pas tel que Harry l'avait décrit à Hermione.

Pour l'instant, Harry se cognait littéralement la tête contre le mur de briques. Pas fort. Il ne voulait pas endommager son précieux cerveau. Mais s'il ne trouvait pas un exutoire quelconque pour sa frustration, il souffrirait de combustion spontanée.

\emph{Bam. Bam. Bam.}

II semblait que l'univers voulait \emph{vraiment} que vous disiez “Wingardium Leviosa”, et il voulait que vous le disiez d'une certaine façon, et il se fichait tout autant de \emph{votre} opinion sur la prononciation correcte du sort que de votre opinion sur la gravité.

\emph{POURQUOOOOOOOOOOOOI~?}

Le pire, c'était l'air amusé et hautain qu'affichait Hermione.

Hermione n'avait \emph{pas} apprécié l'idée de rester assise à obéir aux instructions de Harry sans savoir pourquoi.

Alors Harry lui avait expliqué ce qu'ils testaient.

Harry avait expliqué pourquoi ils le testaient.

Harry avait expliqué pourquoi aucun sorcier ne l'avait probablement essayé avant eux.

Harry avait expliqué qu'il était à vrai dire plutôt confiant dans sa prédiction.

Parce que, avait dit Harry, il était \emph{impossible} que l'univers veuille vraiment vous voir dire “Wingardium Leviosa”.

Hermione avait fait remarquer que ce n'était pas ce que les livres disaient. Hermione avait demandé si Harry pensait vraiment qu'il était plus intelligent, à onze ans, après juste un mois de scolarité à Poudlard, que tous les autres sorciers du monde qui n'étaient pas d'accord avec lui.

Harry avait dit exactement le mot suivant~:

<<~Évidemment.~>>

Maintenant, Harry regardait le mur de briques rouges directement face à lui et estimait avec quelle force il lui faudrait se frapper la tête pour se donner une commotion qui interférerait avec sa formation de souvenirs à long terme et l'empêcherait de se souvenir de ça plus tard. Hermione ne riait pas, mais il pouvait sentir son \emph{intention de rire} qui irradiait de derrière lui comme une épouvantable pression sur sa peau, un peu comme s'il avait su qu'il était suivi par un tueur en série mais en \emph{pire}.

<<~Dis-le, dit Harry.

--- Je \emph{n'allais} pas le dire, dit la voix sympathique de Hermione Granger. Ça ne semblait pas être gentil.

--- Finissons-en, dit Harry.

--- D'accord~! Alors tu m'as fait ce \emph{long cours magistral} en me disant à quel point c'était difficile de pratiquer la science la plus simple, et qu'on aurait peut-être besoin de rester sur ce problème pendant \emph{trente-cinq} ans, et tu t'es ensuite attendu à ce qu'on fasse la plus grande découverte de l'histoire de la magie pendant notre première heure de collaboration. Tu ne faisais pas qu'espérer, tu t'y attendais vraiment. Tu es bête.

--- Merci. Maintenant -

--- J'ai lu tous les livres que tu m'as donnés et je ne sais toujours pas comment appeler ça. Présomption~? Illusion de la planification~? Super méga effet Lac Wobegon~? Ils devront le nommer d'après toi. Le Biais Harry.

--- Ça \emph{va}~!

--- Mais c'\emph{est} mignon. C'est tellement un truc de garçon.

--- \emph{Va mourir}.

--- Ooh, tu dis des choses des plus romantiques.~>>

\emph{Bam. Bam. Bam.}

<<~Quoi maintenant~?~>> dit Hermione.

Harry reposa sa tête contre les briques. Son front commençait à lui faire mal là où il l'avait frappé. <<~Rien. Je dois retourner concevoir d'autres expériences.~>>

Pendant le mois précédent, Harry avait précautionneusement devisé, à l'avance, un parcours expérimental qui leur aurait duré jusqu'à décembre.

Ça aurait été un ensemble d'expériences \emph{géniales} si le \emph{tout premier test} n'avait pas falsifié la théorie initiale.

Harry n'arrivait pas à croire qu'il avait été aussi stupide.

<<~Laisse-moi me corriger, dit Harry. Je dois concevoir \emph{une} nouvelle expérience. Je te dirai quand je l'ai, et nous la ferons, et alors je concevrai la suivante. Qu'est-ce que tu en penses~?

--- J'en pense que \emph{quelqu'un} vient de faire \emph{beaucoup d'efforts} pour rien.~>>

\emph{Bam.} Oh. Il y avait été un peu plus fort que prévu.

<<~Donc~>>, dit Hermione. Elle était penchée dans sa chaise et l'air hautain était de retour sur son visage. <<~Qu'avons-nous appris aujourd'hui~?

--- J'ai découvert, dit Harry à travers ses dents serrées, que quand il s'agit de faire des recherches vraiment élémentaires sur un problème réellement déroutant où on n'a pas la moindre idée de ce qui se passe, mes livres sur la méthode scientifique sont de la merde -

--- Langage, M. Potter~! Certains d'entre nous sont d'innocentes jeunes filles~!

--- Très bien. Mais si mes livres n'étaient pas du \emph{merlan}, c'est un type de poisson tout à fait honorable, ils m'auraient donné l'important conseil suivant~: quand il y a un problème déroutant et que tu ne fais que commencer et que tu as une hypothèse falsifiable, va la tester. Trouve un moyen simple et facile de faire une vérification élémentaire et fais-la tout de suite. Ne t'embête pas à concevoir un parcours expérimental élaboré qui rendrait impressionnante une demande de subvention aux yeux d'une agence de financement. Vérifie juste aussi vite que possible si tes idées sont fausses avant de commencer à y investir d'immenses efforts. Que penses-tu de cette morale~?

--- Mmm… d'accord, dit Hermione. Mais j'espérais aussi quelque chose comme “les livres de Hermione ne sont pas sans valeur. Ils sont écrits par de sages vieux sorciers qui en savent beaucoup plus long sur la magie que moi. Je devrais prêter attention à ce que disent les livres de Hermione.” On peut aussi avoir cette morale~?~>>

La mâchoire de Harry semblait être bien trop crispée pour pouvoir laisser échapper le moindre mot, alors il hocha juste la tête.

<<~Génial~! dit Hermione. J'ai bien aimé cette expérience. On a beaucoup appris et ça ne m'a pris qu'une petite heure.

--- AAAAAAAAAAAAAAHHHHHHHHHHHHHHH~!~>>

\later

Dans les donjons de Serpentard.

Une salle inutilisée éclairée d'une inquiétante lumière verte, bien plus vive cette fois, venant d'un petit globe de cristal temporairement enchanté, mais une inquiétante lumière verte quand même, projetant d'étranges ombres à partir de bureaux poussiéreux.

Deux silhouettes aux dimensions enfantines porteuses de houppelandes grises (pas de masques) étaient entrées en silence et s'étaient assises sur deux chaises l'une en face de l'autre, séparées par un bureau.

C'était le deuxième rassemblement de la Conspiration Bayésienne.

Drago Malfoy n'avait pas su s'il devait l'attendre avec impatience.

Harry Potter, à en juger par l'expression de son visage, ne semblait pas avoir le moindre doute quant à l'humeur appropriée.

Harry Potter avait l'air d'être prêt à tuer quelqu'un.

<<~Hermione Granger~>>, dit Harry Potter alors que Drago ouvrait la bouche. <<~\emph{Ne pose pas de questions}.~>>

\emph{Il n'aurait pas été à un deuxième rendez-vous galant quand même~?} pensa Drago, mais ça n'avait aucun sens.

<<~Harry, dit Drago, je suis désolé, mais je dois quand même te demander, as-tu \emph{vraiment} commandé une bourse en peau de Moke très coûteuse pour l'anniversaire de la fille sang-de-bourbe~?

--- Oui, en effet. Bien sûr, tu as déjà compris pourquoi.~>>

Mû par un sentiment de frustration, Drago leva son bras et se ratissa le cuir chevelu, sa houppelande frottant l'arrière de sa main. Il n'avait \emph{pas} été certain de la raison, mais maintenant il ne pouvait pas le dire. Et Serpentard \emph{savait} qu'il faisait la cour à Harry Potter, il avait rendu cela évident en cours de Défense. <<~Harry,~>> dit Drago, <<~les gens savent que je suis ton ami, ils ne sont pas au courant pour la Conspiration, bien sûr, mais ils savent qu'on est amis, et quand tu fais ce genre de choses, ça nuit à \emph{mon} image.~>>

Le visage de Harry Potter se contracta.

<<~Quiconque à Serpentard ne peut pas comprendre le concept consistant à agir de façon gentille envers quelqu'un que l'on aime pas réellement devrait être donné à manger à des serpents de compagnie.

--- Il y a beaucoup de gens à Serpentard qui ne le \emph{comprennent pas},~>> dit Drago avec sérieux. <<~La plupart des gens sont stupides, et il faut quand même avoir une bonne image auprès d'eux.~>> Harry Potter \emph{devait} comprendre ça s'il voulait parvenir à quoi que ce soit dans sa vie.

<<~Quelle importance, ce que les autres pensent de toi~? Vas-tu vraiment passer toute ta vie à devoir expliquer tout ce que tu fais aux pires idiots de Serpentard, à \emph{les} laisser \emph{te} juger~? Je suis désolé, Drago, mais je n'abaisserai pas mes fourbes complots à un niveau que le plus idiot de Serpentard pourra comprendre juste parce que sinon, tu aurais une mauvaise image. Même notre amitié ne vaut pas ça. La vie ne serait \emph{plus amusante du tout}. Dis-moi que \emph{tu} n'as jamais pensé ça de quelqu'un à Serpentard trop stupide pour respirer~; que c'est en deçà de la dignité d'un Malfoy que de devoir les satisfaire.~>>

Drago n'y avait sincèrement pas pensé. Jamais. Satisfaire les idiots, c'était comme de respirer, vous le faisiez sans réfléchir.

<<~Harry, finit par dire Drago. Ce n'est pas malin de juste faire ce que tu veux sans te soucier de l'image que ça donne. Le Seigneur des Ténèbres se souciait de son image~! Il était craint et haï, il savait \emph{exactement} quelle sorte de peur et de haine il voulait créer. \emph{Tout le monde} doit se soucier de ce que pensent les autres.~>>

La silhouette encapuchonnée haussa les épaules. <<~Peut-être. Rappelle-moi un jour de te parler de quelque chose nommé l'expérience de conformité de Asch, tu pourrais la trouver assez amusante. Pour l'instant, je remarquerai juste qu'il est dangereux de se soucier \emph{instinctivement} de ce que pensent les autres parce que c'est \emph{vraiment important pour toi} et pas parce que c'est le résultat d'un calcul fait de sang-froid. Rappelle-toi, j'ai été battu et malmené par des Serpentard plus âgés pendant quinze minutes, et je me suis ensuite levé et je les ai pardonnés. Exactement comme le bon et vertueux Survivant se doit de faire. Mais mes calculs faits de sang-froid, Drago, me disent que les pires idiots de Serpentard ne me sont \emph{d'aucune utilité}, puisque \emph{je n'ai pas de serpent de compagnie}. Je n'ai donc aucune raison de me soucier de ce qu'ils pensent de la façon dont je conduis mon duel avec Hermione Granger.~>>

Drago ne ferma pas ses poings sous l'effet de la frustration.

<<~C'est juste une sang-de-bourbe~>>, dit Drago, se maîtrisant pour garder sa voix calme et pour ne pas crier. <<~Si tu ne l'aimes pas, pousse-la dans les escaliers.

--- Serdaigle saurait -

--- Dis à Pansy Parkinson de la pousser dans les escaliers~! Tu n'aurais même pas à la manipuler, donne-lui une Mornille et elle le fera~!

--- \emph{Je} saurais~! Hermione m'a battu dans un concours de lecture, elle a de meilleures notes que moi, je dois la battre avec mon \emph{cerveau}, ou ça ne compte pas~!

--- \emph{Ce n'est qu'une sang-de-bourbe~! Pourquoi est-ce que tu la respectes autant~?}

--- \emph{Elle a du pouvoir à Serdaigle~! Pourquoi est-ce que tu te soucies de ce qu'un idiot sans pouvoir de Serpentard peut penser~?}

--- \emph{Ça s'appelle la politique~! Et si tu ne peux pas y jouer tu ne peux pas avoir de pouvoir~!}

--- \emph{Marcher sur la Lune est un pouvoir~! Être un grand sorcier est un pouvoir~! Il y a des genres de pouvoirs qui ne m'obligent pas à passer le reste de ma vie à satisfaire des demeurés~!}~>>

Ils s'arrêtèrent tous les deux, et, presque parfaitement à l'unisson, commencèrent à prendre de profondes inspirations pour se calmer.

<<~Désolé~>>, dit Harry après quelques moments, essuyant de la sueur de son front. <<~Désolé, Drago. Tu as beaucoup de pouvoir politique et c'est logique que tu veuilles le conserver. Tu \emph{devrais} analyser ce que Serpentard pense. C'est un jeu important et je n'aurais pas dû l'insulter. Mais tu ne peux pas \emph{me} demander d'abaisser mon niveau de jeu au sein de Serdaigle juste pour que tu puisses passer du temps avec moi sans avoir une mauvaise image. Dis à Serpentard que tu grinces des dents tout en prétendant être mon ami.~>>

C'était exactement ce que Drago \emph{avait} dit à Serpentard, et il ne savait toujours pas si c'était vrai.

<<~De toute façon, dit Drago. En parlant de ton image. J'ai peur d'avoir de mauvaises nouvelles. Rita Skeeter a entendu certaines histoires te concernant et elle est allée poser des questions.~>>

Harry Potter leva les sourcils.

<<~Qui~?

--- Elle écrit pour \emph{La Gazette du sorcier,}~>> dit Drago. Il essaya de ne pas paraître soucieux. La \emph{Gazette du sorcier} était l'un des principaux outils de Père, il l'utilisait comme il aurait usé d'une baguette magique. <<~C'est le journal auquel les gens prêtent vraiment attention. Rita Skeeter parle des célébrités, et comme elle le dit, elle utilise sa plume pour crever leurs réputations surfaites. Si elle ne peut pas trouver de rumeur te concernant, elle s'en fabriquera une.

--- Je \emph{vois}~>>, dit Harry Potter. Sous la houppelande, son visage verdâtre semblait fort pensif.

Drago hésita quant à ce qu'il devait dire ensuite. Quelqu'un devait déjà avoir informé Père du fait qu'il faisait la cour à Harry Potter, et Père saurait aussi que Drago ne lui en avait pas parlé dans ses lettres, et Père comprendrait que Drago ne pensait pas qu'il pouvait réellement garder ça secret, ce qui envoyait un message clair disant que Drago jouait à son propre jeu, mais toujours du côté de Père, puisque si Drago avait succombé à la tentation, il aurait envoyé de faux rapports.

Par conséquent, Père avait probablement anticipé ce que Drago allait dire.

Jouer au jeu avec Père était une sensation plutôt déroutante. Même s'ils étaient dans le même camp. D'un côté, c'était exaltant, mais Drago savait aussi qu'à la fin, il s'avérerait que Père avait mieux joué que lui. Il était impossible que ça finisse autrement.

<<~Harry, dit enfin Drago. Ce n'est pas une suggestion. Ce n'est pas mon conseil. C'est juste une information. Mon père pourrait presque certainement étouffer cet article. Mais il t'en coûterait.~>>

Que Père se soit attendu à ce que Drago dise exactement cela à Harry Potter ne fut pas quelque chose que Drago dit tout haut. Harry Potter le comprendrait de lui-même, ou pas.

Mais au lieu de ça, Harry Potter secoua la tête, souriant sous sa houppelande. <<~Je n'ai aucune intention d'étouffer Rita Skeeter.~>>

Drago n'essaya même pas de cacher son incrédulité.

<<~Tu ne \emph{peux pas} me dire que tu ne te soucies pas de ce qu'un \emph{journal} dit sur toi~!

--- Je m'en soucie moins que tu ne pourrais le penser, dit Harry Potter. Mais j'ai mes propres méthodes pour traiter avec Skeeter et ses semblables. Je n'ai pas besoin de l'aide de Lucius.~>>

Un air inquiet apparut sur le visage de Drago avant qu'il ne puisse l'en empêcher. Quel que soit le prochain coup de Harry, c'était quelque chose auquel Père ne s'attendait pas, et cela rendait Drago très nerveux que d'imaginer où ça pourrait mener.

Drago se rendit aussi compte qu'il suait sous sa houppelande. Il n'avait jamais vraiment porté une de ces choses auparavant, et il ne s'était pas rendu compte que les houppelandes de Mangemort avaient probablement des Charmes Rafraîchissants et d'autres choses comme ça.

Harry Potter essuya de nouveau un peu de sueur de son front, grimaça, sortit sa baguette, pointa vers le haut, prit une profonde inspiration, et dit <<~\emph{Frigideiro}~!~>>

Quelques instants plus tard, Drago sentit le courant d'air froid.

<<~\emph{Frigideiro~! Frigideiro~! Frigideiro~! Frigideiro~! Frigideiro~!}~>>

Puis Harry Potter rabaissa sa baguette, même si sa main semblait un peu tremblante, et il la rangea dans sa robe.

Toute la pièce était sensiblement plus fraîche. Drago aurait aussi pu le faire, mais quand même, pas mal.

<<~Donc, dit Drago. La Science. Tu vas me parler du sang.

--- Nous allons \emph{découvrir} le sang, dit Harry Potter. En faisant des expériences.

--- Très bien, dit Drago. Quel genre d'expériences~?~>>

De sous sa houppelande, Harry Potter sourit d'un air maléfique~: <<~À toi de me le dire.~>>

\later

Drago avait entendu parler de quelque chose nommé la méthode socratique, qui consistait à enseigner en posant des questions (nommée d'après l'ancien philosophe qui avait été trop intelligent pour être un vrai Moldu et avait donc été un sorcier sang-pur en couverture). L'un de ses précepteurs avait beaucoup utilisé la méthode socratique. Ça avait été agaçant mais efficace.

Et il y avait la méthode Harry Potter, qui était cinglée.

Pour être honnête, Drago devait admettre que Harry Potter avait d'abord essayé la méthode Socratique et que ça n'avait pas très bien marché.

Harry Potter avait demandé comment Drago ferait pour \emph{réfuter} l'hypothèse des puristes du sang selon laquelle les sorciers ne pouvaient pas faire les super trucs qu'ils avaient faits il y a huit siècles parce qu'ils s'étaient métissés avec les nés-Moldus et les Cracmols.

Drago avait dit qu'il ne comprenait pas comment Harry Potter pouvait rester assis là à le regarder en face et à dire que ce n'était pas un piège.

Harry Potter avait répondu, toujours en le regardant en face, que si c'était un piège ça aurait été tellement pathétiquement évident qu'\emph{il} aurait dû être donné à manger à des serpents de compagnie, mais ce n'était \emph{pas} un piège, c'était simplement une règle sur la façon dont les scientifiques opéraient, et elle disait que vous deviez essayer de réfuter vos propres théories, et si vous faisiez un effort honnête et que vous échouiez, alors vous aviez gagné.

Drago avait essayé de faire remarquer la prodigieuse stupidité de cette méthode en suggérant que la clé pour survivre à un duel était de se jeter Avada Kedavra sur le pied et de manquer.

Harry Potter avait \emph{acquiescé}.

Drago avait secoué la tête.

Harry Potter avait alors introduit l'idée que les scientifiques regardaient les idées se battre pour voir lesquelles gagnaient, et qu'on \emph{ne pouvait pas se battre sans opposant}, alors Drago devait trouver des opposants à l'hypothèse des puristes du sang, pour qu'elle puisse les combattre et gagner, ce que Drago comprit un peu mieux, même si Harry Potter l'avait dit d'un air plutôt répugné. Par exemple~: il était clair que si le purisme du sang était la façon dont le monde fonctionnait vraiment, alors le ciel devait être bleu. Et si une autre théorie était vraie, alors le ciel devait être vert. Et personne n'avait encore vu le ciel, alors vous alliez dehors, regardiez, et les puristes du sang gagnaient~; et après que ça eut lieu six fois de suite, les gens commenceraient à remarquer la tendance.

Harry Potter avait ensuite prétendu que tous les opposants que Drago inventait étaient trop faibles, et que le purisme du sang n'aurait pas de mérite à les vaincre parce que la bataille ne serait pas assez impressionnante. Drago avait aussi compris cela. \emph{Les sorciers sont devenus plus faibles parce que les elfes de maison volent notre magie} ne lui avait pas semblé très impressionnant à lui non plus.

(Bien que Harry Potter \emph{avait} dit qu'au moins celle-là était testable, puisqu'ils pouvaient essayer de vérifier si les elfes de maison étaient devenus plus fort au fil du temps, et même dessiner une image représentant l'augmentation de la force des elfes de maison et une autre image représentant la diminution de la force des sorciers et si les deux images correspondaient ça désignerait les elfes de maison, le tout dit d'un ton si sérieux que Drago avait ressenti l'impulsion d'aller poser quelques questions précises à un Dobby sous Veritaserum avant de reprendre ses esprits.)

Et Harry Potter avait enfin dit que Drago \emph{ne pouvait pas} truquer la bataille, les scientifiques n'étaient pas stupides, ce serait \emph{évident} si on la truquait, ça devait être un \emph{vrai combat}, entre deux théories qui pourraient \emph{vraiment} être vraies, avec un test que seule la \emph{vraie} hypothèse devrait réussir, quelque chose qui \emph{aurait} des conséquences différentes selon que l'une ou l'autre hypothèse soit vraie. Harry Potter avait prétendu qu'il voulait juste savoir \emph{comment le sang fonctionnait vraiment} et que pour ça, il avait besoin de voir le purisme de sang gagner \emph{pour de vrai} et que Drago n'allait pas le tromper avec des théories qui n'attendaient que d'être abattues.

C'est alors que Harry Potter avait dit, plutôt frustré, qu'il ne pouvait pas concevoir que Drago ait \emph{vraiment} autant de mal à considérer différents points de vue, il y avait \emph{certainement} des Mangemorts qui avaient joué le rôle d'ennemis du purisme du sang et avaient trouvé des arguments bien plus plausibles contre leur propre camp que ceux que Drago avait offerts. Si Drago avait essayé de jouer le rôle d'un membre de la faction de Dumbledore, et avait trouvé l'hypothèse des elfes de maison, il n'aurait pas un instant trompé qui que ce soit.

Drago avait été forcé d'admettre que Harry marquait un point.

D'où la méthode Harry Potter.

<<~S'il vous plaît, Dr Malfoy, pleurnicha Harry Potter, pourquoi n'acceptez-vous pas mon article~?~>>

Harry Potter avait dû répéter la phrase <<~fais juste semblant de faire semblant d'être un scientifique~>> trois fois avant que Drago ne comprenne.

À cet instant, Drago s'était rendu compte qu'il y avait quelque chose de profondément \emph{tordu} dans le cerveau de Harry, et que quiconque s'essaierait à la Legilimancie à son encontre ne reviendrait probablement jamais.

Harry Potter était alors entré dans des détails considérables~: Drago devait faire semblant d'être Dr Malfoy, un Mangemort faisant semblant d'être rédacteur en chef d'une revue scientifique qui voulait rejeter l'article de son ennemi le Dr Potter, <<~De la transmission héréditaire des capacités magiques~>>, et si le Mangemort ne se comportait pas comme l'aurait fait un vrai scientifique, sa nature de Mangemort serait révélée et il serait exécuté~; en même temps, le Dr Malfoy était aussi observé par ses propres rivaux et il devait \emph{avoir l'air} de rejeter l'article du Dr Potter pour des raisons scientifiquement neutres, ou alors il perdrait son poste de rédacteur en chef.

Il était merveilleux que le Choixpeau ne soit pas à Sainte-Mangouste en train de baragouiner comme un dément.

On n'avait aussi jamais demandé à Drago de jouer un rôle aussi complexe et il n'aurait refusé le défi pour rien au monde.

Pour l'instant, comme Harry Potter avait dit, ils se mettaient dans l'ambiance.

<<~Dr Potter, j'ai bien peur que vous ayez écrit ceci avec une encre de la mauvaise couleur, dit Drago. Suivant~!~>>

Le visage du Dr Potter se débrouilla très bien pour se décomposer sous l'effet du désespoir et Drago ne put s'empêcher de ressentir un éclair de la joie du Dr Malfoy, même si le Mangemort faisait seulement semblant d'être le Dr Malfoy.

Cette partie était \emph{drôle}. Il aurait pu continuer toute la journée.

Le Dr Potter se leva de sa chaise, affalé et consterné, s'éloigna en traînant les pieds et devint Harry Potter, qui lui fit un signe d'encouragement, puis redevint le Dr Potter, s'approchant avec un sourire enthousiaste.

Le Dr Potter s'assit et présenta une feuille de parchemin au Dr Malfoy sur laquelle il était écrit~:
\begin{center}
\emph{De la transmission héréditaire des capacités magiques}

\emph{Dr H. J. Potter-Evans-Verres, Institut pour la Science Suffisamment Avancée} \end{center}

\begin{writtenNote}
Mon observation~:

Les sorciers d'aujourd'hui ne peuvent pas\\
faire des choses aussi impressionnantes que\\
celles que les sorciers faisaient il y a 800 ans.

Ma conclusion~:

Le monde sorcier est devenu plus faible en mêlant\\
son sang à celui des nés-Moldus et des Cracmols.
\end{writtenNote}

<<~Dr Malfoy, dit le Dr Potter en le regardant avec espoir, je me demandais si la \emph{Revue des Résultats Impossibles à Reproduire} pourrait envisager la publication de mon article intitulé “De la transmission héréditaire des capacités magiques”.~>>

Drago regarda le parchemin, souriant pendant qu'il envisageait des motifs de refus possibles. S'il avait été un professeur, il aurait rejeté cet essai parce qu'il était trop court, donc…

<<~C'est trop long, Dr Potter~>>, dit le Dr Malfoy.

Pendant un instant on put voir une véritable incrédulité sur le visage du Dr Potter.

<<~Ah… dit le Dr Potter. Et si je me débarrasse du double interligne entre l'observation et la conclusion, et que je mets juste \emph{“par conséquent”} -

--- Alors ce sera trop court. Suivant~!~>>

Le Dr Potter s'éloigna en traînant les pieds.

<<~Très bien, dit Harry Potter, tu deviens \emph{trop bon} à ce jeu. Encore deux fois pour t'entraîner, et la troisième fois ce sera pour de vrai, sans interruptions, je viendrai juste te voir et cette fois-ci tu refuseras le papier à cause de son véritable contenu, souviens-toi, tes rivaux scientifiques te regardent.~>>

Le prochain article du Dr Potter était parfait à tout point de vue, une merveille en son genre, mais il devait malheureusement être refusé parce que la revue du Dr Malfoy avait un problème avec la lettre E\@. Le Dr Potter offrit de réécrire l'article sans ces mots, et le Dr Malfoy expliqua qu'en fait c'était plus un problème de voyelles.

L'article suivant fut refusé parce qu'on était mardi.

On était en fait samedi.

Le Dr Potter essaya de rappeler ce fait et se vit répondre~: <<~Suivant~!~>>

(Drago commençait à comprendre pourquoi Rogue avait utilisé son emprise sur Dumbledore juste pour obtenir un poste qui le laisserait être horrible envers les élèves.)

Et alors…

Le Dr Potter s'approcha avec un petit sourire supérieur sur le visage.

<<~C'est mon tout dernier article, \emph{De la transmission héréditaire des capacités magiques}~>>, dit le Dr Potter avec assurance, et il jeta le parchemin. <<~J'ai décidé d'autoriser votre revue à le publier, et je l'ai préparée exactement selon vos directives, afin que vous puissiez le publier rapidement.~>>

Le Mangemort décida de pourchasser et de tuer le Dr Potter quand sa mission serait terminée. Le Dr Malfoy garda un sourire poli, puisque ses rivaux le regardaient, et dit…

(La pause continua tandis que le Dr Potter le regardait avec impatience.)

… <<~Laissez-moi regarder ça, s'il vous plaît.~>>

Le Dr Malfoy prit le parchemin et le parcourut avec attention.

Le fait qu'il ne soit pas un vrai scientifique commençait à rendre le Mangemort nerveux, et Drago essayait de se rappeler comment parler comme Harry Potter.

<<~Vous, ah, devez prendre en compte d'autres explications possibles pour votre, euh, observation, à part celle-ci -

--- Vraiment~? l'interrompit le Dr Potter. Comme quoi exactement~? \emph{Les elfes de maison volent notre magie}~? Mes données n'admettent qu'une seule conclusion possible, Dr Malfoy. Il n'y \emph{a} pas d'autres hypothèses plausibles.~>>

Drago essayait furieusement d'ordonner à son cerveau de penser à ce qu'il dirait s'il faisait semblant d'être dans le camp de Dumbledore, ce qu'\emph{ils} prétendaient être l'explication pour le déclin de la sorcellerie, Drago ne s'était jamais embêté à vraiment poser cette question…

<<~Si vous n'arrivez pas à imaginer une autre explication possible pour mes données, vous devrez publier mon article, \emph{Dr Malfoy}.~>>

C'est le sourire railleur sur le visage du Dr Potter qui fit tout basculer.

<<~Ah ouais~? lâcha le Dr Malfoy. Et comment savez-vous que la magie elle-même n'est pas en train de disparaître~?~>>

Le temps s'arrêta.

Drago et Harry Potter échangèrent des regards horrifiés.

Puis Harry Potter cracha quelque chose qui était probablement un mot extrêmement grossier si on avait été élevé par des Moldus. <<~\emph{Je n'avais pas pensé à ça~!}~>> dit Harry Potter. <<~Et j'aurais dû. La magie s'en va. \emph{Mince, mince, mince}~!~>>

Le ton alarmé de la voix de Harry Potter était contagieux. Sans même y penser, la main de Drago alla trouver sa baguette dans sa robe et la serra. Il avait pensé que la Maison des Malfoy était \emph{en sécurité}, tant que vous épousiez des membres de familles capables de retracer leur lignée sur quatre générations, vous étiez censés être \emph{en sécurité}, ça ne lui était jamais venu à l'esprit que personne ne pouvait rien faire pour arrêter la fin de la magie.

<<~Harry, on fait quoi~?~>> la voix de Drago montait vers la panique. <<~\emph{On fait quoi}~?

--- \emph{Laisse-moi réfléchir~!}~>>

Après quelques instants, Harry tendit sa main vers un pupitre proche et attrapa la même plume et le même rouleau de parchemin qu'il avait utilisé pour écrire son prétendu article, et il commença à griffonner quelque chose.

<<~On arrivera à le comprendre~>>, dit Harry, la gorge serrée, <<~si la magie disparaît, on trouvera à quelle vitesse elle s'en va, et combien de temps il nous reste pour faire quelque chose, et alors on trouvera pourquoi elle disparaît, et on fera quelque chose. Drago, les pouvoirs des sorciers ont-ils décliné à un taux constant, ou y a-t-il eu des chutes soudaines~?

--- Je… je ne sais pas…

--- Tu m'as dit que personne n'était arrivé au niveau des fondateurs de Poudlard. Donc ça continue depuis au moins huit siècles~? Tu ne peux pas te rappeler d'une histoire sur des problèmes apparaissant soudainement il y a cinq siècles ou quelque chose comme ça~?~>>

Drago essayait frénétiquement de réfléchir.

<<~J'ai toujours entendu dire que personne n'a été aussi bon que Merlin, et après ça, personne n'a été aussi bon que les fondateurs de Poudlard.

--- Très bien~>>, dit Harry. Il griffonnait toujours. <<~Parce que les Moldus ont commencé à ne plus croire à la magie il y a trois siècles, et je pensais que ça aurait pu avoir un rapport. Et il y a environ un siècle et demi, les Moldus ont commencé à utiliser un genre de technologie qui arrête de fonctionner à proximité de la magie et je me demandais si la réciproque aurait aussi pu être vraie.~>>

Drago explosa sur son siège, tellement en colère qu'il pouvait à peine parler.

<<~Ce sont les \emph{Moldus…}

--- \emph{Bon sang~!} rugit Harry. Est-ce que \emph{tu} t'écoutes au moins parler~? Ça dure depuis huit siècles au moins et les Moldus ne faisaient rien d'intéressant à l'époque~! \emph{Nous devons découvrir la vérité}~! Les Moldus ont \emph{peut-être} quelque chose à voir avec ça, mais si ce n'est \emph{pas} le cas, et que tu te mets à tout leur mettre sur le dos, et que ça nous empêche de découvrir ce qui se passe \emph{vraiment}, alors un matin viendra où tu te réveilleras pour découvrir que ta baguette n'est qu'un bout de bois~!~>>

La respiration de Drago se bloqua dans sa gorge. Son père avait souvent dit \emph{nos baguettes se briseront dans nos mains} dans ses discours, mais Drago n'avait jamais pensé à ce que ça \emph{voulait dire}, après tout, ça n'allait pas lui arriver à \emph{lui}. Et maintenant ça semblait très réel. \emph{Juste un bout de bois}. Drago pouvait imaginer ce que ça serait de sortir sa baguette et d'essayer de jeter un sort et de découvrir que rien ne se produisait…

Ça pouvait arriver à \emph{tout le monde}.

Il n'y aurait plus de sorciers, plus de magie, jamais. Juste des Moldus avec quelques légendes de ce que leurs ancêtres avaient été capables de faire. Certains des Moldus s'appelleraient Malfoy, et c'est tout ce qui resterait du nom.

Pour la première fois de sa vie, Drago comprit pourquoi les Mangemorts existaient.

Il avait toujours considéré comme acquis que devenir un Mangemort, c'était quelque chose qu'on faisait quand on devenait adulte. Maintenant Drago \emph{comprenait}, il savait pourquoi Père et les amis de Père avaient juré de donner leur vie pour empêcher le cauchemar d'avoir lieu, il y avait des choses que vous ne pouviez pas regarder se produire sans rien faire. Mais si ça allait avoir lieu \emph{de toute façon}, et si tous ces sacrifices, tous les amis perdus victimes de Dumbledore, toute la \emph{famille} qu'ils avaient perdue, et si tout avait été pour \emph{rien}.

<<~La magie ne \emph{peut pas} être en train de disparaître~>>, dit Drago. Sa voix se brisait. <<~Ça ne serait pas \emph{juste}.~>>

Harry s'arrêta de griffonner et leva les yeux. Son visage affichait de la colère. <<~Ton père ne t'a jamais dit que la vie n'est pas juste~?~>>

Père avait dit cela à chaque fois que Drago avait utilisé ce mot.

<<~Mais, mais, c'est juste trop horrible d'y croire…

--- Drago, laisse-moi te présenter quelque chose que j'appelle la Litanie de Tarski. Elle change à chaque fois que tu l'utilises. Cette fois-ci, elle se prononce ainsi~: \emph{Si la magie disparaît du monde, je veux croire que la magie disparaît du monde. Si la magie ne disparaît pas du monde, je veux ne pas croire que la magie disparaît du monde. Puis-je ne pas devenir attaché à des croyances que je ne souhaite pas avoir.} Si nous vivons dans un monde où la magie disparaît, \emph{c'est ce que nous devons croire}, nous devons savoir ce qui va arriver pour pouvoir l'arrêter, ou dans le pire des cas pour être prêt à faire ce qu'on peut avec le temps qui nous reste. Ne pas le croire ne l'empêchera pas d'arriver. Donc la \emph{seule} question que nous devons nous poser, c'est si la magie disparaît \emph{vraiment}, et si c'est le monde dans lequel nous vivons, alors c'est ce en quoi nous voulons croire. Litanie de Gendlin~: \emph{Ce qui est vrai l'est déjà, l'admettre ne le rend pas pire.} Dis-le.

--- Ce qui est vrai l'est déjà~>>, répéta Drago, la voix tremblante, <<~l'admettre ne le rend pas pire.

--- Si la magie disparaît, je veux croire que la magie disparaît. Si la magie ne disparaît pas, je veux ne pas croire que la magie disparaît. Dis-le.~>>

Drago répéta les mots, la nausée soulevant son estomac.

<<~Bien, dit Harry, souviens-toi, ça n'est peut-être \emph{pas} en train d'arriver, et alors tu ne devras pas non plus y croire. \emph{D'abord}, on veut juste savoir ce qui se passe vraiment, dans quel monde nous vivons vraiment.~>> Harry revint à son travail, griffonna un peu plus, puis tourna le parchemin pour que Drago puisse le voir. Drago se pencha sur le bureau et Harry rapprocha la lumière verte.

\penalty-10
\begin{center}\itshape
{\scshape Observation~:}\\
La sorcellerie n'est pas aussi puissante qu'elle ne l'était quand Poudlard a été fondée. \penalty101

{\scshape Hypothèses~:}\penalty102
\begin{enumerate}[1.]\firmlist
\item La magie elle-même disparaît.
\item Les sorciers se métissent avec les Moldus et les Cracmols.
\item Le savoir permettant de jeter des sorts puissants se perd.
\item Les sorciers ne mangent pas ce qu'il faut étant enfants, ou quelque chose d'autre à part le sang les fait devenir plus faible.
\item La technologie Moldue interfère avec la magie (depuis 800 ans~?).
\item Les sorciers plus puissants ont moins d'enfants (Drago = fils unique~? Vérifier si trois sorciers puissants, Quirrell / Dumbledore / Seigneur des Ténèbres ont eu des enfants). \end{enumerate} {\scshape Tests~:} \end{center}

\emph{{Tests~:}}

<<~Très bien~>>, dit Harry. Sa respiration semblait un peu plus calme. <<~Maintenant, quand tu fais face à un problème déroutant et que tu ne sais pas ce qui se passe, la réaction intelligente est de trouver des tests très simples, des choses que tu peux tout de suite vérifier. On a besoin de tests rapides pour établir une distinction entre ces hypothèses. Des observations qui seraient différentes pour au moins l'une d'elles par rapport aux autres.~>>

Drago regarda la liste, choqué. Il se rendit soudain compte qu'il connaissait une effroyable quantité de Sang-Purs qui étaient enfants uniques. Lui-même, Vincent, Gregory, presque \emph{tout le monde}. Les deux sorciers les plus puissants dont tout le monde parlait étaient Dumbledore et le Seigneur des Ténèbres et aucun d'eux n'avait d'enfant, comme Harry l'avait soupçonné…

<<~Ça va être très dur de faire la distinction entre 2 et 6,~>> dit Harry, <<~c'est dans le sang dans un cas comme dans l'autre, et il faudrait essayer de suivre le déclin de la sorcellerie, et le comparer au nombre d'enfants qu'ont différents types de sorciers, et mesurer les capacités des nés-Moldus comparées à celles des Sang-Purs…~>> Les doigts de Harry tapotaient nerveusement le bureau. <<~Mettons juste 6 et 2 dans le même panier et appelons-les pour l'instant l'hypothèse du sang. 4 est peu probable parce que tout le monde aurait remarqué une baisse soudaine quand les sorciers seraient passés à un nouveau régime, c'est difficile d'imaginer ce qui aurait pu changer avec régularité pendant les derniers 800 ans. 5 est peu probable pour la même raison, pas de baisses soudaines, et puis les Moldus ne faisaient rien il y a 800 ans. De toute façon 4 ressemble à 2 et 5 ressemble à 1. Donc on devrait surtout essayer de distinguer entre 1, 2 et 3.~>> Harry fit pivoter le parchemin face à lui, dessina une ellipse autour de ces trois chiffres et le fit à nouveau pivoter. <<~La magie disparaît, le sang s'affaiblit, le savoir diminue. Quel test produit un résultat différent selon que l'une de ces trois hypothèses est vraie~? Que pourrions-nous observer qui voudrait dire qu'une de ces trois est fausse~?

--- \emph{Je} ne sais pas~! laissa échapper Drago. Pourquoi tu me demandes~? C'est toi le scientifique~!

--- Drago~>>, dit Harry d'un ton qui s'approchait légèrement du plaidoyer désespéré, <<~je ne sais que ce que savent les scientifiques moldus~! Tu as grandi dans le monde magique, pas moi~! Tu connais plus de magie que moi, tu en sais plus \emph{sur} la magie que moi et c'est toi qui as eu l'idée à la base, alors commence à penser comme un scientifique et résous ça~!~>>

Drago avala sa salive avec difficulté et fixa le papier.

La magie disparaît… les sorciers se métissent avec les Moldus… le savoir est perdu…

<<~À quoi ressemble le monde si la magie disparaît~? dit Harry Potter. Tu en sais plus sur la magie, tu devrais être celui qui essaie de deviner, pas moi~! Imagine que tu racontes une histoire sur le sujet, qu'est-ce qui se passe dans l'histoire~?~>>

Drago imagina. <<~Des sorts qui fonctionnaient ne fonctionnent plus.~>> \emph{Les sorciers se réveillent et découvrent que leur baguette sont des bouts de bois…}

<<~À quoi ressemble le monde si le sang sorcier s'affaiblit~?

--- Les gens ne peuvent pas faire des choses que leurs ancêtres pouvaient faire.

--- À quoi ressemble le monde si le savoir est perdu~?

--- Les gens ne savent même plus comment jeter des sorts…~>> dit Drago. Il s'arrêta, s'étant lui-même surpris. <<~C'est un test, non~?~>>

Harry hocha la tête avec fermeté. <<~C'en est un.~>> Il l'écrivit sur le parchemin, en dessous de \emph{Tests}~:

\emph{A. Y a-t-il des sorts que l'on connaît mais qu'on ne peut pas jeter (1 ou 2) ou des sorts perdus qu'on ne connaît plus (3)~?}

<<~Donc ça distingue entre 1 et 2 d'un côté et 3 de l'autre, dit Harry. Maintenant on a besoin d'un moyen de faire la distinction entre 1 et 2. La magie disparaît, le sang s'affaiblit, comment pourrait-on faire la différence~?

--- Quel genre de sorts les étudiants jetaient-ils dans leur première année à Poudlard~? dit Drago. S'ils jetaient des sorts bien plus puissants, leur sang était plus fort…~>>

Harry Potter secoua sa tête. <<~Ou la magie elle-même était plus forte. Nous devons trouver un moyen de faire la \emph{différence}.~>> Harry se leva de sa chaise et commença à arpenter nerveusement la salle. <<~Non, attends, ça pourrait quand même marcher. Imagine que différents sorts utilisent différentes quantités d'énergie magique. Alors si la magie ambiante s'affaiblissait, les sorts les plus puissants mourraient en premier, mais les sorts que tout le monde apprend en première année resteraient les mêmes…~>> Les déambulations de Harry s'accélérèrent. <<~Ce n'est pas un très bon test, ça distingue plutôt entre la perte de la sorcellerie puissante et la perte de toute la sorcellerie, car le sang de quelqu'un pourrait être trop faible pour la sorcellerie puissante mais assez fort pour les sorts simples… Drago, sais-tu si les sorciers les plus puissants d'\emph{une} époque, par exemple les sorciers puissants de ce siècle, étaient plus puissants que les autres enfants~? Si le Seigneur des Ténèbres avait jeté le sort de Refroidissement, aurait-il pu geler la salle entière~?~>>

Le visage de Drago se tordit tandis qu'il essayait de se souvenir. <<~Je n'arrive pas à me rappeler avoir entendu quoi que ce soit au sujet du Seigneur des Ténèbres, mais je pense que Dumbledore est censé avoir fait quelque chose d'incroyable lors de ses BUSE de Métamorphose en cinquième année… je pense que les autres sorciers puissants étaient puissants à Poudlard aussi…~>>

Harry se renfrogna, déambulant toujours. <<~Peut-être qu'ils étudiaient simplement beaucoup. Mais quand même, si les élèves en première année apprenaient les mêmes sorts et semblaient être aussi puissants à cette époque qu'ils le sont aujourd'hui, alors nous pourrions appeler ça un \emph{faible} élément de preuve en faveur de 1 et 2… attends, ne bouge pas.~>> Harry s'arrêta. <<~J'ai un autre test qui pourrait distinguer entre 1 et 2. Ça prendrait un moment à expliquer, car ça utilise certaines choses que les scientifiques savent au sujet du sang et de l'hérédité, mais sa vérification est facile. Et si on \emph{combine} mon test et ton test et qu'ils fournissent le même résultat, ce sera un bon indice quant à la bonne réponse.~>> Harry courut presque jusqu'au bureau, prit le parchemin, et écrivit~:

\emph{B. Les anciens élèves en première année jetaient-ils le même genre de sort qu'aujourd'hui avec la même puissance~? (Faible élément de preuve pour 1 et contre 2, mais le sang pourrait aussi faire perdre seulement la sorcellerie puissante).}

\emph{C. Test supplémentaire qui distingue entre 1 et 2 en utilisant les connaissances scientifiques sur le sang, j'expliquerai plus tard.}

<<~OK, dit Harry, on peut au moins essayer de déterminer la différence entre 1, 2 et 3, alors commençons tout de suite par ça, on pourra toujours trouver \emph{d'autres} tests après avoir terminé ceux-ci. Cela dit, ça va avoir l'air un petit peu étrange si Drago Malfoy et Harry Potter se promènent en posant plein de questions ensemble, alors voilà mon idée~: tu traverses Poudlard et tu trouves des vieux portraits et tu les interroges sur les sorts qu'ils ont appris à jeter pendant leurs premières années. Ce sont des portraits, alors ils ne sauront pas que c'est étrange que Drago Malfoy pose une question comme ça. J'interrogerai des portraits récents et des vivants au sujet des sorts qu'on connaît mais qu'on ne peut pas jeter, personne ne trouvera ça inhabituel que Harry Potter pose des questions bizarres. Et je devrai faire des recherches compliquées au sujet des sorts oubliés, donc je veux que ce soit toi qui assembles les données dont j'ai besoin pour ma question scientifique. C'est une question simple, et tu devrais pouvoir trouver la réponse en interrogeant les portraits. Tu vas peut-être vouloir noter ça, tu es prêt~?~>>

Drago s'assit à nouveau et fouilla son cartable à la recherche d'un parchemin et d'une plume. Lorsqu'ils furent installés au bureau, Drago releva les yeux, l'air déterminé.

<<~Vas-y.

--- Trouve des portraits qui connaissaient un couple de Cracmols mariés -- ne fais pas cette tête Drago, c'est une information importante. Demande juste aux portraits récents de Gryffondor, je ne sais pas. Interroge les portraits qui connaissaient un tel couple assez bien pour connaître les noms de tous leurs enfants. Note le nom de chaque enfant, et si cet enfant était un sorcier, un Cracmol, ou un Moldu. S'ils ne savent pas si l'enfant était un Cracmol ou un Moldu, écris “non-sorcier”. Note ça pour \emph{chaque} enfant qu'avait le couple, n'en omet aucun. Si le portrait connaissait seulement le nom des enfants sorciers, pas ceux de \emph{tous} les enfants, alors n'écris \emph{aucune} donnée sur ce couple. Il est très important que tu m'apportes uniquement les données de ceux qui connaissaient \emph{tous} les enfants qu'un couple Cracmol avait, assez bien pour connaître leur nom. Essaie au moins d'obtenir quarante noms si tu peux, et si tu as le temps d'en avoir plus, encore mieux. Tu as noté tout ça~?

--- Répète~>>, dit Drago quand il eut fini d'écrire, et Harry se répéta.

<<~C'est noté, dit Drago, mais pourquoi…

--- Ça a à voir avec l'un des secrets du sang que les scientifiques ont déjà découvert. Je t'expliquerai quand tu reviendras. Séparons-nous et retrouvons-nous ici dans une heure, ce qui devrait faire 18h22. On est prêt à y aller~?~>>

Drago hocha la tête avec fermeté. Tout cela était très précipité, mais il avait appris à se précipiter il y a bien longtemps.

<<~Alors \emph{vas-y}~!~>> dit Harry Potter, et il rejeta sa houppelande et la fourra dans sa bourse, qui commença à la manger et, sans même attendre que sa bourse ait terminé, il pivota et commença à avancer à grandes enjambées vers la porte de la salle, se cognant dans un bureau et tombant presque dans sa hâte.

Quand Drago eut fini d'ôter sa propre houppelande et de la ranger dans son cartable, Harry Potter était parti.

Drago courut presque vers la sortie.~
%  LocalWords:  Oogely boogely oo ee AAAAAAAAARRRRRRGHHHH Aw Mmm Asch’s
%  LocalWords:  Whyyyyyyyyyyyyyyyy AAAAAAAAAAAAAAHHHHHHHHHHHHHHH Gendlin
