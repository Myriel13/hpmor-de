\chapter{Théorie de l'identité individuelle}

\lettrine{T}{out} complot voit advenir un moment où la victime commence à avoir des soupçons~; et, jetant un regard en arrière, elle voit une série d'événements pointant tous dans une unique direction. Et Père avait expliqué à Drago que lorsque ce moment survient, la perspective de la perte peut sembler si insupportable à la victime, l'admission de sa crédulité peut lui sembler si humiliante, qu'elle nie malgré tout l'existence d'un complot, et le jeu peut alors continuer longtemps.

Père avait prévenu Drago qu'il ne devrait plus recommencer.

Mais avant, il avait laissé M. Avery finir tous les cookies qu'il avait escroqués à Drago tandis que ce dernier regardait en pleurant. Tout le bocal de délicieux cookies que Père lui avait donné à peine quelques heures plus tôt, car Drago les avaient tous perdus au jeu contre M. Avery, jusqu'au dernier.

C'était donc une sensation familière que Drago avait ressentie au fond de son estomac quand Grégory lui avait parlé du Baiser.

Parfois, vous jetiez un regard en arrière, et vous voyiez des choses…

(Dans une salle sans lumière -- on ne pouvait plus tout à fait la qualifier d'\emph{inusitée}, puisqu'elle avait été mise à profit de façon hebdomadaire pendant les quelques mois précédents -- un garçon était assis, enveloppé dans une houppelande à capuche, avec un globe de cristal éteint posé sur le bureau devant lui. Réfléchissant en silence, réfléchissant dans les ténèbres, attendant qu'une porte ouverte laisse entrer la lumière.)

Harry avait repoussé Granger et il avait dit~: “\emph{Je t'ai dit pas de bisou~!}”

Il dirait probablement quelque chose comme~: “\emph{Elle l'a fait juste pour m'agacer la dernière fois, exactement comme elle m'a fait aller à ce rendez-vous galant.}”

Mais l'histoire vérifiée était que Granger avait été prête à refaire face au Détraqueur afin d'aider Harry~; qu'elle l'avait embrassé en pleurant, alors qu'il était perdu dans les profondeurs du détraquage~; et que son baiser l'avait ramené.

Ça ne ressemblait pas à de la rivalité, pas même à de la rivalité amicale.

Ça ressemblait au genre d'amitié qu'on ne voyait généralement même pas dans les pièces.

Alors pourquoi Harry avait-il fait grimper les murs gelés de Poudlard à son amie~?

Parce que c'était le genre de chose que Harry Potter faisait à ses amis~?

Père avait dit à Drago que pour appréhender un complot étrange, une technique consistait à regarder ce qui \emph{finissait} par arriver, en partant du principe que c'était le résultat \emph{attendu}, et à se demander qui en bénéficiait.

Ce qui avait fini par arriver après que Drago et Granger combattent Harry Potter ensemble… c'était que Drago avait commencé à avoir des sentiments beaucoup plus amicaux envers Granger.

Qui bénéficiait de voir l'héritier Malfoy devenir l'ami d'une sorcière Sang-de-Bourbe~?

Qui en bénéficiait et était connu pour ce genre de complot~?

Qui en bénéficiait et était peut-être capable de manipuler Harry Potter~?

Dumbledore.

Et si c'était vrai alors Drago \emph{devrait} aller voir Père et tout lui dire, quoi qu'il arrive ensuite, peu importe ce qui arriverait ensuite, Drago ne pouvait même pas imaginer ce qui arriverait ensuite, c'était horrible au-delà de toute imagination. Ce qui le poussait à s'accrocher désespérément à la dernière parcelle d'espoir que les choses n'étaient pas du tout ce qu'elles avaient l'air d'être…

… Drago se souvenait aussi de ça, lors de la leçon de M. Avery.

Drago n'avait pas prévu de se confronter immédiatement à Harry. Il essayait encore de deviser un test expérimental, quelque chose que Harry ne percerait pas facilement à jour, où il ne pourrait pas faire semblant. Mais Vincent lui avait alors porté un message disant que Harry voulait le rencontrer tôt cette semaine, vendredi au lieu de samedi.

Et Drago était là, dans la salle obscure, un globe de cristal éteint sur son bureau. Il attendait.

Des minutes s'écoulèrent.

Des pas approchèrent.

La porte émit un doux craquement en s'ouvrant, révélant Harry Potter, qui était habillé de sa propre houppelande~; il s'avança dans la salle obscure, et la solide porte se referma derrière lui avec un léger clic.

Drago toucha le globe de cristal et la salle s'illumina d'une puissante lumière verte. Elle projetait les ombres des bureaux jusqu'au sol et se réfléchissait jusqu'à lui depuis les dossiers incurvés des chaises, faite de photons qui rebondissaient sur le bois d'une façon telle que l'angle d'incidence était égal à l'angle de réflexion.

Au moins, \emph{cette} partie de ce qu'il avait appris n'avait probablement pas été un mensonge.

Harry avait tressailli lorsque le globe s'était illuminé, il s'était arrêté, puis il avait repris son approche.

«Bonjour, Drago», dit doucement Harry, relevant sa capuche à l'approche du bureau de Drago. «Merci d'être venu. Je sais que ce n'est pas ton horaire habituel…

--- De rien,» dit Drago d'une voix sans timbre.

Harry traîna l'une des chaises pour faire face à Drago, et les pieds de la chaise émirent un léger crissement en raclant le sol. Il fit pivoter la chaise, pour qu'elle soit à l'envers, et il l'enfourcha, ses bras croisés contre le dossier. Le visage du garçon était pensif, sourcils froncés, sérieux, un air très adulte, même pour Harry Potter.

«J'ai une question importante à te poser, dit Harry, mais avant cela, il y a autre chose que j'aimerais que nous fassions.»

Drago ne dit rien, et il commença à ressentir une certaine lassitude. Une partie de lui voulait juste en avoir fini avec tout cela.

«Dis-moi, Drago, dit Harry. Pourquoi les Moldus ne laissent-ils jamais de fantômes derrière eux lorsqu'ils meurent~?

--- Parce que les Moldus n'ont pas d'âme, évidemment», dit Drago. Il ne se rendit compte que cela contredisait peut-être les opinons politiques de Harry qu'après avoir prononcé ces mots, et il décida qu'il s'en fichait. Et après tout, c'\emph{était} évident.

Le visage de Harry ne fit montre d'aucune surprise. «Avant de te poser ma question importante, je veux voir si tu peux apprendre le Patronus.»

Pendant un instant, l'absurdité patente estomaqua Drago. Bon vieux Harry impossible-à-prédire-où-à-comprendre. Drago se demandait parfois si Harry utilisait délibérément cette tactique afin de désorienter ses interlocuteurs.

Puis il comprit, et il se redressa et s'éloigna du bureau d'un seul mouvement furieux. C'en était trop. C'était fini.

«Comme les serviteurs de Dumbledore, cracha-t-il.

--- Comme Salazar Serpentard», dit Harry d'une voix ferme.

Drago failli trébucher au beau milieu de son premier pas en direction de la porte.

Il se retourna lentement vers Harry.

«Je ne sais pas où tu es allé dénicher ça, dit Drago, mais c'est faux, tout le monde sait que le Patronus est un sortilège Gryffondor…

--- Salazar Serpentard pouvait lancer un Patronus corporel», dit Harry. Sa main darda vers sa robe et en tira un livre dont le titre avait été écrit en blanc sur vert, ce qui le rendait donc presque impossible à déchiffrer dans la lumière verdâtre~; mais il avait l'air ancien. «J'ai découvert ça en faisant des recherches sur le Patronus. Et j'ai trouvé la référence originale et emprunté le livre à la bibliothèque juste au cas où tu ne me croirais pas. Et l'auteur de ce livre ne pensait pas qu'il y ait quoi que ce soit de \emph{surprenant} à ce que Salazar ait été capable de lancer un Patronus~; la croyance en l'incapacité des Serpentard à lancer ce sort doit être récente. Et une note historique de plus, même si je n'ai pas le livre sur moi~: Godric Gryffondor n'en a jamais été capable.»

Après avoir essayé six fois de révéler l'imposture de Harry, chaque tentative plus ridicule que la précédente, Drago se rendit compte que Harry ne mentait tout simplement \emph{pas} au sujet de ce qui était écrit dans les livres. Malgré cela, lorsque les mains de Harry ouvrirent le livre à un marque-page, Drago se pencha en avant et étudia l'emplacement pointé par le doigt de Harry.

\emph{Puis les feux de Serdaigle s'abattirent sur les ténèbres qui avaient masqué l'aile gauche de l'armée de Lord Foul, brisant celles-ci, et il fut révélé que Lord Gryffondor avait dit vrai~; la peur qu'ils avaient tous ressentie n'avait pas une source naturelle mais provenait d'une triple douzaine de Détraqueurs à qui les âmes des vaincus avaient été promises. Lady Poufsouffle et Lord Serpentard firent immédiatement jaillir leur Patronus, un large blaireau en colère et un serpent d'argent étincelant, et alors que les ombres quittaient leur cœur, les défenseurs relevèrent leur tête. Et Lady Serdaigle rit, et remarqua que Lord Foul était fort simplet, car ce serait maintenant son armée qui serait sujette à la peur, pas les défenseurs de Poudlard. Et pourtant Lord Serpentard dit~: «Simplet, il ne l'est pas, cela j'en suis certain.» Et à ses côtés, Lord Gryffondor étudia le champ de bataille, alors qu'un air inquiet s'abattait sur ses traits…}

Drago releva les yeux. «Alors~?»

Harry ferma le livre et le mit dans sa bourse. «Chaos et Soleil ont tous deux des soldats capables de lancer des Patronus corporels. Ceux-ci peuvent être utilisés pour transmettre des messages. Si tu ne peux pas apprendre ce sort, l'armée Dragon aura un désavantage militaire sévère…»

Drago s'en fichait, et il le dit à Harry. Sa voix était probablement plus acerbe qu'elle n'aurait dû l'être.

Harry ne cilla pas. «Alors j'invoque la faveur que tu me dois, de cette fois où j'ai empêché qu'une émeute ne se déclenche lors de notre première leçon de vol sur balai. Je vais essayer de t'enseigner le Patronus, et je veux que tu me fasses la faveur d'essayer de ton mieux et honnêtement. Par l'honneur de la maison Malfoy, j'espère que tu le feras.»

Drago ressentit de nouveau une certaine lassitude. Si Harry avait demandé à n'importe quel autre moment, ç'aurait été une juste contrepartie à la faveur due, étant donné que le sortilège n'était pas vraiment Gryffondor. Mais…

«\emph{Pourquoi~?} dit Drago.

--- Pour découvrir si tu peux faire cette chose que Salazar Serpentard pouvait faire, dit Harry d'une voix neutre. C'est un test expérimental, et je ne te dirai pas ce qu'il signifie avant que tu ne l'aies exécuté. Vas-tu le faire~?»

… C'\emph{était} probablement une bonne idée de se décharger de cette faveur en faisant quelque chose d'innocent, et ce d'autant plus que l'heure de rompre avec Harry Potter semblait être venue. «Très bien.»

Harry fit émerger une baguette de sa robe et la posa contre le globe. «Pas vraiment la meilleure couleur pour apprendre le Patronus, dit Harry. Je veux dire, une lumière verte, exactement de la même teinte que celle du sort de Mort. Mais l'argent est aussi une couleur Serpentard, non~? \emph{Dulak.}» La lumière disparut, et Harry murmura les deux premières phrases du sortilège de lumière continue, relançant cette partie du sort, même si aucun d'eux n'aurait pu lancer le sortilège en entier. Puis Harry toucha le globe de nouveau, et la pièce s'éclaira d'argent étincelant, mais aussi doux et apaisant.

C'est le temps que Drago mit à comprendre ce que la phrase précédente de Harry impliquait.

«Tu as vu un \emph{sort de Mort} depuis la dernière fois qu'on s'est vu~? Quand -- comment…

--- Lance le Patronus», dit Harry, l'air plus sérieux que jamais, «et je te le dirai.»

Drago comprima ses paupières à l'aide de ses mains, masquant la lumière argentée. «Tu sais, je devrais vraiment me souvenir que tu es trop \emph{bizarre} pour faire des complots \emph{normaux}~!»

Depuis les ténèbres qu'il s'était imposé, il entendit le ricanement de Harry.

\later

Harry observait Drago de près alors que celui-ci terminait sa dernière répétition des gestes préliminaires, la partie du sort qui était difficile à apprendre~; le brandissement final et la prononciation n'avaient pas besoin d'être précis. Harry n'avait pas pu repérer d'imperfection lors des trois dernières répétitions, mais il avait aussi ressenti une étrange impulsion, celle d'ajuster des détails au sujet desquels M. Lupin n'avait rien dit, comme l'angle du coude de Drago, ou la direction vers laquelle son pied pointait~; ça avait peut-être été entièrement le fruit de son imagination, ça l'était même probablement, mais Harry avait décidé de quand même suivre son instinct, juste au cas où.

«Très bien», dit doucement Harry. Il y avait une tension dans sa poitrine, et elle lui rendait la parole légèrement ardue. «Bon, nous n'avons pas de Détraqueur ici, mais ça ne fait rien. Nous n'en avons pas besoin. Drago, quand ton père m'a parlé à la gare, il a dit que tu étais ce qu'il avait de plus précieux au monde, et il a menacé de laisser tomber toutes ses autres ambitions pour se venger de moi si jamais il t'arrivait malheur.

--- Il… quoi~?» La voix de Drago s'était brutalement interrompue, et son regard était devenu étrange. «Pourquoi est-ce que tu me dis \emph{ça}~?

--- Pourquoi ne le ferais-je pas~?» Harry ne laissa pas ses traits bouger, mais il pouvait deviner ce que Drago pensait~; que Harry avait comploté pour séparer Drago de son père et qu'il ne devrait rien dire qui puisse les rapprocher l'un de l'autre. «Il y a toujours eu une seule personne d'importance à tes yeux, et je sais exactement quelle pensée heureuse te permettra de lancer le Patronus. Tu me l'as dit à la gare avant le premier jour de cours. Un jour, tu es tombé de ton balai et tu t'es brisé des côtes. Ça t'a fait plus mal que tout ce que tu avais ressenti auparavant, et tu as pensé que tu allais mourir. Fais comme si cette peur venait d'un Détraqueur qui serait debout devant toi, habillé d'une cape en lambeaux, et qui ressemblerait à quelque chose de mort qu'on aurait laissé dans l'eau. Puis lance le Patronus, et quand tu brandiras la baguette pour repousser le Détraqueur, pense à la façon dont ton père a tenu ta main pour que tu n'aies pas peur~; puis pense à tout l'amour qu'il a pour toi, et à celui que tu as pour lui, et mets tout cela dans ta voix lorsque tu diras \emph{Expecto Patronum}. Pour l'honneur de la maison Malfoy, et pas seulement parce que tu m'as promis une faveur. Montre-moi que tu ne m'as pas menti ce jour-là, à la gare, quand tu m'as dit que Lucius était un bon père. Montre-moi que tu peux faire ce dont Salazar Serpentard était capable.»

Et Harry fit un pas en arrière, derrière Drago, hors de son champ de vision, afin qu'il ne soit face qu'au vieux bureau d'enseignant et au tableau noir au fond de la salle inusitée.

Drago jeta un coup d'œil derrière lui, l'étrange regard toujours présent, puis il se détourna et regarda devant lui. Harry vit l'exhalation, l'inhalation. La baguette pivota une fois, deux fois, trois fois, quatre fois. Les doigts de Drago glissèrent sur la baguette, de la bonne distance…

Drago abaissa sa baguette.

«C'est trop… dit Drago, je ne peux pas y \emph{penser} correctement pendant que tu regardes…»

Harry se détourna et commença à marcher vers la porte. «Je reviendrai dans une minute, dit Harry. Maintiens juste ta pensée heureuse et le Patronus restera.»

\later

De derrière Drago vint le son d'une porte qui s'ouvrait de nouveau.

Il entendit les pas de Harry entrer dans la salle, mais il ne se retourna pas pour regarder.

Harry ne dit rien non plus. Le silence s'étira.

Enfin…

«Qu'est-ce que ça veut \emph{dire}~?» dit Drago. Sa voix vacillait un peu.

«Ça veut dire que tu aimes ton père», dit la voix de Harry. Ce qui était exactement ce que Drago avait pensé, et il essayait de ne pas pleurer devant Harry. C'est trop juste, beaucoup trop juste…

Devant Drago, à même le sol, se trouvait la forme étincelante d'un serpent que Drago reconnu~; un Bungarus Candidus, un serpent initialement amené à leur manoir par Lord Abraxas Malfoy, après un séjour dans un pays lointain quelconque, et depuis ce jour, Père avait conservé un Bungarus Candidus dans l'ophidiarium. La particularité du Bungarus Candidus était que sa morsure ne faisait pas très mal. Père avait dit cela, puis il avait dit à Drago qu'il ne pourrait \emph{jamais} caresser l'animal, peu importe qui le surveillerait. Le venin tuait vos nerfs si vite que vous n'aviez pas le temps de sentir la douleur à mesure que le poison se répandait. Vous pouviez en mourir même après avoir utilisé des sorts de soin. Il mangeait d'autres serpents. Il était aussi Serpentard qu'une créature pouvait l'être.

C'était pourquoi on avait forgé la tête de la cane de Père à l'image de celle d'un Bungarus Candidus.

Le serpent lumineux darda sa langue, qui était d'argent elle aussi~; et il commença, sans que l'on sache comment, à \emph{sourire}, avec plus de chaleur qu'un reptile n'aurait jamais dû le faire.

Et Drago comprit alors…

«Mais», dit Drago, regardant encore le magnifique serpent étincelant, «\emph{tu} ne peux pas lancer le Patronus.» Maintenant que Drago l'avait lancé, il comprenait pourquoi c'était important. On pouvait être méchant, comme Dumbledore, et malgré tout lancer le Patronus, tant qu'on avait \emph{quelque chose} de lumineux à l'intérieur de soi. Mais si Harry Potter n'avait pas la moindre pensée qui puisse ainsi briller…

«Le sortilège du Patronus est plus complexe que tu le penses, dit Harry avec sérieux. Tous ceux qui échouent à le lancer ne sont pas des gens mauvais, ni même des gens tristes. Mais quoi qu'il en soit, je \emph{peux} le lancer. Je l'ai fait à mon deuxième essai, après m'être rendu compte de l'erreur que j'avais faite la première fois que j'avais fait face au Détraqueur. Mais, eh bien, ma vie a tendance à être singulière, et mon Patronus a une forme étrange, et je vais la garder secrète pour le moment…

--- Et je suis juste censé \emph{croire} ça~?

--- Tu peux demander au professeur Quirrell si tu ne me crois pas, dit Harry. Demande-lui si Harry Potter peut lancer un Patronus corporel, et dis-lui que je t'ai dit de demander. Il saura que la requête vient de moi, personne d'autre ne pourrait le savoir.»

Oh, et maintenant Drago devait faire confiance au \emph{professeur Quirrell~?} Mais quand même, connaissant Harry, ça pouvait être vrai~; et le professeur Quirrell ne mentirait pas pour des raisons triviales.

Le serpent radieux tourna sa tête de gauche à droite, comme à la recherche d'une proie inexistante, puis il s'enroula sur lui-même comme pour se reposer.

«Je me demande, dit doucement Harry, quand c'est arrivé, quelle année, quelle génération, quel jour les Serpentard ont arrêté d'essayer d'apprendre le Patronus. Quand les gens ont commencé à penser, quand les Serpentard eux-mêmes ont commencé à penser qu'être trompeur et ambitieux, c'était être froid et malheureux. Et si Serpentard savait que ses élèves ne se fatiguaient même plus à venir apprendre le Patronus, je me demande s'il souhaiterait n'être jamais né~? Je me demande quand tout est allé de travers, quand la maison Serpentard est allée de travers.»

La créature étincelante disparut en un clin d'œil, car l'agitation qui montait en Drago l'empêchait de maintenir le sort. Il pivota pour faire face à Harry et dut se contrôler pour ne pas lever sa baguette. «Qu'est-ce que \emph{tu} sais de la maison Serpentard \emph{ou} de Salazar Serpentard~? \emph{Tu} n'as jamais été réparti dans ma maison, qu'est-ce qui te donne le droit de…»

Et c'est \emph{là} que Drago se rendit \emph{enfin} compte.

«\emph{Tu as été Trié à Serpentard~! } dit Drago. C'est \emph{ça}, et après tu, tu as, je ne sais pas, tu as \emph{claqué des doigts}…» Drago avait un jour demandé à Père s'il serait plus malin de se faire répartir dans une autre maison, pour que tout le monde lui fasse confiance, et Père avait souri et dit qu'il y avait lui aussi pensé à son âge, mais qu'il était impossible de tromper le Choixpeau…

… jusqu'à ce que \emph{Harry Potter} arrive.

Comment avait-il pu croire \emph{une minute} que \emph{Harry} était un \emph{Serdaigle}~?

«Une hypothèse intéressante, dit calmement Harry. Sais-tu que tu es la seconde personne de Poudlard à inventer une théorie de ce genre~? Du moins, tu es le second à me l'avoir dite en face…

--- Rogue», dit Drago avec certitude. Le directeur sa Maison n'était pas un idiot.

«Professeur Quirrell, \emph{bien sûr}, dit Harry. Quoique maintenant que j'y pense, Severus m'a en effet demandé comment j'étais parvenu à rester hors de sa maison et si j'avais été en possession de quelque chose que le Choixpeau voulait. J'imagine qu'on pourrait dire que tu es le troisième. Oh, mais la théorie du professeur Quirrell était légèrement différente de la tienne, cela dit. Pourrais-je avoir ta parole de ne pas le répéter~?»

Drago hocha la tête sans même y penser. Qu'était-il censé faire, répondre non~?

«Le professeur Quirrell pensait que Dumbledore n'était pas satisfait de la décision du Choixpeau au sujet du Survivant.»

Et à l'instant où Harry le dit, Drago sut, il \emph{sut} que c'était vrai, c'était juste \emph{évident}. Qui Dumbledore croyait-il tromper~?

… enfin, à part tous les habitants de Poudlard à l'exception de Rogue et Quirrell, et \emph{Harry} le croyait peut-être…

Drago marcha jusqu'à son bureau, comme frappé de stupeur, et il s'assit assez vite pour que cela lui fasse légèrement mal. Ce genre de chose arrivait à peu près une fois par mois avec Harry, et ça n'avait pas encore eu lieu ce janvier, donc le temps était venu.

Son semblable Serpentard, qui se croyait ou ne se croyait pas être un Serdaigle, se rassit dans la chaise qu'il avait utilisée plus tôt, maintenant en travers, et regarda Drago, une expression aiguë sur le visage.

Drago ne savait pas ce qu'il était censé \emph{faire} à présent, s'il devait essayer de persuader le Serpentard perdu que non, il \emph{n'était pas} vraiment un Serdaigle… ou essayer de découvrir si Harry était secrètement allié à Dumbledore, mais cela sembla soudainement moins probable… mais alors \emph{pourquoi} Harry avait-il monté ce coup entre lui et Granger…

Il aurait \emph{vraiment} dû se souvenir que Harry était trop bizarre pour les complots normaux.

«Harry, dit Drago. t'es-tu délibérément opposé au général Soleil et à moi dans le seul but de nous faire travailler ensemble contre toi~?»

Harry hocha la tête sans hésiter, comme si c'était la chose la plus naturelle du monde et qu'il n'y avait pas à en avoir honte.

«Toute cette histoire avec les gants, nous faire grimper les murs de Poudlard, le \emph{seul but} était de faire que Granger et moi soyons plus amicaux l'un envers l'autre. Et même avant ça. Tu complotes ça depuis très longtemps. Depuis le \emph{début}.»

Encore le hochement de tête.

«POURQUOIIIIIIIIIII~?»

Les sourcils de Harry s'élevèrent un instant, seule réaction face à un Drago qui criait si fort dans la salle fermée qu'il se faisait mal aux oreilles. \scream{Pourquoi, pourquoi, pourquoi} Harry Potter \scream{faisait-il} ce genre de choses…

Puis Harry dit~:

«Pour que les Serpentard puissent à nouveau lancer le Patronus.

--- \emph{Ça… n'a… aucun… SENS~!}» Drago était conscient du fait qu'il commençait à perdre le contrôle de sa voix, mais il ne semblait pas être capable de s'arrêter. «\emph{Qu'est-ce que ça a à voir avec Granger~?}

--- Les motifs», dit Harry. Son visage était maintenant très sérieux, très grave. «Comme un quart d'enfants nés sorciers de parents Cracmols. Un motif simple, impossible à rater, que tu reconnaîtrais instantanément si tu savais quoi chercher~; alors que, étant ignorant de celui-ci, tu ne te rendrais même pas compte que c'est un indice. Le poison de la maison Serpentard a déjà été vu dans le monde Moldu. C'est une prédiction \emph{à l'avance}, Drago, j'aurais pu te l'écrire avant notre premier jour de cours, juste en t'entendant parler à la gare de King's Cross. Laisse-moi te décrire le genre de personnes vraiment pathétiques qui traînent aux rassemblements politiques de ton père, des familles Sang-Pur qui ne seraient jamais invitées au manoir Malfoy. Garde à l'esprit que \emph{je} ne les ai jamais rencontrés, je prédis juste leur existence en reconnaissant le motif de ce qui est en train d'arriver à la maison Serpentard…»

Et Harry Potter commença à décrire les Parkinson et les Montagues et les Boles d'une précision au calme tranchant avec laquelle Drago n'aurait même pas osé \emph{penser} au cas où il y aurait eu un Legilimens dans les environs, c'était \emph{au-delà} de l'insulte, ils \emph{tueraient} Harry si jamais ils l'apprenaient…

«Pour résumer, conclut Harry, ils n'ont aucun pouvoir eux-mêmes. Ils n'ont aucune fortune. S'ils n'avaient pas de Moldus à haïr, si tous les Moldus disparaissaient, comme ils disent le \emph{souhaiter}, ils se réveilleraient un matin et découvriraient qu'ils n'ont \emph{rien}. Mais tant qu'ils peuvent dire que les Sang-Pur sont supérieurs, ils peuvent se sentir eux-mêmes supérieurs, ils peuvent avoir l'impression qu'ils font partie de la classe dominante. Même si ton père ne songerait jamais à les inviter à dîner, même s'il n'y a pas le moindre Gallion dans leurs chambres fortes, même s'ils ont eu de pires scores à leur BUSE que le pire des nés-Moldus de Poudlard. Même s'ils ne peuvent plus jeter le Patronus. Pour eux, tout est de la faute des Moldus, il ont autre chose à blâmer que leurs propres échecs, et cela les rend encore plus faible. C'est ce que la maison Serpentard devient~: \emph{pathétique}. Et la racine du problème, c'est la haine des Moldus.

--- Salazar Serpentard lui-même a dit que les Moldus devaient être chassés~! Qu'ils affaiblissaient notre sang…» la voix de Drago était devenue un cri.

«\emph{Salazar avait tort, c'est un fait}~! Tu le \emph{sais}, Drago~! Et cette \emph{haine} empoisonne toute ta maison, tu ne pourrais pas lancer de Patronus avec une pensée comme celle-là~!

--- Alors pourquoi \emph{Salazar Serpentard} pouvait-il lancer le Patronus~?»

Harry essuyait de la sueur de son front.
«Parce que les choses ont \emph{changé} entre alors et maintenant~! Écoutes, Drago, il y a trois-cents ans, tu aurais pu trouver de grands scientifiques, aussi grands que Salazar, à leur façon, qui auraient dit que d'autres Moldus étaient inférieurs à cause de leur couleur de peau…

--- \emph{Couleur de peau~?} dit Drago.

---Je sais, la couleur de peau au lieu de quelque chose de vraiment important comme la pureté du sang, n'est-ce pas ridicule~? Mais alors, quelque chose dans le monde a changé, et \emph{aujourd'hui}, tu ne trouveras plus aucun grand scientifique pour dire que la couleur de peau devrait avoir de l'importance, seulement des rebuts comme ceux que je t'ai décrit plus tôt. Salazar Serpentard a fait cette erreur à l'époque où tout le monde la faisait, parce qu'il a grandi en le croyant, pas parce qu'il avait \emph{désespérément besoin de quelque chose à haïr}. Il y en a eu une poignée qui ont fait mieux que tous les autres, et \emph{ils} étaient exceptionnellement bons. Mais ceux qui se contentaient d'accepter ce que tout le monde pensait n'étaient pas \emph{exceptionnellement} méchants. La triste vérité, c'est que les gens ne remarquent tout simplement pas les problèmes moraux tant que quelqu'un ne le pointe pas du doigt~; et une fois qu'ils sont aussi âgés que Salazar l'était quand il a rencontré Godric, ils ont perdu leur capacité à changer d'avis. Ce n'est qu'\emph{alors} que Poudlard a été construite, et l'école a commencé à envoyer des lettres aux Moldus, car Godric avait insisté, et de plus en plus de gens commencèrent à remarquer que les Moldus \emph{n'étaient pas} différents. Maintenant c'est un gros problème politique au lieu d'être quelque chose que tout le monde croit sans même y penser. Et la \emph{bonne} réponse, c'est que les Moldus ne sont \emph{pas} plus faibles que les Sang-Pur. Alors \emph{maintenant}, ceux qui se retrouvent à militer pour ce que Salazar croyait autrefois, ce sont soit des gens qui ont grandi dans des environnements Sang-Pur très fermés, comme toi, \emph{soit} des gens qui sont eux-mêmes si pathétiques qu'ils ont désespérément besoin de se sentir supérieurs à d'autres~; des gens qui aiment haïr.

--- Ça ne… ça n'a pas l'air vrai…» dit la voix de Drago. Ses oreilles écoutaient, et il se demandait pourquoi il ne trouvait rien de mieux à répondre.

«Ah bon~? Drago, tu \emph{sais} maintenant qu'il n'y a rien qui cloche chez Hermione Granger. J'ai entendu dire que tu avais eu du mal à la laisser tomber du toit. Alors qu'elle avait bu une potion de Chute Plumée, alors que tu savais qu'elle était en sécurité. Quel genre de personne veut la \emph{tuer}, pas parce qu'elle a fait quelque chose de mal mais seulement parce qu'elle est née-Moldue~? Même si elle n'est… si elle n'est qu'une jeune fille qui les aiderait pour leurs devoirs sans hésiter, s'ils le lui demandaient,» la voix de Harry se brisa, «quel genre de personne veut la voir \emph{mourir}~?»

\emph{Père…}

Drago se sentit déchiré en deux morceaux, il avait l'impression que la vision binoculaire devenait difficile, \emph{Granger est une Sang-de-Bourbe, elle devrait mourir} et une fille suspendue à sa main sur le toit, comme de voir double, voir double…

«Et quelqu'un qui ne veut \emph{pas} la voir mourir ne veut pas traîner avec le genre de personne qui le \emph{veut}~! C'est \emph{tout} ce que les gens voient quand ils regardent Serpentard aujourd'hui, pas les plans intelligents, pas la quête de grandeur, juste la haine des Moldus~! J'ai payé une Mornille à Morag pour qu'elle demande à Padma pourquoi elle n'était pas allée à Serpentard, on sait tous les deux qu'elle a eu le choix. Et Morag m'a dit que Padma l'a juste \emph{regardée d'un drôle d'air} et qu'elle a dit qu'elle n'était pas Pansy Parkinson. Tu vois~? Les meilleurs élèves, ceux qui ont les vertus de plus d'une maison, les élèves qui ont le \emph{choix}, ils vont sous le Choixpeau en pensant \emph{n'importe où mais pas Serpentard}, et quelqu'un comme Padma se retrouve à Serdaigle. Et… je pense que le Choixpeau essaie de maintenir un équilibre dans la répartition, alors il remplit les rangs de Serpentard avec tous ceux qui ne \emph{sont pas} dégoûtés par toute cette haine. Alors au lieu de Padma Patil, Serpentard a Pansy Parkinson. Elle n'est pas très rusée, elle n'est pas très ambitieuse, mais elle correspond au genre de personne qui ne trouve rien à redire à ce que Serpentard est en train de devenir. Et plus de Padmas vont à Serdaigle et plus de Pansies vont à Serpentard plus le processus s'accélère. \emph{C'est en train de détruire Serpentard, Drago~!}»

Cela avait les échos d'une horrible vérité, Padma \emph{avait} appartenu à Serpentard… et Serpentard avait eu Pansy au lieu d'elle… Père se ralliait aux familles inférieures comme celles des Parkinson parce qu'ils étaient une source de soutien pratique, mais Père ne s'était pas rendu compte des \emph{conséquences} qu'il y avait à associer le nom de Serpentard au leur…

«Je ne peux pas…» dit Drago, mais il n'était même pas sûr de ce dont il était incapable -- «Qu'est-ce que tu \emph{veux} de moi~?

--- Je ne sais pas très bien comment soigner la maison Serpentard, dit lentement Harry. Mais je suis certain que c'est quelque chose que nous allons devoir faire, toi et moi. L'aube de la science a mit des siècles à se lever sur le monde Moldu, ça n'a eu lieu que lentement, mais plus la science devenait forte, plus ce genre de haine battait en retraite.» La voix de Harry était maintenant douce. «Je ne sais pas exactement pourquoi ça s'est fait ainsi, mais c'est comme ça que ça a eu lieu, historiquement. Comme s'il y avait quelque chose dans la science qui ressemblait au Patronus, qui repoussait toutes sortes de ténèbres et de folies, pas immédiatement, mais ça a semblé suivre la science partout où elle est allée. Les Lumières, c'est comme ça qu'on a appelé ça dans le monde Moldu. Ça a un rapport avec la quête de vérité, je pense… avec la capacité de changer ses croyances, de quitter celles que l'on avait en grandissant… avec la pensée \emph{logique}, avec la capacité de se rendre compte qu'il n'y a aucune \emph{raison} de haïr quelqu'un parce que sa couleur de peau est différente, tout comme il n'y a aucune raison de haïr Hermione Granger… ou peut-être qu'il y a là autre chose que même moi je ne comprends pas. Mais nous appartenons tous les deux aux Lumières maintenant. Réparer Serpentard n'est qu'une des choses que nous devons faire.

--- Laisse-moi réfléchir», dit Drago, sa voix semblable à un croassement, «s'il te plaît,» et il posa sa tête entre ses mains, et réfléchit.

\later

Drago réfléchit un long moment, les paumes de ses mains sur ses yeux pour masquer le monde, aucun son hormis celui de sa respiration et de celle de Harry. Tout le raisonnement persuasif de Harry, les parcelles de vérité évidentes qui s'y trouvaient~; et contre cela, l'évidence, l'hypothèse parfaite et totalement évidente quant à ce qui était en train de se passer \emph{réellement}…

Après un moment, Drago releva enfin la tête.

«Ça m'a l'air vrai», dit doucement Drago.

Un immense sourire apparut sur le visage de Harry.

«Alors, continua Drago, c'est maintenant que tu m'amènes à Dumbledore, pour officialiser la chose~?»

Il avait gardé un ton très nonchalant.

«Ah, ouais, dit Harry. C'est le truc sur lequel je voulais t'interroger, à vrai dire…»

Le sang de Drago se glaça dans ses veines, devint solide, se brisa…

«Le professeur Quirrell m'a dit quelque chose qui m'a fait réfléchir, et, eh bien, peu importe ta réponse, je suis déjà stupide de n'avoir pas pensé à te poser la question plus tôt. Tout le monde à Gryffondor pense que Dumbledore est un saint, les Poufsouffle pensent qu'il est fou, les Serdaigle sont très fiers d'avoir réussi à comprendre qu'il faisait seulement semblant d'être fou, mais je n'ai jamais interrogé de Serpentard. Je pense que j'étais censé en savoir assez pour ne pas commettre ce genre d'erreur. Mais même si \emph{tu} penses que Dumbledore ne voit aucun problème à l'idée d'une conspiration visant à réparer la maison Serpentard, j'imagine que je n'ai rien manqué d'important.»

…

…

…

«Tu sais», dit Drago, sa voix remarquablement calme, tout bien considéré, «à chaque fois que je me demande si tu fais ce genre de chose juste pour m'agacer, je me dis que ça \emph{doit} être accidentel, que \emph{personne} ne pourrait faire ce genre de chose exprès, même s'il essayait de toutes ses forces jusqu'à ce que du sang lui sorte par les oreilles. C'est la seule raison pour laquelle je ne vais pas t'étrangler.

--- Hein~?»

Puis s'étrangler \emph{lui-même}, parce que Harry \emph{avait} grandi chez les Moldus, puis Dumbledore l'avait écarté de Serpentard, vers Serdaigle, en douceur, et il était donc parfaitement plausible que Harry ne sache \emph{rien}, et Drago n'avait jamais pensé à lui \emph{en parler}.

Ou alors Harry avait deviné que Drago ne se joindrait pas à Dumbledore si facilement, auquel cas ceci était juste la prochaine étape du plan de Dumbledore…

Mais si Harry n'était \emph{vraiment} pas au courant pour Dumbledore, alors le prévenir était plus important que \emph{tout le reste}.

«Très bien», dit Drago après avoir pris le temps d'organiser ses pensées. «Je ne sais pas par où commencer, alors je vais juste commencer quelque part.» Il prit une profonde inspiration. «Dumbledore a tué sa petite sœur, et il s'en est tiré parce que son frère a refusé de témoigner contre lui…»

\later

Harry avait écouté avec une inquiétude et un désarroi croissants. Il pensait avoir été prêt, prêt à écouter la version puriste de l'histoire avec circonspection. Le problème était que même si vous l'écoutiez avec beaucoup de circonspection, c'était \emph{quand même} inquiétant.

Le père de Dumbledore avait été condamné pour usage d'un sortilège impardonnable sur ses enfants, et il était mort à Azkaban. Ce n'était pas un des péchés de Dumbledore, c'était inscrit dans un registre public. Harry pouvait vérifier cette partie et voir si tout avait été inventé par les Puristes du Sang.

La mère de Dumbledore était morte mystérieusement peu de temps avant que la sœur cadette de celui-ci ne meure, un meurtre, selon les Aurors. Sa sœur aurait été brutalisée par des Moldus et n'avait jamais reparlé après ce jour, et Drago remarqua que cela ressemblait remarquablement à un sortilège d'Oubliettes raté.

Après que Harry l'eut interrompu plusieurs fois, Drago avait semblé comprendre le principe, et il présentait maintenant les observations puis les inférences.

« -- donc tu n'as pas à me croire sur parole, dit Drago, tu peux le \emph{voir}, d'accord~? N'importe qui à Serpentard peut le voir. Dumbledore a repoussé son duel contre Grindelwald jusqu'au moment précis où cela le mettrait le plus en valeur, \emph{après} que celui-ci ait mit la majeure partie de l'Europe en ruines et se soit forgé la réputation d'être le mage noir le plus terrible de l'Histoire, juste au moment où il avait perdu l'or et les sacrifices sanguinaires de ses pions Moldus et qu'il commençait à décliner. Si Dumbledore avait vraiment été le noble sorcier qu'il prétendait être, il aurait combattu Grindelwald bien avant ça. Dumbledore \emph{voulait} probablement que l'Europe soit mise en ruines, ça faisait probablement partie de leur plan commun, il ne l'a attaqué qu'après que ses marionnettes l'ont \emph{laissé tomber}. Et ce grand duel de lumière n'était pas réel, il est impossible que deux sorciers soient si parfaitement semblables à l'autre qu'ils puissent se battre vingt heures d'affilée jusqu'à ce que l'un d'eux tombe d'épuisement, c'était juste une façon de rendre les choses plus spectaculaires.» La voix de Drago prit alors un ton plus indigné. «Et ça l'a rendu \emph{Chef du Magenmagot~!} La lignée de Merlin, interrompue, corrompue après mille-cinq-cents ans~! Et \emph{alors} il est devenu Grand Manitou Suprême, en plus du reste, et il avait \emph{déjà} Poudlard comme forteresse invincible -- Directeur \emph{et} Chef Sorcier \emph{et} Grand Manitou, une personne normale n'essaierait jamais de faire tout cela en même temps, \emph{comment peut-on ne pas voir que Dumbledore essaie de conquérir le monde~?}

--- Pause», dit Harry, et il ferma ses yeux pour réfléchir.

Ce n'était pas pire que ce qu'on aurait entendu au sujet de l'Ouest en Russie Stalinienne, et rien de tout cela n'aurait été vrai. Mais ici, les puristes du sang ne pouvaient pas se permettre de tout inventer… si~? La \emph{Gazette du Sorcier} avait montré une tendance prononcée à inventer toutes sortes de choses… mais, une fois encore, lorsqu'elle s'était trop avancée au sujet des fiançailles Weasley, on l'\emph{avait} pointée du doigt et elle \emph{avait} été gênée…

Harry ouvrit ses yeux et vit que Drago le regardait sans ciller, comme s'il attendait quelque chose.

«Donc quand tu m'as demandé s'il était temps de te joindre à Dumbledore, ce n'était qu'un test.»

Drago hocha la tête.

«Et avant ça, quand tu as dit que ce que je disais semblait vrai…

--- Ça \emph{semble} vrai, dit Drago. Mais je ne sais pas si je peux te faire confiance. Allez-vous vous plaindre parce que je vous ai \emph{testé}, M. Potter~? Allez-vous dire que je vous ai \emph{trompé}~? Que je vous ai \emph{induit en erreur}~?»

Harry savait qu'il aurait dû sourire en beau joueur, mais il n'y arrivait pas, la déception était trop forte.

«Tu as raison, c'est équitable, je ne peux pas me plaindre, préféra-t-il dire. Et qu'en est-il de Celui-Dont-On-Ne-Doit-Pas-Prononcer-Le-Nom~? Pas aussi méchant qu'on le dit~?»

En entendant cela, Drago eut l'air amer.

«Alors tu penses qu'il s'agit juste de donner le beau rôle au camp de Père et le mauvais à celui de Dumbledore, et que je crois à tout ça seulement parce que Père me l'a dit.

--- C'est une possibilité que je prends en compte», dit Harry d'une voix neutre.

La voix de Drago était basse et intense. «Ils \emph{savaient}. Mon père savait, ses amis savaient. Ils \emph{savaient} que le Seigneur des Ténèbres était maléfique. \emph{Mais c'était la seule chance que les sorciers avaient contre Dumbledore~!} Le seul sorcier au monde assez puissant pour le combattre~! Certains des Mangemorts étaient aussi vraiment maléfiques, comme Bellatrix Black -- Père n'est pas comme ça -- mais Père et ses amis \emph{devaient} le faire, Harry, ils le \emph{devaient}, Dumbledore était en train de tout conquérir, le Seigneur des Ténèbres était le dernier espoir des sorciers~!»

Drago regardait Harry durement. Harry croisa le regard, essaya de réfléchir. Personne ne se voyait comme le méchant de sa propre histoire -- peut-être Voldemort, peut-être Bellatrix, mais certainement pas Drago. La question n'était pas de savoir si les Mangemorts avaient été méchants. La question était de savoir s'ils avaient été \emph{les} méchants~; s'il y avait \emph{un} ou \emph{deux} méchants dans l'histoire…

«Tu n'es pas convaincu», dit Drago. Il semblait inquiet et légèrement en colère. Ce qui ne surprit pas Harry. Il était quasiment certain que Drago croyait à tout cela.

«\emph{Devrais-}je l'être~?» dit Harry. Il ne détourna pas le regard. «Juste parce que tu le crois~? Es-tu à présent un rationaliste assez puissant pour que ta croyance constitue une preuve d'importance à mes yeux, parce qu'il serait très improbable que tu y croies si ce n'était pas vrai~? Lorsque je t'ai rencontré, tu n'étais pas puissant à ce point. Tout ce que tu m'as dit, y as-tu de nouveau réfléchi suite à ton éveil à la science, ou est-ce seulement une chose à laquelle tu as toujours crue en grandissant~? Peux-tu me regarder dans les yeux et me jurer sur l'honneur de la maison Malfoy que s'il y avait une fausse vérité cachée quelque part dans ce que tu as dit, une chose qui a été ajoutée juste pour ternir un peu le portrait de Dumbledore, tu l'aurais remarqué~?»

Drago commença à ouvrir la bouche, et Harry dit~: «Non. N'entache pas l'honneur de la maison Malfoy. Tu n'es \emph{pas} encore assez puissant, et tu devrais le savoir. Écoutes, Drago, j'ai moi-même commencé à remarquer des choses inquiétantes. Mais il n'y a rien de \emph{définitif}, rien de \emph{certain}, ce sont seulement des déductions et des hypothèses et des témoins peu crédibles… Et il n'y a rien de certain dans ton histoire non plus. Dumbledore avait peut-être une bonne excuse pour expliquer le fait qu'il n'ait pas combattu Grindelwald des années plus tôt -- mais il \emph{vaudrait mieux} que ça ait été une sacrée bonne excuse, compte tenu de ce qui passait côté Moldu… mais quand même. Y a-t-il une chose clairement maléfique que Dumbledore a \emph{certainement} faite, pour que je n'aie plus à m'interroger~?»

La respiration de Drago était saccadée. «Très bien, dit Drago d'une voix chancelante, je vais te dire ce que Dumbledore a fait.» De la robe de Drago sortit une baguette, et Drago dit «Quietus», puis de nouveau~: «Quietus», mais il rata la prononciation une fois encore, et Harry finit par sortir sa propre baguette et par lancer le sort.

«Voilà, dit Drago d'une voix rauque, il était une fois une, une fille, elle s'appelait Narcissa, et elle était la plus jolie, la plus intelligente, la plus rusée à avoir jamais été Triée à Serpentard, et mon père l'aimait, et ils se sont mariés, et elle n'était pas un Mangemort, elle n'était pas une combattante, \emph{tout ce qu'elle a jamais fait, c'est aimer Père…}» Drago s'arrêta là, parce qu'il pleurait.

Harry se sentit malade. Drago n'avait jamais parlé de sa \emph{mère}, pas une seule fois, il aurait dû le remarquer plus tôt. «Elle… elle s'est retrouvée sur la trajectoire d'un sortilège~?»

La voix de Drago était un cri. «\emph{Dumbledore l'a brûlée vive dans sa chambre~!}»

\later

Dans une salle de classe nimbée d'une lumière d'argent, un garçon en regarde un autre. Ce dernier sanglote, essuyant frénétiquement ses yeux du revers de sa robe.

Harry avait du mal à maintenir un équilibre, à continuer de s'empêcher de juger, l'émotion était trop forte, quelque en chose en lui voulait soit se mettre à pleurer, par empathie pour Drago, soit \emph{savoir} que c'était faux…

«\emph{Dumbledore l'a brûlée vive dans sa chambre~!}»

Ça…

… ne ressemblait pas au style de Dumbledore…

… mais il y avait une limite au nombre de fois où l'on pouvait avoir cette pensée avant de commencer à s'interroger sur fiabilité du concept de “style”.

«Ça, ça a dû faire horriblement mal, dit Drago d'une voix tremblante, Père n'en parle jamais, on n'en parle jamais devant lui, mais M. Macnair m'a dit qu'il y avait des marques de griffes partout dans la chambre, parce que Mère avait lutté pendant que Dumbledore la \emph{brûlait vive}. C'est la dette que Dumbledore doit à la maison Malfoy et \emph{il le paiera de sa vie}~!

--- Drago,» dit Harry, il laissa sa voix devenir rauque, avoir l'air calme n'aurait pas été \emph{normal}, «je suis désolé, pardon de te le demander, mais je \emph{dois} savoir, \emph{comment} sais-tu que c'était Dumble-

--- Dumbledore a \emph{dit} qu'il l'avait fait, il a dit à Père que c'était un \emph{avertissement~!} Et Père n'a pas pu témoigner sous Veritaserum parce qu'il est un Occlumens, il n'a même pas pu faire de procès à Dumbledore, même les alliés de Père ne l'ont pas cru lorsque Dumbledore a tout nié en public, mais on le sait, les Mangemorts le savent, Père n'aurait aucune raison de mentir à ce sujet, Père voudrait que notre vengeance porte sur la \emph{bonne} personne, tu peux comprendre ça, Harry~?». Drago avait perdu tout contrôle sur sa voix.

\emph{À moins que Lucius ne l'ait fait lui-même, bien sûr, et qu'il n'ait trouvé plus pratique de blâmer Dumbledore.}

Quoique… ça ne ressemblait pas non plus au style de \emph{Lucius}. Et s'il \emph{avait} tué Narcissa, il aurait été plus intelligent de faire porter le chapeau à une victime plus faible plutôt que de perdre du capital politique et de la crédibilité en s'en prenant à Dumbledore…

Drago finit par s'arrêter de pleurer, et il regarda Harry. «\emph{Eh bien~?} dit Drago comme s'il avait voulu cracher les mots. Est-ce assez \emph{maléfique} pour vous, M. Potter~?»

Harry baissa les yeux et regarda ses bras, posés sur le dossier de la chaise. Il ne pouvait plus croiser les yeux de Drago, la douleur qui s'y trouvait était trop forte.

«Je ne m'attendais pas à entendre ça, dit doucement Harry. Je ne sais plus quoi penser.

--- Tu ne \emph{sais pas~?}» la voix de Drago devint un cri et il se leva brutalement…

«Je me suis souvenu du moment où le Seigneur des Ténèbres a tué mes parents, dit Harry. Quand je me suis retrouvé face au Détraqueur, la première fois, c'est ce dont je me suis souvenu, le pire souvenir. Même si c'était il y a très longtemps. Je les ai entendus mourir. Ma mère a supplié le Seigneur des Ténèbres de ne pas me tuer, \emph{pas Harry, s'il vous plaît, non, prenez-moi, tuez-moi à sa place~!} C'est ce qu'elle a dit. Et le Seigneur des Ténèbres s'est moqué d'elle, et il a ri. Ensuite, je me souviens de l'éclat de lumière verte…»

Harry releva les yeux vers Drago.

«Alors on pourrait se battre, dit Harry, on pourrait juste continuer ce combat. Tu pourrais me dire qu'il était juste que ma mère meure, parce qu'elle était la femme de James, qui avait tué un Mangemort. Mais que \emph{ta} mère n'aurait pas dû mourir, parce qu'\emph{elle} était innocente. Et je pourrais te dire qu'il était juste que ta mère meure, que Dumbledore devait avoir une \emph{raison} qui rendait \emph{acceptable} le fait qu'il la brûle vive dans sa chambre~; mais que \emph{ma} mère n'aurait pas dû mourir. Mais tu sais, Drago, dans un cas comme dans l'autre, n'est-il pas évident qu'on serait juste biaisés~? Parce que la règle dit qu'il ne faut pas tuer d'innocents, et cette règle ne peut pas s'activer pour ma mère et se désactiver pour la tienne, et elle ne peut pas s'activer pour ta mère et se désactiver pour la mienne. Si tu me dis que Lily était l'ennemie des Mangemorts et qu'il est juste de tuer ses ennemis, alors la même règle dit que Dumbledore avait raison de tuer Narcissa puisqu'elle était \emph{son} ennemie.» La voix de Harry redevint rauque. «Alors si nous devons être d'accord sur quelque chose, ça va être que la mort d'\emph{aucun} de nos parents n'était juste et qu'\emph{aucune} mère ne devrait plus avoir à mourir.»

\later

La furie qui bouillonnait en Drago était si intense qu'il parvenait à peine à s'empêcher de quitter la pièce~; seuls le retenaient la conscience que ce moment était critique, et un petit reste d'amitié, un léger éclair de sympathie, parce qu'il avait oublié, il avait \emph{oublié} que le père et la mère de Harry étaient morts aux mains du Seigneur des Ténèbres.

Le silence s'étira.

«Tu peux parler, dit Harry, Drago, parle-moi, je ne me mettrai pas en colère -- est-ce que tu penses, je ne sais pas, que la mort de Narcissa a été bien pire que celle de Lily~? Qu'il est inacceptable ne serait-ce que de les comparer l'une à l'autre~?

--- J'imagine que moi aussi, j'ai été stupide, dit Drago. Pendant tout ce temps, pendant tout ce temps j'ai oublié que tu devais haïr les Mangemorts, parce qu'ils ont tué tes parents, tu dois les haïr autant que je hais Dumbledore.» Et Harry n'avait jamais rien dit, il n'avait jamais réagi quand Drago avait parlé des Mangemorts, il l'avait gardé \emph{caché} -- Drago était un imbécile.

«Non, dit Harry. Ce n'est pas -- ce n'est pas ça Drago, je, je ne sais même pas comment te l'expliquer, à part en te disant qu'une pensée comme celle-ci ne», la voix de Harry s'étrangla, «tu ne pourrais jamais l'utiliser… pour lancer un Patronus…»

Drago sentit un tiraillement soudain dans son cœur, un tiraillement qu'il aurait souhaité ne pas ressentir.

«Prétends-tu que tu vas juste \emph{oublier} tes parents~? Est-ce que tu dis que je devrais \emph{oublier} Mère~?

--- Alors on \emph{doit} être ennemis~?» La voix de Harry devenait aussi incontrôlée que celle de Drago. «Que \emph{nous} sommes-nous faits \emph{l'un à l'autre} qui justifie le fait que nous devions être ennemis~? Je refuse d'être pris au piège comme ça~! “Justice” ne peut pas signifier que nous devrions \emph{tous les deux} attaquer \emph{l'autre}, ça n'a aucun sens~!» Harry s'interrompit, prit une profonde inspiration, fit courir ses doigts le long de sa tignasse étudiée -- les doigts émergèrent drapés de sueur, Drago put le voir. «Drago, écoutes, on ne peut pas espérer tomber d'accord sur tout dès le départ, toi et moi. Alors je ne te demanderai pas de dire que le Seigneur des Ténèbres a eu \emph{tort} de tuer ma mère, seulement de dire que c'est… \emph{triste}. On ne cherchera pas à décider si c'était \emph{nécessaire}, si c'était \emph{justifié}. Je te demanderai juste de dire qu'il est triste que cela ait eu lieu, que la vie de ma mère avait elle aussi de la valeur, tu vas juste dire ça pour l'instant. Et je dirai qu'il est triste que Narcissa soit morte, parce que sa vie aussi avait de la valeur. On ne peut pas s'attendre à tomber d'accord sur tout, tout de suite, mais si on commence par dire que toute vie est précieuse, que la mort de \emph{n'importe qui} est triste, alors je sais que nous nous retrouverons un jour. C'est ce que je veux que tu dises. Pas qui avait raison. Pas qui avait tort. Seulement que la mort de ta mère est triste, et que celle de la mienne l'est aussi, et que ce serait triste si Hermione Granger mourait, que chaque vie est précieuse, peut-on être d'accord à ce sujet et laisser tomber le reste pour l'instant, est-ce assez si on est d'accord seulement sur ça~? Peut-on le faire~? Ça ressemble… plus au genre de pensée que quelqu'un utiliserait pour lancer un Patronus.»

Il y avait des larmes dans les yeux de Harry.

Et Drago était de nouveau en colère. «Dumbledore a \emph{tué} ma mère, ça ne suffit pas de dire que c'était \emph{triste}~! Je ne comprends pas ce que tu penses que \emph{tu} dois faire, mais les Malfoy \emph{doivent} se venger~!» Ne pas venger une mort dans la famille allait \emph{au-delà} de la faiblesse, au-delà du déshonneur, vous auriez aussi bien pu n'avoir jamais \emph{existé}.

«Je ne débats pas ce point, dit doucement Harry. Mais diras-tu que la mort de Lily Potter était triste~? Juste dire ça~?

--- C'est…» Drago avait une fois de plus du mal à trouver ses mots. «Je sais, je sais ce que tu ressens, mais ne comprends-tu pas Harry, même si je dis juste que la mort de Lily Potter est \emph{triste}, ça va \emph{déjà} à l'encontre des Mangemorts~!

--- Drago, tu \emph{dois} être capable de dire que les Mangemorts avaient tort sur certains points~! Tu \emph{dois} en être capable, sinon tu ne pourras pas progresser en tant que scientifique, il y aura une barrière sur ton chemin, une autorité que tu ne pourras pas contredire. Tout changement n'est pas une amélioration, mais toute amélioration est un changement, tu ne peux pas \emph{mieux agir} à moins d'être capable d'agir \emph{différemment}, \emph{tu dois te laisser être meilleur que d'autres}~! Même que ton père, Drago, même que lui. Tu dois être capable de montrer quelque chose que ton père a fait et de dire qu'il a eu tort, parce qu'il n'est pas \emph{parfait}, et si tu ne peux pas dire ça, tu ne peux pas mieux agir.»

Père l'avait mis en garde, tous les soirs avant qu'il aille se coucher, pendant le mois précédant son départ pour Poudlard, que certains auraient ce but.

«Tu essaies de m'éloigner de Père.

--- D'éloigner une \emph{partie} de toi, dit Harry. J'essaie de te laisser réparer certaines choses au sujet desquelles ton père s'est trompé. J'essaie de te laisser \emph{faire mieux}. Mais je n'essaie pas de… briser ton \emph{Patronus~!}» La voix de Harry devint plus douce. «Je ne voudrais jamais briser une chose si lumineuse. Qui sait, peut-être que \emph{ça} aussi, c'est nécessaire à la réparation de Serpentard…»

Cela atteignait Drago, c'était ça le problème, en dépit de tout, cela l'atteignait, il fallait vraiment faire attention aux alentours de Harry parce que ses arguments étaient si convaincants \emph{même quand il avait tort}.

«Et ce que tu n'admets \emph{pas}, c'est que Dumbledore t'a dit que tu pouvais venger la mort de tes parents en prenant son fils à Lord Malfoy…

--- \emph{Non}. Non. C'est juste faux.» Harry prit une profonde inspiration. «Je ne savais pas qui était Dumbledore, ni qui était le Seigneur des Ténèbres, ni qui étaient les Mangemorts, ni comment mes parents étaient morts, pas jusqu'à trois jours avant mon arrivée à Poudlard. Le jour où nous nous sommes rencontrés dans le magasin de vêtements, c'est le jour où j'ai appris tout cela. Et Dumbledore \emph{n'aime} même pas la science Moldue, ou il dit qu'il ne l'aime pas, j'ai eu l'opportunité de le sonder à ce sujet un jour. L'idée de me venger des Mangemorts à travers toi ne m'a \emph{jamais} traversé l'esprit, pas \emph{une seule fois} avant aujourd'hui. Je ne savais pas qui étaient les Malfoy quand je t'ai rencontré dans le magasin de vêtements, et alors je t'ai \emph{apprécié}.»

Il y eut un long silence.

«J'aimerais pouvoir te faire confiance», dit Drago. Sa voix tremblait. «Si je pouvais juste \emph{savoir} que tu disais la vérité, tout serait tellement plus simple…»

Et soudain Drago la vit.

La façon de savoir si Harry Potter était vraiment sincère, quand il disait qu'il voulait réparer Serpentard, quand il disait qu'il était triste au sujet de la mort de Mère.

Ce serait illégal, et \emph{dangereux}, car il devrait le faire sans l'aide Père, il ne pourrait même pas compter sur Harry Potter pour \emph{l'aider}, mais…

«Très bien, dit Drago. J'ai conçu une expérience déterminante.

--- Qu'est-ce que c'est~?

--- Je veux que tu prennes une goutte de Veritaserum, dit Drago. Juste une goutte, pour que tu ne puisses pas mentir, mais pas assez pour te \emph{forcer} à répondre à tout. Je ne sais pas où je l'obtiendrai, mais je \emph{m'assurerai} que c'est sûr…

--- Euh», dit Harry. Il avait l'air désemparé. «Drago, euh…

--- Ne le dis pas», dit Drago. Sa voix était ferme et calme. «Si tu refuses, j'ai le résultat de mon expérience.

--- Drago, je suis un Occlumens…

--- OH C'EST TELLEMENT FAUX…

--- M. Bester m'a entraîné. Le professeur Quirrell a mis cela en place. Écoute, Drago, je \emph{prendrai} une goutte de Veritaserum si tu peux en obtenir, je te \emph{préviens} juste que je suis un Occlumens. Pas un Occlumens parfait, mais M. Bester a dit que j'arrivais à élever une barrière totale, et je pourrais probablement vaincre le Veritaserum.

--- \emph{Tu es en première année~! C'est de la folie~!}

--- Connais-tu un Legilimens en qui tu as confiance~? Je serais heureux de faire une démonstration -- écoute Drago, je suis désolé, mais le fait que je te le \emph{dise} ne compte-t-il pas un peu~? J'aurais \emph{pu} me contenter de te laisser faire l'expérience, tu sais.

--- \emph{\shout{Pourquoi~?} Pourquoi faut-il que tu sois toujours ainsi, Harry~? Pourquoi faut-il que tu chamboules tout même quand c'est \shout{IMPOSSIBLE~?} Et arrête de sourire, ça n'est pas drôle~!}

--- Je suis désolé, je suis désolé, je \emph{sais} que ce n'est pas drôle, je…»

Drago mit un moment à reprendre le contrôle de lui-même.

Mais Harry avait raison. Il aurait \emph{pu} laisser Drago lui administrer le Veritaserum. \emph{S'il} était vraiment un Occlumens… Drago ne savait pas à qui il pourrait demander d'essayer la Légilimancie, mais au moins il pouvait demander au professeur Quirrell si c'était vrai… pouvait-il faire confiance au \emph{professeur Quirrell~?} Peut-être que le professeur Quirrell dirait tout ce que Harry lui dirait de dire.

Drago se souvint de l'autre chose que Harry lui avait dit de demander au professeur Quirrell, et il pensa à un autre test.

«Tu \emph{sais}, dit Drago. Tu \emph{sais} ce que ça me coûte, si j'accepte l'idée que le poison de Serpentard est la haine des Moldus, si je dis que la mort de Lily Potter est triste. Et ça fait \emph{partie} \emph{de ton plan}, ne me dis pas le contraire.»

Harry ne dit rien, ce qui fut sage.

«Il y a quelque chose que je veux en retour, dit Drago. Et avant cela, un test expérimental que je veux essayer…»

\later

Drago poussa la porte vers laquelle les portraits l'avaient envoyé, et cette fois, c'était la bonne. Devant lui, un petit espace vide et pierreux, surplombé par le ciel nocturne. Pas un toit comme celui duquel il avait fait tomber Harry, mais une belle petite cour intérieure, loin au-dessus du sol. Avec de belles balustrades, des nervures de pierre élaborées qui se fondaient en douceur dans le sol de pierre… Le fait que les créateurs de Poudlard y aient insufflé tant de \emph{beauté} continuait d'émerveiller Drago à chaque fois qu'il y pensait. Il devait y avoir eu un moyen de tout faire d'un coup, personne n'aurait pu créer tant de détails, morceau par morceau, le château \emph{changeait} et tous les nouveaux morceaux étaient aussi beaux. Cela allait tant au-delà de la puissance magique faiblissante d'aujourd'hui que personne n'y aurait cru si Poudlard n'avait pas été là pour en témoigner.

Vide et froid, le ciel nocturne d'hiver~; en ces derniers jours de janvier, il s'assombrissait bien avant le couvre-feu des élèves.

Les étoiles brillaient à travers l'atmosphère transparente.

Harry avait dit que les étoiles l'aideraient.

De sa baguette, Drago toucha sa poitrine, fit glisser ses doigts d'un mouvement longuement pratiqué, et dit~: «\emph{Thermos}». Une chaleur se répandit en lui en partant de son cœur~; le vent vint souffler sur son visage, mais il n'avait plus froid.

«\emph{Thermos}», dit la voix de Harry, derrière-lui.

Ils allèrent jusqu'à la balustrade et regardèrent le sol, loin en dessous. Drago essaya de deviner s'ils étaient dans une des tours qui pouvaient être vues depuis l'extérieur et il découvrit qu'à cet instant précis, il ne pouvait pas tout à fait se représenter Poudlard de l'extérieur. Mais le sol était toujours le même, il pouvait voir la vague silhouette de la Forêt Interdite et la lumière de la Lune scintiller depuis le lac.

«Tu sais», dit Harry d'une voix basse, à côté de lui, ses bras appuyés sur la balustrade, «quelque chose que les Moldus font vraiment de travers, c'est qu'ils n'éteignent pas toutes leurs lumières la nuit. Pas même une heure par mois, pas même un quart d'heure une fois par an. Les photons se dispersent dans l'atmosphère et délavent toutes les étoiles, mis à part les plus brillantes, et le ciel nocturne ne ressemble pas du tout à cela, pas à moins de s'éloigner loin des villes. Après avoir observé le ciel au-dessus de Poudlard, c'est difficile de s'imaginer vivre dans une ville Moldue, où on ne pourrait pas voir les étoiles. Je ne voudrais certainement pas passer toute ma vie dans une ville Moldue après avoir vu le ciel nocturne de Poudlard.»

Drago jeta un coup d'œil à Harry et le vit tête renversée, regardant l'arc de la Voie lactée qui traversait l'obscurité.

«Bien sûr, continua Harry toujours à voix basse, on ne peut pas bien voir les étoiles depuis la \emph{Terre}, l'air obstrue le chemin. Il faut les regarder d'ailleurs si on veut les voir vraiment, les étoiles brûlantes et brillantes, telles qu'elles sont réellement. As-tu jamais souhaité pouvoir filer vers le ciel, Drago, et aller voir ce qui se trouve autour de Soleils autres que le nôtre~? Si la magie n'avait aucune limite, est-ce une des choses que tu ferais~? Si tu pouvais faire n'importe quoi~?»

Il y eut un silence, puis Drago se rendit compte qu'on attendait une réponse de lui.

«Je n'y ai jamais pensé», dit Drago. Sans qu'il l'ait consciemment décidé, sa voix avait été aussi douce et étouffée que celle de Harry. «Tu penses vraiment que quelqu'un sera un jour capable de le faire~?

--- Je ne pense pas que ça sera facile, dit Harry. Mais je sais que je ne compte pas passer toute ma vie sur Terre.»

Drago se serait moqué de cette phrase s'il n'avait pas su que certains Moldus étaient déjà partis, sans même utiliser de magie.

«Pour passer ton test, dit Harry, je vais devoir te dire ce que ça veut dire pour \emph{moi}, cette pensée, dans son intégralité, pas la version courte que j'ai essayé de t'expliquer plus tôt. Mais tu devrais pouvoir voir que c'est la même idée, mais plus générale. Alors \emph{ma} version de la pensée, Drago, c'est que lorsqu'on ira dans les étoiles, on trouvera peut-être des gens là-bas. Et dans ce cas, ils ne nous ressembleront certainement pas. Il y aura peut-être des choses faites à base de cristaux, ou des grandes sphères pulsatives… ou ils seront peut-être faits de magie, maintenant que j'y pense. Alors avec toute cette étrangeté, comment reconnaîtras-tu une \emph{personne}~? Pas par sa forme, pas par le nombre de ses bras ou de ses jambes. Pas par la substance qui la compose, que ce soit de la chair ou du cristal ou quelque chose que je ne peux pas imaginer. Et même alors, leurs esprits ne fonctionneraient pas exactement comme les nôtres. Mais pour toute chose qui vit et pense et se connaît elle-même et ne veut pas mourir, Drago, c'est triste, c'est triste que cette personne doive mourir, parce qu'elle ne le désire pas. Comparé à ce qui existe peut-être ailleurs, tous les humains à avoir jamais vécu sont comme des frères et sœurs, on pourrait à peine nous distinguer. Ceux venus d'ailleurs qui nous rencontreraient, ils ne verraient pas des Français ou des Anglais, ils ne pourraient pas faire la différence~; ils verraient juste un humain. Un humain qui peut aimer, et haïr, et rire, et pleurer~; et pour \emph{eux}, pour ceux venus d'ailleurs, nous nous ressemblerions comme des gouttes d'eau. \emph{Eux}, en revanche, ils seraient différents. \emph{Vraiment} différents. Mais si nous voulions tous devenir amis, cela ne nous arrêterait pas, et cela ne les arrêterait pas non plus.»

Harry leva sa baguette, et Drago se détourna, regarda ailleurs, comme il l'avait promis~; il regarda vers le sol de pierre et le mur de pierre dans lequel la porte était sertie. Car il avait promis de ne pas regarder, de ne jamais parler de ce que Harry avait dit, ou de ce qui avait eu lieu cette nuit, même s'il ne savait pas pourquoi ce devait être un secret.

«J'ai fait le rêve, dit la voix de Harry, qu'un jour, tous les êtres sentients\footnotemark{} seront jugés par les motifs de leur esprit et pas par la couleur ou la forme de la matière dont ils sont faits, ou par l'identité de leurs parents. Parce que si on peut un jour s'entendre avec des choses faites de cristal, quelle bêtise ce serait que de ne pas s'entendre avec des nés-Moldus, qui ont la même forme que nous, qui pensent comme nous, qui nous ressemblent comme deux gouttes d'eau~? Les choses de cristal ne pourraient même pas nous différencier. N'est-il pas impossible d'imaginer que le poison de haine de Serpentard mérite qu'on l'emmène avec nous dans les étoiles~? Toute vie est précieuse, tout ce qui pense et se connaît et ne veut pas mourir. La vie de Lily Potter était précieuse, et la vie de Narcissa Malfoy était précieuse, même s'il est trop tard pour elles maintenant, leur mort fut triste. Mais il y a d'autres vies qui sont toujours là, pour lesquelles se battre. Ta vie, et ma vie, et la vie de Hermione Granger, et toutes les vies de la Terre, et les vies au-delà, qui doivent être défendues et protégées, \emph{EXPECTO PATRONUM~!}»
\translatorsnotetext{J'utilise le mot anglais sentient à la façon de Guy Abadia dans sa traduction de \emph{L'Étoile et le Fouet} (\emph{Whipping Star} de Frank Herbert).}

Et la lumière fut.

Tout se transforma en argent, tout fut baigné dans cette lumière, le sol et les murs de pierre, la porte, la balustrade, si flamboyants dans ce simple reflet qu'on pouvait à peine les voir, même l'air semblait briller, et la lumière s'intensifia, encore, et encore…

Lorsque la lumière disparut, ce fut comme un choc soudain, la main de Drago alla automatiquement dans sa robe à la recherche d'un mouchoir, et ce n'est qu'alors qu'il se rendit compte qu'il pleurait.

«Voilà ton résultat expérimental, dit la voix basse de Harry. Cette pensée était sincère.»

Drago se tourna lentement vers Harry, qui avait sa baguette baissée.

«Il, il doit y avoir un truc, c'est ça~?» dit Drago. Il n'arrivait plus à encaisser ces chocs. «Ton Patronus -- ne peut pas \emph{vraiment} être si puissant…» Et pourtant ça \emph{avait} été de la lumière de Patronus, une fois qu'on savait ce qu'on regardait, il était impossible de la confondre avec autre chose.

«C'était la \emph{véritable} forme du Patronus, dit Harry. Elle te laisse mettre toute ta force dans le Patronus, sans aucune retenue. Et avant que tu ne me demandes~: Dumbledore ne me l'a pas apprise. Il n'en connaît pas le secret, et même s'il le savait, il ne pourrait lancer la véritable forme. J'ai résolu ce puzzle moi-même. Et j'ai su, après l'avoir résolu, que ce sort ne doit pas être révélé. Pour ton bien, je me suis soumis à ton test~; mais Drago, tu ne dois pas en parler.»

Drago ne savait plus, il ne savait plus où se trouvait la véritable force, ni ce qui était juste. Il voyait double, il voyait double. Il voulait dire des idéaux de Harry qu'ils étaient faiblesse, bêtise Poufsouffle, le genre de mensonge que les dirigeants racontaient pour apaiser la populace et que Harry avait été assez idiot pour croire, une bêtise prise au sérieux et élevée à une hauteur insensée, projetée jusqu'aux étoiles elles-mêmes…

Quelque chose de beau et caché, mystérieux et étincelant…

«Est-ce que, chuchota Drago, est-ce que je pourrai un jour lancer un Patronus comme celui-ci~?

--- Si tu recherches toujours la vérité, dit Harry, et si tu ne refuses pas les pensées chaleureuses lorsque tu les trouves, alors je suis sûr que oui. Je pense que quelqu'un peut aller n'importe où s'il persévère assez longtemps, même jusqu'aux étoiles.»

De son mouchoir, Drago s'essuya les yeux une fois de plus.

«On devrait retourner à l'intérieur, dit Drago d'une voix instable, quelqu'un aurait pu la voir, toute cette lumière…»

Harry hocha la tête et alla jusqu'à la porte, qu'il franchit~; et avant de le suivre, Drago leva une dernière fois ses yeux vers le ciel nocturne.

Qui \emph{était} le Survivant, qui était déjà un Occlumens, qui pouvait lancer la véritable forme du Patronus, et faire d'autres choses étranges~? Qu'était le Patronus de Harry, dont la forme devait rester cachée~?

Drago ne posa aucune de ces questions, parce que Harry aurait pu y \emph{répondre}, et Drago ne pouvait simplement pas endurer d'autre choc aujourd'hui. Il n'en \emph{pouvait plus}. Un de plus et sa tête allait juste tomber de ses épaules et rebondir, rebondir, rebondir le long des couloirs de Poudlard.

\later

Ils s'étaient agenouillés dans une petite alcôve au lieu de parcourir de nouveau le chemin qui menait à la salle de classe. Drago l'avait voulu~; il s'était senti trop nerveux pour repousser le moment plus longtemps.

Il éleva une barrière de silence et regarda Harry, l'interrogeant silencieusement.

«J'y ai réfléchi, dit Harry. Je vais le faire, mais il y a cinq conditions…

--- \emph{Cinq~?}

--- Oui, cinq. Drago, faire un engagement comme celui-ci, c'est presque \emph{supplier} les choses de mal tourner, tu \emph{sais} que ça tournerait mal si c'était une pièce…

--- Ça n'en est pas une~! dit Drago. Dumbledore a tué Mère. Il est maléfique. C'est une de ces choses dont tu parles, une de celles qui n'ont \emph{pas} à être compliquées.

--- Drago, dit Harry d'une voix précautionneuse, tout ce que je \emph{sais} c'est que \emph{tu} as dit que \emph{Lucius} a dit que \emph{Dumbledore} a dit qu'il a tué Narcissa. Pour y croire de façon inconditionnelle, je dois te faire confiance à toi \emph{et} à Lucius \emph{et} à Dumbledore. Donc, comme je l'ai dit, il y a des conditions. La première est que \emph{tu} peux me libérer de cette promesse n'importe quand, si l'idée cesse de te sembler bonne. Bien sûr, ça devra être un acte délibéré de ta part, pas un jeu sur les mots ou quelque chose comme ça.

--- D'accord», dit Drago. Ça semblait assez sûr.

«La deuxième condition, c'est que je promets de prendre pour ennemi la personne qui a vraiment tué Narcissa, comme je l'aurais déterminé en usant au mieux de mes capacités de rationaliste. Que ce soit Dumbledore ou quelqu'un d'autre. Et tu as ma parole que je ferai mon maximum en tant que rationaliste pour garder mon jugement honnête sur des questions de fait. D'accord~?

--- Je n'aime pas ça», dit Drago. Bien sûr qu'il n'aimait pas ça, tout le but de la manœuvre était que Harry ne rejoigne jamais le camp de Dumbledore. Malgré cela, si Harry \emph{était} honnête, il se ferait son opinion sur Dumbledore bien assez vite~; et s'il ne l'était pas, alors il avait déjà trahi sa promesse… «Mais j'accepte.

--- La troisième condition est que Narcissa doit avoir été \emph{brûlée vive}. Si cette partie de l'histoire se révèle avoir été exagérée juste pour noircir le tableau, alors je serai libre de décider si je veux ou non tenir cette promesse. Les gens bons doivent parfois tuer. Mais ils ne torturent jamais à tort. C'est parce que Narcissa a été \emph{brûlée vive} que je sais que celui qui a fait ça est maléfique.»

Drago parvint à peine à se contrôler.

«La quatrième condition est que si Narcissa s'est salie les mains, et qu'elle a, disons, rendu un enfant fou par \emph{Endoloris}, et que cette personne a brûlé Narcissa pour se venger, notre contrat pourrait de nouveau être rompu. Parce que dans ce cas, il serait toujours mal de l'avoir brûlée, il aurait seulement fallu la tuer sans douleur~; mais ce ne serait pas \emph{maléfique} au sens où ce le serait si elle avait seulement été l'amour de Lucius et qu'elle n'avait jamais rien fait, comme tu me le dis. La cinquième condition est que si la personne qui a tué Narcissa a été manipulée, alors mon ennemi est la personne qui l'a manipulée, pas la personne qui a exécuté le meurtre.

--- Tout ça sonne \emph{vraiment} comme si tu essayais de te faufiler hors des mailles du filet…

--- Drago, je ne me ferai pas ennemi de quelqu'un de bien, ni pour toi ni pour personne. Il faut que je sois vraiment persuadé que la personne a fait quelque chose mal. Mais j'y ai pensé, et il me semble que si Narcissa n'a jamais rien fait de mal, qu'elle est simplement tombée amoureuse de Lucius et qu'elle a décidé de devenir sa femme, alors celui qui l'a brûlée vive dans sa chambre a peu de chances d'être quelqu'un de bien. Et je promets de devenir l'ennemi de cette personne, que ce soit Dumbledore ou quelqu'un d'autre, à moins que tu ne me libères délibérément de cette promesse. En espérant que \emph{cela} ne tourne pas mal comme ça le ferait si on était dans une pièce de théâtre.

--- Je ne suis pas satisfait, dit Drago. Mais d'accord. Tu fais le serment d'être l'ennemi de celui qui a tué ma mère, et je…»

Harry attendit d'un air patient, alors que Drago essayait de refaire fonctionner sa voix.

«Je t'aiderai à réparer le problème de Serpentard avec la haine des nés-Moldus, finit-il d'un soupir. Et je dirai que la mort de Lily Potter était triste.

--- Qu'il en soit ainsi», dit Harry.

Et l'accord fut conclu.

La faille, Drago le savait, la faille venait de s'élargir un peu plus. Non, pas un peu, \emph{beaucoup}. Il avait la sensation de dériver, d'être perdu, toujours plus loin de la rive, toujours plus loin de chez lui…

«Excuse-moi», dit Drago. Il se détourna de Harry, puis il essaya de se calmer, il fallait qu'il fasse ce test, et il ne voulait pas l'échouer à cause de la nervosité ou de la honte.

Il leva sa baguette en position de départ pour un Patronus.

Il se souvint tomber de son balai, la douleur, la peur, il imagina une grande silhouette encapuchonnée qui s'approchait, qui ressemblait à une chose morte qu'on aurait laissée dans l'eau.

Puis il ferma ses yeux, pour mieux se souvenir de Père tenant ses petites mains froides dans sa force chaude.

\emph{N'aie pas peur, mon fils, je suis là…}

La baguette fut brandie pour repousser la peur, et Drago fut surpris par la force du geste~; et il se souvint alors que \emph{Père} n'était pas perdu, qu'il ne le serait jamais, qu'il serait toujours là et qu'il serait toujours fort, peu importe ce qui arriverait à Drago, et sa voix s'exclama~: «\emph{Expecto Patronum~!}»

Drago ouvrit les yeux.

Un serpent étincelant lui rendit son regard, pas moins lumineux qu'avant.

Derrière lui, il entendit \emph{Harry} exhaler, comme s'il était soulagé.

Drago regarda la lumière blanche. Il semblait qu'il n'était pas totalement perdu, après tout.

«Ce qui me rappelle, dit Harry après un moment. Peut-on tester mon hypothèse sur la façon d'utiliser le Patronus pour envoyer des messages~?

--- Est-ce que ça va me surprendre~? dit Drago. Je ne veux plus de surprises pour aujourd'hui.»

\later

Harry avait prétendu que l'idée n'était pas si bizarre que ça et qu'il ne voyait pas en quoi elle pourrait choquer Drago, ce qui n'avait eu pour effet que de rendre ce dernier encore plus nerveux~; mais Drago pouvait comprendre à quel point il était important de pouvoir s'envoyer des messages en cas d'urgence.

Le truc -- c'était du moins l'hypothèse de Harry -- consistait à vouloir partager la bonne nouvelle, à vouloir que le destinataire apprenne la pensée heureuse que vous aviez utilisée pour lancer le Patronus. Sauf qu'au lieu de communiquer avec des mots, le Patronus lui-même était le message. En voulant que le destinataire voit le Patronus, ce dernier irait le rejoindre.

«Dis à Harry», dit Drago au serpent lumineux, même si Harry se tenait à quelques pas de lui, de l'autre côté de la pièce, «de, euh, faire attention au singe vert», ce qui était un code dans une pièce qu'il avait un jour vue.

Et alors, comme à la gare de King's Cross, Drago voulut que Harry sache à quel point Père avait toujours tenu à lui~; sauf que cette fois il n'essaya de le dire par des mots mais par sa pensée heureuse.

Le serpent brillant ondula à travers la pièce, donnant l'impression qu'il ondulait dans les airs plutôt que sur la pierre~; il parcourut une courte distance, parvint à Harry…

… et lui dit, d'une étrange voix que Drago reconnut comme étant probablement celle que les gens entendaient quand il parlait~:

«Attention au singe vert.

--- \emph{Hsssss ssss sshsshssss}», dit Harry.

Le serpent ondula en sens inverse jusqu'à Drago.

«Harry dit que le message est bien reçu, dit le Bungarus Candidus étincelant avec la voix de Drago.

--- Euh, dit Harry. Ça fait bizarre de parler à un Patronus.»

…

…

…

…

«Pourquoi est-ce que tu me regardes comme ça~?» dit l'héritier de Serpentard.

\later

\emph{Après-coup~:}

Harry regardait Drago.

«Tu veux dire seulement les serpents \emph{magiques}, c'est ça~?

--- N-non», dit Drago. Il avait l'air assez pâle, et il bégayait encore, mais au moins il avait arrêté de produire des sons incohérents. «Tu peux parler Fourchelangue, c'est le langage de tous les serpents. Tu peux comprendre n'importe quel serpent quand il parle, et ils peuvent comprendre ce que tu leur dis… Harry, tu ne peux \emph{pas} croire que tu as vraiment été Trié à Serdaigle~! \emph{Tu es l'héritier de Serpentard~!}»

…

…

…

…

…

«\shout{Les serpents sont sentients~?}»
%  LocalWords:  Foul’s Dulak Abraxas WHYYYYY Dumble ssss
%  LocalWords:  Hsssss sshsshssss
