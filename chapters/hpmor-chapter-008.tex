\chapter{Biais positif}

\lettrine{P}{ersonne} n'avait demandé de l'aide, c'était ça le problème. Ils étaient juste restés là à parler, manger, ou regarder en l'air, pendant que leurs parents papotaient. Pour quelque étrange raison que ce soit, personne ne s'était assit pour lire un livre, ce qui voulait dire qu'elle ne pouvait pas juste s'asseoir à côté d'eux et sortir le sien. Et même après qu'elle avait pris l'audacieuse initiative de s'asseoir et de continuer sa troisième lecture de \emph{Poudlard~: Son Histoire}, personne n'avait semblé enclin à venir s'asseoir à côté d'elle.

À part les aider à faire leurs devoirs ou autre chose, elle ne savait pas vraiment comment entrer en contact avec les gens. Elle ne se \emph{trouvait} pas particulièrement timide. Elle se voyait comme le genre de fille qui prend les choses en main. Et pourtant, malgré cela, sans une requête de type <<~Je n'arrive à me souvenir de comment on pose une division~>>, c'était simplement trop \emph{embarrassant} d'aller voir quelqu'un et de lui dire… de lui dire quoi~? Elle n'avait jamais su quoi. Et aucune fiche d'information standard à ce sujet ne semblait exister, ce qui était ridicule. Toutes ces histoires au sujet de rencontrer les gens ne lui avaient jamais parues très sensées. Pourquoi était-ce à \emph{elle} d'endosser toute la responsabilité, alors qu'il y avait deux personnes impliquées~? Pourquoi les adultes ne l'aidaient-ils jamais~? Elle aurait aimé qu'une autre fille se présente à \emph{elle} et lui dise~: <<~Hermione, le professeur m'a dit être ton amie.~>>

Mais soyons clairs~: Hermione Granger, assise seule au premier jour d'école dans un des quelques compartiments laissés vides, dans la dernière voiture du train, avec la porte du compartiment ouverte juste au cas où quelqu'un aurait une raison de vouloir lui parler, n'était \emph{pas} triste, solitaire, mélancolique, déprimée, désespérée, ou obsédée par ses problèmes. Non, elle lisait \emph{Poudlard~: Son Histoire} pour la troisième fois et trouvait cela plutôt agréable, avec seulement en arrière-pensée une légère nuance de contrariété envers la déraison générale du monde.

Il y eut le son d'une porte inter-wagon qui s'ouvrait, puis des pas et un étrange bruit de glissement venant du couloir. Hermione mit \emph{Poudlard~: Son Histoire} de côté, se leva, pencha sa tête dans le couloir — juste au cas où quelqu'un aurait besoin d'aide — et elle vit un jeune garçon habillé d’une robe de sorcier, probablement en première ou deuxième année vu sa taille, à l'air passablement idiot avec une écharpe enroulée autour de sa tête. Une petite malle se tenait à ses côtés. Alors même qu'elle l'apercevait, il frappa à la porte d'un autre compartiment, fermé, et dit d'une voix légèrement étouffée par l'écharpe~: <<~Excusez-moi, pourrais-je vous poser une question rapide~?~>>

Elle n'entendit pas la réponse venue de l'intérieur du compartiment, mais après que le garçon eut ouvert la porte, elle crut l'entendre dire — à moins qu'elle n'ait mal compris — <<~Quelqu'un ici connaît-il les six quarks, ou bien l'endroit où je pourrais trouver une fille de première année nommée Hermione Granger~?~>>

Après que le garçon eut fermé la porte du compartiment, Hermione dit~: <<~Je peux t'aider~?~>>

Le visage voilé se tourna vers elle, et la voix répondit~:

<<~Pas à moins que tu puisses nommer les six quarks ou me dire où je peux trouver une fille de première année nommée Hermione Granger.

--- Haut, bas, étrange, charme, vérité, beauté, et pourquoi cherches-tu une fille de première année nommée Hermione Granger~?~>>

C'était dur à dire à cette distance, mais elle pensa voir le garçon faire un grand sourire sous son écharpe. <<~Ah, donc \emph{tu} es une fille de première année nommée Hermione Granger, dit la jeune voix étouffée. Sur le train de Poudlard, rien de moins.~>> Le garçon commença à marcher vers elle et son compartiment, et sa malle serpenta derrière lui. <<~Techniquement parlant, je suis seulement censé te \emph{chercher}, mais il semble probable que je sois aussi censé te parler ou t'inviter à rejoindre mon groupe ou obtenir de toi un objet magique important ou découvrir que Poudlard a été construite sur les ruines d'un ancien temple ou quelque chose comme ça. PJ ou PNJ, là est la question~?~>>

Hermione ouvrit la bouche pour répondre, mais elle ne pouvait pas imaginer quelle réponse il était \emph{possible} de donner à… \emph{quoi que ce soit} ce qu'elle venait d'entendre, et pendant ce temps le garçon marcha jusqu'à elle, regarda à l'intérieur du compartiment, hocha la tête d'un air satisfait, et s'assit sur le banc vide en face du sien, sur lequel le livre de Hermione se trouvait toujours. Sa malle se précipita à sa suite, crût de trois fois son diamètre initial et se blottit contre celle de Hermione d'une façon étrangement troublante.

<<~Assieds-toi, je t'en prie, dit le garçon, et ferme la porte derrière toi, si tu veux bien. Ne t'en fais pas, \emph{je} ne mords personne qui ne me mord pas avant.~>> Il enlevait déjà l'écharpe d'autour de sa tête.

L'imputation que ce garçon pensait qu'elle avait \emph{peur} de lui suffit à lui faire fermer la porte dans un glissement violent, l'écrasant contre le mur avec une force inutile. Elle fit un demi-tour et vit un jeune visage doté d'yeux vert brillant et rieurs, ainsi qu'une cicatrice rouge-noire à l'air colérique gravée sur son front, ce qui lui remit quelque chose à l'esprit, mais pour le moment elle avait à se soucier de choses plus importantes.

<<~Je n'ai pas dit que j'étais Hermione Granger~!

--- \emph{Je} n'ai pas dit que tu avais \emph{dis} que tu étais Hermione Granger, j'ai juste dit que tu étais Hermione Granger. Si tu veux savoir comment je le sais, c'est parce que je sais tout. Bonsoir mesdames et messieurs, mon nom est Harry James Potter-Evans-Verres ou Harry Potter pour faire plus court, je sais que pour une fois ça ne \emph{te} dira rien…~>>

Le cerveau de Hermione fit enfin le rapprochement. La cicatrice sur son front, la forme en éclair. <<~Harry Potter~! Tu es dans \emph{Histoire Magique Moderne} et \emph{La montée et le déclin des Arts Sombres} et \emph{Grands Évènements Magiques du Vingtième siècle}.~>> C'était à vrai dire la première fois de toute sa vie qu'elle \emph{rencontrait} quelqu'un se trouvant dans un \emph{livre}, et c'était une sensation plutôt bizarre.

Le garçon cligna trois fois des yeux.

<<~Je suis dans des \emph{livres}~? Attends, bien sûr que je suis dans des livres… quelle étrange pensée.

--- Allons, tu ne le savais pas~? dit Hermione. J'aurais appris tout ce que je pouvais à mon sujet si ça avait été moi.~>>

Le garçon parla assez sèchement. <<~Mademoiselle Hermione Granger, il s'est écoulé moins de 72 heures depuis que je suis allé au Chemin de Traverse et ai découvert ce pourquoi j'étais renommé. J'ai passé les deux derniers jours à acheter des livres scientifiques. \emph{Crois-moi}, je compte apprendre tout ce que je peux à ce sujet.~>> Le garçon hésita. <<~\emph{Que} disent les livres à mon sujet~?~>>

L'esprit de Hermione Granger fit un retour dans le temps, elle n'avait pas envisagé d'être testée sur \emph{ces} livres et ne les avait donc lus qu'une seule fois, mais c'était il y a seulement un mois donc le contenu était encore frais. <<~Tu es le seul à avoir survécu au Sortilège de la Mort et tu es donc appelé le Survivant. Tu es l'enfant de James Potter et Lily Potter anciennement Lily Evans, né le 31 juillet 1980. Le 31 octobre 1981 le Seigneur des Ténèbres Celui-Dont-On-Ne-Doit-Pas-Prononcer-Le-Nom a attaqué ta maison bien que je ne sache pas pourquoi, dont l'emplacement avait été révélé par Sirius Black bien qu'ils n'aient pas précisé comment ils savaient que c'était lui. Tu as été retrouvé en vie avec la cicatrice sur ton front dans les ruines de la maison de tes parents non loin des restes calcinés du corps de Tu-Sais-Qui. Le président du Magenmagot Albus Percival Wulfric Brian Dumbledore t'a envoyé quelque part, personne ne sait où. \emph{La montée et le déclin des Arts Sombres} prétend que tu as survécu grâce à l'amour de ta mère et que ta cicatrice contient tous les pouvoirs du Seigneur des Ténèbres et que les centaures ont peur de toi, mais \emph{Grands Événements Magiques du Vingtième siècle} ne mentionne rien de tel et \emph{Histoire Magique Moderne} prévient qu'il existe beaucoup de théories cinglées à ton sujet.~>>

La bouche du garçon pendait béate.

<<~T'a-t-on dit d'attendre Harry Potter dans le train vers Poudlard, ou quelque chose dans le genre~?

--- Non, dit Hermione. Qui t'a parlé de \emph{moi}~?

--- Le Professeur McGonagall, et je crois que je comprends pourquoi. Hermione, as-tu une mémoire eidétique~?~>>

Hermione secoua la tête.

<<~Elle n'est pas photographique, j'ai toujours rêvé qu'elle le soit, mais j'ai dû lire mes manuels cinq fois avant de les avoir tous mémorisés.

--- Vraiment, dit le garçon d'une voix subtilement étranglée. J'espère que ça ne te dérange pas si je teste ça~— ce n'est pas que je ne te crois pas, mais comme on dit~: “Fais confiance, mais vérifie”. Pas la peine que je m'interroge quand je peux juste faire une expérience.~>>

Hermione sourit, l'air plutôt contente d'elle-même. Elle adorait tellement les tests. <<~Vas-y.~>>

Le garçon mit une main dans la bourse qu'il portait au côté et dit <<~Dosages et Potions Magiques par Arsenius Jigger.~>> Lorsqu'il retira sa main, il tenait le livre qu'il avait nommé.

Instantanément, Hermione voulut posséder l'une de ces bourses plus qu'elle n'avait jamais désiré autre chose.

Le garçon ouvrit le livre quelque part au milieu et lut.

<<~Si tu faisais de \emph{l'huile de perspicacité}…

--- Je peux \emph{voir} la page d'ici, tu sais~!~>>

Le garçon inclina le livre pour qu'elle ne puisse plus voir, et tourna à nouveau les pages.

<<~Si tu préparais une \emph{potion d'escalade d'araignée}, quel serait l'ingrédient à ajouter après la soie d'Acromantula~?

--- Après avoir versé la soie, attendre jusqu'à ce que la potion ait pris exactement la teinte du ciel d'aube sans nuage à 8 degrés de l'horizon et 8 minutes avant que le haut du soleil ne devienne visible. Tourner huit fois dans le sens contraire des aiguilles d'une montre et une fois dans le sens horaire, puis ajouter huit gouttes de crottes de nez de licorne.~>>

Le garçon referma le livre d'un bruit sec et le remit dans sa bourse, qui l'avala avec un petit bruit de rot.

<<~Bien bien bien \emph{bien} bien bien. Je voudrais vous faire une proposition, Mademoiselle Granger.

--- Une proposition~?~>> dit Hermione avec méfiance. Les filles ne devaient pas écouter ce genre de choses.

C'est aussi à ce moment que Hermione se rendit compte de l'autre détail~— enfin, l'un des détails~— étrange chez ce garçon. Apparemment, les gens \emph{des} livres \emph{ressemblaient} à des livres quand ils parlaient. C'était une découverte pour le moins surprenante.

Le garçon mit la main dans sa bourse et dit <<~canette de soda~>>, et récupéra un cylindre vert fluo. Il le lui tendit et dit~: <<~Puis-je t'offrir quelque chose à boire~?~>>

Hermione accepta le soda poliment. À vrai dire elle se \emph{sentait} un peu assoiffée à présent. <<~Merci beaucoup, dit Hermione alors qu'elle décapsulait la canette. C'était ça ta proposition~?~>>

Le garçon toussa. <<~Non,~>> dit-il. Et juste quand Hermione commença à boire, il dit~: <<~Je voudrais que tu m'aides à conquérir l'univers.~>>

Hermione finit de boire et rabaissa la canette. <<~Non merci, je ne suis pas maléfique.~>>

Le garçon la regarda avec surprise, comme s'il s'était attendu à une autre réponse. <<~Eh bien, je parlais un peu rhétoriquement, dit-il. Au sens du projet Baconien, tu sais, pas du pouvoir politique. “La mise en application de toutes choses possibles”, et ainsi de suite. Je veux conduire des études expérimentales généralisées sur les sorts, comprendre les lois sous-jacentes, amener la magie dans le domaine de la science, fusionner les mondes magiques et Moldus, élever le niveau de vie de toute la planète, faire avancer l'humanité de plusieurs siècles, découvrir le secret de l'immortalité, coloniser le système solaire, explorer la galaxie, et, le plus important, comprendre ce qui peut bien diable se passer ici, parce que tout ça est absolument impossible.~>>

Ça avait l'air un peu plus intéressant. <<~Et~?~>>

Le garçon la regarda avec incrédulité.

<<~\emph{Et}~? Ce n'est pas \emph{assez}~?

--- Et qu'est-ce que tu veux de moi~? dit Hermione.

--- Je veux que tu m'aides dans mes recherches, bien sûr. Avec ta mémoire encyclopédique ajoutée à mon intelligence et à ma rationalité, nous aurons fini le projet Baconien en un rien de temps, et par “un rien de temps” je veux probablement dire au moins trente-cinq ans.~>>

Hermione commençait à trouver ce garçon agaçant. <<~Je ne t'ai rien vu faire d'intelligent. Peut-être que je \emph{te} laisserai m'aider dans \emph{mes} recherches.~>>

Il y eut un silence certain dans le compartiment.

<<~Tu me demandes de démontrer mon intelligence, donc~>>, dit le garçon après une longue pause.

Hermione acquiesça.

<<~Laisse-moi te prévenir que mettre mon ingénuité en doute est une dangereuse sorte de projet, et pourrait rendre ta vie beaucoup plus surréaliste.

--- Je ne suis toujours pas impressionnée,~>> dit Hermione. La main contenant la canette de soda commença de nouveau à s'élever vers ses lèvres.

<<~Bon, peut-être que \emph{ceci} t'impressionnera~>>, dit le garçon. Il se pencha en avant et la regarda avec intensité. <<~J'ai déjà fait quelques expériences et je me suis rendu compte que je n'ai pas besoin de baguette, je peux faire survenir ce que je veux juste en claquant des doigts.~>>

Il dit cela juste au moment où Hermione était en train de déglutir, et elle s'étouffa et toussa et expulsa le fluide vert fluo.

Sur sa robe de sorcière neuve, jamais portée, le premier jour d'école.

Hermione cria vraiment. C'était un cri aigu qui, dans le compartiment fermé, ressemblait à une sirène de raid aérien.

<<~Beeeurk~! Mes vêtements~!

--- Pas de panique~! dit le garçon. Je peux t'arranger ça. Regarde~!~>> Il leva une main et claqua des doigts.

<<~Tu vas…~>> puis elle baissa les yeux.

Le fluide vert était encore là, mais alors même qu'elle le regardait, il commença à disparaître, à s'effacer, et en seulement quelques instants c'était comme si elle ne s'était jamais rien renversé dessus.

Hermione fixa le garçon, qui arborait à présent un sourire plutôt satisfait.

De la magie muette sans baguette~! À \emph{son} âge~? Alors qu'il n'avait obtenu les manuels que \emph{trois jours} auparavant~?

Puis elle se souvint de ce qu'elle avait lu, hoqueta, et s'écarta de lui. \emph{Tout le pouvoir du Seigneur des Ténèbres~! Dans sa cicatrice~!}

Elle se leva hâtivement. <<~J'ai, j'ai, j'ai besoin d'aller aux lavabos, attends ici~>> — il fallait qu'elle trouve un adulte pour le leur dire —

Le sourire du garçon disparu. <<~C'était juste un tour, Hermione. Je suis désolé, je ne voulais pas te faire peur.~>>

Sa main s'arrêta sur la poignée de la porte.

<<~Un \emph{tour}~?

--- Oui, dit le garçon. Tu m'as demandé de démontrer mon intelligence. Alors j'ai fait quelque chose d'apparemment impossible, ce qui est toujours une bonne façon de frimer. Je ne peux pas \emph{vraiment} faire tout ce que je veux juste en claquant des doigts.~>> Le garçon s'interrompit. <<~Du moins, je ne \emph{pense} pas pouvoir le faire, je n'ai jamais vraiment essayé.~>> Le garçon leva sa main et claqua à nouveau des doigts. <<~Non, pas de banane.~>>

Hermione était plus confuse qu'elle ne l'avait jamais été dans toute sa vie.

Le garçon souriait maintenant à nouveau à la vue de l'expression de Hermione.

<<~Je t'ai \emph{prévenue} que remettre mon ingénuité en doute tendrait à rendre ta vie surréaliste. Souviens-toi bien de ça la prochaine que je te préviens à propos de quelque chose.

--- Mais, mais, bégaya Hermione. Qu'est-ce que tu as \emph{fait} alors~?~>>

Le regard du garçon sembla mesurer et peser comme jamais elle ne l'avait vu chez quelqu'un de son âge.

<<~Tu penses que tu as ce qu'il faut pour être une scientifique à toi toute seule, avec ou sans mon aide~? Alors voyons comment \emph{tu} enquêtes sur un phénomène déroutant.

--- Je…~>> L'esprit de Hermione devint vide pendant un instant. Elle aimait être testée, mais elle n'avait jamais eu un \emph{tel} test auparavant. Elle essaya frénétiquement de se souvenir de tout ce qu'elle avait lu au sujet de ce que les scientifiques étaient censés faire. Son cerveau passa des vitesses, vrombit, et cracha les instructions nécessaires à la réalisation d'un projet scientifique pour une kermesse d'école primaire~:

\emph{Étape~1~: Former une hypothèse.\\
Étape~2~: Faire une expérience pour tester l'hypothèse.\\
Étape~3~: Mesurer les résultats.\\
Étape~4~: Faire une affiche en carton.}

L'étape 1 était de former une hypothèse. Cela voulait dire~: essayer de penser à quelque chose qui \emph{aurait} \emph{pu} avoir eu lieu à l'instant.

<<~Très bien. Mon hypothèse est que tu peux jeter un sort sur ma robe pour faire disparaître tout ce qui y a été renversé.

--- Très bien, dit le garçon, est-ce ta réponse~?~>>

Le choc se dissipait, l'esprit de Hermione commençait à fonctionner correctement. <<~Attends, ce n'est pas une très bonne idée. Je ne t'ai pas vu toucher ta baguette ni dire le moindre sort, alors comment aurais-tu pu en jeter un~?~>>

Le garçon attendit avec une expression neutre.

<<~Mais suppose que toutes les robes du magasin aient \emph{déjà} un charme pour les garder propres~; ce qui serait un charme très utile à avoir. Tu as découvert cela en renversant quelque chose sur \emph{toi} plus tôt.~>>

Les sourcils du garçon s'élevèrent.

<<~\emph{Est-ce} ta réponse~?

--- Non, je n'ai pas fait l'étape 2, “Faire une expérience pour tester l'hypothèse”.~>>

Le garçon ferma à nouveau la bouche et commença à sourire.

Hermione regarda la canette de soda qui se trouvait dans sa main et qu'elle avait automatiquement placée dans le porte-canette de la fenêtre. Elle la soupesa et sentit qu'elle était pleine à environ un tiers.

<<~Bon, dit Hermione, l'expérience que je veux faire est d'en verser sur ma robe pour voir ce qui se passe, et ma prédiction est que la tache disparaîtra. Seulement si ça ne marche \emph{pas}, ma robe sera tachée, et je n'ai pas envie que ça se produise.

--- Fais-le sur la mienne, dit le garçon, comme ça tu n'as pas à t'inquiéter que ta robe soit tâchée.

--- Mais…~>> dit Hermione. Il y avait quelque chose qui \emph{clochait} avec cette façon de penser, mais elle ne savait pas comment le formuler précisément.

<<~J'ai une robe de rechange dans ma malle, dit le garçon.

--- Mais tu n'as nulle part où te changer~>>, objecta Hermione. Puis elle révisa son opinion. <<~Bien que je suppose que je pourrais sortir et fermer la porte…

--- J'ai aussi un endroit où me changer dans ma malle.~>>

Hermione regarda la malle, qui, elle commençait à le soupçonner, était beaucoup plus spéciale que la sienne.

<<~Très bien, dit Hermione, puisque tu le dis, et elle versa délicatement un peu de soda vert sur un coin de la robe du garçon. Puis elle fixa la tache du regard, essayant de se rappeler combien de temps le soda original avait mit à disparaître.~>>

Et le soda disparut~!

Hermione poussa un soupir de soulagement, et pas qu'un peu parce ça voulait dire qu'elle n'avait pas affaire à tous les pouvoirs du Seigneur des Ténèbres.

Eh bien, l'étape 3 était de mesurer les résultats, mais dans ce cas cela consistait juste à voir que le soda avait disparu. Et elle présumait qu'elle pouvait probablement sauter l'étape 4 qui parlait de faire une affiche en carton.

<<~Ma réponse est que la robe a été ensorcelée pour rester propre.

--- Pas tout à fait~>>, dit le garçon.

Hermione sentit une pique de déception. Elle aurait vraiment aimé ne \emph{pas} la ressentir, le garçon n'était pas un enseignant, mais c'était quand même un test et elle avait mal répondu à une question et ça lui faisait toujours l'effet d'un petit coup de poing dans l'estomac.

(Le fait qu'elle n'avait jamais laissé cela l'arrêter ni même interférer avec son envie d'être testée vous disait presque tout ce que vous aviez besoin de savoir au sujet de Hermione Granger).

<<~Ce qui est triste, dit le garçon, c'est que tu as probablement fait tout ce que le livre t'a dit de faire. Tu as fait une prédiction qui distinguerait entre une robe ensorcelée et une robe non ensorcelée, et tu l'as testée, et tu as rejeté l'hypothèse nulle, à savoir que la robe n'était pas ensorcelée. Mais à moins de lire des livres bien, bien meilleurs, tes livres ne t'apprendront pas à pratiquer la science \emph{correctement}. Je veux dire assez bien pour trouver la bonne réponse, et pas juste pondre une publication de plus comme celles dont Papa se plaint toujours. Alors laisse-moi essayer de t'expliquer~— sans te donner la réponse~— ce que tu as mal fait cette fois, et je te donnerai une autre chance.~>>

Elle commençait à ne pas apprécier le ton oh-si-supérieur du garçon, alors qu'il n'était qu'un autre enfant de onze ans comme elle~; mais cela était moins important que de découvrir pourquoi elle s'était trompée. <<~Très bien.~>>

L'expression du garçon devint plus intense. <<~C'est un jeu basé sur une expérience connue nommée l'exercice 2-4-6, et voilà comment elle se déroule. J'ai une \emph{règle}~— connue par moi, mais pas par toi~— qui correspond à des triplets de trois nombres, mais pas à certains autres. 2-4-6 est un exemple de triplet qui correspond à la règle. En fait… laisse-moi écrire la règle, juste pour que tu saches qu'elle est fixe, et la plier pour te la donner. Ne regarde pas s'il te plaît, puisque je déduis de ce qui s'est passé plus tôt que tu peux lire à l'envers.~>>

Le garçon dit <<~papier~>> et <<~criterium~>> à sa bourse, et elle ferma solidement ses yeux pendant qu'il écrivait.

<<~Voilà~>>, dit le garçon. Il tenait une pièce de papier pliée serré. <<~Mets ça dans ta poche~>>, et elle s'exécuta.

<<~Maintenant, la façon dont ce jeu fonctionne, dit le garçon, c'est que tu me donnes un triplet de trois nombres, et je te dis “Oui” si les trois nombres correspondent à la règle, et “Non” s'ils ne correspondent pas. Je suis la Nature, la règle est une de mes lois, et tu m'étudies. Tu sais déjà que 2-4-6 donne un “Oui”. Une fois que tu as effectué tous les tests supplémentaires que tu souhaites~— que tu m'as soumis autant de triplets que tu juges nécessaire~— tu t'arrêtes et tu devines la règle, et là tu peux déplier la feuille et voir si tu as réussi. Comprends-tu le jeu~?

--- Bien sûr que je comprends, dit Hermione

--- Vas-y.

--- 4-6-8 dit Hermione.

--- Oui, dit le garçon.

--- 10-12-14, dit Hermione.

--- Oui, dit le garçon.~>>

Hermione essaya de faire dériver son esprit un peu plus loin, puisqu'il semblait qu'elle avait fait tous les tests nécessaires, et pourtant ça ne pouvait pas être aussi facile.

<<~1-3-5.

--- Oui.

--- Moins 3, moins 1, plus 1.

--- Oui.~>>

Hermione était à court d'idées.

<<~La règle est que les nombres doivent augmenter de deux à chaque fois.

--- Maintenant suppose que je te dise, dit le garçon, que ce test est plus dur qu'il n'y paraît, et que seulement 20~\% des adultes le réussissent.~>>

Hermione fronça les sourcils. Qu'avait-elle manqué~? Puis, soudain, elle pensa à un test qu'elle avait encore besoin de faire.

<<~2-5-8~! dit-elle triomphante.

--- Oui.

--- 10-20-30~!

--- Oui.

--- La vraie réponse est que les nombres doivent augmenter de la \emph{même} quantité à chaque fois. Ça n'a pas besoin d'être 2.

--- Très bien, dit le garçon, prends le papier et regarde si tu as réussi.~>>

Hermione prit le papier hors de sa poche et le déplia.

\emph{Trois nombres réels en ordre croissant, du plus petit au plus grand.}

La mâchoire de Hermione s'affaissa. Elle avait la distincte sensation que quelque chose de terriblement injuste venait d'être commis à son encontre, que le garçon était un sale petit menteur tricheur pourri, mais lorsqu'elle chercha dans ses souvenirs elle ne put trouver une seule mauvaise réponse parmi celles qu'il avait données.

<<~Ce que tu viens de découvrir est appelé “biais positif”, dit le garçon. Tu avais une règle en tête, et tu as continué à penser à des triplets qui feraient dire “Oui” à la règle. Mais tu n'as pas essayé de tester autant de triplets que possible qui feraient dire “Non” à la règle. En fait, tu n'as pas eu un \emph{seul} “Non”, donc “trois nombres, n'importe lesquels” aurait tout aussi bien pu être la règle. C'est comme quand les gens imaginent des expériences qui pourraient confirmer leurs hypothèses au lieu d'essayer d'imaginer des expériences qui pourraient les falsifier — ce n'est pas exactement la même erreur, mais presque. Tu dois apprendre à regarder le côté négatif des choses, à regarder dans les ténèbres. Lorsque cette expérience est réalisée, seuls 20~\% des adultes trouvent la bonne réponse. Et nombreux sont ceux parmi les autres qui inventent des hypothèses fantastiquement compliquées et accordent beaucoup de confiance à leurs mauvaises réponses, puisqu'ils ont fait beaucoup d'expériences et que tout s'est déroulé comme ils s'y attendaient.

--- Maintenant, dit le garçon, veux-tu t'essayer à nouveau au problème initial~?~>>

Ses yeux étaient très attentifs à présent, comme si c'était là le \emph{vrai} test.

Hermione ferma les yeux et essaya de se concentrer. Elle suait sous sa robe. Elle avait l'étrange sensation qu'on ne lui avait jamais demandé de réfléchir aussi dur pour un test, ou peut-être même que c'était la \emph{première} fois qu'on lui avait jamais demandé de réfléchir pour un test.

Quelle autre expérience pouvait-elle réaliser~? Elle avait une Grenouille en Chocolat, pouvait-elle essayer de la frotter un peu sur la robe et de voir si \emph{ça} disparaissait~? Mais ça ne ressemblait toujours pas à la façon de penser tordue et négative que le garçon demandait d'elle. Comme si elle demandait toujours un “Oui”, si la tache de Grenouille en Chocolat disparaissait, alors qu'elle aurait dû demander un “Non”.

Donc… selon son hypothèse… quand le soda devrait-il… ne \emph{pas} disparaître~?

<<~J'ai une expérience à faire, dit Hermione. Je veux verser un peu de soda par terre, et voir s'il ne \emph{disparaît pas}. As-tu quelques mouchoirs en papier dans ta bourse, pour que je puisse éponger le soda si ça ne marche pas~?

--- J'ai des serviettes~>>, dit le garçon. Il avait toujours un air neutre.

Hermione prit le soda, et en versa un peu sur le sol.

Quelques secondes plus tard, il disparut.

<<~Eurêka~>>, dit Hermione doucement. C'était comme une compulsion, elle \emph{devait} le dire. À vrai dire, elle voulait le crier, mais elle était un tout petit peu trop inhibée pour ça. Puis la prise de conscience lui vint, et elle eut envie de se frapper. <<~Bien sûr~! \emph{Tu} m'as donné le soda~! Ce n'est pas la robe qui est ensorcelée, c'était le soda depuis le début~!~>>

Le garçon se leva et s'inclina solennellement. Il avait un énorme sourire.

<<~Dans ce cas… pourrais-je t'aider dans tes recherches, Hermione Granger~?

--- Je, euh…~>> Hermione ressentait toujours l'effet de l'euphorie, mais elle ne savait pas bien comment répondre à \emph{ça}.

Ils furent interrompus par un coup frappé à la porte avec faiblesse, timidité, et même avec \emph{réticence}.

Le garçon se détourna, regarda par la fenêtre, et dit~: <<~Je ne porte pas mon écharpe, tu peux y aller~?~>>

C'est à ce moment-là que Hermione comprit pourquoi le garçon — non, le Garçon-qui-avait-survécu, Harry Potter — avait commencé à porter l'écharpe sur sa tête, et elle se sentit un peu idiote de ne pas s'en être rendu compte plus tôt. C'était en fait assez étrange, car elle aurait cru que Harry Potter était le genre de garçon qui se montrerait fièrement au monde entier~; et l'idée lui vint qu'il pourrait être en réalité plus timide qu'il n'en avait l'air.

Lorsque Hermione ouvrit la porte, elle fut accueillie un jeune garçon tremblant qui ressemblait exactement à la façon dont il avait frappé à la porte.

<<~Excuse-moi, dit le garçon d'une petite voix, Je suis Neville Londubat. Je cherche ma tortue de compagnie, je, je n'arrive à la trouver nulle part dans son wagon… tu aurais vu ma tortue~?

--- Non~>>, dit Hermione, et c'est là que sa nature serviable s'alluma plein gaz. <<~As-tu regardé dans les autres compartiments~?

--- Oui, murmura le garçon.

--- Alors nous devrons vérifier tous les autres wagons, dit vivement Hermione. Je t'aiderai. Mon nom est Hermione Granger, au fait.~>>

On aurait dit que le garçon allait s'évanouir de gratitude.

<<~Attends~>>, fit la voix de \emph{l'autre} garçon — Harry Potter. <<~Je ne suis pas certain que ce soit la meilleure façon de faire.~>>

Sur ce Neville sembla être au bord des larmes, et Hermione pivota, énervée. Si Harry Potter était le genre de personne qui était prêt à abandonner un petit garçon juste parce qu'il n'aimait pas être interrompu…

<<~Quoi~? Pourquoi \emph{pas}~?

--- Eh bien, dit Harry Potter, Ça va prendre un moment de vérifier tout le train manuellement, et on pourrait quand même rater la tortue, et si on ne la trouve pas d'ici à Poudlard il aura des ennuis. Ça serait beaucoup plus sensé d'aller directement à la voiture de tête, où se trouvent les préfets, et de demander leur aide directement. C'est la première chose que j'ai faite quand je t'ai cherchée, Hermione, bien qu'ils n'aient pas su où te trouver. Mais peut-être qu'ils ont des sorts ou des objets magiques qui faciliteraient grandement la recherche de la tortue. Nous ne sommes qu'en première année.~>>

Cette idée… \emph{était} beaucoup plus sensée.

<<~Penses-tu pouvoir te rendre seul à la voiture des préfets par toi-même~? demanda Harry Potter. J'ai plusieurs raisons de ne pas vouloir trop montrer mon visage.~>>

Neville s'étrangla soudain et fit un pas en arrière. <<~Je me souviens de cette voix~! Tu es l'un des Seigneurs du Chaos~! \emph{Tu es celui qui m'a donné des bonbons~!}~>>

Quoi~? Quoi quoi \emph{quoi}~?

Harry Potter se détourna de la fenêtre et se leva dans un élan théâtral. <<~\emph{Jamais~!} dit-il, la voix pleine d'indignation. Est-ce que je ressemble à quelqu'un qui donnerait des bonbons à un enfant~?~>>

Les yeux de Neville s'agrandirent.

<<~\emph{Tu} es Harry Potter~? \emph{Le} Harry Potter~? \emph{Toi}~?

--- Non, juste un Harry Potter, nous sommes trois à bord de ce train…~>>

Neville poussa un petit cri aigu et s'enfuit. Il y eut un crépitement de bruits de pas frénétiques suivis du son d'une porte de wagon s'ouvrant et se refermant.

Hermione se laissa tomber durement sur le banc. Harry Potter ferma la porte et s'assit à côté d'elle.

<<~Pourrais-tu s'il te plaît m'expliquer ce qui se passe~?~>> dit Hermione d'une voix faible. Elle se demandait si traîner avec Harry Potter voulait dire qu'elle serait toujours autant déroutée.

<<~Oh, eh bien ce qui s'est passé c'est que Fred et George et moi avons vu ce pauvre petit garçon à la gare — la femme qui l'accompagnait était partie un moment, et il avait l'air vraiment effrayé, comme s'il était certain qu'il allait être attaqué par des Mangemorts ou quelque chose comme ça. Et il y a un dicton qui dit que la peur est souvent pire que l'objet de la peur, donc je me suis dit que c'était un type à qui ça pourrait être bénéfique de voir son pire cauchemar devenir réalité et de se rendre compte que ce n'était pas si grave que ce qu'il craignait…~>>

Hermione resta assise, sa bouche grande ouverte.

<<… et Fred et George ont trouvé ce sort qui assombrit et floute les écharpes enroulées autour d'un visage, comme si nous étions des rois morts-vivants et que c'étaient nos linceuls…~>>

Elle n'aimait pas du tout où cette histoire allait.

<<… et après que nous avions fini de lui donner tous les bonbons que j'avais achetés, on s'est dit~: “Donnons-lui de l'argent~! Ha ha ha~! Voilà des Noises, garçon~! Prends une Mornille d'argent~!” en dansant autour de lui et en riant d'un air maléfique et tout ça. Je pense qu'il y avait quelques personnes dans la foule qui voulaient intervenir, mais l'apathie du témoin les a retenues assez longtemps pour qu'ils aient le temps de voir ce qu'on faisait, et ensuite ils étaient bien trop confus pour faire quoi que ce soit. Il a fini par dire “allez-vous-en” en chuchotant très doucement alors nous sommes parties tous les trois en criant et en courant, et en disant quelque chose à propos de la lumière qui brûlait. J'espère qu'il n'aura pas autant peur de se faire malmener à l'avenir. Ça s'appelle la thérapie par désensibilisation, au fait.~>>

D'accord, elle n'avait \emph{pas} deviné où cette histoire allait.

Le feu d'indignation brûlante qui était l'un des moteurs principaux de Hermione s'éveilla en vrombissant, même si une part d'elle \emph{voyait bien} ce qu'il avait essayé de faire.

<<~C'est affreux~! \emph{Tu} es affreux~! Ce pauvre garçon~! Tu as fait quelque chose de \emph{méchant}~!

--- Je pense que le mot que tu cherches est \emph{amusant}, et quoi qu'il en soit tu poses la mauvaise question. La question est~: cela a-t-il fait plus de bien que de mal, ou plus de mal que de bien~? Si tu as le moindre argument en rapport avec \emph{cette} question, je serai heureux de l'entendre, mais je n'accepterai aucune autre critique tant que \emph{celle-}ci n'aura pas été réglée. Je suis tout à fait d'accord, ce que j'ai fait \emph{a l'air} affreux, méchant, et brutalisant, vu qu'un petit garçon effrayé est concerné et ainsi de suite, mais ce n'est certainement pas le problème clé, non~? Au fait, ça s'appelle le \emph{conséquentialisme}, et ça signifie qu'un acte n'est pas bon ou mauvais parce qu'il a \emph{l'air} bon ou mauvais, ou autre chose du genre~; la seule question est celle du résultat final — quelles sont les conséquences.~>>

Hermione ouvrit la bouche pour dire quelque chose de \emph{cinglant}, mais il semblait malheureusement qu'elle avait négligé l'étape où elle pensait à quelque chose à dire avant d'ouvrir la bouche. Tout ce qu'elle trouva fut~:

<<~Et s'il a des \emph{cauchemars}~?

--- Honnêtement je ne pense pas qu'il avait besoin de notre aide pour avoir des cauchemars, et s'il a des cauchemars à propos de \emph{ça}, alors ce sera des cauchemars où les monstres vous donnent du chocolat et \emph{ça} c'était notre but \emph{initial}.~>>

Le cerveau de Hermione hoquetait de confusion à chaque fois qu'elle essayait de se mettre correctement en colère. <<~Ta vie est-elle toujours si inhabituelle~?~>> dit-elle enfin.

Le visage de Harry Potter rayonna de fierté.

<<~Je la \emph{rends} inhabituelle. Tu observes le résultat de beaucoup de travail et d'huile de coude.

--- Donc…~>> dit Hermione, et elle se tut avec maladresse.

<<~Donc, dit Harry Potter, quelle est l'étendue exacte de tes connaissances scientifiques~? Je sais résoudre des équations, je connais un peu de théorie de probabilité Bayésienne et de théorie de la décision et beaucoup de science cognitive, et j'ai lu le \emph{Cours de Physique de Feynman} (du moins le volume 1) et \emph{Jugement dans l'Incertitude~: Heuristiques et Biais} et \emph{Langage dans la Pensée et l'Action} et \emph{Influence et Manipulation} et \emph{Choix Rationnel dans un Monde Incertain} et \emph{Gödel, Escher, Bach} et \emph{Un pas plus loin} et…~>>

Le quiz et le contre-quiz qui suivirent durèrent plusieurs minutes avant d'être interrompus par un autre coup timide frappé à la porte. <<~Entrez~>> dirent-ils presque au même instant, et la porte glissa pour révéler Neville Londubat.

Neville pleurait \emph{vraiment} cette fois. <<~J'ai été à la voiture de tête et j'ai trouvé un p-préfet, mais il m'a d-dit que les préfets ne devaient pas être dérangés pour des petites affaires telles que des tortues m-manquantes.~>>

Le Survivant changea d'expression. Ses lèvres devinrent une ligne fine. Sa voix, lorsqu'il parla, était froide et sinistre.

<<~Quelles étaient ses couleurs~? Vert et argent~?

--- N-non, son badge était r-rouge et or.

--- \emph{Rouge et Or~!} s'écria Hermione. Mais ce sont les couleurs de \emph{Gryffondor}~!~>>

Harry Potter \emph{siffla} en entendant ça, un son effrayant qui aurait pu venir d'un serpent et fit tressaillir Neville et Hermione. <<~Je \emph{suppose}, cracha Harry Potter, que trouver la tortue d'un quelconque première année n'est pas assez \emph{héroïque} pour mériter un préfet de \emph{Gryffondor}. Viens Neville, \emph{je} vais venir avec toi cette fois, et on verra si le Survivant obtient plus d'attention. D'abord nous trouverons un préfet qui sait jeter des sorts, et si ça ne marche pas, nous trouverons un préfet qui n'a pas peur de se salir les mains, et si \emph{ça} ne marche pas, je commencerai à recruter parmi mes fans et si nous le devons nous démonterons ce train boulon par boulon.~>>

Le Survivant se leva, attrapa la main de Neville, et Hermione se rendit compte dans un hoquet mental qu'ils avaient presque la même taille, et pourtant une partie d'elle-même insistait sur le fait que Harry Potter faisait au moins trente centimètres de plus, et Neville quinze de moins.

<<~\emph{Reste~!}~>> lui lâcha Harry Potter — non, en fait, à sa \emph{malle} — et il ferma la porte fermement et s'en fut.

Elle aurait probablement dû y aller avec eux, mais pendant un bref moment Harry Potter était devenu si effrayant qu'elle était plutôt contente de ne pas l'avoir suggéré.

L'esprit de Hermione était maintenant si embrouillé qu'elle ne pensait même pas pouvoir lire <<~Histoire~: Son Poudlard~>>. Elle avait l'impression qu'un rouleau compresseur lui était passé dessus et l'avait transformée en pancake. Elle n'était pas sûre de ce qu'elle pensait, ou de ce qu'elle ressentait, et encore moins de pourquoi. Elle resta juste assise à la fenêtre et regarda le paysage en mouvement.

%  LocalWords:  NPC Eek Judgment
