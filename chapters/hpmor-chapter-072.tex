\namedpartchapter{Accomplissement de soi}{SA}{VII}{Déni plausible}

\lettrine{L}{e} soleil d'hiver s'était couché bien avant la fin du dîner et c'est donc au milieu de la paisible lumière des étoiles qui perçait à travers le plafond de la grande salle que Hermione partit pour la tour Serdaigle au côté de son partenaire d'étude, Harry Potter, qui depuis peu semblait avoir \emph{tout le temps du monde} pour étudier. Quand Harry faisait-il ses vrais devoirs~? Elle n'en avait pas la moindre idée, mais ils étaient faits, peut-être par des elfes de maison pendant que Harry dormait.

Presque toutes les paires d'yeux de la grande salle leur tombèrent dessus lorsqu'ils passèrent les imposantes portes du réfectoire, des portes qui ressemblaient plus aux portes de siège d'un château qu'à une cloison que les élèves devraient avoir à traverser après leur dîner.

Ils sortirent sans dire un mot et marchèrent jusqu'à ce que le lointain babillage des conversations estudiantines se soit fondu dans le silence~; puis ils allèrent un peu plus loin dans les couloirs de pierre avant que Hermione ne parle enfin.

"Pourquoi t'as fait ça, Harry~?"

"Fais quoi~?" dit le Survivant d'un ton absent comme si son esprit avait été tout à fait ailleurs, absorbé par des choses immensément plus importantes.

"Je veux dire pourquoi est-ce que tu ne leur as pas juste dit \emph{non}~?"

"Eh bien," dit Harry alors que leurs chaussures battaient les dalles d'un pas léger, "je ne peux pas m'amuser à dire 'non' à chaque fois que quelqu'un m'interroge au sujet de quelque chose que je n'ai pas fait. Par exemple imagine que quelqu'un me demande~: 'Harry, c'est toi qui as fait la blague avec la peinture invisible~?', que je réponde~: 'non', puis que la même personne me dise~: 'Harry, est-ce que tu sais qui a joué avec le balai volant de l'attrapeur Gryffondor~?' et que je dise~: 'Je refuse de répondre à cette question', ça vendrait plutôt la mèche."

"Et c'est pour \emph{ça}," dit Hermione avec précaution, "que tu as dit à tout le monde…" Elle se concentra et essaya de se souvenir des mots exacts, "… que si, hypothétiquement, il y \emph{avait} une conspiration, tu ne pouvais ni confirmer ni infirmer que le véritable chef de cette conspiration était le fantôme de Salazar Serpentard, et qu'à vrai dire tu ne pouvais même pas admettre qu'une conspiration existait et que les gens devraient donc arrêter de te poser des questions à ce sujet."

"Ouaip," dit Harry Potter avec un léger sourire. "Ça leur apprendra à prendre les scénarios hypothétiques trop au sérieux."

"Et tu \emph{m'as} dit de ne répondre à rien -"

"Ils ne te \emph{croiraient} peut-être pas si tu le niais," dit Harry. "Donc il vaut mieux ne rien dire, à moins que tu ne veuilles qu'ils te prennent pour une menteuse."

"Mais -" dit Hermione, désemparée, "mais - mais maintenant les gens croient que je \emph{fais} des choses pour le compte de Salazar Serpentard~!" La façon dont ceux de Gryffondor l'avaient regardée - la façon dont ceux de \emph{Serpentard} l'avaient regardée -

"Ça fait partie de la condition de héros," dit Harry. "Tu as vu ce que le \emph{Chicaneur} dit sur \emph{moi}~?"

L'espace d'un instant, Hermione imagina ses parents lisant un article de journal à son sujet sauf qu'au lieu de parler de sa victoire nationale au concours d'orthographe ou de n'importe lequel des autres moyens par lesquels elle s'était imaginée finir dans le journal, le gros titre disait~: "DRAGO MALFOY TOMBE ENCEINTE DE HERMIONE GRANGER".

Ça suffisait à vous faire réfléchir à deux fois à l'idée de se lancer dans une carrière d'héroïne.

La voix de Harry devint un peu plus formelle. "En parlant de ça, Mlle Granger, comment va votre dernière quête~?"

"Eh bien," dit Hermione, "à moins que le fantôme de Salazar Serpentard ne surgisse \emph{vraiment} et qu'il ne nous dise où trouver les brutes, je ne pense pas que nous allons avoir beaucoup de chance." Non que cela la désolasse.

Elle jeta un coup d'œil à Harry et vit que le garçon l'observait d'un regard particulièrement intense.

"Tu sais, Hermione," dit le garçon d'une voix basse comme s'il voulait s'assurer que personne d'autre au monde ne l'entendrait, "je pense que tu as raison, je pense que certains se font beaucoup plus aider que d'autres pour devenir des héros. Et \emph{je} ne pense pas non plus que ce soit juste."

Et Harry attrapa les robes de sorcières de Hermione au-dessus de son bras, la poussa dans un couloir transversal à celui dans lequel ils marchaient, la bouche de Hermione béate de surprise lorsque la main de Harry sortit sa baguette, puis ils longèrent la courbure du couloir tandis que Harry pointait sa baguette vers l'endroit d'où ils venaient et disait "\emph{Quietus}" d'une voix basse, puis un moment plus tard, dans l'autre direction, "\emph{Quietus}" à nouveau.

Le garçon inspecta les alentours, pas seulement les côtés mais aussi vers le plafond et vers le plancher.

Puis il mit une main dans sa bourse et dit~: "Cape d'Invisibilité".

"\emph{Ehhhhh}~?" dit Hermione.

Harry extrayait déjà des replis de tissu noir scintillant de l'appareil en peau de Moke. "Ne t'inquiète pas," dit le garçon avec un petit sourire, "elles sont tellement rares que personne n'a prit la peine d'inventer un règlement pour les interdire…"

Puis Harry tendit le maillage de soie noir et dit, d'une voix étrangement cérémonieuse~: "Je ne te donne pas, mais je te prête, ma cape, à Hermione Jean Granger. Protège-la bien."

Elle regarda la soie scintillante de la cape, le tissu qui avalait toute la lumière qu'il recevait hormis la lueur d'étranges petits reflets, un tissu si parfaitement noir qu'on aurait dû voir de la poussière ou des peluches ou \emph{quelque chose} mais non, plus on le regardait plus on avait le sentiment que ce qu'on voyait n'était tout simplement pas là, mais alors on clignait des yeux et c'était juste une cape noire.

"Prends-la, Hermione."

Sans même penser, Hermione tendit la main pour se saisir du tissu~; et juste à l'instant où son cerveau s'éveilla et qu'elle commença à retirer sa main, Harry lâcha la cape qui commença à tomber, si bien que Hermione s'en saisit par réflexe. À l'instant où ses doigts touchèrent et tinrent la cape elle sentit une secousse intangible la traverser, comme la première fois qu'elle avait tenu sa baguette~; c'était comme si quelqu'un s'était mis à chanter d'une voix extrêmement faible à l'arrière de son crâne.

"C'est un de mes objets de quête, Hermione," dit Harry d'une voix douce. "Elle appartenait à mon père et ce n'est pas quelque chose que je pourrais remplacer en cas de perte. Ne la prête à personne, ne la montre à personne, ne dis à personne qu'elle existe… mais si tu veux l'emprunter un moment, viens me voir et demande-la moi."

Hermione arracha enfin ses yeux à la contemplation des replis noirs sans profondeur et regarda de nouveau Harry.

"Je ne peux pas -"

"Bien sûr que tu peux," dit Harry. "Parce qu'il n'y a absolument rien de juste au fait que j'ai trouvé ceci emballé dans du papier cadeau dans une boîte un matin, à côté de mon lit, et… pas toi." Harry marqua une pause songeuse. "À moins que tu \emph{n'aies} reçu ta propre cape d'invisibilité, auquel cas, oublie ce que j'ai dit."

Puis ce que \emph{cape d'invisibilité} impliquait lui vint enfin à l'esprit et elle pointa un doigt choqué vers Harry, même s'ils étaient maintenant suffisamment près pour qu'elle ne puisse pas entièrement étendre son bras, et sa voix s'éleva à un niveau d'indignation considérable lorsqu'elle dit~: "Alors c'est \emph{comme ça} que tu as disparu du placard à potions~! Et cette fois où -" et elle resta en suspens, parce que même \emph{avec} une cape d'invisibilité elle ne voyait toujours pas comment Harry avait…

Harry se lustra les ongles sur sa robe avec une nonchalance étudiée et dit~: "Eh bien, tu savais qu'il devait y avoir \emph{un} truc, non~? Et maintenant l'héroïne va mystérieusement savoir où et quand trouver les brutes - exactement comme si elle les avait entendues tout planifier, même si est \emph{impossible} qu'une personne de son âge soit \emph{capable} de devenir invisible pour les espionner."

Il y eut une pause, un silence.

"Harry -" dit-elle. "Je - je ne suis plus sûre que ce soit une si bonne idée de combattre des brutes."

Les yeux de Harry restèrent braqués sur les siens. "Parce que d'autres filles pourraient se faire mal~?"

Elle hocha seulement la tête.

"C'est \emph{leur} choix, Hermione, tout comme c'est le tien. \emph{J'ai} décidé de ne pas faire le choix évidemment stupide que tout le monde fait dans les livres, de ne pas essayer de te garder à l'abri, protégée et incapable, ce qui t'aurait mise très en colère contre moi, et tu m'aurais repoussé, serais partie seule, aurais fait face à encore plus d'ennuis, aurais héroïquement surmonté tes obstacles, après quoi j'aurais eu mon épiphanie et j'aurais compris que bla bla bla et cetera. Je sais comment cette partie de l'histoire de ma vie se déroule alors je la saute. Si je peux prédire ce que je vais penser plus tard, autant que je le pense maintenant. Bref, ce que je veux dire c'est que tu ne devrais pas non plus étouffer \emph{tes} amies pour les garder à l'abri. Dis leur juste à l'avance que ça va horriblement mal tourner et ce de façon tout à fait prévisible et que si elles veulent quand même être des héroïnes malgré ça alors d'accord."

C'était pendant des moments comme celui là que Hermione se demandait si elle allait \emph{jamais} s'habituer à la façon de penser de Harry. "Harry, je ne veux vraiment," sa voix buta pendant une seconde, "vraiment, \emph{vraiment} pas qu'il leur arrive du mal~! Surtout à cause de quelque chose que j'ai commencé~!"

"Hermione," dit Harry avec sérieux, "je suis assez certain que tu as fait ce qu'il fallait. Je ne vois pas de scénario plausible qui serait \emph{pire} pour elles, à long terme, que de \emph{ne rien faire}."

"Et s'il leur arrive quelque chose de \emph{grave}~?" dit Hermione. Sa voix se bloqua dans sa gorge; elle se souvint de capitaine Ernie racontant comment Harry avait juste regardé droit dans les yeux d'une brute pendant que celle-ci tordait son doigt en arrière, avant que le professeur Chourave n'arrive pour le sauver~; et une autre pensée suivit celle-ci, une pensée pour Hannah et ses mains délicates avec ses ongles qu'elle peignait précautionneusement d'un jaune Poufsouffle chaque matin, mais imaginer la suite lui fut alors interdit. "Et alors - alors elles ne feront plus jamais rien de courageux, plus jamais -"

"Je ne pense pas que ça marche comme ça," dit Harry d'un ton ferme. "Même si tout tourne atrocement mal, je ne pense pas que ça marche comme ça dans l'esprit humain. Ce qui est important c'est qu'elles se voient comme des gens capables de dépasser leurs limites. Essayer et avoir mal ne peut pas être pire pour quelqu'un que d'être… \emph{coincé}."

"Et si tu as \emph{tort}, Harry~?"

Harry se tut un moment, puis il haussa les épaules d'un air un peu triste et dit "Et si j'ai raison~?"

Hermione rabaissa les yeux vers le maillage posé noir sur sa main. De l'intérieur, la cape était étrangement douce et pourtant ferme contre sa paume, comme si la cape avait essayé de donner un câlin rassurant à sa main.

Puis elle releva le bras, tendit la cape à Harry.

Harry ne fit pas un geste pour la prendre.

"Je -" dit Hermione. "Enfin, merci, merci beaucoup, mais je dois encore y réfléchir, alors tu peux la reprendre pour l'instant. Et… Harry, je ne pense pas que ce soit bien \emph{d'espionner} les gens -"

"Même pas des brutes, pour sauver leurs victimes~?" dit Harry. "\emph{Je} n'ai jamais été brutalisé, mais j'ai participé à une simulation réaliste et ça n'était pas très agréable. As-tu jamais été brutalisée, Hermione~?"

"Non," dit-elle d'une petite voix, et elle continua de tendre sa cape d'invisibilité à Harry.

Il reprit enfin sa cape - elle eut un léger sentiment de perte lorsque la chanson inaudible disparut de l'arrière de son crâne - et commença à enfoncer le matériau noir dans sa bourse.

Après que la bourse eut mangé le dernier morceau de tissu, Harry se détourna de Hermione pour aller briser la barrière de silence -

"Et, euh," dit Hermione. "Ce n'est pas \emph{la} Cape d'Invisibilité quand même~? Celle qu'on a vue à la bibliothèque à la page dix-huit de la traduction par Paula Vieira du \emph{Parchemin Illustré des Objets Perdus} de Gottschalk~?"

Harry se retourna vers elle, un léger sourire sur le visage, et dit exactement du même ton de voix qu'il avait utilisé plus tôt avec les autres élèves au dîner~: "Je ne peux ni confirmer ni infirmer que je possède des objets magiques dotés d'incroyables pouvoirs."

\later

Cette nuit, après être entrée dans son lit, Hermione essayait encore de prendre une décision. Sa vie avait été plus simple pendant le dîner, lorsqu'il n'y avait pas \emph{eu} de moyen efficace de trouver des brutes~; et maintenant il lui fallait à nouveau choisir, pas pour elle cette fois mais pour ses amis. Elle voyait continuellement en pensée le visage ridé de Dumbledore et la douleur qu'il n'avait pas tout à fait masquée, et elle entendait la voix de Harry qui lui disait~: "C'est leur choix, Hermione, tout comme c'est le tien."

Et sa main continuait de se remémorer la sensation de la cape contre ses doigts, la repassait encore et encore dans son esprit. Il y avait dans cette sensation quelque chose de puissant qui contraignait ses pensées à y revenir, ainsi qu'à la chanson qu'elle avait entendue / pas entendue quelque part dans son esprit et dans sa magie, une chanson maintenant redevenue silencieuse.

Harry avait parlé à la cape comme si celle-ci était une personne, il lui avait dit de bien prendre soin de Hermione. Harry avait dit que la cape avait appartenu à son père, qu'il ne pourrait pas la remplacer si elle était perdue…

Mais… Harry ne ferait pas \emph{vraiment} ça, si~?

De juste \emph{donner} l'une des trois Reliques de la Mort, créées des siècles avant Poudlard~?

Elle aurait pu dire qu'elle se sentait flattée mais cela allait \emph{loin} au-delà de la flatterie, jusqu'à lui faire se demander ce qu'elle représentait exactement aux yeux de Harry.

Peut-être Harry était-il le genre de personne qui aimait prêter d'anciens artefacts magiques perdus à \emph{tous ceux} qu'il considérait comme des amis, mais -

Mais quand elle songeait à \emph{quelle} partie de sa vie Harry disait avoir sauté, la partie où il essayait de la maintenir en sécurité…

Hermione leva les yeux vers le plafond du dortoir de Serdaigle. Quelque part à côté de son lit, Mandy et Su parlaient. Elle avait monté le niveau du sortilège de silence pour ne pas entendre les mots exacts mais elle pouvait toujours percevoir leur léger murmure~; il y avait quelque chose de réconfortant à dormir dans une pièce avec d'autres filles. Elle savait que Harry gardait son propre sortilège de silence au niveau maximum.

Elle commençait à se demander si, peut-être, Harry…

Vous savez…

S'il \emph{l'aimait} bien.

Hermione mit longtemps à s'endormir cette nuit-là.

Et lorsqu'elle se réveilla le lendemain matin il y avait un petit bout de parchemin qui dépassait de sous son oreiller et sur lequel était écrit~: \emph{À dix heures et demie tu trouveras une brute dans la quatrième coursive à gauche du couloir qui mène à la salle des potions - S.}

\later

Ce matin là, lorsque Hermione entra dans la grande salle, son estomac était rempli de papillons aussi grands que des Hippogriffes, et elle n'avait pas encore décidé ce qu'elle ferait alors même qu'elle s'approchait de la table où Serdaigle prenait son petit déjeuner.

Elle vit qu'il y avait une place vide à côté de Padma. Ce serait l'endroit où s'asseoir si elle décidait d'en parler à Padma puis demander à Padma de le dire à Daphné et à Tracey.

Hermione marcha vers la place vide à côté de celle de Padma.

Des mots attendaient dans sa gorge, \emph{Padma, j'ai reçu un message mystérieux}…

Et elle pouvait sentir un immense mur de brique à l'intérieur d'elle qui empêchait les mots de sortir. Elle mettrait Hannah, Susan et Daphné en \emph{danger}. Elle persuaderait, elle les prendrait la main et entraînerait vers le danger. Ce serait Mal.

Ou elle pourrait juste essayer de s'occuper des brutes elle-même sans rien dire à ses amis, et il était tout à fait évident que ça aussi, ce serait Mal.

Hermione savait qu'elle faisait face à un Dilemme Moral, exactement comme tous ces sorciers et ces sorcières dont elle avait lu les histoires. Seulement dans les histoires, les gens avaient toujours un \emph{bon} choix et un mauvais choix, pas deux mauvais, ce qui en l'occurrence semblait un peu injuste. Mais sans qu'elle sache comment, elle avait la sensation - peut-être cela venait-il du ton sur lequel Harry parlait toujours de la façon dont les livres d'Histoire les verraient - qu'elle faisait face à une décision Héroïque, et que toute sa vie pouvait partir dans une direction ou dans une autre selon ce qu'elle décidait là, maintenant, \emph{ce matin}.

Hermione s'assit à la table sans regarder d'un côté ni de l'autre, se contentant de fixer l'assiette et les couverts comme si des réponses avaient été cachées à l'intérieur en réfléchissant aussi vite qu'elle le pouvait, et quelques secondes plus tard elle entendit la voix de Padma chuchoter, presque au creux de son oreille~: "Daphné dit qu'elle sait où trouver une brute à dix heures et demie aujourd'hui."

\later

Foutues.

Selon Susan Bones, elles étaient toutes foutues.

Tantine lui racontait parfois des histoires qui commençaient comme ça, des histoires de gens qui faisaient quelque chose qu'ils \emph{savaient} être stupide, et les histoires finissait généralement avec quelqu'un qui se retrouvait complètement \emph{foutu} à terre, sur les murs, et sur les chaussures de Tantine.

"Hé, Padma," marmonna Parvati, sa voix à peine audible par-dessus les légers impacts de huit filles marchant sur la pointe des pieds dans le couloir qui menait à la salle de potions, "t'sais pourquoi Hermione a soupiré toute la matinée -"

"On se tait~!" siffla Lavande, le dur chuchotement bien plus puissant que le marmonnement de Parvati. "On ne sait jamais quand le Mal écoute~!"

"\emph{Shhh~!}" dirent trois autres filles encore plus fort.

Absolument, totalement, extrêmement foutues.

Alors qu'elles s'approchaient de la quatrième coursive à gauche de la salle de potions, là où l'informateur mystérieux de Daphné avait dit que les brutalisations auraient lieu, les huit filles ralentirent, le bruit de leurs pas devint plus doux, et le général Granger fit enfin le geste qui signifiait \emph{Halte, je vais voir}.

Lavande leva alors une main et, quand Hermione se fut tournée pour la regarder, Lavande, l'air perplexe, pointa un doigt vers le couloir, se désigna elle-même, puis essaya d'exprimer quelque chose que Susan ne comprit pas -

Le général Granger secoua la tête et à nouveau, cette fois avec des mouvements plus lents et plus exagérés, fit le signe pour \emph{Halte, je vais voir}.

Lavande, l'air encore plus perplexe, pointa vers l'endroit d'où elles venaient et de son autre main fit le geste de rebondir.

Maintenant tout le monde avait l'air encore plus perplexe que Lavande et Susan songea avec aigreur que manifestement, une heure de pratique deux jours plus tôt ne suffisait pas à se souvenir d'un nouvel ensemble de signaux codés.

Hermione pointa vers Lavande, puis vers le sol sur lequel elle se tenait, l'expression de son visage rendant limpide le sens voulu~: \emph{Tu. Reste. Ici.}

Lavande hocha la tête.

\emph{Doom doom doom}, les mots de la marche de la légion du Chaos revinrent à l'esprit de Susan, \emph{doom doom doom doom doom doom…} {[}NdT: jeu de mot difficile à traduire, \emph{doomed} peut vouloir dire 'foutu'{]}

Hermione fouilla dans ses robes et en sortit un petit bâton muni d'un miroir et d'un oculaire. Très doucement, oh si doucement, la Serdaigle avança vers le mur jusqu'à l'endroit où la coursive s'ouvrait vers un couloir et fit dépasser juste le bout de l'oculaire au-delà de l'angle.

Puis un peu plus.

Puis un peu plus.

Puis le général Granger fit précautionneusement passer sa tête.

Elle se retourna alors, hocha la tête et fit le geste pour \emph{suivez-moi}.

Susan se sentit un peu mieux alors qu'elle s'avançait. Apparemment, la partie du Plan qui disait qu'elles devaient arriver une demi-heure avant les brutes avait marché. Peut-être n'étaient-elles que \emph{légèrement} foutues~?

\later

À dix heures vingt-neuf, presque à l'heure pile, la brute arriva. Si quiconque avait été présent pour l'entendre - même si le couloir était apparemment vide - cette personne aurait entendu les chaussures de la brute produire un fort cliquetis dans le couloir principal, entrer dans la coursive, marcher vers l'endroit où la coursive prenait son premier tournant, passer ce tournant, puis s'arrêter, assez surprise de constater que cette coursive se terminait maintenant par un solide mur de brique là où il n'y en avait eu aucun auparavant.

Puis la brute haussa les épaules et se détourna tout en se penchant pour observer le couloir principal derrière l'angle.

Après tout, on \emph{était} à Poudlard.

Derrière les cloisons métamorphosées à la hâte dotées de l'apparence d'un mur de brique, les filles attendaient~; sans parler, sans bouger, sans presque respirer, mais en regardant à travers les trous qu'elles avaient laissé dans les cloisons.

Susan put sentir une contraction de sa poitrine jusqu'à ses orteils lorsque la brute entra dans son champ de vision. Le garçon semblait être en septième année, voir même \emph{plus vieux}, ses robes étaient brodées de vert au lieu du rouge qu'elles avaient espéré, il avait des \emph{muscles} et, après avoir regardé un peu plus longtemps, Susan se rendit compte que sa posture révélait qu'il était un \emph{duettiste}.

Puis elles entendirent toutes le son d'autres pas qui s'approchaient dans le couloir. Les Gryffondor et Serpentard de quatrième année venaient de sortir de leur cours de potions.

Le crépitement des pieds passa, diminua et s'estompa, et la brute ne fit rien. L'espace d'un instant Susan se sentit soulagée -

Puis un autre groupe de bruit de pas plus petit approcha.

La brute ne fit toujours rien et les pas s'en allèrent.

Cela se produisit plusieurs fois encore.

Puis, alors qu'approchait un dernier ensemble de pas faiblement audible, les sept filles entendirent la voix de la brute qui disait, claire, froide et basse~: "\emph{Protego}".

Quelqu'un \emph{s'étrangla} alors, mais celle-ci le fit heureusement très très doucement. Si elles ne pouvaient même pas l'atteindre -

Susan se rendit compte que les brutes apprenaient \emph{déjà}. Elle ne s'était pas attendue à ce que la S.P.E.H.S. soit capable de faire ça très souvent avant que les brutes ne comprennent - mais - Hermione avait déjà vaincu trois brutes - et la veille, l'école avait bourdonnée de spéculations quant au fantôme de Salazar Serpentard -

\emph{Il nous attend~!}

Susan aurait dû murmurer d'abandonner, d'annuler le plan, mais il n'y avait aucun moyen de faire passer le message à -

"\emph{Silencio}," dit la brute d'une voix douce et mesurée, baguette pointée vers le couloir, la brume bleue de son sortilège de bouclier chatoyant autour de lui. "\emph{Accio} victime."

Lorsque le garçon en quatrième année entra dans leur champ de vision, il pendait à l'envers, comme si une main invisible le tenait en l'air par une jambe, et ses robes bordées de rouge commençaient à glisser le long de ses cuisses et à révéler son pantalon. Sa bouche s'ouvrait et se fermait en vain~; aucun son ne sortait.

"J'imagine que tu te demandes ce qui se passe," dit le Serpentard en septième année d'une voix basse et froide. "Ne t'en fais pas. C'est tellement simple que même un Gryffondor pourrait comprendre."

Sur ce, le Serpentard serra son poing et l'envoya avec force dans le ventre du Gryffondor. Le corps du garçon convulsa désespérément mais toujours aucun mot ne sortait de sa bouche

"Tu es ma victime," dit le Serpentard plus âgé. "Je suis une brute. Je vais te battre. Et on verra si quelqu'un m'en empêche."

C'est alors que Susan comprit que c'était un piège.

Et presque au même instant sonna le puissant et aigu cri d'une jeune fille qui hurlait~: "\emph{Arrête, malfaisant~! Finite Incantatem~!}"

\emph{Lavande}, songea Susan avec angoisse. La Gryffondor s'était portée volontaire pour faire diversion pendant que les autres l'attaqueraient par derrière au moment où la brute ne s'y attendrait pas, voilà ce qu'avait été le plan, sauf que maintenant -

"Au nom de Poudlard," s'écria la voix de Lavande, même si elles ne pouvaient pas la voir, "et au nom des héroïnes de par le monde, je t'ordonne de lâcher ce \shout{Iiiiiihhh}~!"

"\emph{Expelliarmus,}" dit la brute. "\emph{Stupéfix. Accio} héroïne stupide."

Lorsque Lavande flotta jusqu'à leur champ de vision, pendue par un pied et inconsciente, Susan eut la berlue~: la fille était habillée d'une jupe or et pourpre vif et d'un chemisier à la place de ses robes de Poudlard habituelles.

La brute jetait aussi un regard étrange au corps renversé de la fille, et il dirigea alors sa baguette vers elle, dit~: "\emph{Finite Incantatem}," mais les vêtements ne changèrent pas.

Puis la brute haussa les et épaules et, toujours face à Lavande et non pas face au garçon suspendu, il ramena son poing en arrière -

"\emph{Lagann~!}" s'écrièrent cinq voix, et cinq spirales vertes jaillirent de cinq baguettes à travers cinq trous dans le faux mur, et un instant plus tard la voix de Hermione hurla~: "\emph{Stupéfix~!}"

Cinq spirales vertes se fracassèrent inutilement sur le nuage bleu et le rayon rouge de Hermione rebondit sur ce même nuage et frappa le garçon en quatrième année qui eut une convulsion puis devint immobile.

Et la brute de septième année se retourna avec un sourire lugubre tandis que les filles de première année hurlaient et chargeaient.

\later

Les yeux de Susan s'ouvrirent grand et elle roula immédiatement loin de l'endroit du sol où elle avait été étendue, ses poumons toujours en feu et son tout son corps encore douloureux comme si on l'avait frappée, la bataille n'avait progressé que de quelques secondes à ce qu'elle pouvait en voir, le corps de Hannah tombait et son bras était toujours tendu en direction de Susan, "\emph{Glisseo~!}" hurla Hermione, mais le garçon plus âgé fit un grand geste de baguette vers le sol, créant une traînée verte et lumineuse qui perturba visiblement le sortilège de Hermione et en fit une douche d'étincelles bleu-blanches, puis presque dans le même mouvement la brute dit~: "\emph{Stupéfix~!}" et Hermione fut projetée en arrière, alors Susan fit appel à toute la magie qui lui restait et elle hurla \emph{"Innerver}~!" vers le corps de Hermione alors même que la brute se tournait vers elle, baguette pointée de nouveau vers elle, puis Padma hurla "\emph{Prismatis~!}" juste avant que la brute ne hurle "\emph{Impedimental~!}", la sphère arc-en-ciel se forma \emph{autour de la brute} et le Serpentard vacilla lorsque son propre sort fut renvoyé vers lui, mais un instant plus tard la baguette de ce dernier se retourna et le toucha, et la sphère prismatique de Padma éclata alors comme une bulle de savon, tranchée par la baguette de la brute, et "\emph{Innerver}~!" hurla Parvati en direction du corps de Hannah, et Tracey et Lavande hurlèrent au même instant~: "\emph{Wingardium Leviosa~!}" -

\later

Hannah Abbott leva sa baguette d'une main qui tremblait d'épuisement, elle n'avait même plus assez de magie pour un seul \emph{Innerver} à présent.

Le reste de la coursive était silencieux, des corps étaient éparpillés au sol, Padma, Tracey, Lavande, Hermione et Parvati en tas contre un mur, Susan debout, pétrifiée à mesure qu'elle découvrait la scène avec impuissance, même le corps du Gryffondor était étalé au sol, immobile (Hermione l'avait réveillé et il s'était battu mais ça n'avait pas suffit.)

Ça avait été une très courte bataille.

La brute souriait encore et les seuls signes d'épuisement étaient une ondulation dans l'iridescence bleue qui l'entourait et quelques perles de sueur sur son front.

La brute leva son bras, essuya la sueur et avança jusqu'à elle comme un Moremplis qui aurait pris forme humaine.

Hannah se retourna et fuit, des hurlements réprimés dans sa gorge qui s'étranglait, elle sprinta dépassa les cloisons en faux mur de brique tombées au sol, elle courut le long de la coursive aussi vite qu'elle le pouvait, zigzaguant au maximum -

Juste avant qu'elle ne franchisse l'angle de la coursive, la voix de la brute dit~: "\emph{Cluthe~!"} et d'horribles crampes traversèrent ses jambes, elle tomba, glissa et frappa sa tête contre le mur, sauf qu'elle ne remarqua même pas la douleur de l'impact, car elle avait commencé à hurler à cause de ses muscles qui s'entortillaient -

La brute approchait encore, Hannah le vit lorsqu'elle tourna la tête, elle approchait lentement, toujours avec cet effrayant sourire.

Elle fit une roulade en dépit de la douleur, alors que les muscles de ses jambes se nouaient l'un autour de l'autre, elle roula jusque derrière l'angle de la coursive et cria~: "\emph{Vas-t-en~!}"

"Je ne pense pas," dit la brute, sa voix grave et effrayante, semblable à celle d'un homme adulte, très proche.

La brute passa l'angle et Daphné Greengrass enfonça sa Très Ancienne Lame directement dans son entrejambe.

Il y eut une flash de lumière qui éclaira tout le couloir -

\later

C'est en silence que les sept filles quittèrent le bureau de Madame Pomfresh, laissant l'une des leurs sur un lit d'hôpital.

La guérisseuse avait dit que Hannah irait très bien d'ici trente cinq minutes, les muscles déchirés étaient simples à rapiécer.

Daphné s'était chargée des explications et selon elle, Hannah avait eu une mésaventure avec son sortilège de course à pied, ce qui lui avait occasionné des crampes aux jambes. Madame Pomfresh leur avait jeté un regard sévère mais n'avait pas discuté, bien que le sortilège soit environ six ans au-dessus de leur niveau actuel.

Madame Pomfresh avait aussi donné à Daphné une potion pour pallier à son état d'épuisement magique absolu et l'avait prévenue de ne jeter aucun sortilège pendant les trois prochaines heures. Cet épuisement était officiellement dû à ses efforts répétés pour lancer \emph{Finite} sur les jambes de Hannah plutôt qu'à la Très Ancienne Lame qui avait utilisé toute sa magie pour pouvoir briser le \emph{Protego}.

Les autres avaient décidé de ne pas mentionner les ecchymoses sous leur robe jusqu'à avoir trouvé des aînées qui pourraient lancer \emph{Episkey}. Il y avait des limites aux talents de persuasion de Daphné.

Tout s'était joué sur trop peu de choses, songea Susan, bien trop peu. Si la brute avait ne serait-ce que jeté un coup d'œil derrière l'angle - si elle avait prit un moment pour relancer son sortilège de bouclier -

"On devrait arrêter," dit Susan dès qu'elles furent hors de portée d'oreille du bureau de la guérisseuse. "On devrait arrêter de faire ça."

Et alors, bien qu'elles aient été censées voter pour ce genre de chose, elles se tournèrent toutes vers le général Granger.

Le général Soleil ne sembla pas remarquer qu'on la regardait et continua, le regard braqué vers l'avant.

Après un moment, Hermione Granger, d'une voix un peu pensive et un peu triste, dit~: "Hannah a dit qu'\emph{elle} ne voulait pas qu'on s'arrête. Je ne sais pas si ce serait juste qu'on aie… moins de courage \emph{pour} elle qu'elle n'en a elle-même."

Toutes les autres filles hochèrent la tête, hormis Susan.

"Je pense que le pire est arrivé," dit Parvati. "Et on sait le gérer. On vient de le prouver."

Susan ne savait pas quoi répondre à cela. Elle ne pensait pas que hurler à s'en faire éclater les poumons au sujet de leur stupidité flagrante et du fait qu'elles étaient FOUTUES serait persuasif. Et elle ne pouvait pas juste \emph{quitter} les autres filles non plus. Comme si ça ne suffisait pas d'être maudit par le zèle, pourquoi fallait-il que les Poufsouffle soient \emph{loyaux} en plus de tout le reste~?

"Au fait, Lavande," dit Padma. "Par les sous-vêtements de Merlin, qu'est-ce que tu \emph{portais} là bas~?"

"Mon costume d'héroïne," dit la Gryffondor.

Daphné parla d'une voix lasse, sans tourner la tête, en continuant de marcher d'un pas laborieux~: "C'est le costume du soldat de Gryffondor dans la pièce \emph{Les Chroniques des Soldats Lunariens}."

"Tu l'as métamorphosé~?" dit Parvati d'un ton perplexe. "Mais la brute t'a lancé \emph{Finite} -"

"Nan~!" dit Lavande. "C'est \emph{réel}, j'ai juste métamorphosé mon costume en un t-shirt et une jupe normaux \emph{avant} comme ça j'avais juste à me lancer \emph{Finite} après avoir vu la brute. Tu veux avoir le tien, Parvati~? Katarina et Joshua m'ont fait le mien hier pour douze Mornilles, ils sont en sixième année -"

"Je pense," dit le général Granger avec prudence, "que cela nous donnerait à toutes l'air un peu idiotes."

"Eh bien," dit Lavande, "on devrait voter pour savoir si on -"

"Je pense," dit le général Granger, "que peu importe ce que \emph{vous} votez, je préférerais mourir plutôt que d'être vue avec un de ces costumes sur le dos -"

Susan ignora son argument. Elle essayait de trouver une technique pour qu'elles soient moins foutues.

\later

Toute la grande salle se tut, même si ce ne fut que pour un instant, lorsque les sept filles entrèrent pour le déjeuner.

Puis les applaudissements commencèrent.

Ils étaient épars, ce n'était pas un applaudissement massif au démarrage synchronisé. Nombre d'entre eux venaient de la table Gryffondor, moins de Poufsouffle et Serdaigle et aucun de Serpentard.

Daphné sentit son visage se pincer. Elle avait \emph{espéré} - enfin, peut-être qu'après qu'elles aient trouvé une brute Gryffondor à arrêter et un Serpentard à sauver, ses camarades Serpentard se rendraient compte -

Elle regarda la table Poufsouffle.

Neville Londubat applaudissait mains haut au-dessus de la tête mais il ne souriait pas. Peut-être qu'il avait entendu pour Hannah, ou peut-être qu'il se demandait pourquoi Hannah n'était pas là.

Puis, pas tout à fait capable de s'en empêcher, elle jeta un coup d'œil à la table d'honneur.

Le visage du professeur Chourave était ridé d'inquiétude. Elle et le professeur McGonagall penchaient leur tête vers le directeur, qui avait un air solennel, et leurs lèvres bougeaient rapidement. Le professeur Flitwick avait l'air plus résigné qu'autre chose et Quirrell, visage flasque, buvait sa soupe à cuillerées tremblantes tenues par un poing serré.

Le professeur Rogue avait les yeux braqués sur -

\emph{Elle~?}

Ou - Hermione Granger, juste à côté~?

Un fin et discret sourire passa sur le visage du maître des potions, il leva les mains, les rapprocha l'une de l'autre dans un geste qui était trop lent pour être un véritable applaudissement, puis il revint à son assiette en ignorant les conversations qui l'entouraient.

Daphné sentit un petit frisson descendre le long de son échine et elle se dirigea hâtivement vers la table Serpentard. Susan, Lavande et Parvati se détachèrent du groupe et se rendirent aux tables Poufsouffle et Gryffondor, de l'autre côté de la grande salle.

L'événement eut lieu lorsqu'elles passèrent la partie de la table Serpentard où s'asseyait leur équipe de Quidditch.

C'est à ce moment que Hermione trébucha soudain, \emph{avec force}, comme si on lui avait balayé les pieds, elle s'affala dans l'espace entre Marcus Flint et Lucian Bole et il y eut un triste petit plouf lorsque son visage atterrit dans l'assiette de steak et de purée de pommes de terre de Flint.

Tout sembla alors avoir lieu bien trop vite, ou peut-être que c'était Daphné qui pensait trop lentement, quand Flint laissa échapper un mugissement indigné, repoussa Hermione d'une main et la jeta vers la table Serdaigle où elle rebondit contre le dos d'un étudiant et alla s'effondrer sur le sol -

Le silence se répandit par vagues.

Hermione se releva en poussant sur le sol mais ne se remit pas tout à fait sur pied, Daphné pouvait voir que tout son corps tremblait, que son visage était encore couvert de purée et de morceaux de steak épars.

Pendant un long moment, personne ne parla, personne ne bougea. Comme si personne dans la grande salle ne pouvait imaginer plus que Daphné ce qui allait se produire ensuite.

Puis la puissante voix de Flint, la voix du capitaine Serpentard qui beuglait des ordres sur le terrain de Quidditch, dit avec un grondement menaçant~: "Tu as gâché ma nourriture, fillette."

Un autre moment de silence figé. La tête de Hermione - Daphné pouvait la voir trembler - se retourna pour regarder le capitaine de Quidditch de Serpentard.

"Présente-moi des excuses," dit Flint.

Harry Potter commença à se lever de la table Serdaigle puis s'arrêta soudain à mi-parcours, comme s'il venait de penser à quelque chose -

Puis cinq autres personnes se levèrent à la table Serdaigle.

Toutes l'équipe de Quidditch de Serpentard se leva, leurs baguettes apparurent dans leurs mains, et les élèves se levèrent alors à la table Gryffondor et à la table Poufsouffle, et sans réfléchir Daphné regarda la table d'honneur et vit que le directeur était toujours assis et qu'il regardait, qu'il ne faisait que regarder, Dumbledore \emph{se contentait de regarder}, une main ouverte comme pour retenir le professeur McGonagall - dans une seconde quelqu'un allait crier un sortilège et il serait trop tard, \emph{pourquoi le directeur ne faisait-il rien -}

Et une voix dit~: "Je vous présente mes excuses."

Daphné se retourna, abasourdie, la bouche béante.

"\emph{Récurvite}" dit la voix mielleuse, et la purée disparut du visage de Hermione, révélant une expression de surprise sur le visage de la Serdaigle lorsque Drago Malfoy s'approcha, rengaina sa baguette puis posa un genoux à terre et lui offrit sa main.

"Navré pour tout ceci, Mlle Granger," dit la voix polie de Drago Malfoy. "Je suppose que quelqu'un s'est crû drôle."

Hermione prit la main de Drago et Daphné comprit soudain ce qui allait se produire -

Mais Drago Malfoy \emph{ne laissa pas} Hermione se lever à moitié avant de la laisser retomber par terre.

Il la mit juste sur pieds.

"Merci," dit Hermione.

"De rien," dit Drago d'une voix forte, sans regarder autour de lui, où les quatre maisons de Poudlard le regardaient dans un état de stupéfaction absolue. "Souvenez-vous juste qu'être rusé et ambitieux ne signifie pas qu'on se comporte forcément d'une façon pareille."

Puis il revint à sa place sur le banc Serpentard et s'assit comme s'il ne venait pas - comme s'il ne venait pas juste - \emph{juste de} -

Hermione se rendit à la place libre la plus proche sur le banc Serdaigle et s'assit.

Plusieurs personnes, assez lentement, se rassirent.

"Daphné~?" dit Tracey. "Tu vas bien~?"

\later

Le cœur de Drago tambourinait dans sa poitrine si fort qu'il se demandait si celui-ci n'allait pas exploser dans une gerbe de sang, comme avec ce sortilège qu'Amycus Carrow avait un jour utilisé sur un bébé chien.

Le visage de Drago demeura parfaitement neutre car il savait (on le lui avait fait rentrer dans le crâne encore et encore) que s'il montrait le moindre signe de ce qu'il ressentait à ses camarades, ceux-ci le réduiraient en miettes comme un essaim d'Acromentules.

Il n'avait pas eu le temps de vérifier auprès de Harry Potter, de planifier, juste l'instant où il s'était rendu compte que le moment de commencer à sauver la réputation de Serpentard était \emph{maintenant}.

De tous côté de la longue table Serpentard, des visages en colère regardaient Drago.

Mais les visage perplexes étaient en surnombre.

"D'accord, j'abandonne," dit un garçon de sixième année que Drago ne reconnut pas, assit en face de lui, deux places sur la droite. "Pourquoi tu as fait ça, Drago~?"

Même si sa bouche était très sèche, Drago ne déglutit pas. Cela aurait été un signe de peur. Au lieu de cela, il prit une bouchée de carottes, qui étaient l'aliment le plus humide sa son assiette, mâcha et avala tout en réfléchissant le plus rapidement possible.

"Tu sais," dit Drago, d'une voix aussi tranchante qu'il en était capable - alors que son cœur battait encore plus fort dans sa poitrine, alors que tout le monde s'était arrêté de parler et l'écoutait - "il y a \emph{probablement} un moyen de donner encore plus mauvaise réputation à Serpentard qu'en attaquant huit filles de première années des quatre maisons qui travaillent ensemble pour arrêter les brutalisations, mais je ne le \emph{trouve} pas. Là, on tire bénéfice des agissements de Daphné."

Les visages qui étaient perplexes le demeurèrent.

"Quoi~?" dit le garçon en sixième année, et "Attends, \emph{quel} bénéfice~?" dit une fille en cinquième année assise à sa gauche.

"Ça donne une bonne réputation à Serpentard," dit Drago.

Les Serpentard autour de lui jetaient des regards mi-interloqués mi-moqueurs, comme s'il venait d'essayer de leur expliquer ce qu'était l'algèbre.

"Auprès de \emph{qui} ?" dit le garçon en sixième année.

"Mais tu viens d'aider une \emph{Sang-de-Bourbe}," dit la fille en cinquième année. "Comment est-ce que \emph{ça} c'est censé nous donner une bonne réputation~?"

La gorge de Drago se clôt. Son cerveau vivait un hideux dysfonctionnement qui l'empêchait de trouver autre chose à répondre que la vérité -

Puis~: "C'est probablement une sorte de plan incroyablement malin que Malfoy prépare," dit un garçon en cinquième année. "Vous savez, comme dans \emph{La Tragédie de Light}, où tous les revers font en fait partie du plan. Et ça finit avec la tête de Granger au bout d'une pique et personne pour le soupçonner."

"\emph{Ça}, ça se tient," dit quelqu'un plus loin vers l'autre bout de la table, et il y eut de nombreux hochements de tête.

\later

"Est-ce que \emph{tu} sais ce que le boss prépare~?" marmonna Vincent à voix basse.

Gregory Goyle ne répondit pas. Dans son esprit, il pouvait entendre très distinctement la voix de son maître qui disait~: \emph{Je n'arrive pas à croire que je crois à tout ça}, le jour où la rumeur disant que Salazar Serpentard montrait à Potter et à Granger où trouver des brutes avait commencé.

"M. Goyle~?" chuchota Vincent.

Les lèvres de Gregory Goyle formèrent les mots~: \emph{Oh, non}, mais aucun son ne sortit.

\later

Hermione était partie en avance du déjeuner ce jour là, pour une raison ou une autre elle n'avait pas eu faim. Ces quelques secondes d'horrible humiliations avaient continué de brûler dans son esprit, encore et encore, la sensation de son visage pressé contre la purée, puis d'être envoyée en l'air, puis la voix du Serpentard qui disait~: "Présente-moi tes excuses"… ça avait peut-être été la première fois de sa vie qu'elle avait eu envie de \emph{haïr} quelqu'un. Le garçon qui l'avait projetée (on lui disait qu'il s'appelait Marcus Flint), et aussi cette personne qui avait lancé le sortilège de trébuchement… elle avait sentit, pendant un moment horrible, qu'elle voulait dire à Harry que s'il commençait à être \emph{créatif} pour son compte, elle n'y opposerait aucune objection.

Elle n'était pas sortie de la grande salle depuis plus d'une minute lorsqu'elle entendit le bruit de pieds qui couraient derrière elle, elle se retourna et vit Daphné qui fonçait dans sa direction.

Et elle écouta ce que son soldat Soleil avait à dire…

"Tu ne \emph{comprends pas}~?" dit la voix de Daphné qui était à peine au-dessous d'un glapissement. "Ce n'est pas parce que quelqu'un est gentil avec toi qu'il veut être ton ami~! C'est \emph{Drago Malfoy~!} Son père est un Mangemort, tous les parents de tous ses amis sont des Mangemorts - Nott, Goyle, Crabbe, \emph{tous les gens qui l'entourent}, tu comprends~? Ils méprisent \emph{tous} les nés-Moldus, ils veulent que les gens comme toi \emph{meurent}, ils pensent que tu n'es bonne qu'à être \emph{sacrifiée} pour d'horribles rituels noirs~! Drago est le \emph{prochain Lord Malfoy}, il a été éduqué depuis la naissance pour te haïr et pour \emph{mentir}~!" Les yeux gris-vert de Daphné demandaient farouchement un assentiment ou le signe d'une compréhension.

"Il -" dit Hermione d'une voix hésitante. Elle se souvint du toit, de l'horrible secousse lorsqu'elle avait commencé à tomber, de la main de Drago Malfoy qui avait attrapé la sienne et qui l'avait serrée si fort qu'elle en avait ensuite eu des ecchymoses. Elle avait dû lui demander deux fois avant qu'il ne la laisse enfin tomber. "Peut-être que Drago Malfoy n'est pas comme eux -"

Le chuchotement de Daphné était presque un cri. "S'il ne finit \emph{pas} par t'infliger dix fois pire que ce qu'il a fait pour t'aider, sa \emph{vie} est finie, tu comprends~? Je veux dire que Lucius Malfoy le \emph{déshériterait}, au sens propre~! T'sais quelles sont les chances qu'il ne soit \emph{pas} en train de préparer quelque chose~?"

"Faibles~?" dit Hermione d'une petite voix.

"\emph{Zéro~!}" siffla Daphné. "Je veux dire \emph{aucune}~! Je veux dire \emph{moins} que zéro~! Je veux dire que les chances sont tellement faibles que tu ne les trouverais pas avec trois sortilèges d'agrandissement et un sortilège de Montre-Moi et - et - une ancienne carte et un prophète centaure~! Tout le monde à Serpentard sait qu'il fomente quelque chose contre toi et qu'il ne veut pas qu'on le soupçonne, j'ai entendu quelqu'un dire qu'il avait dirigé sa baguette vers toi juste avant que tu ne trébuches - tu ne comprends pas~? \emph{Tout ça fait partie du plan de Malfoy~!}"

\later

Drago s'assit et mangea son steak au chou-fleur rôti et à la sauce Ashwinder (qui n'était pas faite à partir de vrais œufs Ashwinder et avait exactement le goût du feu) en essayant de ne pas rire et de ne pas pleurer non plus.

Il avait \emph{entendu parler} du concept de déni plausible mais il n'avait pas compris à quel point c'était important jusqu'au jour où il avait découvert que les Malfoys n'en avaient pas du tout.

"Tu veux connaître mon plan~?" dit Drago. "\emph{Voilà} mon plan. Je ne vais \emph{rien faire} cette fois et comme ça la \emph{prochaine} fois que les gens penseront que je fomente quelque chose, ils n'en seront pas certains."

"Euh…" dit le garçon en cinquième année. "Je ne pense pas que je te crois, ça n'a vraiment pas l'air assez fourbe pour être ça -"

"C'est ce qu'il \emph{veut} que tu penses," dit la fille en cinquième année.

\later

"Albus," dit Minerva d'un ton menaçant, "avez-vous \emph{planifié} tout cela~?"

\later

"Allons, si \emph{j'avais} claqué des doigts sous la table, je ne te le \emph{dirais} pas -"

\later

La main tremblante du professeur de Défense fit retomber sa cuillère dans sa soupe.

\later

"Comment ça, \emph{on vous a piégées~?}" dit Millicent. Elles étaient toutes les deux assises sur le lit de Daphné après être venues là directement après le déjeuner dans la grande salle. "Avec mes yeux de voyante capables de voir à travers le Temps lui-même, je vous ai vues \emph{gagner}."

Daphné fixa Millicent de ses yeux de simples mortels qui se trouvaient être fort plissés~: "Ce garçon nous \emph{attendait}."

"Ben ouais~!" dit Millicent. "Tout le monde sait que vous faites la chasse aux brutes~!"

"Hannah s'est prise un sortilège vraiment douloureux en plein visage," dit Daphné. "Elle a dû aller voir un guérisseur, Millicent~! En tant qu'amie, tu aurais dû me \emph{prévenir} !"

"Écoute, Daphné, je te l'ai \emph{dit}, -" la Serpentard s'interrompit, comme si elle essayait de se souvenir de quelque chose, puis elle dit~: "je veux dire, je t'ai dit que ce que je vois \emph{doit} se produire. Si j'essaie de le changer, si \emph{quiconque} essaie de le changer, des choses vraiment terribles, atroces, pas bien, extrêmement mauvaises auront lieu. Et ensuite ça se produira \emph{quand même}. Si je vois que tu vas te faire frapper, je ne \emph{peux pas} te le dire parce que tu \emph{essaieras} de l'éviter, et \emph{alors} -" Millicent se tut.

"Et alors~?" dit Daphné d'un ton sceptique. "Par exemple, qu'est-ce qui se passe si je n'y vais tout simplement pas~?"

"Je ne \emph{sais} pas~!" dit Millicent. "Mais comparé à ça, se faire manger par des Moremplis ressemble probablement à une partie de plaisir~!"

"Écoute, même moi je sais que les prophéties ne marchent pas comme ça," dit Daphné, puis elle marqua une pause. "En tout cas elles ne marchent pas comme ça dans les pièces de théâtre…" Il fallait convenir qu'il existait toutes sortes de tragédies ou essayer d'éviter la prophétie la \emph{déclenchait}, ou d'autres où au contraire, le fait d'essayer de les \emph{suivre} était la seule raison qui les faisait se produire. Mais il était \emph{possible} de s'arranger pour que les prophéties se déroulent à votre avantage si vous étiez assez malin et si quelqu'un de suffisamment amoureux de vous pouvait prendre votre place, et avec assez d'efforts il était possible de carrément briser une prophétie… mais après tout, les voyantes des pièces ne se souvenaient jamais de ce qu'elles avaient vues…

Millicent dut remarquer l'hésitation de Daphné car elle commença à prendre un peu d'assurance. "Eh bien," dit Millicent d'une voix sèche, "ce n'est pas une pièce~! Écoute, je te dirai si je vois une bataille facile ou une bataille difficile. Mais c'est \emph{tout} ce que je peux faire, tu comprends~? Et si je dis 'difficile', tu ne \emph{peux pas} être absente~! Ou - ou -" les yeux de Millicent remontèrent dans ses orbites et elle entonna d'une voix creuse~: "\emph{Ceux qui tentent de tromper leur destin connaîtrons des tristes et sombres fins -"}

\later

Le professeur Chourave secoua la tête, le visage pincé.

"Mais -" dit Susan. "Mais vous avez aidé \emph{Harry Potter} cette fois-là -"

"Et l'on m'a \emph{clairement} fait comprendre," dit le professeur Chourave d'une voix qui donnait l'impression que quelqu'un utilisait un sortilège de rétrécissement pour lui serrer la gorge, "que c'était le travail du professeur Rogue et pas le mien de faire régner l'ordre dans la maison Serpentard - mademoiselle Bones, \emph{s'il vous plaît}, vous n'avez pas à \emph{faire} ça si -"

"Si, je \emph{dois} le faire," dit Susan avec tristesse. "Je suis une Poufsouffle, on doit être loyaux."

\later

"Un parchemin mystérieux sous ton oreiller~?" dit Harry Potter en relevant les yeux dans le renfoncement où ils étaient assis et étudiaient. Les yeux verts du garçon se plissèrent. "Il ne venait pas du père noël~?"

Silence.

"Alors," dit Hermione, "je ne vais \emph{pas} te poser de question, tu ne vas \emph{rien} me dire et on va \emph{tous les deux} faire comme si tu n'avais jamais dit ça et que je n'étais pas au courant -"

\later

Susan s'approcha de la table dès que la fille plus âgée se retrouva seule et jeta un regard autour d'elle pour vérifier que personne dans la salle commune Poufsouffle ne regardait (comme sa tante le lui avait enseigné, de façon à ce qu'il ne soit pas évident qu'elle vérifiait).

"Hé, Susie," dit la Poufsouffle en septième année. "Est-ce que tu as déjà besoin de plus de -"

"Est-ce qu'on pourrait parler en privé un moment~?" dit Susan.

\later

Jaime Astorga, Serpentard en septième année et jusqu'à récemment considéré comme un petit arriviste prometteur sur la scène des duettistes junior se tenait droit comme un piquet dans le bureau du professeur Rogue, dents serrées très fort et de la sueur dans le dos.

"Je me souviens clairement," dit le directeur de sa maison d'une voix traînante et pleine d'ironie, "vous avoir prévenu, ainsi que nombre d'entre vous ce matin même, qu'il y avait certaines filles de première année qui pourraient s'avérer agaçantes si un combattant se montrait \emph{imprudent} et se permettait d'être \emph{pris par surprise}."

Le professeur Rogue déambulait lentement autour de lui.

"Je -" dit Jaime alors que de la sueur perlait sur son front. Il savait à quel point c'était ridicule, à quel point c'était une excuse pathétique. "Monsieur, elles n'auraient pas dû être capables de -" une fille de première année n'aurait pas dû être capable de briser son \emph{Protego}, peu importe le sortilège ancien qu'elle maniait - Greengrass avait dû recevoir de \emph{l'aide} -

Mais il était très clair que son directeur de maison n'allait pas croire ça.

"Oh, je suis tout à fait d'accord," murmura Rogue d'un ton bas, emplit de menace. "Elles n'auraient pas dû. Je commence à me demander si M. Malfoy, quelles que soient ses manigances, n'a pas raison, Astorga. Ça ne peut qu'être mauvais pour la réputation de Serpentard si nos combattants, plutôt que de faire montre de leur force, perdent face à de petites filles~!" la voix de Rogue avait monté d'un ton. "Il est heureux que vous ayez eu le bon goût d'être vaincu par une petite fille elle-même venue de Serpentard, sans quoi je vous aurais moi-même enlevé des points~!"

Les poings de Jaime Astorga se contractèrent contre ses hanches mais il ne trouva rien à répondre.

Il fallut longtemps avant que Jaime Astorga soit autorisé à être excusé auprès de son directeur de maison.

Et ensuite, seuls les murs, le plancher et le plafond furent témoin du sourire de Severus Rogue.

\later

Ce soir là, Drago reçut la visite de la chouette de son père, Tanaxu, qui n'était pas verte uniquement parce que les chouettes vertes n'existaient pas. Le mieux que Père avait pu trouver était une chouette aux ailes de l'argent le plus pur, aux yeux verts étincelants et au bec aussi pointu et cruel que les crocs d'un serpent. Le parchemin enroulé autour de la jambe de Tanaxu était court et allait droit au but~:

\emph{Mon fils, que fais-tu~?}

Le parchemin que Drago renvoya était tout aussi court et disait~:

\emph{J'essaie de mettre fin au mal commis contre la réputation de Serpentard, père.}

En autant de temps qu'il fallait à une chouette pour voler de Poudlard au manoir Malfoy puis d'en revenir, la chouette familiale porta un nouveau message à Drago, et celui-ci disait seulement~:

\emph{Que fais-tu vraiment~?}

Drago regarda le parchemin qu'il avait déroulé de la jambe de la chouette. Ses mains tremblaient alors qu'il le tenait à la lumière de son feu de cheminée. Quatre mots gravés d'une encre noire n'auraient pas dû être plus effrayants que la mort.

Il n'avait pas beaucoup de temps pour réfléchir. Père savait exactement combien de temps un message mettait à aller du manoir à Poudlard et à en revenir~; il saurait si Drago avait différé sa réponse afin de rédiger un mensonge méticuleux.

Mais Drago attendit encore jusqu'à ce que sa main cesse de trembler avant d'écrire sa réponse, la seule réponse dont il pensait que Père pourrait l'accepter~:

\emph{Je me prépare à la prochaine guerre.}

Drago enroula le parchemin autour de la jambe de la chouette, l'attacha, et envoya Tanaxu voler hors de sa chambre à travers les couloirs de Poudlard et dans la nuit.

Il attendit, mais aucune réponse ne vint.

%  LocalWords:  Gleep Gottschalk’s Shhh eek Episkey Lunarian
%  LocalWords:  splutching Tanaxu Tanaxu’s
