\chapter{Accomplissement de soi, Épilogues~: Apparences superficielles}

\section{Après-coup~: Albus Dumbledore et…}

\lettrine{L}{e} vieux sorcier était assis seul à son bureau dans le non-silence de la pièce, entouré de ses appareils, innombrables et inaperçus~; sa robe était d'un jaune et d'un tissu doux, différente des habits qu'il portait habituellement devant les autres. Sa main ridée tenait une plume et grattait un parchemin d'apparence officielle. Si vous vous étiez débrouillé pour être là, pour regarder son visage ridé, vous n'auriez pas été capable d'en déduire plus au sujet de l'homme que ce que vous auriez compris des appareils énigmatiques qui l'entouraient. Peut-être auriez-vous remarqué que le visage semblait un peu triste et un peu fatigué, mais après tout, Albus Dumbledore avait toujours l'air ainsi lorsqu'il était seul.

Dans le foyer de cheminette ne se trouvaient que des cendres éparpillées et pas une trace de flamme~: une porte magique si solidement fermée qu'elle en avait cessé d'exister. Sur le plan matériel, la grande porte de chêne du bureau avait été fermée à double tour~; au-delà, les escaliers infinis demeuraient immobiles~; au pied de ceux-ci, les gargouilles qui bloquaient l'entrée ne coulaient pas, leur pseudo-vie ayant été reprise à la pierre.

Alors, au moment où la plume était au milieu d'un mot, au milieu même d'une lettre -

Le vieux sorcier bondit sur ses pieds à une vitesse qui aurait abasourdi n'importe quel observateur, abandonnant la plume à mi-lettre, la laissant tomber sur le parchemin~; vif comme l'éclair il pivota vers la porte en chêne, sa robe jaune tournoyant autour de lui et une baguette au pouvoir redouté jaillissant dans sa main -

Et tout aussi abruptement, le vieux sorcier se figea, arrêtant son mouvement au moment même où la baguette touchait sa main.

Par trois fois, une main frappa à la porte de chêne.

À présent plus lentement, l'effroyable baguette retourna dans son étui de duel accroché sous la manche du vieux sorcier. Le vieil homme avança de quelques pas, prit une posture plus formelle et ajusta l'expression de son visage. Non loin, sur le bureau, la plume se plaça à côté du parchemin, comme si elle avait précautionneusement été placée là plutôt que jetée avec hâte, et le parchemin lui-même se retourna pour ne plus montrer qu'un fond vierge.

D'un mouvement sec et silencieux de sa volonté, la porte de chêne s'ouvrit grand.

Aussi durs que la pierre, les yeux verts le regardaient avec furie.

"J'admets être impressionné, Harry," dit le vieux sorcier à voix basse. "La Cape d'Invisibilité t'aurait permis d'échapper à mes moyens d'observations les plus faibles~; mais je n'ai pas senti mes golems s'écarter ni les escaliers tourner. Comment es-tu venu ici~?"

Le garçon entra dans le bureau, chaque pas délibéré, étudié, jusqu'à ce que la porte se referme doucement derrière lui. "Je peux aller où bon me semble, avec ou sans votre permission," dit le garçon. Sa voix semblait calme~; trop calme, peut-être. "Je suis dans votre bureau car j'ai décidé de m'y rendre, et au diable les mots de passe. Vous auriez bien tort, M. le directeur, de penser que je ne reste dans cette école que parce que j'y suis prisonnier. Je n'ai simplement pas \emph{encore} choisi de partir. En gardant cela à l'esprit, pourquoi avez-vous ordonné à votre agent, le professeur Rogue, de briser l'accord que nous avions passé dans ce bureau, à savoir qu'il ne tourmenterait plus d'élèves avant leur cinquième année~?"

Le vieux sorcier regarda le jeune héros en colère pendant un long moment. Puis, assez lentement pour ne pas alarmer le garçon, ses doigts flétris ouvrirent l'un des nombreux tiroirs du bureau, en sortirent une feuille de parchemin et l'étalèrent sur le bureau. "Quatorze," dit le vieux sorcier. "Ce n'est pas le nombre de chouettes envoyées la nuit dernière. Seulement celles envoyées aux familles qui possèdent un siège au Magenmagot, une grande fortune, ou qui sont déjà alliées à tes ennemis. Ou, dans le cas de Robert Jugson, les trois à la fois, car son père Lord Jugson est un Mangemort et son grand-père un Mangemort tombé sous la baguette d'Alastor Maugrey. Ce que les lettres disent, je l'ignore, mais je peux le deviner. Ne comprends-tu \emph{toujours} pas, Harry~? Chaque fois que Hermione Granger \emph{gagnait}, comme tu le dis, la menace de Serpentard sur elle augmentait, encore et encore. Mais aujourd'hui ceux de Serpentard ont triomphé d'elle, facilement et sans encombre, sans violence ni dommages durables. Ils ont gagné et ils n'ont plus besoin de se battre…" le vieux sorcier soupira. "C'est du moins ce que j'avais prévu. Ce que j'avais espéré. Et il en aurait été ainsi si le professeur de Défense n'avait pas choisi d'intervenir. La dispute remonte donc maintenant au Conseil des Gouverneurs, ou Severus semblera vaincre le professeur de Défense~; mais ce ne sera pas la même chose pour les Serpentard, ça ne sera pas terminé, achevé en un instant à leur entière satisfaction."

Le garçon avança d'un pas de plus dans la pièce, sa tête se penchant un peu plus en arrière pour faire face aux lunettes en demi-lune, et pourtant il semblait que c'était le garçon qui toisait le directeur plutôt que le contraire. "Donc ce Lord Jugson est un Mangemort~?" dit le garçon d'une voix douce. "Bien. Sa vie nous appartient donc déjà, et je peux en faire ce que je veux sans avoir de problème éthique -

--- \emph{Harry~!}"

La voix du garçon était aussi claire qu'un bloc de glace faite de l'eau la plus pure venue de quelque source reculée. "Vous semblez penser que la Lumière devrait vivre dans la peur des Ténèbres. Je dis que ça devrait être le contraire. Je préférerais ne pas tuer ce Lord Jugson même s'il est un Mangemort. Mais une heure de réflexion avec le professeur de Défense devrait amplement suffire à trouver un moyen imaginatif de le ruiner financièrement ou de le faire exiler d'Angleterre magique. Je pense que cela aiderait à clarifier les choses.

--- Je confesse," dit lentement le vieux sorcier, "que l'idée de ruiner une maison vieille de cinq-cents ans et de provoquer un Mangemort dans une guerre à outrance à cause d'une bagarre dans les couloirs de Poudlard ne m'était pas venue, Harry." Le vieux sorcier leva un doigt pour remonter ses lunettes en demi-lunes depuis le bout de son nez, où elles avaient glissé lors de son mouvement rapide un peu plus tôt. "J'ose avancer qu'elle ne viendrait pas non plus à Mlle Granger, pas plus qu'au professeur McGonagall ou à Fred et à George."

Le garçon haussa les épaules. "Ce ne serait pas \emph{à propos} des couloirs," dit le garçon. "Ce ne serait que justice pour ses crimes passés, et je ne le ferais que si Jugson jouait le premier coup. L'idée n'est pas de m'utiliser comme un joker qui inspire la crainte, après tout. C'est d'enseigner aux gens que les neutres sont parfaitement à l'abri de mes coups et qu'il est incroyablement dangereux de tester mes limites." Le garçon eut un sourire qui n'atteignit pas ses yeux. "Peut-être que j'achèterai un encart publicitaire dans la Gazette du Sorcier pour dire que tous ceux qui veulent poursuivre cette querelle avec moi apprendront le véritable sens du mot Chaos mais que tous ceux qui me laissent tranquille n'auront pas de problème.

--- \emph{Non}," dit le vieux sorcier. Sa voix était à présent plus grave et révélait une partie de son âge et de son véritable pouvoir. "Non, Harry, il ne peut en être ainsi. Tu n'as pas encore appris le sens du combat, ce qui se passe réellement lorsque des ennemis se rencontrent sur le champ de bataille. Et ainsi tu rêves, comme tous les jeunes garçons le font, d'apprendre à tes ennemis à te craindre. Cela m'effraie, car tu as peut-être déjà, bien trop jeune, assez de pouvoir pour transformer une partie de ces rêves en réalité. Sur cette route, il n'est \emph{pas} d'embranchement qui ne mène pas aux ténèbres, Harry, aucun. C'est le chemin du Seigneur des Ténèbres, c'est certain."

Le garçon hésita alors et ses yeux passèrent brièvement sur le perchoir vide où Fumseck reposait parfois ses ailes. C'était un mouvement que peu auraient perçu, mais le vieux sorcier le connaissait très bien.

"D'accord, oubliez la partie où je leur apprends à me craindre," dit alors le garçon. Sa voix n'en était pas moins dure, mais une partie du froid l'avait quittée. "Je ne pense toujours pas que vous devriez laisser les enfants se faire brutaliser de peur de ce que Lord Jugson \emph{pourrait} faire. Les protéger constitue l'essentiel de votre métier. Si Lord Jugson essaie vraiment de vous barrer la route, alors faites tout ce qu'il faut faire pour l'arrêter. Donnez-moi un accès complet à mes coffres et j'endosserai \emph{personnellement} toute responsabilité liée aux retombées qu'il pourrait y avoir à bannir les brutes de Poudlard, qu'elles viennent de Lord Jugson ou de qui que ce soit d'autre."

Lentement, le vieux sorcier secoua la tête. "Harry, il semble que tu crois que je n'aie qu'à employer tout mon pouvoir et que tous les ennemis seront alors balayés. Tu as tort. Lucius Malfoy contrôle Fudge, influence toute l'Angleterre par le biais de la Gazette du Sorcier, et ce n'est que par une très faible marge qu'il ne contrôle pas assez le conseil des gouverneurs pour m'exclure de Poudlard. Amelia Bones et Bartemius Crouch sont des alliés, mais même eux s'écarteraient s'ils nous voyaient agir avec inconséquence. Le monde qui nous entoure est plus fragile que ce que tu sembles croire, et il nous faut avancer avec plus de prudence. La vieille guerre des sorciers n'a jamais pris fin, Harry, elle n'a fait que continuer sous une forme différente~; le roi noir dormait et Lucius Malfoy a déplacé ses pièces pendant un moment. Penses-tu que Lucius te permettrait de lui prendre un de ses pions sans réagir~?"

Le garçon sourit, et une partie de la froideur revint. "D'accord, je trouverai un moyen d'arranger les choses pour que Lord Jugson ait l'air d'avoir trahi son propre camp.

--- Harry -

--- Les obstacles ne sont qu'un appel à la \emph{créativité}, M. le directeur. Ils n'exigent pas que vous abandonniez les enfants que vous étiez censé protéger. Faisons gagner le côté clair, et si cela provoque des problèmes…" le garçon haussa les épaules. "Faisons-le gagner à nouveau.

--- Ainsi parleraient les phénix s'ils en étaient capables," dit le vieux sorcier. "Mais tu ne comprends pas le \emph{prix du phénix}."

Les deux derniers mots furent prononcés d'une voix particulièrement distincte qui sembla résonner dans le bureau, et un immense grondement apparut alors tout autour d'eux.

Entre l'ancien bouclier accroché au mur et le porte-Choixpeau, la pierre des murs commença à couler, à bouger, à se verser dans deux colonnes et à révéler une ouverture entre elles, un passage, qui laissait voir un escalier de pierre qui menait vers le haut, vers les ténèbres.

Le vieux sorcier se détourna et monta ces escaliers, puis il se retourna vers l'endroit où Harry Potter se tenait. "Viens~!" dit le vieux sorcier. Ses yeux ne pétillaient plus. "Puisque tu es déjà allé jusqu'à forcer ton entrée sans être invité, autant que tu ailles un peu plus loin."

\later

Il n'y avait pas de rambarde sur ces escaliers de pierre, et au bout de quelques pas, Harry sortit sa baguette et lança un \emph{Lumos}. Le directeur ne regarda pas en arrière et ne semblait pas regarder ses pieds, comme s'il avait monté ses escaliers assez souvent pour ne pas avoir besoin de les voir.

Le garçon savait qu'il aurait dû être curieux ou effrayé, mais il n'avait plus assez de capacité mentale disponible. Il lui fallait tout son contrôle de lui-même pour ne pas laisser la furie qui frémissait en lui déborder plus qu'elle ne l'avait déjà fait.

Les escaliers continuèrent de monter sur une courte distance, une volée droite sans courbe ni angle.

Au sommet se trouvait une épaisse porte de métal, noire sous la lumière bleue de la baguette de Harry, ce qui signifiait que le métal lui-même était soit noir soit peut-être rouge.

Albus Dumbledore leva sa longue baguette comme on aurait brandi un symbole et parla de nouveau de cette étrange voix qui sembla résonner dans les oreilles de Harry, comme si elle se gravait au fer rouge dans sa mémoire~: "\emph{Destin du phénix}".

La dernière porte s'ouvrit et Harry entra derrière Dumbledore.

La pièce semblait être faite d'un métal noir similaire à celui de la porte qui y menait. Les murs étaient noirs, le sol était noir. Le plafond au-dessus était noir à l'exception d'un unique globe de cristal qui pendait de celui-ci, accroché à une chaîne blanche qui brillait d'une lueur argentée qu'on aurait dit destinée à imiter la lumière d'un Patronus, même si on pouvait voir que ce n'était pas tout à fait ça.

Dans la pièce se trouvaient des piédestaux de métal noir et chacun portait soit une image mouvante, soit un cylindre à moitié empli d'un liquide argenté légèrement brillant, soit un petit objet isolé~: un collier d'argent brûlé, un chapeau écrasé, une bague de mariage en or, neuve. De nombreux piédestaux portaient les trois à la fois, l'image mouvante, le liquide argenté et l'objet. Il semblait y avoir de nombreuses baguettes de sorciers sur ces piédestaux, et nombre de ces baguettes étaient brisées, brûlées, ou semblaient avoir fondu.

Il fallut tout ce temps pour que Harry comprenne ce qu'il regardait et sa gorge se noua soudain. C'était comme si la rage qui l'habitait avait reçu un coup de marteau, peut-être le plus fort qu'il ait jamais reçu de sa vie.

"Tous ceux qui sont tombés à cause de toutes mes guerres ne sont pas là," dit Albus Dumbledore, dont Harry ne pouvait voir que les boucles grises et les robes jaunes. "Loin de là. Seuls mes amis les plus proches et ceux qui sont morts à cause de mes pires décisions. Une partie de ceux-là se trouve ici. Ceux que je regrette le plus résident ici."

Harry n'arrivait pas à compter combien de piédestaux se trouvaient là. Peut-être une centaine. La pièce de métal noire n'était pas petite et il restait clairement de l'espace destiné à accueillir des piédestaux supplémentaires.

Albus Dumbledore se retourna et regarda Harry de ses profonds yeux bleus sertis sur son visage tels des bijoux d'acier, mais sa voix fut calme~: "J'ai l'impression que tu ignores tout du prix du phénix," dit Albus Dumbledore à voix basse. "J'ai l'impression que tu n'es pas quelqu'un de mauvais mais des plus terriblement ignorant, confiant dans son ignorance, comme je l'ai été un jour, il y a longtemps. Et pourtant je n'ai jamais entendu Fumseck aussi clairement que tu sembles l'avoir entendu, ce jour-là. Peut-être étais-je déjà trop vieux et endeuillé lorsque mon phénix est venu me voir. S'il est quelque chose que je ne comprends pas quant à l'empressement que je devrais avoir à me battre, fais-moi part de ta sagesse." La voix du vieux sorcier ne contenait nulle colère~; l'impact qui coupait le souffle comme si on venait de tomber d'un balai volant provenait entièrement des baguettes brûlées et fracassées qui luisaient doucement dans la mort, sous la lumière argentée. "Sinon, détourne-toi et pars de ce lieu, mais alors je ne voudrais plus jamais en entendre parler."

Harry ne savait pas quoi dire. Il n'avait jamais rien vécu de tel et tous les mots semblaient s'écrouler. Il aurait trouvé quelque chose à dire s'il avait cherché, mais il n'arrivait pas à croire, en cet instant, que des mots pourraient avoir un sens. On n'aurait pas dû pouvoir gagner n'importe quel débat uniquement parce que des gens étaient morts à cause de ses décisions, et pourtant, même en sachant cela, Harry avait l'impression qu'il n'avait rien qu'il puisse dire. Rien qu'il ait le droit de dire.

Et il se serait détourné, il aurait quitté ce lieu sans l'éclair de compréhension qui le traversa alors~: qu'une partie d'Albus Dumbledore se tenait probablement ici en permanence, toujours, peu importe où il était. Et que si vous étiez dans un endroit pareil, vous pouviez faire n'importe quoi, \emph{perdre} n'importe quoi tant que cela vous permettait de ne pas avoir à vous battre une fois de plus.

L'un des piédestaux attira le regard de Harry~; la photographie qui s'y trouvait ne bougeait pas, ne souriait pas, n'agitait pas la main~; c'était la photographie d'une femme qui regardait l'objectif avec sérieux, ses cheveux châtains tressés en nattes d'un style moldu ordinaire que Harry n'avait vu porté par aucun sorcière. Un cylindre empli d'un liquide argenté se trouvait à côté de la photo, mais il n'y avait aucun objet, pas d'anneau fondu ni de baguette brisée.

Harry avança lentement jusqu'à se tenir devant le piédestal. "Qui était-elle~?" dit Harry d'une voix qui sembla étrange à ses propres oreilles.

"Elle s'appelait Tricia Glasswell," dit Dumbledore. "La mère d'une fille née-Moldue que les Mangemort ont tué. Elle était détective pour le gouvernement moldu, et elle a ensuite fourni des informations venues des autorités moldues à l'Ordre du Phénix jusqu'à être… trahie… et mise entre les mains de Voldemort." Il y eut comme un accroc dans la voix de Dumbledore. "Elle n'est pas morte en paix, Harry;

--- A-t-elle sauvé des vies~?" dit Harry.

"Oui," dit doucement le sorcier. "En effet."

Harry éleva son regard au-dessus du piédestal pour regarder Dumbledore. "Le monde serait-il un endroit meilleur si elle ne s'était pas battue~?

--- Non," dit le vieux sorcier. Sa voix était fatiguée et endeuillée. Il semblait plus penché, comme s'il était en train de se replier sur lui-même. "Je vois que tu ne comprends toujours pas. Je pense que tu ne comprendras pas avant le jour où tu… oh, Harry. Il y a si longtemps, quand je n'étais pas beaucoup plus âgé que tu ne l'es maintenant, j'ai découvert le véritable visage de la violence, et son prix. Emplir un lieu de sortilèges mortels - qu'elle qu'en soit la raison - \emph{quelle} qu'en soit la raison, Harry - est une chose laide qui corrompt sa nature, aussi terrible que le plus noir des rituels. La violence, une fois commencée, devient comme un Moremplis et attaque toute vie qui l'entoure. Je… voudrais pouvoir t'épargner la façon dont j'ai appris cette leçon."

Harry se détourna des yeux bleus et posa son regard sur le noir métal du sol. Il était clair que le directeur essayait de lui dire quelque chose d'important, et ce n'était pas non plus quelque chose que Harry trouvait idiot.

"Il y avait un Moldu nommé Mohandas Gandhi," dit Harry à l'attention du plancher. "Il pensait que le gouvernement d'Angleterre moldu ne devait pas diriger son pays. Et il refusait de se battre. Il a convaincu son pays entier de ne pas se battre. Au lieu de cela, il a dit aux gens de marcher vers les soldats anglais et de se laisser battre sans résister, et lorsque l'Angleterre n'a plus pu supporter de faire ça, nous avons libéré son pays. J'ai trouvé que c'était très beau lorsque je l'ai lu, que c'était plus digne que toutes les guerres à avoir jamais été menées au canon ou à l'épée. Qu'ils l'aient vraiment fait et que ça ait \emph{vraiment marché}." Harry inspira à nouveau. "Sauf que j'ai alors découvert que pendant la seconde guerre mondiale, Gandhi avait dit à son peuple que si les Nazis envahissaient l'Inde, ils devraient aussi utiliser la résistance non-violente contre eux. Mais les Nazis auraient juste tiré à vue. Et peut-être Winston Churchill s'est-il toujours dit qu'il aurait dû y avoir un meilleur moyen, un moyen intelligent de gagner sans avoir à faire de mal à qui que ce soit, mais il ne l'a jamais trouvé et il a donc dû se battre." Harry releva les yeux vers le directeur, qui le regardait. "Winston Churchill est celui qui a essayé de convaincre le gouvernement d'Angleterre de ne \emph{pas} donner la Tchécoslovaquie à Hitler en échange d'un traité de paix, qui a essayé de les convaincre de se battre tout de suite -

--- Je reconnais le nom, Harry," dit Dumbledore. Les lèvres du vieux sorcier s'élevèrent brièvement. "Même si l'honnêteté me force à dire que ce bon vieux Winston n'a jamais été prompt aux remords, même après une douzaine de shots de Whisky Pur Feu.

--- L'idée étant," dit Harry après une brève pause lors de laquelle il se rappela à qui exactement il était en train de parler et combattit le sentiment, revenu une fois de plus, qu'il était un enfant ignorant devenu fou d'audace qui n'avait aucun droit d'être dans cette pièce et aucun droit de remettre en question Albus Dumbledore sur quoi que ce soit, "l'idée étant que ce n'est pas une \emph{réponse} de dire que la violence, c'est le mal. Ça ne dit pas quand se battre et quand ne pas se battre. C'est une question difficile, Gandhi a refusé de s'y confronter, et c'est pour ça que j'ai perdu une partie de mon respect pour lui.

--- Et ta propre réponse, Harry~?" dit doucement Dumbledore.

"Une réponse est qu'on ne devrait jamais utiliser la violence, sauf pour arrêter la violence," dit Harry. "On ne devrait jamais risquer la vie de quelqu'un, sauf pour sauver encore plus de vies. Ça \emph{sonne} bien quand on le dit comme ça. Sauf que le problème, c'est que si un policier voit un voleur cambrioler une maison, il \emph{devrait} essayer d'arrêter le voleur même si celui-ci risque de se défendre et que quelqu'un risque de souffrir ou même d'être tuer. Même si le voleur n'essaie de prendre que des bijoux, qui ne sont que des \emph{objets}. Parce que si personne n'est là pour ne serait-ce \emph{qu'incommoder} les voleurs, il va y avoir \emph{encore et toujours} plus de voleurs. Et même s'ils ne faisaient que voler des \emph{choses} à chaque fois, ça… le tissu social…" Harry s'interrompit. Dans cette pièce, ses pensées n'étaient pas aussi ordonnées qu'il avait l'habitude de le prétendre. Il aurait dû être capable de fournir une explication parfaitement logique en termes de théorie des jeux, il aurait au moins dû être capable de \emph{voir} les choses sous cet angle, mais cela lui échappait. Ceux qui coopèrent et les traîtres… "Vous ne voyez pas que si les gens méchants sont prêts à prendre le risque d'être violents pour obtenir ce qu'ils veulent et que les bons cèdent toujours parce que la violence est trop terrible, c'est… ce n'est pas une société où il fait bon vivre, M. le directeur~! Vous ne vous rendez pas compte de ce que toutes ces brutalisations font à Poudlard, et à Serpentard encore plus~?

--- La \emph{guerre} est trop terrible", dit le vieux sorcier. "Et pourtant elle viendra. Voldemort revient. Les pièces noires se rassemblent. Dans cette guerre, Severus est l'une des pièces les plus importantes de notre camp. Mais notre mauvais maître des potions doit sauver les apparences, comme on dit. Si Severus peut payer ce tribu en faisant de la peine à des enfants, seulement de la peine, Harry," la voix du vieux sorcier était très douce, "il faudrait que tu sois terriblement innocent en matière de guerre pour penser que tu as fait une mauvaise affaire. Les décisions difficiles ne ressemblent pas à \emph{ça}, Harry. Elles ressemblent… à ça." Le vieux sorcier ne bougea pas. Il resta simplement là, entre les piédestaux.

"Vous ne devriez pas être directeur", dit Harry, une sensation de brûlure dans sa gorge. "Je suis désolé, je suis tellement désolé mais vous ne devriez pas essayer d'être un directeur d'école et de mener une guerre en même temps. Poudlard ne devrait pas être impliquée.

--- Les enfants survivront," dit le vieux sorcier aux yeux fatigués. "Ils ne survivraient pas à Voldemort. T'es-tu déjà demandé pourquoi les enfants de Poudlard ne parlent pas beaucoup de leurs parents, Harry~? C'est parce qu'il y a toujours, à portée d'oreille, quelqu'un qui a perdu son père, sa mère ou les deux. C'est ce que Voldemort a laissé derrière lui la dernière fois qu'il est venu. \emph{Rien} ne mérite que cette guerre recommence ne serait-ce qu'un jour avant qu'il ne puisse en être autrement ni qu'elle dure un jour de plus que nécessaire." Le vieux sorcier fit alors un geste, comme pour montrer toutes les baguettes fracassées. "Nous ne nous battions pas parce que nous voulions être vertueux~! Nous nous battions parce que nous y étions obligés, parce qu'il n'y avait aucun autre moyen. C'était notre réponse.

--- Est-ce pour cela que vous avez attendu si longtemps avant de vous confronter à Grindelwald~?"

Harry avait posé la question sans vraiment réfléchir -

Le temps s'écoula lentement pendant que les yeux bleus le scrutaient.

"À qui as-tu été parler, Harry~?" dit le vieux sorcier. "Non, ne réponds pas. Je sais déjà." Dumbledore soupira. "Nombre d'entre eux m'ont posé cette question et j'ai toujours évité de leur répondre. Pourtant, tu devras un jour apprendre toute la vérité sur cette affaire. Promettras-tu de ne jamais en parler à un autre avant que je ne te le permette~?"

Harry aurait aimé pouvoir le dire à Drago mais - "Je promets," dit-il.

"Grindelwald possédait un objet terrible et ancien," dit Dumbledore. "Tant qu'il l'avait en sa possession, je ne pouvais briser ses défenses. Je ne pouvais gagner notre duel, seulement le combattre pendant de longues heures jusqu'à ce qu'il tombe d'épuisement~; et j'en serais ensuite mort sans Fumseck. Mais tant que ses alliés Moldus continuaient de faire des sacrifices de sang pour le maintenir, il ne serait pas tombé. À cette époque, il était véritablement invincible. De ce sombre objet que possédait Grindelwald, personne ne doit rien savoir, ne doit rien soupçonner, il ne doit pas y avoir un seul indice. Et tu n'en parleras donc pas, et je n'en dirai pas plus pour le moment. C'est tout Harry. Il n'y a pas de morale, pas de sagesse à cette histoire. C'est tout ce qu'il y a."

Harry hocha lentement la tête. Étant donné ce qu'il avait vu de la magie, ce n'était pas entièrement impensable…

"Et alors," continua la voix de Dumbledore, encore plus basse, presque comme s'il se parlait à lui-même, "puisque c'est moi qui l'avais fait tomber, ils m'ont obéi lorsque j'ai dit qu'il ne devrait pas mourir, même s'ils étaient des milliers à réclamer son sang. Il fut donc emprisonné à Nurmengard, dans la prison qu'il avait construite, et c'est là qu'il y endure sa peine aujourd'hui encore. Je suis allé le combattre sans avoir l'intention de le tuer, Harry. Parce que vois-tu, j'avais déjà essayé de tuer Grindelwald une fois auparavant, il y a longtemps, et ça… ça s'était… avéré être… une erreur, Harry." Le vieux sorcier regardait maintenant sa longue baguette gris-noir qu'il tenait à deux mains comme si c'était une boule de cristal sortie d'un fantasme moldu, une source divinatoire où des réponses pouvaient être trouvées. "Et j'ai alors songé… j'ai songé que je ne devrais jamais tuer. Et Voldemort est arrivé."

Le vieux sorcier releva les yeux vers Harry et dit d'une voix enrouée~: "Il n'est pas comme Grindelwald, Harry. Il n'y a plus rien d'humain en lui. \emph{Lui}, tu dois le détruire. Tu ne dois pas hésiter quand le moment viendra. Envers lui seul, de toutes les créatures de ce monde, tu ne dois montrer aucune pitié~; et lorsque tu en auras fini tu dois l'oublier, oublier que tu as jamais commis cet acte et te remettre à vivre. Garde ta furie pour cela et pour cela seulement."

Il n'y avait plus que du silence dans le bureau.

Il dura quelques longues secondes et fut enfin brisé par une unique question.

"Y a-t-il des Détraqueurs à Nurmengard~?

--- Quoi~?" dit le vieux sorcier. "Non~! Je ne lui aurais pas fait ça, même à lui."

\later

Le vieux sorcier regarda le jeune garçon qui s'était raidi et dont le visage avait changé.

"En d'autres termes," dit le garçon, comme s'il se parlait à lui-même dans une pièce vide, "on sait déjà comment garder de puissants mages noirs en prison sans utiliser de Détraqueurs. Les gens \emph{savent} qu'on le sait.

--- Harry…?

--- Non," dit le garçon. Il releva les yeux, et ses yeux brillaient d'un feu vert. "Je n'accepte pas votre réponse, M. le directeur. Fumseck m'a donné une mission, et je sais maintenant pourquoi il l'a donnée à moi et pas à vous. Vous êtes prêt à accepter un équilibre des pouvoirs dans lequel les méchants sont vainqueurs. Pas moi.

--- Ce n'est pas non plus une réponse," dit le vieux sorcier~; son visage ne révélait rien de sa souffrance car il avait longuement pratiqué la dissimulation de celle-ci. "On ne change pas quelque chose en refusant de l'accepter. Je me demande maintenant si tu n'es pas simplement trop jeune pour comprendre cela, Harry, en dépit de ton apparence~; ce n'est que dans les fantasmes d'enfants que toutes batailles peuvent être gagnées et qu'aucun mal n'a à être toléré.

--- Et c'est pour cela que je peux détruire les Détraqueurs et pas vous," dit le garçon. "Parce que je pense que les ténèbres peuvent être brisées."

Le vieux sorcier eut le souffle coupé.

"Le prix du phénix n'est pas inévitable," dit le garçon. "Il ne fait pas partie d'un équilibre profondément ancré dans l'univers. C'est juste une partie du problème où vous n'avez pas encore découvert comment tricher."

Les lèvres du vieux sorcier s'ouvrirent mais aucun mot ne sortit.

Une lumière argentée tombait sur les baguettes brisées.

"Fumseck m'a donné une mission," répéta le garçon, "et je la mènerai à bien même si je dois faire tomber tout le ministère pour cela. C'est la partie de la réponse qui vous manque. L'idée qu'on ne peut pas s'arrêter en disant \emph{oh, eh bien j'imagine que je ne pourrai jamais trouver un moyen d'arrêter les brutalités à Poudlard} et en \emph{rester} là. On continue juste de chercher jusqu'à avoir trouvé un moyen de le faire. Si ça signifie qu'il faut faire voler en éclats toute la conspiration de Lucius Malfoy, \emph{très bien}.

--- Et le véritable combat, celui contre Voldemort~?" dit le vieux sorcier d'une voix chancelante. "Que feras-tu pour gagner \emph{celui-là}, Harry~? Briseras-tu la terre entière~? Même si un jour tu obtenais un tel pouvoir, tu n'es pas encore au-delà des prix à payer, et peut-être ne le seras-tu jamais~! Car agir de la sorte \emph{maintenant} n'est rien d'autre que de la folie~!

--- J'ai demandé au professeur Quirrell pourquoi il avait ri," dit le garçon d'une voix neutre, "après avoir donné cent points à Hermione. Et il m'a répondu, ce n'étaient pas ses mots exacts, mais c'est plus ou moins ce qu'il a dit, qu'il avait trouvé particulièrement amusant de voir que le grand et bon Albus Dumbledore était resté assis là à ne rien faire pendant que cette pauvre fille innocente appelait à l'aide et que c'était \emph{lui} qui l'avait défendue. Et il m'a alors dit que quand les gens bons et moraux avaient fini de s'embrouiller l'esprit, ils ne faisaient généralement rien, ou s'ils agissaient, qu'on pouvait à peine les distinguer des méchants. Tandis que \emph{lui} pouvait aider d'innocentes jeunes filles quand ça lui chantait parce qu'il n'était pas l'un des gentils. Et que je devrais m'en souvenir à chaque fois que j'envisageais de devenir un gentil."

Le vieux sorcier ne révéla pas la force du choc. Si vous l'aviez regardé très attentivement, seul un léger écarquillement de ses yeux l'aurait trahi.

"Ne vous en faites pas, M. le directeur," dit le garçon. "Je ne me suis pas emmêlé les pinceaux. Je sais que je suis censé apprendre ce qui est bon auprès de Hermione et de Fumseck, pas du professeur Quirrell et de vous. Ce qui m'amène à la véritable raison de ma venue. Le temps de Hermione est trop précieux pour être gâché en retenues. Le professeur Rogue les annulera en prétendant que je l'ai fait chanter."

Après une hésitation, le vieux sorcier hocha la tête, faisant légèrement onduler sa barbe d'argent. "Ce ne serait pas idéal pour \emph{elle}, Harry," dit le vieux sorcier. "Mais on peut écrire que la retenue sera surveillée par le professeur Binns et vous pourrez ainsi étudier ensemble dans sa salle de cours.

--- Très bien," dit le garçon. "Je pense que c'est tout ce que nous avions à régler. La prochaine fois que vous semblez travailler avec les méchants ou que vous les laissez gagnez, attendez-vous à ce que je fasse toute ce que je pense que Fumseck me laisserait faire, peu importe les problèmes que cela causera. J'espère que nous nous sommes bien compris."

Sans un mot de plus, le garçon se retourna et quitta la pièce par la porte de métal noir, le mot "\emph{Lumos}~!" et la lumière de sa baguette jaillissant un instant plus tard.

Le vieux sorcier se tint là, silencieux, entre les ruines des vies que la sienne avait laissées en arrière. Sa main ridée s'éleva, tremblante, pour venir toucher ses lunettes en demi-lune -

Le garçon refit passer sa tête à l'intérieur. "Ça vous embêterait de rallumer les escaliers, M. le directeur~? Je préférerais ne pas refaire tout le travail que j'ai dû faire pour venir.

--- Vas, Harry Potter," dit le vieux sorcier. "Les escaliers te recevront."

(Quelques temps plus tard, une version antérieure de Harry qui avait attendu invisible à côté des gargouilles depuis neuf heures du soir, suivit la directrice adjointe à travers l'ouverture qui s'était formée pour elle, se plaça en silence derrière celle-ci sur les escaliers tournants jusqu'à ce qu'ils arrivent à destination, puis, toujours sous la Cape, fit pivoter son Retourneur de Temps par trois fois.)

\latersection{Après-coup~: Professeur Quirrell et…}

Dans la clairière ombragée où le professeur de Défense attendait, le dos négligemment appuyé contre l'écorce grise et rugueuse d'un immense hêtre encore privé de ses feuilles en cette fin de mars et dont le tronc et les branches ressemblaient à un bras pâle jaillit du sol qui aurait explosé en un millier de doigts. Autour du professeur de Défense et au-dessus de lui, les branches étaient si denses que même en ce début de printemps, alors que si peu d'arbres commençaient à peine à bourgeonner, on pouvait à peine voir le ciel depuis le sol. Les fils du maillage boisé se croisaient et proliféraient tant que si vous aviez été sur un balai, à la recherche de quelqu'un, vous auriez trouvé plus simple de vous guider à l'ouïe qu'à la vue. La nuit qui s'approchait de ces bois interdits ne vous aurait pas aidé, pas plus que le soleil presque couché et déjà invisible dont seules quelques lueurs mourantes illuminaient encore le sommet des arbres les plus hauts.

Puis vint le bruit de pas le plus discret qui soit, presque inaudible même sur le sol forestier~; la démarche d'un homme habitué à passer inaperçu. Aucune brindille ne se brisa, aucune feuille ne bruissa…

"Bon après-midi," dit le professeur Quirrell. Il ne prit la peine de déplacer ni ses yeux, ni ses mains nonchalamment laissées le long de ses flancs.

Une silhouette couverte d'une cape noire apparut dans une vibration visuelle tandis que sa tête se tournait pour regarder à gauche et à droite. Dans la main droite de la silhouette, tenue basse, se trouvait une baguette d'un bois si gris qu'elle était presque argentée.

"Je ne sais pas pourquoi vous désiriez que nous nous rencontrions \emph{ici} en particulier," dit Severus Rogue d'une voix fraîche.

"Oh," dit le professeur Quirrell comme pour passer le temps, comme si toute l'affaire n'avait pas la moindre importance, "j'ai pensé que vous préféreriez avoir de l'intimité. Les murs de Poudlard ont des oreilles, et vous ne voudriez pas que le directeur apprenne votre rôle dans l'affaire d'hier, n'est-ce pas~?"

La fraîcheur de mars sembla devenir plus profonde, la température sembla chuter. "Je ne sais pas de quoi vous parlez," dit le maître des potions d'un ton glacial.

"Vous savez parfaitement de quoi je parle," dit le professeur Quirrell d'un ton amusé. "Vraiment, mon bon professeur, vous ne devriez pas vous mêler des affaires d'idiots à moins d'être à chaque instant prêt à vous défendre contre toute la violence dont ils sont capables." (Les mains du professeur Quirrell étaient toujours détendues, le long de ses flancs). "Et pourtant aucun de ces idiots ne semble se souvenir vous avoir vu tomber, pas plus que ces jeunes demoiselles ne se remémorent votre présence. Ce qui soulève une question fascinante~: pourquoi feriez-vous l'effort extraordinaire, j'oserais dire l'effort \emph{désespéré}, de lancer \emph{cinquante-deux} sortilèges d'Oubliettes~?" Le professeur Quirrell inclina la tête. "Auriez-vous tant peur de l'opinion de simples élèves~? Je ne pense pas. Redouteriez-vous que l'affaire devienne connue de votre bon ami, Lord Malfoy~? Mais ces idiots avaient immédiatement inventé une excuse assez satisfaisante quant à votre présence. Non, il n'y a qu'une personne qui a assez de pouvoir sur vous et qui serait des plus perturbées d'apprendre que vous exécutez le moindre plan sans qu'il en aie connaissance. Votre véritable et secret maître, Albus Dumbledore.

--- \emph{Quoi~?}" siffla le maître de potions, la colère visible sur ses traits.

"Mais il semble maintenant que vous jouiez seul, et je me retrouve donc des plus intrigués par ce que vous \emph{pourriez bien} être en train de faire, et pourquoi." Le professeur de Défense regardait la silhouette vêtue de noir du maître des potions avec l'attention qu'un homme aurait pu apporter à un insecte exceptionnellement intéressant mais qui n'en serait pas moins demeuré, au fond, qu'un insecte.

"Je ne suis pas un serviteur de Dumbledore," dit le maître des potions d'une voix froide.

"Vraiment~? Quelle incroyable nouvelle." Le professeur de Défense souriait avec légèreté. "Dites-m'en plus."

Il y eut un long silence. Depuis un arbre, une chouette hulula, un son qui sembla immense~; aucun des deux hommes ne fut surpris ni ne tressaillit.

"Vous n'avez aucune envie d'être mon ennemi," dit Severus Rogue d'une voix très douce.

"Ah bon~?" dit le professeur Quirrell. "Et comment le sauriez-vous~?

--- En revanche," continua le maître des potions, sa voix toujours très douce, "mes amis bénéficient de nombreux avantages."

L'homme appuyé contre l'écorce grise leva les sourcils. "Tels que~?

--- Je sais nombre de choses sur cette école," dit le maître des potions. "Des choses que vous ne me soupçonnez peut-être pas de connaître."

Il y eut un silence, comme l'attente de quelque chose.

"Que c'est fascinant," dit le professeur Quirrell. L'homme examinait ses ongles avec l'air de s'ennuyer. "Continuez, je vous en prie.

--- Je sais que vous avez… \emph{enquêté}… sur le couloir du troisième étage -

--- Vous ne savez rien de tel." Le dos de l'homme se raidit. "N'essayez pas de me bluffer, Severus Rogue, je trouve cela agaçant et vous n'êtes en position de m'agacer. Un simple coup d'œil permettrait à n'importe quel sorcier compétent de savoir que le directeur a tissé des quantités astronomiques de toiles, de barrières, de pièges et d'alertes. De plus, il y a là des charmes venus d'un pouvoir ancien, des constructions magiques au sujet desquelles je n'ai même pas entendu de rumeurs, des techniques qui doivent avoir directement été dégorgées des connaissances accumulées de Flamel lui-même. Même Celui-Dont-On-Ne-Doit-Pas-Prononcer-Le-Nom aurait eu du mal à les franchir sans être repéré." Le professeur Quirrell tapota sa joue d'un air pensif. "Quant au loquet, un simple \emph{Collaporta} placé sur une poignée ordinaire de façon si faible qu'il n'aurait pas pu bloquer Mlle Granger le jour où elle est arrivée à Poudlard. Jamais auparavant je n'ai eu connaissance d'un piège aussi flagrant." Le professeur de Défense plissa alors les yeux. "Je n'ai connaissance d'aucun individu encore de ce monde pour lequel de tels prouesses de détection accompliraient quoi que ce soit. S'il existe un sorcier possesseur de savoirs anciens contre qui ce piège a été tendu - vous pouvez échanger \emph{cette} information contre autant de silence que vous voudrez, mon cher professeur, ainsi qu'une bonne dose de faveurs en prime."

On aurait pu jurer que le professeur Quirrell regardait Severus Rogue avec le plus grand intérêt. Pas une trace de sourire ne déplaça ses lèvres.

Il y eut un autre long silence dans la clairière.

"Je ne sais pas \emph{qui} Dumbledore craint," dit Rogue. "Mais je sais quel appât a été mis en place, et j'ai une idée quant à ce qui le protège vraiment -

--- Quant à ça," dit le professeur comme s'il s'ennuyait de nouveau, "je l'ai volé il y a des mois et j'ai laissé un faux à sa place. Mais merci d'avoir demandé.

--- Vous mentez," dit Severus Rogue au bout d'un moment.

"Oui, en effet." Le professeur Quirrell s'appuya de nouveau contre le bois gris, ses yeux s'élevant vers le dense maillage des branches, la nuit qui tombait à peine visible entre les complexes tresses. "Je souhaitais simplement voir si vous me reprendriez sur ce point puisque vous prétendez en savoir si peu." Le professeur de Défense se sourit à lui-même.

Le maître des potions avait l'air d'être sur le point de s'étrangler de rage. "\emph{Que voulez-vous~?}

--- À vrai dire, rien," dit le professeur de Défense en continuant de regarder le toit forestier. "J'étais seulement curieux. J'imagine que je vais juste continuer d'observer et voir où vont vos plans, et pendant ce temps je ne dirai rien au directeur - tant que vous acceptez de m'octroyer une faveur de temps à autre, bien sûr." Un sourire sec passa sur son visage. "Vous êtes congédié pour le moment, Severus Rogue. Même si j'apprécierai une autre petite conversation sous peu si vous acceptez de me parler avec honnêteté de vos allégeances. Et je dis bien \emph{honnêtement}, pas sous les masques que vous m'avez montré aujourd'hui. Peut-être découvrirez-vous que vous avez plus d'alliés que vous ne le pensiez. Prenez un peu de temps pour y réfléchir, mon ami."

\latersection{Après-coup~: Drago Malfoy et…}

Un hémisphère arc-en-ciel, un dôme de force lui-même doté de peu de chromaticité mais qui renvoyait la lumière incidente sous forme de reflets éclatés et iridescents de plusieurs couleurs en fracturant l'éclat de l'immense lustre de la salle commune de Serpentard.

À l'abri sous l'hémisphère arc-en-ciel, le visage terrifié d'une jeune sorcière qui n'avait jamais combattu de brutes, qui n'avait rejoint aucune des armées du professeur Quirrell, qui obtenait des notes passables au mieux en cours de Défense et qui n'aurait pas pu lancer une barrière prismatique même si sa vie en avait dépendu.

"Oh, arrêtez ça," dit Drago Malfoy en essayant de prendre un ton ennuyé en dépit de la sueur qui venait d'apparaître sous sa robe et en gardant sa baguette pointée vers la barrière qui protégeait Millicent Bulstrode.

Il n'arrivait pas à se rappeler avoir pris une décision, il y avait juste eu deux garçons plus âgés sur le point de lancer un maléfice à Millicent tandis que la salle commune regardait en silence et Drago avait alors juste sorti sa baguette et lancé sa barrière, laissant à son cœur le soin de le remplir d'adrénaline pendant que son pauvre et triste cerveau se torturait désespérément à la recherche d'explications…

Les deux garçons plus âgés se redressaient de leur posture menaçante au-dessus de Millicent et se tournaient vers Drago en le regardant d'un air stupéfait mêlé de colère. Gregory et Vincent, à côté de lui, avaient déjà sorti leurs baguettes, mais elles n'étaient pas pointées. De toute façon, ils n'auraient pas pu gagner à trois.

Mais les garçons plus âgés ne l'attaquèrent pas. Personne n'était assez stupide pour attaquer le prochain Lord Malfoy.

Ce n'était pas la peur d'être attaqué qui avait déclenché les sueurs de Drago, dont il espérait d'ailleurs que les gouttelettes sur son front n'étaient pas visibles.

Drago suait à cause de la certitude naissance et écœurante qui lui disait que même en s'en tirant cette fois-ci, s'il continuait ainsi, le temps viendrait où tout s'écroulerait~; et alors il ne serait peut-être plus le prochain Lord Malfoy.

"M. Malfoy," dit le garçon qui avait l'air le plus âgé. "Pourquoi la protégez-vous~?

--- Vous avez donc localisé la maîtresse de la conspiration," dit Drago avec son sourire méprisant numéro deux, "et c'est, autant être clair maintenant, une fille en première année prénommée Millicent Bulstrode. Elle n'est qu'une \emph{passeuse}, espèce de cornigaud~!

--- Et alors~?" demanda le garçon plus âgé. "Elle les a quand même aidées~!"

Drago leva sa baguette et la sphère prismatique disparut instantanément. Toujours d'une voix ennuyée, Drago dit~: "\emph{Saviez}-vous ce que vous faisiez, Mlle Bulstrode~?

--- N-non," bégaya Millicent, toujours assise à son bureau.

"Saviez-vous où les messages Serpentard que vous passiez allaient~?

--- Non~!" dit Millicent.

"Merci," dit Drago. "Vous tous, merci de la laisser tranquille, elle n'est qu'un pion. Mlle Bulstrode, vous pouvez considérer que la faveur que vous m'avez faite en février a été rendue." Et Drago revint à ses devoirs de potions, en priant Merlin et les autres que Millicent ne dirait rien de stupide, comme par exemple~: 'Quelle faveur~?'…

"Alors pourquoi," dit une voix claire depuis l'autre côté de la pièce, "ces sorcières allaient-elles exactement là où Millicent leur disait d'aller~?"

Suant encore plus, Drago leva la tête pour regarder Randolph Lee. "Que disait le faux message, exactement~?" demanda Drago. Était-ce~: 'Je vous ordonne de vous y rendre, au nom de la Dame des Ténèbres Bulstrode' ou 'Merci de me retrouver ici, amicalement, Millicent'~?"

Randolph Lee ouvrit la bouche, hésita pendant une fraction de seconde -

"C'est ce que je pensais," dit Drago. "Ce n'était pas un très bon test, M. Lee, ça… ça peut…" Un instant de supplice nerveux désespéré pendant lequel il essaya de trouver comment le dire sans utiliser des mots de Harry comme \emph{faux positif}. "Ça peut amener les sorcières n'importe où du moment que l'une d'entre elles est \emph{amie} avec Millicent."

Comme si l'affaire avait été entièrement réglée, Drago baissa de nouveau les yeux vers ses devoirs de potions en ignorant entièrement (mis à part cette sensation de terreur écœurante dans son estomac) les chuchotement de la pièce.

Ce n'est que du coin de l'œil qu'il put voir que Gregory le regardait fixement.

\later

Les yeux posés sur ses devoirs d'astronomie, Drago n'arrivait pas à se concentrer dans cette pièce. Si vous ne vouliez pas penser à ce que Harry Potter pouvait vous avoir dit, la pire chose à faire était probablement de regarder les images du ciel étoilées de votre manuel et d'essayer de vous souvenir que vous n'étiez \emph{pas} censé savoir comment les planètes se déplaçaient. L'astronomie était un art noble et prestigieux, un signe d'éducation et de connaissance, mais seuls les Moldus modernes possédaient des artefacts secrets permettant de la pratiquer un million de milliard de fois mieux en utilisant des méthodes que Harry avait essayé d'expliquer et que Drago ne pouvait toujours pas commencer à comprendre, mis à part qu'apparemment, il n'y avait même pas besoin de \emph{magie} pour que des \emph{objets} soient capables de faire de l'\emph{Arithmancie}.

Drago regarda les images des constellations et se demanda si c'était pareil dans les autres maisons, si les gens se menaçaient en permanence à Serdaigle.

Harry Potter lui avait un jour dit que les soldats sur un champ de bataille ne se battaient pas vraiment pour leur pays. Le patriotisme les amenait peut-être jusqu'au champ de bataille, mais une fois là, ils se battaient pour se protéger \emph{les uns les autres}, leurs amis avec qui ils s'étaient entraînés aux côtés desquels ils se trouvaient. Harry avait fait remarquer, et Drago avait su qu'il disait vrai, qu'on ne pouvait pas utiliser la loyauté envers un chef pour alimenter un Patronus, que ce n'était pas \emph{exactement} le bon type de pensée réjouissante et heureuse. Mais en pensant à protéger quelqu'un à ses côtés…

Cela, avait dit Harry d'un ton pensif, était probablement la raison pour laquelle les Mangemorts s'étaient effondrés à l'instant où le Seigneur des Ténèbres était parti. Ils n'y avait pas eu assez d'affection \emph{entre eux}.

On pouvait recruter un groupe comprenant Bellatrix Black, Amycus Carrow, Lord Malfoy, M. MacNair et les maintenir au pas grâce au sortilège Doloris. Mais à l'instant où le maître de la marque des ténèbres partait, vous n'aviez plus une armée, seulement un cercle de connaissances. C'était pour cela que Père avait échoué. Ça n'avait même pas vraiment été sa faute. Il n'y avait rien que Père \emph{aurait pu} faire après avoir hérité de Mangemorts qui n'étaient pas vraiment \emph{amis}.

Et même si c'était Serpentard qu'il était censé défendre - que c'était pour \emph{sauver} Serpentard que Harry et lui avaient fait un pacte - Drago ne pouvait s'empêcher de penser que c'était juste moins \emph{épuisant} quand il dirigeait les entraînements de son armée. Quand il travaillait avec des élèves des trois autres maisons, avec des élèves qui n'étaient pas de Serpentard. Une fois que vous pouviez voir les problèmes et les nommer, vous ne pouviez plus vous \emph{empêcher} de les voir et ça devenait chaque jour de plus en plus \emph{agaçant}.

"M. Malfoy~?" dit la voix de Gregory Goyle depuis le sol de la chambre de Drago, certes petite mais privée. Gregory faisait ses devoirs de métamorphose, pour lesquels il avait souvent besoin d'aide.

À ce stade, n'importe quelle distraction était la bienvenue. "Oui~?" dit Drago.

"Vous ne maniganciez rien du tout contre Granger," dit Gregory, "en fait, non~?"

La sensation qui se répandit jusqu'à l'estomac de Drago ressemblait beaucoup à la voix qu'avait eu Gregory~: malade et apeurée.

"Vous l'aidiez, en fait, le jour où vous l'avez aidée à se relever," dit Gregory ."Et avant ça, la fois où vous l'avez empêchée de tomber du toit. Vous avez \emph{aidé une Sang-de-Bourbe} -

--- Bien sûr," dit immédiatement Drago d'un ton sarcastique, sans la moindre hésitation, en revenant à ses devoirs d'Astronomie comme s'il n'était pas le moins du monde nerveux. Tout se déroulait exactement comme Drago l'avait craint, mais au moins cela voulait dire qu'il avait rejoué cette conversation dans sa tête, encore et encore, et qu'il avait trouvé un bon gambit d'ouverture. "Allons, Gregory, tu t'es battu contre le général Granger, tu \emph{sais} à quel point ses sortilèges sont puissants. Comme si un vrai rejeton de Moldu allait être plus fort que toi, plus fort que Théodore, plus fort que tous les Sang Purs de l'école à part moi~? Est-ce que tu n'as pas \emph{foi} dans les enseignements de Père~? Elle est \emph{adoptée}. Ses parents sont morts pendant la guerre et quelqu'un la collée avec un couple de Moldus pour la cacher. Il est \emph{impossible} que le général Granger soit une Sang-de-Bourbe."

Une pulsation silencieuse traversa la chambre de Drago. Il voulait savoir, il avait besoin de savoir à quoi ressemblait le visage de Gregory. Mais il ne \emph{pouvait pas} lever les yeux de son bureau, pas avant que Gregory ne parle.

Et alors -

"C'est \emph{ça} que Harry Potter vous a dit~?" demanda Gregory.

La voix vacilla et se brisa. Lorsque Drago leva les yeux de ses devoirs, il vit que des larmes perlaient aux yeux de Gregory.

Apparemment, ça n'avait pas fonctionné.

"Je ne sais pas quoi faire," dit Gregory dans un murmure. "Je ne sais pas quoi faire, M. Malfoy. Votre père ne va - quand il le découvrira - il ne va pas aimer ça, M. Malfoy~!"

\emph{Ce n'est pas à toi de décider ce que Père aimera, Goyle -}

Drago pouvait entendre les mots dans son esprit~; ils avaient la voix de Père, la même dureté. C'était le genre de choses que Père lui avait \emph{dit} de dire si Vincent ou Gregory le remettaient jamais en question, et si cela ne fonctionnait pas, il devrait leur lancer un sortilège. Père avait dit qu'ils n'étaient \emph{pas} des amis sur un pied d'égalité et que Drago ne devrait jamais l'oublier. C'était lui qui dirigeait, ils étaient ses serviteurs, et si Drago ne pouvait faire en sorte que cela reste ainsi alors il n'était pas apte à hériter de la maison Malfoy…

"Tout va bien, Gregory," dit Drago avec autant de gentillesse que possible. "Tout ce dont tu dois t'inquiéter, c'est de me protéger. Personne ne va te blâmer parce que tu as suivi mes ordres, ni mon père ni le tien." Il mit toute la chaleur qu'il avait dans sa voix, comme s'il essayait de lancer un Patronus. "Et de toute façon, la prochaine guerre ne va pas ressembler à la précédente. La maison Malfoy était là longtemps avant le Seigneur des Ténèbres, et tous les Malfoys n'agissent pas de la même façon. Père le sait.

--- Vraiment~?" dit Gregory d'une voix tremblante. "Il le sait \emph{vraiment}~?"

Drago hocha la tête. "Le professeur Quirrell le sait aussi," dit Drago. "C'est le but des armées. Le professeur de Défense a raison~: quand la prochaine guerre surviendra, Père n'arrivera pas à unir tout le pays car les gens se souviendront de la \emph{dernière} guerre. Mais tous ceux qui se seront battus dans les armées du professeur Quirrell sauront qui étaient les meilleurs généraux et ils sauront qui mérite de les mener. Ils le désigneront comme leur Seigneur, je serai son bras droit, et la maison Malfoy finira tout en haut, comme toujours. Les gens se tourneront peut-être même vers \emph{moi}, si Potter n'est pas là, du moment qu'ils pensent que je suis digne de confiance. C'est ce que je mets en place maintenant. Père comprendra."

Gregory leva une main et s'essuya les yeux avant de se replonger dans ses devoirs de métamorphose. "D'accord," dit-il d'une voix chancelante, "si vous le dites, M. Malfoy."

Drago hocha de nouveau la tête, ignora la sensation de néant qui s'emparait de lui face aux mensonges qu'il venait de dire à son ami et revint aux étoiles.

\latersection{Après-coup~: Hermione Granger et…}

Être invisible aurait dû être plus \emph{intéressant} que ça. Les couloirs de Poudlard auraient dû être décorés d'étranges couleurs ou de quelque chose dans le genre. Mais en fait, songeait Hermione, être sous la cape d'invisibilité de Harry était exactement comme de ne \emph{pas} être sous un cape d'invisibilité, mis à part qu'on était sous une cape. Lorsqu'on rabattait le voile de tissu noir et soyeux de la capuche par-dessus son visage, on ne pouvait même pas le voir devant ses yeux et il ne bloquait même pas la respiration. Et le monde était exactement pareil à lui-même, sauf qu'on ne voyait pas de petits reflets de soi-même quand on passait devant des objets en métal. Les portraits ne regardaient jamais là où on était, ils faisaient seulement les choses bizarres que les portraits faisaient quand ils étaient seuls. Elle n'avait pas encore essayé de passer devant un miroir et elle n'était pas certaine de \emph{vouloir} le faire. Surtout, il n'y avait plus de \emph{soi} quand on marchait, plus de mains, plus de pieds, juste un point de vue mouvant. C'était une sensation qui rendait nerveux~; pas tant d'être \emph{invisible} que de ne \emph{pas exister}.

Harry ne lui avait posé aucune question. Il avait juste entendu le mot 'invisibilité' et Harry avait alors sorti la cape de sa poche. Elle n'avait même pas eu une chance de lui expliquer ni sa réunion extrêmement secrète avec Daphné et Millicent Bulstrode ni qu'elle pensait que cela protégerait les autres filles. Harry lui avait juste donné ce qui était probablement une Relique de la Mort. Si on voulait être juste, et Hermione \emph{essayait} d'être juste, il fallait bien admettre que Harry pouvait parfois être un bon, un très bon ami.

La réunion secrète elle-même avec été un échec complet.

Millicent avait proclamé être une voyante.

Hermione avait soigneusement et très longuement expliqué à Millicent et à Daphné que ce n'était tout simplement pas possible.

Elle et Harry s'étaient informés sur la divination assez tôt dans leurs recherches~; Harry avait insisté pour qu'ils lisent tout ce qu'ils pouvaient sur les prophéties hors de la section interdite. Comme Harry l'avait fait remarquer, cela leur éviterait beaucoup d'efforts de pouvoir trouver une voyante qui prophétiserait tout ce qu'ils découvriraient dans trente-cinq ans (ou, pour utiliser les termes de Harry, tout moyen d'obtenir des informations transmises d'un futur lointain pouvait potentiellement constituer une condition de victoire instantanée et totale).

Mais, comme Hermione l'expliqua à Millicent, les prophéties n'étaient pas contrôlables, il n'y avait aucun moyen de \emph{demander} une prophétie sur quelque chose en particulier. Au lieu de cela (avaient dit les livres), c'était une sorte de \emph{pression} qui s'accumulait dans le Temps là où un immense événement essayait d'avoir lieu ou de s'empêcher d'avoir lieu. Et les voyantes étaient comme des points faibles par lesquels cette pression s'échappait lorsque la personne destinée à l'entendre était là. Les prophéties ne concernaient donc que les choses importantes, cruciales, parce que elles seules généraient assez de pression, et on entendait quasiment jamais plus d'une voyante dire la même chose car après la première fois, la pression était partie. De plus, comme Hermione l'avait ensuite expliqué à Millicent, les voyantes elles-mêmes ne se souvenaient pas des prophéties parce que le message n'étaient pas destiné à \emph{elles}. Et les messages ne seraient que des énigmes, et seul quelqu'un qui entendrait la prophétie dite par la voix de la voyante originelle pourrait entendre tout le sens contenu dans l'énigme. Il était \emph{impossible} que Millicent soit capable de fournir une prophétie \emph{quand ça lui chantait} au sujet de \emph{brutes} et de s'en \emph{souvenir ensuite} et si \emph{c'était le cas} elle aurait dit "le squelette est la clé", pas "Susan Bones doit être là" {[}NdT: Le mot «~bones~» signifie «~os~»{]}.

Millicent avait alors eu l'air passablement effrayée et Hermione avait donc détendu ses poings serrés contre ses hanches, puis elle s'était calmée et avait pris le soin de dire qu'elle était heureuse que Millicent les aie aidées mais qu'elles étaient \emph{parfois} tombées dans des pièges en suivant les instructions de celle-ci et que Hermione voulait vraiment savoir d'où les messages étaient \emph{vraiment} venus.

Et Millicent avait dit d'une petite voix~:

\emph{Mais, mais elle} m'a \emph{dit qu'elle était voyante…}

Hermione avait dit à Daphné de ne pas insister après que Millicent eut refusé de révéler sa source. Ce n'était pas juste qu'elle s'était sentie très mal en voyant l'air effrayé de Millicent. Elle avait aussi eu la forte impression que si elles \emph{trouvaient} la personne qui avait révélé des choses à Millicent, eh bien \emph{elles} se retrouvaient juste avec des enveloppes sous leur oreiller le matin.

Elle commençait à ressentir le même désespoir que celui qu'elle avait ressenti avant la bataille qui avait précédé Noël en regardant le diagramme de Zabini, toutes ses lignes et ses rectangles colorés et… et elle ne comprit que maintenant ce que cela signifiait que ce soit \emph{Zabini} qui lui ait montré ce diagramme.

Elle avait l'impression que même pour une Serdaigle, il était possible d'avoir une vie trop compliquée.

Hermione commença à monter une courte spirale de marches en marbre jaune qui émergeait d'un moyeu central, un "secret" mal gardé qui était en fait l'un des moyens les plus rapides de passer des donjons Serpentard à la tour Serdaigle mais que seules les sorcières pouvaient traverser (le fait que les filles en particulier aient besoin d'un moyen rapide de passer de Serdaigle à Serpentard et de Serpentard à Serdaigle laissait Hermione légèrement perplexe). Au sommet de l'escalier, maintenant qu'elle s'éloignait des locaux Serpentard et entrait dans le corps principal de Poudlard, Hermione s'arrêta et ôta la cape d'invisibilité de Harry.

Après que sa bourse eut avalé la cape, Hermione tourna à droite et commença à avancer le long d'une coursive de quelques mètres, gardant maintenant sous surveillance son environnement automatiquement, sans vraiment y penser, et ses yeux en mouvement perpétuel passèrent sur une alcôve ombragée -

\emph{(brève désorientation)}

- puis une décharge de peur qui frappa son corps comme l'aurait fait un sortilège d'étourdissement, et elle découvrit que sans une pensée ni une décision consciente, sa baguette avait bondi dans sa main et pointait déjà vers…

… une cape noire si large et ondulante qu'il était impossible de dire si la silhouette qui s'y trouvait était celle d'un homme ou d'une femme, et au-dessus de la cape, un chapeau noir à larges bords~; et un brouillard noir qui semblait s'amonceler au-dessous et obscurcir le visage de celui qui ou de ce qui se trouvait derrière le brouillard.

"Rebonjour, Hermione," chuchota une voix sifflante située sous le chapeau et derrière le brouillard noir.

Le cœur de Hermione battait déjà à tout rompre dans sa poitrine, ses robes de sorcières étaient déjà humides de sueur et collaient à sa peau, elle pouvait déjà sentir le goût de la peur~; elle ne savait pas pourquoi elle était si soudainement emplie d'adrénaline mais sa main se resserra sur sa baguette. "Qui êtes-vous~?" demanda Hermione d'un ton autoritaire.

Le chapeau s'inclina légèrement~; la voix chuchotante, lorsqu'elle émergea du brouillard noir, était sèche comme de la poussière. "Le dernier allié," dit le chuchotement sifflant. "Celui qui répond enfin, quand aucun autre ne le fera. Je suis peut-être le seul \emph{véritable} ami que tu aies dans tout Poudlard, Hermione. Car tu as maintenant découvert comment les autres sont restés silencieux lorsque tu t'es retrouvée dans le besoin -

--- Quel est votre \emph{nom}~?"

La cape noire tourna légèrement de gauche à droite. Ça ne ressemblait \emph{pas} à un haussement d'épaules mais ça en transmettait le sens. "C'est là l'énigme, jeune Serdaigle. Jusqu'à ce que tu la résolves, tu peux m'appeler comme tu le souhaites."

Hermione pouvait sentir que sa paume suait déjà et elle ressentit de la gratitude pour les lignes striées qui aidaient sa main à garder une prise stable sur le bois. "Eh bien, M. Incroyablement Suspect," dit Hermione, "que me voulez-vous~?

--- Ce n'est pas la bonne question," répondit le chuchotement du brouillard noir. "Tu devrais plutôt me demander ce \emph{je} peux t'offrir à \emph{toi}.

--- Non," dit la jeune fille d'un ton assez calme, "franchement, je ne pense pas que je \emph{devrais} vous demander ça."

Un gloussement aigu émergea du brouillard noir. "Pas du pouvoir," chuchota la voix, "ni des richesses, car te soucies peu de telles choses, n'est-ce pas, jeune Serdaigle~? Du \emph{savoir}. C'est ce que je possède. Je sais ce qui se déroule dans cette école, tous les plans et les joueurs secrets, les réponses à toutes les énigmes. Je connais la raison derrière le froid que tu vois dans les yeux de Harry Potter. Je connais la véritable nature de la maladie mystérieuse du professeur Quirrell. Je sais qui Dumbledore craint vraiment.

--- Tant mieux pour vous," dit Hermione Granger. "Mais savez-vous combien de coups de langues il faut pour arriver au centre chocolat d'une sucette Tootsie~?"

Le brouillard noir sembla s'assombrir légèrement et la voix sembla plus basse, comme déçue. "Alors tu n'es même pas seulement curieuse, jeune Serdaigle, de connaître les vérités qui se cachent derrière les mensonges~?

--- Cent quatre-vingt sept," dit-elle. "J'ai essayé de compter un jour." Sa main glissait presque sur sa baguette et elle avait les doigts fatigués, comme si elle l'avait tenue pendant plusieurs heures au lieu de plusieurs minutes.

La voix siffla~: "Le professeur Rogue est secrètement un Mangemort."

Hermione faillit laisser tomber sa baguette.

--- Ah," chuchota la voix, satisfaite. "Je pensais que cela pourrais t'intéresser. Donc, Hermione. Y a-t-il quoi que ce soit que tu désires savoir au sujet de tes ennemis ou de ceux que tu appelles tes amis~?"

Elle leva les yeux vers le brouillard noir qui surplombait l'immense cape en essayant frénétiquement de mettre de l'ordre dans ses pensées. Le professeur Rogue était un Mangemort~? Qui \emph{lui} dirait une chose pareille, \emph{pourquoi}, qu'est-ce qui se \emph{passait}~? "C'est -" dit Hermione. Sa voix tremblait. "C'est extrêmement sérieux si c'est vrai. Pourquoi le dire à \emph{moi} et pas au directeur~?

--- Dumbledore n'a rien fait pour arrêter Rogue," chuchota le brouillard noir. "Tu l'as vu, Hermione. La pourriture de Poudlard commence à son sommet. Tout ce qui ne va pas dans cette école commence avec le directeur fou. Toi seule a osé le dire haut et fort - et c'est pour cela que je te parle.

--- Et avez-vous aussi parlé à Harry Potter alors~?" dit Hermione en gardant sa voix aussi stable que possible. Si c'était \emph{ça} le fantôme qui aidait Harry -

Le brouillard noir s'assombrit, s'éclaircit, comme un non de la tête. "Harry Potter m'effraie," chuchota-t-il. "La froideur de son regard et les ténèbres qui grandissent derrière celui-ci m'effraient. Harry Potter est un tueur et tous ceux qui lui font obstacle vont mourir. Même toi, Hermione Granger. Si tu oses t'opposer vraiment à lui, alors les ténèbres derrière son regard sortiront et te détruiront. Cela, je le sais.

--- Alors vous ne savez pas la moitié de ce que vous prétendez savoir," dit Hermione d'une voix un peu plus ferme. "Moi aussi, j'ai peur de Harry. Mais pas à cause de ce qu'il pourrait \emph{me} faire. À cause de ce qu'il pourrait faire pour me \emph{protéger} -

--- Faux." Le chuchotement était direct, dur, comme s'il ne comportait pas la possibilité d'un rejet. "Harry Potter \emph{finira} par se retourner contre toi, Hermione, lorsque les ténèbres l'auront entièrement pris. Il ne versera pas une larme, il ne le remarquera même pas, le jour où ses pas t'écraseront enfin.

--- \emph{Doublement} faux~!" dit-elle en élevant la voix même si des frissons descendaient le long de son échine. L'une des paroles de Harry lui revint~: "Que penses-tu savoir, au juste, et comment crois-tu que tu le sais~?

--- Le temps -" la voix sembla se reprendre. "Le temps d'en parler viendra plus tard. Pour l'instant, pour aujourd'hui, Harry Potter n'est en effet pas ton ennemi. Et pourtant tu cours un terrible danger.

--- \emph{Ça}, je veux bien le croire," dit Hermione Granger. Elle souhaitait désespérément faire passer sa baguette à son autre main car elle avait l'impression de devoir saisir son bras juste pour le garder levé~; sa tête la faisait souffrir comme si elle avait regardé le brouillard noir pendant des jours~; elle ne savait pas comment elle s'était épuisée si vite.

"Lucius Malfoy t'a remarqué, Hermione." Le chuchotement avait monté d'un cran, il avait pris un timbre, un ton d'inquiétude audible. "Tu as humilié la maison Serpentard, tu as vaincu son fils au combat. Même avant cela, tu étais une gêne pour tous ceux du camp des Mangemort, car tu es née-Moldue et possèdes pourtant des capacités magiques supérieures à celles de n'importe quel Sang Pur. Et tu commences maintenant à être connue, à être observée par le monde extérieur. Lucius Malfoy cherche à t'écraser, Hermione, à te faire mal, peut-être même à te tuer, et il a les moyens de le faire~!" Le chuchotement était devenu pressant.

Il y eut un silence.

"Ce sera tout~?" dit Hermione. Si elle avait été l'ex-colonel Zabini ou Harry Potter, elle aurait probablement posé des questions intelligentes afin d'obtenir plus d'informations, mais son esprit était ralenti, fatigué. Il fallait vraiment qu'elle parte de là et qu'elle aille s'allonger un moment.

"Tu ne me crois pas," dit le chuchotement, à présent plus doux et plus triste. "Pourquoi pas, Hermione. \emph{J'essaie} de t'aider."

Hermione fit un pas en arrière pour s'éloigner de l'alcôve ombragée.

"\emph{Pourquoi pas}, Hermione~?" exigea la voix, redevenue un sifflement. "Tu me dois cela~! Dis-moi, et alors -" la voix s'interrompit et recommença, plus basse. "Et alors tu pourras partir, je suppose. Seulement dis-moi… pourquoi…"

Peut-être n'aurait-elle pas dû répondre~; peut-être aurait-elle dû juste se retourner et fuir, ou encore mieux, commencer par lancer une barrière prismatique et courir ensuite en hurlant à s'en percer les poumons~; mais la note de douleur sincère dans la voix la saisit et elle répondit.

"Parce que vous avez l'air incroyablement sombre et suspect," dit Hermione en gardant un ton poli et en maintenant sa baguette pointée vers l'immense cape noire et le brouillard sans visage.

"C'est \emph{tout}~?" dit la voix, incrédule. Elle semblait imprégnée de tristesse. "J'attendais mieux de toi, Hermione. Une Serdaigle telle que toi, la plus intelligente à faire l'honneur de sa présence à Poudlard depuis une génération, devrait sûrement savoir que les apparences peuvent être trompeuses.

--- Oh, je sais bien," dit Hermione. Elle fit un pas de plus en arrière, ses doigts fatigués resserrés sur la baguette. "Mais ce que les gens oublient parfois, c'est que même si les apparences \emph{peuvent} être trompeuses, elles ne le sont généralement \emph{pas}."

Il y eut un silence.

"C'est \emph{toi} qui es intelligente," dit la voix, puis le brouillard s'évapora et ne dissimula plus rien~; Hermione vit le visage qui se trouvait en dessous et le reconnaître envoya une décharge d'adrénaline et de terreur à travers son corps -

\emph{(brève désorientation)}

- puis une décharge de peur qui frappa son corps comme l'aurait fait un sortilège d'étourdissement, et elle découvrit que sans une pensée ni une décision consciente, sa baguette avait bondi dans sa main et pointait déjà vers…

… une dame scintillante, sa longue robe blanche ondulante comme agitée de vents invisibles, on ne pouvait voir ni ses mains ni ses pieds, son visage était caché par un voile blanc et elle irradiait de lumière, pas comme un fantôme, pas transparente, juste entourée d'une douce lumière blanche.

Hermione regarda la douce apparition, bouche bée, en se demandant pourquoi son cœur tambourinait déjà dans sa poitrine et pourquoi elle se sentait aussi effrayée.

"Rebonjour, Hermione." Le chuchotement bienveillant émanait de la lueur blanche derrière le voile. "J'ai été envoyée pour vous aider, aussi ne soyez pas effrayée. Je vous servirai en toutes choses, car vous, ma Dame, êtes promise au plus merveilleux des destins -"

\begin{center}
…

…

…
\end{center}

%  LocalWords:  unsilence Glasswell
