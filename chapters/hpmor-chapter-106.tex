\partchapter{La Vérité}{III}

\lettrine{A}{près} avoir fait un pas dans la pièce interdite, Harry poussa un cri perçant et bondit en arrière. Il se cogna contre le professeur Rogue, et ils s'effondrèrent par terre.

Ce dernier se releva et reprit son poste devant la porte. Il fit pivoter sa tête vers Harry~: "Je garde cette porte sur ordre du directeur," dit-il de son ton sardonique habituel. "Allez-vous-en sur le champ, où je déduirai des points de Maison~!"

C'était à glacer le sang, mais Harry était plus préoccupé par l'immense chien à trois têtes qui venait de se jeter sur lui avant d'être retenu par la chaîne attachée à son collier.

"C'est… c'est… c'est…" dit Harry.

"Oui," dit le professeur Quirrell, loin derrière lui, "c'est bien l'occupant normal de cette pièce, qui est interdite à tous les élèves et surtout à ceux qui sont en première année."

"\emph{Même des sorciers devraient trouver ça dangereux~!}" Dans la pièce, la gigantesque créature noire émit un chœur de mugissements. Des flots de salive jaillissaient de ses trois gueules pleines de crocs.

Le professeur Quirrell soupira. "Un sortilège l'empêche de manger les élèves, il les recrache seulement de l'autre côté de la porte. Maintenant, petit, comment penses-tu que nous devrions nous occuper de cette créature~?"

"Euh," bégaya Harry. Il essayait de s'entendre penser malgré le rugissement continu du gardien de cette pièce. "Euh. Si c'est comme le cerbère de la légende Moldue 'Orphée et Eurydice', on doit lui chanter une berceuse pour pouvoir passer…"

"\emph{Avada Kedavra.}"

La bête à trois têtes tomba.

Harry se retourna vers le professeur Quirrell. Celui-ci avait l'air extrêmement déçu et semblait se demander si Harry avait jamais assisté à un seul de ses cours.

"J'ai plus ou moins \emph{supposé}," dit Harry entre deux halètements, "que ne pas surmonter ces épreuves à la façon d'un élève de première année pourrait déclencher une \emph{alarme}."

"Petit, c'est un mensonge. Tu as simplement oublié mes leçons à ta première véritable occasion de t'en souvenir. Quant aux alarmes, j'ai passé des mois à embrouiller les protections et les détecteurs de ces pièces."

"Alors pourquoi est-ce que vous m'avez fait passer devant~?"

Le professeur Quirrell se contenta de sourire. Il avait l'air beaucoup plus maléfique que d'habitude.

"Je vois," dit Harry avant d'entrer lentement dans la pièce, toujours tremblant.

La pièce, entièrement en pierre, était illuminée d'un bleu pâle venu de recoins arrondis creusés dans le mur, comme si la lumière d'un ciel gris avait traversé des fenêtres qui n'étaient pas là. Au fond de la pièce, une trappe de bois au sol, munie d'un seul anneau. Au milieu, un gigantesque chien mort et ses trois têtes sans vie.

Harry se tourna vers l'un des recoins arrondis et alla le regarder de plus près. Il n'y avait rien là sinon la lueur bleue sans source, alors il continua pour observer le suivant, examinant le mur au passage.

"Que fais-tu~?", demanda le professeur Quirrell.

"J'inspecte la pièce," dit Harry. "Il pourrait y avoir un indice, quelque chose d'écrit, une clé utile plus tard, quelque chose comme ça."

"Dis-tu cela sérieusement~? Ou essaies-tu de nous ralentir~? Réponds en Fourchelangue."

Harry se retourna. "\parsel{J'étais ssérieux}," siffla Harry. "\parsel{J'aurais fait pareil ssi j'avaiss été sseul.}"

Le professeur Quirrell se passa une main sur le front. "J'admets," dit-il, "que ton approche aurait un intérêt dans la tombe d'Amon-Set, par exemple. Donc je ne vais pas te traiter d'idiot. Mais quand même. Le faux puzzle, la partie visible du défi, c'est un jeu pour les première année. Nous allons simplement ouvrir cette trappe."

Sous la trappe se trouvait une plante immense, semblable à un dieffenbachia géant. Ses larges feuilles sortaient de sa tige centrale pour former un escalier en colimaçon, mais elles étaient plus sombres que celles d'un dieffenbachia normal~; d'autres tiges veineuses poussaient de la tige centrale et pendaient dans le vide. La base était plus large, recouverte de feuilles encore plus grandes. Elle avait l'air de promettre une douce réception à celui qui tomberait. En dessous, une autre pièce de pierre, proche de la première, avec les mêmes recoins comme de fausses fenêtres, sources de la même lumière bleu-gris.

"Une idée évidente est de voler jusqu'au sol avec le balai dans ma bourse ou de jeter quelque chose de lourd pour voir si ces petites tiges sont piégées," dit Harry en jetant un coup d'œil en bas. "Mais j'imagine que vous allez juste me dire de descendre l'escalier de feuilles." Elles avaient vraiment l'air de vouloir servir d'escalier en colimaçon.

"Après toi," dit le professeur Quirrell.

Harry posa un pied prudent sur la première feuille et découvrit qu'elle supportait bien son poids. Puis il regarda la pièce une dernière fois avant de la quitter, au cas où quelque chose attirerait son attention.

L'énorme chien mort attirait tant le regard qu'il était difficile de se concentrer sur autre chose.

"Professeur Quirrell," et il s'empêcha de continua par~: \emph{votre façon d'écarter les obstacles a des défauts évidents}, "et si quelqu'un passait sa tête par la porte et voyait que le cerbère est mort~?"

"Dans ce cas, la personne aurait aussi vu que Rogue n'est pas dans son état normal," répondit le professeur Quirrell. "Mais puisque tu insistes…" le professeur de Défense s'avança jusqu'au chien et posa sa baguette sur la tête de ce dernier. Il entonna un sortilège aux accents latins qui, pour Harry, s'accompagna d'un sentiment d'appréhension grandissant. Comme depuis toujours, le Survivant ressentait le pouvoir du Seigneur des Ténèbres.

Le dernier mot fut "\emph{Inferius}" et fut, pour Harry, accompagné d'une sorte de pulsion intérieure~: \emph{NON, ARRÊTE}.

Alors le chien à trois têtes se leva, ses six yeux mornes, vides, et il se tourna de nouveau vers la porte.

Harry regarda l'immense Inferius et il sentit son estomac se retourner. Il n'avait connu que deux sensations pires que celle-ci.

Il sut alors qu'il avait vu et ressenti cette procédure auparavant. Seule l'incantation latine avait manqué.

Le centaure qui s'était confronté à lui dans la Forêt Interdite était mort. Le professeur de Défense lui avait envoyé un véritable Avada Kedavra, pas un faux.

Presque inconsciemment, Harry avait continué de croire que si Hermione pouvait \emph{revenir}, lui pourrait revenir à l'éthique de Batman~: personne ne meurt. La plupart des gens traversaient les aventures de leur vie sans que personne ne meure.

Il n'en serait rien.

Il n'avait même pas remarqué le jour où cette possibilité lui avait échappé. Même si Hermione ressuscitait maintenant, il était trop tard pour que cette histoire se termine sans victime.

Il ne savait même pas comment le centaure s'était appelé.

Il se dit que soit le professeur de Défense confirmerait l'accusation en Fourchelangue, soit il mentirait de sa voix normale. Dans un cas comme dans l'autre, ses soupçons envers Harry augmenteraient. Mais Harry savait que - même s'il ignorait \emph{comment} il arrêterait le professeur Quirrell, même s'il n'osait pas le trahir activement ni même \emph{décider} de le faire avant que l'heure de vaincre n'ait presque sonné - il savait que le choses ne se régleraient jamais à l'amiable entre Lord Voldemort et lui~; que ces deux différents esprits ne pouvaient pas vivre dans le même monde.

Et ce fut comme si cette résolution, cette certitude dans l'opposition, avait fait appel à la force que Harry avait toujours considérée comme son côté obscur. Il avait cessé d'y faire délibérément appel après le jour où il avait tué le troll. Mais ce côté obscur n'avait jamais été séparé de lui. C'était un souvenir de Tom Jedusor. Il ne savait pas ce qui avait mené à ça, mais en supposant que ce soit bien le cas, il devait faire usage de ce qui restait d'intelligence dans ce côté obscur. Pas comme un mode de pensé alternatif, comme il l'avait initialement conçu, mais comme des motifs neuronaux fortement enclins à se mêler les uns aux autres - puisqu'ils avaient jadis fait partie d'un tout.

Ce qui ne changeait malheureusement rien au fait que le professeur Quirrell avait les mêmes capacités, beaucoup plus d'expérience derrière, et aussi un pistolet.

Harry se retourna, posa le pied sur la plante géante et commença à descendre l'escalier en colimaçon offert par les feuilles. Il avait mis trop longtemps à se remettre, cette fois, mais, malgré le chagrin qui l'écrasait comme des litres d'eau, il commençait à revenir. Ce n'était pas une barre de métal froid qu'il sentait dans son échine, mais quand même quelque chose de solide, de droit. Il allait jouer à ce jeu, il ferait revenir Hermione, et alors seulement, il arrêterait le professeur Quirrell. Ou il l'arrêterait d'abord et il récupérerait la pierre lui-même. Il fallait bien qu'il y ait une possibilité, qu'une opportunité se présente, une façon d'arrêter Voldemort \emph{et} de rendre la vie à Hermione.

Harry continua sa descente.

Derrière lui, le chien à trois têtes attendait, gardait la porte. 

%  LocalWords:  fter Amon
