\chapter{Rationalisation}

\lettrine{H}{ermione} Granger avait peur d'être devenue Mauvaise.

Hermione Granger avait peur d'être devenue Mauvaise.

La différence entre Bon et Mauvais était généralement simple à saisir, elle n'avait jamais compris pourquoi les autres avaient tant de difficulté. À Poudlard, "Bon", c'était le professeur Flitwick et le professeur McGonagall et le professeur Chourave. "Mauvais", c'était le professeur Rogue et le professeur Quirrell et Drago Malfoy. Harry Potter… était un de ces cas inhabituels où on ne \emph{pouvait pas} savoir juste en regardant. Elle essayait encore de décider à quel groupe il appartenait.

Mais en ce qui \emph{la} concernait…

Hermione s'amusait \emph{beaucoup trop} à écraser Harry Potter.

Elle avait mieux réussi que lui à chacun de leurs cours (à part le vol sur balais, qui était comme la gym~: ça ne comptait pas). Elle avait gagné de \emph{vrais} points pour Serdaigle presque tous les jours de sa première semaine, pas pour des trucs héroïques bizarres mais pour des choses \emph{intelligentes}, comme apprendre des sorts rapidement ou aider d'autres élèves. Elle savait que ces points étaient de meilleurs points, et ce qui était encore mieux c'était que Harry Potter le savait aussi. Elle pouvait le voir dans ses yeux à chaque fois qu'elle gagnait un \emph{vrai} \emph{point} de plus.

On n'était pas censé autant aimer gagner quand on était Bon.

Ça avait commencé pendant le voyage en train, mais il avait fallu un moment avant que le cyclone ne s'apaise. Ce n'était pas avant plus tard cette nuit-là que Hermione avait commencé à comprendre \emph{à quel point} elle avait laissé ce garçon lui marcher dessus.

Avant d'avoir connu Harry Potter, elle n'avait jamais rencontré quelqu'un qu'elle aurait voulu écraser. Si quelqu'un ne se débrouillait pas aussi bien qu'elle en classe, c'était son boulot de l'aider, pas de l'enfoncer. C'était ça, être Bon.

Et maintenant…

… maintenant elle \emph{gagnait}, Harry Potter tressaillait chaque fois qu'elle gagnait un autre point, c'était \emph{tellement} amusant~; ses parents l'avaient mise en garde contre la drogue et elle soupçonnait que gagner, c'était \emph{encore plus} amusant.

Elle avait toujours aimé les sourires que les enseignants lui prodiguaient quand elle faisait bien les choses. Elle avait toujours aimé voir la longue ligne de cases cochées sur les contrôles où elle avait parfaitement répondu. Mais maintenant, quand elle travaillait bien en cours, elle jetait nonchalamment un regard autour d'elle, et elle apercevait Harry Potter qui grinçait des dents, et ça lui donnait envie de se mettre à chanter comme dans un film de Disney.

C'était Mauvais, non~?

Hermione avait peur d'être devenue Mauvaise.

Et une pensée lui était alors venue, qui avait effacé toutes ses peurs.

Elle et Harry vivaient le début d'une romance~! Bien sûr~! Tout le monde savait ce que ça voulait dire quand un garçon et une fille commençaient à se battre en permanence. Ils se \emph{faisaient la cour}~! Il n'y avait rien de Mauvais à \emph{ça}.

Ce n'était pas juste qu'elle \emph{aimait} battre jusqu'à la mort (scolaire) l'un des étudiants les plus connus de l'école, quelqu'un qui était \emph{dans} des livres et \emph{parlait} comme des livres, le garçon qui était parvenu à vaincre le Seigneur des Ténèbres et qui avait même écrasé le \emph{professeur Rogue} comme si c'était un triste petit insecte, le garçon qui, comme le professeur Quirrell l'aurait dit, dominait tout le monde en première année à Serdaigle, \emph{sauf} Hermione Granger qui \emph{pulvérisait} complètement le Survivant dans tous ses cours sauf en vol sur balais.

Parce que ça aurait été Mauvais.

Non. C'était une Romance. C'était \emph{ça}. C'était \emph{pour ça} qu'ils se battaient.

Hermione était heureuse d'avoir compris cela à temps pour aujourd'hui, quand Harry perdrait leur concours de lecture, dont \emph{toute l'école} était au courant, et elle voulait commencer à \emph{danser} sous l'impulsion de joie pure et débordante que cela lui causait.

Il était 14h45, on était samedi, et Harry Potter avait encore la moitié d'\emph{Une Histoire de la Magie} de Bathilda Tourdesac à lire et Hermione regardait sa montre de poche alors qu'elle tic-taquait avec une épouvantable lenteur jusqu'à 14h47.

Et toute la salle commune de Serdaigle regardait.

Ce n'étaient pas juste les première année, la nouvelle s'était répandue comme du lait renversé et au moins la moitié de Serdaigle s'était tassée dans la pièce, serrée dans des sofas, s'appuyant sur des étagères de livres, assise sur des accoudoirs de fauteuils. Les six préfets étaient là, y compris les préfets en chef de Poudlard. Quelqu'un avait dû jeter un sort de Rafraîchissement d'Air juste pour qu'il y ait assez d'oxygène. Et le vacarme de la conversation s'était éteint en des murmures qui s'étaient maintenant effacés pour laisser place à un silence complet.

14h46.

La tension était insupportable. Si ça avait été n'importe qui d'autre, \emph{n'importe qui d'autre}, sa défaite aurait été courue d'avance.

Mais c'était Harry Potter, et on ne pouvait ignorer la possibilité que, dans les secondes à suivre, il lève une main et claque des doigts.

Avec une terreur soudaine, elle se rendit compte à quel point Harry Potter était susceptible de faire exactement ça. Ça lui ressemblerait \emph{tout à fait} d'avoir \emph{déjà fini de lire} la deuxième moitié du livre.

La vue de Hermione commença à se brouiller. Elle essaya de se forcer à respirer, et découvrit qu'elle en était simplement incapable.

Plus que dix secondes, et il n'avait toujours pas levé sa main.

Plus que cinq secondes.

14h47.

Harry Potter replaça précautionneusement le marque-page dans le livre, referma ce dernier et le mit de côté.

"Je voudrais noter, au bénéfice de la postérité," dit le Survivant d'une voix claire, "que je n'avais qu'un demi-livre à lire, et que j'ai fait face à un certain nombre de délais inattendus -"

"\emph{Tu as perdu~!}" glapit Hermione. "\emph{Perdu~!} Tu \emph{as perdu notre concours~!"}

Il y eut une expiration collective tandis que tout le monde recommençait à respirer.

Harry Potter lui jeta un regard de Feu Ardent, mais elle flottait dans un halo de joie pure et blanche et rien ne pouvait l'atteindre.

"\emph{Tu te rends compte du genre de semaine que j'ai eu} ?" dit Harry Potter. "Tout être inférieur aurait eu du mal à lire huit livres de Dr. Seuss~!"

"\emph{Tu} as fixé la limite de temps."

Le regard de Feu Ardent de Harry devint encore plus brûlant. "Je n'avais aucune façon logique de savoir que j'allais devoir sauver toute l'école du professeur Rogue, ou me faire battre en cours de Défense, et si je te disais comment j'ai perdu mon temps entre 17h et le dîner jeudi tu penserais que je suis fou -"

"Ooooh, on dirait que \emph{quelqu'un} est devenu la proie de l'\emph{illusion de la planification}."

Le visage de Harry révéla une profonde stupéfaction.

"Oh, ça me rappelle que j'ai fini de lire le premier lot de livres que tu m'as prêté," dit Hermione de son air le plus innocent. Deux avaient été \emph{difficiles}. Elle se demandait combien de temps ça \emph{lui} avait pris pour finir de les lire.

"Un jour," dit le Survivant, "quand les lointains descendants d'\emph{Homo sapiens} contempleront l'histoire de la galaxie et se demanderont quand les choses ont mal tournées, ils concluront que l'erreur originelle a été commise quand quelqu'un a appris à Hermione Granger à lire."

"Mais tu as quand même perdu," dit Hermione. Elle avait une main sur son menton et semblait contemplative. "Mais je me demande, que devrais-tu perdre exactement~?"

"\emph{Quoi~?}"

"Tu as perdu le pari," expliqua Hermione, "tu dois donc verser un gage."

"Je ne me souviens pas d'avoir consenti à ça~!"

"Vraiment~?" dit Hermione Granger. Elle se donna l'air pensive. Alors, comme si l'idée venait de se présenter à elle~: "Alors faisons un vote. Toute personne à Serdaigle qui pense que Harry Potter doit payer, levez votre main~!"

"\emph{Quoi} ?" glapit de nouveau Harry Potter.

Il pivota et vit qu'il était entouré par une mer de mains levées.

Et si Harry Potter avait fait \emph{plus attention}, il aurait remarqué qu'une énorme partie des observateurs semblaient être des filles et que pratiquement toutes les filles de la pièce avaient leur main levée.

"Stop~!" gémit Harry Potter. "Vous ne savez pas ce qu'elle va demander~! Vous \emph{rendez-vous} compte de ce qu'elle est en train de faire~? Elle vous pousse à vous engager à l'avance maintenant, et alors la pression de la cohérence vous fera accepter tout ce qu'elle dira ensuite~!"

"Ne t'en fais pas," dit la préfète Pénélope Deauclaire. "Si elle demande quelque chose de déraisonnable, nous pourrons toujours changer d'avis. Pas vrai, tout le monde~?"

Et il y eut des hochements de tête empressés venus de toutes les filles auxquelles Pénélope Deauclaire avait parlé du plan de Hermione.

\later

Une silhouette silencieuse se glissa à travers les frais couloirs des donjons de Poudlard. Il devait être présent dans une pièce précise à 18h pour rencontrer un certain quelqu'un, et si possible il valait mieux être en avance, par signe de respect.

Mais quand sa main tourna la poignée et poussa la porte dans cette salle sombre, silencieuse et abandonnée, il y avait une silhouette se tenant déjà là, entre les rangées de vieux bureaux poussiéreux. Une silhouette qui tenait un petit bâton vert luisant projetant une lumière pâle qui éclairait à peine celui qui le tenait, et encore moins la pièce environnante.

La lumière du couloir mourut quand la porte se referma derrière lui, et les yeux de Drago commencèrent à s'ajuster à la faible lueur.

La silhouette se tourna lentement pour l'observer, révélant un visage ombragé, seulement partiellement éclairé par l'étrange lumière verte.

Drago commençait déjà à aimer cette rencontre. Gardez la froide lumière verte, rendez-les tous deux plus grands, donnez-leur des capuches et des masques, déplacez-les d'une salle de classe à un cimetière, et ce serait exactement comme le début de la moitié des histoires que les amis de son père racontaient sur les Mangemorts.

"Je veux que tu saches, Drago Malfoy," dit la silhouette avec un calme mortel, "que je ne te blâme pas pour ma récente défaite."

Sans réfléchir, Drago ouvrit la bouche pour protester, il n'y avait absolument aucune raison pour qu'\emph{il} soit blâmé.

"C'était, plus que toute autre chose, dû à ma propre stupidité," continua la sombre silhouette. "J'aurais pu faire beaucoup d'autres choses, et ce à toutes les étape du parcours. Tu ne m'as pas demandé de faire \emph{exactement} ce que j'ai fait. Tu m'as seulement demandé mon aide. C'est moi qui, malavisé, ai choisi cette méthode en particulier. Mais le fait demeure que j'ai perdu le concours d'un demi-livre. Les actes de ton idiot de compagnie, et la faveur que tu as demandée, et, oui, ma propre idiotie quant à la façon de t'accorder cette faveur, m'ont fait \emph{perdre du temps}. Plus de temps que tu ne le sais. Du temps qui, à la fin, s'est avéré crucial. Le fait demeure, Drago Malfoy, que si tu ne m'avais pas demandé cette faveur, j'\emph{aurais} gagné. Et non… pas… \emph{perdu}."

Drago avait déjà entendu parler de la défaite de Harry, et du gage que Hermione avait exigé de lui. La nouvelle avait circulé plus vite que des chouettes n'auraient pu la transporter.

"Je comprends," dit Drago. "Je suis désolé." Il n'y avait rien d'autre qu'il \emph{puisse} dire s'il voulait que Harry Potter soit ami avec lui.

"Je ne demande ni compassion ni chagrin," dit la sombre silhouette, toujours d'un calme mortel. "Mais je viens de passer deux heures entières en compagnie de Hermione Granger, habillé des vêtements qui m'avaient été fournis, à visiter des endroits fascinants de Poudlard tels que la petite chute gargouillante de ce qui m'a semblé être de la morve, accompagné d'une quantité d'autres filles qui insistaient pour s'adonner à de serviables activités telles que répandre des pétales de rose métamorphosés sur notre chemin. J'ai eu un rendez-vous galant, héritier de Malfoy. Mon \emph{premier} rendez-vous galant. \emph{Et quand je déclarerai que tu me dois cette faveur, tu la repaieras.}"

Drago acquiesça solennellement. Avant d'arriver, il avait pris la prudente précaution d'apprendre tout détail disponible sur le rendez-vous de Harry afin d'en finir avec son hilarité hystérique avant l'heure de rencontre convenue, et de ne pas commettre de \emph{faux pas} en gloussant jusqu'à en perdre conscience.

"Penses-tu," dit Drago, "que quelque chose devrait arriver à cette Granger-"

"Fais passer le mot à Serpentard que la Granger est \emph{mienne} et que quiconque se mêle de \emph{mes} affaires verra ses restes éparpillés dans une zone assez grande pour inclure douze différentes langues vivantes. Et puisque je ne suis pas à Gryffondor et que j'utilise la \emph{ruse} plutôt que les attaques frontales immédiates, ils ne devraient pas paniquer en me voyant sourire à Granger."

"Ou si tu es vu à un second rendez-vous~?" dit Drago, ne laissant percer dans sa voix qu'une toute petite note de scepticisme.

"\emph{Il n'y aura pas de second rendez-vous}," dit la silhouette nimbée de vert d'une voix si effrayante qu'elle semblait être non seulement celle d'un Mangemort mais aussi celle d'Amycus Carrow ce jour-là, juste avant que Père ne lui dise d'arrêter, qu'il n'était pas le Seigneur des Ténèbres.

Bien sûr c'\emph{était} toujours la voix non muée d'un jeune garçon, et quand on la combinait avec les \emph{mots qu'il avait prononcés}, eh bien, ça ne fonctionnait pas. Si Harry Potter \emph{devenait} un jour le prochain Seigneur des Ténèbres, Drago utiliserait une Pensine pour conserver une copie de ce souvenir dans un endroit sûr, et Harry Potter n'oserait plus jamais le trahir.

"Mais parlons d'affaires plus joyeuses," dit la silhouette aux ombres verdâtres. "Parlons de savoir et de pouvoir, Drago Malfoy, parlons de Science."

"Oui," dit Drago. "Parlons donc."

Drago se demanda quelle partie de son visage était visible, et quelle partie était dans l'ombre, dans cette inquiétante lumière verte.

Et bien que Drago ait gardé une expression sérieuse, il y avait un sourire dans son cœur.

Il avait \emph{enfin} une vraie conversation d'adulte.

"Je t'offre un pouvoir," dit la sombre silhouette, "et je te parlerai de ce pouvoir et de son prix. Le pouvoir a son origine dans la connaissance de la forme de la réalité et du contrôle qu'on obtient alors sur elle. Ce que tu comprends, tu peux le commander, et c'est un pouvoir assez grand pour permettre de marcher sur la Lune. Le prix de ce pouvoir est que tu dois apprendre à poser des questions à la Nature, et, encore plus difficile, à accepter ses réponses. Tu feras des expériences, réaliseras des tests et observeras ce qui se produit. Et tu devras accepter le sens de ces résultats quand ils te diront que tu as tort. Tu devras \emph{apprendre à perdre}, pas contre moi, mais contre la Nature. Quand tu te retrouveras à débattre avec la réalité, tu devras laisser la réalité gagner. Ce sera douloureux, Drago Malfoy, et je ne sais pas si tu as la force nécessaire dans ce domaine. Maintenant que tu connais le prix, est-ce toujours ton souhait que d'apprendre le pouvoir humain~?"

Drago prit une profonde inspiration. Il y avait déjà réfléchi. Et il avait du mal à voir comment il pourrait répondre autrement. Il avait reçu l'instruction de s'engager dans toute voie pouvant mener à une amitié avec Harry Potter. Il ne faisait qu'\emph{apprendre}, il ne promettait pas de \emph{faire} quoi que ce soit. Il pouvait toujours arrêter les leçons n'importe quand…

Il y avait un certain nombre de choses dans cette situation qui lui donnaient l'air d'être un piège, mais honnêtement, Drago ne voyait pas comment cela pouvait mal tourner.

En plus Drago voulait quand même diriger le monde.

"Oui," dit Drago.

"Excellent," dit la sombre silhouette. "J'ai eu ce qu'on pourrait appeler une \emph{semaine encombrée}, et ça prendra du temps de planifier ton programme d'études-"

"J'ai moi-même beaucoup de choses à faire pour consolider mon pouvoir à Serpentard," dit Drago, "sans parler des devoirs. Peut-être qu'on devrait juste commencer en octobre~?"

"Ça m'a l'air raisonnable," dit la sombre silhouette, "mais ce que je voulais dire c'est que pour préparer ton programme, je dois savoir ce que je vais t'enseigner. Trois pensées me viennent. La première est que je t'explique l'esprit et le cerveau humain. La deuxième option est que je t'explique l'univers physique, ces arts qui pavent le chemin vers la Lune. Cela nécessite l'utilisation de beaucoup de nombres, mais pour un certain type d'esprit, ces nombres sont plus beaux que tout ce que la Science a à offrir. Aimes-tu les nombres, Drago~?"

Drago secoua la tête.

"Alors tant pis. Tu finiras par apprendre les mathématiques, mais je ne pense pas que ce sera pour tout de suite. La troisième option est que je t'enseigne la génétique et l'évolution et l'hérédité, ce que tu appellerais le sang -"

"Celle-là," dit Drago.

La silhouette hocha la tête. "Je pensais que tu dirais ça. Mais je pense que ce sera le chemin le plus douloureux pour toi, Drago. Et si ta famille et tes amis, les puristes du sang, disent une chose, et que tu découvres que les tests expérimentaux en disent une autre~?"

"Alors je me débrouillerai pour que les tests expérimentaux donnent la \emph{bonne} réponse~!"

Il y eut une pause, et la sombre silhouette se tint là, bouche béante, pendant quelques instants.

"Euh," dit la sombre silhouette. "Ça ne marche pas vraiment comme ça. C'est ce contre quoi j'essayais de te mettre en garde, Drago. Tu ne \emph{peux pas} changer la réponse pour qu'elle te plaise."

"On peut \emph{toujours} faire en sorte que la réponse soit celle qui nous arrange," dit Drago. Ça avait pratiquement été la première chose que ses précepteurs lui avaient enseignée. "Le problème est juste de trouver les bons arguments."

"Non," dit la sombre silhouette, sa voix s'élevant sous l'effet de la frustration, "non, non, non~! Alors tu obtiens la \emph{mauvaise} réponse, et ce n'est pas comme ça que tu pourras aller sur la Lune~! La Nature n'est pas une personne, tu ne peux pas la duper et lui faire croire autre chose, si tu essaies de dire que la Lune est faite de fromage tu pourras argumenter pendant des jours et ça ne changera pas la Lune~! Ce dont tu parles, c'est de \emph{rationaliser}, comme de commencer avec une feuille de papier, d'aller directement à la dernière ligne, d'utiliser de l'encre pour écrire 'et \emph{par conséquent}, la Lune est faite de fromage', puis de remonter tout en haut pour écrire plein d'arguments ingénieux. Mais soit la Lune est faite de fromage, soit elle ne l'est pas. Au moment où tu as écrit la conclusion, c'était déjà vrai ou faux. Que la page se termine par la bonne ou la mauvaise conclusion est déterminé à l'instant où tu écris cette conclusion. Si tu essaies de choisir entre deux malles très chères, et que tu aimes celle qui brille, les arguments ingénieux que tu inventes pour justifier son achat n'ont aucune importance, la \emph{vraie} règle que tu as utilisée pour \emph{choisir pour quelle malle tu allais argumenter} était 'prend celle qui brille', et quelle que soit l'efficacité de cette règle en ce qui concerne la sélection de malles, c'est le genre de malle que tu auras. La rationalité \emph{ne peut pas} être utilisée pour argumenter en faveur d'un camp déjà fixé, il est seulement possible de l'utiliser pour \emph{décider en faveur de quel camp tu vas argumenter}. La Science n'est pas faite pour \emph{convaincre} quiconque que les puristes du sang ont raison. Ça, c'est de la \emph{politique~!} Le pouvoir de la science trouve son origine dans \emph{la découverte du véritable fonctionnement de la Nature, et du fait qu'elle ne peut être changée par le débat}~! Ce que la science \emph{peut} faire est de nous dire \emph{comment le sang fonctionne vraiment}, comment les sorciers héritent vraiment leurs pouvoirs de leurs parents, et si les nés-Moldus sont vraiment plus faibles ou plus forts -"

"\emph{Plus forts~!}" dit Drago. Il avait essayé de suivre, un air perplexe sur le visage, il pouvait comprendre en quoi ça se tenait, mais ça ne ressemblait certainement pas à quoi que ce soit qu'il ait entendu avant. Et Harry Potter avait alors dit quelque chose que Drago ne pouvait absolument pas laisser passer. "Tu penses que les sang-de-bourbe sont \emph{plus forts}~?"

"Je ne pense rien," dit la sombre silhouette. "Je ne sais rien. Je ne crois rien. Ma conclusion n'est pas encore écrite. Je découvrirai comment tester la force magique des nés-Moldus, et la force magique des Sang-Purs. Si mes tests me disent que les nés-Moldus sont plus faibles, je croirai qu'il sont plus faibles. Si mes tests me disent que les nés-Moldus sont plus forts, je croirai qu'ils sont plus forts. Sachant cela et d'autres faits, je gagnerai une certaine quantité de pouvoir -"

"Et tu t'attends à ce que \emph{je} crois tout ce que tu dis~?" demanda vivement Drago.

"Je m'attends à ce que tu exécutes les tests \emph{toi-même}," dit calmement la sombre figure. "As-tu peur de ce que \emph{tu} vas découvrir~?"

Drago fixa la sombre silhouette pendant un moment, ses yeux étroits. "Bon piège, Harry," dit-il. "Il faudra que je m'en souvienne, c'est nouveau."

La sombre silhouette secoua la tête. "Ce n'est pas un piège, Drago. Souviens-toi - je ne \emph{sais pas} ce que nous allons découvrir. Mais on ne comprend pas l'univers en débattant avec lui ou en lui disant de revenir avec une autre réponse la prochaine fois. Quand tu enfiles la robe d'un scientifique, tu dois oublier toute la politique et tous les arguments et toutes les factions et tous les camps, faire taire les récriminations désespérées de ton esprit, et ne souhaiter qu'entendre la réponse de la Nature." La sombre silhouette s'interrompit. "La plupart des gens ne peuvent pas faire ça. C'est pour ça que c'est difficile. Es-tu certain de ne pas préférer apprendre le fonctionnement du cerveau~?"

"Et si je te dis que je préférerais apprendre le fonctionnement du cerveau," dit Drago, sa voix maintenant dure, "tu raconteras à tout le monde que j'avais peur de ce que j'allais découvrir."

"Non," dit la sombre silhouette. "Je ne ferai pas ça."

"Mais tu feras peut-être le même genre de tests toi-même, et si tu obtiens la mauvaise réponse, je ne serai pas là pour dire quelque chose avant que tu ne la montres à quelqu'un d'autre." La voix de Drago était toujours dure.

"Je te demanderais quand même avant, Drago," dit calmement la sombre silhouette.

Drago resta silencieux. Il ne s'était pas attendu à ça, il pensait avoir vu le piège mais… "Tu le \emph{ferais}~?"

"Bien sûr. Comment saurais-\emph{je} qui faire chanter et ce qu'on pourrait tirer d'eux~? Drago, je te dis à nouveau que ce n'est pas un piège que je te tends. Du moins pas à toi personnellement. Si tu avais d'autres opinions politiques, je serais en train de te dire~: et si les tests montrent que les Sang-Pur sont plus forts~?"

"Vraiment."

"\emph{Oui~!} C'est le prix que \emph{tout le monde} doit payer pour devenir un scientifique~!"

Drago leva une main. Il devait réfléchir.

La sombre silhouette nimbée de vert attendit.

En fait, ça ne prit pas longtemps d'y réfléchir. Si on écartait toutes les parties déroutantes… alors Harry Potter se préparait à jouer avec quelque chose qui pouvait provoquer une explosion politique gigantesque, et il serait dément de juste s'en aller et de le laisser le faire seul. "Nous étudierons le sang," dit Drago.

"\emph{Excellent,"} dit la silhouette, et elle sourit. "Je te félicite pour ton désir de poser la question."

"Merci," dit Drago, ne parvenant pas tout à fait à masquer toute trace d'ironie de sa voix.

"Hé, tu pensais qu'aller sur la Lune était \emph{facile}~? Réjouis-toi que cela ne nécessite que de changer parfois d'avis, et pas un sacrifice humain~!"

"Un sacrifice humain serait \emph{beaucoup} plus simple~!"

Il y eut une courte pause, puis la silhouette hocha la tête. "C'est juste."

"Écoute, Harry," dit Drago sans grand espoir, "je pensais que l'idée était de prendre toutes les choses que les Moldus savent, de les combiner avec ce que les sorciers savent, et de devenir maîtres des deux mondes. Ne serait-il pas beaucoup plus simple de juste étudier les choses que les Moldus ont \emph{déjà} découvertes, comme les trucs de la Lune, et d'utiliser \emph{ce} pouvoir -"

"\emph{Non}," dit la silhouette en secouant brutalement la tête de droite à gauche, ce qui envoya des ombres vertes courir sur son nez et ses yeux. Sa voix devint très grave. "Si tu ne peux apprendre l'art scientifique qui consiste à accepter la réalité, alors je ne \emph{dois pas} te dire ce que cette acceptation a permis de découvrir. Ce serait comme si un puissant sorcier te parlait de ces portes qu'il ne faut pas ouvrir et de ces sceaux qui ne doivent pas être brisés avant que tu n'aies prouvé ton intelligence et ta discipline en survivant à de moindres périls."

Un frisson descendit le long de la colonne vertébrale de Drago et il eut un soubresaut involontaire. Il savait que ça avait été visible même dans la faible lumière. "Très bien," dit Drago. "Je comprends." Père lui avait dit cela en de nombreuses occasions. Quand un sorcier plus puissant vous disait que vous n'étiez pas prêt à savoir, et que vous vouliez vivre, vous n'insistiez pas.

La silhouette inclina la tête. "En effet. Mais il y a autre chose que tu devrais comprendre. Les premiers scientifiques, étant des Moldus, ils n'avaient pas vos traditions. Au début, ils ne comprenaient tout simplement pas la notion de savoir dangereux, et ils pensaient que l'on pouvait librement discuter de toute chose. Lorsque leurs recherches devinrent dangereuses, ils dirent à leurs hommes politiques des choses qui auraient dû rester secrètes - ne prends pas cet air, Drago, ce n'est pas simplement de la stupidité. Ils devaient déjà être assez malins pour découvrir ce secret. Mais c'étaient des Moldus, et c'était la première fois qu'ils découvraient quelque chose de \emph{vraiment} dangereux, et ils n'avaient pas \emph{commencé} avec une tradition du secret. Une guerre faisait rage, et les scientifiques de chaque camp avaient peur que, s'ils ne parlaient \emph{pas}, les scientifiques du pays \emph{ennemi} parleraient à \emph{leurs} hommes politiques avant eux…" La voix resta en suspens, lourde de sens. "Ils ne détruisirent pas le monde. Mais il s'en est fallu de peu. Et \emph{nous} n'allons pas reproduire cette erreur."

"Tout à fait," dit Drago, sa voix maintenant très ferme. "Pas \emph{nous}. Nous sommes des sorciers, et étudier la science ne fait pas de nous des Moldus."

"Comme tu dis," dit la silhouette nimbée de vert. "Nous allons établir \emph{notre} propre Science, une Science magique, et cette Science aura des traditions plus intelligentes dès le début." La voix devint très dure. "Le savoir que je partage avec toi sera enseigné en même temps que la discipline consistant à accepter la vérité, et le niveau de ce savoir sera aligné sur ton progrès dans cette discipline, et tu ne partageras pas ce savoir avec ceux n'ayant pas appris cette discipline. Acceptes-tu cela~?"

"Oui," dit Drago. Qu'était-il censé faire, dire non~?

"Bien. Et ce que tu découvriras toi-même, tu le garderas pour toi, à moins que tu ne penses que d'autres scientifiques sont prêts à l'apprendre. Ce que nous partagerons entre nous, nous n'en parlerons pas au monde, à moins que nous ne soyons d'accord sur le fait que posséder ce savoir ne lui fait pas courir de danger. Et quelles que soient nos opinions politiques et nos allégeances, nous punirons ensemble \emph{tous ceux} qui parmi nous révèlent des sortilèges dangereux ou donnent des armes dangereuses, peu importe qu'une guerre fasse rage. À partir d'aujourd'hui, ce sera la tradition et la loi de la science parmi les sorciers. Sommes-nous d'accord là-dessus~?"

"Oui," dit Drago. En fait, ça commençait \emph{vraiment} à avoir l'air sacrément attrayant. Les Mangemorts avaient essayé de prendre le pouvoir en étant plus effrayants que tout le monde, et ils n'avaient pas encore vraiment gagné. Peut-être était-il temps d'essayer de diriger à l'aide de secrets. "Et notre groupe reste caché aussi longtemps que possible, et tous ses membres doivent accepter nos règles."

"Bien sûr. Absolument."

Il y eut une très courte pause.

"Nous aurons besoin de meilleures robes," dit la sombre silhouette, "avec des capuches et tout ça -"

"Je \emph{pensais justement} à ça," dit Drago. "Nous n'avons pas besoin de nouvelles robes intégrales, cela dit, juste des houppelandes à capuche à mettre par-dessus. J'ai une amie à Serpentard, elle prendra tes mesures -"

"Ne lui dis pas pour \emph{quoi} c'est, quand même -"

"Je ne suis pas \emph{stupide~!}"

"Et pas de masques pour l'instant, pas tant que c'est juste toi et moi -" dit la sombre silhouette.

"Oui~! Mais plus tard nous devrions avoir une espèce de marque spéciale que tous nos serviteurs devront porter, la Marque de la Science, comme un serpent mangeant la Lune sur leur bras droit -"

"Ça s'appelle un doctorat et est-ce que ça ne rendrait pas l'identification des nôtres trop facile~?"

"Hein~?"

"Je veux dire que si quelqu'un dit 'OK, maintenant tout le monde relève sa robe au-dessus de du bras droit' et que nos gars sont là 'oups, désolé, on dirait que je suis un espion' -"

"\emph{Oublie ce que je viens de dire}," dit immédiatement Drago, de la sueur perlant soudain sur toute la surface de son corps. Il avait besoin d'une distraction, \emph{vite} - "Et comment nous appellerons-nous~? Les Mangescience~?"

"Non," dit lentement la sombre silhouette. "Ça ne sonne pas bien…"

Du revers d'une de ses manches, Drago s'essuya le front, chassant des gouttes d'humidité. Mais à quoi \emph{avait pensé} le Seigneur des Ténèbres~? Père avait dit que le Seigneur des Ténèbres était \emph{intelligent}~!

"J'ai trouvé~!" dit soudain la sombre silhouette. "Tu ne comprendras pas pour l'instant, mais crois moi, ça convient."

Pour le moment, Drago aurait accepté 'Mangeurs Malfoy' du moment que ça changeait de sujet. "C'est quoi~?"

Et, debout entre les pupitres poussiéreux d'une salle abandonnée des donjons de Poudlard, la silhouette de Harry Potter, nimbée de vert, écarta ses bras d'un geste théâtral et dit, "Ce jour marquera l'aube de… la \emph{Conspiration Bayésienne}."

\later

Une silhouette silencieuse marcha d'un pas lourd et fatigué à travers les couloirs de Poudlard en direction de Serdaigle.

Harry s'était directement rendu au dîner après la réunion avec Drago, et il y était resté à peine assez longtemps pour s'étouffer sur quelques bouchées rapides avant d'aller se coucher.

Il n'était même pas encore 19h, mais il aurait dû être endormi depuis longtemps. Il s'était rendu compte la nuit \emph{dernière} qu'il ne pourrait pas utiliser le Retourneur de Temps samedi avant que le concours de lecture ne soit déjà terminé. Mais il pourrait toujours utiliser le Retourneur de Temps \emph{vendredi} soir, et ainsi gagner du temps. Alors vendredi, Harry s'était forcé à rester éveillé jusqu'à 21h, quand la coque protectrice s'était ouverte, et il avait alors utilisé les quatre heures restantes pour revenir à 17h et s'effondrer de fatigue. Il s'était réveillé samedi matin aux environs de 2h, exactement comme prévu, et il avait alors lu pendant douze heures d'affilée… et ça n'avait quand même pas suffit. Et maintenant Harry allait se coucher assez tôt pendant les jours à venir, jusqu'à ce que son cycle de sommeil se recale.

Le portrait sur la porte lui posa une énigme stupide faite pour les enfants de onze ans à laquelle il répondit sans même que les mots passent par son esprit conscient, puis il tituba jusqu'en haut des escaliers, vers son dortoir, où il enfila son pyjama, s'effondra dans son lit.

Et découvrit que son oreiller semblait plutôt bosselé.

Harry grogna. Il s'assit à contrecœur, se retourna et souleva son oreiller.

Ce qui révéla une note, deux Gallions d'or et un livre intitulé~: \emph{Occlumancie~: L'Art Caché}

Harry prit la note et lut~:


\begin{writtenNote}
Eh bien, tu t'es mis dans le pétrin, et vite. James lui-même n'aurait pas été à la hauteur.

Tu t'es fait un puissant ennemi. Rogue possède la loyauté, l'admiration et la peur de toute la maison Serpentard. Tu ne peux faire confiance à aucun d'entre eux à présent, qu'ils viennent à toi sous des airs amicaux ou effrayants.

Dorénavant, tu ne devras pas croiser le regard de Rogue. C'est un Legilimens et il pourrait en profiter pour lire ton esprit. J'ai joint à ce message un livre qui pourrait t'aider à te protéger, même si ta progression sera limitée sans l'aide d'un précepteur. Tu pourras quand même espérer détecter l'intrusion.

Afin que tu trouves le temps nécessaire à étudier l'Occlumancie, j'ai de plus joint 2 Gallions, ce qui est le prix de la feuille de réponse et des devoirs pour le cours d'Histoire de première année (le professeur Binns ayant donné les mêmes contrôles et les mêmes devoirs chaque année depuis sa mort). Tes nouveaux amis, les jumeaux Weasley, devraient pouvoir t'en vendre des exemplaires. Il va sans dire que tu ne dois pas te faire prendre en leur possession.

Dumbledore fait juste semblant d'être fou. Il est extrêmement intelligent, et si tu continues à entrer dans des placards et à disparaître, il va certainement en déduire que tu as une cape d'invisibilité en ta possession, s'il ne l'a pas déjà fait. Évite-le autant que possible, cache la Cape d'Invisibilité en lieu sûr (PAS dans ta bourse) à chaque fois que tu ne peux pas l'éviter, et fais très attention en sa présence.

S'il te plaît Harry, à l'avenir, sois plus prudent.

- Le Père Noël
\end{writtenNote}

Harry fixa la note.

Ça \emph{semblait} être de bons conseil. Bien sûr, Harry n'allait pas tricher en cours d'Histoire même s'ils lui donnaient un singe mort pour professeur. Mais la Legilimancie de Severus… celui ou celle qui avait envoyé cette note savait beaucoup de choses importantes et secrètes, et il ou elle était prêt à les dire à Harry. La note le mettait encore en garde contre la possibilité que Dumbledore vole la Cape mais à ce stade, Harry ne savait absolument pas si c'était mauvais signe, ça aurait pu être une erreur compréhensible.

Il semblait y avoir une intrigue en cours à Poudlard. Peut-être que si Harry \emph{comparait} \emph{les versions} de Dumbledore et de l'envoyeur-de-note, il pourrait en déduire une image \emph{combinée} qui pourrait être précise~? Par exemple s'ils s'accordaient \emph{tous les deux} sur quelque chose, alors…

… mouais…

Harry fourra le tout dans sa bourse et remonta le Sourdineur et tira la couverture par-dessus sa tête et mourut.

\later

C'était dimanche matin et Harry mangeait des pancakes dans la Grande Salle, des bouchées vives et menues, jetant des coups d'œil nerveux à sa montre à intervalles de quelques secondes.

Il était 8h02, et dans précisément deux heures et une minute, ça ferait \emph{exactement une semaine} qu'il avait vu les Weasley et était passé sur la plate-forme 9¾.

Et l'idée lui était venue… Harry ne savait pas si c'était une bonne façon de voir l'univers, il ne savait plus rien maintenant, mais il \emph{semblait possible}…

Que…

\emph{Pas assez de choses intéressantes ne lui soient arrivées la semaine dernière.}

Une fois qu'il aurait fini de manger son petit déjeuner, Harry comptait monter directement à sa chambre et se cacher au niveau caverne de sa malle et ne parler à personne jusqu'à 10h03.

Et c'est alors que Harry vit les jumeaux Weasley marcher vers lui. L'un d'eux portait quelque chose, caché derrière son dos.

Il devrait crier et s'enfuir.

Il devrait crier et s'enfuir.

Quoi que ce fût… ça pourrait très bien être…

… le \emph{grand final}…

Il devrait vraiment juste crier et s'enfuir.

Avec le sentiment résigné que l'univers finirait \emph{quand même} par le retrouver, Harry continua de découper son pancake avec son couteau et sa fourchette. Il n'arrivait pas à trouver l'énergie nécessaire. C'était la triste vérité. Harry savait maintenant comment les gens se sentaient quand ils étaient fatigués de courir, fatigués d'essayer d'échapper au destin, et qu'ils tombaient juste au sol et laissaient les démons aux horribles dents et tentacules venus des plus sombres abysses les traîner vers leur innommable destinée.

Les jumeaux Weasley vinrent plus près.

Et plus près encore.

Harry mangea une autre bouchée de pancake.

Les jumeaux Weasley arrivèrent, souriant joyeusement.

"Bonjour, Fred," dit Harry d'un ton morne. L'un des deux jumeaux fit un signe de tête. "Bonjour, George." L'autre jumeau hocha la tête.

"Tu as l'air fatigué," dit George.

"Tu devrais te réjouir," dit Fred.

"Regarde ce qu'\emph{on} a pour toi~!"

Et de derrière le dos de Fred, George récupéra -

Un gâteau avec douze bougies allumées.

Il y eut une pause, tandis que toute la table Serdaigle les fixait.

"Ça ne va pas," dit quelqu'un. "Harry Potter est né le trente et un juil-"

"\prophesy{Il vient}," dit une immense voix creuse qui coupa toutes les conversations à la manière d'une épée de glace. "\prophesy{Celui qui déchirera l' -}"

Dumbledore avait bondi de son trône et courut sur la Grande Table et saisit la femme qui proférait ces horribles paroles, Fumseck apparut dans un flash, et tous les trois disparurent dans un craquement de flammes.

Il y eut une pause choquée…

… puis des têtes se tournèrent en direction de Harry Potter.

"Je n'ai pas fait ça," dit Harry d'une voix fatiguée.

"C'était une \emph{prophétie}~!" siffla quelqu'un à la table. "Et je parie que ça \emph{te} concerne~!"

Harry soupira.

Il se redressa, éleva la voix, et dit très fort, par-dessus les conversations qui commençaient~: "\emph{Ça n'est pas à propos de moi~! Clairement~! Je ne viens pas, je suis déjà là~!}"

Harry se rassit.

Ceux qui l'avaient regardé se détournèrent de nouveau.

Quelqu'un d'autre à la table dit~: "Alors de \emph{qui} ça parle~?"

Et, le cœur lourd et maussade, Harry comprit qui \emph{n'était pas} encore à Poudlard.

Appelez ça une folle conjecture, mais Harry avait l'impression qu'un de ces jours, un Seigneur des Ténèbres mort-vivant allait débarquer à Poudlard.

La conversation continua autour de lui.

"Et déchirer \emph{quoi}~?"

"J'ai cru entendre Trelawney dire quelque chose qui commençait par un 'A' juste avant que le Directeur ne l'attrape."

"Comme… âme~? astre~?"

"Si quelqu'un va déchirer le Soleil alors on est \emph{vraiment} mal~!"

Ça semblait assez peu probable à Harry, à moins que le monde ne contienne d'horribles choses au fait des idées de David Criswell au sujet de la dissipation d'étoiles.

"Donc," dit Harry d'un ton fatigué, "ça arrive tous les dimanches au petit déjeuner, c'est ça~?"

"Non," dit un étudiant qui aurait pu être en septième année, et il fronçait les sourcils d'un air sinistre. "Pas du tout."

Harry haussa les épaules. "Bref. Quelqu'un veut du gâteau d'anniversaire~?"

"Mais ce n'est \emph{pas} ton anniversaire~!" dit le même élève que celui qui avait émis une objection auparavant.

C'était bien sûr le signal pour que Fred et George commencent à rire.

Même Harry parvint à avoir un sourire las.

Alors que la première part lui était servie, Harry dit~: "J'ai eu une \emph{semaine vraiment longue}."

\later

Et Harry était assis au niveau caverne de sa malle, qui était fermée à clé pour que personne ne puisse entrer, une couverture sur la tête, attendant que la semaine soit finie.

10h01.

10h02.

10h03, mais juste pour être sûr…

10h04 et la première semaine était finie.

Harry laissa échapper un soupir de soulagement, et retira la couverture de sa tête avec précaution.

Quelques instants plus tard, il avait émergé dans son dortoir ensoleillé.

Peu de temps après, il était dans la salle commune de Serdaigle. Quelques personnes le regardèrent, mais personne ne dit rien, ni n'essaya de lui parler.

Harry trouva un bon et large bureau pour écrire, prit une chaise confortable, et s'assit. Il extirpa un papier et un crayon de sa bourse.

Maman et Papa avaient dit à Harry, en des termes tout sauf incertains, que bien qu'ils comprennent son enthousiasme à l'idée de quitter la maison et d'échapper à ses parents, il devrait leur écrire \emph{toutes les semaines sans exception}, juste pour qu'ils sachent qu'il était en vie, en bonne santé, et pas en prison.

Harry regarda la feuille de papier blanc. \emph{Voyons voir…}

Après avoir quitté ses parents à la gare, il avait…

… fait la connaissance d'un garçon élevé par Dark Vador, s'était lié d'amitié avec les trois pires farceurs de Poudlard, avait rencontré Hermione, puis il y avait eu l'incident avec le Choixpeau… lundi on lui avait donné une machine à remonter le temps pour traiter ses troubles du sommeil, et un bienfaiteur inconnu lui avait donné une cape d'invisibilité légendaire, il avait sauvé sept Poufsouffle en regardant cinq garçons plus âgés et effrayants dans les yeux, et l'un d'eux avait menacé de lui briser les doigts, il s'était rendu compte qu'il avait un mystérieux côté obscur, il avait appris à jeter \emph{Frigideiro} en cours de Sorts et enchantements, il avait commencé à être le rival de Hermione… mardi il avait découvert l'Astronomie, enseignée par le professeur Aurora Sinistra qui était gentille, et l'Histoire de la Magie, qui était enseignée par un fantôme qui aurait dû être exorcisé et remplacé par un lecteur de cassette… mercredi il avait été nommé Étudiant le Plus Dangereux de la Classe… jeudi, ne pensons même pas à jeudi… vendredi, l'Incident en cours de Potions, suivi par son chantage avec le directeur, suivi par le professeur de Défense lui ordonnant de se faire battre en plein cours, suivi par le professeur de Défense se révélant être l'être humain le plus génial à encore marcher à la surface de cette planète… samedi il avait perdu un pari et avait été à son premier rendez-vous galant et avait commencé à racheter Drago… et puis ce matin la prophétie non entendue du professeur Trelawney pourrait ou ne pourrait pas indiquer qu'un Seigneur des Ténèbres immortel était sur le point d'attaquer Poudlard.

Harry organisa mentalement ce qu'il avait, et commença à écrire.
\begin{writtenNote}
\letterAddress{Chers Maman et Papa,}

La vie à Poudlard est très amusante. J'ai appris à violer la Deuxième Loi de la Thermodynamique en cours de Sorts et enchantements, et j'ai rencontré un fille prénommée Hermione Granger qui lit plus vite que moi.

Je ferais mieux de m'arrêter là.

\letterClosing[Votre fils aimant,,]{Harry James Potter-Evans-Verres.}
\end{writtenNote}

%  LocalWords:  ermione Bathilda Bagshot’s Awww
