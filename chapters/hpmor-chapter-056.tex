\namedpartchapter{L'Expérience de Prison de Stanford}{TSPE}{VI}{Optimisation sous contraintes}

\lettrine{S}{ilencieuse}, , heureusement qu'elle était silencieuse, la porte de métal de l'étage suivant. Soit il n'y avait personne derrière, soit ils souffraient en silence, ou peut-être qu'ils criaient mais que leur voix avait déjà lâché, ou peut-être qu'ils monologuaient en marmonnant, seuls dans le noir…

\emph{Je ne suis pas sûr d'en être capable}, pensa Harry, et il ne pouvait pas blâmer les Détraqueurs pour ces pensées désespérées. Il valait mieux descendre, c'était plus sûr, car son plan prendrait du temps à être mis en œuvre et les Aurors progressaient probablement déjà dans sa direction. Mais s'il passait devant une seule porte de métal supplémentaire en restant coi et en gardant une respiration parfaitement régulière il deviendrait peut-être fou~; s'il fallait qu'il délaisse une partie de lui-même à chaque fois, il ne resterait bientôt plus rien de lui -

Un chat à l'éclat lunaire se forma et bondit devant le Patronus de Harry. Ce dernier cria presque, ce qui n'aurait pas amélioré son image auprès de Bellatrix.

<<~Harry~!~>> dit la voix du professeur McGonagall d'un ton plus alarmé que Harry ne l'avait jamais entendu chez elle. <<~Où es-tu~? Tu vas bien~? C'est mon Patronus, réponds-moi~!~>>

Dans un effort convulsif, Harry se vida l'esprit, réajusta sa gorge, se força à redevenir calme, changea de personnalité comme s'il s'était agi d'une barrière Occlumantique. Cela prit quelques secondes et il espéra de toutes ses forces que grâce aux délais habituels de communication, le professeur McGonagall ne remarquerait rien~; il espéra tout aussi fort que les Patronus ne rendaient pas compte de leur environnement à leur possesseur.

Une jeune voix innocente dit~: <<~Je suis à la chambre de Marie, professeur, au Chemin de Traverse. Pour tout vous dire, je suis en train de me rendre aux toilettes. Qu'est-ce qui ne va pas~?~>>

Le chat s'en fut d'un bond, et Bellatrix se mit à glousser doucement, c'était un rire appréciateur et empoussiéré, mais elle s'interrompit abruptement au premier sifflement de Harry.

Un instant plus tard, le chat était de retour, et il dit de la voix du professeur McGonagall~: <<~Je viens te récupérer tout de suite. Ne va \emph{nulle part}, si tu n'es pas près du professeur de Défense ne vas pas le voir, ne dis rien à personne, je serai là aussi vite que possible~!~>>

Et le chat lumineux bondit vers l'avant et disparut.

Harry jeta un coup d'œil à sa montre et nota l'heure afin de pouvoir revenir dans les toilettes de la chambre de Marie au bon moment après avoir sorti tout le monde de là et avoir de nouveau immobilisé le Retourneur de Temps par le professeur Quirrell…

\emph{Tu sais}, dit la partie de son cerveau consacrée à la résolution de problèmes, \emph{qu'il y a une limite au nombre de contraintes qu'on peut ajouter à un problème avant qu'il ne devienne} vraiment \emph{impossible~?}

Ça n'aurait pas dû avoir d'importance, en fait ça n'en avait pas vraiment, ça n'était pas comparable à la souffrance d'un seul prisonnier d'Azkaban, et pourtant Harry se trouva profondément conscient du fait que si son plan ne s'achevait pas par, à la fois, sa récupération à la Chambre de Marie comme s'il n'était jamais parti et que le professeur de Défense semble innocent de tout méfait, le professeur McGonagall allait \emph{le tuer}.

\later

Tandis que leur équipe se préparait à grignoter un bout de terrain supplémentaire le long de la spirale C au moyen d'une série de boucliers, d'analyses et de dissipation du bouclier précédemment lancé, Amélia tapotait des doigts sur sa hanche et se demandait si elle ne devrait pas consulter l'indéniable expert du domaine. Si seulement il n'était pas si -

Elle entendit le craquement de feu familier et devina ce qu'elle allait voir avant même de se retourner.

Un tiers des Aurors se retournaient et levaient leur baguette en direction du vieux sorcier aux lunettes en croissant de lune et à la longue barbe d'argent qui était apparu juste au milieu de leur groupe avec un phénix éclatant d'or et de rouge perché sur son épaule.

<<~Ne tirez pas~!~>> Le Polynectar aurait rendu simple la contrefaçon du visage, mais imiter le voyage par phénix aurait été bien plus difficile - les barrières autorisaient ce moyen rapide d'entrer, même si aucune sortie rapide n'était possible.

La vieille sorcière et le vieux sorcier se regardèrent pendant un long moment.

(Amélia se demanda distraitement lequel de ses Aurors avait fait parvenir le message, après tout, plusieurs anciens membres de l'Ordre du Phénix étaient ici avec elle~; elle essaya de se souvenir si elle avait vu le moineau d'Emmeline ou le chat d'Andy manquer au troupeau de créatures lumineuses~; mais elle savait que c'était futile. Ça pouvait fort bien n'être la faute d'aucun des membres de son équipe car le vieux fouineur savait souvent des choses impossibles à savoir).

Albus Dumbledore inclina sa tête vers Amélia d'un geste courtois.

<<~J'espère ne pas être importun, dit le vieux sorcier avec calme. Nous sommes tous dans le même camp, n'est-ce pas~?

--- Cela dépend, dit Amélia d'une voix dure. Es-tu là pour nous aider à attraper des criminels ou pour les protéger des conséquences de leurs actes~?~>> \emph{Vas-tu essayer de protéger la meurtrière de mon frère de son Baiser bien mérité, vieux fouineur~?} Amélia avait entendu dire que Dumbledore était devenu plus malin vers la fin de la guerre, en grande partie grâce à l'insistance ininterrompue de Fol-Œil~; mais il était retombé dans sa ridicule clémence à l'instant où le corps de Voldemort avait été trouvé.

Une douzaine de points blanc et argent, réflexions des animaux étincelants, luisaient à la surface des lunettes en croissant de lune du vieux sorcier~; il parla~: <<~Je souhaite voir Bellatrix Black libérée encore moins que toi, dit-il. Elle ne \emph{doit} pas quitter cette prison en vie, Amélia.~>>

Avant qu'elle ne puisse de nouveau parler, ne serait-ce que pour exprimer sa gratitude étonnée, le vieux sorcier fit un geste de sa longue baguette noire et un flamboyant phénix d'argent apparut, peut-être plus lumineux que tous leurs Patronus réunis. C'était la première fois qu'elle avait vu quelqu'un jeter ce sortilège sans parler. <<~Ordonne à tous tes Aurors de dissiper leur Patronus pendant dix secondes, dit le vieux sorcier. Ce que les ténèbres ne peuvent trouver, peut-être la lumière saura-t-elle le découvrir.~>>

Amélia aboya l'ordre d'accomplir la volonté de Dumbledore à l'officier des communications qui en fit part à tous les Aurors par le biais de leur miroir.

Cela prit un moment qui devint un silence horrible, aucun des Aurors n'osait parler, et Amélia essaya d'évaluer ses propres pensées. \emph{Elle ne doit pas quitter cette prison en vie…} Albus Dumbledore ne se transformerait pas en Bartemius Croupton sans avoir une bonne raison. S'il avait voulu lui dire \emph{pourquoi}, il l'aurait déjà fait~; mais ce n'était certainement pas bon signe.

Tout de même, il était bon de savoir qu'ils pourraient travailler ensemble, cette fois-ci.

<<~Maintenant,~>> dit un chorus de miroirs, et tous les Patronus s'évanouirent en un clin d'œil, mis à part ce flamboyant phénix d'argent.

<<~Un autre Patronus est-il présent~?~>> dit distinctement le vieux sorcier à l'intention de la créature lumineuse.

La créature lumineuse inclina la tête en signe d'acquiescement.

<<~Peux-tu le trouver~?~>>

La tête d'argent s'inclina de nouveau.

<<~T'en souviendras-tu, si jamais il partait et revenait ensuite~?~>>

Un dernier hochement de tête du phénix flamboyant.

<<~C'est fait, dit Dumbledore.

--- Terminé~>>, dirent tous les miroirs un instant après, et Amélia leva sa baguette et entreprit de relancer son propre Patronus (bien qu'il lui fallut un effort de concentration supplémentaire, avec ce sourire féroce déjà présent sur son visage, pour réussir à penser à la première fois où Susan lui avait embrassé la joue plutôt qu'au sombre de destin de Bellatrix Black. Cet autre Baiser était certainement une pensée heureuse, mais pas d'un genre adéquat au Patronus).

\later

Ils n'étaient même pas parvenus à la fin du couloir que Harry leva sa main, poliment, comme s'il avait été dans une salle de classe.

Il réfléchit rapidement. La question était de savoir comment - non, cela aussi était évident.

<<~Il semble, dit Harry d'une voix froidement amusée, que quelqu'un a demandé à ce Patronus de ne transmettre son message qu'à moi.~>> Il gloussa. <<~Eh bien. Pardonne-moi, chère Bella. \emph{Quietus}.~>>

L'humanoïde d'argent dit immédiatement, avec la voix de Harry~:

<<~Il y a un autre Patronus qui recherche ce Patronus.

--- \emph{Quoi~?}~>> dit Harry. Et alors, sans s'interrompre pour réfléchir à ce qui se passait~: <<~Peux-tu le bloquer~? L'empêcher de te trouver~?~>>

L'humanoïde d'argent secoua la tête.

\later

À peine Amélia et les autres Aurors avaient-ils fini de relancer leur Patronus que -

Le phénix flamboyant s'envola , le vrai phénix rouge et or le suivit et le vieux sorcier s'élança calmement derrière eux, sa longue baguette portée à hauteur de cuisse.

Les boucliers qui entouraient leur bout de terrain s'écartèrent autour du vieux sorcier comme s'ils avaient été faits d'eau et se refermèrent derrière lui sans une ondulation.

<<~\emph{Albus~!} s'écria Amélia. Qu'est-ce que tu t'imagines être en train faire~?~>>

Mais elle le savait déjà.

<<~Ne me suivez pas, dit le vieux sorcier d'une voix sévère. Je peux me protéger, je ne peux pas protéger les autres.~>>

L'injure qu'Amélia lui lança fit trembler jusqu'à ses propres Aurors.

\later

\emph{C'est pas juste, pas juste, pas juste~! Il y a une limite au nombre de contraintes qu'on peut ajouter à un problème avant qu'il ne devienne réellement impossible~!}

Harry bloqua les pensées inutiles, ignora la fatigue qu'il ressentait et força son esprit à se confronter aux nouveaux besoins, il lui fallait réfléchir \emph{vite}, utiliser l'adrénaline pour suivre des chaînes logiques rapidement et sans hésitation plutôt que de la gâcher dans du désespoir.

Pour que la mission réussisse,

(1) Harry devrait dissiper son Patronus.

(2) Bellatrix devrait rester invisible aux yeux des Détraqueurs après que le Patronus aura été dissipé

(3) Harry devrait résister à l'épuisement du Détraquage après que le Patronus aura été dissipé.

…

\emph{Si je résous ça}, dit le cerveau de Harry, \emph{je veux un cookie ensuite, et si tu rends le problème un peu plus difficile, je dis bien un chouïa plus difficile, je sors de ton crâne et je pars pour Tahiti.}

Harry et son cerveau considérèrent le problème.

Azkaban était demeurée invincible pendant des siècles, s'en remettant à l'impossibilité d'échapper au regard des Détraqueurs. Donc si Harry trouvait un \emph{autre} moyen de cacher Bellatrix, ce serait soit grâce à son savoir scientifique soit à sa prise de conscience que les Détraqueurs étaient la Mort.

Le cerveau de Harry suggéra qu'une façon évidente d'empêcher les Détraqueurs de voir Bellatrix serait de faire cesser son existence, c'est-à-dire de la tuer.

Harry félicita son cerveau pour être sorti des sentiers battus et lui dit de continuer à chercher.

\emph{Tue-la puis ramène-la}, dit la suggestion suivante. \emph{Utilise Frigideiro} \emph{pour la refroidir jusqu'à ce que son activité mentale cesse, puis réchauffe-la ensuite en utilisant Thermos}, \emph{comme les gens qui tombent dans de l'eau très froide et qui peuvent ensuite être ranimés une demi-heure plus tard sans dommage cérébral notable.}

Harry considéra cela. Bellatrix ne survivrait peut-être pas dans son état d'affaiblissement actuel. \emph{Et} ça n'empêcherait peut-être pas la Mort de la voir. \emph{Et} il aurait du mal à aller bien loin en portant une Bellatrix inconsciente. \emph{Et} il n'arrivait pas à se souvenir des travaux de recherches sur la température corporelle exacte qui était censée temporairement arrêter le cerveau sans néanmoins être fatale.

C'était une autre idée originale, mais Harry dit à son cerveau de continuer à réfléchir à…

…\emph{comment se cacher de la Mort…}

Une grimace parcourut le visage de Harry. Quelque part, un jour, il avait entendu quelque chose à ce sujet.

\emph{L'un des prérequis pour devenir un grand sorcier est d'avoir une excellente mémoire}, avait dit le professeur Quirrell. \emph{La clé d'un puzzle est parfois quelque chose que vous avez lu il y a vingt ans dans un vieux rouleau de parchemin, ou un anneau particulier que vous avez vu au doigt d'un homme que vous n'avez rencontré qu'une fois…}

Harry se concentra aussi fort qu'il le pouvait, mais il n'arrivait pas à se rappeler, il l'avait sur le bout de la langue, mais il n'y arrivait pas~; alors il ordonna à son inconscient de continuer à essayer de se souvenir et il se concentra sur l'autre moitié du problème.

\emph{Comment puis-je me protéger des Détraqueurs sans Patronus~?}

Le directeur avait été exposé à des Détraqueurs situées à quelques mètres, de façon répétée, encore et encore, pendant toute une journée, et il n'en était ressorti qu'avec un air fatigué. Comment avait-il accompli cela~? Harry pouvait-il aussi le faire~?

C'était peut-être une mutation génétique aléatoire, auquel cas Harry était foutu. Mais en supposant que le problème \emph{était} solvable…

Alors la réponse évidente était que Dumbledore n'avait pas peur de la mort.

Dumbledore n'avait \emph{vraiment} pas peur de la mort. Dumbledore croyait honnêtement et sincèrement que la mort était la prochaine grande aventure. Il le croyait jusqu'au plus profond de son être, pas seulement sous la forme de mots utiles permettant de masquer une dissonance cognitive, pas seulement pour faire semblant d'être sage. Dumbledore avait décidé que la mort faisait partie de l'ordre naturel et normatif, et quel que soit le peu de peur qui restait en lui, il avait fallu une longue durée et des expositions répétées pour que le Détraqueur parvienne à puiser son énergie au travers de cette petite faille.

Cette possibilité n'était pas offerte à Harry.

Puis il pensa au revers de la médaille, à l'évidente question inversée~:

\emph{Pourquoi suis-je beaucoup plus vulnérable que la moyenne~? Les autres élèves ne sont pas tombés par terre quand ils ont fait face au Détraqueur.}

Harry comptait détruire la mort, y mettre un terme s'il le pouvait. Il comptait vivre pour toujours, s'il le pouvait~; il l'espérait, et l'idée de la Mort ne faisait surgir en lui aucun désespoir, aucun sentiment d'inéluctabilité. Il n'était pas aveuglément attaché à sa propre vie~; de fait, il avait dû faire un effort pour ne \emph{pas} la brûler dans le but de protéger d'autres de la Mort. Pourquoi les ombres de la Mort avaient-elles un tel pouvoir sur lui~? Il ne se serait pas cru si effrayé par elles.

Était-ce qu'il avait rationalisé pendant tout ce temps~? Qui avait eu secrètement si peur de la mort que cela avait tordu ses pensées, lorsque Harry avait accusé Dumbledore~?

Harry considéra cela en s'empêchant de se détourner de l'idée. C'était pénible, mais…

Mais…

Mais les pensées pénibles n'étaient pas toujours \emph{vraies}, et celle-ci ne semblait pas tout à fait juste. Comme si elle comportait un élément de vérité mais que cet élément n'était pas \emph{là} où l'hypothèse disait qu'il était -

Et c'est là que Harry comprit.

\emph{Oh.}

\emph{Oh, je comprends à présent.}

\emph{Celui qui a peur, c'est…}

Harry demanda à son côté obscur s'il avait peur de la mort.

Et le Patronus de Harry trembla, se ternit, disparut presque l'espace d'un instant, à cause de cette terreur désespérée et sanglotante, de cette peur indicible qui ferait tout pour ne pas mourir, qui ignorerait tout le reste pour ne pas mourir, qui ne pouvait pas penser ni ressentir en présence de cette horreur absolue, qui ne pouvait pas plus regarder dans l'abysse de la non-existence qu'elle n'aurait pu faire face au Soleil, une chose aveugle et terrifiée qui voulait seulement trouver un coin sombre et ne plus avoir à y penser -

La silhouette d'argent s'était assombrie jusqu'à avoir un éclat lunaire, elle vacillait comme une bougie en fin de vie -

\emph{Tout va bien}, pensa Harry, \emph{tout va bien.}

Il se visualisa, berçant son côté obscur dans ses bras comme si celui-ci avait été un enfant effrayé.

\emph{C'est bien et c'est normal d'être horrifié, parce que la mort est horrible. Tu n'as pas à cacher ton horreur, tu n'as pas à te sentir honteux, tu peux la porter comme une marque d'honneur, à découvert, en plein jour.}

C'était étrange de se sentir ainsi divisé en deux, de suivre son fil de pensée qui donnait du réconfort et celui de l'incompréhension que son côté obscur avait pour l'étrangeté des pensées ordinaires de Harry~; car parmi toutes les choses que son côté obscur avait associées à sa peur de la mort, la seule qu'il n'aurait jamais imaginée, à laquelle il ne serait jamais attendu, c'était d'être accepté, félicité et aidé…

\emph{Tu n'as pas à te battre seul}, dit silencieusement Harry à l'intention de son côté obscur. \emph{Le reste de ma personne t'aidera. Je ne me laisserai pas mourir, et je ne laisserai pas mes amis mourir non plus. Pas toi/moi, pas Hermione, pas Maman ni Papa, pas Neville, pas Drago ni personne, c'est la volonté de protéger…} il visualisa des ailes de lumière solaire semblables aux ailes de Patronus qu'il avait déployées donner abri et réconfort à cet enfant effrayé.

Le Patronus s'éclaira de nouveau, le monde tournait autour de Harry - ou était-ce son esprit qui tournait~?

\emph{Prends ma main}, songea Harry, et il le visualisa, \emph{viens avec moi, et on y parviendra ensemble…}

Il y eut une embardée dans l'esprit de Harry, comme si son cerveau avait fait un pas à gauche ou que l'univers avait fait un pas à droite.

Et dans un couloir puissamment éclairé d'Azkaban, les faibles lampes à gaz largement dépassées par la lumière stable et résolue d'un Patronus à forme humaine, un garçon invisible se tint là, un étrange sourire sur le visage, et son corps ne tremblait que légèrement.

Harry savait qu'il venait d'accomplir quelque chose d'important, quelque chose qui dépassait l'amélioration de sa résistance aux Détraqueurs.

Et plus que ça~: il s'était \emph{souvenu}. Penser à la Mort de façon anthropomorphique avait ironiquement fait l'affaire. Maintenant il pouvait se souvenir de ce qui était réputé cacher quelqu'un au regard de la Mort elle-même…

\later

Dans un couloir d'Azkaban, les jambes d'un sorcier s'arrêtèrent brutalement~; le flamboyant phénix d'argent qui le guidait s'était arrêté en plein vol, il battait des ailes sans se déplacer, en plein désarroi. Il tendit le cou, regarda devant puis derrière, comme s'il était perdu~; puis il se tourna vers son maître et secoua la tête en signe d'excuse.

Sans ajouter un mot, le vieux sorcier se détourna et repartit à grand pas vers là d'où il était venu.

\later

Harry se redressa et sentit la peur se déverser sur lui et tout autour de lui. Une petite partie de son être avait peut-être été érodée par les vagues de néant qui se brisaient continuellement sur sa pierre immuable mais ses membres n'étaient pas froids et sa magie était là, avec lui. Ces vagues finiraient un jour par le corroder et le consumer, elles s'infiltreraient par les petites parties de son être qui tremblaient encore devant la Mort au lieu d'utiliser leur peur pour se donner de l'énergie avant la bataille. Mais ce funeste destin mettrait longtemps à advenir tant que les ombres de la Mort seraient loin de lui, indifférentes. Le défaut, la craquelure, la ligne de fracture qui avait été présente en lui était réparée, et les étoiles flamboyaient puissamment dans son esprit, vastes, ne connaissant pas la peur, elles étincelaient au milieu du froid et des ténèbres.

Aux yeux de n'importe qui d'autre, il aurait semblé que le garçon se tenait seul dans le couloir de métal faiblement éclairé, arborant cet étrange sourire.

Car Bellatrix Black et le serpent enroulé autour de ses épaules étaient masqués par la Cape d'Invisibilité, l'une des trois Reliques de la Mort, réputée masquer son porteur à la vue de la Mort elle-même. L'énigme dont la réponse avait été perdue et que Harry avait redécouverte.

Et Harry savait maintenant que la dissimulation fournie par la Cape était plus que la simple transparence de la Désillusion, que la Cape vous maintenait \emph{caché}, pas seulement invisible, aussi inobservable que l'étaient les Sombrals à ceux qui ignoraient la Mort. Et Harry savait aussi que c'était en sang de Thestral qu'était fait le symbole des Reliques de la Mort situé à l'intérieur de la Cape, liant ainsi à celle-ci cette partie du pouvoir de la Mort, lui permettant de confronter les Détraqueurs sur leur terrain et de les bloquer. Ça lui était venu comme une intuition, mais l'intuition était certaine, le savoir était apparu à l'instant où il avait résolu l'énigme.

Sous la Cape, Bellatrix était toujours transparente, mais elle n'était plus cachée aux yeux de Harry, il savait qu'elle était là, aussi évidente que lui était la présence d'un Thestral. Car Harry n'avait pas donné sa Cape, il l'avait seulement prêtée~; et il avait compris et maîtrisé la Relique de la Mort transmise le long de la lignée Potter.

Il regarda la femme invisible droit dans les yeux et dit~:

<<~Les Détraqueurs peuvent-ils t'atteindre, Bella~?

--- Non~>>, dit la femme d'une voix douce et émerveillée. Puis~: <<~Mais, seigneur… \emph{vous…}

--- Si tu dis quelque chose de stupide, cela va m'agacer, dit Harry d'un ton froid. Ou bien as-tu l'impression que je me sacrifierai pour toi~?

--- Non, seigneur~>>, répondit la servante du Seigneur des Ténèbres d'un ton perplexe et peut-être admiratif.

<<~Suis~>>, dit le froid murmure de Harry.

Et ils continuèrent leur périple vers le fond d'Azkaban, et le Seigneur des Ténèbres fourra la main dans sa bourse, prit un cookie et le mangea. Si Bellatrix l'avait interrogé, Harry aurait dit que c'était pour le chocolat, mais elle ne demanda rien.

\later

Le vieux sorcier revint à grands pas parmi les Aurors, les phénix argent et rouge-or maintenant derrière lui.

<<~\emph{Toi -} commença à mugir Amélia.

--- Ils ont dissipé leur Patronus~>>, dit Dumbledore. Le vieux sorcier ne sembla pas élever la voix mais ses mots calmes prirent le dessus. <<~Je ne peux plus les trouver.~>>

Amélia grinça des dents et mit un certain nombre de remarques acerbes en attente avant de se tourner vers l'officier des communications. <<~Dites à la salle de garde de redemander aux Détraqueurs s'ils peuvent sentir la présence de Bellatrix Black.~>>

La spécialiste des communications parla à son miroir pendant un moment, et elle releva les yeux quelques secondes plus tard avec un air surpris~: <<~Non -~>>

Dans son esprit, Amélia jurait déjà violemment.

<<~- mais ils peuvent voir quelqu'un dans les niveaux inférieurs, quelqu'un qui n'est pas un prisonnier.

--- Très bien~! lâcha Amélia. Dites aux Détraqueurs qu'une douzaine d'entre eux peuvent pénétrer Azkaban et s'emparer de cette personne et de ceux qui l'accompagnent~! Et s'ils voient Bellatrix Black, qu'ils l'Embrassent immédiatement~!~>>

Amélia se retourna et jeta un regard furibond à Dumbledore, le mettant au défi de protester~; mais le vieux sorcier se contenta de la regarder d'un air triste et demeura coi.

\later

L'Auror McCusker finit de converser avec le corps qui dérivait de l'autre côté de la fenêtre, transmettant ainsi les ordres de la directrice.

Le corps lui offrit un sourire mortel qui le démembra presque, puis il flotta vers les profondeurs.

Peu de temps après, douze Détraqueurs émergèrent de l'abysse central d'Azkaban et se dirigèrent vers l'extérieur, vers les murs de la vaste structure de métal qui s'élevait au-dessus d'eux.

Entrant par les trous creusés dans les fondations d'Azkaban, les plus sombres de toutes les créatures entamèrent leur marche d'horreur.
%  LocalWords:  TSPE ilent McCusker
