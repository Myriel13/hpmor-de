\chapter{Être consciencieux}

\lettrinepara[ante=<<~]{F}{\emph{rigideiro~!}}~>>

\hplettrineextrapara
Harry trempa un doigt dans le verre d'eau posé sur son bureau. Il aurait dû être froid. Mais tiède il avait été, et tiède il était resté. Une fois de plus.

Harry se sentait hautement trahi.

Il y avait des centaines de romans de \emph{fantasy} éparpillés dans la maison Verres. Harry en avait lu une bonne quantité. Et il semblait de plus en plus que Harry avait un côté obscur. Donc, après que le verre d'eau eut refusé de coopérer les premières fois, Harry avait jeté un regard tout autour du cours de Sortilèges afin de s'assurer que personne ne regardait, avait pris une profonde inspiration, et s'était mis en colère. Il avait pensé aux Serpentard malmenant Neville et au jeu où on faisait tomber vos livres à chaque fois que vous essayez de les ramasser. Il avait pensé à ce que Drago Malfoy avait dit au sujet de la fille Lovegood, âgée de dix ans, et à la façon dont le Magenmagot fonctionnait vraiment…

Et la fureur était entrée dans son sang, et il avait tenu sa baguette d'une main tremblante de haine, et il avait dit d'un ton glacé~: <<~\emph{Frigideiro~!}~>> et absolument rien ne s'était produit.

Harry avait été \emph{roulé}. Il voulait écrire à quelqu'un et demander un \emph{remboursement} de son côté obscur, qui aurait clairement \emph{dû} avoir d'irrésistibles pouvoirs magiques mais s'était révélé \emph{défectueux}.

<<~\emph{Frigideiro!}~>> dit à nouveau Hermione, depuis le pupitre à côté du sien. Son eau était un solide bloc de glace et il y avait des cristaux blancs se formant sur le rebord de son verre. Elle semblait totalement absorbée par son travail et pas du tout consciente des autres élèves du cours qui la regardaient avec des yeux pleins de haine, ce qui était soit (a) dangereusement inconscient de sa part, soit (b) un numéro parfaitement huilé qui tenait de la performance artistique.

<<~Oh, \emph{très bien}, Mademoiselle Granger~!~>> piailla Filius Flitwick, leur professeur de Sortilèges et le directeur de Serdaigle, un petit homme minuscule qui n'avait pas du tout l'air d'être un ancien champion de duel. <<~Excellent~! Prodigieux~!~>>

Harry s'était attendu à être deuxième après Hermione, dans le pire des cas. Il aurait bien sûr préféré qu'\emph{elle} soit \emph{son} rival, mais il aurait accepté que la situation soit inversée.

En ce lundi, Harry se destinait à être bon dernier de la classe, une position pour laquelle il rivalisait en compagnie de tous les autres élevés-Moldus mis à part Hermione. Laquelle était seule et sans rival au sommet, pauvre petite.

Le professeur Flitwick se tenait au-dessus du pupitre d'une autre des nés-Moldus et ajustait doucement la façon dont elle tenait sa baguette.

Harry jeta un coup d'œil à Hermione. Il déglutit. C'était son rôle naturel dans l'ordre du monde… <<~Hermione~? dit timidement Harry. As-tu la moindre idée de ce que je pourrais faire de travers~?~>>

Ses yeux brillèrent d'une effroyable obligeance, et quelque chose au fond du cerveau de Harry hurla son désespoir et son humiliation.

Cinq minutes plus tard, l'eau de Harry semblait bien être perceptiblement plus froide que la température de la pièce, et Hermione lui avait donné de petites tapes verbales sur la tête, et lui avait dit de le prononcer avec plus de soin la prochaine fois, et était partie aider quelqu'un d'autre.

Le professeur Flitwick lui avait donné un point pour avoir aidé Harry.

Il grinçait des dents si fort que sa mâchoire lui faisait mal, et ça n'aidait pas sa prononciation.

\emph{Je me fiche que ce soit de la compétition déloyale. Je sais exactement ce que je vais faire avec mes deux heures supplémentaires par jour. Je vais m'asseoir dans ma malle et étudier jusqu'à ce que je sois au niveau de Hermione.}

\later

<<~La métamorphose est une des magies les plus complexes et les plus dangereuses qui vous seront données d'apprendre à Poudlard~>>, dit le professeur McGonagall. Il n'y avait pas la moindre trace de sourire sur le visage de la sombre vieille sorcière. <<~Quiconque s'agitant dans mon cours devra partir et ne reviendra pas. Vous avez été prévenus.~>>

Sa baguette s'abaissa et frappa son bureau, qui se remodela doucement en cochon. Deux élèves nés-Moldu émirent de petits glapissements. Le cochon regarda autour de lui avec un air confus puis redevint un bureau.

McGonagall balaya la classe du regard. Ses yeux s'arrêtèrent sur quelqu'un.

<<~M. Potter, dit le professeur McGonagall. Vous n'avez eu vos manuels qu'il y a quelques jours. Avez-vous commencé à lire votre manuel de métamorphose~?

--- Non, pardon professeur, dit Harry.

--- Vous n'avez pas à vous excuser, M. Potter, si vous deviez lire à l'avance, nous vous en aurions fait part.~>> Les doigts de McGonagall donnèrent un coup sec sur le bureau situé juste devant elle. <<~M. Potter, voudriez-vous bien essayer de deviner si c'est un bureau que j'ai brièvement métamorphosé en cochon, ou s'il était un cochon au début et que j'ai brièvement enlevé la métamorphose~? Vous le sauriez si vous aviez lu le premier chapitre de votre manuel.~>>

Les sourcils de Harry se plissèrent quelque peu. <<~J'imagine qu'il serait plus facile de commencer avec un cochon, puisque si ça avait d'abord été un bureau, il ne saurait peut-être pas comment se tenir debout.~>>

Le professeur McGonagall secoua la tête. <<~Ce n'est pas votre faute, M. Potter, mais la réponse correcte est qu'en cours de métamorphose, on ne cherche \emph{pas} à deviner. Les mauvaises réponses seront notées avec une sévérité extrême, les questions laissées vides seront notées avec une grande indulgence. Vous devez apprendre à savoir ce que vous ne savez pas. Si je vous pose n'importe quelle question, peu importe qu'elle soit basique ou évidente, et que vous répondez “Je ne suis pas sûr”, je ne vous en voudrai pas, et celui ou celle qui rira fera perdre des points à sa maison. Pouvez-vous me dire pourquoi cette règle existe, M. Potter~?~>>

\emph{Parce qu'une seule erreur de métamorphose peut être incroyablement dangereuse.}

<<~Non.

--- Correct. La Métamorphose est plus dangereuse que le transplanage, qui n'est pas enseigné avant la sixième année. La métamorphose doit malheureusement être apprise et pratiquée jeune afin de maximiser vos capacités une fois adulte. C'est donc un sujet dangereux, et vous devriez être assez effrayés à l'idée de faire la moindre erreur, car aucun de mes étudiants n'a jamais eu de séquelles permanentes, et je serais \emph{extrêmement ennuyée} si vous étiez la première classe à \emph{entacher mon dossier}.~>>

Certains étudiants déglutirent.

Le professeur McGonagall se leva et alla jusqu'au mur situé derrière son bureau. Il soutenait un tableau blanc accompagné de marqueurs et d'un effaceur. <<~Il existe de nombreuses raisons pour lesquelles la métamorphose est dangereuse, mais l'une d'elles s'élève au-dessus des autres.~>> Elle prit l'un des marqueurs et dessina des lettres rouge vif qu'elle souligna ensuite en bleu~:

\begin{center}
  \newsavebox{\hpbox}%
  \fontspec[ExternalLocation,Color=AA0000]{Halogen}
  \savebox{\hpbox}{\MakeUppercase{Une métamorphose n'est pas permanente~!}}
  \vspace{0.5ex}
  \usebox{\hpbox}
  \settowidth{\versewidth}{\usebox{\hpbox}}
  \vskip -1.7ex
  \fontspec[ExternalLocation,Color=2020FF]{ArchitectsDaughter}
  \resizebox{\versewidth}{.4ex}{\rotatebox{90}{I}}
\end{center}

<<~Une métamorphose n'est pas permanente~! dit McGonagall. Une métamorphose n'est pas permanente~! Une métamorphose n'est pas permanente~! M. Potter, supposez qu'un étudiant change un bloc de bois en un verre d'eau et que vous le buviez. Que pouvez-vous imaginer qu'il se passera lorsque la métamorphose se dissipera~?~>>

Il y eut une pause.

<<~Excusez-moi, M. Potter, je n'aurais pas dû vous demander cela, j'oubliais que vous êtes béni d'une imagination exceptionnellement pessimiste -

--- Pas de souci, dit Harry avalant bruyamment sa salive. Donc la première réponse est que je ne \emph{sais} pas~>>, McGonagall hocha la tête, <<~mais j'\emph{imagine} qu'il pourrait y avoir… du bois dans mon estomac, et dans mon système sanguin, et si une partie de cette eau avait été absorbée par mes tissus corporels -- serait-ce de la pulpe de bois ou du bois solide ou…~>> Le manque de maîtrise de la magie de Harry le mit en défaut. Il ne pouvait déjà pas comprendre comment le bois se transformait en eau~; il ne pouvait donc pas non plus comprendre ce qui se passerait après que les molécules d'eau furent mélangées par des mouvements thermiques normaux, et que la magie se fut dissipée, et que la transformation se soit inversée.

Le visage de McGonagall était rigide. <<~Comme M. Potter l'a correctement déduit, il deviendrait extrêmement malade et devrait être l'objet d'une attention médicale d'urgence. Merci d'ouvrir vos livres à la page 5.~>>

Même sans aucun son pour accompagner l'image mouvante, on pouvait voir que la femme à la peau horriblement décolorée hurlait.

<<~Le criminel qui a initialement métamorphosé de l'or en vin et l'a donné à boire à cette femme, “en paiement de la dette”, comme il l'a dit, a reçu une sentence de dix ans à Azkaban. Merci de vous rendre à la page 6. C'est un Détraqueur. Ce sont les gardiens d'Azkaban. Ils drainent votre magie, votre vie, et toute pensée heureuse que vous essayez d'avoir. L'image page 7 est le criminel dix ans plus tard, à sa sortie. Vous remarquerez qu'il est mort -- oui, M. Potter~?

--- Professeur, dit Harry, dans un cas comme celui-ci, si le pire se produit, y a-t-il un moyen de \emph{maintenir} la métamorphose~?

--- Non, répondit catégoriquement McGonagall. Maintenir une métamorphose est un drain magique permanent qui croît proportionnellement à la taille de la forme cible. Et il vous faudrait entrer en contact avec la cible régulièrement, à quelques heures d'intervalle, ce qui, dans un cas comme celui-ci, est impossible. Les désastres comme celui-ci sont \emph{irrécupérables}~!~>>

Le professeur McGonagall se pencha en avant. Son visage devint très dur.

<<~Vous ne changerez absolument jamais quelque chose en un liquide ou en un gaz, quelles que soient les circonstances. Pas d'eau, pas d'air. Rien qui ressemble à de l'eau, rien qui ressemble à de l'air. Même si ce n'est pas censé être bu. Les liquides \emph{s'évaporent}, des petites parties s'échappent dans les airs. Vous ne métamorphoserez rien qui soit destiné à être brûlé. Ça fera alors de la fumée et quelqu'un pourrait la respirer~! Vous ne métamorphoserez jamais rien qui puisse potentiellement se retrouver dans le corps de quelqu'un par quelque moyen que ce soit. Pas de nourriture. Rien qui \emph{ressemble} à de la nourriture. Même pas une petite blague amusante où vous comptiez les prévenir au sujet de votre tarte à la boue avant qu'ils ne la mangent pour de vrai. Vous ne le ferez jamais. Point. Dans cette classe ou hors de cette classe ou \emph{où que ce soit}. Est-ce bien compris par \emph{chaque élève}~?

--- Oui,~>> dirent Harry, Hermione, et quelques autres. Les autres semblaient sans voix.

--- \emph{Est-ce bien compris par chaque élève~?}

--- Oui, dirent-ils, marmonnèrent-ils, et chuchotèrent-ils.

--- Si vous brisez n'importe laquelle de ces règles, vous n'étudierez plus la métamorphose pendant votre séjour à Poudlard. Répétez après moi. Je ne métamorphoserai jamais rien en liquide ou en gaz.

--- Je ne métamorphoserai jamais rien en liquide ou en gaz, dirent les étudiants en un chorus syncopé.

--- Encore~! Plus fort~! Je ne métamorphoserai jamais rien en liquide ou en gaz.

--- Je ne métamorphoserai jamais rien qui ressemble à de la nourriture ou toute autre chose allant dans le corps humain.

--- Je ne métamorphoserai jamais rien qui puisse être brûlé car cela pourrait faire de la fumée.

--- Je ne métamorphoserai jamais rien qui ressemble à de l'argent, même de l'argent Moldu, dit le professeur McGonagall. Les gobelins ont les moyens de trouver le coupable. Et il est écrit dans la loi que la nation gobeline est dans un état de \emph{guerre} permanent avec les faussaires magiques. Ils n'enverront pas d'Aurors. Ils enverront une armée.

--- Je ne métamorphoserai jamais rien qui ressemble à de l'argent, répondirent les élèves en chœur.

--- Et \emph{par-dessus tout}, dit le professeur McGonagall, vous ne métamorphoserez aucun sujet vivant, en particulier vous-même. Cela vous rendra malade, et vous tuera même peut-être, selon la façon dont vous vous serez métamorphosés et selon la durée pendant laquelle vous aurez maintenu le changement.~>> Le professeur McGonagall marqua une pause. <<~M. Potter a en ce moment une main interrogative levée en l'air parce qu'il a vu une transformation en Animagus -- plus précisément un humain se transformant en chat et à nouveau en humain. Mais la transformation en Animagus n'est pas une métamorphose \emph{libre}.~>>

Le professeur McGonagall extirpa un petit morceau de bois de sa poche. D'un coup de baguette magique, il devint une sphère de verre. Puis elle dit <<~\emph{Crystferrium~!}~>> et la sphère de verre devint une sphère d'acier. Elle frappa une dernière fois et la sphère d'acier devint à nouveau un morceau de bois. <<~\emph{Crystferrium} transforme un sujet de verre solide en un objet d'acier solide doté d'une forme similaire. Il ne peut faire l'inverse, et il ne peut pas non plus transformer un bureau en cochon. La forme la plus générale de la métamorphose -- la métamorphose libre, que vous allez apprendre ici -- est capable de transformer un sujet en n'importe quelle cible, du moins en ce qui concerne la forme physique. C'est pour cela que la métamorphose libre doit être muette. Utiliser des sortilèges demanderait l'utilisation de mots différents pour chaque transformation entre sujet et objet.~>>

Le professeur McGonagall jeta un regard dur à ses élèves.

<<~\emph{Certains} enseignants commencent par les sortilèges de métamorphose et passent ensuite à la métamorphose libre. Oui, ce serait au départ beaucoup plus simple. Mais cela peut constituer un mauvais moule qui détériorera vos capacités ultérieures. Vous apprendrez ici la métamorphose libre \emph{dès le départ}, ce qui exige que vous jetiez le sort sans prononcer un mot, en représentant dans votre esprit la forme du sujet, la forme cible, et la transformation. %~>>

<<~Et pour répondre à la question de M. Potter, continua le professeur McGonagall, c'est la métamorphose \emph{libre} que vous ne devez jamais opérer sur un sujet vivant. Il y a des sortilèges et des potions qui peuvent transformer sans risque des sujets vivants, de façons \emph{limitées} et réversibles. Par exemple, un Animagus à qui il manque un membre ne retrouvera pas ce membre après s'être transformé. La métamorphose libre n'est \emph{pas} sûre. Votre corps changera pendant sa métamorphose -- la respiration par exemple produit une perte constante de matière corporelle qui est déversée dans l'atmosphère. Lorsque la métamorphose s'estompera et que votre corps essaiera de revenir à sa forme \emph{originale}, il ne sera pas tout à fait capable de le faire. Si vous collez votre baguette contre votre corps et que vous vous imaginez avec des cheveux dorés, ils tomberont une fois la métamorphose terminée. Si vous vous voyez avec une peau plus claire, vous passerez un long séjour à Sainte Mangouste. Et si vous vous métamorphosez en une forme adulte, alors, quand la métamorphose se dissipera, vous mourrez.~>>

Voilà qui expliquait l'existence de garçons gras ou de filles n'étant pas parfaitement jolies. Ou même de gens âgés tant qu'on y était. Ça n'existerait pas si on pouvait juste se métamorphoser tous les matins… Harry leva la main et essaya de signaler sa présence au professeur McGonagall du regard.

<<~\emph{Oui}, M. Potter~?

--- Est-il possible de métamorphoser un sujet vivant en une cible statique, comme une pièce -- non, pardon, je suis vraiment désolé, disons juste comme une sphère d'acier.~>>

Le professeur McGonagall secoua la tête. <<~M. Potter, même les objets inanimés subissent de petits changements internes au fil du temps. Il n'y aurait pas de changement visible sur votre corps après la transformation, et vous ne remarqueriez rien d'anormal pendant la première minute. Mais une heure plus tard, vous seriez très malade, et le lendemain, vous seriez mort.

--- Euh, excusez-moi, mais alors si j'avais lu le premier chapitre, j'aurais pu \emph{deviner} que le bureau était initialement un bureau et non un cochon, dit Harry, mais seulement si j'avais \emph{en plus} émis l'hypothèse que vous ne vouliez pas tuer le cochon, ce qui \emph{semble} hautement probable, mais -

--- M. Potter, je puis prédire que noter vos contrôles sera pour moi une immense source de ravissement. Mais si vous avez d'autres questions, pourrais-je s'il vous plaît vous demander d'attendre la fin du cours~?

--- Pas d'autres questions, professeur.

--- Maintenant répétez après moi, dit le professeur McGonagall. Je n'essaierai jamais de métamorphoser un sujet vivant, et en particulier moi-même, à moins que l'on ne m'ait spécifiquement chargé de le faire à l'aide d'un sortilège spécialisé ou d'une potion.

--- Si je ne suis pas certain que la métamorphose est sûre, je ne m'y essaierai pas avant d'avoir demandé l'autorisation au professeur McGonagall ou au professeur Flitwick ou au professeur Rogue ou au professeur Dumbledore, qui sont les seules autorités légitimes en matière de métamorphose à Poudlard. Demander à un autre étudiant n'est \emph{pas} une alternative acceptable, même s'ils disent se souvenir avoir posé la même question.

--- Même si le professeur de Défense actuel de Poudlard me dit qu'une métamorphose est sûre, et même si je vois le professeur de Défense la réaliser et que je ne vois rien de néfaste se produire, je ne m'y essaierai pas moi-même.

--- J'ai le droit inaliénable de refuser d'opérer toute métamorphose au sujet de laquelle je ressens la moindre nervosité. Puisque même le directeur de Poudlard ne peut me donner l'ordre de le faire, je n'accepterai certainement aucun ordre de ce genre venant du professeur de Défense, même si le professeur de Défense menace de déduire cent points à ma maison et de me faire exclure.

--- Si je brise une seule de ces règles je n'étudierai plus la métamorphose pendant mon séjour à Poudlard.

--- Nous répéterons ces règles au début de chaque cours pendant un mois, dit le professeur McGonagall. Et maintenant, nous allons commencer avec pour sujet des allumettes et pour cible des aiguilles… posez vos baguettes, merci bien, par “commencer”, je voulais dire que vous allez commencer à prendre des notes.~>>

Une demi-heure avant la fin du cours, le professeur McGonagall distribua des allumettes.

À la fin du cours, Hermione avait une allumette argentée, et le reste de la classe, né-Moldu ou non, avait exactement la même chose que ce qu'on leur avait donné au départ.

Le professeur McGonagall décerna un point de plus à Serdaigle.

\later

Après que la classe eut été remerciée, Hermione alla jusqu'au bureau de Harry alors que celui-ci rangeait ses livres dans sa bourse.

<<~Tu sais, dit Hermione d'un ton innocent, aujourd'hui, j'ai gagné deux points pour Serdaigle.

--- En effet, dit sèchement Harry.

--- Mais ce n'était pas aussi bien que tes \emph{sept} points, dit-elle. Je suppose que je ne suis pas aussi intelligente que toi.~>>

Harry finit de donner ses devoirs à manger à sa bourse et se tourna vers Hermione, les yeux en fentes. Il avait à vrai dire complètement oublié cet épisode.

Elle \emph{battit des paupières}. <<~Cela dit, nous avons des cours tous les jours. Je me demande combien de temps cela te prendra de trouver d'autres Poufsouffle à sauver~? Nous sommes lundi. Donc cela te donne jusqu'à jeudi.~>>

Ils se regardèrent dans le blanc des yeux, sans ciller.

Harry parla le premier.

<<~Tu te rends bien sûr compte que ça va être la guerre.

--- Je ne savais pas que nous étions en paix.~>>

Tous les autres étudiants regardaient maintenant la scène avec des yeux fascinés. Tous les autres étudiants plus le professeur McGonagall, malheureusement.

<<~Oh, M. Potter, fredonna le professeur McGonagall depuis l'autre extrémité de la pièce, j'ai de bonnes nouvelles pour vous. Madame Pomfresh a approuvé votre suggestion visant à réduire la fragilité des Portillons tournants, et nous comptons avoir terminé les modifications d'ici la fin de la semaine prochaine. J'imagine que cela mérite… disons dix points pour Serdaigle.~>>

La visage de Hermione trahissait un immense sentiment de trahison et d'éberluement. Harry supposa que son propre visage ne devait pas avoir l'air très différent.

<<~P\emph{rofesseur…} siffla Harry.

--- Il ne fait \emph{aucun doute} que vous méritez ces dix points, M. Potter. Je ne donnerais pas des points par caprice. De votre point de vue, vous avez simplement remarqué quelque chose de fragile et avez suggéré une façon de le protéger, mais les Portillons tournants sont coûteux et le Directeur n'était \emph{pas} ravi du tout la dernière fois que quelqu'un en a cassé un. McGonagall eut l'air pensive. Voyons, je me demande si un autre élève a jamais gagné dix-sept points dès son premier jour de cours. Il faudra que je vérifie, mais je pense que nous avons là un nouveau record. Peut-être devrions-nous faire une annonce pendant le dîner~?

--- \emph{PROFESSEUR}~! s'écria Harry. C'est \emph{notre} guerre~! Arrêtez de vous en mêler~!

--- Vous avez maintenant jusqu'à jeudi de la semaine \emph{prochaine}, M. Potter. À moins bien sûr que vous ne vous prêtiez à quelque espièglerie et ne \emph{perdiez} alors des points. En vous adressant à un professeur de façon irrespectueuse, par exemple.~>> Le professeur McGonagall se posa un doigt sur la joue et prit un air songeur. <<~Je m'attends à ce que vous atteignez les nombres négatifs avant vendredi soir.~>>

La bouche de Harry se referma immédiatement. Il jeta son meilleur Regard Mortel à McGonagall, mais elle sembla trouver cela amusant.

<<~Oui, une annonce au dîner, certainement, rêvassa le professeur McGonagall. Mais il ne faudrait pas offenser les Serpentard, nous garderons donc l'annonce brève. Juste le nombre de points et les faits… et si quelqu'un vient vous voir parce qu'il a besoin d'aide pour leurs devoirs et se voit déçu d'apprendre que vous n'avez même pas commencé à lire vos manuels, vous pourrez toujours les renvoyer vers Mademoiselle Granger.

--- P\emph{rofesseur~!} dit Hermione d'une voix relativement aiguë.

Le professeur McGonagall l'ignora. <<~Voyons, je me demande combien de temps Mademoiselle Granger mettra à faire quelque chose digne d'une annonce au dîner~? Quoi que ce soit, j'attends ce moment avec impatience.~>>

Harry et Hermione, par consentement muet, se retournèrent et quittèrent la salle à grands pas. Ils furent suivis par une traîne de Serdaigle hypnotisés.

<<~Euh, dit Harry. C'est toujours bon pour après dîner~?

--- Bien sûr, dit Hermione. Je ne voudrais pas te voir prendre du retard dans tes cours.

--- Que c'est aimable. Et permets-moi de te dire que, aussi brillante que tu aies été aujourd'hui, je ne peux m'empêcher de me demander de quoi tu seras capable une fois que tu auras eu une formation à la rationalité des plus élémentaires.

--- Est-ce vraiment si utile~? Ça n'avait pas l'air de t'aider en Sortilèges ou en Métamorphose.~>>

Il y eut une courte pause.

<<~Eh bien, j'ai eu mes manuels il y a quatre jours à peine. C'est pour ça que j'ai dû gagner ces dix-sept points sans utiliser ma baguette.

--- Il y a quatre jours~? Tu ne peux peut-être pas lire huit livres en quatre jours, mais tu aurais quand même pu en lire \emph{un}. Combien de jours cela te prendra-t-il à ce rythme~? Tu es fort en math, alors peux-tu me dire combien font huit fois quatre divisé par zéro~?

--- J'ai cours pour le moment, contrairement à toi, mais mes week-ends sont libres, donc… limite avec epsilon approchant zéro positif de huit fois quatre divisé par epsilon… dimanche à 10h47.

--- J'ai tout lu en \emph{trois} jours, tu sais.

--- Samedi à 14h47 donc. Je suis sûr que je trouverai un moyen.~>>

Et il y eut le soir, et il y eut le matin, le premier jour.

%  LocalWords:  rigideiro Flooing Crystferrium Erm
