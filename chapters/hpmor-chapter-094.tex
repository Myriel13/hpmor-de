\partchapter{Rôles}{V}

\section{La première rencontre~:}

\lettrine{L}{e} 17 avril 1992 à 6h07, le soleil se levait tout juste au-dessus de l'horizon du château de Poudlard, traversant les rideaux fermés du dortoir des premières années de Serdaigle, offrant une douce lumière rouge-orange d'aube peu altérée par le tissu blanc qui recouvrait les fenêtres et ne réveillant pas les garçons toujours endormis, encore habitués au rythme hivernal.

Dans un lit parmi tant d'autres, Harry Potter dormait du sommeil de l'exténué.

Sans bruit, la porte s'ouvrit.

Sans bruit, une silhouette s'avança.

La silhouette alla jusqu'au lit de Harry Potter.

La silhouette posa une main sur l'épaule du garçon endormit qui se réveilla et cria.

Personne d'autre ne l'entendit.

"M. Potter," couina le petit homme, "le directeur requiert votre présence, immédiatement."

Lentement, le garçon s'assit, ses mains bougeant brièvement sous ses couvertures. Il s'était attendu à se sentir beaucoup plus mal en se réveillant ce matin. Ça semblait… être anormal, que son cerveau fonctionne, que ses pensées se meuvent encore, qu'il ne soit pas rendu infirme par les larmes pendant une semaine. Le garçon savait que les cerveaux n'auraient jamais pu évoluer pour faire une chose pareille, car cela n'aurait constitué une adaptation à rien. Son côté obscur ne ferait certainement pas ça. Mais malgré cela, il lui semblait quand même anormal d'être vivant et lucide, ce matin là.

Mais sa résolution de ressusciter Hermione Granger lui sembla - suffisante, comme s'il agissait déjà comme il fallait, dans le bon sens~; elle serait ramenée, et c'était tout~; porter le deuil serait revenu à abandonner. Il n'y avait plus rien à choisir, plus d'ambiguïté, plus de conflit déchirant, plus besoin de se souvenir de ce qu'il avait \emph{vu}…

"Je vais m'habiller," dit Harry.

Le professeur Flitwick sembla assez réticent avant de parler mais il dit d'une voix haut perchée~: "Le directeur a précisé que vous deviez être amené dans son bureau sans délai, M. Potter. Je suis désolé."

Moins d'une minute plus tard - Le professeur Flitwick l'avait envoyé droit dans le bureau du directeur grâce aux feux de cheminette internes à Poudlard - Harry se retrouva, toujours dans son pyjama, face à Albus Dumbledore. La directrice adjointe était aussi assise, dans une autre chaise, et le maître des potions rôdait non loin entre les étranges appareils, surprit dans un immense bâillement au moment où Harry arrivait par la cheminée.

"Harry," dit le directeur sans préambule, "avant de dire ce que je vais te dire, je t'annonce que Hermione Granger est vraiment morte. Le système de sécurité l'a perçut et m'en a informé. Les pierres elles-mêmes ont parlé d'une sorcière qui avait perdu la vie. J'ai testé son corps là où elle était tombée et c'étaient les véritables restes de Hermione Granger, pas une poupée ou un sosie. Il n'existe aucun moyen connu des sorciers de défaire la mort. Cela étant dit, les restes de Hermione Granger ont maintenant disparu du débarras où ils avaient été placés et où tu les gardais. Les as-tu pris, Harry Potter~?

"Non," dit Harry en plissant les yeux. Un coup d'œil lui révéla que Severus l'observait maintenant avec intérêt.

Dumbledore semblait lui aussi très attentif, mais aussi amical. "Le corps de Hermione Granger est-il en ta possession~?"

"Non."

"Sais-tu où il est~?"

"Non."

"Sais-tu qui l'a pris~?"

"Non," dit Harry, puis il hésita. "Mis à part les évidentes spéculations probabilistes qui ne reposent sur aucune information connue de moi seul."

Le vieux sorcier hocha la tête. "Sais-tu pourquoi il a été pris~?"

"Non. Mis à part les évidentes spéculations et cetera."

"Que seraient ces spéculations~?". Les yeux âgés étaient comme aiguisés.

"Si l'ennemi peut remarquer que vous partez consulter les jumeaux Weasley pendant les cours après que Hermione est arrêtée et découvre cette carte magique dont vous dites qu'elle a été volée, alors l'ennemi peut se demander pourquoi je gardais le corps de Hermione Granger. À mon tour. Avez-vous organisé la mort de Hermione Granger dans l'espoir de récupérer l'argent de Lucius~?"

"\emph{Quoi~?}" dit le professeur McGonagall.

"Non," dit le vieux sorcier.

"Saviez-vous ou soupçonniez-vous que Hermione Granger allait mourir~?"

"Je ne le savais pas. Quant à mes soupçons, je l'avais mise là où, à ma connaissance, elle serait le mieux défendue contre Voldemort. Je n'ai pas voulu sa mort, je ne l'ai pas permise, je n'ai pas prévu d'en bénéficier, Harry Potter. Maintenant, montre-moi ta bourse."

"Elle est dans ma malle…" commença Harry.

"Severus," dit le vieux sorcier, et le maître des potions s'avança. "Vérifie aussi sa malle, chaque compartiment."

"Ma malle est protégée."

Severus Rogue eut un sourire sans joie et marcha dans la flamme verte.

Dumbledore sortit sa longue baguette noire et grise et commença à l'agiter non loin des cheveux de Harry avec l'air d'un Moldu muni d'un détecteur de métaux. Avant d'avoir atteint le cou de Harry, Dumbledore s'arrêta.

"La gemme sur ton anneau," dit Dumbledore. "Ce n'est plus un diamant transparent,. Il est marron, la couleur des yeux de Hermione Granger, la couleur de ses cheveux."

Une tension emplit soudain la pièce.

"C'est le rocher de mon père," dit Harry. "Métamorphosé comme avant. J'ai juste fait ça pour me souvenir de Hermione…"

"Je dois m'en assurer. Enlève cet anneau, Harry, et pose-le sur mon bureau."

Harry s'exécuta lentement, enleva la gemme et déposa l'anneau à l'autre bout du bureau.

Dumbledore pointa sa baguette vers la gemme et…

Un grand et banal rocher gris sauta en l'air sous l'impact de son expansion soudaine, frappa une sorte de barrière invisible au-dessus de lui puis retomba dans un grand craquement sur le bureau du directeur,

"Ça va me prendre une demi-heure de travail de le métamorphoser à nouveau," dit Harry d'un ton neutre.

Dumbledore reprit son examen. Harry dut enlever sa chaussure droite et enlever l'anneau de pied, son Portoloin d'urgence au cas où quelqu'un le kidnapperait et l'emmènerait à l'extérieur des limites de Poudlard (et ne mettrait en place aucun système anti-Transplanage, anti-Portoloin, anti-Phénix et anti boucle temporelle, ce que, avait confié Severus à Harry, n'importe quel Mangemort proche de Voldemort ferait certainement). Il fut vérifié que la magie émanant de l'anneau de doigt de pied était effectivement une magie de Portoloin et pas une magie de métamorphose. Le reste de Harry semblait être en règle.

Peu de temps après, le maître des potions revint avec la bourse de Harry et quelques autres objets magiques trouvés dans la malle de ce dernier que le directeur examina aussi, l'un après l'autre, y compris les éléments non utilisés de son kit de soin.

"Puis-je partir à présent~?" demanda Harry quand tout fut terminé, d'une voix aussi froide qu'il en fut capable. Il prit sa bourse et commença à lui donner son rocher gris à manger. L'anneau vide revint sur son doigt.

Le vieux sorcier expira et glissa sa baguette dans sa manche. "Je \emph{suis} navré," dit-il. "Il fallait que je sache. Harry… il semble que le Seigneur des Ténèbres a prit les restes de Hermione Granger. Je ne puis imaginer ce qu'il aurait à y gagner, mis à part envoyer son corps nous affronter sous la forme d'un Inferi. Severus te donnera certaines potions que tu devras garder sur toi. Sois prévenus, et sois prêt à faire le nécessaire, lorsqu'il le faudra."

"L'Inferi possédera-t-il l'esprit de Hermione~?"

"Non…"

"Alors ce ne sera pas elle. Puis-je partir~? Au moins pour quitter mon pyjama."

"Il y a autre chose, mais je serai bref. Le système de sécurité de Poudlard a détecté qu'aucune créature venue de l'extérieur n'est entrée et que c'est le professeur de Défense qui a tué Hermione Granger."

"Hmm," dit Harry.

\emph{Pensée 1~: Mais j'ai vu le troll tuer Hermione.}

\emph{Pensée 2~: Le professeur Quirrell m'a lancé un sortilège de faux souvenirs et a créé la mise en scène que Dumbledore a vue quand il est arrivé.}

\emph{Pensée 3~: Le professeur Quirrell ne peut pas faire ça, sa magie ne peut pas m'atteindre. Je l'ai vu à Azkaban…}

\emph{Pensée 4~: Puis-je faire confiance à ces souvenirs~?}

\emph{Pensée 5~: Il y a clairement eu une sorte de débâcle à Azkaban, nous n'aurions pas eu besoin d'un missile si le professeur Quirrell ne s'était pas évanoui, et pourquoi se serait-il évanoui sinon à cause de …}

\emph{Pensée 6~: Mais en fait, est-ce que j'ai déjà été à Azkaban~?}

\emph{Pensée 7~: J'ai clairement pratiqué mon contrôle sur les Détraqueurs avant d'effrayer celui du Magenmagot. Et \emph{ça}, c'était dans les journaux.}

\emph{Pensée 8~: Est-ce que je me souviens correctement de ce qu'ont dit les journaux~?}

"Hmm," répéta Harry. "Ce sortilège devrait vraiment être Impardonnable. Pensez-vous que le professeur Quirrell aurait pu lancer un sortilège de Faux Souvenirs…"

"Non. Je suis remonté dans le temps et j'ai placé certains instruments destinés à enregistrer la dernière bataille de Hermione, ne pouvant pas tout à fait supporter de l'observer moi-même." Le vieux sorcier semblait bien sombre. "Tu as deviné juste, Harry Potter. Voldemort a saboté tout ce que nous avions donné à Hermione pour sa protection. Son balais volant gisait entre ses mains. Sa cape d'invisibilité ne la masquait pas. Le troll marchait au soleil, indemne~; ce n'était pas une créature sauvage mais une arme affûtée et brandie. Et c'est bien le troll qui l'a tuée, par sa seule force, si bien que mes systèmes d'alarmes et mes toiles détectrices de magies hostiles ont été inutiles. Elle n'a jamais croisé la route du professeur Quirrell."

Harry déglutit, ferma les yeux, et réfléchit. "C'est donc une tentative de faire porter le chapeau au professeur Quirrell. On dirait. Cela semble effectivement être le mode opératoire de l'ennemi. Le troll a mangé Hermione Granger, que dit le système de sécurité~? Oh, regardez, en fait c'est le professeur de Défense, comme l'année dernière… non. Non, ça ne peut pas être ça."

"Pourquoi pas, M. Potter~?" dit le maître des potions. "Ça me semble assez évident…"

"C'est ça le problème."

\emph{L'ennemi est intelligent.}

Les vapeurs du sommeil quittaient lentement l'esprit de Harry, et après une bonne nuit de repos son cerveau pouvait voir ce qui n'avait pas été évident la veille.

Selon les conventions habituelles de la littérature… l'ennemi n'était pas censé regarder ce qu'on avait fait, saboter les objets magiques qu'on avait distribués, puis envoyer un troll rendu indétectable par quelque moyen que les héros ne pouvaient pas deviner, même une fois le méfait accompli, si bien que le résultat aurait été le même en ne faisant rien pour se défendre. Dans un livre, le point de vue restait généralement sur les personnages principaux. Voir l'ennemi passer outre tout le travail des protagonistes grâce à des plans et des actes entrepris hors du champ de vision de l'histoire aurait constitué un \emph{diabolus ex machina} et une source d'insatisfaction dramatique.

Mais dans la vraie vie, l'ennemi se verrait comme le personnage principal, il serait lui aussi malin, il penserait aux choses à l'avance, même si on ne le verrait pas faire. C'est pour cela que tout semblait si décousu, avec certaines parties inexpliquées et d'autre apparemment inexplicables. Qu'avait ressenti Lucius, quand Harry avait menacé Dumbledore de briser Azkaban~? Qu'avaient ressenti les Aurors d'Azkaban quand le balais s'était élevé au sommet d'une colonne de feu~?

\emph{L'ennemi est intelligent.}

"L'ennemi savait parfaitement que vous retourneriez dans le temps pour voir ce qui était vraiment arrivé à Hermione, en particulier parce que l'arrivée d'un troll à Poudlard nous révèle que quelqu'un peut tromper le système de sécurité." Harry ferma les yeux, intensifia sa réflexion, et essaya de se mettre à la place de l'ennemi. Pourquoi lui ou son côté obscur aurait-il fait une chose comme… "Nous sommes censés en conclure que l'ennemi contrôle ce que le système de sécurité nous dit. Mais c'est en fait une chose que l'ennemi peut seulement faire avec difficulté ou dans des circonstances particulières~; il essaie de créer une apparence trompeuse d'omnipotence." \emph{Comme je le ferais.} "Plus tard, le système de sécurité pourrait nous montrer le professeur Sinistra en train de tuer quelqu'un. Nous penserons que quelqu'un nous trompe à nouveau, mais en fait, le professeur Sinistra aura été victime d'une Légilimancie et \emph{aura} tué quelqu'un."

"Sauf si c'est précisément ce que le Seigneur des Ténèbres veut que nous pensions," dit Severus Rogue, un sourcil froncé sous l'effet de la concentration. "Auquel cas il contrôle bien le système de sécurité et le professeur Sinistra sera innocent."

"Le Seigneur des Ténèbres fomente-t-il \emph{vraiment} des plans avec autant de niveaux de méta…"

"Oui," dirent Dumbledore et Severus.

Harry hocha la tête, comme distrait. "Alors cela pourrait être une piège, soit pour nous faire croire que le système de sécurité dit la vérité quand il ment, soit qu'il ment quand il dit la vérité, selon le niveau de réflexion de notre part auquel l'ennemi s'attend. Mais si l'ennemi voulait que nous fassions confiance au système de sécurité… nous lui aurions fait confiance de toute façon si nous n'avions pas rencontré de raison de ne pas lui faire confiance. Il n'y avait donc pas besoin de faire l'effort de piéger le professeur Quirrell par une méthode dont nous allions comprendre que nous étions censés la découvrir, juste pour nous forcer à passer au niveau méta…"

"Pas tout à fait," dit Dumbledore. "Si Voldemort n'a pas un contrôle complet du système de sécurité, alors il fallait que le système croie que la main d'un professeur était à l'œuvre, sans quoi l'alarme aurait été sonnée à la première blessure de Mlle Granger et pas seulement au moment de sa mort."

Harry leva une main et se frotta un sourcil, juste sous ses cheveux.

\emph{OK, question sérieuse. Si l'ennemi est aussi malin, pourquoi est-ce que je suis toujours en vie~? Est-ce que c'est vraiment si difficile que ça d'empoisonner quelqu'un, est-ce qu'il y a des sortilèges, des potions et des Bézoards qui peuvent me guérir d'absolument tout ce qu'on pourrait glisser dans mon petit déjeuner~? Le système de sécurité le détecterait-il, pourrait-il pister la magie du meurtrier~?}

\emph{Ma} cicatrice \emph{pourrait-elle contenir le fragment d'âme qui maintient le Seigneur des Ténèbres dans ce monde, si bien qu'il ne veut pas me tuer~? Alors il essaie plutôt d'éloigner tous mes amis pour m'affaiblir mentalement et prendre le contrôle de mon corps~? Ça expliquerait l'histoire de Fourchelangue. Le Choixpeau ne pourrait peut-être pas détecter une phylactère-de-liche-machinchose. Problème évident numéro 1~: le Seigneur des Ténèbres est censé avoir fait son phylactère-de-liche-machinchose en 1943 en tuant machine et en faisant accuser M. Hagrid. Problème évident numéro 2, les âmes n'existent pas.}

\emph{Quoi que Dumbledore pensait aussi que mon sang était un ingrédient particulièrement important pour un rituel destiné à rendre toute sa force au Seigneur des Ténèbres, ce qui exigerait de me maintenir en vie jusqu'au moment où… oh, en voilà une pensée réjouissante.}

"Eh bien…" dit Harry. "Je suis sûr d'une chose."

"Et quelle est-elle~?"

"On doit sortir Neville de Poudlard \emph{maintenant}. C'est la prochaine cible évidente et aucun élève de première année ne peut survivre à une attaque de ce genre. On a de la chance que Neville n'ait pas été assassiné hier soir~; l'ennemi n'a pas à attendre qu'on ait fini notre deuil avant de jouer son prochain coup." \emph{Pourquoi l'ennemi n'a-t-il pas frappé pendant que nous étions distraits~?}

Dumbledore eut un échange de regards avec Severus, puis, alors que le visage du professeur McGonagall se pinçait soudain. "Harry," dit le vieux sorcier, "si tu éloignes toi-même tous tes amis, c'est exactement comme si Voldemort…"

"Je m'en sortirai \emph{très bien} sans Neville pendant deux autres mois, ce n'est pas comme si vous comptiez forcer mes amis à rester ici cet été, et ça ne \emph{suffit certainement pas à justifier} qu'on le laisse se faire tuer~! Professeur McGonagall…"

"Je suis plutôt d'accord," dit la sorcière écossaise. Elle fronça les sourcils. "Je suis extrêmement d'accord. Je suis tellement d'accord que… j'ai quelque difficulté à l'exprimer, Albus…"

"Tellement d'accord que vous allez le sortir d'ici vous-même, peu importe ce que quiconque en dit, parce que ce ne sera pas une excuse de dire que vous obéissiez aux ordres si Neville se fait tuer~?" dit Harry.

Le professeur McGonagall ferma brièvement les yeux. "Oui, mais il y existe certainement un moyen d'être responsable sans proférer des menaces d'actions unilatérales."

Le directeur soupira. "Pas besoin. Vas-y, Minerva."

"Attendez," dit le maître des potions alors même que le professeur McGonagall, d'un mouvement plutôt rapide, s'emparait d'une pincée de poudre verte dans un vase de cheminette. "Nous ne devrions pas attirer l'attention sur le garçon comme le directeur a attiré l'attention sur les jumeaux Weasley. Il serait plus sage, je pense, que la grand-mère de M. Londubat le sorte de Poudlard. Laissez-le rester dans la salle commune pour l'instant~; le Seigneur des Ténèbres ne semble pas capable d'agir autant à découvert."

Il y eut un long échange de regards entre eux quatre et Harry finit par hocher la tête, suivit par le professeur McGonagall.

"Quoi qu'il en soit," dit Harry. "Je suis sûr d'une autre chose."

"Et quelle est-elle~?" dit Dumbledore.

"J'ai vraiment besoin d'aller aux toilettes, et j'aimerais aussi quitter ce pyjama."

\later

"Au fait," dit Harry quand lui et le directeur émergèrent de la cheminette dans le bureau vide du directeur de Serdaigle. "Une dernière petite question rapide que je voulais vous poser. Cette épée que les jumeaux Weasley ont sortie du Choixpeau. C'était l'épée de Gryffondor, n'est-ce pas~?"

Le vieux sorcier se retourna~; son visage était neutre. "Qu'est-ce qui te fait penser ça, Harry~?"

"Le Choixpeau a hurlé \emph{Gryffondor~!} juste avant de la donner, elle avait un pommeau en rubis, des lettres d'or sur la lame, et le texte latin disait \emph{rien de mieux}. Juste une intuition."

"\emph{Nihil supernum}," dit le vieux sorcier. "Ce n'est pas \emph{tout à fait} ce que ça veut dire."

Harry hocha la tête. "Mmmh. Qu'est-ce que vous en avez fait~?"

"Je l'ai récupérée et je l'ai mise dans un endroit sûr," dit le vieux sorcier. Il regarda Harry avec sévérité. "J'espère que tu n'as pas l'avarice de la désirer pour toi, jeune Serdaigle."

"Pas du tout, je voulais juste m'assurer que vous n'alliez pas en priver ses maîtres légitimes de façon permanente. Donc les jumeaux Weasley sont l'héritier de Gryffondor~?"

"L'héritier de Gryffondor~?" dit Dumbledore, l'air surpris. Puis le vieux sorcier sourit, ses yeux bleus étincelèrent. "Ah, Harry, Salazar Serpentard a peut-être construit une Chambre des Secrets dans Poudlard, mais Godric Gryffondor ne se laissait pas tant aller à de telles extravagances. Tout ce que nous avons vu, c'est que Godric a légué son épée à la défense de Poudlard, si jamais un élève de valeur devait faire face à un ennemi qu'il ne pourrait vaincre seul."

"Ce n'est pas comme de répondre non. Ne croyez pas que je n'ai pas remarqué que vous n'avez pas dit non."

"Je n'ai pas vécu à cette époque, Harry, et je ne sais pas tout ce que Godric Gryffondor a pu faire, ni ce qui est certain qu'il n'a pas fait…"

"Estimez-vous personnellement que la probabilité qu'il existe quelque chose de semblable à un héritier de Gryffondor et que l'un ou les deux jumeaux Weasley le soient est supérieure à cinquante pour cent~? Oui ou non, une évasion veut dire oui. Vous n'arriverez pas à me distraire, peu importe à quel point j'ai besoin d'aller aux toilettes."

Le vieux sorcier soupira. "Oui, Fred et George Weasley sont l'héritier de Gryffondor. Je te supplie de ne pas leur en parler, pas encore."

Harry hocha la tête et se retourna, se préparant à partir. "Je suis surpris," dit-il. "J'ai un peu lu l'histoire de la vie de Godric Gryffondor. Les jumeaux Weasley sont… eh bien, ils sont géniaux de plus d'une façon, mais ils ne ressemblent pas beaucoup au Godric des livres d'histoire."

"Seul un homme incroyablement fier et vaniteux," dit doucement Dumbledore en se retournant dans la cheminette qui rugissait de flammes vertes, "pourrait croire que son héritier devrait être comme lui plutôt que comme celui qu'il aurait aimé être."

Le directeur entra dans le feu vert, et il disparut.

\latersection{La seconde rencontre (dans un coin à l'écart de la salle commune de Poufsouffle)~:}

Le visage de Neville Londubat était déformé par l'angoisse et il parlait comme si personne n'était là pour l'entendre, face au vide.

"Sérieux," lui répondit le vide. "Je porte une cape d'invisibilité avec des sortilèges de discrétion supplémentaires juste pour traverser les couloirs parce que \emph{je} ne veux pas me faire tuer. Mes parents me feraient sortir de Poudlard immédiatement si le directeur le leur permettait. Neville, te faire sortir de Poudlard, c'est du bon sens, ça n'a \emph{rien} à voir avec…"

"Je vous ai trahi, mon général," dit Neville, avec une voix aussi creuse qu'un enfant de onze ans pouvait raisonnablement en produire. "Je ne l'ai même pas fait à la façon Chaotique. J'ai obéi aux ordres et j'ai essayé de faire que tu leur obéisse toi aussi. Qu'est-ce que tu dis toujours, que dans la Légion du Chaos, un soldat uniquement capable de suivre les ordres est inutile~?"

"Neville", dit le vide d'un ton ferme. Le poids de deux mains drapées d'un fin tissu vinrent s'appuyer avec force sur les épaules de Neville et la voix s'approcha de lui. "Tu n'obéissais pas aveuglément aux ordres, tu essayais de me protéger. C'est vrai que dans notre monde chaotique, les soldats qui ne savent que suivre les ordres et les règles ne valent rien. Mais les soldats qui suivent les règles pour protéger leurs amis valent…"

"Un peu mieux que rien~?" dit Neville avec amertume.

"\emph{Beaucoup} mieux que rien. Neville, tu as commis une erreur de jugement. Elle m'a coûté environ six secondes. Il se pourrait que les blessures de Hermione lui aient été tout juste fatales, mais même alors, je ne pense pas que six secondes auraient suffit au troll pour qu'il prenne une autre bouchée de Hermione. Dans le monde hypothétique où tu ne t'es pas interposé, Hermione meurt quand même. Maintenant, je pourrais rester ici et te faire la liste des dix façons dont j'aurais pu assurer la survie de Hermione si je n'avais pas été stupide…"

"Toi~? \emph{Tu} lui as couru après. C'est \emph{moi} qui ai essayé de t'arrêter. Si c'est la faute de quelqu'un, c'est la mienne," dit Neville d'un ton amer.

Le vide réagit par un silence qui dura un moment.

"Waouh," dit enfin le vide. "Waouh. Je dois dire que c'est une façon sacrément différente de voir les choses. J'essaierai de m'en souvenir la prochaine fois que je ressentirai l'envie de me blâmer pour quelque chose. Neville, le terme technique qui nous intéresse est 'biais égocentrique', c'est le fait que tu ressens tout de ta propre vie mais que tu ne peux pas ressentir tout ce qui se passe dans le reste du monde. Il s'est passé beaucoup, beaucoup de choses avant et après que tu ai déboulé devant moi. Tu vas passer des semaines à te souvenir de ce que tu as fait pendant six secondes, ça se voit, mais personne d'autre ne va prendre la peine d'y penser. Les autres passent beaucoup moins de temps à réfléchir à tes erreurs passées que toi, simplement parce que tu n'es pas au centre de leur monde. Je te \emph{garantis} que personne à part toi n'a même \emph{envisagé} de blâmer Neville Londubat pour ce qui est arrivé à Hermione Granger. Pas pendant une fraction de seconde. Tu te comportes, si tu me pardonnes l'expression, comme un petit idiot. Maintenant tais-toi et dis moi au revoir."

"Je ne veux pas dire au revoir," dit Neville. Sa voix tremblait mais il parvint à ne pas pleurer. "Je veux rester ici et me battre avec toi contre… contre ce qui se passe."

Le vide s'approcha de lui, le prit dans ses bras, et la voix de Harry Potter murmura~: "Pas de bol." 

%  LocalWords:  diabolus Mmhm
