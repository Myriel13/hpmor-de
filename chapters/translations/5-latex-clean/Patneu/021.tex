

\hypertarget{rationalisierung}{% \section{21. Rationalisierung}\label{rationalisierung}}

\textbf{Kapitel 21: Rationalisierung

}

Wer Rowlings Pflicht erfüllt, ist Rowling.

\later

Hermine Granger hatte sich gesorgt, dass sie böse würde.

Der Unterschied zwischen Gut und Böse war üblicherweise einfach zu erkennen, sie hatte nie verstanden warum andere Leute solche Schwierigkeiten damit hatten. In Hogwarts waren „Gut“ Professor Flitwick und Professor McGonagall und Professor Sprout. „Böse“ waren Professor Snape und Professor Quirrell und Draco Malfoy. Harry Potter… war einer jener unüblichen Fälle, bei denen man es \emph{nicht} auf den ersten Blick sagen konnte. Sie versuchte noch immer herauszufinden, wo er hingehörte.

Doch wenn es um \emph{sie selbst} ging…

Hermine hatte \emph{zu viel Spaß} daran, Harry Potter vernichtend zu schlagen.

Sie hatte sich in jedem einzelnen Unterricht, den sie belegten, besser gemacht als er. (Außer Besenstiel-Reiten, was wie Sportunterricht war, es zählte nicht.) Sie hatte an fast jedem Tag ihrer ersten Woche \emph{echte} Hauspunkte bekommen, nicht für merkwürdige heldenhafte Dinge, sondern \emph{schlaue} Dinge, wie Zauber schnell zu lernen und anderen Schülern zu helfen. Sie wusste, Hauspunkte dieser Art waren besser und das beste war, Harry Potter wusste es auch. Sie konnte es in seinen Augen sehen, jedes mal wenn sie einen \emph{echten} Hauspunkt gewann.

Wenn man Gut war, sollte man das Gewinnen nicht so genießen.

Es hatte am Tag der Zugfahrt begonnen, doch es hatte eine Weile gedauert bis der Wirbelwind sich gelegt hatte. Erst später in jener Nacht war Hermine langsam klar geworden, \emph{wie sehr} sie sich von diesem Jungen hatte überrumpeln lassen.

Bevor sie Harry Potter getroffen hatte, hatte sie niemanden, den sie besiegen wollte. Wenn jemand im Unterricht nicht so gut zurechtkam wie sie, war es ihr Job ihm zu helfen, nicht es ihm unter die Nase zu reiben. Das bedeutete es, Gut zu sein.

Und jetzt…

… jetzt \emph{gewann} sie, Harry Potter zuckte jedes mal zusammen, wenn sie einen Hauspunkt bekam und es war \emph{so ein Spaß,} ihre Eltern hatten sie vor Drogen gewarnt und sie vermutete dies hier machte \emph{mehr Spaß.}

Sie hatte immer das Lächeln gemocht, das Lehrer ihr zuwarfen wenn sie etwas richtig machte. Sie hatte es immer gemocht die lange Reihe von Häkchen auf einem perfekt beantworteten Test zu sehen. Doch wenn sie jetzt gut im Unterricht war, sah sie sich unauffällig um und erhaschte einen Blick auf Harry Potter, der mit den Zähnen knirschte und dann war es als wollte sie in Gesang ausbrechen wie in einem Disney-Film.

Das war Böse, oder nicht?

Hermine hatte sich gesorgt, dass sie Böse wurde.

Und dann war ihr ein Gedanke gekommen, der all ihre Ängste hinweg gewischt hatte.

Sie und Harry verliebten sich! Natürlich! Jeder wusste, was es hieß, wenn ein Junge und ein Mädchen anfingen, die ganze Zeit zu streiten. Sie \emph{machten sich den Hof!} \emph{Daran} war nichts Böses.

Es konnte nicht sein, dass es ihr einfach \emph{gefiel} dem berühmtesten Schüler der Schule in den schulischen Hintern zu treten, jemanden, der \emph{in} Büchern war und wie Bücher \emph{sprach}, der Junge der den Dunklen Lord bezwungen und selbst \emph{Professor Snape} wie einen lausigen kleinen Käfer zerquetscht hatte, den Jungen der, wie Professor Quirrell es ausgedrückt hätte, dominant war über jeden Ravenclaw im ersten Jahr \emph{außer} Hermine Granger, die den Jungen-der-überlebt-hatte vollkommen \emph{platt machte} in jedem Unterrichtsfach außer Besenstiel-Reiten.

Denn das wäre Böse gewesen.

Nein. Es war eine Romanze. \emph{Das} war es. \emph{Deshalb} stritten sie sich.

Hermine war froh, dass sie das rechtzeitig zum heutigen Tag herausgefunden hatte, an dem Harry ihren Buch-lese-Wettstreit verlieren würde, über den die \emph{ganze Schule} Bescheid wusste und sie wollte vor schierer überschäumender Freude darüber \emph{tanzen}.

Es war 2:45~Uhr am Samstag Nachmittag und Harry Potter hatte noch die Hälfte von Bathilda Bagshots \emph{Geschichte der Zauberei}* vor sich und sie starrte auf ihre Taschenuhr, die mit verhängnisvoller Langsamkeit auf 2:47~Uhr zu tickte.

Und der gesamte Ravenclaw-Gemeinschaftsraum sah zu.

Es waren nicht nur die Erstklässler, die Nachricht hatte sich wie ein Lauffeuer verbreitet und halb Ravenclaw drängte sich in dem Raum zusammen, auf Sofas gequetscht, an Bücherschränke gelehnt und auf Armlehnen von Stühlen sitzend. Alle sechs Vertrauensschüler waren da, einschließlich der Schulsprecherin von Hogwarts. Jemand hatte einen Lufterfrischungs-Zauber wirken müssen, nur damit es genug Sauerstoff gab. Und das Getöse der Gespräche war zu einem Flüstern und nun zu völliger Stille verstummt.

2:46~Uhr.

Die Spannung war unerträglich. Wenn es irgendwer sonst, \emph{irgendjemand} anders gewesen wäre, seine Niederlage wäre bereits Gewissheit gewesen.

Doch dies war Harry Potter und man konnte die Möglichkeit nicht ausschließen, dass er, irgendwann in den nächsten paar Sekunden, eine Hand heben und mit den Fingern schnipsen würde.

Mit plötzlichem Schrecken wurde ihr klar, dass Harry Potter vielleicht genau das tun könnte. Es würde \emph{genau zu ihm passen,} die zweite Hälfte des Buches \emph{bereits vorher fertig gelesen} zu haben…

Hermines Sicht begann zu verschwimmen. Sie versuchte, sich zum Atmen zu zwingen und stellte fest, dass sie es nicht konnte.

Zehn Sekunden blieben und noch immer hatte er seine Hand nicht gehoben.

Noch fünf Sekunden.

2:47~Uhr.

Harry Potter steckte sorgsam ein Lesezeichen in sein Buch, klappte es zu und legte es beiseite.

„Ich würde gern für die Nachwelt festhalten,“ sagte der Junge-der-überlebt-hat mit klarer Stimme, „dass ich nur noch ein halbes Buch übrig hatte und mit ein paar unerwarteten Verspätungen zu kämpfen hatte—“

„\emph{Du hast verloren!}“ kreischte Hermine. „Das \emph{hast} du! Du \emph{hast unseren Wettstreit verloren!}“

Ein kollektives Ausatmen war zu vernehmen, als alle wieder anfingen zu atmen.

Harry Potter schoss ihr einen Feurig Lodernden Blick zu, doch sie schwebte in einer Aura reinster weißer Glückseligkeit und nichts konnte ihr etwas anhaben.

„\emph{Ist dir klar, was ich für eine Woche hatte?}“ sagte Harry Potter. „Jedem geringeren Wesen wäre es schwer gefallen auch nur acht Dr.-Suess-Bücher zu lesen!“

„\emph{Du} hast das Zeitlimit festgelegt.“

Harrys Blick Lodernden Feuers wurde noch heißer. „Es gab keinen logischen Weg für mich zu erahnen, dass ich die ganze Schule vor Professor Snape würde retten müssen oder im Verteidigungs-Unterricht augemischt würde und wenn ich dir sagen würde, wie ich all die Zeit zwischen 5~Uhr nachmittags und dem Abendessen am Donnerstag verloren habe, hieltest du mich für verrückt—“

„Oooh, klingt als wäre da \emph{jemand} dem \emph{Planungs-Fehlschluss} zum Opfer gefallen.“

Der blanke Schock zeigte sich auf Harry Potters Gesicht.

„Oh, das erinnert mich daran, ich habe den ersten Stapel Bücher, die du mir geliehen hast, fertig gelesen,“ sagte Hermine mit ihrer besten Unschuldsmiene. Ein paar von denen waren auch noch ziemlich \emph{schwer} gewesen. Sie fragte sich wie lange \emph{er} gebraucht hatte, um sie zu lesen.

„Eines Tages,“ sagte der Junge-der-überlebt-hat, „wenn die fernen Nachfahren des \emph{Homo} \emph{sapiens} auf die Geschichte der Galaxis zurückblicken und sich fragen, wie alles so schief gehen konnte, werden sie zu dem Schluss kommen, dass der ursprüngliche Fehler war, dass jemand Hermine Granger das Lesen beigebracht hat.“

„Doch du hast trotzdem verloren,“ sagte Hermine. Sie hielt sich mit einer Hand das Kinn und blickte nachdenklich drein. „Nun, was genau solltest du verlieren, frage ich mich?“

„\emph{Was?}“

„Du hast die Wette verloren,“ erklärte Hermine, „also musst du einen Preis zahlen.“

„Ich kann mich nicht erinnern, dem zugestimmt zu haben!“

„Wirklich?“ sagte Hermine Granger. Sie setzte einen nachdenklichen Gesichtsausdruck auf. Dann, als sei ihr die Idee gerade erst gekommen, „Dann stimmen wir doch ab. Alle die glauben, Harry Potter muss einen Preis zahlen, hebt die Hand!“

„\emph{Was?}“ kreischte Harry Potter erneut.

Er wirbelte herum und sah, dass er von einem Meer erhobener Hände umgeben war.

Und hätte Harry Potter \emph{genauer} hingesehen, hätte er bemerkt, dass verdammt viele der Zuschauer Mädchen zu sein schienen und praktisch jede weibliche Person im Raum die Hand gehoben hatte.

„Stop!“ klagte Harry Potter. „Ihr wisst gar nicht, was sie fordern wird! \emph{Merkt} ihr nicht, was sie tut? Sie lässt euch jetzt eine Vorverpflichtung eingehen und dann wird der Druck dem zu entsprechen euch dazu bringen, dem zuzustimmen, was immer sie auch sagt!“

„Keine Sorge,“ sagte die Vertrauensschülerin Penelope Clearwater. „Wenn sie etwas unvernünftiges fordert, können wir unsere Meinung einfach ändern. Oder, Leute?“

Und es folgte eifriges Nicken von allen Mädchen, denen Penelope von Hermines Plan erzählt hatte.

\later

Eine stille Gestalt schlüpfte leise durch die kühlen Hallen der Verliese von Hogwarts. Er sollte um 6:00~Uhr nachmittags in einem bestimmten Raum sein, um eine bestimmte Person zu treffen und wenn irgend möglich war es besser früh dran zu sein, um Respekt zu zeigen.

Doch als seine Hand den Türknauf drehte und die Tür dieses finsteren, stillen, ungenutzten Klassenraumes öffnete, stand dort bereits eine Silhouette inmitten der Reihen staubiger alter Pulte. Eine Silhouette, die einen kleinen grün-glühenden Stab hielt, der fahles Licht verströmte welches ihn, der sie hielt, kaum zu erleuchten vermochte, geschweige denn das umgebende Zimmer.

Das Licht des Korridors erstarb als die Tür hinter ihm ins Schloss fiel und Dracos Augen begannen, sich an das dämmerige Glühen zu gewöhnen.

Die Silhouette drehte sich langsam, um ihn zu betrachten, enthüllte dabei ein überschattetes Gesicht, nur zum Teil erhellt von dem unheimlichen grünen Licht.

Draco gefiel dieses Treffen bereits. Behielte man das kalte grüne Licht, machte sie beide größer, gäbe ihnen Kutten und Masken, versetzte sie vom Klassenzimmer auf einen Friedhof und es wäre genau so wie die Hälfte der Geschichten, die Freunde seines Vaters über die Todesser erzählten.

„Ich will, dass du weißt, Draco Malfoy,“ sagte die Silhouette mit tödlicher Ruhe in der Stimme, „dass ich dir für meine jüngste Niederlage nicht die Schuld gebe.“

Draco öffnete ohne Nachdenken protestierend den Mund, es gab keinen \emph{Grund} ihm die Schuld zu geben—

„Der Grund war, mehr als alles andere, meine eigene Dummheit,“ fuhr die schattenhafte Gestalt fort. „Ich hätte viele Dinge anders machen können, an jedem Punkt des Weges. Du hast mich nicht gebeten, \emph{genau das} zu tun was ich tat. Du hast nur um Hilfe gebeten. Ich war derjenige, der unklugerweise diese bestimmte Vorgehensweise gewählt hat. Doch der Fakt bleibt, dass ich den Wettstreit um ein halbes Buch verloren habe. Die Handlungen deines idiotischen Domestiken und der Gefallen, den du erbeten hast und, ja, meine eigene Dummheit haben mich \emph{Zeit gekostet.} Mehr Zeit als du ahnst. Zeit die sich, letztendlich, als entscheidend herausgestellt hat. Der Fakt bleibt, Draco Malfoy, dass, hättest du jenen Gefallen nicht erbeten, ich gewonnen \emph{hätte}. Und nicht… stattdessen… \emph{verloren.}“

Draco hatte bereits von Harrys Niederlage gehört und dem Preis, den Granger ihm abverlangt hatte. Die Neuigkeiten hatten sich schneller verbreitet als Eulen sie hätten tragen können.

„Ich verstehe,“ sagte Draco. „Es tut mir leid.“ Er \emph{konnte} nichts anderes sagen, wenn er Harry Potter zum Freund haben wollte.

„Ich bitte nicht um Verständnis oder Sorge,“ sagte die dunkle Silhouette, noch immer mit tödlicher Ruhe. „Doch ich habe gerade zwei volle Stunden in der Gegenwart von Hermine Granger verbracht, in der Kleidung, die mir dargereicht wurde, während des Besuchs solch faszinierender Orte in Hogwarts wie einem kleinen gluckernden Wasserfall aus etwas, das für mich aussah wie Rotz, begleitet von einer Anzahl weiterer Mädchen, die auf solch hilfreiche Maßnahmen bestanden, wie unseren Weg mit transfigurierten Rosenblüten zu bestreuen. Ich war bei einem Date, Spross von Malfoy. Mein \emph{erstes} Date. \emph{Und wenn ich diesen Gefallen einfordere, wirst du dafür bezahlen müssen.}“

Draco nickte ernsthaft. Vor seinem Erscheinen hatte er in weiser Voraussicht jedes verfügbare Detail über Harrys Date in Erfahrung gebracht, damit er all sein hysterisches Gelächter vor ihrem vereinbarten Treffen hinter sich bringen konnte und ihm kein \emph{Fauxpas} unterlief, indem er die ganze Zeit kicherte, bis er ohnmächtig wurde.

„Glaubst du,“ sagte Draco, „dem Granger-Mädchen sollte etwas trauriges zustoßen—“

„Verbreite in Slytherin, dass das Granger-Mädchen \emph{mir} gehört und alle, die sich in \emph{meine} Angelegenheiten einmischen, ihre Überreste über ein Gebiet verteilt finden werden, in dem zwölf verschiedene Sprachen gesprochen werden. Und da ich kein Gryffindor bin und auf \emph{List} statt offenen Angriff setze, sollte man sich nichts dabei denken, wenn man sehen sollte, wie ich sie anlächle.“

„Oder wenn du auf einem zweiten Date gesehen wirst?“ sagte Draco und erlaubte sich nur einen kleinen Hauch von Skepsis in der Stimme.

„\emph{Es wird kein zweites Date geben,}“ sagte die grün-erleuchtete Silhouette mit einer so beängstigenden Stimme, dass sie nicht nur wie bei einem Todesser klang, sondern wie Amycus Carrow das eine mal kurz bevor ihm Vater sagte, er solle aufhören, er sei nicht der Dunkle Lord.

Natürlich \emph{war} es immer noch die hohe ungebrochene Stimme eines kleinen Jungen und in Verbindung mit den \emph{tatsächlichen Worten,} nun, es funktionierte einfach nicht. Wenn Harry Potter eines Tages der Dunkle Lord \emph{wurde}, würde Draco ein Denkarium verwenden, um eine Kopie dieser Erinnerung an einem sicheren Ort zu verwahren und Harry Potter würde es niemals wagen ihn zu verraten.

„Doch sprechen wir von fröhlicheren Dingen,“ sagte die grün-beschattete Gestalt. „Sprechen wir über Wissen und über Macht. Draco Malfoy, sprechen wir über Wissenschaft.“

„Ja,“ sagte Draco. „Reden wir.“

Draco fragte sich wie viel von seinem eigenen Gesicht zu sehen war und wie viel im Schatten lag, in jenem unheimlichen grünen Licht.

Und obwohl Dracos Gesichtsausdruck ernst blieb, war ein Lächeln in seinem Herzen.

Er führte \emph{endlich} eine richtige erwachsene Unterhaltung.

„Ich biete dir Macht,“ sagte die schattenhafte Gestalt, „und ich werde dir berichten von dieser Macht und ihrem Preis. Die Macht stammt aus dem Wissen um die Beschaffenheit der Realität, mit dem man die Kontrolle über sie erlangt. Was man versteht, kann man beherrschen und das ist Macht genug, um auf dem Mond zu spazieren. Der Preis dieser Macht ist, dass man lernen muss, der Natur Fragen zu stellen und sehr viel schwieriger, die Antworten der Natur zu akzeptieren. Du wirst experimentieren, Tests durchführen und sehen was passiert. Und du musst die Bedeutung dieser Ergebnisse akzeptieren, wenn sie dir sagen, dass du falsch liegst. Du wirst \emph{lernen müssen, wie man verliert,} nicht gegen mich, gegen die Natur. Wenn du mit der Realität im Streit liegst, wirst du die Realität gewinnen lassen müssen. Du wirst dies schmerzhaft finden, Draco Malfoy und ich weiß nicht ob du in dieser Hinsicht die nötige Stärke besitzt. Im Wissen um den Preis, ist es noch immer dein Wunsch, von der Macht des Menschen zu erfahren?“

Draco nahm einen tiefen Atemzug. Er hatte darüber nachgedacht. Und er konnte schwerlich etwas anderes antworten. Er war angewiesen worden, jeden Weg zur Freundschaft mit Harry Potter zu nutzen. Er \emph{lernte} nur, er versprach nicht irgendetwas zu \emph{tun}. Er konnte die Lektionen zu jedem beliebigen Zeitpunkt beenden…

Es gab sicherlich eine Anzahl von Aspekten dieser Situation, die es wie eine Falle wirken ließen, doch ganz ehrlich, Draco sah nicht wie das schief gehen konnte.

Außerdem wollte Draco schon irgendwie die Welt regieren.

„Ja,“ sagte Draco.

„Exzellent,“ sagte die schattenhafte Gestalt. Ich hatte eine etwas \emph{vollgestopfte Woche} und es wird Zeit in Anspruch nehmen, deinen Lehrplan auszuarbeiten -"

„Ich muss selbst so einige Dinge erledigen, um meine Macht in Slytherin zu festigen,“ sagte Draco, „ganz zu schweigen von Hausaufgaben. Vielleicht sollten wir einfach im Oktober beginnen?“

„Klingt vernünftig,“ sagte die schattenhafte Gestalt, „doch was ich sagen wollte war, um deinen Lehrplan auszuarbeiten, muss ich wissen, was ich dich lehren werde. Drei Möglichkeiten drängen sich auf. Die erste ist, dass ich dich über den menschlichen Geist und das Gehirn unterweisen werde. Die zweite Option ist, dass ich dir vom physikalischen Universum erzähle, jene Künste die den Pfad zu einem Besuch auf dem Mond ebnen. Dafür sind sehr viele Zahlen von Nöten, doch für bestimmte Geister sind diese Zahlen so wunderschön wie nichts anderes, das die Wissenschaft zu bieten hat. Magst du Zahlen, Draco?“

Draco schüttelte den Kopf.

„Dann soviel dazu. Du wirst trotzdem deinen Teil Mathematik erlernen, doch nicht sofort, denke ich. Die dritte Möglichkeit ist, ich unterrichte dich in Genetik, Evolution und Vererbung, was du als Blut bezeichnen würdest—“

„Das,“ sagte Draco.

Die Gestalt nickte. „Ich dachte mir, du würdest das sagen. Doch ich denke, dies wird der schmerzhafteste Pfad für dich sein, Draco. Was wenn deine Familie und Freunde, die Blutreinheits-Verfechter, das eine sagen und du feststellst, das Experiment sagt etwas anderes?“

„Dann finde ich heraus, wie ich das Experiment dazu bringe, die \emph{richtige} Antwort zu geben!“

Es entstand eine Pause, während die schattenhafte Gestalt kurz mit offen stehendem Mund dastand.

„Ähm,“ sagte die schattenhafte Gestalt. „So funktioniert es aber nicht. Davor habe ich dich zu warnen versucht, Draco. Du \emph{kannst nicht} dafür sorgen, dass die Antwort immer so lautet, wie es dir gefällt.“

„Man kann \emph{immer} dafür sorgen, dass die Antwort im eigenen Sinne ist,“ sagte Draco. Das war praktisch das erste gewesen, was seine Privatlehrer ihm beigebracht hatten. „Es geht nur darum, die richtigen Argumente zu finden.“

„Nein,“ sagte die schattenhafte Gestalt zunehmend frustrierter, „nein, nein, nein! Dann bekommst du die \emph{falsche Antwort} und so kannst du nicht zum Mond fliegen! Die Natur ist keine Person, du kannst sie nicht durch Tricks dazu bringen, etwas anderes zu glauben; wenn du versuchst dem Mond zu erzählen, dass er aus Käse gemacht ist, kannst du tagelang argumentieren, doch das verändert nicht den Mond! Wovon du sprichst, ist \emph{Rationalisierung}, als würdest du ein Stück Papier nehmen, direkt bis zur letzten Zeile springen und mit Tinte hinschreiben 'und \emph{deshalb} ist der Mond aus Käse gemacht' und dann wieder nach oben springen um alle möglichen cleveren Argumente darüber zu schreiben. Aber entweder ist der Mond aus Käse gemacht oder er ist es nicht. In dem Moment, als du die Schlusszeile geschrieben hast, war sie bereits wahr oder bereits falsch. Ob das ganze Blatt Papier mit der richtigen Schlussfolgerung oder der falschen endet, steht bereits fest in dem Moment, wenn du die Schlusszeile schreibst. Wenn du versuchst zwischen zwei teuren Koffern zu wählen und du magst den glänzenden, spielt es keine Rolle welche cleveren Argumente dir dafür einfallen, ihn zu kaufen; die \emph{wirkliche} Regel nach der du den Koffer, für den du \emph{argumentieren wolltest, ausgewählt hast,} war 'nimm den glänzenden' und ob diese Regel effektiv ist um gute Koffer auszuwählen oder nicht, das ist der Koffer, den du bekommst. Rationalität \emph{kann man nicht} benutzen, um für eine feste Seite zu argumentieren, man kann sie nur dafür benutzen zu \emph{entscheiden, für welche Seite man streiten soll.} Wissenschaft ist nicht dazu da, irgendwen davon zu \emph{überzeugen}, dass die Blutreinheits-Verfechter recht haben. Das ist \emph{Politik!} Die Macht der Wissenschaft ist, \emph{das durch Argumente unveränderbare wahre Wesen der Natur zu ergründen!} Was die Wissenschaft \emph{kann}, ist uns zu sagen, \emph{wie Blut wirklich funktioniert,} wie Zauberer wirklich ihre Kräfte von ihren Eltern erben und ob Muggelgeborene wirklich schwächer oder stärker sind—“

„\emph{Stärker!}“ sagte Draco. Er hatte versucht, dem zu folgen, mit Verwirrung auf dem Gesicht, er konnte sehen, dass es \emph{irgendwie} Sinn machte, doch es war definitiv anders als alles von dem er je zuvor gehört hatte. Und dann hatte Harry Potter etwas gesagt, das Draco unmöglich durchgehen lassen konnte. „Du denkst Schlammblüter sind \emph{stärker?}“

„Ich denke gar nichts,“ sagte die schattenhafte Gestalt. „Ich weiß nichts. Ich glaube nichts. Meine Schlusszeile ist noch nicht geschrieben. Ich werde herausfinden, wie sich die magische Kraft von Muggelgeborenen und die magische Kraft von Reinblütern testen lassen. Wenn meine Tests mir sagen Muggelgeborene sind schwächer, werde ich glauben sie sind schwächer. Wenn meine Tests mir sagen Muggelgeborene sind stärker, werde ich glauben sie sind stärker. Durch das Wissen um diese und andere Wahrheiten, werde ich ein gewisses Maß an Macht erlangen—“

„Und du erwartest von \emph{mir}, zu glauben was immer du sagst?“ verlangte Draco hitzig.

„Ich erwarte von dir, die Tests \emph{persönlich} auszuführen,“ sagte die schattenhafte Gestalt leise. „Fürchtest du das, was \emph{du} erfahren wirst?“

Draco starrte die schattenhafte Gestalt eine Weile an, seine Augen verengten sich. „Nette Falle, Harry,“ sagte er. „Die werde ich mir merken müssen, sie ist neu.“

Die schattenhafte Gestalt schüttelte den Kopf. „Es ist keine Falle, Draco. Denk daran - ich \emph{weiß nicht,} was wir herausfinden werden. Doch man kann das Universum nicht verstehen, indem man mit ihm streitet oder ihm sagt, es solle beim nächsten mal mit einer anderen Antwort kommen. Wenn du den Umhang eines Wissenschaftlers anlegst, musst du all deine Politik und Argumente und Fraktionen und Seiten vergessen, von dem ablassen, woran dein Geist sich verzweifelt klammert und nur den Wunsch verspüren, die Antwort der Natur zu hören.“ Die schattenhafte Gestalt hielt inne. „Die meisten Menschen können es nicht. Deshalb ist es schwierig. Bist du sicher, du würdest nicht lieber nur etwas über das Gehirn lernen?“

„Und wenn ich dir sagte, ich würde lieber etwas über das Gehirn lernen,“ sagte Draco, nun mit Härte in der Stimme, „wirst du herum erzählen, dass ich Angst vor dem hatte, was ich erfahren würde.“

„Nein,“ sagte die schattenhafte Gestalt. „So etwas werde ich nicht tun.“

„Aber du könntest die gleichen Tests selbst durchführen und wenn du die falsche Antwort hättest, wäre ich nicht da um irgendetwas zu sagen, bevor du es jemandem zeigst.“ Dracos Stimme war noch immer hart.

„Ich würde dich noch immer vorher fragen, Draco,“ sagte die schattenhafte Gestalt leise.

Draco hielt inne. Das hatte er nicht erwartet, er hatte gedacht er sähe die Falle, doch… „\emph{Würdest} du?“

„Natürlich. Woher wüsste \emph{ich}, wen wir erpressen sollten oder was wir von ihm verlangen können? Draco, noch einmal, ich habe dir damit keine Falle gestellt. Nicht dir persönlich zumindest. Wenn deine politischen Ansichten anders wären, würde ich sagen, was wenn die Tests zeigten, dass Reinblüter stärker sind.“

„Wirklich.“

„\emph{Ja!} Das ist der Preis, den \emph{jeder} zahlen muss, um ein Wissenschaftler zu werden!“

Draco hob eine Hand. Er musste nachdenken.

Die schattenhafte, grün-erleuchtete Gestalt wartete.

Er musste allerdings nicht lange darüber nachdenken. Wenn man all die verwirrenden Teile beiseite ließ… dann plante Harry Potter mit etwas herum zu spielen, was eine gewaltige politische Bombe platzen lassen konnte und es wäre wahnsinnig, sich einfach abzuwenden und es ihn allein tun zu lassen. „Wir studieren Blut,“ sagte Draco.

„\emph{Exzellent,}“ sagte die Gestalt und lächelte. „Glückwunsch, dass du die Frage stellen möchtest.“

„Danke,“ sagte Draco und konnte die Ironie nicht ganz aus seiner Stimme verbannen.

„Hey, denkst du zum Mond zu fliegen war \emph{einfach?} Sei froh, dass man hierfür nur ab und an seine Meinung ändern muss und keinen Menschen opfern!“

„Ein Menschenopfer wäre \emph{viel} einfacher!“

Eine kleine Pause entstand, dann nickte die Gestalt. „Guter Punkt.“

„Sieh mal, Harry,“ sagte Draco ohne große Hoffnung, „ich dachte, die Idee wäre alles zu nehmen, was Muggel wissen, es mit dem zu kombinieren was Zauberer wissen und beide Welten zu meistern. Wäre es nicht viel einfacher nur all die Sachen zu lernen, die Muggel bereits herausgefunden \emph{haben}, wie das Mond-Zeugs und \emph{diese} Macht zu nutzen—“

„\emph{Nein,}“ sagte die Gestalt mit energischem Kopfschütteln, ließ dabei grüne Schatten um seine Nase und Augen wandern. Seine Stimme hatte sich sehr verdüstert. „Wenn du nicht die Kunst des Wissenschaftlers erlernen kannst, die Realität zu akzeptieren, dann \emph{darf} ich dir nicht sagen, was durch diese Akzeptanz entdeckt wurde. Es wäre als erzählte dir ein mächtiger Zauberer von jenen Toren, die nicht geöffnet und Siegeln, die nicht gebrochen werden dürfen, bevor du deine Intelligenz und Disziplin durch das Überleben der geringeren Gefahren unter Beweis gestellt hast.“

Ein Schauer lief Draco über den Rücken und er erzitterte unwillkürlich. Er wusste, es war sichtbar gewesen, selbst in dem gedämpften Licht. „In Ordnung,“ sagte Draco. „Ich verstehe.“ Vater hatte es ihm oft gesagt. Wenn ein mächtigerer Zauberer einem sagte, man sei nicht bereit, sollte man nicht weiter drängen, wenn man leben wollte.

Die Gestalt neigte den Kopf. „In der Tat. Doch es gibt noch etwas, das du verstehen solltest. Die ersten Wissenschaftler, da sie Muggel waren, hatten nicht eure Traditionen. Am Anfang war ihnen die Vorstellung gefährlichen Wissens einfach nicht vertraut und sie dachten, über alles was man wisse, solle frei gesprochen werden. Als ihre Suche gefährliches zutage brachte, berichteten sie ihren Politikern von Dingen, die hätten geheim bleiben sollen - schau nicht so, Draco, sie waren nicht einfach dämlich. Immerhin mussten sie schlau genug sein, das Geheimnis überhaupt erst zu entdecken. Doch sie waren Muggel, es war das erste mal, dass sie etwas \emph{wahrhaft} gefährliches entdeckt hatten und sie besaßen nicht \emph{von Beginn an} eine Tradition der Geheimhaltung. Es war ein Krieg im Gange und die Wissenschaftler auf einer Seite sorgten sich, dass wenn sie \emph{nicht} redeten, die Wissenschaftler des \emph{feindlichen} Landes es \emph{ihren} Politikern zuerst sagen würden…“ Die Stimme verklang bedeutungsvoll. „Sie haben nicht die Welt zerstört. Doch es war knapp. Und \emph{wir} werden diesen Fehler nicht wiederholen.“

„Richtig,“ sagte Draco, seine Stimme nun sehr fest. „Werden \emph{wir} nicht. Wir sind Zauberer und Wissenschaft zu erlernen macht uns nicht zu Muggeln.“

„Du sagst es,“ sagte die grün-erleuchtete Silhouette. „Wir werden unsere \emph{eigene} Wissenschaft etablieren, eine magische Wissenschaft und jene Wissenschaft wird von Beginn an schlauere Traditionen haben.“ Die Stimme wurde hart. „Das Wissen, das ich mit dir teile wird zusammen gelehrt mit der Disziplin, die Wahrheit zu akzeptieren, der Grad dieses Wissens wird an deinen Fortschritt in jener Disziplin gebunden sein und du wirst dieses Wissen mit niemandem teilen, der diese Disziplin nicht gemeistert hat. Akzeptierst du das?“

„Ja,“ sagte Draco. Was sollte er schon tun, nein sagen?

„Gut. Und was du selbst entdeckst, wirst du für dich behalten, es sei denn du denkst, dass andere Wissenschaftler für das Wissen bereit sind. Was wir untereinander teilen, werden wir der Welt nicht verraten, es sei denn wir stimmen überein, dass das Wissen für die Welt sicher ist. Und wie auch immer unsere eigenen politischen Ansichten und Loyalitäten beschaffen sind, wir \emph{alle} werden \emph{jeden} aus unserer Mitte bestrafen, der gefährliche Magie oder Waffen weitergibt, egal was für ein Krieg gerade vor sich geht. Von diesem Tage an wird dies die Tradition und das Gesetz der Wissenschaft unter Zauberern sein. Stimmen wir darin überein?“

„Ja,“ sagte Draco. Tatsächlich \emph{hörte} sich das langsam ziemlich attraktiv an. Die Todesser hatten versucht, die Macht zu ergreifen, indem sie furchterregender als alle anderen waren und sie hatten noch nicht wirklich gewonnen. Vielleicht war es Zeit für den Versuch, stattdessen durch Geheimnisse zu regieren. „Und unsere Gruppe bleibt so lange wie möglich im Verborgenen und jedes Mitglied wird unseren Regeln zustimmen müssen.“

„Natürlich. Definitiv.“

Eine kurze Pause entstand.

„Wir werden bessere Umhänge brauchen,“ sagte die schattenhafte Gestalt, „mit Kapuzen und so weiter—“

„Das habe ich \emph{auch gerade} gedacht,“ sagte Draco. Wir brauchen allerdings keine völlig neuen Umhänge, nur Mäntel mit Kapuzen zum Überstreifen. Ich habe eine Freundin in Slytherin, sie nimmt deine Maße -"

„Sag ihr aber nicht \emph{wofür—} “

„Ich bin nicht \emph{blöd!}“

„Und noch keine Masken, nicht wenn es nur wir beide sind—“ sagte die schattenhafte Gestalt.

„Stimmt! Doch später dann sollten wir eine Art spezielles Mal haben, dass all unsere Diener tragen, das Mal der Wissenschaft, wie eine Schlange, die den Mond frisst auf dem rechten Arm—“

„Das nennt man einen Doktortitel und könnte man unsere Leute damit nicht zu einfach identifizieren?“

„Häh?“

„Ich meine, was wenn jemand sowas sagt wie 'okay, jetzt krempeln mal alle den rechten Ärmel hoch' und unser Kerl so 'uups, sorry, sieht aus als wäre ich ein Spion'—“

„\emph{Vergiss dass ich was gesagt habe,}“ sagte Draco, dem plötzlich der Schweiß am ganzen Körper ausbrach. Er brauchte eine Ablenkung, \emph{schnell—} „Und wie nennen wir uns? Die Wissenschaftesser?“

„Nein,“ sagte die schattenhafte Gestalt langsam. „Das klingt nicht richtig…“

Draco fuhr sich mit dem Ärmel seines Umhangs über die Stirn, wischte kleine Perlen aus Flüssigkeit davon. Was hatte sich der Dunkle Lord nur \emph{gedacht?} Vater hatte gesagt, der Dunkle Lord war \emph{schlau!}

„Ich hab's!“ sagte die schattenhafte Gestalt plötzlich. „Du wirst noch nicht verstehen, aber vertrau mir, es passt.“

In jenem Moment hätte Draco auch 'Malfoy-Mampfer' akzeptiert, solange sie nur das Thema wechselten. „Was ist es?“

Und zwischen den staubigen Pulten in einem ungenutzten Klassenzimmer in den Verliesen von Hogwarts stehend, breitete die grün-erleuchtete Silhouette dramatisch die Arme aus und sagte, „Dieser Tag markiert den Aufstieg der… \emph{Verschwörung von Bayes.}“

\later

Eine leise Gestalt schleppte sich erschöpft durch die Gänge von Hogwarts in Richtung von Ravenclaw.

Harry war direkt von dem Treffen mit Draco aus zum Abendessen gegangen und gerade lang genug dort geblieben um ein paar schnelle Bissen herunter zu schlingen bevor er sich ins Bett aufmachte.

Es war noch nicht einmal 7~Uhr abends, doch für Harry bereits weit nach Schlafenszeit. Ihm war \emph{letzte} Nacht klar geworden, dass er den Zeitumkehrer am Samstag, bis der Buch-lese-Wettstreit bereits vorüber war, nicht würde benutzen können. Doch er konnte den Zeitumkehrer noch immer \emph{Freitag} nachts benutzen und auf diese Weise Zeit gewinnen. Also hatte sich Harry gezwungen am Freitag bis 9~Uhr abends wach zu bleiben, wenn die Schutzhülle sich öffnete und dann die verbleibenden vier Stunden auf dem Zeitumkehrer genutzt, um bis 5~Uhr abends zurück zu gehen und vor Müdigkeit zusammenzubrechen. Er war Samstag um 2~Uhr morgens aufgewacht, genau wie geplant und hatte die nächsten zwölf Stunden hindurch gelesen… und es hatte trotzdem nicht gereicht. Und nun würde Harry die nächsten paar Tage ziemlich früh ins Bett gehen, bis sein Schlafrythmus wieder aufgeholt hatte.

Das Porträt an der Tür stellte ihm irgendein dummes Rätsel für Elfjährige, das er löste, ohne dass die Worte überhaupt sein Bewusstsein passierten und dann stolperte Harry die Treppenstufen zu seinem Schlafsaal hinauf, schlüpfte in seinen Pyjama und ließ sich aufs Bett fallen.

Und stellte fest, dass sein Kissen sich ziemlich klumpig anfühlte.

Harry ächzte. Er setzte sich widerstrebend auf, drehte sich im Bett und hob sein Kissen an.

Dieses enthüllte eine Notiz, zwei goldene Galleonen und ein Buch mit dem Titel \emph{Okklumentik: Die Verborgene Chunst.}**

Harry nahm die Notiz und las:

\emph{Meine Güte, hast du dir einen Ärger eingehandelt und auch noch so schnell. Dein Vater konnte dir nicht das Wasser reichen.}

\emph{Du hast dir einen mächtigen Feind gemacht. Snape verfügt über Treue, Bewunderung und Furcht des gesamten Hauses Slytherin. Du kannst jetzt niemandem in diesem Haus mehr trauen, ob sie dir nun freundlich oder furchterregend scheinen.}

\emph{Von nun an darfst du Snapes Blick nicht begegnen. Er ist ein Legilimentor und kann anderenfalls deinen Geist lesen. Ich habe ein Buch beigefügt, welches dir helfen mag, dich zu verteidigen, obwohl die Fortschritte, die man ohne einen Privatlehrer machen kann, ihre Grenzen haben. Dennoch darfst du hoffen, ein Eindringen zumindest zu bemerken.}

\emph{Damit du etwas extra Zeit findest, in der du Okklumentik studieren kannst, habe ich 2 Galleonen beigelegt, was der Preis für die Antwortbögen und Hausaufgaben der ersten Klasse in Geschichte der Magie ist (da Professor Binns die selben Tests und Aufgaben ausgibt, seit dem Jahr in dem er starb). Deine neuen Freunde, die Weasley-Zwillinge sollten dir eine Kopie verkaufen können. Es braucht wohl nicht erwähnt zu werden, dass du dich damit in deinem Besitz nicht erwischen lassen darfst.}

\emph{Über Professor Quirrell ist mir nur wenig bekannt. Er ist ein Slytherin und ein Verteidigungs-Professor und das sind zwei Punkte, die gegen ihn sprechen. Überdenke sorgfältig jeden Rat, den er dir gibt und erzähle ihm nichts, das du nicht bekannt werden lassen willst.}

\emph{Dumbledore gibt nur vor, verrückt zu sein. Er ist extrem intelligent und wenn du weiterhin in Schränke steigst und dann verschwindest, wird er sicherlich darauf schließen, dass du einen Unsichtbarkeitsumhang besitzt, wenn er es nicht bereits getan hat. Meide ihn, soweit es möglich ist, verstecke den Unsichtbarkeitsumhang an einem sicheren Ort (NICHT dein Beutel), wenn du ihn nicht meiden kannst und gehe in seiner Gegenwart mit äußerster Vorsicht vor.}

\emph{Bitte sei in Zukunft vorsichtiger, Harry Potter.}

\emph{- Santa Claus}

Harry starrte die Notiz an.

Es schien \emph{in der Tat} ein ziemlich guter Rat zu sein. Natürlich würde Harry im Geschichtsunterricht nicht betrügen, auch wenn man ihm einen toten Affen als Professor vorsetzte. Doch Severus Legilimentik… wer immer die Notiz geschickt hatte, wusste eine Menge wichtiger, geheimer Dinge und war willens, sie Harry mitzuteilen. Die Notiz warnte ihn noch immer davor, Dumbledore könne den Umhang stehlen, doch an diesem Punkt hatte Harry ehrlich keine Ahnung ob das ein schlechtes Zeichen war, es könnte auch nur ein verständlicher Irrtum sein.

Irgendeine Art von Intrige schien in Hogwarts vor sich zu gehen. Vielleicht konnte Harry, wenn er die Geschichten von Dumbledore und dem Notizen-Schreiber \emph{miteinander abglich,} ein \emph{kombiniertes} Bild ermitteln, das Klarheit brachte? Etwa, wenn sie \emph{beide} bei etwas übereinstimmten, dann…

… was auch immer…

Harry stopfte alles in seinen Beutel, drehte den Ruhezauber auf, zog sich die Decke über den Kopf und fiel wie tot in den Schlaf.

\later

Es war Sonntag morgens und Harry aß Pfannkuchen in der Großen Halle, mit schnellen kleinen Bissen, sah nervös alle paar Sekunden auf seine Armbanduhr.

Es war 8:02 am Morgen und in exakt zwei Stunden und einer Minute wäre es \emph{genau eine Woche} her, seit er die Weasleys getroffen hatte und auf das Gleis Neundreiviertel hinüber gewechselt war.

Und ihm war der Gedanke gekommen… Harry wusste nicht, ob es zulässig war in dieser Weise über das Universum zu denken, er wusste eigentlich gar nichts mehr, aber es \emph{schien möglich…}

Dass…

\emph{Ihm während der letzten Woche noch nicht genug interessante Dinge widerfahren waren.}

Als er mit dem Frühstück fertig war, wollte Harry direkt hinauf auf sein Zimmer gehen, sich in der Keller-Etage seines Koffers verstecken und bis 10:03~Uhr mit niemandem reden.

Und da sah Harry die Weasley-Zwillinge auf sich zukommen. Einer von ihnen hielt etwas hinter seinen Rücken versteckt.

Er sollte schreiend weg rennen.

Er sollte schreiend weg rennen.

Was immer das war… es konnte durchaus…

… das \emph{große Finale} sein…

Er sollte wirklich einfach schreiend weg rennen.

Das Universum würde ihn \emph{so oder} \emph{so} erwischen, dachte Harry schicksalsergeben und zerschnitt weiter seinen Pfannkuchen mit Messer und Gabel. Er konnte die Energie nicht aufbringen. Das war die traurige Wahrheit. Harry wusste jetzt, wie sich Leute fühlten, wenn sie es leid waren wegzulaufen, ihrer Bestimmung zu entfliehen und sich einfach zu Boden fallen und von den entsetzlichen klauen- und tentakel-bewehrten Dämonen in den dunkelsten Abgrund ziehen ließen, wo sie ihr unaussprechliches Schicksal erwartete.

Die Weasley-Zwillinge kamen näher.

Und immer näher.

Harry biss noch einmal von seinem Pfannkuchen ab.

Die Weasley-Zwillinge erreichten ihn mit breitem Grinsen.

„Hallo, Fred,“ sagte Harry dumpf. Einer der Zwillinge nickte. „Hallo, George.“ Der andere Zwilling nickte.

„Du klingst müde,“ sagte George.

„Kopf hoch,“ sagte Fred.

„Sieh mal, was \emph{wir} für dich haben!“

Und George zog hinter Freds Rücken—

Einen Kuchen mit zwölf flammenden Kerzen hervor.

Eine Pause entstand, als der Ravenclaw-Tisch sie anstarrte.

„Das stimmt nicht,“ sagte jemand. „Harry Potter wurde am einunddreißigsten Jul—“

„\emph{ER WIRD KOMMEN,}“ ertönte eine gewaltige hohle Stimme, die durch alle Gespräche fuhr wie ein Schwert aus Eis. „\emph{ER ZERSTÖRT DIE—} “

Dumbledore war von seinem Thron aufgesprungen und quer über den Lehrertisch gerannt und ergriff nun die Frau, die jene entsetzlichen Worte sprach, Fawkes war mit einem Aufleuchten erschienen und alle drei verschwanden in einem feurigen Spalt.

Eine schockierte Pause entstand…

… gefolgt von Köpfedrehen in Richtung von Harry Potter.

„Ich war's nicht,“ sagte Harry mit müder Stimme.

„Das war eine \emph{Prophezeiung!}“ zischte jemand am Tisch. „Und ich wette, es geht um \emph{dich!}“

Harry seufzte.

Er erhob sich von seinem Platz, hob die Stimme und übertönte laut die Unterhaltungen, die gerade einsetzten, „\emph{Es geht nicht um mich!} \emph{Offensichtlich! Ich komme nicht her, ich bin bereits hier!}“

Harry setzte sich wieder hin.

Die Leute, die zu ihm hin gesehen hatten, drehten sich wieder weg.

Jemand anders am Tisch sagte, „Um wen ging es \emph{dann?}“

Und mit einem dumpfen, bleiernen Gefühl wurde Harry klar, wer \emph{nicht} bereits in Hogwarts war.

Man mochte es eine vage Vermutung nennen, doch Harry hatte das Gefühl, der untote Dunkle Lord würde sich demnächst blicken lassen.

Die Gespräche um ihn herum gingen weiter.

„Nicht zu vergessen, die \emph{was} zerstören?“

„Ich glaube, ich hörte Trelawney etwas sagen, das mit 'S' anfing, bevor der Schulleiter sie geschnappt hat.“

„Wie… Seele? Sonne?“

„Wenn jemand die Sonne zerstören wird, sitzen wir \emph{wirklich} in der Tinte!“

Das schien Harry eher unwahrscheinlich, es sei denn es gab ein paar beängstigende Dinge in der Welt, die von David Criswells Ideen über Star Lifting gehört hatten.***

„Also,“ sagte Harry in müdem Tonfall, „das passiert jeden Sonntag beim Frühstück, oder?“

„Nein,“ sagte ein Schüler, der in seinem siebten Jahr sein mochte mit grimmigem Stirnrunzeln. „Tut es nicht.“

Harry zuckte mit den Schultern. „Wie auch immer. Möchte jemand Geburtstagskuchen?“

„Aber du \emph{hast nicht} Geburtstag!“ sagte der selbe Schüler, der schon beim letzten mal Einwände hatte.

Das war für Fred und George natürlich das Stichwort, um in Gelächter auszubrechen.

Selbst Harry brachte ein müdes Lächeln zustande.

Als ihm das erste Stück gereicht wurde, sagte Harry, „Ich hatte eine \emph{echt lange Woche.}“

\later

Und nun saß Harry in der Keller-Etage seines Koffers, zugeschoben und verschlossen, damit niemand herein konnte, ein Laken über den Kopf gezogen und wartete darauf, dass die Woche vorbei sein würde.

10:01.

10:02.

10:03, aber nur um sicher zu gehen…

10:04 und die erste Woche war um.

Harry entließ einen Seufzer der Erleichterung und zog zögerlich das Laken von seinem Kopf.

Einige Augenblicke später war er hinaus getreten in die klare, sonnendurchflutete Luft seines Schlafsaals.

Kurz darauf war er im Ravenclaw-Gemeinschaftsraum. Ein paar Leute sahen ihn an, doch niemand sagte etwas oder versuchte mit ihm zu reden.

Harry fand ein nettes breites Schreibpult, zog sich einen bequemen Stuhl heran und setzte sich. Aus seinem Beutel zog er ein Blatt Papier und einen Bleistift.

Mum und Dad hatten Harry deutlich zu verstehen gegeben, dass obwohl sie Verständnis hatten für seine Begeisterung, von zu Hause und seinen Eltern wegzukommen, er dennoch \emph{keinesfalls} \emph{versäumen dürfe, ihnen jede Woche} zu schreiben, nur damit sie wussten, dass er am Leben, unverletzt und nicht im Gefängnis war.

Harry starrte auf das leere Blatt Papier hinab. \emph{Mal sehen…}

Nachdem er seine Eltern am Bahnhof verlassen hatte, hatte er…

… Bekanntschaft mit einem Jungen geschlossen, der von Darth Vader erzogen worden war, sich mit den drei berüchtigtsten Tunichtguten von Hogwarts angefreundet, Hermine getroffen, dann war da noch der Vorfall mit dem Sprechenden Hut… Am Montag hatte er eine Zeitmaschine bekommen, um seine Schlafstörung zu behandeln, den legendären Unsichtbarkeitsumhang von einem unbekannten Wohltäter erhalten, sieben Hufflepuffs gerettet, indem er fünf ältere Slytherins in die Knie gezwungen hatte, von denen einer gedroht hatte, ihm seinen Finger zu brechen, erkannt dass er eine mysteriöse dunkle Seite besaß, im Zauberkunst-Unterricht den \emph{Frigideiro}-Spruch gelernt und seine Rivalität mit Hermine begonnen… Am Dienstag hatte er mit Astronomie bei Professor Aurora Sinistra begonnen, die nett war und mit Geschichte der Magie, unterrichtet von einem Geist, der exorziert und durch einen Kassetten-Rekorder ersetzt werden sollte… Mittwoch war er zum Gefährlichsten Schüler im Klassenraum erklärt worden… Donnerstag, gar nicht erst nachdenken über Donnerstag… Freitag, der Vorfall im Zaubertränke-Unterricht, gefolgt von seiner Erpressung des Schulleiters, gefolgt davon, dass der Verteidigungs-Professor ihn im Unterricht aufmischen ließ, gefolgt davon, dass der Verteidigungs-Professor sich als das umwerfendste menschliche Wesen herausstellte, das noch immer auf Erden wandelte… Samstag hatte er eine Wette verloren und war auf seinem ersten Date gewesen und hatte angefangen Draco zur Umkehr zu bewegen… und heute morgen dann mochte Professor Trelawneys ungehörte Prophezeiung darauf hindeuten oder auch nicht, dass ein unsterblicher Dunkler Lord im Begriff war Hogwarts anzugreifen.

Harry ordnete im Geiste sein Material und begann zu schreiben.

\emph{Liebe Mum und Dad:}

\emph{Hogwarts macht eine Menge Spaß. Ich habe im Zauberkunst-Unterricht gelernt, wie man das Zweite Gesetz der Thermodynamik verletzt und ein Mädchen namens Hermine Granger getroffen, das schneller liest als ich.}

\emph{Ich belasse es lieber dabei.}

\emph{Euer euch liebender Sohn,}

\emph{Harry James Potter-Evans-Verres.}

* engl.: \emph{A History of Magic}

** engl.: \emph{Occlumency: The Hidden Arte;} es schien mir als solle der Titel mit dem zusätzlichen „e“ bewusst altertümlich klingen, daher habe ich hierfür eine laut Wiktionary althochdeutsche Form des Wortes \emph{Kunst} verwendet (vielleicht schon etwas zu alt).

*** Unter \emph{star lifting} versteht man verschiedene Technologien, die eine fortschrittliche raumfahrende Zivilisation nutzen könnte, um einem Stern bedeutende Teile seiner Masse zu entziehen, etwa als Quelle von Rohmaterialien.

