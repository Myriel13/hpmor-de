

\hypertarget{bonus-akten-1–2–3}{% \section{11. Bonus-Akten 1, 2, 3}\label{bonus-akten-1–2–3}}

\textbf{Kapitel 11: Bonus-Akten 1, 2, 3

}

Heil dem Dunklen Lord Rowling.

Dieses Kapitel besteht aus nicht-kanonischen Extras.

\later

\uline{BONUS-AKTEN \#1: 72 Stunden zum Sieg}

(Oder „Was geschieht, wenn man Harry verändert, aber alle anderen Charaktere gleich bleiben“)

Dumbledore spähte über seinen Schreibtisch, mit gutmütig funkelnden Augen, zum jungen Harry hinüber. Der Junge war zu ihm gekommen mit fürchterlich ernstem Ausdruck auf seinem jungen Gesicht - Dumbledore hoffte, was immer der Grund sei, es wäre keine allzu ernste Angelegenheit. Harry war bei weitem zu jung, als dass die Prüfungen seines Lebens jetzt schon beginnen mochten. „Worüber wünschtest du mit mir zu sprechen, Harry?“

Harry James Potter-Evans-Verres beugte sich in seinem Stuhl nach vorn, ein grimmiges Lächeln auf seinem Gesicht. „Schulleiter, während des Festessens nach der Auswahlzeremonie spürte ich einen stechenden Schmerz in meiner Narbe. Angesichts dessen, unter welchen Umständen ich mir diese Narbe zugezogen habe, schien das nichts zu sein, was man einfach ignorieren sollte. Ich dachte zuerst, es sei wegen Professor Snape, doch ich bin der Bacon'schen experimentellen Methodik gefolgt, was bedeutet, die Bedingungen für sowohl das Vorhandensein als auch die Abwesenheit eines Phänomens zu ermitteln und ich konnte bestimmen, dass meine Narbe schmerzt, wenn und nur wenn ich die Rückseite von Professor Quirrells Kopf betrachte, was immer da auch unter seinem Turban ist. Obwohl es durchaus auch etwas harmloseres sein \emph{könnte}, denke ich, wir sollten vorsichtshalber das Schlimmste annehmen, nämlich dass es Sie-wissen-schon-wer ist - Moment, jetzt schauen Sie doch nicht gleich so entsetzt, eigentlich ist das doch eine einmalige Gelegenheit—“

\later

\uline{BONUS-AKTEN \#2: \emph{Dunkle Lords machen mir keine Angst}}

Dies war die ursprüngliche Version von Kapitel 9. Sie wurde ersetzt, da sich - trotzdem sie vielen Lesern gefallen hat - herausstellte, dass viele andere Leser auf Lieder in Fanfics \emph{massiv} allergisch reagieren, aus Gründen die nicht groß erläutert zu werden brauchen. Ich wollte die Leser nicht bereits vor dem zehnten Kapitel in die Flucht schlagen.

Lee Jordan ist (im Kanon) der Komplize von Fred und George, was das Streiche spielen angeht. „Lee Jordan“ klang für mich wie ein Muggelgeborenen-Name, was heißen würde, er könnte Fred und George eine Melodie beibringen, die Harry erkennen würde. Das war für einige Leser nicht so offensichtlich, wie für euren Autor.

\later

Draco ging nach Slytherin und Harry stieß einen kleinen Seufzer der Erleichterung aus. Es \emph{schien} eine sichere Sache zu sein, doch man konnte nie wissen, welch winziges Ereignis den eigenen Masterplan vereiteln mochte.

Sie näherten sich jetzt den Ps…

Und drüben am Gryffindor-Tisch war eine geflüsterte Unterhaltung zu vernehmen.

„\emph{Was, wenn's ihm nicht gefällt?}“

„\emph{Er hat kein Recht, es nicht zu mögen—} “

„\emph{- nicht nach dem Streich, den er} \emph{diesem} \emph{Jungen} \emph{gespielt hat—} “

„\emph{- Neville Longbottom war sein Name—} “

„\emph{- er ist ein so gerechtes Ziel, wie es nur geht.}“

„\emph{Alles klar. Passt nur auf, dass ihr euren} \emph{Part} \emph{nicht vergesst.}“

„\emph{Wir sind es oft genug durchgegangen—} “

„\emph{- in den letzten drei Stunden.}“

Und Minerva McGonagall blickte am Sprecherpodium des Lehrertisches zum nächsten Namen auf ihrer Liste hinab. \emph{Bitte, lass ihn kein Gryffindor sein bitte, lass ihn kein Gryffindor sein OH BITTE, lass ihn kein Gryffindor sein…} Sie atmete tief durch und rief:

„Potter, Harry!“

Plötzlich wurde es still in der Halle, als alle geflüsterten Gespräche endeten.

Eine Stille, durchbrochen von einem schrecklichen summenden Geräusch, das sich modulierte und veränderte in scheußlicher Verhöhnung musikalischer Melodie.

Minervas Kopf fuhr, schockiert, herum und ermittelte den Ursprung des summenden Geräusches in Richtung des Gryffindor-Tischs, wo Sie \emph{auf der Tischplatte standen} und in irgendwelche winzigen Geräte bliesen, die sie an Ihre Lippen hielten. Ihre Hand fuhr hinab zu ihrem Zauberstab, um die Bande mit einem \emph{Silencio} zum Schweigen zu bringen, doch ein anderes Geräusch ließ sie innehalten.

Dumbledore kicherte.

Minerva wandte den Blick wieder Harry Potter zu, der gerade erst aus der Reihe hervorgetreten war, als er ins Stolpern geriet und zum Stehen kam.

Dann setzte sich der Junge wieder in Bewegung, vollführte mit den Beinen seltsame ausladende Bewegungen, wogte mit den Armen vor und zurück und schnippte mit den Fingern im Takt Ihrer Musik.

\emph{Zur Melodie von „Ghostbusters“*}

\emph{(Gespielt auf dem Kazoo von Fred und George Weasley

und gesungen von Lee Jordan.)}

\emph{.}

\emph{Ein Dunkler Lord euch plagt,

Dann seid nicht verzagt.

Nach wem ruft ihr dann?}

HARRY POTTER!" rief Lee Jordan und die Weasley-Zwillinge bildeten einen triumphierenden Chor.

\emph{Mit dem Tödlichen Fluch,

Kriegt ihr von ihm Besuch.

Nach wem ruft ihr dann?}

„HARRY POTTER!“ Dieses mal riefen sehr viel mehr Stimmen.

Die Weasley-Gräuel verfielen in ein ausgedehntes Heulen, nunmehr begleitet von einigen älteren Muggelgeborenen, die ihre eigenen kleinen Gerätschaften produziert hat, ohne Zweifel aus dem Tafelsilber der Schule transfiguriert. Als ihre Musik ihren Tiefpunkt erreichte rief Harry Potter:

\emph{\emph{Dunkle Lords machen mir keine Angst!}}

Daraufhin ertönte Jubel, besonders vom Gryffindor-Tisch und immer mehr Schüler stellten ihre eigenen unmusikalischen Instrumente her. Ihr scheußliches Summen verdoppelte sich noch einmal und schwoll zu einem weiteren entsetzlichen Crescendo an:

\emph{\emph{Dunkle Lords machen mir keine Angst!}}

Minerva warf einen Seitenblick zu beiden Seiten des Lehrertisches, fürchtete sich davor, doch ahnte nur allzu gut, was sie dort sehen würde.

Trelawney fächelte sich hektisch Luft zu, Flitwick schaute neugierig zu, Hagrid klatschte im Takt der Musik, Sprout blickte streng und Quirrell betrachtete den Jungen mit zynischer Belustigung. Direkt zu ihrer Linken summte Dumbledore mit und zu ihrer Rechten umklammerte Snape seinen leeren Weinkelch mit weiß hervortretenden Knöcheln so hart, dass das massive Silber sich langsam verformte.

\emph{Ein Dementoren-Kuss?

Bereitet euch Verdruss?

Nach wem ruft ihr dann?

HARRY POTTER!}

\emph{Ein Irrwicht im Haus?

'Ne alte Fledermaus?

Nach wem ruft ihr dann?

HARRY POTTER!}

Minervas Lippen wurden zu einer weißen Linie. Mit Ihnen würde sie noch ein Wörtchen zu reden haben, über diesen letzten Vers, wenn Sie dachten, sie sei machtlos, weil es der erste Schultag war und Gryffindor noch keine Punkte zu verlieren hatte. Wenn Sie sich um Nachsitzen nicht scherten, würde ihr schon etwas anderes einfallen.

Dann, mit einem Anflug plötzlichen Entsetzens, blickte sie in Snapes Richtung, ihm musste doch \emph{sicherlich} klar sein, dass der Potter-Junge keine Ahnung haben konnte, von wem dort die Rede war—

Snapes Gesicht war über Zorn weit hinaus und zeigte jetzt eine Art zufriedener Gleichgültigkeit. Ein schwaches Lächeln umspielte seine Lippen. Er blickte in Richtung von Harry Potter, nicht zum Gryffindor-Tisch und in seinen Händen hielt er die zerknüllten Überreste dessen, was einmal ein Weinkelch gewesen war…

Und Harry schritt voran, fegte mit Armen und Beinen durch die Bewegungen des Ghostbusters-Tanzes, mit einem Lächeln auf dem Gesicht. Es war ein toller Auftritt, er hatte ihn völlig überrascht. Das Mindeste, was er tun konnte, war mitzuspielen und nicht alles zu ruinieren.

Alle jubelten ihm zu. Er fühlte sich innerlich ganz warm und gleichzeitig irgendwie furchtbar.

Sie bejubelten ihn für einen Job, den er erledigt hatte, als er ein Jahr alt war. Einen Job, den er nicht wirklich zu Ende gebracht hatte. Irgendwo, irgendwie war der Dunkle Lord noch immer am Leben. Hätten sie auch dann noch so sehr gejubelt, wenn sie das gewusst hätten?

Doch die Macht des Dunklen Lords \emph{war}schon einmal gebrochen worden.

Und Harry würde sie erneut beschützen. Wenn es tatsächlich eine Prophezeiung gab und es das war, was sie besagte. Nun, eigentlich egal, was irgendeine verdammte Prophezeiung sagte.

All diese Menschen, die an ihn glaubten und ihm zujubelten - Harry könnte es nicht ertragen, das zu enttäuschen. Aufzublitzen und zu vergehen, wie so viele andere Wunderkinder. Eine Enttäuschung zu sein. Seinem Ruf als ein Symbol des Lichts nicht gerecht zu werden, egal wie er dazu gekommen war. Er würde absolut, definitiv, wie lange es auch dauerte und wenn es ihn umbrachte, ihren Erwartungen gerecht werden. Und dann weitermachen und diese Erwartungen noch übertreffen, damit die Leute sich, zurückblickend, fragten, wieso sie einst so wenig von ihm erwartet hatten.

Und so rief er die Lüge hinaus, die er erfunden hatte, weil sie so gut passte und das Lied danach verlangte:

\emph{Dunkle Lords machen mir keine Angst!

Dunkle Lords machen mir keine Angst!}

Harry nahm die letzten Schritte zum Sprechenden Hut, als die Musik zum Ende kam. Er verbeugte sich in Richtung des Gryffindor-Tisches, dann wandte er sich um und richtete eine Verbeugung an die andere Seite der Halle und wartete, bis der Applaus und das Kichern erstarben…

\later

\uline{BONUS-AKTEN \#3: Alternative Enden für 'Selbstbewusstsein'}

Das Angebot, jedem die gesamte Handlung zu verraten, der errät, was 'noch nie zuvor passiert ist' hat zu einer \emph{Menge} interessanter Versuche angespornt. Das erste Extra unten stammt direkt aus meiner persönlichen Lieblingsantwort von Meteoricshipyards. Das zweite basiert auf Kazumas Vorschlag, was „noch nie zuvor passiert ist“, das dritte auf einer Kombination von yoyoente und dougal74, das vierte auf wolf550es Review von Kapitel 10. Dasjenige, das mit 'K' anfängt und das direkt darüber sind von DarkHeart81. Die anderen sind von mir. Jeder, der eine meiner Ideen aufgreifen und etwas daraus machen will, besonders mit der letzten, darf das gerne tun. Und bevor ich 100 empörte Beschwerden erhalte, ja, ich bin mir sehr wohl bewusst, dass das gesetzgebende Organ des Vereinigten Königreichs das House of Commons im Parlament ist.**

\later

… In seinem Hinterkopf fragte er sich, ob der Sprechende Hut wirklich \emph{bei Bewusstsein} war in dem Sinn, sich seines eigenen Bewusstseins bewusst zu sein und wenn ja, ob er zufrieden damit war, nur einmal im Jahr mit Elfjährigen zu sprechen. Sein Lied deutete darauf hin: \emph{Oh, ich bin der Sprechende Hut und mir geht's gut,} \emph{nur} \emph{einmal im Jahr} \emph{packt mich} \emph{die Arbeitswut…}

Als es im Raum einmal mehr still wurde, setzte Harry sich auf den Stuhl und setzte sich \emph{vorsichtig} das 800 Jahre alte telepathische Artefakt vergessener Magie auf den Kopf.

Dabei dachte er, so stark er nur konnte: \emph{Sortier' mich noch nicht ein! Ich habe Fragen, die ich dir stellen muss! Wurden mir je die Erinnerungen gelöscht? Hast du den Dunklen Lord ausgewählt, als er ein Kind war und kannst du mir etwas über seine Schwächen verraten? Kannst du mir sagen, warum ich den Bruder des Zauberstabs des Dunklen Lords bekommen habe? Ist der Geist des Dunklen Lords an meine Narbe gebunden und ist das der Grund, warum ich manchmal so zornig werde? Das sind die wichtigen Fragen, aber wenn du noch einen Moment hast, kannst du mir irgendwas darüber verraten, wie man die verlorenen magischen Künste wiederentdecken kann, die dich erschaffen haben?}

Und der Sprechende Hut antwortete, „\emph{Nein. Ja. Nein. Nein. Ja und nein, stell beim nächsten mal keine Doppelfragen. Nein.}“ und laut rief er, „RAVENCLAW!“

\later

„\emph{Oh je. Das ist noch nie zuvor passiert…}“

\emph{Was?}

„\emph{Ich bin allergisch auf dein Haarshampoo—} “

Und dann nieste der Sprechende Hut mit einem mächtigen „HA-TSCHUU!“, dass in der gesamten Großen Halle widerhallte.

„Nun denn!“ rief Dumbledore fröhlich. „Es scheint, Harry Potter wurde in das neue Haus Hatschuu einsortiert! McGonagall, Sie können Hauslehrerin des Hauses Hatschuu werden. Sie beeilen sich besser mit den Vorbereitungen für Lehrplan und Unterricht von Hatschuu, morgen ist der erste Tag!“

„Aber, aber, aber,“ stammelte McGonagall, ihr Geist komplett aus der Fassung geraten, „wer wird dann Hauslehrer von Gryffindor?“ Das war alles, was ihr einfiel, sie \emph{musste} das irgendwie verhindern…

Dumbledore legte nachdenklich einen Finger an seine Wange. „Snape.“

In Snapes protestierendem Schrei ging der von McGonagall beinahe unter, „Wer wird dann Hauslehrer von \emph{Slytherin?}“

„Hagrid.“

\later

\emph{Sortier' mich noch nicht ein! Ich habe Fragen, die ich dir stellen muss! Wurden mir je die Erinnerungen gelöscht? Hast du den Dunklen Lord ausgewählt, als er ein Kind war und kannst du mir etwas über seine Schwächen verraten? Kannst du mir sagen, warum ich den Bruder des Zauberstabs des Dunklen Lords bekommen habe? Ist der Geist des Dunklen Lords an meine Narbe gebunden und ist das der Grund, warum ich manchmal so zornig werde? Das sind die wichtigen Fragen, aber wenn du noch einen Moment hast, kannst du mir irgendwas darüber verraten, wie man die verlorenen magischen Künste wiederentdecken kann, die dich erschaffen haben?}

Es gab eine kurze Pause.

\emph{Hallo? Muss ich die Fragen wiederholen?}

Der Sprechende Hut schrie, ein abscheulich schrilles Geräusch, dass in der Großen Halle widerhallte und die meisten Schüler dazu veranlasste, sich die Hände über die Ohren zu schlagen. Mit einem verzweifelten Jaulen sprang er von Harry Potters Kopf und machte einen Satz über den Boden, schob sich mit der Krempe voran und schaffte es auf halbe Strecke bis zum Lehrertisch bevor er explodierte.

\later

„SLYTHERIN!“

Als er den Ausdruck des Entsetzens auf Harry Potters Gesicht sah, dachte Fred Weasley schneller als je zuvor in seinem Leben. In einer flüssigen Bewegung riss er seinen Zauberstab hervor, flüsterte „\emph{Silencio!}“ und dann „\emph{Verändermeinestimmio!}“ und schließlich „\emph{Ventriliquo!}“***

„War nur'n Scherz!“ sagte Fred Weasley. „GRYFFINDOR!“

\later

„\emph{Oh je. Das ist noch nie zuvor passiert…}“

\emph{Was?}

„\emph{Für gewöhnlich würde ich bei solchen Fragen auf den Schulleiter verweisen, der mich} \emph{wiederum selbst} \emph{fragen könnte, wenn er es wünschte. Aber einige der Informationen, nach denen du fragst, gehen nicht nur über deine} \emph{Nutzerrechte} \emph{hinaus, sondern auch über die des Schulleiters.}“

\emph{Wie kann ich meine Nutzerrechte} \emph{erweitern?}

„\emph{Ich fürchte, deine derzeitigen Nutzerrechte erlauben es mir nicht, diese Frage zu beantworten.}“

\emph{Welche Möglichkeiten} stehen\emph{denn} \emph{mit meinen} \emph{Nutzerrechten zur Verfügung?}

Danach dauerte es nicht lange—

„ROOT!“

\later

„\emph{Oh je. Das ist noch nie zuvor passiert…}“

\emph{Was?}

„\emph{Ich habe früher schon Schülerinnen berichten müssen, dass sie Mütter waren - es würde dir das Herz brechen, zu erfahren, was ich in ihrem Geist gesehen habe - doch das ist das erste mal, dass ich jemandem sagen musste, dass er Vater ist.}“

\emph{WAS?}

„\emph{Draco Malfoy trägt dein Baby aus.}“

\emph{WAAAAAAAS?}

„\emph{Ich wiederhole: Draco Malfoy trägt dein Baby aus.}“

\emph{Aber wir sind erst elf—}

„\emph{Eigentlich ist Draco insgeheim dreizehn Jahre alt.}“

\emph{A-a-aber Männer können nicht schwanger werden—}

„\emph{Und ein Mädchen unter den Klamotten.}“

\emph{ABER WIR HATTEN NIEMALS SEX, DU IDIOT!}

„\emph{SIE HAT DIR NACH DER VERGEWALTIGUNG DAS GEDÄCHTNIS GELÖSCHT, SCHWACHKOPF!}“

Harry Potter fiel in Ohnmacht. Sein bewusstloser Körper fiel mit einem dumpfen Klatschen vom Stuhl.

„RAVENCLAW!“ verkündete der Hut, der auf seinem Kopf lag. Das war sogar noch lustiger gewesen als seine erste Idee.

\later

„ELF!“

Häh? Harry erinnerte sich, dass Draco ein 'Haus Elf' erwähnt hatte, aber was war das genau?

Dem bestürzten Ausdruck nach zu urteilen, der sich langsam auf den Gesichtern um ihn herum abzeichnete, jedenfalls nichts gutes—

\later

„BLOCK!“****

\later

„REPRÄSENTANTEN!“

\later

„\emph{Oh je. Das ist noch nie zuvor passiert…}“

\emph{Was?}

„\emph{Ich habe noch nie jemanden einsortiert, der die Wiedergeburt} \emph{war} \emph{von Godric Gryffindor UND Salazar Slytherin UND Naruto.}“

\later

„ATREIDES!“*****

\later

„Wieder reingelegt! HUFFLEPUFF! SLYTHERIN! HUFFLEPUFF!“

\later

„ERDBEER-EINTOPF!“******

\later

„KHAAANNNN!“*******

\later

Am Lehrertisch lächelte Dumbledore weiterhin wohlwollend; leise metallene Geräusche waren gelegentlich aus Richtung von Snape zu vernehmen, während er müßig die verdrehten Überreste dessen zusammenpresste, was einmal ein schwerer silberner Weinkelch gewesen war und Minerva umklammerte das Podium mit weiß hervortretenden Knöcheln, in dem Wissen, dass Harry Potters ansteckendes Chaos jetzt selbst den Sprechenden Hut erfasst hatte.

Szenario um Szenario spielte sich in Minervas Kopf ab, jedes schlimmer als das vorherige. Der Hut würde entscheiden, Harry sei zu ausgeglichen, um in ein Haus sortiert zu werden und gehöre in alle von ihnen. Der Hut würde verkünden, dass Harrys Geist zu seltsam sei, um einsortiert zu werden. Der Hut würde verlangen, dass Harry von Hogwarts ausgeschlossen würde. Der Hut war ins Koma gefallen. Der Hut würde darauf bestehen, dass ein ganz neues Haus der Verdammnis geschaffen wurde, nur um Harry Potter zu beherbergen und \emph{Dumbledore würde sie dazu zwingen…}

Minerva dachte daran, was Harry ihr bei ihrem desaströsen Ausflug in die Winkelgasse erzählt hatte, über den… Planungsfehlschluss, war es gewesen, glaubte sie… und wie die Leute üblicherweise zu optimistisch waren, selbst wenn sie glaubten, pessimistisch zu sein. Es war die Art von Information, die einem durch den Verstand jagte, sich darin einnistete und Alpträume hervorbrachte…

Doch was war das \emph{Schlimmste}, was passieren mochte?

Nun… im schlimmsten Fall würde der Hut Harry einem völlig neuen Haus zuweisen. Dumbledore würde darauf bestehen, dass sie es tat - ein vollkommen neues Haus für ihn zu schaffen - und sie würde die Lehrpläne für den gesamten Unterricht neu zusammenstellen müssen, am ersten Tag des Schuljahres. Und Dumbledore würde sie als Hauslehrerin von Gryffindor abziehen und ihr geliebtes Haus übergeben an… Professor Binns, den Geschichtsgeist und sie würde zur Hauslehrerin von Harrys Haus der Verdammnis ernannt und würde vergeblich versuchen, dem Kind Anweisungen zu erteilen, Punkt um Punkt abziehen ohne Wirkung, während ihr für Desaster um Desaster die Schuld gegeben würde.

War das der schlimmstmögliche Fall?

Minerva konnte ehrlich nicht sehen, wie es noch schlimmer kommen sollte als das.

Und selbst im allerschlimmsten Fall - egal \emph{was} mit Harry geschah - würde alles nach sieben Jahren vorüber sein.

Minerva spürte, wie ihre weiß hervortretenden Knöchel langsam ihren Griff um das Podium lockerten. Harry hatte recht gehabt, es lag etwas Tröstendes darin, in die tiefsten Abgründe der Finsternis zu starren, zu wissen, dass man sich nunmehr seinen schlimmsten Ängsten gestellt hatte und vorbereitet war.

Die ängstliche Stille wurde von einem einzelnen Wort durchbrochen.

„Schulleiter!“ rief der Sprechende Hut.

Am Lehrertisch erhob sich Dumbledore mit Verwirrung auf dem Gesicht. „Ja?“ wandte er sich an den Hut. „Was gibt es?“

„Ich habe nicht mit Ihnen gesprochen,“ sagte der Hut. „Ich habe Harry Potter den Platz in Hogwarts zugewiesen, an den er am ehesten gehört, nämlich das Büro des Schulleiters—“

* Ich hoffe die deutsche Version weiß auch zu gefallen, obwohl ich etwas improvisieren musste, besonders um die Anspielung auf Snape zu behalten. Für alle, die auch das Original lesen (oder singen) möchten:

\emph{There's a Dark Lord near?

Got no need to fear

Who you gonna call?}

\emph{HARRY POTTER!}

\emph{With a Killing Curse?

Well it could be worse.

Who you gonna call?}

\emph{HARRY POTTER!}

\emph{I ain't afraid of Dark Lords!}

\emph{I ain't afraid of Dark Lords!}

\emph{Dark robes and a mask?

Impossible task?

Who you gonna call?}

\emph{HARRY POTTER!}

\emph{Giant Fire-Ape?

Old bat in a cape?

Who you gonna call?}

\emph{HARRY POTTER!}

\emph{I ain't afraid of Dark Lords!}

\emph{I ain't afraid of Dark Lords!}

\emph{** Dies dürfte sich auf das neunte Extra beziehen, in welchem der Sprechende Hut Harry wohl ins House of Commons (das britische Unterhaus) hätte einsortieren müssen, aber da der Autor US-Bürger ist, hat er sich wohl lieber für das dortige Repräsentantenhaus entschieden.}

\emph{*** Der Wortbedeutung und später dem Kontext nach offenbar eine Art „Bauchredner-Zauber“, mit dem man seine Stimme an einem gewünschten Ort erklingen lassen kann, auf den man seinen Zauberstab richtet.}

\emph{****} engl.: „\emph{PANCAKES!}“\emph{,} eine Anspielung auf die Restaurantkette \emph{IHOP} (\emph{International House of Pancakes}) in den USA. Ich habe mich hier für die in Deutschland zu findende Restaurantkette \emph{Block House} entschieden.

\emph{***** Das Haus Atreides ist eine Adels-Familie aus den} \emph{Dune}-Zyklen von Frank Herbert.

\emph{******} engl.: „\emph{PICKLED STEWBERRIES!}“\emph{,} in Weiterführung des „Haus XY“-Themas wohl eine Verballhornung hausgemachter Gerichte, in diesem Fall ungefähr \emph{hausgepökelte Schmorbeeren.}

\emph{******* Offenbar eine Anspielung auf} \emph{Der Zorn des Khan.}

