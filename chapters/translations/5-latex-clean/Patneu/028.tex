

\hypertarget{reduktionismus}{% \section{28. Reduktionismus}\label{reduktionismus}}

\textbf{Kapitel 28: Reduktionismus}

Whatever can go Rowling will go Rowling.

\emph{Erneut} sollte das \emph{nicht gesagt} werden müssen, doch von Severus Snape geäußerte Ansichten sind nicht notwendigerweise diejenigen des Autors.

\later

"Okay," sagte Harry und schluckte. "Okay, Hermine, es ist genug, du kannst aufhören."

Die weiße Zuckerpille vor Hermine hatte Form oder Farbe noch immer nicht im Mindesten verändert, trotzdem sie sich stärker konzentrierte als Harry es je zuvor gesehen hatte, mit zusammengekniffenen Augen, Schweißperlen auf ihrer Stirn und zitternder Hand, die ihren Zauberstab umklammert hielt -

"Hermine, \emph{hör auf!} Es wird nicht funktionieren, Hermine, wir können wohl keine Dinge erschaffen, die es noch nicht gibt!"

Langsam lockerte Hermines Hand ihren Griff um den Zauberstab.

"Ich dachte, ich hätte es gespürt," sagte sie, ihre Stimme ein bloßes Flüstern. "Ich dachte, ich hätte gespürt, wie die Transfiguration beginnt, nur für eine Sekunde."

Ein Kloß saß Harry im Hals. "Du hast es dir wahrscheinlich eingebildet. Zu sehr darauf gehofft."

"Ja, wahrscheinlich," sagte sie. Sie sah aus, als wolle sie gleich in Tränen ausbrechen.

Zögernd nahm Harry seinen mechanischen Bleistift zur Hand, langte hinüber zu dem Stück Papier mit all den ausgestrichenen Gegenständen und zog eine Linie durch den Punkt namens 'HEILMITTEL FÜR ALZHEIMER'.

Sie hätten niemandem eine transfigurierte Pille verabreichen können. Doch Transfiguration, zumindest auf die Art, zu der sie in der Lage waren, verzauberte das Zielobjekt nicht - sie würde einen gewöhnlichen Besen nicht in einen fliegenden verwandeln. Wäre Hermine also überhaupt in der Lage gewesen, die Pille zu erschaffen, wäre es eine \emph{nicht-magische} Pille gewesen, die aus gewöhnlichen materiellen Gründen funktionierte. Sie hätten insgeheim Pillen für ein Muggel-Forschungslabor anfertigen können, um sie dort \emph{untersuchen} zu lassen und zu versuchen, sie zurück zu entwickeln, bevor die Transfiguration nachließ… in keiner der beiden Welten hätte irgendjemand davon erfahren müssen, dass Magie im Spiel gewesen war, es wäre einfach ein weiterer wissenschaftlicher Durchbruch…

Es gehörte auch nicht zu der Art von Dingen, an die Zauberer denken würden. Sie respektierten die reine \emph{Anordnung der} \emph{Atome} nicht so sehr, sie hielten unverzauberte, \emph{materielle} Dinge nicht für Objekte, die Macht besitzen könnten. War es nicht magisch, war es nicht interessant.

Zuvor hatte Harry \emph{ganz} im Geheimen - er hatte es nicht einmal Hermine erzählt - versucht, Nanotechnologie nach Eric Drexler zu transfigurieren. (Er hatte, natürlich, versucht, eine Nanofabrik zu produzieren, keine winzigen selbst-replizierenden Assembler, Harry war ja nicht verrückt.*) Hätte es funktioniert, wäre es Göttlichkeit auf einen Schlag gewesen.

"Das war's für heute, oder?" sagte Hermine. Sie hatte sich in ihrem Stuhl zurückfallen lassen, lehnte ihren Kopf gegen die Rückenlehne und auf ihrem Gesicht zeigte sich Erschöpfung, was für Hermine sehr ungewöhnlich war. Sie tat gern so, als kenne sie keine Grenzen, zumindest wenn Harry in der Nähe war.

"Eins noch," sagte Harry behutsam, "aber das ist etwas kleines, außerdem könnte es tatsächlich funktionieren. Ich habe es bis zum Schluss aufgehoben, weil ich gehofft habe, wir könnten mit einem Erfolg aufhören. Es ist etwas echtes, nicht so wie Phaser. Man hat es bereits in einem Labor hergestellt, anders als das Heilmittel für Alzheimer. Und es ist eine generische Substanz, nichts spezifisches, wie die verlorenen Bücher, von denen du Kopien zu transfigurieren versucht hast. Ich habe ein Diagramm der Molekularstruktur erstellt, damit ich es dir zeigen kann. Wir wollen es nur \emph{länger} machen als je zuvor und mit allen Röhren auf Linie und den Endpunkten in Diamant eingefasst." Harry brachte ein Blatt Millimeterpapier zum Vorschein.

Hermine setzte sich auf, nahm es entgegen und studierte es stirnrunzelnd. "Das sind \emph{alles} Kohlenstoff-Atome? Und Harry, wie ist der Name? Ich kann es nicht transfigurieren, wenn ich nicht weiß, wie es heißt."

Harry zogangewidert eine Grimasse. Er hatte noch immer Probleme, solche Sachen zu akzeptieren, es sollte keine Rolle spielen, wie man etwas \emph{nannte,} wenn man wusste, was es \emph{war.} "Man nennt sie Buckytubes oder Kohlenstoffnanoröhren. Es ist eine Art Fulleren, das erst in diesem Jahr entdeckt wurde. Es ist etwa hundertmal stärker als Stahl und wiegt nur ein Sechstel davon."

Hermine blickte überrascht von dem Millimeterpapier auf. "Das gibt es \emph{wirklich?}"

"Ja," sagte Harry, "ist nur schwierig herzustellen auf Muggel-Art. Könnten wir genug davon kriegen, dann könnten wir damit einen Weltraumlift bauen bis ganz nach oben in den geosynchronen Orbit oder noch höher und was Delta-V** angeht, ist das schon die halbe Strecke zu jedem Ort im Sonnensystem. Außerdem könnten wir Satelliten für Solarenergie rauswerfen wie Konfetti."

Hermine runzelte erneut die Stirn. "Ist das Zeug \emph{sicher?}"

"Ich sehe nicht wieso nicht," sagte Harry. "Eine Buckytube ist, im Prinzip, nur ein zu einer Röhre gerolltes Blatt aus Graphit und Graphit ist das selbe Zeug, das man für Bleistifte benutzt -"

"Ich \emph{weiß,} was Graphit ist, Harry," sagte Hermine. Sie strich gedankenverloren ihr Haar zurück, während sie mit gefurchten Augenbrauen das Blatt Papier anstarrte.

Harry langte in eine Tasche seines Umhangs und brachte einen weißen Faden zum Vorschein, gebunden an zwei kleine Ringe aus grauem Plastik an den Enden. Wo der Faden auf die Ringe traf, hatte er Tropfen von Superkleber hinzugefügt, um aus allem ein einzelnes Objekt zu machen, das im Ganzen transfiguriert werden konnte. Cyanacrylat funktionierte, wenn Harry sich richtig erinnerte, durch kovalente Bindungen und das kam einem "soliden Objekt" so nahe, wie es jemals möglich war in einem Universum, das letztendlich aus winzigen individuellen Atomen zusammengesetzt war. "Wenn du bereit bist," sagte Harry, "versuch das zu einer Reihe von ausgerichteten Buckytube-Fasern zu transfigurieren, die in zwei Ringe aus solidem Diamant eingefasst sind."

"In Ordnung…" sagte Hermine langsam. "Harry, ich hab das Gefühl, ich hätte gerade was vergessen."

Harry zuckte hilflos mit den Schultern. \emph{Vielleicht bist du nur müde.} Obwohl er es besser wusste als das laut zu sagen.

Hermine legte ihren Zauberstab gegen einen der Plastik-Ringe und starrte eine Weile vor sich hin.

Zwei kleine Kreise aus glitzerndem Diamant lagen auf dem Tisch, verbunden durch einen langen schwarzen Faden.

"Es hat sich verwandelt," sagte Hermine. Sie klang als versuche sie, Enthusiasmus zu zeigen, habe aber nicht mehr die Kraft. "Was jetzt?"

Harry fühlte sich leicht ernüchtert von der mangelnden Leidenschaft seiner Forschungspartnerin, gab aber sein bestes, es nicht zu zeigen; vielleicht gelang es dem gleichen Prozess, nur umgekehrt, sie aufzumuntern. "Jetzt prüfe ich, ob es Gewicht hält."

Harry hatte für ein früheres Experiment mit Diamant-Stäben bereits ein Gestell vorbereitet - man konnte mittels Transfiguration mit Leichtigkeit solide Objekte aus Diamant hervorbringen, sie waren nur nicht von Dauer. Das frühere Experiment hatte festgestellt, ob einen langen Diamant-Stab in einen kürzeren zu transfigurieren, diesem erlauben würde, während des Zusammenziehens ein daran befestigtes schweres Gewicht zu heben, also ob man gegen Spannung antransfigurieren konnte, was in der Tat möglich war.

Harry streifte vorsichtig einen Kreis aus glitzerndem Diamant über den dicken Metallhaken oben am Gestell, hing einen starken Metall-Aufhänger an den unteren Ring und begann dann, Gewichte am Aufhänger zu befestigen.

(Harry hatte die Weasley-Zwillinge gebeten, die Apparatur für ihn zu transfigurieren und die Weasley-Zwillinge hatten ihn mit einem ungläubigen Blick bedacht, als könnten sie sich nicht vorstellen, für \emph{welche} Art Streich er das nur wollen könnte, doch sie hatten keine Fragen gestellt. Und ihre Transfigurationen hielten, ihnen zufolge, ungefähr drei Stunden, also hatten Harry und Hermine noch eine Weile.)

"Einhundert Kilogramm," sagte Harry etwa eine Minute später. "Ich glaube kaum, dass ein so dünner Stahlfaden das aushalten würde. Es sollte noch viel mehr gehen, aber mehr Gewichte habe ich nicht."

Die Stille dehnte sich aus.

Harry richtete sich auf und ging zu ihrem Tisch zurück, setzte sich in seinen Stuhl und machte feierlich ein Häkchen neben 'Buckytubes'. "Na also," sagte Harry. "\emph{Das} hat funktioniert."

"Es ist aber nicht wirklich \emph{nützlich,} Harry, oder?" sagte Hermine von dem Platz, wo sie saß, den Kopf in die Hände gestützt. "Ich meine, selbst würden wir sie einem Wissenschaftler geben, könnte man aus unseren nicht lernen, wie man viele Buckytubes herstellen kann."

"Man könnte vielleicht \emph{etwas} daraus lernen," sagte Harry. "Hermine, sieh dir das \emph{an,} dieser winzig kleine Faden trägt all dieses Gewicht, wir haben gerade etwas gemacht, das kein Muggel-Labor hätte herstellen können -"

"Doch jede andere Hexe könnte es," sagte Hermine. Die Erschöpfung schlug sich jetzt in ihrer Stimme nieder. "Harry, ich glaube, das funktioniert nicht."

"Du meinst unsere Beziehung?" sagte Harry. "Toll! Machen wir Schluss."

Das entlockte ihr ein schwaches Grinsen. "Ich meine unsere Forschung."

"Oh, Hermine, wie \emph{kannst} du nur?"

"Du bist süß, wenn du gemein bist," sagte sie. "Aber Harry, das ist doch Wahnsinn, ich bin zwölf, du bist elf, es ist \emph{dumm,} zu glauben, wir würden irgendwas entdecken, was noch niemand zuvor herausgefunden hat."

"Willst du wirklich sagen, wir sollten die Enthüllung der Geheimnisse der Magie aufgeben, nachdem wir es nicht einmal einen \emph{Monat} lang versucht haben?" sagte Harry und versuchte, eine herausfordernde Note in seine Stimme zu legen. Um ehrlich zu sein verspürte er zum Teil die selbe Erschöpfung wie Hermine. Keine der \emph{guten} Ideen funktionierte jemals. Er hatte erst eine erwähnenswerte Entdeckung gemacht, das Mendelsche Muster und davon konnte er Hermine nicht erzählen, ohne sein Versprechen an Draco zu brechen.

"Nein," sagte Hermine. Ihr junges Gesicht wirkte jetzt sehr ernst und erwachsen. "Ich will sagen, wir sollten all die Magie \emph{studieren,} die Zauberer bereits kennen, damit wir solche Sachen nach unserem Abschluss von Hogwarts machen können."

"Ähm…" sagte Harry. "Hermine, ich sage das nur ungern so, aber stell dir vor, wir hätten entschieden, unsere Forschung bis später aufzuschieben und das erste, was wir versuchten, wäre ein Heilmittel für Alzheimer zu transfigurieren und es hätte \emph{funktioniert.} Wir würden uns fühlen… ich glaube das Wort \emph{dämlich} beschreibt nicht ansatzweise, wie wir uns fühlen würden. Was wenn es irgendwas anderes in der Art gibt und es funktioniert tatsächlich?"

"Das ist nicht \emph{fair,} Harry!" sagte Hermine. Ihre Stimme zitterte, als stünde sie kurz davor in Tränen auszubrechen. "Das kannst du Leuten nicht \emph{aufladen!} Es ist nicht unser \emph{Job,} solche Sachen zu machen, wir sind \emph{Kinder!}"

Einen Moment lang fragte sich Harry, was passieren mochte, wenn jemand Hermine erzählte, sie müsse einen unsterblichen Dunklen Lord bekämpfen, ob sie zu einem jener weinerlichen, selbstmitleidigen Helden würde, von denen zu lesen Harry in seinen Büchern nie ausstehen konnte.

"Jedenfalls," sagte Hermine. Ihre Stimme bebte. "Ich will das nicht mehr machen. Ich glaube nicht, dass Kinder Dinge tun können, zu denen Erwachsene nicht in der Lage sind, das gibt es nur in Geschichten."

Stille herrschte im Klassenraum.

Hermine sah langsam ein wenig verängstigt aus und Harry wusste, sein eigener Gesichtsausdruck war kälter geworden.

Vielleicht hätte es nicht so sehr geschmerzt, wäre Harry der gleiche Gedanke nicht auch schon gekommen - dass, obwohl dreißig Jahre für einen wissenschaftlichen Durchbruch alt sein mochten und zwanzig etwa richtig, obwohl es Leute gab, die mit siebzehn einen Doktorgrad erwarben und vierzehnjährige Erben, die große Könige und Generäle gewesen waren, es nicht wirklich irgendwen gab, der es mit elf in die Geschichtsbücher geschafft hatte.

"In Ordnung," sagte Harry. "Herausfinden, wie man etwas tut, das ein Erwachsener nicht kann. Ist das deine Herausforderung?"

"So habe ich es nicht gemeint," sagte Hermine, ihre Stimme erklang als ängstliches Flüstern.

Mit einigem Aufwand riss Harry seinen Blick von Hermine los. "Ich bin nicht wütend auf \emph{dich,}" sagte Harry. Seine Stimme war kalt, trotz aller Bemühungen. "Ich bin wütend auf, ich weiß nicht, auf alles. Doch ich bin nicht gewillt, zu verlieren, Hermine. Zu verlieren ist nicht immer das Richtige. Ich finde heraus, wie man etwas tut, das ein erwachsener Zauberer nicht kann und dann komme ich auf dich zurück. Wie klingt das?"

Mehr Stille.

"Okay," sagte Hermine, ihre Stimme wankte ein wenig. Sie drückte sich aus dem Stuhl hoch und trat hinüber zur Tür des verlassenen Klassenzimmers, in dem sie gearbeitet hatten. Ihre Hand wanderte zum Türknauf. "Wir sind noch Freunde, ja? Und wenn dir nichts einfällt -"

Ihre Stimme stockte.

"Dann lernen wir zusammen," sagte Harry. Seine Stimme war jetzt noch kälter.

"Ähm, dann mach's gut, für's erste," sagte Hermine und sie verließ schnell das Zimmer und schloss die Tür hinter sich.

Manchmal hasste Harry es, eine dunkle Seite zu haben, selbst wenn er sich gerade in ihr befand.

Und der Teil von ihm, der exakt das gleiche gedacht hatte wie Hermine, dass nein, Kinder \emph{nicht} tun konnten, wozu Erwachsene nicht fähig waren, sagte all die Dinge, die auszusprechen Hermine zu große Angst gehabt hatte, wie \emph{Da hast du dir ja gerade mal eine höllisch schwierige Herausforderung ausgesucht} und \emph{Junge, wirst du diesmal mit runter gelassener Hose dastehen} und \emph{Wenigstens weißt du so, dass du's vermasselt hast.}

Und der Teil von ihm, dem es nicht gefiel, zu verlieren, erwiderte mit kalter Stimme, \emph{Schön, na dann halt mal die Klappe und schau gut hin.}

\later

Es war beinahe Zeit zum Mittagessen und Harry war es egal. Er hatte sich nicht einmal die Mühe gemacht, einen Müsliriegel aus seinem Beutel zu schnappen. Sein Magen konnte ein wenig Hunger ertragen.

Die Zauberwelt war winzig, sie dachten nicht wie Wissenschaftler, sie wussten nichts von Wissenschaft, sie stellten nicht infrage, womit sie aufgewachsen waren, sie hatten ihre Zeitmaschinen nicht mit Schutzhüllen ausgestattet, sie spielten Quidditch, das ganze magische Britannien war kleiner als eine Muggel-Kleinstadt, die größte Zauberschule unterrichtete nur bis zum Alter von siebzehn, \emph{dumm} war es nicht, das mit elf Jahren herauszufordern, \emph{dumm} war es, davon \emph{auszugehen,} dass Zauberer wussten, was sie taten und bereits all die niedrig hängenden Früchte erschöpft hatten, die ein wissenschaftlich Universalgebildeter erkennen würde.

Schritt Eins war gewesen, eine Liste aller magischen Beschränkungen aufzustellen, die Harry einfielen, all der Dinge, die man angeblich nicht tun konnte.

Schritt Zwei, die Beschränkungen markieren, die aus wissenschaftlicher Perspektive am \emph{wenigsten} Sinn machten.

Schritt Drei, Beschränkungen priorisieren, die ein Zauberer wahrscheinlich nicht infrage stellen würde, wenn er \emph{nichts} von Wissenschaft wusste.

Schritt Vier, sich Methoden einfallen zu lassen, sie in Angriff zu nehmen.

\later

Hermine fühlte sich noch immer ein wenig zittrig, als sie sich neben Mandy an den Ravenclaw-Tisch setzte. Hermines Mittagessen bestand aus zwei Früchten (Tomatenscheiben und geschälten Mandarinen), drei Gemüsen (Karotten, Karotten und noch mehr Karotten), einem Stück Fleisch (frittierte Diricawl-Keulen, deren ungesunden Überzug sie sorgsam entfernen würde) und einem kleinen Stück Schokoladen-Kuchen, den sie sich verdienen würde, indem sie die anderen Teile aß.

Es war nicht so schlimm gewesen, wie im Zaubertränke-Unterricht, darüber hatte sie manchmal noch \emph{Alpträume.} Doch diesmal hatte \emph{sie} es ausgelöst und \emph{sie hatte sich als das Ziel empfunden.} Nur für einen Augenblick, bevor die schreckliche Dunkelheit weg gesehen hatte und gesagt, sie sei nicht wütend auf sie, weil sie ihr keine Angst machen wollte.

Und noch immer hatte sie das Gefühl, sie habe vorhin etwas vergessen, etwas wirklich wichtiges.

Doch sie hatten keine der Regeln der Transfiguration verletzt… oder? Sie hatten keine Flüssigkeiten hergestellt, keine Gase, sie hatten keine Anweisungen vom Verteidigungs-Professor entgegen genommen…

Die \emph{Pille!} Das war etwas zu essen gewesen!

… aber, nein, niemand würde einfach eine herumliegende Pille schlucken, es hatte auch nicht wirklich \emph{funktioniert,} sie hätten einfach \emph{Finite Incantatem} benutzen können, falls doch, aber sie würde Harry trotzdem davon erzählen müssen und sichergehen, dass sie es nicht Professor McGonagall gegenüber erwähnten, falls sie sonst niemals wieder Transfiguration würden studieren dürfen…

Hermine bekam langsam ein wirklich übles Gefühl im Magen. Sie schob ihren Teller auf dem Tisch zurück, so konnte sie nicht zu Mittag essen.

Und sie schloss ihre Augen und begann im Geiste die Regeln der Transfiguration zu rezitieren.

"\emph{Ich werde niemals etwas in eine Flüssigkeit oder ein Gas} \emph{transfigurieren.}"

"\emph{Ich werde niemals etwas transfigurieren, das wie Essen aussieht oder irgendetwas anderes, das in einen menschlichen Körper gelangt.}"

Nein, sie hätten wirklich \emph{nicht} versuchen sollen, die Pille zu transfigurieren oder es hätte ihnen zumindest \emph{klar werden} müssen… sie hatte sich von Harrys brillanter Idee so gefangen nehmen lassen, dass sie nicht \emph{nachgedacht} hatte…

Das üble Gefühl in Hermines Magen wurde schlimmer. Im Geiste hatte sie das Gefühl als schwebe etwas genau am Rande ihrer Wahrnehmung, eine Auffassung, die davor stand, sich ins Gegenteil zu verkehren, eine junge Frau, die bald ein altes Weib würde, eine Vase, aus der zwei Gesichter wurden…

Und sie rief sich weiter die Regeln der Transfiguration ins Gedächtnis.

\later

Harrys Knöchel waren um seinen Zauberstab herum weiß hervorgetreten, als er endlich den Versuch aufgab, die Luft vor seinem Zauberstab in eine Büroklammer zu transfigurieren. Es wäre natürlich nicht sicher gewesen, eine Büroklammer in Gas zu transfigurieren, doch Harry sah keinen Grund, warum es andersherum unsicher sein sollte. Es sollte nur einfach nicht \emph{möglich} sein. Aber warum nicht? Luft war eine ebenso reale Substanz wie alles andere…

Nun, vielleicht \emph{machte} diese Beschränkung Sinn. Luft war unorganisiert, alle Moleküle veränderten ständig ihre Beziehung zueinander. Vielleicht konnte man der Substanz keine neue Form aufprägen, wenn sie nicht lange genug still hielt, um sie zu meistern, obwohl die Atome in Feststoffen ebenso ständig vibrierten…

Umso mehr Harry scheiterte, desto mehr fühlte er die Kälte und umso klarer schien alles zu werden.

Okay. Das nächste auf der Liste.

Man konnte nur ganze Objekte im Ganzen transfigurieren. Man konnte kein \emph{halbes} Streichholz in eine Nadel verwandeln, es musste das \emph{ganze Ding} sein. Damals als Harry von Draco in dem Klassenraum eingesperrt worden war, war das der Grund gewesen, weshalb er nicht einfach einen dünnen, zylindrischen Querschnitt der Wände zu Schwamm transfigurieren und einen Steinbrocken heraus stoßen konnte, der groß genug war, dass er durch das Loch passte. Er hätte der gesamten Wand eine neue Form aufprägen müssen und vielleicht dem ganzen Abschnitt von Hogwarts, nur um diesen kleinen Querschnitt zu verändern.

Und das war \emph{lächerlich.}

\emph{Dinge waren aus Atomen gemacht.} Lauter winzig kleinen Pünktchen. Es \emph{gab} keinen Zusammenhang, es \emph{gab} keine Solidität, nur elektromagnetische Kräfte, die die kleinen Pünktchen miteinander verbunden hielten…

\later

Mandy Brocklehurst hielt inne, mit der Gabel auf dem Weg zum Mund. "Huh," sagte sie zu Su Li, die dem jetzt leeren Platz neben ihr gegenüber saß, "was ist denn in Hermine gefahren?"

\later

Harry wollte seinen Radierer am liebsten umbringen.

Er hatte versucht, einen einzigen Flecken auf dem pinken Rechteck in Stahl zu verwandeln, unabhängig vom Rest des Gummis und der Radierer kooperierte nicht.

Es \emph{musste} eine gedankliche Beschränkung sein, keine reale. \emph{Musste.}

\emph{Dinge waren aus Atomen gemacht} und jedes Atom war ein kleines separates Ding. Atome wurden zusammengehalten von einem Quantennebel aus geteilten Elektronen für kovalente Bindungen oder manchmal auch nur Magnetismus auf kurze Distanz für ionische Bindungen oder Van-der-Waals-Kräfte.

Wenn man es so betrachtete, waren die Protonen und Neutronen in den Atomkernen winzige separate Dinge. Die Quarks in den Protonen und Neutronen waren winzige separate Dinge! Es \emph{existierte} einfach nichts in der Realität, der Welt-da-draußen, das der menschlichen Wahrnehmung solider Objekte entsprach. Es waren alles nur kleine Pünktchen.

Und freie Transfiguration war im Grunde eine rein geistige Angelegenheit, nicht wahr? Keine Worte, keine Gesten. Nur das reine Konzept der Form, strikt getrennt von Substanz, der Substanz aufgeprägt, unabhängig von ihrer Form gedacht. Das und der Zauberstab und was immer es war, das einen zum Zauberer machte.

Die Zauberer konnten keine Teile von Dingen transfigurieren, konnten nur transformieren, was ihr Geist als Ganzes erfasste, weil sie es nicht \emph{in den Knochen spürten,} dass es alles nur Atome waren bis ganz nach unten.

Harry hatte sich, so sehr er nur konnte, auf dieses Wissen konzentriert, den \emph{wahren Fakt,} dass der Radierer nur eine Ansammlung von Atomen war, alles nur eine Ansammlung von Atomen war und die Atome des kleinen Fleckens, den er zu transfigurieren versuchte, eine \emph{ebenso gültige} Ansammlung bildeten, wie jede andere Ansammlung, die ihm einfallen mochte.

Und Harry war noch immer nicht in der Lage gewesen, den einzelnen Teil des Radierers zu verändern, die Transfiguration führte einfach nirgendwo hin.

\emph{Das. War. Lächerlich.}

Harrys Knöchel schlossen sich erneut weiß hervortretend um seinen Zauberstab. Er war es \emph{leid,} experimentelle Resultate zu erhalten, die \emph{keinen Sinn ergaben.}

Vielleicht hielt die Tatsache, dass \emph{irgendein} Teil seines Geistes noch immer in Begriffen von Objekten dachte, die Transfiguration davon ab, durch zu gehen. Er hatte an eine Ansammlung von Atomen gedacht, die ein \emph{Radierer} war. Er hatte an eine Ansammlung von Atomen gedacht, die ein \emph{kleiner Flecken} war.

Zeit, einen Gang höher zu schalten.

Harry presste seinen Zauberstab stärker gegen den winzigen Abschnitt des Radierers und versuchte, durch die Illusion zu blicken, die Nicht-Wissenschaftler für die Realität hielten, die Welt von Schreibpulten und Stühlen, von Luft und Radierern und Menschen.

Wenn man durch einen Park spazierte, war die immersive Welt, die einen umgab, etwas das im eigenen Gehirn als Muster feuernder Neuronen existierte. Der Eindruck eines strahlend blauen Himmels war nichts, was sich hoch über einem befand, es war etwas im eigenen visuellen Kortex und der visuelle Kortex befand sich im hinteren Teil des eigenen Gehirns. All die Eindrücke jener lichten Welt spielten sich in Wirklichkeit in dieser stillen Knochenhöhle ab, die man Schädel nannte, dem Ort, an demdas eigene \emph{Ich} lebte und den es nie und nimmer verließ. Wollte man wirklich jemandem Hallo sagen, der \emph{tatsächlichen Person,} dann schüttelte man ihm nicht die Hand, man klopfte sanft an seinen Schädel und sagte "Wie geht's dir da drin?" Das war, was Menschen ausmachte, wo sie wirklich lebten. Und das \emph{Bild} des Parks, durch den man \emph{zu spazieren} glaubte, war etwas, das im eigenen Gehirn visualisiert wurde, indem es Signale von den Augen und der Netzhaut verarbeitete.

Es war keine \emph{Lüge,} wie die Buddhisten glaubten, es lag nichts furchtbar mystisches und unerwartetes hinter dem Schleier der Maya, was hinter der Illusion des Parks lag, war einfach nur der \emph{wirkliche Park,} doch trotzdem war alles eine \emph{Illusion.}

Harry saß nicht in dem Klassenzimmer.

Harry blickte nicht auf den Radierer.

Harry befand sich in Harrys Schädel.

Er erlebte ein verarbeitetes Bild, das sein Gehirn aus den von seiner Netzhaut gesendeten Signalen dekodierte.

Der wirkliche Radierer befand sich anderswo, irgendwo außerhalb des Bildes.

Und der wirkliche Radierer war nicht so, wie das Bild, das Harrys Gehirn von ihm hatte. Die Idee des Radierers als \emph{solides Objekt} war etwas, das nur in seinem eigenen Gehirn existierte, in seinem parietalen Kortex, der seinen Sinn für Form und Raum verarbeitete. Der wirkliche Radierer war eine Ansammlung von Atomen, die von elektromagnetischen Kräften und geteilten Valenzelektronen zusammengehalten wurde, während in der Nähe Luftmoleküle voneinander und den Molekülen des Radierers abprallten.

Der wirkliche Radierer war weit entfernt und Harry, in seinem Schädel, konnte niemals wirklich mit ihm in Berührung kommen, sich nur Ideen darüber vorstellen. Doch \emph{sein Zauberstab hatte die Macht,} er konnte Dinge da draußen in der \emph{Realität} verändern, es waren nur Harrys eigene Voreingenommenheiten, die ihn \emph{einschränkten.} Irgendwo hinter dem Schleier der Maya berührte die \emph{Wahrheit} hinter Harrys Konzept "mein Zauberstab" die Ansammlung von Atomen, von denen Harrys Geist als "ein Flecken auf dem Radierer" dachte und wenn dieser Zauberstab die Ansammlung von Atomen verändern konnte, die Harry als "den ganzen Radierer" betrachtete, dann gab es absolut keinen Grund, warum er die andere Ansammlung nicht ebenfalls verändern konnte…

Die Transfiguration ging noch immer nicht durch.

Harry biss die Zähne zusammen und schaltete \emph{noch einen} Gang höher.

Das Konzept das Harrys Geist von dem Radierer als solidem Objekt hatte, war \emph{offensichtlich Unsinn.}

Es war eine Karte, die dem Territorium nicht entsprach, nicht entsprechen \emph{konnte.}

Menschliche Wesen erstellten ihr Modell der Welt mithilfe vielschichtiger Organisationsebenen, sie hatten \emph{separate Ansichten} darüber, wie Länder funktionierten, wie Menschen funktionierten, wie Organe funktionierten, wie Zellen funktionierten, wie Moleküle funktionierten, wie Quarks funktionierten.

Wenn Harrys Gehirn über Radierer nachdenken musste, dächte es an die Regeln, denen Radierer gehorchten, wie "Radierer können Bleistift-Striche entfernen". Nur wenn Harrys Hirn vorhersagen musste, was auf der niedrigeren chemischen Ebene geschah, nur dann würde Harrys Gehirn anfangen - als sei es ein separater Fakt - über Radierer-Moleküle nachzudenken.

Doch das alles war nur im \emph{Geiste.}

Harrys Geist mochte separate \emph{Ansichten} haben, über die Regeln denen Radierer gehorchten, doch Radierer gehorchten keinem \emph{separaten} \emph{physikalischen} \emph{Gesetz.}

Harrys Geist erstellte sein Modell der Realität, mithilfe mehrerer Organisationsebenen, mit verschiedenen Ansichten über jede Ebene. Doch das alles lag nur in der \emph{Karte,} das wahre Territorium entsprach dem nicht, die \emph{Realität selbst} kannte nur eine \emph{einzige} Organisationsebene, die Quarks, ein einheitlicher Prozess niedrigster Ordnung, der mathematisch simplen Regeln gehorchte.

Oder zumindest war es das, was Harry geglaubt hatte, bevor er von Magie erfahren hatte, aber der Radierer war nicht magisch.

Und selbst \emph{wäre} der Radierer magisch gewesen, war die Vorstellung, es könne \emph{tatsächlich} ein einzelner solider Radierer existieren, einfach \emph{unmöglich.} Dinge wie Radierer \emph{konnten} keine grundlegenden Bestandteile der Realität sein, sie waren zu groß und kompliziert, um Atome zu sein, sie \emph{mussten} aus Teilen bestehen. Es konnte keine Dinge geben, die \emph{fundamental kompliziert} waren. Der implizite Glaube, den Harrys Gehirn an den Radierer als einzelnes solides Objekt hatte, war nicht nur \emph{falsch,} es war eine Verwechslung von Karte und Territorium, der Radierer existierte nur als separates Konzept in Harrys mehrschichtigem \emph{Modell} der Welt, nicht als separates Element einer einschichtigen Realität.

… die Transfiguration \emph{geschah noch immer nicht.}

Harry atmete schwer, fehlgeschlagene Transfiguration war beinahe ebenso erschöpfend wie erfolgreiche Transfiguration, aber \emph{verdammt} sollte er sein, wenn er jetzt aufgab.

Okay, scheiß auf diesen Neunzehntes-Jahrhundert-Müll.

Die Realität bestand nicht aus Atomen, sie war keine Ansammlung winziger herum springender Billard-Kugeln. Das war nur eine weitere Lüge. Die Auffassung von Atomen als kleine Pünktchen war nur eine weitere bequeme Halluzination, an die Menschen sich klammerten, weil sie sich nicht der unmenschlich fremdartigen Struktur der zugrunde liegenden Realität stellen wollten. Kein Wunder, dann, das seine darauf basierenden Transfigurationsversuche nicht erfolgreich gewesen waren. Wenn er Macht wollte, musste er seine Menschlichkeit ablegen und seine Gedanken zwingen, sich der wahren Mathematik der Quantenmechanik anzupassen.

Es \emph{gab keine Partikel,} es gab nur \emph{Wolken aus Amplitude} in einem \emph{Konfigurationsraum über mehrere Partikel} und was sein Gehirn sich so gern als einen Radierer vorstellte, war nichts weiter als ein gigantischer \emph{Faktor} in einer \emph{faktorisierenden} Wellenfunktion, er hatte ebenso wenig eine eigenständige Existenz, wie ein solider Faktor von 3 versteckt in der Zahl 6 existierte, wenn sein Zauberstab in der Lage war, \emph{Faktoren in einer angemessen faktorisierbaren Wellenfunktion zu verändern,} dann sollte er verdammt nochmal auch in der Lage sein, den etwas \emph{kleineren} Faktor zu ändern, den Harrys Gehirn als Flecken auf dem Radierer visualisierte -***

\later

Hermine stürmte durch die Korridore, die Schuhe schlugen hart auf dem Stein auf, ihr Atem kam stoßweise, der Adrenalin-Schock raste noch immer durch ihre Adern.

Wie das Bild einer jungen Frau, die zu einem alten Weib wurde, wie ein Kelch, aus dem zwei Gesichter wurden.

Was hatten sie getan?

\emph{Was hatten sie getan?}

Sie erreichte den Klassenraum und ihre Finger glitten zunächst am Türknauf ab, zu schwitzig, sie packte härter zu und die Tür öffnete sich -

- in einer blitzartigen Einschätzung erblickte sie Harry, der ein kleines pinkes Rechteck auf dem Tisch vor ihm anstarrte -

- während ein paar Schritte entfernt der winzige schwarze Faden, aus der Entfernung fast unsichtbar, all das Gewicht trug -

"\emph{Harry, raus aus dem Klassenraum!}"

Der reine Schock zeigte sich auf Harrys Gesicht und er stand so schnell auf, dass er fast gestürzt wäre, hielt nur inne, um das kleine pinke Rechteck vom Tisch zu klauben und er platzte aus der Tür, sie war bereits beiseite getreten, den Zauberstab schon in der Hand, hob ihn und zielte auf den Faden -

"\emph{Finite Incantatem!}"

Und Hermine warf die Tür wieder zu, genau als der ohrenbetäubende Krach von einhundert Kilogramm fallenden Metalls aus dem Inneren erklang.

Sie keuchte, schnappte nach Luft, sie war ohne anzuhalten den ganzen Weg hierher gerannt, sie war schweißgebadet und ihre Beine und Schenkel brannten wie lodernde Flammen, sie hätte Harrys Fragen um alle Galleonen der Welt nicht beantworten können.

Hermine blinzelte und stellte fest, dass ihre Beine nachgegeben hatten und Harry sie aufgefangen hatte und sie jetzt sanft zu Boden ließ.

"… gesund…" brachte sie flüsternd hervor.

"\emph{Was?}" sagte Harry und sah blasser aus, als sie ihn je zuvor gesehen hatte.

"… fühlst du, dich, gesund…"

Harry sah noch verängstigter aus, als die Frage langsam sackte. "Ich, ich glaube, ich habe keine Symptome -"

Hermine schloss für einen Moment die Augen. "Gut," flüsterte sie. "Atem, holen."

Das dauerte eine Weile. Harry sah noch immer verängstigt aus. Das war auch gut, vielleicht lernte er was daraus.

Hermine langte in den Beutel, den Harry ihr gekauft hatte, flüsterte "Wasser" durch ihre raue Kehle, nahm die Flasche heraus und trank in großen Schlucken.

Und dann dauerte es noch immer eine Weile, bevor sie wieder sprechen konnte.

"Wir haben die Regeln gebrochen, Harry," sagte sie mit heiserer Stimme. "Wir haben die Regeln gebrochen."

"Ich…" schluckte Harry. "Ich verstehe noch immer nicht wie, ich habe darüber \emph{nachgedacht,} aber -"

"Ich fragte, ob die Transfiguration sicher sei und \emph{du hast mir geantwortet!}"

Eine Pause entstand.

"Das war's?" sagte Harry.

Sie hätte schreien können.

"Harry, begreifst du nicht?" sagte sie. "Es ist aus winzigen Fasern gemacht, was wenn es sich \emph{aufdröselt,} wer \emph{weiß} was hätte schief gehen können, \emph{wir haben Professor McGonagall nicht gefragt!} Verstehst du nicht, was wir getan haben? Wir haben experimentiert mit Transfiguration. Wir haben \emph{experimentiert} mit \emph{Transfiguration!}"

Eine weitere Pause.

"Richtig…" sagte Harry langsam. "Das ist wahrscheinlich eins dieser Dinge, die sie einem nicht einmal \emph{verbieten,} weil es zu offensichtlich ist. Teste keine brillanten Ideen für Transfiguration ganz allein in einem ungenutzten Klassenraum ohne irgendeinen Professor zu konsultieren."

"Du hättest uns umbringen können, Harry!" Hermine wusste, es war nicht fair, sie hatte den Fehler auch gemacht, aber sie war immer noch wütend auf ihn, er klang immer so zuversichtlich und das hatte sie in seinem Kielwasser mitgezogen. "Wir hätten \emph{Professor McGonagalls perfekten Rekord ruinieren können!}"

"Ja," sagte Harry, "davon erzählen wir ihr besser nichts, oder?"

"Wir müssen aufhören," sagte Hermine. "Wir müssen damit aufhören oder uns passiert noch was. Wir sind zu jung, Harry, wir können das nicht tun, noch nicht."

Ein schwaches Grinsen huschte über Harrys Gesicht. "Ähm, da liegst du irgendwie falsch."

Und er hielt ihr ein kleines pinkes Rechteck entgegen, einen Radiergummi mit einem hellen metallenen Flecken darauf.

Hermine starrte ihn verwirrt an.

"Quantenmechanik reichte nicht aus," sagte Harry. "Ich musste bis ganz runter zu Zeitloser Physik***, bevor es klappte. Musste es so betrachten, dass der Zauberstab eine \emph{Verbindung} zwischen verschiedenen vergangenen und zukünftigen Realitäten schafft, anstatt irgendetwas über Zeit zu \emph{verändern} - aber ich hab's geschafft, Hermine, ich habe hinter die Illusion von Objekten geblickt und ich wette, es gibt nicht einen anderen Zauberer auf der Welt, der das gekonnt hätte. Selbst wenn ein Muggelgeborener von zeitlosen Formulierungen der Quantenmechanik wüsste, es wäre nur eine seltsame Idee über fernes seltsames Quanten-Zeugs, man würde nicht \emph{verstehen,} dass es die \emph{Realität} ist, nicht akzeptieren, dass die Welt, die man kennt, nur eine Halluzination ist. Ich habe einen \emph{Teil} des Radierers transfiguriert, ohne das \emph{ganze Ding} zu verwandeln."

Hermine hob ihren Zauberstab erneut, deutete damit auf den Radierer.

Für einen Augenblick zeigte sich Ärger auf Harrys Gesicht, doch er unternahm nichts, um sie aufzuhalten.

"\emph{Finite Incantatem,}" sagte Hermine. "Sprich mit Professor McGonagall, bevor du es nochmal versuchst."

Harry nickte, doch sein Gesicht war noch immer leicht verkniffen.

"Und wir müssen trotzdem aufhören," sagte Hermine.

"\emph{Warum?}" sagte Harry. "Verstehst du nicht, was das \emph{bedeutet,} Hermine? Die Zauberer wissen \emph{nicht} alles! Es gibt zu wenige von ihnen, noch weniger, die irgendwas von Wissenschaft verstehen, sie haben die niedrig hängenden Früchte noch nicht erschöpft -"

"Es ist nicht \emph{sicher,}" sagte Hermine. "Wenn wir neue Dinge herausfinden \emph{können,} ist es sogar noch \emph{weniger} sicher! Wir sind \emph{zu jung!} Wir haben schon einen großen Fehler gemacht, beim nächsten mal könnten wir einfach \emph{sterben!}"

Dann zuckte Hermine zusammen.

Harry wandte den Blick von ihr ab und atmete langsam und tief durch.

"Bitte versuch es nicht allein, Harry," sagte Hermine mit bebender Stimme. "Bitte."

\emph{Bitte zwing mich nicht, zu entscheiden, ob ich es Professor Flitwick sagen soll.}

Eine lange Pause entstand.

"Du willst also, dass wir lernen," sagte Harry. Sie spürte, wie er versuchte, den Ärger aus seiner Stimme zu verbannen. "Nur lernen."

Hermine war nicht sicher, ob sie etwas sagen sollte, doch… "Wie du, ähm, Zeitlose Physik gelernt hast, richtig?"

Harry blickte sie wieder an.

"Was du da gemacht hast," sagte Hermine zaghaft, "es hatte nichts mit \emph{unseren} Experimenten zu tun, nicht wahr? Du konntest es tun, weil du eine Menge Bücher gelesen hast."

Harry öffnete den Mund und schloss ihn dann wieder. Ein frustrierter Ausdruck lag auf seinem Gesicht.

"In Ordnung," sagte Harry. "Wie wär's damit. Wir lernen und wenn mir etwas einfällt, das \emph{wirklich} scheint, als wäre es einen Versuch wert, versuchen wir es, nachdem ich einen Professor frage."

"Okay," sagte Hermine. Sie fiel nicht um vor Erleichterung, aber nur weil sie bereits saß.

"Sollen wir zum Mittagessen?" sagte Harry vorsichtig.

Hermine nickte. Ja. Mittagessen klang gut. Wirklich, diesmal.

Sie begann sich vorsichtig vom Steinboden hoch zu stemmen, verzog das Gesicht, als ihr Körper protestierte -

Harry richtete den Zauberstab auf sie und sagte "\emph{Wingardium Leviosa.}"

Hermine blinzelte, als das massive Gewicht, das auf ihren Beinen lastete, zu etwas erträglichem schwand.

Ein schiefes Lächeln fuhr über Harrys Gesicht. "Man kann etwas \emph{heben,} ohne in der Lage zu sein, es tatsächlich schweben zu lassen," sagte er. "Erinnerst du dich an das Experiment?"

Hermine lächelte hilflos zurück, obwohl sie dachte, sie sollte immer noch sauer sein.

Und sie trat den Weg zurück zur Großen Halle an, fühlte sich merklich und wunderbar leicht auf den Füßen, während Harry sorgsam seinen Zauberstab auf sie gerichtet hielt.

Er konnte es nur für fünf Minuten aufrechterhalten, doch es war der Gedanke, der zählte.

\later

Minerva sah Dumbledore an.

Dumbledore blickte fragend zu ihr zurück. "Verstehen Sie irgendetwas davon?" sagte der Schulleiter und klang verwirrt.

Es war das vollkommenste und heilloseste Kauderwelsch gewesen, das Minerva sich erinnern konnte, jemals gehört zu haben. Es war ihr ein wenig peinlich, den Schulleiter herbeigerufen zu haben, um es zu hören, doch sie hatte explizite Anweisungen erhalten.

"Ich fürchte nicht," sagte Professor McGonagall steif.

"Also," sagte Dumbledore. Der silberne Bart schwang von ihr weg, als der funkelnde Blick des alten Zauberers einmal mehr andernorts weilte. "Du vermutest, du könntest etwas tun, zu dem andere Zauberer nicht in der Lage sind, etwas das wir für unmöglich erachten."

Die drei standen im privaten Transfigurations-Werkraum des Schulleiters, wohin der glänzende Phönix von Dumbledores Patronus sie beordert hatte, Augenblicke nachdem ihr eigener Patronus ihn erreichte. Licht schien herab durch die Oberlichter und erleuchtete das große sieben-zackige alchemistische Diagramm, das im Zentrum des kreisrunden Raumes aufgezeichnet war, offenbarte dass es ein wenig Staub angesetzt hatte, was Minerva traurig stimmte. Transfigurations-Forschung war eine von Dumbledores großen Freuden und sie hatte gewusst, dass er letztlich unter Zeitdruck gestanden hatte, doch nicht wie \emph{sehr.}

Und nun würde Harry Potter noch mehr von der Zeit des Schulleiters verschwenden. Doch das konnte sie sicherlich \emph{Harry} nicht zum Vorwurf machen. Er hatte das richtige getan, indem er zu ihr gekommen war, um zu berichten, ihm sei eine Idee für etwas gekommen, das in der Transfiguration bislang für unmöglich gehalten wurde und sie hatte ihrerseits exakt das getan, was ihr aufgetragen worden war: Sie hatte Harry angewiesen, Stillschweigen zu bewahren und nichts mit ihr zu besprechen, bis sie den Schulleiter kontaktiert hätte und sie zu einer sicheren Örtlichkeit gewechselt wären.

Hätte Harry von vornherein gesagt, was \emph{genau} er tun zu können glaubte, hätte sie sich nicht bemüht.

"Sehen Sie, ich weiß, es ist schwer zu erklären," sagte Harry ein wenig verlegen. "Es läuft darauf hinaus, dass was Sie glauben, mit dem in Konflikt steht, was Wissenschaftler glauben, in einem Fall, bei dem ich ehrlich erwarte, dass Wissenschaftler mehr wissen als Zauberer."

Minerva hätte einen lauten Seufzer ausgestoßen, wenn der Schulleiter die ganze Sache nicht sehr ernst zu nehmen schiene.

Harrys Idee entstammte simpler Unwissenheit, nichts weiter. Verwandelte man die Hälfte einer Metallkugel in Glas, hatte die \emph{ganze Kugel} anschließend eine andere Form. Den Teil zu verändern, \emph{hieß} das Ganze zu verändern und das bedeutete, die gesamte Form zu entfernen und durch eine andere zu ersetzen. Was würde es überhaupt \emph{bedeuten,} nur die Hälfte einer Metallkugel zu transfigurieren? Dass die Metallkugel \emph{als Ganzes} die selbe Form hatte wie zuvor, aber die \emph{Hälfte} der Kugel jetzt eine andere Form hatte?

"Mr. Potter," sagte Professor McGonagall, "was Sie tun möchten, ist nicht nur unmöglich, es ist \emph{unlogisch.} Wenn Sie die Hälfte von etwas verändern, \emph{haben} Sie das Ganze verändert."

"In der Tat," sagte Dumbledore. "Doch Harry ist der Held, also mag er in der Lage sein, Dinge zu bewerkstelligen, die logisch unmöglich sind."

Minerva hätte mit den Augen gerollt, wäre sie nicht seit langem schon dagegen abgestumpft.

"Angenommen, es \emph{wäre} möglich," sagte Dumbledore, "fällt Ihnen irgendein Grund ein, weshalb die Resultate sich in irgendeiner Weise von denen einer gewöhnlichen Transfiguration unterscheiden würden?"

Minerva runzelte die Stirn. Die Tatsache, dass das Konzept buchstäblich unvorstellbar war, stellte sie vor einige Probleme, doch sie versuchte es als gegeben hinzunehmen. Eine Transfiguration, die nur der Hälfte einer Metallkugel aufgeprägt wurde…

"Seltsame Dinge geschehen an der Verbindungsstelle?" sagte Minerva. "Doch das sollte sich nicht vom Transfigurieren des Objektes als Ganzes unterscheiden, in eine Form mit zwei verschiedenen Teilen…"

Dumbledore nickte. "Das war auch mein Gedanke. Und Harry, wenn deine Theorie korrekt ist, impliziert sie, dass was du tun willst \emph{genau} wie jede andere Transfiguration verläuft, nur auf einen Teil des Gegenstandes angewendet, anstatt auf das Ganze? \emph{Überhaupt} keine Änderungen?"

"Ja," sagte Harry bestimmt. "Nur darum geht es."

Dumbledore blickte wieder zu ihr. "Minerva, fällt Ihnen ein irgendwie gearteter Grund ein, weshalb das gefährlich sein sollte?"

"Nein," sagte Minerva, nachdem sie ihr Gedächtnis zu Ende durchstöbert hatte.

"Mir ebenso wenig," sagte der Schulleiter. "In Ordnung, dann, da dies in jeder Hinsicht exakt einer gewöhnlichen Transfiguration entsprechen sollte und da uns keinerlei Grund einfallen mag, wieso es gefährlich sein sollte, halte ich Sicherheitsvorkehrungen der zweiten Stufe für ausreichend."

Minerva war überrascht, doch sie wandte nichts dagegen ein. Dumbledore war bei weitem der Erfahrenere, was Transfiguration betraf und er hatte sich buchstäblich an tausenden neuer Transfigurationen versucht, ohne jemals eine zu niedrige Sicherheitsstufe gewählt zu haben. Er hatte Transfiguration \emph{im Kampf} eingesetzt und er war \emph{noch am Leben.} Wenn der Schulleiter die zweite Sicherheitsstufe für ausreichend hielt, dann war sie ausreichend.

Dass Harry mit Sicherheit scheitern würde, war, natürlich, vollkommen irrelevant.

Die beiden begannen die Schutz- und Aufspürzauber in Kraft zu setzen. Das wichtigste Sicherheitsnetz war dasjenige, welches sicherstellte, dass kein transfiguriertes Material in die Luft gelangt war. Harry würde in einem separaten Kraftfeld mit eigener Luftversorgung eingeschlossen, nur um sicher zu gehen, nur sein Zauberstab würde den Schild verlassen dürfen und der Übergang war eng gestrickt. Sie befanden sich in Hogwarts, also konnten sie nicht jegliches Material, das Anzeichen von spontaner Verbrennung zeigte, automatisch hinaus disapparieren, doch sie konnten es fast ebenso schnell zu einem der Oberlichter hinaus befördern, aus welchem Grund auch alle Fenster nach außen hin zu öffnen waren. Harry selbst würde beim ersten Anzeichen von Problemen zu einem anderen Oberlicht hinaus befördert.

Harry beobachtete ihre Arbeiten mit leicht verängstigtem Gesichtsausdruck.

"Seien Sie unbesorgt," sagte Professor McGonagall mitten in ihrer Beschreibung, "dies wird nahezu sicher nicht notwendig sein, Mr. Potter. \emph{Erwarteten} wir, dass etwas schief ginge, würde Ihnen nicht erlaubt, es zu versuchen. Es sind lediglich die üblichen Vorsichtsmaßnahmen für jede Transfiguration, die noch nie zuvor jemand versucht hat."

Harry schluckte und nickte.

Und wenige Minuten später war Harry an einen Sicherheitsstuhl geschnallt und sein Zauberstab ruhte auf der Metallkugel - welche, basierend auf seinen aktuellen Testresultaten, zu groß hätte sein sollen, als dass er sie in weniger als dreißig Minuten transfigurieren könnte.

Und wenige Minuten \emph{danach} lehnte Minerva an einer Wand und fühlte sich einer Ohnmacht nahe.

Ein kleiner Flecken Glas war auf der Kugel erschienen, wo Harrys Zauberstab geruht hatte.

Harry sagte nicht \emph{Ich hab's Ihnen doch gesagt,} doch der selbstzufriedene Ausdruck auf seinem verschwitzten Gesicht sprach für sich selbst.

Dumbledore wirkte Analyse-Zauber auf die Kugel, sah von Moment zu Moment erstaunter aus. Dreißig Jahre schienen von seinem Gesicht abgefallen.

"Faszinierend," sagte Dumbledore. "Genau wie er gesagt hat. Er hat einfach einen Teil des Gegenstandes transfiguriert, ohne das Ganze zu transfigurieren. Du sagst, es ist wirklich nur eine gedankliche Beschränkung, Harry?"

"Ja," sagte Harry, "aber eine tiefgreifende, nur zu wissen, dass es eine gedankliche Beschränkung sein musste, genügte nicht. Ich musste den Teil meines Geistes unterdrücken, der den Fehler beging und stattdessen an die zugrunde liegende Realität denken, die Wissenschaftler verstanden haben."

"Wahrlich faszinierend," sagte Dumbledore. "Wie ich verstanden habe, würde es jeden anderen Zauberer Monate des Studiums kosten, dasselbe zu tun, wenn er es überhaupt zustande brächte? Und darf ich dich bitten, noch andere Gegenstände partiell zu transfigurieren?"

"Wahrscheinlich schon und natürlich," sagte Harry.

Eine halbe Stunde später fühlte sich Minerva noch ebenso perplex, doch wesentlich beruhigter, was die Sicherheitsaspekte betraf.

Es \emph{war} das gleiche, abgesehen davon, dass es logisch unmöglich war.

"Ich glaube, das genügt, Schulleiter," sagte Minerva schließlich. "Ich vermute, partielle Transfiguration ist ermüdender als die gewöhnliche Variante."

"Mit der Übung immer weniger," sagte der erschöpfte und blasse Junge mit unsteter Stimme, "aber ja, da haben Sie recht."

Der Vorgang, Harry aus den Schutzzaubern heraus zu holen, brauchte eine weitere Minute und dann geleitete Minerva ihn zu einem sehr viel bequemeren Stuhl und Dumbledore brachte eine Eiskrem-Soda zum Vorschein.

"\emph{Meinen Glückwunsch,} Mr. Potter!" sagte Professor McGonagall und meinte es auch so. Sie hätte fast alles dagegen verwettet, dass das funktionierte.

"Herzlichen Glückwunsch, in der Tat," sagte Dumbledore. "Selbst ich habe noch keine originären Entdeckungen auf dem Gebiet der Transfiguration vor dem Alter von vierzehn gemacht. Seit den Tagen von Dorothea Senjak**** ist kein Genius so früh erblüht."

"Danke," sagte Harry und klang leicht überrascht.

"Nichtsdestotrotz," sagte Dumbledore nachdenklich, "hielte ich es für äußerst weise, dieses glückliche Ereignis als Geheimnis zu bewahren, zumindest einstweilen. Harry, hast du deine Idee mit irgendeiner anderen Person diskutiert, bevor du mit Professor McGonagall gesprochen hast?"

Es herrschte Stille.

"Ähm…" sagte Harry. "Ich will ja niemanden der Inquisition übergeben, aber ich habe es einer anderen Schülerin erzählt -"

Das Wort explodierte fast von Professor McGonagalls Lippen. "\emph{Was?} Sie haben eine völlig neue Form der Transfiguration mit einer \emph{Schülerin} diskutiert, bevor Sie eine anerkannte Autoritätsperson konsultiert haben? Haben Sie eine Vorstellung davon, wie \emph{verantwortungslos} das war?"

"Es tut mir leid," sagte Harry. "Es war mir nicht bewusst."

Der Junge sah angemessen verängstigt aus und Minerva spürte, wie sich etwas in ihr entspannte. Immerhin verstand Harry, wie töricht er gewesen war.

"Du musst Miss Granger zur Verschwiegenheit verpflichten," sagte Dumbledore ernst. "Und erzähle es niemandem sonst, wenn du keinen wirklich guten Grund dafür hast und er es nicht ebenfalls geschworen hat."

"Ah… warum?" sagte Harry.

Das fragte Minerva sich ebenfalls. Einmal mehr dachte der Schulleiter zu weit voraus, als dass sie hätte mithalten können.

"Weil du etwas tun kannst, von dem niemand sonst glauben wird, dass du dazu fähig bist," sagte Dumbledore. "Etwas vollkommen unerwartetes. Es mag sich als dein kritischer Vorteil erweisen, Harry, und wir müssen ihn bewahren. Bitte, vertrau mir in der Sache."

Professor McGonagall nickte, ihr gefasster Gesichtsausdruck offenbarte nichts von ihrer inneren Verwirrung. "Bitte tun Sie es, Mr. Potter," sagte sie.

"In Ordnung…" sagte Harry langsam.

"Sobald wir deine Aufzeichnungen vollständig ausgewertet haben," fügte Dumbledore hinzu, "ist es dir gestattet, dich in partieller Transfiguration zu üben, \emph{ausschließlich} von Glas zu Stahl und Stahl zu Glas, mit Miss Granger als Beobachterin. Selbstverständlich informiert ihr umgehend einen Professor, sollte irgendeiner von euch jeglichen Verdacht auf Symptome von Transfigurationskrankheit haben."

Unmittelbar bevor Harry den Werkraum verließ, mit der Hand schon am Türknauf, drehte sich der Junge noch einmal um und sagte, "Wo wir gerade hier sind, ist einem von Ihnen irgendeine Veränderung an Professor Snape aufgefallen?"

"Veränderung?" sagte der Schulleiter.

Minerva unterdrückte in schiefes Lächeln auf ihrem Gesicht. Natürlich war der Junge besorgt wegen des 'bösen Meisters der Zaubertränke', da er nicht wissen konnte, weshalb Severus vertrauenswürdig war. Es wäre gelinde gesagt seltsam gewesen, Harry zu erklären, dass Severus seine Mutter noch immer liebte.

"Ich meine, hat sich sein Verhalten in letzter Zeit irgendwie verändert?" sagte Harry.

"Nicht, dass es mir aufgefallen wäre…" sagte der Schulleiter langsam. "Warum fragst du?"

Harry schüttelte den Kopf. "Ich will ihren eigenen Beobachtungen nicht vorgreifen, indem ich etwas sage. Vielleicht halten Sie einfach die Augen offen?"

Das ließ einen Schauer des Unbehagens durch Minerva fahren, wie es keine direkte Anschuldigung gegen Severus vermocht hätte.

Harry verneigte sich vor beiden respektvoll und verabschiedete sich.

\later

"Albus," sagte Minerva, nachdem der Junge verschwunden war, "woher \emph{wussten} Sie, dass Sie Harry ernst nehmen sollten? Ich hätte diese Idee für schlichtweg unmöglich gehalten!"

Das Gesicht des alten Zauberers wurde ernst. "Aus dem selben Grund, weshalb es ein Geheimnis bleiben muss, Minerva. Dem selben Grund, weshalb ich Ihnen aufgetragen habe, zu mir zu kommen, sollte Harry irgendetwas in dieser Richtung andeuten. Weil es eine Macht ist, die Voldemort nicht kennt."

Es dauerte ein paar Sekunden, bis die Worte sackten.

Und dann wanderte der kalte Schauer ihren Rücken hinunter, wie immer wenn sie sich erinnerte.

Es hatte als gewöhnliches Bewerbungsgespräch begonnen, Sybill Trelawney bewarb sich um die Stelle als Professorin für Wahrsagen.

\emph{DER EINE MIT DER MACHT, DEN DUNKLEN LORD ZU BESIEGEN, NAHT HERAN,

JENEN GEBOREN, DIE IHM DREIMAL DIE STIRN GEBOTEN HABEN,

GEBOREN, WENN DER SIEBTE MONAT STIRBT,

UND DER DUNKLE LORD WIRD IHN ALS SICH EBENBÜRTIGEN KENNZEICHNEN,

ABER ER WIRD EINE MACHT BESITZEN, DIE DER DUNKLE LORD NICHT KENNT,

UND DER EINE MUSS DEN ANDEREN ZERSTÖREN, AUF DAS NICHT MEHR ALS EIN ÜBERREST VON IHM BLEIBE,

DENN JENE} \emph{ZWEI UNGLEICHENGEISTER} \emph{KÖNNEN NICHT AUSHARREN IN DER} \emph{SELBEN} \emph{WELT.}

Jene grauenvollen Worte*****, gesprochen mit jener furchtbaren, donnernden Stimme, schienen nicht zu passen auf etwas wie partielle Transfiguration.

"Nun, vielleicht nicht," sagte Dumbledore auf Minervas Versuch hin, zu erklären. "Ich gestehe, ich hatte auf etwas gehofft, das bei der Suche nach Voldemorts Horkrux helfen würde, wo immer er ihn versteckt haben mag. Aber…" Der alte Zauberer zuckte mit den Schultern. "Prophezeiungen sind eine verzwickte Angelegenheit, Minerva und es ist besser, kein Risiko einzugehen. Die kleinste Sache mag sich als entscheidend erweisen, wenn sie unerwartet bleibt."

"Und was, glauben Sie, meinte er wegen \emph{Severus?}" sagte Minerva.

"Da habe ich keine Ahnung," seufzte Dumbledore. "Es sei denn, Harry will gegen Severus vorgehen und glaubte, eine offene Frage könnte ernst genommen werden, wo ein direkter Vorwurf verworfen würde. Und sollte das tatsächlich geschehen sein, schloss Harry korrekt, dass ich nicht darauf vertrauen würde. Halten wir einfach Ausschau, ohne vorschnell zu urteilen, wie er sagt."

\later

\emph{Nachspiel, 1:}

"Ähm, Hermine?" sagte Harry mit sehr kleiner Stimme. "Ich glaube, ich schulde dir eine wirklich, wirklich, wirklich große Entschuldigung."

\later

\emph{Nachspiel, 2:}

Alissa Cornfoots Augen waren leicht glasig als sie den Meister der Zaubertränke betrachtete, wie er der Klasse eine strenge Predigt hielt, eine kleine bronzene Bohne empor haltend und irgendwas erzählend über schreiende Pfützen aus menschlichem Fleisch. Schon seit Beginn des Schuljahres hatte sie Schwierigkeiten, in Zaubertränke zuzuhören. Sie starrte weiter ihren furchtbaren, gemeinen, schmierigen Professor an und fantasierte von speziellen Stunden des Nachsitzens. Irgendetwas stimmte mit ihr wahrscheinlich \emph{ganz und gar nicht,} aber sie schien einfach nicht damit aufhören zu können -

"Au!" sagte Alissa dann.

Snape hatte die bronzene Bohne gerade zielsicher gegen Alissas Stirn schnipsen lassen.

"Miss Cornfoot," sagte der Meister der Zaubertränke mit schneidender Stimme, "dies ist ein heikler Zaubertrank und wenn Sie nicht aufpassen können, werden Sie Ihre Klassenkameraden verletzen, nicht nur sich selbst. Ich sehe Sie nach dem Unterricht."

Die letzten sechs Worte waren überhaupt nicht hilfreich, doch sie strengte sich noch mehr an und schaffte es, durch den Tag zu kommen, ohne irgendjemanden zu schmelzen.

Nach dem Unterricht trat Alissa vor den Schreibtisch. Ein Teil von ihr wollte einfach demütig mit beschämtem Gesichtsausdruck und reuevoll hinter dem Rücken verschränkten Händen dastehen, nur für den Fall, doch eine leise innere Stimme sagte ihr, das könnte eine \emph{schlechte Idee} sein. Also stand sie stattdessen einfach mit neutralem Gesichtsausdruck dort, mit für eine junge Dame sehr angemessener Haltung und sagte, "Professor?"

"Miss Cornfoot," sagte Snape, ohne von den Bögen, die er benotete, aufzusehen, "ich erwidere Ihre Zuneigungen nicht, ich beginne Ihr Starren als störend zu empfinden und Sie werden Ihre Blicke fortan im Zaum halten. Haben ich mich klar ausgedrückt?"

"Ja,„ sagte Alissa, es erklang als ersticktes Quieken und Snape entließ sie, worauf sie aus dem Klassenzimmer floh, mit Wangen flammend rot wie geschmolzene Lava.******

* Molekulare \emph{Assembler} (dt. etwa: „Zusammensetzer“) sind winzige Maschinen in Molekül-Größe, die Materie auf atomarer oder molekularer Ebene manipulieren können und so verschiedenste Strukturen und Produkte fertigen und sich ggf. auch selbst replizieren können. Die Möglichkeiten, die eine theoretisch unbegrenzte und exponentiell anwachsende Anzahl von Assemblern bietet, sind beinahe unbegrenzt. Ließen die Assembler sich programmieren und steuern, liefe das im Sinne von Harrys Zielen quasi auf eine „Programmiersprache für die Realität“ hinaus, was göttlichen Kräften so nahe käme, wie man es sich nur vorstellen könnte. Allerdings sind Assembler bei missbräuchlicher oder unsachgemäßer Nutzung auch eine vergleichsweise gefährliche Art der Nanotechnologie, da sie beispielsweise in einem sogenannten „Grey-goo“-Szenario (deut.: „Graue Schmiere“) sämtliche Materie eines Planeten (etwa der Erde) für ihre Selbstreplikation nutzen oder dessen Oberfläche komplett einebnen könnten, was für das darauf befindliche Leben offensichtlich fatal wäre. Durch ihre Fähigkeit zu exponentiellem Wachstum wären sie überdies in einem solchen Fall kaum aufzuhalten oder einzudämmen. \emph{Nanofabriken} stellen dem gegenüber eine weniger flexible, aber deutlich sicherere Lösung dar, da sie in der Regel spezialisierter, nicht mobil und nicht zur (vollständigen) Selbstreplikation fähig sind.

** Der Begriff \emph{Delta-V} bezeichnet das Ausmaß der Veränderung der Geschwindigkeit eines Objektes, also beispielsweise seine Beschleunigung und wird in der Raumfahrt auch zur Angabe der Manövrierfähigkeit von Raumfahrzeugen verwendet. Wie Harry schon sagt, ist das nötige Delta-V, um von einem geosynchronen Orbit aus einen beliebigen Ort im Sonnensystem zu erreichen, im Vergleich zum notwendigen Delta-V, um überhaupt erst aus dem „Gravitationsschacht“ der Erde in einen solchen Orbit „hinauf klettern“ zu können, relativ gering. Ein Weltraumlift würde es erlauben, das an ihm hinauf kletternde Fahrzeug, im Gegensatz zu einer Rakete, mit der zum Erreichen dieses Delta-V nötigen Energie von der Erdoberfläche oder einer geosynchronen Raumstation aus versorgen und je nach Konstruktionsweise beim Abstieg eventuell auch einen Teil der Energie wieder zurückgewinnen zu können. Allein dadurch, dass der Treibstoff für ihre energieaufwendigste Phase nicht mehr mitgeführt werden müsste, ergäbe sich also eine drastische Senkung der Kosten für Weltraummissionen, sowie eine deutliche Erhöhung ihrer möglichen Nutzlast.

*** Hier sollte angemerkt werden, dass der Autor selbstverständlich den Leser an dieser Stelle für die Konzepte der Quantenmechanik und der Zeitlosen Physik interessieren möchte, es aber auch ohne ein tiefgreifenderes Verständnis dieser möglich sein wird, der Geschichte zu folgen. Es sollte trotzdem ersichtlich sein, worauf der Autor hinaus möchte. Wer sich dafür interessiert, findet eine interessante Einleitung des Autors in englischer Sprache in dem LessWrong-Artikel \emph{The Quantum Physics Sequence} (http://lesswrong.com/lw/r5/the\_quantum\_physics\_sequence/). Außerdem sollte an dieser Stelle darauf hingewiesen werden, dass der Autor selbst kein Quantenphysiker ist und dieses Kapitel in der Vergangenheit (meiner Erinnerung nach) schon mehrmals (wohl seinem aktuellen Wissensstand nach) überarbeitet hat, von Ausführungen zur Stringtheorie, über die Schleifenquantengravitation, zum jetzigen Stand, welcher daher vielleicht auch nicht der Weisheit letzter Schluss sein mag. Das Konzept der Zeitlosen Physik ist (wie der Autor zu Beginn des nächsten Kapitels anmerken wird) auch noch keine etablierte wissenschaftliche Erkenntnis, sondern aktueller Stand des Autors.Der Autor geht in dem LessWrong-Artikel \emph{Timeless Physics} (http://lesswrong.com/lw/qp/timeless\_physics/), der Teil der \emph{Quantum Physics Sequence} ist, näher darauf ein. Ich könnte versuchen, es grob zu vereinfachen, aber dabei käme auf dem Platz einer Fußnote wohl nur grober Unsinn heraus. Will man es genauer wissen, wird man dann doch sein Englisch auffrischen müssen -- es lohnt sich auf jeden Fall!Aber wie schon gesagt: Die Geschichte funktioniert auch so.

**** Eine Figur aus den \emph{Black Company-}Romanen von Glen Cook (dt.: \emph{Die schwarze Schar}), beschrieben als eine außerordentlich mächtige Magierin, brillante Taktikerin und geborene Anführerin.

***** Für Interessierte gibt es auch hier von mir wieder den Original-Text (wer weiß, vielleicht findet manch einer noch eine versteckte Bedeutung, die mir entgangen ist…?):

\emph{THE ONE WITH THE POWER TO VANQUISH THE DARK LORD APPROACHES,

BORN TO THOSE WHO HAVE THRICE DEFIED HIM,

BORN AS THE SEVENTH MONTH DIES,

AND THE DARK LORD WILL MARK HIM AS HIS EQUAL,

BUT HE WILL HAVE POWER THE DARK LORD KNOWS NOT,

AND EITHER MUST DESTROY ALL BUT A REMNANT OF THE OTHER,

FOR THOSE TWO DIFFERENT SPIRITS CANNOT EXIST IN THE SAME WORLD.}

****** Bevor jemand fragt: \emph{Nein,} das wird auch nicht \emph{so ein} Fanfic, der Autor möchte mit dieser Szene auf etwas \emph{anderes} hinweisen und wir werden fortan nichts mehr von Miss Cornfoot lesen.

