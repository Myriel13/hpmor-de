

\hypertarget{die-todsuxfcnde}{% \section{38. Die Todsünde}\label{die-todsuxfcnde}}

\textbf{Kapitel 38: Die Todsünde}

Hell strahlend die Sonne und klar die Luft, bunt und fröhlich das Gewimmel der Schüler und ihrer Eltern, blitzblank das Pflaster von Gleis Neundreiviertel und noch tief stand die Wintersonne am Himmel um 9:45 Uhr am Morgen des 5. Januar 1992. Einige der jüngeren Schüler trugen Schals und Handschuhe, doch die meisten einfach nur ihre Umhänge, immerhin waren sie Zauberer.

Nachdem Harry den Ankunftsbereich verlassen hatte, nahm er Schal und Mantel ab, öffnete ein Fach seines Koffers und verstaute darin seine Wintersachen.

Einen langen Moment stand er einfach nur da und ließ sich beißen von der kalten Luft des Januars, nur um zu sehen wie es sich anfühlte.

Dann zog nahm Harry seinen Zaubererumhang heraus und warf ihn sich über.

Und schließlich zog Harry seinen Zauberstab und konnte nicht anders, als an seine Eltern zu denken, die er gerade zum Abschied geküsst hatte und an die Welt, deren Probleme er hinter sich ließ...

Mit einem seltsamen Schuldgefühl ob des Unvermeidlichen sprach Harry, „\emph{Thermos}."

Die Wärme durchströmte ihn.

Und der Junge-der-überlebt-hat war zurück.

Harry gähnte und streckte sich, zum Abschluss seiner Ferien fühlte er sich eher lethargisch als alles andere. Ihm war an diesem Morgen nicht danach, in seinen Lehrbüchern zu lesen oder auch nur nach ernsthafter Science Fiction; er brauchte jetzt etwas vollkommen sinnfreies, um sich zu beschäftigen...

Nun, das sollte nicht allzu schwer werden, sofern er bereit war sich von vier bronzenen Knuts zu trennen.

Außerdem mochten sich, wenn der \emph{Klitterer} das einzige konkurrierende Nachrichtenerzeugnis zum korrupten \emph{Tagespropheten} war, darin ja vielleicht doch einige unterdrückte echte Neuigkeiten finden.

Harry trottete zu dem gleichen Kiosk vom letzten Mal hinüber und fragte sich, ob der \emph{Klitterer} wohl die zuletzt gesehene Schlagzeile noch toppen konnte.

Der Verkäufer setzte zu einem Lächeln an als Harry näher kam, dann veränderte sich plötzlich der Ausdruck auf seinem Gesicht als er die Narbe bemerkte.

„\emph{Harry Potter?}“ keuchte der Verkäufer.

„Nein, Mr. Durian,“ erwiderte Harry mit kurzem Blick auf das Namensschild des Mannes, „nur eine gelungene Imitation -"

Dann blieb Harry die Stimme im Halse stecken als sein Blick auf die Frontseite des Klitterers fiel.

\emph{BESOFFENE SEHEHRIN SPUCKT'S AUS:\\ DUNKLER LORD WIRD ZURÜCKKEHREN,}

\hfill\break Für einen kurzen Augenblick versuchte Harry sich nichts anmerken zu lassen, bevor ihm klar wurde, dass nicht schockiert zu sein auf andere Weise ebenso verräterisch sein mochte -

„Verzeihen Sie,“ sagte Harry. Er klang ein wenig aufgewühlt und wusste nicht einmal, ob dies nun zu verräterisch war oder was für eine Reaktion er eigentlich zeigen müsste, wüsste er wirklich von nichts. Er hatte offenbar wirklich zu viel Zeit in der Gesellschaft von Slytherins verbracht und dabei vergessen, wie man etwas vor ganz normalen Leuten geheim hielt. Vier Knuts trafen auf den Tresen. „Eine Ausgabe des \emph{Klitterers}, bitte."

„Oh, schon in Ordnung, Mr. Potter!“ sagte der Verkäufer hastig und winkte ab. „Ist schon -- nur zu, nehmen Sie -"

Eine Zeitung kam durch die Luft geflogen, direkt in Harrys Finger und er entfaltete sie.

\emph{BESOFFENE SEHERIN SPUCKT'S AUS:\\ DUNKLER LORD WIRD ZURÜCKKEHREN,\\ HEIRATET DRACO MALFOY}

\hfill\break „Ist umsonst,“ sagte der Verkäufer, „für Sie, meine ich -"

„Nein,“ sagte Harry, „ich wollte ohnehin eine kaufen."

Der Verkäufer nahm die Münzen an sich und Harry las weiter.

„Sowas,“ meinte Harry eine halbe Minute später, „eine besoffene Seherin spuckt nach sechs Schluck Scotch ja \emph{so einige} seltsame Sachen aus.* Ich meine, wer hätte gedacht, dass Sirius Black und Peter Pettigrew insgeheim dieselbe Person sind?"

„Ich nicht,“ sagte der Verkäufer.

"Sie haben sogar ein Bild abgedruckt auf dem die beiden zusammen drauf sind, damit wir wissen, wer die beiden sind, die insgeheim dieselbe Person sind."

„Jepp,“ meinte der Verkäufer. „Ziemlich clevere Verkleidung, nich' wahr?"

"Und ich bin insgeheim fünfundsechzig Jahre alt."

„Sehen aber nicht halb so alt aus,“ meinte der Verkäufer schmeichlerisch.

"Und ich bin verlobt mit Hermine Granger \emph{und} Bellatrix Black \emph{und} Luna Lovegood und, oh ja, mit Draco Malfoy auch..."

„Wird sicher 'ne interessante Hochzeit,“ meinte der Verkäufer.

Harry blickte von der Zeitung auf und sinnierte fröhlich, „Wissen Sie, zuerst hörte ich, Luna Lovegood wäre wahnsinnig und fragte mich, ob sie es wirklich ist oder sich einfach nur irgendwas ausdenkt und insgeheim aus dem Lachen gar nicht mehr rauskommt. Als ich dann meine zweite \emph{Klitterer}-Schlagzeile gelesen habe, da hatte ich entschieden, sie könne \emph{unmöglich} wahnsinnig sein; ich meine, es kann nicht einfach sein sich sowas auszudenken, das kommt doch nicht einfach \emph{zufällig} aus einem raus. Und wissen Sie, was ich \emph{jetzt} glaube? Ich glaube, sie muss wohl doch wahnsinnig sein. Wenn normale Menschen versuchen sich irgendwas auszudenken, dann kommt nicht \emph{sowas} dabei raus. Irgendwas muss in deinem Kopf wirklich \emph{falsch} laufen, bevor \emph{das} dabei rauskommt, wenn du anfängst dir irgendwas auszudenken!"

Der Verkäufer starrte Harry an.

„Ernsthaft,“ fragte Harry. „Wer \emph{liest} dieses Zeug?"

„Na Sie,“ erwiderte der Verkäufer.

Daraufhin wanderte Harry mit seiner Zeitung davon.

Er setzte sich nicht an denselben nahegelegenen Tisch, an dem er vor seiner \emph{ersten} Fahrt mit diesem Zug zusammen mit Draco gesessen hatte. Das schien ihm nur die Geschichte zu provozieren, sich zu wiederholen.

Es lag nicht \emph{nur} daran, dass - dem Klitterer zufolge - Harrys erste Woche in Hogwarts etwa vierundfünfzig Jahre gedauert hatte. Harrys bescheidener Ansicht nach konnte sein Leben momentan einfach keine weiteren Kompliziertheiten mehr \emph{gebrauchen}.

Also suchte sich Harry anderswo einen kleinen eisernen Sitz aus, abseits der Menge und des gelegentlichen leisen Ploppens der Eltern, die mit ihren Kindern herein apparierten, ließ sich dort nieder und las den \emph{Klitterer}, um zu sehen ob er irgendwelche unterdrückten Nachrichten enthielt.

Abgesehen von den offensichtlichen Verrücktheiten (der Himmel helfe ihnen allen, wenn auch nur irgendetwas \emph{davon} stimmte), gab es noch eine ganze Reihe abfälliger romantischer Klatschgeschichten; doch nichts das von allzu großer \emph{Wichtigkeit} wäre, sollte es tatsächlich der Wahrheit entsprechen.

Gerade las Harry über ein neues Gesetzesvorhaben des Ministeriums, das alle Hochzeiten verbieten sollte, als -

„Harry Potter,“ erklang eine seidene Stimme, die Harrys Blut augenblicklich einen Adrenalinstoß versetzte.

Harry blickte auf.

„Lucius Malfoy,“ erwiderte Harry mit schwerer Stimme. Nächstes mal würde er so schlau sein, bis 10:55 Uhr draußen im Muggel-Teil von King's Cross warten.

Lucius neigte höflich den Kopf, wobei ihm sein langes weißes Haar über die Schultern fiel. Der Mann trug immer noch den gleichen Gehstock, schwarz lackiert mit einem silbernen Schlangenkopf als Griff und irgendetwas an der Art, wie er ihn hielt, drückte mit stiller Gewissheit aus \emph{dies} \emph{hier} \emph{ist eine tödliche Waffe} und nicht \emph{ich bin schwächlich} \emph{und stütze mich darauf}. Sein Gesicht war ausdruckslos.

Zwei Männer flankierten ihn, mit ständig suchendem Blick, die Zauberstäbe gesenkt in Händen. Die beiden bewegten sich wie ein Organismus mit vier Beinen und vier Armen - Crabbe-und-Goyle-Senior - und Harry meinte erraten zu können, welcher welcher war, es spielte allerdings nicht wirklich eine Rolle. Die zwei waren lediglich Lucius Anhängsel, so sicher als wären sie die beiden rechten Zehen an seinem linken Fuß.

„Ich bitte um Verzeihung für die Störung, Mr. Potter,“ sagte die seidig glatte Stimme. „Doch Sie haben auf keine meiner Eulen geantwortet und dies hier, so schien es mir, mag vielleicht meine einzige Gelegenheit sein, Sie zu treffen."

„Ich habe keine Ihrer Eulen erhalten,“ erwiderte Harry ruhig. „Dumbledore hat sie abgefangen, vermute ich. Doch ich hätte sie auch anderenfalls nicht beantwortet, es sei denn durch Draco. Meiner Ansicht nach wäre ein direkter Umgang mit Ihnen, ohne Dracos Wissen, ein Vertrauensbruch was unsere Freundschaft anbelangt.„

\emph{Bitte geh weg, geh einfach weg...}

Die grauen Augen funkelten ihn an. „So möchten Sie sich also geben...“ sagte der ältere Malfoy. „Nun denn. Ich spiele das Spielchen einstweilen mit. Was bezweckten Sie damit, Ihren guten Freund, meinen Sohn, zu einer öffentlichen Allianz mit diesem Mädchen zu bewegen?"

„Oh,“ sagte Harry leichthin, „das ist doch offensichtlich, nicht wahr? Dracos Zusammenarbeit mit Granger wird ihn erkennen lassen, dass Muggelgeborene doch richtige Menschen sind. Mua. Ha. Ha."

Ein dünnes Lächeln umspielte Lucius Lippen. „Ja, das klingt ganz nach einem von Dumbledores Plänen. Doch \emph{ist es keiner}."

„In der Tat,“ sagte Harry. „Es ist Teil meines Spiels mit Draco und kein Werk von Dumbledore und mehr werde ich dazu nicht sagen."

„Lassen wir die Spiele einmal beiseite,“ sagte der ältere Malfoy und plötzlich verhärteten sich die grauen Augen. „Wenn meine Vermutungen zutreffen, würden Sie Dumbledore ohnehin kaum je zu Willen sein, \emph{Mr. Potter}."

Kurz herrschte Stille.

„Also wissen Sie es,“ sagte Harry mit kalter Stimme. „Sagen Sie mir, wann genau wurde es Ihnen klar?"

„Als ich Ihre Antwort auf Professor Quirrells kleine Ansprache zu lesen bekam,“ sagte der weißhaarige Mann und lachte grimmig. „Zunächst war ich verwirrt, denn sie schien nicht in Ihrem Interesse zu liegen; ich brauchte mehrere Tage, um zu verstehen, wessen Interessen hier gedient wurde, doch dann wurde endlich alles klar. Und ebenso offensichtlich ist Ihre Schwäche, wenn auch nicht in mancher Hinsicht, so doch in anderer."

„Sehr schlau von Ihnen,“ sagte Harry, noch immer kalt. „Doch vielleicht verkennen Sie meine Interessen."

„Vielleicht tue ich das.“ Ein Hauch von Stahl schlich sich in die seidene Stimme. „In der Tat ist es genau das, was ich fürchte. Sie spielen seltsame Spiele mit meinem Sohn, zu einem Zweck, den ich nicht ermessen kann. Dies ist gewiss kein freundlicher Akt und meine Besorgnis können Sie mir wohl \emph{kaum} verdenken!"

Lucius lehnte sich nunmehr mit beiden Händen auf seinen Stock; weiß traten die Knöchel hervor und seine Leibwächter wirkten plötzlich angespannt.

Irgendein Instinkt in Harry hielt es für eine ausgesprochen schlechte Idee, in diesem Moment Angst zu zeigen und Lucius wissen zu lassen, dass er sich einschüchtern ließ. Ohnehin waren sie hier in einem öffentlichen Bahnhof -

„Ich finde es interessant,“ sagte Harry und legte nunmehr Stahl in seine Stimme, „dass Sie zu glauben scheinen, ich könne davon profitieren, Draco Schaden zuzufügen. Doch es ist unwichtig, Lucius. \emph{Er} ist mein Freund und ich verrate meine Freunde nicht."

„\emph{Was?}“ flüsterte Lucius. Auf seinem Gesicht zeigte sich blanker Schock.

Dann -

„Gesellschaft,“ sagte einer der Lakaien und anhand der Stimme glaubte Harry zu erkennen, es müsse der ältere Crabbe sein.

Lucius richtete sich auf, wandte sich um und stieß ein missbilligendes Zischen aus.

Ihnen näherte sich Neville, sichtbar verängstigt aber dennoch zielstrebig, im Schlepptau einer groß gewachsenen Frau, die keineswegs eingeschüchtert schien.

„Madam Longbottom,“ sagte Lucius in eisigem Ton.

„Mr. Malfoy,“ gab die Frau ebenso eisig zurück. „Belästigen Sie etwa unseren Harry Potter?„

Darauf entließ Lucius ein seltsam bitteres, bellendes Lachen. „Oh, ich denke doch eher nicht. Sie sind wohl gekommen, ihn vor mir zu beschützen?“ Der weißhaarige Schopf wandte sich Neville zu. „Und das dürfte dann wohl Mr. Potters treuer Lieutenant sein, der letzte Spross von Longbottom, Neville, selbsternannt von Chaos. Welch seltsame Pfade das Schicksal doch wählt. Manchmal denke ich, es müsse dem Wahnsinn verfallen sein."

Harry hatte nicht die geringste Ahnung, was er darauf erwidern sollte und Neville wirkte verwirrt und verängstigt.

„Ich bezweifle, dass es das Schicksal ist, das hier verrückt spielt,“ meinte Madam Longbottom und ihre Stimme nahm einen höhnischen Tonfall an. „Sie scheinen schlechter Stimmung zu sein, Mr. Malfoy. Hat die Ansprache unseres lieben Professor Quirrell sie ein paar Unterstützer gekostet?"

„Nun, es war ein recht scharfsinniger Verriss meiner Fähigkeiten,“ erwiderte Lucius kalt, „doch wohl nur erfolgreich bei jenen Narren, die glauben ich sei wahrhaftig ein Todesser gewesen."

„\emph{Was?}“ platzte Neville heraus.

„Ich stand unter dem \emph{Imperius}-Fluch, junger Mann,“ sagte Lucius, nun in erschöpftem Ton. „Der Dunkle Lord hätte wohl kaum erfolgreich Anhänger aus den Reihen der Reinblüter-Familien rekrutieren können, ohne die Unterstützung des Hauses Malfoy. Ich zögerte und so hat er sich schlicht meiner versichert. Auch seine eigenen Todesser wussten bis zum Ende hin nichts davon, daher das falsche Mal, das ich trage; wenngleich es mich ohne echtes Einverständnis nicht zu binden vermag. Einige der Todesser glauben noch immer, ich sei einer der ersten unter ihnen gewesen und um des Friedens der Nation willen lasse ich sie in dem Glauben, damit sie unter Kontrolle bleiben. Doch war ich nicht ein solcher Narr, jenem unglückseligen Abenteurer aus eigener Wahl zu folgen-"

„Ignoriere ihn einfach,“ sagte Madam Longbottom, sowohl an Harry als auch an Neville gerichtet. „Er muss den Rest seines Lebens den Ahnungslosen mimen, aus Angst vor der Aussage unter Veritaserum.“ Sie sagte es mit offensichtlicher Genugtuung.

Abfällig wandte Lucius ihr den Rücken zu und sich nun wieder an Harry. „Würden Sie diese Xanthippe** wohl auffordern, zu gehen, \emph{Mr. Potter?}"

„Ich denke, eher nicht,“ sagte Harry mit trockener Stimme. „Ich bevorzuge den Umgang mit jenem Teil von Haus Malfoy, der meinem Alter entspricht."

Daraufhin folgte lange nichts. Grüne Augen durchbohrten ihn forschend.

„Natürlich...“ sagte Lucius bedächtig. „\emph{Jetzt} erspüre ich den Narren. Die ganze Zeit über gaben Sie nur vor, Sie hätten Sie keine Ahnung worüber wir sprachen."

Harry erwiderte seinen Blick und blieb stumm.

Lucius hob seinen Stock um ein paar Zentimeter an und stieß ihn dann hart wieder zu Boden.

Die Welt verschwand in einem blassen Nebel, alle Geräusche verstummten und das Universum bestand nur noch aus Harry und Lucius und dem schlangenköpfigen Gehstock.

„Mein Sohn ist mein Herz,“ intonierte der ältere Malfoy, „das letzte von Wert, was ich auf dieser Welt noch besitze und so sage ich Ihnen dies im Geiste der Freundschaft: Sollte er zu Schaden kommen, so widme ich mein Leben der Vergeltung. Doch solange mein Sohn \emph{nicht} zu Schaden kommt, so wünsche ich Ihnen gutes Gelingen. Da Sie nichts mehr von mir verlangt haben, so will auch ich von Ihnen nichts mehr verlangen."

Dann verzog sich der blasse Nebel und enthüllte eine aufgebrachte Madam Longbottom, zurückgehalten von dem älteren Crabbe; nun den Zauberstab in Händen.

„Wie können Sie es \emph{wagen!}“ zischte sie.

Lucius dunkler Umhamg und sein weißes Haar umwehten ihn als er sich dem älteren Goyle zuwandte. „Wir kehren nach Malfoy Manor zurück."

Mit einem dreifachen Ploppen disapparierten sie und waren verschwunden.

Stille folgte.

„Gütiger \emph{Himmel,}“ sagte Madam Longbottom. „Was hatte das denn zu bedeuten?"

Hilflos zuckte Harry mit den Schultern. Dann blickte er zu Neville.

Auf Nevilles Stirn zeichneten sich Schweißperlen ab.

„Vielen Dank, Neville,“ sagte Harry. „Ich weiß deine Hilfe wirklich zu schätzen, Neville. Und jetzt, Neville, denke ich solltest du dich setzen."

„Ja, General,“ sagte Neville und anstatt sich zu einem der Sitze neben Harry zu begeben, ließ er sich einfach in sitzender Position aufs Pflaster fallen.

„Sie haben bei meinem Enkelsohn viele Veränderungen bewirkt,“ sagte Madam Longbottom. „Manche davon sagen mir zu, andere jedoch nicht."

„Schicken Sie mir eine Liste, welche welche sind,“ sagte Harry. „Ich werde sehen, was ich da machen kann.„

Neville grunzte, sagte aber nichts.

Madam Longbottom gluckste. „Das werde ich, junger Mann, vielen Dank.“ Sie senkte die Stimme. „Mr. Potter... die Ansprache von Professor Quirrell war etwas, dass unsere Nation schon seit langem hat hören müssen. Von Ihrem Kommentar darauf vermag ich das nicht zu sagen."

„Ich werde mir Ihren Ratschlag zu Herzen nehmen,“ meinte Harry versöhnlich.

„Das will ich sehr hoffen,“ erwiderte Madam Longbottom und wandte sich wieder ihrem Enkel zu. „Muss ich dich noch -"

„Du kannst ruhig gehen, Oma,“ sagte Neville. „Diesmal schaffe ich es allein."

„Na, \emph{das} lobe ich mir,“ sagte sie und verschwand mit einem Plop wie eine Seifenblase.

Einen Moment lang saßen die beiden Jungen still da.

Neville brach das Schweigen zuerst, Erschöpfung in der Stimme. „Du willst dich sicher um die Veränderungen kümmern, die ihr \emph{zusagen}, nicht wahr?"

„Nicht um \emph{alle} davon,“ meinte Harry mit Unschuldsmiene. „Ich will nur sichergehen, dass ich dich nicht allzu sehr verderbe."

--------------------------------------------------------------------------------------------------------------------------------------------

\hfill\break Draco wirkte äußerst besorgt. Immerzu fuhr er mit dem Kopf herum, trotzdem Draco bereits darauf bestanden hatte, dass sie in Harrys Koffer hinunter gingen und diesmal einen echten Stillezauber verwendeten, nicht nur die geräuschverzerrende Barriere.

„\emph{Was} hast du zu Vater gesagt?“ platzte Draco sofort heraus als der Stillezauber einsetzte und die Geräusche von Gleis Neundreiviertel verstummten.

„Ich... hör mal, kannst du mir sagen, was er dir gesagt hat, bevor er dich abgesetzt hat?“ fragte Harry.

„Dass ich es ihm sofort sagen soll, falls es scheint als würdest du mich bedrohen,“ sagte Draco. „Dass ich es ihm sofort sagen soll, falls \emph{ich} irgendetwas tue, dass für \emph{dich} eine Bedrohung könnte! Vater hält dich für \emph{gefährlich}, Harry, was immer du ihm heute gesagt hast, es hat ihm \emph{Angst} gemacht! \emph{Es ist keine gute Idee, Vater Angst zu machen!}"

\emph{Oh, zur Hölle...}

„\emph{Worüber} habt ihr gesprochen?“ verlangte Draco.

Harry lehnte sich schwer in dem kleinen Klappstuhl zurück, der am Boden seines Koffers stand. „Weißt du, Draco, genau wie es die Grundlage aller Rationalität ist, zu fragen 'Was denkst du, das du weißt und woher denkst du, weißt du es?', gibt es dabei auch eine Todsünde, eine Art zu denken, die das Gegenteil davon ist. Wie die griechischen Philosophen der Antike. Sie hatten überhaupt keine Ahnung, was vor sich ging, also sagten sie andauernd solche Sachen wie 'Alles ist Wasser' oder 'Alles ist Feuer' und sie fragten sich nie 'Moment mal, selbst wenn tatsächlich alles Wasser \emph{ist}, woher will ich das eigentlich \emph{wissen?}' Sie fragten sich nicht, ob sie irgendwelche Belege hatten, die \emph{diese} Möglichkeit von all den \emph{anderen} vorstellbaren Möglichkeiten unterschieden, Belege die ihnen sehr wahrscheinlich nicht unterkommen dürften, wenn ihre Theorie \emph{nicht} stimmte -"

„\emph{Harry,}“ wiederholte Draco mit angespannter Stimme, „\emph{Worüber hast du mit Vater gesprochen?}"

„Tatsächlich weiß ich das gar nicht so genau,“ sagte Harry, „also ist es jetzt sehr wichtig, dass ich mir \emph{nicht} einfach irgendetwas zusammenreime -"

Einen so hohen und lauten Schreckensschrei hatte Harry von Draco noch nie gehört.

* Versucht das dreimal ganz schnell zu sagen!\\ ** engl. \emph{harridan}, könnte auch mit \emph{alter Drachen} übersetzt werden, abwertende Bezeichnung für eine zänkische alte Dame; Xanthippe war historisch die Ehefrau des Sokrates und gilt laut vieler Anekdoten quasi als Prototyp einer launischen und streitsüchtigen Ehefrau.

