

\hypertarget{die-falschen-fragen-stellen}{% \section{13. Die falschen Fragen stellen}\label{die-falschen-fragen-stellen}}

\textbf{Kapitel 13: Die falschen Fragen stellen\\ }

Elen sila J. K. Rowling omentielvo.

EDIT: Keine Panik. Ich schwöre feierlich, dass es eine logische, vorausschauende, Kanon-kompatible Erklärung für alles gibt, was in diesem Kapitel geschieht. Es ist ein Rätsel, ihr sollt versuchen es zu lösen und schafft ihr es nicht, lest einfach das nächste Kapitel.

--------------------------------------------------------------------------------------------------------------------------------------------

\emph{"Solcheinoffensichtliches Rätsel habe ich selten gehört."}

--------------------------------------------------------------------------------------------------------------------------------------------

Gleich als Harry die Augen öffnete, am Morgen seines ersten vollen Schultages in Hogwarts im Jungen-Schlafsaal der Erstklässler von Ravenclaw, wusste er, dass etwas nicht stimmte.

Es war still.

\emph{Zu} still.

Ach ja, richtig… Da war ein Quietus-Zauber am Kopfende seines Bettes, der sich mit einem kleinen Schieberegler kontrollieren ließ, was die einzige Möglichkeit darstellte, wie in Ravenclaw jemals irgendwer einschlafen konnte.

Harry setzte sich auf und schaute sich um, in der Erwartung die anderen zu sehen, die sich für den Tag fertig machten -

Der Schlafsaal, leer.

Die Betten, zerwühlt und ungemacht.

Die Sonne, schien aus einem ziemlich hohen Winkel herein.

Sein Quietus-Zauber bis zum Anschlag aufgedreht.

Und seine mechanische Uhr lief immer noch, aber der Wecker war abgeschaltet.

Man hatte ihn offenbar bis 9:52 Uhr schlafen lassen. Trotzdem er sich nach Kräften bemüht hatte, seinen 26-Stunden-Schlafzyklus seiner Ankunft in Hogwarts anzupassen, hatte er letzte Nacht nicht vor 1 Uhr morgens einschlafen können. Er hatte vorgehabt, um 7 Uhr mit den anderen Schülern aufzuwachen, an seinem ersten Tag ein wenig verschlafen zu sein konnte er verkraften, solange das vor dem morgigen Tag irgendwie auf magische Weise behoben wurde. Aber jetzt hatte er das Frühstück verpasst. Und seine erste Unterrichtsstunde in Hogwarts, in Kräuterkunde, hatte vor einer Stunde und zweiundzwanzig Minuten angefangen.

Langsam, ganz langsam regte sich der Ärger in ihm. Ach, was für ein toller kleiner Streich. Seinen Wecker abzustellen. Den Quietus-Zauber aufzudrehen. So dass der Große Harry Potter seine erste Stunde verpasste und als Langschläfer verschrien würde.

Wenn Harry herausbekam, wer das getan hatte…

Nein, das wäre nur durch die Zusammenarbeit aller zwölf anderen Jungen im Ravenclaw-Schlafsaal möglich gewesen. Sie alle hätten seine schlafende Gestalt bemerkt. Sie alle hatten ihn das Frühstück verschlafen lassen.

Der Ärger verebbte, stattdessen war er nun verwirrt und fühlte sich furchtbar verletzt. Sie hatten ihn doch \emph{gemocht.} Hatte er gedacht. Letzte Nacht, da hatte er geglaubt, sie würden ihn mögen. \emph{Warum…}

Als Harry aus dem Bett stieg, erblickte er ein Stück Papier, das aus dem Kopfteil ragte.

Auf dem Papier stand:

\emph{An meine Mit-Ravenclaws,}

\emph{Es war ein wirklich langer Tag. Bitte lasst mich ausschlafen und macht euch keine Sorgen, dass ich das Frühstück verpasse. Ich habe meine erste Stunde nicht vergessen.}

\emph{Euer,\\ Harry Potter.}

Und Harry stand dort wie erstarrt, Eiswasser kroch in seine Venen.

Das Papier war in seiner Handschrift geschrieben, mit seinem eigenen mechanischen Bleistift.

Und er erinnerte sich nicht daran, es geschrieben zu haben.

Und… Harry schielte auf das Stück Papier. Und wenn er sich nicht täuschte, waren die Worte "nicht vergessen" in anderem Stil geschrieben, als versuchte er sich selbst etwas mitzuteilen…?

Hatte er \emph{gewusst,} dass sein Gedächtnis gelöscht werden würde? War er lange aufgeblieben und hatte irgendeine Art Verbrechen begangen oder eine verdeckte Operation durchgeführt und dann… aber er \emph{beherrschte} den Obliviate-Zauber nicht… hatte jemand anders… was…

Harry kam ein Gedanke. Wenn er \emph{gewusst} hatte, dass sein Gedächtnis gelöscht würde…

Noch immer im Pyjama stürmte Harry um sein Bett herum zu seinem Koffer, presste den Daumen gegen das Schloss, zog den Beutel heraus, steckte die Hand hinein und sagte "Notiz an mich selbst."

Und ein weiteres Stück Papier sprang in seine Hand.

Harry nahm es heraus und starrte es an. Es war ebenfalls in seiner Handschrift verfasst.

Die Notiz besagte:

\emph{An Mich,}

\emph{Bitte spiele das Spiel. Du kannst das Spiel nur einmal im Leben spielen. Dies ist eine einmalige Gelegenheit.}

\emph{Erkennungscode 927, ich bin eine Kartoffel.}

\emph{Dein,\\ Du.}

Harry nickte langsam. "Erkennungscode 927, ich bin eine Kartoffel" war in der Tat die Nachricht, die er sich - einige Jahre zuvor beim Fernsehen - im Vorhinein hatte einfallen lassen und die nur er kennen würde. Falls er prüfen musste, ob eine Kopie seiner selbst wirklich \emph{er} war oder sowas. Nur für den Fall. Sei vorbereitet.

Harry konnte der Nachricht nicht \emph{trauen,} es mochten noch andere Zauber im Spiel sein. Doch einen simplen Streich schloss sie zumindest aus. Er hatte das definitiv geschrieben und er erinnerte sich definitiv nicht daran, es geschrieben zu haben.

Während er auf das Papier starrte, bemerkte Harry Tinte, die von der anderen Seite hindurch schien.

Er drehte es um.

Auf der Rückseite stand:

\emph{ANWEISUNGEN FÜR DAS SPIEL:}

\emph{die Regeln des Spiels weißt du nicht\\ den Einsatz des Spiels weißt du nicht\\ den Zweck des Spiels weißt du nicht\\ wer das Spiel kontrolliert weißt du nicht\\ wie man das Spiel beendet weißt du nicht}

\emph{Du beginnst mit 100 Punkten.\\ Fang an.}

Harry starrte die "Anweisungen" an. Diese Seite war nicht handgeschrieben; die Schrift war vollkommen ebenmäßig, von daher künstlich. Sie sah aus als stammte sie von einer Flotte-Schreibe-Feder, wie Harry sie gekauft hatte, um Dinge zu diktieren.

Er hatte \emph{absolut keine Ahnung,} was hier vor sich ging.

Nun… der erste Schritt war, sich anzuziehen und etwas zu essen. Vielleicht in umgekehrter Reihenfolge. Seine Magen fühlte sich reichlich leer an.

Das Frühstück hatte er natürlich verpasst, aber darauf war er vorbereitet, da er diese Eventualität vorhergesehen hatte. Harry steckte die Hand in seinen Beutel und sagte "Imbiss", in der Erwartung, die Schachtel mit Müsliriegeln zu erhalten, die er vor seiner Abreise nach Hogwarts gekauft hatte.

Was da auftauchte, fühlte sich nicht nach einer Schachtel Müsliriegel an.

Als Harry die Hand in sein Sichtfeld führte, erblickte er zwei winzige Schokoriegel - nicht annähernd genug für eine Mahlzeit - an denen eine Nachricht befestigt war und die Nachricht war in der selben Schrift verfasst, wie die Spielanweisungen.

Die Notiz besagte:

VERSUCH FEHLGESCHLAGEN: -1 PUNKT\\ AKTUELLER PUNKTESTAND: 99\\ KÖRPERLICHE VERFASSUNG: IMMER NOCH HUNGRIG\\ GEISTIGE VERFASSUNG: VERWIRRT

"Gnäähhhhh" sagte Harrys Mund ohne jegliche bewusste Beteiligung oder Entscheidung seinerseits.

Etwa eine Minute stand er so da.

Eine Minute später machte es \emph{immer noch} keinen Sinn und er hatte \emph{noch immer} keine Ahnung, was hier vor sich ging und sein Hirn hatte noch nicht einmal \emph{begonnen,} auch nur irgendwelche \emph{Hypothesen} aufzustellen, als wären seine geistigen Hände gerade in Watte gepackt und könnten nichts erfassen.

Sein Magen, der seine eigenen Prioritäten hatte, schlug ein mögliches Experiment vor.

"Ah…" sagte Harry zu dem leeren Raum. "Ich nehme nicht an, ich könnte einen Punkt dazu verwenden, meine Müsliriegel-Schachtel zurück zu bekommen?'

Nichts als Stille.

Harry steckte die Hand in seinen Beutel und sagte "Müsliriegel-Schachtel."

Eine Schachtel mit etwa der richtigen Form sprang in seine Hand… doch sie war zu leicht und bereits geöffnet und leer und die an ihr befestigte Notiz besagte:

PUNKTE VERBRAUCHT: 1\\ AKTUELLER PUNKTESTAND: 98\\ DU ERHÄLTST: EINE MÜSLIRIEGEL-SCHACHTEL

"Ich würde gern einen Punkt dazu verwenden, \emph{die Müsliriegel selbst} wieder zu bekommen," sagte Harry.

Erneute Stille.

Harry steckte seine Hand in den Beutel und sagte "Müsliriegel".

Nichts tauchte auf.

Harry zuckte verzweifelt mit den Schultern und ging hinüber zu dem ihm zugeteilten Schrank neben seinem Bett, um seinen Zauberumhang für den Tag zu holen.

Auf dem Boden des Schrankes, unter seinem Umhang, lagen die Müsliriegel und eine Notiz:

PUNKTE VERBRAUCHT: 1\\ AKTUELLER PUNKTESTAND: 97\\ DU ERHÄLTST: 6 MÜSLIRIEGEL\\ DU TRÄGST IMMER NOCH: PYJAMAS\\ ISS NICHT WÄHREND DU DEINEN PYJAMA TRÄGST\\ SONST ERHÄLTST DU EINE PYJAMA-STRAFE

\emph{Und jetzt weiß ich, dass wer immer das Spiel kontrolliert, verrückt ist.}

"Ich vermute, dass Dumbledore das Spiel kontrolliert," sprach Harry laut. Vielleicht konnte er \emph{diesmal} einen neuen Landgeschwindigkeitsrekord fürs schnell von Begriff sein aufstellen.

Stille.

Doch Harry erkannte langsam ein Muster; die Notiz würde am dem Ort sein, wo er als nächstes nachschaute. Also guckte Harry unter sein Bett.

HA! HA HA HA HA HA!\\ HA HA HA HA HA HA!\\ HA! HA! HA! HA! HA! HA!\\ DUMBLEDORE KONTROLLIERT DAS SPIEL NICHT\\ SCHLECHT GERATEN\\ GANZ SCHLECHT GERATEN\\ -20 PUNKTE\\ UND DU TRÄGST IMMER NOCH PYJAMAS\\ DAS IST DEIN VIERTER ZUG\\ UND DU TRÄGST IMMER NOCH PYJAMAS\\ PYJAMA-STRAFE: -2 PUNKTE\\ AKTUELLER PUNKTESTAND: 75

Okay, na das war ja eine harte Nuss. Es war erst sein erster Schultag und wenn man Dumbledore ausschloss, war ihm niemand sonst hier bekannt, der so verrückt wäre.

Mehr oder weniger auf Autopilot sammelte Harry einen Umhang und etwas Unterwäsche zusammen, zog das Kellergeschoss seines Koffers heraus (er schätzte seine Privatsphäre und jemand könnte den Schlafsaal betreten), zog sich an und stieg dann wieder hinauf, um seinen Pyjama wegzuräumen.

Harry hielt inne bevor er die Schublade mit seinen Pyjamas heraus zog. Wenn sich das Muster hier als richtig erwies…

"Wie kann ich mehr Punkte verdienen?" sagte Harry laut.

Dann zog er die Schublade heraus.

GELEGENHEITEN UM GUTES ZU TUN GIBT ES ÜBERALL\\ DOCH FINSTERNIS IST WO DAS LICHT SEIN MUSS\\ KOSTEN DER FRAGE: 1 PUNKT\\ AKTUELLER PUNKTESTAND: 74\\ NETTE UNTERWÄSCHE\\ HAT DEINE MUTTER DIE AUSGESUCHT?

Harry zerknüllte die Notiz in seiner Hand, mit flammend rotem Gesicht. Dracos Fluch kam ihm wieder in den Sinn. \emph{Sohn eines Schlammbluts -}

An diesem Punkt war er allerdings schlauer, als das laut zu sagen. Er bekäme wahrscheinlich eine Strafe für Obszönitäten aufgedrückt.

Harry bewaffnete sich mit seinem Eselsfell-Beutel und seinem Zauberstab. Er zog die Verpackung von einem der Müsliriegel ab und warf sie in den Mülleimer des Zimmers, wo sie auf einem kaum angerührten Schokofrosch, einem zerknitterten Umschlag und etwas rot-grünem Einwickelpapier landete. Die anderen Müsliriegel steckte er in seinen Eselsfell-Beutel.

In einem letzten verzweifelten und letztlich vergeblichen Versuch, irgendeinen Hinweis zu entdecken, sah er sich noch einmal um.

Dann verließ Harry den Schlafsaal, im Gehen essend, auf der Suche nach den Verliesen von Slytherin. Zumindest \emph{glaubte} er, dass es das sei, worauf jener Satz hinwies.

Sich in den Fluren von Hogwarts zurechtzufinden, war wie… wahrscheinlich \emph{nicht} genau so schlimm, wie durch ein Escher-Gemälde zu spazieren; solche Dinge sagte man um des rethorischen Effekts willen, nicht ihres Wahrheitsgehaltes wegen.

Nach kurzer Zeit kam Harry zu dem Schluss, dass ein Escher-Gemälde verglichen mit Hogwarts tatsächlich sowohl Vor- als auch Nachteile bot. Nachteil: Keine konsistente Ausrichtung der Schwerkraft. Vorteil: Zumindest bewegten die Treppen sich nicht, \emph{WÄHREND MAN NOCH AUF IHNEN DRAUF STAND.}

Harry hatte ursprünglich vier Treppenfluchten erklommen, um zu seinem Schlafsaal zu gelangen. Nachdem er nicht weniger als zwölf Treppen hinunter geklettert war, ohne auch nur in die Nähe der Verliese zu kommen, gelangte Harry zu dem Schluss, dass (1) ein Escher-Gemälde im Vergleich dazu ein \emph{Zuckerschlecken} wäre, (2) er sich jetzt aus irgendeinem Grund \emph{höher} im Schloss befand als zu Anfang und (3) er sich dermaßen \emph{gründlich} verirrt hatte, dass es ihn nicht überrascht hätte, aus dem nächsten Fenster zu schauen und zwei Monde am Himmel zu erblicken.

Backup-Plan A war gewesen, anzuhalten und nach dem Weg zu fragen, doch es schien ein erheblicher Mangel an umherwandernden Leuten zu herrschen, fast so als würden die Burschen alle den Unterricht besuchen, so wie sie es sollten.

Backup-Plan B…

"Ich habe mich verlaufen," sagte Harry laut. "Kann mir, ähm, vielleicht der Geist von Hogwarts helfen oder so?"

"Ich glaube nicht, dass dieses Schloss über sehr viel Geist verfügt," bemerkte eine verwelkte alte Dame in einem der Gemälde an der Wand. "Leben, vielleicht, doch keinen Geist."

Eine kurze Pause entstand.

"Sind Sie -" sagte Harry, dann schloss er den Mund. Wenn er es näher bedachte, würde er NEIN das Gemälde nicht fragen, ob es über ein echtes Bewusstsein verfügte, in dem Sinne sich seines Bewusstseins bewusst zu sein.

"Ich bin Harry Potter," sagte sein Mund, mehr oder weniger auf Autopilot. Ebenso mehr oder weniger automatisch streckte Harry dem Gemälde die Hand entgegen.

Die Frau in dem Bild blickte auf Harrys Hand hinab und zog die Augenbrauen hoch.

Langsam sank die Hand wieder zurück an Harrys Seite.

"Sorry," sagte Harry, "ich bin noch neu hier."

"Soviel kann ich sehen, junger Rabe. Wohin versuchen Sie denn zu gelangen?"

Harry zögerte. "Ich bin nicht ganz sicher," sagte er.

"Dann sind Sie vielleicht bereits da."

"Nun, wo auch immer ich hin \emph{will,} ich glaube \emph{hier} ist es nicht…"* Harry schloss den Mund als er merkte, wie idiotisch er sich anhörte. "Ich fang nochmal an. Ich spiele da dieses Spiel, nur dass ich die Regeln nicht kenne -" Das funktionierte wohl auch nicht wirklich. "Okay, dritter Versuch. Ich suche nach Gelegenheiten, um Gutes zu tun und habe nur diesen kryptischen Hinweis, dass Finsternis ist, wo das Licht sein muss, also habe ich versucht, nach unten zu gelangen, aber ich scheine immer weiter hinauf zu kommen…"

Die alte Dame in dem Gemälde blickte ihn ziemlich skeptisch an.

Harry seufzte. "Mein Leben neigt ein wenig zur Merkwürdigkeit."

"Wäre es zutreffend, zu sagen, dass Sie weder wissen, wo Sie hin wollen, noch wieso Sie dorthin zu gelangen versuchen?"

"\emph{Vollkommen} zutreffend."

Die alte Dame nickte. "Ich bin nicht sicher, ob sich verlaufen zu haben Ihr größtes Problem ist, junger Mann."

"Das ist wahr, aber anders als die wichtigeren Probleme ist es ein Problem, von dem ich weiß, wie es zu lösen ist und \emph{wow} wird diese Unterhaltung gerade zu einer Metapher für die menschliche Existenz, das ist mir bis gerade eben gar nicht aufgefallen."

Die Dame betrachtete Harry abschätzend. "Sie sind also \emph{doch} ein feiner junger Rabe, nicht wahr? Einen Moment lang habe ich mich schon gewundert. Nun denn, eine allgemeine Regel, solange Sie immer links abbiegen, gelangen Sie sicher nach unten."

Das klang merkwürdig vertraut, doch Harry konnte sich nicht erinnern, wo er es schon einmal gehört hatte. "Ähm… Sie scheinen eine sehr intelligente Person zu sein. Oder das Bild einer sehr intelligenten Person… jedenfalls, haben Sie schon einmal von einem mysteriösen Spiel gehört, das man nur einmal spielen kann, aber keiner erklärt einem, wie die Regeln sind?"

"Das Leben," sagte die Dame sofort. "Solch ein offensichtliches Rätsel habe ich selten gehört."

Harry blinzelte. "Nein," sagte er langsam. "Ich meine, ich habe tatsächlich eine Nachricht bekommen und alles, die sagt, ich müsse das Spiel spielen, aber ich würde die Regeln nicht erfahren und jemand hinterlässt mir kleine Zettel, wie viele Punkte ich für Regelverstöße verloren habe, wie eine Minus-zwei-Punkte-Strafe fürs Pyjama tragen. Kennen Sie jemanden hier in Hogwarts, der verrückt und mächtig genug ist, so etwas zu tun? Außer Dumbledore, meine ich?"

Das Bild der alten Dame seufzte. "Ich bin nur ein Abbild, junger Mann. Ich erinnere mich an Hogwarts wie es war - nicht Hogwarts wie es ist. Ich kann Ihnen nur sagen, wenn dies ein Rätsel wäre, so wäre die Antwort, dass das Spiel das Leben ist und obgleich wir nicht alle Regeln selbst gestalten, sind es immer wir selbst, die Punkte vergeben oder nehmen. Wenn es kein Rätsel sondern Realität ist - dann weiß ich es nicht."

Harry verbeugte sich sehr tief vor dem Bild. "Dank Ihnen, Milady."

Die Dame vollführte vor ihm einen Knicks. "Ich wünschte ich könnte sagen, ich würde Sie in guter Erinnerung behalten," sagte sie, "doch ich werde mich wahrscheinlich überhaupt nicht an Sie erinnern. Leben Sie wohl, Harry Potter."

Er verneigte sich zur Antwort noch einmal und machte sich auf, die nächste Treppenflucht hinab zu steigen.

Vier linke Abzweigungen später fand er sich am Beginn eines Korridors wieder, der zur anderen Seite abrupt in einem herabgestürzten Haufen großer Felsen endete - als hätte es einen Höhleneinsturz gegeben, nur dass die umgebenden Mauern und die Decke intakt waren und aus völlig gewöhnlichen Steinen des Schlosses bestanden.

"Okay," sprach Harry zur leeren Luft, "ich geb's auf. Ich bitte um einen weiteren Hinweis. Wie komme ich dorthin, wo ich hin muss?"

"Ein Hinweis! Ein Hinweis sagst du?"

Die aufgeregte Stimme entstammte einem Gemälde an einer nicht weit entfernten Wand, dieses ein Porträt eines Mannes in mittleren Jahren im schreiend-pinksten Umhang, den Harry je gesehen hatte oder sich auch nur hätte vorstellen können. In dem Porträt trug er einen schlaffen alten Spitzhut mit einem Fisch darauf (keiner Zeichnung eines Fisches, wohlgemerkt, sondern einem Fisch).

"Ja!" sagte Harry. "Ein Hinweis! Ein Hinweis sage ich! Bloß nicht \emph{irgendein} Hinweis, ich suche nach einem \emph{ganz bestimmten} Hinweis, er ist für ein Spiel, das ich spiele -"

"Ja, ja! Ein Hinweis für das Spiel! Du bist Harry Potter, nicht wahr? Ich bin Cornelion Flubberwalt! Es wurde mir von Erin dem Gefährten gesagt, dem es wiederum Lord Wieselnas erzählt hat, dem es berichtet wurde von, ich hab's wirklich vergessen. Aber es war eine Nachricht für \emph{mich,} die ich dir geben soll! Für \emph{mich!} Um mich hat sich niemand mehr geschert seit, ich weiß nicht wie lang, vielleicht schon ewig, ich habe hier unten festgesteckt in diesem völlig nutzlosen alten Korridor - ein Hinweis! Ich habe deinen Hinweis! Er wird dich nur drei Punkte kosten! Willst du ihn?"

"Ja! Ich will ihn!" Harry war sich bewusst, dass er seinen Sarkasmus wahrscheinlich besser zügeln sollte, doch er konnte einfach nicht an sich halten.

Die Finsternis ist zu finden zwischen den grünen Studierzimmern und McGonagalls Transfigurationsklasse! Das ist der Hinweis! Und gib mal ein bisschen Gas, du bist langsamer als ein Sack Schnecken! Minus zehn Punkte fürs rumtrödeln! Jetzt hast du noch 61 Punkte! Das war der Rest der Nachricht!"

"Danke," sagte Harry. Langsam lag er in dem Spiel wirklich zurück. "Ähm… ich nehme nicht an, dass Sie wissen wo die Nachricht \emph{ursprünglich} herstammt, oder?"

"Sie wurde gesprochen von einer hohlen Stimme, die aus einer Leere in der Luft selbst erklang, einer Kluft die sich öffnete über einem feurigen Abgrund! Das hat man mir erzählt!"

Harry war nicht einmal mehr sicher, ob er solchen Dinge noch mit Skepsis begegnen oder sie einfach als gegeben hinnehmen sollte. "Und wo finde ich den Gang zwischen den grünen Studierzimmern und der Transfigurationsklasse?"

"Dreh dich einfach um und geh links, rechts, runter, runter, rechts, links, rauf und wieder links,** dann bist du am grünen Studierzimmer und wenn du rein und direkt zur anderen Seite wieder raus gehst, bist du in einem breiten, gewundenen Korridor, der zu einer Kreuzung führt und zur Rechten dieser Kreuzung ist ein langer gerader Gang, der zum Transfigurations-Klassenraum führt!" Die Gestalt des Mannes in mittleren Jahren hielt inne. "Zumindest war es so als \emph{ich} in Hogwarts war. Das \emph{ist} doch ein Montag in einem ungeraden Jahr, oder?"

"Bleistift und mechanisches Papier," sagte Harry zu seinem Beutel. "Äh, abbrechen, Papier und mechanischer Bleistift." Er blickte auf. "Könnten Sie das wiederholen?"

Nachdem er sich noch zwei weitere Male verlaufen hatte, glaubte Harry so langsam die Grundregel des Navigierens im sich immerfort verändernden Labyrinth namens Hogwarts zu verstehen, nämlich, \emph{frag} \emph{ein Gemälde nach dem Weg.} Wenn das irgendeine Art unglaublich tiefsinniger Lebensweisheit widerspiegelte, so hatte er nicht herausfinden können welche.

Das grüne Studierzimmer war ein überraschend gemütlicher Raum, den Sonnenlicht aus Fenstern mit grün-getöntem Glas durchströmte, das Drachen in idyllisch gelassenen Szenen zeigte. Es verfügte über ungeheuer bequem wirkende Sessel und Tische, die äußerst gut geeignet schienen, um daran in Gesellschaft von ein bis drei Freunden zu lernen.

Harry konnte allerdings nicht \emph{tatsächlich} direkt hindurch und zur Tür auf der anderen Seite hinausgehen. Es waren \emph{Bücherregale} in die Wand eingelassen und er musste hinüber gehen und einige der Titel lesen, um seinen Anspruch auf den Verres-Familiennamen aufrechtzuerhalten. Doch er tat es zügig, eingedenk der Mahnung wegen Langsamkeit und trat dann zur anderen Seite hinaus.

Er lief gerade den "breiten, gewundenen Korridor" entlang, als er den Aufschrei eines kleinen Jungen hörte.

In solchen Momenten konnte Harry es rechtfertigen, ungebremst los zu sprinten, ohne Rücksicht darauf seine Kräfte zu schonen oder die richtigen Aufwärmübungen zu machen oder sich um mögliche Hindernissen zu sorgen, ein plötzlicher, rasender Flug, der zu einem ebenso plötzlichen Ende kam, als er beinahe eine Gruppe von sechs Erstklässlern aus Hufflepuff über den Haufen rannte…

… die sich eng zusammen drängten und ziemlich verängstigt wirkten, als wollten sie verzweifelt etwas unternehmen, wüssten aber nicht was, was vermutlich mit der Gruppe aus fünf älteren Slytherins zusammenhing, die einen weiteren kleinen Jungen zu umzingeln schienen.

Harry wurde plötzlich sehr zornig.

"\emph{Entschuldigt mal!}" rief Harry so laut er konnte.

Es mochte unnötig gewesen sein. Man blickte bereits zu ihm. Doch es brachte mit Sicherheit erst einmal alle Aktionen zum Erliegen.

Harry schritt an dem Knäuel aus Hufflepuffs vorbei auf die Slytherins zu.

Sie blickten auf ihn herunter mit Ausdrücken auf dem Gesicht, die von Zorn über Belustigung bis hin zu Erleichterung reichten.\\ Ein Teil von Harrys Gehirn schrie voller Panik, dass dies sehr viel ältere und größere Jungen waren, die ihn in den Boden stampfen konnten.

Ein anderer Teil erwiderte trocken, dass jeder, der sich ernsthaft dabei erwischen ließ, auf den Jungen-der-überlebt-hatte einzustampfen, sich eine ganze \emph{Wagenladung} voller Ärger einhandelte, besonders wenn es sich um eine Bande älterer Slytherins handelte und sieben Hufflepuffs alles mit ansahen und dass die Chance, sie könnten ihm in Gegenwart von Zeugen irgendeinen permanenten Schaden zufügen, gegen null ging. Ihre einzige echte Waffe, die die älteren Jungen gegen ihn in der Hand hatten, war seine eigene Angst, wenn er es zuließ.

Dann bemerkte Harry, dass der Junge, den sie gefangen hielten, Neville Longbottom war.

Natürlich.

Das besiegelte die Sache. Harry hatte entschieden, sich reumütig bei Neville zu entschuldigen und das bedeutete, Neville gehörte \emph{ihm,} wie konnten sie es \emph{wagen?}

Harry langte nach Neville, ergriff ihn am Handgelenk und \emph{riss} ihn aus dem Kreis der Slytherins heraus, der Junge taumelte schockiert als Harry ihn hinaus zerrte und sich beinahe mit der gleichen Bewegung selbst einen Weg durch die entstandene Lücke bahnte.

Und so stand Harry inmitten der Slytherins, an Nevilles Stelle und blickte zu den viel älteren, größeren und stärkeren Jungen hinauf.

"Hallo," sagte Harry. "Ich bin der Junge-der-überlebt-hat."

Eine ziemlich peinliche Pause entstand. Niemand schien zu wissen, wie das Gespräch von hier an weiter verlaufen sollte.

Harrys senkte den Blick und erspähte einige Bücher und Papiere über den Boden verstreut. Oh, das nette alte Spiel, bei dem man den Jungen versuchen lässt, seine Bücher aufzusammeln, um sie ihm dann wieder aus der Hand zu schlagen. Harry konnte sich nicht erinnern, jemals selbst Ziel dieses Spieles gewesen zu sein, doch er verfügte über eine ziemlich gute Vorstellungskraft und seine Vorstellung machte ihn wütend. Nun, sobald die größere Lage hier gelöst war, könnte Neville immer noch problemlos zurückkommen und seine Sachen aufheben, vorausgesetzt dass die Slytherins zu sehr auf ihn fokussiert blieben, um wegen der Bücher auf irgendwelche Ideen zu kommen.

Unglücklicherweise war sein abschweifender Blick bemerkt worden. "Ooh," sagte der größte der Jungen, "wolldes'du die Bücher wieda -"

"Sei still," sagte Harry kalt. \emph{Bring sie immer schön aus dem Gleichgewicht. Tu nicht, was sie erwarten. Verfall nicht in ein Muster, das sie verleitet, dich zu mobben.} "Ist das etwa Teil eines unglaublich cleveren Plans zu deinem zukünftigen Vorteil oder tatsächlich nur eine solch sinnlose Entehrung von Salazar Slytherins Namen, wie es -"

Der größte Junge schubste Harry Potter kräftig und er fiel der Länge nach aus dem Kreis der Slytherins hinaus auf den steinernen Boden von Hogwarts.

Und die Slytherins lachten.

Harry erhob sich auf, wie ihm schien, furchterregend langsame Weise. Er wusste noch nicht, wie sein Zauberstab zu verwenden war, doch unter diesen Umständen gab es keinen Grund, sich davon aufhalten zu lassen.

"Ich würde gern \emph{so viele Punkte wie nötig} zahlen, um diese Person loszuwerden," sagte Harry und deutete mit dem Finger auf den größten der Slytherins.

Dann hob Harry die andere Hand, sagte "Abrakadabra" und schnippte mit den Fingern.

Bei dem Wort \emph{Abrakadabra} schrien zwei der Hufflepuffs auf, Neville eingeschlossen, drei andere Slytherins hechteten verzweifelt aus der Schusslinie von Harrys Finger und der größte Slytherin stolperte schockiert zurück als plötzlich ein roter Spritzer Gesicht, Brust und Nacken zierte.

\emph{Das} hatte Harry \emph{nicht} erwartet.

Langsam griff sich der größte Slytherin an den Kopf und zog die Schale mit Kirschkuchen herunter, die gerade auf ihm drapiert worden war. Einen Augenblick lang hielt der größte der Slytherins die Schale in der Hand, starrte sie an, dann ließ er sie zu Boden fallen.

Es war wahrscheinlich nicht der beste Augenblick dafür, dass einer der Hufflepuffs zu lachen anfinge, doch genau das war es, was einer der Hufflepuffs jetzt tat.

Dann fiel Harry die Notiz am Boden der Schale auf.

"Warte kurz," sagte Harry und schoss nach vorn, um die Notiz aufzulesen. "Die Notiz ist für mich, glaube ich -"

"\emph{Du,}" grollte der größte der Slytherins, "\emph{du, wirst, gleich -}"

"\emph{Guck} dir das an!" rief Harry und schwenkte dem älteren Slytherin die Notiz entgegen. "Ich meine, \emph{guck} dir das nur an! Ist es zu fassen, dass man mir 30 Punkte berechnet für Lieferung und Abfertigung von einem lausigen Kuchen? 30 Punkte! Dabei zahle ich noch drauf, selbst nachdem ich einen unschuldigen Jungen in Nöten gerettet habe! Und Lagergebühren? Transportzuschläge? Verladekosten? Was für \emph{Verladekosten} fallen denn bei einem \emph{Kuchen} an?"

Eine weitere dieser peinlichen Pausen. Harry bedachte denjenigen Hufflepuff, welcher scheinbar einfach nicht zu kichern aufhören konnte, mit tödlichen Verwünschungen; dieser Idiot würde ihn noch in Schwierigkeiten bringen.

Harry trat zurück und warf den Slytherins seinen besten tödlichen Blick zu. "Jetzt verschwindet oder ich mache euer Leben immer surrealer, bis ihr es tut. Seid gewarnt… mischt ihr euch in \emph{mein} Leben ein, könnte \emph{euer} Leben… \emph{ein wenig haarig} werden. Verstanden?"

In einer einzigen schrecklichen Bewegung riss der größte Slytherin seinen Zauberstab hervor und richtete ihn auf Harry und wurde exakt im selben Moment von einem weiteren Kuchen hinten am Kopf getroffen, dieses mal Blaubeere.

Die Notiz auf diesem Kuchen war ziemlich groß und deutlich zu lesen. "Du solltest vielleicht die Notiz auf dem Kuchen lesen," bemerkte Harry. "Ich glaube, diesmal ist sie für dich."

Der Slytherin langte langsam empor, nahm die Kuchenschale, drehte sie um, wobei mit einem feuchten Flatsch noch mehr Blaubeere auf dem Boden landete und las die Notiz, die besagte:

\uline{WARNUNG}\\ WÄHREND DAS SPIEL LÄUFT\\ IST \uline{JEGLICHER} EINSATZ VON MAGIE GEGEN DEN TEILNEHMER UNTERSAGT\\ WEITERE EINMISCHUNGEN IN DAS SPIEL\\ \uline{WERDEN} DER SPIELLEITUNG ZUGETRAGEN

Der Ausdruck reiner Verblüffung auf dem Gesicht des Slytherin war ein Augenschmaus. So langsam fing Harry an, diesen Spielleiter zu mögen.

"Schau mal," sagte Harry, "willst du's für heute nicht einfach gut sein lassen? Ich denke, das hier gerät gerade etwas außer Kontrolle. Wie wär's wenn du nach Slytherin zurück gehst und ich zurück nach Ravenclaw und wir alle erstmal wieder etwas runterkommen, okay?"

"Ich habe eine bessere Idee," zischte der größte Slytherin. "Wie wär's wenn du dir aus Versehen all deine Finger brichst?"

"Wie in Merlins Namen willst du das glaubhaft als Unfall dastehen lassen, nachdem du vor einem Dutzend Zeugen damit gedroht hast, du \emph{Schwachkopf -}"

Der größte Slytherin griff betont langsam nach Harrys Händen und Harry erstarrte, der Teil seines Hirns, der das Alter und die Stärke des anderen Jungen bemerkte, vermochte sich endlich Gehör zu verschaffen und schrie, \emph{WAS ZUM TEUFEL MACHE ICH DA?}

"Warte!" sagte einer der anderen Slytherins, mit plötzlicher Panik in der Stimme. "Stopp, das solltest wirklich nicht machen!"

Der größte Slytherin ignorierte ihn, nahm Harrys rechte Hand fest in seine linke und Harrys Zeigefinger in die rechte Hand.

Harry starrte dem Slytherin direkt in die Augen. Ein Teil von Harry schrie, das sollte nicht passieren, das \emph{durfte} nicht passieren, die Erwachsenen würden so etwas niemals \emph{tatsächlich} zulassen -

Langsam bog der Slytherin seinen Mittelfinger zurück.

\emph{Er hat mir noch nicht tatsächlich den Finger gebrochen und es ist unter meiner Würde, auch nur zu zucken, bis er es tut. Bis dahin ist das nur ein weiterer Einschüchterungsversuch.}

"Stopp!" sagte der Slytherin, der schon zuvor Einwände erhoben hatte. "Hör auf, das ist wirklich eine schlechte Idee!"

"Das würde ich auch meinen," sagte eine eisige Stimme. Die Stimme einer älteren Frau.

Der größte der Slytherins ließ Harrys Hand los und sprang zurück als habe er sich verbrannt.

"Professor Sprout!" schrie einer der Hufflepuffs und klang so froh, wie Harry es noch von niemandem gehört hatte.

Als Harry sich umdrehte, trat in sein Blickfeld eine kleine plumpe Frau mit wirrem grau-gelocktem Haar und schmutzbedeckter Kleidung. Sie richtete einen anklagenden Finger auf die Slytherins. "Erklären Sie sich," sagte sie. "Was machen Sie mit meinem Hufflepuffs und…" sie blickte ihn an. "Meinem guten Schüler Harry Potter."

\emph{Oh oh. Richtig, es war IHR Unterricht, den ich heute Morgen verpasst habe.}

"Er hat gedroht uns zu töten!" platzte einer der anderen Slytherins heraus, der gleiche der einen Stop gefordert hatte.

"Was?" sagte Harry verdutzt. "Habe ich \emph{nicht!} Wollte ich euch umbringen, würde ich nicht vorher noch öffentlich damit drohen!"

Ein dritter Slytherin fing hilflos an zu kichern, bis ihn die tödlichen Blicke der anderen Jungen abrupt zum Schweigen brachten.

Professor Sprout hatte eine reichlich skeptische Miene aufgesetzt. "Was für eine Todesdrohung wäre das genau?"

"Der Tödliche Fluch! Er hat so getan, als würde er den Tödlichen Fluch anwenden!"

Professor Sprout wandte sich Harry zu. "Ja, eine gar schreckliche Drohung von einem elfjährigen Jungen. Doch trotzdem nichts, was Sie auch nur \emph{im Traum} vorgeben sollten, Harry Potter."

"Ich kenne nicht einmal die \emph{Worte} für den Tödlichen Fluch," sagte Harry prompt. "Und ich hatte zu keinem Zeitpunkt den Zauberstab gezogen."

Jetzt warf Professor Sprout Harry einen reichlich skeptischen Blick zu. "Dann nehme ich an, dieser Junge hat sich \emph{selbst} mit zwei Kuchen beworfen."

"Er \emph{hat} seinen Zauberstab nicht benutzt!" platzte einer der jungen Hufflepuffs heraus. "Ich weiß auch nicht, wie er es gemacht hat, er hat einfach mit den Fingern geschnippt und da war Kuchen!"

"Wirklich," sagte Professor Sprout nach einem Moment. Sie zog ihren eigenen Zauberstab. "Ich werde nicht darauf bestehen, da Sie hier das Opfer zu sein scheinen, doch hätten Sie etwas dagegen, wenn ich Ihren Zauberstab überprüfe, um sicher zu gehen?"

Harry holte seinen Zauberstab hervor. "Was muss ich -"

"\emph{Prior Incantato,}" sagte Sprout. Sie runzelte die Stirn. "Seltsam, Ihr Zauberstab scheint überhaupt noch nicht verwendet worden zu sein."

Harry zuckte mit den Schultern. "Wurde er auch nicht, ich habe den Zauberstab und meine Schulbücher erst vor ein paar Tagen bekommen."

Sprout nickte. "Dann haben wir hier einen klaren Fall unabsichtlicher Magie durch einen Jungen, der sich bedroht fühlte. Und die Regeln besagen schlicht, dass Sie dafür nicht verantwortlich zu machen sind. Was \emph{Sie} betrifft…" wandte sie sich den Slytherins zu. Ihre Augen fielen bewusst auf Nevilles Bücher, die auf dem Boden lagen.

Es herrschte lange Stille, während sie die fünf Slytherins betrachtete.

"Drei Punkte Abzug von Slytherin, für \emph{jeden,}" sagte sie schließlich. "Und sechs von \emph{ihm,}" deutete sie auf den mit Kuchen bedeckten Jungen. "Legen Sie sich \emph{niemals} wieder mit meinen Hufflepuffs an oder mit meinem Schüler Harry Potter. Jetzt \emph{gehen Sie.}"

Sie musste sich nicht wiederholen; die Slytherins drehten sich um und entfernten sich rasch.

Neville ging und begann seine Bücher aufzuheben. Er schien zu weinen, doch nur ein wenig. Es mochte am verspäteten Schock liegen oder weil die anderen Jungen ihm halfen.

"Ihnen \emph{vielen} Dank, Harry Potter," wandte sich Professor Sprout an ihn. "Sieben Punkte für Ravenclaw, einen für jeden Hufflepuff, den Sie beschützt haben. Und mehr habe ich nicht zu sagen."

Harry blinzelte. Er hatte mehr so etwas wie eine Standpauke darüber erwartet, sich aus Schwierigkeiten herauszuhalten und eine ziemliche Schimpftirade, weil er seinen ersten Unterricht verpasst hatte.

Vielleicht hätte er \emph{doch} nach Hufflepuff gehen sollen. Sprout war cool.

"\emph{Ratzeputz,}" sagte Sprout in Richtung der Kuchenschweinerei auf dem Boden, die sich prompt in Luft auflöste.

Dann entfernte sie sich in Richtung des Flures, der zum grünen Studierzimmer führte.

"Wie hast du das \emph{gemacht?}" zischte einer der Jungen aus Hufflepuff, sobald sie verschwunden war.

Harry lächelte selbstzufrieden. "Ich kann alles geschehen lassen, was ich will, indem ich nur mit den Fingern schnippe."

Der Junge machte große Augen. "\emph{Wirklich?}"

"Nein," sagte Harry. "Aber wenn du allen diese Geschichte erzählst, dann auf jeden Fall auch Hermine Granger im ersten Jahr von Ravenclaw, sie hat eine Anekdote dazu, die du vielleicht ganz lustig findest." Er hatte absolut keinen Schimmer, was hier vor sich ging, aber diese Gelegenheit, zu seiner wachsenden Legende beizutragen, würde er nicht verstreichen lassen. "Oh und was war das alles wegen dem Tödlichen Fluch?"

Der Junge warf ihm einen seltsamen Blick zu. "Du weißt es wirklich nicht?"

"Wenn doch würde ich nicht fragen."

"Die Worte des Tödlichen Fluches sind," der Junge schluckte und senkte die Stimme zu einem Flüstern, wobei er die Hände zur Seite streckte als wolle er völlig klarstellen, dass er keinen Zauberstab trug, "\emph{Avada Kedavra.}"

\emph{Natürlich sind sie das.}

Harry setzte das auf die anwachsende Liste von Dingen, von denen er seinem Dad, Professor Michael Verres-Evans, nie und nimmer erzählen würde. Es war schon schlimm genug, davon zu sprechen, man sei die einzige Person, die den gefürchteten Tödlichen Fluch überlebt hatte, auch ohne zugeben zu müssen, dass der Tödliche Fluch "Abrakadabra" lautete.

"Ich verstehe," sagte Harry nach einem Moment. "Nun, das ist auf jeden Fall das letzte mal, dass ich \emph{das} sage, bevor ich mit den Fingern schnipse." Obwohl es einen interessanten Effekt hatte, der von taktischem Nutzen sein mochte.

"\emph{Warum} hast du -"

"Von Muggeln aufgezogen, Muggel halten es für einen Witz und glauben, es sei lustig. Das ist ehrlich was passiert ist. Sorry, aber kannst du mir nochmal deinen Namen sagen?"

"Ich bin Ernie Macmillan," sagte der Hufflepuff. Er streckte die Hand aus und Harry schüttelte sie. "Ist mir eine Ehre, dich kennenzulernen."

Harry vollführte eine leichte Verbeugung. "Freut mich auch dich kennenzulernen, aber lassen wir das mit der Ehre."

Dann drängten sich die anderen Jungen um ihn und es folgte eine plötzliche Flut von Bekanntmachungen.

Als sie fertig waren, schluckte Harry. Das würde sehr schwierig werden. "Ähm… wenn ihr mich alle entschuldigen würdet… ich muss Neville etwas sagen -"

Aller Augen richteten sich auf Neville, der mit ängstlichem Gesichtsausdruck einen Schritt zurücktrat.

"Ich nehme an," sagte Neville mit kleiner Stimme, " du wirst sagen, ich hätte mutiger sein sollen -"

"Oh nein, nichts dergleichen!" sagte Harry hastig. "Hat nichts \emph{damit} zu tun. Es ist nur, ähm, wegen etwas, das der Sprechende Hut mir gesagt hat -"

Plötzlich wirkten die Jungen \emph{sehr} interessiert, außer Neville, der sogar \emph{noch} ängstlicher wirkte.

Irgendetwas schien in Harrys Kehle festzusitzen. Er wusste, er sollte einfach damit herausrücken, doch es war als habe er einen großen Klotz verschluckt, der genau im Weg steckte.

Es war als müsse Harry manuell die Kontrolle über seine Lippen an sich reißen und jede Silbe einzeln hervorbringen, doch er schaffte es. "Es, tut, mir, leid." Er stieß die Luft aus und nahm einen tiefen Atemzug. "Was ich, ähm, gestern gemacht habe. Du musst jetzt nicht dankbar sein oder so, ich verstehe es, wenn du mich einfach hasst. Es geht nicht darum, dass ich versuche cool zu wirken, indem ich mich entschuldige oder dass du es annehmen musst. Was ich getan habe, war falsch."

Es gab eine Pause.

Neville drückte seine Bücher fester an die Brust. "Warum hast du es dann getan?" sagte er mit dünner, unsteter Stimme. Er blinzelte als versuchte er Tränen zurückzuhalten. "Warum machen das \emph{alle} mit mir, selbst der Junge-der-überlebt-hat?"

Harry fühlte sich plötzlich kleiner als je zuvor in seinem Leben. "Es tut mir leid," sagte Harry noch einmal, nun mit rauer Stimme. "Es ist nur… du sahst so ängstlich aus, als hättest du ein Schild über dem Kopf, auf dem 'Opfer' steht und ich wollte dir zeigen, dass die Dinge \emph{nicht} immer schlecht ausgehen, dass einem die Monster manchmal Schokolade schenken… ich dachte, wenn ich dir das zeige, dann würde dir klar, dass es nicht so vieles gibt, vor dem man sich fürchten muss -"

"Aber das \emph{gibt es,}" flüsterte Neville. "Du hast es heute gesehen, das \emph{gibt es!}"\\ "Sie hätten dir vor Zeugen nichts wirklich schlimmes angetan. Ihre größte Waffe ist die Angst. Deswegen zielen sie auf \emph{dich,} weil sie sehen können, dass du Angst hast. Ich wollte dafür sorgen, dass du weniger Angst hast… dir zeigen, dass die Angst schlimmer ist als die Sache selbst… oder zumindest habe ich mir das eingeredet, aber der Sprechende Hut sagte mir, ich hätte mich nur selbst belogen und hätte es wirklich getan, weil ich Spaß daran hatte. Deswegen entschuldige ich mich -"

"Du hast mir weh getan," sagte Neville. "Gerade eben. Als du mich gepackt und von ihnen weggezogen hast." Neville streckte den Arm aus und deutete auf die Stelle, wo Harry ihn ergriffen hatte. "Ich kriege da später vielleicht einen blauen Fleck, weil du mich so heftig weggezogen hast. Tatsächlich hast du mir mehr weh getan als die Slytherins, indem sie mich herumgeschubst haben."

"\emph{Neville!}" zischte Ernie. "Er hat versucht, dich zu \emph{retten!}"

"Es tut mir leid," flüsterte Harry. "Als ich das gesehen habe, wurde ich einfach… wirklich zornig…"

Neville blickte ihm fest in die Augen. "Also hast du mich da ganz heftig herausgerissen, dich an meine Stelle gestellt und einen gemacht auf "Hallo, ich bin der Junge-der-überlebt-hat."

Harry nickte.

"Ich glaube, eines Tages wirst du richtig cool sein," sagte Neville. "Aber jetzt gerade bist du es nicht."

Harry schluckte den plötzlichen Kloß in seinem Hals und ging. Er lief weiter den Korridor hinunter bis zur Kreuzung, bog dann links in einen Flur ab und ging weiter, ohne auf den Weg zu achten.

Was hätte er denn machen \emph{sollen?} Niemals wütend werden? Er war nicht sicher, ob er überhaupt etwas hätte unternehmen können, ohne zornig zu sein und wer weiß, was dann mit Neville und seinen Büchern geschehen wäre. Außerdem hatte Harry genug Fantasy-Bücher gelesen, um zu wissen wie \emph{das} ablief. Er würde versuchen, den Ärger zu unterdrücken und scheitern und er würde wieder hervorbrechen. Und nach der ganzen langen Reise der Selbsterkenntnis würde er am Ende lernen, dass sein Zorn ein Teil von ihm war und er nur, indem er ihn akzeptierte, lernen konnte, weise von ihm Gebrauch zu machen. \emph{Star Wars} war das einzige Universum, in dem die Lösung \emph{tatsächlich} darin bestand, dass man sich völlig von negativen Emotionen trennen sollte und irgendwas an Yoda hatte in Harry immer schon einen Hass auf den kleinen grünen Schwachkopf ausgelöst.

Also war der offensichtlich zeitsparendere Plan, die Reise der Selbsterkenntnis zu überspringen und gleich zu dem Teil zu kommen, wo er erkannte, dass er nur dann seinen Ärger kontrollieren könnte, indem er ihn als Teil seiner selbst akzeptierte.

Das Problem war, er \emph{fühlte} sich nicht außer Kontrolle, wenn er zornig war. Die kalte Wut weckte in ihm das Gefühl, als habe er die Kontrolle. Nur im Rückblick schienen sich \emph{die Ereignisse insgesamt} irgendwie… hochgeschaukelt zu haben.

Er fragte sich, inwieweit den Spielleiter solche Dinge interessierten und ob er dafür Punkte verloren oder gewonnen hatte. Harry selbst hatte auf jeden Fall das Gefühl so einige Punkte verloren zu haben und er war sicher, die alte Dame in dem Bild hätte ihm gesagt, dass sei die einzige Meinung, die zählte.

Außerdem fragte sich Harry ob der Spielleiter auch Professor Sprout geschickt hatte. Es war ein logischer Gedanke: die Notiz hatte angedroht, die Spielleitung zu informieren und da war Professor Sprout. Vielleicht \emph{war} Professor Sprout der Spielleiter - die \emph{Hauslehrerin von Hufflepuff} wäre die \emph{letzte} Person, die irgendwer verdächtigen würde, was sie ziemlich weit oben auf Harrys Liste platzieren sollte. Er hatte auch ein oder zwei Krimis gelesen.

"Also, wie mache ich mich so in dem Spiel?" sagte Harry laut.

Ein Blatt Papier segelte über seinen Kopf, als habe es jemand von hinten über ihn geworfen - Harry drehte sich um, doch da war niemand - und als Harry sich wieder nach vorn wandte, ruhte die Notiz auf dem Boden.

Die Notiz besagte:***

PUNKTE FÜR STIL: 10\\ PUNKTE FÜR GUTMÜTIGKEIT: -3.000.000\\ RAVENCLAW-HAUSPUNKTE-BONUS: 70\\ AKTUELLER PUNKTESTAND: -2.999.871\\ VERBLEIBENDE ZÜGE: 2

"\emph{Minus drei Millionen Punkte?}" beschwerte sich Harry empört bei dem leeren Flur. "Das scheint doch etwas übertrieben! Ich möchte Einspruch bei der Spielleitung einlegen! Wie soll ich denn drei Millionen Punkte in den nächsten zwei Zügen wieder aufholen?"

Eine weitere Notiz kam über seinen Kopf gesegelt.

EINSPRUCH: FEHLGESCHLAGEN\\ DIE FALSCHEN FRAGEN GESTELLT: -1.000.000.000.000 PUNKTE\\ AKTUELLER PUNKTESTAND: -1.000.002.999.871\\ VERBLEIBENDE ZÜGE: 1

Harry gab es auf. Mit nur einem verbleibenden Zug konnte er nur noch seinen besten Tipp abgeben, auch wenn er nicht sehr gut war. "Ich denke, das Spiel steht für das Leben."

Ein letztes Stück Papier flog über seinen Kopf hinweg, auf dem stand:

VERSUCH FEHLGESCHLAGEN\\ VERSAGT VERSAGT VERSAGT\\ AIIIIIIIIIIEEEEEEEEEEEEEE\\ AKTUELLER PUNKTESTAND: MINUS UNENDLICH\\ \uline{DU HAST DAS SPIEL VERLOREN}

LETZTE ANWEISUNG:\emph{\hfill\break geh zu Professor McGonagalls Büro}

Die letzte Zeile war in seiner eigenen Handschrift verfasst.

Harry starrte die letzte Zeile ein Weilchen an, zuckte dann mit den Schultern. Schön. Dann also Professor McGonagalls Büro. Wenn \emph{sie} der Spielleiter war…

Okay, ehrlich gesagt hatte Harry nicht den leisesten Schimmer, wie er sich fühlen würde, wenn Professor McGonagall der Spielleiter war. Sein Verstand brachte absolut nichts zutage. Es war, buchstäblich, unvorstellbar.

Ein paar Porträts später - es war kein langer Weg, Professor McGonagalls Büro lag nicht weit von ihrer Transfigurationsklasse entfernt, zumindest nicht an Montagen in ungeraden Jahren - stand Harry vor der Tür zu ihrem Büro.

Er klopfte.

"Herein," erklang die gedämpfte Stimme von Professor McGonagall.

Er trat ein.

* Diese Unterhaltung ähnelt einem Gespräch zwischen Alice und der Grinsekatze aus \emph{Alice im Wunderland} von Lewis Carroll.\\ ** Das klingt ein wenig nach einem Cheat-Code für ein altes Konsolenspiel (auch wenn sich kein konkret passender finden ließ) oder, vielleicht auch wahrscheinlicher, nach einer Wegbeschreibung, wie man sie in einer Lösung für ein altes (in diesem Fall offenbar wenig hilfreich bis böswillig programmiertes) textbasiertes Computerspiel (wie z.B. \emph{Zork}) finden mag.\\ *** Das ist ganz offensichtlich eine Anspielung auf \emph{Per Anhalter durch die Galaxis} von Douglas Adams. Die leider bereits im Original ungenaue Übersetzung des entsprechenden Zitates (\emph{Ten out of ten for style, but minus several million for good thinking.} - \emph{Zehn von zehn Punkten für Stil, aber ein paar Millionen miese für Gutmütigkeit.}) habe ich des Wiedererkennungswertes wegen beibehalten. Korrekter kritisiert der Spielleiter hier natürlich nicht Harrys Gutmütigkeit - er forderte ihn ja auf, Gutes zu tun - sondern seine mangelnde Intelligenz (\emph{good thinking} = \emph{logisches Denken}).

