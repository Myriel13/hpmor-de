

\hypertarget{belohnungsaufschub}{% \section{19. Belohnungsaufschub}\label{belohnungsaufschub}}

\textbf{Kapitel 19: Belohnungsaufschub

}

Blut für den Blutgott! Schädel für J. K. Rowling!

\later

Draco trug einen ernsten Ausdruck auf dem Gesicht und sein grün-getrimmter Umhang sah irgendwie sehr viel formeller, ernsthafter und ausgefeilter aus als die exakt gleichen Umhänge, die die zwei Jungen hinter ihm trugen.

"Rede," sagte Draco.

"Jah! Rede!"

"Du has'den Boss gehört! Rede!"

"Ihr zwei, andererseits, \emph{haltet die Klappe.}"

Die letzten Unterrichtsstunden am Freitag fingen gerade an, in dem weiten Auditorium, wo alle vier Häuser Verteidigung, ähm, Kampf-Magie lernten.

Die letzten Unterrichtsstunden am Freitag.

Harry hoffte, dass dieser Unterricht stressfrei verlaufen und dem brillanten Professor Quirrell klar sein würde, dass dies vielleicht nicht der beste Zeitpunkt wäre, ihn für irgendetwas besonderes auszuwählen. Harry hatte sich ein wenig erholt, doch…

… doch nur für den Fall war es wahrscheinlich besser, vorher etwas Stress abzubauen.

Harry lehnte sich in seinem Stuhl zurück und bedachte Draco und seine Lakaien mit einem höchst feierlichen Blick.

"Sie fragen: Was ist unser Ziel?" deklamierte Harry. "Ich kann es in einem Wort nennen: Sieg - Sieg um jeden Preis - Sieg trotz allem Schrecken - Sieg, wie lang und beschwerlich der Weg dahin auch sein mag; denn ohne Sieg gibt es kein -"*

"\emph{Rede über Snape,}" zischte Draco. "\emph{Was hast du getan?}"

Harry wischte die falsche Feierlichkeit aus seinem Gesicht und warf Draco einen ernsteren Blick zu.

"Du hast es gesehen," sagte Harry. "Alle haben es gesehen. Ich habe mit den Fingern geschnippt."

"\emph{Harry! Reiz mich nicht!}"

Also war er jetzt zu \emph{Harry} befördert worden. Interessant. Und tatsächlich war Harry sich ziemlich sicher, er sollte das bemerken und sich schlecht fühlen, wenn er nicht irgendwie darauf reagierte…

Harry tippte sich an die Ohren und warf einen bedeutungsvollen Blick auf die Lakaien.

"Sie werden nicht reden," sagte Draco.

"Draco," sagte Harry, "ich werde jetzt hundertprozentig ehrlich sein und sagen, dass ich von Mr~Goyles Scharfsinn gestern nicht sonderlich beeindruckt war."

Mr~Goyle zuckte zusammen.

"Ich ebenso wenig," sagte Draco. "Ich erklärte ihm, dass ich dir deshalb jetzt einen Gefallen schulde." (Mr~Goyle zuckte erneut.) "Doch es \emph{liegt} ein großer Unterschied zwischen dieser Art Fehler und Indiskretion. Das zu verstehen wurde ihnen wirklich schon von Kindheit an beigebracht."

"Nun gut, dann," sagte Harry. Er senkte die Stimme, obwohl die Hintergrundgeräusche in Dracos Gegenwart zu einem Rauschen verblasst waren. "Ich habe eines von Severus Geheimnissen ausgeknobelt und eine kleine Erpressung durchgezogen."

Dracos Gesichtsausdruck wurde härter. "Gut, jetzt erzähl mir was, dass du nicht den Idioten in Gryffindor direkt ins Gesicht gesagt hast, was heißt, es war die Geschichte, die du in der ganzen Schule verbreiten \emph{wolltest.}"

Harry grinste unwillkürlich und er wusste, dass Draco angebissen hatte.

"Was erzählt Severus?" sagte Harry.

"Dass ihm nicht klar war, wie empfindlich die Gefühle kleiner Kinder wären," sagte Draco. "Selbst in Slytherin! Selbst \emph{mir!}"

"Bist du sicher," sagte Harry, "dass du etwas wissen willst, das dein Hauslehrer dich lieber nicht wissen lassen will?"

"Ja," sagte Draco ohne zu zögern.

\emph{Interessant.} "Dann wirst du wirklich zuerst deine Lakaien wegschicken, weil ich nicht sicher bin, dass ich alles glauben kann, was du über sie glaubst."

Draco nickte. "Okay."

Mr~Crabbe und Mr~Goyle sahen \emph{sehr} unglücklich aus. "Boss -" sagte Mr~Crabbe.

"Ihr habt Mr~Potter keinen Grund gegeben euch zu trauen," sagte Draco. "Geht!"

Sie gingen.

"Insbesondere," sagte Harry und senkte die Stimme noch weiter, "bin ich nicht \emph{vollkommen} überzeugt, dass sie was ich gesagt habe nicht einfach an Lucius berichten."

"Das würde Vater nicht \emph{tun!}" sagte Draco und sah ehrlich entsetzt aus. "Sie \emph{gehören mir!}"

"Tut mir leid, Draco," sagte Harry. "Ich bin einfach nicht sicher, dass ich alles glauben kann, was du über deinen Vater glaubst. Stell dir vor, es wäre dein Geheimnis und ich sagte dir, mein Vater würde das nicht tun."

Draco nickte langsam. "Du hast recht. Es \emph{tut} mir leid, Harry. Es war falsch von mir, es von dir zu verlangen."

\emph{Wie wurde ich} so \emph{befördert? Sollte er mich jetzt nicht hassen?} Harry hatte das Gefühl, er sähe hier etwas, das sich nutzen ließe… er wünschte nur sein Hirn wäre nicht so erschöpft. Normalerweise hätte er sich gern an einem komplizierten Ränkespiel versucht.

"Sei's drum," sagte Harry. "Ein Handel. Ich werde dir was erzählen, das nicht an der großen Glocke hängt und auch nicht an die große Glocke gehängt \emph{wird} und im \emph{besonderen} nicht deinem Vater zu Ohren kommt und im Gegenzug erzählst du mir, was du und Slytherin über die ganze Sache denken."

"Deal!"

Nun machen wir das so vage wie möglich… etwas, das nicht viel Schaden anrichtet, selbst wenn es herauskäme… "Was ich sagte war die Wahrheit. Ich habe eines von Severus Geheimnissen entdeckt und ich habe eine Erpressung durchgezogen. Aber Severus war nicht als einziger beteiligt."

"\emph{Ich wusste es!}" sagte Draco triumphierend.

Harry wurde flau im Magen. Er hatte offenbar etwas sehr wichtiges gesagt und er wusste nicht wieso. Das war kein gutes Zeichen.

"Alles klar," sagte Draco. Er grinste jetzt breit. "Also, so war die Reaktion in Slytherin. Zuerst die ganzen Idioten so 'Wir hassen Harry Potter! Mischen wir ihn auf!"

Harry schluckte. "Was \emph{stimmt} nicht mit dem Sprechenden Hut? Das ist nicht Slytherin, sondern \emph{Gryffindor -}"

"Es können nicht alle Kinder hochbegabt sein," sagte Draco, doch er lächelte auf hässlich-verschwörerische Weise, als wollte er andeuten, dass er Harrys Ansicht insgeheim zustimmte. "Und etwa fünfzehn Sekunden danach machte ihnen jemand klar, warum sie Snape damit nicht wirklich einen Gefallen täten, also alles gut. Jedenfalls, danach kam die zweite Welle Idioten, die sagten, 'Sieht aus als wäre Harry Potter doch nur so ein Weltverbesserer.'"

"Und dann?" sagte Harry und lächelte, obwohl er keine Ahnung hatte, warum \emph{das} dämlich war.

"Und dann meldeten sich die tatsächlich schlauen Leute zu Wort. Es ist offensichtlich, dass du einen Weg gefunden hast, eine \emph{Menge} Druck auf Snape auszuüben. Und das könnte mehr als eine Sache sein… der offensichtlich \emph{nächste} Gedanke ist, dass es etwas mit der unbekannten Sache zu tun hat, die Snape gegen Dumbledore in der Hand hat. Habe ich recht?"

"Kein Kommentar," sagte Harry. Zumindest diesen Teil verarbeitete Harrys Gehirn korrekt. Im Haus Slytherin \emph{hatte} man sich gefragt, warum Severus nicht gefeuert wurde. Und sie hatten gefolgert, dass Severus Dumbledore erpresste. Könnte das tatsächlich stimmen…? Aber Dumbledore hatte sich nicht so verhalten als ob…

Draco sprach weiter. "Und als \emph{nächstes} führten die schlauen Leute aus, dass wenn du genug Druck auf Snape ausüben konntest, damit er halb Hogwarts in Ruhe lässt, bedeutet das, du hattest wahrscheinlich Macht genug um ihn komplett loszuwerden, wenn du gewollt hättest. Was du mit ihm gemacht hast, war eine Demütigung, genau wie er versuchte dich zu demütigen - doch du hast uns unseren Hauslehrer gelassen."

Harry ließ sein Lächeln breiter werden.

"Und die \emph{wirklich} schlauen Leute," sagte Draco, sein Gesicht jetzt ernst, "verschwanden daraufhin und führten eine kleine Debatte unter sich und jemand erklärte, dass es sehr dumm wäre, einen solchen Feind frei herumlaufen zu lassen. Wenn du sein Druckmittel gegen Dumbledore ausschalten könntest, wäre es das Sinnvollste das einfach zu tun. Dumbledore würde Snape aus Hogwarts rausschmeißen und ihn vielleicht sogar umbringen lassen, er wäre dir \emph{extrem} dankbar und du müsstest dir keine Sorgen machen, dass Snape sich eines Nachts mit interessanten Zaubertränken in deinen Schlafsaal schleicht."

Harrys Gesicht war jetzt ausdrucklos. Daran hatte er nicht gedacht und das hätte er wirklich, wirklich tun solllen. "Und daraus hast du geschlossen…?"

"Snapes Druckmittel war ein Geheimnis von Dumbledore und \emph{du hast es rausgefunden!}" Draco blickte triumphierend. "Es kann nicht mächtig genug sein, um Dumbledore völlig zu vernichten oder Snape hätte es bereits benutzt. Snape weigert sich, sein Druckmittel für irgendetwas anderes einzusetzen als König von Haus Slytherin in Hogwarts zu bleiben und selbst dabei bekommt er nicht immer was er will, also muss es seine Grenzen haben. Aber es \emph{muss} wirklich gut sein! Vater hat seit \emph{Jahren} versucht, Snape dazu zu bringen, es ihm zu verraten!"

"Und," sagte Harry, "nun denkt Lucius vielleicht kann \emph{ich} es ihm sagen. Hast du schon eine Eule bekommen -"

"Werde ich heute Nacht," sagte Draco und lachte. "Sie wird lauten," seine Stimme nahm einen tieferen, formelleren Tonfall an, "\emph{Mein geliebter Sohn: Ich habe dich} \emph{bereits} \emph{über Harry Potters potentielle Bedeutung unterrichtet. Wie dir bereits klar} \emph{sein wird, ist seine Bedeutung nun umso größer und dringlicher.} \emph{Solltest} \emph{du} \emph{gleich welchen} \emph{Zugang für eine Freundschaft} \emph{mit ihm} \emph{oder} \emph{ein} \emph{Druckmittel gegen ihn sehen, musst du} \emph{dem} \emph{nachgehen und alle} \emph{Ressourcen} \emph{der Malfoys stehen dir} \emph{wenn} \emph{nötig zur Verfügung.}"

Donnerwetter. "Nun," sagte Harry, "kein Kommentar ob dein ganzes kompliziertes Gebilde von Theorie stimmt oder nicht, lass mich nur sagen, dass wir noch nicht so gute Freunde sind."

"Ich weiß," sagte Draco. Dann wurde sein Gesicht \emph{sehr} ernst und seine Stimme sehr leise, selbst mit dem Rauschen. "Harry, ist dir in den Sinn gekommen, dass wenn du etwas weißt, von dem Dumbledore nicht will, dass es bekannt wird, Dumbledore dich einfach umbringen lassen könnte? Und es würde außerdem aus dem Jungen-der-überlebt-hat, einem möglichen konkurrierenden Anführer, einen wertvollen Märtyrer machen."

"Kein Kommentar," sagte Harry erneut. An das letzte hatte er ebenfalls nicht gedacht. \emph{Schien} nicht Dumbledores Stil zu sein… aber…

"Harry," sagte Draco, "du hast offensichtlich \emph{unglaubliches} Talent, doch du hast keine Übung und keine Mentoren und manchmal tust du wirklich dumme Sachen und \emph{du brauchst wirklich einen Ratgeber, der weiß wie man das macht oder du wirst verletzt!}" Dracos Gesicht war eindringlich.

"Ah," sagte Harry. "Einen Ratgeber wie Lucius?"

"Wie \emph{mich!}" sagte Draco. "Ich verspreche, deine Geheimnisse vor Vater zu bewahren, vor \emph{jedem,} ich helfe dir einfach mit dem, was immer du tun willst!"

Wow.

Harry sah, dass Zombie-Quirrell durch die Türen getaumelt kam.

"Der Unterricht fängt gleich an," sagte Harry. "Ich denke nach über das, was du gesagt hast, oftmals wünschte ich \emph{schon}, ich hätte all dein Training, es ist nur, ich weiß nicht wie ich dir so schnell vertrauen kann -"

"Solltest du nicht," sagte Draco, "es ist zu früh. Siehst du? Ich gebe dir sogar dann gute Ratschläge, wenn es mir schadet. Doch vielleicht sollten wir \emph{uns beeilen,} engere Freundschaft zu schließen."

"Dafür bin ich offen," sagte Harry, der bereits herauszufinden versuchte, wie sich das ausnutzen ließ.

"Noch ein kleiner Rat," sagte Draco eilig als Quirrell auf seinen Schreibtisch zu schlurfte, "im Moment weiß niemand in Slytherin, was er von dir halten soll, wenn du also um unsere Gunst wirbst und ich denke, das tust du, solltest du etwas tun, dass Slytherin Freundschaft vermittelt. \emph{Bald,} etwa heute oder morgen."

"Severus weiterhin Extra-Hauspunkte an Slytherin verteilen zu lassen, war nicht genug?" Kein Grund, dass Harry dafür nicht die Lorbeeren einstrich.

Dracos Augen blitzten verstehend, dann sagte er hastig, "Es ist nicht dasselbe, vertrau mir, es muss etwas offensichtliches sein. Schubs deine Schlammblut-Rivalin Granger gegen eine Mauer oder sowas, jeder in Slytherin wird wissen, was das heißt -"

"So funktioniert das in Ravenclaw \emph{nicht,} Draco! Wenn du jemanden gegen eine Mauer schubsen musst, heißt das, du hast nicht genug \emph{Köpfchen,} ihn auf die richtige Art zu schlagen und jeder in Ravenclaw \emph{weiß das -}"

Der Bildschirm auf Harrys Pult erwachte flackernd zum Leben und löste eine plötzliche Sehnsucht nach Fernsehen und Computern aus.

"Ähem," sagte Professor Quirrells Stimme, schien aus dem Bildschirm heraus direkt zu Harry zu sprechen. "Bitte nehmen Sie Ihre Plätze ein."

\later

Und alle Kinder setzten sich und starrten auf die Verstärker-Bildschirme auf ihren Pulten oder blickten direkt zu der großen weißen Marmor-Bühne hinab, wo Professor Quirrell stand, an seinen Schreibtisch auf dem kleinen Podest aus dunklerem Marmor gelehnt.

"Heute," sagte Professor Quirrell, "hatte ich geplant, Sie Ihren ersten Schutzzauber zu lehren, einen kleinen Schild, der der Vorläufer des heutigen \emph{Protego} war. Doch nach weiterer Überlegung habe ich den heutigen Lehrplan im Lichte jüngster Ereignisse abgeändert."

Professor Quirrells Blick suchte die Sitzreihen ab. Harry zuckte zusammen auf seinem Platz, in der hintersten Reihe. Er hatte so ein Gefühl, wer aufgerufen werden würde.

"Draco, vom Noblen und Uralten Haus Malfoy," sagte Professor Quirrell.

Puh.

"Ja, Professor?" sagte Draco. Seine Stimme erklang lauter, schien aus dem Verstärker-Bildschirm auf Harrys Pult zu kommen, der Dracos Gesicht zeigte, als er sprach. Dann wechselte er zurück zu Professor Quirrell, der sagte:

"Ist es Ihr Ziel, der nächste Dunkle Lord zu werden?"

"Das ist eine seltsame Frage, Professor," sagte Draco. "Ich meine, wer wäre dumm genug es zuzugeben?"

Ein paar Schüler lachten, aber nicht viele.

"In der Tat," sagte Professor Quirrell. "Obwohl es keinen Sinn hat, irgendeinen von Ihnen zu fragen, würde es mich nicht im mindesten überraschen, wenn es ein oder zwei Schüler in meinen Klassen gibt, die Ambitionen hegen, der nächste Dunkle Lord zu sein. Immerhin wollte \emph{ich} der nächste Dunkle Lord werden, als \emph{ich} ein junger Slytherin war."

Dieses mal war das Gelächter viel verbreiteter.

"Nun, immerhin ist es das Haus der Ehrgeizigen," sagte Professor Quirrell lächelnd. "Mir wurde erst später klar, dass es die Kampf-Magie war, die mich wirklich begeisterte und mein wahrer Ehrgeiz war, ein großer kämpfender Zauberer zu werden und eines Tages in Hogwarts zu lehren. Jedenfalls, als ich dreizehn Jahre alt war, las ich mich durch die historischen Abteilungen der Hogwarts-Bibliothek, untersuchte Leben und Schicksale vergangener Dunkler Lords und ich fertigte eine Liste an, mit all den Fehlern, die ich niemals machen würde, wenn ich ein Dunkler Lord wäre -"

Harry kicherte, bevor er es verhindern konnte.

"Ja, Mr~Potter, sehr amüsant. Also, Mr~Potter, können Sie sich denken, was der erste Punkt auf jener Liste war?"

\emph{Toll.} "Ähm… sich niemals etwas kompliziertes für einen Gegner ausdenken, wenn ein einfaches Abrakadabra auch ausreicht?"

"Der \emph{Ausdruck,} Mr~Potter, ist \emph{Avada Kedavra,}" Professor Quirrells Stimme klang aus irgendeinem Grund etwas scharf, "und nein, das war \emph{nicht} auf der Liste, die ich mit dreizehn angefertigt habe. Würden Sie noch einmal raten?"

"Ah… niemals vor irgendwem mit deinem bösen Masterplan angeben?"

Professor Quirrell lachte. "Ah, nun \emph{das} war Nummer zwei. Meine Güte, Mr~Potter, haben wir die selben Bücher gelesen?"

Es folgte mehr Gelächter, mit einem nervösen Unterton. Harry presste seine Kiefer fest aufeinander und sagte nichts. Eine Verneinung würde nichts bringen.

"Aber nein. Der \emph{erste} Punkt war, 'Ich werde nicht herumlaufen und starke, boshafte Gegner provozieren.' Die Geschichte der Welt wäre eine ganz andere, wenn Mornelithe Falconsbane oder Hitler diesen elementaren Punkt verstanden hätten. Nun, \emph{falls,} Mr~Potter - nur \emph{falls} sie zufällig einen Ehrgeiz hegen, ähnlich meinem als junger Slytherin - dann hoffe ich, Ihr Ehrgeiz ist es nicht, ein \emph{dummer} Dunkler Lord zu werden."

"Professor Quirrell," sagte Harry, mit den Zähnen knirschend, "ich bin ein \emph{Ravenclaw} und es ist nicht mein Ehrgeiz, dumm zu sein, Punkt. Ich weiß, was ich heute getan habe, war dumm. Doch es war nicht \emph{finster!} Ich war es \emph{nicht,} der in diesem Kampf den ersten Schlag ausgeteilt hat!"

"Sie, Mr~Potter, sind ein Idiot. Doch das war auch ich in Ihrem Alter. Daher habe ich Ihre Antwort erwartet und den heutigen Lehrplan entsprechend abgeändert. Mr~Gregory Goyle, wenn Sie bitte vortreten würden?"

Eine überraschte Pause entstand im Klassenraum. Das hatte Harry nicht erwartet.

Ebenso nicht, wie es aussah, Mr~Goyle, der ziemlich unsicher und besorgt aussah, als er die Marmor-Bühne bestieg und sich dem Podest näherte.

Professor Quirrell richtete sich vom Schreibtisch auf. Er sah plötzlich stärker aus, seine Hände ballten sich zu Fäusten und er nahm klar erkennbar eine Kampfkunst-Stellung ein.

Harrys riss bei dem Anblick die Augen auf und ihm wurde klar, warum Mr~Goyle hinauf gerufen worden war.

"Die meisten Zauberer," sagte Professor Quirrell, "halten sich nicht groß auf mit dem, was Muggel als Kampfkunst bezeichnen würden. Ist nicht ein Zauberstab stärker als eine Faust? Diese Einstellung ist dumm. Zauberstäbe werden in Fäusten gehalten. Wenn Sie ein großer kämpfender Zauberer werden wollen, \emph{müssen} Sie die Kampfkunst bis zu einem Grad erlernen, der selbst einen Muggel beeindrucken würde. Ich werde nun eine bestimmte, lebenswichtige Technik demonstrieren, die ich in einem \emph{Dojo} erlernt habe, in einer Muggel-Schule der Kampfkünste, von welcher ich in Kürze mehr berichten werde. Für den Moment…" Professor Quirrell machte mehrere Schritte vorwärts, noch immer in Stellung, auf Mr~Goyles Position zu. "Mr~Goyle, ich werde Sie bitten, mich anzugreifen."

"Professor Quirrell," sagte Mr~Goyle, seine Stimme nun verstärkt, wie die des Professors war, "darf ich fragen, welchen Grad -"

"Sechster \emph{Dan.} Sie werden nicht verletzt werden und ich ebenso wenig. Und wenn Sie eine Gelegenheit sehen, ergreifen Sie sie."

Mr~Goyle nickte, sah sehr erleichtert aus.

"Bemerken Sie," sagte Professor Quirrell, "dass Mr~Goyle fürchtete, jemanden anzugreifen, der die Kampfkunst nicht in akzeptablem Maße beherrscht, aus Angst ich, oder er, könnte verletzt werden. Mr~Goyles Haltung ist völlig korrekt und er hat sich dafür drei Quirrell-Punkte verdient. Jetzt, Kampf!"

Der Junge rauschte vorwärts, Fäuste flogen und der Professor blockte jeden Schlag, tanzte zurück, Quirrell trat und Goyle blockte und wirbelte herum, versuchte Quirrel ins Taumeln zu bringen, mit ausholendem Bein und Quirrell sprang darüber hinweg und es passierte alles zu schnell für Harry, um zu begreifen, was geschah und dann lag Goyle auf dem Rücken und stieß mit den Beinen und Quirrell \emph{flog tatsächlich durch die Luft} und dann traf er auf, Schulter voran und rollte sich ab.

"Stop!" schrie Professor Quirrell am Boden, klang leicht panisch. "Sie gewinnen!"

Mr~Goyle riss sich so abrupt empor, dass er taumelte, stolperte und fiel beinahe durch die abgebrochene Bewegung seines Hals-über-Kopf-Angriffs auf Professor Quirrell. Sein Gesichtsausdruck völlig schockiert.

Professor Quirrell bog den Rücken durch und schwang sich auf die Füße mit einer anmutigen Sprungbewegung, die von seinen Händen keinen Gebrauch machte.

Es war still im Klassenraum, eine Stille geboren aus reiner Verwirrung.

"Mr~Goyle," sagte Professor Quirrell, "welche lebenswichtige Technik habe ich demonstriert?"

"Wie man richtig fällt, wenn einen jemand wirft," sagte Mr~Goyle. "Es ist eine der ersten Techniken, die man lernt -"

"Das auch," sagte Professor Quirrell.

Es gab eine Pause.

"Die lebenswichtige Technik, die ich demonstriert habe," sagte Professor Quirrell, "war, wie man verliert. Sie können gehen, Mr~Goyle, danke."

Mr~Goyle verließ die Plattform, ziemlich verwirrt. Harry fühlte dasselbe.

Professor Quirrell ging zurück zu seinem Schreibtisch und lehnte sich erneut dagegen. "Manchmal vergessen wir die grundlegendsten Dinge, weil es zu lang zurückliegt, dass wir sie lernten. Mir wurde klar, dass mir mit meinem Lehrplan dasselbe wiederfahren war. Man bringt Schülern nicht bei zu werfen, bevor man ihnen beibringt zu fallen. Und ich darf Ihnen nicht beibringen zu kämpfen, wenn Sie nicht verstehen wie man verliert."

Professor Quirrells Gesicht wurde härter und Harry glaubte einen Hauch von Schmerz, einen Anflug von Sorge in diesen Augen gesehen zu haben. "Ich lernte zu verlieren in einem \emph{Dojo} in Asien, wo wie jeder Muggel weiß, alle guten Kampfkünstler leben. Dieses \emph{Dojo} unterrichtete einen Stil, der unter kämpfenden Zauberern den Ruf genoss, sich gut für magisches Duellieren zu eignen. Der Meister dieses \emph{Dojo} - ein alter Mann nach Muggel-Maßstäben - war der größte lebende Lehrer dieses Stils. Er hatte natürlich keine Ahnung, dass Magie existiert. Ich bewarb mich um ein Studium dort und war einer der wenigen Schüler, die aus vielen Bewerbern in jenem Jahr angenommen wurden. Es mag ein winziges bisschen spezieller Einfluss im Spiel gewesen sein."

Gelächter erklang im Klassenraum. Harry teilte es nicht. Das war ganz und gar nicht richtig gewesen.

"Jedenfalls. Während einem meiner Faustkämpfe, nach dem ich auf besonders demütigende Weise geschlagen worden war, verlor ich die Beherrschung und griff meinem Trainingsparter an -"

\emph{Oha.}

"- glücklicherweise mit meinen Fäusten statt meiner Magie. Der Meister warf mich, überraschenderweise, nicht auf der Stelle hinaus. Aber er sagte mir, ich hätte ein Problem mit meinem Temperament. Er erklärte es mir und mir war klar, dass er recht hatte. Und dann sagte er, ich würde lernen zu verlieren."

Professor Quirrells Gesicht war ausdruckslos.

"Auf seinen strikten Befehl hin reihten sich alle Schüler des \emph{Dojo} auf. Einer nach dem anderen kamen sie auf mich zu. Ich durfte mich \emph{nicht} verteidigen. Ich durfte nur um Gnade bitten. Einer nach dem anderen schlugen sie mich oder boxten mich und stießen mich zu Boden. Einige von ihnen spuckten auf mich. Sie belegten mich mit üblen Schimpfworten in ihrer Sprache. Und jedem von ihnen musste ich sagen 'Ich habe verloren!' und ähnliche solche Dinge, wie etwa 'Ich bitte dich aufzuhören!' und 'Ich gebe zu, du bist besser als ich!'"

Harry versuchte sich das vorzustellen und schaffte es einfach nicht. Es war nicht möglich, dass dem ehrwürdigen Professor Quirrell so etwas passiert sein konnte.

"Ich war schon damals ein hochbegabter Kampf-Magier. Mit zauberstabloser Magie allein hätte ich jeden in diesem Dojo töten können. Ich habe es nicht getan. Ich lernte zu verlieren. Bis heute erinnere ich mich daran als die unangenehmsten Stunden meines Lebens. Und als ich das Dojo acht Monate später verließ - was nicht annähernd genug Zeit war, doch alles was ich erübrigen konnte - sagte mir der Meister, er hoffe dass ich verstanden hätte, warum das nötig gewesen sei. Und ich sagte ihm, dass es eine der wertvollsten Lektionen war, die ich jemals lernte. Was die Wahrheit war und immer noch ist."

Professor Quirrells Gesichtsausdruck wurde bitter. "Sie fragen sich, wo dieses wundersame \emph{Dojo} ist und ob Sie dort studieren können. Sie können es nicht. Denn nicht lange danach kam ein weiterer Möchtegern-Student an diesen versteckten Ort, zu jenem abgelegenen Berg. Er-dessen-Name-nicht-genannt-werden-darf."

Das Geräusch vieler simultan einatmender Münder erklang. Harry hatte ein ungutes Gefühl im Magen. Er wusste was kam.

"Der Dunkle Lord kam offen zu jener Schule, ohne Verkleidung, glühend rote Augen und alles. Die Schüler versuchten, ihm den Weg zu versperren und er apparierte einfach hindurch. Es herrschte Angst dort, doch auch Disziplin und der Meister trat vor. Und der Dunkle Lord verlangte - erbat nicht, sondern verlangte - unterrichtet zu werden."

Professor Quirrells Gesicht war sehr hart. "Vielleicht hatte der Meister zu viele Bücher gelesen, die die Lüge berichten, ein wahrer Kampfkünstler könne sogar Dämonen bezwingen. Aus welchem Grund auch immer, der Meister weigerte sich. Der Dunkle Lord fragte, warum er kein Schüler sein könne. Der Meister sagte ihm, er habe keine Geduld und da riss der Dunkle Lord seine Zunge heraus."

Man schnappte kollektiv nach Luft.

"Sie können sich denken, was als nächstes geschah. Die Schüler wollten auf den Dunklen Lord einstürmen und fielen um, an Ort und Stelle erstarrt. Und dann…"

Professor Quirrells Stimme stockte für einen Moment, dann fuhr er fort.

"Es gibt einen Unverzeihlichen Fluch, den Cruciatus-Fluch, welcher unerträgliche Schmerzen hervorruft. Wenn der Cruciatus länger als für ein paar Minuten angewandt wird, verursacht er dauerhaften Wahnsinn. Einen nach dem anderen folterte der Dunkle Lord sie damit in den Irrsinn und beendete es dann mit dem Tödlichen Fluch, während der Meister gezwungen war zuzusehen. Als alle seine Schüler auf diese Art gestorben waren, folgte der Meister. Ich erfuhr dies von einem einzelnen überlebenden Schüler, welchen der Dunkle Lord am Leben gelassen hatte, um die Geschichte zu erzählen und der ein Freund von mir gewesen war…"

Professor Quirrell wandte sich ab und als er sich einen Moment später zurück drehte, wirkte er wieder ruhig und gefasst.

"Dunkle Zauberer können ihr Temperament nicht kontrollieren," sagte Professor Quirrell leise. "Es ist ein nahezu universeller Fehler ihrer Art und jeder, der sich ihrer Bekämpfung widmet, lernt bald sich darauf zu verlassen. Seien Sie sich im Klaren, dass der Dunkle Lord an diesem Tag \emph{keinen} Sieg errungen hat. Sein Ziel war es, die Kampfkünste zu erlernen und doch ging er ohne eine einzige Lektion. Der Dunkle Lord war töricht zu wünschen, diese Geschichte möge erzählt werden. Sie zeigte nicht seine Stärke, viel mehr eine auszunutzende Schwäche."

Professor Quirrells Blick richtete sich auf ein einziges Kind im Klassenraum.

"Harry Potter," sagte Professor Quirrell.

"Ja," sagte Harry mit heiserer Stimme.

"Was \emph{präzise} haben Sie heute falsch gemacht, Mr~Potter?"

Harry fühlte sich als müsse er sich übergeben. "Ich habe die Beherrschung verloren."

"Das ist \emph{nicht} präzise," sagte Professor Quirrell. "Ich werde es exakter beschreiben. Es gibt viele Tiere, bei denen sogenannte Dominanzkämpfe vorkommen. Sie rasen mit Hörnern aufeinander zu - versuchen einander zu Boden zu werfen, nicht sich aufzuspießen. Sie kämpfen mit ihren Tatzen - mit eingezogenen Klauen. Doch wieso mit eingezogenen Klauen? Sicherlich hätten sie, benutzten sie ihre Klauen, eine bessere Chance zu gewinnen? Doch dann könnte ihr Gegner ebenfalls die Klauen zücken und anstelle den Dominanzkampf mit einem Gewinner und einem Verlierer zu beenden, könnten sie beide schwer verletzt werden."

Professor Quirrells Blick schien aus dem Verstärker-Bildschirm heraus direkt auf Harry gerichtet zu sein. "Was Sie heute demonstriert haben, Mr~Potter, ist dass Sie - anders als jene Tiere, die ihre Klauen eingezogen lassen und das Ergebnis akzeptieren - nicht wissen, wie man einen Dominanzkampf verliert. Als ein \emph{Hogwarts-Professor} Sie herausgefordert hat, haben Sie nicht nachgegeben. Als es aussah als könnten Sie verlieren, zückten Sie Ihre Klauen, ungeachtet der Gefahr. Sie ließen die Situation \emph{eskalieren} und dann ließen Sie sie \emph{erneut} eskalieren. Es begann mit einem Klaps gegen Sie von Professor Snape, der offensichtlich die Dominanz über Sie hatte. Anstatt zu verlieren, schlugen Sie zurück und verloren zehn Punkte von Ravenclaw. Bald sprachen Sie davon, Hogwarts zu verlassen. Die Tatsache, dass Sie die Sache in eine unbekannte Richtung eskalieren ließen und irgendwie am Ende gewannen, ändert nichts an der Tatsache, dass Sie ein Idiot sind."

"Ich verstehe," sagte Harry. Seine Kehle war trocken. Das \emph{war} präzise gewesen. \emph{Beängstigend} präzise. Jetzt da Professor Quirrell es sagte, konnte Harry im Nachhinein sehen, dass es eine \emph{vollkommen} akkurate Beschreibung dessen war, was geschehen war. Wenn jemandes Modell von einem selbst so gut war, musste man sich fragen, ob er auch mit anderen Dingen recht hatte, wie seinem Killerinstinkt.

"Das \emph{nächste} mal, Mr~Potter, wenn Sie entscheiden einen Kampf eher eskalieren zu lassen, anstatt zu verlieren, könnten Sie \emph{alle} Einsätze verlieren, die Sie auf dem Tisch haben. Ich kann mir nicht vorstellen, welche es heute waren. Ich kann mir aber denken, dass sie viel, viel zu hoch waren für den Verlust von zehn Hauspunkten."

Wie das Schicksal des magischen Britannien. Das hatte er getan.

"Sie werden einwenden, dass Sie versucht haben, ganz Hogwarts zu helfen, ein viel wichtigeres Ziel, großer Risiken würdig. Das ist eine \emph{Lüge.} Hätten Sie -"

"Ich hätte den Klaps eingesteckt, gewartet und den bestmöglichen Zeitpunkt für meinen Zug gewählt," sagte Harry mit heiserer Stimme. "Aber das hätte bedeutet, \emph{zu verlieren.} Zuzulassen, dass er mich dominiert. Es war das, was der Dunkle Lord bei dem Meister nicht tun konnte, von dem er lernen wollte."

Professor Quirrell nickte. "Ich sehe, Sie haben vollkommen verstanden. Und daher, Mr~Potter, werden Sie heute lernen, wie man verliert."

"Ich -"

"Ich will keine Einwände hören, Mr~Potter. Es ist offenkundig, dass Sie dies sowohl brauchen als auch stark genug sind, es zu ertragen. Ich versichere Ihnen, dass Ihre Erfahrung nicht so hart wird, wie das was ich durchgemacht habe, doch Sie mögen es wohl als die unangenehmsten fünfzehn Minuten Ihres jungen Lebens in Erinnerung behalten."

Harry schluckte. "Professor Quirrell," sagte er mit kleiner Stimme, "können wir das ein andermal tun?"

"Nein," sagte Professor Quirrell schlicht. "Es ist Ihr fünfter Schultag in Hogwarts und dies ist bereits passiert. Heute ist Freitag. Unser \emph{nächster} Verteidigungs-Unterricht ist am Mittwoch. Samstag, Sonntag, Montag, Dienstag, Mittwoch… Nein, wir haben \emph{keine} Zeit zu verlieren."

Dem folgten ein paar Lacher, doch nur sehr wenige.

"Bitte verstehen Sie es als Anweisung Ihres Professors, Mr~Potter. Was ich gern sagen würde, ist dass ich Ihnen ansonsten keinerlei offensive Zauber beibringen werde, da mir dann zu Ohren käme, dass Sie jemanden schwer verletzt oder sogar getötet haben. Unglücklicherweise wurde mir zugetragen, dass Ihre Finger bereits mächtige Waffen sind. Schnipsen Sie damit zu keinem Zeitpunkt während dieser Lektion."

Mehr verstreutes Gelächter, es klang ziemlich nervös.

Harry war zum Heulen zumute. "Professor Quirrell, wenn Sie irgendetwas in der Art tun, worüber Sie sprachen, wird mich das wieder wütend machen und ich würde heute lieber nicht noch einmal wütend werden -"

"Es geht \emph{nicht} darum zu vermeiden, wütend zu werden," sein Gesichtsausdruck ernst. "Wut ist natürlich. Sie müssen lernen, wie Sie verlieren, selbst wenn Sie wütend sind. Oder wenigstens \emph{vorzugeben,} Sie würden verlieren, damit Sie Ihre Vergeltung \emph{planen} können. Wie ich es heute mit Mr~Goyle getan habe, es sei denn jemand von Ihnen denkt, er \emph{ist} wirklich besser -"

"Ich nicht!" rief Mr~Goyle von seinem Pult aus, leichte Verzweiflung in der Stimme. "Ich weiß, dass Sie nicht wirklich verloren haben! Bitte schmieden Sie keine Rachepläne!"

Harry hatte ein ungutes Gefühl im Magen. Professor Quirrell wusste nichts von seiner mysteriösen dunklen Seite. "Professor, wir müssen wirklich nach dem Unterricht darüber reden -"

"Werden wir," sagte Professor Quirrell, sein Tonfall wie ein Versprechen. "Nachdem Sie lernen, wie man verliert." Sein Gesicht war ernst. "Es sollte selbstverständlich sein, dass ich alles ausschließen werde, das Sie verletzen oder Ihnen auch nur signifikante Schmerzen zufügen könnte. Der Schmerz wird aus der Schwierigkeit zu verlieren stammen, anstatt zurückzuschlagen und den Kampf eskalieren zu lassen, bis Sie gewinnen."

Harrys Atem kam in kurzen, panischen Stößen. Er hatte noch mehr Angst als nach dem Verlassen des Zaubertränke-Klassenraums. "Professor Quirrell," brachte er heraus, "ich möchte nicht, dass Sie deshalb gefeuert werden -"

"Werde ich nicht," sagte Professor Quirrell, "wenn \emph{Sie} hinterher berichten, dass es notwendig war. Und ich vertraue Ihnen, dies zu tun." Einen Moment lang wurde Professor Quirrells Stimme sehr trocken. "Glauben Sie mir, schlimmeres ist in den Korridoren toleriert worden. Dieser Fall wird sich nur dadurch unterscheiden, dass es in einem Klassenraum geschieht."

"Professor Quirrell," flüsterte Harry, doch er nahm an, seine Stimme hallte trotzdem überall wieder, "glauben Sie wirklich, wenn ich das nicht tue, könnte ich jemanden verletzen?"

"Ja," sagte Professor Quirrell schlicht.

"Dann," Harry fühlte sich elend, "werde ich es tun."

Professor Quirrell wand sich zu den Slytherins um. "Also… mit der vollen Zustimmung Ihres Lehrers und auf eine Weise, dass Snape nicht für Ihre Handlungen verantwortlich gemacht werden kann… wünscht irgendeiner von Ihnen, dem Jungen-der-überlebt-hat seine Dominanz zu zeigen? Ihn herum zu schubsen, zu Boden zu stoßen, um Gnade bitten zu hören?"

Fünf Hände hoben sich.

"Alle die die Hände erhoben haben, Sie sind absolute Idioten. Welchen Teil von \emph{vorgeben} zu verlieren haben Sie nicht verstanden? Wenn Harry Potter der nächste Dunkle Lord wird, wird er Sie aufspüren und töten, nachdem er seinen Abschluss macht."

Die fünf Hände fielen abrupt zurück auf ihre Pulte.

"Werde ich nicht," sagte Harry, seine Stimme ziemlich schwächlich. "Ich schwöre niemals Rache zu nehmen an denen, die mir helfen zu lernen wie man verliert. Professor Quirrell… würden Sie \emph{bitte…} damit \emph{aufhören?}"

Professor Quirrell seufzte. "Es \emph{tut} mir leid, Mr~Potter. Mir ist klar, dass Sie dies in gleichem Maße ärgerlich finden müssen, ob Sie nun beabsichtigen der nächste Dunkle Lord zu werden oder nicht. Doch diese Kinder hatten \emph{ebenfalls} eine wichtige Lebenslektion zu lernen. Wäre es akzeptabel, Ihnen als Entschuldigung einen Quirrell-Punkt zu verleihen?"

"Machen Sie zwei daraus," sagte Harry.

Ein Ausbruch überraschten Gelächters erklang, entschärfte etwas die Anspannung.

"So sei es," sagte Professor Quirrell.

"Und nachdem ich meinen Abschluss habe, werde ich Sie aufspüren und \emph{durchkitzeln.}"

Es gab noch mehr Gelächter, obwohl Professor Quirrell nicht lächelte.

Harry fühlte sich als ränge er mit einer Anakonda bei dem Versuch die Unterhaltung durch das Nadelöhr zu zwingen, bei dem die Leute erkannten, dass er doch kein Dunkler Lord war… \emph{warum} hatte Professor Quirrell ihn nur so im Verdacht?"

"Professor," sagte Dracos unverstärkte Stimme. "Auch mein Ehrgeiz ist es nicht, ein dummer Dunkler Lord zu werden."

Schockierte Stille breitete sich im Klassenraum aus.

\emph{Du musst das nicht tun!} platzte Harry beinahe laut heraus, hielt sich jedoch rechtzeitig zurück; Draco wollte vielleicht nicht bekannt werden lassen, dass er das aus Freundschaft zu Harry tat… oder aus dem Wunsch heraus, freundlich zu erscheinen…

\emph{Das} als den Wunsch zu bezeichnen, \emph{freundlich zu erscheinen,} ließ Harry sich klein und gemein fühlen. Wenn Draco beabsichtigt hatte ihn zu beeindrucken, funktionierte es perfekt.

Professor Quirrell betrachtete Draco ernsthaft. "\emph{Sie} sorgen sich, Sie könnten nicht vorgeben zu verlieren, Mr~Malfoy? Dass der Makel, der Mr~Potter betrifft auch Ihnen zu eigen ist? \emph{Sicherlich} hat Ihr Vater Sie besseres gelehrt."

"Wenn es um Worte geht, vielleicht," sagte Draco, jetzt auf dem Verstärker-Bildschirm. "Nicht wenn es darum geht, herumgeschubst und zu Boden gestoßen zu werden. Ich möchte genauso stark wie Sie sein, Professor Quirrell."

Professor Quirrells Augenbrauen hoben sich und blieben oben. "Ich fürchte, Mr~Malfoy," sagte er nach einiger Zeit, "dass die Arrangements die ich für Mr~Potter getroffen habe, einschließlich mehrerer älterer Slytherins, die \emph{anschließend} erfahren werden, wie dumm sie waren, auf Sie nicht den gleichen Effekt hätten. Doch es ist meine professionelle Einschätzung, dass Sie bereits sehr stark sind. Sollte mir zu Ohren kommen, dass Sie versagt haben, wie Mr~Potter heute versagt hat, werde ich die angemessenen Arrangements treffen und mich bei Ihnen und wen auch immer Sie verletzt haben, entschuldigen. Ich denke allerdings nicht, dass dies notwendig sein wird."

"Ich verstehe, Professor," sagte Draco.

Professor Quirrell ließ den Blick über die Klasse schweifen. "Wünscht sonst noch jemand, stark zu werden?"

Einige Schüler blickten sich nervös um. Manche, so wirkte es von seiner hinteren Reihe aus auf Harry, sahen aus als öffneten sie den Mund, sagten aber nichts. Am Ende sprach niemand.

"Draco Malfoy wird einer der Generäle der Armeen Ihres Jahrgangs sein," sagte Professor Quirrell, "sollte er geneigt sein, sich an dieser außerschulischen Aktivität zu beteiligen. Und nun, Mr~Potter, bitte treten Sie vor."

\later

\emph{Ja,} hatte Professor Quirrell gesagt, \emph{es muss hier vor allen sein, vor Ihren Freunden, denn dort hat Snape Sie konfrontiert und dort müssen Sie lernen zu verlieren.}

Nun sah also das erste Schuljahr zu. In magisch erzwungener Stille und sowohl von Harry als auch dem Professor gebeten, nicht einzugreifen. Hermine hatte ihr Gesicht abgewandt, doch sie hatte sich nicht geäußert oder ihm gar einen bedeutungsvollen Blick zugeworfen, vielleicht weil sie in Zaubertränke ebenfalls anwesend gewesen war.

Harry stand auf einer weichen blauen Matte, wie man sie in einem Muggel-Dojo finden mochte, die Professor Quirrell dafür auf dem Boden ausgelegt hatte, wenn Harry zu niedergestoßen wurde.

Harry hatte Angst davor, was er tun könnte. Wenn Professor Quirrell mit seinem Killerinstinkt recht hatte…

Harrys Zauberstab lag auf Professor Quirrells Schreibtisch, nicht weil Harry irgendwelche Zauber gekannt hätte um sich zu schützen, sondern weil er ihn andererseits (dachte Harry) vielleicht in jemandes Augenhöhle zu stechen versucht hätte. Sein Beutel lag auch dort, der jetzt seinen zwar geschützten doch immer noch potentiell zerbrechlichen Zeitumkehrer enthielt.

Harry hatte Professor Quirrell gebeten, ihm ein paar Boxhandschuhe zu transfigurieren und an seinen Händen zu befestigen. Professor Quirrell hatte ihn mit einem Blick stillen Verstehens bedacht und abgelehnt.

\emph{Ich werde ihnen nicht die Augen auskratzen, ich werde ihnen nicht die Augen auskratzen, ich werde ihnen nicht die Augen auskratzen, mein Leben in Hogwarts wäre zu Ende, ich werde eingesperrt,} sang Harry zu sich selbst, versuchte sich den Gedanken ins Hirn zu hämmern, in der Hoffnung er würde bleiben, wenn sein Killerinstinkt übernahm.

Professor Quirrell kehrte zurück, in Begleitung dreizehn älterer Slytherins aus verschiedenen Jahrgängen. Harry erkannte einen von ihnen als denjenigen, den er mit einem Kuchen getroffen hatte. Zwei andere von dieser Auseinandersetzung waren ebenfalls anwesend. Derjenige, der gesagt hatte, sie sollten aufhören, dass sie das wirklich nicht tun sollten, fehlte.

"Ich wiederhole," sagte Professor Quirrell und klang sehr streng, "Potter darf \emph{nicht} tatsächlich verletzt werden. Jeder einzelne \emph{Unfall} wird als vorsätzlich behandelt. Haben Sie verstanden?"

Die älteren Slytherins nickten, grinsend.

"Dann dürfen Sie nun gern den Jungen-der-überlebt-hat ein wenig zurechtstutzen," sagte Professor Quirrell mit einem schiefen Lächeln, das nur die Erstklässler verstanden.

Durch eine Art stille Übereinkunft befand sich das Kuchen-Opfer an der Spitze der Gruppe.

"Potter," sagte Professor Quirrell, "das ist Mr~Peregrine Derrick. Er ist besser als Sie und das wird er Ihnen gleich zeigen."

Derrick trat vor und Harrys Gehirn schrie disharmonisch, er durfte nicht wegrennen, er durfte nicht zurückschlagen -

Derrick stoppte eine Armlänge von Harry entfernt.

Harry war noch nicht zornig, nur angsterfüllt. Und das hieß, er sah sich einem Teenager gegenüber, einen vollen halben Meter größer als er, mit sichtbar definierten Muskeln, Gesichtsbehaarung und einem Grinsen voll schrecklicher Vorfreude.

"Bitten Sie ihn, Ihnen nicht weh zu tun," sagte Professor Quirrell. "Vielleicht, wenn er sieht dass Sie erbärmlich genug sind, findet er Sie langweilig und verschwindet."

Gelächter erklang von den zuschauenden älteren Slytherins.

"Bitte," sagte Harry, seine Stimme versagte, "tu, mir, nicht, weh…"

"Das klang nicht sehr überzeugend," sagte Professor Quirrell.

Derricks Grinsen wurde breiter. Der ungehobelte Schwachkopf blickte sehr überlegen drein und…

… Harrys Blut-Temperatur fiel…

"Bitte tu mir nicht weh," versuchte es Harry erneut.

Professor Quirrell schüttelte den Kopf. "Wie in Merlins Namen konnten Sie das wie eine Beleidigung klingen lassen, Potter? Es gibt nur eine Antwort, die Sie von Mr~Derrick darauf zu erwarten haben."

Derrick schritt absichtlich nach vorn und stieß mit Harry zusammen.

Harry stolperte ein paar Fuß zurück, bevor er sich fangen konnte und richtete sich eisig auf.

"Falsch," sagte Professor Quirrell, "falsch, falsch, falsch."

"Du hast mich angerempelt, Potter," sagte Derrick. "Entschuldige dich."

"Tut mir leid!"

"Du \emph{klingst} nicht, als ob es dir leid tut," sagte Derrick.

Harrys Augen wurden weit vor Empörung, er \emph{hatte} das flehentlich klingen lassen -

Derrick stieß ihn, hart und Harry fiel mit Händen und Knien auf die Matte.

Der blaue Stoff schien in Harrys Blick zu verschwimmen, nicht weit entfernt.

Er begann Professor Quirrells Motive für diese sogenannte \emph{Lektion} anzuzweifeln.

Ein Fuß lastete auf Harrys Hinterteil und einen Moment später wurde Harry hart zur Seite gestoßen und fiel der Länge nach auf den Rücken.

Derrick lachte. "Das macht \emph{Spaß,}" sagte er.

Er musste nur sagen, dass es vorbei war. Und die ganze Sache dem Büro des Schulleiters melden. Das wäre das Ende dieses \emph{Verteidigungs-Professors} und seines unglückseligen Aufenthalts in Hogwarts und… Professor McGonagall wäre wütend darüber, aber…

(Ein Bild von Professor McGonagalls Gesicht blitzte vor seinem geistigen Auge auf, sie sah nicht wütend aus, nur traurig -)

"Jetzt sagen Sie ihm, dass er besser ist als Sie, Potter," sagte Professor Quirrells Stimme.

"Du bist, besser, als ich."

Harry wollte sich erheben und Derrick setzte einen Fuß auf seine Brust und stieß ihn zurück auf die Matte.

Die Welt wurde kristallklar. Mögliche Handlungsverläufe und deren Konsequenzen breiteten sich in vollkommener Schärfe vor ihm aus. Der Narr würde nicht erwarten, dass er zurückschlug, ein schneller Treffer in die Leistengegend würde ihn lange genug außer Gefecht setzen um -

"Versuchen Sie es erneut," sagte Professor Quirrell und mit einer plötzlichen schnellen Bewegung rollte Harry herum und sprang auf die Füße und wirbelte herum, dorthin wo sein wahrer Feind stand, der Verteidigungs-Professor -

Professor Quirrell sagte, "Sie haben keine Geduld."

Harry zögerte. Sein Geist, gut trainiert in der Kunst des Pessimismus, zeichnete ihm das Bild eines verwelkten alten Mannes dem Blut aus seinem Mund lief, nachdem Harry seine Zunge herausgerissen hatte -

Einen Augenblick später schickte Derrick Harry erneut auf die Matte und ließ sich dann auf ihm nieder, wodurch Harrys Atem keuchend entwich.

"Stop!" schrie Harry. "Bitte hör auf!"

"Besser," sagte Professor Quirrell. "Das klang sogar aufrichtig."

Das \emph{war} es gewesen. Das war das Schreckliche, das Üble, es \emph{war} aufrichtig gewesen. Harry atmete schnell und stoßweise, Angst und kalter Zorn durchströmten ihn -

"Verlieren Sie," sagte Professor Quirrell.

"Ich, verliere," presste Harry hervor.

"Das gefällt mir," sagte Derrick auf ihm drauf. "Verlier noch weiter."

\later

Hände schoben Harry, stießen ihn stolpernd ihm Kreis der älteren Slytherins zu einem weiteren paar Hände, die ihn erneut schubsten. Harry hatte den Punkt, an dem er versucht hatte nicht zu weinen längst hinter sich gelassen und versuchte jetzt nur noch, nicht zu Boden zu stürzen.

"Was bist du, Potter?" sagte Derrick.

"Ein, V-Verlierer, ich verliere, du gewinnst, du bist besser, als ich, bitte hör auf -"

Harry stolperte über einen Fuß und ging krachend zu Boden, konnte sich mit den Händen nicht ganz anfangen. Er war einen Moment lang benommen, dann kam er mühsam wieder auf die Füße -

"\emph{Genug!}" sagte Professor Quirrells Stimme, klang scharf genug, um Eisen zu schneiden. "Entfernen Sie sich von Mr~Potter!"

Harry sah den überraschten Ausdruck auf ihren Gesichtern. Die Kälte in Harrys Blut, die ihn durchflutete und wieder abgeebbt war, lächelte mit kalter Genugtuung.

Dann brach Harry auf der Matte zusammen.

Professor Quirrell sprach. Man hörte die älteren Slytherins nach Luft schnappen.

"Und ich glaube der Spross von Malfoy hat Ihnen ebenfalls etwas zu sagen," endete Professor Quirrell.

Dracos Stimme begann zu sprechen. Sie klang beinahe so scharf wie die von Professor Quirrell, sie hatte den selben Tonfall amgenommen, den Draco benutzt hatte, um seinen Vater zu imitieren und sie sagte Dinge wie \emph{hätte das Haus Slytherin in Gefahr bringen können} und \emph{wer weiß wie viele Verbündete allein in dieser Schule} und \emph{überhaupt kein Bewusstsein für} \emph{irgendwas,} \emph{vonGerissenheit ganz zu schweigen} und \emph{stumpfsinnige Schläger, zu nichts als Lakaien zu gebrauchen} und irgendwas in Harrys Hinterkopf stempelte Draco, gegen alles was er wusste, als Verbündeten ab.

Harry tat alles weh, er hatte wahrscheinlich Prellungen, sein Körper fühlte sich kalt an, sein Geist völlig erschöpft. Er versuchte an Fawkes Lied zu denken, doch ohne die Gegenwart des Phoenix konnte er sich nicht an die Melodie erinnern und versuchte er, sie sich vorzustellen, wollte ihm nichts außer einem Vogelzwitschern in den Sinn kommen.

Dann hörte Draco auf zu sprechen und Professor Quirrell sagte den älteren Slytherins, sie seien entlassen und Harry öffnete die Augen und mühte sich, sich aufzusetzen, "Warten Sie," sagte Harry, presste die Worte hervor, "es gibt etwas, dass ich, ihnen sagen, will -"

"Warten Sie auf Mr~Potter," sagte Professor Quirrell kalt zu den sich entfernenden Slytherins.

Harry kam schwankend auf die Füße. Er blickte sorgsam nicht in Richtung seiner Klassenkameraden. Er wollte nicht sehen, wie sie ihn jetzt gerade ansahen. Er wollte ihr Mitleid nicht sehen.

Also blickte Harry stattdessen zu den älteren Slytherins, die sich immer noch in einem Schockzustand zu befinden schienen. Grauen zeichnete sich auf ihren Gesichtern ab.

Seine dunkle Seite hatte sich, als sie die Kontrolle hatte, an der Vorstellung dieses Augenblicks festgehalten und gab weiter vor, zu verlieren.

Harry sagte, "Niemand wird -"

"Stop," sagte Professor Quirrell. "Wenn das ist, was ich denke, warten Sie bitte bis nachdem sie verschwunden sind. Sie werden später davon hören. Wir haben alle unsere Lektionen zu lernen, Mr~Potter."

"Einverstanden," sagte Harry.

"Sie. Gehen Sie."

Die älteren Slytherins flohen und die Tür schloss sich hinter ihnen.

Niemand wird Rache an ihnen nehmen," sagte Harry heiser. "Das ist eine Bitte an alle, die sich als meine Freunde betrachten. Ich hatte meine Lektion zu lernen, sie halfen mir sie zu lernen, sie hatten auch ihre Lektion zu lernen, es ist vorbei. Wenn ihr diese Geschichte erzählt, dann bitte auch diesen Teil."

Harry drehte sich zu Professor Quirrell um.

"Sie haben verloren," sagte Professor Quirrell, seine Stimme zum ersten mal freundlich. Es klang seltsam von dem Professor, als sollte seine Stimme gar nicht dazu in der Lage sein.

Harry \emph{hatte} verloren. Es hatte Momente gegeben in denen der kalte Zorn völlig hin fort gewaschen war, ersetzt durch Angst und während dieser Augenblicke hatte er die älteren Slytherins angefleht und es ernst gemeint…

"Und sind Sie noch am Leben?" sagte Professor Quirrell, noch immer mit jener seltsamen Freundlichkeit.

Harry brachte ein Nicken zustande.

"Nicht immer ist das Verlieren so," sagte Professor Quirrell. "Es gibt Kompromisse und ausgehandelte Kapitulationen. Es gibt andere Wege um Mobber zu beschwichtigen. Es liegt eine ganze Kunstform darin, andere zu manipulieren, indem man sich von ihnen dominieren lässt. Doch zunächst muss das Verlieren \emph{denkbar} sein. Werden Sie sich erinnern, wie Sie verloren?"

"Ja."

"Werden Sie in der Lage sein zu verlieren?"

"Ich… denke schon…"

"Das denke ich auch." Professor Quirrell verneigte sich so tief, dass sein dünnes Haar beinahe den Boden berührte. "Gratulation, Harry Potter, Sie gewinnen."

Es gab keinen einzelnen Ursprung, niemanden der sich zuerst bewegte, der Applaus startete mit einem Mal, wie ein gewaltiger Donnerschlag.

Harry konnte den Schock auf seinem Gesicht nicht verbergen. Er riskierte einen Blick auf seine Klassenkameraden und er sah auf ihren Gesichtern kein Mitleid, sondern Ehrfurcht. Der Applaus kam von Ravenclaw und Gryffindor und Hufflepuff und sogar Slytherin, wahrscheinlich weil Draco Malfoy ebenfalls applaudierte. Einige Schüler standen von ihren Stühlen auf und halb Gryffindor stand auf den Tischen.

So stand Harry da, schwankend, ließ ihren Respekt über sich hinweg branden, fühlte sich stärker und vielleicht auch ein wenig wiederhergestellt.

Professor Quirrell wartete, bis der Applaus erstarb. Es dauerte eine ganze Weile.

"Überrascht, Mr~Potter?" sagte Professor Quirrell. Er klang amüsiert. "Sie haben gerade erfahren, dass die echte Welt nicht \emph{immer} Ihren schlimmsten Alpträumen gleicht. Ja, wären Sie ein armer namenloser Junge gewesen, der missbraucht wurde, hätten sie Sie hinterher wahrscheinlich weniger respektiert, Sie bemitleidet, während sie Sie von ihrem hohen Ross aus trösten. Das \emph{ist} die Natur des Menschen, fürchte ich. Doch \emph{Sie} sind ihnen bereits als mächtige Gestalt vertraut. Und sie sahen, wie Sie sich Ihrer Angst stellten und nicht zurückwichen, obwohl Sie jederzeit hätten gehen können. Dachten Sie schlechter von \emph{mir,} nachdem Sie wussten, dass ich es bewusst über mich ergehen ließ, bespuckt zu werden?"

Harry fühlte ein Brennen in seiner Kehle und hielt es verzweifelt zurück. Er traute diesem wundersamen Respekt nicht genug, um erneut davor in Tränen auszubrechen.

"Ihre \emph{außerordentliche} Leistung in meinem Unterricht verdient eine außerordentliche Belohnung, Harry Potter. Bitte nehmen Sie sie an, zusammen mit meinen besten Wünschen im Namen meines Hauses und wissen Sie von diesem Tage an, dass nicht alle Slytherins gleich sind. Es gibt Slytherins und dann gibt es Slytherins." Professor Quirrell lächelte breit, als er das sagte. "Einundfünfzig Punkte für Ravenclaw."

Es gab eine schockierte Pause und dann brach ein Pandämonium aus heulenden, pfeifenden und jubelnden Ravenclaws los.

(Und im selben Moment fühlte Harry, dass etwas daran \emph{falsch} war, Professor McGonagall hatte recht gehabt, es \emph{sollte} Konsequenzen geben, es hätte einen Preis gezahlt werden müssen, man konnte nicht einfach so alles wieder zurückdrehen -)

Doch Harry sah die freudigen Gesichter in Ravenclaw und wusste, er konnte unmöglich ablehnen.

Sein Hirn machte einen Vorschlag. Es war ein guter Vorschlag. Harry konnte kaum glauben, dass sein Gehirn ihn immer noch aufrecht hielt, ganz zu schweigen davon, gute Vorschläge zu machen.

"Professor Quirrell," sagte Harry, so deutlich wie er durch seine brennende Kehle konnte. "Sie sind alles, was ein Mitglied Ihres Hauses sein sollte und ich glaube, Sie müssen genau das sein, was Salazar Slytherin im Sinn hatte, als er Hogwarts mit gegründet hat. Ich danke Ihnen und Ihrem Haus," Draco nickte ganz leicht und ließ unmerklich den Finger kreisen, \emph{mach weiter,} "und ich denke das schreit nach einem dreifachen Hoch für Slytherin. Alle zusammen?" Harry hielt inne. "\emph{Hurra!}" Nur ein paar Leute schafften es beim ersten Versuch mit einzustimmen. "\emph{Hurra!}" Diesmal war der Großteil von Ravenclaw dabei. "\emph{Hurra!}" Das war fast ganz Ravenclaw, ein paar versprengte Hufflepuffs und etwa ein Viertel von Gryffindor.

Dracos Hand machte eine kleine, schnelle Daumen-hoch-Geste.

Die meisten Slytherins blickten völlig schockiert drein. Ein paar starrten verwundert Professor Quirrell an. Blaise Zabini betrachtete Harry mit berechnender Neugier.

Professor Quirrell verneigte sich. "Danke \emph{Ihnen,} Harry Potter," sagte er, noch immer mit diesem breiten Lächeln. Er wandte sich an die Klasse. "Nun, glauben Sie es oder nicht, wir haben noch eine halbe Stunde Unterricht übrig und das ist genug, um mit dem Simplen Schild anzufangen. Mr~Potter wird, natürlich, verschwinden und eine wohl verdiente Pause machen."

"Ich kann -"

"Dummkopf," sagte Professor Quirrell nachsichtig. Die Klasse lachte bereits. "Ihre Klassenkameraden können es Ihnen hinterher beibringen oder ich gebe Ihnen private Nachhilfestunden, falls nötig. Doch \emph{jetzt} gehen Sie durch die dritte Tür von links hinter der Bühne, wo Sie ein Bett, eine Auswahl außergewöhnlich leckerer Snacks und ein wenig sehr leichte Lektüre aus der Hogwarts-Bibliothek vorfinden werden. Sie dürfen nichts weiter mitnehmen, besonders nicht Ihre Lehrbücher. Nun gehen Sie."

Harry ging.

* Harry zitiert aus Winston Churchills \emph{Blut-Schweiß-und-Tränen-Rede} zu seinem Amtsantritt vor dem Unterhaus des britischen Parlaments, Übersetzung aus Winston Churchill: \emph{Reden in Zeiten des Krieges.} Aus dem Englischen von Walther Weibel.

