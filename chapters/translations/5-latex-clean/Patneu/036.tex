

\hypertarget{statusunterschiede}{% \section{36. Statusunterschiede}\label{statusunterschiede}}

\textbf{Kapitel 36: Statusunterschiede}\\

Schmerzliche Desorientierung, so fühlte es sich an, aus dem Zugang zum Gleis neundreiviertel hinaus und wieder in den Rest der Welt einzutreten, von dem Harry einst geglaubt hatte, er sei die einzig echte Welt. Die Menschen kleideten sich in lockere Hemden und Hosen anstelle der würdevolleren Umhänge der Hexen und Zauberer. Verstreute Häufchen von Abfall hier und da zwischen den Bänken. Ein vergessener Geruch, die Abgase verbrannten Benzins, lag scharf und stechend in der Luft. Das Ambiente des Bahnhofs King's Cross, weniger hell und wuselig als Hogwarts oder die Winkelgasse; die Menschen schienen geduckter, ängstlicher und hätten wahrscheinlich mit Freuden ihre Probleme gegen einen dunklen Zauberer eingetauscht, den es zu bekämpfen galt. Harry hätte gern den Dreck und Müll mit einem \emph{Ratzeputz} und \emph{Everto} beseitigt und wenn er den Spruch dafür gekannt hätte, auch einen Kopfblasenzauber verwendet, um die Luft nicht einatmen zu müssen. Aber an diesem Ort konnte er seinen Zauberstab nicht verwenden...

So, wurde Harry klar, musste es sich wohl anfühlen, aus einem Land der Ersten Welt in ein Dritte-Welt-Land zu kommen.

Nur war es die Nullte Welt, die Harry verlassen hatte, die Zauberwelt der Reinigungszauber und Hauselfen; wo man mit Hilfe der Künste von Heilern und eigener Magie durchaus hundertsiebzig werden mochte, bevor einem das Alter wirklich zu schaffen machte.

Und das nicht-magische London, die Muggel-Erde, auf die Harry zeitweise zurückgekehrt war. Hier würden Mum und Dad den Rest ihres Lebens verbringen, falls nicht die Technologie einen gewaltigen Sprung nach vorn machen und die Lebensqualität der Zauberer noch übertreffen oder sich irgendetwas in der Welt tiefgreifend verändern sollte.

Unbewusst wandte Harry den Kopf und hielt Ausschau nach dem hölzernen Koffer, der ihm - von sämtlichen Muggeln unbemerkt - auf klauenbewehrten Tentakteln nacheilend, eine rasche Bestätigung bot, dass - ja - er sich das nicht alles nur eingebildet hatte...

Und dann war da noch der andere Grund, warum es ihm die Brust zuschnürte.

Seine Eltern hatte keine Ahnung.

Sie wussten \emph{überhaupt nichts.}

Sie wussten nicht...

„Harry?“ rief eine dünne, blonde Frau, deren vollkommen weiche und makellose Haut sie erheblich jünger erscheinen ließ als dreiundreißig und mit einem Mal erkannte Harry: Ja, es \emph{war} Magie; zuvor hatte er die Anzeichen nicht erkannt, doch jetzt konnte er sie klar und deutlich sehen. Und was für ein Zaubertrank auch immer derart lange wirken mochte, er musste unglaublich gefährlich gewesen sein, denn die meisten Hexen taten sich das nicht an, so verzweifelt waren sie nicht...

Wasser sammelte sich in Harrys Augen.

„\emph{Harry?}“ rief ein älter wirkender Mann, dessen Bauch sich über der Hose leicht wölbte, mit auffällig zur Schau gestellter akademischer Lässigkeit gekleidet in eine schwarze Weste über einem dunklen grau-grünen Hemd; jemand der immer und überall wohin er auch ging stets ein Professor sein würde, der sicherlich einer der brillantesten Zauberer seiner Generation geworden wäre, wäre er nur mit zwei Kopien jenes Gens geboren worden, anstatt mit gar keinem...

Harry hob die Hand und winkte ihnen zu. Er konnte nicht sprechen. Er brachte kein einziges Wort heraus.

Sie kamen zu ihm herüber, nicht rennend, sondern stetigen, würdevollen Schrittes; denn so schnell ging Professor Michael Verres-Evans nun einmal und Mrs. Petunia Evans-Verres würde ihn keinesfalls überholen.

Das Lächeln auf dem Gesicht seines Vaters war nicht sehr breit, allerdings war sein Vater auch nie ein großer Lächler gewesen; es war jedenfalls mindestens so breit wie Harry es je gesehen hatte, breiter als wenn ein neuer Forschungszuschuss bewilligt wurde oder einer seiner Studenten eine neue Stelle bekam und ein breiteres Lächeln konnte man sich nicht wünschen.

Mum blinzelte heftig und versuchte zu lächeln, doch es wollte ihr nicht so recht gelingen.\\ „Also!“ sagte sein Vater als er auf ihn zukam. „Schon irgendwelche revolutionären Entdeckungen gemacht?"

Natürlich glaubte Dad, er mache Witze.

Es hatte noch nicht ganz so sehr geschmerzt, dass seine Eltern nicht an ihn glaubten, als es auch sonst niemand getan hatte; damals als Harry noch nicht \emph{gewusst} hatte, wie es sich anfühlte, wenn Menschen wie Schulleiter Dumbledore oder Professor Quirrell einen ernst nahmen.

Und da wurde Harry bewusst, dass der Junge-der-überlebt-hatte nur im magsichen Britannien existierte, dass es eine solche Person im London der Muggel einfach nicht gab, nur einen süßen kleinen elfjährigen Jungen, der über Weihnachten nach Hause kam.

„Entschuldigt bitte,“ sagte Harry mit zitternder Stimme, „ich werde jetzt wahrscheinlich gleich in Tränen ausbrechen; das bedeutet aber nicht, dass in der Schule irgendetwas passiert wäre."

Harry machte einen Schritt nach vorn und hielt dann inne, hin- und hergerissen zwischen der Umarmung seines Vaters und seiner Mutter, er wollte auf keinen Fall, dass sich einer von ihnen zurückgesetzt fühlte oder glaubte, Harry liebe ihn mehr als den anderen -

„Sie,“ sagte sein Vater, „sind ein höchst alberner junger Mann, Mr. Verres“ und mit diesen Worten fasste er ihn sanft an den Schultern und schob ihn in die Arme seiner Mutter, die vor ihm kniete und der bereits die Tränen über die Wange liefen.

„Hallo, Mum,“ sagte Harry mit zitternder Stimme, „ich bin zurück.“ Und er umarmte sie, inmitten der lärmenden Geräusche der Mechanik und des Geruchs nach verbranntem Benzin und Harry fing an zu weinen, denn er wusste, dass es kein Zurück mehr geben \emph{konnte,} am allerwenigsten für ihn.

--------------------------------------------------------------------------------------------------------------------------------------------

\hfill\break Der Himmel war bereits vollkommen dunkel und die Sterne kamen zum Vorschein als sie sich endlich ihren Weg durch den Weihnachtsverkehr der Universitätsstadt namens Oxford gebahnt hatten und in der Auffahrt des kleinen, leicht heruntergekommen wirkenden alten Hauses zum Stehen, das ihrer Familie dazu diente, den Regen von ihren Büchern fernzuhalten.

Als sie das kurze Stück des Pflasters bis zu ihrer Haustür überquerten, kamen sie an einer Reihe von Blumentöpfen vorbei, in denen kleine, gedimmte elektrische Leuchten steckten (gedimmt deshalb, weil sie sich des Tags durch Solarenergie selbst wiederaufladen mussten) und die Lichter leuchteten bei ihrem Eintreffen auf. Am schwierigsten war es gewesen, Bewegungssensoren zu finden, die wasserdicht waren und bei genau der richtigen Entfernung auslösten...

In Hogwarts gab es echte Fackeln in der Art.

Und dann öffnete sich die Haustür und Harry trat ins Wohnzimmer, blinzelte mehrmals.

\emph{Jeder freie Zoll an Wandfläche wurde von einem Bücherregal verdeckt. Jedes Bücherregal hatte sechs Einlegeböden und reichte fast bis zur Decke. Manche Bücherregale waren bis zum Bersten mit gebundenen Büchern vollgestopft: Wissenschaft, Mathematik, Geschichte und alles was es sonst noch gab. Andere Fächer enthielten zwei Reihen von Taschenbüchern voller Science Fiction, die rückwärtige Reihe auf Taschentuch-Verpackungen oder zwei mal vier Zoll großen Holzkästen aufgereiht, so dass man die Buchrücken über die Bücher davor noch sehen konnte. Und es reichte immer noch nicht aus. Bücher stapelten sich auf den Tischen und Sitzgelegenheiten und bildeten kleine Häuflein unter den Fenstern...}

Der Haushalt der Verres war genau so, wie er ihn zurückgelassen hatte, nur mit mehr Büchern und auch das war ganz genau so, wie er es kannte.

Und ein Weihnachtsbaum, zwei Tage vor Heiligabend noch nackt und ungeschmückt, was Harry kurz aus der Bahn warf, bevor ihm mit einem warmen Gefühl in seiner Brust die Erkenntnis kam, dass seine Eltern natürlich \emph{gewartet} hatten.

„Wir haben übrigens das Bett aus deinem Zimmer geräumt, um Platz für mehr Bücherregale zu machen,“ sagte sein Vater. „Du kannst doch in deinem Koffer schlafen, nicht wahr?"

„\emph{Du} kannst in meinem Koffer schlafen,“ sagte Harry.

„Da fällt mir wieder ein,“ sagte sein Vater. „Was \emph{haben} sie denn nun eigentlich wegen deinem Schlafzyklus unternommen?"

„Magie,“ sagte Harry und raste schnell zur Tür, die in sein Zimmer führte, nur für den Fall, dass Dad \emph{keine} Witze machte...

„Das ist keine Erklärung!“ sagte Professor Verres-Evans, genau als Harry rief, „\emph{Du hast den ganzen freien Platz auf meinen Bücherregalen belegt?}„

--------------------------------------------------------------------------------------------------------------------------------------------

\hfill\break Den 23. Dezember hatte Harry damit verbracht, noch ein paar Muggel-Dinge einzukaufen, die er nicht einfach transfigurieren konnte; sein Vater war beschäftigt gewesen und hatte gemeint, Harry müsse wohl zu Fuß gehen oder den Bus nehmen, was Harry nur allzu gut passte. Im Baumarkt hatten ihm zwar einige Leute fragende Blicke zugeworfen, aber er hatte mit unschuldiger Stimme verkündet, dass sein Vater nebenan beim Einkaufen und sehr beschäftigt sei und ihn deshalb vorausgeschickt hatte, um ein paar Dinge zu besorgen (wobei er eine Liste in erwachsen wirkender, kaum zu entziffernder Handschrift emporstreckte) und letztendlich war Geld immer noch Geld.

Den Weihnachtsbaum hatten sie alle zusammen geschmückt und Harry hatte eine winzig kleine tanzende Fee auf die Spitze gesetzt (für zwei Sickel und fünf Knuts bei Gambol \& Japes).

Gringotts hatte ihm bereitwillig Galleonen gegen Papiergeld eingetauscht, doch schienen sie keine einfache Möglichkeit zu haben, um größere Mengen an Gold als steuerfreies, unauffälliges Muggelgeld auf ein Schweizer Nummernkonto zu transferieren. Das hatte Harrys Plan einen ziemlichen Dämpfer versetzt, den Großteil des Geldes, das von sich selbst zu stehlen ihm gelungen war, in einem vernünftigen Mix aus 60\% in internationalen Indexfonds und 40\% in Aktien von Berkshire Hathaway* anzulegen. Für den Moment hatte sich Harry damit begnügt, seine Vermögenswerte etwas breiter anzulegen, indem er sich unsichtbar und mit Hilfe des Zeitumkehrers spät in der Nacht hinausgeschlichen und einhundert goldene Galeonen in seinem Hinterhof vergraben hatte. Das hatte er ohnehin schon immer, immer, \emph{immer} einmal tun wollen.

Ein Teil des 24. Dezembers ging damit herum, dass der Professor Harrys Bücher durchlas und ihm Fragen dazu stellte. Die meisten Experimente, die sein Vater vorgeschlagen hatte, waren unpraktikabel, zumindest für den Moment und von den verbleibenden hatte Harry viele bereits durchgeführt. („Ja, Dad, ich habe schon versucht, was passiert, wenn Hermine eine veränderte Aussprache beigebracht wird und sie wusste nicht, ob sie verändert wurde, das war das erste Experiment, das ich gemacht habe, Dad!“)

Die letzte Frage, die Harrys Vater gestellt hatte, als er mit einem Ausdruck verwirrten Abscheus auf dem Gesicht von \emph{Zaubertränke und Zauberbräue} aufblickte, war ob das alles Sinn ergebe, wenn man ein Zauberer sei und Harry hatte geantwortet, nein.

Woraufhin sein Vater erklärt hatte, Magie sei einfach unwissenschaftlich.

Harry war immer noch ein wenig schockiert ob der Vorstellung, etwas das Teil der \emph{Realität} war für unwissenschaftlich zu erklären. Dad schien zu glauben, dass der Widerspruch zwischen seiner Intuition und dem Universum bedeutete, das Universum hätte ein Problem.

(Andererseits gab es auch eine Menge Physiker, die die Quantenmechanik für seltsam hielten, anstatt die Quantenmechanik für normal und sich selbst für seltsam.)

Harry hatte seiner Mutter die Heilausrüsting gezeigt, die er für ihr Zuhause gekauft hatte, obwohl die meisten Tränke bei Dad nicht funktionieren würden. Die Art und Weise wie Mum die Ausrüstung anstarrte, veranlasste Harry zu der Frage, ob Mums Schwester jemals so etwas für Opa Edwin und Oma Elaine gekauft hatte. Als Mum nicht darauf antwortete, meinte Harry hastig, sie habe wohl einfach nie daran gedacht. Schließlich hatte er schnell das Zimmer verlassen.

Das Traurige war, Lily Evans \emph{hatte} wahrscheinlich wirklich nie daran gedacht. Harry wusste, dass andere Menschen dazu neigten, nicht über schmerzhafte Themen nachzudenken; auf die selbe Weise, wie sie dazu neigten, die Hände nicht auf einer heißen Herdplatte abzulegen und Harry begann zu vermuten, dass die meisten Muggelgeborenen rasch eine Neigung dazu entwickelten, nicht allzu viel über ihre Familienangehörigen nachzudenken, die ohnehin alle sterben würden, bevor sie ihr hundertstes Lebensjahr erreichten.

Natürlich nicht, dass Harry auch nur die geringste Absicht hatte, \emph{das} zuzulassen.

Und dann kam der Abend des 24. Dezembers und sie fuhren los zu ihrer Essensverabredung am Weihnachtsabend.

--------------------------------------------------------------------------------------------------------------------------------------------

\hfill\break Das Haus war riesig; nicht nach den Maßstäben von Hogwarts, doch sicherlich gemessen an dem, was man erwarten konnte, wenn der eigene Vater als angesehener Professor versuchte in Oxford zu leben. Zwei Stockwerke aus Ziegelsteinen glänzten in der untergehenden Sonne, mit zwei Reihen Fenstern und einem großen Fenster, dass viel weiter nach oben reichte als man hätte erwarten sollen; das würde wohl ein großes Wohnzimmer sein...

Harry atmete tief durch und betätigte die Türklingel.

Ein entfernter Ruf erklang, „Liebling, kannst du aufmachen?"

Gefolgt vom gemächlichen Trapsen sich nähernder Schritte.

Dann öffnete sich die Tür und offenbarte einen gesellig wirkenden Mann mit feisten, rosigen Wangen und zurückgehendem Haaransatz in einem leicht gewölbten, blauen Button-Down-Hemd.

„Dr. Granger?“ sagte Harrys Vater forsch, bevor Harry auch nur den Mund aufbekam. „Ich bin Michael und das sind Petunia und unser Sohn Harry. Das Essen ist in dem magischen Koffer,“ woraufhin Dad mit einer vagen Geste hinter sich deutete - nicht ganz in Richtung des Koffers, wie sich zeigte.

„Ja, bitte, kommen Sie herein,“ sagte Leo Granger. Er trat nach vorn und entnahm mit einem gemurmelten „Vielen Dank“ den ausgestreckten Händen des Professors die mitgebrachte Weinflasche, dann trat er zurück und deutete in Richtung Wohnzimmer. „Bitte, setzen Sie sich doch. Und,“ senkte er den Blick zu Harry hinab, „die Spielsachen sind alle unten im Keller; ich bin sicher Hermi kommt gleich runter, es ist die erste Tür rechts von dir,“ deutete er in Richtung eines Flurs.

Harry blickte ihn nur einen Moment lang an, sich bewusst dass er seinen Eltern den Eintritt versperrte.

„Spielzeug?“ sagte Harry mit kindlich hoher Stimme und weit aufgerissenen Augen. „Ich liebe Spielzeug!"

Hinter ihm sog seine Mutter die Luft ein und Harry schritt ins Haus, wobei er es fertig brachte nicht allzu fest aufzutreten.

Das Wohnzimmer war ganz genau so groß, wie es von außen gewirkt hatte, mit getäfelter Decke, von der ein gigantischer Kronleuchter hing und einem Weihnachtsbaum, den durch die Tür zu befördern ein mörderisches Unterfangen gewesen sein musste. Die unteren Etagen des Baumes waren sorgsam und gründlich geschmückt worden, in einem hübschen Muster aus rot, grün und gold, mit ein paar neueren Einsprengseln aus blau und bronze; jene Bereiche, die nur ein Erwachsener zu erreichen vermochte, waren achtlos mit zufällig drapierten Lichterketten und Gebinden aus Lametta behängt. Ein Flur erstreckte sich bis zum Mobiliar einer Küche und eine hölzerne Treppe mit polierten Metallgeländern erstreckte sich zu einem zweiten Stockwerk hinauf.

„Mann!“ sagte Harry. „Ist das ein großes Haus! Ich hoffe, hier verlaufe ich mich nicht!„

--------------------------------------------------------------------------------------------------------------------------------------------

\hfill\break Dr. Roberta Granger wurde ein wenig nervös als das Abendessen näher rückte. Der Truthahn und der Braten, ihre eigenen Beiträge zu diesem gemeinsamen Projekt, kochten im Ofen stetig vor sich hin; die anderen Speisen sollten von ihren Gästen beigesteuert werden, der Familie Verres, die einen Sohn namens Harry adoptiert hatte. Der der Zauberwelt bekannt war als der Junge-der-überlebt-hat. Und der ebenfalls der einzige Junge war, den Hermine je als „süß“ bezeichnet oder eigentlich überhaupt jemals wahrgenommen hatte.

Die Verres hatten gemeint, Hermine sei ebenfalls das einzige Kind in Harrys Altersgruppe, dessen Existenz ihr Sohn jemals in irgendeiner Weise zur Kenntnis genommen hatte.

Und vielleicht war es noch etwas voreilig; doch beide Paare hegten den leisen Verdacht, dass in ein paar Jahren womöglich die Hochzeitsglocken klingen mochten.

Während sie also den ersten Weihnachtstag wie immer mit der Familie ihres Mannes verbrachten, hatten sie entschieden, am Weihnachtsabend die möglichen künftigen Schwiegereltern ihrer Tochter kennenzulernen.

Die Türklingel schellte, während sie gerade dabei war den Truthahn zu begießen, daher hob sie die Stimme und rief, „\emph{Liebling, kannst du aufmachen?}"

Kurz waren das Knarzen eines Sessels und seines Insassen zu vernehmen, dann erklangen die schweren Schritte ihres Mannes und die Tür schwang auf.

„Dr. Granger?“ sagte forsch die Stimme eines älteren Mannes. „Ich bin Michael und das sind Petunia und unser Sohn Harry. Das Essen ist in dem magischen Koffer."

„Ja, bitte, kommen Sie herein,“ sagte ihr Mann, gefolgt von einem gemurmelten „Vielen Dank,“ mit welchem offenbar irgendein Geschenk entgegengenommen worden war und „Bitte, setzen Sie sich doch.“ Dann nahm Leos Stimme einen Tonfall gespielter Begeisterung an und er sagte, „Und die Spielsachen sind alle unten im Keller; ich bin sicher Hermi kommt gleich runter, es ist die erste Tür rechts von dir."

Es gab eine kurze Pause.

Dann erklang die helle Stimme eines kleinen Jungen, „Spielzeug? Ich liebe Spielzeug!„

Das Geräusch von Schritten war zu vernehmen, die ins Haus eintraten, dann sagte die selbe hohe Stimme, „Mann! Ist das ein großes Haus! Ich hoffe, hier verlaufe ich mich nicht!

Lächelnd schloss Roberta die Ofentür. Nach der Art, wie Hermine den Jungen-der-überlebt-hat in ihren Briefen beschrieben hatte, war sie doch ein wenig besorgt gewesen - obwohl ihre Tochter sicherlich in keiner Weise angedeutet hatte, Harry Potter könne irgendwie gefährlich sein; nichts in der Art der finsteren Andeutungen in jenen Büchern, die Roberta - vermeintlich für Hermine - während ihres Ausflugs in die Winkelgasse erstanden hatte. Allzu viel hatte ihre Tochter eigentlich überhaupt nicht gesagt, nur dass Harry klinge als käme er aus einem Buch und Hermine sich beim Lernen mehr anstrengte als je zuvor in ihrem Leben, nur um ihm im Unterricht voraus zu bleiben. Aber so wie es klang, war Harry Potter ein ganz gewöhnlicher elfjähriger Junge.\\ Sie erreichte die Eingangstür just in dem Moment als ihre Tochter mit einem Tempo die Treppe heruntergerasselt kam, die ihr ganz und gar nicht sicher scheinen mochte; Hermine hatte zwar behauptet, dass Hexen gegen Stürze erheblich unempfindlicher seien, aber Roberta hatte da so ihre Zweifel -

Roberta nahm einen ersten Eindruck von Professor und Mrs. Verres auf, die beide recht nervös wirkten, als der Junge mit der legendären Narbe auf seiner Stirn sich gerade zu ihrer Tochter umwandte und, nun mit tieferer Stimme, sagte, „Wie schön Sie zu treffen, an diesem freudigsten aller Abende, Miss Granger.“ Er streckte die Hand aus, wie um seine Eltern auf einem silbernen Tablett zu darzubieten. „Ich präsentiere Ihnen meinen Vater, Professor Michael Verres-Evans und meine Mutter, Mrs. Petunia Evans-Verres."

Und während Robertas Mund noch offen stand, wandte sich der Junge wieder seinen Eltern zu und sagte, nun wieder mit der gleichen hellen Stimme, „Mum, Dad, das ist Hermine! Sie ist wirklich schlau!"

„\emph{Harry!}“ zischte ihre Tochter. „Lass das!„

Der Junge wirbelte wieder zu Hermine herum. „Ich fürchte wohl, Miss Granger,“ sagte der Junge tragisch, „dass man Sie und mich in die labyrinthenen Abgründe des Kellers verbannt hat. Überlassen wir sie also ihren erwachsenen Gesprächen, welche zweifellos sich weit über unseren kindischen Intellekt erheben würden und setzen wir doch unsere andauernde Diskussion der Implikationen des Humanen Projektivismus für die Kunst der Transfiguration fort."

„Entschuldigt uns, bitte,“ sagte ihre Tochter in äußerst festem Ton, ergriff den Jungen am linken Ärmel und zog ihn mit sich in den Flur - Roberta wirbelte herum in dem hilflosen Versuch ihnen zu folgen als sie an ihr vorbei stürmten; der Junge winkte ihr fröhlich zu - dann zog Hermine den Jungen in den Kellerzugang und knallte die Tür hinter sich zu.

„Ich, äh, ich möchte mich entschuldigen für...“ setzte Mrs. Verres zögerlich an.

„Es tut mir Leid,“ sagte der Professor und lächelte nachsichtig, „Harry kann ein bisschen überempfindlich sein, wenn es um solche Dinge geht. Aber ich nehme an, er liegt durchaus richtig, was unser Interesse an ihrer Unterhaltung betrifft.„

\emph{Ist er gefährlich?} wollte Roberta fragen, doch sie blieb still und sann nach unverfänglicheren Fragen. Ihr Mann an ihrer Seite gluckste, als fände er das, was sie soeben beobachtet hatten, eher amüsant als beängstigend.

Der schrecklichste Dunkle Lord der Geschichte hatte versucht diesen Jungen zu töten und die verbrannten Überreste seines Körpers hatte man neben seiner Krippe gefunden.

Womöglich ihr künftiger Schwiegersohn.

Roberta hegte wachsende Vorbehalte, was die Vorstellung anging, ihre Tochter der Hexenkunst zu überlassen - besonders nachdem sie die Bücher gelesen und die Daten verglichen und dabei erkannt hatte, dass ihre magisch begabte Mutter höchstwahrscheinlich auf der Höhe von Grindelwalds Schreckensherrschaft getötet worden und nicht bei ihrer Geburt gestorben war, wie ihr Vater es immer behauptet hatte. Doch nach ihrem ersten Ausflug hatte ihnen Professor McGonagall noch weitere Besuche abgestattet, „um zu sehen, wie Miss Granger sich macht“ und Roberta wurde das Gefühl nicht los, dass falls Hermine sich dahingehend äußern sollte, ihre Eltern legten ihrer Hexenkarriere Steine in den Weg, man etwas dagegen \emph{unternehmen} würde...

Roberta setzte ihr schönstes Lächeln auf und gab ihr bestes, um ein wenig gespielte Weihnachtsfreude zu verbreiten.

--------------------------------------------------------------------------------------------------------------------------------------------

\hfill\break Der Tisch im Esszimmer erwies sich als erheblich länger als es für sechs Personen - ähm, vier Personen und zwei Kinder - wirklich nötig war, doch war er vollständig mit einer Decke aus feinem, weißen Leinen bedeckt und das Essen war unnötigerweise auf extravagante Servierplatten verbracht worden, welche immerhin aus rostfreiem Edelstahl bestanden und nicht aus echtem Silber.

Harry bereitete es einige Mühe sich auf den Truthahn zu konzentrieren.

Die Unterhaltung hatte sich, natürlich, Hogwarts zugewandt und für Harry war es offenkundig, dass seine Eltern sich erhofften, Hermine würde sich verplappern und mehr über Harrys Schulzeit verraten als er selbst ihnen erzählt hatte. Und entweder hatte Hermine das ebenfalls erkannt oder sie wich einfach nur von selbst allen Themen aus, die sich als problematisch erweisen mochten.

Soweit war \emph{Harry} also auf der sicheren Seite.

Doch unglücklicherweise hatte Harry den Fehler gemacht, seinen Eltern per Eule allerhand über Hermine zu berichten, das sie \emph{ihren} Eltern nicht erzählt hatte.

Wie etwa, dass sie während ihrer außerschulischen Aktivitäten als General einer Armee fungierte.

Hermines Mutter wirkte ob dieser Vorstellung sehr beunruhigt, woraufhin Harry hier schnell unterbrochen und sein bestes gegeben hatte zu erklären, dass alles nur Betäubungszauber waren, Professor Quirrell immer aufpasste und wegen der Existenz magischer Heilmethoden viele Sachen sehr viel weniger gefährlich waren als sie sich anhörten - wobei Hermine ihm unter dem Tisch einen harten Tritt versetzt hatte. Dankenswerterweise hatte Harrys Vater, der wie Harry zugeben musste ihm in einigen Punkten überlegen war, hierzu mit unumstößlicher professoraler Autorität verkündet, dass er deswegen überhaupt nicht besorgt sei, da er sich nicht vorstellen könne, dass es Kindern erlaubt sei daran teilzunehmen wenn es gefährlich wäre.

Das war jedoch nicht der Grund aus dem es Harry Schwierigkeiten bereitete das Essen zu genießen.

...das Problem dabei, sich selbst zu bemitleiden, war dass es nie lange dauerte, bis man auf jemanden traf, der noch schlimmer dran war als man selbst.

An einem Punkt hatte Dr. Leo Granger gefragt, ob die nette Lehrerin, Professor McGonagall, die Hermine so gemocht zu haben schien, ihr in der Schule viele Punkte geben würde.

Hermine hatte geantwortet, ja, mit einem offenbar aufrichtigen Lächeln auf den Lippen.

Harry war es, mit einigem Aufwand, gelungen nicht eisig anzumerken, dass Professor McGonagall niemals irgendeinen Schüler von Hogwarts unfairerweise bevorzugen würde und dass Hermine deshalb viele Punkte bekam, weil sie sich \emph{jeden, einzelnen, verdient} hatte.

Bei einer anderen Gelegenheit hatte Leo Granger am Tisch seine Ansicht zum Besten gegeben, wie ausgesprochen klug Hermine doch sei und dass sie es auf die medizinische Fakultät hätte schaffen und Zahnärztin werden können, wenn diese ganze Hexerei-Geschichte nicht gewesen wäre.

Hermine hatte erneut gelächelt und ein schneller Blick von ihr hatte Harry gerade noch davon abgehalten zu bemerken, dass Hermine ebenso eine \emph{weltberühmte Wissenschaftlerin} hätte werden können und zu fragen, ob den Grangers dieser Gedanke wohl gekommen wäre, hätten sie einen \emph{Sohn} anstelle einer \emph{Tochter} gehabt oder ob es für ihren Nachwuchs in jedem Falle inakzeptabel sei, sich höhere Ziele zu stecken als sie selbst.

Doch mittlerweile näherte Harry sich rapide immer weiter seinem Siedepunkt.

Und wusste den Umstand \emph{deutlich} mehr zu schätzen, dass sein eigener Vater \emph{immer} alles in seiner Macht stehende getan hatte, um Harrys Entwicklung als Hochbegabter zu unterstützen und ihn \emph{immer} ermutigt hatte, nach Höherem zu streben und \emph{niemals} eine seiner Errungenschaften herabgewürdigt hatte, selbst wenn ein Wunderkind letztendlich trotzdem nur ein Kind war. War diess die Art von Haushalt, in der er womöglich gelandet wäre, hätte Mum damals Vernon Dursley geheiratet?

Dennoch gab Harry sein Bestes.

„Und sie schlägt dich wirklich in \emph{all} deinen Fächern außer im Flugunterricht und Verwandlung?“ sagte Professor Michael Verres-Evans.

„Ja,“ sagte Harry bemüht gelassen, während er sich einen weiteren Bissen Weihnachtstruthahn abschnitt. „In den meisten davon mit deutlichem Abstand.“ Unter anderen Umständen hätte Harry das deutlich weniger bereitwillig zugegeben, weshalb er bis jetzt auch nicht dazu gekommen war, es seinem Vater zu erzählen.

„Hermine war immer schon sehr gut in der Schule,“ meinte Dr. Leo Granger in zufriedenem Ton.

„Harry tritt auch bei Wettkämpfen auf Nationalebene an!“ sagte Professor Michael Verres-Evans.

„Liebling!“ sagte Petunia.

Hermine kicherte, wodurch sich Harry ob ihrer Situation jedoch keinen Deut besser fühlte. Es schien Hermine tatsächlich nichts auszumachen und \emph{das machte Harry etwas aus.}

„Es ist mir nicht peinlich gegen sie zu verlieren, Dad,“ sagte Harry. Jetzt, in diesem Moment, war es das tatsächlich nicht. „Habe ich schon erwähnt, dass sie alle ihre Schulbücher vor ihrem ersten Schultag auswendig gelernt hat? Und ja, ich habe es überprüft."

„Ist das, äh, \emph{normal} für sie?“ fragte Professor Verres-Evans die Grangers.

„Oh, ja, Hermine prägt sich immerzu alle möglichen Sachen ein,“ sagte Dr. Roberta Granger mit fröhlichem Lächeln. „Sie kennt jedes Rezept in all meinen Kochbüchern auswendig. Ich vermisse sie schon jedesmal wenn ich zu Abend koche."

Dem Ausdruck auf dem Gesicht seines Vaters nach zu urteilen, teilte Dad zumindest teilweise Harrys Gefühle.

„Keine Sorge, Dad,“ sagte Harry, „mittlerweile bekommt sie so viel fortgeschrittenen Lehrstoff, wie sie nur will. Ihre Lehrer in Hogwarts wissen wie klug sie ist, \emph{anders als ihre Eltern!}"

Zum Ende hin hatte er deutlich die Stimme erhoben und als sich ihm alle Gesichter zuwandten und Hermine ihm erneut einen Tritt versetzte, da wusste Harry, dass er zu weit gegangen war, aber es war einfach zu viel, einfach viel zu viel.

„Natürlich wissen wir wie klug sie ist,“ sagte Leo Granger und wirkte allmählich verärgert ob dieses Kindes, das die Unverfrorenheit besessen hatte, an ihrem Esstisch die Stimme zu erheben.

„Sie haben nicht die leiseste Ahnung,“ sagte Harry und ein eisiger Hauch schlich sich in seine Stimme. „Sie glauben es wäre süß, dass sie so viele Bücher liest, nicht wahr? Sie sehen ein perfektes Zeugnis und es gefällt Ihnen, dass sie sich im Unterricht gut macht. Ihre Tochter ist die talentierteste Hexe ihrer Generation und der hellste Stern von Hogwarts und eines Tages, Dr. und Dr. Granger, wird die Tatsache, dass Sie ihre Eltern sind das einzige sein, weshalb sich die Geschichte Ihrer erinnern wird!"

Hermine, die sich in aller Seelenruhe von ihrem Platz erhoben hatte und um den Tisch herum gekommen war, wählte diesen Moment um Harrys Hemd bei der Schulter zu packen und ihn aus seinem Stuhl zu zerren. Harry ließ es zu, doch während Hermine ihn weg schleifte, sagte er und erhob dabei die Stimme sogar noch lauter, „Es wäre durchaus möglich, dass in eintausend Jahren die Tatsache, dass Hermine Grangers Eltern Zahnärzte waren, der einzige Grund sein wird, dass sich überhaupt noch jemand an die Zahnmedizin erinnert!"

--------------------------------------------------------------------------------------------------------------------------------------------

\hfill\break \emph{Roberta starrte dorthin, wo ihre Tochter gerade noch den Jungen-der-überlebt-hat aus dem Zimmer geschleift hatte, mit einem Ausdruck unglaublicher Geduld auf ihrem jungen Gesicht.}

\emph{„Es tut mir fürchterlich Leid,“ sagte Professor Verres mit einem belustigten Lächeln. „Aber machen Sie sich bitte keine Sorgen, Harry redet immer so. Sind sie nicht schon wie ein altes Ehepaar?"}

\emph{Erschreckenderweise} \emph{\emph{waren}} \emph{sie das.}

--------------------------------------------------------------------------------------------------------------------------------------------

\hfill\break Eigentlich hatte Harry von Hermine jetzt eine ordentliche Standpauke erwartet.

Doch nachdem sie beide den Zugang zum Keller hinter sich gelassen hatten und Hermine die Tür hinter ihnen geschlossen hatte, hatte sie sich umgewandt -

- und lächelte nun und das auch noch aufrichtig, soweit Harry das beurteilen konnte.

„Bitte lass es gut sein, Harry,“ sagte sie mit sanfter Stimme. „Obwohl es wirklich sehr nett von dir ist. Es ist schon in Ordnung.„

Harry blickte sie nur an. „Wie hältst du es nur aus?“ sagte er. Er musste die Stimme senken; sie wollten ja nicht, dass die Eltern sie hörten, doch sie wurde höher, wenn auch nicht lauter. „\emph{Wie hältst du das aus?}"

Hermine zuckte mit den Schultern und sagte, „Weil Eltern nun einmal so sein \emph{sollten?}"

„Nein,“ sagte Harry mit leiser und eindringlicher Stimme, „so nicht, mein Vater würde \emph{nie} auf mich herabsehen - nun ja, doch \emph{schon,} aber niemals auf diese Weise -„

Hermine hob einen Finger und Harry hielt inne während sie nach Worten suchte. Es verging eine Weile, doch dann sagte sie, „Harry... Professor McGonagall und Professor Flitwick mögen mich, weil ich die talentierteste Hexe meiner Generation bin und der hellste Stern von Hogwarts. Und Mum und Dad wissen das nicht und du wirst es ihnen niemals erklären können, aber sie lieben mich trotzdem. Was bedeutet, dass alles genau so ist wie es sein sollte, sowohl in Hogwarts als auch zu Hause. Und da sie \emph{meine} Eltern sind, Mr. Potter, haben \emph{Sie} nichts dagegen einzuwenden.“ Einmal mehr lächelte sie ihr mysteriöses Lächeln vom Abendessen und bedachte Harry mit einem zärtlichen Blick. „\emph{Ist} das klar, Mr. Potter?"

Harry nickte knapp.

„Gut,“ sagte Hermine, beugte sich vor und küsste ihn auf die Wange.

--------------------------------------------------------------------------------------------------------------------------------------------

\hfill\break Die Unterhaltung war gerade erst wieder in Gang gekommen als sie aus der Ferne ein hoher, gellender Aufschrei erreichte,\\ "\emph{Hey!} \emph{KeinKüssen!}"

Die beiden Väter brachen zum gleichen Zeitpunkt in schallendes Gelächter aus als die zwei Mütter von ihren Stühlen hochfuhren und mit dem selben Entsetzen auf dem Gesicht Richtung Keller sprinteten.

Nachdem sie die Kinder zurückgeholt hatten, meinte Hermine in eisigem Ton, sie würde Harry garantiert nie wieder küssen und Harry erwiderte empört, eher würde die Sonne zu einem kalten, toten Häufchen Asche verglühen, bevor er sie noch einmal nah genug an sich heran ließe, es zu versuchen.

Was bedeutete, dass alles ganz genau so war wie es sein sollte und sie alle setzten sich wieder zum Weihnachtsessen.

* Eine US-amerikanische Holdinggesellschaft; Vorsitzender ist Warren Buffet. Gehört laut Wikipedia zu den 20 größten Unternehmen der USA und war durch konstant hohe Ertragskraft und große Finanzreserven über lange Zeit eines der wenigen Unternehmen mit der höchsten Bonitätsstufe.

