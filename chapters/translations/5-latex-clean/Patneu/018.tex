

\hypertarget{dominanz-hierarchien}{% \section{18. Dominanz-Hierarchien}\label{dominanz-hierarchien}}

\textbf{Kapitel 18: Dominanz-Hierarchien\\ }

\hfill\break Jede hinreichend fortgeschrittene J. K. Rowling ist von Magie nicht zu unterscheiden.

--------------------------------------------------------------------------------------------------------------------------------------------

\hfill\break

\emph{"Das klingt wie etwas, das ich tun würde, nicht wahr?"}

\hfill\break

--------------------------------------------------------------------------------------------------------------------------------------------

\hfill\break Es war Frühstückszeit am Freitagmorgen. Harry nahm einen weiteren gewaltigen Bissen von seinem Toast und erinnerte dann sein Gehirn daran, dass sein Frühstück herunter zu schlingen ihn nicht tatsächlich schneller in die Verliese bringen würde. Immerhin hatten sie noch eine volle Stunde Zeit zum Lernen zwischen dem Frühstück und dem Beginn von Zaubertränke.

Aber Verliese! In Hogwarts! Harrys Vorstellung malte sich bereits die Abgründe, schmalen Brücken, kerzenbesetzten Wandleuchter und Flecken von glühendem Moos aus. Würde es Ratten geben? Würde es \emph{Drachen} geben?

\emph{„Harry Potter,“ sagte eine leise Stimme hinter ihm.}

\emph{Harry blickte über seine Schulter und sah sich Ernie Macmillan gegenüber, elegant gekleidet in gelb-getrimmtem Umhang und etwas besorgt dreinschauend.}

\emph{„Neville dachte, ich sollte dich warnen,“ sagte Ernie mit verhaltener Stimme. „Ich denke er hat recht. Sei in unserer Stunde heute auf der Hut vor dem Meister der Zaubertränke. Die älteren Hufflepuffs erzählten, dass Professor Snape gegenüber Leuten, die er nicht mag, wirklich widerlich werden} \emph{kann} \emph{und er mag fast niemanden, der nicht in Slytherin ist. Wenn du irgendwas neunmalkluges sagst... könnte es wirklich schlimm für dich werden, nach dem was ich gehört habe. Halt einfach den Kopf unten und gib ihm keinen Grund auf dich aufmerksam zu werden.„}

\emph{Es entstand eine Pause als Harry das verarbeitete, dann hob er die Augenbrauen. (Harry wünschte, er könnte nur eine Augenbraue hochziehen, wie Spock, aber das hatte er nie hinbekommen.) „Danke,“ sagte Harry. „Du hast mir vielleicht gerade eine Menge Ärger erspart.„}

\emph{Ernie nickte und ging zurück zum Hufflepuff-Tisch.}

\emph{Harry} \emph{aß weiter seinen} \emph{Toast.}

\emph{Es war etwa vier Bissen später, als jemand sagte „Entschuldige bitte“ und Harry sich zu einem älteren Ravenclaw} \emph{umwandte, der ein wenig besorgt aussah -}

\emph{Einige Zeit später vernichtete Harry gerade seine dritte Portion Speckscheiben. (Er hatte gelernt, beim Frühstück ordentlich zuzulangen. Er konnte sich immer noch beim Mittagessen zurückhalten, wenn er seinen Zeitumkehrer nicht benutzte.) Und da ertönte} \emph{schon wieder eine} \emph{Stimme hinter ihm, die sagte „Harry?"}

\emph{„Ja,“ sagte Harry} \emph{matt, „ich werde versuchen, Professor Snapes Aufmerksamkeit nicht auf mich zu ziehen -"}

\emph{„Oh, das ist aussichtslos,“ sagte Fred.}

\emph{„Vollkommen aussichtlos,“ sagte George.}

\emph{„Also haben wir die Hauselfen einen Kuchen für dich backen lassen,“ sagte Fred.}

\emph{„Wir stecken eine Kerze darauf für jeden Punkt} \emph{von} \emph{Ravenclaw, den du verlierst,“ sagte George.}

\emph{„Und schmeißen beim Mittagessen eine Party für dich am Gryffindor-Tisch,“ sagte Fred.}

\emph{„Wir hoffen, das wird dich hinterher aufmuntern,“ beendete George.}

\emph{Harry schluckte den letzten Bissen seiner Speckscheibe und drehte sich um. „Okay,“ sagte Harry. „Ich wollte das nach Professor Binns eigentlich nicht fragen, wirklich nicht, aber wenn Professor Snape} \emph{\emph{so}} \emph{furchtbar ist, warum} \emph{hat man ihn} \emph{nicht gefeuert?"}

\emph{„Gefeuert?“ sagte Fred.}

\emph{Du meinst, entlassen?“ sagte George.}

\emph{„Ja,“ sagte Harry. „Das macht man mit schlechten Lehrern. Man feuert sie. Dann stellt man stattdessen einen besseren Lehrer ein. Ihr habt hier keine Gewerkschaften oder Kündigungsfristen, oder?"}

\emph{Fred und George runzelten auf die selbe Weise die Stirn, wie Stammes-Älteste von Jäger-und-Sammler-Völkern es tun mochten, wenn man versuchte ihnen Infinitesimalrechnung zu erklären.}

\emph{„Ich weiß nicht,“ sagte Fred nach einer Weile. „Ich habe nie darüber nachgedacht."}

\emph{„Ich auch nicht,“ sagte George.}

\emph{„Jaah,“ sagte Harry, „Das höre ich oft. Wir sehen uns beim Mittagessen, Jungs und seht es mir nach, wenn keine Kerzen auf dem Kuchen sind.„}

\emph{Fred und George lachten beide, als habe Harry etwas witziges gesagt, verneigten sich vor ihm und gingen nach Gryffindor zurück.}

\emph{Harry wandte sich wieder dem Frühstückstisch zu und nahm sich einen Cupcake. Sein Magen fühlte sich bereits voll an, aber er hatte das Gefühl, dieser Morgen könnte eine Menge Kalorien verbrauchen.}

\emph{Als er seinen Cupcake aß, dachte Harry an den schlechtesten Lehrer, den er bisher getroffen hatte, Professor Binns aus Geschichte. Professor Binns war ein Geist. Nach dem, was Hermine über Geister erzählt hatte, schien es nicht, als wären sie sich ihrer selbst voll bewusst. Es gab keine berühmten Entdeckungen, die von Geistern gemacht wurden oder überhaupt etwas an origineller Arbeit, egal wer sie im Leben gewesen waren. Geister hatten oft Probleme, sich zu erinnern, welches Jahrhundert gerade war. Hermine sagte, sie waren wie zufällige Porträts,} \emph{eingeprägt} \emph{in die umgebende Materie von einem Ausbruch psychischer Energie beim plötzlichen Tod eines Zauberers.}

\emph{Harry waren während seiner erfolglosen Abstecher in das öffentliche Muggel-Bildungssystem schon einige dumme Lehrer untergekommen -- sein Vater war bei der Auswahl von Schulabsolventen als Tutoren natürlich sehr viel wählerischer gewesen -- doch beim Geschichtsunterricht war er zum ersten mal einem Lehrer begegnet, der buchstäblich nicht empfindungsfähig war.}

\emph{Und das zeigte sich auch. Harry hatte nach fünf Minuten aufgegeben und angefangen, ein Lehrbuch zu lesen. Als klar wurde, dass „Professor Binns“ sich nicht beschweren würde, hatte Harry noch in seinen Beutel gegriffen und Ohrenstöpsel herausgeholt.}

\emph{Benötigten Geister kein Gehalt? War das der Grund? Oder war es in Hogwarts buchstäblich unmöglich jemanden zu feuern,} \emph{\emph{selbst wenn er starb?}}

\emph{Nun schien es, dass Professor Snape sich einfach jedem gegenüber, der kein Slytherin war, absolut scheußlich benahm und es war nicht einmal jemandem} \emph{\emph{in den Sinn gekommen,}} \emph{seinen Vertrag zu beenden.}

\emph{Und der Schulleiter hatte ein Hühnchen in Brand gesteckt.}

\emph{„Entschuldige bitte,“ erklang eine besorgte Stimme hinter ihm.}

\emph{„Ich schwöre,“ sagte Harry ohne sich umzudrehen,“ hier ist es fast achteinhalb Prozent so schlimm, wie} \emph{das, was Dad über Oxford erzählt."}

--------------------------------------------------------------------------------------------------------------------------------------------

\hfill\break Harry stampfte die steinernen Korridore entlang und sah gleichzeitig beleidigt, genervt und wütend aus.

„Verliese!“ zischte Harry. „\emph{Verliese!} Das sind keine Verliese! Das ist ein Keller! Ein \emph{Keller!}„

Einige der Ravenclaw-Mädchen warfen ihm seltsame Blicke zu. Die Jungs waren mittlerweile alle an ihn gewöhnt.

Es schien, dass das Stockwerk, in welchem sich der Zaubertränke-Klassenraum befand, aus keinem besseren Grund als „Verliese“ bezeichnet wurde, als dass es unter der Erde lag und ein wenig kälter als der Hauptteil des Schlosses war.

In \emph{Hogwarts!} In \emph{Hogwarts!} Harry hatte sein ganzes Leben lang gewartet und jetzt wartete er \emph{immer noch} und wenn es \emph{einen Ort auf der Welt} gab, der anständige Verliese hatte, sollte es Hogwarts sein! Musste Harry sein eigenes Schloss bauen, wenn er auch nur einen kleinen bodenlosen Abgrund sehen wollte?

Kurze Zeit später kamen sie zum eigentlichen Zaubertränke-Klassenraum und Harrys Laune verbesserte sich erheblich.

Im Zaubertränke-Klassenraum befanden sich seltsame konservierte Kreaturen, die in großen Gläsern entlang jedem Zentimeter Wand zwischen den Schränken schwammen. Harry war mit dem Lesen mittlerweile weit genug gekommen, dass er tatsächlich einige der Kreaturen identifizieren konnte, wie die Zabriskan Fontema. Obwohl die fünfzig Zentimeter durchmessende Spinne wie eine Acromantula \emph{aussah}, war sie doch zu klein, um eine zu \emph{sein.} Er hatte versucht, Hermine zu fragen, doch sie schien nicht sehr interessiert, auch nur annähernd in die Richtung zu schauen, die er andeutete.

Harry betrachtete eine große Staubkugel mit Augen und Füßen, als der Attentäter in den Raum huschte.

Das war der erste Gedanke, der Harry in den Sinn kam, als er Professor Severus Snape erblickte. Es lag etwas stilles und tödliches in der Art und Weise, wie der Mann zwischen den Pulten der Kinder entlangpirschte. Sein Umhang war ungepflegt, sein Haar schmutzig und fettig. Etwas an ihm erinnerte an Lucius, obwohl die beiden sich nicht einmal annähernd ähnlich sahen und man bekam den Eindruck, dass während Lucius einen mit makelloser Eleganz töten würde, dieser Mann einen einfach umbrächte.

„Setzt euch,“ sagte Professor Severus Snape. „Sofort."

Harry und ein paar andere Kinder, die herumgestanden und miteinander gesprochen hatten, rangen um die Pulte. Harry hatte geplant neben Hermine zu landen, aber irgendwie verschlug es ihn am Ende an das nächste freie Pult neben Justin Finch-Fletchley (es war eine Doppel-Klasse, Ravenclaw und Hufflepuff), was ihn zwei Pulte zur Linken von Hermine platzierte.

Severus setzte sich hinter den Lehrer-Schreibtisch und sagte ohne die geringste Überleitung oder Vorstellung, „Hannah Abbott."

„Hier,“ sagte Hannah, mit etwas zittriger Stimme.

"Susan Bones."

"Anwesend."

Und so ging es weiter, sonst sagte niemand ein Wort, bis:

"Ah, ja. Harry Potter. Unsere neue... \emph{Berühmtheit.}"

"Die Berühmtheit ist anwesend, \emph{Sir.}"

Die Hälfte der Klasse zuckte zusammen und einige der schlaueren Schüler sahen plötzlich aus, als wollten sie aus der Tür stürmen, solange das Klassenzimmer noch existierte.

Severus lächelte auf erwartungsvolle Weise und rief den nächsten Namen auf der Liste auf.

Harry seufzte innerlich. Das war viel zu schnell passiert, als dass er etwas dagegen hätte tun können. Nun gut. Offenbar konnte dieser Mann ihn schon jetzt nicht ausstehen, aus welchem Grund auch immer. Und wenn Harry darüber nachdachte, war es viel besser, wenn dieser Professor für Zaubertränke auf \emph{ihm} herumhackte, als auf, etwa, Neville oder Hermine. Harry war sehr viel eher in der Lage sich zu verteidigen. Jup, war wahrscheinlich am besten so.

Als die komplette Anwesenheitsliste abgehakt war, ließ Severus seinen Blick über die versammelte Klasse schweifen. Seine Augen waren so leer, wie ein Nachthimmel ohne Sterne.

„Sie sind hier,“ sagte Severus mit leiser Stimme, die die Schüler ganz hinten nur mit Mühe verstanden, „um die subtile Wissenschaft und exakte Kunst der Zaubertrank-Brauerei zu erlernen. Da es hier wenig albernes Zauberstab-Gefuchtel geben wird, werden viele von Ihnen kaum glauben können, dass es sich hierbei um Magie handelt. Ich ewarte von Ihnen kein wirkliches Verständnis für die Schönheit eines brodelnden Kessels mit seinen flirrenden Dämpfen, die grazile Macht der Flüssigkeiten, die sich ihren Weg durch menschliche Venen bahnen,“ dies in eher zärtlichem, hämischem Ton, „den Geist verhexen und die Sinne auf eine Reise schicken,“ das wurde immer gruseliger und gruseliger. „Ich kann Sie lehren, wie man Glanz und Ansehen brodelnd zusammenbraut, wie man Ruhm aus Flaschen zieht und sogar wie man den Tod verkorkt -- wenn Sie nicht ein ebenso großer Haufen Narren sind, wie ich sie üblicherweise unterrichten muss."

Severus schien irgendwie den skeptischen Ausdruck auf Harrys Gesicht bemerkt zu haben, zumindest sprangen seine Augen plötzlich dorthin, wo Harry saß.

„Potter!“ schnappte der Professor für Zaubertränke. „Was bekäme ich, wenn ich einem Wermutaufguss geriebene Affodill-Wurzel beifügte?„

Harry blinzelte. „Kam das in \emph{Zaubertränke und Zauberbräue}* vor?“ sagte er. „Ich habe es gerade fertig gelesen und ich erinnere mich an nichts, für das man Wermut brauchte -"

Hermines Hand schoss in die Höhe und Harry funkelte sie an, was sie zum Anlass nahm, ihre Hand noch höher zu recken.

„Ts, ts,“ sagte Severus mit seidiger Stimme. „Ruhm ist offenbar nicht alles."

„Wirklich?“ sagte Harry. „Doch Sie haben uns gerade gesagt, Sie würden uns beibringen, wie man Ruhm aus Flaschen zieht. Sagen Sie, wie \emph{funktioniert} das eigentlich? Man trinkt ihn und wird zu einem Prominenten?"

Dreiviertel der Klasse zuckten zusammen.

Hermines Hand fiel langsam wieder herab. Nun, das war nicht überraschend. Sie mochte seine Rivalin sein, doch sie war nicht die Art Mädchen, die mitspielen würde, nachdem klar war, dass der Professor absichtlich versuchte, ihn zu demütigen.

Harry bemühte sich sehr, sein Temperament unter Kontrolle zu halten. Die erste Erwiderung, die ihm in den Sinn kam, war 'Abrakadabra'.

„Versuchen wir es nochmal,“ sagte Severus. „Potter, wo würden Sie suchen, wenn es hieße, sie sollten mir einen Bezoar beschaffen?"

„Das steht ebenfalls nicht im Lehrbuch,“ sagte Harry, „aber in einem Muggel-Buch habe ich gelesen, dass ein Trichobezoar** eine Masse verfestigter Haare ist, die man in einem menschlichen Magen findet und Muggel glaubten, er würde jede Vergiftung heilen -"

„Falsch,“ sagte Severus. „Einen Bezoar findet man im Magen einer Ziege, er besteht nicht aus Haaren und heilt die meisten Vergiftungen, aber nicht alle."

"Ich habe nicht \emph{gesagt,} er würde es, ich sagte, das sei, was ich in einem Muggel-Buch gelesen habe -"

"Niemanden interessieren hier ihre \emph{erbärmlichen} Muggel-Bücher. Letzter Versuch. Was ist der Unterschied, Potter, zwischen Eisenhut und Wolfswurz?"

Das war's.

„Wissen Sie,“ sagte Harry eisig, „in einem meiner wirklich \emph{faszinierenden} Muggel-Bücher wird eine Studie beschrieben, in der Leute sich selbst sehr schlau erscheinen ließen, in dem sie Fragen nach beliebigen Fakten stellten, die nur sie kannten. Offenbar bemerkten die Zuschauer nur, dass die Fragesteller es wussten und die Antwortenden nicht und berücksichtigten nicht die Unfairness des zugrundeliegenden Spiels. Also, Professor, können Sie mir sagen, wieviele Elektronen sich auf der äußersten Hülle eines Kohlenstoff-Atoms befinden?„

Severus Lächeln wurde breiter. „Vier,“ sagte er. „Doch es ist ein nutzloser Fakt, den niemand sich die Mühe machen sollte zu notieren. Und zu ihrer Information, Potter, Affodill und Wermut ergeben einen Schlaftrank, der so stark ist, dass er auch als Trank der Lebenden Toten bekannt ist. Was Eisenhut und Wolfswurz angeht, so bezeichnen sie die selbe Pflanze, auch bekannt als Akonit, was Sie wissen würden, wenn Sie \emph{Tausend Zauberkräuter und -pilze}*** gelesen hätten. Dachten wohl, Sie bräuchten das Buch nicht zu öffnen, bevor Sie herkommen, was, Potter? Der ganze Rest von Ihnen sollte sich das aufschreiben, damit Sie nicht so unwissend bleiben, wie er.“ Severus hielt inne, sah sehr selbstzufrieden aus. „Und das wären... fünf Punkte? Nein, machen wir lieber zehn Punkte von Ravenclaw daraus, für Widerworte."

Hermine keuchte, zusammen mit einigen anderen.

„Professor Severus Snape,“ sagte Harry gepresst. „Ich weiß von nichts, was ich getan hätte, mir Ihre Feindschaft zu verdienen. Wenn Sie ein Problem mit mir haben, von dem ich nichts weiß, schlage ich vor wir -"

"Ruhe, Potter. Zehn weitere Punkte von Ravenclaw. Der Rest von Ihnen öffnet die Bücher auf Seite 3."

Das leichte brennende Gefühl ganz hinten in Harrys Kehle währte nur kurz und es lag fast keine Feuchtigkeit in seinen Augen. Wenn zu weinen keine effektive Strategie war, um diesen Zaubertränke-Professor zu vernichten, gab es keinen Grund zu weinen.

Langsam, setzte Harry sich kerzengerade auf. Es war, als sei all sein Blut abgesaugt und durch flüssigen Stickstoff ersetzt worden. Ihm war klar, dass er seine Beherrschung hatte wahren wollen, doch er konnte sich nicht mehr erinnern warum.

„Harry,“ flüsterte Hermine hektisch über zwei Tische hinweg, „lass es, bitte, ist schon gut, wir zählen das nicht -"

"Sprechen im Unterricht, Granger? Drei -"

„Also,“ sagte eine Stimme kälter als null Kelvin, „wie reicht man eine formelle Beschwerde gegen Missbrauch durch einen Lehrer ein? Spricht man mit der Stellvertretenden Schulleiterin, schreibt man einen Brief an den Schulbeirat... würden Sie bitte erklären, wie es funktioniert?"

Die Klasse war völlig erstarrt.

„Einen Monat Nachsitzen, Potter,“ sagte Severus, noch breiter lächelnd.

"Ich weise Ihre Autorität als Lehrer zurück und werde an keinem von Ihnen verhängten Nachsitzen teilnehmen.„

Alle hielten den Atem an.

Severus Lächeln verschwand. „Dann werden Sie -“ er stockte kurz.

„Ausgeschlossen, wollten Sie sagen?“ Harry, seinerseits, lächelte jetzt dünn. „Doch dann schienen Sie Ihre Fähigkeit, die Drohung wahrzumachen, anzuzweifeln oder fürchteten die Konsequenzen, wenn Sie es täten. Ich, andererseits, hege keine Zweifel oder Befürchtungen hinsichtlich der Aussicht eine andere Schule mit weniger missbräuchlichen Professoren zu finden. Oder vielleicht sollte ich private Tutoren anstellen, wie es meine gewohnte Praxis ist, die meinem Lerntempo voll entsprechen können. Ich habe genug Geld in meinem Verlies. Irgendwas von wegen Kopfgeldern für einen Dunklen Lord, den ich besiegt habe. Aber es \emph{gibt} Lehrer in Hogwarts, die ich mag, daher denke ich, dass es einfacher wäre einen Weg zu finden, nur Sie loszuwerden."

„Mich loszuwerden?“ sagte Severus, lächelte nun ebenfalls dünn. „Welch amüsante Selbstüberschätzung. Wie, nehmen Sie an, werden Sie das anstellen, Potter?"

„So weit ich weiß, gibt es eine Reihe von Beschwerden über Sie, von Schülern und ihren Eltern,“ geraten, aber ziemlich sicher, „was nur die Frage offen lässt, warum Sie noch nicht weg sind. Ist Hogwarts finanziell zu eingeschränkt, um sich einen echten Professor für Zaubertränke zu leisten? Wenn ja, könnte ich was zuschießen. Ich bin sicher, man könnte einen höherklassigen Lehrer finden, wenn man das Doppelte Ihres Gehalts anböte."

Zwei Eisblöcke verströmten eisigen Winter im Klassenzimmer.

„Sie werden feststellen,“ sagte Severus leise, „dass der Schulbeirat ihrem Angebot in keiner Weise zugeneigt ist."

„Lucius...“ sagte Harry. „\emph{Deshalb} sind Sie noch hier. Vielleicht sollte ich mit Lucius darüber plaudern. Ich glaube, er würde sich gern mit mir treffen. Ich frage mich, ob ich irgendetwas habe, das er will?"

Hermine schüttelte hektisch den Kopf. Harry bekam es aus dem Augenwinkel mit, doch seine Aufmerksamkeit galt nur Severus.

„Du bist sehr töricht, Junge,“ sagte Severus. Er lächelte nun überhaupt nicht mehr. „Du hast nichts, was Lucius mehr bedeutet als meine Freundschaft. Und selbst wenn, hätte ich andere Verbündete.“ Seine Stimme wurde hart. „Und ich finde es immer unwahrscheinlicher, dass du nicht nach Slytherin sortiert wurdest. Wie kam es, dass du es schafftest, meinem Haus fern zu bleiben? Ah, ja, weil der Sprechende Hut vorgab, er habe \emph{gescherzt.} Zum ersten mal seit Anbeginn der Geschichtsschreibung. Worüber haben Sie wirklich mit dem Sprechenden Hut \emph{geplaudert,} Potter? Hatten Sie etwas, das er wollte?"

Harry starrte Severus kaltem Blick entgegen und erinnerte sich, dass der Sprechende Hut ihn gewarnt hatte, niemandem in die Augen zu sehen, wenn er nachdachte über - Harry schlug den Blick nieder zu Severus Schreibtisch.

"Sie scheinen seltsam abgeneigt, mir in die Augen zu sehen, Potter!"

Ein Schock plötzlichen Verstehens - „Also waren \emph{Sie} es, vor dem der Sprechende Hut mich gewarnt hat!"

„Was?“ sagte Severus Stimme und klang ehrlich überrascht, wobei Harry natürlich sein Gesicht nicht ansah.

Harry stand von seinem Pult auf.

„Setzen, Potter,“ sagte eine zornige Stimme, von irgendwo, wo er nicht hinsah.

Harry ignorierte sie und sah sich im Klassenraum um. „Ich habe nicht die Absicht, meine Zeit in Hogwarts von einem unprofessionellen Lehrer ruinieren zu lassen,“ sagte Harry mit tödlicher Ruhe. „Ich denke, ich werde diesen Unterricht verlassen und entweder einen Tutor einstellen, der mir Zaubertränke beibringt, während ich hier bin oder wenn der Beirat wirklich so unzugänglich ist, den Sommer über lernen. Wenn jemand von euch entscheiden sollte, sich nicht von diesem Mann herumschubsen zu lassen, sind meine Stunden für euch offen."

"\emph{Setzen, Potter!}"

Harry durchschritt den Raum und ergriff den Türknauf.

Er drehte sich nicht.

Harry drehte sich langsam um und erhaschte einen Blick auf Severus gemeines Lächeln, bevor er daran dachte wegzusehen.

"Öffnen Sie diese Tür."

„Nein,“ sagte Severus.

„Sie geben mir das Gefühl, bedroht zu werden,“ sagte eine Stimme, so eisig, sie klang nicht einmal wie Harrys, „und das ist ein Fehler."

Severus Stimme lachte. „Was willst du dagegen tun, kleiner Junge?"

Harry machte sechs lange Schritte weg von der Tür, bis er nahe der hintersten Reihe von Pulten stand.

Dann stellte Harry sich aufrecht hin und erhob die rechte Hand mit einer schrecklichen Bewegung, die Finger zu einem Schipsen erhoben.

Neville schrie auf und tauchte unter sein Pult ab. Andere Kinder schrumpften zusammen oder hoben instinktiv die Arme, um sich zu schützen.

„\emph{Harry, tu's nicht!}“ kreischte Hermine. „Was immer du mit ihm machen wolltest, tu's nicht!"

„Seid ihr alle \emph{verrückt} geworden?“ bellte Severus Stimme.

Langsam senkte Harry die Hand. „Ich wollte ihm nicht wehtun, Hermine,“ sagte Harry, seine Stimme etwas leiser. „Ich wollte nur die Tür aufsprengen."

Doch jetzt da Harry darüber nachdachte, sollte man ja keine Dinge transfigurieren, die verbrannt werden sollen, was bedeutete, dass später in der Zeit zurückzugehen und Fred und George dazu zu bringen eine sorgfältig bemessene Menge Sprengstoff zu transfigurieren, keine wirklich gute Idee gewesen wäre...

„\emph{Silencio,}“ sagte Severus Stimme.

Harry versuchte zu sagen „Was?“ und stellte fest, dass er keinen Laut herausbrachte.

"Das ist lächerlich geworden. Ich denke Sie konnten sich für einen Tag bereits genug in Schwierigkeiten bringen, Potter. Sie sind der störendste, aufsässigste Schüler, den ich je gesehen habe und ich weiß nicht, wieviele Punkte Ravenclaw gerade hat, aber ich bin sicher, ich kann sie alle auslöschen. Zehn Punkte von Ravenclaw! Zehn Punkte von Ravenclaw! Fünfzig Punkte von Ravenclaw! Jetzt setzen Sie sich hin und sehen Sie zu, wie der Rest der Klasse den Unterricht beendet!"

Harry steckte die Hand in seinen Beutel und versuchte 'Filzstift' zu sagen, doch natürlich kamen keine Worte heraus. Für einen kurzen Moment stoppte ihn das; doch dann kam Harry auf den Gedanken F-I-L-Z-S-T-I-F-T mit Fingerbewegungen zu formulieren, was funktionierte. B-L-O-C-K und er hatte einen Block Papier. Harry schritt hinüber zu einem leeren Pult, nicht das an dem er ursprünglich gesessen hatte und kritzelte eine kurze Nachricht. Er riss das Blatt Papier ab, steckte den Filzstift und den Block für schnelleren Zugriff in eine Tasche seines Umhangs und hielt seine Nachricht empor, nicht für Snape, aber zum Rest der Klasse.

ICH VERSCHWINDE\\ MUSS SONST NOCH\\ JEMAND NACH DRAUßEN?

„Sie sind verrückt, Potter,“ sagte Snape mit kalter Verachtung.

Davon abgesehen sprach niemand.

Harry vollführte ironisch eine schwungvolle Verbeugung vor dem Lehrer-Schreibtisch, ging hinüber zur Wand und mit einer flüssigen Bewegung riss er eine Schranktür auf, schritt hinein und schlug die Tür hinter sich zu.

Man hörte ein unterdrücktes Geräusch wie von einem Fingerschnippen und dann nichts mehr.

Im Klassenzimmer sahen sich die Schüler verwirrt und verängstigt an.

Das Gesicht des Meisters der Zaubertränke war jetzt vollkommen wutentbrannt. Er durchquerte mit schrecklichen großen Schritten den Raum und riss die Schranktür auf.

Der Schrank war leer.

--------------------------------------------------------------------------------------------------------------------------------------------

\hfill\break Eine Stunde früher horchte Harry aus dem Inneren des geschlossenen Schranks. Von draußen kam kein Laut, aber wozu ein Risiko eingehen.\\ U-M-H-A-N-G, buchstabierten seine Finger.

Sobald er unsichtbar war, öffnete er die Schranktür sehr vorsichtig und langsam einen Spalt weit und spähte hinaus. Niemand schien sich im Klassenraum zu befinden.

Die Tür war nicht verschlossen.

Als Harry aus dem gefährlichen Ort heraus und im Gang war, sicher unsichtbar, ließ sein Ärger etwas nach und ihm wurde klar, was er gerade getan hatte.

Was er gerade getan hatte.

Harrys unsichtbares Gesicht war in absolutem Entsetzen erstarrt.

Er hatte sich einen Lehrer um drei Größenordnungen stärker zum Feind gemacht als jemals zuvor. Er hatte damit gedroht, Hogwarts zu verlassen und musste das möglicherweise durchziehen. Er hatte alle Punkte von Ravenclaw verloren und dann den Zeitumkehrer benutzt...

Seine Vorstellung zeigte ihm seine Eltern, die ihn anschrien, nachdem er der Schule verwiesen war, Professor McGonagall, die von ihm enttäuscht war und er konnte es nicht ertragen und ihm \emph{fiel kein} \emph{Ausweg ein} \emph{-}

Harry erlaubte sich zu denken, dass wenn zornig zu werden ihm diesen ganzen Ärger eingebrockt hatte, dann würde ihm, wenn er zornig war, vielleicht ein Ausweg einfallen; die Dinge schienen irgendwie klarer, wenn er zornig war.

Und der Gedanke, den Harry sich nicht zu denken erlaubte, war, dass er seiner Zukunft einfach nicht in's Auge sehen konnte, wenn er nicht zornig war.

Also verdrängte er seine Gedanken und rief sich die brennende Demütigung in Erinnerung -

\emph{Ts, ts. Ruhm ist offenbar nicht alles.}

\emph{Zehn Punkte von Ravenclaw für Widerworte.}

Die beruhigende Kälte spülte erneut durch seine Venen, wie eine Welle, die von einem Brecher reflektiert und zurückgeworfen wurde und Harry stieß den Atem aus.

Okay. Jetzt werden wir wieder vernünftig.

Tatsächlich war er etwas enttäuscht von seinem nicht-zornigen Selbst, dass es einfach so zusammengebrochen war und sich nur aus Schwierigkeiten heraushalten wollte. Professor Severus Snape war \emph{jedermanns} Problem. Normal-Harry hatte das vergessen und nur sich \emph{selbst} schützen wollen. Und all die anderen Opfer im Stich lassen? Die Frage war nicht, wie er sich schützen konnte, sondern wie er diesen Professor für Zaubertränke aus dem Weg räumen konnte.

\emph{Das also ist meine dunkle Seite, ja? Ein etwas vorurteilsbehafteter Begriff das, meine helle Seite scheint egoistischer und feiger zu sein, von verwirrt und panisch ganz zu schweigen.}

Und nun, da er wieder klar dachte, war ebenso klar, was als nächstes zu tun war. Er hatte sich bereits eine extra Stunde zur Vorbereitung verschafft und konnte noch bis zu fünf weitere Stunden bekommen, wenn nötig...

--------------------------------------------------------------------------------------------------------------------------------------------

\hfill\break Minerva McGonagall wartete im Büro des Schulleiters.

Dumbledore saß in seinem gepolsterten Thron hinter seinem Schreibtisch, gekleidet in vier Schichten formeller lavendelfarbener Umhänge. Minerva saß in einem Stuhl vor ihm, auf der anderen Seite Severus in einem weiteren Stuhl. Den dreien gegenüber stand ein leerer hölzerner Hocker.

Sie warteten auf Harry Potter.

\emph{Harry,} dachte Minerva verzweifelt, \emph{du hast versprochen, du würdest keine Lehrer beißen!}

Und vor ihrem geistigen Auge konnte sie deutlich die Erwiderung sehen, Harrys zorniges Gesicht und seine empörte Antwort: \emph{Ich sagte, ich würde niemanden beißen, der mich nicht zuerst beißt!}

Ein Klopfen kam von der Tür.

„Herein!“ rief Dumbledore.

Die Tür schwang auf und Harry Potter trat ein. Minerva keuchte beinahe laut auf. Der Junge schien kühl, gesammelt und voller Selbstkontrolle.

„Guten Mor-“ Harrys Stimme brach plötzlich ab. Seine Kinnlade fiel herunter.

Minerva folgte Harrys Blick und sie sah, dass Harry Fawkes anstarrte, der auf seinem goldenen Podest saß. Fawkes flatterte mit seinen rot-goldenen Flügeln, wie das Flackern einer Flamme, neigte den Kopf und nickte dem Jungen gemessen zu.

Harry wand sich um und starrte Dumbledore an.

Dumbledore zwinkerte ihm zu.

Minerva hatte das Gefühl, ihr entging etwas.

Plötzliche Unsicherheit streifte über Harrys Gesicht. Seine kühle Haltung geriet ins Wanken. Angst zeigte sich in seinen Augen, dann Zorn und der Junge war wieder gefasst.

Ein Schauer lief Minerva über den Rücken. Etwas stimmte hier nicht.

„Bitte setz dich,“ sagte Dumbledore. Sein Gesicht war jetzt wieder ernst.

Harry saß.

„Also, Harry,“ sagte Dumbledore. „Ich habe einen Bericht über den heutigen Tag von Professor Snape gehört. Würdest du mir in deinen eigenen Worten berichten, was passiert ist?„

Harrys Blick sprang abschätzig zu Severus. „Es ist nicht kompliziert,“ sagte der Junge und lächelte dünn. „Er versuchte mich genau so zu mobben, wie er es mit jedem Nicht-Slytherin in der Schule tut, seitdem Lucius ihn bei Ihnen abgeladen hat. Was die weiteren Einzelheiten anbelangt, so wünsche ich eine private Unterredung mit Ihnen darüber. Von einem Schüler, der missbräuchliches Verhalten seitens eines Professors berichtet, kann immerhin schwerlich erwartet werden, vor eben diesem Professor offen zu sprechen.„

Dieses mal konnte Minerva ein lautes Aufkeuchen nicht vermeiden.

Severus lachte einfach.

Und das Gesicht des Schulleiters wurde ernst. „Mr. Potter,“ sagte der Schulleiter, „man spricht nicht in solcher Weise über einen Professor von Hogwarts. Ich fürchte, dass Sie einer fatalen Fehleinschätzung unterliegen. Professor Severus Snape genießt mein vollstes Vertrauen und dient Hogwarts auf mein eigenes Geheiß hin, nicht das von Lucius Malfoy."

Einige Momente lang war es still.

Als der Junge wieder sprach, war seine Stimme eisig. „Entgeht mir hier etwas?"

„Sogar einige Dinge, Mr. Potter,“ sagte der Schulleiter. „Sie sollten, zunächst einmal, verstehen, dass der Zweck dieses Treffens ist, zu erörtern, wie Sie für die Ereignisse dieses Morgens zu disziplinieren sind."

"Dieser Mann hat Ihre Schule jahrelang terrorisiert. Ich habe mit Schülern gesprochen und genug Geschichten zusammengetragen für eine Zeitungskampagne, um die Eltern gegen ihn zusammen zu trommeln. Einige der jüngeren Schüler weinten, als sie es mir erzählten. Mir kamen fast die Tränen, als ich sie gehört habe! \emph{Sie haben seinem Missbrauch tatenlos zugesehen? Sie haben Ihren Schülern das angetan? Warum?}"

Minerva schluckte einen Kloß in ihrem Hals. Sie hatte - sich das auch gefragt, manchmal, aber irgendwie hatte sie nie wirklich -\\ „Mr. Potter,“ sagte der Schulleiter, seine Stimme nun streng, „bei diesem Treffen geht es nicht um Professor Snape. Es geht um Sie und Ihre Missachtung der Schuldisziplin. Professor Snape hat vorgeschlagen und ich habe zugestimmt, dass drei volle Monate Nachsitzen angemessen sein werden -"

„Abgelehnt,“ sagte Harry eisig.

Minerva war sprachlos.

„Dies ist keine Bitte, Mr. Potter,“ sagte der Schulleiter. Die gesamte, volle Gewalt des Blickes des Zauberers war auf den Jungen gerichtet. „Dies ist Ihre Bestrafu-"

"Sie werden mir erklären, warum Sie diesem Mann erlaubt haben, den Kindern die in Ihre Obhut gegeben wurden Leid zuzufügen und wenn Ihre Erklärung nicht zufriedenstellend ist, werde ich meine Zeitungskampagne beginnen, mit \emph{Ihnen} als Ziel."

Minervas Körper schwankte angesichts der Gewalt dieses Ausbruchs, der schieren, groben \emph{Majestätsbeleidigung.}

Selbst Severus wirkte schockiert.

„Das, Harry, wäre in höchstem Maße unklug,“ sagte Dumbledore langsam. „Ich bin die wichtigste Figur, die Lucius auf dem Spielbrett entgegen steht. Solltest du etwas dergleichen tun, würde ihn das außerordentlich stärken und es war nicht mein Eindruck, dies sei die Seite, die du gewählt hast.„

Der Junge blieb einen langen Moment lang still.

Diese Unterhaltung wird langsam privat,“ sagte Harry. Seine Hand zuckte in Severus Richtung. „Schicken Sie ihn weg."

Dumbledore schüttelte den Kopf. „Harry, habe ich dir nicht gesagt, dass Severus Snape mein vollstes Vertrauen genießt?"

Auf dem Gesicht des Jungen zeigte sich der Schock darüber. „Das Mobbing dieses Mannes macht Sie verwundbar! Ich bin nicht der einzige, der eine Zeitungskampagne gegen Sie starten könnte! Das ist wahnsinnig! Wieso tun Sie das?"

Dumbledore seufzte. „Es tut mir leid, Harry. Es hat mit Dingen zu tun, die du, zu diesem Zeitpunkt, noch nicht bereit bist zu hören."

Der Junge starrte Dumbledore an. Dann drehte er sich um und blickte zu Severus. Dann wieder zurück zu Dumbledore.

„Es \emph{ist} Wahnsinn,“ sagte der Junge langsam. „Sie haben ihn nicht zurückgehalten, weil Sie glauben er ist \emph{Teil des Musters.} Dass Hogwarts einen bösen Meister der Zaubertränke braucht um eine richtige Zauberschule zu sein, so wie einen Geist, der Geschichte unterrichtet."

„Das klingt wie etwas, das ich tun würde, nicht wahr?“ sagte Dumbledore lächelnd.

„Inakzeptabel,“ sagte Harry entschieden. Sein Blick war nun kalt und finster. „Ich werde Mobbing oder Missbrauch nicht dulden. Ich hatte viele Möglichkeiten in Betracht gezogen, mit diesem Problem umzugehen, aber ich werde es einfach machen. Entweder geht dieser Mann oder ich tue es."

Minerva keuchte erneut. Etwas seltsames flackerte in Severus Augen.

Nun wurde Dumbledores Blick ebenfalls kalt. „Ein Schulverweis, Mr. Potter, ist die letzte Drohung, die gegen einen Schüler gerichtet werden kann. Er wird gewöhnlich nicht von Schülern als Drohung gegen den Schulleiter verwendet. Dies ist die beste Zauberschule auf der ganzen Welt und eine Ausbildung hier ist eine Gelegenheit, die nicht jedem zuteil wird. Haben Sie den Eindruck, dass Hogwarts nicht ohne Sie zurecht kommt?"

Und Harry saß dort und lächelte dünn.

Plötzliches Entsetzen befiel Minerva. Sicherlich würde Harry nicht -

„Sie vergessen,“ sagte Harry, „dass Sie nicht der einzige sind, der Muster erkennen kann. \emph{Das hier wird privat. Jetzt schicken Sie ihn -}“ Harry deutete wieder mit der Hand auf Severus, dann stoppte er mitten im Satz und der Geste.

Minerva konnte es auf Harrys Gesicht erkennen, der Moment als es ihm einfiel.

Sie hatte es ihm immerhin erzählt.

„Mr. Potter,“ sagte der Schulleiter, „noch einmal, Severus Snape genießt mein vollstes Vertrauen."

„Sie haben es ihm gesagt,“ flüsterte der Junge. „Sie törichter Narr."

Dumbledore reagierte nicht auf die Beleidigung. „Ihm was gesagt?"

"Dass der Dunkle Lord lebt."

„\emph{Was in Merlins Namen reden Sie da, Potter?}“ schrie Severus völlig überrascht und empört.

Harry blickte kurz zu ihm herüber, grimmig lächelnd. „Oh, dann sind wir also \emph{doch} ein Slytherin,“ sagte Harry. „Ich habe mich schon gewundert."

Und dann war Stille.

Schließlich sprach Dumbledore. Seine Stimme war sanft. „Harry, wovon \emph{sprichst} du?"

„Es tut mir leid, Albus,“ flüsterte Minerva.

Severus und Dumbledore drehten sich zu ihr um.

„Professor McGonagall hat es mir nicht verraten,“ sagte Harrys Stimme, schnell und nicht so ruhig wie vorher. „Ich habe geraten. Ich sagte Ihnen, auch ich kann die Muster sehen. Ich habe geraten und sie reagierte kontrolliert, wie Severus auch. Doch ihre Kontrolle war nicht ganz perfekt und ich konnte erkennen, dass es nicht echt war."

„Und ich sagte ihm,“ sagte Minerva, ihre Stimme zitterte ein wenig, „dass Sie und ich und Severus die einzigen seien, die es wüssten."

„Was Sie als Zugeständnis tat um zu verhindern, dass ich einfach herumlaufe und Fragen stelle, womit ich gedroht habe, falls sie nicht reden würde,“ sagte Harry. Der Junge kicherte kurz. „Ich hätte wirklich einen von Ihnen allein erwischen und sagen sollen, sie habe mir alles erzählt, um zu sehen ob Ihnen etwas herausrutscht. Hätte wahrscheinlich nicht funktioniert, wäre aber einen Versuch wert gewesen.“ Der Junge lächelte erneut. „Die Drohung ist noch auf dem Tisch und ich erwarte irgendwann \emph{umfassend} informiert zu werden."

Severus warf ihr einen Blick voller Genugtuung zu. Minerva reckte ihr Kinn und ertrug es, sie wusste es war verdient.

Dumbledore lehnte sich in seinem gepolsterten Thron zurück. Seine Augen waren so kalt wie Minerva es je bei ihm gesehen hatte, seit dem Tag an dem sein Bruder starb. „Und du drohst, uns mit Voldemort allein zu lassen, wenn wir deinen Wünschen nicht entsprechen."

Harrys Stimme war rasiermesserscharf. „Ich bedaure Ihnen mitteilen zu müssen, dass Sie nicht der Mittelpunkt des Universums sind. Ich drohe nicht, mich vom magischen Britannien abzuwenden. Ich drohe, mich von Ihnen abzuwenden. Ich bin kein demütiger kleiner Frodo. Dies ist \emph{meine} Aufgabe und wenn Sie dabei sein wollen, werden Sie nach \emph{meinen} Regeln spielen."

In Dumbledores Gesicht lag noch immer Kälte. „Ich fange an zu bezweifeln, dass Sie als der Held geeignet sind, Mr. Potter."

Der Blick den Harry erwiderte war ebenso eisig. „Ich beginne an Ihrer Eignung als mein Gandalf zu zweifeln, \emph{Mr. Dumbledore.} Boromir war wenigstens ein verständlicher Fehler. Was hat dieser \emph{Nazgul} unter meinen Gefährten zu suchen?"

Minerva verstand überhaupt nichts mehr. Sie blickte zu Severus, um zu sehen, ob er dem folgen konnte und sie sah, dass Severus sein Gesicht aus Harrys Blickfeld weggedreht hatte und lächelte.

„Ich nehme an,“ sagte Dumbledore langsam, „dass dies von Ihrer Warte aus eine berechtigte Frage ist. Also, Mr. Potter, wenn Professor Snape Sie fortan in Ruhe lässt, wird dies dann das letzte mal sein, dass diese Angelegenheit zur Sprache kommt oder werde ich Sie jede Woche mit einer neuen Forderung hier wiedersehen?"

„\emph{Mich} in Ruhe lässt?“ Harrys Stimme klang empört. „Ich bin nicht sein einziges Opfer und sicherlich nicht das verwundbarste! \emph{Haben Sie vergessen, wie schutzlos Kinder sind? Welches Leid ihnen geschieht?} Fortan wird Severus jeden Schüler von Hogwarts mit angemessener und professioneller Höflichkeit behandeln oder Sie werden sich einen neuen Meister der Zaubertränke suchen oder einen neuen Helden!"

Dumbledore fing an zu lachen. Ein warmes, amüsiertes Lachen aus voller Kehle, als habe Harry gerade einen komischen Tanz vor ihm vollführt.

Minerva wagte nicht, sich zu bewegen. Ihre Augen flackerten und sie sah, dass Severus ebenso regungslos da saß.

Harrys Ausdruck wurde noch kälter. „Sie missverstehen mich, Schulleiter, wenn Sie das für einen Witz halten. Dies ist keine Bitte. Dies ist Ihre Bestrafung."

„Mr. Potter -“ sagte Minerva. Sie wusste nicht einmal was sie sagen sollte. Doch sie konnte das nicht einfach durchgehen lassen.

Harry bedeutete ihr still zu sein und sprach weiter mit Dumbledore. „Und wenn Ihnen das unhöflich erscheint,“ sagte Harry, seine Stimme jetzt etwas weniger hart, „schien es nicht weniger unhöflich, als Sie es zu mir sagten. So etwas würden Sie zu niemandem sagen, den Sie als echtes menschliches Wesen betrachten anstatt nur als Ihnen untergeordnetes Kind und ich werde Ihnen genau die selbe Höflichkeit entgegenbringen, wie Sie mir -"

„Oh, in der Tat, das ist sie, das ist meine Bestrafung, wenn es jemals eine gab! \emph{Natürlich} erpresst du mich hier um deine Mitschüler zu retten, nicht dich selbst! Ich weiß nicht, wie ich etwas anderes annehmen konnte!“ Dumbledore lachte jetzt noch stärker. Er klopfte dreimal mit der Faust auf den Schreibtisch.

Harrys Blick wurde unsicher. Er wand sein Gesicht ihr zu, sprach sie zum ersten mal an. „Verzeihen Sie,“ sagte Harry. Seine Stimme schien zu schwanken. „Muss er seine Medikamente nehmen oder sowas?"

„Ah...“ Minerva hatte nicht die leiseste Ahnung, was sie sagen sollte.

„Also,“ sagte Dumbledore. Er wischte sich Tränen aus den Augen. „Entschuldige. Tut mir leid wegen der Unterbrechung. Bitte mach weiter mit der Erpressung."

Harry öffnete den Mund, schloss ihn dann wieder. Er schien jetzt etwas unsicher. „Ah... er muss außerdem aufhören, die Gedanken der Schüler zu lesen."

„Minerva,“ sagte Severus mit tödlicher Stimme, „du -"

„Der Sprechende Hut warnte mich,“ sagte Harry.

"\emph{Was?}"

"Kann nicht mehr sagen. Jedenfalls, ich denke das war's. Ich bin fertig."

Stille.

„Was jetzt?“ sagte Minerva, als klar wurde, dass niemand sonst etwas sagen würde.

„Was jetzt?“ echote Dumbledore. „Nun, jetzt gewinnt der Held, natürlich."

„\emph{Was?}“ sagten Severus, Minerva und Harry.

„Nun, er scheint uns wirklich in die Ecke gedrängt zu haben,“ sagte Dumbledore fröhlich lächelnd. „Doch Hogwarts \emph{braucht} einen bösen Meister der Zaubertränke oder es wäre keine richtige Zauberschule, nicht wahr? Also, wie wäre es damit, dass Professor Snape sich nur Schülern ab dem fünften Jahr gegenüber furchtbar verhält?"

„\emph{Was?}“ sagten alle drei erneut.

"Wenn es die verletzlichsten Opfer sind, um die du besorgt bist. Vielleicht hast du recht, Harry. Vielleicht \emph{habe} ich über die Jahrzehnte vergessen wie es ist, ein Kind zu sein. Also schließen wir einen Kompromiss. Severus wird weiterhin Slytherin unfairerweise Punkte verleihen und die Disziplin in seinem Haus schleifen lassen und sich furchtbar verhalten gegenüber Schülern im fünften Jahr und darüber. Den anderen wird er Angst einjagen, sie aber nicht missbrauchen. Er wird versprechen nur Gedanken zu lesen, wenn die Sicherheit eines Schülers es erfordert. Hogwarts wird seinen bösen Meister der Zaubertränke haben und die verletzlichsten Opfer, wie du sagtest, werden sicher sein."

Minerva McGonagall war so schockiert wie niemals zuvor in ihrem Leben. Sie blickte unsicher zu Severus, dessen Gesicht vollkommen neutral geblieben war, als wisse er nicht, welchen Ausdruck es tragen solle.

„Ich nehme an, das ist akzeptabel,“ sagte Harry. Seine Stimme klang etwas seltsam.

„Das können Sie nicht ernst meinen,“ sagte Severus, seine Stimme so ausdruckslos wie sein Gesicht.

„Ich bin sehr dafür,“ sagte Minerva langsam. Sie war so sehr dafür, dass ihr Herz unter ihrem Umhang wild pochte. „Aber was sollen wir nur den Schülern sagen? Sie haben das vielleicht nicht in Frage gestellt, als Severus sich allen gegenüber... furchtbar verhalten hat, aber -"

„Harry kann den anderen Schülern erzählen, dass er ein schreckliches Geheimnis über Severus entdeckt und eine kleine Erpressung durchgezogen hat,“ sagte Dumbledore. „Es ist ja auch wahr; er hat herausgefunden, dass Severus Gedanken gelesen hat und er hat uns definitiv erpresst."

„Das ist Wahnsinn!“ explodierte Severus.

„Buah ha ha!“ sagte Dumbledore.

„Ah...“ sagte Harry unsicher. „Und wenn mich jemand fragt, warum Fünftklässler und darüber leer ausgingen? Ich könnte mir vorstellen, dass sie sauer sind und dieser Teil war nicht unbedingt meine Idee -"

„Sag ihnen,“ sagte Dumbledore, „dass nicht du den Kompromiss vorgeschlagen hast, dass das alles war, was du kriegen konntest. Und dann weigere dich, noch irgendetwas zu sagen. Das ist, ebenfalls, wahr. Es ist eine Kunst für sich, mit etwas Übung lernst du es schon."

Harry nickte langsam. „Und die Punkte, die er Ravenclaw abgezogen hat?"

"Sie dürfen nicht zurückgegeben werden."

Es war Minerva, die das gesagt hatte.

Harry sah sie an.

„Es tut mir leid, Mr. Potter,“ sagte sie. Es \emph{tat} ihr leid, doch es musste getan werden. „Es \emph{muss} Konsequenzen für Ihr Fehlverhalten geben oder diese Schule wird auseinander brechen.„

Harry zuckte mit den Schultern. „Akzeptabel,“ sagte er schlicht. „Aber in Zukunft wird Severus weder mein Haus mit hineinziehen, indem er mir Punkte abzieht, noch wird er meine wertvolle Zeit mit Nachsitzen verschwenden. Sollte er der Ansicht sein, dass mein Verhalten Korrekturbedarf zeigt, möge er seine Sorgen Professor McGonagall mitteilen."

„Harry,“ sagte Minerva, „wirst du dich weiterhin der Schuldisziplin unterwerfen oder jetzt über dem Gesetz stehen, wie Severus es tat?„

Harry sah sie an. Etwas warmes lag kurz in seinem Blick, bevor es unterdrückt wurde. „Ich werde weiterhin gegenüber allen Mitgliedern des Lehrkörpers, die nicht wahnsinnig oder böse sind, ein gewöhnlicher Schüler sein, solange sie nicht unter Druck gesetzt werden, von denen die es sind.“ Harry blickte kurz zu Severus, wandte sich dann wieder Dumbledore zu. „Lassen Sie Minerva zufrieden und ich werde in ihrer Gegenwart ein normaler Hogwarts-Schüler sein. Keine besonderen Privilegien oder Immunitäten."

„Wundervoll,“ sagte Dumbledore ernsthaft. „Gesprochen wie ein wahrer Held."

„Und,“ sagte sie,“ Mr. Potter muss sich für sein heutiges Verhalten öffentlich entschuldigen."

Harry warf ihr noch einen Blick zu. Dieser war ein wenig skeptisch.

„Der Schuldisziplin wurde durch Ihre Handlungen schwerer Schaden zugefügt, Mr. Potter,“ sagte Minerva. „Sie muss wiederhergestellt werden."

"Ich glaube, Professor McGonagall, dass Sie die Wichtigkeit dessen, was Sie Schuldisziplin nennen, erheblich überschätzen, verglichen damit Geschichte von einem lebendigen Lehrer unterrichten zu lassen oder ihre Schüler nicht zu foltern. Die geltende Hierarchie aufrechtzuerhalten und die Einhaltung ihrer Regeln durchzusetzen scheint sehr viel weiser, richtiger und wichtiger zu sein, wenn man an der Spitze steht und das Durchsetzen übernimmt, als wenn man sich am unteren Ende befindet und ich kann wenn nötig Studien zu diesem Effekt zitieren. Ich könnte zu diesem Punkt noch stundenlang weitermachen, aber ich werde es dabei belassen.„

Minerva schüttelte den Kopf. „Mr. Potter, Sie unterschätzen die Wichtigkeit von Disziplin, weil Sie sie selbst nicht benötigen -“ Sie hielt inne. Das war nicht richtig herausgekommen und Severus, Dumbledore und selbst Harry warfen ihr seltsame Blicke zu. „Zum Lernen, meine ich. Nicht jedes Kind kann ohne Autorität lernen. Und es sind die anderen Kinder, Mr. Potter, denen es schaden wird, wenn sie Ihrem Beispiel nacheifern."

Harrys Lippen verzogen sich zu einem schiefen Lächeln. „Die erste und letzte Zuflucht ist die Wahrheit. Die Wahrheit ist, dass ich nicht hätte wütend werden sollen, den Unterricht nicht hätte unterbrechen sollen, nicht hätte tun sollen, was ich getan habe und allen ein schlechtes Vorbild war. Die Wahrheit ist auch, dass Severus Snape sich auf eine Weise benommen hat, die eines Hogwarts-Professors unwürdig ist und von nun an mehr Rücksicht auf die verletzten Gefühle seiner Schüler in ihrem vierten Jahr und darunter nehmen wird. Wir könnten beide aufstehen und die Wahrheit sagen. Damit könnte ich leben."

„Davon träumen Sie, Potter!“ spuckte Severus.

„Außerdem,“ sagte Harry, grimmig lächelnd, „wenn die Schüler sehen, dass die Regeln für \emph{jeden} gelten... auch für Professoren, nicht nur für arme hilflose Schüler, die nichts als Elend aus dem System erfahren... nun, die positiven Effekte auf die Schuldisziplin sollten \emph{enorm} sein."

Es gab eine kurze Pause, dann kicherte Dumbledore. „Minerva denkt, dass du rechter hast, als sie dir zugestehen will."

Harrys Blick zuckte weg von Dumbledore, hinunter zum Fußboden. „Lesen \emph{Sie ihre} Gedanken?"

„Gesunder Menschenverstand wird oft fälschlich für Legilimentik gehalten,“ sagte Dumbledore. „Ich werde diese Angelegenheit mit Severus besprechen und keine Entschuldigung wird von dir verlangt werden, es sei denn er entschuldigt sich ebenso. Und nun erkläre ich diese Sache für beendet, zumindest bis zum Mittagessen. „Allerdings, Harry, fürchte ich, dass Minerva mit dir noch über eine andere Angelegenheit sprechen möchte. Und in dieser Sache habe ich keinerlei Druck ausgeübt. Minerva, würdest du bitte?"

Minerva erhob sich von ihrem Stuhl und fiel beinahe. Es war zu viel Adrenalin in ihrem Blut, ihr Herz schlug zu schnell.

„Fawkes,“ sagte Dumbledore, „bitte begleite sie."

„Nicht nötig -“ setzte sie an.

Dumbledore schoss ihr einen Blick zu und sie schwieg still.

Der Phoenix schwebte durch den Raum, wie eine geschmeidig leckende Flammenzunge und landete auf ihrer Schulter. Sie fühlte die Wärme durch ihren Umhang, sie durchdrang ihren ganzen Körper.

„Bitte folgen Sie mir, Mr. Potter,“ sagte sie, jetzt mit fester Stimme und sie verschwanden durch die Tür.

--------------------------------------------------------------------------------------------------------------------------------------------

\hfill\break Sie standen auf den rotierenden Stufen, stiegen still herab.

Minerva wusste nicht, was sie sagen sollte. Sie kannte die Person nicht, die neben ihr stand.

Und Fawkes begann zu summen.

Es war zart und sanft, wie die Melodie eines Lagerfeuers klingen mochte und es legte sich über ihren Geist, lindernd, tröstend, heilend was es berührte...

„\emph{Was} ist \emph{das?}“ flüsterte Harry neben ihr. Seine Stimme war unsicher, schwankend, veränderte ihre Tonlage.

„Das Lied des Phoenix,“ sagte Minerva, sich nicht wirklich bewusst, was sie sagte, ihre Aufmerksamkeit ganz bei der seltsamen, leisen Musik. „Sie hat, ebenfalls, Heilkräfte."

Harry wandte das Gesicht von ihr ab, doch sie erhaschte einen Blick auf etwas gequältes an ihm.

Der Abstieg schien sehr lange zu dauern oder vielleicht war es nur die Musik, die lange anzudauern schien und als sie aus der Nische traten, wo ein Wasserspeier gestanden hatte, hielt sie Harrys Hand fest in ihrer.

Als der Wasserspeier wieder an seine Stelle schritt, erhob sich Fawkes von ihrer Schulter, schwang herum und schwebte vor Harry.

Harry starrte Fawkes an wie hypnotisiert vom ewig wechselnden Schein eines Feuers.

„Was soll ich tun Fawkes?“ flüsterte Harry. „Ich hätte sie nicht beschützen können, wäre ich nicht zornig geworden."

Die Schwingen des Phoenix schlugen weiter, er schwebte weiterhin an Ort und Stelle. Kein Geräusch ertönte, bis auf das Schlagen der Flügel. Dann gab es ein Aufleuchten, wie von einem Feuer, das aufloderte und verglühte und Fawkes war verschwunden.

Sie beide blinzelten, als erwachten sie aus einem Traum oder vielleicht als schliefen sie wieder ein.

Minerva blickte hinab.

Harry Potters leuchtendes junges Gesicht blickte zu ihr auf.

„Sind Phoenixe Leute?“ sagte Harry. „Ich meine, sind sie klug genug, als Person zu gelten? Könnte ich mit Fawkes sprechen, wenn ich wüsste wie?„

Minerva blinzelte schwer. Dann blinzelte sie erneut. „Nein,“ sagte Minerva, ihre Stimme wankte. „Phoenixe sind Wesen mächtiger Magie. Ihre Magie gibt ihrer Existenz eine Bedeutung, die kein einfaches Tier besitzen könnte. Sie sind Feuer, Licht, Heilung, Wiedergeburt. Doch schlussendlich, nein."

"Wo kann ich einen bekommen?"

Minerva beugte sich hinab und umarmte ihn. Sie hatte es nicht beabsichtigt, doch sie schien in der Sache keine große Wahl zu haben.

Als sie sich erhob, hatte sie Schwierigkeiten zu sprechen. Doch sie musste fragen. „Was ist heute geschehen, Harry?"

"Ich weiß auch keine Antwort auf irgendeine der wichtigen Fragen. Davon abgesehen würde ich lieber eine Weile nicht darüber nachdenken."

Minerva nahm seine Hand erneut in ihre und sie gingen den Rest des Weges still zusammen.

Es war nur eine kurze Strecke, da sich das Büro der Stellvertreterin natürlicherweise nahe dem des Schulleiters befand.

Minerva setzte sich hinter ihren Schreibtisch.

Harry setzte sich vor ihren Schreibtisch.

„Nun denn,“ flüsterte Minerva. Sie hätte fast alles gegeben, das nicht zu tun oder nicht diejenige zu sein, die es tun musste oder dass es irgendwann, nur nicht jetzt geschehen müsse. „Es geht um eine Frage der Schuldisziplin. Von der Sie nicht ausgenommen sind."

"Nämlich?"

Er wusste es nicht. Er hatte es sich noch nicht zusammengereimt. Sie fühlte, wie es ihr die Kehle zuschnürte. Doch es gab etwas zu tun und sie würde sich nicht davor drücken.

„Mr. Potter,“ sagte Professor McGonagall, „ich muss Sie um Ihren Zeitumkehrer bitten."

All der Friede des Phoenix verschwand augenblicklich von seinem Gesicht und Minerva fühlte sich, als habe sie ihn gerade erstochen.

„\emph{Nein!}“ sagte Harry. Panik lag in seiner Stimme. „Ich brauche ihn, ich werde Hogwarts nicht mehr besuchen können, ich werde nicht schlafen können!"

„Sie werden schlafen können,“ sagte sie. „Das Ministerium hat die Schutzhülle für Ihren Zeitumkehrer geliefert. Ich werde sie so verzaubern, dass sie sich nur zwischen 9 Uhr abends und Mitternacht öffnet."

Harrys Gesicht verzerrte sich. „Aber - aber ich -"

"Mr. Potter, wie oft haben Sie Ihren Zeitumkehrer seit Montag benutzt? Wie viele Stunden?"

„Ich...“ sagte Harry. „Moment, lassen Sie mich nachrechnen -“ Er blickte hinab auf seine Armbanduhr.

Minerva fühlte einen Anflug von Trauer. Das hatte sie sich gedacht. „Dann waren es nicht nur zwei Stunden pro Tag. Ich vermute, würde ich Ihre Schlafsaal-Kameraden fragen, fände ich heraus, dass Sie Mühe hatten lange genug aufzubleiben, um zu einer vernünftigen Zeit zu schlafen und jeden Morgen früher und früher aufwachten. Korrekt?"

Harrys Gesicht sagte alles, was sie wissen musste.

„Mr. Potter,“ sagte sie sanft, „es gibt Schüler, denen kein Zeitumkehrer anvertraut werden kann, sie werden süchtig danach. Wir geben ihnen einen Trank, der ihren Schlafzyklus um den notwendigen Betrag verlängert, doch sie verwenden den Zeitumkehrer für mehr als nur das Besuchen ihrer Klassen. Und deshalb müssen wir sie zurücknehmen. Mr. Potter, Sie haben begonnen, Ihren Zeitumkehrer als Ihre Lösung für alles zu gebrauchen und das oft auf sehr törichte Weise. Sie haben ihn benutzt, um ein Erinnermich zurück zu bekommen. Sie sind aus einem Schrank verschwunden, so dass es andere Schüler mitbekamen, anstatt zurückzugehen, nachdem Sie draußen waren und mich oder jemand anders zu holen, um die Tür zu öffnen."

Harrys Gesichtsausdruck zufolge hatte er daran nicht gedacht.

„Und was noch wichtiger ist,“ sagte sie, „Sie hätten einfach in Professor Snapes Unterricht sitzen sollen. Und zusehen. Und zum Ende des Unterrichts gehen. Wie Sie es getan hätten, wenn Sie keinen Zeitumkehrer besessen hätten. Es gibt einige Schüler, denen man keinen Zeitumkehrer anvertrauen kann, Mr. Potter. Sie sind einer von ihnen. Es tut mir leid."

„Aber ich \emph{brauche} ihn!“ platzte Harry heraus. „Was wenn Slytherins mich bedrohen und ich entkommen muss? Er gibt mir \emph{Sicherheit -}"

"Alle anderen Schüler in diesem Schloss trägt das gleiche Risiko und ich versichere Ihnen, sie überleben es. In diesem Schloss ist seit fünfzig Jahren kein Schüler gestorben. Mr. Potter, Sie werden Ihren Zeitumkehrer übergeben und zwar jetzt."

Harrys Gesicht verzog sich vor Qual, doch er zog den Zeitumkehrer unter seinem Umhang hervor und gab ihn ihr.

Aus ihrem Schreibtisch zog Minerva eine der Schutzhüllen, die nach Hogwarts gesandt worden waren. Sie ließ die Abdeckung über dem rotierenden Stundenglas des Zeitumkehrers in Position schnappen und legte dann ihren Zauberstab auf die Hülle, um die Verzauberung zu vervollständigen.

„\emph{Das ist nicht fair!}“ kreischte Harry. „Ich habe heute halb Hogwarts vor Professor Snape gerettet, ist es richtig, dass ich dafür bestraft werde? Ich habe den Ausdruck auf Ihrem Gesicht gesehen, Sie \emph{hassten} was er tat!"

Minerva sprach einige Augenblicke lang nicht. Sie wirkte den Zauber.

Als sie fertig war und aufblickte, wusste sie, dass ihr Gesichtsausdruck streng war. Vielleicht war es falsch das zu tun. Und dann wieder war es vielleicht das richtige. Vor ihr saß ein trotziges Kind und das war \emph{nicht} das Ende des Universums.

„\emph{Fair,} Mr. Potter?“ schnappte sie. „Ich musste \emph{zwei Berichte} über den öffentlichen Gebrauch von Zeitumkehrern an das Ministerium an \emph{zwei aufeinander folgenden Tagen} schreiben! Seien Sie \emph{extrem} dankbar, dass es Ihnen erlaubt wurde, den Zeitumkehrer auch nur in eingeschränkter Form zu behalten! Der Schulleiter hat persönlich per Floh-Netzwerk bei ihnen darum gebeten und wären Sie nicht der Junge-der-überlebt-hat hätte selbst das nicht ausgereicht!"

Harry starrte sie an.

Sie wusste, dass er das wütende Gesicht von Professor McGonagall sah.

Harrys Augen füllten sich mit Tränen.

„Es, tut mir, leid,“ flüsterte er, nun mit erstickter und gebrochener Stimme. „Es, tut mir, leid Sie, enttäuscht, zu haben..."

„Auch mir tut es leid, Mr. Potter,“ sagte sie streng und übergab ihm den nun eingeschränkten Zeitumkehrer. „Sie können gehen."

Harry drehte sich um und floh aus ihrem Büro, schluchzend. Sie hörte, wie seine Füße sich trappelnd den Korridor entlang entfernten, dann wurde das Geräusch abgeschnitten als die Tür zuschwang.

„Mir tut es auch leid, Harry,“ flüsterte sie in den stillen Raum. „Mir tut es auch leid."

--------------------------------------------------------------------------------------------------------------------------------------------

\hfill\break Fünfzehn Minuten seit Beginn des Mittagessens.

Niemand sprach mit Harry. Einige Ravenclaws schossen ihm wütende Blicke zu, andere mitfühlende, einige der jüngsten Schüler sahen ihn sogar bewundernd an, doch niemand sprach mit ihm. Selbst Hermine hatte nicht versucht herüber zu kommen.

Fred und George waren vorsichtig näher getreten. Sie hatten nichts gesagt. Das Angebot war klar, ebenso dass es optional war. Harry hatte ihnen gesagt, dass er hinüber kommen würde, wenn der Nachtisch serviert wurde, nicht früher. Sie hatten genickt und sich schnell entfernt.

Es lag wahrscheinlich an dem vollkommen ausdruckslosen Gesicht, das Harry zur Schau trug.

Die anderen dachten wahrscheinlich, er hielte seinen Ärger oder Bestürzung im Zaum. Sie wussten, weil sie gesehen hatten, wie Flitwick kam und ihn holte, dass er ins Büro des Schulleiters gerufen worden war.

Harry versuchte, nicht zu grinsen, denn wenn er grinste, würde er zu lachen anfangen und wenn er zu lachen anfing, würde er nicht mehr aufhören bis die netten Leute in weißen Jacken kamen und ihn fortbrachten.

Es war zu viel. Es war einfach alles zu viel. Harry hatte fast auf die Dunkle Seite hinüber gewechselt, seine dunkle Seite hatte Dinge getan, die im Rückblick wahnsinnig erschienen, seine dunkle Seite hatte einen unmöglichen Sieg errungen, der vielleicht echt gewesen war und vielleicht auch nur die reine Willkür eines verrückten Schulleiters, seine dunkle Seite hatte seine Freunde beschützt. Er hielt es einfach nicht mehr aus. Er brauchte Fawkes, der noch einmal für ihn sang. Er brauchte den Zeitumkehrer, um sich abzusetzen und sich eine ruhige Stunde zur Erholung zu gönnen, doch das war keine Option mehr und der Verlust war wie ein Loch in seiner Existenz, aber er konnte nicht darüber nachdenken, denn dann fing er vielleicht zu lachen an.

Zwanzig Minuten. Alle Schüler, die am Mittagessen teilnehmen würden, waren erschienen, fast niemand fehlte.

Das Klopfen eines Löffels klang durch die Große Halle.

„Wenn ich um Ihre Aufmerksamkeit bitten darf,“ sagte Dumbledore. „Harry Potter hat etwas mitzuteilen."

Harry atmete tief durch und erhob sich. Er ging zum Lehrertisch hinüber, alle Augen auf ihn gerichtet.

Harry wand sich um und ließ den Blick über die vier Tische schweifen.

Es wurde schwerer und schwerer, nicht zu grinsen, doch Harry behielt sein ausdrucksloses Gesicht bei, während er seine kurze, einstudierte Rede hielt.

„Die Wahrheit ist heilig,“ sagte Harry tonlos. „Eines meiner wertvollsten Besitztümer ist ein Button auf dem steht 'Sage immer die Wahrheit, selbst wenn deine Stimme zittert'. Nun denn, dies ist die Wahrheit. Denkt daran. Ich sage es nicht, weil ich dazu gezwungen bin, ich sage es, weil es wahr ist. Was ich in Professor Snapes Unterricht getan habe, war töricht, dumm, kindisch und eine unentschuldbare Verletzung der Regeln von Hogwarts. Ich habe den Unterricht gestört und meine Mitschüler ihrer unersetzlichen Lernzeit beraubt. Alles nur, weil ich mein Temperament nicht zügeln konnte. Ich hoffe, dass nicht einer von euch meinem Beispiel folgen wird. Ich werde ernsthaft versuchen, es selbst nie wieder zu tun.„

Viele der Schüler trugen jetzt einen ernsten, unglücklichen Ausdruck auf ihren Gesichtern, wie man ihn bei einer Zeremonie anlässlich des Verlustes eines gefallenen Champions beobachten könnte. Unter den jüngeren am Gryffindor-Tisch war der Ausdruck fast universell vertreten.

Bis Harry seine Hand hob.

Er hob sie nicht hoch. Das wäre vielleicht voreilig erschienen. Er hob sie sicherlich nicht in Severus Richtung. Harry hob einfach die Hand auf Brusthöhe und schnippte leise mit den Fingern, eine Geste, die man mehr sah als hörte. Es war möglich, dass man sie vom Großteil des Lehrertisches aus überhaupt nicht sah.

Die scheinbare Geste des Widerstands hatte plötzliches Lächeln von einigen der jüngeren Schüler und Gryffindors, kalt-überhebliches höhnisches Grinsen von Slytherin und Stirnrunzeln und besorgte Blicke von allen anderen zur Folge.

Harry hielt sein Gesicht ausdruckslos. „Danke,“ sagte er. „Das ist alles."

„Danke, Mr. Potter,“ sagte der Schulleiter. „Und nun hat uns Professor Snape ebenfalls etwas mitzuteilen.„

Severus erhob sich geschmeidig von seinem Platz am Lehrertisch. „Ich wurde darauf aufmerksam gemacht,“ sagte er, „dass meine eigenen Handlungen ihren Anteil daran hatten, den zugegeben unentschuldbaren Zorn von Mr. Potter zu provozieren und in der darauf folgenden Diskussion wurde mir klar, dass ich vergessen hatte, wie leicht die Gefühle der jungen und unmündigen verletzt werden können -"

Erstickte Geräusche waren zur gleichen Zeit von vielen Leuten zu vernehmen.

Severus fuhr fort als habe er es nicht gehört. „Der Zaubertränke-Klassenraum ist ein gefährlicher Ort und ich bin immer noch der Ansicht, dass strikte Disziplin von Nöten ist, doch von nun an werde ich der... emotionalen Fragilität... meiner Schüler in ihrem vierten Jahr und darunter mehr Beachtung schenken. Mein Punktabzug von Ravenclaw bleibt in Kraft, aber ich werde das Nachsitzen von Mr. Potter zurückziehen. Danke.„

Ein einzelnes Klatschen ertönte aus Richtung Gryffindor und schneller als der Blitz war Severus Zauberstab in seiner Hand und „\emph{Quietus!}“ brachte den Täter zum Schweigen.

„Ich werde auch weiterhin Disziplin und Respekt in \emph{all} meinen Klassen einfordern,“ sagte Severus kalt, „und jeder der sich mit mir anlegt, wird es bereuen."

Er setzte sich.

„Danke auch Ihnen!“ sagte Schulleiter Dumbledore heiter. „Weitermachen!„

Und Harry, immer noch ausdruckslos, ging zu seinem Sitz in Ravenclaw zurück.

Es gab eine Gesprächsexplosion. Zwei Worte waren zu Anfang klar zu erkennen. Das erste war ein einleitendes „Was -“, mit dem viele verschiedene Sätze begannen, wie „Was ist gerade passiert -“ und „Was zur Hölle -“ Das zweite war „\emph{Ratzeputz!}“ als Schüler fallengelassenes Essen und verspuckte Getränke von sich selbst, dem Tischtusch und einander abputzten.

Einige Schüler weinten ganz offen. Ebenso Professor Sprout.

Am Gryffindor-Tisch, wo ein Kuchen mit einundfünfzig unangezündeten Kerzen wartete, flüsterte Fred, „Ich glaube das ist hier vielleicht nicht unsere Liga, George."

Und von diesem Tag an würde es, egal was Hermine auch allen erzählen mochte, eine anerkannte Legende von Hogwarts sein, dass Harry Potter absolut alles möglich machen konnte, indem er mit den Fingern schnippte.

* engl.: \emph{Magical Drafts and Potions}\\ ** engl.: \emph{trichinobezoar, siehe Wikipedia-Artikel \emph{Bezoar} und \emph{Rapunzelsyndrom.}}\\ *** engl.: \emph{One Thousand Magical Herbs and Fungi}

