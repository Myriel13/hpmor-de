

\hypertarget{alles-was-ich-glaube-ist-falsch}{% \section{2. Alles, was ich glaube, ist falsch}\label{alles-was-ich-glaube-ist-falsch}}

\textbf{Kapitel 2: Alles, was ich glaube, ist falsch}\\

\#hinzufügen "stddisclaimer.h"

--------------------------------------------------------------------------------------------------------------------------------------------

\emph{"Natürlich war es meine Schuld. Es gibt hier niemand anderen, der für irgendetwas verantwortlich sein könnte."}

--------------------------------------------------------------------------------------------------------------------------------------------

"Also, nur um das klarzustellen," sagte Harry, "wenn die Professorin dich schweben lässt, Dad, weißt du, dass du nicht an irgendwelchen Drähten befestigt wurdest, das wird als Beleg geeignet sein. Du wirst nicht einfach deine Meinung ändern und sagen, dass es ein Zaubertrick ist. Das wäre nicht fair. Wenn du das denkst, solltest du es \emph{jetzt} sagen und wir können uns stattdessen ein anderes Experiment ausdenken."

Harrys Vater, Professor Michael Verres-Evans, rollte mit den Augen. "Ja, Harry."

"Und du, Mum, deine Theorie besagt, dass die Professorin das tun können sollte und wenn es nicht passiert, gibst du zu, dass du Unrecht hattest. Nichts darüber, dass Magie nicht funktioniert, wenn Leute sie skeptisch sehen oder so etwas."

Die Stellvertretende Schulleiterin Minerva McGonagall sah Harry mit amüsiertem Gesichtsausdruck an. Sie sah ziemlich hexenhaft aus, mit ihrem schwarzen Umhang und spitzem Hut, aber wenn sie sprach, klang sie sehr formell und schottisch, was mit dem Aussehen überhaupt nicht zusammenpasste. Auf den ersten Blick sah sie aus wie jemand, der gackern und kleine Babys in Kessel werfen sollte, aber der ganze Effekt wurde ruiniert, sobald sie den ihren Mund aufmachte. "Ist das ausreichend, Mr. Potter?" sagte sie. "Soll ich mit der Demonstration beginnen?"

"\emph{Ausreichend?} Wahrscheinlich nicht," sagte Harry. "Aber es wird zumindest \emph{helfen}. Beginnen Sie, Stellvertretende Schulleiterin."

"Nur Professor reicht aus," sagte sie und dann, "\emph{Wingardium Leviosa}."

Harry sah seinen Vater an.

"Huh," sagte Harry.

Sein Vater schaute zurück. "Huh," echote er.

Dann sah Professor Verres-Evans wieder zu Professor McGonagall. "Alles klar, Sie können mich jetzt runter lassen."

Sein Vater wurde vorsichtig auf den Boden abgesenkt.

Harry fuhr sich mit einer Hand durchs Haar. Vielleicht lag es nur an diesem seltsamen Teil von ihm der schon \emph{vorher} überzeugt gewesen war, aber… "Das ist ein bisschen enttäuschend," sagte Harry. "Man würde meinen, es ginge irgendeine dramatischere Art von geistigem Ereignis damit einher, eine Beobachtung von verschwindend geringer Wahrscheinlichkeit gemacht zu haben -" Harry hielt inne. Mum, die Hexe und selbst sein Dad sahen ihn schon wieder mit \emph{diesem Blick} an. "Ich meine damit, herauszufinden, dass alles, was ich glaube, falsch ist."

Ernsthaft, es hätte dramatischer sein sollen. Sein Gehirn hätte seine komplette momentane Ansammlung von Hypothesen über das Universum die Toilette herunterspülen sollen, von denen keine erlaubte, dass das passierte. Aber stattdessen schien sein Gehirn nur zu sagen, \emph{Alles klar, ich habe die Hogwarts-Professorin ihren Zauberstab schwenken und deinen Vater in die Luft heben sehen, na und?}

Die Hexen-Dame lächelte wohlwollend auf sie herab und sah ziemlich amüsiert aus. "Hätten Sie gern eine weitere Demonstration, Mr. Potter?"

"Das müssen Sie nicht," sagte Harry. "Wir haben ein definitives Experiment durchgeführt. Aber…" Harry zögerte. Er konnte sich nicht beherrschen. Tatsächlich, \emph{sollte} er sich unter diesen Umständen auch nicht beherrschen. Es war vollkommen richtig, neugierig zu sein. "Was \emph{können} Sie sonst noch tun?"

Professor McGonagall verwandelte sich in eine Katze.

Harry stolperte unwillkürlich zurück, so schnell zurückrudernd, dass er über einen Bücherstapel fiel und mit einem \emph{zack} auf seinem Hintern landete. Seine Hände fuhren nach unten, um ihn abzufangen, ohne vollkommen auszureichen und es gab ein warnendes Stechen in seiner Schulter als sein Gewicht ungebremst herunterfiel.

Sofort verformte sich die kleine getigerte Katze wieder zu einer Frau mit Umhang. "Es tut mir leid, Mr. Potter," sagte die Hexe und klang aufrichtig, obwohl ihre Mundwinkel nach oben zuckten. "Ich hätte Sie warnen sollen."

Harry schnappte nach Luft. Seine Stimme klang erstickt. \emph{"Das können Sie nicht TUN!"}

"Es ist nur eine Verwandlung," sagte Professor McGonagall. "Eine Animagus-Transformation, um genau zu sein."

"Sie haben sich in eine Katze verwandelt! Eine \emph{KLEINE} Katze! Sie haben den Energieerhaltungssatz verletzt! Das ist nicht nur eine beliebige Regel, sie wird von der Gestalt des Hamilton-Operators vorausgesetzt! Sie zu verwerfen zerstört die Unitarität und das bedeutet Überlichtgeschwindigkeits-Kommunikation! Und Katzen sind \emph{KOMPLIZIERT!} Ein menschlicher Geist kann sich nicht einfach die Anatomie einer kompletten Katze vorstellen und die ganze Biochemie der Katze und was ist mit der \emph{Neurologie?} Wie können Sie überhaupt weiter \emph{denken} mit einem Gehirn in Katzen-Größe?"

Professor McGonagalls Lippen zuckten jetzt stärker. "Magie."

"Magie \emph{ist nicht genug} um das zu tun! Sie müssten Gott sein!"

Professor McGonagall blinzelte. "Das ist das erste mal, dass man mich \emph{so} genannt hat."

Ein Schleier legte sich über Harrys Sicht, als er versuchte zu verstehen, was gerade zerschlagen worden war. Die ganze Idee eines einheitlichen Universums mit mathematisch gleichförmigen Regeln, das war es, was gerade die Toilette heruntergespült worden war; die komplette Vorstellung von \emph{Physik}. Dreitausend Jahre des Auflösens großer komplizierter Dinge in kleinere Teile; zu entdecken, dass die Musik der Planeten die selbe Melodie hatte, wie ein fallender Apfel; herauszufinden, dass die wahren Gesetze vollkommen universell waren und nirgendwo Ausnahmen hatten und die Form einfacher Mathematik annahmen, die die kleinsten Teile beherrschte, \emph{ganz abgesehen davon}, dass der Geist das Gehirn war und das Gehirn aus Neuronen gemacht, ein Gehirn war, was eine Person \emph{war} -

Und dann verwandelte sich eine Frau in eine Katze, so viel zu all dem.\\ Einhundert Fragen kämpften um Vorrang über Harrys Lippen und der Sieger strömte heraus: "Und, und was für eine Beschwörung ist \emph{Wingardium Leviosa?} Wer erfindet die Worte zu diesen Zaubern, Vorschüler?"

"Das wird reichen, Mr. Potter," sagte Professor McGonagall knapp, obwohl unterdrückte Belustigung in ihren Augen glänzte. "Wenn Sie wünschen, Magie zu erlernen, schlage ich vor, dass wir die Formalitäten erledigen, damit Sie nach Hogwarts gehen können."

"Richtig," sagte Harry, etwas benommen. Er ordnete seine Gedanken. Der Marsch der Vernunft würde einfach von vorn beginnen müssen, das war alles; sie hatten immer noch die experimentelle Methode und das war das Entscheidende. "Wie komme ich dann nach Hogwarts?"

Ein ersticktes Lachen entrang sich Professor McGonagall, als ob von Pinzetten herausgezogen.

"Warte einen Moment, Harry," sagte sein Vater. "Weißt du noch, weshalb du bisher nicht zur Schule gegangen bist? Was ist mit deinem Zustand?"

Professor McGonagall drehte sich zu Michael um. "Sein Zustand? Was heißt das?"

"Ich schlafe nicht richtig," sagte Harry. Er schwenkte hilflos seine Hand. "Mein Schlafzyklus ist sechsundzwanzig Stunden lang, ich gehe jedesmal zwei Stunden später schlafen, jeden Tag. Ich kann nicht früher einschlafen und am nächsten Tag gehe ich noch zwei Stunden später schlafen. 10 Uhr abends, 12 Uhr nachts, 2 Uhr morgens, 4 Uhr morgens, bis einmal um die Uhr herum. Slebst wenn ich versuche früh aufzuwachen, macht es keinen Unterschied und ich bin den ganzen Tag lang ein Wrack. Deshalb bin ich bis jetzt nicht auf eine normale Schule gegangen."

"Einer der Gründe," sagte seine Mutter. Harry warf ihr einen zornigen Blick zu.

McGonagall gab ein langes \emph{Hmmmmm} von sich. "Ich kann mich nicht erinnern,von einem solchen Zustand schon einmal gehört zu haben…" sagte sie langsam. "Ich werde mich mit Madam Pomfrey beraten, um zu sehen, ob ihr irgendwelche Abhilfe einfällt." Dann hellte sich ihr Gesicht auf. "Nein, ich bin sicher, das wird kein Problem sein - ich werde rechtzeitig eine Lösung finden. Nun," und ihr Blick wurde wieder schärfer, "was sind diese \emph{anderen} Gründe?"

Harry warf seinen Eltern einen zornigen Blick zu. "Ich bin ein Gegner der Zwangsverpflichtung von Kindern, aus dem Grund, dass ich nicht unter dem Versagen eines in Auflösug begriffenen Schulsystems leiden sollte, Lehrer oder Lernmaterialien von auch nur geringstmöglicher angemessener Qualität bereitzustellen."

Daraufhin brachen Harrys Eltern in schallendes Gelächter aus, als ob sie das alles für einen großen Witz hielten. "Oh," sagte Harrys Vater mit leuchtenden Augen, "ist \emph{das} der Grund, warum du in deinem dritten Schuljahr eine Mathematik-Lehrerin gebissen hast?"

"\emph{Sie wusste nicht, was ein Logarithmus war!}"

"Natürlich," legte Harrys Mutter nach. "Sie zu beißen, war eine sehr erwachsene Reaktion darauf."

Harrys Vater nickte. "Eine anerkannte Methode um auf das Problem von Lehrern, die Logarithmen nicht verstehen, aufmerksam zu machen."

"Ich war \emph{sieben Jahre alt!} Wie lange werdet ihr das noch wieder aufwärmen?"

"Ich weiß," sagte seine Mutter mitfühlend, "da beißt du nur \emph{eine} Mathematik-Lehrerin und sie lassen dich das nie vergessen, oder?"

Harry drehte sich zu Professor McGonagall um. "Da haben Sie's! Sehen Sie, womit ich mich herumschlagen muss?"

"Entschuldigen Sie mich," sagte Petunia und floh durch die Hintertür in den Garten,von wo aus ihr schreiendes Gelächter klar zu hören war.

"Nun, ja, nun," Professor McGonagall schien aus irgendeinem Grund Probleme zu haben zu sprechen, "in Hogwarts haben keine Lehrer gebissen zu werden, haben sie verstanden, Mr. Potter?"

Harry blickte sie finster an. "Schön, ich werde niemanden beißen, der mich nicht zuerst beißt."

Professor Michael Verres-Evans musste auch schnell den Raum verlassen, als er das hörte.

"Nun," seufzte Professor McGonagall, nachdem Harrys Eltern sich wieder gefangen hatten und zurückgekommen waren. "Nun. Ich denke, unter diesen Umständen sollte ich es vermeiden mit Ihnen ihre Schulsachen einkaufen zu gehen, bis ein oder zwei Tage vor Schulbeginn."

"Was? Warum? Die anderen Kinder wissen schon Bescheid über Magie, oder nicht? Ich muss sofort damit anfangen, aufzuholen!"

Seien Sie versichert, Mr. Potter," antwortete Pofessor McGonagall, "dass Hogwarts vollkommen in der Lage ist, die Grundlagen zu lehren. Und ich vermute, Mr. Potter, dass, wenn ich Sie mit ihren Schulbüchern zwei Monate lang allein lasse, selbst ohne einen Zauberstab, ich bei der Rückkehr zu diesem Haus nur einen Krater vorfinden werde, aus dem lila Rauch aufsteigt, eine entvölkerte Stadt ringsum und eine Plage brennender Zebras, die das terrorisieren, was von England noch übrig ist."

Harrys Mutter und Vater nickten in perfektem Einklang.

"\emph{Mum! Dad!}"

