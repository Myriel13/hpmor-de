

\hypertarget{ein-sehr-unwahrscheinlicher-tag}{% \section{1. Ein sehr unwahrscheinlicher Tag}\label{ein-sehr-unwahrscheinlicher-tag}}

\textbf{Kapitel 1: Ein sehr unwahrscheinlicher Tag}

Haftungsausschluss: J. K. Rowling gehört Harry Potter und niemandem gehören die Methoden der Rationalität.

Man ist weithin der Meinung dieses Fanfic käme etwa um Kapitel 5 herum richtig in die Gänge. Wenn ihr es nach Kapitel 10 immer noch nicht mögt, gebt auf.

Rezensionen freuen mich. Ihr könnt Rezensionen für jedes Kapitel hinterlassen, keine Registrierung nötig und es ist nicht notwendig, alle Kapitel gelesen zu haben, bevor ihr anfangt, Kapitel zu rezensieren—aber hinterlasst bitte maximal \emph{eine} Rezension pro Kapitel.

Dieses Fanfic beschränkt sich nicht streng auf nur einen Ausgangspunkt—es existiert ein hauptsächlicher Ausgangspunkt an einem Zeitpunkt in der Vergangenheit, aber auch andere Abweichungen. Der beste Begriff, den ich für dieses Fanfic gehört habe, ist „Parallel-Universum“.

Der Text enthält viele Hinweise: Offensichtliche Hinweise, nicht-so-offensichtliche Hinweise, wahrhaftig undeutliche Andeutungen, deren erfolgreiche Entschlüsselung durch manche Leser mich schockiert hat und massenhaft Belege für aller Augen. Dies ist eine rationalistische Geschichte; ihre Mysterien sind lösbar und sollen gelöst werden.

Die Abfolge der Geschichte ist die einer Serie, z.B. einer TV-Serie, die für eine vorher bestimmte Anzahl von Staffeln läuft, deren Episoden in sich geschlossen sind, aber mit einem übergeordneten Handlungsbogen, der zu einem finalen Abschluss hinführt.

Alle erwähnte Wissenschaft ist echte Wissenschaft. Aber bitte behaltet im Hinterkopf, dass außerhalb des Reiches der Wissenschaft, die Ansichten der Charaktere nicht diejenigen des Autors sein können. Nicht alles, was der Protagonist tut, ist eine Lektion in Weisheit und Ratschläge von dunkleren Charakteren können nicht vertrauenswürdig oder gefährlich zweischneidig sein.*

\later

\emph{Im Mondlicht glitzert ein winziges Stückchen Silber, ein Bruchteil einer Linie…}

\emph{(schwarze Umhänge fallen)}

\emph{… Blut fließt in Strömen und jemand schreit ein Wort.}

\later

Jeder Zoll Platz an der Wand wird von einem Bücherschrank verdeckt. Jedes Bücherregal hat sechs Fächer und geht fast bis zur Decke. Manche Bücherschränke sind bis zum Anschlag mit gebundenen Büchern vollgestopft: Wissenschaft, Mathematik, Geschichte und alles weitere. Andere Regale enthalten zwei Reihen Science-Fiction-Taschenbücher, die hintere Reihe auf alte Taschentuchpackungen oder Holzlatten gestellt, so dass man die hintere Buchreihe über die Bücher davor sehen kann. Und es ist immer noch nicht genug. Bücher überschwemmen die Tische und die Sofas und bilden kleine Stapel unter den Fenstern.

Das ist das Wohnzimmer des Hauses, welches von dem angesehenen Professor Michael Verres-Evans, seiner Frau Petunia Evans-Verres und ihrem Adoptivsohn Harry James Potter-Evans-Verres bewohnt wird.

Es liegt ein Brief auf dem Wohnzimmertisch und ein ungestempelter Umschlag aus gelblichem Pergament, in emerald-grüner Tinte adressiert an \emph{Mr~H.~Potter}.

Der Professor und seine Frau sprechen in scharfem Ton miteinander, aber sie schreien nicht. Der Professor sieht Schreien als unzivilisiert an.

„Du machst Witze,“ sagte Michael zu Petunia. Sein Tonfall wies darauf hin, dass er sehr fürchtete, sie könne es ernst meinen.

„Meine Schwester war eine Hexe,“ wiederholte Petunia. Sie sah ängstlich aus, verteidigte aber ihren Standpunkt. „Ihr Ehemann war ein Zauberer.“

„Das ist absurd!“ sagte Michael scharf. „Sie waren bei unserer Hochzeit - sie waren an Weihnachten zu Besuch—“

„Ich sagte ihnen, du dürftest es nicht wissen,“ flüsterte Petunia. „Aber es ist wahr. Ich habe Dinge gesehen—“

Der Professor rollte mit den Augen. „Liebling, ich verstehe, dass du mit der skeptischen Literatur nicht vertraut bist. Dir ist vielleicht nicht klar, wie einfach es für einen geübten Zauberkünstler ist, das Unmögliche zu vorzugaukeln. Erinnerst du dich, wie ich Harry beigebracht habe, Löffel zu verbiegen? Falls es schien, als konnten sie immer erahnen, was du dachtest, das nennt sich Cold Reading—“

„Es war kein Löffel verbiegen—“

„Was war es dann?“

Petunia biss sich auf die Unterlippe. „Ich kann's dir nicht einfach erzählen. Du wirst denken, ich bin—“ Sie schluckte. „Hör zu. Michael. Ich war nicht - immer so—“ Sie deutete auf sich selbst, als wolle sie auf ihre elegante Figur hinweisen. „Lily hat das getan. Weil ich - weil ich sie \emph{angefleht} habe. Jahrelang habe ich sie angefleht. Lily war \emph{immer} schon hübscher als ich und ich war… gemein zu ihr deswegen und dann bekam sie \emph{Magie}, kannst du dir vorstellen wie ich mich fühlte? Und ich \emph{flehte} sie an, etwas von dieser Magie bei mir anzuwenden, damit ich auch hübsch sein könnte, auch wenn ich ihre Magie nicht haben könnte, zumindest hübsch könnte ich sein.“

Tränen sammelten sich in Petunias Augen.

„Und Lily weigerte sich und erfand die lächerlichsten Ausreden, als ob die Welt untergehen würde, wenn sie nett zu ihrer Schwester wäre oder ein Zentaur ihr gesagt hätte, nicht zu - die lächerlichsten Sachen und ich hasste sie dafür. Und als ich gerade die Universität absolviert hatte, ging ich mit diesem Jungen aus, Vernon Dursley, er war fett und der einzige Junge, der mit mir redete. Und er erzählte mir, dass er Kinder wollte und er seinen ersten Sohn Dudley nennen würde. Und ich dachte mir, \emph{was für ein Elternteil nennt sein Kind Dudley Dursley?} Es war, als ob sich mein ganzes zukünftiges Leben vor meinen Augen ausbreitete und ich hielt es nicht aus. Und ich schrieb meiner Schwester und erklärte ihr, dass wenn sie mir nicht helfen würde, ich lieber einfach—“

Petunia stoppte.

„Jedenfalls,“ sagte Petunia mit kleiner Stimme, „gab sie nach. Sie erzählte mir, dass es gefährlich war und ich sagte, es interessiere mich nicht mehr und ich trank diesen Zaubertrank und ich war wochenlang krank, aber als es mir besser ging, wurde meine Haut reiner und ich wurde schließlich kurviger und… ich war wunderschön, die Leute waren \emph{nett} zu mir,“ ihre Stimme brach, „und danach konnte ich meine Schwester nicht mehr hassen, besonders als ich herausfand, was ihre Magie ihr am Ende bescherte—“

„Liebling,“ sagte Michael freundlich, „du wurdest krank, du hast etwas zugenommen, während du im Bett lagst und deine Haut ist von allein reiner geworden. Oder, dass du krank warst, brachte dich dazu deine Ernährung umzustellen—“

„Sie war eine Hexe,“ wiederholte Petunia. „Ich habe es gesehen.“

„Petunia,“ sagte Michael. Die Verärgerung schlich sich in seine Stimme. „Du \emph{weißt}, das kann nicht wahr sein. Muss ich wirklich erklären wieso?“

Petunia rang mit den Händen. Sie schien den Tränen nahe. „Mein Lieber, ich weiß, ich kann Diskussionen mit dir nicht gewinnen, aber bitte, du musst mir dabei vertrauen—“

„\emph{Dad! Mum!}\,“

Die beiden hielten inne und sahen Harry an, als hätten sie vergessen, dass eine dritte Person im Raum war.

Harry atmete tief ein. „Mum, \emph{deine} Eltern hatten keine Magie, oder?“

„Nein,“ sagte Petunia und wirkte verwirrt.

„Dann wusste niemand in deiner Familie von Magie, als Lily ihren Brief bekam. Wie wurden \emph{sie} überzeugt?“

„Ah…“ sagte Petunia. „Sie schickten nicht nur einen Brief. Sie schickten einen Professor aus Hogwarts. Er—“ Petunias Augen schnellten zu Michael. „Er hat uns etwas Magie gezeigt.“

„Dann müsst ihr nicht darüber streiten,“ sagte Harry fest. Gegen alle Erfahrung hoffend, dass sie diesmal, nur dieses eine mal, auf ihn hören würden. „Wenn es wahr ist, können wir einfach einen Hogwarts-Professor herholen und die Magie selbst sehen und Dad wird zugeben, dass es wahr ist. Und wenn nicht, wird Mum zugeben, dass es falsch ist. Dafür gibt es die experimentelle Methode, so dass wir Dinge nicht nur durch diskutieren klären müssen.“

Der Professor drehte sich um und sah auf ihn herab, geringschätzig wie üblich. „Oh, komm schon, Harry. Wirklich, \emph{Magie?} Ich dachte, \emph{du} wüsstest es besser, als das ernst zu nehmen, Sohn, auch wenn du erst zehn bist. Magie ist so ziemlich die unwissenschaftlichste Sache, die es gibt!“

Harry blickte verbittert drein Er wurde gut behandelt, wahrscheinlich besser, als die meisten biologischen Väter ihre eigenen Kinder behandelten. Harry wurde auf die besten Grundschulen geschickt—und als das nicht funktionierte, wurde ihm Tutoren aus der endlosen Schar hungernder Studenten zur Verfügung gestellt. Immer wurde Harry ermutigt, alles zu lernen, was sein Interesse weckte, ihm wurden alle Bücher gekauft, die seine Vortellungskraft gefangen nahmen und er wurde bei jedwedem Mathematik- oder Wissenschaftswettberb unterstützt, an dem er teilnahm. Ihm wurde jeder vernünftige Wunsch gewährt, außer, vielleicht, das winzigste Fetzchen Respekt. Von einem Doktor, der Biochemie in Oxford lehrte, konnte schwerlich erwartet werden, auf den Rat eines kleinen Jungen zu hören. Man würde, natürlich, zuhören, um Interesse zu zeigen; das ist es, was ein gutes Elternteil tun würde und deshalb, wenn man sich selbst als ein gutes Elternteil ansah, würde man es tun. Aber einen Zehnjährigen \emph{ernst nehmen}? Wohl kaum.

Manchmal wollte Harry seinen Vater anschreien.

„Mum,“ sagte Harry, „wenn du diese Diskussion mit Dad gewinnen willst, sieh in Kapitel zwei des ersten Buchs der Feynman-Vorlesungen über Physik nach. Dort gibt es ein Zitat darüber, dass Philosophen viel darüber sagen, was Wissenschaft unbedingt ausmacht und es ist alles falsch, weil die einzige Regel in der Wissenschaft ist, dass die letzte Entscheidung durch Beobachtung gefällt wird - dass du nur die Welt ansehen und berichten musst, was du siehst. Ähm… spontan fällt mir nicht ein, wo man etwas darüber findet, dass es ein Ideal der Wissenschaft ist, Dinge durch Experimente anstatt Diskussionen beizulegen—“

Seine Mutter sah zu ihm herab und lächelte. „Danke, Harry. Aber—“ sie hob den Kopf wieder zu ihrem Ehemann. „Ich will keine Diskussion mit deinem Vater gewinnen. Ich möchte, dass mein Mann seiner, seiner liebenden Ehefrau zuhört und ihr nur dieses eine mal vertraut—“

Harry schloss kurz die Augen. \emph{Hoffnungslos}. Seine beiden Eltern waren einfach hoffnungslos.

Jetzt würden seine Eltern sich wieder auf eine \emph{dieser} Streitereien einlassen; bei der seine Mutter versuchte, seinen Vater dazu zu bringen, sich schuldig zu fühlen und sein Vater versuchte, seine Mutter dazu zu bringen, sich dumm zu fühlen.

„Ich werde in mein Zimmer gehen,“ verkündete Harry. Seine Stimme zitterte ein wenig. „Bitte versucht euch nicht zu sehr darüber zu streiten, Mum, Dad, wir werden früh genug wissen, wie es ausgeht, richtig?“

„Natürlich, Harry,“ sagte sein Vater und seine Mutter gab ihm einen beruhigenden Kuss und dann stritten sie sich weiter, während Harry die Stufen zu seinem Schlafzimmer hinaufstieg.

Er schloss die Tür hinter sich und versuchte nachzudenken.

Das Komische war, dass er Dad hätte zustimmen \emph{sollen}. Niemand hatte jemals irgendwelche Belege für Magie gesehen und gemäß Mum gab es da draußen eine ganze magische Welt. Wie könnte irgendjemand so etwas geheim halten? Mehr Magie? Das schien eine eher verdächtige Ausrede zu sein.

Es hätte ein klarer Fall sein sollen, dass Mum scherzte, log oder verrückt war, in aufsteigender Reihenfolge der Furchtbarkeit. Wenn Mum den Brief selbst geschickt hätte, würde das erklären, wie er im Briefkasten ohne einen Stempel auftauchte. Ein bisschen Verrücktheit war viel, viel weniger unwahrscheinlich, als dass das Universum tatsächlich so funktionierte.

Außer, dass irgendein Teil von Harry zutiefst davon überzeugt war, dass Magie real war, seit dem Moment als er den vermeintlichen Brief von der Hogwarts-Schule für Hexerei und Zauberei gesehen hatte.

Harry rieb sich die Stirn und zog eine Grimasse. \emph{Glaube nicht alles, was du denkst} hatte eines seiner Bücher gesagt.

Aber diese bizarre Gewissheit… Harry ertappte sich dabei, einfach zu erwarten, dass, ja, ein Hogwarts-Professor auftauchen und einen Zauberstab schwenken und Magie dort herauskommen würde. Die seltsame Gewissheit machte keinen Versuch sich gegen eine Falsifikation zu verteidigen—erfand keine Ausreden im Voraus dafür, warum kein Professor da sein oder der Professor nur in der Lage sein würde, Löffel zu verbiegen.

\emph{Woher kommst du, seltsame kleine Vorhersage?} richtete Harry den Gedanken an sein Gehirn. \emph{Warum glaube ich, was ich glaube?}

Üblicherweise war Harry ziemlich gut darin, diese Frage zu beantworten, aber in diesem speziellen Fall hatte er keinen \emph{Schimmer}, was sein Gehirn dachte.

Harry zuckte im Geiste mit den Schultern. Eine flache Metallplatte an einer Tür erfordert es, zu drücken und ein Griff an einer Tür erfordert es, zu ziehen und das, was es mit einer überprüfbaren Hypothese zu tun gilt, ist loszuziehen und sie zu überprüfen.

Er nahm ein Blatt liniertes Papier von seinem Schreibtisch und fing an, zu schreiben.

\emph{Sehr geehrte Stellvertretende Schulleiterin}

Harry hielt inne, überlegend; verwarf dann das Papier für ein anderes, einen weiteren Millimeter Graphit von seinem mechanischen Bleistift abklopfend. Das verlangte nach sorgfältiger Schönschrift.

\emph{Sehr geehrte Stellvertretende Schulleiterin Minerva McGonagall,}

\emph{oder Wer auch immer, den es betreffen mag:}

\emph{Ich habe vor kurzem Ihren Aufnahme-Brief für Hogwarts, adressiert an} \emph{Mr~H.~Potter, erhalten. Ihnen ist möglicherweise nicht bewusst, dass meine biologischen Eltern, James Potter und Lily Potter (ehemals Evans), tot sind. Ich wurde adoptiert von Lily's Schwester, Petunia Evans-Verres und ihrem Ehemann, Michael Verres-Evans.}

\emph{Ich bin überaus interessiert daran, Hogwarts zu besuchen, vorausgesetzt, ein solcher Ort existiere tatsächlich. Nur meine Mutter Petunia sagt, dass sie von Magie weiß, kann sie aber nicht selbst anwenden. Mein Vater ist äußerst skeptisch. Ich selbst bin unsicher. Auch weiß ich nicht, wo ich irgendwelche der Bücher oder Ausrüstungsgegenstände, die in Ihrem Aufnahme-Brief aufgezählt werden, beschaffen sollte.}

\emph{Mutter erwähnte, dass Sie einen Hogwarts-Repräsentanten zu Lily Potter (damals Lily Evans) schickten, um ihrer Familie zu demonstrieren, dass Magie real war und, wie ich annehme, Lily bei der Beschaffung ihrer Schulsachen zu helfen. Wenn Sie das auch für meine Familie tun könnten, wäre das überaus hilfreich.}

\emph{Mit besten Grüßen}

\emph{Harry James Potter-Evans-Verres.}

Harry fügte ihre aktuelle Adresse hinzu, faltete dann den Brief und steckte ihn in einen Umschlag, den er an Hogwarts adressierte. Weitere Überlegung bewog ihn dazu, eine Kerze zu beschaffen und Wachs auf die Lasche des Umschlags zu tropfen, in welches er mit der Spitze eines Taschenmessers seine Initialen H.J.P.E.V. eindrückte. Wenn er schon in den Wahnsinn abgleiten sollte, würde er es mit Stil tun.

Dann öffnete er seine Tür und ging zurück nach unten. Sein Vater saß im Wohnzimmer und las ein Buch über höhere Mathematik, um zu zeigen, wie schlau er war und seine Mutter war in der Küche dabei, eines der Lieblingsgerichte seines Vaters zuzubereiten, um zu zeigen, wie liebevoll sie war. Es sah nicht so aus, als würden sie überhaupt miteinander reden. So beängstigend Streitereien auch waren, \emph{nicht zu streiten} war irgendwie noch schlimmer.

„Mum,“ sagte Harry in die zermürbende Stille, „ich werde die Hypothese überprüfen. Wie schicke ich deiner Theorie nach eine Eule an Hogwarts?“

Seine Mutter wandte sich von der Küchenspüle ab, um ihn schockiert anzustarren. „Ich - ich weiß nicht, ich denke, du musst einfach eine magische Eule besitzen.“

Das hätte höchst verdächtig klingen sollen, \emph{oh, dann gibt es also keinen} \emph{Weg, deine Hypothese zu überprüfen,} aber die sonderbare Gewissheit in Harry schien willens, sich noch etwas weiter hervor zu wagen.

„Nun, der Brief ist irgendwie hierher gekommen,“ sagte Harry, „also werde ich einfach draußen damit herumwedeln und rufen 'Brief für Hogwarts' und sehen, ob eine Eule ihn aufschnappt. Dad, willst du mitkommen und zusehen?“

Sein Vater schüttelte unmerklich den Kopf und las weiter. \emph{Natürlich}, dachte Harry bei sich selbst. Magie war eine schändliche Sache, an die nur dumme Leute glaubten; wenn sein Vater so weit ginge, die Hypothese zu \emph{überprüfen} oder auch nur dabei zuzusehen, würde sich das anfühlen, als würde er sich damit \emph{gemein machen}…

Erst als Harry zur Hintertür in den Garten hinausstapfte, wurde ihm klar, dass wenn \emph{tatsächlich} eine Eule herunterkommen und den Brief aufschnappen würde, er einige Probleme haben würde, Dad davon zu erzählen.

\emph{Aber - nun - das kann nicht} wirklich \emph{passieren, oder? Egal, was mein Gehirn zu glauben scheint. Wenn tatsächlich eine Eule herunterkommt und diesen Umschlag packt, habe ich sehr viel wichtigere Sorgen, als was Dad darüber denkt.}

Harry atmete tief ein und hob den Umschlag hoch in die Luft.

Er schluckte.

Laut \emph{Brief für Hogwarts!} auszurufen, während man mitten im eigenen Garten einen Umschlag hoch in die Luft hält, war… tatsächlich ziemlich peinlich, jetzt wo er darüber nachdachte.

\emph{Nein. Ich bin besser als Dad. Ich werde die wissenschaftliche Methode benutzen, auch wenn ich mir dadurch dämlich vorkomme.}

„Brief—“ sagte Harry, aber es kam tatsächlich mehr als ein leises Krächzen heraus.

Harry stählte seinen Willen und rief in den leeren Himmel, „\emph{Brief für Hogwarts! Kann ich eine Eule bekommen?}“

„Harry?“ fragte die amüsierte Stimme einer Frau, eine der Nachbarinnen.

Harry zog seine Hand ein, als habe sie Feuer gefangen und er versteckte den Umschlag hinter seinem Rücken als wäre es Drogengeld. Sein ganzes Gesicht glühte heiß vor Scham.

Das Gesicht einer alten Frau blickte über den Nachbarzaun, graue Haare lugten unter ihrem Haarnetz hervor. Mrs~Figg, die gelegentlich seine Babysitterin war. „Was machst du, Harry?“

„Gar nichts,“ sagte Harry mit erstickter Stimme. „Überprüfe nur - eine wirklich dumme Theorie—“

„Hast du deinen Aufnahme-Brief von Hogwarts bekommen?“

Harry erstarrte.

„Ja,“ sagten Harrys Lippen nach einer kleinen Weile. „Ich habe einen Brief von Hogwarts bekommen. Sie sagen, sie wollen meine Eule bis zum 31. Juli, aber—“

„Aber du \emph{hast} keine Eule. Armes Schätzchen! Ich kann mir nicht vorstellen, \emph{was} jemand sich dabei gedacht hat, dir nur den Standard-Brief zu schicken.“

Ein schrumpliger Arm wurde über den Zaun gestreckt und öffnete erwartungsvoll eine Hand. An diesem Punkt kaum noch nachdenkend, übergab Harry seinen Umschlag.

„Überlass das einfach mir, Schätzchen,“ sagte Mrs~Figg, „und im Handumdrehen kommt jemand vorbei.“

Und ihr Gesicht verschwand vom Zaun.

Es war lange still im Garten.

Dann sagte die Stimme eines Jungen, still und leise, „Was.“

* Die geneigten Leser mögen hier bitte beachten, dass die Anmerkungen an Kapitelanfängen oder -enden üblicherweise Anmerkungen des ursprünglichen Autors sind und keine Anmerkungen des Übersetzers, insbesondere wenn er euch auffordert, ihm zu schreiben oder ähnliches. Meine Anmerkungen beschränken sich im Allgemeinen auf diese Fußnoten. Bitte beachtet weiterhin, dass die Anmerkungen des Autors nicht notwendigerweise vollständig sind, beispielsweise wenn die Übersetzung einzelner Abschnitte im Deutschen oder für die deutsche Version der Geschichte keinen Sinn mehr ergibt. Informationen zum Autor Eliezer Yudkowsky alias LessWrong und der originalen Geschichte „Harry Potter and the Methods of Rationality“ in englischer Sprache findet ihr wahlweise auf FanFiction.net (\url{https://www.fanfiction.net/s/5782108/1/Harry-Potter-and-the-Methods-of-Rationality}) oder auf der autorisierten Mirror-Seite HPMOR (\url{http://hpmor.com}). Die Erlaubnis zur Verwendung seiner „Charaktere, Situationen und Welten“ hat der Autor auf letzterer freundlicherweise erteilt.

