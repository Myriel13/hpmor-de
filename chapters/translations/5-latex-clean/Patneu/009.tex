

\hypertarget{selbstbewusstsein-teil-1}{% \section{9. Selbstbewusstsein, Teil 1}\label{selbstbewusstsein-teil-1}}

\textbf{Kapitel 9: Selbstbewusstsein, Teil 1}

All your base are belong to J. K. Rowling.

\later

1.000 REZENSIONEN IN 26 TAGEN, WUUHUU, SUPA LEISTUNG! 30 TAGE 1.189-REZENSIONEN-KOMBO UND STEIGEND! YEAH! IHR LEUTE SEID DIE BESTEN! DAS IST SPARTAAAAA!

Ähem.

Die Quarks der dritten Generation wurden auch "Truth" und "Beauty" genannt, bevor "Top" und "Bottom" sich durchgesetzt haben; mein Geburtsdatum stimmt in etwa mit dem von Hermine überein und als ich elf war, habe ich "Truth" und "Beauty" verwendet.

Als Teil 1 dieses Kapitels zum ersten mal gepostet wurde, sagte ich, dass, wenn irgendjemand vor dem nächsten Update erraten würde, wovon im letzten Satz die Rede ist, ich ihm den ganzen Rest der Handlung verraten würde.*

\later

\emph{Man wusste nie, welches winzige Ereignis den eigenen Masterplan vom Kurs abbringen würde.}

\later

"Abbott, Hannah!"

Pause.

"HUFFLEPUFF!"

"Bones, Susan!"

Pause.

"HUFFLEPUFF!"

"Boot, Terry!"

Pause.

"RAVENCLAW!"

Harry blickte kurz zu seinem neuen Hausgenossen hinüber, mehr um einen kurzen Blick auf das Gesicht zu erhaschen, als wegen irgendwas sonst. Er versuchte immer noch, sich nach seiner Begegnung mit den Geistern wieder unter Kontrolle zu kriegen. Das Traurige, das wirklich Traurige, das wirklich, wahrhaftig Traurige war, er \emph{schien} sich wieder unter Kontrolle zu bekommen. Es erschien unpassend. Als hätte er wenigstens einen Tag brauchen sollen. Vielleicht ein ganzes Leben. Vielleicht einfach für immer.

"Corner, Michael!"

Eine lange Pause.

"RAVENCLAW!"

An dem Pult vor dem großen Lehertisch stand Professor McGonagall, elegant aussehend und scharf umherblickend, während sie einen Namen nach dem anderen ausrief, wobei sie nur Hermine und einige andere angelächelt hatte. Hinter ihr, im größten Stuhl an der Tafel -- tatsächlich mehr ein goldener Thron -- saß ein schrumpliger und bebrillter Alter mit einem silbrig-weißen Bart, der aussah, als würde er fast bis zum Fußboden reichen, wenn er zu sehen wäre, der die Auswahlzeremonie gütigem Gesichtsausdruck beobachtete; so stereotyp, wie ein Weiser Alter Mann nur sein konnte, ohne orientalisch zu sein. (Obwohl Harry von seiner ersten Begegnung mit Professor McGonagall gelernt hatte, stereotypen Erscheinungen zu misstrauen, als er gedacht hatte, sie müsse gackern.) Der uralte Zauberer hatte jedem ausgewählten Schüler applaudiert, mit einem unbeirrbaren Lächeln, das irgendwie für jeden wieder aufs neue begeistert schien.

Zur Linken des goldenen Thrones saß ein Mann mit stechenden Augen und mürrischem Gesicht, der niemandem applaudierte und der es irgendwie schaffte, jedes mal genau auf Harry zurückzublicken, wenn Harry ihn ansah. Weiter zur Linken, der blassgesichtige Mann, den Harry im Tropfenden Kessel gesehen hatte, dessen Augen umherhuschten, wie vor Panik ob der umgebenden Menge und der gelegentlich in seinem Sitz zu rucken und zu zucken schien; aus irgendeinem Grund ertappte Harry sich dabei, ihn fortwährend anzustarren. Zur Linken dieses Mannes eine Reihe von drei älteren Hexen, die an den Schülern nicht besonders interessiert zu sein schienen. Dann zur Rechten des hohen goldenen Stuhles, eine rundgesichtige Hexe mittleren Alters mit einem gelben Hut, die jedem applaudierte außer den Slytherins. Ein winziger Mann, der auf seinem Stuhl stand, mit buschigem weißem Bart, der jedem Schüler applaudiert hatte, aber nur auf die Ravenclaws herablächelte. Und an der äußersten Rechten, den selben Platz wie drei geringere Wesen beanspruchend, die gewaltige Gestalt, die sie alle beim Ausstieg aus dem Zug begrüßt hatte und sich Hagrid, Hüter der Schlüssel und Ländereien, nannte.

"Ist der Mann, der auf seinem Stuhl steht der Hauslehrer von Ravenclaw?" flüsterte Harry Hermine zu.

Zum ersten mal antwortete Hermine nicht sofort; sie trat permanent von einem Fuß auf den anderen, starrte den Sprechenden Hut an und zappelte so heftig, dass Harry dachte, ihre Füße könnten vom Boden abheben.

"Ja, ist er," sagte eine der Vertrauensschülerinnen, die sie begleitet hatten, eine junge Frau, die das Blau von Ravenclaw trug. Miss Clearwater, wenn Harry sich richtig erinnerte. Ihre Stimme war leise, aber vermittelte einen Hauch von stolz. "Das ist der Zauberkunst-Professor von Hogwarts, Filius Flitwick, der gelehrteste lebende Meister der Zauberkunst und ein ehemaliger Duellier-Meister -"

"Warum ist er so \emph{klein?}" zischte ein Schüler, an dessen Namen Harry sich nicht erinnerte. "Ist er ein \emph{Halbblut?}"

Ein kühler Blick von der jungen Vertrauensschülerin. "Der Professor hat in der Tat Kobold-Vorfahren -"

"Was?" sagte Harry unwillkürlich, woraufhin Hermine und vier andere Schüler ihn zur Ruhe riefen.

Jetzt wurde Harry ein überraschend einschüchternden zornigen Blick von der Ravenclaw-Vertrauensschülerin zuteil.

"Ich meine -" flüsterte Harry. "Nicht, dass ich ein \emph{Problem} damit hätte - es ist nur - ich meine - wie ist das \emph{möglich?} Man kann nicht einfach zwei verschieden Spezies zusammenmixen und bekommt lebensfähigen Nachwuchs! Es sollte die genetischen Anweisungen für alle Organe durcheinanderbringen, die sich bei den beiden Spezies unterscheiden -- es wäre, als baute man eine," sie hatten keine Autos, also konnte er keinen Vergleich über vermischte Motor-Blaupausen verwenden, "eine Mischung aus Kutsche und Boot oder sowas…"

Die Ravenclaw-Vetrauensschülerin sah Harry immer noch streng an. "Warum \emph{könnte} man denn keine Mischung aus Kutsche und Boot bauen?"

"\emph{Sscchht!}" zischte ein anderer Vertrauensschüler, obwohl die Ravenclaw-Hexe weiterhin leise gesprochen hatte.

"Ich meine -" sagte Harry noch leiser und überlegte, wie er fragen sollte, ob Kobolde sich aus Menschen entwickelt hatten oder aus einem gemeinsamen Vorfahren der Menschen, wie \emph{Homo erectus} oder ob Kobolde irgendwie aus Menschen \emph{gemacht} worden waren -- ob sie, beispielsweise, immer noch genetisch gesehen Menschen waren, unter einer vererblichen Verzauberung, deren magischer Effekt verwässert wurde, wenn nur ein Elternteil ein 'Kobold' war, was erklären würde, wie Kreuzungen möglich waren und in welchem Fall Kobolde \emph{keine} unglaublich wertvolle Datenquelle für die Entwicklung von Intelligenz in anderen Spezies als dem \emph{Homo sapiens} wären -- jetzt wo Harry darüber nachdachte, hatten die Kobolde in Gringotts nicht besonders den Eindruck wirklich fremdartiger, nicht-menschlicher Intelligenzen gemacht, nicht wie die Dirdir** oder Puppenspieler*** -- "ich meine, woher \emph{kommen} Kobolde eigentlich?"

"Litauen," flüsterte Hermine abwesend, ihre Augen fest auf den Sprechenden Hut gerichtet.

Jetzt bekam ein Hermine ein Lächeln von der Vertrauensschülerin.

"Nicht so wichtig," flüsterte Harry.

Am Pult rief Professor McGonagall aus, "Goldstein, Anthony!"

"RAVENCLAW!"

Neben Harry, wippte Hermine so sehr auf den Zehen, dass ihre Füße tatsächlich bei jedem Wippen vom Boden abhoben.

"Goyle, Gregory!"

Es gab einen langen, angespannten Moment der Stille unter dem Hut. Fast eine Minute.

"SLYTHERIN!"

"Granger, Hermine!"

Hermine stürmte los und rannte mit Volldampf auf den Sprechenden Hut zu, nahm ihn auf und zwängte sich die flickenübersäte alte Klamotte heftig über den Kopf, was Harry zusammenzucken ließ. Hermine hatte \emph{ihm} alles über den Sprechenden Hut erklärt, aber sie \emph{behandelte} ihn sicher nicht wie ein unersetzliches, unerlässlich wichtiges, 800-Jahre-altes Artefakt vergessener Magie, das kurz davor war komplizierte Telepathie bei ihrem Verstand anzuwenden und in nicht besonders gutem physischem Zustand zu sein schien.

"RAVENCLAW!"

Und so viel zu voreiligen Schlüssen. Harry verstand nicht, warum Hermine deshalb so nervös gewesen war. In welchem verrückten Paralleluniversum würde dieses Mädchen \emph{nicht} nach Ravenclaw sortiert? Wenn Hermine Granger nicht nach Ravenclaw ging, gab es keinen Grund, warum ein Haus Ravenclaw existieren sollte.

Hermine kam am Ravenclaw-Tisch an und erntete einen pflichschuldigen Applaus; Harry fragte sich, ob der Applaus lauter oder leiser gewesen wäre, wenn sie auch nur die geringste Vorstellung gehabt hätten, welche Konkurrenz sie gerade an ihrem Tisch willkommen geheißen hatten. Harry kannte Pi bis zu 3,141592, weil eine Genauigkeit von einem Millionstel für die meisten praktischen Zwecke ausreichte. Hermine kannte Pi bis zur hundertsten Stelle, weil so viele Stellen auf der Rückseite ihres Mathematik-Lehrbuchs abgebildet gewesen waren.

Neville Longbottom ging nach Hufflepuff, was Harry froh war, zu sehen. Wenn es in diesem Haus tatsächlich die Loyalität und Kameradschaft gab, für die es beispielhaft sein sollte, würde Neville ein ganzes Haus voller verlässlicher Freunde sehr gut tun. Clevere Kinder in Ravenclaw, böse Kinder in Slytherin, Möchtegern-Helden in Gryffindor und alle, die die echte Arbeit erledigen in Hufflepuff.

(Obwohl Harry Recht gehabt \emph{hatte,} zuerst einen Ravenclaw-Vertrauensschüler anzusprechen. Die junge Frau hatte nicht einmal von ihrer Lektüre hochgesehen oder Harry erkannt, sie stieß nur ihren Zauberstab in Nevilles Richtung und murmelte etwas. Woraufhin Nevilles Gesicht einen benommenen Ausdruck annahm und er davon lief zum fünften Waggon von vorn, viertes Abteil auf der linken Seite, welches tatsächlich seine Kröte enthalten hatte.)

"Malfoy, Draco!" ging nach Slytherin und Harry stieß einen kleinen Seufzer der Erleichterung aus. Es \emph{schien} eine sichere Sache zu sein, doch man konnte nie wissen, welch winziges Ereignis den eigenen Masterplan vereiteln mochte.

Professor McGonagall rief "Perks, Sally-Anne!" und aus der Menge der Kinder löste sich ein blasses, spindeldürres Mädchen, das seltsam ätherisch aussah -- als könne sie in dem Moment, in dem man aufhörte sie anzusehen, auf mysteriöse Weise verschwinden und niemand würde sie jemals wiedersehen oder sich auch nur an sie erinnern.

Und dann (mit einer Spur von Beklommenheit, die sie so fest von ihrer Stimme und ihrem Gesicht fern hielt, dass man sie tatsächlich sehr gut kennen musste, um sie zu bemerken) atmete Minerva McGonagall tief ein und rief aus, "Potter, Harry!"

Es herrschte plötzliche Stille in der Halle.

Alle Unterhaltungen endeten.

Aller Augen drehten sich, zum Starren.

Zum ersten mal in seinem ganzen Leben, fühlte Harry sich, als bekäme er die Gelegenheit, Lampenfieber zu erleben.

Harry unterdrückte dieses Gefühl sofort. Ganze Räume voller Menschen, die ihn anstarrten, waren etwas, woran er sich gewöhnen musste, wenn er im magischen Britannien leben oder, um genau zu sein, auch nur irgendetwas anderes interessantes mit seinem Leben anfangen wollte. Ein zuversichtliches und falsches Lächeln auf dem Gesicht tragend, setzte er einen Fuß, um voranzuschreiten -

"Harry Potter!" rief die Stimme von entweder Fred oder George Weasley und dann rief der andere Weasley-Zwilling "Harry Potter!" und einen Moment später der ganze Gryffindor-Tisch und bald danach hatten ein guter Teil von Ravenclaw und Hufflepuff den Ruf aufgenommen.

\emph{"Harry Potter! Harry Potter! Harry Potter!}"

Und Harry Potter ging nach vorn. Viel zu langsam, wurde ihm bewusst, als er angefangen hatte, aber da war es schon zu spät, um seinen Gang noch anzupassen, ohne dass es peinlich aussah.

\later

"\emph{Harry Potter! Harry Potter! HARRY POTTER!}"

Sich nur zu sehr bewusst, was sie sehen würde, drehte sich Minerva McGonagall um, um hinter sich zum Rest des Lehrertisches zu sehen.

Trelawney fächelte sich hektisch Luft zu, Filius sah neugierig zu, Hagrid klatschte mit, Sprout blickte streng, Vector und Sinistra verwirrt und Quirrell starrte ausdruckslos ins Nichts. Albus lächelte wohlwollend. Und Severus Snape umklammerte mit weißen Knöcheln seinen Weinkelch, so stark, dass das Silber sich langsam zu verformen begann.

Mit einem breiten Grinsen, sich mal zur einen und dann zur anderen Seite verbeugend, während er zwischen den vier Haustischen entlang ging, schritt Harry mit erhaben gemessenem Tempo voran, ein Prinz, der seine Burg in Besitz nahm.

"\emph{Rette uns vor noch ein paar Dunklen Lords!}" rief einer der Weasley-Zwillinge und dann der andere, "\emph{Besonders wenn sie Professoren sind!}" zum allgemeinen Gelächter aller Tische, außer dem der Slytherins.

Minervas Lippen wurden zu einer weißen Linie. Sie würde mit den Weasley-Gräueln hinterher noch ein ernstes Wort über diesen letzten Teil sprechen, wenn sie dachten, sie wäre machtlos, weil es erst der erste Schultag war und Gryffindor noch keine Punkte zum Absehen hatte. Wenn sie sich nicht um's Nachsitzen scherten, würde ihr etwas anderes einfallen.

Dann sah sie mit einem plötzlichen entsetzten Keuchen in Severus Richtung, ihm war doch \emph{sicher} klar, dass der Potter-Junge keine Ahnung haben konnte, von wem dort die Rede war -

Severus Gesicht war über die Wut hinaus zu einer Art von wohliger Gleichgültigkeit übergegangen. Ein schwaches Lächeln umspielte seine Lippen. Er blickte in die Richtung von Harry Potter, nicht des Gryffindor-Tisches und seine Hände hielten die zerknüllten Überreste dessen, was einmal ein Weinkelch gewesen war.

\later

Harry Potter schritt mit einem festgefrorenen Lächeln voran, fühlte sich innerlich warm und gleichzeitig furchtbar.

Sie jubelten ihm zu für einen Job, den er im Alter von einem Jahr erledigt hatte. Einen Job, den er nicht wirklich zu Ende gebracht hatte. Irgendwo, irgendwie, war der Dunkle Lord immer noch am Leben. Hätten sie ganz so laut gejubelt, wenn sie das wüssten?

Aber die Macht des Dunklen Lords \emph{war} einmal gebrochen worden.

Und Harry würde sie wieder beschützen. Wenn es tatsächlich eine Prophezeiung gab und es das war, was sie besagte. Nun, eigentlich egal was irgendeine verfluchte Prophezeiung sagte.

All diese Leute, die an ihn glaubten und ihm zujubelten -- Harry konnte es nicht ertragen, das falsch sein zu lassen. Aufzublitzen und zu vergehen, wie so viele andere Wunderkinder. Eine Enttäuschung zu sein. Dabei zu versagen, seinem Ruf als ein Symbol des Lichts gerecht zu werden, egal \emph{wie} er ihn bekommen hatte. Er würde absolut, unbedingt, egal wie lange es dauerte und selbst wenn es ihn umbrachte, ihre Erwartungen erfüllen. Und dann weitermachen, diese Erwartungen zu \emph{übertreffen,} so dass die Leute sich rückblickend wundern würden, dass sie einst so wenig von ihm verlangt hatten.

"\emph{HARRY POTTER! HARRY POTTER! HARRY POTTER!}"

Harry vollführte die letzten Schritte bis zum Sprechenden Hut. Er verbeugte sich zum Orden des Chaos am Gryffindor-Tisch und drehte sich dann um, für eine Verbeugung zur anderen Seite der Halle und wartete, bis der Applaus und das Gelächter erstarben.

(In seinem Hinterkopf fragte er sich, ob der Sprechende Hut wirklich \emph{bei Bewusstsein} war in dem Sinn, sich seines eigenen Bewusstseins bewusst zu sein und wenn ja, ob er zufrieden damit war, nur einmal im Jahr mit Elf-jährigen zu sprechen. Sein Lied deutete darauf hin: \emph{Oh, ich bin der Sprechende Hut und mir geht es gut, darf das ganze Jahr ruh'n, muss nur einmal was tun…}****)

Als es im Raum einmal mehr still wurde, setzte Harry sich auf den Stuhl und setzte sich \emph{vorsichtig} das 800-Jahre-alte telepathische Artefakt vergessener Magie auf den Kopf.

Denkend, so stark er nur konnte: \emph{Sortier' mich noch nicht ein! Ich habe Fragen, die ich dir stellen muss! Wurden mir je die Erinnerungen gelöscht? Hast du den Dunklen Lord ausgewählt, als er ein Kind war und kannst du mir etwas über seine Schwächen verraten? Kannst du mir sagen, warum ich den Bruder des Zauberstabs des Dunklen Lords bekommen habe? Ist der Geist des Dunklen Lords an meine Narbe gebunden und ist das der Grund, warum ich manchmal so zornig werde? Das sind die wichtigen Fragen, aber wenn du noch einen Moment hast, kannst du mir irgendwas darüber verraten, wie man die verlorenen magischen Künste wiederentdecken kann, die dich erschaffen haben?}

In der Stille von Harrys Geist, wo es nie zuvor mehr als nur eine Stimme gegeben hatte, ertönte eine zweite und ungewohnte Stimme, die deutlich beunruhigt klang:

\emph{"Oh je. Das ist noch nie zuvor passiert…"}

* Natürlich dürft ihr auch gerne noch mitraten, obwohl ich nicht weiß, wie viel Zeit die Leser der englischen Fassung hatten und ich von daher nicht versprechen kann, euch auch nur annähernd so viel Zeit einzuräumen.

** Eine außerirdische Spezies aus dem Roman \emph{Im Reich der Dirdir,} dem dritten Band des \emph{Tschai-Zyklus.}

*** engl.: \emph{Puppeteers}, auch \emph{Pierson's Puppeteers,} eine außerirdische Spezies aus den \emph{Known} \emph{Space}-Romanen von Larry Niven, angefangen mit \emph{Der schwebende Wald.}

**** engl.: \emph{Oh, I'm the Sorting Hat and I'm okay, I sleep all year and I work one day…}

