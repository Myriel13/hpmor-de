

\hypertarget{abstimmungsprobleme-teil-1}{% \section{33. Abstimmungsprobleme, Teil 1}\label{abstimmungsprobleme-teil-1}}

\textbf{Kapitel 33: Abstimmungsprobleme, Teil 1

}

I just recite to myself, over and over, until I can choose sleep: It all adds up to J. K. Rowling.

Die in diesem Kapitel verwendete Version der Entscheidungstheorie ist \emph{nicht} die in der akademischen Welt dominierende. Sie basiert auf etwas namens "Zeitlose Entscheidungstheorie", die derzeit entwickelt wird von (unter anderem) Gary Drescher, Wei Dai, Vladimir Nesov und, nun ja… \emph{(hüstel)} mir.

\later

Das Erschreckende war, wie schnell die ganze Sache außer Kontrolle geraten war.

"Albus," sagte Minerva und versuchte nicht einmal, die Besorgnis in ihrer Stimme zu verbergen, während sie beide die Große Halle betraten, "es muss etwas geschehen."

Für gewöhnlich war die Stimmung in Hogwarts vor den Weihnachtstagen heiter und ausgelassen. Die Große Halle war bereits in Grün und Rot geschmückt, zum Gedenken zweier Schüler aus Gryffindor und Slytherin*, deren Hochzeit zur Weihnachtszeit zu einem Symbol der Freundschaft geworden war, die Häuser und Bündnisse überwand, eine Tradition beinahe ebenso alt wie Hogwarts selbst, die sich sogar bis in die Länder der Muggel verbreitet hatte.

Doch nun blickten die Schüler beim Abendessen nervös über die Schulter oder warfen den anderen Tischen boshafte Blicke zu und an so mancher Tafel wurde hitzig gestritten. Man hätte die Stimmung als \emph{angespannt} bezeichnen können, doch der Ausdruck, der sich Minerva aufdrängte, war \emph{Alarmstufe} \emph{Fünf.}

Man nehme eine Schule, in vier Häuser geteilt…

Nun noch in jedem Jahrgang, drei Armeen im Krieg.

Und inzwischen ergriffen längst nicht mehr nur die Erstklässler Partei für Drachen und Sunshine und Chaos; sie waren zu den Armeen derer geworden, die selbst keine hatten. Die Schüler trugen Armbänder mit den Zeichen von Feuer und Smiley und Erhobener Hand und verhexten einander in den Korridoren. Alle drei Generäle der Erstklässler hatten versucht, ihnen Einhalt zu gebieten - selbst Draco Malfoy hatte sie angehört und schließlich grimmig genickt - doch ihre vorgeblichen Anhänger hatten ihnen keine Beachtung geschenkt.

Entrückt ließ Dumbledore den Blick über die Tische schweifen. "\emph{In einer jeden Stadt,}" rezitierte der alte Zauberer sachte, "\emph{ist alles Volk von alters her gespalten in die Fraktionen der Blauen und der Grünen… Sie bekämpfen ihre Gegner, ohne auch nur recht zu wissen,} \emph{zu} \emph{welchem} \emph{Sinnsie sich insolcherlei Gefahr} \emph{begeben} \emph{mögen…} \emph{Und so} \emph{erwächst in ihnen eine Feindschaft gegen ihre Nächsten, die keinen} \emph{Ursprung} \emph{kennt und nie} \emph{erlahmt,} \emph{gar} \emph{dass sie je verschwände;denn} \emph{selbst den Banden der Ehe, noch der Verwandtschaft, noch der Freundschaft} \emph{weicht sie nicht} \emph{und mögen die, die sich nach jenen Farben scheiden, auch Brüder oder sonst dergleichen sein.} \emph{Ich, für meinen Teil, vermag dies nichts anderes zu heißen, als eine Krankheit, die die Seele befällt…}"**

"Verzeihung," sagte Minerva, "ich verstehe nicht -"

"Prokopios," sagte Dumbledore. "Sie nahmen ihre Wagenrennen sehr ernst, im Alten Rom. Ja, Minerva, ich stimme zu, dass etwas geschehen muss."

"Bald," sagte Minerva und senkte die Stimme noch weiter. "Albus, ich glaube, dass es noch vor dem Samstag geschehen muss."

Am Sonntag würden die meisten Schüler Hogwarts verlassen, um die Feiertage mit ihren Familien zu verbringen; daher fand am Samstag die finale Schlacht der drei Armeen der Erstklässler statt, die entscheiden würde, wer Professor Quirrells dreimal verfluchten Weihnachtswunsch gewann.

Dumbledore wandte sich zu ihr und studierte sie mit schwermütigem Blick. "Sie fürchten, dass die ganze Sache dann hochgeht und jemand verletzt werden wird."

Minerva nickte.

"Und dass Professor Quirrell die Schuld gegeben wird."

Minerva nickte erneut, ihr Gesicht angespannt. Sie war schon seit langem erfahren darin, auf welche Weise Verteidigungs-Professoren gefeuert wurden. "Albus," sagte Minerva, "wir dürfen Professor Quirrell jetzt nicht verlieren, wir \emph{dürfen es} \emph{nicht!} Wenn er auch nur bis zum Januar bleibt, werden unsere Fünftklässler ihre ZAGs bestehen, schafft er es bis zum Mai, bestehen unsere Siebtklässler ihre UTZs, er holt Jahre der Vernachlässigung in Monaten wieder auf, eine ganze Generation wird heranwachsen, die in der Lage sein wird, sich zu verteidigen, trotz dem Fluch des Dunklen Lords - Sie müssen die Schlacht verhindern, Albus! Verbieten Sie die Armeen, jetzt!"

"Ich bin nicht sicher, wie gut der Verteidigungsprofessor das aufnehmen würde," sagte Dumbledore und blickte zum Lehrertisch, wo Professor Quirrell in seine Suppe sabberte. "Seine Armeen schienen ihm sehr wichtig zu sein, obwohl ich natürlich glaubte, es würde vier Armeen in jedem Jahrgang geben, als ich dem zustimmte." Der alte Zauberer seufzte. "Ein intelligenter Mann, wahrscheinlich mit den besten Absichten, doch nicht intelligent genug, wie ich fürchte. Und das Verbieten der Armeen mag die Explosion ebenso auslösen."

"Aber Albus, was werden Sie dann \emph{tun?}"

Der alte Zauberer bedachte sie mit einem nachsichtigen Lächeln. "Nun, ich schmiede natürlich einen Plan. So scheint es in Hogwarts die neue Mode zu sein."

Und inzwischen waren sie dem Lehrertisch zu nahe, als dass Minerva noch etwas hätte erwidern können.

\later

Das Erschreckende war, wie schnell die ganze Sache außer Kontrolle geraten war.

Die erste Schlacht im Dezember war… chaotisch verlaufen, so hatte Draco jedenfalls gehört.

Die zweite Schlacht war völlig \emph{verworren.}

Und die nächste würde noch \emph{schlimmer} werden, es sei denn sie drei hatten Erfolg mit ihrem letzten verzweifelten Versuch, das zu verhindern.

"Professor Quirrell, das ist Wahnsinn," sagte Draco gerade heraus. "Das hat nichts mehr damit zu tun, wie ein Slytherin zu denken, es ist einfach…" Draco fehlten die Worte. Er wedelte hilflos mit den Händen. "Es ist einfach unmöglich, irgendwelche echten Pläne zu schmieden, mit all diesem Unsinn, der vor sich geht. In der letzten Schlacht hat einer meiner Soldaten seinen eigenen Selbstmord vorgetäuscht. Wir haben \emph{Hufflepuffs,} die versuchen, Verschwörungen auszuhecken und sie glauben, sie können es, doch das können sie \emph{nicht.} Es passieren jetzt einfach nur noch rein zufällig irgendwelche Dinge, das hat nichts mehr damit zu tun, wer am cleversten ist oder wessen Armee am besten kämpft, es ist…" Er konnte es nicht einmal beschreiben.

"Ich muss Mr~Malfoy zustimmen," sagte Granger und kland dabei wie jemand, der niemals erwartet hätte, sich diese Worte sagen zu hören. "Verräter zu erlauben funktioniert nicht, Professor Quirrell."

Draco hatte zu verbieten versucht, dass irgendwer in seiner Armee Pläne schmiedete, außer ihm und hatte damit die Verschwörungen nur in den Untergrund getrieben; niemand wollte außen vor bleiben, wenn die Soldaten in \emph{anderen} Armeen Pläne schmieden durften. Nachdem sie ihre letzte Schlacht jämmerlich verloren hatten, hatte er schließlich eingelenkt und seine Weisung zurückgezogen; doch zu diesem Zeitpunkt hatten seine Soldaten bereits ihre eigenen Pläne in Gang gesetzt, ohne jede Form von zentraler Koordination.

Nachdem er sich von all ihren Plänen hatte berichten lassen oder zumindest das, wovon seine Soldaten behaupteten, es wären ihre Pläne, hatte er versucht, einen eigenen Plan zu entwerfen, um die letzte Schlacht zu gewinnen. Er setzte erheblich mehr als drei verschiedene Dinge voraus, die funktionieren mussten und Draco hatte das Papier mit einem \emph{Incendio} in Brand gesetzt und mit \emph{Everto}die Asche verschwinden lassen, denn hätte Vater das gesehen, wäre er enterbt worden.

Professor Quirrells Augenlider waren halb geschlossen und sein Kinn ruhte auf seinen Händen, als er sich auf seinen Schreibtisch vorbeugte. "Und Sie, Mr~Potter?" sagte der Verteidigungs-Professor. "Schließen Sie sich dem ebenfalls an?"

"Wir bräuchten jetzt nur noch Franz Ferdinand zu erschießen und wir könnten den Ersten Weltkrieg lostreten," sagte Harry. "Das Chaos ist perfekt. Mir gefällt's."

"\emph{Harry!}" sagte Draco völlig geschockt.

Ihm wurde sogar erst nach einer Sekunde klar, dass er es exakt zur selben Zeit und im exakt gleichen Tonfall der Empörung gesagt hatte, wie Granger.

Granger warf ihm einen erstaunten Blick zu und Draco gab sich betont nichtssagend. Uups.

"Genau, so ist es!" sagte Harry. "Ich hintergehe euch! Alle beide! Schon wieder! Haha!"

Professor Quirrell lächelte dünn, doch seine Augen waren noch immer halb geschlossen. "Und weshalb das, Mr~Potter?"

"Weil ich glaube, mit dem Chaos besser umgehen zu können, als Miss~Granger oder Mr~Malfoy," sagte der Verräter. "Unser Krieg ist ein Nullsummenspiel und es spielt keine Rolle, ob es im absoluten Sinne einfach oder schwierig ist, sondern nur wer besser oder schlechter abschneidet."

Harry Potter lernte erheblich zu schnell.

Professor Quirrells Augen wanderten unter den Lidern umher, fixierten erst Draco, dann Granger. "Um die Wahrheit zu sagen, Mr~Malfoy, Miss~Granger, ich könnte es mir einfach nicht verzeihen, das große Debakel vor seinem Höhepunkt zum Ende zu bringen. Einer ihrer Soldaten ist sogar zueinem Vierfachagenten geworden."

"\emph{Vierfach?}" sagte Granger. "Aber es gibt nur drei Seiten in diesem Krieg!"

"Ja," sagte Professor Quirrell, "so sollte man meinen, nicht wahr. Ich glaube nicht, dass es jemals in der Geschichte einen Vierfachagenten oder irgendeine Armee mit einem solch hohen Anteil an vorgeblichen und echten Verrätern gegeben hat. Wir erforschen hier neue Gefilde, Miss~Granger und wir können jetzt nicht umkehren."

Draco verließ das Büro des Vereidigungs-Professors mit heftig knirschenden Zähnen und Granger neben ihm blickte sogar noch wütender drein.

"Ich kann nicht glauben, dass du das getan hast, Harry!" sagte Granger.

"Sorry," sagte Harry und klang dabei ganz und gar nicht, als täte es ihm Leid, die Lippen zu einem diabolischen Grinsen verzogen. "Denk dran, Hermine, es \emph{ist} nur ein Spiel und warum sollten Generäle wie wir die einzigen sein, die Pläne schmieden dürfen? Außerdem, was wollt ihr zwei schon dagegen machen? Euch gegen mich zusammentun?"

Draco tauschte Blicke mit Granger aus und wusste, dass sein Gesicht ebenso angespannt war wie ihres. Harry hatte mit der Zeit immer öfter und mit immer unverhohlener Schadenfreude seinen Vorteil gezogen aus Dracos Weigerung, mit einem Schlammblut gemeinsame Sache zu machen und Draco hatte es langsam \emph{satt,} dass das gegen ihn benutzt wurde. Wenn das noch lange so weiter ging, dann würde er sich mit Granger verbünden, wenn auch nur um Harry Potter zu vernichten und dann würden sie ja sehen, wie diesem Sohn eines Schlammbluts \emph{das} gefiel.

\later

Das Erschreckende war, wie schnell die ganze Sache außer Kontrolle geraten war.

Hermine betrachtete das Pergament, das Zabini ihr gegeben hatte und füllte sich absolut und vollkommen hilflos.

Es gab Namen und Linien, die Namen mit anderen Namen verbanden und einige der Linien hatten verschiedene Farben und…

"Sagen Sie mir," sagte General Granger, "gibt es irgendjemanden in meiner Armee, der \emph{kein} Spion ist?"

Die beiden befanden sich nicht im Büro, sondern in einem anderen verlassenen Klassenraum und sie waren allein, denn wie Colonel Zabini gemeint hatte, war es nun beinahe sicher, dass zumindest einer der Captains ein Verräter war. Wahrscheinlich Captain Goldstein, doch Zabini wusste es nicht mit Sicherheit.

Ihre Frage hatte den jungen Slytherin zu einem ironischen Lächeln veranlasst. Blaise Zabini wirkte ihr gegenüber immer ein wenig herablassend, doch schien er ihr keine offene Verachtung entgegen zu bringen; anders als der Hohn, den er Draco Malfoy entgegen brachte oder der Groll, den er für Harry Potter entwickelt hatte. Zunächst war sie besorgt gewesen, Zabini könnte sie verraten, doch der Junge schien verzweifelt beweisen zu wollen, dass er den anderen beiden Generälen in nichts nach stand und Hermine war der Ansicht, obwohl Zabini sie wahrscheinlich bereitwillig an jeden \emph{anderen} verraten würde, so würde er doch niemals zulassen, dass Malfoy oder Harry gewannen.

"Die meisten Ihrer Soldaten \emph{sind}Ihnen noch immer treu ergeben, da bin ich ziemlich sicher," sagte Zabini. "Es will einfach nur niemand bei dem Spaß außen vor bleiben." Der verächtliche Ausdruck auf dem Gesicht des Slytherin machte deutlich, was er von Leuten hielt, die Verschwörungen nicht ernst nahmen. "Also denken sie, sie können Doppelagenten sein und insgeheim für unsere Seite arbeiten, während sie vorgeben, uns zu hintergehen."

"Und das gilt ebenso für jeden aus den \emph{anderen} Armeen, der sagt, er wolle \emph{unser} Spion sein," sagte Hermine vorsichtig.

Der junge Slytherin zuckte mit den Schultern. "Ich denke, ich habe eine ziemlich genaue Vorstellung davon, wer Malfoy wirklich an Sie verraten will, obwohl ich nicht sicher bin, ob Ihnen \emph{irgendjemand} wirklich Potter ausliefern will. Aber Nott wird ganz sicher Potter an Malfoy verraten und da ich Entwhistle vorgeblich auf Malfoys Geheiß habe an ihn herantreten lassen und Entwhistle in Wirklichkeit an uns berichtet, ist das beinahe so gut -"

Hermine schloss einen Moment lang die Augen. "Wir werden verlieren, nicht wahr?"

"Sehen Sie," sagte Zabini geduldig, "Im Augenblick sind Sie in Führung, was Quirrell-Punkte angeht. Wir brauchen nur diese letzte Schlacht nicht \emph{völlig} zu verlieren und Sie haben genug Quirrell-Punkte um den Weihnachtswunsch zu gewinnen."

Professor Quirrell hatte angekündigt, dass bei der letzten Schlacht ein formelles Punktesystem verwendet werden würde, worum er gebeten worden war, um zu vermeiden, dass das Ergebnis im Nachhinein angefochten wurde. Für jeden Treffer bekam der General der eigenen Armee zwei Quirrell-Punkte. Ein Gong würde auf dem Schlachtfeld ertönen (sie wussten noch nicht, wo sie kämpfen würden, doch Hermine hoffte erneut auf den Wald, wo Sunshine sich gut anstellte) und die Tonhöhe würde anzeigen, welche Armee die Punkte erzielt hatte. Und gab irgendjemand vor, getroffen worden zu sein, würde der Gong trotzdem ertönen und etwas später dann, nach einer unbestimmten Zeitspanne, ein doppelter Gong um den Widerruf zu verkünden. Und rief man denn Namen einer Armee aus, etwa "Für Sunshine!" oder "Für Chaos!" oder "Für Drachen!" wechselte die eigene Zugehörigkeit zu jener Armee…

Selbst Hermine hatte den Fehler in \emph{diesem} Regelwerk erkennen können. Doch wie Professor Quirrell weiterhin verkündet hatte, konnte jemanden, der etwa ursprünglich Sunshine zugeordnet war, niemand im Namen Sunshines erschießen - oder viel mehr, man konnte schon, doch dann verlor Sunshine einen einzelnen Quirrell-Punkt, symbolisiert von einem dreifachen Gong. Was verhinderte, dass man seine eigenen Soldaten für Punkte erschoss und es unsinnig machte, sich selbst zu erschießen, bevor einen der Feind erwischte, doch falls nötig konnte man noch immer Spione erschießen.

Gegenwärtig hatte Hermine zweihundert und vierundvierzig Quirrell-Punkte, Malfoy hatte zweihundert und neunzehn und Harry hatte zweihundert und einundzwanzig und es gab vierundzwanzig Soldaten in jeder Armee.

"Wir kämpfen also vorsichtig," sagte Hermine, "und versuchen einfach nur, nicht allzu hoch zu verlieren."

"Nein," sagte Zabini. Der Gesichtsausdruck des jungen Slytherin wurde ernst. "Das Problem ist, Malfoy und Potter wissen beide, sie können nur gewinnen, indem sie sich zusammentun und uns vernichten, um den Kampf dann unter sich auszutragen. Also folgendes, denke ich, sollten wir tun -"

Hermine verließ den Klassenraum wie leicht benebelt. Zabinis Plan hatte nicht gerade auf der Hand gelegen, er war seltsam und kompliziert und verschachtelt und eher die Art von Plan, die sie von Harry erwartet hätte, nicht von Zabini. Es fühlte sich falsch an, überhaupt dazu in der Lage zu sein, einen solchen Plan zu begreifen. Junge Mädchen sollten keine solchen Pläne verstehen können. Der Hut hätte sie nach Slytherin geschickt, hätte er gesehen, dass sie dazu fähig wäre, einen solchen Plan zu verstehen…

\later

Das Fantastische war, wie schnell er das Chaos hatte eskalieren können, sobald er es mit voller Absicht tat.

Harry saß in seinem Büro; man hatte ihm die Erlaubnis erteilt, von den Hauselfen Möbel zu ordern, also hatte er einen Thron herbeischaffen lassen und in Schwarz und Purpurrot gemusterte Vorhänge. Scharlachrotes Licht, gemischt mit Schatten, floss wie Blut über den Boden.

Irgendetwas tief in Harry hatte das Gefühl, er sei endlich daheim.

Vor ihm standen die vier Lieutenants von Chaos, seine vertrauenswürdigsten Lakaien, einer von ihnen ein Verräter.

Das. Genau so sollte das Leben sein.

"Wir sind versammelt," sagte Harry.

"Lasst Chaos regieren," erwiderten seine vier Lieutenants im Chor.

"Mein Luftkissenfahrzeug ist voller Aale," sagte Harry.

"Ich werde diese Schallplatte nicht kaufen, sie ist zerkratzt," erwiderten die vier Lieutenants.

"Und aller-mümsige Burggoven."

"Die mohmen Räth' ausgraben!"

Damit waren die Formalitäten abgeschlossen.

"Wie schreitet die Verwirrung voran?" sagte Harry in trockenem Flüsterton wie Imperator Palpatine.

"Sehr gut, General Chaos," sagte Neville im selben Tonfall, den er bei militärischen Angelegenheiten immer anzustimmen pflegte, so tief, dass der Junge oft innehalten und sich räuspern musste. Der Chaos-Lieutenant war formell gekleidet in seinen schwarzen Schulumhang, getrimmt mit dem Gelb des Hauses Hufflepuff und sein Haar war gescheitelt und gekämmt, wie es angemessen war für einen aufrechten jungen Mann. Harry hatte dieser Widerspruch besser gefallen, als alle Kutten, die sie ausprobiert hatten. "Unsere Legionäre haben seit gestern Abend fünf neue Pläne in Gang gesetzt."

Harry grinste bösartig. "Hat irgendeiner davon Aussicht auf Erfolg?"

"Ich denke nicht," sagte Neville von Chaos. "Hier der Bericht."

"Exzellent," sagte Harry frostig lachend, als er Neville das Pergament aus der Hand nahm, und gab sich alle Mühe, es so klingen zu lassen als rassle der Staub in seiner Kehle. Das brachte die Gesamtanzahl auf sechzig.

Sollte Draco mal \emph{versuchen,} damit umzugehen. Sollte er es nur \emph{versuchen.}

Und was Blaise Zabini betraf…

Harry lachte erneut und dieses mal musste er sich nicht einmal anstrengen, damit es böse klang. Er musste sich für die Treffen mit seinem Stab wirklich von irgendwem einen Kniesel leihen, damit er unterdessen eine Katze streicheln könnte.

"Kann die Legion jetzt mit den Plänen aufhören?" sagte Finnigan von Chaos. "Ich meine, haben wir nicht schon genug -"

"Nein," sagte Harry schlicht. "Es kann \emph{niemals} genug Verschwörungen geben."

Professor Quirrell hatte es perfekt auf den Punkt gebracht. Sie loteten die Grenzen aus, vielleicht weiter als jemals zuvor und Harry hätte es sich nie verzeihen können, hätte er jetzt einen Rückzieher gemacht.

Ein Klopfen erklang an der Tür.

"Das dürfte der General der Drachen sein," sagte Harry und zeigte ein bösartig wissendes Lächeln. "Ganz wie erwartet. Führt ihn herein und euch selbst hinaus."

Und die vier Lieutenants schoben sich hinaus, warfen Draco finstere Blicke zu während der feindliche General Harrys geheimen Unterschlupf betrat.

Sollte er das nicht mehr tun dürfen wenn er älter wurde, so würde Harry einfach für immer elf Jahre alt bleiben.

\later

Die Sonne ergoss sich durch die roten Vorhänge hinter Harry Potters gepolstertem Sessel in Erwachsenengröße, den er mit gold- und silbernem Glitter bestreut hatte und hartnäckig als seinen Thron bezeichnete und ließ rote Strahlen aus Blut von über den Boden tanzen.

(Draco war allmählich bedeutend zuversichtlicher, dass es die richtige Entscheidung gewesen war, Harry Potter zu stürzen, bevor er die Weltherrschaft übernehmen konnte. Draco wagte sich ein Leben unter seiner Regentschaft nicht einmal \emph{vorzustellen.})

"Guten Abend, Drachen-General," sagte Harry Potter mit eisigem Flüstern. "Sie treffen ein, genau wie von mir erwartet."

Was nicht überraschend war angesichts der Tatsache, dass Draco und Harry den Zeitpunkt dieses Treffens im Vorhinein vereinbart hatten.

Und abends war es ebenfalls nicht, doch Draco verbiss sich eine entsprechende Bemerkung, denn mittlerweile wusste er es besser.

"General Potter," sagte Draco so würdevoll wie irgend möglich, "Ihnen ist bewusst, dass unser beider Armeen zusammenarbeiten müssen, damit auch nur \emph{einer} von uns Professor Quirrells Weihnachtswunsch gewinnen kann, nicht wahr?"

"Sssehr richtig," zischte Harry, als glaubte der Junge er sei ein Parselmund. "Wir müssen kooperieren, um Sunshine zu zerstören, bevor wir es dann unter uns austragen können. Doch sollte einer von uns den anderen zuvor betrügen, könnte dieser im späteren Kampf einen Vorteil erlangen. Und General Sunshine, der all dies ebenfalls bewusst ist, wird versuchen, jedem von uns vorzutäuschen, der andere habe ihn betrogen. Und Sie und ich, denen das wiederum bewusst ist, werden versucht sein, den anderen zu betrügen und es auf Grangers Hinterlist zu schieben. Was Granger \emph{ebenfalls} weiß."

Draco nickte. So viel war offensichtlich. "Und… uns beiden geht es \emph{nur} um den Sieg und es gibt keinen Dritten, der einen von uns bestrafen könnte, wenn wir einander verraten…"

"Exakt," sagte Harry Potter, nun mit Ernst auf dem Gesicht. "Wir stehen vor einem \emph{wahren} Gefangenendilemma."

Das Gefangenendilemma verlief, Harrys Lektionen zufolge, wie folgt: Zwei Gefangene wurden in getrennte Zellen gesperrt. Es gab Beweise gegen jeden Gefangenen, doch nur geringfügige, genug für eine Verurteilung zu je zwei Jahren Haft. Jeder der Gefangenen konnte sich entscheiden sich \emph{gegen den anderen zu wenden,} ihn zu verraten und vor Gericht gegen ihn auszusagen, was seine eigene Haftstrafe um ein Jahr verkürzen, die des anderen jedoch um ein Jahr verlängern würde. Oder er konnte \emph{kooperieren} und schweigen. Sollten also beide Gefangenen gestehen und jeder gegen den anderen aussagen, würden sie jeweils drei Jahre absitzen; wenn beide kooperierten oder schwiegen, würden sie je zwei Jahre absitzen; doch wenn einer gestand und der andere kooperierte, würde der Geständige nur ein einziges Jahr absitzen, der Kooperierende dagegen vier.

Und beide Gefangene mussten sich entscheiden, ohne die Wahl des anderen zu kennen und keinem würde die Chance zuteil, seine Entscheidung im Nachhinein zu ändern.

Draco hatte angemerkt, wären die beiden Gefangenen Todesser während des Zaubererkrieges gewesen, so hätte der Dunkle Lord jeden Verräter getötet.

Harry hatte genickt und gemeint, das sei \emph{eine} mögliche Auflösung des Gefangenendilemmas - und in der Tat würden beide Todesser aus genau diesem Grund \emph{wünschen,} es gäbe einen Dunklen Lord.

(An jener Stelle hatte Draco Harry um eine Pause gebeten, um dies eine Weile zu überdenken, bevor sie fortfuhren. Das erklärte so \emph{einiges} darüber, wieso Vater und seine Freunde willentlich einem Dunklen Lord zu Diensten gewesen waren, der oft nicht nett zu ihnen war…)

Tatsächlich, so hatte Harry gemeint, war das ziemlich genau der Grund, warum Menschen Regierungen hatten - ein \emph{Einzelner} mochte einen Vorteil daraus ziehen, wenn er jemand anderem etwas stahl, so wie jeder Gefangene für sich besser da stünde, wenn er beim Gefangenendilemma gestand. Doch wenn \emph{jeder} so dachte, würde das Land im Chaos versinken und alle stünden schlechter da, so wie es geschähe, wenn beide Gefangenen sich gegeneinander wandten. Also ließen sich die Menschen von Regierungen beherrschen, ebenso wie die Todesser sich vom Dunklen Lord hatten beherrschen lassen.

(Erneut hatte Draco Harry um eine Pause gebeten. Draco hatte es immer als gegeben hingenommen, dass Zauberer die ehrgeizig waren, die Macht ergriffen, weil sie herrschen wollten und die Leute sich beherrschen ließen, weil sie ängstliche kleine Hufflepuffs waren. Und das schien, nach einiger Überlegung, noch immer wahr zu sein; doch Harrys Perspektive war zweifellos faszinierend, auch wenn er falsch lag.)

Allerdings, fuhr Harry daraufhin fort, war die Furcht vor der Strafe durch eine dritte Partei nicht der \emph{einzig} mögliche Grund, im Gefangenendilemma zu kooperieren.

Nehmen wir an, hatte Harry gemeint, man spielte das Spiel gegen eine magisch erzeugte, identische Kopie seiner selbst.

Draco hatte gemeint, gäbe es zwei Dracos, so würde natürlich kein Draco wollen, dass dem anderen etwas Schlimmes widerfuhr; ganz zu schweigen davon, dass kein Malfoy sich je einen Verräter würde schimpfen lassen.

Erneut hatte Harry genickt und gesagt, dies sei eine \emph{weitere} Lösung des Gefangenendilemmas - die Menschen könnten kooperieren, weil sie einander wichtig waren oder aufgrund ihres Ehrgefühls oder weil sie bestrebt waren, ihren Ruf zu wahren. Tatsächlich, hatte Harry gemeint, war es recht schwierig, ein \emph{wahres} Gefangenendilemma zu schaffen - im echten Leben kümmerten sich die Menschen üblicherweise um einander oder auch ihre Ehre oder ihren Ruf oder die Strafe eines Dunklen Lords oder \emph{sonst irgendetwas} außer Gefängnisstrafen. Doch angenommen, es sei die Kopie von jemand \emph{absolut} selbstsüchtigem -

(Pansy Parkinson war ihr Beispiel der Wahl gewesen)

- so dass sich jede Pansy nur darum scherte, was mit \emph{ihr} geschah und nicht mit der anderen Pansy.

\emph{Angenommen,} das sei alles, was Pansy kümmerte… und es keinen Dunklen Lord gäbe… und Pansy sich nicht um ihren Ruf scherte… und Pansy entweder keinerlei Ehrgefühl besaß oder sich der anderen Gefangenen nicht verpflichtet fühlte… wäre \emph{dann} die rationale Entscheidung für Pansy, zu kooperieren oder zu gestehen?

Manche Leute, sagte Harry, behaupteten, die rationale Entscheidung sei, dass Pansy gegen ihre Kopie aussagte, doch Harry und außerdem noch jemand namens Douglas Hofstadter, glaubten diese Leute lägen falsch. Denn, so hatte Harry gesagt, wenn Pansy gestand - nicht aufs Geratewohl, sondern weil sie glaubte, dafür \emph{rationale Gründe} zu haben - so würde die andere Pansy exakt genauso denken. Zwei identische Kopien würden nicht verschiedene Dinge entscheiden. Also hatte Pansy zu wählen zwischen einer Welt, in der beide Pansys kooperierten und einer Welt, in der beide Pansys gestanden und sie kam besser dabei weg, wenn beide Kopien kooperierten. Und würde Harry glauben, \emph{dass} 'rationale' Menschen im Gefangenendilemma gestanden, so hätte er diese Form von 'Rationalität' bestimmt nicht verbreitet, denn ein Land oder eine Verschwörung voller 'rationaler' Menschen würde im Chaos versinken. 'Rationalität' wäre etwas, wovon man nur seinen \emph{Feinden} erzählte.

Was sich zu jenem Zeitpunkt auch alles vernünftig \emph{angehört} hatte, doch \emph{nun} kam Draco der Gedanke…

"\emph{Du} hast gesagt," sagte Draco, "die rationale Lösung des Gefangenendilemmas sei, zu kooperieren. Aber natürlich würdest \emph{du} wollen, dass ich das glaube, nicht wahr?" Und ließe Draco sich täuschen und kooperierte, so würde Harry hinterher nur sagen, \emph{Haha, wieder reingelegt!} und ihn auslachen.

"Ich würde dir keine falschen Lektionen erteilen," sagte Harry ernst. "Doch ich muss dich daran erinnern, Draco, ich sagte nicht, du solltest einfach automatisch kooperieren. Nicht bei einem \emph{wahren} Gefangenendilemma, wie diesem hier. Was ich sagte, war, dass wenn du entscheidest, du nicht so denken solltest, als entschiedest du nur für dich selbst \emph{oder} als entschiedest du für jedermann. Du solltest so denken, als entschiedest du für alle Menschen, die dir \emph{ähnlich genug} sind, dass sie wahrscheinlich dasselbe tun werden wie du, aus denselben Gründen. Und dabei auch die Vorhersagen berücksichtigen, die ein jeder treffen kann, der dich gut genug kennt, um dein Verhalten zutreffend vorherzusagen, damit du dein rationales Handeln nie bereuen musst, aufgrund der korrekten Vorhersagen, die andere Leute über dich treffen - erinnere mich daran, dir bei Gelegenheit einmal Newcombs Problem zu erklären. Die Frage, die du und ich uns also stellen müssen, Draco, ist folgende: Sind wir einander ähnlich genug, dass wir wahrscheinlich \emph{dasselbe} tun werden, was immer das sein mag und unsere Entscheidungen größtenteils auf die selbe Weise treffen? Oder kennen wir einander gut genug, um die Handlungen des anderen vorherzusagen, so dass \emph{ich} vorhersagen kann, ob du kooperieren oder mich verraten wirst und \emph{du} vorhersagen kannst, dass ich entschieden habe, dasselbe zu tun, was du meiner Vorhersage nach tun wirst, weil \emph{ich} weiß, dass du vorhersehen wirst, dass ich so entscheide?"

… und Draco konnte nur denken, da er sich schon anstrengen musste, auch nur die \emph{Hälfte} davon zu verstehen, so lautete die Antwort offensichtlich 'Nein'.

"Ja," sagte Draco.

Eine Pause entstand.

"Ich verstehe," sagte Harry und klang enttäuscht. "Na schön. Ich schätze, müssen wir uns wohl etwas anderes einfallen lassen."

Draco hatte nicht geglaubt, dass das funktionieren würde.

Draco und Harry diskutierten es wieder und wieder durch. Sie waren beide bereits zuvor überein gekommen, dass ihre Taten auf dem Schlachtfeld \emph{nicht} als gebrochene Versprechen im wahren Leben zählten - obwohl Draco ein wenig verärgert darüber war, was Harry in Professor Quirrells Büro getan hatte und das sagte er auch.

Doch wenn sie beide sich weder auf Freundschaft noch Ehre verlassen konnten, so \emph{blieb} die Frage offen, wie sie ihre Armeen zur Zusammenarbeit bewegen sollten, um Sunshine zu schlagen, trotz allem was Granger versuchen mochte, um einen Keil zwischen sie zu treiben. Professor Quirrells Regeln ließen es nicht allzu verlockend erscheinen, die Soldaten der anderen Armee von Sunshine töten zu lassen - das machte die Hürde, die man selbst zu nehmen hatte, nur umso höher - doch sie verleiteten jede Seite dazu, Abschüsse zu stehlen, anstatt wie eine einzige Armee zu agieren oder einige Soldaten der anderen Seite in der Hitze des Gefechts zu erschießen…

\later

Hermine war auf dem Rückweg nach Ravenclaw und achtete nicht wirklich darauf, wohin sie lief, im Geiste beschäftigt mit Krieg und Verrat und anderen altersunangemessenen Konzepten und so bog sie um eine Ecke und stieß prompt mit einem Erwachsenen zusammen.

"Sorry," sagte sie automatisch und dann, völlig unwillkürlich, "\emph{Iiiiek!}"

"Nur keine Sorge, Miss~Granger," sagte das freundliche Lächeln, inmitten der funkelnden Augen und oberhalb des silbernen Bartes des SCHULLEITERS VON HOGWARTS. "Ihnen sei gänzlich vergeben."

Vollkommen hilflos konnte sie den Blick nicht abwenden von dem freundlichen Gesicht des mächtigsten Zauberers der Welt, der außerdem Großmeister war und Ganz Hohes Tier, vor Jahren durch die Belastungen des Kampfes gegen den Dunklen Lord in den Wahnsinn getrieben und unzählige andere Fakten poppten einer nach dem anderen in ihrem Geiste auf, während ihre Kehle nur peinliche kleine Quiekgeräusche zustande brachte.

"Tatsächlich, Miss~Granger," sagte Albus Percival Wulfric Brian Dumbledore, "ist unser Aufeinandertreffen ein höchst glücklicher Umstand. Ich habe mich nämlich gerade in diesem Moment neugierig gefragt, was Sie drei sich wohl als ihre Wünsche haben einfallen lassen…"

\later

Der Samstag erstrahlte hell und klar am Horizont und die Schüler hörte man nur in verhaltenem Flüsterton sprechen, als könne das Pulverfass beim ersten lauten Geräusch in die Luft gehen.

\later

Draco hatte gehofft, sie würden erneut in den oberen Stockwerken von Hogwarts kämpfen. Professor Quirrell hatte gemeint, dass Kämpfe im echten Leben viel eher in Städten stattfänden als in Wäldern, was durch das Kämpfen in den Räumen und Korridoren der Schule simuliert werden sollte, wobei Bänder das Schlachtfeld begrenzten. Die Drachen-Armee hatte sich in jenen Schlachten gut geschlagen.

Stattdessen hatte sich Professor Quirrell, genau wie Draco befürchtet hatte, für diese Schlacht etwas \emph{besonderes} einfallen lassen.

Das Schlachtfeld war der See von Hogwarts.

Und sie kämpften auch nicht in Booten.

Sondern \emph{unter Wasser.}

Der Riesenkrake war vorübergehend betäubt; es waren Zauber eingesetzt worden, um die Grindelohs fernzuhalten; Professor McGonagall hatte mit den Wassermenschen gesprochen und an alle Schüler waren Zaubertränke ausgegeben worden, mit denen sie unter Wasser agieren konnten und die ihnen erlaubten zu atmen, klar zu sehen, miteinander zu sprechen und nur wenig langsamer zu schwimmen, als es ein schneller Lauf an Land gestattet hätte, indem sie sich mit den Füßen abstießen.

Eine gewaltige silberne Sphäre schwebte in der Mitte des Schlachtfeldes und erschien wie ein kleiner Unterwasser-Mond. Er würde ihnen dabei helfen, einen Sinn für die Richtung zu bewahren - zunächst. Der Mond würde im Verlauf der Schlacht langsam abnehmen und sobald er sich vollständig verdunkelt hätte, würde die Schlacht vorüber sein, falls sie nicht bereits vorher endete.

Krieg im Wasser. Man konnte keine Verteidigungslinie bilden, Angreifer konnten sich aus jeder Richtung nähern und selbst mit dem Zaubertrank konnte man in der Finsternis des Sees nicht allzu weit sehen.

Und entfernte man sich zu weit vom Ort des Geschehens, so würde man nach einer Weile zu glühen beginnen und sehr einfach aufzuspüren sein - für gewöhnlich würde Professor Quirrell eine Armee, die sich zerstreute und floh, anstatt zu kämpfen, einfach für besiegt erklären; doch heute arbeiteten sie mit einem Punktesystem. Natürlich bekam man noch immer etwas Zeit, \emph{bevor} man zu leuchten begann, falls man etwa Attentäter spielen wollte.

Die Drachen-Armee war zu Beginn des Spiels im tiefen Wasser platziert worden; hoch oben und weit entfernt schien der ferne Unterwasser-Mond. Das trübe Wasser wurde größtenteils von \emph{Lumos}-Zaubern erhellt, doch seine Soldaten würden natürlich die Lichter löschen, sobald sie ihre Manöver begannen. Es ergab keinen Sinn, sich dem Gegner zu erkennen zu geben, bevor man selbst ihn sah.

Draco trat ein paar mal mit den Füßen und schob sich in eine höhere Position, von der er auf seine im Wasser schwebenden Soldaten herab schauen konnte.

Unter Dracos eisigem Blick erstarben die Unterhaltungen fast augenblicklich und seine Soldaten blickten zu ihm auf mit einer erfreulichen Mischung aus Besorgnis und Furcht.

"Hört mir jetzt sehr genau zu," sagte General Malfoy. Seine Stimme erklang ein wenig tiefer und blubbernd vor Blasen, \emph{hörd mir jedwd} \emph{wehr genau su}, doch der Klang wurde deutlich weiter getragen. "Wir können das hier nur auf eine Art gewinnen. Wir müssen uns mit Chaos gegen Sunshine verbünden und Sunshine besiegen. \emph{Dann} tragen wir die Sache mit Potter aus und gewinnen. Das muss \emph{unbedingt} geschehen, verstanden? Egal was sonst noch vor sich geht, dieser Teil \emph{muss} genau so passieren -"

Und Draco erklärte den Plan, den er und Harry sich hatten einfallen lassen.

Die Soldaten tauschten erstaunte Blicke aus.

"- und sollte irgendeiner \emph{eurer} Pläne dem in die Quere kommen," schloss Draco, "dann werde ich den Verantwortlichen, nachdem wir aus dem Wasser sind, in \emph{Brand} stecken."

Ein nervöser Chor von Jawolls erklang.

"Und an all diejenigen, die geheime Befehle erhalten haben, seht zu, dass ihr sie \emph{buchstabengetreu} ausführt," sagte Draco.

Etwa die Hälfte seiner Soldaten \emph{nickte ganz offen} und Draco merkte sie zur Exekution vor, nachdem er an die Macht käme.

Natürlich waren all diese persönlichen Befehle völliger Unsinn, so sollte etwa einer der Drachen einen Auftrag von einem falschen Verräter an einen anderen Drachen übermitteln und dem zweiten Drachen wurde heimlich im Vertrauen aufgetragen, alles zu berichten, was der erste Drache sagte. Draco hatte jedem Drachen erzählt, der gesamte Krieg könne von dieser einen Sache abhängen und er hoffe, sie verstünden, dass dies wichtiger sei als ihre bereits bestehenden Pläne. Mit etwas Glück würde das all die Dummköpfe bei Laune halten und vielleicht obendrein noch an paar Spione offenbaren, wenn die Berichte nicht den Anweisungen entsprachen.

Dracos wirklicher Plan um gegen Chaos zu gewinnen… nun, er war einfacher als der, den er verbrannt hatte, doch Vater hätte er noch immer nicht gefallen. Allerdings, so sehr er sich bemüht hatte, war Draco nichts besseres eingefallen. Es war ein Plan der \emph{unmöglich} gegen irgendjemand anderen außer Harry Potter hätte funktionieren können. Eigentlich war es, dem Verräter zufolge, ursprünglich sogar Harrys Plan \emph{gewesen,} obwohl Draco das ohnehin schon vermutet hatte. Draco und der Verräter hatten ihn nur ein wenig modifiziert…

\later

Harry nahm einen tiefen Atemzug, fühlte wie das Wasser harmlos in seinen Lungen plätscherte.

Sie hatten im Wald gekämpft und er hatte nicht die Chance gehabt, es zu sagen.

Sie hatten in den Korridoren von Hogwarts gekämpft und er hatte nicht die Chance gehabt, es zu sagen.

Sie hatten in der Luft gekämpft, wo jedem Soldaten ein Besen zugeteilt wurde und noch immer hatte es nicht ganz gepasst.

Harry hatte schon geglaubt, er würde nie die Chance erhalten, jene Worte zu sagen, nicht solange er noch jung genug war, dass es sich echt anfühlte…

Die Chaos-Legionäre blickten Harry voller Verwirrung an, während ihr General mit den Füßen nach oben gerichtet schwamm, hin zum fernen Licht der Oberfläche und mit dem Kopf nach unten in Richtung der trüben Tiefen.

"\emph{Warum steht ihr} \emph{alle} \emph{auf dem Kopf?}" schalt der junge Kommandeur seine Armee und begann zu erklären, wie man kämpfte, nachdem man sich frei gemacht hatte von seiner bevorzugten Ausrichtung der Schwerkraft.

\later

Ein dumpfer, dröhnender Gong hallte durch das Wasser und augenblicklich tauchten Zabini und Anthony mit fünf weiteren Soldaten hinab in die trüben Tiefen des Sees. Parvati Patil, die einzige Gryffindor in der Gruppe, wandte einen Moment lang den Kopf und winkte ihnen allen fröhlich nach, während sie hinab tauchte und einen Moment später taten Scott und Matt dasselbe. Die anderen versanken einfach und verschwanden.

General Granger schluckte einen Kloß in ihrem Hals, als sie sie ziehen sah. Sie riskierte alles dafür, teilte ihre Armee auf, anstatt einfach zu versuchen, so viele feindliche Soldaten mit sich zu nehmen, wie möglich.

Was Ihnen klar sein muss, hatte Zabini ihr gesagt, ist dass keine Armee sich rühren würde, bis sie nicht einen Plan hatten, nach dem sie den Sieg erwarten durften. Sunshine konnte nicht einfach selbst einen Plan machen, um zu gewinnen, sie mussten die beiden anderen Armeen \emph{glauben} machen, sie würden gewinnen, bis es zu spät war.

Ernie und Ron wirkten noch immer geschockt. Susan blickte den verschwindenden Soldaten mit berechnendem Blick hinterher. Ihre Armee, oder was davon übrig war, wirkte einfach nur verwirrt; Ornamente aus Licht sprenkelten ihre Uniformen, wie sie dort knapp unter der sonnenbeschienenen Seeoberfläche trieben.

"Was \emph{jetzt?}" sagte Ron.

"Jetzt warten wir," sagte Hermine, laut genug, dass es alle Soldaten mitbekamen. Es fühlte sich eigenartig an, mit dem Mund voller Wasser zu reden, sie hatte noch immer das Gefühl, irgendeine furchtbare Unhöflichkeit bei Tisch zu begehen und sich jeden Moment von oben bis unten voll zu sabbern. "Uns alle, die wir hier bleiben, wird es erwischen, aber das wäre sowieso passiert, wo Drachen und Chaos sich gegen uns zusammentun. Wir müssen nur so viele von ihnen mitnehmen, wie wir können."

"Ich habe einen Plan," sagte eine der Sunshine-Soldaten… Hannah, ihre Stimme war zunächst schwer zu erkennen gewesen. "Er ist, na ja, ziemlich kompliziert, aber ich weiß, wie wir die Drachen und Chaos dazu bringen, gegeneinander zu kämpfen -"

"Ich auch!" sagte Fay. "Ich habe auch einen Plan! Wisst ihr, Neville Longbottom ist insgeheim auf unserer Seite -"

"\emph{Du} hast mit Neville geredet?" sagte Ernie. "Das stimmt nicht, \emph{ich} war doch derjenige, der -"

Daphne Greengrass und ein paar der anderen Slytherins, die nicht mit Zabini gegangen waren, konnten sich vor Kichern kaum halten, als die zahlreichen Rufe "Nein, wartet, \emph{ich} war der, der Longbottom rekrutiert hat" nach und nach aus den Reihen der Soldaten erschallten.

Aus Hermines Blick sprach nur totale Erschöpfung.

"In Ordnung," sagte Hermine, sobald sie alle verstummt waren, "haben wir es jetzt alle verstanden? Alle eure Pläne waren von Chaos erfunden oder manche auch von den Drachen. Jeder, der Harry oder Malfoy \emph{wirklich} verraten wollte, ist direkt zu mir oder Zabini gekommen, nicht zu euch. Vergleicht einfach all eure Geheimpläne, dann seht ihr es selbst." Was Verschwörungen anging, mochte sie nicht so gut sein wie Zabini, doch sie verstand immer, was all ihre Offiziere ihr mitteilten, deshalb hatte Professor Quirrell sie zum General gemacht. "Also haltet euch nicht mit irgendwelchen Plänen auf, wenn die anderen Armeen hier eintreffen. Kämpft einfach, okay? Bitte?"

"Aber," sagte Ernie und wirkte schockiert, "Neville ist in \emph{Hufflepuff!} Willst du sagen, er hat uns \emph{angelogen?}"

Daphne lachte so laut und so hemmungslos, dass die entweichende Luft sie im Wasser kopfüber herumgewirbelt hatte.

"Ich bin nicht sicher, \emph{was} Longbottom ist," sagte Ron finster, "aber ich glaube nicht, dass er noch ein Hufflepuff ist. Nicht mehr, seit \emph{Harry Potter} ihn in die Finger bekommen hat."

"Wisst ihr," sagte Susan, "dass ich ihn danach \emph{gefragt} habe und Neville meinte, er sei jetzt ein Chaos-Hufflepuff?"

"Jedenfalls," sagte Hermine laut. "hat Zabini jeden mitgenommen, den wir für einen Spion hielten, also müssen wir uns in \emph{unserer} Armee jetzt nicht mehr allzu sehr über die Schulter schauen, hoffe ich."

"\emph{Anthony} war ein Spion?" rief Ron.

"\emph{Parvati} war eine Spionin?" keuchte Hannah.

"Parvati war die \emph{totale} Spionin," sagte Daphne. "Sie hat in einem Spion-Schuhladen eingekauft und Spion-Lippenstift getragen und eines Tages wird sie einen netten Spion-Ehemann heiraten und viele kleine Spione kriegen."

Dann hallte ein Gong durch das Wasser und verkündete, dass Sunshine soeben zwei Punkte gemacht hatte.

Kurz darauf folgte ein dreifacher Gong; die Drachen hatten einen Punkt verloren.

Verrätern war es nicht erlaubt, Generäle zu töten, nicht nach dem Desaster der ersten Schlacht im Dezember, bei der alle drei Generäle in der ersten Minute erschossen worden waren. Doch mit etwas Glück…

"Oooh," sagte Hermine. "Klingt als macht Mr~Crabbe ein kleines Nickerchen."

\later

Wie zwei Fischschwärme schwammen die Armeen nebeneinander her.

Neville Longbottom trat mit langen, fließenden Bewegungen mit den Füßen aus. Tauchend, immer tauchend, egal in welche Richtung man sich auch bewegte. Man wollte dem Feind immer das kleinstmögliche Profil präsentieren, ihm den Kopf oder die Füße zuwenden. Also tauchte man immer hinab und mit dem Kopf zuerst und der Feind war immer \emph{unten.}

Wie bei jedem anderen Chaos-Legionär in der Armee rotierte Nevilles Kopf immerzu, während er schwamm, blickte nach oben, nach unten, umher, zu jeder Seite. Nicht nur Ausschau haltend nach Sunshine-Soldaten, sondern auch nach jeglichem Anzeichen, dass ein Chaos-Legionär seinen Zauberstab gezogen hatte und im Begriff war, sie zu verraten. Üblicherweise warteten Verräter die Wirren des Gefechts ab, um zuzuschlagen, doch dieser frühe Gong hatte sie alle in Alarmbereitschaft versetzt.

… um ehrlich zu sein, machte es Neville traurig. Im November war er Soldat gewesen, in einer geeinten Armee, sie alle zogen am gleichen Strang und halfen einander und nun suchten sie alle bei einander fortwährend nach den ersten Anzeichen des Verrats. General Chaos mochte es so vielleicht mehr Spaß machen, doch es war kein annähernd so großer Spaß für Neville.

Die früher als 'oben' bekannte Richtung wurde stetig heller, als sie der Oberfläche und Sunshine näher kamen.

"Zauberstäbe raus," sagte General Chaos.

Nevilles Trupp zog die Zauberstäbe, richtete sie direkt voraus auf den Feind, während ihre Köpfe alles noch schneller absuchten. Wenn es unter ihnen Sunny-Verräter gab, so rückte für sie die Zeit näher,zuzuschlagen.

Der andere Fischschwarm, die Drachen-Armee, tat dasselbe.

"\emph{Jetzt!}" rief die ferne Stimme des Drachen-Generals.

"\emph{Jetzt!}" rief General Chaos.

"\emph{Für Sunshine!}" riefen alle Soldaten in beiden Armeen und stürmten nach unten.

\later

"\emph{Was?}" entfuhr es Minerva unwillkürlich, die die Schirme von ihrem Platz nahe dem See beobachtete, ein Aufschrei der an vielen anderen Orten widerhallte; ganz Hogwarts beobachtete diese Schlacht, so wie schon die erste.

Professor Quirrell lachte trocken. "Ich habe Sie gewarnt, Schulleiter. Es ist unmöglich, Regeln aufstellen, in denen Mr~Potter keine Lücke findet."

\later

Für einige lange, kostbare Sekunden, während die siebenundvierzig Soldaten auf ihre eigenen siebzehn zustürmten, war Hermines Geist wie leergefegt.

Wieso…

Dann ergab mit einem mal alles Sinn.

Jedes mal wenn einer der ursprünglichen Sunshine-Soldaten von jemandem im Namen Sunshines erschossen wurde, würde sie einen Quirrell-Punkt verlieren. Wenn eine der beiden Armeen zwei Sunshine-Soldaten erschoss, wären \emph{beide} feindlichen Armeen zwei Punkte näher daran, sie zu überholen, es war der selbe Gewinn, nur \emph{teilten} sie ihn. Und erschoss irgendwer einen anderen Soldaten \emph{nicht} im Namen von Sunshine, so würde dieser Gong selbst in solcher Verwirrung \emph{nicht} unbemerkt bleiben…

Plötzlich war Hermine sehr froh darüber, dass Zabini nicht dem offensichtlichen Plan gefolgt war, Unfrieden zwischen den beiden anderen Armeen zu stiften, während sie Sunshine attackierten.

Dennoch, es war entmutigend, dieses Gefühl, wenn die Chancen schwanden, einem die Hoffnung entrissen wurde.

Die meisten von Hermines Soldaten wirkten noch immer verwirrt, doch auf einigen Gesichtern zeigte sich bereits ein erster Anflug des Entsetzens, als sie begriffen.

"Wir kriegen das hin," sagte Susan Bones mit fester Stimme. Köpfe drehten sich und blickten auf den Sunshine-Captain. "Unser Job bleibt der gleiche, so viele von ihnen mitzunehmen, wie wir können. Und denkt daran, Zabini hat alle Spione mitgenommen! Wir müssen nicht auf der Hut sein, so wie \emph{sie!}" Das Mädchen lächelte trotzig, beschwor damit zustimmendes Nicken von vielen der anderen Soldaten herauf, selbst von Hermine. "Es kann wieder so sein, wie im November. Wir müssen uns nur behaupten, im Kampf unser Bestes geben und einander vertrauen -"

Daphne erschoss sie.

\later

"\emph{Blut für den Blutgott!}" kreischte Neville von Chaos, doch da er unter Wasser war, klang es eher nach 'Blubblt für den Blubbltglott!'

Captain Weasley schnellte herum, richtete den Zauberstab auf Neville und feuerte. Doch Neville schwamm \emph{hinunter} auf ihn zu, den Zauberstab direkt voraus gerichtet und das bedeutete, der Simple Schild konnte Nevilles gesamtes Profil schützen; sollte ihn jetzt irgendjemand erschießen, so würde es nicht Sunny Ron sein.

Ein grimmiger Ausdruck der Entschlossenheit trat auf Captain Weasleys Gesicht und er hielt pfeilgerade auf Neville zu, sein Mund bildete das Wort \emph{Contego,} obwohl der Schuld im Wasser nicht zu sehen war.

Die zwei verfeindeten Champions schossen aufeinander zu, wie von der Sehne geschnellte Pfeile, ein jeder entschlossen, den anderen in der Mitte zu spalten. Sie hatten sich schon viele Male zuvor duelliert, doch dieses Mal würde alles entscheiden.

(Weit entfernt am Seeufer hielten einhundert Kehlen den Atem an.)

"\emph{Regenbögen und Einhörner!}" brüllte der Sunshine-Captain.

"\emph{Die Schwarze Ziege mit eintausend Jungen!}"

"\emph{Mach deine Hausaufgaben!}"

Näher und immer näher stürmten die zwei Champions aufeinander zu, keiner willens auszuweichen; der erste der abdrehte, würde eine verwundbare Breitseite bieten und erschossen werden, doch wenn keiner die Nerven verlor, würden sie direkt ineinander krachen…

Er fiel geradewegs nach unten, während der Feind hinauf stieg, ihn zu treffen, Hammer senkte sich auf Amboss herab, auf einem Pfad, von dem keiner zu weichen gewillt war…

"\emph{Spezialattacke, Chaotischer} \emph{Wirbel!}"

Neville sah den Ausdruck des Entsetztens auf Captain Weasleys Gesicht, als der Schwebezauber ihn erfasste. Vor Beginn der Schlacht hatten sie es getestet und genau wie Harry vermutet hatte, wurde \emph{Wingardium Leviosa} zu einer völlig neuen Art von Waffe, sobald sie unter Wasser schwammen.

"\emph{Verflucht} \emph{nochmal, Longbottom!}" kreischte Ron Weasley, "\emph{Kannst du nicht einmal} \emph{kämpfen,} \emph{ohne deine dämlichen Spezialattacken -}"

und in dem Moment war der Sunshine-Captain bereits seitlich herumgewirbelt worden und Neville schoss ihm ins Bein.

"Ich kämpfe nicht fair," sagte Neville der schlafenden Gestalt, "ich kämpfe wie Harry Potter."

\later

Granger: 237 / Malfoy: 217 / Potter: 220

Es schmerzte ihn noch immer jedes mal, wenn er Hermine erschießen musste. Harry konnte den Anblick kaum ertragen, den friedvollen Ausdrucks, der sich auf ihr schlafendes Gesicht gelegt hatte, die nun ziellos dahin treibenden Arme und die Sonnenstrahlen, die über ihre Tarnuniform und die Wolke ihres kastanienbraunen Haares glitten.

Und hätte Harry sich davor gedrückt, derjenige zu sein, der sie erschoss… nicht nur hätte Draco gewusst, was es bedeutete, \emph{Hermine} wäre beleidigt gewesen.

\emph{Sie ist nicht tot,} schalt Harry seinHirn, während seine Fußtritte ihn fort trugen, \emph{sie schläft nur. DUMMKOPF.}

\emph{Bist du sicher?} sagte sein Hirn. \emph{Was wenn sie jetzt eine Ex-Hermine ist?}*** \emph{Könnten wir zurück und nachsehen?}

Harry warf einen flüchtigen Blick zurück.

\emph{Siehst du, ihr geht's gut, da kommen Blasen aus ihrem Mund.}

\emph{Könnte ihr letzter Atemzug sein, der entweicht.}

\emph{Ach, halt die Klappe. Was bist du überhaupt so überfürsorglich-paranoid?}

\emph{Ähm, erste echte Freundin, die wir in unserem ganzen Leben hatten? Hey, weißt du noch, was mit unserem Haus-Stein passiert ist?}

\emph{Gibst du wohl endlich RUHE wegen diesem wertlosen Stückchen Schutt, es war nicht mal lebendig, geschweige denn empfindungsfähig, das ist so ungefähr das armseligste Kindheitstrauma aller Zeiten -}

Die beiden Armeen trennten sich eilig und wurden wieder zu zwei separaten Fischschwärmen.

General Granger war siebzehn Punkte zurückgefallen und hatte drei Chaoten und zwei Drachen mit sich genommen; ein weiterer Chaot und zwei Drachen waren als Verräter erschossen worden. Also hatte sie netto sieben Punkte verloren, Harry einen und Draco hatte zwei verloren; damit war Sunshine den Drachen zwanzig Punkte voraus und Chaos um siebzehn Punkte. Chaos konnte noch immer mit Leichtigkeit gewinnen, wenn sie alle zwanzig verbleibenden Drachen ausschalteten. Blieben natürlich noch jene verbleibenden sieben Sunshine-Soldaten…

… wenn man sie so nennen konnte.

Die zwei Fischschwärme schwammen unruhig nebeneinander, die Soldaten jeder Armee nur den Befehl erwartend, ihre wahre Zugehörigkeit zu verkünden und anzugreifen…

"An alle, die sie bekommen haben," sagte Harry laut, "denkt an die Spezialbefehle Eins bis Drei. Und vergesst nicht, für Nummer Drei gilt Merlin Sagt. Nicht bestätigen."

Die vertrauenswürdigen Zweidrittel der Armee nickten nicht und das andere Drittel wirkte lediglich verwirrt.

\emph{Spezialbefehl Eins: Haltet euch in dieser Schlacht nicht mit irgendwelchen Codewörtern auf, verschwendet keinen Aufwand auf irgendwelche Pläne, die nicht explizit vom Commander autorisiert wurden; einfach schwimmen, schützen und feuern.}

Hermine und Draco hatten beide ihre Soldaten bekämpft und den ganzen Dezember lang versucht, sie vom Schmieden eigener Pläne abzuhalten. Harry hatte seine Soldaten darin bestärkt und ihre Verschwörungen während der letzten zwei Schlachten unterstützt… während er ihnen gleichzeitig vermittelt hatte, er müsse sie vielleichtirgendwann \emph{zukünftig}bitten, einen oder zwei Pläne ihrer Pläne auf Eis zu legen, woraufhin sie dem Anliegen freudig nachkamen. Daher würden sie jetzt, in dieser kritischen Schlacht, bereitwillig gehorchen.

Weder Hermine noch Draco hättendiesen Befehlmit Aussicht auf Erfolg erteilen können, da war sich Harry sicher. Es machte den Unterschied, ob deine Soldaten dich als Mitverschwörer sahen oder als einen griesgrämigen alten Spielverderber, der ihnen keinen Spaß gönnte. Das Aufrechterhalten der Ordnung war gleichbedeutend mit der Eskalation des Chaos und umgekehrt ebenso…

"Da sind sie!" rief jemand und deutete auf sie.

Aus der Tiefe des Sees stiegen die Vergessenen, diejenigen denen die letzte Schlacht versagt geblieben war, die sieben verschollenen Sunshine-Soldaten, leuchtend mit der Aura von Feiglingen, die nun verblasste, da sie zur Schlacht zurückkehrten.

Ein Flattern durchzog die zwei Fischschwärme, die Zauberstäbe wankten.

"Nicht feuern!" rief Harry und ein ähnlicher Ruferklang von General Malfoy.

Für einen Moment hielten alle den Atem an.

Dann schlossen sich die sieben Sunshine-Soldaten der Drachen-Armee an.

Ein triumphaler Jubel erklang von der Drachen-Armee.

Schreie des Entsetzens von einem Drittel der Chaos-Legion.

Einige der anderen Zweidrittel lächelten, obwohl es nicht der Fall sein sollte.

Harry lächelte nicht.

\emph{Oh, das wird sowas von nicht funktionieren…}

Aber Harry war einfach nichts besseres eingefallen.

"Spezialbefehle Zwei und Drei immer noch in Kraft!" rief Harry. "Kämpft!"

"\emph{Für die Chaos-Legion!}" brüllten die zwanzig Chaos-Legionäre.

"\emph{Für die Drachen-Armee!}" brüllten zwanzig Drachen-Krieger und sieben Sunshine-Soldaten.

Und die Chaoten tauchten direkt nach unten, als sich all die Verräter zum Angriff bereit machten.

\later

Granger: 237 / Malfoy: 220 / Potter: 226

Dracos Kopf fuhr wie rasend herum, versuchte zu erfassen, was geschah; irgendwie, trotz seiner größeren Streitkräfte, hatte er \emph{die Initiative verloren.} Vier kleine Streitkräfte der Chaoten wurden von vier größeren der Drachen verfolgt, doch da Dracos Streitkräfte eine Konfrontation zu erzwingen versuchten, mussten sie dorthin \emph{folgen,} wohin Chaos \emph{floh} und irgendwie entstanden dadurch Ansammlungen Chaotischer Truppen, die in die ungeschützten Flanken der Drachen feuerten -

Es passierte \emph{schon wieder!}

"\emph{Prismatis!}" rief Draco, erhob seinen Zauberstab und diesen Schild konnte man sogar im Wasser ausmachen, eine funkelnde mehrfarbige Fläche, breit genug um Draco und die anderen fünf Drachen an seiner Seite vor den Chaotischen Streitkräften zu schützen, die gerade im Vorbeischwimmen das Feuer auf sie eröffnet hatten und \emph{das}erlaubte den anderen fünf Drachen, \emph{ihre} Aufmerksamkeit wieder der Chaotischen Streitmachtzuzuwenden, die sie verfolgt hatten -

Es entstand ein angespannter Moment, als Schlafzauber um Schlafzauber in Dracos Prismatische Barriere einschlug und Draco hoffte bei Merlin, dass keiner dieser vier Chaoten den Schlagbohrfluch gelernt hatte -

Dann ertönte der Gong für einen Sieg der Drachen und die Chaotische Streitmacht trat Hals über Kopf den Rückzug an und machten sich schwimmend davon; worauf Draco, nun mit leicht zitternden Händen, die Prismatische Barriere und den Zauberstab sinken ließ.

Im Wasser zu kämpfen, war sogar noch anstrengender als auf Besenstielen.

"\emph{Keine Verfolgung aufnehmen!}" rief Draco seinen Soldaten zu, die ansetzten, ihnen nachzujagen; dann, "\emph{Sonorus! SAMMELT EUCH UM MICH!}"

Die Drachen-Streitkräfte strömten um Draco herum zusammen und die Kräfte der Chaoten wirbelten herum und begannen augenblicklich, die Drachen zu \emph{verfolgen} - Draco fluchte laut als er den Klang eines Chaotischen Sieges vernahm, jemand hatte seinen Simplen Schild nicht richtig ausgerichtet - und dann gelangten die Drachen-Kräfte in Reichweite, sich gegenseitig zu unterstützen und die Chaoten zogen sich in die ferne Dunkelheit zurück.

Irgendwie, trotz ihrer zahlenmäßigen Überlegenheit, hatten die Drachen drei Treffer gegen die Chaoten erzielt und die Chaoten im Gegenzug vier und er hatte vernommen, wie ein Drachen-Spion exekutiert wurde. Entweder hatte Harry Potter sich sehr schnell verdammt viele wirklich gute Ideen einfallen lassen oder er hatte sich aus irgendeinem unvorstellbaren Grund bereits sehr viele Gedanken darüber gemacht, wie man unter Wasser kämpfte. Das funktionierte nicht, also musste Draco seine Strategie überdenken.

Es schien auch, dass alle Probleme hatten beim Schwimmen zu spielen und die Schlacht mochte lange genug dauern, dass ihnen die Zeit davon lief… der ferne Unterwasser-Mond war nur noch halb voll, das war nicht gut… er musste sich \emph{schnell} etwas neues einfallen lassen…

"Wie lauten die Befehle?" sagte Padma Patil, die zu Draco hinüber schwamm.

Padma war sein Erster Offizier; sie war schlau und stark und was noch besser war, sie hasste Granger und betrachtete Harry als Rivalen, was sie \emph{vertrauenswürdig} machte. Mit Padma zu arbeiten, rief ihm die alte Redensart wieder ins Gedächtnis, Ravenclaw sei Slytherins Schwester; Draco war überrascht gewesen, von seinem Vater zu hören, es sei akzeptables Haus für seine zukünftige Braut, doch nun erkannte er den Sinn darin.

"Wartet, bis alle versammelt sind," sagte Draco. In Wahrheit brauchte er eine kleine Verschnaufpause. Das war das Problem, wenn man der General \emph{und} der mächtigste Zauberer war, man musste immerzu Magie einsetzen.

Zabini traf als nächster ein, der eine Streitmacht aus zwei Sunnies und vier Drachen anführte, unter ihnen Gregory, der ein Auge auf Zabini haben sollte. Draco traute Zabini nicht. Und weder Draco noch Zabini trauten den Sunnies genug, um sie in irgendeiner Einheit die Mehrheit stellen zu lassen; sie \emph{sollten} entweder direkt loyal zu Draco stehen oder zu Granger, die mit dem Versprechen getäuscht worden war, dass die Drachen am Ende betrogen würden, nachdem beide Streitkräfte dezimiert worden waren, so wie auch Harrys vertrauenswürdigere Chaoten getäuscht worden sein sollten, nicht auf die Sunnies zu schießen, durch das Versprechen, diese würden falsche Schlafflüche abfeuern und später Chaos unterstützen; doch es bestand die Möglichkeit, dass einige der Sunnies tatsächlich loyal zu Chaos \emph{standen} und \emph{keine} echten Schlafflüche feuerten, was es den Drachen nicht gestattete, so zu gewinnen, wie ihre zahlenmäßige Überlegenheit es ihnen hätte erlauben sollen…

Die nächste eintreffende Einheit war dezimiert, drei Soldaten hielten die Zauberstäbe auf zwei weitere gerichtet, die mit leeren Händen schwammen.

Draco biss die Zähne zusammen. Noch mehr Probleme mit Verrätern. Er musste mit Professor Quirrell darüber sprechen, Verräter zumindest irgendwie \emph{bestrafen} zu können, solche Verhältnisse waren \emph{unrealistisch,} im wahren Leben folterte man seine Verräter zu Tode.

"General Malfoy!" rief der Kommandeur der problematischen Einheit als sie näher schwammen, ein Ravenclaw namens Terry. "Wir wissen nicht, was wir machen sollen - Cesi hat Bogdan erschossen, aber Cesi sagt, Kellah hätte ihm erzählt, Bogdan habe Specter erschossen -"

"Hab ich \emph{nicht!}" sagte Kellah.

"Doch \emph{hast du!}" kreischte Cesi. "General Malfoy, \emph{sie} ist der Spion, ich hätte es wirk-"

"\emph{Somnium,}" sagte Draco.

Ein dreifacher Gong kündigte den Verlust eines Punktes für die Drachen an, dann begann Kellahs schlaffer Körper im Wasser davon zu treiben.

Inzwischen \emph{hatte}Draco bereits Bekanntschaft mit dem Wort 'Rekursion' gemacht und er erkannte einen Harry-Potter-Plan, wenn er einen sah.

(Unglücklicherweise hatte Draco noch \emph{nichts} von Autoimmunerkrankungen gehört und ihm kam \emph{nicht}mehr rechtzeitig der Gedanke, dass ein cleverer Virus seinen Angriff damit beginnen würde, Symptome einer Autoimmunerkrankung zu erzeugen, damit der Körper seinem eigenen Immunsystem zu misstrauen begann…)

"\emph{Allgemeiner Befehl!}" sagte Draco und hob die Stimme. "Niemand erschießt irgendwelche Verräter außer mir selbst, Gregory, Padma und Terry. Wenn irgendwem irgendetwas verdächtiges auffällt, kommt er zu einem von \emph{uns.}"

Und dann -

Ein Gong verkündete, dass Sunshine zwei Punkte gemacht hatte.

"\emph{Was?}" sagten Draco und Zabini fast zeitgleich; die Köpfe fuhren herum. Niemand schien getroffen worden zu sein und alle Sunshine-Soldaten waren anwesend und erfasst. (Außer Parvati, die von einem noch immer unbekannten Verräter in Padmas Trupp erschossen worden war und natürlich hatte Padma erneut auf Parvati geschossen, falls sie es nur vortäuschte, also konnte sie es nicht sein…)

"Ein Sunny-Verräter bei Chaos?" sagte Zabini verwirrt. "Aber alle von denen ich wusste, sollten während Chaos Angriff auf Sunshine losschlagen -"

"Nein!" sagte Padma im Tonfall plötzlichen Verstehens. "Das war \emph{Chaos,}sie haben einen Verräter exekutiert!"

"\emph{Was?}" sagte Zabini. "Aber warum -"

Und Draco begriff. \emph{Verdammt!} "Weil Potter denkt, sein Vorsprung vor Sunshine sei sicher, aber er ist nicht sicher, wie hoch er \emph{uns} schlagen kann! Also will er keinen Punkt verlieren, wenn er einen Verräter exekutiert! \emph{Allgemeiner Befehl!} Wenn ihr einen Verräter exekutieren müsst, ruft vorher zuerst Sunshine! Und vergesst nicht, danach wieder zu den Drachen zu wechseln -"

\later

Granger: 253 / Malfoy: 252 / Potter: 252

Longbottoms Körper driftete chaotisch, mit verworrenen Armen und Beinen, durchs Wasser. Nachdem Draco ihn endlich erwischt hatte, hatten sie ihn alle \emph{noch einmal}erschossen, nur um sicherzugehen.

In der Nähe blickte sie Harry Potter, nun geschützt von einer Prismatischen Sphäre, allesamt grimmig an, während irgendwo in weiter Ferne der letzte Splitter abnehmenden Mondes langsam verblasste. Wenn Longbottom es nur geschafft hätte, noch einen weiteren Soldaten zu erschießen (so wusste Draco, dass Harry dachte), wenn die Chaoten nur ein wenig länger durchgehalten hätten, dann hätten sie vielleicht \emph{gewonnen…}

Nachdem Draco seine Streitkräfte wieder formiert und einen neuen Angriff gestartet hatte, hatten die folgende Schlacht und die Exekution von Spionen in Sunshines Namen dafür gesorgt, dass Sunshine nun genau einen Punkt sowohl vor den Drachen als auch Chaos lag. Sobald Harry damit angefangen hatte, war Draco keine andere Wahl geblieben als es ihm gleichzutun.

Doch nun waren sie General Chaos zahlenmäßig überlegen, die Überlebenden der Drachen-Armee und der letzte verbleibende Sunny-Verräter: Draco, Padma und Zabini.

Und Draco, der kein Narr war, hatte Padma angewiesen, Zabinis Zauberstab an sich zu nehmen, nachdem Longbottom Gregory erwischt hatte und gleich darauf Draco zum Opfer gefallen war. Der Junge hatte ihn mit einem beleidigten Blick bedacht, Draco gesagt, dafür schulde er ihm was und ihn übergeben.

Blieben also noch Draco und Padma, um General Chaos auszuschalten.

"Ich nehme nicht an, Sie möchten sich ergeben?" sagte Draco und war Harry Potter sein bösartigstes Grinsen zu.

"Schlaf vor Kapitulation!" rief General Chaos.

"Nur damit du's weißt," sagte Draco, "Zabini \emph{hat} gar keine ältere Schwester, die du vor Mobbern aus Gryffindor beschützen könntest. Aber Zabini \emph{hat} eine Mutter, die nicht viel von Muggelgeborenen wie Granger hält und ich habe ihr ein paar Zeilen geschrieben und Zabini ein paar Gefallen angeboten - nichts was meinen Vater betrifft, nur Sachen die \emph{ich}hier in der Schule bewerkstelligen kann. Und nebenbei, Zabinis Mutter hat auch für den Jungen-der-überlebt-hat nicht allzu viel übrig. Nur für den Fall, dass du immer noch dachtest, Zabini sei in Wirklichkeit auf deiner Seite."

Harrys Gesichtsausdruck wurde noch grimmiger.

Draco hob den Zauberstab und begann rhythmisch zu atmen, sammelte Kraft für den Schlagbohrfluch. Grangers Prismatische Sphäre war mittlerweile beinahe so stark, wie die von Draco und Harrys war auch nicht viel schwächer, woher nahmen die beiden bloß die Zeit?

"\emph{Lagann!}" sprach Draco undlegte alles was er hatte in den Zauber, eine grüne Spirale kam lodernd zum Vorschein und Harrys Schild sprang in Scherben und fast im selben Moment -

"\emph{Somnium!}" sagte Padma.

\later

Granger: 253 / Malfoy: 252 / Potter: 254

Harry ließ einen langen Seufzer der Erleichterung vernehmen und nicht nur, weil er die Prismatische Sphäre nicht mehr aufrecht erhalten musste. Seine Hand zitterte, als er den Zauberstab sinken ließ.

"Weißt du," sagte Harry, "einen Moment lang hab ich mir schon Sorgen gemacht."

\emph{Spezialbefehl Zwei: Wenn ein Sunny-Verräter nicht wirklich auf euch zu schießen scheint, gebt gelegentlich vor, getroffen worden zu sein. Nehmt eher Drachen als Sunnies ins Visier, aber schießt auch auf Sunnies, wenn keine Drachen da sind.}

\emph{Spezialbefehl Drei: Merlin sagt, schießt nicht auf Blaise Zabini oder eine der Patil-Zwillinge.}

Mit einem breiten Grinsen streifte Parvati Patil den transfigurierten Flicken vom Zeichen ihrer Uniform und ließ ihn im Wasser davon treiben.

"Gryffindors für Chaos," sagte sie und händigte Zabini seinen Zauberstab wieder aus.

"\emph{Verbindlichsten} Dank," sagte Harry und verbeugte sich schwungvoll vor dem Mädchen aus Gryffindor. "Und danke auch \emph{dir,}" verbeugte er sich zu Zabini. "Weißt du, als du mit diesem Plan zu mir kamst, habe ich mich gefragt, ob du wahnsinnig bist oder brillant und ich habe entschieden, du bist beides. "Und nebenbei," sagte Harry und wandte sich nun zu Dracos Körper um, "Zabini \emph{hat} eine Cousine -"

"\emph{Somnium,}" sagte Zabinis Stimme.

\later

Granger: 255 / Malfoy: 252 / Potter: 254

Und Harry Potters Körper trieb davon, sein Ausdruck von Schock und Entsetzen wich rasch der Entspannung des Schlafes.

"Wenn ich's mir recht überlege," sagte Parvati fröhlich, "wohl doch eher Gryffindors für Sunshine."

Sie brach in Gelächter aus, so begeistert wie noch nie in ihrem Leben, sie hatte es \emph{endlich} geschafft, ihre Zwillingsschwester aus dem Weg zu räumen und ihren Platz einzunehmen und das hatte sie schon \emph{immer} machen wollen und das war einfach \emph{perfekt} gewesen, alles war einfach \emph{perfekt} -

- und dann fuhr ihr Zauberstab in einer blitzschnellen Bewegung herum, gerade als sich Zabinis Zauberstab auf sie richtete.

"Stop!" sagte Zabini. "Nicht schießen und leiste keinen Widerstand. Das ist ein Befehl."

"\emph{Was?}" sagte Parvati.

"Sorry," sagte Zabini und blickte dabei nicht-ganz-aufrichtig entschuldigend drein, "aber ich kann nicht absolut \emph{sicher} sein, dass du auf Sunshines Seite stehst. Also befehle ich dir, dich von mir erschießen zu lassen."

"\emph{Moment mal!}" sagte Parvati. "Wir sind Chaos nur einen Punkt voraus! Wenn du mich jetzt erschießt -"

"Ich werde dich \emph{natürlich} im Namen der Drachen erschießen," sagte Zabini und klang nun ein wenig überheblich. "Nur weil wir \emph{sie} damit hereingelegt haben, heißt nicht dass wir es nicht für uns nutzen können."

Parvati starrte ihn an, ihre Augen verengten sich. "General Malfoy sagte, deine Mutter kann Hermine nicht ausstehen."

"Mag sein," sagte Zabini, noch immer mit diesem überheblichen Grinsen. "Aber einige von uns sind eher willens ein Elternteil zu verärgern als Draco Malfoy."

"Und Harry Potter sagte, du hast eine Cousine -"

"Nope," sagte Zabini.

Parvati starrte ihn an, überlegte fieberhaft, doch sie war nicht wirklich gut, was Verschwörungen anging; Zabini hatte gesagt, der Plan sei, den Punktestand von Chaos und Drachen insgeheim so weit wie möglich anzugleichen, damit sie Sunshines Namen verwendeten, um Verräter zu exekutieren, anstatt auch nur einen Punkt zu verlieren und das hatte \emph{geklappt}… aber… sie hatte das Gefühl, ihr entging hier etwas, sie war ja keine Slytherin…

"Warum erschieße \emph{ich} nicht \emph{dich} im Namen der Drachen?" sagte Parvati.

"Weil ich den höheren Rang habe," sagte Zabini.

Parvati hatte ein mieses Gefühl dabei.

Sie starrte ihn einen langen Moment lang an.

Und dann -

"\emph{Somni-}" setzte sie an, dann wurde ihr klar, dass sie nicht \emph{für Drachen} gesagt hatte und hastig schnitt sie sich das Wort ab -

\later

Granger: 255 / Malfoy: 254 / Potter: 254

"Hey, Leute," sagte das Gesicht von Blaise Zabini auf den Schirmen, sichtlich amüsiert, "schätze, jetzt hängt alles von mir ab."

Am gesamten Seeufer hielten die Leute den Atem an.

Sunshine lag um exakt einen Punkt vor Chaos und den Drachen.

Blaise Zabini konnte sich selbst im Namen von Chaos oder auch der Drachen erschießen oder die Dinge lassen, wie sie waren.

Eine Serie von Gongschlägen deutete an, dass die letzte Minute verstrich.

Und der Slytherin zeigte ein seltsames, schiefes Lächeln und spielte wie geistesabwesend mit seinem Zauberstab, das dunkle Holz kaum zu erkennen im dunklen Wasser.

"Wisst ihr," sagte Blaise Zabinis Stimme, im Tonfall von jemandem, der die Worte schon eine Weile einstudiert hatte, "es ist doch wirklich nur ein Spiel. Und Spiele sollten \emph{Spaß} machen. Wie wär's also, wenn ich einfach mache, wonach mir gerade der Sinn steht?"

* Übersetzung leicht abgewandelt: Aus dem englischen Original geht das jeweilige Geschlecht der beiden nicht hervor; ebenso nicht, ob sie zum Zeitpunkt der Hochzeit noch Schüler waren; nur dass sie den Häusern Gryffindor und Slytherin entstammten.

** engl.: \emph{In every city,the population has been divided for a long time past into the Blue and the Green factions…And they fight against their opponents knowing not for what end they imperil themselves…So there grows up in them against their fellow men a hostility which has no cause, and at no time does it cease or disappear, for it gives place neither to the ties of marriage nor of relationship nor of friendship, and the case is the same even though those who differ with respect to these colours be brothers or any other kin. I, for my part, am unable to call this anything except a disease of the soul…} (aus \emph{History of the Wars, I})

deutsch: \emph{Die Demen in jeder Stadt sind seit alters in Venetoi (Blaue) und Prasinoi (Grüne) gespalten} {[}…{]} \emph{Sogar Schlachten fechten sie mit der Gegenpartei aus, ohne recht zu wissen, warum sie sich in solche Gefahr stürzen} {[}…{]} \emph{So wächst in ihnen ohne wirkliche Veranlassung der Hass gegen den Nächsten, und dieses Gefühl dauert in alle Ewigkeit und lässt weder Verschwägerung noch Blutsverwandtschaft noch das Gesetz der Freundschaft gelten, selbst wenn die nach den Farben getrennten Parteigänger Brüder oder sonst dergleichen sind. Ich kann daher dies nur als eine Seelenkrankheit bezeichnen.}(Übersetzung von Otto Veh aus \emph{Perserkriege}, abgeändert um den Ton des englischen Originals besser zu treffen)

Siehe hierzu auch die Parabel \emph{A Fable of Science and Politics} (\url{https://lesswrong.com/lw/gt/a_fable_of_science_and_politics/}) aus der Sequence \emph{How To Actually Change Your Mind.}

*** Das ist eine Anspielung auf den Sketch \emph{Der Papagei ist tot} aus der Comedy-Show \emph{Monty Python's Flying Circus.}

