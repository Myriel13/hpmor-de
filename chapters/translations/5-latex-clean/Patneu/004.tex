

\hypertarget{die-effiziente-muxe4rkte-hypothese}{% \section{4. Die Effiziente-Märkte-Hypothese}\label{die-effiziente-muxe4rkte-hypothese}}

\textbf{Kapitel 4: Die Effiziente-Märkte-Hypothese}

Haftungsausschluss: J. K. Rowling beobachtet euch, von dort, wo sie auf ewig in der Leere zwischen den Welten wartet.

Anm. d. Autors: Wie einige bemerkten, scheint diese Geschichte inkonsistent zu sein, bezüglich des Werts einer Galleone; Ich greife mir einen einheitlichen Wert heraus und bleibe dabei. Fünf Pfund Sterling die Galleone lässt sich nicht mit sieben Galleonen für einen Zauberstab und Kindern, die gebrauchte Zauberstäbe verwenden, in Einklang bringen.

\later

"\emph{Weltherrschaft ist so ein hässliches Wort. Ich nenne es lieber Weltoptimierung.}" *

\later

Massen goldener Galleonen. Stapel silberner Sickel. Haufen bronzener Knuts.

Harry stand dort und starrte mit offenem Mund auf das Familien-Verlies. Er hatte so viele Fragen, er wusste nicht, wo er \emph{anfangen} sollte.

Von direkt außerhalb des Verlieses beobachtete ihn Pofessor McGonagall, scheinbar locker gegen die Wand gelehnt, aber ihre Augen aufmerksam. Nun, das machte Sinn. Vor einen riesigen Haufen Goldmünzen gesetzt zu werden, war ein so unverkennbarer Charaktertest, dass es schon klischeehaft war.

"Sind diese Münzen aus dem reinen Metall?" sagte Harry schließlich.

"Was?" zischte der Kobold Griphook, der neben der Tür wartete. "Stellen Sie die Integrität von Gringotts in Frage, Mr. Potter-Evans-Verres?"

"Nein," sagte Harry abwesend, "überhaupt nicht, entschuldigen Sie, wenn das falsch rüberkam, Sir. Ich habe nur überhaupt keine Ahnung wie ihr Finanzsystem funktioniert. Ich frage, ob Galleonen im Allgemeinen aus purem Gold bestehen."

"Natürlich," sagte Griphook.

"Und kann jeder sie prägen oder werden sie durch ein Monopol herausgegeben, das dadurch Geldschöpfungsgewinne generiert?"

"Was?" sagte Professor McGonagall.

Griphook grinste und zeigte scharfe Zähne. "Nur ein Narr würde anderen als Kobold-Münzen trauen!"

"Mit anderen Worten," sagte Harry, "sollten die Münzen nicht mehr wert sein, als das Gold aus dem sie gemacht sind?"

Griphook starrte Harry an. Professor McGonagall sah irritiert aus.

"Ich meine, angenommen, ich käme hierher mit einer Tonne Silber. Könnte ich eine Tonne Sickel daraus gemacht bekommen?"

"Gegen eine Gebühr, Mr. Potter-Evans-Verres." Der Kobold sah ihn mit glitzernden Augen an. "Gegen eine bestimmte Gebühr. Wo würden Sie eine Tonne Silber finden, frage ich mich?"

"Das war hypothetisch gesprochen," sagte Harry. \emph{Zumindest für den Moment.} "Also… wie viel würden Sie an Gebühren verlangen, als Anteil am Gesamtgewicht?"

Griphooks Augen waren aufmerksam. "Ich müsste meine Vorgesetzten konsultieren…"

"Geben Sie mir eine ungefähre Schätzung. Ich werde Gringotts nicht darauf festnageln."

"Ein zwanzigster Teil des Metalls würde wohl für die Prägung aufkommen."

Harry nickte. "Vielen Dank, Mr. Griphook."

\emph{Also ist die Zauberer-Wirtschaft nicht nur fast vollkommen von der Muggel-Wirtschaft getrennt, hier hat auch noch niemand etwas von Wechselkurs-Unterschieden gehört.}Die größere Muggel-Wirtschaft hatte eine veränderliche Handelsspanne von Gold zu Silber, also sollten jedesmal, wenn sich das Muggel-Gold-zu-Silber-Verhältnis um mehr als 5\% von dem Verhältnis des Gewichts von siebzehn Sickeln zu einer Galleone entfernt, entweder Gold oder Silber aus der Zauberer-Wirtschaft abgezogen werden, bis es unmöglich wurde den Umtauschkurs aufrechtzuhalten. Man schaffe eine Tonne Silber her, tausche sie in Sickel um (und zahle 5\%), tausche die Sickel in Galleonen um, schaffe das Gold in die Muggel-Welt, tausche es gegen mehr Silber um, als das, mit dem man angefangen hat und wiederhole das.

War das Muggel-Gold-zu-Silber-Verhältnis nicht irgendwo bei etwa fünfzig zu eins? Harry nahm zumindest an, dass es nicht siebzehn war. Und es sah so aus, als wären die Silbermünzen tatsächlich \emph{kleiner} als die Goldmünzen.

Dann wieder stand Harry in einer Bank, die jemandes Geld \emph{buchstäblich} in von Drachen bewachten Verliesen voller Goldmünzen aufbewahrte, in die man hineingehen und Münzen aus seinem Verlies nehmen musste, jedesmal wenn man Geld ausgeben wollte. Ihnen die Feinheiten von Wechsel-Kursunterschieden, ganz zu schweigen von Marktineffizienzen, aufzuzeigen, wäre wohl Verschwendung. Er war versucht gewesen, abfällige Bemerkungen über die Unfertigkeit ihres Finanzsystems zu machen…

\emph{Aber das Traurige ist, ihre Art ist wahrscheinlich besser.}

Auf der anderen Seite könnte einem kompetenten Hedge-Fonder wahrscheinlich innerhalb einer Woche die Zauberwelt gehören. Harry legt diese Idee ab für den Fall, dass ihm je das Geld ausging oder er mal eine Woche frei hatte.

In der Zwischenzeit sollten die riesigen Massen an Goldmünzen im Potter-Verlies für seine kurzfristigen Bedürfnisse ausreichen.

Harry stapfte vorwärts und fing an, mit einer Hand Goldmünzen aufzusammeln und sie in die andere zu stapeln.

Als er bei zwanzig angelangt war, hustete Professor McGonagall. "Ich denke, das wird mehr als ausreichend sein, um Ihre Schulsachen zu bezahlen, Mr. Potter."

"Hm?" sagte Harry, gedanklich woanders. "Warten Sie kurz, ich mache eine Fermi-Kalkulation."

"Eine \emph{was?}" sagte Professor McGonagall und klang irgendwie alarmiert.

"Das ist etwas Mathematisches. Benannt nach Enrico Fermi. Ein Weg um schnell grobe Zahlen im Kopf zu behalten…"

Zwanzig goldene Galleonen wogen vielleicht ungefähr ein zehntel Kilogramm? Und Gold brachte, was, zehntausend Britische Pfund pro Kilogramm? Also wäre eine Galleone etwa fünfzig Pfund wert… Die Hügel aus Goldmünzen sahen etwa sechzig Münzen hoch und zwanzig Münzen breit in jede Grundrichtung aus und ein Hügel war pyramidenförmig, also machte er etwa ein Drittel des Würfels aus. Achttausend Galleonen pro Hügel, in etwa, und es gab etwa fünf Hügel dieser Größe, also vierzigtausend Galleonen oder 2 Millionen Pfund Sterling.

Nicht schlecht. Harry lächelte mit grimmiger Genugtuung. Es war eine Schande, dass er gerade mitten dabei war, die aufregende neue Welt der Magie zu entdecken und sich keine Auszeit nehmen konnte, um die aufregende neue Welt des Reichtums zu erkunden, die, wie ihm eine schnelle Fermi-Abschätzung sagte, ungefähr eine Milliarde mal weniger interessant war.

\emph{Trotzdem ist das das letzte mal, dass ich einen Rasen mähe für ein lausiges Pfund.}

Harry wandte sich von dem riesigen Goldhaufen ab. "Entschuldigen, dass ich frage, Professor McGonagall, aber wie ich das verstehe, waren meine Eltern in ihren Zwanzigern, als sie starben. Ist das eine übliche Menge Geld, die ein junges Paar in der Zauberwelt in ihrem Verlies hat?" Wenn ja, kostete eine Tasse Tee wahrscheinlich fünftausend Pfund. Regel eins der Wirtschaft: Man kann Geld nicht essen.

Professor McGonagall schüttelte den Kopf. "Ihr Vater war der letzte Erbe einer alten Familie, Mr. Potter. Es ist auch möglich…" Die Hexe zögerte. "Manches von diesem Geld könnte aus Kopfgeldern stammen, die auf Sie-wissen-schon-wen ausgesetzt waren, zu zahlen an seinen Ki- äh, an wer auch immer ihn besiegen würde. Oder diese Kopfgelder wurden bisher noch nicht eingetrieben. Ich bin nicht sicher."

"Interessant…" sagte Harry langsam. "Also gehört einiges davon tatsächlich, in gewisser Weise, mir. Soll heißen, ich habe es verdient. Irgendwie. Möglicherweise. Selbst wenn ich mich an die Gelegenheit nicht erinnere." Harrys Finger tippten gegen sein Hosenbein. "Dadurch fühle ich mich weniger schuldig, wenn ich \emph{einen sehr kleinen Teil davonausgebe! Keine Panik, Professor McGonagall!}"

"Mr. Potter! Sie sind ein Minderjähriger und als solcher wird Ihnen erlaubt \emph{vernünftige} Abhebungen zu machen aus -"

"Ich bin \emph{absolut} venünftig! Ich bin vollkommen für finanzielle Voraussicht und Impulskontrolle! Aber ich \emph{habe} auf dem Weg hierher einige Dinge gesehen, die \emph{sinnvolle, erwachsene} Erwerbungen darstellen würden…"

Harry verschränkte den Blick mit Professor McGonagall in einem stillen Anstarr-Wettbewerb.

"Wie was?" sagte Professor McGonagall schließlich.

"Koffer, die von innen größer sind als von außen?"

Professor McGonagalls Gesicht wurde ernst. "Die sind \emph{sehr} teuer, Mr. Potter!"

"Ja, aber -" flehte Harry. "Ich bin sicher, dass, wenn ich erwachsen bin, ich einen wollen werde. Und ich \emph{kann} ihn mir leisten. Logischerweise würde es genau so viel Sinn machen, ihn jetzt zu kaufen, anstatt später und ich kriege den Nutzen sofort. Es ist in jedem Fall das gleiche Geld, richtig? Ich meine, ich \emph{würde} einen guten wollen, mit \emph{viel} Platz drinnen, gut genug, dass ich mir später nicht einfach einen besseren holen müsste…" ließ Harry den Satz hoffnungsvoll ausklingen.

Professor McGonagalls Blick wankte nicht. "Und was genau würden Sie in einem solchen Koffer \emph{aufbewahren}, Mr. Potter -"

"Bücher."

"Natürlich," seufzte Professor McGonagall.

"Sie hätten mir viel früher erzählen sollen, dass diese Art magischer Gegenstand existiert! Und dass ich mir einen leisten kann! Jetzt werden mein Vater und ich die nächsten zwei Tage damit verbringen müssen \emph{hektisch} alle Secondhand-Buchläden nach alten Lehrbüchern abzuklappern, damit ich eine angemesse Wissenschafts-Bibliothek nach Hogwarts mitnehmen kann - und vielleicht eine kleine Science Fiction-Sammlung, wenn ich etwas annehmbares aus den Wühltischen zusammenstellen kann. Oder besser noch, ich werde ihnen das Geschäft ein bisschen versüßen, okay? Ich kaufe nur ein paar -"

"\emph{Mr. Potter!} Denken Sie, Sie können mich \emph{bestechen?}"

"Was? \emph{Nein!} Nicht so! Ich meinte, Hogwarts kann einige der Bücher, die ich mitbringe behalten, wenn Sie denken, dass einige von ihnen gute Ergänzungen für die Bibliothek abgeben würden. Ich werde sie günstig bekommen und \emph{ich} will sie nur irgendwo in meiner Nähe haben. Es ist in Ordnung Leute mit \emph{Büchern} zu bestechen, richtig? Das ist -"

"Familien-Tradition."

"Ja, ganz genau."

Professor McGonagalls Körper schien zusammenzusacken, ihre Schultern senkten sich unter ihrem schwarzen Umhang. "Ich kann den Sinn Ihrer Worte nicht leugnen, obwohl ich sehr wünschte, ich könnte es. Ich werde Ihnen gestatten, weitere hundert Galleonen abzuheben, Mr. Potter." Sie seufzte erneut. "Ich \emph{weiß}, dass ich das bereuen werde und ich tue es bereits."

"Das ist die richtige Einstellung! Und tut ein 'Eselsfell-Beutel'** das, was ich glaube?"

"Es kann nicht so viel, wie ein Koffer," sagte die Hexe mit sichtbarem Widerstreben, "aber… ein Eselsfell-Beutel mit einem Aufrufezauber und Unaufspürbarem Ausdehnungs-Zauber kann eine Anzahl Gegenstände beinhalten, bis sie von demjenigen herbeigerufen werden, der sie dort platziert hat -"

"Ja! Ich brauche definitv auch einen davon! Das wäre wie die Super-Gürteltasche der ultimativen Hammermäßigkeit***! Batmans alles haltender Werkzeuggürtel! Vergessen Sie mein Schweizer Armeemesser, ich könnte ein ganzes Werkzeugset dadrin mitnehmen! Oder \emph{Bücher}! Ich könnten die besten 3 Bücher, die ich gerade lese, jederzeit bei mir haben und einfach irgendwo eins herausholen! Ich muss nie wieder eine Minute meines Lebens verschwenden! Was sagen Sie, Professor McGonagall? Es ist immerhin dafür, dass Kinder lesen, der beste aller möglichen Gründe."

"… Ich nehme an, Sie mögen noch zehn Galleonen hinzufügen."

Griphook bedachte Harry mit einem Blick aufrichtigen Respekts, möglicherweise sogar offener Bewunderung.

"Und ein bisschen Taschengeld, wie Sie vorhin erwähnt haben. Ich denke ich erinnere mich, ein oder zwei Dinge gesehen zu haben, die ich gern in diesem Beutel aufbewahren würde."

"\emph{Treiben Sie es nicht zu weit, Mr. Potter.}"

"Oh, aber Professor McGonagall, warum wollen Sie mir denn die Parade verhageln? Das ist doch sicher ein \emph{glücklicher} Tag, wenn ich all die Zaubersachen zum ersten mal entdecke! Warum den Part der grummeligen Erwachsenen spielen, wenn Sie stattdessen lächeln und sich an Ihre eigene unschuldige Kindheit erinnern können, während Sie den freudigen Ausdruck auf meinem jungen Gesicht sehen, wenn ich ein paar Spielzeuge kaufe; mit einem unbedeutenden Teil des Reichtums, den ich durch den Sieg über den schrecklichsten Zauberer, den Britannien je kannte, verdiente; nicht, dass ich Sie etwa beschuldige, undankbar zu sein oder so, aber trotzdem, was sind ein paar Spielzeuge verglichen damit?"

"\emph{Sie,}" grollte Professor McGonagall. Der Ausdruck auf ihrem Gesicht war so angsteinflößend und schrecklich, dass Harry quiekte und zurücktrat, mit einem lauten klimpernden Geräusch über einen Haufen Goldmünzen stolperte und rücklinks in eine Masse Geld fiel. Griphook seufzte und schlug sich die Hand vor's Gesicht. "Ich würde ganz Zauberer-Britannien einen großen Dienst erweisen, wenn ich Sie in diesem Verließ einschließen und hier lassen würde."

Und Sie gingen ohne weitere Schwierigkeiten.

* Hier ging ein kleineres Wortspiel bei der Übersetzung drauf, bezüglich dem ähnlichen Klang von \emph{world domination} und \emph{world optimisation}.

** Der originale Ausdruck hier ist \emph{mokeskin pouch}. Laut dem deutschsprachigen Harry-Potter-Wiki, wird ein \emph{Moke} in Joanne K. Rowlings Buch \emph{Phantastische Tierwesen und wo sie zu finden sind} (engl.: \emph{Fantastic Beasts und Where to Find Them}) als eine Echse beschrieben, die schrumpft, bis sie nicht mehr zu sehen ist, sobald Fremde sich ihr nähern. Diese Eigenschaft soll dafür verantwortlich sein, dass ein Beutel aus ihrer Haut sich soweit verengt, dass man nichts mehr herausnehmen kann, wenn ein Unbefugter hineingreift. In der deutschen Übersetzung der originalen Harry-Potter-Romane wurde -- wahrscheinlich aus Gründen der Verständlichkeit -- aus \emph{mokeskin} allerdings \emph{Eselsfell}, dem man gemeinhin eigentlich keine magischen Eigenschaften unterstellen würde. Da mir keine wirklich bessere (oder weniger dumm klingende) Übersetzung einfallen will, übernehme ich die originale, wie sie ist.

*** Wenn jemand bessere Übersetzungen für die Begriffe \emph{awesome} und \emph{awesomeness} anzubieten hat, immer raus damit.

