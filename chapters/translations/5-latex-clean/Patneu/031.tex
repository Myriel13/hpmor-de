

\hypertarget{gruppenarbeit-teil-2}{% \section{31. Gruppenarbeit, Teil 2}\label{gruppenarbeit-teil-2}}

\textbf{Kapitel 32: Gruppenarbeit, Teil 2

}

\emph{Nachspiele:}

\later

Harry schritt in seinem Generalsbüro auf und ab, das dafür wunderbar geeignet war, soweit er sagen konnte, schien es keinen anderen Nutzen zu haben.

\emph{Wie?}

\emph{Wie?}

Hermine hätte diese Schlacht nicht gewinnen sollen! Nicht bei ihrem ersten Versuch, nicht trotz ihrer überhaupt nicht gewalttätigen Natur, zu allem Überfluss einfach so auch noch eine großartige militärische Kommandeurin zu sein, war einfach zu viel, selbst für \emph{sie.}

Hatte sie die Taktik in einem Buch über Militärgeschichte entdeckt? Aber es war nicht nur diese eine Taktik gewesen, sie hatte ihre Truppen perfekt positioniert, um jeden Weg zum Rückzug abzuschneiden, ihre Truppen waren besser koordiniert als seine \emph{oder} die von Draco…

Hatte Professor Quirrell sein Versprechen gebrochen, ihr nicht zu helfen? Hatte er ihr das Tagebuch von General Tacticus* gegeben oder sowas?

Harry entging hier etwas, etwas wirklich wichtiges und sein Geist drehte sich immer wieder im Kreis, doch er bekam es nicht heraus.

Schließlich seufzte Harry. Er kam damit einfach nicht weiter und er musste vor der nächsten Schlacht von Hermine oder irgendjemand anderem den Schlagbohrfluch** lernen - Professor Quirrell hatte Harry klar gemacht, belustigt, doch mit scharfem, warnendem Unterton in der Stimme, dass "keine magischen Gegenstände, außer denen, die ich zur Verfügung stelle" Muggel-Technologie mit einschloss, egal wie sehr das \emph{keine Magie war.} Außerdem musste Harry sich auch noch etwas einfallen lassen, wie er beim nächsten mal Mr~Goyle vom Himmel holen sollte…

Die Schlachten machten für Generäle eine Menge Quirrell-Punkte aus und Harry musste sich langsam ins Zeug legen, wenn er Professor Quirrells Weihnachtswunsch gewinnen wollte.

\later

In seinem Privatzimmer in Slytherin starrte Draco Malfoy ins Leere, als sei die Mauer vor seinem Schreibtisch die faszinierendste Oberfläche auf der Welt.

\emph{Wie?}

\emph{Wie?}

Im Rückblick war es ein ziemlich offensichtlicher Einfall gewesen, was listige Pläne anging, doch Granger \emph{sollte} gar nicht gerissen sein! Sie hatte zu viel von einem Hufflepuff in sich gehabt, um einen Simplen Schagzauber auszuführen! Hatte Professor Quirrell ihr trotz seines Versprechens Ratschläge erteilt oder…

Und dann machte Draco endlich, was er schon viel früher hätte tun sollen.

Was er schon nach seinem ersten Treffen mit Granger hätte tun sollen.

Was Harry Potter ihm \emph{gesagt} hatte, das er tun sollte, wofür er ihn \emph{trainiert} hatte und doch hatte Harry Draco auch gewarnt, dass es Zeit brauchen würde, bis sein Hirn klar begriff, dass die Methoden auf das reale Leben anzuwenden waren und bis heute hatte Draco das nicht \emph{verstanden.} Jeden einzelnen seiner Fehler hätte er vermeiden können, wenn er nur die Dinge \emph{angewandt} hätte, die Harry ihm bereits \emph{erklärt} hatte -

Draco sagte laut, "Ich erkenne, dass ich verwirrt bin."

\emph{Deine Stärke als Rationalist ist deine Fähigkeit, von Fiktion verwirrter zu sein als von der Realität…}

Draco war verwirrt.

Daher musste etwas, das er glaubte, Fiktion sein.

Granger hätte zu all dem nicht in der Lage sein sollen.

Daher, war sie es wahrscheinlich auch nicht.

\emph{Ich verspreche General Granger in keiner Weise zu helfen, von der Sie beide nicht wissen.}

In einem Anflug plötzlicher, entsetzter Erkenntnis fegte Draco seine Papiere aus dem Weg, jagte durch die Unordnung auf seinem Schreibtisch, bis er es fand.

Und da war es.

Genau dort, in der Liste von Menschen und Material, die jeder der drei Armeen zugewiesen waren.

\emph{Verfluchter} Professor Quirrell!

Draco hatte es \emph{gelesen} und hatte es doch nicht \emph{gesehen -}

\later

Das Sonnenlicht des Nachmittags ergoss sich ins Büro des Sunshine-Regiments, erleuchtete General Granger in ihrem Stuhl, als glühe sie mit einer goldenen Aura.

"Wie lange glaubst Sie, braucht Malfoy bis er es rausfindet?" sagte General Granger.

"Nicht lange," sagte Colonel Blaise Zabini. "Hat er vielleicht schon. Wie lange wird Potter dafür brauchen?"

"Ewig," sagte General Granger, "es sei denn Malfoy erzählt es ihm oder einem seiner Soldaten wird es klar. Harry Potter denkt einfach nicht so."

"Wirklich?" sagte Captain Ernie Macmillan und blickte von einem der Ecktische auf, wo er im Schach von Captain Ron Weasley vernichtend geschlagen wurde. (Sie hatten natürlich all die anderen Stühle zurückgebracht, nachdem Malfoy verschwunden war.) "Ich meine, es scheint mir irgendwie offensichtlich zu sein. Wer würde denn versuchen, sich alles ganz allein einfallen zu lassen?"

"Harry," sagte Hermine, exakt im gleichen Moment als Zabini sagte, "Malfoy."

"Malfoy hält sich für um Welten besser als alle anderen," sagte Zabini.

"Und Harry… \emph{sieht} einfach die meisten anderen Menschen nicht wirklich so," sagte Hermine.

Eigentlich war es irgendwie traurig. Harry war sehr, sehr allein aufgewachsen. Es war nicht so, dass er in solchen Begriffen dachte, nur Genies hätten ein Recht zu existieren. Es würde ihm nur einfach nicht \emph{in den Sinn kommen,} dass irgendjemand in Hermines Armee außer Hermine irgendwelche guten Ideen haben könnte.

"Jedenfalls," sagte Hermine. "Captains Goldstein und Weasley, Sie haben die Aufgabe, sich neue Strategien für unsere nächste Schlacht einfallen zu lassen. Captains Macmillan und Susan - Sorry, ich meinte Macmillan und Bones - suchen Sie nach neuen Taktiken, die wir verwenden können, außerdem jede Art von Training, das wir Ihrer Meinung nach versuchen sollten. Oh und ich gratuliere zu ihrem Marschlied, Captain Goldstein, ich denke es hat der \emph{Truppenmoral} wirklich gut getan."

"Was werden Sie tun?" sagte Susan. "Und Colonel Zabini?"

Hermine erhob sich aus ihrem Stuhl und streckte sich. "Ich werde herauszufinden versuchen, was Harry Potter wohl denkt und Colonel Zabini wird sich überlegen, was Draco Malfoy tun könnte und wir gesellen uns beide wieder zu Ihnen, sobald uns etwas eingefallen ist. Ich mache einen Spaziergang, während ich nachdenke. Zabini, wollen Sie mitkommen?"

"Ja, General," sagte Zabini steif.

Es hatte kein Befehl sein sollen. Hermine seufzte innerlich ein wenig. Daran würde sie sich erst noch gewöhnen müssen und obwohl Zabinis erste Idee zweifellos funktioniert hatte, war sie nicht \emph{ganz} sicher, dass Professor Quirrells Zitat Mischung aus positiven und negativen Anreizen Zitat Ende ausreichen würde, den Slytherin bis zum Dezember gänzlich auf ihrer Seite zu halten, wenn Verräter zum ersten mal erlaubt würden…

Sie hatte auch noch immer keine Idee, was sie mit Professor Quirrells Weihnachtswunsch anstellen sollte. Vielleicht würde sie einfach Mandy fragen, ob sie irgendetwas wollte, wenn es soweit war.

* Es ist nicht ganz klar, ob Harry hier auf eine bestimmte Person anspielt oder ob "Tacticus" einfach nur einen Platzhalter-Name für irgendein militärisches Genie ist. Falls ersteres der Fall ist, kämen mehrere (historische und fiktive) Personen in Frage, wie etwa die griechischen Militärschriftsteller Aineias Taktikos (4. Jahrhundert v. Chr.) und Aelianus Tacticus (2. Jahrhundert n. Chr.), ersterer gilt als frühester europäischer kriegswissenschaftlicher Schriftsteller und hatte zumindest Erfahrung als militärischer Kommandeur, wenn auch nicht klar ist, ob er ein General war. Außerdem infrage kommen der römische Historiker und Senator Cornelius Tacitus (1. Jahrhundert n. Chr.) sowie der fiktive General Callus Taktikus, der Erwähnung in mehreren \emph{Scheibenwelt}-Romanen von Terry Pratchett findet.

** engl.: \emph{Breaking Drill Hex;} der Zauber den Hermine verwendet hat, um Dracos Schild zu brechen.

