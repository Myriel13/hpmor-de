

\hypertarget{abstimmungsprobleme-teil-2}{% \section{34. Abstimmungsprobleme, Teil 2}\label{abstimmungsprobleme-teil-2}}

\textbf{Kapitel 34: Abstimmungsprobleme, Teil 2}

\hfill\break Minerva und Dumbledore hatten gemeinsam ihr gesammeltes Talent aufgeboten, um die große Bühne heraufzubeschwören, auf die Quirrell nun langsam zu schritt; sie bestand, im Grunde, aus Sperrholz, doch ihre Außenflächen schimmerten vor funkelndem Marmor, mit Einlagen aus Platin und gesprenkelt mit Edelsteinen in den Farben eines jeden Hauses. Weder sie noch der Schulleiter konnten sich mit irgendeinem der Gründer von Hogwarts messen, doch die Beschwörung musste auch nur wenige Stunden lang halten. Für gewöhnlich erfreute Minerva sich an den seltenen Anlässen, da sie die Gelegenheit bekam, sich an großen Transfigurationen vollkommen zu verausgaben; die vielen kleinen Möglichkeiten zur künstlerischen Betätigung und die Illusion der Opulenz hätten ihr Vergnügen bereiten sollen; doch dieses Mal hatte sie die Arbeit mit dem schrecklichen Gefühl verrichtet, sich ihr eigenes Grab zu schaufeln.

Aber mittlerweile ging es Minerva schon wieder etwas besser. Einen kurzen Moment hatte es den Anschein gehabt, als ginge nun die ganze Sache hoch; doch Dumbledore hatte sich bereits erhoben und herzlich applaudiert und niemand hatte sich als töricht genug erwiesen, im Beisein des Schulleiters den Aufstand zu proben.

So war die explosive Stimmung rasch einer kollektiven Empfindung gewichen, die sich vielleicht am besten beschreiben lassen mochte mit den Worten: \emph{Jetzt macht aber mal halblang!}

Blaise Zabini hatte sich selbst im Namen von Sunshine erschossen und der finale Punktestand betrug 254 zu 254 zu 254.

--------------------------------------------------------------------------------------------------------------------------------------------

\hfill\break Hinter der Bühne warteten drei Kinder darauf, sie zu erklimmen und funkelten einander mit einer Mischung aus Zorn und Frustration böse an. Die Situation wurde auch dadurch nicht gerade besser, dass sie noch immer durchnässt waren, nachdem man sie aus dem See gefischt hatte und die Wärmezauber nicht ganz auszureichen schienen, die kühle Dezember-Luft völlig auszugleichen; vielleicht lag es aber auch nur an ihrer Stimmung.

„Das \emph{war's},“ sagte Granger. „Es \emph{reicht!} Keine Verräter mehr!"

„Da kann ich Ihnen nur zustimmen, Miss Granger,“ sagte Draco eisig. „Genug ist genug."

„Und was habt ihr beiden vor dagegen zu unternehmen?“ schnappte Harry Potter. „Professor Quirrell hat bereits gesagt er würde die Spione nicht verbieten!"

„Dann übernehmen wir das \emph{für} ihn,“ sagte Draco grimmig. Als er die Worte aussprach, war ihm noch gar nicht klar gewesen, was er damit eigentlich meinte, doch allein durch den Akt selbst schien sich nun ein Plan herauszubilden -

--------------------------------------------------------------------------------------------------------------------------------------------

\hfill\break Die Bühne war wirklich gut gemacht, jedenfalls für eine Struktur, die nicht von Dauer war; ihre Schöpfer hatten nicht den üblichen Fehler begangen, sich von der eigenen Illusion des Reichtums berauschen zu lassen und tatsächlich etwas von Architektur und Ästhetik verstanden. Von Dracos Standpunkt aus sähen die Schüler ihn eingehüllt in einen blassen Schein aus Emerald und Granger würde, an dem Platz auf den Draco sie subtil manövriert hatte, umgeben sein von Ravenclaws Saphir. Was Harry Potter betraf, so würdigte Draco ihn momentan keines Blickes.

Professor Quirrell war... erwacht oder wie auch immer man das nennen sollte und lehnte nun an einem Podium aus Platin, bar aller Edelsteine. Mit offenkundigem Showtalent klopfte und rüttelte der Verteidigungsprofessor nun jene drei Umschläge zurecht, welche die Pergamente enthielten, auf denen die drei Generäle ihre Wünsche notiert hatten, während alle Schüler von Hogwarts in gebannter Erwartung verharrten.

Schließlich blickte Professor Quirrell von den Umschlägen auf. „Nun,“ sagte der Verteidigungsprofessor. „Das ist jetzt etwas unangenehm."

Ein leichter Anflug von Gelächter fuhr durch die Menge, jedoch mit einem scharfen Unterton.

„Ich nehme an, Sie fragen sich alle, was ich nun wohl tun werde?“ sagte Professor Quirrell. „Es nützt alles nichts; ich werde das tun müssen, was nur fair ist. Obwohl ich zunächst noch eine kleine Ansprache halten wollte, doch davor, so scheint es mir, wünschen Mr. Malfoy und Miss Granger noch etwas mitzuteilen."

Draco blinzelte, dann tauschten er und Granger rasche Blicke aus - \emph{darf ich?} - \emph{ja,} \emph{na los} - und Draco erhob die Stimme.

„Sowohl General Granger als auch ich wünschen zu verkünden,“ schlug Draco seinen förmlichsten Tonfall an, im Wissen dass seine Stimme verstärkt und gehört wurde, „dass keiner von uns noch länger die Dienste von Verrätern in Anspruch nehmen wird. Und falls wir in irgendeiner Schlacht feststellen sollten, dass Mr. Potter Verräter aus irgendeiner unserer Armeen rekrutiert hat, so werden wir unsere Kräfte bündeln und ihn zerschmettern."

Und Draco warf dem Jungen-der-überlebt-hat einen bösen Blick zu, voller Groll. \emph{Nimm das, General Chaos!}

„Ich stimme vollkommen mit General Malfoy überein,“ hört er Granger neben sich, ihre hohe Stimme stark und klar. „Keiner von uns wird von Verrätern Gebrauch machen und sollte General Potter es tun, dann werden wir ihn vom Schlachtfeld fegen."

Ein überraschtes Flüstern ging durch die Schülerschaft.

„Sehr schön,“ sagte der Verteidigungsprofessor lächelnd. „Auch wenn Sie beide lange genug gebraucht haben, so darf man Ihnen trotzdem gratulieren, dass Sie noch vor irgendeinem der anderen Generäle darauf gekommen sind."

Das musste einen Moment sacken -

„In Zukunft, Mr. Malfoy, Miss Granger, bedenken Sie bitte, bevor Sie mit irgendeinem Anliegen mein Büro betreten sollten, ob es nicht eine Möglichkeit für Sie gibt, Ihr Ziel ohne meine Hilfe zu erreichen. Ich werde zu diesem Anlass keine Quirrell-Punkte abziehen, doch das nächste Mal dürfen Sie erwarten, die vollen fünfzig zu verlieren.“ Professor Quirrell trug ein belustigtes Grinsen zur Schau. „Und was haben Sie dazu zu sagen, Mr. Potter?"

Der Blick von Harry Potter wanderte erst zu Granger, dann zu Draco. Sein Gesichtsausdruck schien ruhig; doch Draco war sicher, \emph{beherrscht} wäre der bessere Ausdruck gewesen.

Schließlich sprach Harry Potter, mit ausdrucksloser Stimme. „Die Chaos-Legion nimmt auch weiterhin gern die Dienste aller Verräter in Anspruch. Wir sehen uns auf dem Schlachtfeld.„

Draco wusste, dass sich auf seinem eigenen Gesicht der Schock abzeichnete; aus dem Publikum war entsetztes Raunen zu vernehmen und als Draco den Blick über die erste Reihe schweifen ließ, sah er selbst Harrys eigene Chaoten sprachlos.

Auf Grangers Gesicht zeigte sich Zorn, welcher sich nur noch verstärkte. „Mr. Potter,“ sagte sie in scharfem Ton, als hielte sie sich für eine Lehrerin, „versuchen Sie \emph{absichtlich} unausstehlich zu sein?"

„Ganz sicher nicht,“ sagte Harry Potter ruhig. „Ich werde euch auch nicht jedesmal auf die Probe stellen. Besiegt mich und ich bleibe besiegt. Aber Drohungen allein sind nicht immer ausreichend, General Sunshine. Ihr habt mich nicht darum gebeten mich euch anzuschließen, sondern schlicht versucht euren Willen durchzusetzen und manchmal muss man den Feind auch tatsächlich besiegen, um ihm den eigenen Willen aufzuzwingen. Wisst ihr, ich bin einfach noch skeptisch, ob Hermine Granger, der hellste Stern am schulischem Himmel von Hogwarts und Draco, Sohn von Lucius, Spross des noblen und uralten Hauses Malfoy, tatsächlich zur Zusammenarbeit fähig sind, um ihren gemeinsamen Feind, Harry Potter, zu schlagen.“ Ein belustigtes Grinsen lief über Harry Potters Gesicht. „Vielleicht mache ich einfach das, was Draco mit Zabini versucht hat und schreibe einen kleinen Brief an Lucius Malfoy; mal sehen was \emph{er} davon hält."

„\emph{Harry!}“ keuchte Granger und wirkte absolut fassungslos und auch aus dem Publikum war ähnliches zu vernehmen.

Draco hielt den Ärger im Zaum, der ihn durchströmte. Es war \emph{dummer} Zug von Harry gewesen, das in der Öffentlichkeit zu sagen. Hätte Harry es einfach \emph{getan}, mochte es vielleicht sogar funktioniert haben, soweit hatte Draco noch gar nicht gedacht, doch wenn Vater \emph{jetzt} handelte, dann sähe es aus als spielte er Harry in die Hände -

„Wenn du glaubst, mein Vater, Lord Malfoy, ließe sich von \emph{dir} so leicht manipulieren,“ sagte Draco eisig, „dann steht dir noch eine Überraschung bevor, Harry Potter."

Und noch während ihm die Worte über die Lippen kamen, wurde Draco bewusst, dass er soeben \emph{seinen eigenen Vater}, mehr oder weniger unabsichtlich, direkt in die Ecke gedrängt hatte. Vater würde das wahrscheinlich \emph{gar} \emph{nicht} gefallen, nicht das allerkleinste bisschen, doch nun konnte er das unmöglich sagen... Draco würde sich dafür entschuldigen müssen, es \emph{war} ganz ehrlich nur ein Versehen gewesen, doch er fand es seltsam, dass es überhaupt geschehen war.

„Dann nur zu und besiegt den bösen General Chaos,“ sagte Harry und wirkte noch immer belustigt. „Gegen euer beider Armeen kann ich nicht gewinnen - nicht wenn ihr \emph{wirklich} zusammenarbeitet. Aber ich frage mich, ob es mir nicht vorher gelingen mag, euch zu entzweien."

„Wird es nicht und wir werden dich \emph{vernichten!}“ sagte Draco Malfoy.

An seiner Seite nickte Hermine Granger entschieden.

„Nun,“ sagte Professor Quirrell, nachdem sich die verblüffte Stille nun schon eine Weile hinzog. „Diesen Verlauf der Unterhaltung hatte ich durchaus \emph{nicht} erwartet.“ Der Gesichtsausdruck des Verteidigungsprofessors zeigte tatsächlich eine gelinde Faszination. „Um ehrlich zu sein, Mr. Potter, hatte ich erwartet, Sie würden sofort nachgeben und dann lächelnd verkünden, Sie hätten die von mir beabsichtigte Lektion schon vor längerem erkannt, jedoch entschieden, den anderen die Überraschung nicht zu verderben. Tatsächlich habe ich sogar meine Rede darauf abgestimmt, Mr. Potter.„

Harry Potter zuckte nur mit den Schultern. „Das tut mir leid,“ sagte er und dann nichts mehr.

„Oh, keine Sorge,“ sagte Professor Quirrell. „Auch dies wird den Zweck erfüllen."

Und mit diesen Worten wandte sich Professor Quirrell von den drei Kindern ab und richtete sich an seinem Podium auf, seine Aufmerksamkeit galt nun der versammelten Menge der Zuschauer; die übliche Aura distanzierter Belustigung fiel von ihm ab wie eine Maske und als er erneut sprach war seine Stimme noch lauter verstärkt als zuvor.

„Hätte es Harry Potter nicht gegeben,“ sagte Professor Quirrell, seine Stimme so schneidend und kalt wie die Dezemberluft, „dann hätte Sie-wissen-schon-wer gewonnen.

Augenblicklich trat allumfassende Stille ein.

--------------------------------------------------------------------------------------------------------------------------------------------

\hfill\break Täuschen Sie sich nicht,“ sagte Professor Quirrell. „Der Dunkle Lord war im Begriff, den Sieg davonzutragen. Immer weniger und weniger Auroren wagten es, sich ihm entgegenzustellen, die Wachsamen, die sich ihm widersetzten wurden gejagt und getötet. Ein Dunkler Lord und vielleicht fünfzig Todesser \emph{besiegten} ein Land von Tausenden. Dies spottet jeder Beschreibung! Ich habe keine Noten, die mir ausreichen würden ein solches Ausmaß an Unfähigkeit zu bewerten!"

Ein Stirnrunzeln zeichnete sich ab auf dem Gesicht von Schulleiter Dumbledore, auf jenen der Zuschauer Verwirrung und die vollkommene Stille dauerte an.

"Verstehen Sie nun, wie es geschehen konnte? Sie haben es am heutigen Tage gesehen. Ich erlaubte den Einsatz von Verrätern und gab den Generälen keinerlei Mittel an die Hand, ihnen Einhalt zu gebieten. Das Resultat haben Sie gesehen. Sie alle verfolgten gerissene Pläne und verrieten einander, bis sich der letzte Soldat auf dem Schlachtfeld selbst erschossen hat! Es kann \emph{keinerlei} Zweifel daran bestehen, dass jede dieser drei Armeen von jedem äußeren Feind geschlagen würde, der in sich geeint ist.„

Professor Quirrell stützte sich auf das Podium, seine Stimme nun voll grimmigem Nachdruck. Er streckte die rechte Hand aus, die Finger gespreizt. „Teilung ist Schwäche,“ sagte der Verteidigungsprofessor. Seine Hand schloss sich zur Faust. „Einigkeit ist Stärke. Der Dunkle Lord verstand das gut, wo immer er sonst auch irren mochte und er \emph{nutzte} dieses Wissen, um die eine Errungenschaft zu kreieren, die ihn zum schrecklichsten Dunklen Lord der Geschichte machte. Ihre Eltern standen gegen einen Dunklen Lord. Und fünfzig Todesser, in vollkommener Einheit, wissend dass jeder Mangel an Loyalität mit dem Tod bestraft würde und jede Nachlässigkeit oder Unfähigkeit mit Schmerzen. Keiner von ihnen entging dem Griff des Dunklen Lords, sobald sie sein Mal akzeptierten. Und die Todesser akzeptierten jenes schreckliche Mal, denn sie wussten, sobald sie es trugen, stünden sie \emph{vereint} gegen ein geteiltes Land. Ein Dunkler Lord und fünfzig Todesser hätten ein gesamtes Land bezwungen, durch die Macht des Dunklen Mals.„

Professor Quirrells Stimme klang rau und hart. „Ihre Eltern \emph{hätten} sich ihnen in gleicher Weise erwehren können. Sie taten es nicht. Es gab einen Mann namens Yermy Wibble, der die Nation aufrief, eine allgemeine Einberufung zu verfügen, obgleich ihm die Vision fehlte, ein Mal von Britannien zu propagieren. Yermy Wibble wusste, was mit ihm geschehen würde; er hoffte sein Tod würde andere inspirieren. Also nahm sich der Dunkle Lord gleichsam seiner Familie an. Ihre abgezogenen Häute inspirierten zu nichts anderem als Furcht und niemand wagte es, erneut die Stimme zu erheben. Und Ihre Eltern hätten die Konsequenzen ihrer verabscheuungswürdigen Feigheit tragen müssen, wären sie nicht von einem erst ein Jahr alten Jungen gerettet worden.“ Professor Quirrells Gesicht zeigte nichts als Verachtung. „Ein Dramaturg würde es einen \emph{Geist aus der Maschine} nennen, denn sie taten nichts, ihre Erlösung zu verdienen. Er-dessen-Name-nicht-genannt-werden-darf verdiente vielleicht nicht zu gewinnen, doch Ihre Eltern verdienten zweifellos zu verlieren."

Die Stimme des Verteidigungsprofessors erklang wie Eisen. „Und seien Sie sich bewusst: Ihre Eltern haben nichts daraus gelernt! Die Nation ist noch immer gespalten und schwach! Wie wenige Jahrzehnte vergingen zwischen Grindelwald und Sie-wissen-schon-wem? Glauben Sie etwa, \emph{Sie} hätten sich der nächsten Bedrohung Zeit Ihres Lebens nicht zu stellen? Werden \emph{Sie} sodann die Fehler Ihrer Eltern wiederholen, nun da Ihnen das Ergebnis am heutigen Tage so klar und deutlich vor Augen geführt wurde? Denn ich kann Ihnen versichern, was Ihre Eltern tun werden, wenn der Tag der Finsternis da ist! Ich kann Ihnen sagen, welche Lektion sie gelernt haben! Sie haben gelernt, sich zu verstecken wie Feiglinge und gar nichts zu tun, während sie warten, dass Harry Potter sie retten kommt!"

Ein abschätzender Blick lag in den Augen von Schulleiter Dumbledore und andere Schüler blickten zu ihrem Verteidigungsprofessor auf, mit Verwirrung und Zorn und Erstaunen.

Professor Quirrells Blick war nun ebenso kalt wie seine Stimme. „Merken Sie sich dies und merken Sie es sich gut. Er-dessen-Name-nicht-genannt-werden-darf wünschte über dieses Land zu herrschen, es für alle Zeit in seinem grausamen Griff zu halten. Doch immerhin wünschte er über ein Land der \emph{Lebenden} zu herrschen und nicht nur ein Häuflein Asche! Es hat bereits andere Dunkle Lords gegeben, die wahnsinnig waren, die nichts begehrten, als die ganze Welt gleich einem lodernden Scheiterhaufen brennen zu sehen! Es hat Kriege gegeben, in denen ganze Länder gegeneinander marschiert sind! Ihre Eltern hätten beinahe verloren, gegen eine halbe Hundertschaft, die dieses Land lebendig zu ergreifen gedachte! Wie schnell wären sie wohl vernichtet worden, von einem Feind, der ihnen zahlenmäßig überlegen ist und nichts im Sinn hat als ihre Vernichtung? Dies sage ich voraus: Wenn sich die nächste Bedrohung erhebt, so wird Lucius Malfoy behaupten, Sie müssten ihm folgen oder untergehen, dass Ihnen nur die Hoffnung bleibt, auf seine Grausamkeit und Stärke zu vertrauen. Und obgleich Lucius Malfoy selbst es glauben wird, es wird eine Lüge sein. Denn als der Dunkle Lord scheiterte, vereinte Lucius Malfoy die Todesser nicht, sie verstreuten sich augenblicklich, sie flohen wie geprügelte Hunde und verrieten einander! Lucius Malfoy besitzt nicht die Stärke, ein wahrer Lord zu sein, ob nun ein Dunkler oder sonst einer."

Draco Malfoy hielt die weiß-glühenden Fäuste geballt, in seinen Augen standen Tränen und Zorn und unerträgliche Scham.

„Nein,“ sagte Professor Quirrell, „ich glaube nicht, dass es Lucius Malfoy sein wird, der Sie errettet. Und mögen Sie glauben, ich würde hier meine eigenen Dienste anpreisen, so wird die Zeit bald genug zeigen, dass dem nicht so ist. Ich spreche Ihnen keine Empfehlung aus, meine Schüler. Doch ich sage Ihnen, sollte ein ganzes Land einen Anführer finden, so stark wie der Dunkle Lord, aber ehrenhaft und rein und akzeptierten sie sein Mal; so könnten sie jeden Dunklen Lord zerquetschen wie ein Insekt und auch der ganze Rest unserer gespaltenen magischen Welt könnte sie nicht schrecken. Und sollte uns dereinst, in einem Auslöschungskrieg, ein noch größerer Feind erwachsen, so könnte nur eine geeinte magische Welt ihn überstehen."

Hierbei keuchten einige auf, die meisten Muggelgeborene; die Schüler in grün-getrimmten Umhängen wirkten lediglich verwirrt. Nun war es an Harry Potter, die Hände bebend zur Faust zu ballen und Hermine neben ihm wirkte zornig und bestürzt.

Der Schulleiter erhob sich nun von seinem Platz, mit strengem Gesicht; noch sagte er kein Wort, doch die Botschaft war klar.

„Ich kann nicht sagen, welche \emph{Art} von Bedrohung wohl kommen mag,“ sagte Professor Quirrell. „Doch werden Sie kaum Ihr ganzes Leben in Frieden verbringen, nicht wenn die Geschichte der Welt uns auch nur etwas über ihre Zukunft zu verraten vermag. Und wenn Sie in Zukunft so handeln, wie Sie es von drei Armeen am heutigen Tage gesehen haben, wenn sie ihre kleinlichen Streitigkeiten nicht beilegen und das Mal eines Anführers akzeptieren können, so mögen sie in der Tat wünschen, dass der Dunkle Lord noch lebte, um über Sie zu herrschen und den Tag bereuen, da Harry Potter jemals geboren wurde -"

„\emph{Genug!}“ bellte Albus Dumbledore.

Stille trat ein.

Langsam wandte Professor Quirrell den Blick zu Albus Dumbledore, der dort stand in der ganzen rasenden Macht seiner Zauberkraft; ihre Augen trafen sich und ein lautloser Druck senkte sich wie eine gewaltige Last über alle Schüler, die lauschten und sich nicht zu rühren wagten.

„Auch Sie haben dieses Land im Stich gelassen,“ sagte Professor Quirrell. „Und Sie kennen die Gefahr so gut wie ich."

„Weder sind solche Reden für die Ohren von Schülern bestimmt,“ sagte Albus Dumbledore und hob bedrohlich die Stimme. „Noch sollten Professoren sie im Munde führen!"

Woraufhin Professor Quirrell, in trockenem Tonfall, sprach: „Es wurden viele Reden gehalten, bestimmt für die Ohren Erwachsener, als der Dunkle Lord sich erhob. Und die Erwachsenen applaudierten und jubelten, dann gingen sie, für den Tag unterhalten, heim. Doch ich werde gehorchen, Schulleiter und keine weiteren Reden halten, so sie Ihnen nicht zusagen. Meine Lektion ist simpel. Ich werde auch weiterhin nichts gegen Verräter unternehmen und wir werden sehen, was die Schüler dagegen zu tun vermögen, wenn sie nicht warten, dass ihre Lehrer sie retten kommen.„

Dann widmete sich Professor Quirrell erneut seinen Schülern und sein Mund verzog sich zu einem schiefen Grinsen, dass den bedrohlichen Druck von der Menge zu nehmen schien, wie ein Gott dessen Atem die Wolken zerstreute. „Doch gehen Sie mit den Verrätern nicht zu hart ins Gericht,“ sagte Professor Quirrell. „Sie hatten ja nur ihren Spaß."

Gelächter erklang, zunächst noch nervös, dann schwoll es langsam an während Professor Quirrell mit schiefem Grinsen da stand und sich ein Teil der Anspannung löste.

--------------------------------------------------------------------------------------------------------------------------------------------

\hfill\break Noch immer schwirrte Dracos Verstand vor eintausend Fragen, wie betäubt vor Entsetzen, als Professor Quirrell sich anschickte die Umschläge zu öffnen, denen die drei ihre Wünsche anvertraut hatten.

Es war Draco nie zuvor in den Sinn gekommen, dass mondfahrende Muggel eine größere Bedrohung sein könnten als der langsame Verfall der Zauberkunst oder dass Vater sich als zu schwach erwiesen hatte, sie aufzuhalten.

Und sogar noch seltsamer war die offensichtliche Andeutung: Professor Quirrell glaubte, dass \emph{Harry} es konnte. Der Verteidigungsprofessor gab vor, er habe keine Empfehlung ausgesprochen, doch er hatte Harry Potter in seiner Rede immer wieder und wieder erwähnt; andere würden bereits dasselbe ahnen wie Draco.

Es war einfach lächerlich. Der Junge, der einen gepolsterten Sessel mit Glitter bestreut und ihn einen Thron genannt hatte -

\emph{Der Junge, der sich Severus Snape entgegengestellt und gewonnen hatte,} flüsterte eine verräterische Stimme, \emph{dieser Junge könnte} \emph{dereinst zu einem} \emph{Lord} \emph{werden, stark genug zu herrschen, stark genug uns alle zu retten} -

Harry war von Muggeln \emph{aufgezogen} worden! Praktisch war er selbst ein Schlammblut, er würde sich doch nicht gegen seine Adoptivfamilie stellen -

\emph{Er kennt ihre Künste, ihre Geheimnisse und ihre Methoden; er kann alle Wissenschaft der Muggel gegen sie verwenden, zusammen mit unserer eigenen Macht als Zauberer.}

Aber was, wenn er sich weigert? Was, wenn er zu schwach ist?

\emph{Dann,} sagte die innere Stimme, \emph{wirst du es} \emph{tun} \emph{müssen, nicht wahr, Draco Malfoy?}

Dann verstummte die Menge von Neuem, als Professor Quirrell den ersten Umschlag öffnete.

„Mr. Malfoy,“ sagte Professor Quirrell, „Ihr Wunsch ist es, dass... Slytherin den Hauspokal gewinnt."

Verwirrung aus der Menge der Zuschauer.

„Ja, Professor,“ sagte Draco mit klarer Stimme, wissend dass sie nun wieder verstärkt wurde. „Wenn das nicht möglich ist, dann etwas anderes für Slytherin -"

„Ich werde Hauspunkte nicht auf unfaire Weise vergeben,“ sagte Professor Quirrell. Mit nachdenklichem Ausdruck auf dem Gesicht, tippte er sich an die Wange. „Was ihren Wunsch schwierig genug macht, um interessant zu sein. Möchten Sie uns den Grund verraten, Mr. Malfoy?"

Draco wandte sich vom Verteidigungsprofessor ab und blickte in den Schein von Platin und Emeralden gehüllt über die Menge. Nicht alle Slytherins hatten die Drachen-Armee unterstützt; es gab Anti-Malfoy-Fraktionen, die ihrem Unmut Ausdruck verliehen hatten, indem sie den Jungen-der-überlebt-hat unterstützten oder sogar Granger und jene Gruppen würden sich durch Zabinis Handlungen enorm ermutigt fühlen. Er musste sie daran erinnern, dass Slytherin für Malfoy stand und Malfoy für Slytherin -

„Nein,“ sagte Draco. „Sie sind Slytherins, sie werden verstehen."

Ein leichtes Gelächter aus dem Publikum, besonders von Slytherin, selbst von Schülern, die sich noch einen Moment zuvor als Gegner von Malfoy bekannt hätten.

Schmeichelei war doch etwas Wunderbares.

Draco wandte sich wieder Professor Quirrell zu und war überrascht, einen verlegenen Ausdruck auf Grangers Gesicht zu sehen.

„Und was Miss Granger angeht...“ sagte Professor Quirrell. Das Geräusch eines zerreißenden Umschlages erklang. „Ihr Wunsch ist es, dass... Ravenclaw den Hauspokal gewinnt?"

Erhebliches Gelächter von Seiten des Publikums, selbst Draco entfuhr ein leichtes Glucksen. Er hatte nicht gedacht, dass Granger dieses Spiel spielte.

„Nun ja, ähm,“ sagte Granger und klang als verhaspele sie sich plötzlich bei einer einstudierten Rede, „ich wollte sagen, dass...“ Sie holte tief Luft. „In meiner Armee gab es Soldaten aus einem jeden Haus und niemals würde ich die Leistung irgendeines von ihnen schmälern wollen. Doch auch die Häuser sollten weiterhin von Bedeutung sein. Es war traurig, als Schüler des selben Hauses sich gegenseitig verhexten, nur weil sie in verschiedenen Armeen waren. Man sollte sich auf die Mitglieder seines Hauses verlassen können. Deshalb haben Godric Gryffindor, Salazar Slytherin, Rovena Ravenclaw und Helga Hufflepuff die vier Häuser überhaupt erst geschaffen. Ich bin General Sunshine, aber davor noch bin ich Hermine Granger von Ravenclaw und ich bin stolz, Teil eines achthundert Jahre alten Hauses zu sein."

„Gut gesprochen, Miss Granger!“ erklang Dumbledores donnernde Stimme.

Harry Potter runzelte die Stirn und etwas machte sich ganz am Rande von Dracos Wahrnehmung bemerkbar.

„Ein interessanter Gedanke, Miss Granger,“ sagte Professor Quirrell. „Und doch gibt es Zeiten, da ist es gut für einen Slytherin, wenn er Freunde in Ravenclaw hat und für einen Gryffindor, hat er Freunde in Hufflepuff. Sicherlich wäre es das Beste, Sie könnten sich auf die Freunde in Ihrem Haus und auch auf die in Ihrer Armee verlassen?"

Grangers Blick flackerte kurz hinüber zu den zuschauenden Schülern und Lehrern, doch sie sagte nichts.

Professor Quirrell nickte, wie zu sich selbst und wandte sich dann wieder dem Podium zu, nahm den letzten Umschlag und riss ihn auf. An Dracos Seite versteifte sich Harry Potter sichtlich, als der Verteidigungsprofessor das Pergament hervor zog. „Und Mr. Potters Wunsch ist es, dass -"

Eine Pause entstand und Professor Quirrell betrachtete das Pergament.

Dann, ohne auch nur die geringste Regung auf Professor Quirrells Gesicht, ging das Pergament in Flammen auf und verbrannte in einem kurzen, intensiven Feuerstoß, nach dem nur noch ein Häuflein schwarzer Asche aus seiner Hand rieselte.

„Bitte beschränken Sie sich auf das Mögliche, Mr. Potter,“ sagte Professor Quirrell und klang nun wirklich äußerst trocken.

Es entstand eine lange Pause; Harry, der neben Draco stand, wirkte sichtlich erschüttert.

\emph{Was in Merlins Namen hatte er sich nur gewünscht?}

„Ich will doch sehr hoffen,“ sagte Professor Quirrell, „dass Sie einen weiteren Wunsch vorbereitet haben, falls ich diesen nicht gewähren könnte.„

Eine weitere Pause.

Harry holte tief Luft. „Habe ich nicht,“ sagte er, „doch mir ist bereits einer eingefallen.“ Harry Potter wandte sich an das versammelte Publikum und seine Stimme erklang fest, als er sprach. „Die Menschen fürchten Verräter, wegen des Schadens, den sie unmittelbar verursachen können, der Soldaten, die sie erschießen oder der Geheimnisse, die sie verraten. Doch das ist nur ein Teil der Gefahr. Was die Menschen aus \emph{Angst} vor Verrätern tun, kommt sie ebenfalls teuer zu stehen. Ich benutzte diese Strategie am heutigen Tage gegen Sunshine und gegen die Drachen. Ich wies meine Verräter nicht an, so viel direkten Schaden anzurichten, wie möglich. Vielmehr wies ich sie an, so viel Verwirrung und Misstrauen zu schaffen, wie sie nur konnten, um die Generäle zu zwingen, auch den höchsten Preis zu zahlen, nur in dem Versuch, zu verhindern dass sie es wieder tun. Wenn es nur ein paar wenige Verräter gibt und ein ganzes Land, dass sich ihnen entgegenstellt, so steht es zu fürchten, dass die Handlungen von ein paar wenigen Verrätern weniger Schaden anrichten, als dass was ein ganzes Land unternimmt, um sie aufzuhalten, dass die Heilung schlimmer sein mag als die Seuche -"

„Mr. Potter,“ sagte der Verteidigungsprofessor, mit plötzlich schneidender Stimme, „was uns unsere Geschichte lehrt, ist dass Sie schlichtweg falsch liegen. Die Generation Ihrer Eltern tat zu wenig, um nach Einigkeit zu streben, nicht zu viel! Dieses gesamte Land, Mr. Potter, wäre beinahe gefallen, auch wenn Sie nicht hier waren, es zu sehen. Ich schlage vor, Sie fragen Ihre Mitschüler in Ravenclaw, wie viele von ihnen Familie an den Dunklen Lord verloren haben. Oder fragen Sie besser \emph{nicht,} es wäre klüger! \emph{Haben} Sie einen Wunsch zu äußern, Mr. Potter?"

„So es Ihnen nichts ausmacht,“ sagte die sanfte Stimme von Albus Dumbledore, „würde ich gern hören, was der Junge-der-überlebt-hat zu sagen hat. Er hat mehr Erfahrung als sonst einer von uns, wenn es darum geht, wie man Kriege beendet."

Ein paar Leute lachten, doch viele waren es nicht.

Harry Potters wandte den Blick zu Dumbledore und einen Moment lang wirkte er nachdenklich. „Ich will nicht sagen, dass Sie Unrecht haben, Professor Quirrell. In dem letzten Krieg hielten die Menschen nicht zusammen und ein gesamtes Land fiel beinahe an ein paar Dutzend Angreifer und, ja, das war jämmerlich. Und wenn wir beim nächsten Mal wieder den selben Fehler machen, dann ja, so wäre das noch erbärmlicher. Doch man kämpft niemals zweimal den selben Krieg. Und das Problem ist, der Feind kann \emph{ebenfalls} gerissen sein. Ist man geteilt und uneins, so ist man auf eine Art verwundbar; doch will man Einigkeit um jeden Preis, so geht man andere Risiken ein und muss andere Opfer bringen und auch das wird der Feind für sich auszunutzen suchen. Man kann nicht aufhören zu denken und das Spiel nicht nur auf einem Level spielen."

„Schlichtheit hat ebenso einiges für sich, Mr. Potter,“ sagte die trockene Stimme des Verteidigungsprofessors. „Ich möchte hoffen, Sie haben an diesem Tag etwas darüber gelernt, welche Gefahren kompliziertere Strategien mit sich bringen, als für Einigkeit zu sorgen und sich gegen den Feind zu wenden. Und sollte nun all dies nicht irgendwie mit Ihrem Wunsch verflochten sein, so wäre ich höchst verärgert."

„Ja,“ sagte Harry Potter, „es war recht schwierig, sich einen Wunsch einfallen zu lassen, der die Opfer der Einigkeit symbolisiert. Doch das Problem der Zusammenarbeit stellt sich nicht nur in Kriegen, es ist etwas, dem wir uns unser ganzes Leben lang stellen müssen, an jedem einzelnen Tag. Wenn alle sich auf die gleichen Regeln verabreden und die Regeln sind dumm, dann verstößt ein \emph{Einzelner,} der entscheidet etwas zu ändern, gegen die Regeln. Doch wenn \emph{alle gemeinsam} entscheiden etwas zu ändern, dann können sie es. Es ist genau dasselbe Problem, es müssen alle zusammenwirken. Doch für den \emph{ersten,} der etwas anspricht, scheint es als würde er allein gegen die Menge stehen. Und glaubte man, es sei das einzig wichtige, dass die Menschen immer einig sind, so könnte man niemals das Spiel verändern, egal wie dumm die Regeln auch sind. Mein Wunsch nun also - als Symbol dafür was geschieht, wenn Menschen in gemeinsam den falschen Weg einschlagen - ist es, dass wir in Hogwarts Quidditch ohne den Schnatz spielen sollten."

„\emph{WAS?}“ schrie es aus hundert Kehlen in der Menge und Draco klappte die Kinnlade herunter.

„Der Schnatz ruiniert das gesamte Spiel,“ sagte Harry Potter. „Alles, was die anderen Spieler tun, wird dadurch völlig irrelevant. Es würde überwältigend mehr Sinn ergeben, einfach eine Uhr zu kaufen. Das ist eine dieser unglaublich dummen Sachen, die man nicht bemerkt, weil man damit aufgewachsen ist, die Menschen tun, nur weil es alle anderen tun -„

Doch an diesem Punkt war Harry Potters Stimme bereits nicht mehr zu hören, denn der Aufruhr hatte begonnen.

--------------------------------------------------------------------------------------------------------------------------------------------

\hfill\break Der Aufruhr endete etwa fünfzehn Sekunden später, nachdem ein gigantischer Feuerstoß zum Klang von einhundert Donnerschlägen aus dem höchsten Turm von Hogwarts hervorbrach. Draco hatte gar nicht gewusst, dass Dumbledore das \emph{konnte}.

Die Schüler setzten sich, sehr vorsichtig und leise, wieder hin.

Professor Quirrell lachte ohne Unterlass. „So sei es, Mr. Potter. Ihr Wille geschehe.“ Der Verteidigungsprofessor hielt inne. „Natürlich versprach ich nur \emph{einen} gerissenen Plan. Und das ist auch alles, was Sie drei bekommen werden."

Draco hatte diese Worte zuvor schon halb erwartet, doch nun trafen sie ihn trotzdem wie ein Schock; Draco tauschte einen schnellen Blick mit Granger, sie wären offensichtliche Verbündete gewesen, doch ihre Wünsche waren direkt entgegengesetzt -

„Sie meinen,“ sagte Harry, „wir müssen uns alle auf einen Wunsch einigen?"

„Oh, das wäre doch nun \emph{wirklich} zu viel verlangt,“ sagte Professor Quirrell. „Schließlich haben Sie drei ja keinen gemeinsamen Feind, nicht wahr?"

Und für einen kurzen Augenblick, so schnell dass Draco schon glaubte, er habe es sich eingebildet, zuckten die Augen des Verteidigungsprofessors in Richtung von Dumbledore.

„Nein,“ sagte Professor Quirrell, „ich meine, dass ich drei Wünsche mit einem einzigen Plan gewähren werde."

Es entstand eine verwirrte Stille.

„Das können Sie nicht,“ sagte Harry neben Draco schlicht. „Nicht einmal \emph{ich} könnte das. Zwei dieser Wünsche sind vollkommen inkompatibel. Es ist \emph{logisch unmöglich} -“, dann schnitt sich Harry selbst das Wort ab.

„Sie sind noch ein paar Jahre zu jung, mir zu sagen, was ich nicht tun kann, Mr. Potter,“ sagte Professor Quirrell mit einem kurzen, trockenen Lächeln.

Dann wandte sich der Verteidigungsprofessor wieder der Menge der Schüler zu. „Um die Wahrheit zu sagen, ich hege keinerlei Zuversicht mehr, dass Sie die Lektion des heutigen Tages begreifen werden. Daher fahren Sie heim und genießen Sie die Zeit mit Ihren Familien oder was von ihnen noch übrig ist, solange sie noch am Leben sind. Meine eigene Familie ist schon seit langem tot durch die Hand des Dunklen Lords. Ich sehe Sie wieder zum Schulbeginn.„

In der sprachlosen Stille, die darauf folgte, als Professor Quirrell sich bereits von der Bühne abwandte, vernahm Draco die Stimme des Verteidigungsprofessors, nun leise und nicht länger verstärkt, „Doch mit Ihnen, Mr. Potter, habe ich zu reden.“

