

\hypertarget{egozentrische-verzerrung}{% \section{29. Egozentrische Verzerrung}\label{egozentrische-verzerrung}}

\textbf{Kapitel 29: Egozentrische Verzerrung}\\

Dummerweise ist es schwer, jemandem zu erklären, wer J. K. Rowling ist. Jeder muss sie selbst erleben.

Wissenschafts-Disclaimer: Loshua weist darauf hin, dass die Empathie-Theorie in Kapitel 27 (man nutzt sein eigenes Gehirn, um andere zu simulieren) noch nicht ganz als bekannter wissenschaftlicher Fakt gilt. Die Belege deuten soweit in diese Richtung, doch wir haben die Hirnschaltkreise noch nicht analysiert und es bewiesen. Gleichsam sind die zeitlosen Formulierungen der Quantenmechanik (erwähnt in Kapitel 28) so elegant, dass es mich schockieren würde zu hören, dass die letztendliche Theorie Zeit beinhaltet, aber sie sind ebenfalls noch nicht etabliert.

--------------------------------------------------------------------------------------------------------------------------------------------

Ein flaues Gefühl breitete sich neuerdings in Hermines Magen aus, wann immer sie andere Schüler über sich und Harry sprechen hörte. An diesem Morgen war sie in einer Duschkabine gewesen als sie eine Unterhaltung zwischen Morag und Padma mitangehört hatte, die der letzte Tropfen war, der das Fass endgültig zum Überlaufen gebracht hatte.

Sie glaubte langsam, dass eine Rivalität mit Harry Potter anzufangen, ein furchtbarer Fehler gewesen war.

Wenn sie sich nur von Harry Potter \emph{ferngehalten} hätte, hätte sie Hermine Granger sein können, der hellste Stern am akademischen Himmel von Hogwarts, die mehr Punkte für Ravenclaw verdiente als irgendjemand sonst. Sie wäre nicht \emph{so} berühmt gewesen, wie der Junge-der-überlebt-hat, doch sie wäre berühmt gewesen um \emph{ihrer selbst} willen.

Stattdessen hatte der Junge-der-überlebt-hat eine akademische Rivalin und ihr Name war Hermine Granger.

Und schlimmer noch, sie war mit ihm auf ein Date gegangen.

Die Idee mit Harry eine Romanze zu beginnen, war ihr zunächst verlockend erschienen. Sie hatte solche Bücher gelesen und wenn es irgendjemand in Hogwarts gab, der als Love Interest* der Heldin in Frage kam, war es offensichtlich Harry Potter. Schlau, lustig, berühmt, manchmal unheimlich…

Also hatte sie Harry zu einem Date mit ihr gebracht.

Und nun war \emph{sie} ein Love Interest für \emph{ihn.}

Oder noch schlimmer, eine der Möglichkeiten auf seiner Speisekarte.

Sie war an diesem Morgen in einer Duschkabine gewesen und kurz davor, das Wasser aufzudrehen, als sie von draußen Gekicher gehört hatte. Und sie hatte Morag davon reden hören, dieses muggelgeborene Mädchen würde wahrscheinlich nicht hart genug kämpfen, um gegen Ginevra Weasley zu gewinnen und Padma hatte spekuliert, Harry Potter könne beschließen, dass er \emph{beide} wolle.

Als verstünden sie gar nicht, dass MÄDCHEN die Auswahl auf ihrer Speisekarte hatten und JUNGEN um sie kämpften.

Doch das war noch nicht einmal der Teil, der wirklich weh tat. Das war als sie 98 Punkte in einem von Professor McGonagalls Tests erreicht hatte und die Neuigkeit nicht war, dass Hermine Granger das beste Ergebnis der Klasse erzielt hatte, sondern dass Harry Potters Rivalin sieben Punkte mehr erzielt hatte als er.

Kam man dem Jungen-der-überlebt-hat zu nahe, wurde man Teil seiner Geschichte.

Man bekam nicht seine eigene.

Und Hermine war der Gedanke gekommen, dass sie sich einfach von ihm abwenden sollte, aber das wäre zu traurig gewesen.

Doch sie wollte sich \emph{zurückholen,} was sie unabsichtlich verspielt hatte, als sie als Harrys Rivalin bekannt geworden war. Sie wollte wieder eine eigenständige Person sein, anstatt nur Harrys drittes Bein, war das zu viel verlangt?

Es war schwer aus dieser Falle wieder heraus zu klettern sobald man einmal hinein gefallen war. Egal wie viele Punkte man im Unterricht auch machte, selbst wenn man etwas tat, das eine spezielle Erwähnung beim Abendessen verdiente, es bedeutete nur, dass man wieder Harry Potter Konkurrenz machte.

Doch sie glaubte, ihr sei ein Weg eingefallen.

Etwas zu tun, das \emph{nicht} nur als Tauziehen gegen Harry angesehen würde.

Es würde schwer werden.

Es würde gegen ihre Natur gehen.

Sie würde gegen jemand sehr Böses antreten müssen.

Und sie würde jemand \emph{noch} viel Böseres um Hilfe bitten müssen.

Hermine hob die Hand, um an jene schreckliche Tür zu klopfen.

Sie zögerte.

Hermine wurde klar, dass sie sich \emph{dumm} aufführte und sie hob die Hand etwas höher.

Sie versuchte erneut zu klopfen.

Ihre Hand verfehlte die Tür vollkommen.

Und dann schwang die Tür trotzdem auf.

"Meine Güte," sagte die Spinne in ihrem Netz. "War es wirklich so schwer, einen Quirrell-Punkt zu verlieren, Miss Granger?"

Hermine stand dort mit erhobener Hand und ihre Wangen färbten sich rosa. Das \emph{war} es gewesen.

"Nun, Miss Granger, ich werde gnädig sein," sagte der böse Professor Quirrell. "Betrachten Sie ihn als bereits verloren. Da habe ich Ihnen doch eine schwere Entscheidung abgenommen. Sind Sie nicht dankbar?"

"Professor Quirrell," brachte Hermine mit leicht quiekender Stimme heraus. "Ich habe viele Quirrell-Punkte, nicht wahr?"

"In der Tat haben Sie die," sagte Professor Quirrell. "Wenn auch einen weniger als zuvor. Schrecklich, nicht wahr? Denken Sie nur, wenn mir der Grund für Ihr Hiersein nicht zusagt, könnten Sie noch weitere fünfzig verlieren. Vielleicht ziehe ich sie Ihnen einzeln ab, einen… und noch einen… und noch einen…"

Hermines Wangen wurden noch roter. "Sie sind wirklich böse, hat Ihnen das schon mal jemand gesagt?"

"Miss Granger," sagte Professor Quirrell ernst, "es kann gefährlich sein, Leuten solcherlei Komplimente zu machen, wenn sie nicht wahrhaft verdient wurden. Der Angesprochene könnte sich verlegen fühlen und ungenügend und versucht sein, etwas zu tun, das des Lobes würdig ist. Was also wollten Sie mit mir besprechen, Miss Granger?"

--------------------------------------------------------------------------------------------------------------------------------------------

Es war am Donnerstag-Nachmittag nach dem Mittagessen und Hermine und Harry waren versteckt in einer kleinen Ecke der Bibliothek, mit einem aufgespannten \emph{Quietus}-Feld damit sie reden konnten. Harry lag mit dem Bauch nach unten am Boden, mit aufgestützten Ellenbogen und dem Kopf in den Händen und seine Füße schwangen hinter ihm locker durch die Luft. Hermine besetzte einen gepolsterten Stuhl, der viel zu groß für sie war, als sei sie die Hermine-Füllung inmitten einer Süßigkeit.

Harry hatte vorgeschlagen, sie könnten in einem ersten Durchgang nur die \emph{Titel} aller Bücher in der Bibliothek lesen und danach könnte Hermine alle Inhaltsverzeichnisse durchlesen.

Hermine hatte das für eine brillante Idee gehalten. Das hatte sie zuvor noch nie mit einer Bibliothek gemacht.

Unglücklicherweise wies dieser Plan eine kleine Schwachstelle auf.

Nämlich, dass sie beide Ravenclaws waren.

Hermine las ein Buch namens \emph{Magische Merksprüche.}**

Harry las ein Buch namens \emph{Der Skeptische Zauberer.}**

Beide hatten gedacht, es sei nur eine besondere Ausnahme, die sie nur dieses eine mal machen würden und beiden war bis jetzt nicht klar geworden, dass keiner von ihnen jemals alle Buchtitel fertig lesen würde, egal wie sehr sie es auch versuchten.

Die Stille ihrer kleinen Nische wurde von zwei Worten durchbrochen.

"Oh, \emph{nein,}" sagte Harry plötzlich laut, es klang als seien ihm die Worte entrissen worden.

Ein wenig mehr Stille.

"Hat er \emph{nicht,}" sagte Harry im selben Tonfall.

Dann hörte sie Harry ohnmächtig kichern.

Hermine blickte von ihrem Buch auf.

"Okay," sagte sie, "was \emph{ist?}"

"Ich habe gerade rausgefunden, warum man die Weasleys niemals nach der Familien-Ratte fragt," sagte Harry. "Es ist \emph{wirklich} furchtbar und ich sollte nicht lachen und ich bin ein ganz schrecklicher Mensch."

"Ja," sagte Hermine steif, "bist du. Erzähl's mir schon."

"Okay, zuerst der Hintergrund. In diesem Buch gibt es ein ganzes Kapitel über Sirius-Black-Verschwörungstheorien. Du weißt noch wer das ist, nicht wahr?"

"Natürlich," sagte Hermine. Sirius Black war ein Verräter, ein Freund von James Potter, der Voldemort in das geschützte Heim der Potters eingelassen hatte.

"Nun, es stellt sich raus, es gab da ein paar, sagen wir mal, \emph{Unregelmäßigkeiten} in Zusammenhang damit, dass Black nach Askaban kam. Er bekam kein Verfahren und der Juniorminister, der die Verantwortung trug als die Auroren Black verhaftet haben, war kein anderer als Cornelius Fudge, der unser aktueller Zaubereiminister wurde."

Das klang auch für Hermine etwas verdächtig und das sagte sie auch.

Harry machte mit den Schultern eine zuckende Bewegung während er auf dem Boden liegend in das Buch blickte. "Verdächtige Sachen passieren andauernd und wenn man ein Verschwörungstheoretiker ist, kann man immer \emph{irgendwas} finden."

"Aber \emph{kein Verfahren?}" sagte Hermine.

"Es war unmittelbar nach dem Fall des Dunklen Lords," sagte Harry und seine Stimme klang ernst. "Es war alles unglaublich chaotisch und als die Auroren Black aufgespürt hatten, stand er lachend in einer Straße, knöcheltief in Blut, mit zwanzig Augenzeugen, die berichteten, wie er einen Freund meines Vaters namens Peter Pettigrew und zwölf Passanten getötet hat. Ich sage nicht, dass ich gutheiße, dass Black kein Verfahren bekommen hat. Aber das sind Zauberer von denen wir hier sprechen, also ist es nicht wirklich verdächtiger als, keine Ahnung, die Sachen auf die Leute zeigen, wenn sie darüber streiten, wer John F. Kennedy erschossen hat. Also, jedenfalls ist Sirius Black der Lee Harvey Oswald der Zaubererwelt. Es gibt alle möglichen Verschwörungstheorien darüber, wer meine Eltern an seiner Stelle \emph{wirklich} verraten hat und eine der beliebtesten ist Peter Pettigrew und da wird es jetzt kompliziert."

Hermine lauschte fasziniert. "Aber wie kommt man von da zur \emph{Hausratte} der Weasleys -"

"Warte noch," sagte Harry, "ich komme gleich dazu. Nun, nach Pettigrews Tod kam heraus, dass er ein Spion für die Seite des Lichts gewesen war - kein Doppelagent, nur jemand der herumgeschlichen ist und Sachen rausgefunden hat. Darin war er schon als Teenager gut gewesen, selbst in Hogwarts stand er in dem Ruf, alle möglichen Geheimnisse rauszufinden. Die Verschwörungstheorie ist also, dass Pettigrew ein unregistrierter Animagus wurde, noch während er in Hogwarts war, der sich in etwas Kleines verwandeln konnte, das herumflitzen und Gesprächen lauschen konnte. Das Hauptproblem ist, dass erfolgreiche Animagi selten sind und es als Jugendlicher zu schaffen, wäre wirklich unwahrscheinlich, also besagt die Verschwörungstheorie natürlich, dass mein Vater und Black ebenfalls unregistrierte Animagi waren. Und in dieser Verschwörungstheorie hat Peter Pettigrew selbst die zwölf Passanten getötet, sich in seine kleine Animagus-Form verwandelt und ist geflohen. Nun, Michael Shermer sagt, es gibt vier zusätzliche Probleme dabei. Erstens, Black war der einzige außer meinen Eltern, der wusste, wie man durch die Schutzzauber um ihr Haus gelangt." (Harrys Stimme verhärtete sich leicht, als er das sagte.) "Zweitens, Black war als Verdächtiger von Anfang an wahrscheinlicher als Pettigrew, es gibt Gerüchte, dass Black während seiner Zeit in Hogwarts absichtlich versucht hat, einen Schüler umzubringen und er stammte aus dieser wirklich üblen Reinblüter-Familie, Bellatrix Black war praktisch seine Cousine. Drittens, Black war als magischer Kämpfer etwa zwanzig mal so gut wie Pettigrew, auch wenn er nicht so schlau war. Das Duell zwischen ihnen wäre gewesen wie Professor Quirrell gegen Professor Sprout. Pettigrew bekam wahrscheinlich nicht mal die Chance, seinen Zauberstab zu ziehen, ganz zu schweigen davon, all die Beweise zu fälschen, auf denen die Verschwörungstheorie beruht. Und viertens, Black stand auf der Straße, \emph{lachend.}"

"Aber die \emph{Ratte -}" sagte Hermine.

"Richtig," sagte Harry. "Nun, um es kurz zu machen, Bill Weasley beschloss, die Hausratte seines kleinen Bruders Percy sei Pettigrew in Animagus-Form -"

Hermines Kinnlade klappte runter.

"Ja," sagte Harry, "man würde nicht unbedingt erwarten, dass der böse Pettigrew ein trauriges Leben im Verborgenen als Hausratte einer verfeindeten Zauberer-Familie führt, er wäre entweder bei den Malfoys oder hätte sich, eher wahrscheinlich, in die Karibik abgesetzt nach ein wenig plastischer Chirurgie. Jedenfalls, Bill setzt seinen Bruder Percy außer Gefecht, betäubt und schnappt sich die Ratte, schickt all diese Eulen mit Notfall-Nachrichten los -"

"Oh, \emph{nein!}" entfuhr es Hermine.

"- und schafft es irgendwie Dumbledore, den Zaubereiminister und den Leiter der Aurorenzentrale zusammen zu trommeln -"

"Hat er \emph{nicht!}" sagte Hermine.

"Und natürlich glauben sie, er ist verrückt, als sie ankommen, aber sie testen die Ratte trotzdem mit \emph{Veritas Oculum,}*** nur um sicher zu gehen und was haben sie wohl entdeckt?"

Sie wäre \emph{gestorben.} "Eine Ratte."

"Du hast dir einen Keks verdient! Also haben sie den armen Bill Weasley ins St. Mungo geschleift und es stellte sich raus, er hatte einen ziemlich gewöhnlichen schizophrenen Zusammenbruch, manchen Leuten passiert das einfach, besonders jungen Männern in dem Alter, in dem man bei uns aufs College gehen würde. Der Kerl war überzeugt, er sei siebenundneunzig Jahre alt, wäre gestorben und durch einen Bahnhof in der Zeit zu seinem jüngeren Selbst zurückgereist. Und er hat wirklich gut auf die Antipsychotika angesprochen und ist jetzt wieder normal und alles ist wieder gut, außer dass die Leute nicht mehr so viel über Sirius-Black-Verschwörungstheorien reden und man die Weasleys niemals nach der Familien-Ratte fragt."

Hermine kicherte ohnmächtig. Es war wirklich furchtbar und sie sollte nicht lachen und sie war ein ganz schrecklicher Mensch.

"Was ich \emph{nicht} verstehe," sagte Harry, nachdem ihr Kichern nachgelassen hatte, "ist, warum Black Pettigrew aufspüren sollte, anstatt zu fliehen so schnell er konnte. Er musste wissen, dass die Auroren hinter ihm her waren. Ich frage mich, ob sie den Grund aus Black herausbekommen haben, bevor sie ihn nach Askaban brachten? Siehst du, deswegen durchlaufen Leute, die absolut unbestreitbar schuldig sind, trotzdem das Rechtssystem und bekommen ein Verfahren."

Dem musste Hermine zustimmen.

Bald darauf war Harry fertig mit seinem Buch, während Hermine noch die Hälfte von ihrem vor sich hatte - ihr Buch war sehr viel schwieriger als Harrys, doch trotzdem war es ihr peinlich. Und dann musste sie \emph{Magische} \emph{Merksprüche} zurück ins Regal stellen und sich losreißen, weil es Zeit für sie wurde sich dem meist gefürchteten Unterricht in ganz Hogwarts zu stellen, BESENSTIEL-REITEN.

Harry begleitete sie auf ihrem Weg dorthin, obwohl sein eigener Unterricht erst eineinhalb Stunden später begann, wie ein Kampfjet, der ein trauriges kleines Propellerflugzeug auf dem Weg zu seiner Beerdigung eskortierte.

Der Junge sagte ihr mit leiser, mitfühlender Stimme auf Wiedersehen und sie trat hinaus auf die grasbewachsenen Felder der Verdammnis.

Dort gab es viel Gekreische und Beinahe-Stürze und Begegnungen mit dem Tod und den Boden, der am völlig \emph{falschen Platz} war und die Sonne, die ihr in die Augen schien und Morag, die um sie herum schwirrte und Mandy, die sich für \emph{subtil} hielt bei dem Versuch, immer in ihrer Nähe zu bleiben, um sie zu fangen, falls sie fiel und sie \emph{wusste,} dass die anderen Schüler sie beide auslachten, doch sie sagte nie etwas zu Mandy, weil sie wirklich nicht sterben wollte.

Zehn Millionen Jahre später war der Unterricht vorbei und sie war wieder am Boden, wo sie hingehörte, bis zum nächsten Donnerstag. Manchmal hatte sie Alpträume, es wäre immer Donnerstag.

\emph{Warum} jeder das lernen musste, wenn man als Erwachsener doch einfach Apparieren oder per Flohnetzwerk oder Portschlüssel reisen konnte, wo immer man hin wollte, war Hermine ein vollkommen unlösbares Rätsel. Niemand musste als Erwachsener tatsächlich auf Besenstielen reiten, es war als würde man im Sportunterricht gezwungen, Völkerball zu spielen.

Wenigstens besaß Harry den Anstand, sich angemessen dafür zu schämen, dass er gut darin war.

--------------------------------------------------------------------------------------------------------------------------------------------

Es war ein paar Stunden später und sie saß im Lernzimmer von Hufflepuff, zusammen mit Hannah, Susan, Leanne und Megan. Professor Flitwick hatte, ungewöhnlich zurückhaltend für einen Lehrer, gefragt ob sie diesen vier vielleicht eine Weile bei ihren Hausaufgaben für Zauberkunst behilflich sein könnte, auch wenn sie keine Ravenclaws seien und Hermine wäre vor Stolz fast \emph{geplatzt.}

Hermine nahm ein Stück Pergament, verschüttete etwas Tinte darauf, riss es in vier Teile, zerknitterte sie und warf die Stückchen auf den Tisch.

\emph{Sie} hätte es auch nur mit dem Zerknittern geschafft, doch all das ließ sie deutlich mehr wie Abfall aussehen und das half, wenn jemand zum ersten mal den Entsorgungszauber übte.

Hermine schärfte Augen und Ohren und sagte, "Okay, versucht es."

"\emph{Everto.}"

"\emph{Everto.}"

"\emph{Everto.}"

"\emph{Everto.}"

Hermine glaubte, sie hatte noch nicht wirklich alle Probleme erfasst. "Könnt ihr es alle nochmal versuchen?"

Eine Stunde später hatte Hermine festgestellt, dass (1) Leanne und Megan etwas nachlässig waren, doch wenn man sie bat, etwas zu üben, dann taten sie es, (2) Hannah und Susan fokussiert und getrieben waren, bis an den Punkt, da man sie bitten musste, \emph{langsam zu machen} und \emph{locker zu werden} und Sachen zu \emph{überdenken,} statt sich so zu \emph{überanstrengen} -- der Gedanke, dass diese zwei bald \emph{ihr} gehören würden, war seltsam - und (3) sie es wirklich mochte, Hufflepuffs zu helfen, im ganzen Lernzimmer lag eine sehr fröhliche Stimmung in der Luft.

Als sie zum Abendessen ging, stellte sie fest, dass der Junge-der-überlebt-hat ein Buch lesend darauf wartete, sie zu begleiten. Sie fühlte sich geschmeichelt und auch ein wenig besorgt, denn Harry schien nicht wirklich mit \emph{irgendwem} zu reden, außer ihr.

"Wusstest du, dass es in Hufflepuff ein Mädchen gibt, die ein Metamorphmagus ist?" sagte Hermine als sie Kurs auf die Große Halle nahmen. "Sie macht ihr Haar richtig rot, wie Stoppschild-Rot, nicht Weasley-Rot und als sie sich mal mit heißem Tee bekleckert hat, hat sie sich in einen schwarzhaarigen Jungen verwandelt, bis sie es wieder unter Kontrolle hatte."

"Wirklich? Cool," sagte Harry und klang ein wenig abgelenkt. "Ähm, Hermine, nur um sicherzugehen, du weißt dass morgen der letzte Tag ist, um sich für Professor Quirrells Armeen einzuschreiben, richtig?"

"Ja," sagte Hermine. "Die Armeen des bösen Professor Quirrell." Ihre Stimme klang ein wenig zornig, obwohl Harry natürlich nicht wusste, wieso.

"Hermine," sagte Harry in verärgertem Ton, "er ist nicht böse. Er ist etwas düster und durch und durch Slytherin. Doch das ist nicht dasselbe, wie \emph{böse} zu sein."

Harry Potter hatte zu viele Worte für die Dinge, das war das Problem. Er wäre viel besser damit gefahren, hätte er das Universum einfach in Gut und Böse unterteilt. "Professor Quirrell hat mich vor versammelter Klasse aufgerufen und mir gesagt, ich solle \emph{auf jemanden schießen!}"

"Er hatte recht," sagte Harry sachlich. "Tut mir leid, Hermine, aber das hatte er. Du hättest auf \emph{mich} schießen sollen, es hätte mir nichts ausgemacht. Du kannst Kampf-Magie nicht lernen, wenn du nicht mit echten Zaubern gegen echte Gegner trainieren kannst. Und jetzt machst du dich ganz gut im Sparring, oder?"

Hermine war erst zwölf und so wusste sie es, doch sie konnte es nicht in Worte fassen, sie wusste nichts zu sagen, das Harry überzeugen würde.

Professor Quirrell hatte ein junges Mädchen vor aller Augen gezerrt und ihr befohlen, unprovoziert das Feuer auf einen Klassenkameraden zu eröffnen.

Es \emph{spielte keine Rolle,} ob Professor Quirrell recht hatte, dass sie es lernen musste.

Professor McGonagall hätte das niemals getan.

Professor Flitwick hätte das niemals getan.

Vielleicht hätte sogar Professor Snape das nicht getan.

Professor Quirrell war \emph{BÖSE.}

Doch sie fand die Worte nicht und sie wusste, dass Harry ihr niemals glauben würde.

"Hermine, ich habe mit älteren Schülern gesprochen," sagte Harry. "Professor Quirrell könnte der \emph{einzige} kompetente Verteidigungs-Professor sein, den wir in allen sieben Jahren in Hogwarts bekommen. Alles andere können wir später lernen. Wenn wir Verteidigung studieren wollen, müssen wir es \emph{dieses Jahr} tun. Die Schüler, die sich für das außerschulische Zeug einschreiben, werden eine Menge lernen, weit über das hinaus, wovon das Ministerium glaubt, dass Erstklässler es lernen sollten - wusstest du, dass wir den Patronus-Zauber lernen werden? Im \emph{Januar?}"

"Den \emph{Patronus-Zauber?}" sagte Hermine und ihre Stimme hob sich überrascht.

Ihre Bücher sagten, das war einer der lichtesten Zauber, die man kannte, eine Waffe gegen die dunkelsten Kreaturen, gewirkt mit reinen positiven Gefühlen. Das war nichts, wovon sie erwarten würde, dass der böse Professor Quirrell es unterrichtete - oder den Unterricht dafür arrangierte, denn Hermine konnte sich nicht vorstellen, dass er den Zauber selbst beherrschte.

"Ja," sagte Harry. "Normalerweise lernen Schüler den Patronus-Zauber nicht vor dem fünften Schuljahr oder später! Doch Professor Quirrell sagt, die Lehrpläne des Ministeriums stammten von sprechenden Flubberwürmern und die Fähigkeit, den Patronus-Zauber zu wirken, hängt mehr von Gefühlen ab als von magischer Stärke. Professor Quirrell sagt, die meisten Schüler machen \emph{weit} weniger als sie können und dieses Jahr wird er das beweisen."

Der übliche Ton ehrfürchtiger Verehrung lag in Harrys Stimme als er über Professor Quirrell sprach und Hermine biss die Zähne zusammen und ging weiter.

"Eigentlich habe ich mich schon angemeldet," sagte Hermine, mit etwas kleiner Stimme. "Erst heute morgen. Für alles, wie du gesagt hast."

\emph{Wenn schon, denn schon} hieß es.

Außerdem wollte sie nicht \emph{verlieren} und wenn sie gewinnen wollte, musste sie lernen.

"Dann \emph{bist} du also bei den Armeen dabei?" Harry klang plötzlich enthusiastisch. "Das ist klasse, Hermine! Ich habe die Liste meiner Soldaten schon zusammen, aber ich bin sicher Professor Quirrell erlaubt mir noch einen mehr oder zu tauschen -"

"Ich trete nicht \emph{deiner} Armee bei." Hermines Stimme war scharf. Sie wusste, die Annahme war einleuchtend, doch es ärgerte sie \emph{trotzdem.}

Harry blinzelte. "Nicht der von Draco Malfoy, sicherlich. Du willst also in der dritten Armee sein? Obwohl wir noch gar nicht wissen, wer der General \emph{ist?}" Harry klang überrascht und ein wenig verletzt und sie konnte es ihm nicht vorwerfen, obwohl sie es ihm natürlich vorwarf, immerhin war alles seine Schuld. "Aber warum nicht in meiner?"

"Denk mal drüber nach," schnappte Hermine, "und vielleicht kommst du noch drauf!"

Und sie beschleunigte ihre Schritte und ließ Harry mit offenem Mund hinter sich zurück.

--------------------------------------------------------------------------------------------------------------------------------------------

"Professor Quirrell," sagte Draco in seinem formellsten Tonfall, "ich muss protestieren gegen Ihre Ernennung von Hermine Granger als dritten General."

"Oh?" sagte Professor Quirrell, in seinem Stuhl lässig und entspannt zurückgelehnt. "Protestieren Sie los, Mr. Malfoy."

"Granger ist für die Position ungeeignet," sagte Draco.

Professor Quirrell tippte sich nachdenklich mit dem Finger an die Wange. "Aber ja, ja das ist sie. Haben Sie noch andere Einwände?"

"Professor Quirrell," sagte Harry Potter neben ihm, "bei allem gebührenden Respekt für Miss Grangers außergewöhnliche schulische Talente und die Quirrell-Punkte, die sie in Ihrem Unterricht zurecht verdient hat, ihre Persönlichkeit ist nicht geeignet für eine militärische Führungsposition."

Draco war erleichtert gewesen, als Harry zugestimmt hatte, ihn zu Professor Quirrells Büro zu begleiten. Es lag nicht \emph{nur}daran, dass Harry offenkundig ein riesengroßer Lehrer-Liebling war, was Professor Quirrell anging. Draco hatte auch angefangen, sich zu sorgen, dass Harry \emph{tatsächlich} mit Granger befreundet war, es war jetzt schon eine Weile her und er hatte seinen Zug \emph{noch immer} nicht gemacht… aber das war schon besser.

"Ich stimme Mr. Potter zu," sagte Draco. "Sie zum General zu ernennen, macht daraus eine Farce."

"Unsanft ausgedrückt," sagte Harry, "doch ich kann Mr. Malfoy nicht widersprechen. Um es unverblümt zu sagen, Professor Quirrell, Hermine Granger hat so viel Killerinstinkt wie eine Schüssel feuchter Weintrauben."

"Das," sagte Professor Quirrell nachsichtig, "ist nichts, was zu bemerken mir selbst entgehen würde. Sie sagen mir nichts, das ich nicht bereits wüsste."

Draco war an der Reihe, etwas zu sagen, doch die Unterredung war plötzlich ins Stocken geraten. Diese Antwort war \emph{keine} der Möglichkeiten gewesen, die er und Harry in ihrem Brainstorming erarbeitet hatten, bevor sie hierher kamen. Was \emph{sagte} man, nachdem einem der Lehrer mitteilte, er wisse alles, was man selbst wüsste und er trotzdem im Begriff stand, einen offensichtlichen Fehler zu begehen?

Die Stille dehnte sich aus.

"Steckt dahinter irgendein Plan?" sagte Harry langsam.

"Muss hinter allem, was ich tue, irgendein Plan stecken?" sagte Professor Quirrell. "Kann ich nicht einmal Chaos erzeugen, nur um des Chaos willen?"

Draco verschluckte sich fast.

"Nicht in ihrem Kampf-Magie-Unterricht," sagte Harry mit ruhiger Gewissheit. "Andernorts, vielleicht, doch nicht dort."

Professor Quirrell hob langsam die Augenbrauen.

Harry blickte fest auf ihn zurück.

Draco erschauerte.

"Nun denn," sagte Professor Quirrell. "Keiner von Ihnen scheint eine sehr simple Frage zu berücksichtigen. Wen könnte ich anstelle von Miss Granger ernennen?"

"Blaise Zabini," sagte Draco ohne zu zögern.

"Irgendwelche anderen Vorschläge?" sagte Professor Quirrell und klang deutlich amüsiert.

\emph{Anthony Goldstein und Ernie Macmillan,} kam es ihm in den Sinn, bevor sein gesunder Menschenverstand einsetzte und alle Schlammblüter und Hufflepuffs ausschloss, egal wie aggressiv sie sich auch duellieren mochten. Also sagte Draco stattdessen, "Was stimmt nicht mit Zabini?"

"Ich \emph{verstehe…}" sagte Harry langsam.

"Ich \emph{nicht,}" sagte Draco. "Wieso nicht Zabini?"

Professor Quirrell sah Draco an. "Weil, Mr. Malfoy, egal wie sehr er es auch versucht, er niemals mit Ihnen oder Mr. Potter wird mithalten können."

Der Schock brachte Draco ins Wanken. "Sie können nicht ernsthaft glauben, dass \emph{Granger -}"

"Er spekuliert auf sie," sagte Harry leise. "Es gibt keine Garantie. Die Chancen stehen nicht einmal gut. Sie wird uns wahrscheinlich niemals einen guten Kampf liefern und selbst wenn doch, wird sie Monate brauchen, um es zu lernen. Doch sie ist die einzige in unserem Jahrgang, die überhaupt eine Chance hat, uns eines Tages zu schlagen."

Dracos Hände zuckten, doch er ballte sie nicht zu Fäusten. Als Unterstützer aufzutreten und dann einen Rückzieher zu machen, war eine klassische Unterminierungstaktik, also \emph{steckte} Potter mit Granger unter einer Decke und \emph{das} bedeutete -\\ "Aber Professor," fuhr Harry nahtlos fort, "ich habe Sorge, dass Hermine der Posten als General einer Armee nicht \emph{gut tun} wird. Ich spreche jetzt als ihr Freund, Professor Quirrell. Die Konkurrenz mag gut sein für Draco und mich, doch worum Sie sie bitten, ist nicht gut für \emph{sie!}"

Schon gut.

"Ihre Freundschaft für Hermine Granger ist Ihnen hoch anzurechnen," sagte Professor Quirrell trocken. "Besonders da Sie es fertig bringen, zur gleichen Zeit auch noch mit Draco Malfoy befreundet zu sein. Wahrlich eine Leistung, das."

Harry sah plötzlich nervös aus, was bedeutete, er war wahrscheinlich noch deutlich \emph{nervöser} und Draco fluchte innerlich. Natürlich würde Harry Professor Quirrell nicht täuschen.

"Und ich bezweifle, Miss Granger würde ihre freundschaftliche Sorge zu schätzen wissen," sagte Professor Quirrell. "Sie hat mich um die Position gebeten, ich kam nicht auf sie zu."

Harry blieb einen Moment lang still. Dann schoss er Draco einen schnellen Blick zu, entschuldigend und warnend zugleich, der besagte, \emph{Tut mir leid, ich hab mein bestes gegeben} und \emph{Wir} \emph{lassen es besser gut sein.}

"Bezüglich dessen, es würde ihr nicht gut tun," fuhr Professor Quirrell fort, ein leichtes Lächeln spielte nun um seine Lippen, "so vermute ich, die Lasten ihrer Position dürften ihr um einiges leichter fallen, als Sie beide vermuten und sie wird sehr viel schneller einen guten Kampf liefern, als Sie glauben."

Harry und Draco keuchten beide auf vor Entsetzen.

"Sie \emph{beraten} sie doch wohl nicht?" sagte Draco völlig fassungslos.

"Ich habe mich nie dafür gemeldet, gegen \emph{Sie} anzutreten!" sagte Harry.

Das Lächeln, das Professor Quirrells Lippen umspielte, wurde breiter. "Wie es der Zufall will, habe ich \emph{tatsächlich} angeboten, Miss Granger einige Vorschläge für ihre ersten Schlachten zu unterbreiten."

"\emph{Professor Quirrell!}" sagte Harry.

"Oh, keine Sorge," sagte Professor Quirrell. "Sie hat abgelehnt. Genau wie ich erwartet hatte."

Dracos Augen verengten sich.

"Meine Güte, Mr. Potter," sagte Professor Quirrell, "hat Ihnen noch niemand gesagt, dass es unhöflich ist, Leute anzustarren?"

"Sie werden ihr doch nicht insgeheim auf \emph{andere} Weise behilflich sein, oder?" sagte Harry.

"Würde ich das tun?" sagte Professor Quirrell.

"Ja," sagten Draco und Harry gleichzeitig.

"Ihr Mangel an Vertrauen verletzt mich. Nun denn, ich verspreche General Granger in keiner Weise zu helfen, von der Sie beide nicht wissen. Und nun schlage ich vor, Sie wenden sich Ihren militärischen Belangen zu. Der November rückt näher und das schnell."

--------------------------------------------------------------------------------------------------------------------------------------------

Draco war sich im Klaren, was das bedeutete, noch bevor sich die Tür zu Professor Quirrells Büro zur Gänze hinter ihnen geschlossen hatte.

Harry hatte einst abfällig von "sozialem Zeugs" gesprochen.

Und nun war das Dracos einzige Hoffnung.

Lass es ihn nicht merken, lass es ihn nicht merken…

"Wir sollten General Granger zuerst angreifen und sie aus dem Weg schaffen," sagte Draco. "Nachdem wir sie zerquetschen, können wir uns selbst miteinander messen, ohne Ablenkungen."

"Das scheint ihr gegenüber aber nicht fair zu sein, oder?" sagte Harry nachsichtig.

"Was kümmert's \emph{dich?}" sagte Draco. "Sie ist deine Rivalin, richtig?" Dann, mit genau der richtigen Note von Misstrauen in der Stimme, "Sag mir nicht, du magst sie langsam \emph{wirklich,} nachdem du so lange ihr Rivale warst…"

"Bei den Gründern, nein," sagte Harry. "Was soll ich sagen, Draco? Ich habe einfach einen natürlichenSinn für Gerechtigkeit. Granger ebenfalls, wie du weißt. Sie hat sehr feste Ansichten, was Gut und Böse betrifft und wahrscheinlich wird sie sich zuerst auf das Böse stürzen. Und mit einem Namen wie 'Malfoy' bettelt man geradezu darum, nicht wahr."

\emph{VERDAMMT!}

"Harry," sagte Draco und klang verletzt und vielleicht ein wenig herablassend, "willst du nicht \emph{fair} gegen mich kämpfen?"

"Du meinst, statt dich anzugreifen, nachdem du bereits einige deiner Truppen beim Sieg gegen Granger verloren hast?" sagte Harry. "Oh, ich weiß nicht. Vielleicht, wenn es mich irgendwann langweilt, einfach nur zu gewinnen, versuche ich's mal mit dieser 'fair'-Sache."

"Vielleicht wird sie auch\emph{dich} angreifen," sagte Draco. "\emph{Du} bist ihr Rivale."

"Aber ihr \emph{freundlicher} Rivale," sagte Harry mit einem bösen Grinsen. "Ich hab ihr ein nettes Geburtstagsgeschenk gekauft und alles. Man sabotiert seinen freundlichen Rivalen nicht einfach so."

"Was ist damit, die Chance eines \emph{Freundes} auf einen fairen Kampf zu sabotieren?" sagte Draco aufgebracht. "Ich dachte, wir wären Freunde!"

"Lass mich das anders formulieren," sagte Harry. "\emph{Granger} würde einen freundlichen Rivalen nicht sabotieren. Aber deshalb, weil sie den Killerinstinkt einer Schüssel feuchter Weintrauben hat. \emph{Du} würdest. Und \emph{wie} du das würdest. Und weißt du was, ich auch."

\emph{VERDAMMT!}

--------------------------------------------------------------------------------------------------------------------------------------------

Wäre das ein Theaterstück gewesen, hätte an dieser Stelle dramatische Musik eingesetzt.

Der Held, tadellos gewandet in einen grün-getrimmten Umhang und mit perfekt gekämmtem weiß-blondem Haar, trat der Schurkin gegenüber.

Die Schurkin, zurückgelehnt in einem schlichten hölzernen Stuhl, mit klar erkennbaren Hasenzähnen und kastanienbraunen Locken, die über ihre Wangen fielen, begegnete dem Helden.

Es war Mittwoch, der 30. Oktober und die erste Schlacht würde am Sonntag beginnen.

Draco stand in General Grangers Büro, einem Raum von der Größe eines kleinen Klassenzimmers. (Draco war nicht ganz sicher, \emph{warum} die Büros aller Generäle so groß waren. Ein Stuhl und ein Schreibtisch hätten ihm genügt. Ihm erschloss sich nicht einmal, warum die Generäle überhaupt Büros benötigten, seine Soldaten wussten, wo er zu finden war. Außer, Professor Quirrell hätte es absichtlich so arrangiert, damit die riesigen Büros als Zeichen ihres Status dienten, in welchem Fall Draco absolut dafür war.)

Granger saß auf dem einzigen Stuhl im Raum, wie auf einem Thron, ganz am anderen Ende des Büros gegenüber der Tür. Ein langer Tisch erstreckte sich in der Mitte des Raumes zwischen ihnen und vier kleinere kreisrunde Tische waren über die Ecken verstreut, doch es gab nur diesen einen Stuhl, ganz am anderen Ende. An der Wand des Raumes entlang erstreckten sich Fenster und ein Sonnenstrahl erschien an der Spitze von Grangers Kopf, glühend wie eine Krone.

Es wäre nett gewesen, wenn Draco gemessen hätte voran schreiten können. Doch es war ein Tisch im Weg und Draco musste schräg darum herum gehen und es gab keinen guten Weg, dies auf dramatische und würdevolle Weise zu bewerkstelligen. War das Absicht gewesen? Wenn es sich um Vater gehandelt hätte, mit Sicherheit; doch das hier war Granger, also sicher nicht.

Er konnte sich nirgendwo hinsetzen und Granger war auch nicht aufgestanden.

Draco verbannte die Empörung zur Gänze aus seinem Gesicht.

"Nun, Mr. Draco Malfoy," sagte Granger sobald er vor ihr stand, "Sie haben um eine Audienz bei mir gebeten und ich war so gütig, sie zu gewähren. Was ist Ihr Begehr?"

\emph{Komm doch mal} \emph{zu Besuch ins Herrenhaus der Malfoys, mein Vater und ich würden dir gern ein paar interessante Zaubersprüche zeigen.}

"Ihr Rivale, Potter, ist mit einem Angebot an mich heran getreten," sagte Draco und setzte eine ernste Miene auf. "Es macht ihm nichts aus, gegen mich zu verlieren, doch würden Sie gewinnen, wäre es eine Demütigung für ihn. Daher will er sich mit mir zusammenschließen und Sie unverzüglich auslöschen, nicht nur in unserer ersten Schlacht, sondern in allen von ihnen. Wenn ich dazu nicht bereit bin, will Potter, dass ich mich zurückhalte oder Sie in Bedrängnis bringe, während er in einem Erstschlag mit allen Kräften einen Angriff auf Sie ausführt."

"Ich verstehe," sagte Granger und sah überrascht aus. "Und Sie bieten an, mir gegen ihn zur Seite zu stehen?"

"Natürlich," sagte Draco geschmeidig. "Ich hielt das, was er mit Ihnen machen wollte, für nicht gerade fair."

"Nun, das ist aber sehr nett von Ihnen, Mr. Malfoy," sagte Granger. "Es tut mir leid, wie ich zuvor mit Ihnen gesprochen habe. Wir sollten Freunde sein. Darf ich dich Drakey nennen?"

Alarmglocken schrillten los in Dracos Kopf, doch es bestand die \emph{Möglichkeit,} dass sie es ernst meinte…\\ "Natürlich," sagte Draco, "wenn ich dich Hermy nennen darf."

Draco war ziemlich sicher, erhabe ein Flackern auf ihrem Gesicht bemerkt.

"Jedenfalls," sagte Draco, "dachte ich, es würde Potter recht geschehen, wenn wir ihn beide angriffen und ausradierten."

"Aber das wäre nicht fair gegenüber Mr. Potter, oder?" sagte Granger.

"Ich halte es für außerordentlich fair," sagte Draco. "Er plante, es Ihnen zuerst anzutun."

Granger warf ihm einen strengen Blick zu, dem ihm wahrscheinlich Angst eingejagt hätte, wenn er ein Hufflepuff gewesen wäre und kein Malfoy. "Sie halten mich für ziemlich dämlich, nicht wahr, Mr. Malfoy?"

Draco lächelte charmant. "Nein, Miss Granger, doch ich dachte, einen \emph{Versuch} wäre es wert. Also, was wollen Sie?"

"Bieten Sie an, mich zu \emph{bestechen?}" sagte Granger.

"Sicher," sagte Draco. "Kann ich Ihnen einfach eine Galleone zukommen lassen, damit Sie den Rest des Jahres Potter fertig machen, statt mich?"

"Nope," sagte Granger, "doch Sie können mir zehn Galleonen anbieten, damit ich Sie beide gleichermaßen angreife, anstatt nur Sie."

"Zehn Galleonen sind eine Menge Geld," sagte Draco vorsichtig.

"Ich wusste gar nicht, dass die Malfoys so arm sind," sagte Granger.

Draco starrte Granger an.

Er bekam langsam ein komisches Gefühl dabei.

Diese bestimmte Antwort klang nicht, als hätte sie von diesem bestimmten Mädchen kommen sollen.

"Nun," sagte Draco, "man wird nicht reich, indem man Geld verschwendet, wissen Sie."

"Ich weiß nicht, ob Sie wissen, was ein Zahnarztist, Mr. Malfoy, doch meine Eltern sind \emph{Zahnärzte} und alles, was unter zehn Galleonen liegt, ist meine Zeit gar nicht wert."

"Drei Galleonen," sagte Draco, mehr als Versuch denn alles andere.

"Nope," sagte Granger. "Wenn Sie überhaupt einen ausgeglichenen Kampf wollen, so glaube ich nicht, dass ein Malfoy einen ausgeglichenen Kampf weniger begehrt als zehn Galleonen."

Draco bekam ein \emph{sehr} merkwürdiges Gefühl bei der Sache.

"Nein," sagte Draco.

"Nein?" sagte Granger. "Dies ist ein zeitlich befristetes Angebot, Mr. Malfoy. Sind Sie sicher, Sie wollen riskieren, vom Jungen-der-überlebt-hat ein ganzes Jahr lang jämmerlich in den Boden gestampft zu werden? Das wäre ziemlich peinlich für das Haus Malfoy, nicht wahr?"

Es war ein bestechendes Argument, eines das sich schwer verneinen ließ, doch man wurde nicht reich, indem man Geld ausgab, wenn einem sein Herz sagte, es war eine Falle.

"Nein," sagte Draco.

"Wir sehen uns am Sonntag," sagte Granger.

Draco wandte sich um und verließ das Büro ohne ein weiteres Wort.

Das war \emph{nicht so} gelaufen, wie es sollte…

--------------------------------------------------------------------------------------------------------------------------------------------

"Hermine," sagte Harry geduldig, "wir \emph{sollen} uns gegeneinander verschwören. Du könntest mich sogar hintergehen und es würde außerhalb des Schlachtfelds überhaupt nichts bedeuten."

Hermine schüttelte den Kopf. "Es wäre nicht nett, Harry."

Harry seufzte. "Ich glaube, du hast ganz und gar nicht die richtige Einstellung dafür."

\emph{Es wäre nicht nett.} Sie hatte das tatsächlich gesagt. Hermine wusste nicht, ob sie empört darüber sein sollte, wie Harry über sie dachte oder besorgt, ob sie gewöhnlich \emph{wirklich} so sehr wie ein kleines unschuldiges Lämmchen klang.

Es war wahrscheinlich Zeit, das Thema zu wechseln.

"Na jedenfalls, hast du morgen irgendwas besonderes vor?" sagte Hermine. "Es ist -"

Sie schnitt sich abrupt das Wort ab, als es ihr klar wurde.

"Ja, Hermine," sagte Harry etwas verkniffen, "welcher Tag ist morgen?"

--------------------------------------------------------------------------------------------------------------------------------------------

\emph{Zwischenspiel:}

Es hatte einmal eine Zeit gegeben, da war der 31. Oktober im magischen Britannien noch als Halloween bezeichnet worden.

Jetzt war es der Harry-Potter-Tag.

Harry hatte alle Einladungen ausgeschlagen, selbst die von Minister Fudge, die vielleicht gut für zukünftige politische Gefälligkeiten gewesen wäre und die er wirklich mit zusammengebissenen Zähnen hätte annehmen sollen. Aber für Harry würde der 31. Oktober immer der Der-Dunkle-Lord-hat-meine-Eltern-getötet-Tag sein. Es hätte irgendwo eine kleine, würdevolle Gedenkfeier geben sollen und wenn es eine gab, so war er nicht eingeladen worden.

Hogwarts bekam den Tag frei, um zu feiern. Selbst die Slytherins wagten es nicht, außerhalb ihres eigenen Schlafsaales Schwarz zu tragen. Es gab spezielle Events und besonderes Essen und die Lehrer schauten weg, wenn jemand in den Gängen rannte. Immerhin war es der zehnte Jahrestag.

Harry verbrachte den Tag in seinem Koffer, damit er ihn allen anderen nicht verdarb, aß Müsliriegel anstelle der Mahlzeiten, las ein paar seiner traurigeren Science-Fiction-Bücher (keine Fantasy) und schrieb einen Brief an Mum und Dad, der sehr viel länger ausfiel als die, die er üblicherweise schickte.

* Ich denke, ich lasse den englischen Begriff hier so stehen, da die Übersetzungen alle irgendwie… falsch klingen und der Begriff inzwischen auch im Deutschen geläufig sein sollte. Wer nichts damit anfangen kann: Als \emph{Love Interest} wird gewöhnlich ein Charakter bezeichnet, dessen hauptsächliche (oft ausschließliche) Funktion in einer Geschichte es ist, eine Liebesbeziehung mit dem Hauptcharakter einzugehen.\\ ** engl.: \emph{Magical Mnemonics}und \emph{The Skeptical Wizard}\\ *** Der Zauberspruch scheint in den Original-Romanen nicht erwähnt zu werden; der Wortbedeutung nach macht er die Wahrheit sichtbar, zeigt einem in diesem Fall also wohl die wahre Gestalt eines Wesens oder Menschen. Im Original wird ebenfalls ein Zauber erwähnt, der einen Animagus in seine menschliche Form zwingen kann, doch die konkrete Zauberformel wird nicht genannt.

