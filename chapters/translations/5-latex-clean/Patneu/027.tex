

\hypertarget{empathie}{% \section{27. Empathie}\label{empathie}}

\textbf{Kapitel 27: Empathie}

J. K. Rowling ist zu 87\% sicher, dass du in Flammen aufgehen wirst.

Roger Bacon lebte im 13. Jahrhundert und wird als einer der ersten Verfechter der wissenschaftlichen Methode angesehen. Einem Wissenschaftler sein experimentelles Tagebuch zu überreichen, ist etwa so, als gäbe man einem Schriftsteller den Stift, nicht von Shakespeare, sondern von jemandem, der an der Erfindung des Schreibens beteiligt war.

\later

Es kam nicht jeden Tag vor, dass man Harry Potter betteln sah.

„\emph{Biiiiitte,}“ heulte Harry Potter.

Fred und George schüttelten grinsend erneut die Köpfe.

Ein gequälter Ausdruck lag auf Harry Potters Gesicht. „Aber ich habe euch \emph{erzählt}, wie ich das mit Kevin Entwhistles Katze gemacht habe und mit Hermine und der verschwindenden Limo und ich \emph{kann} euch nicht vom Sprechenden Hut oder dem Erinnermich oder Professor Snape erzählen…“

Fred und George zuckten mit den Schultern und gingen.

„Wenn du es jemals herausbekommst,“ sagten die Weasley-Zwillinge, „dann lass es uns wissen.“

„\emph{Ihr seid böse! Ihr seid beide böse!}“

Fred und George schlossen fest die Tür zu dem leeren Klassenraum hinter ihnen und behielten das Grinsen noch eine Weile länger auf ihren Gesichtern, nur für den Fall, dass Harry Potter durch Türen sehen konnte.

Dann bogen sie um eine Ecke und ihre Gesichter fielen in sich zusammen.

„Ich nehme nicht an, von dem worauf Harry getippt hat—“

„- sind dir irgendwelche Ideen gekommen?“ sagten sie gleichzeitig zueinander und ließen dann die Schultern noch weiter hängen.

Das letzte woran sie sich erinnern konnten, war Flume, der sich weigerte ihnen zu helfen, doch sie konnten sich nicht erinnern, \emph{worum} sie ihn gebeten hatten…

… doch sie mussten sich anderswo umgehört und \emph{irgendjemanden} gefunden haben, der ihnen bei \emph{irgendwas} illegalem geholfen hatte, sonst hätten sie nicht zugestimmt, dass man hinterher ihre Erinnerung löschte.

Wie \emph{hatten} sie bloß all das mit nur vierzig Galleonen schaffen können?

Zunächst waren sie besorgt, sie hätten so gute Beweise gefälscht, dass Harry am Ende \emph{tatsächlich} Ginny würde heiraten müssen… doch daran hatten sie, wie es schien, auch gedacht. Die Verhandlungen des Zaubergamots waren \emph{noch einmal} manipuliert worden, um sie wieder in den ursprünglichen Zustand zurückzuversetzen, der gefälschte Verlobungs-Vertrag war aus seinem von Drachen bewachten Verlies in Gringotts verschwunden und so weiter. Ehrlich gesagt, war es ziemlich beängstigend. Die meisten Leute glaubten jetzt, der \emph{Tagesprophet} habe sich die ganze Sache aus unerfindlichen Gründen ausgedacht und der \emph{Klitterer} hatte das Messer hilfreicherweise noch tiefer in die Wunde gebohrt, mit der Schlagzeile des nächsten Tages, HARRY POTTER HEIMLICH VERLOBT MIT LUNA LOVEGOOD.

Wen immer sie auch angeheuert hatten, sie hofften verzweifelt, er würde es ihnen verraten, nachdem die Verjährungsfrist um war. Doch in der Zwischenzeit war es entsetzlich, sie hatten den großartigsten Streich aller Zeiten abgezogen, vielleicht den größten Streich in der Geschichte des Streichespielens und sie \emph{konnten sich nicht erinnern, wie.} Es war wie verhext, sie hatten sich doch beim \emph{ersten} mal etwas einfallen lassen können, warum also konnten sie es jetzt nicht erkennen, nachdem sie von alldem \emph{wussten}, was sie getan hatten?

Ihr einziger Trost war, dass Harry nicht wusste, dass sie es nicht wussten.

Nicht einmal Mum hatte sie dazu befragt, trotz der offensichtlichen Verbindung zu den Weasleys. Was immer getan worden war, es lag weit außer Reichweite irgendeines Hogwarts-Schülers… außer \emph{einem} vielleicht, der es, wenn bestimmte Gerüchte stimmten, getan haben könnte, indem er mit den Fingern schnippte. \emph{Harry} war unter Veritaserum befragt worden, er hatte es ihnen erzählt… in Anwesenheit von Dumbledore, der den Auroren einschüchternde Blicke zuwarf. Die Auroren hatten gerade genug Fragen gestellt, um sicherzugehen, dass Harry den Streich nicht selbst durchgezogen oder irgendjemanden hatte verschwinden lassen und dann gemacht, dass sie aus Hogwarts herauskamen.

Fred und George hatten sich gefragt, ob sie beleidigt sein sollten, dass Harry Potter von den Auroren für \emph{ihren} Streich befragt wurde, doch der Ausdruck auf \emph{Harrys} Gesicht, wahrscheinlich aus dem selben Grund, machte alles wett.

Wenig überraschend, waren Rita Kimmkorn und der Redakteur des \emph{Tagespropheten} beide verschwunden und mittlerweile wahrscheinlich außer Landes. Von diesem Teil \emph{hätten} sie ihrer Familie gern erzählt. Dad hätte ihnen, so glaubten sie, gratuliert, nachdem Mum damit fertig war, sie umzubringen und Ginny ihre Überreste verbrannt hatte.

Doch es war trotzdem in Ordnung, sie würden es Dad eines Tages erzählen und in der Zwischenzeit…

… in der Zwischenzeit hatte Dumbledore, während er im Flur an ihnen vorbeiging, geniest und dabei aus Versehen ein kleines Päckchen aus seiner Tasche fallen lassen und darin befunden hatten sich zwei identische Fluchbrecher-Monokel von \emph{unglaublicher} Qualität. Die Weasley-Zwillinge hatten ihre neuen Monokel am „verbotenen“ Korridor im dritten Stock ausprobiert, einen kleinen Trip zum magischen Spiegel und zurück und sie hatten nicht \emph{alle}Sicherheitsvorkehrungen klar erkennen können, doch die Monokel hatten \emph{viel} mehr gezeigt als sie beim ersten mal gesehen hatten.

Natürlich würden sie sehr vorsichtig sein müssen, sich niemals mit den Monokeln in ihrem Besitz erwischen zu lassen oder sie würden im Büro des Schulleiters enden und sich eine strenge Gardinenpredigt anhören müssen und vielleicht sogar mit Schulverweis bedroht.

Es war gut zu wissen, dass nicht jeder, der nach Gryffindor sortiert wurde, zu Professor McGonagall heran wuchs.

\later

Harry befand sich in einem weißen Raum, fensterlos, ohne auffällige Merkmale und saß vor einem Schreibtisch, gegenüber einem ausdruckslosen Mann im formellen, tiefschwarzen Umhang.

Der Raum war abgeschirmt und der Mann hatte genau siebenundzwanzig Zauber ausgeführt, bevor er auch nur sagte „Hallo, Mr~Potter.“

Es schien seltsam angemessen, dass der Mann in schwarz versuchen würde, in Harrys Geist zu lesen.

„Bereiten Sie sich vor,“ sagte der Mann tonlos.

Ein menschlicher Geist, hatte Harrys Okklumentik-Buch besagt, lag einem Legilimentor nur entlang bestimmter \emph{Oberflächen} offen. Wenn man seine Oberflächen nicht schützen konnte, würde der Legilimentor \emph{hindurch} gehen und auf jeden Teil des eigenen Geistes zugreifen können, den zu verstehen er in der Lage war…

… was tendenziell nicht sehr viel war. Der menschliche Geist, so schien es, war für Menschen auf mehr als der oberflächlichsten Stufe schwer zu begreifen. Harry hatte sich gefragt, ob eine Menge über kognitive Wissenschaft zu wissen, ihn zu einem unglaublich mächtigen Legilimentor machen würde, doch wiederholte Erfahrung hatte ihm \emph{schlussendlich} die Lektion eingeprügelt, in solchen Angelegenheiten nicht allzu erwartungsvoll zu sein. Es war auch nicht so, als verstünde irgendein Geisteswissenschaftler Menschen gut genug, um einen zu machen.

Um die Abwehr, Okklumentik, zu erlernen, war der erste Schritt, sich selbst in seiner Vorstellung als eine andere Person zu sehen, es so konsequent vorzugeben, wie nur möglich, sich selbst vollkommen in diese alternative Persönlichkeit zu versenken. Man würde das nicht immer tun müssen, doch zu Beginn, lernte man auf diese Weise, wo die eigenen Oberflächen lagen. Der Legilimentor würde einen zu lesen versuchen und man würde spüren wie es geschah, wenn man genau genug darauf achtete, würde man den Versuch seines Eindringens erkennen. Und die eigene Aufgabe war es, dafür zu sorgen, dass er immer mit der eigenen imaginären Persönlichkeit in Berührung kam und nicht mit der wirklichen.

Wenn man gut genug darin war, konnte man sich vorstellen, eine sehr \emph{einfache} Person zu sein, etwa ein Stein und es sich zur Gewohnheit machen, diese Vorgabe anstelle all seiner Oberflächen zu belassen. Das war eine Standard-Okklumentik-Barriere. Vorzugeben, ein Stein zu sein, war schwer zu lernen, doch im Nachhinein einfach auszuführen und die frei liegende Oberfläche eines Geistes war viel flacher als das Innere, mit genug Übung konnte man sie also als gewohnheitsmäßig im Hintergrund aufrechterhalten.

Oder, wenn man ein \emph{perfekter Okklumentiker} war, konnte man jedem Eindringen \emph{voraus}eilen, die Anfragen so schnell beantworten, wie sie gestellt wurden, so dass der Legilimentor durch die Oberflächen stoßen und einen Geist betrachten würde, der ununterscheidbar war, von dem desjenigen, wer immer man zu sein vorgab.

Selbst der beste Legilimentor konnte so getäuscht werden. Wenn ein perfekter Okklumentiker behauptete, er ließe seine Okklumentik-Barrieren fallen, konnte man unmöglich wissen, ob er log. Oder schlimmer noch, man mochte gar nicht wissen, dass man es mit einem perfekten Okklumentiker zu tun hatte. Sie waren selten, doch die Tatsache ihrer Existenz bedeutete, dass Legilimentik bei \emph{niemandem} verlässlich war.

Es war traurig, was es darüber aussagte, wie wenig menschliche Wesen einander verstanden, wie wenig ein Zauberer von den unterhalb der geistigen Oberfläche liegenden Tiefen begriff, dass man die besten menschlichen Telepathen täuschen konnte, indem man vorgab, jemand anders zu sein.

Doch dann wieder verstanden Menschen einander überhaupt nur, indem sie etwas vorgaben. Man machte keine Vorhersagen über Menschen, indem man sich die Billionen von Synapsen als einzelne Objekte ausmalte. Man bitte den besten Manipulationskünstler auf Erden, einem eine künstliche Intelligenz zu entwerfen und er würde einen nur verständnislos anglotzen. Man sagte Menschen vorher, indem man das \emph{eigene} Gehirn anwies, sich wie ihres zu verhalten. Man \emph{versetzte sich selbst an ihre Stelle.} Wenn man wissen wollte, was eine zornige Person tun würde, aktivierte man die Bereiche für Zorn im eigenen Gehirn und je nach dem was sie ausspuckten, fiel die Vorhersage aus. Wie funktionierten die Gehirnbereiche für Zorn im Innersten? Wer wusste das schon? Der beste Manipulationskünstler auf Erden mochte gar nicht wissen, was Neuronen \emph{waren} und ebenso wenig der beste Legilimentor.

Alles was ein Legilimentor \emph{verstehen} konnte, konnte ein Okklumentiker \emph{vorgeben} zu sein. Es war in jedem Fall der selbe Trick—wahrscheinlich in beiden Fällen ausgeführt von den selben neuronalen Schaltkreisen, eine einzige Gruppe von Kontrollschaltkreisen für die Rekonfiguration des Gehirns, damit es als Modell für das eines anderen agierte.

Und so war das Rennen zwischen telepathischem Angriff und telepathischer Verteidigung ein entschiedener Sieg für die Verteidigung. Ansonsten wäre die gesamte magische Welt, vielleicht sogar die ganze Erde, ein sehr andersartiger Ort gewesen…

Harry atmete tief ein und konzentrierte sich. Ein dünnes Lächeln lag auf seinem Gesicht.

\emph{Einmal,} nur \emph{ein einziges Mal,} war Harry, was mysteriöse Kräfte betraf, nicht enttäuscht worden.

Nach fast einem Monat Arbeit und mehr aus einer Laune heraus als aufgrund eines tatsächlichen Verdachts, hatte Harry entschieden, sich nachdem er absichtlich kalt und zornig wurde, noch einmal an den Übungen des Okklumentik-Buches zu versuchen. An jenem Punkt hatte er die Hoffnung, was solche Sachen betraf, schon fast aufgegeben, doch es schien trotzdem einen schnellen Versuch wert—

Er hatte alle der schwersten Übungen des Buches in zwei Stunden durchgespielt und am nächsten Tag hatte er Professor Quirrell mitgeteilt, er sei bereit.

Seine dunkle Seite, so hatte sich herausgestellt, war sehr, \emph{sehr} gut darin, vorzugeben, jemand anders zu sein.

Harry dachte an seinen Standard-Auslöser, das erste mal als er gänzlich auf seine dunkle Seite hinüber gewechselt war…

\emph{Severus hielt inne, sah sehr selbstzufrieden aus. "Und das wären… fünf Punkte? Nein, machen wir lieber zehn Punkte von Ravenclaw daraus, für Widerworte.„}

Harrys Lächeln wurde kühler und er betrachtete den Mann im schwarzen Umhang, der glaubte, er würde gleich Harrys Geist lesen.

Und dann wurde Harry zu jemand vollkommen anderem, jemandem der ihm für diesen Anlass angemessen erschienen war.

… in einem weißen Raum, fensterlos, ohne auffällige Merkmale und saß vor einem Schreibtisch, gegenüber einem ausdruckslosen Mann im formellen, tiefschwarzen Umhang.

Kimball Kinnison betrachtete den Mann im schwarzen Umhang, der glaubte, er würde gleich den Geist eines Lensman der Galaktischen Patrouille des zweiten Imperiums lesen.

Zu sagen, dass Kimball Kinnison zuversichtlich war, was den Ausgang dessen betraf, wäre eine Untertreibung gewesen. Er war trainiert worden von Mentor von Arisia, dem mächtigsten Geist dieses oder jedes anderen Universums und der bloße Zauberer, der ihm gegenüber saß, würde genau das sehen, was der Graue Herrscher ihn sehen lassen wollte…

… den Geist des Jungen, als der er sich tarnte, ein unschuldiges Kind namens Harry Potter.

„Ich bin bereit,“ sagte Kimball Kinnison in nervösem Tonfall, der genau angemessen war für einen elfjährigen Jungen.

„\emph{Legilimens,}“ sagte der Zauberer im schwarzen Umhang.

Es gab eine Pause.

Der Zauberer im schwarzen Umhang blinzelte, als habe er etwas so schockierendes gesehen, dass es sogar ausgereicht hatte, \emph{seine} Augenlider zum Zucken zu bringen. Seine Stimme war nicht mehr ganz so tonlos als er sagte, „Der Junge-der-überlebt-hat hat eine \emph{mysteriöse dunkle Seite?}“

Die Hitze kroch Harry langsam in die Wangen.

„Nun,“ sagte der Mann. Sein Gesicht nun wieder völlig gesetzt. „Verzeihen Sie. Mr~Potter, es ist gut ihre Stärken zu kennen, doch das ist nicht dasselbe, wie allzu sehr auf sie zu vertrauen. Sie mögen in der Tat imstande sein, die Okklumentik bereits im Alter von elf Jahren zu erlernen. Das erstaunt mich. Ich hatte geglaubt, Mr~Dumbledore gäbe wieder einmal vor, verrückt zu sein. Ihr dissoziatives Talent ist so groß, dass ich überrascht bin, keine anderen Anzeichen von Kindesmissbrauch vorzufinden und Sie könnten beizeiten ein perfekter Okklumentiker werden. Doch es besteht ein erheblicher Unterschied dazu, zu erwarten, bei Ihrem ersten Versuch eine erfolgreiche Okklumentik-Barriere errichten zu können. Das ist einfach nur lächerlich. Haben Sie irgendetwas gespürt als ich Ihren Geist las?“

Harry schüttelte den Kopf, wurde nun rot vor Zorn.

„Dann passen Sie beim nächsten mal besser auf. Das Ziel ist nicht, an ihrem ersten Unterrichtstag ein perfektes Bild zu erschaffen. Das Ziel ist, zu lernen, wo Ihre Oberflächen liegen. Bereiten Sie sich vor.“

Harry versuchte wieder vorzugeben Kimball Kinnison zu sein, versuchte sich stärker zu konzentrieren, doch seine Gedanken waren leicht zerstreut und er war sich plötzlich all der Dinge nur allzu bewusst, an die er nicht denken sollte…

Oh, würde das ätzend.

Harry biss die Zähne zusammen. Zumindest würde die Erinnerung des Ausbilders hinterher gelöscht.

"\emph{Legilimens}.„

Es gab eine Pause—

\later

… in einem weißen Raum, fensterlos, ohne auffällige Merkmale und saß vor einem Schreibtisch, gegenüber einem ausdruckslosen Mann im formellen, tiefschwarzen Umhang.

Es war ihr vierter Tag, am Sonntagabend. Wenn man so viel bezahlte, bekam man seine Sitzungen zu jeder verdammten Zeit, die man wollte, das Konzept von Wochenenden spielte keine Rolle.

„Hallo, Mr~Potter,“ sagte der Telepath tonlos, nachdem er die volle Garnitur von Privatsphäre-Zaubern gewirkt hatte.

„Hallo, Mr~Bester,“ sagte Harry erschöpft. „Wollen wir den anfänglichen Schock gleich hinter uns bringen?“

„Sie konnten mich überraschen?“ sagte der Mann, klang nun leicht interessiert. „Nun denn.“ Er deutete mit dem Zauberstab und starrte Harry in die Augen. „\emph{Legilimens.}“

Es gab eine Pause, dann zuckte der Zauberer im schwarzen Umhang zurück, als habe ihn jemand mit einem Viehtreiber berührt.

„Der Dunkle Lord ist \emph{am Leben?}“ stieß er erstickt hervor. Seine Augen blickten plötzlich wild umher. „\emph{Dumbledore macht sich unsichtbar und schleicht sich in Mädchen-Schlafsäle?}“

Harry seufzte und blickte auf seine Armbanduhr. In etwa drei Sekunden…

„Also,“ sagte der Mann und hatte seine Tonlosigkeit noch nicht ganz zurück erlangt. „Sie glauben wirklich, Sie werden die geheimen Gesetze der Magie entdecken und allmächtig werden.“

„Das stimmt,“ sagte Harry gleichgültig, noch immer auf seine Uhr blickend. „\emph{So} vermessen bin ich.“

„Ich staune. Es scheint, der Sprechende Hut glaubt, sie werden der nächste Dunkle Lord werden.“

„Und \emph{Sie} wissen, ich gebe mir alle Mühe, es \emph{nicht} zu sein und Sie sahen, dass wir bereits eine lange Diskussion darüber hatten, ob Sie mir Okklumentik beibringen wollen und Sie am Ende entschieden haben, es zu tun, also können wir einfach weitermachen?“

„In Ordnung,“ sagte der Mann exakt sechs Sekunden später, genau wie beim letzten mal. „Bereiten Sie sich vor.“ Er hielt inne und sagte dann, mit ziemlich wehmütiger Stimme, „Obwohl ich \emph{wünschte}, ich könnte mich an den Trick mit dem Gold und Silber erinnern.“

Harry fand es sehr verstörend, wie reproduzierbar menschliche Gedanken waren, wenn man Leute in die gleichen Anfangsbedingungen zurückversetzte und sie den selben Reizen aussetzte. Es entzauberte Illusionen, die man als guter Reduktionist überhaupt nicht erst hätte haben sollen.

\later

Harry war in reichlich schlechter Laune als er am nächsten Montagmorgen aus seinem Kräuterkunde-Unterricht stampfte.

Hermine schäumte neben ihm her.

Die anderen Kinder waren noch drinnen, sammelten etwas langsam ihre Sachen zusammen, weil sie aufgeregt darüber brabbelten, dass Ravenclaw das zweite Quidditch-Spiel des Jahres gewonnen hatte.

Wie es aussah, war letzte Nacht nach dem Abendessen ein Mädchen dreißig Minuten lang auf einem Besenstiel herum geflogen und hatte dann eine Art riesigen Moskito gefangen. Es gab noch andere Fakten über das, was während des Spiels passiert war, doch sie waren irrelevant.

Harry hatte dieses aufregende sportliche Ereignis wegen seiner Okklumentik-Stunden verpasst und weil er außerdem ein Leben hatte.

Er war daraufhin allen Unterhaltungen im Ravenclaw-Schlafsaal aus dem Weg gegangen, waren Stillezauber und magische Koffer nicht wunderbar. Gefrühstückt hatte er am Gryffindor-Tisch.

Doch Kräuterkunde konnte Harry nicht vermeiden und die Ravenclaws hatten vor dem Unterricht und nach dem Unterricht und \emph{während dem} \emph{Unterricht} darüber geredet, bis Harry von dem Baby-Pelziger*, dessen Windel er gerade wechselte, aufblickte und laut verkündete, einige von ihnen versuchten etwas über \emph{Pflanzen} zu lernen und Schnatze wuchsen an überhaupt nichts, also könnten sie \emph{bitte} aufhören über Quidditch zu reden. Alle anderen hatten ihm schockierte Blicke zugeworfen, außer Hermine, die gewirkt hatte, als wolle sie applaudieren und Professor Sprout, die ihm einen Punkt für Ravenclaw verliehen hatte.

\emph{Einen} Punkt für Ravenclaw.

\emph{Einen} Punkt.

Die sieben Idioten auf ihre idiotischen Besen, die ihr idiotisches Spiel spielten, hatten \emph{einhundertneunzig Punkte} für Ravenclaw geholt.

Offenbar wurden Quidditch-Ergebnisse \emph{direkt auf die Hauspunkte angerechnet.}

Mit anderen Worten war einen kleinen goldenen Moskito zu fangen 150 Hauspunkte wert.

Harry konnte sich nicht einmal \emph{vorstellen}, was er würde tun müssen, um einhundertfünfzig Hauspunkte zu verdienen.

Außer, na ja, \emph{hundertfünfzig Hufflepuffs} zu retten oder sich \emph{fünfzehn Ideen einfallen zu lassen, die so gut waren, wie Zeitmaschinen in Schutzhüllen zu stecken} oder sich \emph{eintausendfünfhundert kreative Arten, Leute umzubringen}, auszudenken oder das \emph{ganze Jahr lang} Hermine Granger zu sein.

„Wir sollten sie umbringen,“ sagte Harry zu Hermine, die neben ihm ging mit gleichermaßen zorniger Ausstrahlung.

„Wen?“ sagte Hermine, „Die Quidditch-Mannschaft?“

„Ich dachte an jeden, der irgendwo, irgendwie, irgendwas mit Quidditch zu tun hat, aber die Ravenclaw-Mannschaft wäre ein guter Anfang, ja.“

Hermines Lippen verzogen sich missbilligend. „Du \emph{weißt}, dass Leute umzubringen falsch ist, Harry?“

„Ja,“ sagte Harry.

„Okay, wollte nur sichergehen,“ sagte Hermine. „Nehmen wir uns zuerst die Sucherin vor. Ich habe ein paar Agatha Christie-Romane gelesen, weißt du, wie wir sie in einen Zug locken können?“

„Zwei Schüler planen ein Mordkomplott,“ sagte eine trockene Stimme. „Wie schockierend.“

Um eine nahe liegende Ecke herum strich ein Mann in leicht beflecktem Umhang, sein öliges Haar fiel lang und ungekämmt über seine Schultern. Tödliche Gefahr schien von ihm auszuströmen, schien den Gang zu fluten mit Vorstellungen von ungeschickt zubereiteten Zaubertränken und versehentlichen Ausrutschern und Leuten, die in ihren Betten starben, an etwas was die Auroren als natürliche Umstände ansehen würden.

Ohne auch nur darüber nachzudenken, trat Harry vor Hermine.

Ein Luftholen erklang hinter ihm und einen Moment später rauschte Hermine an ihm vorbei und trat vor \emph{ihn}. „Lauf, Harry!“ sagte sie. „Jungs sollten sich nicht in Gefahr begeben müssen.“

Severus Snape lächelte freudlos. „Amüsant. Ich bitte um einen Augenblick Ihrer Zeit, Potter, wenn Sie sich von Ihren Liebeleien mit Miss~Granger losreißen können.“

Plötzlich lag ein sehr besorgter Ausdruck auf Hermines Gesicht. Sie wandte sich Harry zu und öffnete den Mund, dann hielt sie inne, mit gequältem Blick.

„Oh, keine Sorge, Miss~Granger,“ sagte Severus glatte Stimme. „Ich verspreche, Ihren Liebsten unversehrt zurückzubringen.“ Sein Lächeln verschwand. „Nun werden Potter und ich verschwinden und uns privat unterhalten, allein. Ich hoffe es ist klar, dass Sie nicht eingeladen sind, doch nur für den Fall, betrachten Sie es als Anordnung eines Professors von Hogwarts. Ich bin sicher, ein gutes kleines Mädchen wie Sie wird nicht ungehorsam sein.“

Und Severus drehte sich um und verschwand wieder hinter der Ecke. „Kommen Sie, Potter?“ sagte seine Stimme.

„Ähm,“ sagte Harry zu Hermine. „Kann ich einfach gehen und ihm folgen und es \emph{du} überlegst dir, was ich sagen sollte, damit du nicht allzu besorgt und gekränkt bist?“

„Nein,“ sagte Hermine mit zitternder Stimme.

Severus Gelächter hallte von hinter der Ecke wieder.

Harry senkte den Kopf. „Tut mir leid,“ sagte er leise, „wirklich,“ und er folgte dem Meister der Zaubertränke nach.

\later

„Also,“ sagte Harry. Es waren jetzt keine anderen Geräusche zu hören als zwei paar Beine, lang und kurz, die irgendeinen Steinkorridor entlang stapften. Der Meister der Zaubertränke schritt zügig voran, doch nicht zu schnell für Harry, um mitzuhalten und insoweit Harry das Hogwarts eigene Konzept der Direktionalität begriff, entfernten sie sich von den belebten Bereichen. „Worum geht's hier?“

„Ich nehme nicht an, Sie könnten erklären,“ sagte Severus trocken, „wieso Sie zwei planten, Cho Chang zu ermorden?“

„Ich nehme nicht an, \emph{Sie} könnten erklären,“ sagte Harry trocken, „in Ihrer Eigenschaft als offizieller Repräsentant des Hogwarts-Schulsystems, warum einen goldenen Moskito zu fangen als akademische Leistung angesehen wird, die einhundertfünfzig Hauspunkte wert ist?“

Ein Lächeln strich über Severus Lippen. „Meine Güte und ich nahm an, Sie seien scharfsinnig. Sind Sie wahrhaftig so unfähig, Ihre Klassenkameraden zu verstehen, Potter oder verachten Sie sie zu sehr, um es zu versuchen? Würden Quidditch-Ergebnisse nicht für den Hauspokal zählen, dann würde keiner von ihnen sich überhaupt um die Hauspunkte scheren. Es wäre nur ein obskurer Wettbewerb für Schüler wie Sie und Miss~Granger.“

Das war eine schockierend gute Antwort.

Und dieser Schock weckte Harrys Geist erst richtig auf.

Im Rückblick hätte es nicht überraschend sein sollen, dass Severus seine Schüler verstand, sie sogar sehr gut verstand.

Er hatte ihre Gedanken gelesen.

Und…

… das Buch hatte gesagt, ein erfolgreicher Legilimentor sei extrem selten, seltener als ein perfekter Okklumentiker, weil fast niemand die erforderliche geistige Disziplin besaß.

\emph{Geistige Disziplin?}

Harry hatte Geschichten gesammelt über einen Mann, der regelmäßig im Unterricht die Beherrschung verlor und kleinen Kindern gegenüber in die Luft ging.

… doch der selbe Mann hatte, als Harry davon sprach, dass der Dunkle Lord noch am Leben war, sofort und perfekt reagiert—exakt so reagiert, wie es jemand tun würde, der vollkommen unwissend war.

Der Mann strich durch Hogwarts mit der Aura eines Attentäters, strahlte die Gefahr nur so aus…

… was exakt das wäre, was ein echter Attentäter \emph{nicht} tun würde. Echte Attentäter sollten wie biedere kleine Buchhalter wirken, bis sie einen töteten.

Er war der Hauslehrer des stolzen und aristokratischen Slytherin und er trug einen Umhang, verschmutzt mit Flecken von Zaubertränken und Zutaten, die mit Magie in zwei Minuten entfernt sein könnten.

Harry erkannte, dass er verwirrt war.

Und seine Einschätzung der Bedrohung dievom\emph{Hauslehrer von Slytherin}ausging, schoss in astronomische Höhen.

Dumbledore schien zu denken, Severus gehöre ihm und nichts dem widersprechendes war geschehen; der Meister der Zaubertränke war „Angst einflößend aber nicht missbräuchlich“ gewesen, wie versprochen. Also, hatte Harry zuvor geschlossen, war dies Angelegenheit der Gefährten. Wenn Severus plante, ihm etwas zu leide zu tun, hätte er Harry sicher nicht vor den Augen von Hermine geholt, einer Zeugin, wenn er einfach hätte warten können, bis Harry allein war…

Harry biss sich unbemerkt auf die Lippe.

„Ich kannte einmal einen Jungen, der wahrlich in Quidditch vernarrt war,“ sagte Severus Snape. „Er war ein Vollidiot. Genau wie Sie und ich erwarten würden, wir zwei.“

„Worum \emph{geht's} hier?“ sagte Harry langsam.

„Geduld, Potter.“

Severus drehte den Kopf und glitt dann auf seine Attentäter-Art in eine nahe gelegene Öffnung in der Wand des Korridors, die in einen kleineren und schmaleren Flur führte.

Harry folgte ihm, fragte sich ob es nicht klüger wäre, einfach wegzulaufen.

Sie bogen ab und noch ein weiteres mal und kamen an eine Sackgasse, nur eine leere Wand. Wenn Hogwarts tatsächlich gebaut, anstatt einfach beschworen oder geboren worden war oder was auch immer, würde Harry ein paar scharfe Worte an den Architekten richten, über das Bauen von Gängen, die nirgendwohin führten.

„\emph{Quietus,}“ sagte Severus und noch ein paar andere Dinge.

Harry lehnte sich zurück, die Arme vor der Brust verschränkt und beobachtete Severus Gesicht.

„Sie sehen mir in die Augen, Potter?“ sagte Severus Snape. „Ihre Okklumentik-Stunden können nicht weit genug fortgeschritten sein, damit Sie Legilimentik abwehren könnten. Doch vielleicht sind sie fortgeschritten genug, um sie zu entdecken. Da ich nichts gegenteiliges wissen kann, werde ich keinen Versuch riskieren.“ Der Mann lächelte dünn. „Und dasselbe wird für Dumbledore gelten, denke ich. Weshalb wir \emph{jetzt} diese kleine Unterhaltung führen.“

Harrys Augen weiteten sich unwillkürlich.

„Zu Anfang,“ sagte Severus, mit funkelnden Augen, „sollte ich Ihnen das Versprechen abnehmen, zu \emph{niemandem} über unsere Unterhaltungen zu sprechen. Soweit es die Schule betrifft, diskutieren wir Ihre Hausaufgaben für Zaubertränke. Ob man es für unwichtig hält oder nicht. Soweit es Dumbledore und McGonagall betrifft, verletze ich Draco Malfoys Vertrauen in mich und keiner von uns hält es für angemessen, weiter ins Detail zu gehen.“

Harrys Hirn versuchte die Tragweite und Implikationen dessen zu berechnen und ihm ging der Arbeitsspeicher aus.

„Nun?“ sagte der Meister der Zaubertränke.

„In Ordnung,“ sagte Harry langsam. Es war schwer zu erkennen, wie eine Unterhaltung zu führen und niemandem davon erzählen zu können, einen stärker einschränken konnte, als sie \emph{nicht} zu führen, in welchem Fall man den Inhalt \emph{ebenfalls} nicht preisgeben konnte. „Ich verspreche es.“

Severus beobachtete Harry eindringlich. „Sie sagten einst im Büro des Schulleiters, Sie würden keinerlei Mobbing oder Missbrauch dulden. Und so frage ich mich, Harry Potter. Wie sehr ähneln Sie wohl Ihrem Vater?“

„Insoweit wir hier nicht von Michael Verres-Evans sprechen,“ sagte Harry, „ist die Antwort, dass ich sehr wenig über James Potter weiß.“

Severus nickte, wie zu sich selbst. „Es gibt da einen Slytherin-Fünftklässler. Einen Jungen namens Lesath Lestrange. Er wird von Gryffindors gemobbt. Meine Möglichkeiten, mich mit solchen Situationen zu befassen, sind… begrenzt. \emph{Sie} könnten ihm vielleicht helfen. Wenn Sie es wollten. Ich bitte Sie nicht um einen Gefallen und werde Ihnen auch keinen schuldig sein. Es ist nur eine Gelegenheit, zu tun was Sie wollen.“

Harry starrte Severus nachdenklich an.

„Fragen Sie sich, ob es eine Falle ist?“ sagte Severus, ein schwaches Lächeln strich über seine Lippen. „Ist es nicht. Es \emph{ist} ein Test. Nennen Sie es Neugier meinerseits. Doch Lesaths Probleme sind echt, wie auch meine eigenen Schwierigkeiten zu intervenieren.“

Das war das Problem wenn andere Leute wussten, dass man ein Guter war. Selbst wenn man wusste, dass sie es wussten, konnte man den Köder nicht einfach ignorieren.

Und wenn sein Vater ebenfalls Schüler vor Mobbern beschützt hatte… es spielte keine Rolle, ob Harry wusste, wieso Snape es ihm erzählt hatte. Es gab ihm noch immer ein warmes und stolzes Gefühl im Inneren und machte es ihm unmöglich, sich einfach abzuwenden.

„Schön,“ sagte Harry. „Erzählen Sie mir von Lesath. Warum wird er gemobbt?“

Das schwache Lächeln verschwand von Severus Gesicht. „Sie glauben, es gibt \emph{Gründe}, Potter?“

„Vielleicht nicht,“ sagte Harry leise. „Doch mir kam der Gedanke, er könnte vielleicht ein unwichtiges Schlammblut-Mädchen die Treppe runter gestoßen haben.“

„Lesath Lestrange,“ sagte Severus, nun mit kalter Stimme, „ist der Sohn von Bellatrix Lestrange, der fanatischsten und bösartigsten Dienerin des Dunklen Lords. Lesath ist der anerkannte Bastard von Rabastan Lestrange. Kurz nach dem Tod des Dunklen Lords wurden Bellatrix und Rabastan und Rabastans Bruder Rodolphus gefangen genommen, während sie Alice und Frank Longbottom folterten. Alle drei sind lebenslang in Askaban. Die Longbottoms wurden durch wiederholten Einsatz des Cruciatus-Fluchs in den Wahnsinn getrieben und verbleiben auf der St. Mungo-Station für unheilbar Versehrte. Ist irgendetwas davon ein guter Grund ihn zu mobben, Potter?“

„Es ist überhaupt kein Grund,“ sagte Harry, noch immer leise. „Und Lesath selbst hat nichts Falsches getan, von dem Sie wüssten?“

Das schwache Lächeln zeigte sich erneut auf Severus Lippen. „Er ist nicht mehr ein Heiliger als irgendjemand sonst. Doch er hat keine Schlammblut-Mädchen die Treppe runter gestoßen, nicht soweit ich gehört hätte.“

„Oder in seinem Geist gesehen,“ sagte Harry.

Severus Ausdruck war kühl. „Ich habe seine Privatsphäre nicht verletzt, Potter. Viel mehr habe ich in die Gryffindors hinein gesehen. Er ist einfach ein bequemes Ziel für ihre kleinen Genugtuungen.“

Ein Schwall aus kaltem Zorn lief Harrys Rücken hinab und er musste sich in Erinnerung rufen, dass Severus keine vertrauenswürdige Informationsquelle sein mochte.

„Und Sie glauben,“ sagte Harry, „dass eine Intervention durch Harry Potter, den Jungen-der-überlebt-hat, sich als effektiv erweisen könnte.“

„In der Tat,“ sagte Severus Snape und teilte Harry mit, wann und wo die Gryffindors ihr nächstes kleines Spielchen planten.

\later

Es gibt einen Hauptgang, der auf der Nord-Süd-Achse quer durch den zweiten Stock von Hogwarts führt und zur Mitte dieses Ganges hin gibt es eine kleine Öffnung zu einem kurzen Korridor, der ein Dutzend Schritte zurück führt, bis er in einem rechten Winkel zu einer L-Form abknickt und dann ein Dutzend Schritte weiterführt, bevor er an einem hellen, breiten Fenster endet, von dem man drei Geschosse hinab auf den leichten Nieselregen blicken kann, der über den östlichen Ländereien von Hogwarts fällt. Wenn man am Fenster steht, kann man keine Geräusche aus dem Hauptgang hören und niemand im Hauptgang würde bemerken, was am Fenster vor sich ging. Wenn einem daran irgendetwas seltsam vorkam, war man noch nicht sehr lange in Hogwarts.

Vier Jungen in rot-getrimmten Umhängen lachen und ein Junge im grün-getrimmten Umhang schreit und klammert sich verzweifelt mit den Händen an den Rahmen des geöffneten Fensters, während die vier Jungen so tun, als würden sie ihn hinaus werfen. Es ist, natürlich, nur ein Scherz und außerdem, ein Sturz aus dieser Höhe würde einen Zauberer nicht umbringen. Alles guter, sauberer Spaß. Wenn man irgendwas daran für seltsam hält—

„\emph{Was macht ihr da?}“ sagte die Stimme eines sechsten Jungen.

Die vier Jungen in rot-getrimmten Umhängen wirbeln mit plötzlichem Schrecken herum und der Junge im grün-getrimmten Umhang stößt sich verzweifelt vom Fenster ab und fällt zu Boden, sein Gesicht tränenüberströmt.

„Oh,“ sagte der gutaussehendste der Jungen in rot-getrimmten Umhängen und klang erleichtert, „\emph{du} bist es. Hey, Lessy, weißt du, wer das ist?“

Es kommt keine Antwort von dem Jungen am Boden, der versucht sein Schniefen unter Kontrolle zu bekommen und ein Junge im rot-getrimmten Umhang zieht seinen Fuß zu einem Tritt zurück—

„\emph{Aufhören!}“ ruft der sechste Junge.

Der Junge im rot-getrimmten Umhang gerät ins Schwanken als er seinen Tritt abbricht. „Ähm,“ sagte er, „weißt \emph{du}, wer das ist?“

Der Atem des sechsten Jungen klingt merkwürdig. „Lesath Lestrange,“ sagt er, sein Atem geht in kleinen Stößen, „und \emph{er} hat meinen Eltern nichts angetan, er war fünf Jahre alt.“

\later

Neville Longbottom starrte die vier riesigen Fünftklässler-Mobber vor sich an, sehr darum bemüht, sein Zittern unter Kontrolle zu halten.

Er hätte Harry Potter einfach nein sagen sollen.

„Warum beschützt \emph{du} ihn?“ sagte der Gutaussehende, langsam, klang verwirrt und zeigte erste Anzeichen von Empörung. „Er ist ein \emph{Slytherin}. Und ein \emph{Lestrange.}“

„Er ist ein Junge, der seine Eltern verloren hat,“ sagte Neville Longbottom. „Ich weiß, wie das ist.“ Er wusste nicht wo diese Worte hergekommen waren. Es klang zu cool, wie etwas, das Harry Potter sagen würde.

Das Zittern jedoch blieb.

„Was glaubst du, \emph{wer} du \emph{bist?}“ sagte der Gutaussehende und klang langsam zornig.

\emph{Ich bin Neville, der letzte Spross des Noblen und Uralten Hauses Longbottom—}

Neville konnte es nicht sagen.

„Ich glaube, er ist ein \emph{Verräter,}“ sagte einer der anderen Gryffindors und Neville verspürte plötzlich ein flaues Gefühl im Magen.

Er hatte es gewusst, er hatte es einfach gewusst. Harry Potter hatte sich doch geirrt. Mobber hörten nicht auf, nur weil Neville Longbottom es ihnen sagte.

Der Gutaussehende trat einen Schritt vor und die anderen drei folgten.

„Solche seid ihr also,“ sagte Neville, verwundert wie fest seine Stimme klang. „Es spielt für euch keine Rolle, ob Lesath Lestrange oder Neville Longbottom.“

Lesath Lestrange ließ, von dort wo er auf dem Boden lag, ein plötzliches Keuchen hören.

„Böse ist böse,“ knurrte der selbe Junge, der zuvor gesprochen hatte, „und wenn du ein Freund des Bösen bist, bist du auch böse.“

Die vier traten einen weiteren Schritt vor.

Lesath kam, schwankend, auf die Füße. Sein Gesicht war grau und er machte ein paar Schritte nach vorn, lehnte sich gegen die Wand und sagte nichts. Seine Augen klebten an der Biegung des Ganges, seinem Ausweg.

„Freunde,“ sagte Neville. Seine Stimme wurde jetzt etwas höher. „Ja, ich habe Freunde. Einer von ihnen ist der Junge-der-überlebt-hat.“

Ein paar der Gryffindors blickten plötzlich besorgt drein. Der Gutaussehende ließ sich nichts anmerken. „Harry Potter ist nicht hier,“ sagte er mit harter Stimme. „und wenn er es wäre, glaube ich nicht, dass er Gefallen fände an einem Longbottom, der einen Lestrange verteidigt.“

Und die Gryffindors traten einen weiteren langen Schritt nach vorn und hinter ihnen schob Lesath sich an der Wand entlang, wartete auf seine Chance.

Neville schluckte und hob seine rechte Hand mit zusammengepresstem Daumen und Zeigefinger.

Er schloss die Augen, denn er hatte Harry Potter versprechen müssen, nicht zu linsen.

Wenn das nicht funktionierte, würde er nie wieder irgendwem vertrauen.

Seine Stimme erklang, angesichts der Umstände, überraschend klar.

„Harry James Potter-Evans-Verres. Harry James Potter-Evans-Verres. Harry James Potter-Evans-Verres. Bei deiner Schuld mir gegenüber und deinem wahren Namen beschwöre ich dich, ich öffne den Weg für dich, ich rufe dich an, erscheine vor mir.“

Neville schnippte mit den Fingern.

Und dann öffnete Neville die Augen.

Lesath Lestrange starrte ihn an.

Die vier Gryffindors starrten ihn an.

Der Gutaussehende fing an zu kichern und das erlöste die anderen drei.

„Sollte Harry Potter etwa um die Ecke kommen oder so?“ sagte der Gutaussehende. „Oh. Sieht aus, als wärst du geleimt worden.“

Der Gutaussehende machte einen bedrohlichen Schritt auf Neville zu.

Die anderen drei folgten im Schlepptau.

„Ähem,“ sagte Harry Potter hinter ihnen, lehnte an der Mauer neben dem Fenster, am Ende des Ganges, wo unmöglich irgendjemand ungesehen hätte hin gelangen können.

Wenn Leute schreien zu sehen sich immer so gut anfühlte, konnte Neville irgendwie verstehen, warum Leute zu Mobbern wurden.

Harry Potter schritt voran, stellte sich zwischen Lesath Lestrange und die anderen. Er ließ seinen eisigen Blick über die Jungen in den rot-getrimmten Umhängen schweifen, dann blieben seine Augen auf dem Gutaussehenden ruhen, dem Rädelsführer. „Mr~Carl Sloper,“ sagte Harry Potter. „Ich glaube, ich habe diese Situation zur Gänze verstanden. Wenn Lesath Lestrange selbst auch nur eine einzige böse Tat zu verantworten hat, abgesehen davon, den falschen Eltern geboren worden zu sein, so ist\emph{Ihnen} davon nichts bekannt. Wenn ich mich in diesem Punkt irren sollte, Mr~Sloper, schlage ich vor, Sie unterrichten mich umgehend.“

Neville sah die Angst und die Ehrfurcht auf den Gesichtern der anderen Jungen. Er fühlte es selbst. Harry hatte \emph{behauptet}, es sei alles ein Trick, doch wie sollte das möglich sein?

„Aber er ist ein \emph{Lestrange,}“ sagte der Anführer.

„Er ist ein Junge, der \emph{seine Eltern verloren hat,}“ sagte Harry Potter und seine Stimme wurde noch kälter.

Diesmal zuckten alle anderen drei Gryffindors zusammen.

„Also,“ sagte Harry Potter. „Ihr habt gesehen, dass Neville nicht wünschte, dass ihr einen unschuldigen Jungen im Namen der Longbottoms quält. Das konnte euch nicht bewegen. Wenn ich euch sagte, dass der Junge-der-überlebt-hat \emph{ebenfalls} denkt, ihr seid im Unrecht, dass euer heutiges Handeln ein furchtbarer Fehler war, macht das einen Unterschied?“

Der Anführer machte einen Schritt auf Harry zu.

Die anderen folgten ihm \emph{nicht}.

„Carl,“ sagte einer von ihnen, schluckend. „Vielleicht sollten wir gehen.“

„Man sagt, du wirst der nächste Dunkle Lord sein,“ sagte der Rädelsführer und starrte Harry an.

Ein Grinsen huschte über Harry Potters Gesicht. „Man sagt auch, ich wäre heimlich mit Ginevra Weasley verlobt und es gäbe eine Prophezeiung, wir würden zusammen Frankreich erobern.“ Das Lächeln verblasste. Da Sie offenbar auf der Angelegenheit beharren wollen, Mr~Carl Sloper, lassen Sie mich das klarstellen. \emph{Lassen Sie Lesath in Ruhe.} Wenn nicht, werde ich es erfahren.“

„Also hat Lessy bei dir gepetzt,“ sagte der Anführer kalt.

„Sicher,“ sagte Harry Potter kalt, „und er hat mir auch erzählt, was ihr heute nach dem Zauberkunst-Unterricht getan habt, an einem privaten, abgeschiedenen Ort, wo euch niemand sehen konnte, mit einem bestimmten Hufflepuff-Mädchen, das eine weiße Schleife im Haar trägt—“

Die Kinnlade des Anführers fiel schockiert herab.

„Iip,“ machte einer der anderen Gryffindors mit hochgestochener Stimme und wirbelte auf dem Absatz herum und verschwand um die Ecke. Seine Schritte trappelten rasch davon und verklangen.

Und da waren es sechs.

„Ah,“ sagte Harry Potter, „da geht ein einigermaßen intelligenter junger Mann. Der Rest von euch könnte sich ein Beispiel an Bertram Kirke nehmen, bevor ihr, sagen wir, in Schwierigkeiten kommt.“**

„Drohst du, uns zu verpetzen?“ sagte der gutaussehende Gryffindor und versuchte zornig zu klingen, mit eher zittriger Stimme. „Petzen stoßen schlimme Dinge zu.“

Die anderen zwei Gryffindors setzten langsam zurück.

Harry Potter lachte auf. „Oh, das hast du jetzt nicht gesagt. Versuchst du \emph{wirklich} mich einzuschüchtern? \emph{Mich?} Jetzt ernsthaft, hältst du dich für furchteinflößender als Peregrine Derrick, Severus Snape oder was das betrifft, Du-weißt-schon-wen?“

Selbst der Anführer zuckte dabei zusammen.

Harry Potter hob die Hand, die Finger bereit und alle drei Gryffindors machten einen Satz zurück und einer von ihnen platzte heraus „Nicht - !“

„Seht ihr,“ sagte Harry Potter, „an diesem Punkt schnippe ich mit den Fingern und ihr werdet Teil einer urkomischen Geschichte, die heute beim Abendessen mit viel nervösem Gelächter erzählt wird. Doch die Sache ist die, Leute denen ich vertraue, sagen mir immer wieder, ich solle das nicht tun, Professor McGonagall sagte mir, ich nähme immer den leichtesten Ausweg und Professor Quirrell sagt, ich muss lernen, wie man verliert. Also, erinnert ihr euch an die Geschichte, wo ich mich von ein paar älteren Slytherins habe aufmischen lassen? Das könnten wir machen. Ihr könntet mich eine Weile mobben und ich könnte euch lassen. Nur, erinnert ihr euch an den Teil am Ende, wenn ich meinen vielen, vielen Freunden in dieser Schule sage, sie sollen nichts deswegen unternehmen? Diesmal überspringen wir diesen Teil. Also los. Mobbt mich.“

Harry Potter schritt vorwärts, die Arme einladend weit geöffnet.

Die drei Gryffindors knickten ein und ergriffen die Flucht und Neville musste einen schnellen Seitenschritt machen, um nicht über den Haufen gerannt zu werden.

Es herrschte Stille als ihre Schritte verklangen und danach noch mehr Stille.

Und da waren es drei.

Harry Potter nahm einen tiefen Atemzug, dann stieß er ihn aus. „Puh,“ sagte er. „Wie geht's dir, Neville?“

Nevilles Stimme erklang als hoher Quieken. „Okay, \emph{das} war wirklich cool.“

Ein Grinsen huschte über Harry Potters Gesicht. „\emph{Du} warst auch ziemlich cool, weißt du.“

Neville wusste, dass Harry Potter das nur sagte, damit er sich gut fühlte und doch entfachte es eine warme Glut in seiner Brust.

Harry wandte sich Lesath Lestrange zu—

„Bist du okay, Lestrange?“ sagte Neville, bevor Harry den Mund aufmachen konnte.

Nun gab es etwas, dass man sich selbst niemals sagen zu hören erwartenwürde.

Lesath Lestrange drehte sich langsam um und starrte Neville an, mit verkniffenem Gesicht, nicht länger weinend, trocknende Tränen glitzerten.

„Du glaubst, du weißt wie das ist?“ sagte Lesath mit hoher, bebender Stimme. „\emph{Du glaubst, du weißt es?} Meine Eltern sind in \emph{Askaban}, ich versuche nicht daran zu denken und sie erinnern mich immer daran, sie finden es \emph{toll}, dass Mutter dort in der Kälte und der Finsternis sitzt, mit den Dementoren, die ihr das Leben aussaugen; ich wünschte ich wäre wie Harry Potter, zumindest leiden seine Eltern nicht, meine Eltern leiden immer, jede Sekunde von jedem Tag; ich wünschte ich wäre wie du, wenigstens kannst du deine Eltern manchmal sehen, wenigstens weißt du, dass sie dich geliebt haben, wenn Mutter mich je geliebt hat, haben die Dementoren den Gedanken schon längst aufgezehrt—“

Nevilles Augen weiteten sich vor Schock. Das hatte er nicht erwartet.

Lesath wandte sich Harry Potter zu, dessen Augen angefüllt waren mit Entsetzen.

Lesath ließ sich vor Harry Potter zu Boden fallen, berührte mit der Stirn den Boden und flüsterte, „Hilf mir, mein Lord.“

Eine furchtbare Stille entstand. Neville fiel nichts ein, was er sagen konnte und dem nackten Entsetzen auf Harrys Gesicht nach zu urteilen, wusste er ebenfalls nichts zu sagen.

„Man sagt, Ihr könnt alles fertig bringen, bitte, bitte mein Lord, holt meine Eltern aus Askaban heraus, ich werde für immer Euer treu ergebener Diener sein, mein Leben soll Euch gehören und ebenso mein Tod, nur bitte—“

„Lesath,“ sagte Harry, mit berstender Stimme, „Lesath, ich kann nicht, ich kann nicht wirklich solche Dinge tun, es sind alles nur dumme Tricks.“

„Sind sie \emph{nicht!}“ sagte Lesath, seine Stimme hoch und verzweifelt. „Ich habe es \emph{gesehen}, die Geschichten sind wahr, Ihr \emph{könnt es!}“

Harry schluckte. „Lesath, ich habe die ganze Sache mit Neville eingefädelt, wie haben alles im Voraus geplant, frag ihn!“

Das hatten sie, obwohl Harry nicht gesagt hatte, \emph{wie} er irgendwas davon anstellen würde…

Als Lesath vom Boden hoch sah, war sein Gesicht grausig verzerrt und seine Stimme erklang als ein Kreischen, das Neville in den Ohren schmerzte. „\emph{Du Sohn eines Schlammbluts! Du könntest sie rausholen, du willst nur nicht! Ich habe dich} \emph{auf Knien} \emph{angefleht und trotzdem willst du nicht helfen! Ich hätte es wissen sollen, du bist der Junge-der-überlebt-hat, du denkst, sie gehört dorthin!}“

„Ich \emph{kann nicht!}“ sagte Harry, seine Stimme so verzweifelt wie die von Lesath. „Es ist keine Frage, was ich will, ich habe nicht die \emph{Macht!}“

Lesath stand auf und spuckte vor Harry auf den Boden, dann wandte er sich um und ging davon. Als er um die Ecke war, beschleunigte sich das Geräusch seiner Schritte und als sie verblassten, glaubte Neville ein einsames Schluchzen zu hören.

Und da waren es zwei.

Neville sah Harry an.

Harry sah Neville an.

„Wow,“ sagte Neville leise. „Er schien nicht sehr dankbar für seine Rettung zu sein.“

„Er glaubte, ich könnte ihm helfen,“ sagte Harry mit heiserer Stimme. „Er hatte zum ersten mal seit Jahren wieder Hoffnung.“

Neville schluckte und sprach es aus. „Es tut mir leid.“

„Was?“ sagte Harry vollkommen verwirrt.

„Ich war nicht dankbar, als du mir geholfen hast—“

„Jedes einzelne Wort, das du gesagt hast, war absolut richtig,“ sagte der Junge-der-überlebt-hat.

„Nein,“ sagte Neville, „war es nicht.“

Gleichzeitig lächelten sie beide traurig, jedes wurde dem anderen nicht gerecht.

„Ich weiß, das war nicht real,“ sagte Neville, „Ich weiß, ich hätte nichts tun können, wenn du nicht da gewesen wärst, aber danke, dass ich so tun durfte.“

„Jetzt mach mal halblang,“ sagte Harry.

Harry hatte sich von Neville abgewandt und starrte aus dem Fenster zu den düsteren Wolken.

Neville kam ein vollkommen lächerlicher Gedanke. „Fühlst du dich schuldig, weil du Lesaths Eltern nicht aus Askaban herausholen kannst?“

„Nein,“ sagte Harry.

Ein paar Sekunden verstrichen.

„Ja,“ sagte Harry.

„Du bist dumm,“ sagte Neville.

„Ich bin mir dessen bewusst,“ sagte Harry.

„Musst du buchstäblich \emph{alles} tun, worum dich jemand bittet?“

Der Junge-der-überlebt-hat wandte sich wieder Neville zu. „Es \emph{tun?} Nein. Mich schuldig fühlen, es nicht zu tun? Ja.“

Neville fiel es schwer, Worte zu finden. „Als der Dunkle Lord starb, war Bellatrix Black buchstäblich die böseste Person auf der ganzen Welt und das war, \emph{bevor} sie nach Askaban geschickt wurde. Sie folterte meine Mutter und meinen Vater in den Wahnsinn, weil sie herausfinden wollte, was mit dem Dunklen Lord geschehen war—“

„Ich weiß,“ sagte Harry leise. „Das verstehe ich, aber—“

„Nein! Tust du \emph{nicht!} Sie hatte einen \emph{Grund}, das zu tun und meine Eltern waren beide Auroren! Es ist nicht einmal annähernd das schlimmste, was sie je getan hat!“ Nevilles Stimme bebte.

„Trotzdem,“ sagte der Junge-der-überlebt-hat, sein Blick in weiter Ferne, an einem Ort, den Neville sich nicht vorzustellen vermochte. „Es mag irgendeine clevere Lösung geben, durch die es möglich ist, jeden zu retten und alle leben danach glücklich und zufrieden und wäre ich nur schlau genug, wäre sie mir schon eingefallen—“

„Du hast Probleme,“ sagte Neville. „Du denkst, du solltest sein, wofür Lesath Lestrange dich hält.“

„Ja,“ sagte der Junge-der-überlebt-hat, „das trifft es ziemlich gut. Jedes mal, wenn jemand schreit im Gebet und ich nichts tun kann, fühle ich mich schuldig, dass ich nicht Gott bin.“

Neville verstand das nicht ganz, aber… „Das klingt nicht gut.“

Harry seufzte, „Ich weiß, dass ich ein Problem habe und ich weiß, dass ich es lösen muss, in Ordnung? Ich arbeite dran.“

\later

Harry sah zu, wie Neville verschwand.

Natürlich hatte Harry nicht gesagt, worin die Lösung bestand.

Die Lösung war, offensichtlich, sich zu beeilen und Gott zu werden.

Nevilles Schritte entfernten sich und waren nicht mehr zu hören.

Und da war es einer.

„Ähem,“ erklang Severus Snapes Stimme direkt hinter ihm.

Harry stieß einen kleinen Schrei aus und hasste sich sofort dafür.

Langsam drehte Harry sich um.

Der große, ölige Mann im befleckten Umhang lehnte an der Mauer, am selben Ort, den Harry eingenommen hatte.

„Ein guter Unsichtbarkeitsumhang, Potter,“ sagte der Meister der Zaubertränke gedehnt. „Erklärt einiges.“

Oh, verdammter Mist.

„Und vielleicht habe ich zu viel Zeit in Dumbledores Gesellschaft verbracht,“ sagte Severus, „doch ich kann nicht umhin, mich zu fragen, ob es \emph{der} Unsichtbarkeitsumhang ist.“

Augenblicklich wurde Harry zu jemandem, der niemals etwas von dem Unsichtbarkeitsumhang gehört hatte und der \emph{genau} so schlau war, wie Harry glaubte, dass Severus glaubte, dass Harry war.

„Oh, schon möglich,“ sagte Harry. „Ich gehe davon aus, Ihnen ist klar, was das bedeutet?“

Severus Stimme klang herablassend. „Sie haben keine Ahnung, wovon ich rede, oder, Potter? Ein ziemlich ungeschickter Versuch, mich zu ködern.“

(Professor Quirrell hatte bei ihrem Mittagessen angemerkt, dass Harry seinen Geisteszustand wirklich besser verbergen müsse, als nur eine ausdruckslose Miene aufzusetzen, wenn jemand ein gefährliches Thema anschnitt und hatte ihm erzählt von Täuschungen der ersten Stufe, Täuschungen der zweiten Stufe und so weiter. Also stellte Severus sich Harry entweder \emph{tatsächlich} als Spieler der ersten Stufe vor, was Severus zu einem Spieler der zweiten Stufe machte und Harrys Zug der dritten Stufe war erfolgreich gewesen oder Severus war ein Spieler der vierten Stufe und wollte, dass Harry \emph{glaubte}, die Täuschung sei erfolgreich gewesen. Harry hatte Professor Quirrell, lächelnd, gefragt, auf welcher Stufe \emph{er} spielte und Professor Quirrell hatte, ebenfalls lächelnd, erwidert, \emph{Eine Stufe höher als Sie.})

„Also haben Sie die ganz Zeit zugesehen,“ sagte Harry. „Desillusionierung, nennt man es, glaube ich.“

Ein dünnes Lächeln. „Es wäre töricht von mir gewesen, auch nur das kleinste Risiko in Kauf zu nehmen, dass Sie zu Schaden kämen.“

„Und Sie wollten die Ergebnisse mit eigenen Augen sehen,“ sagte Harry. „Also. Bin ich wie mein Vater?“

Ein seltsam trauriger Ausdruck überfiel den Mann, der fremd wirkte auf seinem Gesicht. „Ich würde eher sagen, Harry Potter, sie ähneln—“

Severus hielt inne.

Er starrte Harry an.

„Lestrange hat Sie Sohn eines Schlammbluts genannt,“ sagte Severus langsam. „Es schien Sie nicht groß zu stören.“

Harry furchte die Augenbrauen. „Nicht unter diesen Umständen, nein.“

„Sie hatten ihm gerade geholfen,“ sagte Severus. Seine Augen beobachteten Harry eindringlich. „Und er hat es Ihnen ins Gesicht geschlagen. Sicherlich ist das nichts, was Sie einfach vergeben würden?“

„Er hatte gerade ein ziemlich grauenhafte Erfahrung hinter sich,“ sagte Harry. „Und ich denke, von Erstklässlern gerettet zu werden, hat seinem Stolz auch nicht gerade gut getan.“

„Ich nehme an, es war recht einfach zu vergeben,“ sagte Severus mit seltsamer Stimme, „da Lestrange Ihnen nichts bedeutet. Nur ein seltsamer Slytherin. Wäre es, vielleicht, ein Freund gewesen, hätten Sie sich wohl weit verletzter gefühlt, von dem, was er sagte.“

„Wäre er ein Freund,“ sagte Harry, „um so mehr Grund, ihm zu vergeben.“

Es blieb lange still. Harry fühlte und er konnte nicht sagen woher oder warum, dass die Luft sich mit furchtbarer Spannung füllte, wie Wasser, dass stieg und stieg und stieg.

Dann lächelte Severus, sah mit einem mal wieder entspannt aus und all die Anspannung verschwand.

„Sie sind ein sehr versöhnlicher Mensch,“ sagte Severus, noch immer lächelnd. „Ich nehme an, ihr Stiefvater, Michael Verres-Evans, hat es Sie gelehrt.“

„Eher Dads Science Fiction- und Fantasy-Sammlung,“ sagte Harry. „Fast wie mein fünftes Elternteil, eigentlich. Ich habe all die Leben der Charaktere in all meinen Büchern gelebt und all ihre mächtige Weisheit hallt wieder in meinem Kopf. Irgendwo da drin, vermute ich, war jemand wie Lesath, obwohl ich nicht sagen könnte wer. Es war nicht schwer, mich in seine Lage zu versetzen. Und es waren auch meine Bücher, die mir sagten, was zu tun war. Die Guten vergeben.“

Severus ließ ein leichtes, belustigtes Lachen vernehmen. „Ich fürchte, ich wüsste nicht allzu viel darüber, was gute Menschen tun.“

Harry sah ihn an. Das war eigentlich irgendwie traurig. „Ich kann Ihnen ein paar Romane mit guten Menschen darin leihen, wenn Sie wollen.“

„Ich würde gern Ihren Rat zu etwas einholen,“ sagte Severus in beiläufigem Ton. „Ich kenne einen anderen Slytherin-Fünftklässler, der von Gryffindors gemobbt wurde. Er umwarb ein wunderschönes muggelgeborenes Mädchen, das dazu stieß als er gemobbt wurde und versuchte, ihn zu retten. Und er nannte sie ein Schlammblut und daraufhin war es vorbei zwischen ihnen. Er hat sich entschuldigt, viele male, doch sie vergab ihm nie. Fällt Ihnen irgendetwas ein, was er hätte sagen oder tun können, um von ihr jene Vergebung zu gewinnen, die Sie Lestrange zuteil werden ließen?“

„Ähm,“ sagte Harry, „nur basierend auf diesen Informationen, bin ich nicht sicher, dass \emph{er} derjenige mit dem Hauptproblem war. Ich hätte ihm geraten, mit niemandem auszugehen, der so unfähig zur Vergebung ist. Angenommen, sie hätten geheiratet, können Sie sich vorstellen, in diesem Haushalt zu leben?“

Eine Pause entstand.

„Oh, aber sie \emph{konnte} vergeben,“ sagte Severus mit Belustigung in der Stimme. „Denn danach ging sie hin und wurde die Freundin des Mobbers. Sagen Sie mir, warum würde sie dem Mobber vergeben und nicht dem Gemobbten?“

Harry zuckte mit den Schultern. „Wenn ich raten müsste, weil der Mobber jemand \emph{anderem} sehr weh getan hatte und der Gemobbte \emph{ihr} nur ein wenig und für sie fühlte sich das irgendwie unverzeihlicher an. Oder, um es nicht allzu sehr zu beschönigen, war der Mobber gutaussehend? Oder reich, was das betrifft?“

Es gab eine weitere Pause.

„Ja zu beidem,“ sagte Severus.

„Und da haben Sie's,“ sagte Harry. „Nicht dass ich jemals selbst in der Highschool gewesen wäre, aber meine Bücher geben mir zu verstehen, dass es eine bestimmte Sorte Teenager-Mädchen gibt, die außer sich gerät, bei einer kleinen Kränkung, wenn der Junge reizlos und arm ist, die jedoch irgendwie Platz in ihrem Herzen finden kann, um einem reichen und gutaussehenden Jungen sein Mobbing zu vergeben. Mit anderen Worten, sie war oberflächlich. Sagen Sie ihm, wer immer es ist, dass sie seiner nicht wert war und er darüber hinweg kommen und beim nächsten mal mit Mädchen ausgehen soll, die Tiefgang statt Schönheit haben.“

Severus starrte Harry still an, ein Glitzern in seinen Augen. Sein Lächeln war verblasst und obwohl Severus Gesicht zuckte, kehrte es nicht zurück.

Harry begann langsam etwas nervös zu werden. „Ähm, nicht das ich selbst auf dem Gebiet irgendwelche Erfahrung hätte, offensichtlich, aber ich denke, dass ist es, was ein weiser Ratgeber aus meinen Büchern sagen würde.“

Es folgte noch mehr Stille und mehr Glitzern.

Es war vermutlich ein guter Zeitpunkt, das Thema zu wechseln.

„Also,“ sagte Harry. „Habe ich Ihren Test bestanden, was immer es war?“

„Ich denke,“ sagte Severus, „dass es keine weiteren Unterhaltungen zwischen uns geben sollte, Potter und es überaus weise von Ihnen wäre, von dieser hier niemals zu sprechen.“

Harry blinzelte. „Hätten Sie wohl etwas dagegen, mir zu sagen, was ich falsch gemacht habe?“

„Sie haben mich gekränkt,“ sagte Severus. „Und ich traue Ihrem Urteil nicht länger.“

Harry starrte Severus an, ziemlich verdutzt.

„Doch Sie haben mir einen gut gemeinten Rat gegeben,“ sagte Severus Snape, „und daher werde ich Ihnen im Gegenzug ebenfalls einen wahrhaftigen Ratschlag erteilen.“ Seine Stimme war fast völlig ruhig. Wie ein Faden, fast perfekt horizontal gespannt trotz dem massiven Gewicht, das von seiner Mitte hing, durch Millionen Tonnen Spannung, die an jedem Ende zerrten. „Sie wären heute fast gestorben, Potter. In Zukunft, teilen Sie Ihre Weisheit mit niemandem, wenn Sie nicht genau wissen, worüber Sie beide sprechen.“

Harrys Geist stellte endlich die Verbindung her.

„\emph{Sie} waren das—“

Harrys Mund schnappte zu, als der \emph{fast gestorben} Teil sackte, zwei Sekunden zu spät.

„Ja,“ sagte Severus, „das war ich.“

Und die schreckliche Spannung flutete zurück in den Raum, wie Wasser, das am Grund des Ozeans zusammengedrückt wurde.

Harry konnte nicht atmen.

\emph{Verliere. Jetzt.}

„Ich wusste es nicht,“ flüsterte Harry. „Es t—“

„Nein,“ sagte Severus. Nur jenes eine Wort.

Harry stand dort in aller Stille, sein Geist suchte verzweifelt nach Optionen. Severus stand zwischen ihm und dem Fenster, was wirklich ein Pech war, da ein Sturz aus dieser Höhe einen Zauberer nicht töten würde.

„Ihre Bücher haben Sie betrogen, Potter,“ sagte Severus, noch immer mit jener Stimme, gespannt durch Millionen Tonnen Zug. „Sie haben Ihnen das eine nicht verraten, das Sie wissen mussten. Man kann nicht aus Geschichten erfahren, wie es ist, denjenigen zu verlieren, den man liebt. Das ist etwas, was man niemals verstehen kann, ohne es selbst zu fühlen.“

„Mein Vater,“ flüsterte Harry. Es war seine bester Versuch, die eine Sache, die ihn retten mochte. „Mein Vater versuchte, Sie vor den Mobbern zu schützen.“

Ein grässliches Lächeln erstreckte sich über Severus Gesicht und der Mann schritt auf Harry zu.

Und an ihm vorbei.

„Leben Sie wohl, Potter,“ sagte Severus und blickte auf seinem Weg nach draußen nicht zurück. „Von heute an dürften wir einander wenig zu sagen haben.“

Und an der Ecke, hielt der Mann inne und ohne sich umzudrehen, sprach er ein letztes mal.

„Ihr Vater war der Mobber,“ sagte Severus Snape, „und was Ihre Mutter in ihm sah, war etwas, das ich niemals verstanden habe, bis zum heutigen Tage.“

Er ging.

Harry wandte sich um und ging zum Fenster. Seine zitternden Hände legten sich auf das Fensterbrett.

\emph{Gib niemals jemandem einen weisen Rat, wenn du nicht genau weißt, worüber ihr beide sprecht. Ist angekommen.}

Harry starrte eine Weile hinaus zu den Wolken und dem leichten Nieselregen. Das Fenster war auf die östlichen Ländereien gerichtet und es war Nachmittag, wenn die Sonne also überhaupt durch die Wolken zu erkennen war, konnte Harry sie nicht sehen.

Seine Hände hatten aufgehört zu zittern, doch es war ein Gefühl der Enge in Harrys Brust, als würde sie von metallenen Bändern zusammengepresst.

Sein Vater war also ein Mobber gewesen.

Und seine Mutter oberflächlich.

Vielleicht waren sie später erwachsen geworden. Gute Menschen wie Professor McGonagall schienen die Welt auf sie zu geben und es mochte nicht \emph{nur} deshalb sein, weil sie heldenhafte Märtyrer waren.

Natürlich war das ein schwacher Trost, wenn man elf Jahre alt war und gerade im Begriff ein Teenager zu werden und sich fragte, was für eine Art Teenager man werden mochte.

So furchtbar.

So traurig.

Solch ein furchtbares Leben, das Harry führte.

Zu erfahren, dass seine genetischen Eltern nicht perfekt gewesen waren, nun, er sollte deshalb wohl eine Weile Trübsal blasen, sich selbst bemitleiden.

Vielleicht konnte er sich bei Lesath Lestrange darüber beklagen.

Harry hatte von Dementoren gelesen. Kälte und Finsternis umgaben sie und Angst, sie saugten alle glücklichen Gedanken aus einem heraus und in deren Abwesenheit stiegen all die eigenen schlimmsten Erinnerungen an die Oberfläche.

Er konnte sich vorstellen, in Lesaths Schuhen zu stecken, zu wissen, dass seine Eltern lebenslänglich in Askaban waren, dem Ort, von dem niemand jemals entkommen war.

Und Lesath würde sich vorstellen, anstelle seiner Mutter zu sein, in der Kälte und der Finsternis und der Angst, allein mit ihren schlimmsten Erinnerungen, selbst in ihren Träumen, jede Sekunde von jedem Tag.

Einen Augenblick lang stellte Harry sich seine eigene Mum und Dad in Askaban vor, wie die Dementoren das Leben aus ihnen heraus saugten, ihnen die glücklichen Erinnerungen an ihre Liebe für ihn entzogen. Nur für einen Moment, bevor seine Vorstellungskraft die Sicherung zog und eine Notfallabschaltung veranlasste und ihm verbat, sich das jemals wieder vorzustellen.

War es recht, das irgendjemandem anzutun, selbst der zweit-bösesten Person auf der Welt?

\emph{Nein,} sprach die Weisheit aus Harrys Büchern, \emph{nicht wenn es irgendeinen anderen Weg gibt, egal welchen anderen Weg.}

Und außer, das Justizsystem der Zauberwelt war so perfekt wie ihre Gefängnisse—und das klang recht unwahrscheinlich, wenn man alles bedachte—gab es irgendwo in Askaban einen Menschen, der vollkommen unschuldig war und wahrscheinlich mehr als nur einen.

Ein brennendes Gefühl breitete sich in Harrys Kehle aus und Feuchtigkeit sammelte sich in seinen Augen und er wollte alle Gefangenen von Askaban in Sicherheit teleportieren und Feuer vom Himmel herab rufen und den schrecklichen Ort zersprengen bis auf sein Fundament. Doch er konnte es nicht, weil er nicht Gott war.

Und Harry erinnerte sich, was Professor Quirrell unter dem Sternenlicht gesagt hatte: \emph{Manchmal}, \emph{wenn diese unvollkommene Welt} \emph{mirungewöhnlich} \emph{hassenswert erscheint, frage ich mich, ob es nicht einen anderen,} \emph{weit entfernten} \emph{Ort} \emph{geben mag, an dem ich hätte sein sollen… Doch die Sterne sind so weit, weit entfernt… Und ich frage mich, was ich wohl träumen würde, wenn ich für eine lange, lange Zeit schliefe.}

Jetzt gerade schien diese unvollkommene Welt ungewöhnlich hassenswert zu sein.

Und Harry konnte Professor Quirrells Worte nicht begreifen, es mochte ebenso gut ein Außerirdischer sein, der sie gesprochen hatte oder eine Künstliche Intelligenz, etwas das so vollkommen anders beschaffen war als Harry, dass sein Gehirn nicht dazu veranlasst werden konnte, auf diese Weise zu arbeiten.

Man konnte seinen Heimatplaneten nicht hinter sich lassen, solange es dort noch einen Ort gab, wie Askaban.

Man musste bleiben und kämpfen.

* Ein \emph{Pelziger} (engl.: \emph{Furcot}) ist ein grünes, pelziges Wesen mit sechs Gliedmaßen aus dem \emph{Homanx-Commonwealth}-Universum von Alan Dean Foster, trat zuerst im Roman \emph{Die denkenden Wälder} auf und lebt dort in einer außerirdischen Dschungelwelt mit 3–4 Meilen hohen Bäumen meist als lebenslanger Begleiter eines Menschen.

** Ich denke, ich sollte an dieser Stelle erwähnen, dass nicht \emph{alle} Charaktere, deren Namen nicht aus den ursprünglichen Harry-Potter-Romanen stammen, Anspielungen auf andere Werke aus Science Fiction und Fantasy sind. Der Autor verschafft ebenfalls Schöpfern von Fan-Kunstwerken zu dieser Geschichte, die ihm gefallen haben, Cameo-Auftritte darin. Wobei ich aber nicht sicher sagen kann, ob das in diesem Fall so ist, da zumindest die Nachnamen \emph{Sloper} und \emph{Kirke} eine Verbindung zu Original-Charakteren aufzuweisen scheinen.

