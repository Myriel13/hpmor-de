

\hypertarget{verantwortungsbewusstsein}{% \section{15. Verantwortungsbewusstsein}\label{verantwortungsbewusstsein}}

\textbf{Kapitel 15: Verantwortungsbewusstsein}\\

\hfill\break Love as thou Rowling.

Der heutige historische Leckerbissen: Die alten Hebräer betrachteten als Anfang des Tages den Sonnenuntergang, nicht das Morgengrauen, deshalb sagten sie „abends und morgens“, nicht „morgens und abends“. (Und wie viele Rezensenten anmerkten, besagt die moderne jüdische Halacha das selbe.)

--------------------------------------------------------------------------------------------------------------------------------------------

\hfill\break

\emph{"Ich bin sicher, ich finde die Zeit irgendwo."}

\hfill\break

--------------------------------------------------------------------------------------------------------------------------------------------

\hfill\break "\emph{Frigideiro!}„

Harry tauchte einen Finger in das Wasserglas auf seinem Pult. Es hätte kühl sein sollen. Aber es war lauwarm geblieben. Schon wieder.

Harry fühlte sich sehr, sehr betrogen.

Es waren hunderte Fantasy-Romane über den Verres-Haushalt verteilt. Harry hatte schon ein paar gelesen. Und es sah langsam so aus, als hätte er eine mysteriöse dunkle Seite. Als sich also das Glas Wasser die ersten paar male geweigert hatte zu kooperieren, hatte Harry sich im Zauberkunst-Klassenraum umgesehen, um sicher zu sein, dass niemand hinsah und dann konzentriert tief eingeatmet und sich in Zorn versetzt. Hatte daran gedacht, wie die Slytherins Neville gemobbt hatten und an das Spiel, bei dem man jemandem die Bücher wieder aus der Hand schlug, immer wenn er sie aufzuheben versuchte. Daran was Draco Malfoy über das zehn Jahre alte Lovegood-Mädchen gesagt hatte und wie der Zaubergamot wirklich arbeitete...

Und der Zorn steig ihm ins Blut, er streckte seinen Zauberstab aus mit einer Hand die zitterte vor Hass und sagte in kaltem Tonfall „\emph{Frigideiro!}“ und absolut nichts passierte.

Harry war \emph{beschwindelt} worden. Er wollte jemandem schreiben und eine \emph{Rückerstattung} verlangen für seine dunkle Seite, die offensichtlich unaufhaltsame magische Kräfte hätte haben \emph{sollen,} sich aber als \emph{defekt} herausstellte.

\emph{„Frigideiro!“ sagte Hermine erneut am Tisch neben ihm. Ihr Wasser war} \emph{festes Eis und es bildeten sich weiße Kristalle am Rand ihres Glases. Sie schien vollkommen auf ihre eigene Arbeit konzentriert und sich der hasserfüllten Blicke der anderen Schüler gar nicht bewusst zu sein, was entweder (a) gefährlich gedankenlos oder (b) ein perfekt verfeinertes Schauspiel von ihr war, dass sie zu einer Kunstform erhoben hatte.}

\emph{„Oh,} \emph{\emph{sehr}} \emph{gut, Miss Granger!“ quiekte Filius Flitwick, ihr Zauberkunst-Professor und Hauslehrer von Ravenclaw, ein winzig kleiner Mann ohne erkennbare Anzeichen einer Vergangenheit als Duellier-Meister. „Ausgezeichnet! Fantastisch!„}

\emph{Harry hatte erwartet, im schlimmsten Fall, zweiter hinter Hermine zu sein. Harry hätte es natürlich vorgezogen, wenn} \emph{\emph{sie}} \emph{mit} \emph{\emph{ihm}} \emph{konkurriert hätte, aber er hätte es auch anders herum akzeptieren können.}

\emph{Ab dem ersten Montag, schien Harry auf dem Weg zum Klassenletzten zu sein, eine Position, um die er freundschaftlich mit allen anderen muggelgeborenen Schülern außer Hermine konkurrierte. Die allein und konkurrenzlos an der Spitze stand, das arme Ding.}

\emph{Professor Flitwick stand über dem Pult einer der anderen Muggelgeborenen gebeugt und korrigierte leise die Art, wie sie ihren Zauberstab hielt.}

\emph{Harry sah zu Hermine hinüber. Er schluckte schwer. Es war die offensichtliche Rolle für sie, so wie die Dinge lagen... „Hermine?“ sagte Harry zögerlich. „Hast du eine Ahnung, was ich vielleicht falsch mache?"}

\emph{Hermines Augen leuchteten auf in furchtbarer Hilfsbereitschaft und etwas in Harrys Hinterkopf schrie vor Erniedrigung.}

\emph{Fünf Minuten später schien Harrys Wasser bemerkenswert kühler als die Raumtemperatur und Hermine hatte ihm ein paar mal verbal den Kopf getätschelt und ihm geraten, es beim nächsten mal sorgfältiger zu auszusprechen und war abgezogen, um jemand anderem zu helfen.}

\emph{Professor Flitwick hatte ihr einen Hauspunkt dafür gegeben, dass sie ihm geholfen hatte.}

\emph{Harry bis so stark die Zähne zusammen, dass sein Kiefer weh tat, was für seine Aussprache nicht hilfreich war.}

\emph{\emph{Ist mir egal ob das unfaires Spiel ist. Ich weiß genau, was ich mit diesen zwei Extra-Stunden pro Tag mache. Ich werde in meinem Koffer sitzen und lernen, bis ich mit Hermine Granger mithalte.}}

--------------------------------------------------------------------------------------------------------------------------------------------

\hfill\break „Verwandlungen sind eine der kompliziertesten und gefährlichsten Arten von Magie, die Sie in Hogwarts lernen werden,“ sagte Professor McGonagall. Es lag keine Spur von Ungezwungenheit in den Zügen der strengen alten Hexe. „Jeder, der in meinem Unterricht Unfug anstellt, wird uns verlassen und nicht zurückkehren. Sie wurden gewarnt."

Ihr Zauberstab fuhr herab und tippte ihren Schreibtisch an, der geschmeidig seine Form zu der eines Schweines veränderte. Ein paar der muggelgeborenen Schüler schrien leise auf. Das Schwein sah sich um und grunzte, schien verwirrt und wurde dann wieder zu einem Schreibtisch.

Die Verwandlungs-Professorin sah sich im Klassenzimmer um, dann ruhten ihre Augen auf einem Schüler.

„Mr. Potter,“ sagte Professor McGonagall. „Sie haben Ihre Schulbücher erst vor wenigen Tagen erhalten. Haben Sie bereits begonnen, ihr Verwandlungs-Lehrbuch zu lesen?"

„Nein, tut mir leid Professor,“ sagte Harry.

„Sie müssen sich nicht entschuldigen, Mr. Potter, wenn von Ihnen verlangt würde, voraus zu lesen, wäre Ihnen das mitgeteilt worden.“ McGonagalls Finger klopften auf dem Schreibtisch vor ihr. „Mr. Potter, würden Sie eine Vermutung anstellen, ob dies ein Schreibtisch ist, den ich in ein Schwein verwandelt habe oder es von Anbeginn an ein Schwein war und ich die Verwandlung kurzfristig aufgehoben habe? Wenn Sie das erste Kapitel ihres Lehrbuches gelesen hätten, wüssten Sie es."

Harrys runzelte leicht die Augenbrauen. „Ich würde vermuten, es wäre einfacher mit einem Schwein anzufangen, denn wäre es von Anfang an ein Schreibtisch, wüsste es nicht, wie es aufrecht steht."

Professor McGonagall schüttelte den Kopf. „Nicht Ihr Fehler, Mr. Potter, aber die korrekte Antwort ist, dass man in Verwandlung \emph{nicht} vermutet. Falsche Antworten werden mit extremer Strenge benotet, offen gelassene Fragen mit großer Nachsicht. Sie müssen lernen, zu wissen, was Sie nicht wissen. Wenn ich Ihnen irgendeine Frage stelle, egal wie offensichtlich oder elementar und Sie antworten 'Ich bin nicht sicher', werde ich es Ihnen nicht vorhalten und jeder, der lacht, verliert Hauspunkte. Können Sie mir sagen, warum diese Regel existiert, Mr. Potter?"

\emph{Weil ein einziger Fehler bei Verwandlungen unglaublich gefährlich sein kann.}

"Nein."

"Korrekt. Verwandlung ist gefährlicher als das Apparieren, was nicht vor dem sechsten Schuljahr unterrichtet wird. Unglücklicherweise muss die Verwandlung schon in jungem Alter erlernt und geübt werden, um Ihre Fähigkeiten als Erwachsene zu maximieren. Dies ist also eine gefährliche Angelegenheit und Sie sollten eine angemessene Furcht davor besitzen, irgendwelche Fehler zu machen, weil keine meiner Schüler bisher dauerhaft verletzt wurden und ich \emph{extrem außer mir sein werde,} wenn Sie die erste Klasse sind, die \emph{meinen Rekord gefährdet.}„

Mehrere Schüler schluckten.

Professor McGonagall stand auf und ging zur Wand hinter ihrem Schreibtisch, an der eine hölzerne Tafel befestigt war. „Es gibt viele Gründe, warum Verwandlungen gefährlich sind, aber ein Grund steht über allen anderen.“ brachte einen kurzen Federkiel mit breitem Ende hervor und formte mit ihm rote Buchstaben, die sie dann mit dem selben Marker blau unterstrich:

\uline{VERWANDLUNGEN SIND NICHT DAUERHAFT!}

„Verwandlungen sind nicht dauerhaft!“ sagte Professor McGonagall. „Verwandlungen sind nicht dauerhaft! Verwandlungen sind nicht dauerhaft! Mr. Potter, nehmen Sie an, ein Schüler verwandelte einen Holzblock in eine Tasse mit Wasser und Sie würden es trinken. Was stellen Sie sich vor, würde mit ihnen geschehen, wenn die Verwandlung endete?“ Es gab eine Pause. Entschuldigen Sie, ich hätte Sie das nicht fragen sollen, Mr. Potter, ich vergaß, dass Sie mit einer ungewöhnlich pessimistischen Vorstellungskraft gesegnet sind -"

„Ist schon in Ordnung,“ sagte Harry und schluckte schwer. „Also die erste Antwort ist, dass ich es nicht \emph{weiß,}“ die Professorin nickte anerkennend, „aber ich kann mir \emph{vorstellen...} es würde Holz in meinem Magen sein und in meinem Blutkreislauf und wenn etwas von dem Wasser von meinem Gewebe aufgenommen wurde - wäre es Holzbrei oder festes Holz oder...“ Harrys Verständnis für Magie versagte. Er konnte nicht verstehen, wie Holz überhaupt erst zu Wasser wurde, deshalb konnte er nicht verstehen, was geschehen würde, nachdem die Wassermoleküle von gewöhnlichen Wärmebewegungen durchmischt wurden und die Magie endete und die Verwandlung sich umkehrte.

McGonagalls Gesicht war hart. „Wie Mr. Potter korrekt gefolgert hat, würde er extrem krank werden und müsste sofort per Floh-Transport ins St.-Mungo-Hospital gebracht werden, wenn er auch nur die geringste Überlebenschance haben sollte. Bitte schlagen Sie Ihre Lehrbücher auf Seite 5 auf."

Selbst ohne jedes Geräusch von dem bewegten Bild konnte man erkennen, dass die Frau mit entsetzlich fahler Hautfarbe schrie.

"Der Verbrecher, der Gold in Wein verwandelt und dieser Frau zu trinken gegeben hat, 'in Begleichung der Schuld', wie er sich ausdrückte, wurde zu zehn Jahren in Askaban verurteilt. Bitte schlagen Sie Seite 6 auf. Dies ist ein Dementor. Sie sind die Wächter von Askaban. Sie rauben Ihnen Ihre Magie, Ihre Lebenskraft und jeden glücklichen Gedanken, den Sie versuchen zu haben. Das Bild auf Seite 7 zeigt den Verbrecher zehn Jahre später, bei seiner Entlassung. Sie werden bemerken, dass er tot ist - ja, Mr. Potter?"

„Professor,“ sagte Harry, „wenn in so einem Fall das schlimmste passiert, gibt es irgendeinen Weg die Verwandlung \emph{aufrecht zu erhalten?}"

„Nein,“ sagte Professor McGonagall schlicht. „Eine Verwandlung zu erhalten, entzieht Ihnen kontinuierlich Magie, deren Menge der Größe der verwandelten Form entspricht. Und Sie müssten mit dem Ziel alle paar Stunden erneut in Kontakt treten, was, in solch einem Fall, unmöglich ist. „Katastrophen wie diese sind \emph{unwiderruflich!}"

Professor McGonagall lehnte sich nach vorn, ihr Gesicht unnachgiebig. „Sie werden absolut niemals, unter keinen Umständen, irgendetwas in eine Flüssigkeit oder ein Gas verwandeln. Kein Wasser, keine Luft. Nichts wie Wasser, nichts wie Luft. Selbst wenn es nicht zum Trinken da ist. Flüssigkeit \emph{verdunstet,} winzig kleine Teile davon gelangen in die Luft. Sie werden nichts verwandeln, was verbrannt wird. Es wird Rauch erzeugen und jemand könnte diesen Rauch einatmen! Sie werden niemals etwas verwandeln, was auf irgendeine vorstellbare Weise in irgendjemandes Körper gelangen könnte. Kein Essen. Nichts, was wie Essen \emph{aussieht.} Nicht einmal als witziger kleiner Scherz, bei dem Sie vorhaben, der Person von ihrem Schlammkuchen zu erzählen, bevor sie ihn tatsächlich isst. Sie werden das niemals tun. Punkt. Nicht in diesem Klassenzimmer, nicht außerhalb davon, \emph{nirgendwo.} Hat \emph{jeder einzelne Schüler} das verstanden?"

„Ja,“ sagten Harry, Hermine und ein paar andere. Die anderen schienen sprachlos zu sein.

"\emph{Hat jeder einzelne Schüler das verstanden?}"

„Ja,“ sagten oder murmelten oder flüsterten sie.

"Wenn Sie irgendeine dieser Regeln brechen, werden Sie während Ihrer Zeit in Hogwarts nicht länger Verwandlung studieren. Sprechen Sie mir nach. Ich werde niemals etwas in eine Flüssigkeit oder ein Gas verwandeln."

„Ich werde niemals etwas in eine Flüssigkeit oder ein Gas verwandeln,“ sagten die Schüler unbeholfen im Chor.

"Noch einmal! Lauter! Ich werde niemals etwas in eine Flüssigkeit oder ein Gas verwandeln."

"Ich werde niemals etwas in eine Flüssigkeit oder ein Gas verwandeln."

"Ich werde niemals etwas verwandeln, was aussieht wie Essen oder irgendetwas anderes, das in einen menschlichen Körper gelangt."

"Ich werde niemals etwas verwandeln, was verbrannt wird, weil es Rauch erzeugen könnte."

„Sie werden niemals etwas verwandeln, was wie Geld aussieht, einschließlich Muggel-Geld,“ sagte Professor McGonagall. „Die Kobolde haben Wege, um herauszufinden, wer es getan hat. Nach anerkanntem Recht befindet sich die Kobold-Nation in permanentem \emph{Kriegszustand} mit allen magischen Geldfälschern. Sie werden keine Auroren schicken. Sie schicken eine Armee."

„Ich werde niemals etwas verwandeln, was wie Geld aussieht,“ wiederholten die Schüler.

„Und \emph{vor allem,}“ sagte Professor McGonagall, „werden Sie kein lebendes Wesen verwandeln, \emph{insbesondere nicht Sie selbst.} Es wird Sie sehr krank machen und möglicherweise töten, abhängig davon, wie Sie sich verwandeln und wie lange Sie Ihre Veränderung aufrecht erhalten.“ Professor McGonagall hielt inne. „Mr. Potter hebt im Augenblick seine Hand, weil er eine Animagus-Transformation gesehen hat -- um genau zu sein, einen Menschen, der sich in eine Katze und wieder zurück verwandelt hat. Aber eine Animagus-Transformation ist keine \emph{freie} Verwandlung.„

Professor McGonagall nahm einen keinen Holzblock aus ihrer Tasche. Mit einem Tippen ihres Zauberstabs wurde er zu einem Glas-Ball. Dann sagte sie „\emph{Crystferrium!}“ und der Glas-Ball wurde zu einem Stahl-Ball. Sie tippte ihn ein letztes mal mit ihrem Zauberstab an und der Stahl-Ball wurde wieder ein Stück Holz. „\emph{Crystferrium} verwandelt ein Objekt aus massivem Glas in ein ähnlich geformtes Ziel aus massivem Stahl. Es kann weder das Gegenteil bewirken, noch kann es einen Schreibtisch in ein Schwein verwandeln. Die üblichste Form der Verwandlung -- freie Verwandlung, die Sie hier lernen werden -- kann jedes Objekt in jedes Ziel verwandeln, zumindest soweit es die physische Form betrifft. Aus diesem Grund, muss freie Verwandlung wortlos angewandt werden. Zaubersprüche zu verwenden, würde verschiedene Worte für jede einzelne Transformation zwischen Objekt und Ziel erfordern."

Professor McGonagall warf ihren Schülern einen scharfen Blick zu. „\emph{Einige} Lehrer beginnen mit Verwandlungs-Zaubersprüchen und gehen danach zu freier Verwandlung über. Ja, das wäre am Anfang viel einfacher. Aber es könnte Ihnen schlechte Gewohnheiten antrainieren und Ihre späteren Fähigkeiten beeinträchtigen. Hier werden Sie freie Verwandlung \emph{von Anfang an} lernen, die erfordert, dass Sie den Zauber wortlos wirken, indem Sie die Objekt-Form, die Ziel-Form und die Transformation in ihrem Geist vollziehen."

„Und um Mr. Potters Frage zu beantworten,“ fügte Professor McGonagall hinzu, „es ist \emph{freie} Verwandlung, die Sie niemals bei irgendeinem lebenden Wesen anwenden dürfen. Es gibt Zaubersprüche und Tränke, die lebende Wesen in \emph{begrenzter} Weise sicher und umkehrbar transformieren können. Einem Animagus, dem eine Gliedmaße fehlt, wird diese beispielsweise nach der Verwandlung immer noch fehlen. Freie Verwandlung ist \emph{nicht} sicher. Ihr Körper wird sich verändern, während er verwandelt ist -- zu atmen beispielsweise führt zu einem konstanten Verlust körpereigenen Materials an die umgebende Luft. Wenn die Verwandlung nachlässt und Ihr Körper versucht, zu seiner \emph{ursprünglichen} Form zurückzukehren, wird er das nicht vollständig tun können. Wenn Sie Ihren Zauberstab an Ihren Körper halten und stellen sich sich selbst mit goldenem Haar vor, wird Ihr Haar hinterher ausfallen. Wenn Sie sich sich selbst als jemanden mit reinerer Haut vorstellen, werden Sie einen langen Aufenthalt in St.-Mungo verbringen. Und wenn Sie sich selbst in eine erwachsene körperliche Form verwandeln, werden Sie sterben, wenn die Verwandlung nachlässt."

Das erklärte warum er so etwas wie fette Jungs gesehen hatte oder nicht vollkommen hübsche Mädchen. Oder alte Menschen, was das betraf. Das würde nicht passieren, wenn man sich einfach jeden Morgen selbst verwandeln könnte... Harry hob seine Hand und versuchte Professor McGonagall mit den Augen zu erreichen.

"\emph{Ja,} Mr. Potter?"

"Ist es möglich ein lebendes Wesen in ein statisches Ziel zu verwandeln, wie etwa eine Münze - nein, entschuldigen Sie, es tut mir schrecklich leid, sagen wir einfach eine Stahl-Kugel."

Professor McGonagall schüttelte den Kopf. „Mr. Potter, selbst unbewegte Objekte unterliegen mit der Zeit kleinen inneren Veränderungen. Es gäbe hinterher keine sichtbaren Veränderungen an Ihrem Körper und in der ersten Minute würde Ihnen nichts falsches auffallen. Aber nach einer Stunde wären Sie krank und nach einem Tag wären Sie tot."

„Ähm, entschuldigen Sie, also wenn ich das erste Kapitel gelesen hätte, hätte ich \emph{vermuten} können, dass der Schreibtisch ursprünglich ein Schreibtisch war und kein Schwein,“ sagte Harry, „aber nur wenn ich die \emph{weitere} Annahme machte, dass Sie das Schwein nicht töten wollten, was höchst wahrscheinlich zu sein \emph{scheint,} aber -"

"Ich weiß jetzt bereits, dass Ihre Prüfungen zu benoten ein endloser Quell der Freude für mich sein wird, Mr. Potter. Aber wenn Sie irgendwelche anderen Fragen haben, darf ich Sie bitten, damit bis nach dem Unterricht zu warten?"

"Keine weiteren Fragen, Professor."

„Jetzt sprechen Sie mir nach,“ sagte Professor McGonagall. „Ich werde niemals versuchen, irgendein lebendes Wesen zu verwandeln, insbesondere nicht mich selbst, es sei denn ich wurde direkt angewiesen, das mit einem spezialisierten Zauberspruch oder Trank zu tun."

"Wenn ich nicht sicher bin, ob eine Verwandlung sicher ist, werde ich sie nicht ausprobieren, bis ich Professor McGonagall oder Professor Flitwick oder Professor Snape oder den Schulleiter gefragt habe, die die einzigen anerkannten Autoritäten für Verwandlung in Hogwarts sind. Einen anderen Schüler zu fragen, ist \emph{nicht} akzeptabel, selbst wenn sie sagen, sie erinnern sich, die selbe Frage gestellt zu haben."

"Selbst wenn der momentane Verteidigungs-Professor von Hogwarts mit sagt, dass eine Verwandlung sicher ist und selbst wenn ich den Verteidigungs-Professor sie ausführen sehe und nichts schlimmes zu passieren scheint, werde ich sie nicht selbst ausprobieren."

"Ich habe das absolute Recht, das Ausführen jeder Verwandlung zu verweigern, bei der ich auch nur das kleinste bisschen unsicher bin. Da nicht einmal der Schulleiter von Hogwarts mir befehlen kann, etwas anderes zu tun, werde ich sicher keinen solchen Befehl vom Verteidigungs-Professor akzeptieren, selbst wenn der Verteidigungs-Professor droht, mir einhundert Hauspunkte abzuziehen und mich der Schule verweisen zu lassen."

"Wenn ich irgendeine dieser Regeln breche, werde ich während meiner Zeit in Hogwarts keine Verwandlung mehr studieren."

„Wir werden diese Regeln den ersten Monat lang zu Beginn jeder Stunde wiederholen,“ sagte Professor McGonagall. „Und nun werden wir mit Streichhölzern als Objekte und Nadeln als Ziele beginnen... stecken Sie Ihre Zauberstäbe weg, danke, mit 'beginnen' meinte ich, Sie werden beginnen, sich Notizen zu machen."

Eine halbe Stunde vor Ende des Unterrichts gab Professor McGonagall die Streichhölzer aus.

Am Ende des Unterrichts hatte Hermine ein silbrig aussehendes Streichholz und der Rest der Klasse, ob Muggelgeborene oder nicht, hatten genau das, womit sie angefangen hatten.

Professor McGonagall verlieh ihr einen weiteren Punkt für Ravenclaw.

--------------------------------------------------------------------------------------------------------------------------------------------

\hfill\break Nachdem die Verwandlungs-Klasse entlassen war, kam Hermine zu Harrys Pult herüber, als Harry seine Bücher in seinem Beutel verstaute.

„Du weißt,“ sagte Hermine mit unschuldigem Ausdruck auf ihrem Gesicht, „ich habe heute zwei Punkte für Ravenclaw verdient."

„Hast du also,“ sagte Harry knapp.

„Aber das war nicht so gut wie deine \emph{sieben} Punkte,“ sagte sie. „Ich schätze, ich bin einfach nicht so intelligent wie du."

Harry fütterte seine Hausaufgaben zu Ende in den Beutel und drehte sich mit zu Schlitzen verengten Augen zu Hermine um. Das hatte er tatsächlich vergessen.

Sie \emph{warf ihm einen Augenaufschlag zu.} „Wir haben allerdings jeden Tag Unterricht. Ich frage mich, wie lange du brauchst, um noch ein paar Hufflepuffs zu finden, die du retten kannst? Heute ist Montag. Also hast du bis Donnerstag Zeit."

Die beiden starrten einander in die Augen, ohne zu blinzeln.

Harry sprach zuerst. „Du weißt natürlich, dass das Krieg bedeutet."

"Ich wusste nicht, dass wir Frieden hatten."

Alle anderen Schüler sahen sie jetzt fasziniert an. Alle Schüler und zusätzlich, unglücklicherweise, Professor McGonagall.

„Oh, Mr. Potter,“ trällerte Professor McGonagall von der anderen Seite des Raums, „ich habe gute Nachrichten für Sie. Madam Pomfrey hat ihrem Vorschlag zugestimmt, wie man das Zerbrechen ihrer Spimster Wickets verhindern kann und wir planen, Ende der Woche damit fertig zu sein. Ich würde sagen das verdient... sagen wir zehn Punkte für Ravenclaw."

Hermines Mund stand offen vor Schock und Verrat. Harry konnte sich vorstellen, dass sein eigenes Gesicht nicht viel anders aussah.

„\emph{Professor...}“ zischte Harry.

„Diese zehn Punkte sind \emph{unbestreitbar} verdient, Mr. Potter. Ich würde Hauspunkte nicht aus einer Laune heraus vergeben. Für Sie ging es vielleicht nur darum, dass Sie etwas zerbrechliches sahen und einen Weg vorschlugen, es zu schützen, aber Spimster Wickets sind teuer und der Schulleiter war \emph{nicht} erfreut, als das letzte mal eines zerbrach.“ Professor McGonagall blickte nachdenklich drein. „Meine Güte, ich frage mich, ob jemals ein anderer Schüler siebzehn Hauspunkte an seinem ersten Schultag verdient hat. Ich werde nachsehen müssen, aber ich nehme an, das ist ein neuer Rekord. Vielleicht sollten wir eine Ankündigung beim Abendessen machen?"

„\emph{PROFESSOR!}“ kreischte Harry. „Das ist \emph{unser} Krieg! Hören Sie auf, sich einzumischen!"

„Jetzt haben Sie bis Donnerstag der \emph{nächsten} Woche Zeit, Mr. Potter. Außer natürlich, Sie richten vorher irgendwelchen Unfug an und \emph{verlieren} Hauspunkte. Zum Beispiel einen Lehrer respektlos anzusprechen.“ Professor McGonagall legte einen Finger an ihre Wange und blickte versonnen. „Ich nehme an, Sie erreichen negative Zahlen, bevor der Freitag endet."

Harrys Mund schnappte zu. Er warf MacGonagall seinen besten Todesblick zu, aber sie schien es nur amüsant zu finden.

„Ja, definitiv eine Ankündigung beim Abendessen,“ grübelte Professor McGonagall. „Aber es wäre nicht gut, die Slytherins zu verärgern, also sollte die Ankündigung kurz sein. Nur die Anzahl der Punkte und die Tatsache des Rekords... und wenn irgendwer wegen Hilfe mit seinen Hausaufgaben zu Ihnen kommt und enttäuscht ist, dass Sie noch nicht einmal angefangen haben, ihre Lehrbücher zu lesen, können Sie ihn immer noch auf Miss Granger verweisen."

„\emph{Professor!}“ sagte Hermine mit ziemlich hoher Stimme.

Professor McGonagall ignorierte sie. „Meine Güte, ich frage mich, wie lange es dauert, bis Miss Granger etwas tut, was einer Ankündigung beim Abendessen würdig ist? Ich freue mich schon darauf, was immer es ist."

Harry und Hermine stürmten in unausgesprochener Übereinstimmung aus dem Klassenraum. Es folgte ihnen eine Schar hypnotisierter Ravenclaws.

„Ähm,“ sagte Harry. „Steht unsere Verabredung für nach dem Abendessen noch?"

„Natürlich,“ sagte Hermine. „Ich würde nicht wollen, dass du mit dem Lernen noch weiter zurück fällst."

"Ja, danke. Und darf ich sagen, dass, so brilliant du auch bereits bist, ich mich frage, wie du sein wirst, nachdem du ein bisschen grundlegende Übung in Rationalität hattest."

"Ist es wirklich so nützlich? Es schien dir bei Zauberkunst und Verwandlung nicht zu helfen."

Es gab eine kurze Pause.

"Nun, ich habe meine Schulbücher erst vor vier Tagen bekommen. Deshalb musste ich diese siebzehn Hauspunkte verdienen, ohne meinen Zauberstab zu benutzen."

"Vor vier Tagen? Vielleicht kannst du keine acht Bücher in vier Tagen lesen, aber du hättest zumindest \emph{eins} lesen können. Wie viele Tage brauchst du bei der Geschwindigkeit um fertig zu werden? Du kennst doch diese ganze Mathematik, also kannst du mir sagen, wie viel acht mal vier geteilt durch null ergibt?"

"Ich habe jetzt Unterricht, den du nicht hattest, aber die Wochenenden sind frei, also... maximal acht mal vier geteilt durch Epsilon, wobei Epsilon auf null zustrebt, plus... 10:47 Uhr am Sonntag."

"Ich habe es tatsächlich in \emph{drei} Tagen geschafft."

"14:47 Uhr am Samstag dann. Ich bin sicher, ich finde die Zeit irgendwo."

Und es wurde abends und es wurde morgens, am ersten Tag.

