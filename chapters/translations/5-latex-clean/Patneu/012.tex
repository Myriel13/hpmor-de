

\hypertarget{impulskontrolle}{% \section{12. Impulskontrolle}\label{impulskontrolle}}

Kapitel 12: Impulskontrolle

ph'nglui mglw'nafh J. K. Rowling wgah'nagl fhtagn

\later

\emph{"Ich frag mich, was wohl mit} ihm \emph{nicht stimmt."}

\later

"Turpin, Lisa!"

Flüster flüster flüster Harry Potter flüster flüster Slytherin flüster flüster nein ernsthaft, was zur Hölle flüster flüster

"RAVENCLAW!"

Harry stimmte in den Applaus ein, der das junge Mädchen begrüßte, das schüchtern auf den Ravenclaw-Tisch zuging, ihr Umhang jetzt in dunklem Blau getrimmt. Lisa Turpin schien hin- und hergerissen zwischen dem Drang, sich so weit wie möglich von Harry Potter weg zu setzen und dem Drang zu ihm rüber zu laufen, sich gewaltsam neben ihn zu quetschen und ihm ein paar Antworten zu entreißen.

Der Mittelpunkt eines außergewöhnlichen und merkwürdigen Ereignisses zu sein und kurz darauf ins Haus Ravenclaw sortiert zu werden, war nahe daran, in Barbecue-Soße getunkt und dann in eine Grube voll verhungerter Kätzchen geworfen zu werden.

"Ich habe dem Sprechenden Hut versprochen, nicht darüber zu reden," flüsterte Harry zum x-ten mal.

"Ja, wirklich."

"Nein, ich habe dem Sprechenden Hut wirklich versprochen, nicht darüber zu reden."

"Na schön, ich habe dem Sprechenden Hut versprochen, über das \emph{meiste} davon nicht zu reden und der Rest ist \emph{privat,} genau wie es \emph{eure Sachen auch waren,} also \emph{hört auf zu fragen.}"

"Ihr wollt wissen, was passiert ist? Schön! Hier ist ein Teil von dem, was passiert ist! Ich habe dem Sprechenden Hut gesagt, dass Professor McGonagall gedroht hat, ihn in Brand zu stecken und er hat mir gesagt, ich solle Professor McGonagall sagen, sie sei ein freches Gör und solle sich von seinem Rasen runter scheren!"

"Wenn ihr mir nicht glaubt, was ich sage, warum \emph{fragt ihr dann überhaupt?}"

"Nein, ich weiß auch nicht, wie ich den Dunklen Lord besiegt habe! Sagt ihr's mir, wenn ihr es rausfindet!"

"\emph{Ruhe!}" rief Professor McGonagall vom Podium des Lehrertisches. "\emph{Kein Gerede bis die Auswahl-Zeremonie vorüber ist!}"

Die Lautstärke fiel kurz ab, als alle warteten, ob sie irgendwelche spezifischen und glaubhaften Drohungen abgeben würde, dann erhob sich das Geflüster erneut.

Dann erhob sich der Alte mit dem silbernen Bart aus seinem großen goldenen Stuhl und lächelte fröhlich.

Sofortige Stille. Jemand stieß Harry hektisch mit dem Ellbogen an, als er versuchte weiter zu flüstern und Harry schnitt sich mitten im Satz das Wort ab.

Der fröhlich dreinblickende alte Mann setzte sich wieder.

\emph{Notiz an mich: Nicht mit Dumbledore anlegen.}

Harry versuchte noch immer, all das zu verarbeiten, was während des Vorfalls mit dem Sprechenden Hut geschehen war. Wovon nicht das geringste war, was in dem Moment passiert war, als Harry den Hut von seinem Kopf hob; in jenem Moment hatte er wie aus dem Nichts ein schwaches Flüstern vernommen, etwas das seltsam nach Englisch und gleichzeitig nach einem Zischen klang und es hatte gesagt, "\emph{Grüssse von Sslytherin zzu Sslytherin: Ssuchsst du meine Geheimnissse, sso ssprich zzu meiner Sschlange.}"

Harry vermutete irgendwie, dass das nicht zum offiziellen Auswahl-Prozess gehören sollte. Und dass es sich um ein wenig extra Magie handelte, die Salazar Slytherin dem Sprechenden Hut bei seiner Schöpfung auferlegt hatte. Und dass der Hut selbst nichts davon wusste. Und dass es ausgelöst wurde, wenn der Hut "SLYTHERIN" sagte, plus oder minus ein paar anderer Bedingungen. Und dass ein Ravenclaw wie er es war, es \emph{wirklich, wirklich nicht hätte hören sollen.} Und dass, sollte er einen verlässlichen Weg finden, Draco zur Verschwiegenheit zu verpflichten, damit er ihn danach fragen konnte, das eine exzellente Gelegenheit wäre, um etwas Comed-Tee zur Hand zu haben.

\emph{Junge, da entschließt man sich, nicht dem Pfad eines Dunklen Lords zu folgen und das Universum pfuscht einem dazwischen, sobald der Hut vom} \emph{Kopf runter ist. An manchen Tagen zahlt es sich einfach nicht aus, gegen das Schicksal anzukämpfen. Vielleicht warte ich mit meinem Vorsatz, kein Dunkler Lord zu werden, noch bis morgen.}

"GRYFFINDOR!"

Ron Weasley bekam eine \emph{Menge} Applaus und nicht nur von den Gryffindors. Offenbar war die Weasley-Familie hier weithin beliebt. Einen Augenblick später lächelte Harry und begann, zusammen mit den anderen zu applaudieren.

Andererseits gab es keinen besseren Tag als heute, sich von der Dunklen Seite abzuwenden.

Zur Hölle mit dem Schicksal und zur Hölle mit dem Universum. Er würde es dem Hut schon zeigen.

"Zabini, Blaise!"

Pause.

"SLYTHERIN!" rief der Hut.

Harry applaudierte Zabini ebenfalls und ignorierte die seltsamen Blicke, die ihm alle, einschließlich Zabini, zu warfen.

Danach wurde kein weiterer Name aufgerufen und Harry wurde klar, dass "Zabini, Blaise" ziemlich nach dem Ende des Alphabets klang. Klasse, also hatte er jetzt \emph{nur} Zabini applaudiert… Na toll.

Dumbledore erhob sich erneut und hielt auf das Podium zu. Offenbar sollten sie mit einer Rede beglückt werden -

Und Harry überkam die Inspiration zu einem \emph{brillanten} experimentellen Test.

Hermine hatte doch gesagt, Dumbledore sei der mächtigste lebende Zauberer, oder?

Harry langte in seinen Beutel und flüsterte "Comed-Tee".

Damit der Comed-Tee funktionierte, musste er Dumbledore dazu bringen, während seiner Rede etwas \emph{so} lächerliches zu sagen, dass Harry sich selbst bei seinem Grad mentaler Vorbereitung \emph{trotzdem} noch verschlucken würde. Wie etwa, dass alle Hogwarts-Schüler das ganze Schuljahr lang keine Kleidung tragen dürften oder sie alle in Katzen verwandelt würden.

Doch wenn \emph{irgendjemand auf der Welt} der Macht des Comed-Tees zu widerstehen vermochte, dann wäre es Dumbledore. Wenn das also funktionierte, war der Comed-Tee buchstäblich \emph{unschlagbar.}

Harry zog unter dem Tisch die Lasche des Comed-Tees, wollte es so unauffällig wie möglich machen. Die Dose gab ein zischendes Geräusch von sich. Ein paar Köpfe wandten sich zu ihm um, doch drehten sich bald zurück als -

"Willkommen! Willkommen zu einem neuen Jahr in Hogwarts!" sagte Dumbledore und strahlte sie alle mit weit ausgebreiteten Armen an, als könne ihn nichts glücklicher machen, als sie alle hier zu sehen.

Harry nahm einen ersten Mund voll Comed-Tee und setzte die Dose wieder ab. Er würde die Limo nach und nach schlucken und versuchen, nicht afzustoßen, egal \emph{was} Dumbledore auch sagte -

"Bevor wir mit unserem Festmahl beginnen, würde ich gern ein paar Worte sagen. Und hier sind sie: Schwachkopf! Schwabbelspeck! Krimskrams! Quiek! Danke sehr!"

Alle klatschten und jubelten und Dumbledore setzte sich wieder.

Harry saß da wie festgefroren, während die Limo aus seinen Mundwinkeln tröpfelte. Er hatte, immerhin, geschafft, sich \emph{leise} zu verschlucken.

Das hätte er wirklich wirklich \emph{wirklich} nicht machen sollen. Erstaunlich wie viel \emph{offensichtlicher} das wurde, eine \emph{Sekunde} nachdem es \emph{zu spät} war.

Rückblickend hätte ihm wahrscheinlich auffallen sollen, dass etwas nicht stimmte, als er gedacht hatte, sie würden alle in Katzen verwandelt… oder schon vorher, wenn man seine geistige Notiz darüber, sich nicht mit Dumbledore anzulegen, bedachte… oder seinen neuerlichen Vorsatz, mehr Rücksicht auf die Gefühle anderer zu nehmen… oder vielleicht wenn er auch nur \emph{ein Fünkchen gesunden Menschenverstand} besäße…

Es war hoffnungslos. Er war verdorben bis zum Kern. Heil dem Dunklen Lord Harry. Gegen das Schicksal hatte man nichts zu bestellen.

Jemand fragte Harry, ob alles in Ordnung sei. (Andere nahmen sich von dem Essen, das auf magische Weise auf dem Tisch erschienen war, warum auch nicht.)

"Alles okay," sagte Harry. "Entschuldigt. Ähm. War das eine… \emph{normale} Ansprache für den Schulleiter? Ihr schient alle… nicht wirklich… überrascht zu sein…"

"Oh, Dumbledore ist natürlich verrückt," sagte ein neben ihm sitzender, älter aussehender Ravenclaw, der sich mit irgendeinem Namen vorgestellt hatte, an den Harry sich nicht einmal ansatzweise erinnerte. "Verdammt lustig, unglaublich mächtiger Zauberer, aber total meschugge." Er hielt inne. "Bei Gelegenheit würde ich auch gerne fragen, warum grüne Flüssigkeit hinter deinen Lippen vor gekommen und dann verschwunden ist, aber ich gehe davon aus, du hast dem Sprechenden Hut versprochen, auch darüber nicht zu reden."

Mit einiger Anstrengung widerstand Harry dem Drang, auf die belastende Dose Comed-Tee in seiner Hand hinunter zu blicken.

Immerhin hatte der Comed-Tee nicht einfach so eine Klitterer-Schlagzeile über ihn und Draco \emph{herbei materialisiert.} Draco hatte es so erklärt, als sei es einfach… ganz natürlich passiert? Als hätte er\emph{die Geschichte passend verändert?}

Harry stellte sich vor, wie er die Stirn gegen die Tischplatte schlug. \emph{Wamm, wamm, wamm} machte sein Kopf im Geiste.

Eine andere Schülerin senkte die Stimme zu einem Flüstern. "Ich hab gehört, dass Dumbledore insgeheim ein Superhirn mit einer Menge Einfluss ist und die Verrücktheit als Fassade benutzt, damit ihn niemand verdächtigt."

"Das hab ich auch gehört," flüsterte ein dritter Schüler und am Tisch entlangfolgte verstohlenes Nicken.

Das musste einfach Harrys Aufmerksamkeit wecken.

"Ich verstehe," flüsterte Harry und senkte die Stimme. "Also weiß jeder, dass Dumbledore insgeheim ein Superhirn ist."

Die meisten Schüler nickten. Der ein oder andere blickte plötzlich nachdenklich drein, einschließlich des älteren Schülers, der neben Harry saß.

\emph{Seid ihr sicher, dass das der Ravenclaw-Tisch ist?} schaffte Harry, nicht laut zu sagen.

"Brillant!" flüsterte Harry. "Wenn es jeder weiß, wird keiner vermuten, dass es ein Geheimnis ist!"

"Ganz genau," flüsterte ein Schüler, dann runzelte er die Stirn. "Moment, das klingt nicht ganz richtig -"

\emph{Notiz an mich: Die 25 Prozent der Hogwartsschüler bekannt als Haus Ravenclaw sind nicht das exklusivste Begabtenförderprogramm der Welt.}

Doch zumindest hatte er heute eine wichtige Erkenntnis gewonnen. Der Comed-Tee war allmächtig. Und \emph{das} bedeutete…

Harry blinzelte überrascht, als sein Verstand endlich die offensichtliche Verbindung herstellte.

…\emph{das} bedeutete, dass sobald er einen Zauber lernte, mit dem er zeitweilig seinen Sinn für Humor verändern konnte, er \emph{alles} geschehen lassen konnte, indem er dafür sorgte, dass er \emph{nur} diese eine Sache überraschend genug fände, umdeshalb heraus zu prusten und dann eine Dose Comed-Tee trank.

\emph{Nun, das war ja eine kurze Reise zur Göttlichkeit. Selbst ich hätte erwartet, dass es länger als bis zu meinem ersten Schultag dauern würde.}

Ganz zu schweigen davon, dass er ganz Hogwarts in den kurzen zehn Minuten seiner Auswahl zu Grunde gerichtet hatte.

Ein wenig schuldig fühlte Harry sich deshalb schon - Merlin allein wusste, was ein wahnsinniger Schulleiter in seinen nächsten sieben Schuljahren anstellen mochte - aber er konnte sich \emph{auch} einen Anflug von Stolz nicht verkneifen.

Morgen. Nicht später als allerspätestens morgen würde er sich von dem Pfad abwenden, der zum Dunklen Lord Harry führte. Eine Aussicht, die mit jeder Minute erschreckender klang.

Und doch, irgendwie, immer attraktiver. Ein Teil seines Geistes malte sich gerade die Uniformen der Lakaien aus.

"Iss," grollte der ältere Schüler, der neben ihm saß und stieß Harry in die Rippen. "Denk nicht. Iss."

Harry begann automatisch, sich den Teller voll zu laden, mit dem was auch immer vor ihm stand, blaue Würstchen mit kleinen leuchtenden Stückchen, wieso nicht.

"Was haltet ihr von der Auswahl -" setzte Padma Patil an, eine der anderen Ravenclaws im ersten Jahr.

"Während der Mahlzeiten wird niemand gelöchert!" sagten mindestens drei Leute im Chor. "Hausregel!" fügte ein anderer hinzu. "Sonst würden wir hier alle verhungern."

Harry stellte fest, dass er wirklich, wirklich hoffte, dass seine clevere neue Idee nicht \emph{tatsächlich}klappte. Und das der Comed-Tee irgendwie anders funktionierte und nicht \emph{wirklich} die allmächtige Kraft besaß, die Realität zu verändern. Nicht dass er nicht allmächtig sein \emph{wollte.} Er konnte nur einfach den Gedanken nicht ertragen, in einem Universum zu leben, das wirklich so funktionierte. Es hatte etwas \emph{würdeloses} an sich, den Aufstieg zu erreichen durch cleveren Einsatz von Sprudelgetränken.

Doch er \emph{würde} es experimentell überprüfen.

"Weißt du," sagte der ältere Schüler neben ihm in überaus freundlichem Tonfall, "wir haben da Methoden, um solche wie dich zum Essen zu bringen, würdest du gern rausfinden, wie wir es anstellen?"

Harry gab es auf und fing an, sein blaues Würstchen zu essen. Es war gar nicht schlecht, besonders die leuchtenden Stückchen.

Das Abendessen ging überraschend schnell vorüber. Harry bemühte sich, zumindest ein wenig von all den seltsamen neuen Speisen zu versuchen, die er sah. Seine Neugier konnte den Gedanken nicht ertragen, \emph{nicht zu wissen,} wie etwas schmeckte. Zum Glück war das nicht eines jener Restaurants, in denen man nur eine Sache bestellen durfte und niemals herausfand, wie die anderen Sachen auf der Speisekarte schmeckten. Harry \emph{hasste} das, es war wie eine Folterkammer für jeden, der auch nur einen Funken Neugier besaß: \emph{Finde alles über nur eines der Mysterien auf dieser Liste heraus, ha ha ha!}

Dann war es Zeit für den Nachtisch, für den Harry völlig vergessen hatte, noch Platz zu lassen. Nachdem er ein wenig Siruptorte* versucht hatte, gab er auf. Sicherlich würden alle diese Sachen im Verlauf des Schuljahres wenigstens noch einmal vorkommen.

Also, was stand noch auf seiner To-do-Liste, abgesehen von den gewöhnlichen Schulsachen?

\emph{Aufgabe 1: Forsche nach Zaubern für geistige Veränderungen, um den Comed-Tee testen zu können und herauszufinden, ob du tatsächlich einen Weg zur Allmacht entdeckt hast. Eigentlich, erforsche einfach jede Art von geistiger Magie, die du finden kannst. Der Geist ist die Grundlage unserer Macht als Menschen, jede Art von Magie, die ihn beeinflusst, ist die wichtigste Sorte Magie, die es gibt.}

\emph{Aufgabe 2: Eigentlich ist das Aufgabe 1 und die andere ist Aufgabe 2. Gehe die Bücherregale der Hogwarts- und Ravenclaw-Bibliotheken} \emph{durch, mache dich mit dem System vertraut und stelle sicher, dass du zumindest alle Buchtitel gelesen hast. Zweiter Durchgang: Lies alle Inhaltsverzeichnisse. Stimme dich mit Hermine ab, die ein viel besseres Gedächtnis hat als du. Finde heraus ob es ein Fernleihe-System in} \emph{Hogwarts gibt und finde heraus, ob ihr beide, besonders Hermine, diese anderen Bibliotheken ebenfalls besuchen könnt. Wenn andere Häuser über private Bibliotheken verfügen, finde heraus, wie du rechtmäßig Zutritt erhältst oder dich dort einschleichen kannst.}

\emph{Option 3a: Lass Hermine schwören, Stillschweigen zu bewahren und fange an, nach 'Von Sslytherin zzu Sslytherin: Ssuchsst du meine Geheimnissse, sso ssprich zzu meiner Sschlange.' zu forschen. Problem: Das klingt höchst vertraulich und es könnte eine ganze Weile dauern, zufällig über ein Buch zu stolpern, das einen Hinweis enthält.}

\emph{Aufgabe 0: Finde heraus, welche} \emph{Zauber es gibt, um Informationen aufzuspüren und abzurufen,} \emph{wenn überhaupt. Bibliotheksmagie ist} \emph{letzten Endes} \emph{nicht so wichtig wie geistige Magie, hat aber viel höhere Priorität.}

\emph{Option 3b: Finde einen Zauber, um Draco Malfoy auf magische Weise zur Verschwiegenheit zu verpflichten oder Dracos Versprechen, ein Geheimnis zu bewahren, magisch zu verifizieren (Veritaserum?) und frage} ihn \emph{nach Slytherins Botschaft…}

Eigentlich… hatte Harry ein ziemlich schlechtes Gefühl, was Option 3b betraf.

Jetzt wo Harry darüber nachdachte, kam ihm Option 3a auch nicht mehr wirklich so toll vor.

Harrys Gedanken vollführten eine Rückblende zum womöglich schlimmsten Moment seines bisherigen Lebens, jenen langen Sekunden des Entsetztens unter dem Hut, die ihm das Blut in den Adern gefrieren ließen, als er glaubte, bereits gescheitert zu sein. In jenem Augenblick hatte er sich gewünscht, die Zeit nur ein paar Minuten zurückdrehen und etwas, irgendetwas anders machen zu können, bevor es zu spät war…

Und dann hatte sich herausgestellt, dass es noch gar nicht zu spät war.

Wunsch gewährt.

Man konnte die Geschichte nicht ändern. Aber man konnte es gleich zu Anfang richtig machen. Gleich beim \emph{ersten} Versuch etwas anders machen.

Diese ganze Angelegenheit mit der Suche nach Slytherins Geheimnissen… schien verdächtig genau die Art von Sache zu sein, bei der man, Jahre später, zurückblicken und sagen würde 'Und \emph{das} war der Punkt, ab dem einfach alles schief gelaufen ist.'

Und er würde sich verzweifelt die Fähigkeit ersehnen, in der Zeit zurück zu gehen und eine andere Wahl zu treffen…

Wunsch gewährt. Was jetzt?

Ein Lächeln breitete sich aus auf Harrys Gesicht.

Es war ein eher \emph{unintuitiver} Gedanke… aber…

Aber er \emph{könnte,} es gab keine Regel, die besagte, dass er es nicht konnte, er \emph{könnte} einfach so tun, als habe er dieses leise Flüstern niemals gehört. Das Universum exakt so fortfahren lassen, als habe sich dieser kritische Augenblick niemals ereignet. In zwanzig Jahren wäre es das, wovon er sich verzweifelt wünschen würde, dass es vor zwanzig Jahren geschehen wäre und zwanzig Jahre vor in zwanzig Jahren war zufällig genau jetzt. Die ferne Vergangenheit zu verändern war einfach, man musste nur rechtzeitig daran denken.

Oder… was sogar \emph{noch} unintuitiver war… er könnte, oh, sagen wir, \emph{Professor McGonagall} informieren, anstatt Draco \emph{oder} Hermine. Und sie könnte ein paar fähige Leute zusammentrommeln und diesen kleinen extra Zauber von dem Hut entfernen lassen.

Aber ja. Das klang nach einer \emph{bemerkenswert} guten Idee, jetzt wo Harry tatsächlich \emph{darauf gekommen} war.

So vollkommen offensichtlich im Rückblick und doch, irgendwie, hatte eran Option 3c und Option 3d einfach nicht gedacht.

Harry verlieh sich selbst +1 Punkt für sein Anti-Dunkler-Lord-Harry-Programm.

Es war ein furchtbar grausamer Streich gewesen, den der Hut ihm da gespielt hatte, doch auf konsequentialistischer Ebene ließ sich über das Ergebnis nicht streiten. Jedenfalls verhalf er ihm mit Sicherheit zu einem deutlich besseren Einblickin die Perspektive des Opfers.

\emph{Aufgabe 4: Entschuldige dich bei Neville Longbottom.}

Okay, er hatte hier gerade einen Lauf, jetzt musste er nur am Ball bleiben. \emph{Ich werde mit jedem Tag, auf jede Art, immer ein bisschen lichter…}**

Die meisten Leute um Harry herum beendeten an diesem Punkt ebenfalls ihr Essen und die Schüsseln mit den Nachspeisen begannen zu verschwinden, sowie auch die benutzten Teller.

Als alle Teller verschwunden waren, erhob sich Dumbledore erneut von seinem Platz.

Harry konnte nicht anders, als den Drang zu verspüren, einen weiteren Comed-Tee zu trinken.

\emph{Du machst} \emph{doch} \emph{wohl WITZE,} dachte Harry zu jenem Teil seiner selbst.

Doch das Experiment zählte nicht, wenn es nicht wiederholt wurde, nicht wahr? Und der Schaden war bereits angerichtet, oder nicht? Wollte er etwa nicht sehen, was \emph{dieses} mal geschah? War er denn nicht \emph{neugierig?} Was, wenn er ein anderes Ergebnis erhielt?

\emph{Hey, ich wette du bist der gleiche Teil meines Hirns, der den Streich gegen Neville} \emph{Longbottom} \emph{durchgedrückt hat.}

Äh, kann sein?

\emph{Und ist es nicht} überwältigend \emph{offensichtlich, dass ich das bereuen werde, eine Sekunde nachdem es zu spät ist?}

Ähm…

\emph{Genau. Also, NEIN.}

"Ähem," sagte Dumbledore vom Podium aus und strich über seinen langen silbernen Bart. "Nur ein paar wenige Worte noch, nun da wir alle gefüttert und gewässert sind. Ich habe ein paar kurzfristige Ankündigungen für Euch zum Beginn des neuen Schuljahres."

"Die Erstklässler nehmen bitte zur Kenntnis, dass der Wald auf unserem Schulgelände für alle Schüler verboten ist. Deshalb wird er der Verbotene Wald genannt. Wäre der Zutritt erlaubt, würde er der Erlaubte Wald genannt werden."

Klar und deutlich. \emph{Notiz an mich: Verbotener Wald ist verboten.}

"Desweiteren wurde ich von unserem Hausmeister Mr. Filch gebeten, euch daran zu erinnern, dass der Einsatz von Magie außerhalb des Unterrichts in den Korridoren unterlassen werden sollte. Leider wissen wir alle, dass was sein \emph{sollte} und was \emph{ist,} zwei verschiedene Dinge sind. Bitte behaltet das im Hinterkopf."

Äh…

"Die Quidditch-Auswahlen finden in der zweiten Woche des Schuljahres statt. Alle, die daran interessiert sind, für ihre Hausmannschaften zu spielen, sollten mit Madam Hooch in Verbindung treten. Alle, die daran interessiert sind, das gesamte Spiel Quidditch neu zu definieren, wenden sich bitte an Harry Potter."

Harry verschluckte sich an seinem eigenen Speichel und bekam einen Hustenanfall, gerade als sich aller Augen auf ihn richteten. Wie zur \emph{Hölle!} Er war Dumbledores Blick zu keinem Zeitpunkt begegnet… er hatte gar nichts \emph{gedacht.} Er hatte zu dem Zeitpunkt ganz sicher nicht über Quidditch nachgedacht! Er hatte mit niemandem darüber geredet außer Ron Weasley und er glaubte \emph{kaum,} dass Ron es irgendwem erzählt hätte… oder war Ron zu einem Professor gerannt, um sich zu beschweren? \emph{Wie} in aller \emph{Welt…}

"Außerdem muss ich euch mitteilen, dass in diesem Jahr der rechte Korridor im dritten Stockwerk für alle tabu ist, die nicht den Wunsch verspüren, eines äußerst schmerzvollen Todes zu sterben. Er wird von einer ausgefeilten Reihe von gefährlichen und potentiell tödlichen Fallen geschützt, die ihr unmöglich allesamt überwinden könnt, besonders wenn ihr erst in eurem ersten Schuljahr seid."

An diesem Punkt beschlich Harry ein taubes Gefühl.

"Und schlussendlich möchte ich Quirinus Quirrell meinen größten Dank aussprechen, der sich heldenhaft bereit erklärt hat, die Stelle als Professor für Verteidigung gegen die Dunklen Künste in Hogwarts anzutreten." Dumbledores Blick wanderte suchend über die Schülerschaft. "Ich hoffe, dass alle Schüler Professor Quirrell mit der äußersten Höflichkeit und \emph{Toleranz} begegnen, die seinem außerordentlichen Einsatz für euch und diese Schule gebühren und dass ihr uns \emph{nicht} mit \emph{kleinlichen Beschwerden} jedweder Art über ihn behelligen werdet, falls ihr euch nicht \emph{selbst} an seinem Job versuchen möchtet."

Was hatte \emph{das} nun wieder zu bedeuten?

"Ich räume nun das Feld für unser neues Fakultätsmitglied Professor Quirrell, der gern ein paar Worte sagen würde."

Der junge, schmale, nervöse Mann, dem Harry zum ersten mal im Tropfenden Kessel begegnet war, bahnte sich langsam seinen Weg zum Podium hinauf, sich nach allen Seiten ängstlich umschauend. Harry erhaschte einen Blick auf seinen Hinterkopf und es wirkte, als würde Professor Quirrell bereits kahl, trotz seiner augenscheinlichen Jugend.

"Ich frag mich, was wohl mit \emph{ihm} nicht stimmt," flüsterte der älter wirkende Schüler, der neben Harry saß. Ähnliche verhaltene Kommentare wurden andernorts am Tisch ausgetauscht.

Professor Quirrell bewältigte den Weg nach oben zum Podium und verharrte dort, blinzelnd. "Ah…" sagte er. "Ah…" Dann schien ihn der Mut vollends zu verlassen und er stand dort in Schweigen gehüllt und zuckte gelegentlich.

"Oh, klasse," flüsterte der ältere Schüler, "sieht mal wieder nach einem \emph{langen} Jahr in Verteidigung aus -"

"Ich grüße Sie, meine jungen Lehrlinge," sagte Professor Quirrell in trockenem, zuversichtlichem Ton. "Wir alle wissen, dass Hogwarts zu einem gewissen \emph{Misserfolg}neigt, was seine Auswahl für diese Stelle betrifft und ohne Zweifel werden sich einige unter Ihnen bereits fragen, welches Unheil mir dieses Jahr wohl zum Verhängnis werden mag. Ich versichere Ihnen, dass meine Inkompetenz nicht jenes Unheil sein wird." Er lächelte dünn. "Ob Sie es glauben oder nicht, ich hegte schon lange den Wunsch, mich eines Tages als Professor für Verteidigung gegen die Dunklen Künste hier an der Hogwarts-Schule für Hexerei und Zauberei zu versuchen. Der erste, der dieses Fach unterrichtete, war Salazar Slytherin selbst und bis ins vierzehnte Jahrhundert hinein war es Tradition, für die größten kämpfenden Zauberer jedweder Überzeugung sich an der Lehre hier zu versuchen. Zu den vergangenen Professoren für Verteidigung zählten nicht nur der legendäre wandernde Held Harold Shea, sondern auch die Zitat unsterbliche Zitat Ende Baba Yaga, ja, ich sehe, einige unter Ihnen erschauern noch immer beim Klang ihres Namens, obgleich sie bereits seit sechshundert Jahren tot ist. Das muss eine interessante Zeit gewesen sein, um Hogwarts zu besuchen, meinen Sie nicht auch?"

Harry schluckte schwer, versuchte den plötzlichen Schwall von Emotionen zu beherrschen, der ihn überkommen hatte, als Professor Quirrell zu sprechen begann. Die präzise Ausdrucksweise erinnerte ihn so sehr an einen Oxford-Dozenten und allmählich traf ihn die Erkenntnis, dass er sein Zuhause und seine Mum und seinen Dad bis Weihnachten nicht wiedersehen würde.

"Sie sind es gewohnt, dass die Verteidigungs-Position eingenommen wurde von den Unfähigen, Scharlatanen und Unglückseligen. Für jeden mit einem Hauch von Geschichtsbewusstsein trägt sie jedoch einen vollkommen anderen Ruf. Nicht jeder, der hier lehrte, zählte zu den Besten, doch die Besten lehrten alle in Hogwarts. In solch erlesener Gesellschaft und nach so langer Zeit der Erwartung dieses Tages, würde ich mich schämen, mir irgendein geringeres Ziel zu setzen, als Perfektion. Und so beabsichtige ich, dass jeder einzelne von Ihnen sich immer an dieses Jahr erinnern wird, als den \emph{besten}Unterricht in Verteidigung, den Sie je hatten. Was Sie in diesem Jahr lernen, wird auf ewig als Ihr festes Fundament in der Kunst der Vereidigung dienen, ungeachtet wer mir als Ihr Lehrmeister vorausging oder noch folgen mag."

Professor Quirrells Gesicht nahm einen ernsten Ausdruck an. "Wir haben \emph{sehr} viel Verlorenes aufzuholen und uns bleibt nur wenig Zeit dafür. Daher beabsichtige ich, von den Lehrgewohnheiten von Hogwarts in vielerlei Hinsicht abzuweichen, sowie einige freiwillige außerschulische Aktivitäten einzuführen." Er hielt inne. "Sollte sich das als nicht ausreichend erweisen, so mag ich neue Wege finden, Sie zu motivieren. Sie sind meine lang erwarteten Studenten und Sie \emph{werden} in meinem lang erwarteten Unterricht in Verteidigung Ihr \emph{aller}bestes geben. Ich würde noch eine verhängnisvolle Drohung hinzufügen, wie 'Ansonsten werden Sie schrecklich zu leiden haben', doch das wäre so klischeehaft, finden Sie nicht? Ich bin stolz darauf, sehr viel einfallsreicher zu sein als das. Ich danke Ihnen."

Daraufhin schienen die Kraft und Zuversicht aus Professor Quirrell zu weichen. Sein Mund klaffte offen, als fände er sich urplötzlich vor einem unerwarteten Publikum wieder und mit einem krampfhaften Ruck wandte er sich um und schlurfte zurück an seinen Platz, sackte vornüber, als bräche er gleich in sich zusammen und implodierte.

"Er scheint etwas merkwürdig," flüsterte Harry.

"Nah," sagte der älter wirkende Schüler. "Du hast ja noch gar nichts geseh'n."

Dumbledore nahm erneut das Podium ein.

"Und nun," sagte Dumbledore, "bevor wir zu Bett gehen, lasst uns alle die Schulhymne singen! Jeder sucht sich seine liebste Melodie und seine Lieblingsworte aus und ab geht's!"

* auch \emph{Zuckergusstorte,} engl.: \emph{treacle tart,} eine traditionelle britische Nachspeise.

** engl.: \emph{In every day, in every way, I'm getting Lighter and Lighter…} Das ist offenbar eine Abwandlung des Mantras von Émile Coué, dem Begründer der modernen, bewussten Autosuggestion.

