

\hypertarget{verwirrung-erkennen}{% \section{26. Verwirrung erkennen}\label{verwirrung-erkennen}}

\textbf{Kapitel 26: Verwirrung erkennen}

Yakka foob mog. Grug pubbawup zink wattoom gazork. Chumble spuzz J. K. Rowling.

\later

Professor Quirrells Sprechzeiten waren von 11:40 bis 11:55 am Donnerstagmorgen. Das galt für all seine Schüler in allen Schuljahren. Es kostete einen Quirrell-Punkt, auch nur an die Tür zu klopfen und wenn er nicht der Ansicht war, das Anliegen sei seine Zeit wert, verlor man weitere fünfzig.

Harry klopfte an die Tür.

Es gab eine Pause. Dann sagte eine beißende Stimme, „Ich nehme an, Sie können ebenso gut eintreten, Mr. Potter.“

Und noch bevor Harry den Türknauf drehen konnte, wurde die Tür aufgeschlagen, traf die Wand mit einem scharfen Krachen als habe etwas das Holz zerbersten lassen oder den Stein oder beides.

Professor Quirrell lehnte sich in seinem Stuhl zurück und las ein verdächtig alt-aussehendes Buch, gebunden in nacht-blaues Leder mit silbernen Runen auf dem Rücken. Seine Augen waren nicht von den Seiten gewichen. „Ich bin in keiner guten Stimmung, Mr. Potter. Und wenn ich nicht in guter Stimmung bin, bin ich keine angenehme Gesellschaft. Um Ihrer selbst willen, erläutern Sie Ihr Anliegen zügig und gehen Sie.“

Ein kalter Hauch ging von dem Raum aus als beinhalte er etwas, das Dunkelheit verströmte, wie Lampen Licht verströmten und das nicht ganz abgeschirmt worden war.

Harry schrak ein wenig zurück. \emph{Nicht in guter Stimmung} schien nicht ganz der richtige Ausdruck. Was konnte Professor Quirrell so zu schaffen machen…?

Nun, man ließ Freunde nicht einfach allein, wenn es ihnen schlecht ging.

Harry trat vorsichtig in den Raum vor. „Kann ich irgendwie helfen -“

„Nein,“ sagte Professor Quirrell und blickte noch immer nicht von dem Buch auf.

„Ich meine, wenn Sie sich mit Dummköpfen befassen mussten und mit jemand vernünftigem reden möchten…“

Eine überraschend lange Pause entstand.

Professor Quirrell schlug das Buch zu und es verschwand mit einem flüsternden Geräusch. Dann blickte er auf und Harry zuckte zusammen.

„Ich nehme an, eine intelligente Unterhaltung wäre für \emph{mich} zu diesem Zeitpunkt erfreulich,“ sagte Professor Quirrell im selben beißenden Tonfall, mit dem er Harry hereingebeten hatte. „Dass \emph{Sie} es so empfinden, ist unwahrscheinlich, seien Sie gewarnt.“

Harry nahm einen tiefen Atemzug. „Ich verspreche, es macht mir nichts aus, wenn Sie die Beherrschung verlieren. Was ist passiert?“

Die Kälte in dem Raum schien sich noch zu vertiefen. „Ein Sechstklässler aus Gryffindor hat einen Fluch gegen einen meiner vielversprechenderen Schüler gerichtet, einen Sechstklässler aus Slytherin.“

Harry schluckte. „Was… für ein Fluch?“

Und Professor Quirrell hielt den Zorn auf seinem Gesicht nicht länger im Zaum. „Warum sich die Mühe machen, eine solch unwichtige Frage zu stellen, Mr. Potter? Unser Freund der Gryffindor-Sechstklässler hielt es nicht für bedeutsam!“

„Ist das Ihr \emph{Ernst?}“ sagte Harry, bevor er es verhindern konnte.

„Nein, ich bin heute in schrecklicher Laune aus keinem bestimmten Grund. \emph{Ja, das ist mein Ernst, Sie Narr!} Er wusste es nicht. Er \emph{wusste es tatsächlich nicht.} Ich habe es nicht geglaubt, bis die Auroren es unter Veritaserum bestätigten. Er ist in seinem \emph{sechsten Jahr in Hogwarts} und er wirkte einen hochrangigen Dunklen Fluch \emph{ohne zu wissen, was er bewirkte.}“

„Sie meinen nicht,“ sagte Harry, „dass er sich irgendwie \emph{geirrt} hat, was er bewirkte, dass er irgendwie die falsche Zauber-Beschreibung gelesen hat -“

„Er wusste nur, dass er gegen einen Feind gerichtet werden sollte. Er \emph{wusste,} das war alles, was er wusste.“

Und das hatte ausgereicht, um den Zauber zu wirken. „Ich verstehe nicht, wie irgendwas mit einem so kleinen Hirn aufrecht gehen kann.“

„In der Tat, Mr. Potter,“ sagte Professor Quirrell.

Eine Pause entstand. Professor Quirrell lehnte sich vor und ergriff das silberne Tintenfass auf seinem Schreibtisch, drehte es in den Händen, starrte es an als frage er sich, wie er ein Tintenfass zu Tode foltern könnte.

„Wurde der Slytherin-Sechstklässler ernsthaft verletzt?“ sagte Harry.

„Ja.“

„Wurde der Gryffindor-Sechstklässler von Muggeln aufgezogen?“

„\emph{Ja.}“

„Weigert sich Dumbledore, ihn der Schule zu verweisen, weil der arme Junge nicht wusste, was er tat?“

Professor Quirrells Hände schlossen sich mit weißen Knöcheln um das Tintenfass. „\emph{Wollen Sie auf etwas hinaus, Mr. Potter oder sprechen Sie nur das} \emph{Offensichtliche aus?}“

„Professor Quirrell,“ sagte Harry ernst, „alle von Muggeln aufgezogenen Schüler in Hogwarts brauchen eine Sicherheitsunterweisung, in der ihnen die Dinge beigebracht werden, die so lächerlich offensichtlich sind, dass keinem Zauberergeborenen jemals in den Sinn käme, sie anzusprechen. Sprich keine Flüche aus, wenn du nicht weißt, was sie tun, wenn du etwas gefährliches entdeckst, erzähle nicht der ganzen Welt davon, braue keine hochrangigen Zaubertränke ohne Aufsicht in einem Klo, der Grund, warum es Gesetze zur Beschränkung der Zauberei Minderjähriger gibt, die Grundlagen eben.“

„Warum?“ sagte Professor Quirrell. „Sollen die Dummen sterben, bevor sie sich vermehren.“

„Wenn es Ihnen nichts ausmacht, dass sie ein paar Slytherin-Sechstklässler mit sich reißen.“

Das Tintenfass ging in Professor Quirrells Händen in Flammen auf und brannte mit schrecklicherLangsamkeit nieder, grässliche schwarz-orange Flammen zehrten an dem Metall und schienen kleine Stücke heraus zu beißen, das Silber verzog sich während es schmolz, als versuche es vergeblich zu entkommen. Ein blechernes, kreischendes Geräusch ertönte, als würde das Metall schreien.

„Ich nehme an, Sie haben recht,“ sagte Professor Quirrell mit ergebenem Lächeln, „ich werde eine Unterweisung ausarbeiten, um sicherzustellen, dass Muggelgeborene, die zum Leben zu dumm sind, niemand wertvollen mit sich nehmen, wenn sie gehen.“

Das Tintenfass schrie weiter und brannte in Professor Quirrells Händen, winzige Tröpfchen aus Metall, noch immer in Flammen stehend, tropften jetzt auf den Schreibtisch, als würde das Tintenfass weinen.

„Sie laufen nicht weg,“ beobachtete Professor Quirrell.

Harry öffnete den Mund -

„Wenn Sie sagen wollen, dass Sie keine Angst vor mir haben,“ sagte Professor Quirrell, „\emph{tun Sie's nicht.}“

„Sie sind die beängstigendste Person, die ich kenne,“ sagte Harry, „und einer der wichtigsten Gründe dafür, ist ihre Selbstkontrolle. Ich kann mir nicht vorstellen, davon zu hören, Sie hätten jemanden verletzt, den zu verletzen Sie nicht bewusst entschieden hätten.“

Das Feuer in Professor Quirrells Händen verlosch und er setzte das zerstörte Tintenfass sorgsam auf seinem Schreibtisch ab. „Sie sagen die nettesten Dinge, Mr. Potter. Haben Sie Unterricht in Schmeichelei genommen? Vielleicht von Mr. Malfoy?“

Harry behieltein ausdrucksloses Gesicht bei und erkannte eine Sekunde zu spät, dass dies sehr wohl als unterschriebenes Geständnis verstanden werden mochte. Professor Quirrell interessierte es nicht, wie der eigene Gesichtsausdruck wirkte, sondern auf welchen Geisteszustand er hindeutete.

„Ich verstehe,“ sagte Professor Quirrell. „Mr. Malfoy ist ein nützlicher Freund, Mr. Potter und es gibt vieles, was er Ihnen beizubringen vermag, doch ich hoffe, dass Sie nicht den Fehler begangen haben, ihm allzu vieles anzuvertrauen.“

„Er weiß nichts, von dem ich fürchten müsste, dass es bekannt wird,“ sagte Harry.

„Sehr gut,“ sagte Professor Quirrell mit dünnem Lächeln, „Also, welches Anliegen führte Sie ursprünglich hierher?“

„Ich bin fertig mit den vorbereitenden Übungen in Okklumentik und bereit für den Privatlehrer.“

Professor Quirrell nickte. „Ich werde Sie am Sonntag zu Gringotts begleiten.“ Er hielt inne, blickte Harry an und lächelte. „Und vielleicht machen wir sogar einen kleinen Umweg, wenn Sie mögen. Mir kam gerade ein erfreulicher Gedanke.“

Harry nickte, erwiderte das Lächeln.

Als Harry das Büro verließ, hörte er Professor Quirrell eine kleine Melodie summen.

Harry war froh, dass er ihn hatte aufmuntern können.

\later

An diesem Sonntag schien eine ziemlich große Zahl von Leuten in den Gängen zu flüstern, zumindest wenn Harry Potter an ihnen vorbeiging.

Und viele deutende Finger.

Und sehr viel weibliches Gekicher.

Es hatte beim Frühstück angefangen, als jemand Harry gefragt hatte, ob er die Nachrichten gehört habe und Harry ihn schnell unterbrochen und gemeint hatte, dass wenn die Nachrichten von Rita Kimmkorn verfasst worden waren, er nichts davon \emph{hören,} sondern sie selbst in der Zeitung lesen wolle.

Daraufhin hatte es sich ergeben, dass nicht allzu viele Schüler eine Ausgabe des \emph{Tagespropheten} besaßen und dass die Kopien, die nicht bereits von ihren Eigentümern aufgekauft worden waren, in einer komplizierten Reihenfolge herumgereicht wurden und niemand wirklich wusste, wer gerade eine hatte…

Also hatte Harry einen Stillezauber angewandt und war mit dem Frühstück fortgefahren, im Vertrauen darauf, dass seine Sitznachbarn die vielen, vielen Fragesteller abwimmelten und gab sein Bestes, den Unglauben zu ignorieren, das Gelächter, die gratulierenden Lächler, die mitfühlenden Blicke, die furchtsamen Seitenblicke und die fallenden Teller, als neue Leute zum Frühstück herunterkamen und die Neuigkeiten vernahmen.

Harry war \emph{ziemlich} neugierig, aber es wäre \emph{wirklich} nicht angemessen gewesen, die Kunstfertigkeit durch Berichte aus zweiter Hand herabzuwürdigen.

Er hatte die nächsten paar Stunden seine Hausaufgaben in der Sicherheit seines Koffers gemacht, nachdem er seinen Schlafsaal-Kameraden gesagt hatte, sie sollten ihn holen kommen, wenn jemand eine Original-Zeitung für ihn gefunden habe.

Harry wusste noch immer nichts, als er um 10 Uhr morgens Hogwarts in einer Kutsche zusammen mit Professor Quirrell verlassen hatte, der vorne rechts saß und augenblicklich im Zombie-Modus zusammengesunken war. Harry saß ihm schräg gegenüber, so weit entfernt, wie es die Kutsche gestattete, hinten links. Trotzdem überkam Harry ein andauerndes Gefühl des Unheils während die Kutsche über einen schmalen Pfad durch einen Teil des nicht-verbotenen Waldes ratterte. Das erschwerte das Lesen etwas, besonders da der Stoff anspruchsvoll war und Harry wünschte plötzlich, er lese stattdessen eines seiner kindischen Science-Fiction-Bücher -

„Wir sind außerhalb der Schutzzauber,* Mr. Potter,“ erklang Professor Quirrells Stimme von vorn. „Zeit zu gehen.“

Professor Quirrell stieg vorsichtig aus der Kutsche, hielt sich beim Aussteigen fest. Harry, auf seiner Seite, sprang hinunter.

Harry fragte sich, wie genau sie dorthin kämen, als Professor Quirrell sagte „Fang!“ und ihm einen bronzenen Knut zuwarf und Harry fing ihn auf ohne nachzudenken.

Ein riesiger immaterieller Haken packte Harry hinter dem Bauchnabel und zog ihn nach hinten, kräftig, doch ohne ein Gefühl von Beschleunigung und einen Augenblick später standen sie mitten in der Winkelgasse.

(\emph{Moment mal, was?} sagte sein Hirn.)

(\emph{Wir wurden gerade teleportiert,} erklärte Harry.)

(\emph{Das kam in unserer angestammten Umgebung nicht vor,} beschwerte sich Harrys Hirn und desorientierte ihn.)

Harry stolperte als seine Füße sich an Straßenpflaster anpassten, anstelle dem Dreck des Waldweges, den sie zuvor durchquert hatten. Er richtete sich auf, noch immer benommen, wobei die hin und her eilenden Hexen und Zauberer leicht zu schwanken und die Schreie der Ladeninhaber in seinem Gehör zu wandern schienen, während sein Hirn versuchte, eine Welt auszumachen, in der es sich befand.

Augenblicke später ertönte eine Art saugend-ploppendes Geräusch ein paar Schritte hinter Harry und als Harry sich umdrehte, stand dort Professor Quirrell.

„Macht es Ihnen etwas aus -“ sagte Harry, im selben Moment als Professor Quirrell sagte, „Ich fürchte ich -“

Harry hielt inne, Professor Quirrell nicht.

„- muss gehen und einiges in die Wege leiten, Mr. Potter. Da mir ausdrücklich erklärt wurde, dass ich für alles verantwortlich bin, was auch immer Ihnen zustoßen sollte, lasse ich Sie bei -“

„Zeitungsstand,“ sagte Harry.

„Verzeihung?“

„Oder irgendwo, wo ich eine Kopie des \emph{Tagespropheten} kaufen kann. Lassen Sie mich dort und mir geht's gut.“

Kurz darauf war Harry in einem Buchladen abgeliefert worden, begleitet von einigen leise gesprochenen, mehrdeutigen Drohungen. Und der Ladeninhaber hatte \emph{weniger} mehrdeutige Drohungen erhalten, danach zu urteilen, wie er erschaudert war und wie seine Augen nun zwischen Harry und dem Eingang hin und her huschten.

Falls der Buchladen nieder brannte, würde Harry inmitten des Feuers warten, bis Professor Quirrell zurückkehrte.

In der Zwischenzeit -

Harry blickte sich schnell um.

Der Buchladen schien ziemlich klein und schäbig, mit nur vier sichtbaren Reihen von Bücherregalen und das nächste Regal, auf das Harrys Blick gefallen war, schien sich mit schmalen, billig gebundenen Büchern zu befassen, die trostlose Titel trugen, wie \emph{Das Massaker von Albanien im Fünfzehnten Jahrhundert.}**

Das wichtigste zuerst. Harry ging hinüber zum Verkaufstresen.

„Verzeihen Sie,“ sagte Harry, „Eine Ausgabe des \emph{Tagespropheten,} bitte.“

„Fünf Sickel,“ sagte der Ladeninhaber. „Sorry, Junge, ich hab nur noch drei.“

Fünf Sickel fielen auf den Tresen. Harry hatte das Gefühl, er hätte ihn noch etwas runter handeln können, doch an diesem Punkt, schien es ihm nicht wirklich wichtig.

Die Augen des Ladeninhabers weiteten sich und er schien Harry zum ersten mal wirklich zu bemerken. „\emph{Du!}“

„\emph{Ich!}“

„Ist es \emph{wahr?} Bist du \emph{wirklich -}“

„\emph{Ruhe!} Sorry, ich habe den \emph{ganzen Tag} darauf gewartet, es original in der Zeitung zu lesen, anstatt es aus zweiter Hand zu hören, also bitte \emph{geben Sie einfach her,} okay?

Der Ladeninhaber starrte Harry einen Augenblick lang an, griff dann wortlos unter den Tresen und überreichte ihm eine gefaltete Kopie des \emph{Tagespropheten.}

Die Schlagzeile:

\emph{HARRY POTTER

HEIMLICH VERLOBT

MIT GINEVRA WEASLEY}

Harry starrte.

Er hob die Zeitung vom Tresen, langsam, ehrfürchtig, als hielte er ein Original-Gemälde von Escher in Händen und entfaltete sie, um zu lesen…

… von den Beweisen, die Rita Kimmkorn überzeugt hatten.

… und einigen weiteren interessanten Details.

… und noch mehr Beweisen.

Fred und George hatten es doch sicherlich vorher mit ihrer Schwester abgeklärt? Ja, natürlich hatten sie das. Es gab ein Bild von Ginevra Weasley, sehnsüchtig seufzend über etwas, dass, soweit Harry erkennen konnte, ein Foto von ihm selbst war. Das musste gestellt worden sein.

Aber \emph{wie} in aller \emph{Welt…?}

Harry saß in einem billigen Klappstuhl, las die Zeitung bereits zum vierten mal, als die Tür leise seufzte und Professor Quirrell in den Laden trat.

„Bitte entschuldigen Sie - \emph{was} in Merlins Namen lesen Sie da?“

„Es scheint,“ sagte Harry, mit Ehrfurcht in der Stimme, „dass ein gewisser Mr. Arthur Weasley unter den Imperius-Fluch eines Todessers geriet, den mein Vater getötet hat, woraus eine Schuld dem Haus Potter gegenüber entstand, zu deren Begleichung mein Vater die Hand der neugeborenen Ginevra Weasley zum Bund der Ehe verlangt hat. Tun Leute so etwas hier wirklich?“

„Wie könnte Miss Kimmkorn \emph{jemals} dumm genug sein, zu glauben -“

Und Professor Quirrells Stimme riss ab.

Harry hatte die Zeitung hochkant und entfaltet gehalten, was bedeutete, dass Professor Quirrell, von dort wo er stand, den Text unter der Schlagzeile sehen konnte.

Der schockierte Ausdruck auf Professor Quirrells Gesicht war ein Kunstwerk, das der Zeitung selbst fast ebenbürtig war.

„Keine Sorge,“ sagte Harry fröhlich, „ist alles nur fake.“

Von anderswo in dem Laden war ein Keuchendes Ladeninhaberszu vernehmen. Das Geräusch eines fallenden Bücherstapels war zu hören.

„Mr. Potter…“ sagte Professor Quirrell langsam, „sind Sie da \emph{sicher?}“

„Vollkommen sicher. Sollen wir gehen?“

Professor Quirrell nickte, sah ziemlich abgelenkt aus und Harry faltete die Zeitung wieder zusammen und folgte ihm aus der Tür.

Aus irgendeinem Grund schien Harry jetzt keinen Straßenlärm mehr zu hören.

Sie gingen dreißig Sekunden still nebeneinander her, bevor Professor Quirrell sprach. „Miss Kimmkorn hat die originalen Verhandlungsprotokolle der geschlossenen Sitzung des Zaubergamots eingesehen.“

„Ja.“

„Die \emph{originalen Verhandlungsprotokolle des Zaubergamots.}“

„Ja.“

„\emph{Ich} hätte Schwierigkeiten, das fertig zu bringen.“

„Wirklich?“ sagte Harry. „Denn sollten meine Vermutungen korrekt sein, wurde dies von einem Hogwarts-Schüler vollbracht.“

„Das ist mehr als unmöglich,“ sagte Professor Quirrell schlicht. „Mr. Potter… ich bedaure Ihnen sagen zu müssen, dass diese junge Dame erwartet, Sie zu ehelichen.“

„Doch \emph{das} ist unwahrscheinlich,“ sagte Harry. „Um Douglas Adams zu zitieren, das Unmögliche besitzt oft eine Geschlossenheit, die dem bloß Unwahrscheinlichen fehlt.“

„Ich verstehe, was Sie meinen,“ sagte Professor Quirrell langsam, „Aber… nein, Mr. Potter. Es mag unmöglich sein, doch ich kann mir \emph{vorstellen,} die Verhandlungen des Zaubergamots zu manipulieren. Doch es ist \emph{unvorstellbar,} dass der Großmeister von Gringotts sein Amtssiegel für einen gefälschten Verlobungs-Vertrag hergeben sollte und Miss Kimmkorn hat jenes Siegel eigenhändig überprüft.“

„In der Tat,“ sagte Harry, „würde man erwarten, dass der Großmeister von Gringotts involviert ist, wenn so viel Geld den Besitzer wechselt. Es scheint, Mr. Weasley war hochverschuldet und verlangte daher eine zusätzliche Zahlung von zehntausend Galleonen -“

„\emph{Zehntausend} Galleonen für eine \emph{Weasley?} Dafür könnte man die Tochter eines Noblen Hauses kaufen!“

„Verzeihung,“ sagte Harry. „Ich muss an diesem Punkt wirklich nachhaken, tun Leute so etwas hier wirklich -“

„Selten,“ sagte Professor Quirrell stirnrunzelnd. „Und seit der Dunkle Lord verschwand, wie ich vermute, überhaupt nicht mehr. Ich nehme an, der Zeitung zufolge, hat Ihr Vater einfach gezahlt?“

„Er hatte keine Wahl,“ sagte Harry. „Nicht wenn er die Bedingungen der Prophezeiung erfüllen wollte.“

„\emph{Geben Sie mir das,}“ sagte Professor Quirrell und die Zeitung sprang so schnell aus Harrys Hand, dass er sich am Papier schnitt.

Harry steckte sich automatisch den Finger in den Mund und saugte daran, ziemlich schockiert und wandte sich protestierend Professor Quirrell zu -

Professor Quirrell war mitten auf der Straße wie angewachsen stehengeblieben und seine Augen flackerten rasant vor und zurück, während eine unsichtbare Kraft die Zeitung vor ihm ausgebreitet hielt.

Harry blickte mit offenem Staunen, als die Zeitung sich öffnete, um die Seiten zwei und drei zu enthüllen. Und nicht lange danach, vier und fünf. Es war als habe der Mann den Anschein der Sterblichkeit abgelegt.

Und nach verstörend kurzer Zeit faltete das Blatt sich wieder zusammen. Professor Quirrell pflückte sie aus der Luft und warf sie Harry zu, der sie reflexartig auffing und dann setzte Professor Quirrell sich wieder in Bewegung und Harry trottete automatisch hinterher.

„Nein,“ sagte Professor Quirrell, „diese Prophezeiung klang mir auch nicht ganz richtig.“

Harry nickte, noch immer wie betäubt.

„Die Zentauren könnten unter einem \emph{Imperius} gestanden haben,“ sagte Professor Quirrell stirnrunzelnd, „\emph{das} scheint verständlich. Was Magie erschaffen kann, kann Magie korrumpieren und es ist nicht undenkbar, dass das Großsiegel von Gringotts in unbefugte Hand gelangen könnte. Der Unsägliche könnte mit Vielsafttrank verkörpert worden sein, ebenso der Bayerische Seher.*** Und mit \emph{genügend} Aufwand mag es möglich sein, die Verhandlungen des Zaubergamots zu manipulieren. Haben Sie eine Ahnung, wie das angestellt wurde?“

„Ich habe keine einzige plausible Hypothese,“ sagte Harry. „Was ich weiß, ist, es wurde mit einem Budget von vierzig Galleonen getan.“

Professor Quirrell blieb wie angewurzelt stehen und wirbelte zu Harry herum. Nun mit einem Ausdruck absoluten Unglaubens auf dem Gesicht. „Vierzig Galleonen bezahlen einen kompetenten Fluchbrecher,**** der einen Weg in ein Haus ebnet, das man auszurauben wünscht! Vierzig\emph{tausend}Galleonen \emph{könnten} als Bezahlung ausreichen für ein Team der besten professionellen Verbrecher der Welt, um die Verhandlungen des Zaubergamots zu manipulieren!“

Harry zuckte hilflos mit den Schultern. „Ich werde daran denken, wenn ich das nächste mal Dreiundneunzigtausend und Neunhundertsechzig Galleonen sparen will, indem ich den richtigen Vertragspartner aussuche.“

„Ich sage das nicht oft,“ sagte Professor Quirrell. „Ich bin beeindruckt.“

„Ich ebenso,“ sagte Harry.

„Und wer ist dieser unglaubliche Hogwarts-Schüler?“

„Ich fürchte, das kann ich nicht sagen.“

Zu Harrys gelinder Überraschung erhob Professor Quirrell dagegen keinen Einwand.

Sie gingen in Richtung des Gringotts-Gebäudes, nachdenklich, da keiner von ihnen jemand war, der ein Problem auf sich beruhen lassen würde, ohne wenigstens fünf Minuten darüber nachzudenken.

„Ich habe das Gefühl,“ sagte Harry schließlich, „wir gehen das von der falschen Seite an. Da gibt es eine Geschichte, die ich einmal hörte, über ein paar Studenten, die in eine Physik-Klasse kamen und die Lehrerin zeigte ihnen eine große Metallplatte neben einem Feuer. Sie sagte ihnen, sie sollten die Metallplatte anfassen und sie fühlten, dass das Metall näher am Feuer kühler war und das weiter entfernte Metall wärmer. Und sie sagte, schreiben Sie ihre Vermutung auf, warum dies passiert. Also schrieben einige Schüler 'wegen der Art, wie das Metall die Hitze leitet' und einige Schüler schrieben 'wegen der Art, wie die Luft zirkuliert' und niemand schrieb 'das scheint einfach unmöglich' und die richtige Antwort war, dass bevor die Schüler den Raum betreten hatten, die Lehrerin die Platte herumgedreht hatte.“

„Interessant,“ sagte Professor Quirrell. „Das klingt ähnlich. Gibt es eine Moral aus der Geschichte?“

„Dass die eigene Stärke als Rationalist die Fähigkeit ist, von Fiktion verwirrter zu sein als von der Realität,“ sagte Harry. „Wenn man jeden Ausgang gleichermaßen erklären kann, weiß man gar nichts. Die Schüler dachten, sie könnten Worte wie 'wegen der Hitzeverteilung' benutzen, um alles zu erklären, selbst eine Metallplatte, die auf der Seite näher dem Feuer kühler ist. Daher bemerkten sie nicht, wie verwirrt sie waren und das hieß, sie konnten vom Falschen nicht verwirrter sein als von der Wahrheit. Wenn Sie mir sagen, dass die Zentauren unter dem \emph{Imperius}-Fluch standen, habe ich immer noch das Gefühl, das etwas nicht ganz stimmt, ich merke, dass ich noch immer verwirrt bin, selbst nach Ihrer Erklärung.“

„Hm,“ sagte Professor Quirrell.

Sie gingen weiter.

„Ich nehme nicht an,“ sagte Harry, „dass es möglich ist, Leute \emph{tatsächlich} in alternative Universen zu transportieren? Wie etwa, dass es nicht unsere eigene Rita Kimmkorn ist oder man sie zeitweise irgendwo anders hin geschickt hat?“

„Wenn \emph{das} möglich wäre,“ sagte Professor Quirrell in ziemlich trockenem Tonfall, „wäre ich dann noch \emph{hier?}“

Und gerade als sie fast an der großen weißen Fassade des Gringotts-Gebäudes angelangt waren, sagte Professor Quirrell:

„Ah. \emph{Natürlich.} Jetzt verstehe ich. Lassen Sie mich raten, die Weasley-Zwillinge?“

„\emph{Was?}“ sagte Harry, dessen Stimme noch eine Oktave höher sprang. „\emph{Wie?}“

„Ich fürchte, das kann ich nicht sagen.“

„… Das ist \emph{nicht} fair.“

„Ich denke, es ist außerordentlich fair,“ sagte Professor Quirrell und sie traten durch bronzene Türen ein.

\later

Es war kurz vor Mittag und Harry und Professor Quirrell saßen an Fuß- und Kopfende eines breiten, langen, flachen Tisches in einem kostspielig ausgestatteten Privatzimmer mit üppig gepolsterten Sofas und Stühlen entlang der Wände und weichen Vorhängen überall.

Sie würden im Restaurant Marys Platz essen, von dem Professor Quirrell gemeint hatte, es sei ihm als eines der besten in der Winkelgasse bekannt, besonders für -- er hatte bedeutungsvoll die Stimme gesenkt -- \emph{bestimmte Zwecke.}

Es war das beste Restaurant in dem Harry je gewesen war und es nagte wirklich an Harry, dass Professor Quirrell \emph{ihn} zu dem Essen einlud.

Der erste Teil ihrer Mission, einen Okklumentik-Lehrer zu finden, war ein Erfolg gewesen. Professor Quirrell hatte Griphook, mit einem bösen Lächeln, gebeten, den besten zu empfehlen, den er kannte und Geld spiele keine Rolle, da Dumbledore bezahlte und der Kobold hatte zurück gelächelt. Es mochte auch von Harrys Seite einiges an Lächeln gegeben haben.

Der zweite Teil des Plans war ein vollkommener Fehlschlag gewesen.

Harry durfte kein Geld aus seinem Verlies entnehmen, ohne die Anwesenheit von Schulleiter Dumbledore oder einem anderen offiziellen Repräsentanten der Schule und Professor Quirrell war der Schlüssel zum Verlies nicht anvertraut worden. Harrys Muggel-Eltern konnten es nicht genehmigen, weil sie Muggel waren und Muggel etwa den gleichen rechtlichen Status wie Kinder oder Kätzchen hatten: sie waren ganz süß, daher konnte man eingesperrt werden, wenn man sie öffentlich quälte, doch sie waren keine \emph{Leute.} Man hatte widerstrebend eine Verordnung erlassen, die Eltern von Muggelgeborenen in begrenztem Umfang als menschlich anzuerkennen, doch Harrys Adoptiveltern fielen nicht in jene rechtliche Kategorie.

Es schien, dass Harry in den Augen der Zauberwelt effektiv ein Waise war. Als solcher waren der Schulleiter von Hogwarts und seine Beauftragten \emph{innerhalb} des Schulsystems Harrys Vormunde bis er seinen Abschluss machte. Harry \emph{konnte} Atmen ohne Dumbledores Erlaubnis, aber nur solange er es nicht explizit verbat.

Harry hatte dann gefragt, ob er Griphook einfach \emph{beauftragen} könnte, seine Investitionen zu diversifizieren, anstatt sie nur als Gold in seinem Verlies aufzustapeln.

Griphook hatte ihn mit leerem Blick angestarrt und gefragt, was 'diversifizieren' bedeutete.

Banken, so schien es, tätigten keine Investitionen. Banken bewahrten die Goldmünzen in sicheren Verliesen für einen auf, gegen eine jährliche Gebühr.

Die Zauberwelt kannte das Konzept von Aktien nicht. Oder Eigenkapital. Oder Unternehmen. Geschäfte wurden von Familien mit den Mitteln aus ihren persönlichen Verliesen geführt.

Kredite wurden von reichen Personen gewährt, nicht von Banken. Obwohl Gringotts den Vertrag beglaubigen würde, gegen Gebühr und seine Erfüllung sicherstellte, gegen eine viel höhere Gebühr.

Gute reiche Leute ließen ihre Freunde Geld leihen und es irgendwann zurückzahlen. \emph{Böse} reiche Leute nahmen \emph{Zinsen.}*****

Es gab keinen Sekundärmarkt für Kredite.

Böse reiche Leute nahmen jährliche Zinssätze von mindestens 20\%.

Harry erhob sich, wandte sich ab und legte den Kopf gegen die Wand.

Harry hatte gefragt, ob er die Erlaubnis des Schulleiters brauche, um eine eigene Bank zu gründen.

Professor Quirrell hatte an diesem Punkt unterbrochen, gemeint es sei Zeit fürs Mittagessen und einen vor Wut rauchenden Harry umgehend durch die bronzenen Türen von Gringotts hinaus geführt, durch die Winkelgasse und in ein gutes Restaurant namens Marys Platz, wo ein Zimmer für sie reserviert worden war. Die Eigentümerin hatte schockiert ausgesehen als sie Professor Quirrell begleitet von Harry Potter erblickt hatte, hatte sie jedoch ohne weiteres zu ihrem Zimmer geführt.

Und Professor Quirrell hatte ganz geflissentlich verkündet, er würde die Rechnung übernehmen, der Ausdruck auf Harrys Gesicht schien ihm einiges Vergnügen zu bereiten.

„Nein,“ sagte Professor Quirrell der Bedienung, „wir brauchen die Karte nicht. Ich nehme die Spezialität des Tages mit einer Flasche Chianti und Mr. Potter wird mit der Diracawl-Suppe****** beginnen, gefolgt von einem Teller Roopo-Bällchen******* und zum Nachtisch Sirup-Pudding.“

Die Bedienung, gekleidet in Umhänge die, trotzdem sie kürzer als üblich waren, noch immer streng und formell wirkten, verbeugte sich respektvoll und entfernte sich, die Tür hinter sich schließend.

Professor Quirrell winkte mit einer Hand in Richtung der Tür und ein Bolzen schob sich zu. „Beachten Sie den Bolzen an der Innenseite. Dieser Raum, Mr. Potter, ist als Marys Zimmer bekannt. Er ist geschützt gegen alle Arten von magischer Ausspähung und ich meine \emph{alle;} Dumbledore selbst könnte nicht in Erfahrung bringen, was hier geschieht. Marys Zimmer wird von zwei Arten von Menschen genutzt. Diejenigen der ersten Sorte ergehen sich in verbotenen Liebeleien. Und jene der zweiten Sorte führen interessante Leben.“

„\emph{Wirklich,}“ sagte Harry.

Professor Quirrell nickte.

Harrys Lippen waren erwartungsvoll geöffnet. „Dann wäre es eine Verschwendung hier nur zu sitzen und zu Mittag zu essen, ohne etwas besonderes zu tun.“

Professor Quirrell grinste, holte seinen Zauberstab heraus und ließ ihn in Richtung der Tür zucken. „Natürlich,“ sagte er, „treffen Menschen, die interessante Leben führen, \emph{sorgfältigere} Vorkehrungen als die Müßiggänger. Ich habe unser Zimmer gerade versiegelt. Nichts dringt nun in diesen Raum ein oder aus ihm heraus - etwa durch den Spalt unter der Tür. Und…“

Professor Quirrell sprach daraufhin nicht weniger als vier verschiedene Zauber, von denen Harry keinen erkannte.

„Selbst dies ist nicht \emph{wirklich} ausreichend,“ sagte Professor Quirrell. „Würden wir etwas wahrhaftig wichtiges tun, wäre es notwendig, noch dreiundzwanzig weitere Prüfungen durchzuführen. Wenn, sagen wir, Rita Kimmkorn wüsste oder erräte, dass wir hierher kommen würden, könnte sie möglicherweise in diesem Raum sein, wenn sie den wahrhaftigen Unsichtbarkeitsumhang trägt. Oder sie könnte vielleicht ein Animagus sein, in winzig kleiner Form. Es gibt Tests, um solche abwegigen Möglichkeiten auszuschließen, doch sie alle durchzuführen, wäre mühsam. Wobei ich mich frage, ob ich sie trotzdem durchführen sollte, um Sie keine schlechten Angewohnheiten zu lehren?“ Und Professor Quirrell tippte sich mit einem Finger an die Wange, blickte nachdenklich drein.

„Schon in Ordnung,“ sagte Harry, „ich verstehe und werde daran denken.“ Obwohl er etwas enttäuscht war, dass sie nichts von wahrhaftiger Wichtigkeit tun würden.

„Nun gut,“ sagte Professor Quirrell. Er lehnte sich in seinem Stuhl zurück und lächelte breit. „Sie haben sich heute gut geschlagen, Mr. Potter. Die Grundidee kam von Ihnen, auch wenn Sie die Vollstreckung delegiert haben. Ich denke, wir werden hiernach nicht mehr allzu viel von Rita Kimmkorn hören. Lucius Malfoy wird von ihrem Versagen nicht erfreut sein. Wenn sie klug ist, flieht sie aus dem Land, sobald ihr klar wird, dass sie zum Narren gehalten wurde.“

Harry wurde flau im Magen. „Lucius stand hinter Rita Kimmkorn…?“

„Oh, das war Ihnen nicht klar?“ sagte Professor Quirrell.

Harry hatte nicht darüber nachgedacht, was mit Rita Kimmkorn im Nachhinein passieren würde.

Überhaupt nicht.

Nicht einmal ansatzweise.

Doch sie würde gefeuert, \emph{natürlich} würde sie gefeuert, nach allem was Harry wusste, mochte sie gerade Kinder in Hogwarts haben und nun war es noch schlimmer, viel schlimmer -

„Wird Lucius sie umbringen lassen?“ sagte Harry mit kaum hörbarer Stimme. Irgendwo in seinem Kopf schrie der Sprechende Hut auf ihn ein.

Professor Quirrell lächelte trocken. „Wenn Sie es noch nicht mit Reportern zu tun hatten, lassen Sie sich von mir sagen, dass die Welt jedes mal ein wenig besser wird, wenn einer von ihnen stirbt.“

Harry sprang mit einer krampfhaften Bewegung aus seinem Stuhl hoch, er musste Rita Kimmkorn finden und sie warnen, bevor es zu spät war -

„\emph{Setzen Sie sich,}“ sagte Professor Quirrell scharf. „\emph{Nein,} Lucius wird sie nicht töten. Doch Lucius macht jenen, die ihm schlechte Dienste leisten, das Leben \emph{extrem}schwer. Miss Kimmkorn wird fliehen und ihr Leben unter neuem Namen wiederaufnehmen. \emph{Setzen Sie sich,} Mr. Potter; es gibt zu diesem Zeitpunkt nichts, das Sie tun könnten und Sie haben eine Lektion zu lernen.“

Harry setzte sich, langsam. Es lag ein enttäuschter, verärgerter Ausdruck auf Professor Quirrells Gesicht, der mehr als die Worte dazu beitrug, dass er inne hielt.

„Es gibt Augenblicke,“ sagte Professor Quirrell mit schneidender Stimme, „da fürchte ich, dass Ihr brillanter Slytherin-Verstand einfach an Sie verschwendet ist. Sprechen Sie mir nach. Rita Kimmkorn war eine niederträchtige, abscheuliche Frau.“

„Rita Kimmkorn war eine niederträchtige, abscheuliche Frau,“ sagte Harry. Er fühlte sich nicht wohl dabei, es zu sagen, doch es schien keine andere Möglichkeit zu geben, überhaupt keine.

„Rita Kimmkorn versuchte, meinen Ruf zu zerstören, doch ich führte einen genialen Plan aus und zerstörte \emph{ihren} Ruf zuerst.“

„Rita Kimmkorn hat mich herausgefordert. Sie hat das Spiel verloren, ich habe gewonnen.“

„Rita Kimmkorn war meinen zukünftigen Plänen hinderlich. Ich hatte keine Wahl, als mich um sie zu kümmern, wenn jene Pläne Erfolg haben sollten.“

„Rita Kimmkorn war mein Feind.“

„Ich kann es im Leben unmöglich zu etwas bringen, wenn ich nicht willens bin, meine Feinde zu besiegen.“

„Ich habe heute einen meiner Feinde besiegt.“

„Ich bin ein guter Junge.“

„Ich habe eine spezielle Belohnung verdient.“

„Ah,“ sagte Professor Quirrell, der während der letzten paar Zeilen ein wohlwollendes Grinsen gezeigt hatte, „ich sehe, ich habe erfolgreich Ihre Aufmerksamkeit erregt.“

Das stimmte. Und wenn es Harry auch vorkam, als habe er sich hier zu etwas drängen lassen -- nein, das kam ihm nicht nur so vor, er \emph{war} gedrängt worden -- konnte er nicht leugnen, durch das Sagen dieser Dinge und Professor Quirrells Lächeln zu sehen, \emph{fühlte} er sich besser.

Professor Quirrell griff in seinen Umhang, eine langsame und bewusst bedeutungsvolle Geste und zum Vorschein kam…

… ein \emph{Buch.}

Es unterschied sich von jedem anderen Buch, das Harry je gesehen hatte, die Ecken und Kanten sichtbar verschlissen; der Ausdruck \emph{grob-behauen}drängte sich auf, als habe man es aus einer Bücher-Mine heraus gehackt.

„Was ist das?“ hauchte Harry.

„Ein Tagebuch,“ sagte Professor Quirrell.

„Wessen?“

„Das einer berühmten Person.“ Professor Quirrell lächelte breit.

"Okay…„

Professor Quirrells Gesichtsausdruck wurde ernster. „Mr. Potter, eine der Voraussetzungen, um ein mächtiger Zauberer zu werden, ist ein exzellentes Gedächtnis. Der Schlüssel zu einem Rätsel ist oftmals etwas, das man vor zwanzig Jahren in einer alten Schriftrolle las oder ein auffälliger Ring, den man am Finger eines Mannes sah, den man nur ein einziges mal traf. Ich erwähne dies um zu erklären, wie ich es schaffte mich an diesen Gegenstand und die daran befestigte Plakette zu erinnern, nachdem ich Ihnen um einiges später begegnet bin. Wissen Sie, Mr. Potter, im Laufe meines Lebens bin ich Zeuge einiger privater Sammlungen geworden, von Individuen, welche, vielleicht, nicht aller ihrer Besitztümer würdig sind -“

„Sie haben es \emph{gestohlen?}“ sagte Harry ungläubig.

„Das ist korrekt,“ sagte Professor Quirrell. „Tatsächlich vor relativ kurzer Zeit. Ich denke, Sie werden diesen bestimmten Gegenstand deutlich mehr zu schätzen wissen, als der niederträchtige kleine Mann, der es zu keinem anderen Zwecke innehatte, als seine ebenso niederträchtigen Freunde mit seiner Seltenheit zu beeindrucken.“

Harry stand jetzt einfach nur der Mund offen.

„Doch sollten Sie das Gefühl haben, meine Handlungen seien inkorrekt gewesen, Mr. Potter, so nehme ich an, dass Sie mein besonderes Geschenk nicht akzeptieren müssen. Obwohl ich mir natürlich nicht die Mühe machen werde, es \emph{zurück} zu stehlen. Was also soll es sein?“

Professor Quirrell warf das Buch von einer Hand in die andere, was Harry dazu veranlasste, mit einem Ausdruck des Unbehagens die Hand auszustrecken.

„Oh,“ sagte Professor Quirrell, „sorgen Sie sich nicht um etwas rüde Behandlung. Sie könnten dieses Tagebuch in eine Feuerstelle fallen lassen und es würde unversehrt wieder herauskommen. In jedem Fall erwarte ich Ihre Entscheidung.“

Professor Quirrell warf das Buch beiläufig hoch in die Luft und fing es grinsend wieder auf.

\emph{Nein,} sagten Gryffindor und Hufflepuff.

\emph{Ja,} sagte Ravenclaw. \emph{Welchen Teil des Wortes 'Buch' habt ihr zwei nicht} \emph{verstanden?}

\emph{Den Teil mit dem Diebstahl,} sagte Hufflepuff.

\emph{Ach, kommt schon,} sagte Ravenclaw, \emph{ihr könnt nicht ernsthaft verlangen, dass wir nein sagen und uns den Rest unseres Lebens fragen, was es war.}

\emph{Vom utilitaristischen Standpunkt aus gesehen, klingt es nach einem Nettogewinn,} sagte Slytherin. \emph{Stellt es euch wie eine wirtschaftliche Transaktion vor, die Handelsgewinne generiert, nur ohne den Teil mit dem Handel. Außerdem,} wir \emph{haben es nicht gestohlen und es hilft niemandem, wenn Professor Quirrell es behält.}

\emph{Er versucht dich in die Dunkelheit zu ziehen!} kreischte Gryffindor und Hufflepuff nickte entschieden.

\emph{Sei kein naiver kleiner Junge,} sagte Slytherin, \emph{er} \emph{will} \emph{dir beizubringen, ein} \emph{besserer} \emph{Slytherin zu sein.}

\emph{Ja,} sagte Ravenclaw. \emph{Wem immer das Buch ursprünglich gehörte, er war wahrscheinlich ein Todesser oder sowas. Es gehört zu uns.}

Harrys Mund öffnete sich, blieb dann so stehen, ein gequälter Ausdruck auf seinem Gesicht.

Professor Quirrell schien es ziemliches Vergnügen zu bereiten. Er hatte das Buch auf einer Ecke ausbalanciert, mit einem Finger, und hielt es aufrecht, während er eine kleine Melodie summte.

Da ertönte ein Klopfen an der Tür.

Das Buch verschwand wieder in Professor Quirrells Umhang und er erhob sich von seinem Stuhl. Professor Quirrell ging hinüber zu Tür -

- und taumelte, schlingerte plötzlich gegen die Wand.

„Alles in Ordnung,“ sagte Professor Quirrells Stimme, die plötzlich sehr viel schwächer als üblich klang. „Setzen Sie sich, Mr. Potter, es ist nur ein Schwächeanfall.******** Setzen Sie sich.“

Harrys Finger umklammerten seine Stuhlkante, unsicher was er tun sollte, was er tun \emph{konnte,} Harry konnte Professor Quirrell nicht einmal zu nahe kommen, wenn er nicht jenesGefühl des Unheils herausfordern wollte -

Dann richtete Professor Quirrell sich wieder auf, schien etwas schwer zu atmen und öffnete die Tür.

Die Bedienung trat ein, trug eine Platte mit Essen und als sie die Teller verteilte, ging Professor Quirrell langsam zurück zum Tisch.

Doch bis die Bedienung sich unter Verbeugungen zurückgezogen hatte, saß Professor Quirrell aufrecht und lächelte wieder.

Trotzdem, die kurze Episode von was-immer-es-war hatte für Harry die Entscheidung gebracht. Er konnte nicht nein sagen, nicht nachdem Professor Quirrell sich solche Mühe gemacht hatte.

„Ja,“ sagte Harry.

Professor Quirrell hob warnend einen Finger, nahm dann seinen Zauberstab heraus, verschloss die Tür erneut und wiederholte drei der vorherigen Zauber.

Dann nahm Professor Quirrell das Buch wieder aus seinem Umhang und warf es Harry zu, dem es fast in seine Suppe fiel.

Harry schoss Professor Quirrell einen Blick hilfloser Empörung zu. Man \emph{ging} einfach nicht so mit Büchern um, ob nun verzaubert oder nicht.

Harry öffnete das Buch mit tief verwurzelter, instinktiver Sorgfalt. Die Seiten schienen zu dick, mit einer Textur die weder Muggel-Papier noch Zauberer-Pergament entsprach. Und der Inhalt war…

… leer?

„Sollte ich sehen -“

„Suchen Sie näher zum Anfang hin,“ sagte Professor Quirrell und Harry (erneut mit jener hilflosen, tiefsitzenden Sorgfalt) blätterte einige Seiten zurück.

Die Buchstaben waren offensichtlich von Hand geschrieben und sehr schwer zu lesen, doch Harry glaubte, die Worte mochten auf Latein sein.

„Was \emph{ist} das?“ sagte Harry.

„Das,“ sagte Professor Quirrell, „sind die Aufzeichnungen der magischen Forschungen eines Muggelgeborenen, der niemals nach Hogwarts kam. Er wies seinen Brief zurück und führte seine eigenen kleinen Nachforschungen durch, die nie sehr weit gediehen ohne einen Zauberstab. Aus der Beschreibung auf der Plakette schließe ich, dass sein Name für Sie eine viel größere Bedeutung birgt als für mich. Dies, Harry Potter, ist das Tagebuch von Roger Bacon.“

Harry fiel beinahe in Ohnmacht.

Und an einer der Wände, wo Professor Quirrell getaumelt war, glitzerten die zerquetschten Überreste eines wunderschönen blauen Käfers.

* engl.: \emph{wards;} hierfür konnte ich auf Anhieb keine Entsprechung in den Original-Romanen ausmachen, der Begriff bezeichnet wohl zusammenfassend alle magischen Schutzmaßnahmen, mit denen Hogwarts versehen ist, um das Schloss und Schulgelände selbst und die Personen darin zu schützen (darunter z.B. auch Zauber gegen das Apparieren und Portschlüssel).

** engl.: \emph{The Massacre of Albania in the Fifteenth Century}

*** engl.: \emph{Bavarian seer;} ich konnte nicht ermitteln, ob Professor Quirrell von einer bestimmten Person spricht oder nur irgendeinen bayerischen Propheten meint, der die von Harry erwähnte Prophezeiung gemacht hätte. Eine Internet-Suche ergab Hinweise auf die angeblichen bayerischen Propheten/Weissager/Hellseher \emph{Mühlhiasl} alias \emph{Matthäus Lang} und \emph{Alois Irlmaier,} sowie zum \emph{Diskordianismus} und den \emph{Illuminaten,} allerdings nichts eindeutiges.

**** engl.: \emph{ward-breaker;} auch hierfür scheint es keine Entsprechung in den Original-Romanen zu geben, die wörtliche Übersetzung impliziert jemanden, der sich auf das Knacken von Schutz-Zaubern spezialisiert hat, wie es wohl für einen Einbrecher in der magischen Welt nützlich wäre und klingt ähnlich einem \emph{Fluchbrecher}(engl.: \emph{Curse Breaker}) von Gringotts, obwohl es unwahrscheinlich ist, dass diese sich für kriminelle Machenschaften engagieren lassen, doch es könnte sich um ihr kriminelles Pendant handeln. Mangels eines besseren Begriffes, behalte ich daher vorerst die Bezeichnung \emph{Fluchbrecher} bei.

***** Der hier verwendete englische Begriff \emph{interest} könnte mehrdeutig sein. Er bedeutet einerseits \emph{Zinsen,} kann andererseits allerdings auch \emph{Interesse} oder \emph{Aufmerksamkeit} bedeuten, was darauf hinweisen mag, dass \emph{böse} reiche Leute mit der Auswahl ihrer Kredit\emph{nehmer} noch andere (eventuell finsterere)Absichten verfolgen als nur die Vermehrung ihres Geldes.

****** Wenn das eine Anspielung ist, konnte ich nicht herausfinden worauf. Laut Newt Scamanders Werk \emph{Phantastische Tierwesen und wo sie zu finden sind,} gibt es allerdings eine magische Tierart namens \emph{Diricawl,} in Muggel-Kreisen auch als \emph{Dodo} bekannt (entgegen dem, was Muggel glauben, sind sie nicht ausgestorben). Es besteht die Möglichkeit, dass der Autor sich mit \emph{Diracawl} hier einfach verschrieben hat.

******* engl.: \emph{Roopo balls,} ein beliebtes Nahrungsmittel der Centauri aus dem \emph{Babylon 5}-Universum, offenbar identisch mit irdischen \emph{Schwedischen Fleischbällchen} (auch\emph{Köttbullar}genannt).

******** Auch hier liegt ein mehrdeutiger Begriff vor: Ein \emph{dizzy spell} bezeichnet im Englischen für gewöhnlich einen Schwindel- oder Schwächeanfall, in einem magischen Umfeld könnte \emph{spell} aber auch Zauber bedeuten, wenn auch unklar ist, wer hier einen Schwindel-Zauber wirken sollte oder aus welchem Grund.

