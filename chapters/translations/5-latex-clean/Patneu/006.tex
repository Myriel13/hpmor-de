

\hypertarget{der-planungs-fehlschluss}{% \section{6. Der Planungs-Fehlschluss}\label{der-planungs-fehlschluss}}

\textbf{Kapitel 6: Der Planungs-Fehlschluss}\\

\hfill\break Blah blah Haftungsausschluss blah blah Rowling blah blah Eigentumsrecht.\\ Anm. d. Autors: Der „Nachspiel“-Abschnitt dieses Kapitels ist Teil der Geschichte, \emph{kein} Extra.

--------------------------------------------------------------------------------------------------------------------------------------------

\hfill\break

\emph{\emph{Sie denken, Ihr Tag war surreal? Versuchen Sie's mit meinem.}}

\hfill\break

--------------------------------------------------------------------------------------------------------------------------------------------

\hfill\break \emph{\emph{Manche}} \emph{Kinder hätten bis} \emph{\emph{nach}} \emph{ihrem ersten Ausflug in die Winkelgasse gewartet.}

\emph{„Säckchen voll Element 79,“ sagte Harry und zog seine Hand, leer, aus dem Eselsfell-Beutel heraus.}

\emph{Die meisten Kinder hätten wenigstens gewartet, bis sie überhaupt ihre} \emph{\emph{Zauberstäbe}} \emph{bekommen.}

\emph{„Säckchen voll} \emph{\emph{okane,}“ sagte Harry. Das schwere Säckchen voll Gold tauchte in seiner Hand auf.}

\emph{Harry zog das Säckchen heraus und ließ es dann wieder in den Eselsfell-Beutel fallen. Er zog seine Han heraus, steckte sie wieder hinein und sagte, „Säckchen voller Symbole für wirtschaftlichen Austausch.“ Dieses mal kam seine Hand leer wieder heraus.}

\emph{„Gib mir das Säckchen zurück, das ich gerade hineingetan habe.“ Heraus kam ein weiteres mal das Säckchen voll Gold.}

\emph{Harry James Potter-Evans-Verres hatte zumindest einen magischen Gegenstand in die Finger bekommen. Warum also warten?}

\emph{„Professor McGonagall,“ sagte Harry zu der verwirrten Hexe, die neben ihm her spazierte, „können Sie mir zwei Worte nennen, eines für Gold und eines für etwas anderes, das kein Geld ist, in einer Sprache, die ich nicht kenne? Aber sagen Sie mir nicht, welches welches ist."}

\emph{„\emph{Ahava}} \emph{and} \emph{\emph{zahav,}“ sagte Professor McGonagall. „Das ist hebräisch und das andere Wort bedeutet Liebe."}

\emph{„Danke, Professor. Säckchen voll} \emph{\emph{ahava.}“ Leer.}

\emph{„Säckchen voll} \emph{\emph{zahav.}“ Und es tauchte in seiner Hand auf.}

\emph{„Zahav ist Gold?“ fragte Harry und Professor McGonagall nickte.}

\emph{Harry dachte über seine gesammelten experimentellen Daten nach. Es war nur ein sehr grober und vorläufiger Versuch, aber es war genug, um zumindest eine Schlussfolgerung zu stützen:}

\emph{"\emph{Aaaaaaarrrgh, das macht überhaupt keinen Sinn!}"}

\emph{Die Hexe neben ihm hob vornehm eine Augenbraue. „Probleme, Mr. Potter?"}

\emph{„Ich habe gerade jede Hypothese widerlegt, die ich hatte! Wie kann es wissen, dass 'Säckchen mit 115 Galleonen' in Ordnung ist, aber 'Säckchen mit 90 plus 25 Galleonen' nicht?“ Es kann} \emph{\emph{zählen}, aber nicht} \emph{\emph{addieren}? Es kann Nomen verstehen, aber keine Ausdrücke, die das selbe bedeuten? Die Person die es hergestellt hat, sprach wahrscheinlich kein Japanisch und} \emph{\emph{ich}} \emph{spreche kein Hebräisch, also benutzt es nicht} \emph{\emph{ihr}} \emph{Wissen und es benutzt nicht} \emph{\emph{mein}} \emph{Wissen -“ Harry schwenkte hilflos eine Hand. „Die Regeln scheinen} \emph{\emph{irgendwie}} \emph{konsistent, aber sie} \emph{\emph{bedeuten}} \emph{nichts! Ich werde lieber nicht mal danach fragen, wie ein} \emph{\emph{Beutel}} \emph{über Stimmerkennung und ein Verständnis natürlicher Sprache verfügen kann, wenn die besten Programmierer für Künstliche Intelligenz, die besten Supercomputer nach 35 Jahren harten Arbeit nicht dazu bringen können,“ Harry schnappte nach Luft, „aber} \emph{\emph{was passiert}} \emph{da?"}

\emph{„Magie,“ sagte Professor McGonagall.}

\emph{"Das ist nur ein} \emph{\emph{Wort!}} \emph{Auch nachdem Sie mir das sagen, kann ich keine neuen Vorhersagen machen! Es ist genau so, als würden Sie 'Phlogiston' oder 'unverzichtbarer Elan' oder 'Erscheinung' oder 'Komplexität' sagen!"}

\emph{Die Hexe im schwarzen Umhang lachte laut auf. „Aber es} \emph{\emph{ist}} \emph{Magie, Mr. Potter."}

\emph{Harry sackte etwas zusammen. „Bei allem Respekt, Professor McGonagall, ich bin mir nicht sicher, ob sie wissen, was ich hier zu tun versuche."}

\emph{"Bei allem Respekt, Mr. Potter, ich bin mir ganz sicher, das tue ich nicht. Es sei denn - das ist natürlich nur eine Vermutung - Sie versuchen die Weltherrschaft an sich zu reißen?"}

\emph{"Nein! Ich meine ja - nur,} \emph{\emph{nein!}"}

\emph{"Ich denke, es sollte mich vielleicht beunruhigen, dass sie scheinbar Probleme haben, diese Frage zu beantworten.„}

\emph{Harry dachte verdrießlich an die Dartmouth-Konferenz über Künstliche Intelligenz im Jahr 1956. Es war die erste Konferenz überhaupt zu diesem Themengebiet gewesen, diejenige welche den Ausdruck „Künstliche Intelligenz“ geprägt hatte. Sie hatten Schlüsselprobleme identifiziert, etwa, wie man Computer dazu bringen könnte, Sprache zu verwenden, zu lernen und sich selbst zu verbessern. Sie hatten behauptet, mit absoluter Ernsthaftigkeit, signifikante Fortschritte könnten bei diesen Problemen gemacht werden, wenn zehn Wissenschaftler gemeinsam zwei Monate daran arbeiteten.}

\emph{\emph{Nein. Kopf hoch. Du fängst gerade erst}} \emph{an\emph{,}\emph{am Problem der Enthüllung aller Geheimnisse der Magie zu arbeiten. Du}} \emph{weißt} \emph{\emph{noch gar nicht wirklich, ob es zu schwierig sein wird, um es in zwei Monaten zu schaffen.}}

\emph{„Und Sie haben} \emph{\emph{wirklich}} \emph{nicht von anderen Zauberern gehört, die diese Art Fragen stellen oder auf diese Art wissenschaftlich Experimentieren?“ fragte Harry noch einmal. Es schien ihm einfach so} \emph{\emph{offensichtlich.}"}

\emph{Aber dann wieder waren mehr als zweihundert Jahre} \emph{\emph{nach}} \emph{der Erfindung der wissenschaftlichen Methode vergangen, bevor irgendein Muggel-Wissenschaftler daran gedacht hatte, systematisch zu untersuchen, welche Sätze ein} \emph{\emph{menschlicher Vierjähriger}} \emph{verstehen oder nicht verstehen konnte. Die Entwicklungspsychologie der Linguistik hätte, prinzipiell, bereits im achtzehnten Jahrhundert entdeckt werden können, aber niemand hatte bis zum zwanzigsten auch nur darüber nachgedacht. Also konnte man der wesentlich kleineren Zauberwelt nicht wirklich vorwerfen, den Aufrufezauber nicht untersucht zu haben.}

\emph{Professor McGonagall schürzte die Lippen und zuckte dann mit den Schultern. „Ich bin immer noch nicht sicher was Sie mit 'wissenschaftlichem Experimentieren' meinen, Mr. Potter. Wie ich schon sagte, ich habe viele muggelgeborene Schüler gesehen, die versucht haben, Wissenschaft in Hogwarts zum Funktionieren zu bringen und jedes Jahr erfinden Leute neue Zauber und Tränke.„}

\emph{Harry schüttelte den Kopf. „Technologie ist überhaupt nicht das gleiche wie Wissenschaft. Und viele verschiedene Wege auszuprobieren, um etwas zu tun, ist nicht das gleich, wie zu experimentieren, um die Regeln herauszufinden.“ Es hatte viele Leute gegeben, die versucht hatten, Flugmaschinen zu erfinden, indem sie viele Ding-mit-Flügeln ausprobiert hatten, aber nur die Gebrüder Wright hatten einen Windkanal gebaut, um den Auftrieb zu messen... „Ähm, wie viele von Muggeln aufgezogene} \emph{Kinder} \emph{\emph{haben}} \emph{Sie denn jedes Jahr in Hogwarts?"}

\emph{"Vielleicht zehn oder so?"}

\emph{Harry setzte einen Schritt lang aus und fiel fast über seine eigenen Füße. „\emph{Zehn?}"}

\emph{Die Muggelwelt hatte eine Bevölkerung von sechs Milliarden und steigend. Wenn man einer in einer Million war, gab es sieben von einem in London und nochmal tausend in China. Es war unvermeidlich, dass die Muggelbevölkerung} \emph{\emph{ein paar}} \emph{Elfjährige hervorbrachte, die Infinitesimalrechnung berherrschten -- Harry wusste, dass er nicht der einzige war. Er hatte andere Wunderkinder bei Mathematikwettbewerben getroffen. Tatsächlich war er vollkommen fertig gemacht worden von Wettbewerbern, die wahrscheinlich buchstäblich} \emph{\emph{den ganzen Tag}} \emph{damit verbrachten, das Lösen mathematischer Probleme zu üben und die} \emph{\emph{niemals}} \emph{ein Science-Fiction-Buch gelesen hatten und die schon vor der} \emph{\emph{Pubertät}\emph{komplett}} \emph{ausbrennen würden und in ihren zukünftigen Leben} \emph{\emph{niemals irgendetwas}} \emph{erreichen würden, weil sie nur} \emph{\emph{bekannte}} \emph{Techniken trainiert hatten, anstatt zu lernen,} \emph{\emph{kreativ}} \emph{zu denken. (Harry war ein etwas schlechter Verlierer.)}

\emph{Aber... in der Zauberwelt...}

\emph{Zehn von Muggeln aufgezogene Kinder pro Jahr, die alle ihre Muggel-Ausbildung im Alter von elf beendet hatten? Und Professor McGonagall könnte voreingenommen sein, aber sie hatte behauptet, dass Hogwarts die größte und wichtigste Zauberschule der Welt wäre... und sie unterrichtete nur bis zum Alter von siebzehn.}

\emph{Professor McGonagall wusste unzweifelhaft jedes kleine Detail darüber, wie man sich in eine Katze verwandelte. Aber sie schien buchstäblich noch nie von der wissenschaftlichen Methode} \emph{\emph{gehört}} \emph{zu haben. Für sie war es nur Muggel-Magie. Und sie schien nicht einmal} \emph{\emph{neugierig}} \emph{zu sein, was für Geheimnisse sich hinter dem Verständnis natürlicher Sprache des Aufrufezaubers verbergen mochten.}

\emph{Das ließ nur zwei Möglichkeiten offen.}

\emph{Möglichkeit eins: Magie war so unglaublich undurchsichtig, verschlungen und unzugänglich, dass, obwohl Zauberer und Hexen ihr bestes gegeben hatten, zu verstehen, sie wenig oder keinen Fortschritt gemacht und schließlich aufgegeben hatten; und Harry würde es nicht besser ergehen.}

\emph{\emph{Oder...}}

\emph{Harry ließ entschlossen seine Fingergelenke knacken, aber sie machten nur ein leises klickendes Geräusch, anstatt unheilverkündend von den Wänden der Winkelgasse wiederzuhallen.}

\emph{Möglichkeit zwei: Er würde die Weltherrschaft übernehmen.}

\emph{Schlussendlich. Vielleicht noch nicht sofort.}

\emph{Solche Sachen} \emph{\emph{dauerten}} \emph{manchmal länger als zwei Monate. Die Muggel-Wissenschaft hatte es nicht in der ersten Woche nach Galileo auf den Mond geschafft.}

\emph{Aber Harry konnte trotzdem das immense Lächeln nicht unterdrücken, das seine Wangen so sehr dehnte, dass es schon zu schmerzen begann.}

\emph{Harry hatte sich immer davor gefürchtet, als eines dieser Wunderkinder zu enden, die es in ihrem Leben niemals zu etwas brachten und den Rest ihres Lebens damit verbrachten, damit anzugeben, wie weit voraus sie im Alter von zehn gewesen waren. Aber dann wieder brachten es die meisten erwachsenen Genies auch nie zu irgendetwas. Es kamen wahrscheinlich tausend Menschen, die so intelligent waren, wie Einstein, auf jeden tatsächlichen Einstein der Geschichte. Weil diese anderen Genies niemals das hatten, was man absolut brauchte, um zu wahrer Größe zu gelangen. Sie fanden niemals ein bedeutendes Problem.}

\emph{\emph{Du gehörst jetzt mir,}} \emph{sprach Harry in Gedanken zu den Wänden der Winkelgasse, all den Geschäften und Gegenständen, all den Ladenbesitzern und Kunden und all den Ländern und Leuten des Zauberer-Britanniens und der ganzen größeren Zauberwelt und dem ganzen größeren Universum, von dem die Muggel-Wissenschaftler so viel weniger verstanden, als sie glaubten.} \emph{\emph{Ich, Harry James Potter-Evans-Verres, nehme hiermit dieses Gebiet im Namen der Wissenschaft in Besitz.}}

\emph{Blitz und Donner misslang es vollkommen in dem wolkenlosen Himmel zu zucken und zu krachen.}

\emph{„Worüber lächeln Sie so?“ erkundigte sich Professor McGonagall misstrauisch und erschöpft.}

\emph{„Ich frage mich, ob es einen Zauber gibt, um jedes mal im Hintergrund Blitze zucken zu lassen, wenn ich einen verhängnisvollen Entschluss fasse,“ erklärte Harry. Er prägte sich die exakten Worte seines verhängnisvollen Entschlusses sorgältig ein, damit zukünftige Geschichtsbücher sie richtig wiedergeben würden.}

\emph{„Ich habe das deutliche Gefühl, dass ich irgendetwas deswegen} \emph{unternehmen sollte,“ seufzte Professor McGonagall.}

\emph{„Ignorieren Sie's, das geht vorbei. Uuh, glänzt das schön!“ Harry schob seine Gedanken über die Welteroberung für den Moment zur Seite und sprang hinüber zu einem Laden mit einem offenen Schaufenster und Professor McGonagall folgte ihm.}

--------------------------------------------------------------------------------------------------------------------------------------------

\hfill\break Harry hatte jetzt seine neuen Zaubertrank-Zutaten und seinen Kessel gekauft und, oh, noch ein paar andere Sachen. Gegenstände, die schienen, als würden sie sich gut in Harrys alles enthaltendem Sack (alias Super-Eselsfell-Beutel QX31 mit Unaufspürbarem Ausdehnungszauber, Aufrufezauber und sich weitender Öffnung) machen. Kluge, vernünftige Einkäufe.

Harry verstand beim besten Willen nicht, warum Professor McGonagall so \emph{misstrauisch} aussah.\\ In diesem Augenblick, war Harry in einem Laden, der teuer genug war, um seine Waren in der sich dahinschlängelnden Hauptstraße der Winkelgasse zu präsentieren. Der Laden hatte eine offene Front mit auf abgeschrägten Holzreihen ausgestellten Artikeln, nur geschützt von einem schwachen grauen Leuchten und einem jung-aussehenden Verkäufer-Mädchen in einer sehr gekürzten Version eines Hexenumhangs, der ihre Knie und Ellbogen entblößte.

Harry untersuchte das Zauberer-Äquivalent eines Erste-Hilfe-Kastens, das Notfall-Heiler-Pack Plus. Es gab zwei sich selbst verengende Stauschläuche. Eine Spritze mit etwas, das aussah wie flüssiges Feuer, welches die Blutzirkulation in einem behandelten Bereich drastisch verlangsamen und dabei die Sauerstoffsättigung des Blutes für bis zu drei Minuten aufrechterhalten sollte, wenn man ein Gift davon abhalten musste, sich im Körper auszubreiten. Weißer Stoff, der über ein Körperteil gelegt werden konnte, um Schmerz für eine gewisse Zeit zu betäuben. Zustätzlich noch einige andere Gegenstände bei deren Verständnis Harry vollkommen versagte, wie die „Behandlung für Dementor-Exposition“, die aussah und roch wie ordinäre Schokolade. Oder der „Bafflesnaffle-Konter"*, der aussah, wie ein kleines zitterndes Ei und einen Beipackzettel aufwies, der zeigte, wie man ihn in jemandes Nasenloch stopfen sollte.

„Eine klare Kaufempfehlung für fünf Galleonen, meinen Sie nicht auch?“ sagte Harry zu Professor McGonagall und die herbeischwebende Teenager-Verkäuferin nickte eifrig.

Harry hatte erwartet, dass Professor McGonagall eine lobende Bemerkung über seine Umsicht und gute Vorbereitung machen würde.\\ Was er stattdessen bekam, was nur als der Böse Blick zu beschreiben.

„Und \emph{warum} genau,“ sagte Professor McGonagall äußerst skeptisch, „erwarten Sie, eine Heil-Ausrüstung zu \emph{brauchen,} junger Mann?“ (Nach dem unglücklichen Zwinschenfall im Zaubertrank-Laden versuchte Professor McGonagall zu vermeiden, „Mr. Potter“ zu sagen, wenn jemand anders in der Nähe war.)\\ Harrys Mund ging auf und zu. „Ich \emph{erwarte} nicht, sie zu brauchen! Sie ist nur für den Fall!"

"Nur für \emph{welchen} Fall?"

Harrys Augen weiteten sich. „Sie denken, ich \emph{plane} etwas gefährliches zu unternehmen und \emph{deshalb} will ich eine medizinische Ausrüstung?"

Ein Blick voll grimmigem Verdacht und ironischem Unglauben war die Antwort.

„Grundgütiger!“ sagte Harry. (Das war ein Ausdruck, den er von dem verrückten Wissenschaftler Doc Brown aus \emph{Zurück in die Zukunft} gelernt hatte.) Haben Sie das auch gedacht, als ich den Zaubertrank für federleichtes Fallen, das Dianthuskraut** und die Flasche mit Nahrungs- und Wasserpillen gekauft habe?"

"Ja."

Harry schüttelte verwundert den Kopf. „Was für einen Plan glauben Sie eigentlich genau, habe ich hier \emph{am Laufen?}"

„Ich weiß es nicht,“ sagte Professor McGonagall finster, „aber er endet entweder damit, dass sie eine Tonne Silber nach Gringotts bringen oder mit der Weltherrschaft."

"Weltherrschaft ist so ein hässliches Wort. Ich nenne es lieber Weltoptimierung."

Dieser urkomische Witz konnte die Hexe nicht beruhigen, die ihm den Blick der Verdammnis zuwarf.

„Wow,“ sagte Harry, als er erkannte, dass sie es ernst meinte. „Sie denken das wirklich. Sie denken wirklich, ich plane etwas gefährliches zu tun."

"Ja."

"Als ob das der einzige Grund wäre, warum jemals jemand eine Erste-Hilfe-Ausrüstung kaufen würde? Verstehen Sie das nicht falsch, Professor McGonagall, aber \emph{mit welcher Art von verrückten Kindern haben Sie es normalerweise zu tun?}"

„Gryffindors,“ spuckte Professor McGonagall aus, das Wort mit einer Fracht aus Bitterkeit und Verzweiflung einhergehend, die wie ein ewiger Fluch allen jugendlichen Enthusiasmus und Ausgelassenheit fiel.

„Stellvertretende Schulleiterin Professor Minerva McGonagall,“ sagte Harry und legte seine Hände streng an die Hüften. „Ich werde nicht in Gryffindor sein, -„

An diesem Punkt warf die Stellvertretende Schulleiterin etwas darüber ein, dass wenn doch, sie herausfinden würde, wie man einen Hut umbringe; welch seltsame Bemerkung Harry ohne Kommentar durchgehen ließ, obwohl das Verkäufer-Mädchen einen plötzlichen Hustenanfall zu eleiden schien.

“- sondern in Ravenclaw. Und wenn Sie wirklich denken, dass ich etwas gefährliches plane, dann, um ehrlich zu sein, verstehen Sie mich nicht \emph{im Geringsten.} Ich \emph{mag} die Gefahr nicht, sie ist \emph{beängstigend.} Ich bin \emph{besonnen.} Ich bin \emph{vorsichtig.} Ich bereite mich auf \emph{unvorhergesehene Eventualtitäten} vor. Wie meine Eltern mir vorgesungen haben: \emph{Be prepared! That's the Boy Scout's marching song! Be prepared! As through life you march along! Don't be nervous, don't be flustered, don't be scared - be prepared!"}

\emph{(Harrys Eltern hatten ihm tatsächlich immer nur diese} \emph{\emph{bestimmten}} \emph{Zeilen des Tom Lehrer-Songs vorgesungen und Harry weilte in seliger Unwissenheit über den Rest.)***}

\emph{Professor McGonagalls Haltung hatte sich etwas gelockert -- wenn auch größtenteils als Harry gesagt hatte, dass er nach Ravenclaw wolle. „Welche Art von} \emph{\emph{Eventualität}} \emph{stellen Sie sich vor, auf die Sie diese Ausrüstung vorbereiten würde,} \emph{\emph{junger Mann?}"}

\emph{"Eine meiner Klassenkameradinnen wird von einem schrecklichen Monster gebissen und während ich verzweifelt in meinem Eselsfell-Beutel nach etwas suche, was ihr helfen könnte, sieht sie mich traurig an und sagt mit ihrem letzten Atemzug,} \emph{\emph{'Warum warst du nicht vorbereitet?'}} \emph{Und dann stirbt sie und ich weiß, während sich ihre Augen schließen, dass sie mir niemals vergeben wird -"}

\emph{Harry hörte das Verkäufer-Mädchen keuchen und er sah zu ihr auf, als sie} \emph{ihn mit schmal zusammengepressten Lippen anstarrte. Dann wirbelte die junge Frau herum und floh in die tieferen Eingeweide des Laden.}

\emph{\emph{Was...?}}

\emph{Professor McGonagall langte nach unten, nahm Harrys an der Hand, sanft aber bestimmt und zog ihn weg von der Hauptstraße der Winkelgasse, führt ihn in einen schmalen Durchgang zwischen zwei Geschäften, der mit schmutzigen Ziegeln gepflastert war und in einer Sackgasse an einer Mauer aus festem schwarzem Schmutz endete.}

\emph{Die große Hexe richtete ihren Zauberstab auf die Hauptstraße, sprach} \emph{\emph{"Quietus„}} \emph{und ein Barriere aus Stille senkte sich um sie, die allen Straßenlärm abschirmte.}

\emph{\emph{Was habe ich falsch gemacht...}}

\emph{Professor McGonagall drehte sich zu Harry um. Sie hatte kein volles erwachsenes} \emph{Missetat-Gesicht} \emph{aufgesetzt, aber ihr Gesichtsausdruck war ebenmäßig, kontrolliert. „Sie müssen bedenken, Mr. Potter,“ sagte sie, „dass in diesem Land vor noch nicht einmal zehn Jahren ein Krieg herrschte. Jeder hat jemanden verloren und von Freunden zu sprechen, die in den eigenen Armen sterben - sollte man nicht leichtfertig tun."}

\emph{„Ich - ich wollte nicht -“ Die Schlussfolgerung schlug wie eine Bombe in Harrys außerordentlich lebhafte Vorstellungskraft ein. Er hatte davon gesprochen, wie jemand seinen letzten Atemzug tat - und das Verkäufer-Mädchen war weggerannt - und der Krieg endete vor zehn Jahren, also wäre dieses Mädchen höchstens acht oder neun Jahre alt gewesen als, als, „Es tut mir leid, ich wollte nicht...“ Harrys Kehle schnürte sich zu und er drehte sich um, um vor dem Blick der älteren Hexe zu fliehen, aber eine Wand aus Dreck war ihm im Weg und er hatte seinen Zauberstab noch nicht. „Es tut mir leid, es tut mir leid, es tut mir} \emph{\emph{leid!}"}

\emph{Von hinter ihm war ein schwerer Seufzer zu vernehmen. „Das weiß ich, Mr. Potter.„}

\emph{Harry wagte es, einen Blick hinter sich zu riskieren. Professor McGonagall sah jetzt nur noch traurig aus. „Es tut mir leid,“ sagte Harry noch einmal und fühlte sich elend. „Ist Ihnen so etwas -“ und dann schloss Harry seine Lippen und schlug sich zusätzlich die Hand vor den Mund.}

\emph{Das Gesicht der älteren Hexe wurde noch ein wenig trauriger. „Sie müssen lernen, zu denken bevor Sie sprechen, Mr. Potter, oder Sie werden ohne viele Freunde durch's Leben gehen. Das ist das Schicksal vieler} \emph{Ravenclaws gewesen und ich hoffe, es wird nicht ihres sein."}

\emph{Harry wollte einfach nur wegrennen. Er wollte einen Zauberstab herausholen und die ganze Sache aus Professor McGonagalls Erinnerung löschen, wieder mit ihr draußen vor dem Laden sein,} \emph{\emph{machen, dass es nie passiert war -}}

\emph{"Aber um Ihre Frage zu beantworten, Mr. Potter, nein, nichts} \emph{\emph{dergleichen}} \emph{ist mir jemals passiert. Sicherlich habe ich Freunde ihren letzten Atemzug tun sehen, ein oder sieben mal. Aber nicht einer von ihnen verfluchte mich, als er starb, und ich dachte niemals, dass sie mir nicht vergeben würden. Warum würden Sie so etwas} \emph{\emph{sagen,}} \emph{Mr. Potter? Warum würden Sie es überhaupt} \emph{\emph{denken?}"}

\emph{„Ich, ich, ich,“ Harry schluckte. „Es ist einfach so, dass ich immer versuche, mir das schlimmste vorzustellen, was passieren könnte“ und vielleicht hatte er auch ein bisschen herumgescherzt, aber er hätte sich lieber die Zunge abgebissen, als das jetzt zu sagen.}

\emph{„Was?“ sagte Professor McGonagall. Aber} \emph{\emph{warum?}"}

\emph{"Damit ich verhindern kann, dass es passiert!"}

\emph{„Mr. Potter...“ die Simmte der älteren Hexe versagte. Dann seufzte sie und kniete sich neben ihm hin. „Mr. Potter,“ sagte sie, jetzt sanft, „es ist nicht Ihre Verantwortung, für die Sicherheit der Schüler in Hogwarts zu sorgen. Es ist meine. Ich werde nicht zulassen, dass Ihnen oder irgendjemand anderem etwas schlimmes passiert. Hogwarts ist der sicherste Ort für magische Kinder in der gesamten Zauberwelt und Madam Pomfrey hat eine voll ausgestattete Krankenstation. Sie werden überhaupt keine Heil-Ausrüstung brauchen, schon gar keine im Wert von fünf Galleonen."}

\emph{„Aber das} \emph{\emph{tue}} \emph{ich!“ platzte Harry heraus. „\emph{Nirgends}} \emph{ist es vollkommen sicher! Und was, wenn meine Eltern einen Herzanfall oder einen Unfall haben, wenn ich an Weihnachten nach Hause komme - Madam Pomfrey wird nicht dort sein, ich brauche selbst eine Heil-Ausrüstung -"}

\emph{„\emph{Was}} \emph{in Merlins Namen...“ sagte Professor McGonagall. Sie stand auf und sah auf Harry mit einem Ausdruck irgendwo zwischen Verärgerung und Sorge hinab. „Es gibt keinen Grund über solch schreckliche Dinge nachzudenken, Mr. Potter!"}

\emph{Harrys Gesichtsausdruck wurde bitter, als er das hörte. „\emph{Naürlich}} \emph{gibt es den! Wenn man nicht nachdenkt, wird man nicht nur selbst verletzt, man verletzt am Ende auch andere!"}

\emph{Professor McGonagall öffnete den Mund und schloss ihn wieder. Die Hexe rieb sich den Nasenrücken und sah nachdenklich aus. „Mr. Potter... wenn ich anbieten würde, Ihnen eine Weile zuzuhören... gibt es irgendetwas, was Sie mir sagen wollen?"}

\emph{"Worüber?"}

\emph{"Darüber, warum Sie überzeugt sind, immer auf der Hut vor schrecklichen Dingen sein zu müssen, die Ihnen zustoßen könnten.„}

\emph{Harry starrte sie verwirrt an. Das war ein selbsterklärendes Axiom. „Nun...“ sagte Harry langsam. Er versuchte sein Gedanken zu ordnen. Wie} \emph{\emph{konnte}} \emph{er sich einer Professoren-Hexe erklären, wenn sie nicht einmal die Grundlagen beherrschte? „Muggel-Forscher haben herausgefunden, dass Menschen immer sehr optimistisch sind, verglichen mit der Realität. Wie wenn sie sagen, etwas würde zwei Tage dauern und es dauert zehn Tage oder sie sagen es wird zwei Monate dauern und es braucht über fünfundreißig Jahre. Zum Beispiel fragte man Schüler nach Zeiträumen, bei welchen sie zu 50\%, 75\% und 99\% sicher wären, ihre Hausaufgaben danach fertig zu haben und nur 13\%, 19\% und 45\% der Schüler waren nach diesen Zeiträumen fertig. Und sie fanden heraus, dass der Grund dafür war, dass wenn sie eine Gruppe nach ihren Schätzungen für den besten Fall fragten, wenn alles so gut wie möglich liefe und eine andere Gruppe nach ihren durchschnittlichen Schätzungen, wenn alles wie üblich liefe, sie Antworten bekamen, die statistisch nicht voneinander zu unterscheiden waren. Verstehen Sie, wenn Sie Leute fragen, was sie im} \emph{\emph{Normalfall}} \emph{erwarten, stellen sie sich vor, was nach dem wahrscheinlichsten Fall für jeden Punkt entlang des Weges aussieht - alles läuft nach Plan, keine Überraschungen. Aber eigentlich, da mehr als die Hälfte der Schüler nach der Zeit, von der sie zu 99\% sicher waren, danach fertig zu sein, nicht fertig waren, liefert die Realität normalerweise Ergebnisse, die ein wenig schlimmer sind als das 'Worst-case-Szenario'. Es wird als der Planungs-Fehlschluss bezeichnet und der beste Weg ihn zu vermeiden, ist, sich zu fragen, wie lange man für die Sachen beim letzten mal gebraucht hat. Das nennt man die Außensicht statt der Innensicht zu verwenden. Aber wenn man etwas neues macht und das nicht tun kann, muss man einfach wirklich, wirklich, wirklich pessimistisch sein. So pessimistisch, dass es in der Realität tatsächlich ungefähr so oft} \emph{\emph{besser}} \emph{ausgeht, als man gedacht hat, wie es schlechter ausgeht. Es ist tatsächlich} \emph{\emph{wirklich schwierig, so}} \emph{pessimistisch zu sein, dass man eine angemessene Chance hat, das echte Leben tatsächlich noch zu} \emph{\emph{untertreffen.}} \emph{Etwa wie, wenn ich diesen riesigen} \emph{Aufwand treibe, mir vorzustellen, dass einer meiner Klassenkameraden gebissen wird, aber was eigentlich passiert, ist, dass die überlebenden Todesser die ganze Schule angreifen, um an mich heranzukommen. Aber andererseits -"}

\emph{„Stop,“ sagte Professor McGonagall.}

\emph{Harry stoppte. Er hatte gerade ausführen wollen, dass sie zumindest wussten, dass der Dunkle Lord nicht angreifen würde, weil er tot war.}

\emph{„Ich denke, ich habe mich vielleicht nicht klar ausgedrückt,“ sagte die Hexe und ihre präzise schottische Stimme klang sogar noch vorsichtiger. „Ist} \emph{\emph{Ihnen persönlich}} \emph{etwas zugestoßen, was Ihnen Angst gemacht hat, Mr. Potter?"}

\emph{„Was mir persönlich zugestoßen ist, ist nur ein Einzelbeleg,“ erklärte Harry. „Es hat nicht das selbe Gewicht, wie ein wiederholt begutachteter Zeitschriftenartikel über eine kontrollierte Studie mit zufälliger Auswahl, vielen Teilnehmern, umfangreichen Effektgrößen und statistischer Signifikanz.„}

\emph{Professor McGonagall kniff sich in den Nasenrücken, atmete ein und aus. „Ich würde es trotzdem gern hören,“ sagte sie.}

\emph{„Ähm...“ sagte Harry. Er atmete tief ein. „Es hatte einige Überfälle in unserer Nachbarschaft gegeben und meine Mutter wollte, dass ich eine Pfanne, die sie sich geliehen hatte, zu einem Nachbarn, zwei Straßen weiter, zurückbringe und ich sagte, dass ich nicht wollte, weil ich vielleicht überfallen würde und sie sagte 'Harry, sag sowas nicht!' Als ob darüber zu reden, dafür} \emph{\emph{sorgen}} \emph{würde, dass es passierte und wenn ich also nicht darüber redete, wäre ich sicher. Ich versuchte zu erklären, warum mich das nicht beruhigte und sie hat mich die Pfanne trotzdem rübertragen lassen. Ich war zu jung, um zu wissen, wie statistisch unwahrscheinlich es für einen Räuber wäre, mich als Ziel zu wählen, aber ich war alt genug um zu wissen, dass nicht über etwas nachzudenken, es nicht davon abhält, zu passieren, also hatte ich wirklich Angst."}

\emph{„Sonst nichts?“ sagte Professor McGonagall nach einer Pause, als klar wurde, dass Harry fertig war. „Es ist Ihnen nichts} \emph{\emph{anderes}} \emph{passiert?"}

\emph{„Ich weiß, es} \emph{\emph{klingt}} \emph{nicht nach viel,“ verteidigte sich Harry. „Aber es war einfach einer dieser kritischen Momente im Leben, verstehen Sie? Ich meine, ich} \emph{\emph{wusste}, dass nicht über etwas nachzudenken, es nicht davon abhalten würde, zu passieren, ich} \emph{\emph{wusste}} \emph{das, aber ich konnte sehen, dass} \emph{Mum wirklich so dachte.“ Harry stoppte und kämpfte mit dem Ärger, der wieder in ihm hochstieg, als er daran dachte. „Sie} \emph{\emph{hörte nicht zu.}} \emph{Ich versuchte, es ihr zu sagen, ich} \emph{\emph{flehte sie an}} \emph{mich nicht nach draußen zu schicken und sie} \emph{\emph{tat es mit einem Lachen ab.}} \emph{Sie behandelte alles, was ich sagte, wie eine Art großen Witz...“ Harry drückte den schwarzen Zorn wieder hinunter. „Das war der Moment, in dem ich merkte, dass alle, die mich beschützen sollten, eigentlich verrückt waren und dass sie nicht auf mich hören würden, egal wie sehr ich sie anflehte und dass ich mich niemals darauf verlassen könnte, dass sie irgendetwas richtig machen.“ Manchmal waren gute Absichten nicht genug, manchmal musste man zurechnungsfähig sein...}

\emph{Es gab einen langen Moment der Stille.}

\emph{Harry nutzte die Zeit, um tief durchzuatmen und sich zu beruhigen. Es machte keinen Sinn, zornig zu werden. Es machte keinen Sinn, zornig zu werden.} \emph{\emph{Alle}} \emph{Eltern waren so,} \emph{\emph{kein}} \emph{Erwachsener würde sich genug erniedrigen, um sich mit einem Kind auf gleiche Ebene zu begeben und zuzuhören, seine biologischen Eltern wären nicht anders gewesen. Vernunft war ein winziger Funke in der Nacht, eine unendlich seltene Ausnahme von der Regel des Wahnsinns, also machte es keinen Sinn, zornig zu werden.}

\emph{Harry mochte sich nicht, wenn er zornig war.}

\emph{„Danke, dass Sie mir das mitgeteilt haben, Mr. Potter,“ sagte Professor McGonagall nach einer Weile. Es war ein abwesender Ausdruck auf ihrem Gesicht (fast der selbe Ausdruck, der auf Harrys Gesicht erschienen war, als er mit dem Beutel experimentiert hatte, wenn Harry sich nur selbst im Spiegel gesehen hätte, um es zu bemerken). „Ich werde darüber nachdenken müssen.“ Sie drehte sich zur Einmündung der Gasse und hob ihren Zauberstab -}

\emph{„Ähm,“ sagte Harry, „können wir jetzt die Heil-Ausrüstung holen?"}

\emph{Die Hexe hielt inne und erwiderte seinen Blick fest. „Und wenn ich nein sage - dass es zu teuer ist und Sie es nicht brauchen werden - was dann?"}

\emph{Harrys Gesicht verzog sich in Verbitterung. „Genau, was Sie denken, Professor McGonagall.} \emph{\emph{Ganz genau}\emph{,}} \emph{was Sie denken. Ich schließe daraus, dass Sie nur ein weiterer verrückter Erwachsener sind, mit dem ich nicht reden kann und fange an, mir zu überlegen, wie ich trotzdem an eine Heil-Ausrüstung komme."}

\emph{„Ich bin ihr Vormund auf diesem Ausflug,“ sagte Professor McGonagall mit einem Hauch von Gefahr. „Ich werde Ihnen} \emph{\emph{nicht}} \emph{erlauben, mich herumzuschubsen."}

\emph{„Ich verstehe,“ sagte Harry. Er hielt den Unmut aus seiner Stimme heraus und sagte nichts anderes von dem, was ihm in den Sinn kam. Professor McGonagall hatte ihm geraten, zu denken, bevor er sprach. Er würde sich vermutlich morgen nicht mehr daran erinnern, aber er konnte es zumindest für fünf Minuten behalten.}

\emph{Der Zauberstab der Hexe vollführte einen kleinen Kreis in ihrer Hand und der Lärm der Winkelgasse kehrte zurück. „In Ordnung, junger Mann,“ sagte sie. „Lassen Sie uns die Heil-Ausrüstung besorgen."}

\emph{Harry fiel vor Überraschung die Kinnlade herunter. Dann eilte er ihr hinterher, fast stolpernd in seiner plötzlichen Hast.}

--------------------------------------------------------------------------------------------------------------------------------------------

\hfill\break Der Laden war noch so, wie sie ihn verlassen hatten, erkennbare und unerkennbare Gegenstände lagen immer noch in der schrägen hölzernen Auslage, immer noch beschützt von dem grauen Leuchten und das Verkäufer-Mädchen wieder an ihrem alten Platz. Das Verkäufer-Mädchen sah auf, als sie näher kamen, ihr Gesicht zeigte Überraschung.

„Es tut mir leid,“ sagte sie, als sie näher kamen und Harry sprach fast im selben Moment, „Ich entschuldige mich für -„\\ Sie brachen ab und sahen einander an und das Verkäufer-Mädchen lachte ein wenig. „Ich wollte dich bei Professor McGonagall nicht in Schwierigkeiten bringen,“ sagte sie. Ihre Stimme senkte sich verschwörerisch. „Ich hoffe, sie war nicht \emph{zu} schrecklich zu dir."

„\emph{Della!}“ sagte Professor McGonagall und klang schockiert.

„Säckchen voll Gold,“ sagte Hary zu seinem Beutel und sah wieder zu dem Verkäufer-Mädchen auf, während er fünf Galleonen abzählte. „Keine Sorge, ich weiß, dass sie nur schrecklich zu mir ist, weil sie mich liebt."

Er zählte dem Verkäufer-Mädchen fünf Galleonen ab, während Professor McGonagall irgendetwas unwichtiges stotterte. „Ein Notfall-Heiler-Pack Plus, bitte."

Es war wirklich irgendwie verunsichernd, zu sehen, wie die sich weitende Öffnung die aktenkoffer-große medizinische Ausrüstung schluckte. Harry konnte nicht anders, als sich zu fragen, was passieren würde, wenn er versuchte, selbst in den Eselsfell-Beutel hineinzusteigen, angesichts dessen, dass nur derjenige, der etwas hineingetan hatte, es auch wieder herausholen können sollte.

Als der Beutel damit fertig war, seinen hart-erkämpften Einkauf zu... verspeisen..., hätte Harry schwören können, hinterher ein kleines rülpsendes Geräusch gehört zu haben. Das \emph{musste} dort mit Absicht hineingezaubert worden sein. Die Alternativ-Hypothese war zu entsetzlich, um darüber nachzudenken... tatsächlich fielen Harry nicht einmal irgendwelche Alternativ-Hypothesen \emph{ein.} Harry sah wieder zu der Professorin auf, als sie begannen, erneut durch die Winkelgasse zu spazieren. „Wohin als nächstes?"

Professor McGonagall deutete auf ein Geschäft, das aussah, als wäre es aus Fleisch statt Steinen gemacht und mit Fell anstelle von Farbe überzogen. „Kleine Haustiere sind in Hogwarts gestattet - Sie könnten sich zum Beispiel eine Eule beschaffen, um Briefe zu versenden -"

"Kann ich einen Knut oder so bezahlen und eine Eule \emph{mieten,} wenn ich Post verschicken muss?"

„Ja,“ sagte Professor McGonagall.

"Dann denke ich entschieden \emph{nein.}"

Professor McGonagall nickte, als würde sie einen Punkt abhaken. „Dürfte ich fragen, wieso nicht?"

"Ich hatte mal einen Stein als Haustier. Er ist gestorben."

"Sie denken, Sie könnten sich nicht um ein Haustier kümmern?"

„Ich \emph{könnte,}“ sagte Harry, „aber ich würde mich am Ende nur den ganzen Tag fragen, ob ich daran gedacht habe, es an diesem Tag zu füttern oder es langsam in seinem Käfig verhungert und sich fragt, wo sein Meister ist und warum es nichts zu essen gibt."

„Die arme Eule,“ sagte die ältere Hexe mit sanfter Stimme. „So verlassen zu werden. Ich frage mich, was ich dann tun würde."

„Nun, ich nehme an, sie würde wirklich hungrig werden und versuchen, sich mit ihren Klauen einen Weg aus dem Käfig oder der Box oder was auch immer zu bahnen, obwohl sie damit wahrscheinlich nicht viel Glück hätte -“ Harry stoppte kurz.

Die Hexe sprach weiter, immer noch mit dieser sanften Stimme. „Und was würde danach mit ihr passieren?"

„Entschuldigen Sie,“ sagte Harry und griff nach oben, um sanft, aber bestimmt, Professor McGonagalls Hand zu ergreifen und steuerte sie in noch einen weiteren schmalen Durchgang; nachdem sie sich schon vor so vielen Danksagern weggeduckt hatten, war der Vorgang schon fast Routine geworden. „Bitte wirken Sie diesen Stille-Zauber."

"\emph{Quietus.}"

Harrys Stimme zitterte. „Diese Eule steht \emph{nicht} für mich, meine Eltern haben mich \emph{niemals} in einem Schrank eingesperrt und hungern lassen, ich habe \emph{keine} Verlassensängste und mir \emph{gefällt die Richtung nicht, die Ihre Gedanken einschlagen, Professor McGonagall!}"

Die Hexe sah ernst auf ihn herab. „Und was für Gedanken wären das, Mr. Potter?"

„Sie denken, ich wurde,“ Harry hatte Schwierigkeiten es auszusprechen, „ich wurde \emph{missbraucht?}"

"Wurden Sie?"

„\emph{Nein!}“ rief Harry. „Nein, wurde ich niemals! Denken Sie, dass ich \emph{dämlich} bin? Ich \emph{weiß} Bescheid über das Konzept des Kindesmissbrauchs, ich \emph{weiß} Bescheid über unangemessene Berührungen und all das und wenn irgendwas davon passieren würde, würde ich die Polizei rufen! Und es dem Schulleiter sagen! Und nach sozialen Diensten im Telefonbuch suchen! Und es Opa und Oma und Mrs. Figg sagen! Aber meine Eltern haben \emph{niemals} so etwas getan, nie im \emph{Leben!} Wie können sie es \emph{wagen}, so etwas anzudeuten!"

Die ältere Hexe sah ihn fest an. „Es ist meine Pflicht als Stellvertretende Schulleiterin, möglichen Anzeichen für Missbrauch bei den Kindern unter meiner Obhut nachzugehen.„

Harrys Zorn geriet außer Kontrolle, bis er nur noch blanke, schwarze Wut war. „\emph{Wagen} Sie es ja nicht, auch nur ein Wort über diese, diese \emph{Unterstellungen} gegenüber irgendjemandem zu verlieren! \emph{Niemandem} gegenüber, haben Sie mich verstanden, Professor McGonagall? So eine Anschuldigung kann Leute ruinieren und Familien zerstören, selbst wenn die Eltern vollkommen unschuldig sind! Ich habe in den Zeitungen darüber gelesen!“ Harry Stimme steigerte sich zu einem schrillen Schrei. „Das \emph{System} weiß nicht, wie man \emph{aufhört,} es glaubt weder den Eltern noch den Kindern, wenn sie sagen, dass nichts passiert ist! \emph{Wagen Sie es nicht meine Familie damit zu bedrohen! Ich werde nicht zulassen, dass Sie mein} \emph{Zuhause zerstören!}"

„Harry,“ sagte die ältere Hexe sanft und streckte eine Hand nach ihm aus -

Harry machte einen schnellen Schritt zurück und seine Hand schnellte nach oben und stoß ihre weg.

McGonagall erstarrte, zog dann ihre Hand zurück und trat einen Schritt zurück. „Harry, es ist alles in Ordnung,“ sagte sie. „Ich glaube dir."

„\emph{Tun Sie das,}“ zischte Harry. Die Wut schäumte immer noch in seinem Blut. „Oder warten Sie nur, bis Sie von mir wegkommen, damit Sie die Papiere ausfüllen können?"

"Harry, ich habe dein Zuhause gesehen. Ich habe deine Eltern gesehen. Sie lieben dich. Du liebst sie. Ich glaube dir, wenn du sagst, dass deine Eltern dich nicht missbrauchen. Aber ich \emph{musste} fragen, weil hier irgendetwas seltsames vor sich geht."

Harry starrte sie kalt an. „Wie was?"

„Harry, ich habe viele missbrauchte Kinder in meiner Zeit in Hogwarts gesehen, es würde dir das Herz brechen, wenn du wüsstest wie wie viele. Und, wenn du glücklich bist, benimmst du dich nicht wie eines dieser Kinder, \emph{überhaupt} nicht. Du lächelst Fremde an, du umarmst Leute, ich habe meine Hand auf deine Schulter gelegt und du hast nicht gezuckt. Aber manchmal, nur manchmal, sagst oder tust du etwas, dass \emph{sehr} nach... jemandem aussieht, der seine ersten elf Lebensjahre in einem Keller eingschlossen verbracht hat. Nicht wie die liebende Familie, die ich sah.“ Professor McGonagall neigte den Kopf, ihr Gesichtsausdruck wieder verwirrt.

Harry nahm das auf, verarbeitete es. Die schwarze Wut begann zu weichen, als ihm klar wurde, dass ihm respektvoll zugehört wurde und seine Familie nicht in Gefahr war.

"Und wie \emph{erklären} Sie dann ihre Beobachtungen, Professor McGonagall?"

„Ich weiß nicht,“ sagte sie. „Aber es ist möglich, dass dir etwas zugestoßen ist, an das du dich nicht erinnerst."

Die Wut stieg erneut in Harry hoch. Das klang viel zu sehr nach dem, was er in der Zeitungsgeschichten über zerrüttete Familien gelesen hatte. „Unterdrückte Erinnerungen sind ein Haufen \emph{Pseudowissenschaft!} Menschen verdrängen traumatische Erinnerungen \emph{nicht,} sie erinnern sich für den Rest ihres Lebens nur \emph{zu} gut daran!"

"Nein, Mr. Potter. Es gibt einen Zauber namens Obliviate."

Harry erstarrte. „Ein Zauber, der Erinnerungen auslöscht?"

Die ältere Hexe nickte. „Aber nicht alle Auswirkungen der Erfahrung, wenn Sie wissen, was ich meine, Mr. Potter."

Ein Schauer lief Harrys den Rücken hinunter. \emph{Diese} Hypothese... konnte \emph{nicht} einfach widerlegt werden. „Aber meine Eltern könnten das nicht tun!"

„In der Tat nicht,“ sagte Professor McGonagall. „Es hätte jemand aus der Zauberwelt sein müssen. Es gibt... kein Weg, um sicher zu gehen, fürchte ich."

Harry rationalistische Fähigkeiten fingen an, wieder hochzufahren. „Professor McGonagall, wie sicher sind Sie sich Ihrer Beobachtungen und welche alternativen Erklärungen könnte es noch geben?"

Die Hexe streckte die Hände, wie um zu zeigen, dass sie leer waren. „Sicher? Ich bin über \emph{gar nichts} sicher, Mr. Potter. In meinem ganzen Leben habe ich niemanden wie Sie getroffen. Manchmal scheinen Sie einfach nicht elf Jahre alt oder auch nur wirklich \emph{menschlich} zu sein.„

Harrys Augenbrauen wanderten nach oben -

Es tut mir leid!“ sagte Professor McGonagall schnell. „Es tut mir sehr leid, Mr. Potter. Ich wollte etwas klar machen und fürchte, dass es anders klang als von mir beabsichtigt -"

„Ganz im Gegenteil, Professor McGonagall,“ sagte Harry und lächelte leicht. „Ich werde das als sehr großes Kompliment auffassen. Aber würde es Ihnen etwas aus machen, wenn ich eine alternative Erklärung anböte?"

"Nur zu."

„Man erwartet nicht, dass Kinder schlauer als ihre Eltern sind,“ sagte Harry. „Oder vielleicht auch viel vernünftiger - mein Vater könnte mich an Schlauheit wahrscheinlich überbieten, wenn er, Sie wissen schon, es tatsächlich \emph{versuchen} würde, anstatt seine Erwachsenen-Intelligenz nur dazu zu benutzen, sich neue Ausreden einfallen zu lassen, um seine Meinung nicht ändern zu müssen -“ Harry brach ab. „Ich bin zu schlau, Professor. Ich habe normalen Kindern nichts zu sagen. Erwachsene respektieren mich nicht genug, um wirklich mit mir zu sprechen. Und, um ehrlich zu sein, selbst wenn sie es täten, würden sie nicht so schlau klingen wie Richard Feynman, also lese ich lieber etwas, was Richard Feynman geschrieben hat. Ich bin \emph{isoliert,} Professor McGonagall. Ich war mein ganzes Leben lang isoliert. Vielleicht hat das einige der selben Auswirkungen, wie in einem Keller eingesperrt zu sein. Und ich bin zu intelligent, um so zu meinen Eltern aufzusehen, wie Kinder es tun sollten. Meine Eltern lieben mich, aber sie fühlen sich nicht verpflichtet, auf Vernunft zu reagieren und manchmal fühlt es sich so an, als seien sie die Kinder - Kinder die \emph{nicht zuhören} und absolute Autorität über meine gesamte Existenz haben. Ich versuche, deshalb nicht zu verbittert zu sein, aber ich versuche auch, mir gegenüber \emph{ehrlich} zu sein, also, ja, bin ich verbittert. Und außerdem habe ich ein Aggressionsbewältigungs-Problem, aber ich arbeite daran. Das ist alles."

"\emph{Das ist alles?}„

Harry nickte bestimmt. „Das ist alles. Sicherlich, Professor McGonagall, ist auch im magischen Britannien die normale Erklärung es immer wert, \emph{berücksichtigt} zu werden?“

--------------------------------------------------------------------------------------------------------------------------------------------

\hfill\break Es war später am Tag, die Sonne sank am Sommerhimmel und der Strom der Einkäufer auf der Straße begann zu versiegen. Einige Geschäfte hatten bereits geschlossen; Harry und Professor McGonagall hatten seine Lehrbücher bei Flourish und Blotts noch gerade so vor Ladenschluss gekauft. Mit nur einer kleinen Explosion, als Harry einen Abstecher bei dem Schlüsselwort „Arithmantik“ gemacht und festgestellt hatte, dass die Lehrbücher für das siebte Schuljahr nichts mathematisch fortgeschritteneres als Trigonometrie enthielten.\\ In diesem Moment allerdings, waren Harrys Geist die Träume von den tief hängenden Früchten wissenschaftlicher Erkenntnisse fern.\\ In diesem Moment, spazierten die beiden aus dem Geschäft von Ollivander hinaus und Harry starrte seinen Zauberstab an. Er schwang ihn und brachte mehrfarbige Funken hervor, was wirklich kein sehr großer Schock mehr hätte sein sollen, nach allem, was er bereits gesehen hatte, aber irgendwie -

\emph{Ich kann Magie wirken.}

\emph{Ich. Wie in, ich persönlich. Ich bin magisch; ich bin ein Zauberer.}

\emph{Harry hatte} \emph{\emph{gefühlt,}} \emph{wie die Magie seinen Arm hinaufstieg und in diesem Moment erkannt, dass er schon immer diesen Sinn gehabt hatte, sein ganzes Leben lang, den Sinn der weder Sehen, noch Hören, Riechen, Schmecken oder Anfassen war, sondern einfach Magie. Als hätte man} \emph{Augen, hielte sie aber ständig geschlossen, so dass man gar nicht wusste, dass man Dunkelheit sah und dann eines Tages öffneten sich die Augen und sahen die Welt. Der Schock davon hatte ihn durchströmt, Teile seines Selbst berührt, sie erwachen lassen und war dann in Sekunden vergangen, nur das sichere Wissen hinterlassend, dass er jetzt ein Zauberer war und immer gewesen war und es, auf seltsame Weise, sogar immer gewusst hatte.}

\emph{Und -}

\emph{\emph{"Es ist wirklich sehr eigenartig, dass Sie nun für diesen Zauberstab bestimmt zu sein scheinen, wo doch sein Bruder sie mit dieser Narbe gezeichnet hat.„}}

\emph{Das konnte} \emph{\emph{unmöglich}} \emph{ein Zufall sein. Es waren} \emph{\emph{tausende}} \emph{von Zauberstäben in diesem Laden gewesen. Nun, okay, es} \emph{\emph{hätte}} \emph{Zufall sein können, es gab sechs Milliarden Menschen auf der Welt tausend-zu-eins Zufälle passierten jeden Tag. Aber das Bayes-Theorem besagte, dass jede vernünftige Hypothese, die es} \emph{\emph{wahrscheinlicher}} \emph{als tausend-zu-eins machte, dass er am Ende den Bruder des Zauberstabes des Dunklen Lords bekommen würde, einen Vorteil hatte.}

\emph{Professor McGonagall hatte einfach nur gesagt} \emph{\emph{wie eigenartig}} \emph{und es dabei belassen, was Harry in einen Schockzustand versetzt hatte, angesichts des schieren, überwältigenden} \emph{\emph{Mangels an Neugier}} \emph{von Zauberern und Hexen. In keiner} \emph{\emph{denkbaren}} \emph{Welt, hätte Harry nur gedacht „Hm“ und wäre aus dem Laden hinausgegangen, ohne auch nur zu} \emph{\emph{versuchen}} \emph{eine Hypothese darüber zu erstellen, was vor sich ging.}

\emph{Er hob seine rechte Hand und berührte seine Narbe.}

\emph{Was...} \emph{\emph{genau...}}

\emph{„Sie sind jetzt ein vollwertiger Zauberer,“ sagte Professor McGonagall. „Glückwunsch."}

\emph{Harry nickte.}

\emph{„Und was halten Sie von der Zauberwelt?“ sagte sie.}

\emph{„Es ist seltsam,“ sagte Harry. „Ich sollte an all das denken, was ich von der Magie gesehen habe... an alles von dem ich jetzt weiß, dass es möglich ist und alles von dem ich jetzt weiß, dass es eine Lüge ist und all die Arbeit, die vor mir liegt, um es zu verstehen. Und doch stelle ich fest, dass ich abgelenkt bin von relativen Nebensächlichkeiten, wie,“ Harry senkte die Stimme, „der ganzen Junge-der-überlebt-hat-Sache.“ Es war niemand in} \emph{der Nähe, aber man musste das Schicksal ja nicht herausfordern.}

\emph{Professor McGonagall} \emph{\emph{ähem-te.}} \emph{„Wirklich? Was Sie nicht sagen."}

\emph{Harry nickte. „Ja. Es ist einfach} \emph{\emph{merkwürdig.}} \emph{Herauszufinden, dass man Teil dieser großartigen Geschichte war, der Quest, den großen und schrecklichen Dunklen Lord zu besiegen und es ist schon} \emph{\emph{vollbracht.}} \emph{Fertig. Komplett vorbei. Als wäre man Frodo Beutlin und fände heraus, dass einen die eigenen Eltern zum Schicksalsberg getragen und einen den Ring haben reinwerfen lassen und man erst ein Jahr alt war und sich nicht mal daran erinnert."}

\emph{Professor McGonagalls Lächeln war etwas eingeforen.}

\emph{„Wissen Sie, wenn ich irgendwer anderes wäre, ich meine irgendjemand anders, wäre ich wahrscheinlich ziemlich besorgt, ob ich diesem Anfang gerecht werden könnte.} \emph{\emph{Mann, Harry, was hast du denn so gemacht, seit du den Dunklen Lord besiegt hast? Deine eigene Buchhandlung? Großartig! Sag mal, wusstest du eigentlich, dass ich mein Kind nach dir benannt habe?}} \emph{Aber ich habe Hoffnungen, dass das kein Problem sein wird.“ Harry seufzte. „Trotzdem... ist es fast genug, dass ich mir wünschte, es gäbe noch} \emph{\emph{irgendwelche}} \emph{losen Enden der Quest, so dass ich wirklich, Sie wissen schon, irgendwie daran} \emph{\emph{beteiligt}} \emph{war."}

\emph{„Oh?“ sagte Professor McGonagall in einem merkwürdigen Tonfall. „Was hatten Sie da im Sinn?"}

\emph{"Nun, zum Beispiel erwähnten Sie, dass meine Eltern verraten wurden. Wer verriet sie?"}

\emph{„Sirius Black,“ sagte die Hexe und zwischte den Namen fast. Er sitzt in Askaban. Zauberer-Gefängnis."}

\emph{"Wie wahrscheinlich ist es, dass Sirius Black aus dem Gefängnis ausbrechen wird und ich ihn werde aufspüren und in einer Art spektakulärem Duell werde besiegen müssen oder besser noch eine hohe Belohnung auf seinen Kopf aussetzen und mich in Australien verstecken, während ich die Ergebnisse abwarte?"}

\emph{Professor McGonagall blinzelte. Zweimal. „Unwahrscheinlich. Niemand ist jemals aus Askaban entkommen und ich bezweifle, dass} \emph{\emph{er}} \emph{der erste sein wird.„}

\emph{Harry war ein bisschen skeptisch bezüglich dieser „\emph{niemand}} \emph{ist} \emph{\emph{jemals}} \emph{aus Askaban entkommen“-Zeile. Trotzdem, vielleicht konnte man mit Magie einem zu 100\% sicheren Gefängnis tatsächlich nahe kommen, besonders} \emph{wenn man selbst einen Zauberstab hatte und sie nicht. Der beste Weg da raus zu kommen, wäre, gar nicht erst rein zu kommen.}

\emph{„Na gut, dann,“ sagte Harry. „Klingt als wäre alles hübsch verpackt.“ Er seufzte und kratzte sich mit der Hand die Stirn. „Oder vielleicht ist der Dunkle Lord in jener Nacht nicht} \emph{\emph{wirklich}} \emph{gestorben. Nicht ganz. Sein Geist besteht fort, flüstert Menschen in Alpträumen zu, die hinübersickern in die wache Welt und sucht nach einem Weg zurück in das Land der Lebenden, welches zu zerstören er geschworen hat und jetzt sind er und ich, wie es die alte Prophezeiung besagt, in einen tödlichen Kampf verstrickt, in dem der Sieger verlieren und der Verlierer siegen soll -"}

\emph{Professor McGonagalls Kopf drehtesich und ihre Augen fuhren herum, als suchten sie die Straße nach Zuhörern ab.}

\emph{„Ich} \emph{\emph{scherze,}} \emph{Professor,“ sagte Harry etwas ärgerlich. Jesses, warum nahm sie nur immer alles so ernst -}

\emph{Langsam begann sich ein sackendes Gefühl in Harrys Magengrube auszubreiten.}

\emph{Professor McGonagall sah ihn mit ruhigem Gesichtsausdruck an. Einem sehr,} \emph{\emph{sehr}} \emph{ruhigen Gesichtsausdruck. Dann setzte sie ein Lächeln auf. „Natürlich tun Sie das, Mr. Potter.„}

\emph{\emph{Ach, Mist.}}

\emph{Wenn Harry die wortlose Schlussfolgerung, die ihm gerade in den Kopf geschossen war, hätte formalisieren müssen, hätte das in etwa geklungen, wie, 'Wenn ich die Wahrscheinlichkeit, dass das, was Professor McGonagall gerade getan hat, das Resultat sorgfältiger Selbstkontrolle war, gegen die Wahrscheinlichkeit all dessen abschätze, was sie} \emph{\emph{natürlicherweise}} \emph{tun würde, wenn ich einen schlechten Scherz gemacht hätte, dann ist ihr Verhalten ein deutlicher Hinweis darauf, dass sie etwas verbirgt.'}

\emph{Aber was Harry tatsächlich dachte, war,} \emph{\emph{Ach, Mist.}}

\emph{Harry drehte selbst den Kopf, um die Straße abzusuchen. Nein, keiner in der Nähe. „Er ist} \emph{\emph{nicht}} \emph{tot, oder,“ seufzte Harry.}

\emph{"Mr. Potter -"}

\emph{"Der Dunkle Lord ist am Leben.} \emph{\emph{Natürlich}} \emph{ist er am Leben. Es war ein} \emph{\emph{Paradebeispiel}} \emph{für vollkommenen} \emph{\emph{Optimismus,}} \emph{dass ich mir auch nur etwas anderes} \emph{\emph{erträumt}} \emph{habe. Ich} \emph{\emph{muss}} \emph{meinen} \emph{\emph{Verstand}} \emph{verloren haben, ich kann mir nicht} \emph{\emph{vorstellen,}} \emph{was ich mir dabei} \emph{\emph{gedacht}} \emph{habe. Nur weil} \emph{\emph{irgendjemand}} \emph{sagte, sein Körper wäre bis zur} \emph{\emph{Unkenntlichkeit}} \emph{verbrannt gefunden worden; ich kann mir nicht vorstellen, warum ich ihn für} \emph{\emph{tot}} \emph{hätte halten sollen.} \emph{\emph{Offensichtlich}} \emph{habe ich noch einiges zu lernen über die Kunst des angemessenen} \emph{\emph{Pessimismus.}"}

"Mr. Potter -"

„Sagen Sie mir wenigstens, dass es nicht wirklich eine Prophezeiung gibt...“ Professor McGonagall warf ihm noch immer dieses strahlende, eingefrorene Lächeln zu. „Oh, sie \emph{wollen} mich doch wohl veralbern."

"Mr. Potter, Sie sollten keine Dinge erfinden, um die sich Sorgen machen müssten -"

"Wollen Sie mir \emph{tatsächlich das} sagen? Stellen Sie sich meine Reaktion später vor, wenn ich herausfinde, dass es doch etwas gab, worüber man sich sorgen sollte.

Ihr eingefrorenes Lächeln fiel in sich zusammen.

Harrys Schultern sackten zusammen. „Ich habe eine ganze Welt der Magie zu analysieren. Ich habe \emph{keine} Zeit dafür."

Dann waren beide still, als ein Mann in einem wehenden orangenen Umhang auf der Straße erschien und langsam vorbeiging; Professor McGonagalls Augen folgten ihm unauffällig. Harrys Mund bewegte sich, als er heftig auf seiner Lippe herumkaute und jemand der genau hinsah, hätte einen winzigen Tropfen Blut erscheinen sehen.

Als der Mann im orangenen Umhang in der Ferne verschwunden war, sprach Harry wieder, seine Stimme ein leises Murmeln. „Werden Sie mir jetzt die Wahrheit sagen, Professor McGonagall? Und versuchen Sie nicht, es abzustreiten, ich bin nicht dämlich."

„Sie sind \emph{elf Jahre alt,} Mr. Potter!“ sagte sie mit einem strengen Flüstern.

"Und von daher kein vollwertiger Mensch. Tut mir leid... ich hab's einen Moment lang \emph{vergessen.}"

"Dies sind furchtbare und wichtige Angelegenheiten! Sie sind \emph{geheim,} Mr. Potter! Es ist ein \emph{Desaster,} dass Sie, noch ein Kind, auch nur so viel wissen! Sie dürfen es \emph{niemandem} sagen, verstehen Sie? Absolut niemandem!"

Wie es manchmal passierte, wenn Harry \emph{ausreichend} ärgerlich wurde, wurde sein Blut kalt anstatt heiß und eine schreckliche Klarheit senkte sich über seinen Geist, potentielle Taktiken ausmachend und ihre Konsequenzen mit eisernem Realismus bewertend.

\emph{Bestehe darauf, dass du ein Recht darauf hast, es zu erfahren: Fehlschlag. In McGonagalls Augen haben elf-jährige Kinder nicht das Recht irgendetwas zu erfahren.}

\emph{Sag, dass ihr sonst keine Freunde mehr seid: Fehlschlag. Ihr liegt nicht genug an eurer Freundschaft.}

\emph{Erkläre, dass es dich in Gefahr bringen wird, wenn du es nicht weißt: Fehlschlag. Es wurden bereits Pläne gemacht, die deine Unwissenheit voraussetzen. Die} \emph{sichere} \emph{\emph{Unannehmlichkeit, sie zu überdenken, wird sehr viel unangenehmer erscheinen, als die nur}} \emph{ungewisse} \emph{\emph{Aussicht, dass du zu schaden kommen könntest.}}

\emph{Gerechtigkeit und Vernunft werden beide versagen. Du musst entweder etwas finden, das du hast, was sie will oder etwas, das du tun kannst, was sie fürchtet...}

Ah.

„Nun denn, Professor,“ sagte Harry mit leiser, eisiger Stimme, „es klingt, als ob ich etwas habe, was sie wollen. Sie können mir, wenn Sie mögen, die Wahrheit sagen, die \emph{ganze} Wahrheit und im Gegenzug, werde ich Ihre Geheimnisse wahren. Oder Sie können versuchen, mich unwissend zu lassen, um mich als Pfand einzusetzen, in welchem Fall ich Ihnen nichts schuldig sein werde."

McGonagall stoppte mitten auf der Straße. Ihre Augen blitzten und ihre Stimme sank vollkommen zu einem Zischen herab. „Wie können Sie es wagen!"

„\emph{Wie können Sie es wagen!}“ flüsterte er ihr zurück.

"Sie würden mich \emph{erpressen?}"

Harry verzog die Lippen. „Ich \emph{biete} Ihnen einen \emph{Gefallen} an. Ich \emph{gebe} Ihnen die Chance, \emph{Ihr} wertvolles Geheimnis zu schützen. Wenn Sie ablehnen, werde ich \emph{jeden} nachvollziehbaren Grund haben, anderswo Nachforschungen anzustellen, nicht um Sie zu verärgern, sondern weil ich \emph{es wissen muss!} Kommen Sie über Ihre sinnlose Verärgerung über ein \emph{Kind}, von dem Sie denken, es sollte Ihnen gehorchen, hinweg und Sie werden erkennen, dass ein vernünftiger Erwachsener das selbe tun würde! \emph{Sehen Sie es aus meiner Perspektive! Wie würden Sie sich fühlen, wenn SIE es wären?}„

Harry sah Professor McGonagall an, beobachtete ihr heftiges Atmen. Es kam ihm in den Sinn, dass jetzt die Zeit wäre, den Druck etwas zu mildern, sie eine Weile köcheln zu lassen. „Sie müssen sich nicht gleich entscheiden,“ sagte Harry in normalerem Tonfall. „Ich verstehe, wenn Sie etwas Zeit möchten, um über mein \emph{Angebot} nachzudenken... aber ich werde Sie vor einer Sache warnen,“ sagte Harry und seine Stimme wurde kälter. „Probieren Sie nicht diesen Obliviate-Zauber an mir aus. Vor einiger Zeit habe ich mir ein Signal ausgedacht und es mir bereits selbst geschickt. Wenn ich das Signal bekomme und mich nicht \emph{erinnere,} es gesendet zu haben...“ ließ Harry den Satz bedeutungsvoll ausklingen.

McGonagalls Gesicht arbeitete, als ihr Ausdruck sich veränderte. „Ich... habe nicht daran gedacht, Ihre Erinnerungen zu löschen, Mr. Potter... aber waren hätten Sie sich ein solches Signal \emph{ausgedacht,} wenn Sie gar nicht wussten -"

"Ich kam darauf, während ich ein Muggel-Science-Fiction-Buch gelesen habe und mir gedacht \emph{na ja, nur für den Fall...} Und nein, ich werde Ihnen das Signal nicht verraten, ich bin nicht blöd."

„Ich wollte nicht danach fragen,“ sagte McGonagall. Sie schien in sich zusammenzufallen und sah plötzlich sehr alt und sehr müde aus. „Das war ein anstrengender Tag, Mr. Potter. Können wir Ihren Koffer besorgen und Sie nach Hause schicken? Ich vertraue darauf, dass Sie nicht über diese Angelegenheit sprechen, bis ich Zeit zum Nachdenken hatte. Merken Sie sich, dass nur zwei andere Menschen auf der Welt von dieser Sache wissen und zwar Schulleiter Albus Dumbledore und Professor Severus Snape."

Also. Neue Informationen; das war ein Friedensangebot. Harry akzeptierte mit einem Nicken, drehte den Kopf, um nach vorn zu sehen und begann, weiter zu gehen, als sein Blut sich langsam wieder zu erwärmen begann.

„Also muss ich jetzt einen Weg finden, um einen unsterblichen Dunklen Zauberer zu töten,“ sagte Harry und seufzte vor Frustration. „Ich wünschte wirklich, Sie hätten mir das gesagt, \emph{bevor} ich mit dem Einkaufen angefangen habe."

--------------------------------------------------------------------------------------------------------------------------------------------

\hfill\break Das Koffer-Geschäft war reichlicher ausgestattet als jedes andere, das Harry besucht hatte. Die Vorhänge waren luxuriös und geschmackvoll gemustert, die Böden und Wände aus gebeiztem und poliertem Holz und die Koffer nahmen Ehrenplätze auf polierten elfenbeinernen Plattformen ein. Der Verkäufer war in einen Umhang gekleidet, dessen Pracht nur einen Hauch unter dem von Lucius Malfoy lag und sprach mit ausgesuchter, schmieriger Höflichkeit sowohl zu Harry als auch Professor McGonagall.

Harry hatte seine Fragen gestellt und wurde angezogen von einem Koffer aus massiv aussehendem Holz, nicht poliert, sondern warm und solide, mit dem eingravierten Muster eines Wächter-Drachens, dessen Augen sich bewegten, um jeden, der sich näherte, anzusehen. Ein Trank, der verzaubert war, um leicht zu sein, auf Kommando zu schrumpfen, kleine klauenbewehrte Tentakel aus seiner Unterseite wachsen zu lassen und seinem Besitzer nachzufolgen. Ein Koffer mit zwei Schubfächern auf jeder der vier Seiten, die beim Aufziehen Fächer, so tief wie der ganze Koffer, offenbarten. Ein Deckel mit vier Schlössern, von denen jedes einen anderen Raum im Inneren enthüllte. Und -- dies war das Wichtige -- ein Griff am Boden, um einen Rahmen herauszuziehen, der eine Treppe enthielt, die in einen kleinen erleuchteten Raum hinunterführte, der, wie Harry schätzte, Platz für etwa zwölf Bücherregale enthielt.

Wenn man solche Gepäckstücke herstellte, wusste Harry nicht, warum irgendjemand ein Haus besaß.

Einhundertundacht goldene Galleonen. Das war der Preis für einen guten Koffer, leicht benutzt. Bei etwa fünfzig Britischen Pfund die Galleone war das genug, um einen Gebrauchtwagen zu kaufen. Er würde teurer sein, als alles andere, was Harry jemals in seinem Leben gekauft hatte, zusammen.

Neunundsiebzig Galleonen. Soviel war noch in dem Säckchen voll Gold verblieben, welches Harry aus Gringotts mitzunehmen erlaubt worden war.

Professor McGonagalls Gesicht zeigte einen verlegenen Ausdruck. Nach einem langen Einkaufstag hatte sie Harry nicht fragen müssen, wie viel Gold in dem Säckchen noch übrig war, nachdem der Verkäufer seinen Preis genannt hatte, was bedeutete, dass die Professorin ohne Stift und Papier gut im Kopf rechnen konnte. Wieder einmal ermahnte Harry sich selbst, dass \emph{wissenschaftlich ungebildet} nicht das selbe war wie \emph{dämlich.}

\emph{„Es tut mir leid, junger Mann,“ sagte Professor McGonagall. „Das ist vollkommen meine Schuld. Ich würde anbieten, Sie nach Gringotts zurückzubringen, aber die Bank wird bereits für alles, außer Notfalldiensten, geschlossen sein."}

\emph{Harry sah sie an und fragte sich...}

\emph{„Nun,“ seufzte Professor McGonagall, als sie sich auf dem Absatz} \emph{umdrehte, „wir können ebenso gut gehen, nehme ich an."}

\emph{...sie} \emph{\emph{war nicht}} \emph{vollkommen durchgedreht, als ein Kind es gewagt hatte, ihr zu trotzen. Sie war nicht glücklich damit gewesen, aber sie hatte} \emph{\emph{nachgedacht,}} \emph{anstatt vor Wut zu explodieren. Es hätte nur sein können, weil es einen unsterblichen Dunklen Lord zu bekämpfen galt -- weil sie auf Harrys guten Willen angewiesen war. Aber die meisten Erwachsenen wären nicht einmal fähig gewesen, auch nur soweit zu denken; würden} \emph{\emph{zukünftige Konsequenzen}} \emph{überhaupt nicht berücksichtigen, wenn jemand mit geringerer Stellung ihnen nicht gehorcht hätte...}

\emph{„Professor?“ sagte Harry.}

\emph{Die Hexe drehte sich zurück und sah ihn an.}

\emph{Harry atmete tief ein. Er musste ein wenig ärgerlich werden, für das, was er nun versuchen wollte, er brächte ansonsten keinesfalls den Mut dafür auf.} \emph{\emph{Sie hat nicht auf mich gehört,}} \emph{dachte er zu sich selbst,} \emph{\emph{ich hätte mehr Gold mitgenommen, aber sie wollte nicht hören...}} \emph{Seine gesamte Welt auf McGonagall und die Notwendigkeit, dieses Gespräch seinem Willen zu beugen, konzentrierend, sprach er.}

\emph{"Professor, Sie dachten, einhundert Galleonen würden mehr als ausreichend für einen Koffer sein. Deshalb hielten Sie es nicht für nötig, mich zu warnen, bevor es nur noch neunundsiebzig wurden. Was genau das ist, was die Studien zeigen -- dass ist, was passiert, wenn Leute denken, sie lassen sich einen} \emph{\emph{kleinen}} \emph{Sicherheitspuffer. Sie sind nicht pessimistisch genug. Wenn ch dafür zuständig gewesen wäre, hätte ich} \emph{\emph{zweihundert}} \emph{Galleonen genommen, nur um sicherzugehen. Da war massenhaft Gold in diesem Verlies und ich hätte alles Überschüssige später zurücklegen können. Aber ich dachte, Sie würden es mich nicht tun lassen. Ich dachte, Sie würden zornig auf mich werden, nur bei der Frage. Lag ich falsch?"}

\emph{„Ich nehme an, ich muss gestehen, dass Sie recht haben,“ sagte Professor McGonagall. „Aber, junger Mann -"}

\emph{„Solche Sachen sind der Grund, warum ich Schwierigkeiten habe, Erwachsenen zu vertrauen.“ Irgendwie behielt Harry seine feste Stimme bei. „Weil sie zornig werden, wenn man auch nur} \emph{\emph{versucht,}} \emph{vernünftig mit ihnen zu reden. Für sie ist das Trotz und Anmaßung und eine Anfechtung ihres höheren Status im Stamm. Wenn man versucht, mit ihnen zu reden, werden sie} \emph{\emph{zornig.}} \emph{Wenn ich also etwas} \emph{\emph{wirklich wichtiges}} \emph{tun müsste, könnte ich Ihnen nicht vertrauen. Selbst wenn sie mit tiefer Sorge, dem, was auch immer ich sage, zuhörten -- weil das auch Teil von jemandes} \emph{\emph{Rolle}} \emph{ist, der einen besorgten Erwachsenen spielt -- würden Sie niemals Ihre Handlungen ändern, würden sich nicht tatsächlich anders verhalten, wegen irgendwas, das ich gesagt habe."}

\emph{Der Verkäufer beobachtete sie beide mit unverhohlener Faszination.}

\emph{„Ich kann Ihren Standpunkt verstehen,“ sagte Professor McGonagall schließlich. „Wenn ich manches mal zu streng erscheine, denken Sie bitte daran, dass ich als Hauslehrerin von Gryffindor gefühlte mehrere tausend Jahre gedient habe."}

\emph{Harry nickte und fuhr fort. „Also - nehmen Sie an, ich hätte einen Weg, um mehr Galleonen aus meinem Verlies zu bekommen,} \emph{\emph{ohne}} \emph{dass wir nach Gringotts zurückgehen, aber das würde voraussetzen, dass ich die Rolle eines gehorsamen Kindes verletze. Könnte ich Ihnen damit vertrauen, auch wenn Sie aus Ihrer eigenen Rolle als Professor McGonagall fallen müssten, um davon Gebrauch zu machen?"}

\emph{„\emph{Was?}“ sagte Professor McGonagall.}

\emph{"Um es anders auszudrücken, wenn ich dafür sorgen könnte, dass der heutige Tag anders verlaufen wäre, so dass wir} \emph{\emph{nicht}} \emph{zu wenig Geld mitgenommen haben, wäre das in Ordnung, obwohl das beinhalten würde, dass, rückblickend gesehen, ein Kind anmaßend gegenüber einem Erwachsenen war?"}

\emph{„Ich... nehme an...“ sagte die Hexe und sah ziemlich verwirrt aus.}

\emph{Harry zog seinen Eselsfell-Beutel heraus und sagte, „Elf Galleonen, ursprünglich aus meinem Familien-Verlies."}

\emph{Und es lag Gold in Harrys Hand.}

\emph{Für einen Moment stand Professor McGonagalls Mund weit offen, dann schnappte ihre Kinnlade zu, ihre Augen wurden schmal und die Hexe brachte heraus, „\emph{Wo}} \emph{haben Sie das her -"}

\emph{"Aus meinem Familien-Verlies, wie ich sagte."}

\emph{"\emph{Wie?}"}

\emph{"Magie."}

\emph{„Das ist wohl kaum eine Antwort!“ schnappte Professor McGonagall und hielt dann inne, blinzelnd.}

\emph{"Nein, ist es nicht, nicht wahr? Ich} \emph{\emph{sollte}} \emph{jetzt behaupten, dass ich experimentell die wahren Geheimnisse entdeckt habe, wie dieser Beutel funktioniert und dass er tatsächlich Objekte von überall her abrufen kann,} \emph{nicht nur aus seinem eigenen Inneren, wenn man die Abfrage richtig formuliert. Aber eigentlich stammt es daher, dass ich vorher in diesen Haufen Gold gefallen bin und einige Galleonen in meine Tasche geschoben habe. Jeder, der etwas von Pessimismus versteht, weiß, dass Geld etwas ist, was man im Zweifel schnell und ohne große Vorwarnung braucht. Also sind Sie jetzt zornig auf mich, weil ich mich Ihrer Autorität widersetzt habe? Oder froh, dass wir bei unserer wichtigen Mission erfolgreich waren?"}

\emph{Die Augen des Verkäufers wurden weit wie Untertassen.}

\emph{Und die große Hexe stand dort, still.}

\emph{„Die Disziplin in Hogwarts} \emph{\emph{muss}} \emph{durchgesetzt werden,“ sagte sie nach fast einer vollen Minute. „Um} \emph{\emph{aller}} \emph{Schüler willen. Und dies} \emph{\emph{muss}} \emph{Ihr Entgegenkommen und Ihren Gehorsam gegenüber} \emph{\emph{allen}} \emph{Professoren beinhalten."}

\emph{"Ich verstehe, Professor McGonagall."}

\emph{"Gut. Nun lassen Sie uns diesen Koffer kaufen und nach Hause gehen."}

\emph{Harry wollte sich übergeben oder jubeln oder in Ohnmacht fallen oder} \emph{\emph{irgendwas.}} \emph{Das war das erste mal, dass seine sorgfältige Argumentation bei} \emph{\emph{irgendwem}} \emph{funktioniert hatte. Vielleicht, weil es auch das erste mal war, dass er etwas wirklich bedeutendes hatte, was ein Erwachsener von ihm wollte, aber trotzdem -}

\emph{Minerva McGonagall, +1 Punkt.}

\emph{Harry verbeugte sich und legte das Säckchen voll Gold und die zusätzlichen elf Galleonen in McGonagalls Hände. „Vielen Dank, Professor. Können Sie den Kauf für mich abschließen. Ich muss die Toilette aufsuchen."}

\emph{Der Verkäufer, wieder salbungsvoll, deutete auf eine in die Wand eingelassene Tür mit vergoldetem Knauf. Als Harry anfing, zu gehen, hörte er den Verkäufer mit seiner schmierigen Stimme fragen, „Dürfte ich es wagen, zu fragen, wer das war, Madam McGonagall? Ich gehe davon aus, er ist ein Slytherin - im dritten Schuljahr vielleicht? Und aus einer bekannten Familie, aber ich erkannte nicht -"}

\emph{Das Zuschlagen der Toilettentür schnitt seine Worte ab und nachdem Harry das Schloss erkannt und zugedrückt hatte, ergriff er das magische selbstreinigende Handtuch und wischte mit zittrigen Händen Flüssigkeit von seiner Stirn. Harry ganzer Körper war in Schweiß gebadet, der} \emph{offensichtlich durch seine Muggelkleider gedrungen war, aber zumindest durch seinen Umhang nicht zu sehen.}

Die Sonne ging unter und es war tatsächlich schon sehr spät, als sie wieder im Hinterhof des Tropfenden Kessels standen, der staubigen und von Blättern bedeckten Schnittstelle zwischen der Winkelgasse des magischen Britanniens und der gesamten Muggelwelt. (Das war eine \emph{furchtbar} entkoppelte Wirtschaft...) Harry würde zu einer Telefonzelle gehen und seinen Vater anrufen, sobald er auf der anderen Seite war. Offenbar brauchte er sich nicht darum kümmern, dass ihm sein Gepäck gestohlen werden könnte. Sein Koffer hatte den Status eines bedeutenden magischen Gegenstandes, etwas, was die meisten Muggel nicht bemerken würden; das war Teil dessen, was man in der Zauberwelt bekommen konnte, wenn man willens war, den Preis eines Gebrauchtwagens zu zahlen.

„Hier trennen sich also unsere Wege, eine Zeit lang,“ sagte Professor McGonagall. Sie schüttelte verwundert den Kopf. „Das war der seltsamste Tag meines Lebens für... ein gutes Jahr. Seit dem Tag, an dem ich erfuhr, dass ein Kind Sie-wissen-schon-wen besiegt hatte. Ich frage mich jetzt, zurückblickend, ob das der letzte vernünftige Tag der Welt war."

Oh, als ob \emph{sie} sich über etwas zu beschweren hätte. \emph{Sie denken, Ihr Tag war surreal? Versuchen Sie's mit meinem.}

\emph{„Ich war heute sehr beeindruckt von Ihnen,“ sagte Harry zu ihr. „Ich hätte daran denken sollen, Ihnen das Lob laut zu machen, ich habe Ihnen in meinem Kopf Punkte verliehen und alles."}

\emph{„Danke, Mr. Potter,“ sagte Professor McGonagall. „Wenn Sie bereits einem Haus zugewiesen wären, hätte ich so viele Punkte abgezogen, dass noch Ihre Enkelkinder den Hauspokal verlieren würden."}

\emph{„Ich danke} \emph{\emph{Ihnen,}} \emph{Professor.“ Es war wahrscheinlich zu früh, sie Minnie zu nennen.}

\emph{Diese Frau mochte gute der vernünftigste Erwachsene sein, den Harry je getroffen hatte, trotz ihrem fehlenden wissenschaftlichen Hintergrund. Harry zog sogar in Betracht, ihr die Nummer-Zwei-Position anzubieten, in welcher Gruppierung auch immer, die er bilden würde, um den Dunklen Lord zu bekämpfen, obwohl er nicht dumm genug war, das laut zu sagen.} \emph{\emph{Nun, was wäre ein guter Name dafür...? Die Todesser-Fresser?}}

\emph{„Ich werde Sie bald wiedersehen, wenn die Schule beginnt,“ sagte} \emph{Professor McGonagall. Und, Mr. Potter, wegen Ihres Zauberstabes -"}

\emph{„Ich weiß, was sie sagen wollen,“ sagte Harry. Er holte seinen wertvollen Zauberstab heraus und mit einem tief stechenden inneren Schmerz drehte er ihn in seiner Hand und hielt ihr den Griff hin. „Nehmen Sie ihn. Ich hatte nicht vor etwas zu tun, nicht eine Sache, aber ich möchte nicht, dass Sie Alpträume darüber haben, ich könnte mein Haus in die Luft jagen."}

\emph{Professor McGonagall schüttelte schnell den Kopf. „Oh nein, Mr. Potter! Das wird nicht getan. Ich wollte Sie nur warnen, Ihren Zauberstab zu Hause nicht zu benutzen, weil das Ministerium Magie von Minderjährigen feststellen kann und sie ohne Aufsicht verboten ist."}

\emph{„Ah,“ sagte Harry. „Das klingt nach einer sehr sinnvollen Regel. Ich bin froh, dass die Zauberwelt diese Dinge ernst nimmt."}

\emph{Professor McGonagall blickte ihn prüfend an. „Das meinen Sie wirklich."}

\emph{„Ja,“ sagte Harry. „Ich verstehe es. Magie ist gefährlich und die Regeln gibt es aus gutem Grund. Bestimmte andere Dinge sind auch gefährlich. Ich verstehe auch das. Denken Sie daran, dasss ich nicht dumm bin."}

\emph{"Das werde ich kaum jemals vergessen. Danke, Harry, das gibt mir ein besseres Gefühl dabei, Ihnen bestimmte Dinge anzuvertrauen. Leben Sie wohl, für den Moment."}

\emph{Harry drehte sich um zu gehen, in den Tropfenden Kessel und nach draußen in die Muggelwelt.}

\emph{Als seine Hand den Griff der Hintertür berührte, hörte er ein letztes Flüstern hinter sich.}

\emph{"Hermine Granger."}

\emph{„Was?“ sagte Harry, seine Hand immer noch auf der Tür.}

\emph{"Suchen Sie nach einer Erstklässlerin namens Hermine Granger im Zug nach Hogwarts."\\ "Wer ist sie?"}

\emph{Es gab keine Antwort und als Harry sich umdrehte, war Professor McGonagall fort.}

--------------------------------------------------------------------------------------------------------------------------------------------

\hfill\break \emph{Nachspiel:}

\emph{Schulleiter Albus Dumbledore lehnte sich nach vorn über seinen Schreibtisch. Seine glitzernden Augen betrachteten Minerva. „Also, meine} \emph{Liebe, wie fanden Sie Harry?"}

\emph{Minerva öffnete den Mund. Dann schloss sie ihn. Dann öffnete sie ihn wieder. Keine Worte kamen heraus.}

\emph{„Ich verstehe,“ sagte Albus feierlich. „Danke für Ihren Bericht, Minerva. Sie können gehen."}

* Ich konnte leider keine eindeutige Erklärung dafür finden, was ein \emph{Bafflesnaffle} ist, es scheint in den originalen Harry-Potter-Romanen nicht erwähnt zu werden und klingt, ehrlich gesagt, wie etwas, was Luna Lovegood sich ausgedacht haben würde. Der Versuch eine sinngemäße Übersetzung zu finden, lieferte mir nur das Stichwort \emph{Verwirrung} und Bezüge zu Pferdezaumzeug. Wenn jemand eine Idee hat, immer her damit, ansonsten bleibt es eben erstmal so stehen, da es für den späteren Verlauf der Geschichte auch keine Relevanz zu haben scheint.\\ ** Auch \emph{Kiemenkraut}, eine Pflanze die jemandem Schwimmhäute wachsen und ihn unter Wasser atmen lassen kann.\\ *** Das erwähnte Lied ist eine Anspielung auf die Ideale der amerikanischen Pfadfinder. Die Strophe, die Harry bekannt ist, scheint dazu aufzurufen, gut vorbereitet durchs Leben zu gehen, sodass man sich nicht fürchten oder nervös sein müsse. Der Rest des Liedes setzt dies eher in den Kontext, sich bei Missetaten nicht erwischen zu lassen.

