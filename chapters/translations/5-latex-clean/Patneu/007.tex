

\hypertarget{wechselwirkung}{% \section{7. Wechselwirkung}\label{wechselwirkung}}

\textbf{Kapitel 7: Wechselwirkung}\\

Boah. Ein Sprecher von Rowlings Literaturagent sagte, dass sie mit Fanfiktion einverstanden ist, so lange niemand dafür Geld verlangt und allen klar ist, dass die ursprünglichen Urheberrechte bei ihr liegen? Das ist richtig cool von ihr. Also danke, JKR, und dein sei das Königreich!

--------------------------------------------------------------------------------------------------------------------------------------------

Ich habe das Gefühl, klarstellen zu müssen das bestimmte Teile dieses Kapitels nicht als "Bashing" zu verstehen sind. Es ist nicht so, dass ich einen Groll hege, die Geschichte schreibt sich einfach selbst und sobald man damit anfängt, Ambosse auf einen Charakter fallen zu lassen, ist es schwer aufzuhören.

Ein paar Rezensenten haben gefragt, ob die Wissenschaft in dieser Geschichte echt oder erfunden ist. Ja, sie ist echt und wenn ihr mein Profil anschaut, werdet ihr einen Link zu einer bestimmten nicht-fiktionalen Seite finden, die euch so ziemlich alles beibringen wird, was Harry James Potter-Evans-Verres weiß \emph{und noch einiges mehr.}

Vielen Dank an alle meine Rezensenten. (Besonders an Darkandus von Viridian Dreams für den überraschend inspirierenden Kommentar "Lungen und Tee sollten nicht wechselwirken").

--------------------------------------------------------------------------------------------------------------------------------------------

\emph{"Dein Dad ist fast so super wie mein Dad."}

--------------------------------------------------------------------------------------------------------------------------------------------

Petunia Evans-Verres Lippen bebten und ihre Augen brannten als Harry sie mitten auf Bahnsteig Neun des Bahnhofs King's Cross umarmte. "Bist du sicher, du willst nicht, dass ich mitkomme, Harry?"

Harry blickte kurz hinüber zu seinem Vater Michael Verres-Evans mit seinem stereotypen ernsten-aber-stolzen Ausdruck und dann zurück zu seiner Mutter, die wirklich eher… ungefasst aussah. "Mum, ich weiß, du magst die Zauberwelt nicht besonders. Du musst nicht mitkommen. Ich mein's ernst."

Petunia zuckte zusammen. "Harry, du solltest dich nicht um mich sorgen, ich bin deine Mutter und wenn du jemanden dabei haben musst -"

"Mum, ich werde in Hogwarts \emph{monatelang} allein sein. Wenn ich mit einem Bahnsteig nicht alleine klarkomme, finden wir es besser jetzt heraus als später, so dass wir noch abbrechen können." Er senkte seine Stimme zu einem Flüstern. "Außerdem, Mum, lieben sie mich dort alle. Wenn ich irgendwelche Probleme habe, muss ich nur mein Schweißband abnehmen," er tippte das Übungsband an, das seine Narbe verdeckte, "und ich werde \emph{viel} mehr Hilfe haben, als ich brauchen kann."

"Oh, Harry," flüsterte Petunia. Sie kniete sich hin und umarmte ihn fest, auf Augenhöhe, ihre Wangen nebeneinander gelehnt. Harry konnte ihr abgehacktes Atmen fühlen und dann hörte er sie sich mit einem unterdrückten Schluchzen zurückziehen. "Oh, Harry, ich liebe dich, denk immer daran."

\emph{Es ist, als ob sie Angst hat, mich nie wieder zu sehen,} schoss Harry der Gedanke in den Kopf. Er wusste, dass es wahr war, aber nicht warum Mum solche Angst hatte.

Also wagte er eine Vermutung. "Mum, du weißt, dass ich mich nicht in deine Schwester verwandeln werde, nur weil ich Magie lerne, oder? Ich werde alle Magie wirken, um die du bittest -- wenn ich es kann, meine ich -- oder wenn du \emph{nicht} willst, dass ich im Haus Magie benutze, mache ich das auch, ich verspreche, ich werde die Magie niemals zwischen uns kommen lassen -"

Eine enge Umarmung schnitt ihm das Wort ab. "Du hast ein gutes Herz," flüsterte ihm seine Mutter ins Ohr. "Ein sehr gutes Herz, mein Sohn."

Harry klang daraufhin selbst etwas erstickt.

Seine Mutter gab ihn frei und stand auf. Sie nahm ein Taschentuch aus ihrer Handtasche und tupfte mit bebender Hand an dem verlaufenden Make-up um ihre Augen.

Es war keine Frage, ob sein Vater ihn auf die magische Seite des Bahnhofs King's Cross begleiten würde. Dad hatte Probleme, auch nur Harrys Koffer direkt anzusehen. Magie lag in der Familie und Michael Verres-Evans ging sie vollkommen ab.

Also räusperte sich sein Vater stattdessen nur. "Viel Glück in der Schule, Harry," sagte er. "Denkst du, ich habe dir genug Bücher gekauft?"

Harry hatte seinem Vater erklärt, dass er dies für seine große Chance hielt, etwas wirklich revolutionäres und wichtiges zu tun und Professor Verres-Evans hatte genickt und seinen extrem vollen Terminkalender für zwei volle Tage über den Haufen geworfen, um auf den Größten-Secondhand-Buchladen-Raubzug-aller-Zeiten zu gehen, der vier Städte umfasst und \emph{dreißig} Kartons voller wissenschaftlicher Bücher zu Tage gefördert hatte, die jetzt in der Kelleretage von Harrys Koffer lagen. Die meisten der Bücher hatte es für ein oder zwei Pfund gegeben, aber einige von ihnen definitiv \emph{nicht,} wie das neueste \emph{Handbuch der Chemie und Physik} oder die komplette 1972er Ausgabe der \emph{Encyclopaedia Britannica.} Sein Vater hatte versucht zu verhindern, dass Harry die Preisschilder sah, aber Harry nahm an, sein Vater müsse \emph{mindestens} eintausend Pfund ausgegeben haben. Harry hatte seinem Vater gesagt, er würde es ihm zurückzahlen, sobald er herausfände, wie er Zauberer-Gold in Muggel-Geld konvertieren könne und sein Vater hatte ihm gesagt, er solle hingehen, wo der Pfeffer wächst.

Und dann hatte sein Vater ihn gefragt: \emph{Denkst du, ich habe dir genug Bücher gekauft?} Es war vollkommen klar, welche Antwort Dad hören wollte.

Harrys Kehle war aus irgendeinem Grund rau. "Man kann niemals genug Bücher haben," rezitierte er das Verres-Familienmotto und sein Vater kniete nieder und umarmte ihn schnell und fest. "Aber du hast es \emph{definitiv} versucht," sagte Harry und fühlte, wieder einen Kloß im Hals. "Es war ein wirklich, wirklich, \emph{wirklich} guter Versuch."

Sein Dad richtete sich auf. "Also…" sagte er. "Siehst \emph{du} ein Gleis Neundreiviertel?"

Der Bahnhof King's Cross war groß und geschäftig, die Wände und Böden belegt mit gewöhnlichen, verschmutzten Kacheln. Er war voller gewöhnlicher Leute, die zu ihren gewöhnlichen Beschäftigungen eilten, gewöhnliche Unterhaltungen führten, die sehr viel gewöhnlichen Lärm erzeugten. Der Bahnhof King's Cross hatte ein Gleis Neun (bei dem sie standen) und ein Gleis Zehn (genau daneben) aber es gab nichts zwischen Gleis Neun und Zehn außer einer schmalen, wenig vielversprechenden Absperrung. Ein großes Oberlicht über ihren Köpfen ließ viel Licht herein, welches das vollkommene Fehlen jedwedes Gleises Neundreiviertel beleuchtete.

Harry starrte umher, bis sein Augen tränten, dachte \emph{komm schon Magier-Blick, komm schon Magier-Blick,} aber absolut nichts fiel ihm auf. Er dachte daran, seinen Zauberstab herauszuholen und zu schwingen, aber Professor McGonagall hatte ihn davor gewarnt, seinen Zauberstab zu benutzen. Außerdem, sollte es wieder einen Regen mehrfarbiger Funken geben, könnte das zu einer Verhaftung wegen Abschießens von Feuerwerk in einem Bahnhof führen. Und das nur unter der Annahme sein Zauberstab entschied sich nicht, etwas anderes zu tun, wie etwa ganz King's Cross in die Luft zu jagen. Harry hatte seine Schulbücher nur kurz überflogen (obwohl das schon bizarr genug gewesen war) in einem sehr schnellen Versuch zu bestimmen, welche Wissenschafts-Bücher er in den nächsten 48 Stunden kaufen sollte.

Nun, er hatte -- Harry sah auf seine Uhr -- eine ganze Stunde, um es herauszukriegen, weil er um elf Uhr im Zug sein sollte. Vielleicht war das die Entsprechung eines IQ-Tests und die dummen Kinder konnten keine Zauberer werden. (Und die Dauer der Extra-Zeit, die man sich dafür gab, würde die eigene Gewissenhaftigkeit bestimmen, die der zweit-wichtigste Faktor für schulischen Erfolg war.)

"Ich kriege es schon raus," sagte Harry seinen wartenden Eltern. "Es ist wahrscheinlich eine Art Test-Dingens."

Sein Vater runzelte die Stirn. "Hm… halte vielleicht nach einer Spur durcheinander laufender Fußabdrücke Ausschau, die in eine Richtung führen, die keinen Sinn zu machen scheint -"

"\emph{Dad!}" sagte Harry. "Hör' auf damit! Ich habe noch nicht mal \emph{versucht} es selbst herauszufinden!" Es war auch noch ein sehr guter Vorschlag, was noch schlimmer war.

"Tut mir leid," entschuldigte sich sein Vater.

"Ah…" sagte seine Mutter. "Ich denke nicht, dass sie das mit einem Schüler machen würden, oder? Bist du sicher, Professor McGonagall hat dir nichts gesagt?"

"Vielleicht war sie abgelenkt," sagte Harry ohne nachzudenken.

"\emph{Harry!}" zischten sein Vater und seine Mutter im Einklang. "\emph{Was hast du gemacht?}"

"Ich, ähm -" Harry schluckte. "Seht mal, wir haben dafür jetzt keine Zeit -"

"\emph{Harry!}"

"Ich mein's ernst! Wir haben jetzt keine Zeit dafür! Weil es eine sehr lange Geschichte ist und ich rausfinden muss, wie ich zur Schule komme!"

Seine Mutter hatte eine Hand über'm Gesicht. "Wie schlimm war es?"

"Ich, ah," \emph{ich kann nicht darüber reden, aus Gründen der nationalen Sicherheit,} "etwa halb so schlimm, wie der Vorfall mit dem Wissenschaftsprojekt?"

"\emph{Harry!}"

"Ich, äh, oh seht mal, da sind ein paar Leute mit einer Eule, ich werde sie fragen gehen, wie ich rein komme!" und Harry rannte weg von seinen Eltern und auf die Familie feuriger Rotschöpfe zu, sein Koffer automatisch hinter ihm her schlitternd.

Die füllige Frau sah ihn an, als er ankam. "Hallo, Lieber. Das erste mal in Hogwarts? Ron ist auch neu -" und dann blickte sie ihn prüfend an. "\emph{Harry Potter?}"

Vier Jungs und ein rothaariges Mädchen und eine Eule drehten sich alle herum und erstarrten.

"Oh, \emph{kommt schon!}" protestierte Harry. Er hatte geplant als Harry Verres durchzugehen, zumindest bis er in Hogwarts ankam. "Ich habe ein Schweißband gekauft und alles! Woher wisst ihr, wer ich bin?"

"Ja," sagte Harrys Vater, der mit langen, mühelosen Schritten hinter ihm herkam, "woher \emph{wissen} Sie, wer er ist?" Seine Stimme deutete eine bestimmte Befürchtung an.

"Dein Bild war in der Zeitung," sagte einer der zwei identisch-aussehenden Zwillinge.\\ "\emph{HARRY!}"

"\emph{Dad!} So ist das nicht! Es ist, weil ich den Dunklen Lord Du-weißt-schon-wer besiegt habe, als ich ein Jahr alt war!"

"\emph{WAS?}"

"Mum, ich kann's erklären."

"\emph{WAS?}"

"Ah… Michael Liebling, es gibt bestimmte Sachen, von denen ich dachte, es wäre das beste, dich damit bis jetzt nicht zu beunruhigen -"

"Entschuldigt," sagte Harry zu der rotschöpfigen Familie, die ihn alle anstarrten, "aber es wäre ganz extrem hilfreich, wenn ihr mir sagen könntet, wie ich nach Gleis Neundreiviertel komme, \emph{jetzt sofort.}"

"Ah…" sagte die Frau. Sie hob eine Hand und deutete auf die Wand zwischen den Gleisen. "Lauf einfach genau auf die Absperrung zwischen den Gleisen neun und zehn zu. Halte nicht an und hab' keine Angst, dass du in sie reinkrachen könntest, das ist sehr wichtig. Renn' lieber ein bisschen, wenn du nervös bist."

"Und was immer du machst, denk nicht an einen Elefanten."

"\emph{George!} Ignorier ihn, Harry Lieber, es gibt keinen Grund, nicht an einen Elefanten zu denken."

"Ich bin Fred, Mum, nicht George -"

"Danke!" sagte Harry und setzte zu einem Lauf in Richtung der Absperrung an -

Moment mal, es würde nicht funktionieren, \emph{wenn er nicht daran glaubte?}

Es waren solche Momente, in denen Harry seinen Geist dafür hasste, tatsächlich schnell genug zu sein, um zu merken, dass dies ein Fall war, in dem "resonanter Zweifel"* anzuwenden war; nämlich, wenn er von Anfang an gedacht hätte, er würde es durch die Absperrung schaffen, wäre alles gut gewesen, nur jetzt war er besorgt darüber, ob er ausreichend \emph{glaubte} durch die Absperrung zu kommen, was bedeutete, dass er sich tatsächlich Sorgen darum \emph{machte} in sie reinzukrachen -

"\emph{Harry! Komm wieder her, du hast einiges zu erklären!}" Das war sein Dad.

Harry schloss die Augen und ignorierte alles, was er über gerechtfertigte Glaubwürdigkeit wusste und versuchte nur \emph{wirklich stark} zu glauben, dass er durch die Absperrung kommen würde und -

- die Geräusche um ihn veränderten sich.

Harry öffnete die Augen und kam stolpernd zum Stehen, sich auf diffuse Weise beschmutzt fühlend, von dem vorsätzlichen Versuch etwas zu glauben.

Er stand auf einem hellen Freiluft-Bahnsteig, neben einem einzelnen großen Zug, vierzehn lange Waggons aufgereiht hinter einer massiven Dampflokomotive aus scharlachrotem Metall mit einem großen Schornstein, der für die Luftqualität todversprechend wirkte. Der Bahnsteig war bereits leicht bevölkert (obwohl Harry eine ganze Stunde zu früh war); dutzende Kinder und ihre Eltern schwärmten um die Bänke, Tische und verschiedenen Straßenhändler und Stände.

Es brauchte nicht extra erwähnt zu werden, dass es keinen solchen Ort im Bahnhof King's Cross gab und keinen Platz, um ihn zu verstecken.

\emph{Okay, also entweder (a) ich bin gerade komplett woanders hinteleportiert worden (b) sie können den Raum falten, wie niemand sonst oder (c) sie ignorieren einfach alle Regeln.}

Es gab ein schlitterndes Geräusch hinter ihm und Harry drehte sich um, um zu beobachten, dass sein Koffer ihm tatsächlich auf seinen kleinen klauenbewehrten Tentakeln gefolgt war. Offensichtlich hatte sein Koffer es, für magische Zwecke, auch geschafft, ausreichend stark zu glauben, um durch die Absperrung zu kommen. Wenn Harry darüber nachdachte, war das eigentlich ein wenig verstörend.

Einen Moment später kam der am jüngsten aussehende rothaarige Junge durch den eisernen Torbogen (eiserner Torbogen?) gerannt, zog seinen Koffer an einer Leine hinter sich her und lief fast in Harry hinein. Harry, der sich dumm fühlte, weil er stehen geblieben war, begann sich schnell von der Landezone wegzubewegen und der rothaarige Junge folgte ihm, heftig an der Leine seines Koffers zerrend, um Schritt zu halten. Einen Augenblick später kam eine weiße Eule durch den Torbogen geflattert und ließ sich auf der Schulter des Jungen nieder.

"Mein Gott," sagte der rothaarige Junge, "bist du \emph{wirklich} Harry Potter?"

\emph{Nicht das schon wieder.} "Ich habe keine logische Möglichkeit, um das sicher zu wissen. Meine Eltern haben mich dazu erzogen, zu \emph{glauben,} dass mein Name Harry James Potter-Evans-Verres lautet und viele Leute hier haben mir erzählt, dass ich wie meine Eltern \emph{aussehe,} ich meine meine anderen Eltern, aber," Harry runzelte die Stirn als es ihm klar wurde, "nach allem, was ich weiß, wäre es gut möglich, dass es Zauber gibt, um ein Kind in eine bestimmte Erscheinung umzuwandeln -"

"Äh, was, Kumpel?"

\emph{Nicht für Ravenclaw bestimmt, nehme ich an.} "Ja, ich bin Harry Potter."

"Ich bin Ron Weasley," sagte das große, dürre, sommersprossige, langnasige Kind und streckte eine Hand aus, die Harry höflich schüttelte, als sie weitergingen. Die Eule warf Harry ein seltsam angemessenes und höfliches Huhen zu (eigentlich mehr ein iihhhhh-Geräusch, was Harry überraschte).

An diesem Punkt wurde Harry das Potential einer drohenden Katastrophe klar. "Nur eine Sekunde," sagte er zu Ron und öffnete einen der Schieber seines Koffers, denjenigen, der, wenn er sich richtig erinnerte, für Winterklamotten vorgesehen war -- er war es -- und dann fand er den dünnsten Schal den er hatte, unter seinem Wintermantel. Harry nahm sein Schweißband ab und entfaltete genau so schnell seinen Schal und band ihn sich ums Gesicht. Es war ein bisschen heiß, besonders im Sommer, aber damit konnte Harry leben.

Dann schloss er diesen Schieber und zog eine andere Schublade auf und zog einen schwarzen Zauberer-Umhang hervor, den er über den Kopf streifte, jetzt wo er nicht mehr auf Muggelgebiet war.

"So," sagte Harry. Das Geräusch kam leicht gedämpft durch den Schal über seinem Gesicht. Er drehte sich zu Ron. "Wie sehe ich aus? Dämlich, ich weiß, aber bin ich als Harry Potter zu erkennen?"

"Äh," sagte Ron. Er schloss seinen Mund, der sich geöffnet hatte. "Nicht wirklich, Harry."

"Sehr gut," sagte Harry. "Aber, um den Sinn der ganzen Übung nicht zunichte zu machen, wirst du mich fortan ansprechen als," Verres könnte nicht mehr funktionieren, "Mr. Spoo."

"Okay, Harry," sagte Ron unsicher.

\emph{Die Macht ist nicht besonders stark in diesem da.} "Nenn… mich… Mister… Spoo."

"Okay, Mister Spoo -" Ron hielt inne. "Ich kann das nicht machen, da fühle ich mich dämlich."

\emph{Das ist nicht nur ein Gefühl.} "Okay. Such \emph{du} einen Namen aus."

"Mr. Cannon," sagte Ron sofort. "Nach den Chudley Cannons."

"Ah…" Harry wusste, er würde es schrecklich bereuen, das gefragt zu haben. "Wer oder was sind die Chudley Cannons?"

"\emph{Wer sind die Chudley Cannons?} Nur das brillianteste Team in der gesamten Geschichte des Quidditch! Sicher, sie waren am Ende auf dem letzten Platz der Liga letztes Jahr, aber-"

"Was ist Quidditch?"

Das zu fragen, war ebenfalls ein Fehler.

"Also, nur damit ich das richtig verstehe," sagte Harry als es schien, dass Rons (von Handgesten begleitete) Erklärung zum Ende kam. "Den Schnatz zu fangen, ist \emph{einhundertfünfzig Punkte} wert?"

"Ja -"

"Wie viele Zehn-Punkte-Tore macht eine Seite normalerweise, den Schnatz \emph{nicht} eingeschlossen?"

"Ähm, vielleicht fünfzehn oder zwanzig in Profi-Spielen -"

"Das ist einfach falsch. Das verletzt jede mögliche Regel des Spieldesigns. Sieh mal, der Rest des Spiels klingt als könnte er Sinn machen, irgendwie, für einen Sport meine ich, aber im Grunde sagst du, dass den Schnatz zu fangen fast jeden normal erzielten Punkt des Spiels zunichte macht. Die beiden Sucher fliegen da oben rum, suchen nach dem Schnatz und interagieren üblicherweise mit niemandem sonst; den Schnatz zu fangen, wird größtenteils Glück sein -"

"Es ist kein Glück!" protestierte Ron. "Du musst deine Augen im richtigen Muster bewegen -"

"Das ist nicht \emph{interaktiv,} es gibt kein Hin-und-Her mit dem anderen Spieler und wie viel Spaß macht es, jemandem zuzusehen, der unglaublich gut darin ist, seine Augen zu bewegen? Und dann stürzt der erste Sucher, der Glück hat, herab, greift sich den Schnatz und macht den Aufwand aller anderen irrelevant. Das ist, als ob jemand ein echtes Spiel genommen und dann diese sinnlose Extra-Position oben drauf gepfropft hätte, so dass man der Aller-Wichtigste-Spieler sein kann, ohne wirklich mitmachen oder den Rest des Spiels lernen zu müssen. Wer war der erste Sucher, der Idioten-Sohn des Königs, der Quidditch spielen wollte, aber die Regeln nicht verstehen konnte?" Eigentlich, jetzt wo Harry darüber nachdachte, klang das nach einer überraschend guten Hypothese. Setz ihn auf einen Besenstiel und sag ihm, fang das glänzende Ding…

Rons Gesicht verfinsterte sich. "Wenn du Quidditch nicht magst, musst du dich nicht drüber lustig machen!"

"Wenn man nicht kritisieren kann, kann man nicht optimieren. Ich schlage vor, wie man \emph{das Spiel verbessert.} Und es ist wirklich einfach. Werdet den Schnatz los."

"Sie werden das Spiel nicht ändern, nur weil \emph{du} das sagst!"

"Ich \emph{bin} der Junge-der-überlebt-hat, wie du weißt. Die Leute werden auf mich hören. Und vielleicht, wenn ich sie dazu bringen kann, das Spiel in Hogwarts zu verändern, wird sich die Neuerung verbreiten."

Ein Ausdruck absoluten Horrors breitete sich auf Rons Gesicht aus. "Aber, aber wenn man den Schnatz loswird, wie wird irgendjemand wissen, wann das Spiel endet?"

"\emph{Kauft… eine… Uhr.} Es wäre sehr viel fairer, als wenn das Spiel manchmal nach zehn Minuten endet und manchmal für Stunden nicht und der Zeitrahmen wäre auch viel vorhersehbarer für die Zuschauer." Harry seufzte. "Oh, hör schon auf, mich so vollkommen entsetzt anzusehen, ich werde mir wahrscheinlich nicht \emph{wirklich} die Zeit nehmen, diesen armseligen Ersatz für einen Nationalsport zu zerstören und ihn nach meinem Bilde stärker und besser neu zu formen. Ich habe viel, viel, \emph{viel} wichtigere Dinge, um die ich mir Gedanken machen muss." Harry sah nachdenklich aus. "Aber es würde auch wieder nicht viel Zeit \emph{brauchen}, die Fünundneunzig Thesen der Schnatzlosen Reformation aufzuschreiben und an eine Kirchentür zu nageln -"

"Potter," erklang die gedehnte Stimme eines kleinen Jungen, "\emph{was} ist das auf deinem Gesicht und \emph{was} steht da neben dir?"

Rons entsetzter Gesichtsausdruck wich purem Hass. "\emph{Du!}"

Harry drehte den Kopf und tatsächlich war es Draco Malfoy, der vielleicht gezwungen war, einen Standard-Schulumhang zu tragen, aber das mit einem Koffer wettmachte, der mindestens genau so magisch und viel eleganter aussah, als Harrys eigener, dekoriert mit Silber und Emeralden und das trug, was Harry für das Malfoy-Familienwappen hielt, eine wunderschöne Schlange mit Fangzähnen über gekreuzten eichenen Zauberstäben.

"Draco!" sagte Harry. "Äh, oder Malfoy, wenn du magst, obwohl das für mich ein bisschen nach Lucius klingt. Es freut mich zu sehen, dass es dir so gut geht, nach, ähm, unserem letzten Treffen. Das ist Ron Weasley. Und ich versuche inkognito zu bleiben, also nenn mich, äh," Harry sah an seinem Umhang hinunter, "Mister Black."

"\emph{Harry!}" zischte Ron. "\emph{Diesen} Namen kannst du nicht benutzen!"

Harry blinzelte. "Warum nicht?" Es \emph{klang} schön dunkel, wie ein mysteriöser Mann von Welt -

"Ich würde sagen, es ist ein \emph{guter} Name," sagte Draco, "aber er gehört dem Noblen und Uralten Haus der Blacks. Ich werde dich Mr. Silver nennen."

"\emph{Du} hältst dich fern von… von Mr. Gold," sagte Ron eisig und machte einen Schritt nach vorn. "Er hat es nicht nötig, mit Deinesgleichen zu sprechen!"

Harry hob beschwichtigend eine Hand. "Mir reicht Mr. Bronze, danke für das Namens-Schema. Und, Ron, ähm," Harry rang darum, wie er das ausdrücken sollte, "Ich finde es toll, dass du so… enthusiastisch dabei bist, mich zu verteidigen, aber ich habe nicht wirklich etwas dagegen, mit Draco zu sprechen -"

Das war anscheinend der letzte Strohhalm für Ron, der zu Harry herumwirbelte, mit Augen flammend vor Empörung. "\emph{Was? Weißt} du, wer das ist?"

"Ja, Ron," sagte Harry, "du erinnerst dich vielleicht, dass ich ihn Draco genannt habe, ohne dass er sich vorstellen musste."

Draco kicherte. Dann fielen seine Augen auf die weiße Eule auf Rons Schulter. "Oh, was ist \emph{das?}" sagte Draco gedehnt und voller Häme. "Wo ist die berühmte Weasley-Familienratte?"

"Im Garten vergraben," sagte Ron kalt.

"Och, wie traurig. Pot… äh, Mr. Bronze, ich sollte erwähnen, dass man weithin der Meinung ist, die Weasley-Familie habe \emph{die beste Haustier-Geschichte aller Zeiten.} Willst du sie erzählen, Weasley?"

Rons Gesicht verzerrte sich. "Du würdest es nicht lustig finden, wenn es \emph{deiner} Familie passiert wäre!"

"Oh," säuselte Draco, "aber es würde den Malfoys nie \emph{passieren.}"

Rons Hände ballten sich zu Fäusten -

"Das ist genug," sagte Harry und legte so viel ruhige Autorität in seine Stimme wie möglich. Es war klar, dass, worum auch immer es dabei ging, es eine schmerzliche Erinnerung für das rothaarige Kind war. "Wenn Ron nicht darüber reden will, muss er nicht darüber reden und ich würde darum bitten, dass du auch nicht darüber redest."

Draco warf Harry einen überraschten Blick zu und Ron nickte. "Das stimmt, Harry! Ich meine Mr. Bronze! Siehst du was für ein Mensch er ist? Jetzt sag ihm, er soll verschwinden!"

Harry zählte im Geiste bis zehn, was für ihn ein sehr schnelles \emph{12345678910} war -- eine merkwürdige Gewohnheit, aus dem Alter von fünf übriggeblieben, als seine Mutter ihm das erste mal geraten hatte, es zu tun und Harry war zu dem Schluss gekommen, dass sein Weg schneller war und genau so effektiv sein sollte. "Ich werde ihm nicht sagen, dass er verschwinden soll," sagte Harry ruhig. "Er darf gern mit mir sprechen, wenn er will."

"Nun, ich hänge nicht mit jemandem rum, der mit Draco Malfoy rumhängt," verkündete Ron kalt.

Harry zuckte mit den Achseln. "Das liegt bei dir. \emph{Ich} habe nicht die Absicht, mir von irgendwem sagen zu lassen, mit wem ich rumhängen kann oder nicht." Im Stillen wiederholend, \emph{bitte geh weg, bitte geh weg…}

Ron blickte verblüfft, als hätte er tatsächlich geglaubt, dieser Satz würde funktionieren. Dann wirbelte Ron herum, zerrte an der Leine seines Gepäcks und stürmte den Bahnsteig hinunter.

"Wenn du ihn nicht mochtest," sagte Draco neugierig, "warum bist du nicht einfach weggegangen?"

"Ähm… seine Mutter hat mir geholfen, herauszufinden, wie ich vom Bahnhof King's Cross aus auf diesen Bahnsteig komme, deshalb war es etwas schwierig ihm zu sagen, er solle sich verdrücken. Und es ist nicht so, dass ich diesen Ron-Typen \emph{hasse}," sagte Harry, "Es ist nur, nur…" Harry suchte nach Worten.

"Dass du keinen Grund siehst, warum er existieren sollte?" bot Draco an.

"Trifft es ziemlich gut."

"Jedenfalls, Potter… wenn du wirklich von Muggeln aufgezogen wurdest -" Draco hielt an dieser Stelle inne, als warte er auf ein Dementi, aber Harry sagte nichts "- dann weißt du vielleicht nicht, wie es ist, berühmt zu sein. Die Leute wollen \emph{all} deine Zeit beanspruchen. Du \emph{musst} lernen, nein zu sagen."

Harry nickte und setzte ein nachdenkliches Gesicht auf. "Das klingt nach einem guten Rat."

"Wenn du versuchst, nett zu sein, verbringst du am Ende nur die meiste Zeit mit den Aufdringlichsten. Entscheide, mit wem du deine Zeit verbringen \emph{willst} und lass alle anderen verschwinden. Du kommst gerade erst an, Potter, also werden dich alle danach beurteilen, mit wem sie dich sehen und du willst nicht mit Leuten wie Ron Weasley gesehen werden."

Harry nickte wieder. "Wenn ich fragen darf, wie hast du mich erkannt?"

"\emph{Mister Bronze,}" sagte Draco gedehnt, "Ich \emph{habe} dich schon getroffen, weißt du. Ich habe jemanden mit einem Schal um seinen Kopf herumlaufen sehen. Also habe ich \emph{geraten.}"

Harry beugte sein Haupt und akzeptierte das Kompliment. "Das tut mir \emph{schrecklich} leid," sagte Harry. "Unser erstes Treffen meine ich. Ich wollte dich vor Lucius nicht in eine peinliche Lage bringen."

Draco winkte ab, während er Harry seltsam ansah. "Ich wünschte nur, Vater wäre hereingekommen, als \emph{du} dich bei \emph{mir} eingeschmeichelt hast -" Draco lachte. "Aber danke \emph{dir,} für das, was du zu Vater gesagt hast. Wenn das nicht gewesen wäre, wäre es schwerer gewesen, das zu erklären."

Harry verbeugte sich tiefer. "Und danke \emph{dir,} dass du Professor McGonagall gegenüber mitgespielt hast."

"Keine Ursache. Obwohl eine der Assistentinnen ihre beste Freundin zu absoluter Verschwiegenheit verpflichtet haben muss, weil Vater sagt, dass merkwürdige Gerüchte umgehen, als hätten du und ich eine Auseinandersetzung gehabt oder sowas."

"Autsch," sagte Harry, zusammenzuckend. "Es tut mir \emph{wirklich} leid -"

"Nein, wir sind dran gewöhnt, Merlin weiß es gibt schon viele Gerüchte über die Familie Malfoy."

Harry nickte. "Ich bin froh zu hören, dass du nicht in Schwierigkeiten steckst."

Draco schmunzelte. "Vater hat, ähm, einen \emph{ausgesuchten} Sinn für Humor, aber er \emph{versteht,} wie man sich Freunde macht. Er versteht es \emph{sehr} gut. Er hat es mich im letzten Monat jede Nacht vor dem Schlafengehen wiederholen lassen, 'Ich werde mir in Hogwarts Freunde machen.' Als ich ihm alles erklärte und er verstand, dass es das war, was ich tat, hat er mir ein Eis gekauft."

Harrys Kinnlade fiel runter. "\emph{Du hast es geschafft, daraus ein Eis zu machen?}"

Draco nickte und sah ganz genau so selbstzufrieden aus, wie es dem Kunststück gebührte. "Nun, Vater \emph{wusste} natürlich, was ich tat, aber er ist derjenige, der mir beigebracht hat, \emph{wie} man es tut und wenn ich auf die richtige Art grinse, \emph{während} ich es tue, macht es das zu einem Vater-Sohn-Ding und dann \emph{muss} er mir ein Eis kaufen oder ich sehe ihn auf diese Art traurig an, als ob ich denke, ich müsse ihn enttäuscht haben."

Harry beäugte Draco berechnend, die Präsenz eines weiteren Meisters wahrnehmend. "Du hast \emph{Unterricht} darin gehabt, wie man Leute manipuliert?"

"Natürlich," sagte Draco stolz. "Ich bin ein \emph{Malfoy.} Vater hat mir Privatlehrer gekauft."

"Wow, " sagte Harry. Robert Cialdinis \emph{Einfluss: Wie und warum sich Menschen überzeugen lassen}** gelesen zu haben, war wahrscheinlich nicht viel verglichen damit (obwohl es immer noch ein Hammer von einem Buch war). "Dein Dad ist fast so super wie mein Dad."

Dracos Augenbrauen hoben sich. "Oh? Und was macht \emph{dein} Vater?"

"Er kauft mir Bücher."

Draco dachte darüber nach. "Das klingt nicht sehr beeindruckend."

"Du hättest dabei sein müssen. Auf jeden Fall bin ich froh, das alles zu hören. So wie Lucius dich angesehen hat, dachte ich, er würde dich k-kreuzigen."

"Mein Vater liebt mich wirklich," sagte Draco fest. "Das würde er niemals tun."

"Ähm…" sagte Harry. Er rief sich die weißhaarige, elegante Gestalt im schwarzen Umhang in Erinnerung, die mit diesem wunderschönen, tödlichen Stock mit Silbergriff durch Madam Malkins Laden gestürmt war. Es war nicht einfach, ihn sich als liebenden Vater vorzustellen. "Nimm das jetzt nicht falsch auf, aber woher \emph{weißt} du das?"

"Häh?" Es war klar, dass das eine Frage war, die Draco sich nicht regelmäßig stellte.

"Ich stelle die fundamentale Frage der Rationalität: Warum glaubst du, was du glaubst? Was denkst du, weißt du und woher denkst du, weißt du es? Was lässt dich glauben, Lucius würde dich nicht opfern, so wie er alles andere für Macht opfern würde?"

Draco warf Harry einen weiteren seltsamen Blick zu. "Was weißt \emph{du} denn über Vater?"

"Ähm… Sitz im Zaubergamot, Sitz im Schulbeirat von Hogwarts, unglaublich reich, hat das Ohr von Minister Fudge, hat das Vertrauen von Minister Fudge, hat wahrscheinlich ein paar höchst peinliche Fotos von Minister Fudge, bekanntester Blutreinheits-Verfechter nachdem der Dunkle Lord fort ist, ehemaliger Todesser, bei dem das Dunkle Mal gefunden wurde, kam aber davon mit der Behauptung, er hätte unter dem Imperius-Fluch gestanden, was lächerlich unglaubwürdig war und so ziemlich jeder wusste es… böse mit großem 'B' und ein geborener Killer… ich denke, das war's so weit."

Dracos Augen hatten sich zu Schlitzen verengt. "McGonagall hat dir das erzählt, nicht."

"Nein, sie wollte hinterher \emph{gar nichts} über Lucius sagen, außer dass ich mich von ihm fernhalten soll. Also habe ich mir während des Zwischenfalls im Zaubertrank-Laden, während Professor McGonagall damit beschäftigt war, den Ladenbesitzer anzuschreien und alles unter Kontrolle zu bekommen, einen der Kunden geschnappt und \emph{ihn} über Lucius ausgefragt."

Dracos Augen weiteten sich wieder. "Hast du das \emph{wirklich?}"

Harry sah Draco verwirrt an. "Wenn ich beim ersten mal gelogen habe, werde ich dir nicht die Wahrheit sagen, weil du zweimal fragst."

Es gab eine kurze Pause als Draco das aufnahm.

"Du wirst so dermaßen nach Slytherin kommen."

"Ich werde so dermaßen nach Ravenclaw kommen, vielen Dank. Ich will nur Macht, damit ich Bücher kriege."

Draco kicherte. "Ja, klar. Wie auch immer… um deine Frage zu beantworten…" Draco atmete tief ein und sein Gesicht wurde ernst. "Vater hat mal eine Abstimmung im Zaubergamot für mich verpasst. Ich saß auf einem Besen und bin runtergefallen und habe mir viele Rippen gebrochen. Es tat wirklich weh. Ich hatte mir noch nie so weh getan und dachte, ich würde sterben. Also verpasste Vater diese wirklich wichtige Abstimmung, weil er dort an meinem Bett im St. Mungo's war, meine Hände hielt und mir versprach, dass ich wieder in Ordnung kommen würde."

Harry sah unangenehm berührt weg, dann, mit einigem Aufwand, zwang er sich wieder Draco anzusehen. "Warum erzählst du mir \emph{das?} Es scheint irgendwie… privat…"

Draco warf Harry einen ernsten Blick zu. "Einer meiner Privatlehrer sagte einmal, dass Leute enge Freundschaften schließen, indem sie private Dinge übereinander wissen und der Grund, warum die meisten Leute keine engen Freunde finden, ist, weil es ihnen zu peinlich ist, irgendetwas wirklich wichtiges über sich selbst mitzuteilen." Draco drehte auffordernd die Hände nach außen. "Dein Zug?"

Zu wissen, dass Dracos hoffnungsvolles Gesicht ihm wahrscheinlich in monatelanger Übung eingetrichert worden war, machte es nicht weniger effektiv, wie Harry feststellte. Tatsächlich \emph{machte} es das \emph{weniger} effektiv, aber unglücklicherweise nicht \emph{ineffektiv.} Das selbe konnte man von Dracos cleverem Einsatz des Drucks zur Erwiderung eines unaufgeforderten Geschenks sagen, eine Technik, über die Harry in seinen Büchern über Sozialpsychologie gelesen hatte (ein Experiment hatte gezeigt, dass ein unaufgefordertes Geschenk von 5 \$ zweimal so effektiv war, wie ein bedingtes Angebot von 50 \$, wenn es darum ging, Leute zum Ausfüllen von Umfragen zu bewegen). Draco hatte aus seinem Vertrauen ein unaufgefordertes Geschenk gemacht und lud Harry nun ein, das Vertrauen zu erwidern… und die Sache war die, das Harry sich unter Druck gesetzt \emph{fühlte.} Verneinung, war Harry sich sicher, würde mit einem Blick trauriger Enttäuschung begegnet werden und vielleicht einer kleinen Spur Missbilligung, um anzuzeigen, dass Harry Punkte verloren hatte.

"Draco," sagte Harry, "nur damit du's weißt, ich erkenne genau, was du gerade machst. Meine eigenen Bücher nannten es \emph{Wechselwirkung} und sie sagten etwas darüber, wie jemandem ein direktes Geschenk von 2 Sickeln zu machen zweimal so effektiv ist, wie ihm zwanzig Sickel anzubieten, um ihn dazu zu bringen, zu tun, was du willst…" ließ Harry ausklingen.

Draco sah traurig und enttäuscht aus. "Es ist nicht als Trick gemeint, Harry. Es ist ein echter Weg, um Freunde zu werden."

Harry hob eine Hand. "Ich sagte nicht, ich würde nicht antworten. Ich brauche nur Zeit um etwas auszuwählen, was privat, aber genau so unschädlich ist. Sagen wir… ich wollte dich wissen lassen, dass man mich nicht zu Dingen drängen kann." Eine Pause zum Überdenken konnte einiges bewirken, um vielen Zustimmungs-Techniken etwas an Macht zu nehmen, sobald man lernte, sie als das zu erkennen, was sie waren.

"Alles klar," sagte Draco. "Ich warte, während du dir was überlegst. Oh, und bitte nimm den Schal ab, während du es sagst."

\emph{Einfach, aber effektiv.}

Und Harry konnte nicht umhin, zu bemerken, wie tölpelhaft, ungelenk und wenig elegant sein Versuch, der Manipulation zu widerstehen / sein Gesicht zu wahren / anzugeben, im Vergleich zu Draco gewirkt hatte. \emph{Ich brauche diese Privatlehrer.}

"Alles klar," sagte Harry nach einer Weile. "Hier ist meins." Er blickte sich um und rollte dann den Schal wieder sein Gesicht hinauf, alles außer der Narbe entblößend. "Ähm… es klingt, als ob du dich wirklich auf deinen Vater verlassen kannst. Ich meine… wenn du ernsthaft mit ihm redest, wird er dir immer zuhören und dich ernstnehmen."

Draco nickte.

"Manchmal," sagte Harry und schluckte. Das war überraschend schwierig, aber dann wieder sollte es das auch sein. "Manchmal wünschte ich, mein Dad wäre wie deiner." Harrys Augen wichen vor Dracos Gesicht zurück, mehr oder weniger automatisch, dann zwang sich Harry, Draco wieder anzusehen.

Dann traf es Harry \emph{was zur Hölle er gerade gesagt hatte} und Harry fügte hastig hinzu, "Nicht, dass ich mir wünsche, mein Dad wäre ein makelloses Werkzeug des Todes, wie Lucius, ich meine nur, mich ernst zu nehmen -"

"Ich verstehe," sagte Draco mit einem Lächeln. "So… fühlt es sich nicht an, als ob wir etwas näher dran sind, Freunde zu werden?"

Harry nickte. "Ja. Tut es eigentlich schon. Ähm… nichts für ungut, aber ich werde meine Verkleidung wieder anlegen, ich will mich \emph{wirklich} nicht rumschlagen mit -"

"Ich verstehe."

Harry rollte den Schal wieder runter über sein Gesicht.

"Mein Vater nimmt alle seine Freunde ernst," sagte Draco. "Deswegen hat er viele Freunde. Du solltest ihn treffen."

"Ich werde drüber nachdenken," sagte Harry in neutralem Tonfall. Er schüttelte verwundert den Kopf. "Also bist du wirklich sein einer Schwachpunkt. Huh."

Jetzt warf Draco Harry einen \emph{wirklich} seltsamen Blick zu. "Wollen wir uns was zu trinken holen und uns irgendwo hinsetzen?"

Harry stellte fest, dass er zu lange an einem Ort gestanden hatte und streckte sich, versuchte seinen Rücken zu strecken. "Sicher."

Der Bahnsteig begann sich jetzt zu füllen, aber es gab immer noch einen ruhigeren Bereich auf der von der roten Dampflokomotive entfernten Seite. Auf dem Weg kamen sie an einem Stand mit einem kahlen, bärtigen Mann vorbei, der Zeitungen und Comic-Bücher und gestapelte neon-grüne Dosen anbot.

Tatsächlich lehnte sich der Standinhaber gerade zurück und trank aus einer der neon-grünen Dosen in exakt dem Augenblick, als er den vornehmen und eleganten Draco Malfoy in Begleitung eines mysteriösen Jungen, der mit einem Schal über dem Gesicht unglaublich dämlich aussah, herannahen sah, was ihn mitten im Trinken einen plötzlichen Hustenanfall erleiden und eine große Menge neon-grüner Flüssigkeit auf seinen Bart tropfen ließ.

"'Tschuldigen Sie," sagte Harry, "aber was genau \emph{ist} dieses Zeug?"

"Comed-Tee," sagte der Standbesitzer. "Wenn du ihn trinkst, passiert sicher etwas überraschendes, was ihn dich über dich selbst oder jemand anderen schütten lässt. Aber er ist verzaubert, um ein paar Sekunden später zu verschwinden -" Tatsächlich verschwand der Fleck auf seinem Bart bereits.

"Wie drollig," sagte Draco. "Wie wirklich, wirklich drollig. Komm, Mr. Bronze, lass uns einen anderen -"

"Warte mal," sagte Harry.

"\emph{Oh komm schon!} Das ist einfach, einfach \emph{kindisch!}"

"Nein, tut mir leid Draco, ich \emph{muss} das untersuchen. Was passiert, wenn ich Comed-Tee trinke, während ich mein bestes gebe, die Unterhaltung vollkommen ernst zu halten?"

Der Standbesitzer lächelte mysteriös. "Wer weiß? Ein Freund läuft in einem Froschkostüm vorbei? Etwas unerwartetes passiert sicher -"

"Nein. Tut mir leid. Das glaube ich einfach nicht. Das verletzt meine viel-geschundene skeptische Einstellung auf so vielen Ebenen, ich habe nicht mal Worte, um es zu beschreiben. Es kann, es kann einfach \emph{nicht sein,} dass ein verdammtes \emph{Getränk} die Realität manipulieren kann, um \emph{Comedy-Einspieler} zu produzieren oder ich gebe auf und zieh mich auf die Bahamas zurück -"

Draco stöhnte. "Machen wir das \emph{wirklich?}"

"Du musst es nicht trinken, aber ich \emph{muss} es untersuchen. \emph{Muss.} Wie viel?"

"Fünf Knuts die Dose," sagte der Standbesitzer.

"\emph{Fünf Knuts?} Sie können realitäts-manipulierende Softdrinks für \emph{fünf Knuts die Dose} verkaufen?" Harry griff in seinen Beutel, sagte "vier Sickel, vier Knuts" und klatschte sie auf den Tresen. "Zwei dutzend Dosen, bitte."

"Ich werde auch eine nehmen," seufzte Draco und setzte an, in seine Taschen zu langen.

Harry schüttelte schnell den Kopf. "Nein, ich übernehme das, zählt auch nicht als Gefallen, ich will sehen, ob es auch für dich funktioniert." Er nahm eine Dose von dem Stapel, der jetzt auf dem Tresen platziert war und warf sie Draco zu und fing dann an, seinen Beutel zu füttern. Die dehnbare Öffnung des Beutels verschlang die Dosen, begleitet von kleinen rülpsenden Geräuschen, die nicht wirklich dazu beitrugen, Harrys Glauben wiederherzustellen, dass er eines Tages für all das eine vernünftige Erklärung entdecken würde.

Zweiundzwanzig Rülpser später hatte Harry die letzte gekaufte Dose in seiner Hand, Draco sah ihn erwartungsvoll an und beide zogen die Lasche zur selben Zeit.

Harry rollte seinen Schal nach oben, um seinen Mund freizumachen und sie kippten die Köpfe nach hinten und tranken den Comed-Tee.

Es \emph{schmeckte} irgendwie hellgrün -- extra-sprudelnd und limoniger als Limone.

Abgesehen davon passierte nichts.

Harry sah den Standbesitzer an, der sie wohlwollend beobachtete.

\emph{Alles klar, wenn dieser Typ nur einen normalen Zufall dazu genutzt hat, mir vierundzwanzig Dosen gar nichts zu verkaufen, werde ich ihm zu seinem kreativen Geschäftssinn gratulieren und ihn dann umbringen.}

"Es passiert nicht immer sofort," sagte der Standbesitzer. "Aber garantiert einmal pro Dose oder Geld zurück."

Harry nahm einen weiteren langen Schluck.

Wieder passierte nichts.

\emph{Vielleicht sollte ich einfach das ganze Ding so schnell wie möglich runterstürzen… und hoffen, dass mein Magen nicht explodiert von all der Kohlensäure oder dass ich nicht rülpse, während ich es trinke…}

Nein, er konnte es sich leisten, ein \emph{bisschen} Geduld zu haben. Aber ehrlich, Harry konnte sich nicht vorstellen, wie das funktionieren würde. Man konnte nicht zu jemandem hingehen und sagen "Jetzt werde ich dich überraschen" oder "Und jetzt erzähle ich dir die Pointe des Witzes und sie wird wirklich lustig sein." Es ruinierte den Schockmoment. In Harrys Zustand mentaler Vorbereitung hätte Lucius Malfoy in einem Ballerina-Outfit vorbeilaufen können und es hätte ihn nicht angemessen herausprusten lassen. Was für einen verrückten Mumpitz genau sollte das Universum \emph{jetzt} ausspucken?

"Lass uns auf jeden Fall hinsetzen," sagte Harry. Er setzte zu einem weiteren Schluck an und setzte sich zum entfernten Sitzbereich hin in Bewegung, was ihn in den richtigen Winkel brachte, um zurück zu blicken und den Abschnitt des Zeitungsständers des Standes zu sehen, der einer Zeitung namens \emph{Der Klitterer} zugedacht war, welche die folgende Schlagzeile zeigte:

\emph{JUNGE-DER-ÜBERLEBT-HAT\\ SCHWÄNGERT DRACO MALFOY}

"\emph{Gah!}" schrie Draco als er von hellgrüner Flüssigkeit aus Harrys Richtung bespritzt wurde. Draco drehte sich zu Harry, mit Feuer in den Augen und griff nach seiner eigenen Dose. "Du Sohn eines Schlammbluts! Mal sehen, wie es \emph{dir} gefällt, vollgespuckt zu werden!" Draco nahm bewusst einen Schluck aus der Dose, als seine eigenen Augen die Schlagzeile entdeckten.

In reinem Reflex versuchte Harry sein Gesicht zu schützen, als der Sprühnebel aus Flüssigkeit in seine Richtung flog. Unglücklicherweise blockte er mit der Hand, die den Comed-Tee enthielt und ließ den Rest der grünen Flüssigkeit über seine Schulter platschen.

Harry starrte auf die Dose in seiner Hand, selbst als er weiterhin hustete und spuckte und die grüne Farbe von Dracos Umhang zu verschwinden begann.

Dann sah er auf und starrte die Zeitungsschlagzeile an.

\emph{JUNGE-DER-ÜBERLEBT-HAT\\ SCHWÄNGERT DRACO MALFOY}

Harrys Lippen öffneten sich und sagten, "ba-bla-ba-ba…"

Zu viele konkurrierende Einwände, das war das Problem. Jedes mal, wenn Harry zu sagen versuchte "Aber wir sind erst elf!" verlangte der Einwand "Aber Männer können nicht schwanger werden!" oberste Priorität und wurde von "Aber da ist nichts zwischen uns, wirklich!" niedergetrampelt.

Dann sah Harry wieder hinunter auf die Dose in seiner Hand.

Er fühlte ein tief-sitzendes Verlangen lauthals schreiend wegzurennen, bis er vor Sauerstoffmangel umfiel und das einzige, was ihn zurückhielt, war, dass er einmal gelesen hatte, dass totale Panik ein Zeichen für ein \emph{wahrhaft} wichtiges wissenschaftliches Problem war.

Harry knurrte, warf die Dose heftig in einen nahen Mülleimer und stapfte wieder zu dem Stand zurück. "Eine Ausgabe des \emph{Klitterers,} bitte." Harry bezahlte vier weitere Knuts, holte eine weitere Dose Comed-Tee aus seinem Beutel und stapfte hinüber zu dem Picknick-Bereich mit dem blonden Jungen, der seine eigene Dose mit einem Ausdruck aufrichtiger Bewunderung anstarrte.

"Ich nehm's zurück," sagte Draco, "das war ziemlich gut."

"Hey, Draco, weißt du was, wette ich, noch viel besser ist, um Freunde zu werden, als Geheimnisse auszutauschen? Jemand umzubringen."

"Ich habe einen Privatlehrer, der das sagt," räumte Draco ein. Er griff in seinen Umhang und kratzte sich mit einer leichten, natürlichen Bewegung. "Wen hast du im Sinn?"

Harry knallte den \emph{Klitterer} hart auf den Picknick-Tisch. "Den Kerl, der sich diese Schlagzeile ausgedacht hat."

Draco stöhnte. "Kein Kerl. Ein Mädchen. Ein \emph{zehnjähriges} Mädchen, kannst du das glauben? Sie ist verrückt geworden, nachdem ihre Mutter gestorben ist und ihr Vater, dem diese Zeitung gehört, ist \emph{überzeugt,} dass sie eine Seherin ist, wenn ihm also nichts einfällt, fragt er Luna Lovegood und glaubt \emph{alles,} was sie sagt."

Nicht wirklich darüber nachdenkend zog Harry die Lasche seiner nächsten Dose Comed-Tee und setzte zum Trinken an. "Veralberst du mich? Das ist sogar noch schlechter als Muggel-Journalismus, was ich für physikalisch unmöglich gehalten hätte."

Draco knurrte. "Sie ist auch auf irgendeine perverse Art besessen von den Malfoys und ihr Vater steht politisch gegen uns, also druckt er jedes Wort. So bald ich alt genug bin, werde ich sie vergewaltigen."

Grüne Flüssigkeit spritzte aus Harrys Nasenlöchern und tränkte den Schal, der immer noch diesen Bereich bedeckte. Comed-Tee und Lungen vermischten sich nicht und Harry verbrachte die nächsten paar Sekunden damit, krampfhaft zu husten.

Draco sah ihn scharf an. "Stimmt was nicht?"

An diesem Punkt kam Harry zu der plötzlichen Erkenntnis, dass (a) aus den Geräuschen vom Rest des Bahnsteigs ein noch undeutlicheres Rauschen geworden war, zur etwa gleichen Zeit, als Draco in seinen Umhang gegriffen hatte und (b) als er einen Mord zu begehen als bindungsstärkende Maßnahme diskutiert hatte, es ganz genau eine Person in der Unterhaltung gegeben hatte, die dachte, sie würden scherzen.

\emph{Klar. Weil er so sehr} schien \emph{wie ein normales Kind. Und er} ist \emph{ein normales Kind, er ist genau so, wie man} erwarten \emph{würde, dass ein normales männliches Kind wäre, wenn Darth Vader sein liebender Vater wäre.}

"Ja, nun," hustete Harry, oh Gott, wie sollte er sich nur aus dieser Klemme wieder herausreden, "ich war nur überrascht, dass du das so offen ansprechen wolltest, du schienst nicht wirklich beunruhigt zu sein, erwischt zu werden oder so."

Draco schnaubte. "Machst du Witze? \emph{Luna Lovegoods} Wort gegen meins?"

Heilige Scheiße am Stiel. "Es gibt nicht sowas wie einen magischen Lügendetektor, nehme ich an?" \emph{Oder DNS-Tests… vorerst.}

Draco sah sich um. Seine Augen verengten sich. "Stimmt, du weißt ja gar nichts. Pass auf, ich erklär's dir, ich meine wie es wirklich läuft, als ob du bereits in Slytherin wärst und mir die selbe Frage stellen würdest. Aber du musst schwören, nichts darüber zu verraten.

"Ich schwöre," sagte Harry.

"Die Gerichte benutzen Veritaserum, aber das ist ein Witz, ehrlich, man lässt bei sich einfach einen Gedächtnis-Zauber anwenden, bevor man aussagt und behauptet dann, der anderen Person wäre per Gedächtnis-Zauber eine falsche Erinnerung eingesetzt worden. Natürlich entscheiden die Gerichte, wenn man nur eine normale Person ist, im Zweifel eher auf einen Gedächtniszauber, anstatt einen Falsche-Erinnerung-Zauber. Aber das Gericht hat Schweigepflicht und wenn \emph{ich} involviert bin, betrifft das die Ehre eines Noblen Hauses, also geht es zum Zaubergamot, wo Vater die Abstimmung kontrolliert. Nachdem ich für nicht schuldig befunden wurde, muss die Lovegood-Familie mich für das Beschmutzen meiner Ehre entschädigen. Und sie wissen von Anfang an, dass es so laufen wird, also werden sie einfach den Mund halten."

Ein kalter Schauer überlief Harry, begleitet von Anweisungen, seine Stimme und sein Gesicht normal zu halten. \emph{Notiz an mich: Regierung des magischen Britannien bei der ersten Gelegenheit stürzen.}

Harry räusperte sich noch einmal. "Draco, bitte bitte \emph{bitte} nimm das nicht falsch auf, mein Wort bindet mich, aber wie du sagtest, ich könnte in Syltherin sein und ich muss wirklich der Information wegen fragen, also was würde passieren \emph{theoretisch gesprochen,} wenn ich aussagen \emph{würde,} ich hätte es dich planen gehört?"

"Dann, wenn ich irgendjemand anders als ein Malfoy wäre, würde ich in Schwierigkeiten stecken," sagte Draco süffisant. "Da ich ein Malfoy \emph{bin…} Vater hat die nötigen Stimmen. Und danach würde er dich zerquetschen… nun, ich nehme an, nicht einfach so, immerhin \emph{bist} du der Junge-der-überlebt-hat, aber Vater ist ziemlich gut in diesen Dingen." Draco runzelte die Stirn. "Nebenbei, \emph{du} hast darüber gesprochen, sie zu ermorden, warum warst du nicht besorgt, \emph{ich} könnte aussagen, nachdem man sie tot auffinden würde?"

\emph{Wie, oh wie, konnte mein Tag nur so schief laufen?} Harrys Mund bewegte sich bereits schneller als er denken konnte. "Das war, als ich dachte, sie wäre \emph{älter!} Ich weiß nicht, wie es \emph{hier} läuft, aber im Muggel-Britannien würden die Gerichte sich deutlich mehr darüber aufregen, wenn jemand ein Kind tötet -"

"Das macht Sinn," sagte Draco, immer noch ein wenig misstrauisch aussehend. "Aber trotzdem, es ist immer besser wenn die Auroren gar nicht erst davon Wind bekommen. Wenn wir vorsichtig genug sind, nur Sachen zu machen, die Heilzauber wieder hinbekommen, können wir einfach hinterher ihre Erinnerung löschen und alles nächste Woche nochmal machen." Dann kicherte der blonde Junge, ein jugendliches, hochgestochenes Geräusch. "Obwohl, stell dir nur vor, sie sagt, es wäre ihr von Draco Malfoy \emph{und} dem Jungen-der-überlebt-hat gemacht worden, nicht einmal \emph{Dumbledore} würde ihr glauben."

\emph{Ich werde euren erbärmlichen, kleinen, magischen Überrest der Finsteren Zeitalter in kleinere Stücke reißen, als die Atome aus denen er besteht.} "Können wir an dieser Stelle anhalten? Nachdem ich herausgefunden hatte, dass diese Schlagzeile von einem Mädchen stammt, das ein Jahr jünger ist als ich, kam mir ein anderer Gedanke für meine Rache."

"Häh? Erzähl'," sagte Draco und setzte zu einem weiteren Schluck seines Comed-Tees an.

Harry wusste nicht, ob die Verzauberung mehr als einmal pro Dose wirkte, aber er \emph{wusste} er könnte es darauf schieben, also bestimmte er sorgfältig genau den richtigen Zeitpunkt:

"Ich habe mir gedacht \emph{eines Tages heirate ich diese Frau.}"

Draco machte ein entsetzliches ka-platschendes Geräusch und ließ grüne Flüssigkeit aus seinen Mundwinkeln fließen, wie eine kaputte Autobatterie. "Bist du \emph{irre?}"

"Ganz im Gegenteil, ich bin so klar, es brennt wie Eis."

"Du hast einen seltsameren Geschmack als ein Lestrange," sagte Draco und klang fast bewundernd. "Und ich nehme an, du willst sie ganz für dich selbst, was?"

"Jep. Ich könnte dir dafür einen Gefallen schulden -"

Draco winkte ab. "Ne, der erste ist umsonst."

Harry starrte auf die Dose in seiner Hand hinab, die Kälte kroch ihm ins Blut. Charmant, fröhlich, großzügig mit Gefallen für seine Freunde, war Draco kein Psychopath. Das war das Traurige und Entsetzliche, genug von der menschlichen Psychologie zu verstehen, um zu wissen, dass Draco \emph{kein} Monster war. Es hatte zehntausende Gesellschaften im Laufe der Weltgeschichte gegeben, in denen diese Unterhaltung hätte stattfinden können. Nein, die Welt wäre tatsächlich ein sehr andersartiger Ort gewesen, wenn es einen \emph{bösen Mutanten} brauchen würde, um zu sagen, was Draco gesagt hatte. Es war sehr einfach, sehr menschlich, es war die Standardeinstellung, wenn nichts anderes dazwischen kam. Für Draco waren seine Feinde keine Menschen.

Und in der verlangsamten Zeit dieses aus der Zeit gefallenen Landes, hier und jetzt, wie in der Finsternis-vor-dem-Morgengrauen, die dem Zeitalter der Vernunft voranging, würde der Sohn eines ausreichend mächtigen Adligen einfach voraussetzen, dass er über dem Gesetz steht, zumindest, wenn es um irgendein einfaches Mädchen ging. Es gab Orte in der Muggelwelt, wo es immer noch genau so war, Länder, in denen diese Art Adel immer noch existierte und die noch immer so dachten und noch finsterere Landstriche, wo es nicht nur der Adel war. So war es an jedem Ort und in jeder Zeit, die nicht direkt von der Aufklärung abstammten. Eine Abstammungslinie, die, wie es schien, das magische Britannien nicht ganz einschloss, trotz aller interkulturellen Ansteckungen mit Dingen wie Dosen mit Aufziehlaschen.

\emph{Und wenn Draco seine Meinung, Rache zu wollen, nicht ändert und ich nicht meine eigene Chance auf Glück im Leben wegwerfe, um ein armes verrücktes Mädchen zu heiraten, dann habe ich gerade nur Zeit erkauft und nicht allzu viel davon…}

Für ein Mädchen. Nicht für andere.

\emph{Ich frage mich, wie schwierig es wäre, einfach eine Liste aller hochrangigen Blutreinheits-Verfechter zu machen und sie umzubringen.}

Genau das hatte man während der Französischen Revolution versucht, mehr oder weniger -- eine Liste aller Feinde des Fortschritts zu machen und alles oberhalb des Nackens zu entfernen -- und es war nicht gut ausgegangen, soweit Harry sich erinnerte. Vielleicht musste er einige der Geschichtsbücher, die sein Vater ihm gekauft hatte, entstauben und nachsehen, ob das, was bei der Französischen Revolution schiefgegangen war, etwas einfach zu behebendes war.

Harry blickte hinauf zum Himmel und auf die blassen Umrisse des Mondes, an diesem Morgen sichtbar durch die wolkenlose Luft.

\emph{Also, die Welt ist kaputt und unvollkommen und wahnsinnig und grausam und blutig und finster. Ist das was Neues? Das wusstest du doch so wie so} \emph{immer schon…}

"Du siehst ganz ernst aus," sagte Draco. "Lass mich raten, deine Muggel-Eltern haben dir erzählt, dass solche Sachen schlecht sind."

Harry nickte, er vertraute seiner Stimme nicht ganz.

"Nun, wie Vater sagt, es mag zwar vier Häuser geben, aber am Ende gehört jeder entweder nach Slytherin oder Hufflepuff. Und ehrlich, du bist nicht auf der Hufflepuff-Seite. Wenn du dich entscheidest, dich insgeheim auf die Seite der Malfoys zu stellen… unsere Macht und dein Ruf… du könntest mit Dingen durchkommen, die sogar \emph{ich} nicht tun kann. Willst du's eine Weile ausprobieren? Sehen, wie es so ist?"

\emph{Sind wir nicht eine clevere kleine Schlange. Elf Jahre alt und redest deiner Beute schon das Verstecken aus…}

Harry dachte nach, wog ab, wählte seine Waffe. "Draco, willst du mir dieses ganze Blutreinheits-Ding erklären? Ich bin noch neu."

Ein breites Lächeln lief über Dracos Gesicht. "Du solltest wirklich Vater treffen und \emph{ihn} fragen, weißt du, er ist unser Anführer."

"Gib mir die Dreißig-Sekunden-Version."

"Okay," sagte Draco. Er atmete tief ein und seine Stimme wurde etwas leiser und rythmisch. "Unsere Kräfte wurden schwächer, Generation für Generation, während der Schlammblut-Makel wächst. Während Salazar und Godric und Rowena und Helga einst Hogwarts errichteten mit ihrer Macht und das Medaillon und das Schwert und das Diadem und den Kelch schufen, kommt kein Zauberer unserer verblichenen Tage ihnen gleich. Wir schwinden dahin, schwinden alle zu Muggeln, weil wir uns kreuzen mit ihrer Brut und unseren Squibs erlauben, zu leben. Wenn der Makel nicht ausgemerzt wird, werden bald unsere Zauberstäbe zerbrechen und unsere Künste vergehen, die Linie von Merlin wird enden und das Blut von Atlantis versiegen. Unseren Kindern wird nichts bleiben, als im Staub zu scharren, um zu überleben, wie die gemeinen Muggel und Finsternis wird die Welt für immer umhüllen." Draco nahm einen weiteren Schluck aus seiner Dose und sah zufrieden aus, das schienen alle Belege zu sein, die er brauchte.

"Klingt überzeugend," sagte Harry und meinte es eher beschreibend als normativ. Es war ein Standard-Muster: Der Sündenfall, die Notwendigkeit, die verbliebene Reinheit vor Verschmutzung zu schützen, die Vergangenheit wurde erhöht und die Zukunft konnte nur noch tiefer sinken. Und dieses Muster hatte auch sein \emph{Gegenstück…} "Allerdings muss ich dich bei einer Tatsache korrigieren. Deine Informationen über die Muggel sind ein wenig veraltet. Wir scharren nicht mehr buchstäblich im Staub."

Dracos Kopf fuhr herum. "\emph{Was?} Was meinst du mit \emph{wir?}"

"Wir. Die Wissenschaftler. Die Linie von Francis Bacon und das Blut der Aufklärung. Die Muggel haben nicht nur herumgesessen und darüber geheult, dass sie keine Zauberstäbe haben, wir haben jetzt unsere \emph{eigenen} Kräfte, mit oder ohne Magie. Wenn all eure Mächte versagen, dann werden wir etwas wirklich kostbares verloren haben, weil eure Magie der einzige Hinweis darauf ist, wie das Universum \emph{wirklich} funktionieren muss -- aber ihr werdet nicht im Boden herumscharren müssen. Eure Häuser werden immer noch kühl sein im Sommer und warm im Winter, es wird Ärzte und Medizin geben. Die Wissenschaft kann euch am Leben halten, wenn die Magie versagt. Es wäre eine Tragödie, aber nicht buchstäblich das Ende alle Lichts in der Welt. Nur mal so gesagt."

Draco war mehrere Fuß zurückgewichen und sein Gesicht war voller Angst vermischt mit Unglauben. "\emph{Wovon in Merlins Namen sprichst du da, Potter?}"

"Hey, ich habe mir \emph{deine} Geschichte angehört, wirst du dir meine nicht anhören?" \emph{Ungeschickt,} schalt Harry sich selbst, aber Draco schien tatsächlich nicht mehr zurückzuweichen und zuzuhören.

\emph{"Jedenfalls," sagte Harry, "scheint es mir, als ob du nicht besonders viel Aufmerksamkeit auf das verwendet hast, was in der Muggelwelt vorgeht." Vielleicht weil die ganze Zauberwelt den Rest der Erde als einen Slum anzusehen schien, der ungefähr genau so viel Nachrichten-Abdeckung verdiente, wie die \emph{Financial Times} dem schon gewohnten Elend in Burundi zuteil werden ließ. "Okay. Schnell-Test. Sind Zauberer jemals auf dem Mond gewesen? Du weißt schon, dieses Ding?" Harry zeigte auf die große und weit entfernte Kugel.}

"\emph{Was?}" sagte Draco. Es war ziemlich klar, dass dem Jungen dieser Gedanke noch nie gekommen war. "Auf \emph{den} - das ist nur ein -" Sein Finger deutete auf das kleine, blasse Dingens am Himmel. "Man kann nirgendwo hin apparieren, wo man noch nie \emph{gewesen} ist und wie würde man überhaupt zum \emph{ersten} mal dorthin kommen?"

"Warte kurz," sagte Harry zu Draco, "ich würde dir gern ein Buch zeigen, dass ich mitgenommen habe, ich denke, ich weiß in welchem Karton es ist." Und Harry stand auf und kniete sich hin und zerrte die Stufen der Keller-Ebene seines Koffers heraus, stürmte dann die Treppe hinunter und hob einen Karton von einem anderen Karton, dem respektlosen Umgang mit seinen Büchern gefährlich nahe kommend und riss den Karton-Deckel herunter und hob schnell, aber vorsichtig einen Stapel Bücher heraus -

(Harry hatte die beinahe-magische Verres-Fähigkeit geerbt, sich zu erinnern, wo alle seine Bücher waren, nachdem er sie auch nur einmal gesehen hatte, was eher mysteriös war, wegen des Fehlens jeder genetischen Verbindung.)

Und Harry raste die Stufen wieder hinauf und schob die Treppe mit der Hacke in seinen Koffer zurück und blätterte, keuchend, durch die Seiten des Buches, bis er das Bild fand, das er Draco zeigen wollte.

Das mit dem weißen, trockenen, kraterübersäten Land und den Menschen in Anzügen und der über allem hängenden blau-weißen Kugel.

Dieses Bild.

\emph{Das} Bild, wenn nur ein Bild auf der Welt überleben würde.

"\emph{So,}" sagte Harry und seine Stimme bebte, weil er den Stolz nicht ganz heraushalten konnte, "sieht die Erde, vom Mond aus gesehen, aus."

Draco beugte sich langsam herüber. Es war ein merkwürdiger Ausdruck auf seinem jungen Gesicht. "Wenn das ein \emph{echtes} Bild ist, warum bewegt es sich nicht?"

\emph{Bewegt?} Oh. "Muggel können sich bewegende Bilder erzeugen, aber sie brauchen einen größeren Kasten, um sie zu zeigen, sie bekommen sie noch nicht auf einzelne Buchseiten."

Dracos Finger bewegten sich zu einem der Anzüge. "Was sind die?" Seine Stimme begann zu zitten.

"Das sind menschliche Wesen. Sie tragen Anzüge, die ihre ganzen Körper bedecken, um sie mit Luft zu versorgen, weil es auf dem Mond keine Luft gibt."

"Das ist unmöglich," flüsterte Draco. Es lag Entsetzen in seinen Augen und vollkommene Verwirrung. "Kein Muggel könnte das jemals tun. \emph{Wie…}"

Harry nahm das Buch zurück, blätterte die Seiten durch, bis er fand, was er suchte. "Das ist eine startende Rakete. Das Feuer treibt sie höher und höher, bis sie den Mond erreicht." Blätterte wieder. "Das ist eine Rakete am Boden. Dieser kleine Fleck daneben ist ein Mensch." Draco keuchte. "Zum Mond zu reisen kostet die Entsprechung von… wahrscheinlich etwa tausend Millionen Galleonen." Draco würgte. "Und es brauchte die Anstrengungen von… wahrscheinlich mehr Menschen, als im gesamten magischen Britannien leben." \emph{Und als sie ankamen, hinterließen sie eine Plakette, auf der stand 'Wir kamen in Frieden, für die gesamte Menschheit.' Obwohl du noch nicht bereit bist, diese Worte zu hören, Draco Malfoy…}

"Du sagst die Wahrheit," sagte Draco langsam. "Du würdest kein ganzes Buch nur dafür fälschen -- und ich kann es in deiner Stimme hören. Aber… aber…"

"Wie, ohne Zauberstäbe oder Magie? Es ist eine lange Geschichte, Draco. Wissenschaft funktioniert nicht, indem man Zauberstäbe schwingt und Zaubersprüche aufsagt, sie funkioniert, indem man auf so tiefer Ebene versteht, wie das Universum funktioniert, dass man genau weiß, was man tun muss, um das Universum dazu zu bringen, zu tun, was man will. Wenn Magie so ist, wie \emph{Imperio} auf jemanden zu wirken, um ihn tun zu lassen, was man will, dann ist Wissenschaft so, als ob man ihn so gut kennt, dass man ihn überzeugen kann, dass es seine eigene Idee war. Es ist viel schwieriger als einen Zauberstab zu schwingen, aber es funktioniert, wenn Zauberstäbe versagen, so wie, wenn der \emph{Imperius} versagt, du eine Person immer noch zu überreden versuchen kannst. Und Wissenschaft wird von Generation zu Generation erschaffen. Man muss wirklich \emph{wissen,} was man tut, um Wissenschaft zu nutzen -- und wenn man etwas wirklich versteht, kann man es anderen erklären. Die größten Wissenschaftler des vergangenen Jahrhunderts, die berühmtesten Namen, die immer noch mit Ehrfurcht genannt werden, ihre Kräfte sind wie \emph{nichts} für die größten Wissenschaftler von heute. In der Wissenschaft gibt es keine Entsprechung eurer verlorenen Künste, die Hogwarts erschaffen haben. In der Wissenschaft nehmen unsere Mächte mit jedem Jahr zu. Und wir fangen an, zu verstehen und die Geheimnisse des Lebens und unserer Ursprünge zu enthüllen. Wir werden in der Lage sein, uns genau das Blut, von dem du sprachst, anzusehen und zu verstehen, was einen zu einem Zauberer macht und in ein oder zwei weiteren Generationen, werden wir dieses Blut überzeugen können, auch alle unsere Kinder zu mächtigen Zauberern zu machen. Also, wie du siehst, ist euer Problem nicht annähernd so schlimm, wie es aussieht, weil in ein paar Jahrzehnten, die Wissenschaft in der Lage sein wird, es für euch zu lösen."

"Aber…" sagte Draco. Seine Stimme bebte. "Wenn \emph{Muggel} diese Art von Macht besitzen… dann… wer sind \emph{wir} dann?"

"Nein, Draco, darum geht es nicht, verstehst du nicht? Wissenschaft speist sich aus der Karft des menschlichen Verständnisses, die Welt zu beobachten und herauszufinden, wie sie funktioniert. Sie kann nicht scheitern, wenn nicht die Menschheit selbst scheitert. Deine Magie könnte versagen und du würdest es hassen, aber du wärst immer noch \emph{du.} Du wärst immer noch am Leben, um es zu bereuen. Aber weil die Wissenschaft sich auf meine menschliche Intelligenz stützt, kann sie nicht entfernt werden, ohne \emph{mich} zu entfernen. Selbst wenn die Regeln des Universums sich für mich verändern, so dass all mein Wissen nichtig ist, finde ich einfach die neuen Regeln heraus, so wie es immer getan wurde. Es ist kein \emph{Muggel}-Ding, es ist ein \emph{Menschen}-Ding, es verbessert und trainiert nur die Macht, die du jedesmal nutzt, wenn du etwas siehst, was du nicht verstehst und fragst 'Warum?' Du bist ein Slytherin, Draco, verstehst du nicht, was daraus folgt?"

Draco sah von dem Buch zu Harry auf. Sein Gesicht zeigte dämmerndes Verständnis. "Zauberer können lernen, diese Macht zu nutzen."

Sehr vorsichtig jetzt… der Köder ist ausgelegt, jetzt der Haken… "Wenn du lernen kannst, von dir selbst als einem \emph{Menschen} zu denken, anstatt als einem \emph{Zauberer,} kannst du deine Kräfte als ein Mensch verbessern und trainieren."

Und wenn \emph{diese} Anweisung nicht in \emph{jedem} Wissenschafts-Lehrplan stand, musste Draco das nicht wissen, oder?

Dracos Augen waren jetzt nachdenklich. "Du hast… das schon gemacht?"

"In gewissem Ausmaß," gestand Harry ein. "Meine Ausbildung ist nicht abgeschlossen. Nicht mit elf Jahren. Aber -- mein Vater hat mir auch Privatlehrer gekauft, wie du siehst." Sicher, das waren hungernde Studienabsolventen gewesen und auch nur, weil Harry in einem 26-Stunden-Zyklus schlief, aber das lassen wir für den Moment beiseite…

Langsam nickt Draco. "Du denkst, du kannst \emph{beide} Künste meistern, die Mächte zusammenführen und…" Draco starrte Harry an. "Dich zum Herrn der zwei Welten machen?"

Harry gab ein böses Lachen von sich, es schien an diesem Punkt einfach zu passen. "Du musst verstehen, Draco, dass die ganze Welt, die du kennst, das ganze magische Britannien, nur ein Feld auf einem viel größeren Spielbrett ist. Das Spielbrett beinhaltet Orte wie den Mond oder die Sterne am Nachthimmel, die Lichter genau wie unsere Sonne sind, nur unvorstellbar weit entfernt und Dinge wie Galaxien, die noch viel größer sind als die Erde und die Sonne, Dinge, so groß, dass nur Wissenschaftler sie sehen können und du nicht einmal weißt, dass sie existieren. Aber ich \emph{bin} wirklich ein Ravenclaw, kein Slytherin. Ich will nicht das Universum beherrschen. Ich denke nur, es könnte sinnvoller organisiert sein."

Es lag Ehrfurcht auf Dracos Gesicht. "Warum erzählst du \emph{mir} das?"

"Oh… es gibt nicht viele Leute, die wissen, wie man \emph{wahre} Wissenschaft bertreibt -- etwas zum allerersten mal zu verstehen, selbst wenn einen die Verwirrung fast in den Wahnsinn treibt. Hilfe wäre hilfreich."

Draco starrte Harry mit offenem Mund an.

"Aber mach keinen Fehler, Draco, wahre Wissenschaft ist wirklich \emph{nicht} wie Magie, du kannst es nicht einfach machen und davon unverändert bleiben, als wenn du die Worte eines neuen Zaubers lernst. Die Macht hat einen Preis und der Preis ist so hoch, dass die meisten Menschen sich weigern, ihn zu zahlen."

Draco nickte bei diesem Satz, als ob er, endlich, etwas gehört hatte, dass er verstehen konnte. "Und der Preis?"

"Sich einzugestehen lernen, dass man falsch liegt."

"Ähm," sagte Draco, nachdem sich die dramatische Pause eine Weile hingezogen hatte. "Wirst du das erklären?"

"Wenn du versuchst auf dieser tiefen Ebene zu verstehen, wie etwas funktioniert, sind die ersten neunundneunzig Erklärungen, die dir einfallen, falsch. Die hundertste ist richtig. Also musst du lernen, zuzugeben, dass du falsch liegst, immer und immer wieder. Es klingt nicht nach viel, aber es ist so schwierig, dass die meisten Menschen keine Wissenschaft betreiben können. Dich selbst immer in Frage zu stellen, immer Dinge noch einmal zu überdenken, die du immer vorausgesetzt hast," wie einen Schnatz beim Quidditch zu haben, "und jedesmal, wenn du deine Meinung änderst, veränderst du dich selbst. Aber ich greife hier zu weit vor. Viel zu weit vor. Ich will dich nur wissen lassen… ich biete an, etwas von meinem Wissen zu teilen. Wenn du willst. Es gibt nur eine Bedingung."

"Aha," sagte Draco. "Weißt du, Vater sagt, wenn jemand das zu dir sagt, ist das nie und nimmer ein gutes Zeichen."

Harry nickte. "Nun, versteh mich nicht falsch und glaube nicht, dass ich einen Keil zwischen dich und deinen Vater treiben will. Darum geht es nicht. Es geht nur darum, dass ich es mit jemandem in meinem Alter zu tun haben will, anstatt das zwischen mir und Lucius auszumachen. Ich denke dein Vater wäre auch damit einverstanden, er weiß, dass du irgendwann erwachsen werden musst. Aber deine Züge in unserem Spiel müssen deine eigenen sein. Das ist meine Bedingung -- das ich es mit dir zu tun habe, Draco, nicht mit deinem Vater."

"Ich muss gehen," sagte Draco. Er stand auf. "Ich muss verschwinden und darüber nachdenken."

"Nimm dir Zeit," sagte Harry.

Die Geräusche des Bahnsteigs veränderten sich von verschwommen zu Gemurmel, als Draco weg ging.

Harry entließ den Atem, den er angehalten hatte, ohne es wirklich zu bemerken und sah dann auf die Uhr an seinem Handgelenk, ein einfaches mechanisches Modell, dass sein Vater ihm in der Hoffnung gekauft hatte, es würde in der Gegenwart von Magie funktionieren. Der Sekundenzeiger tickte immer noch und wenn der Minutenzeiger recht hatte, war es noch nicht ganz elf. Er sollte wahrscheinlich bald in den Zug steigen und anfangen nach wie-auch-immer-sie-aussieht Ausschau zu halten, aber es schien ein paar Minuten wert zu sein, ein paar Atemübungen zu machen und zu sehen, ob sein Blut sich wieder erwärmte.

Aber als Harry von seiner Armbanduhr aufblickte, sah er zwei Gestalten näher kommen, die vollkommen lächerlich aussahen, mit ihren von Winterschals bedeckten Gesichtern.

"Hallo, Mr. Bronze," sagte eine der maskierten Gestalten. "Können wir dich dafür interessieren, dem Orden des Chaos beizutreten?"

--------------------------------------------------------------------------------------------------------------------------------------------

\emph{Nachspiel:}

Nicht zu lange danach, als all die Aufregung dieses Tages sich endlich gelegt hatte, beugte sich Draco mit einem Federkiel in der Hand über einen Schreibtisch. Er hatte ein privates Zimmer in den Slytherin-Kerkern, mit eigenem Schreibtisch und eigenem Feuer -- traurigerweise wurde nicht einmal ihm eine Verbindung zum Flohnetzwerk zugebilligt, aber zumindest nahm Slytherin nicht an diesem vollkommenen Unsinn teil, \emph{jeden} in Schlafsälen unterzubringen. Es gab nicht viele private Zimmer, man musste der \emph{beste} aus einem der besseren Häuser sein, aber das konnte bei Haus Malfoy vorausgesetzt werden.

\emph{Lieber Vater,} schrieb Draco.

Dann hielt er inne.

Tinte tropfte langsam von seinem Federkiel, beschmutzte das Pergament neben den Worten.

Draco war nicht dumm. Er war jung, aber seine Privatlehrer hatten ihn gut ausgebildet. Draco wusste, dass Potter wahrscheinlich wesentlich mehr Sympathie für Dumbledores Seite fühlte, als er durchblicken ließ… obwohl Draco glaubte, Potter könne in Versuchung geführt werden. Aber es war kristallklar, dass Potter versuchte, Draco in Versuchung zu führen, so wie Draco es bei ihm versuchte.

Und es war ebenso klar, dass Potter brilliant war und ein ganzes Stück mehr als nur leicht verrückt und dass er ein Spiel spielte, das Potter selbst größtenteils nicht verstand, improvisiert bei Höchstgeschwindigkeit, mit der Subtilität eines tobenden Nundu. Aber Potter hatte es geschafft, eine Taktik zu wählen, die Draco nicht einfach ignorieren konnte. Er hatte Draco einen Teil seiner eigenen Macht angeboten, darauf setzend, dass Draco sie nicht nutzen konnte, ohne mehr wie er zu werden. Sein Vater hatte das als eine fortgeschrittene Technik bezeichnet und Draco gewarnt, dass sie oft nicht funktionierte.

Draco wusste, er hatte nicht alles verstanden, was passiert war… aber Potter hatte \emph{ihm} die Chance geboten, zu spielen und im Augenblick war es \emph{seine}. Und wenn er die ganze Sache herausplapperte, würde es die von Vater werden.

Am Ende war es genau so einfach. Die einfacheren Techniken setzten die Unwissenheit des Ziels voraus oder zumindest seine Unsicherheit. Schmeichelei musste glaubwürdig als Bewunderung getarnt werden. ("Du hättest in Slytherin sein sollen" war ein alter Klassiker, höchst effektiv bei einem bestimmten Typ von Person, die es nicht erwartet und wenn es funktionierte, konnte man es wiederholen.) Aber wenn man jemandes ultimativen Hebel gefunden hatte, spielte es keine Rolle, ob derjenige wusste, dass man ihn kennt. Potter hatte in seinem wahnsinnigen Ansturm einen Schlüssel zu Dracos Seele erraten. Und wenn Draco wusste, dass Potter es wusste -- selbst wenn es eine sehr naheliegende Vermutung gewesen war -- machte das nicht den geringsten Unterschied.

Und so hatte er jetzt, zum ersten mal in seinem Leben, echte Geheimnisse zu bewahren. Er spielte sein eigenes Spiel. Es lag ein undeutlicher Schmerz darin, aber er wusste, dass Vater stolz wäre und deshalb war es in Ordnung.

Die Tintentröpfchen dort lassend wo sie waren -- es lag eine Nachricht darin und eine, die sein Vater verstehen würde, immerhin hatten sie das Spiel der Subtilitäten mehr als einmal gespielt -- formulierte Draco die eine Frage aus, die bei dieser ganzen Angelegenheit an ihm genagt hatte, der Teil den er, wie es schien, verstehen \emph{sollte,} aber er tat es nicht, überhaupt nicht.

\emph{Lieber Vater:}

\emph{Angenommen, ich würde dir erzählen, dass ich einen Schüler in Hogwarts getroffen habe, der nicht bereits zum Kreis unserer Vertrauten zählt und dich als ein 'makelloses Instrument des Todes' bezeichnet und gesagt hat, ich wäre dein 'einer Schwachpunkt'. Was würdest du über ihn sagen?}

Danach dauerte es nicht lange, bis die Familien-Eule die Antwort überbrachte.

\emph{Mein geliebter Sohn:}

\emph{Ich würde sagen, dass du das Glück hattest, jemanden zu treffen, der das persönliche Vertrauen unseres Freundes und wertvollen Verbündeten, Severus Snape, genießt.}

Draco starrte den Brief eine Weile an und warf ihn schlussendlich ins Feuer.

* Die englische Formulierung ist \emph{resonant doubt.} Wenn Harry sich hier auf einen feststehenden wissenschaftlichen/psychologischen Begriff beziehen sollte, konnte ich ihn nicht finden, also belasse ich es bei einer wörtlichen Übersetzung. Der Begriff \emph{resonant} bezieht sich hier wohl auf das Konzept, dass in einer Situation, in der man daran glauben solle, dass etwas funktioniert, damit es funktioniert, selbst ein kleiner anfänglicher Zweifel sich durch Wechselwirkung mit sich selbst immer weiter aufschaukeln und fatale Auswirkungen haben kann.\\ ** engl.: \emph{Influence: Science and Practice}

