

\hypertarget{abstimmungsprobleme-teil-3}{% \section{35. Abstimmungsprobleme, Teil 3}\label{abstimmungsprobleme-teil-3}}

\textbf{Kapitel 35: Abstimmungsprobleme, Teil 3}

Sie hatten das Büro des Verteidigungsprofessors aufgesucht und Professor Quirrell hatte die Tür versiegelt, bevor er sich in seinem Stuhl zurücklehnte und sprach.

Die Stimme des Verteidigungsprofessors war sehr ruhig und das machte Harry noch ein gutes Stück nervöser als hätte er geschrien.

"Ich versuche," sagte Professor Quirrell leise, "auf die Tatsache Rücksicht zu nehmen, dass Sie noch sehr jung sind. Dass auch ich selbst, als ich in Ihrem Alter war, ein ganz außerordentlicher Narr gewesen bin. Sie sprachen auf erwachsene Art und mischen sich in die Angelegenheiten Erwachsener ein und manchmal vergesse ich, dass Sie nur ein Amateur sind. Ich hoffe sehr, Mr. Potter, dass Ihre kindische Einmischung Sie nicht gerade getötet, Ihr Land ruiniert und den nächsten Krieg verloren gegeben hat."

Es fiel Harry äußerst schwer, nicht die Beherrschung zu verlieren. "Professor Quirrell, ich habe noch einiges weniger gesagt als ich hätte sagen wollen, aber ich musste etwas sagen. Ihre Vorschläge sind extrem besorgniserregend für jeden, der auch nur im Ansatz mit der Muggel-Geschichte der letzten hundert Jahre vertraut ist. Die Italienischen Faschisten, ein paar wirklich üble Leute, leiteten ihren Namen von den \emph{Fascen} ab, einem Bündel zusammengebundener Ruten als Symbol der Idee, dass Einigkeit Stärke ist -"

"Also glaubten die bösen Italienischen Faschisten, dass Einigkeit stärker ist als Teilung," sagte Professor Quirrell. Langsam wurde sein Tonfall schärfer. "Vielleicht glaubten sie auch noch, dass der Himmel blau ist und wollten eine Politik durchsetzen, man solle sich keine Steine an den Kopf schlagen."

\emph{Umgekehrte Dummheit ist noch keine Intelligenz; der dümmste Mensch der Welt mochte sagen, das die Sonne schien, doch deshalb musste sie noch nicht verlöschen}…* "Schön, Sie haben recht, das war ein ad hominem; es ist nicht deshalb falsch, nur \emph{weil} die Faschisten es sagten. Aber Professor Quirrell, Sie können nicht einfach jeden in einem Land das Mal eines einzigen Diktators annehmen lassen! Das ist eine viel zu große Schwachstelle!** Na gut, ich will es mal so ausdrücken. Nehmen Sie an, der Feind zwingt einfach denjenigen, der das Mal kontrolliert, unter den Imperius-Fluch -"

"Mächtige Zauberer sind nicht so leicht zu kontrollieren," sagte Professor Quirrell trocken. "Und können Sie keinen würdigen Anführer finden, so sind Sie in jedem Fall verdammt. Doch würdige Anführer existieren; die Frage ist, ob die Menschen ihnen folgen werden."

Frustriert raufte sich Harry die Haare. Er hätte gern eine Auszeit erklären und Professor Quirrell \emph{Aufstieg und Fall des Dritten Reiches}*** lesen lassen wollen, um die Unterhaltung danach von neuem zu beginnen. "Ich nehme nicht an, dass wenn ich die Demokratie als bessere Regierungsform gegenüber der Diktatur vorschlüge -"

"Ich verstehe," sagte Professor Quirrell. Er schloss für einen Moment die Augen und öffnete sie wieder. "Mr. Potter, die Unsinnigkeit des Quidditch-Spiels ist Ihnen offensichtlich, weil Sie nicht mit der Verehrung für das Spiel aufgewachsen sind. Hätten Sie noch nie etwas von Wahlen gehört, Mr. Potter und \emph{sähen Sie nur das was ist}, so würde Ihnen nicht gefallen, was Sie sehen. Nehmen Sie nur unseren gewählten Zaubereiminister. Ist er der klügste, der stärkste, der größte unserer Nation? Nein, er ist schlicht ein Hanswurst, der in Lucius Malfoys Diensten steht. Die Zauberer gingen zur Wahl und entschieden zwischen Cornelius Fudge und Tania Leach, die in einem großartigen und unterhaltsamen Wettstreit gegeneinander angetreten waren, nachdem der \emph{Tagesprophet}, den Lucius Malfoy ebenfalls kontrolliert, entschieden hatte sie seien die einzig ernstzunehmenden Kandidaten. Dass Cornelius Fudge tatsächlich gewählt wurde als der beste Anführer, den unser Land zu bieten hatte, ist eine Behauptung, die niemand allen Ernstes wagen würde. In der Muggelwelt ist es nicht anders, nachdem was ich gehört und gesehen habe; die letzte Muggelzeitung die ich las, erwähnte dass der letzte Präsident der Vereinigten Staaten ein Schauspieler im Ruhestand gewesen sei.**** Wären Sie nicht mit Wahlen aufgewachsen, Mr. Potter, so würden sie Ihnen ebenso offensichtlich unsinnig erscheinen wie Quidditch."

Harry saß mit offenem Mund da und rang nach Worten. "Der Sinn von Wahlen ist es doch nicht, den einen besten Anführer hervorzubringen, sondern Politikern genug Angst vor den Wählern einzujagen, dass sie nicht vollkommen durchdrehen, wie es Diktatoren tun -"

"Der letzte Krieg, Mr. Potter, wurde ausgetragen zwischen dem Dunklen Lord und Dumbledore. Und auch wenn Dumbledore ein unzulänglicher Anführer war, der im Begriff war, den Krieg zu verlieren, so wäre es \emph{lächerlich} anzudeuten, dass \emph{auch nur einer} der während dieser Zeit gewählten Zaubereiminister seinen Platz hätte einnehmen können! Stärke entspringt mächtigen Zauberern und ihren Anhängern, nicht aus Wahlen und den Narren, die sie hervorbringen. Das ist die Lektion aus der jüngsten Geschichte des magischen Britannien und ich bezweifle, dass der nächste Krieg Ihnen eine andere erteilen wird. \emph{Falls} Sie ihn überleben, Mr. Potter, was \emph{nicht} der Fall sein wird, es sei denn Sie lassen die Träumereien Ihrer Kindheit hinter sich!"

"Wenn Sie glauben, dass der Kurs, den einzuschlagen Sie empfehlen, keine Gefahren birgt," sagte Harry und trotz allem lag Schärfe in seiner Stimme, "so ist das ebenfalls eine kindische Träumerei."

Harry blickte Professor Quirrell grimmig in die Augen, der den Blick erwiderte ohne zu blinzeln.

"Solcherlei Gefahren," sagte Professor Quirrell kalt, "sollten in Büros wie diesem hier besprochen werden, nicht in Ansprachen. Die Narren, die Cornelius Fudge gewählt haben, interessieren sich nicht für Bedenken und Komplikationen. Verlangen Sie ihnen irgendetwas nuancierteres ab als einen mitreißenden Jubel und Sie kämpfen Ihren Krieg allein. \emph{Das}, Mr. Potter, war Ihr kindischer Fehler, den Draco Malfoy selbst im Alter von acht Jahren nicht begangen hätte. Es hätte selbst \emph{Ihnen} offensichtlich sein sollen, dass Sie die Ruhe hätten bewahren und sich \emph{zuerst mit mir besprechen} sollen, anstatt Ihre Sorgen vor der Menge auszubreiten!"

"Ich bin ganz sicher kein Freund von Albus Dumbledore," sagte Harry und die Kälte in seiner Stimme kam der von Professor Quirrell gleich. "Aber er ist kein Kind und weder schien er meine Sorgen für kindisch zu halten, noch meinte er ich hätte zögern sollen, sie auszusprechen."

"Oh," sagte Professor Quirrell, "also nehmen Sie sich jetzt ein Beispiel an unserem Schulleiter, ja?" und erhob sich von seinem Schreibtisch.

--------------------------------------------------------------------------------------------------------------------------------------------

Als Blaise auf dem Weg zum Büro um die Ecke bog, sah er dass Professor Quirrell dort bereits an der Wand lehnte.

"Blaise Zabini," sagte der Verteidigungsprofessor und richtete sich auf; seine Augen saßen wie zwei dunkle Steine in seinem Gesicht und seine Stimme sandte Blaise einen ängstlichen Schauer über den Rücken.

\emph{Er kann mir nichts anhaben, ich muss nur} \emph{immer} \emph{daran denken} -

"Ich glaube," sagte Professor Quirrell mit klarer, kalter Stimme, "dass ich den Namen Ihres Auftraggebers bereits erraten habe. Doch ich würde es gern von Ihren Lippen hören und auch den Preis erfahren, für den Sie zu haben waren."

Blaise wusste, dass er unter seinem Umhang schwitzte und das die Feuchtigkeit sich bereits auf seiner Stirn abzeichnen musste. "Ich bekam die Chance, zu zeigen dass ich besser bin als alle drei Generäle und ich habe sie ergriffen. Eine Menge Leute hassen mich jetzt, aber es gibt auch viele Slytherins, die mich dafür lieben werden. Weshalb glauben Sie, dass ich -"

"Den Ablauf der heutigen Schlacht haben nicht Sie geplant, Mr. Zabini. Sagen Sie mir, wer es gewesen ist."

Blaise schluckte schwer. "Nun… ich meine, in dem Fall… dann wissen Sie doch bereits, wer es war, oder nicht? Der einzige der so verrückt ist, ist Dumbledore. Und er wird mich beschützen, wenn Sie irgendwas versuchen."

"In der Tat. Nennen Sie mir den Preis." Die Augen des Verteidigungsprofessors blieben hart.

"Es ist meine Cousine Kimberly," sagte Blaise, schluckte erneut und versuchte, seine Stimme zu beherrschen. "Es gibt sie wirklich und sie wird wirklich gemobbt, Potter hat das nachgeprüft, er war ja nicht dumm. Nur dass Dumbledore sagte, er hätte die Mobber dazu angestiftet, einfach wegen des Plans und wenn ich für \emph{ihn} arbeite, dann hätte sie danach keine Probleme mehr, aber wenn ich mich für \emph{Potter} entscheide, könnte Kimberly noch mehr Schwierigkeiten bekommen!"

Professor Quirrell blieb einen langen Moment still.

"Ich verstehe," sagte Professor Quirrell, nun mit deutlich sanfterer Stimme. "Mr. Zabini, sollte sich ein solches Ereignis noch einmal wiederholen, so dürfen Sie sich damit direkt an mich wenden. Ich habe eigene Möglichkeiten, meine Freunde zu schützen. Nun, eine letzte Frage: Selbst mit all der Macht, die Sie in Händen hielten, wäre es schwierig gewesen ein Unentschieden zu erzwingen. Gab Dumbledore Ihnen Anweisung, wer andernfalls gewinnen sollte?"

"Sunshine," sagte Blaise.

Professor Quirrell nickte. "Das dachte ich mir." Der Verteidigungsprofessor seufzte. "Für die Zukunft, Mr. Zabini, würde ich Ihnen von solch komplizierten Plänen abraten. Sie neigen dazu, fehlzuschlagen."

"Ähm, tatsächlich habe ich das auch dem Schulleiter gesagt," sagte Blaise, "und er meinte, deshalb sei es wichtig, immer mehr als nur einen Plan zu haben."

Erschöpft fuhr sich Professor Quirrell mit der Hand über die Stirn. "Es ist ein Wunder, dass der Dunkle Lord im Kampf gegen \emph{ihn} nicht wahnsinnig wurde. Sie dürfen nun weiter zu Ihrem Treffen mit dem Schulleiter, Mr. Zabini. Ich werde über diese Begegnung schweigen, doch sollte der Schulleiter irgendwie von unserem Gespräch erfahren, so denken Sie an mein stehendes Angebot, Ihnen jeden Schutz zu bieten, den ich kann. Sie dürfen gehen."

Blaise wartete kein weiteres Wort ab, sondern wandte sich um und floh.

Eine Zeit lang wartete Professor Quirrell noch ab, dann sagte er, "Nur zu, Mr. Potter."

Harry riss sich den Unsichtbarkeitsumhang vom Kopf und stopfte ihn in seinen Beutel. Er zitterte so sehr vor Zorn, dass er kaum sprechen konnte. "\emph{Was?} Er hat \emph{was?}"

"Sie hätten selbst darauf kommen sollen, Mr. Potter," sagte Professor Quirrell sanft. "Sie müssen lernen, die Dinge distanzierter zu betrachten, bis Sie den Wald sehen, den die Bäume verbergen. Jeder, der die Geschichten über Sie hören würde und nicht wüsste, dass Sie der mysteriöse Junge-der-überlebt-hat sind, könnte mit Leichtigkeit darauf schließen, dass Sie im Besitz eines Unsichtbarkeitsumhanges sind. Nun zu diesen Ereignissen hier; treten Sie einen Schritt zurück, konzentrieren Sie sich nicht zu sehr auf die Details und was sehen wir dann? Eine große Rivalität zwischen den Schülern und ihr Wettstreit endete in einem perfekten Unentschieden. Solche Dinge passieren nur in Geschichten, Mr. Potter und es gibt einen Menschen in dieser Schule, der in Geschichten denkt. Ein seltsamer und komplizierter Plan, den Sie als uncharakteristisch für den jungen Slytherin hätten erkennen sollen, mit dem Sie es zu tun hatten. Aber es gibt einen Menschen in dieser Schule, der sich mit solch elaborierten Plänen befasst und sein Name ist nicht Zabini. Und ich warnte Sie davor, es gäbe einen Vierfachagenten unter Ihnen; Sie wussten dass Zabini mindestens ein Dreifachagent war und hätten mit hoher Wahrscheinlichkeit darauf tippen sollen, dass er es war. Nein, ich werde die Schlacht nicht für ungültig erklären. Sie alle drei haben den Test nicht bestanden und gegen Ihren gemeinsamen Feind verloren."

Irgendwelche Tests waren Harry mittlerweile völlig egal. "Dumbledore hat Zabini \emph{erpresst} und \emph{seine Cousine bedroht?} Nur damit unsere Schlacht in einem Unentschieden endet? \emph{Warum?}"

Professor Quirrell lachte freudlos. "Vielleicht war der Schulleiter der Ansicht, die Rivalität täte seinem kleinen Helden gut und wünschte, dass sie nicht endet. Für das größere Wohl, Sie verstehen. Oder vielleicht ist er einfach wahnsinnig. Sehen Sie, Mr. Potter, jedermann weiß dass Dumbledores Wahnsinn eine Maske ist, dass er bei Verstand ist und vorgibt, er wäre verrückt. Man brüstet sich mit dieser cleveren Erkenntnis und da man die geheime Wahrheit kennt, forscht man nicht weiter nach. Es kommt ihnen nicht in den Sinn, dass es \emph{ebenfalls} möglich ist, eine Maske hinter der Maske zu tragen; ein Wahnsinniger zu sein, der vorgibt bei Verstand zu sein und so zu tun als sei er verrückt. Und ich fürchte, Mr. Potter, dass ich nun gehen und mich andernorts wichtigen Angelegenheiten widmen muss; doch ich lege Ihnen dringend ans Herz, sich nicht nach Albus Dumbledore zu richten, wenn Sie einen Krieg zu führen haben. Bis später, Mr. Potter."

Der Verteidigungsprofessor neigte mit einiger Ironie den Kopf und schritt dann fort in die Richtung, in die Zabini geflohen war, während Harry noch immer schockiert und mit offenem Mund da stand.

--------------------------------------------------------------------------------------------------------------------------------------------

\emph{Nachspiel: Harry Potter.}

Harry trottete langsam in Richtung des Schlafsaals der Ravenclaws, nahm weder die Wände oder Gemälde, noch die anderen Schüler wahr; er ging Stufen hinauf und Rampen hinab ohne das Tempo zu ändern oder zu merken, wohin er ging.

Nach Professor Quirrells Verschwinden hatte es über eine Minute gedauert, bis er feststellte, dass seine einzigen Informationsquellen bezüglich Dumbledores Beteiligung (a) Blaise Zabini, dem noch einmal zu trauen er schon ein vollkommener, sabbernder Idiot hätte sein müssen und (b) Professor Quirrell waren, der spielend einfach selbst einen Plan in Dumbledores Stil hätte fälschen können, der ein wenig Rivalität unter den Schülern ebenfalls für eine gute Sache halten mochte und der, wenn man mal die Details außer Acht ließ, gerade vorgeschlagen hatte, das Land in eine magische Diktatur zu verwandeln.

Und es war ebenso möglich, dass Dumbledore \emph{tatsächlich} hinter Zabini steckte und dass Professor Quirrell wirklich nur versucht hatte, dem Dunklen Mal in gleicher Weise entgegenzutreten und zu verhindern, dass sich eine Vorstellung, die er als erbärmlich betrachtete, noch einmal wiederholte. Versucht hatte, dafür zu sorgen, dass Harry dem Dunklen Lord am Ende nicht allein entgegentreten musste, während sich alle anderen ängstlich versteckten, versuchten aus der Schusslinie zu bleiben und darauf warteten, dass Harry Potter sie retten kam.

Aber um ehrlich zu sein…

Nun…

Harry war das eigentlich irgendwie ganz recht.

Ihm war klar, dass durch genau solche Sachen Helden für gewöhnlich wütend und verbittert wurden.

Zur Hölle damit. Harry war es verdammt recht, wenn alle anderen \emph{einfach} \emph{außer Gefahr blieben}, während der Junge-der-überlebt-hat den Dunklen Lord allein ausschaltete, plus oder minus einer kleinen Anzahl von Gefährten. Sollte der nächste Konflikt mit dem Dunklen Lord zu einem Zweiten Zaubererkrieg eskalieren, der ein ganzes Land erfassen würde und viele Menschen mit in den Tod riss, so hätte Harry \emph{bereits versagt.}

Und falls im Anschluss daran ein Krieg zwischen Zauberern und Muggeln ausbräche, so würde es keine Rolle spielen wer gewann, Harry hätte bereits versagt, indem er es soweit kommen ließ. Außerdem, wer sagte denn, dass die Gesellschaften nicht friedlich ineinander übergehen konnten, wenn die Geheimhaltung unausweichlich versagte? (Natürlich hörte Harry im Geiste schon Professor Quirrells trockene Stimme, die ihn fragte ob er ein Narr sei und all die offensichtlichen Dinge sagte…) Und wenn die Magier und Muggel nicht in Frieden zusammenleben konnten, so würde Harry eben Magie und Wissenschaft vereinen und herausfinden, wie er alle Zauberer zum Mars evakuieren konnte***** oder sonst irgendwohin, bevor er einen Krieg ausbrechen ließe.

Denn wenn ein Auslöschungskrieg tatsächlich unvermeidlich wäre…

Das war die eine Sache, die Professor Quirrell nicht erkannt hatte, die eine wichtigste Frage, die er seinem jungen General zu stellen vergessen hatte.

Der wahre Grund warum Harry keinesfalls die Absicht hatte, sich zur Unterstützung eines Lichten Mals überreden zu lassen, egal \emph{wie} sehr es ihm in seinem Kampf gegen den Dunklen Lord auch helfen mochte.

Ein Dunkler Lord und fünfzig Anhänger, die das Mal trugen, waren eine Geißel für das ganze magische Britannien gewesen.

Trüge ganz Britannien das Mal eines starken Anführers, so wäre es eine Geißel für die gesamte Zauberwelt.

Und trüge die gesamte Zauberwelt ein einziges Mal, so wäre sie eine Gefahr für den Rest der Menschheit.

Niemand wusste ganz genau, wie viele Zauberer es auf der Welt gab. Er hatte zusammen mit Hermine ein paar Hochrechnungen angestellt und sie waren auf Zahlen grob im Bereich von etwa einer Million gekommen.

Aber es gab sechs Milliarden Muggel.

Wenn es zu einem finalen Krieg käme…

Professor Quirrell hatte vergessen Harry zu fragen, welche Seite er wohl beschützen würde.

Eine wissenschaftliche Zivilisation, den Blick zum Himmel gewandt in dem Wissen, dass es ihr Schicksal war, nach den Sternen zu greifen.

Und eine magische Zivilisation, langsam verblassend, während ihr Wissen verloren ging und noch immer beherrscht von einer Aristokratie, die Muggel nicht als vollwertige Menschen ansah.

Es war ein unendlich trauriger Gedanke, doch er enthielt nicht den leisesten Hauch eines Zweifels.

--------------------------------------------------------------------------------------------------------------------------------------------

\emph{Nachspiel: Blaise Zabini.}

Blaise durchstreifte die Korridore mit sorgsamer, selbstauferlegter Gemächlichkeit während sein Herz wie wild pochte und er versuchte zur Ruhe zu kommen -

"Ähem," erklang eine trockene, flüsternde Stimme als er eine schattige Nische in der Wand passierte.

Blaise fuhr zusammen, schrie aber nicht auf.

Langsam wandte er sich um.

In jener kleinen dunklen Ecke war ein schwarzer Mantel auszumachen, so ausladend und bauschig, dass nicht einmal erkennbar war, ob die Gestalt darunter männlich oder weiblich war und darüber ein breitkrempiger schwarzer Hut unter dem sich ein schwarzer Nebel zu sammeln schien, der das Gesicht dessen verbarg, wer oder was auch immer sich darunter befinden mochte.

"Bericht," flüsterte Mr. Hut-und-Mantel.

"Ich habe genau das gesagt, was Sie mir aufgetragen haben," sagte Blaise. Seine Stimme war ein wenig ruhiger, nun da es nicht mehr nötig war, zu lügen. "Und Professor Quirrell hat genauso reagiert, wie Sie es erwartet haben."

Der breite schwarze Hut neigte sich und richtete sich wieder auf, als habe der Kopf darunter genickt. "Exzellent," erklang das unidentifizierbare Flüstern. "Die Belohnung, die ich Ihnen versprach ist bereits unterwegs zu Ihrer Mutter, per Eule."

Blaise zögerte, aber die Neugier fraß ihn bei lebendigem Leibe auf. "Darf ich nun fragen, weshalb Sie für Ärger zwischen Professor Quirrell und Dumbledore sorgen wollen?" Der Schulleiter hatte, soweit Blaise das wusste, nichts mit den Mobbern aus Gryffindor zu tun und zusätzlich dazu, Kimberly zu helfen, hatte der Schulleiter auch noch angeboten, dafür zu sorgen, dass Professor Binns ihm exzellente Noten in Geschichte der Zauberei geben würde, selbst wenn er als Hausaufgaben nur noch leere Pergamentblätter einreichte, wobei er allerdings noch immer den Unterricht besuchen und so tun musste als würde er sie einreichen. Tatsächlich hätte Blaise alle drei Generäle auch ganz umsonst verraten und seine Cousine kümmerte ihn genauso wenig, aber das hatte er ja nicht unbedingt sagen müssen.

Der breite schwarze Hut neigte sich leicht zur Seite als wolle er einen fragenden Blick vermitteln. "Sagen Sie, Freund Blaise, kam es Ihnen jemals in den Sinn, dass Verräter, die andere so viele Male verraten, oft ein schlimmes Schicksal ereilt?"

"Nope," sagte Blaise und blickte geradewegs in den schwarzen Nebel unter dem Hut. "Jeder weiß doch, dass Schülern in Hogwarts niemals etwas \emph{wirklich} schlimmes zustößt."

Mr. Hut-und-Mantel gab ein leises Lachen von sich. "In der Tat," sagte das Flüstern. "Mit der Ausnahme des Mordes an einem Schüler vor fünfzig Jahren, die die Regel bestätigt, da Salazar Slytherin sein Monster auf noch höherer Ebene in die alten Schutzzauber von Hogwarts verwoben hatte als der Schulleiter selbst."

Blaise starrte auf den schwarzen Nebel und wurde nun doch ein wenig nervös. Aber es sollte schon ein Professor von Hogwarts nötig sein, um ihm signifikanten Schaden zuzufügen, ohne das Alarm ausgelöst würde. Quirrell und Snape waren die einzigen Professoren, die so etwas tun würden; Quirrell würde kein Interesse daran haben, \emph{sich selbst} zu täuschen und Snape würde doch keinen seiner eigenen Slytherins verletzen… oder doch?

"Nein, Freund Blaise," flüsterte der schwarze Nebel, "ich wollte Ihnen lediglich nahe legen, so etwas niemals als Erwachsener zu versuchen. So viele Fälle von Verrat würden sicherlich mindestens einen Racheakt nach sich ziehen."

"An meiner \emph{Mutter} hat sich niemals jemand gerächt," sagte Blaise stolz. "Obwohl sie \emph{sieben} Ehemänner geheiratet hat und jeder einzelne von ihnen unter mysteriösen Umständen verstarb und ihr eine Menge Geld hinterließ."

"Wirklich?" sagte das Flüstern. "Wie aber hat sie den siebten dazu gebracht, sie zu heiraten, nachdem er gehört hatte, was mit den ersten sechs geschehen war?"

"Das habe ich Mum auch gefragt," sagte Blaise, "und sie sagte, das würde ich erst erfahren, wenn ich alt genug wäre und als ich fragte, wie alt denn alt genug sei, da meinte sie älter als sie."

Erneut das leise Lachen. "Nun denn, Freund Blaise, meinen Glückwunsch, dass Sie in Ihrer Mutter Fußstapfen getreten sind. Nun gehen Sie und so Sie nichts hiervon erwähnen, werden wir uns nicht wieder treffen."

Nervös wich Blaise zurück und fühlte ein seltsames Widerstreben sich umzudrehen.

Der Hut neigte sich. "Oh, nun kommen Sie schon, kleiner Slytherin. Wenn Sie Harry Potter oder Draco Malfoy wirklich ebenbürtig wären, so hätten Sie doch bereits begriffen, dass meine versteckten Drohungen nur Ihr Stillschweigen Albus gegenüber sicherstellen sollten. Hätte ich Ihnen etwas antun wollen, so hätte ich keinen Hinweis gegeben; hätte ich nichts gesagt, \emph{dann} hätten Sie sich sorgen sollen."

Blaise straffte sich, ein klein wenig beleidigt und nickte Mr. Hut-und-Mantel zu; dann wandte er sich entschieden um und schritt davon zu seinem Treffen mit dem Schulleiter.

Bis zum Ende hatte er gehofft, es würde noch jemand auftauchen und ihm die Chance geben Mr. Hut-und-Mantel ans Messer zu liefern.

Allerdings hatte Mum auch keine sieben verschiedenen Ehemänner \emph{gleichzeitig} betrogen. Wenn man es \emph{so} betrachtete, machte er sich immer noch besser als sie.

Und so setzte Blaise Zabini lächelnd seinen Weg zum Büro des Schulleiters fort, zufrieden damit ein Fünffachagent zu sein -

Einen Moment lang geriet der Junge ins Stolpern, dann fing er sich wieder und schüttelte das merkwürdige Gefühl der Desorientierung ab.

Und so setzte Blaise Zabini lächelnd seinen Weg zum Büro des Schulleiters fort, zufrieden damit ein Vierfachagent zu sein -

--------------------------------------------------------------------------------------------------------------------------------------------

\emph{Nachspiel: Hermine Granger.}

Der Bote näherte sich erst als sie allein war.

Hermine verließ gerade die Mädchen-Toilette, wo sie sich manchmal versteckte, um nachzudenken als eine hell glänzende Katze aus dem Nirgendwo hervorsprang und sagte, "Miss Granger?"

Ihr entfuhr ein kleiner Schrei bevor sie erkannte, dass die Katze mit Professor McGonagalls Stimme gesprochen hatte.

Ohnehin hatte sie sich nicht gefürchtet, nur erschreckt; die Katze war hell und strahlend und wunderschön, sie leuchtete mit einer silberweißen Aura wie mondscheinfarbenes Sonnenlicht und sie konnte sich einfach nicht vorstellen, sich davor zu fürchten.

"Was bist du?" sagte Hermine.

"Dies ist eine Nachricht von Professor McGonagall," sagte die Katze, noch immer mit der Stimme der Professorin. "Könnten Sie in mein Büro kommen und mit niemandem hierüber sprechen?"

"Ich bin sofort da," sagte Hermine, noch immer überrascht und die Katze machte einen Satz und verschwand; nur verschwand sie nicht, sondern entfernte sich auf geheimnisvolle Weise; zumindest sagte ihr das ihr Verstand, obwohl ihre Augen sie gerade hatten verschwinden sehen.

Als Hermine am Büro ihrer Lieblingsprofessorin eintraf, schwirrte ihr der Kopf vor lauter Mutmaßungen. Stimmte irgendetwas nicht mit ihren Noten in Verwandlung? Aber warum würde Professor McGonagall dann darum bitten, es niemandem zu sagen? Es ging wahrscheinlich um Harrys Übungen in partieller Transfiguration…

Professor McGonagalls Gesichtsausdruck zeigte Sorge, keine Strenge, als Hermine sich vor dem Schreibtisch setzte - wobei sie sich bemühte ihre Augen nicht zu der Ansammlung kleiner Kämmerchen wandern zu lassen, die Professor McGonagall Hausarbeiten enthielten; sie hatte sich schon immer gefragt, was für Arbeiten die Erwachsenen wohl zu verrichten hatten, um die Schule am Laufen zu halten und ob sie dabei vielleicht ihre Hilfe brauchen konnten…

"Miss Granger," sagte Professor McGonagall, "lassen Sie mich damit beginnen, Ihnen zu sagen, dass ich bereits weiß, dass der Schulleiter Sie gebeten hat, diesen Wunsch zu äußern -"

"Er hat es Ihnen erzählt?" platzte Hermine erschrocken heraus. Der Schulleiter hatte doch gesagt, niemand sonst sollte es wissen!

Professor McGonagall hielt inne, blickte Hermine an und ließ ein trauriges leises Lachen vernehmen. "Es tut gut zu sehen, dass Mr. Potter Sie noch nicht allzu sehr verdorben hat. Miss Granger, Sie sollten etwas nicht unbedingt \emph{zugeben}, nur weil ich sage, ich wüsste es schon. Tatsächlich hat der Schulleiter es mir \emph{nicht} erzählt, ich kenne ihn einfach nur zu gut."

Hermine stieg die Zornesröte ins Gesicht.

"Ist schon in Ordnung, Miss Granger!" sagte Professor McGonagall hastig. "Sie sind eine Ravenclaw in Ihrem ersten Schuljahr, niemand erwartet von Ihnen eine Slytherin zu sein."

\emph{Das} tat weh.

"Schön," erwiderte Hermine leicht säuerlich, "dann werde ich Harry Potter wohl um Slytherin-Lehrstunden bitten müssen."

"Das wollte ich damit \emph{nicht}…" setzte Professor McGonagall an, doch ihre Stimme verklang. "Miss Granger, ich mache mir Sorgen, denn junge Mädchen aus Ravenclaw \emph{sollten} keine Slytherins sein müssen! Wenn der Schulleiter Sie um etwas bittet, bei dem Sie sich nicht wohl fühlen, Miss Granger, dann ist es wirklich völlig in Ordnung, wenn Sie nein sagen. Und falls Sie sich unter Druck gesetzt fühlen, dann sagen Sie dem Schulleiter doch bitte, dass Sie mich gern dabei hätten oder mich lieber erst fragen würden."

Hermines Augen wurden sehr groß. "Tut der Schulleiter denn Dinge, die falsch sind?"

Daraufhin wirkte Professor McGonagall ein wenig traurig. "Nicht absichtlich, Miss Granger, doch ich glaube… nun, es ist \emph{trifft} wahrscheinlich zu, dass es dem Schulleiter manchmal schwer fällt sich zu erinnern, wie es ist ein Kind zu sein. Auch als er selbst noch ein Kind war, muss er sicherlich brillant gewesen sein und stark im Geist und Herzen, mit Mut genug für ganze drei Gryffindors. Manchmal verlangt der Schulleiter zu viel von seinen jungen Schülern, Miss Granger oder er ist nicht genug auf ihr Wohlergehen bedacht. Er ist ein guter Mann, doch manchmal gehen seine Pläne zu weit."

"Aber es ist doch \emph{gut}, wenn Schüler stark und mutig sind," sagte Hermine. "Deshalb hatten Sie mir doch Gryffindor ans Herz gelegt, nicht wahr?"

Professor McGonagall lächelte schief. "Vielleicht war ich auch nur selbstsüchtig, Sie für mein eigenes Haus zu wollen. Hat der Sprechende Hut Ihnen angeboten - nein, ich hätte nicht fragen sollen."

"Er sagte mir, ich könne überall hin außer nach Slytherin," sagte Hermine. \emph{Beinahe} hätte sie gefragt, weshalb sie nicht gut genug für Slytherin sei, bevor sie sich gerade noch beherrschen konnte… "Also \emph{habe} ich Mut, Professor!"

Professor McGonagall lehnte sich über den Schreibtisch vor. Die Sorge zeichnete sich nun noch deutlicher in ihrem Gesicht ab. "Miss Granger, hier geht es nicht um Mut, sondern darum was für junge Mädchen auch gesund ist! Der Schulleiter zieht Sie in seine Pläne mit hinein, Harry Potter vertraut Ihnen seine Geheimnisse an und jetzt schmieden Sie Allianzen mit Draco Malfoy! Und ich habe Ihrer Mutter versprochen, dass Sie in Hogwarts sicher seien!"

Hermine hatte keine Ahnung, was sie darauf erwidern sollte. Aber ihr kam der Gedanke, dass Professor McGonagall sie womöglich nicht gewarnt hätte, wäre sie ein Junge in Gryffindor gewesen anstatt ein Mädchen in Ravenclaw und \emph{das} war, nun ja… "Ich werde versuchen, gut zu sein," sagte sie "und mir von niemandem etwas anderes sagen lassen."

Professor McGonagall presste sich mit den Händen auf die Augen. Als sie sie wieder sinken ließ, wirkte ihr gezeichnetes Gesicht sehr alt. "Ja," sagte sie mit einem Flüstern, "Sie hätten sich gut gemacht in meinem Haus. Passen Sie auf sich auf, Miss Granger und seien Sie vorsichtig. Und sollten Sie jemals über irgendetwas besorgt oder beunruhigt sein, dann zögern Sie bitte nicht, direkt zu mir zu kommen. Ich werde Sie nicht länger aufhalten."

--------------------------------------------------------------------------------------------------------------------------------------------

\emph{Nachspiel: Draco Malfoy.}

Keiner von ihnen wollte an diesem Samstag irgendetwas kompliziertes machen, nicht nachdem sie zuvor eine Schlacht geschlagen hatten. Also saß Draco einfach nur in einem ungenutzten Klassenzimmer und versuchte ein Buch zu lesen, mit dem Titel \emph{Denksport-Physik}.****** Es gehörte zum faszinierendsten, was Draco je in seinem Leben gelesen hatte, zumindest die Teile, die er verstand; jedenfalls wenn dieser \emph{verfluchte Idiot}, der sich weigerte seine Bücher aus den Augen zu lassen, es endlich fertig brächte \emph{die Klappe zu halten}, damit er sich \emph{konzentrieren} könnte -

"Hermine Granger ist ein \emph{Schlaaammbluuut}," sang Harry Potter von seinem nahegelegenen Schreibpult aus, wo er selbst ein sehr viel fortgeschritteneres Buch las.

"Ich weiß, was du da versuchst," sagte Draco ruhig, ohne von den Seiten aufzublicken. "Es wird nicht funktionieren. Wir tun uns trotzdem zusammen und machen dich fertig."

"Ein \emph{Maaaalfoy} arbeitet zusammen mit einem \emph{Schlaaammbluuut}, was werden all die Freunde deines \emph{Vaaaaters} nur denken -"

"Sie werden denken, dass Malfoys nicht so einfach zu manipulieren sind, wie du anscheinend glaubst, \emph{Potter!}"

Der Verteidigungsprofessor war noch verrückter als Dumbledore, kein zukünftiger Retter der Welt könnte sich jemals so \emph{kindisch} und \emph{würdelos} aufführen, egal in welchem Alter.

"Hey, Draco, weißt du was dich wirklich ankotzen wird? \emph{Du} weißt, dass Hermine Granger zwei Kopien des magischen Allels hat, genauso wie du und genauso wie ich, aber all deine Klassenkameraden in Slytherin wissen das nicht und \emph{duuuhuuu} darfst es ihnen nicht \emph{saaaaagen} -"

Dracos Finger schlossen sich immer fester um das Buch und traten weiß hervor. Sich herumschubsen und anspucken zu lassen, konnte unmöglich so viel Selbstbeherrschung erfordern und wenn er es Harry nicht bald heimzahlen könnte, würde er sich noch zu etwas strafbarem hinreißen lassen -

"Also, was \emph{hast} du dir nun eigentlich beim ersten Mal gewünscht?" sagte Draco.

Harry antwortete nicht, also blickte Draco von seinem Buch auf und fühlte einen Stich boshafter Genugtuung als er den traurigen Ausdruck auf Harrys Gesicht bemerkte.

"Ähm," sagte Harry. "Das haben mich schon eine Menge Leute gefragt, aber ich glaube nicht, dass Professor Quirrell gewollt hätte, dass ich darüber rede."

Draco selbst setzte eine ernste Miene auf. "Aber mit \emph{mir} kannst du doch darüber reden. Es ist doch wahrscheinlich gar nicht so wichtig, verglichen mit den anderen Geheimnissen, die du mir schon verraten hast und wofür sonst sind Freunde da?" \emph{Ja, genau, ich bin dein Freund! Fühl dich schuldig!}

"So interessant war es überhaupt nicht," sagte Harry mit offensichtlich gespielter Leichtfertigkeit. "Nur, \emph{ich wünschte Professor Quirrell würde auch im nächsten Jahr wieder Kampf-Magie unterrichten.}"

Harry seufzte und vertiefte sich wieder in sein Buch.

Nach ein paar Sekunden sagte er, "Dein Vater wird dieses Weihnachten wahrscheinlich ziemlich sauer auf dich sein, aber wenn du ihm versprichst, das Schlammblut-Mädchen zu verraten und ihre Armee auszulöschen, kommt bestimmt alles wieder in Ordnung und du kriegst trotzdem noch deine Weihnachtsgeschenke."

Vielleicht würde Professor Quirrell, wenn er und Granger extra höflich fragten und mit ein paar ihrer Quirrell-Punkte nachhalfen, ihnen beiden ja erlauben, mit General Chaos etwas interessanteres anzustellen, als ihn nur schlafen zu schicken.

* Ein Zitat aus \emph{Zen und die Kunst ein Motorrad zu warten}(engl.: \emph{Zen and the Art of Motorcycle Maintenance}) von Robert M. Pirsig, frei übersetzt.\\ ** Harry verwendet hier im englischen den Begriff \emph{Single Point of Failure}, für den ich leider keine direkte Übersetzung finden konnte. Er bezeichnet einen einzelnen Bestandteil eines technischen (oder in diesem Falle Gesellschafts-)Systems, dessen Versagen das gesamte System gefährdet.\\ *** engl.: \emph{Rise and Fall of the Third Reich}\\ **** Hier dürfte die Rede sein von Ronald Reagan, der von 1981 bis 1989 US-Präsident war, bevor er von George Bush sr. abgelöst wurde. Womöglich auch eine Anspielung auf den bereits zuvor erwähnten Film \emph{Zurück in die Zukunft}, in dem der Doc Brown des Jahres 1955 seinen Unglauben darüber äußert, "Ronald Reagan? Der Schauspieler?!" könne im Jahr 1985 Präsident sein.\\ ***** Offenbar eine Anspielung auf die Manga- und Animeserie \emph{Magister Negi Magi}, dort befindet sich die Magische Welt auf dem Mars.\\ ****** engl.: \emph{Thinking Physics}. Die Erstausgabe ist offenbar ohne deutsche Übersetzung, allerdings ist die Dritte Edition mit dem Titel \emph{Thinking Physics, Understandable Practical Reality} als \emph{Denksport-Physik: Fragen und Antworten} auf deutsch verfügbar.

