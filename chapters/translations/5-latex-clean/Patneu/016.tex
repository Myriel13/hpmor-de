

\hypertarget{querdenken}{% \section{16. Querdenken}\label{querdenken}}

\textbf{Kapitel 16: Querdenken}

Das Tor des Gegners ist Rowling.

\later

\emph{Ich bin kein Psychopath, ich bin nur sehr kreativ.}

\later

Gleich als er am Mittwoch den Klassenraum für Verteidigung betrat, wusste Harry, \emph{dieses} Fach würde \emph{anders} sein.

Es war, zunächst einmal, das größte Klassenzimmer, das er bislang in Hogwarts gesehen hatte, ähnlich einem Hörsaal einer bedeutenden Universität, mit in Stufen ansteigenden Reihen von Schreibpulten, einer gigantischen ebenen Bühne aus weißem Marmor zugewandt. Der Klassenraum lag hoch oben im Schloss -- im fünften Stock -- und Harry wusste, mehr an Erklärung würde er dafür nicht bekommen, wo ein solcher Raum Platz finden sollte. Es wurde klar, dass Hogwarts einfach über keinerlei Geometrie \emph{verfügte,} euklidisch oder sonstwie; es hatte Verbindungen, keine Richtungen.

Anders als in einem Universitätssaal, gab es keine Reihen mit Klappsitzen; stattdessen ganz gewöhnliche hölzerne Hogwarts-Schreibpulte und -Stühle, in einer Kurve auf jeder Ebene des Klassenraumes aufgereiht. Außer dass auf jedem Pult ein flaches, weißes, viereckiges, mysteriöses Objekt aufgerichtet war.

Im Zentrum der riesigen Bühne, auf einem kleinen erhöhten Podium aus dunklerem Marmor, befand sich ein einzelnes Lehrerpult. An welchem Quirrell zusammengesackt in seinem Stuhl saß, den Kopf nach hinten hängend und leicht auf seinen Umhang sabbernd.

\emph{An was erinnert mich das jetzt…?}

Harry war so früh zum Unterricht erschienen, dass noch keine anderen Schüler anwesend waren. (Die englische Sprache erwies sich als unzureichend, was die Beschreibung von Zeitreisen betraf; insbesondere ließ das Englische jedwede Worte vermissen, um ausdrücken zu können, wie praktisch sie waren.) Quirrell schien im Augenblick nicht… funktionstüchtig… zu sein und Harry fühlte ohnehin kein großes Verlangen, ihn anzusprechen.

Harry wählte ein Schreibpult aus, stieg zu ihm hinauf und zog das Lehrbuch für Verteidigung heraus. Er war etwa zu sieben Achteln durch -- er hatte eigentlich geplant, mit dem Buch vor dieser Stunde fertig zu sein, aber er lag hinter dem Zeitplan und hatte seinen Zeitumkehrer heute bereits zweimal benutzt.

Bald erklangen Geräusche, als der Klassenraum sich zu füllen begann. Harry ignorierte sie.

"Potter? Was machst \emph{du} hier?"

\emph{Diese} Stimme gehörte nicht hierher. Harry blickte auf. "Draco? Was tust \emph{du} denn in oh mein Gott, du hast \emph{Lakaien.}"

Einer der Burschen, die hinter Draco standen, schien für einen Elfjährigen auffällig viele Muskeln zu haben und der andere nahm eine verdächtig ausbalanciert aussehende Haltung an.

Der Junge mit den weiß-blonden Haaren lächelte ziemlich selbstzufrieden und deutete hinter sich. "Potter, darf ich vorstellen, Mr~Crabbe," seine Hand wanderte von Muskeln zu Gleichgewicht, "Mr~Goyle. Vincent, Gregory, das ist Harry Potter."

Mr~Goyle neigte den Kopf und warf Harry einen Blick zu, der wahrscheinlich irgendetwas bedeuten sollte, aber letztendlich nur nach einem Schielen aussah. Mr~Crabbe sagte "Sehr erfreut" in einem Tonfall, der klang, als versuche er, mit so tiefer Stimme zu sprechen, wie er konnte.

Ein flüchtiger Ausdruck der Verblüffung huschte über Dracos Gesicht, wurde aber schnell von seinem überlegenen Grinsen abgelöst.

"Du hast \emph{Lakaien!}" wiederholte Harry. "Wo kriege \emph{ich} Lakaien her?"

Dracos Grinsen wurde breiter. "Ich fürchte, Potter, der erste Schritt besteht darin, nach Slytherin sortiert zu werden -"

"Was? Das ist unfair!"

"- und dann müssen eure Familien bereits vor eurer Geburt eine Abmachung getroffen haben."

Harry blickte zu Mr~Crabbe und Mr~Goyle. Beide schienen nach Kräften darum bemüht, bedrohlich zu wirken. Was hieß, sie lehnten sich nach vorn, ließen ihre Schultern hervortreten, drückten die Nacken durch und starrten ihn an.

"Ähm… warte mal," sagte Harry. "Das wurde schon vor \emph{Jahren} arrangiert?"

"So ist es, Potter. Ich befürchte, da hast du kein Glück."

Mr~Goyle zauberte einen Zahnstocher hervor und begann seine Zähne zu reinigen, noch immer bedrohlich aussehend.

"Und," sagte Harry, "Lucius bestand darauf, dass du \emph{nicht} mit deinen Bodyguards zusammen aufwachsen und sie erst an deinem ersten Schultag treffen solltest."

Das wischte das Grinsen aus Dracos Gesicht. "Ja, Potter, wir wissen alle, dass du brilliant bist, das weiß inzwischen die ganze Schule, du kannst mit dem Angeben aufhören -"

"Also wurde ihnen ihr \emph{ganzes Leben} lang gesagt, sie würden deine Lakaien sein und sie haben sich \emph{jahrelang} vorgestellt, wie Lakaien sein sollten -"

Draco zuckte zusammen.

"- und was noch schlimmer ist, sie \emph{kennen einander} und sie haben \emph{geübt} -"

"Der Boss sagt', sei still," grummelte Mr~Crabbe. Mr~Goyle biss auf seinen Zahnstocher, hielt ihn zwischen den Zähnen und ließ mit einer Hand die Knöchel der anderen knacken.

"\emph{Ich habe euch gesagt, ihr sollt das vor Harry Potter nicht machen!}"

Die beiden sahen ein wenig belämmert drein und Mr~Goyle steckte schnell seinen Zahnstocher zurück in eine Tasche seines Umhangs.

Aber in dem Moment, als Draco sich weg drehte und wieder Harry zuwand, setzten sie die Drohgebärden fort.

"Ich entschuldige mich," sagte Draco steif, "für die Beleidigung, die diese \emph{Schwachköpfe} dir geboten haben."

Harry warf Mr~Crabbe und Mr~Goyle einen bedeutungsvollen Blick zu. "Ich denke, du bist etwas zu hart zu ihnen, Draco. \emph{Ich} finde, sie benehmen sich genauso, wie ich es von \emph{meinen} Lakaien erwarten würde. Ich meine, wenn ich irgendwelche Lakaien hätte."

Draco fiel die Kinnlade herunter.

Hey, Gregory, du glaubs'doch nich', er versucht uns von unser'm Boss abzuwerb'n, oder?"

"Ich bin sicher, Mr~Potter wäre nicht so töricht."

"Oh, nicht einmal im Traum," sagte Harry glatt. "Es ist nur etwas, was man im Hinterkopf behalten sollte, falls euer derzeitiger Arbeitgeber undankbar erscheint. Außerdem schadet es nie, andere Angebote zu haben, wenn man seine Arbeitsbedingungen aushandelt, nicht wahr?"

"Was hat'n \emph{er} in Ravenclaw zu such'n?"

"Ich habe keine Ahnung, Mr~Crabbe."

"Haltet alle beide \emph{die Klappe,}" sagte Draco durch zusammengebissene Zähne. "Das ist ein \emph{Befehl.}" Mit sichtbarer Anstrengung wand er seine Aufmerksamkeit wieder Harry zu. "Jedenfalls, was machst du im Verteidigungs-Unterricht der Slytherins?"

Harry stutzte. "Moment." Seine Hand schlüpfte in seinen Beutel. "Stundenplan." Er überflog das Pergament. "Verteidigung 2:30 Uhr nachmittags und jetzt gerade ist es…" Harry sah auf seine mechanische Armbanduhr, welche 11:23 Uhr zeigte. "2:23 Uhr, wenn ich nicht irre. Oder?" Wenn doch, nun, Harry wusste, wie er zu der Stunde kam, in welcher auch immer er jetzt sein \emph{sollte}. Gott, er liebte seinen Zeitumkehrer und eines Tages, wenn er alt genug war, würden sie heiraten.

"Nein, das klingt richtig," sagte Draco verwirrt. Seine Blick schweifte über den Rest des Hörsaals, der sich mit grün-besetzten Umhängen füllte und…

"\emph{Gryffindödel!}" spuckte Draco aus. "Was machen \emph{die} hier?"

"Hm," sagte Harry. "Professor Quirrell sagte… ich weiß nicht mehr die genauen Worte… er würde einige von Hogwarts Lehrgewohnheiten ignorieren. Vielleicht hat er einfach alle seine Klassen zusammengelegt."

"Huh," sagte Draco. "Du bist der erste Ravenclaw hier drin."

"Jep. Bin früh gekommen."

"Was machst du dann ganz in der hintersten Reihe?"

Harry blinzelte. "Weiß nicht, schien ein guter Platz zu sein?"

Draco machte ein verächtliches Geräusch. "Du könntest nicht weiter vom Lehrer wegkommen, wenn es Absicht wäre." Der blonde Junge beugte sich etwas näher zu ihm. "Ist es eigentlich wahr, was du zu Derrick und seinen Leuten gesagt haben sollst?"

"Wer ist Derrick?"

"Du hast ihn mit einem Kuchen getroffen?"

"Zwei Kuchen, eigentlich. Was soll ich zu ihm gesagt haben?"

"Das er nichts gerissenes oder ehrgeiziges täte und er eine Schande für Salazar Slytherin wäre." Draco betrachtete Harry aufmerksam.

"Das… klingt ungefähr richtig," sagte Harry. "Ich glaube es war mehr wie 'ist das eine Art unglaublich cleverer Plan, der dir zukünftig Vorteile bringen wird oder wirklich eine solche Entehrung von Salazar Slytherins Andenken, wie es aussieht' oder etwas in der Art. Ich weiß die exakten Worte nicht mehr."

"Du verwirrst einfach jeden, weißt du das," sagte der blonde Junge.

"Häh?" sagte Harry ehrlich verwirrt.

"Warrington sagte, lange Zeit unter dem Sprechenden Hut zu verbringen ist eines der Warnzeichen eines großen Dunklen Zauberers. Alle haben darüber gesprochen und sich gefragt, ob sie sich bei dir einschleimen sollten, nur für den Fall. Dann bist du losgezogen und hast eine Bande \emph{Hufflepuffs} beschützt, um Merlins Willen. \emph{Dann} sagtest du Derrick, er ist eine Schande für das Andenken von Salazar Slytherin! Was \emph{sollen} denn alle denken?"

"Das der Sprechende Hut beschlossen hat, mich in's Haus 'Slytherin! Nur ein Scherz! Ravenclaw!' zu stecken und ich mich dem entsprechend verhalten habe."

Mr~Crabbe und Mr~Goyle kicherten beide, woraufhin Mr~Goyle sich schnell die Hand vor den Mund schlug.

"Wir gehen besser auf unsere Plätze," sagte Draco. Er zögerte, straffte sich ein wenig, sprach etwas formeller. "Aber ich möchte unsere letzte Unterhaltung fortsetzen und ich akzeptiere deine Bedingungen."

Harry nickte. "Würde es dir furchtbar viel ausmachen, wenn ich bis Samstag-Nachmittag warten würde? Ich stecke gerade in sowas wie einem Wettstreit."

"Ein Wettstreit?"

"Um rauszufinden, ob ich alle meine Lehrbücher so schnell lesen kann wie Hermine Granger."

"Granger," echote Draco. Seine Augen wurden schmal. "Das Schlammblut, das glaubt, sie ist Merlin? Wenn du versuchst es \emph{ihr} zu zeigen, wünscht dir ganz Slytherin \emph{viel} Glück, Potter und ich werde dich bis Samstag nicht stören." Draco neigte respektvoll den Kopf und ging weg, gefolgt von seinen Lakaien.

\emph{Oh, ich sehe schon, was für ein Spaß es wird, das zu jonglieren.}

Der Klassenraum füllte sich jetzt rasch mit allen vier Farben: grün, rot, gelb und blau. Draco und seine zwei Freunde schienen gerade dabei zu sein, sich drei nebeneinander liegende Sitze in der vorderen Reihe zu sichern -- natürlich bereits besetzt. Mr~Crabbe und Mr~Goyle versuchten nach Kräften, bedrohlich zu wirken, aber viel zu bewirken schien es nicht.

Harry beugte sich über sein Verteidigungs-Lehrbuch und las weiter.

\later

Um 2:35 Uhr nachmittags, als die meisten Sitze schon besetzt waren und niemand mehr herein zu kommen schien, fuhr Professor Quirrell in seinem Stuhl plötzlich auf und saß gerade und sein Gesicht erschien auf all den flachen, weißen viereckigen Objekten, die auf den Pulten der Schüler aufgerichtet waren.

Harry wurde überrascht, sowohl vom plötzlichen Erscheinen von Professor Quirrells Gesicht als auch von der Ähnlichkeit mit dem Muggel-Fernsehen. Es lag etwas sowohl nostalgisches als auch trauriges darin, es schien so sehr wie ein Stückchen Zuhause und war es doch nicht wirklich…

"Guten Tag, meine jungen Lehrlinge," sagte Professor Quirrell. Seine Stimme schien aus dem Bildschirm zu kommen und direkt zu Harry zu sprechen. "Willkommen zu Ihrer ersten Stunde in Kampf-Magie, wie die Gründer von Hogwarts es ausgedrückt hätten; oder, wie es im späten zwanzigsten Jahrhundert genannt wird, Verteidigung gegen die Dunklen Künste."

Es erhob sich das Geräusch hektisch kramender, überraschter Schüler, die nach ihren Pergamenten oder Notizbüchern griffen.

"Nein," sagte Professor Quirrel. "Bemühen Sie sich nicht, niederzuschreiben, wie dieses Fach einst genannt wurde. Keine solch sinnlose Frage wird Teil Ihrer Benotung in irgendeiner meiner Stunden werden. Das ist ein Versprechen."

Viele Schüler saßen daraufhin kerzengerade, sahen einigermaßen schockiert aus.

Professor Quirrell lächelte dünn. "Diejenigen unter Ihnen, die Zeit damit vergeudet haben, Ihre nutzlosen Erstklässler-Lehrbücher zu lesen -"

Jemand gab ein ersticktes Geräusch von sich. Harry fragte sich, ob es Hermine war.

"- mögen den Eindruck gewonnen haben, obwohl dieses Fach sich Verteidigung gegen die Dunklen Künste nennt, handle es eigentlich davon, wie man sich verteidigt gegen Alptraum-Falter, die mäßig schlimme Träume verursachen oder Säure-Schnecken, die sich innerhalb eines knappen Tages komplett durch einen zwei Zoll dicken Holzbalken fressen können."

Professor Quirrell stand auf, schob seinen Stuhl von seinem Pult zurück. Der Bildschirm auf Harrys Schreibpult folgte jeder seiner Bewegungen. Professor schritt zur Front des Klassenraumes und brüllte:

"Der Ungarische Hornschwanz ist größer als ein Dutzend Männer! Er speit Feuer so schnell und so zielsicher, dass er einen Schnatz mitten im Flug schmelzen kann! Ein Tödlicher Fluch bringt ihn zu Fall!"

Die Schüler keuchten auf.

"Der Bergtroll ist gefährlicher als der Ungarische Hornschwanz! Er ist stark genug, um Stahl zu zerbeißen! Seine Haut ist widerstandsfähig genug, um Schock- und Schneidezaubern standzuhalten! Sein Geruchssinn ist so scharf, dass er schon von weitem erkennen kann, ob seine Beute teil einer Meute oder allein und verwundbar ist! Am furchtbarsten von allem, ist der Troll einzigartig unter den magischen Kreaturen darin, ständig eine Art von Verwandlung auf sich selbst aufrecht zu erhalten -- er verwandelt sich immer in seinen eigenen Körper. Wenn es Ihnen irgendwie gelingt, ihm einen Arm auszureißen, wächst ihm in Sekunden ein neuer! Feuer und Säure erzeugen Narbengewebe, die zeitweilig die regenerativen Kräfte des Trolls durcheinander bringen können -- für ein oder zwei Stunden! Sie sind schlau genug, um Keulen als Werkzeug zu verwenden! Der Bergtroll ist die dritt-perfekteste Tötungsmaschine der Natur! Ein Tödlicher Fluch bringt ihn zu Fall."

Die Schüler sahen ziemlich geschockt aus.

Professor Quirrell lächelte grimmig. "Ihr trauriger Ersatz für ein Drittklässler-Verteidigungs-Lehrbuch wird Ihnen vorschlagen, den Bergtroll dem Sonnenlicht auszusetzen, wodurch er an Ort und Stelle erstarrt. Dies, meine jungen Lehrlinge, ist jene Art nutzloses Wissen, wie Sie es niemals in meinem Unterricht vorfinden werden. Man begegnet keinen Bergtrollen am hellichten Tag! Die Idee, dass Sie Sonnenlicht einsetzen sollten, um sie aufzuhalten, ist das Ergebnis törichter Lehrbuch-Schreiber, die ihr meisterhaftes Trivialwissen zur Schau stellen, auf Kosten der Zweckmäßigkeit. Nur weil es einen lächerlich verworrenen Weg gibt, mit Bergtrollen fertig zu werden, heißt das nicht, dass Sie ihn tatsächlich ausprobieren sollten! Der Tödliche Fluch kann nicht abgewehrt werden, ist unaufhaltsam und funktioniert jedes einzelne Mal bei allem, was ein Gehirn hat. Wenn Sie, als erwachsener Zauberer, nicht in der Lage sind, den Tödlichen Fluch zu verwenden, dann können Sie einfach disapparieren! Ebenso, wenn Sie der zweit-perfektesten Tötungsmaschine begegnen, einem Dementor. Sie disapparieren einfach!"

"Außer, natürlich," sagte Professor Quirrell, seine Stimme nun tiefer und härter, "Sie befinden sich unter dem Einfluss eines Anti-Apparier-Zaubers. Nein, es gibt nur ein einziges Ungetüm, welches Sie bedrohen kann, sobald Sie erwachsen sind. Das allergefährlichste Monster auf der ganzen Welt, so gefährlich, dass nichts anderes ihm nahe kommt. Der Dunkle Zauberer. Das ist das Einzige, was Sie immer noch wird bedrohen können."

Professor Quirrells Lippen waren zu einer schmalen Linie geworden. "Ich werde Ihnen widerstrebend genug Trivialwissen beibringen für eine Note, mit der Sie die vom Ministerium verlangten Teile Ihrer Abschlussprüfungen im ersten Jahr bestehen. Da Ihre exakte Note in diesen Bereichen keinen Einfluss auf Ihr zukünftiges Leben haben wird, steht es jedem, der mehr als ein "Bestanden" als Note wünscht, frei, seine eigene Zeit mit dem Studium unseres erbärmlichen Ersatzes für ein Lehrbuch zu verschwenden. Der Name dieses Fachs lautet nicht Verteidigung gegen kleine Ärgernisse. Sie sind hier um zu lernen, wie Sie sich gegen die Dunklen Künste verteidigen. Was bedeutet, lassen Sie uns dies völlig klar stellen, sich gegen Dunkle Zauberer zu verteidigen. Leute mit Zauberstäben, die Ihnen weh tun wollen und die damit wahrscheinlich Erfolg haben werden, wenn Sie ihnen nicht zuerst weh tun! Es gibt keine Verteidigung ohne Angriff! Es gibt keine Verteidigung ohne Kampf! Diese Wahrheit wird als zu brutal für Elfjährige erachtet, von den fetten, überbezahlten, von Auroren beschützten Politikern, die Ihre Lehrpläne erstellt haben. Zur Hölle mit diesen Narren! Sie sind hier für das Fach, das in Hogwarts achthundert Jahre lang gelehrt wurde! Willkommen zu Ihrem ersten Jahr in Kampf-Magie!"

Harry fing an zu applaudieren. Er konnte nicht anders, es war zu inspirierend.

Als Harry zu klatschen begann, gab es vereinzelte Antwort von Gryffindor und noch mehr von Slytherin, aber die meisten Schüler schienen einfach zu geschockt, um zu reagieren.

Professor Quirrell machte eine schneidende Geste und der Applaus erstarb augenblicklich. "Vielen Dank," sagte Professor Quirrell. "Nun zum Praktischen. Ich habe alle meine Verteidigungs-Klassen im ersten Jahr zu einer kombiniert, was mir erlaubt, ihnen doppelt so viel Zeit im Klassenraum zu bieten, als mit je zwei Klassen -"

Einige keuchten vor Entsetzen.

"- eine erhöhte Last, die ich wiedergutmachen werde, indem ich keine Hausaufgaben aufgebe."

Das Keuchen erstarb augenblicklich.

"Ja, Sie haben mich richtig verstanden. Ich werde Sie lehren zu kämpfen, nicht bis zum Montag zwölf Zoll über das Kämpfen zu schreiben."

Harry wünschte sich verzweifelt, er hätte sich neben Hermine gesetzt, damit er jetzt den Ausdruck auf ihrem Gesicht sehen könnte, aber andererseits war er sich ziemlich sicher, dass seine Vorstellung korrekt war.

Außerdem war Harry verliebt. Es würde eine Dreier-Hochzeit werden: er, der Zeitumkehrer und Professor Quirrell.

"Für jene von Ihnen, welche sich dafür entscheiden, habe ich einige außerschulische Aktivitäten arrangiert, von denen ich glaube, Sie werden sie sowohl recht interessant als auch lehrreich finden. Möchten Sie der Welt Ihre \emph{eigenen} Fähigkeiten zeigen, anstatt vierzehn anderen Leuten beim Quidditch spielen zuzusehen? Mehr als sieben Leute können in einer Armee kämpfen."

Heißer \emph{Scheiß.}

"Diese und andere außerschulische Aktivitäten bringen Ihnen außerdem Quirrell-Punkte ein. Was sind Quirrell-Punkte, fragen Sie? Das Hauspunkte-System wird meinen Bedürfnissen nicht gerecht, weil es Hauspunkte zu selten macht. Ich bevorzuge, meine Schüler häufiger wissen zu lassen, wie sie sich anstellen. Und in den seltenen Fällen, in denen ich Ihnen einen schriftlichen Test zu bieten habe, wird er sich wie von selbst benoten und wenn Sie zu viele Fragen falsch beantworten, wird Ihr Test die Namen von Schülern aufzeigen, die diese Fragen richtig hatten und diese Schüler werden Quirrell-Punkte verdienen können, indem Sie Ihnen helfen."

… wow. Warum verwendeten die anderen Professoren kein solches System?

"Wozu sind Quirrell-Punkte gut, fragen Sie? Zunächst, werden zehn Quirrell-Punkte einen Hauspunkt wert sein. Aber Sie werden Ihnen auch andere Gefälligkeiten einbringen. Möchten Sie Ihre Prüfung zu einem unüblichen Zeitpunkt ablegen? Gibt es eine bestimmte Unterrichtseinheit, die Sie viel lieber überspringen würden? Sie werden sehen, dass ich im Interesse von Schülern, die genug Quirrell-Punkte gesammelt haben, äußerst flexibel sein kann. Quirrell-Punkte werden über die Generalsränge der Armeen bestimmen. Und zu Weihnachten -- genau vor den Weihnachtsferien -- werde ich jemandem einen Wunsch gewähren. Jedwedes schulbezogene Kunststück, welches in meiner Macht, meinem Einfluss und vor allem, meiner Genialität liegt. Ja, ich war in Slytherin und ich biete an, zu Ihren Gunsten ein gerissenes Komplott auszuarbeiten, wenn es das ist, wonach Ihr Begehren verlangt. Dieser Wunsch geht an denjenigen, der in allen sieben Jahrgängen die meisten Punkte verdient hat."

Das würde Harry sein.

"Nun lassen Sie Ihre Bücher und losen Gegenstände an Ihren Pulten -- sie sind sicher, die Bildschirme wachen für Sie darüber -- und kommen Sie nach unten auf die Bühne. Es ist Zeit ein Spiel zu spielen namens Wer ist der Gefährlichste Schüler im Klassenzimmer."

Harry drehte seinen Zauberstab in seiner rechten Hand und sagte "\emph{Ma-ha-su!}"

Es gab ein weiteres hohes "Bing" von der schwebenden blauen Kugel, die Professor Quirrell Harry als Ziel zugewiesen hatte. Dieser bestimmte Ton zeigte einen perfekten Treffer an, den Harry bei neun seiner letzten zehn Versuche erzielt hatte.

Irgendwo hatte Professor Quirrell einen Zauber ausgegraben, der unglaublich einfach auszusprechen war \emph{und} eine lächerlich einfache Zauberstab-Bewegung erforderte \emph{und} dazu neigte, dorthin zu treffen, wo immer man auch gerade hinsah. Professor Quirrell hatte verächtlich verkündet, dass echte Kampf-Magie viel schwieriger war als das. Dass der Zauber in einem echten Gefecht vollkommen nutzlos war. Dass er ein kaum geordneter Ausbruch von Magie war, dessen einziger echter Inhalt das Zielen war und dass er, wenn er traf, einen flüchtigen Schmerz verursachte, vergleichbar mit einem harten Schlag auf die Nase. Dass der einzige Sinn dieses Tests war, herauszufinden, wer schnell lernen würde, da Professor Quirrell sicher war, dass niemandem dieser Zauber oder etwas ähnliches bisher begegnet war.

Harry kümmerte nichts davon.

"\emph{Ma-ha-su!}"

Ein \emph{roter Blitz aus Energie} schoss aus seinem Zauberstab und traf das Ziel und die blaue Kugel machte ein weiteres Mal das Bing, was bedeutete, der Zauber hatte \emph{tatsächlich für ihn funktioniert.}

Harry fühlte sich zum ersten Mal seit er nach Hogwarts gekommen war, wie ein richtiger Zauberer. Er wünschte, das Ziel würde ausweichen, wie die kleinen Kugeln, die Ben Kenobi dazu benutzt hatte, Luke zu trainieren, aber aus irgendeinem Grund hatte Professor Quirrell stattdessen alle Schüler und ihre Ziele in geraden Reihen aufgestellt, um sicher zu stellen, dass sie nicht aufeinander feuerten.

Also senkte Harry seinen Zauberstab, sprang zur Rechten, riss seinen Zauberstab nach oben und drehte ihn und rief "\emph{Ma-ha-su!}"

Es gab ein tieferes "Dong", was bedeutete, dass er es fast richtig gemacht hatte.

Harry steckte seinen Zauberstab in seine Tasche, sprang zurück zur Linken und drehte und feuerte einen weiteren Energie-Blitz.

Das hochtönende Bing war mit Sicherheit eines der befriedigendsten Geräusche, die er in seinem Leben gehört hatte. Harry wollte seinen Triumph herausschreien, so laut er konnte. \emph{ICH KANN ZAUBERN! FÜRCHTET MICH, GESETZE DER PHYSIK, ICH KOMME, UM EUCH ZU BRECHEN!}

"\emph{Ma-ha-su!}" Harrys Stimme war laut, aber kaum zu erkennen über den ständigen Chor ähnlicher Schreie von der Bühne des Klassenzimmers.

"Genug," sagte Professor Quirrells verstärkte Stimme. (Sie erklang nicht laut. Sie erklang in normaler Lautstärke und kam direkt von hinter der linken Schulter, egal wo man in Bezug zu Professor Quirrell stand.) "Ich sehe, dass Sie nun alle wenigstens einmal Erfolg hatten." Die Ziel-Kugeln wurden rot und begannen, zur Decke empor zu schweben.

Professor Quirrel stand auf dem erhöhten Podium im Zentrum der Bühne und lehnte sich mit einer Hand leicht auf sein Lehrerpult.

"Ich sagte Ihnen," sagte Professor Quirrell, "dass wir ein Spiel spielen würden, namens Wer ist der Gefährlichste Schüler im Klassenraum. Es gibt eine Schülerin in diesem Klassenraum, die den Simplen Sumerischen Schlag-Zauber schneller als jeder andere gemeistert hat -"

Oh, bla bla bla.

"- und damit weitermachte, sieben anderen Schülern zu helfen. Wofür sie die ersten sieben Quirrell-Punkte verdient hat, die Ihrem Jahrgang verliehen werden. Treten Sie vor, Hermine Granger. Es ist Zeit für die nächste Stufe des Spiels."

Hermine Granger begann nach vorn zu schreiten, eine Mischung aus Triumph und Besorgnis auf ihrem Gesicht. Die Ravenclaws blickten stolz zu ihr, die Slytherins zornig und Harry mit offener Verärgerung. Harry war dieses mal gut gewesen. Er war wahrscheinlich sogar in der oberen Hälfte der Klasse, nun da jeder mit einem gleichermaßen unbekannten Zauber konfrontiert worden war und Harry hatte Adalbert Schwahfels \emph{Theorie der Magie} komplett durchgelesen. Und doch \emph{machte Hermine sich immer noch besser.}

Irgendwo in seinem Hinterkopf saß die Angst, dass Hermine einfach klüger war als er.

Aber für den Moment machte Harry seine Hoffnungen an den bekannten Fakten fest, dass (a) Hermine viel mehr als die Standard-Lehrbücher gelesen hatte und (b) Adalbert Schwahfel ein uninspirierter Sack war, der \emph{Theorie der Magie} geschrieben hatte, um sich an einem Schulrat zu bereichern, der von Elfjährigen nicht viel hielt.

Hermine erreichte das zentrale Podium und stieg hinauf.

"Hermine Granger meisterte einen vollkommen unbekannten Zauber in zwei Minuten, fast eine volle Minute schneller als der Nächstplatzierte." Professor Quirrell drehte sich langsam auf der Stelle um alle Schüler anzusehen, die sie beobachteten. "Könnte Miss~Grangers Intelligenz sie zur gefährlichsten Schülerin im Klassenzimmer machen? Nun? Was denken Sie?"

Niemand schien im Moment irgendetwas zu denken. Selbst Harry war nicht sicher, was er sagen sollte.

"Finden wir es also heraus, nicht wahr?" sagte Professor Quirrelll. Er wand sich wieder an Hermine und deutete über die versammelte Klasse. " Wählen Sie irgendeinen Schüler aus, der Ihnen gefällt und wirken Sie einen Simplen Schlag-Zauber auf ihn."

Hermine erstarrte an Ort und Stelle.

"Kommen Sie," sagte Professor Quirrell glatt. "Sie haben diesen Zauber über fünfzig mal perfekt ausgeführt. Er fügt keinen dauerhaften Schaden zu und ist nicht einmal sehr schmerzhaft. Er tut so weh, wie ein harter Schlag und dauert nur wenige Sekunden." Professor Quirrells Stimme wurde härter. "Das ist eine direkte Anweisung Ihres Professors, Miss~Granger. Wählen Sie ein Ziel und feuern Sie einen Simplen Schlag-Zauber."

Hermines Gesicht verzog sich vor Entsetzen und ihr Zauberstab zitterte in ihrer Hand. Harrys eigene Finger umklammerten mitfühlend seinen Zauberstab. Obwohl er verstand, was Professor Quirrell bezweckte. Obwohl er verstand, was Professor Quirrell zu demonstrieren versuchte.

"Wenn Sie \emph{nicht} Ihren Zauberstab erheben und feuern, Miss~Granger, verlieren Sie einen Quirrell-Punkt."

Harry starrte Hermine an, wollte, dass sie in seine Richtung sah. Seine rechte Hand tippte sachte auf seine Brust. \emph{Nimm mich, ich fürchte mich nicht…}

Hermines Zauberstab zuckte in ihrer Hand; dann entspannte sich ihr Gesicht und sie ließ den Zauberstab zu ihrer linken Seite herabsinken.

"Nein," sagte Hermine Granger.

Ihre Stimme war ruhig und obwohl sie nicht laut war, hörte sie jeder in der Stille.

"Dann muss ich Ihnen einen Punkt abziehen," sagte Professor Quirrell. "Dies ist ein Test und Sie haben ihn nicht bestanden."

Das erreichte sie. Harry konnte es sehen. Aber sie hielt ihre Schultern gerade.

Professor Quirrells Stimme war mitfühlend und schien den ganzen Raum auszufüllen. "Dinge zu wissen reicht nicht immer aus, Miss~Granger. Wenn Sie nicht einmal Gewalt im Ausmaß eines angestoßenen Zehs austeilen oder einstecken können, können Sie sich nicht schützen und werden in Verteidigung nicht bestehen. Bitte gesellen Sie sich wieder zu Ihren Klassenkameraden."

Hermine ging zurück in Richtung der Ansammlung von Ravenclaws. Ihr Gesicht wirkte friedlich und Harry wollte, aus irgendeinem merkwürdigen Grund, anfangen zu klatschen. Obwohl Professor Quirrell \emph{recht} gehabt hatte.

"Also," sagte Professor Quirrell. "Es wird klar, dass Hermine Granger nicht die gefährlichste Schülerin im Klassenraum ist. Wer, denken Sie, könnte tatsächlich die gefährlichste Person hier sein? - außer mir, natürlich."

Ohne zu überlegen, wand Harry seinen Kopf zur Abordnung der Slytherins.

"Draco, aus dem Noblen und Uralten Haus von Malfoy," sagte Professor Quirrell. "Es scheint, viele Ihrer Mitschüler blicken in Ihre Richtung. Treten Sie bitte vor."

Draco tat es, mit gewissem Stolz in seiner Haltung. Er stieg auf das Podium und sah mit einem Lächeln zu Professor Quirrell auf.

"Mr~Malfoy," sagte Professor Quirrell. "Feuer."

Harry hätte versucht es aufzuhalten, wenn Zeit gewesen wäre, aber in einer einzigen flüssigen Bewegung wirbelte Draco zur Ravenclaw-Abordnung herum und erhob seine Zauberstab und sagte "\emph{Mahasu!}", als wäre es alles eine Silbe und Hermine sagte "Au!" und das war es.

"Guter Treffer," sagte Professor Quirrell. "Zwei Quirrel-Punkte für Sie. Aber sagen Sie mir, warum zielten Sie auf Miss~Granger?"

Es gab eine Pause.

Schließlich sagte Draco, "Weil sie am meisten herausstach."

Professor Quirrells Lippen hoben sich zu einem Lächeln. "Und das ist der wahre Grund, warum Draco Malfoy gefährlich ist. Hätte er irgendjemand anders gewählt, würde ihm dies Kind wahrscheinlicher verübeln, herausgegriffen worden zu sein und Mr~Malfoy sich mit größerer Sicherheit einen Feind schaffen. Und obwohl Mr~Malfoy eine andere Rechtfertigung, sie auszuwählen, hätte geben können, so hätte es ihm nichts eingebracht, denn einige von Ihnen zu verwirren, während andere ihn bereits feiern, ob er etwas sagt oder nicht. Was besagt, Mr~Malfoy ist gefährlich, weil er es versteht, wen er treffen sollte und wen nicht, wie man sich Verbündete schafft und Feinde zu schaffen vermeidet. Zwei weitere Quirrel-Punkte für Sie, Mr~Malfoy. Und da Sie eine beispielhafte Tugend von Slytherin demonstriert haben, hat sich, wie ich denke, das Haus von Salazar ebenfalls einen Punkt verdient. Sie mögen sich wieder Ihren Freunden zugesellen."

Draco verbeugte sich leicht und ging zurück zur Slytherin-Abordnung. Manches Klatschen erhob sich von den grün-verzierten Umhängen, aber Professor Quirrell machte eine schneidende Geste und Stille senkte sich erneut herab.

"Es mag scheinen, als sei unser Spiel vorbei," sagte Professor Quirrell. "Und doch gibt es einen einzigen Schüler in diesem Klassenraum, der gefährlicher ist, als der Spross von Malfoy."

Und \emph{nun} schienen sich aus irgendeinem Grund furchtbar viele Blicke zu wenden auf…

"Harry Potter. Treten Sie vor."

Das ließ nichts Gutes erahnen.

Harry schritt zögerlich zu dem Platz, wo Professor Quirrell auf seinem erhöhten Podium stand, immer noch leicht gegen sein Lehrerpult gelehnt.

Die Nervosität, weil er in's Rampenlicht gestellt wurde, schien Harrys Sinne zu schärfen, als er auf das Podium zutrat und sein Geist raste durch Möglichkeiten, was Professor Quirrells Ansicht nach wohl Harrys Gefährlichkeit demonstrieren könnte. Würde er aufgefordert, einen Zauber zu wirken? Einen Dunklen Lord zu besiegen?

Seine vermeintliche Immunität gegen den Tödlichen Fluch zu demonstrieren? Sicher war Professor Quirrell \emph{dafür} zu klug…

Harry stoppte kurz vor dem Podium und Professor Quirrell bat ihn nicht, näher zu kommen.

"Die Ironie ist," sagte Professor Quirrell, "Sie schauten alle zur richtigen Person aus den vollkommen falschen Gründen. Sie denken," Professor Quirrell verzog die Lippen, "dass Harry Potter den Dunklen Lord besiegt hat und muss daher sehr gefährlich sein. Pah. Er war ein Jahr alt. Welche Laune des Schicksals auch immer den Dunklen Lord getötet hat, dürfte wenig mit Mr~Potters Fähigkeiten als Kämpfer zu tun gehabt haben. Aber nachdem ich Gerüchte darüber hörte, wie ein Ravenclaw fünf ältere Syltherins in die Knie zwang, befragte ich mehrere Augenzeugen und kam zu dem Schluss, Harry Potter würde mein gefährlichster Schüler sein."

Ein Adrenalin-Stoß fuhr durch Harrys Körper und ließ ihn aufschnellen. Er wusste nicht, zu welchem Schluss Professor Quirrell gelangt war, aber das konnte nicht gut sein.

"Äh, Professor Quirrell -" begann Harry.

Professor Quirrell schien amüsiert. "Sie denken, ich sei zu einem falschen Schluss gekommen, nicht wahr, Mr~Potter? Sie werden lernen, besseres von \emph{mir} zu erwarten." Professor Quirrell erhob sich von seinem Pult. "Mr~Potter, alle Dinge haben ihren gewöhnlichen Zweck. Nennen Sie mir zehn ungewöhnliche Möglichkeiten, Objekte in diesem Raum im Kampf zu verwenden!"

Für einen Moment war Harry sprachlos ob des schieren, simplen Schocks, verstanden worden zu sein.

Und dann begannen die Ideen, aus ihm herauszusprudeln.

"Es gibt Pulte, die schwer genug sind, um tödlich zu sein, wenn sie aus großer Höhe fallengelassen werden. Es gibt Stühle mit Metall-Beinen, mit denen man jemanden aufspießen könnte, wenn man sie kräftig genug antreibt. Die Luft in diesem Klassenzimmer wäre tödlich, wenn sie verschwände, weil Menschen im Vakuum sterben und sie kann zum Transport von Giftgasen benutzt werden."

Harry musste kurz innehalten, um zu atmen und in diese Pause hinein sagte Professor Quirrell:

"Das sind drei. Sie brauchen zehn. Der Rest der Klasse denkt, Sie haben bereits allen Inhalt des Klassenraumes aufgebraucht."

"\emph{Ha!} Der Boden kann entfernt werden, um eine Stachelgrube zu schaffen, in die man fallen kann, die Decke kann man auf jemanden herabstürzen lassen, die Wände können als Rohmaterial für Verwandlungen in alle möglichen tödlichen Dinge verwendet werden -- wie etwa Messer."

"Das sind sechs. Aber sicher haben Sie nun alles ausgeschöpft?"

"Ich hab' noch nicht mal angefangen! Sehen Sie nur all die Leute an! Einen Gryffindor den Feind attackieren zu lassen, ist natürlich ein \emph{gewöhnlicher} Gebrauch -"

"Das werde ich nicht zählen."

"- aber ihr Blut kann auch benutzt werden, um jemanden zu ertränken. Ravenclaws sind für ihre Intelligenz bekannt, aber ihre Organe könnten für genug Geld auf dem Schwarzmarkt verkauft werden, um einen Attentäter anzuheuern. Slytherins sind nicht nur nützlich als Attentäter, sie können auch mit ausreichender Geschwindigkeit geworfen werden, um einen Feind zu zerschmettern. Und Hufflepuffs verfügen, zusätzlich dazu, hart arbeiten zu können, auch über Knochen, die entfernt, angespitzt und benutzt werden können, um jemanden zu erstechen."

Inzwischen starrte der Rest der Klasse Harry mit einigem Entsetzen an. Selbst die Slytherins wirkten geschockt.

"Das macht zehn, obwohl ich großzügig bin, das mit den Ravenclaws gelten zu lassen. Nun, als Zusatzaufgabe, ein Quirrell-Punkt für jeden Verwendungszweck für Objekte in diesem Raum, die Sie noch nicht genannt haben." Professor Quirrell bedachte Harry mit einem freundschaftlichen Lächeln. "Der Rest der Klasse glaubt, Sie sind nun in Schwierigkeiten, da Sie alles außer den Zielen genannt haben und Sie keine Ahnung haben, was sich mit diesen anstellen ließe."

"Pah! Ich habe alle Menschen genannt, aber nicht meinen Umhang, der benutzt werden kann, um einen Feind zu ersticken, wenn er oft genug um seinen Kopf gewickelt wird oder Hermine Grangers Umhang, der in Streifen gerissen und zu einem Seil geknotet und benutzt werden kann, um jemanden zu hängen oder Draco Malfoys Umhang, mit dem man ein Feuer anzünden kann -"

"Drei Punkte," sagte Professor Quirrell, "jetzt keine Kleidung mehr."

"Mein Zauberstab kann einem Gegner durch seine Augenhöhle in's Gehirn gestoßen werden" und jemand gab ein entsetztes, ersticktes Geräusch von sich.

"Vier Punkte, keine Zauberstäbe mehr."

"Meine Armbanduhr könnte jemanden erstickten, wenn sie ihm in die Kehle gerammt wird -"

"Fünf Punkte und genug."

"Hmpf," sagte Harry. "Zehn Quirrell-Punkte für einen Hauspunkt, richtig? Sie hätten mich weitermachen lassen sollen, bis ich den Hauspokal gewonnen hätte, ich habe noch nicht einmal mit all den ungewöhnlichen Verwendungszwecken angefangen für alles, was ich in meinen Taschen habe" oder dem Eselsfell-Beutel selbst und er konnte nicht über den Zeitumkehrer oder den Unsichtbarkeitsumhang sprechen, aber es musste \emph{irgendetwas} geben, was er über diese roten Kugeln sagen könnte…

"\emph{Genug,} Mr~Potter. Nun, glauben Sie alle zu verstehen, was Mr~Potter zum gefährlichsten Schüler im Klassenraum macht?"

Es gab leises, zustimmendes Gemurmel.

"Sagen Sie es bitte laut. Terry Boot, was macht Ihren Schlafsaal-Kameraden gefährlich?"

"Ah… ähm… er ist kreativ?"

"\emph{Falsch!}" bellte Professor Quirrell und seine Faust fuhr scharf auf sein Pult herab, mit einem verstärkten Geräusch, dass alle hoch schrecken ließ. "Jede von Mr~Potters Ideen war mehr als nutzlos!"

Harry erstarrte überrascht.

"Den Fußboden entfernen, um eine Stachelfalle zu bauen? Absurd! Im Kampf hat man keine solche Vorbereitungszeit und wenn doch, gäbe es hundert bessere Verwendungsmöglichkeiten! Material aus den Mauern verwandeln? Mr~Potter kann keine Verwandlungen durchführen! Mr~Potter hatte exakt eine einzige Idee, welche er unmittelbar, in diesem Augenblick verwenden könnte, ohne umfangreiche Vorbereitung oder einen kooperierenden Gegner oder Magie, die er nicht beherrscht. Diese Idee war, seinen Zauberstab durch die Augenhöhle seines Feindes zu stoßen. Was wahrscheinlicher seinen Zauberstab zerbrechen würde, als seinen Gegner zu töten! Kurz gesagt, Mr~Potter, ich fürchte Ihre Vorschläge waren allesamt grauenhaft."

"Was?" sagte Harry empört. "Sie \emph{fragten} nach ungewöhnlichen Ideen, nicht nach praktischen! Ich habe unkonventionell gedacht! Wie würden \emph{Sie} etwas in diesem Klassenraum benutzen, um jemanden zu töten?"

Professor Quirrells Gesichtsausdruck war missbilligend, aber mit Lachfalten um die Augen. "Mr~Potter, ich sagte nie, Sie sollten \emph{töten.} Es gibt einen Ort und eine Zeit um Ihren Gegner am Leben zu lassen und ein Klassenraum von Hogwarts ist für gewöhnlich einer jener Orte. Aber, um Ihre Frage zu beantworten, schlagen Sie ihm mit einer Stuhlkante in's Genick."

Es gab einiges Gelächter von Seiten der Slytherins, aber sie lachten mit Harry, nicht über ihn.

Alle anderen sahen ziemlich entsetzt aus.

"Aber Mr~Potter hat nun demonstriert, weshalb er der gefährlichste Schüler im Klassenraum ist. Ich fragte nach ungewöhnlichen Möglichkeiten Gegenstände in diesem Raum im Kampf zu verwenden. Mr~Potter hätte vorschlagen können, ein Schreibpult zu verwenden, um einen Fluch abzuwehren oder einen Stuhl zu gebrauchen, um einen nahenden Angreifer stolpern zu lassen oder Kleidung um seinen Arm zu wickeln, um einen improvisierten Schild zu schaffen. Stattdessen, war jede einzelne Möglichkeit, die Mr~Potter nannte, offensiv statt defensiv und entweder tödlich oder potenziell tödlich."

Was? Moment, das konnte nicht stimmen… Harry verspührte einen plötzlichen Schwindel, als er sich zu erinnern versuchte, was genau er vorgeschlagen hatte, sicherlich musste es ein Gegenbeispiel geben…

"Und deshalb," sagte Professor Quirrell, "waren Mr~Potters Ideen so seltsam und nutzlos -- weil er weit in's Reich des Unpraktischen hinausgreifen musste, um seinem Anspruch gerecht zu werden, den \emph{Gegner zu töten.} Für ihn war jede Idee, die dahinter zurückblieb, nicht wert berücksichtigt zu werden. Dies zeigt eine Eigenschaft, die wir \emph{Killerinstinkt} nennen könnten. Ich habe ihn. Mr~Potter hat ihn, weshalb er fünf älteren Slytherins die Stirn bieten konnte. Draco Malfoy hat ihn nicht, noch nicht. Mr~Malfoy würde kaum davor zurückschrecken, über gewöhnlichen Mord zu sprechen, aber selbst er war geschockt -- ja, waren Sie, Mr~Malfoy, ich habe Sie beobachtet -- als Mr~Potter beschrieb, wie man die Körper seiner Klassenkameraden als Rohmaterial verwenden könnte. Es gibt Zensoren in seinem Geist, die ihn vor solchen Gedanken zurückschrecken lassen. Mr~Potter denkt \emph{einzig} daran, den Feind zu töten, er wird jede Gelegenheit dazu nutzen, er weicht nicht zurück, seine Zensoren sind aus. Obwohl sein jugendliches Genie so undiszipliniert und unpraktisch ist, dass es nutzlos wird, macht sein \emph{Killerinstinkt} Harry Potter zum Gefährlichsten Schüler im Klassenraum. Ein letzter Punkt für ihn -- nein, machen wir daraus einen Punkt für Ravenclaw -- für diese unverzichtbare Voraussetzung eines wahren kämpfenden Zauberers."

Harrys Mund stand offen vor unaussprechlichem Schock, während er hektisch nach irgendetwas suchte, was er darauf erwidern konnte. \emph{Das ist so absolut nicht meine Art!}

Aber er konnte sehen, dass die anderen Schüler begannen, es zu glauben. Harrys Geist schnellte durch mögliche Leugnungen und fand nichts, was gegen die entschiedene Stimme von Professor Quirrell bestehen konnte. Das beste, was Harry einfiel war "Ich bin kein Psychopath, ich bin nur sehr kreativ" und das klang irgendwie unheilverkündend. Er musste etwas unerwartetes sagen, etwas dass die Leute dazu bringen würde, innezuhalten und noch einmal darüber nachzudenken -

"Und nun," sagte Professor Quirrell. "Mr~Potter. Feuer."

Natürlich passierte nichts.

"Ah, nun gut," sagte Professor Quirrell. Er seufzte. "Ich nehme an, wir alle müssen irgendwo beginnen. Mr~Potter, wählen Sie irgendeinen Schüler, der Ihnen beliebt, für einen Simplen Schlag-Zauber. Sie \emph{werden} es tun, bevor ich Ihre Klasse für heute entlasse. Falls Sie es nicht tun, werde ich mit dem Abzug von Hauspunkten beginnen und damit fortfahren, bis Sie es tun."

Harry erhob vorsichtig den Zauberstab. Wenigstens das musste er tun oder Professor Quirrell finge vielleicht sofort mit dem Abziehen der Hauspunkte an.

Langsam, als stünde er auf einem Drehteller, fuhr Harry zu den Slytherins herum.

Und Harrys Augen trafen auf Dracos.

Draco Malfoy zeigte nicht die geringste Spur von Angst. Der blonde Junge ließ kein sichtbares Zeichen der Zustimmung erkennen, wie Harry es Hermine gegeben hatte, aber man konnte es auch kaum von ihm erwarten. Die anderen Slytherins würden das für ziemlich seltsam halten.

"Warum das Zögern?" sagte Professor Quirrell. "Sicherlich gibt es nur eine offensichtliche Wahl."

"Ja," sagte Harry. "Nur eine \emph{offensichtliche} Wahl."

Harry drehte den Zauberstab und sagte "\emph{Ma-ha-su!}"

Es herrschte vollkommene Stille im Klassenraum.

Harry schüttelte seinen linken Arm, im Versuch das anhaltende Stechen loszuwerden.

Es blieb weiterhin still.

Schließlich seufzte Professor Quirrell. "Ja, brilliant in der Tat, aber hier war eine Lektion zu lernen und Sie sind ihr ausgewichen. Ein Punkt Abzug von Ravenclaw, für das Prahlen mit Ihrer Cleverness auf Kosten des eigentlichen Ziels. Klasse entlassen."

Und bevor irgendwer sonst etwas sagen konnte, tönte Harry:

"Nur ein Scherz! RAVENCLAW!"

Danach herrschte für einen kurzen Moment Stille, das Geräusch denkender Menschen und dann begann das Gemurmel und die Gespräche schwollen rasch zu einem Rauschen an.

Harry drehte sich zu Professor Quirrell um, sie beide mussten reden -

Quirrell war zusammen gesackt und trottete zu seinem Stuhl zurück.

Nein. Inakzeptabel. Sie mussten \emph{wirklich} reden. Schluss mit dem Zombie-Getue, Professor Quirrell würde wahrscheinlich aufwachen, wenn Harry ihn ein paar mal anstupste. Harry schritt voran -

\emph{FALSCH

NICHT

SCHLECHTE IDEE}

Harry schwankte und stoppte abrupt, fühlte sich schwindlig.

Und dann ging ein Schwarm von Ravenclaws auf ihn nieder und die Debatten begannen.

