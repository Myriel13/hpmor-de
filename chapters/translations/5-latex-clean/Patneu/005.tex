

\hypertarget{der-fundamentale-zuschreibungsfehler}{% \section{5. Der fundamentale Zuschreibungsfehler}\label{der-fundamentale-zuschreibungsfehler}}

\textbf{Kapitel 5: Der fundamentale Zuschreibungsfehler}\\

\hfill\break J. K. Rowling starrt euch an. Könnt ihr ihren Blick auf euch spüren? Sie liest eure Gedanken mit ihren Rowling-Strahlen.

--------------------------------------------------------------------------------------------------------------------------------------------

\hfill\break

\emph{\emph{"Es hätte ein übernatürliches Eingreifen gebraucht, damit er über deine Moral verfügt, angesichts seines Umfelds.„}}

\hfill\break

--------------------------------------------------------------------------------------------------------------------------------------------

\hfill\break Das Eselsfell-Beutel-Geschäft war ein uriger (mancher könnte sogar sagen süßer) kleiner Laden, versteckt hinter einem Gemüsestand, der sich hinter einem Geschäft für magische Handschuhe befand, welches in einer kleinen Gasse abseits einer Seitenstraße der Winkelgasse lag. Enttäuschenderweise war die Ladenbesitzerin kein verhutzeltes altes Weib; nur eine nervös aussehende junge Frau in einem ausgeblichenen gelben Umhang. Gerade hielt sie ihm einen Super-Eselsfell-Beutel QX31 entgegen, dessen Verkaufsargumente aus einer sich weitenden Öffnung, sowie einem Unaufspürbaren Ausdehnungszauber bestanden: Man konnte tatsächlich große Dinge hinein tun, obwohl das absolute Volumen immer noch begrenzt war.

Harry hatte darauf \emph{bestanden} direkt hierher zu kommen, als aller erstes -- so sehr darauf bestanden, wie er es nur glaubte zu können, ohne Professor McGonagall misstrauisch zu machen. Harry hatte etwas, das er so schnell wie möglich in den Beutel stecken musste. Es war nicht das Säckchen Galleonen, welches Professor McGonagall ihm aus Gringotts abzuheben erlaubt hatte. Es waren all die anderen Galleonen, die Harry heimlich in seine Tasche geschoben hatte, nachdem er in den Haufen von Goldmünzen gefallen war. Das \emph{war} ein echter Unfall gewesen, aber Harry war niemals jemand, der eine Gelegenheit vergeudete... obwohl es wirklich eher ein spontaner Entschluss gewesen war. Seitdem hatte Harry das Säckchen mit den erlaubten Galleonen peinlich genau neben seiner Hosentasche getragen, so dass jedes Klimpern vom richtigen Ort zu kommen schien.

Das ließ noch die Frage offen, wie er es tatsächlich schaffen sollte, die \emph{anderen} Goldmünzen in seinen Beutel zu bekommen, ohne erwischt zu werden. Die Goldmünzen mochten ihm gehören, aber sie waren trotzdem gestohlen - selbst-gestohlen? Auto-entwendet?

Harry sah von dem Super-Eselsfell-Beutel QX31, auf dem Tresen vor ihm, auf. „Kann ich ihn eine Weile ausprobieren? Um sicher zu sein, dass er, ähm, zuverlässig funktioniert?“ Er weitete seine Augen in einem Gesichtsausdruck jungenhafter, verspielter Unschuldigkeit.

Und tatsächlich, nach zehn Wiederholungen, das Münzen-Säckchen in den Beutel zu packen, „Säckchen voll Gold“ zu flüstern und es wieder herauszuholen, trat Professor McGonagall zur Seite und begann einige der anderen Gegenstände im Laden in Augenschein zu nehmen und die Ladenbesitzerin drehte den Kopf um zuzusehen.

Harry ließ das Säckchen voll Gold mit seiner \emph{linken} Hand in den Eselsfell-Beutel fallen; seine \emph{rechte} Hand kam aus seiner Tasche, einige der Goldmünzen festhaltend, griff in den Eselsfell-Beutel, ließ die losen Galleonen hineinfallen und rief (mit einem Flüstern „Säckchen voll Gold“) das ursprüngliche Säckchen wieder auf. Dann wanderte das Säckchen zurück in seine \emph{linke} Hand, um es wieder hineinfallen zu lassen und Harrys \emph{rechte} Hand wanderte zurück in seine Tasche...

Professor McGonagall sah einmal zu ihm zurück, aber Harry schaffte es, weder zu erstarren, noch zu zucken und sie schien nichts zu bemerken. Obwohl man das bei den Erwachsenen, die einen Sinn für Humor hatten, \emph{nie} so genau wissen konnte. Es brauchte drei Wiederholungen, um den Job zu erledigen und Harry schätzte, dass er es geschafft hatte, sich selbst etwa dreißig Galleonen zu stehlen.

Harry hob seine Hand, wischte sich etwas Schweiß von seiner Stirn und atmete aus. „Ich hätte gern diesen hier, bitte."

Fünfzehn Galleonen leichter (offenbar der doppelte Preis eines Zauberstabes) und einen Super-Eselsfell-Beutel QX31 schwerer drückten sich Harry und Professor McGonagall durch die Tür nach draußen. Der Tür bildete eine Hand, als sie gingen und winkte ihnen zum Abschied, ihren Arm auf eine Weise ausstreckend, die Harry ein etwas mulmiges Gefühl bereitete.

Und dann, unglücklicherweise...

„Sind sie \emph{wirklich} Harry Potter?“ flüsterte der alte Mann und eine große Träne glitt seine Wange hinab. „Sie würden darüber keine Lügen erzählen, oder? Ich hatte nur Gerüchte gehört, dass Sie den Tödlichen Fluch nicht \emph{wirklich} überlebt hätten und dass deshalb niemand jemals wieder von Ihnen gehört hat.„

...es schien, als ob Professor McGonagalls Verkleidungszauber gegen in magischen Praktiken Erfahrenere nicht ganz perfekt funktionierte.\\ Professor McGonagall hatte eine Hand auf Harrys Schulter gelegt und ihn in die nächstgelegene Gasse gezogen, sofort als sie „Harry Potter?“ gehört hatte. Der alte Mann war ihnen gefolgt, aber zumindest hatte es anscheinend niemand sonst mitbekommen.

Harry bedachte die Frage. \emph{War} er wirklich Harry Potter? „ Ich weiß nur, was andere Leute mir erzählt haben,“ sagte Harry. „Es ist nicht so, als ob ich mich erinnere, geboren worden zu sein.“ Er rieb sich mit der Hand über die Stirn. „Ich hatte diese Narbe, so lange ich mich erinnern kann und mir wurde gesagt, mein Name wäre Harry Potter, so lange ich mich erinnern kann. Aber,“ sagte Harry nachdenklich, „wenn es bereits ausreichende Gründe gibt, eine Verschwörung anzunehmen, gibt es wohl keinen Grund, warum sie nicht ein anderes Waisenkind hätten finden und dazu erziehen können, zu glauben, \emph{er} wäre Harry Potter -"

Professor McGonagall fuhr sich verzweifelt mit der Hand durch ihr Gesicht. „Sie sehen fast genau so aus, wie Ihr Vater James in seinem ersten Jahr in Hogwarts. Und ich kann \emph{allein aufgrund der Persönlichkeit} sagen bezeugen, dass Sie mit der Geißel von Gryffindor verwandt sind."

„\emph{Sie} könnte auch mit drin stecken,“ bemerkte Harry.

„Nein,“ sagte der alte Mann zittrig. „Sie hat recht. Sie haben die Augen Ihrer Mutter."

„Hmm,“ machte Harry stirnrunzelnd. „Ich nehme an, \emph{Sie} könnten auch mit drin stecken-"

"Genug, Mr. Potter.„

Der alte Mann hob eine Hand, wie um Harry zu berühren, ließ sie dann aber fallen. „Ich bin nur froh, dass Sie am Leben sind,“ murmelte er. „Danke, Harry Potter. Danke für das, was Sie getan haben... Ich lasse Sie jetzt allein.„

Und das Klopfen seines Gehstocks entfernte sich, aus der Gasse hinaus und die Hauptstraße der Winkelgasse hinunter.

Die Professorin sah sich um, ihr Gesichtsausdruck grimmig und angespannt. Harry sah sich automatisch ebenfalls um. Aber die Gasse schien leer bis auf ein paar alte Blätter und von der in die Winkelgasse hinausführenden Einmündung waren nur schnell ausschreitende Passanten zu sehen.

Schließlich schien sich Professor McGonagall zu entspannen. „Das war nicht in Ordnung,“ sagte sie mit leiser Stimme. „Ich weiß, dass Sie das nicht gewohnt sind, Mr. Potter, aber die Leute machen sich Gedanken um Sie. Bitte seien Sie nett zu ihnen.„\\ Harry blickte auf seine Schuhe hinab. „Das sollten sie nicht,“ sagte er mit einer Spur Verbitterung. „Sich Gedanken um mich machen, meine ich."

„Sie haben sie vor Sie-wissen-schon-wem gerettet,“ sagte Professor McGonagall. „Wie sollten sie nicht?"

Harry sah zu dem strengen Gesichtsausdruck der Hexen-Dame unter ihrem spitzen Hut auf und seufzte. „Ich nehme nicht an, dass, wenn ich \emph{fundamentaler Zuschreibungsfehler} sagen würde, sie irgendeine Ahnung hätten, was das bedeutet."

„Nein,“ sagte die Professorin mit ihrem präzisen schottischen Akzent, „aber bitte erklären Sie es, Mr. Potter, wenn sie so nett wären."

„Nun...“ sagte Harry, sich überlegend, wie er dieses spezielle Stück Muggel-Wissenschaft beschreiben sollte. „Nehmen Sie an, Sie kommen zur Arbeit und sehen einen Kollegen gegen seinen Schreibtisch treten. Sie denken, 'Was für eine zornige Person er sein muss'. Ihr Kollege denkt daran, wie ihn jemand auf dem Weg zur Arbeit gegen eine Wand geschubst und dann angeschrien hat. \emph{Jeder} wäre darüber wütend, denkt er. Wenn wir andere Leute betrachten, sehen wir Charakterzüge, die ihr Verhalten erklären, aber wenn wir uns selbst betrachten, sehen wir die Umstände, die unser Verhalten erklären. Die Geschichten von Leuten machen internen Sinn für sie, von innen gesehen, aber wir sehen die Geschichten der Leute nicht auf ihrer Stirn geschrieben. Wir sehen sie nur in einer Situation und wir sehen nicht, wie sie in einer anderen Situation wären. Der funamentale Zuschreibungsfehler ist also, dass wir mit permanenten, dauerhaften Charakterzügen erklären, was eher durch Umstände und Kontext zu erklären wäre. Es gab einige elegante Experimente, die das bestätigten, aber Harry wollte sich nicht darin vertiefen.

Die Hexe ihre Augenbrauen unter ihrer Hutkrempe hoch. „Ich denke, ich verstehe...“ sagte Professor McGonagall langsam. „Aber was hat das mit Ihnen zu tun?"

Harry trat heftig genug gegen die Ziegelmauer der Gasse, dass ihm der Fuß schmerzte. „Die Leute denken, ich hätte sie vor Sie-wissen-schon-wem gerettet, weil ich sowas wie ein großer Krieger des Lichts wäre."

„Der eine mit der Macht, den Dunklen Lord zu besiegen...“ murmelte die Hexe, eine seltsame Ironie säuerte ihre Stimme.

„Ja,“ sagte Harry, Verärgerung und Frustration kämpften in ihm, „als ob ich den Dunklen Lord zerstört hätte, weil ich irgendeine Art von permanenter, andauernder zerstöre-den-Dunklen-Lord Charaktereigenschaft hätte. Ich war zu dem Zeitpunkt fünfzehn Monate alt! Ich \emph{weiß} nicht, was passiert ist, aber ich würde annehmen, dass es etwas mit, wie man sagen würde, zufälligen umweltbedingten Umständen zu tun hatte. Und ganz sicher nichts mit meiner Persönlichkeit. Die Leute machen sich keine Gedanken um \emph{mich}, sie beachten \emph{mich} nicht einmal, sie wollen einer \emph{schlechten Erklärung} die Hand geben.“ Harry hielt inne und sah Professor McGonagall an. „Wissen \emph{Sie}, was wirklich passierte?"

„Mir \emph{ist} eine Idee dazu gekommen...“ sagte Professor McGonagall. „Und zwar nachdem ich Sie getroffen habe."

"Ja?"

"Sie haben über den Dunklen Lord triumphiert, indem Sie noch entsetzlicher waren als er und haben den Tödlichen Fluch überlebt, indem Sie noch schrecklicher als der Tod waren."

„Ha. Ha. Ha.“ Harry trat wieder gegen die Wand.

Professor McGonagall lachte leise. „Schaffen wir Sie als nächstes zu Madam Malkins. Ich fürchte Ihre Muggel-Kleidung könnte Aufmerksamkeit erregen."

Auf dem Weg trafen sie noch auf zwei weitere Danksager.

Madam Malkins Umhänge hatte eine wirklich langweilige Ladenfassade; rote, ordinäre Ziegel und Schaufenster mit einfachen schwarzen Umhängen darin. Keine Umhänge, die glänzten oder sich verwandelten oder sich drehten oder seltsame Strahlen abgaben, die direkt durch dein Hemd zu gehen und dich zu kitzeln schienen. Nur einfache schwarze Umhänge, dass war alles, was man durch das Fenster sehen konnte. Die Tür stand weit offen, wie um anzukündigen, dass es hier keine Geheimnisse und nichts zu verstecken gab.

„Ich werde für einige Minuten verschwinden, während Ihnen Ihre Umhänge angepasst werden,“ sagte Professor McGonagall. „Wird das für Sie in Ordnung sein, Mr. Potter."

Harry nickte. Er hasste Kleidung einkaufen mit brennender Hingabe und konnte der älteren Hexe nicht verübeln, genau so zu fühlen.

Professor McGonagalls Zauberstab kam aus ihrem Ärmel und tippte leicht gegen Harrys Kopf. „Und da sie für Madam Malkins Sinne klar zu erkennen sein müssen, entferne ich die Verschleierung."

„Äh...“ sagte Harry. Das beunruhigte ihn ein wenig, er war immer noch nicht an die ganze 'Harry Potter'-Sache gewöhnt.

„Ich war mit Madam Malkin zusammen in Hogwarts,“ sagte Professor McGonagall. „Selbst zu dieser Zeit war sie einer der \emph{beherrschtesten} Menschen, die ich kannte. Sie würde nicht mit der Wimper zucken, wenn Sie-wissen-schon-wer höchstpersönlich in ihr Geschäft hinein spazieren würde.“ Professor McGonagalls Stimme klang nostalgisch und sehr anerkennend. „Madam Malkins wird Sie nicht belästigen und Sie wird nicht zulassen, dass Sie irgendjemand anders belästigt."

„Wohin \emph{gehen} Sie denn?“ erkundigte sich Harry. „Nur im Fall, dass, Sie wissen schon, irgendetwas \emph{passiert}.„

McGonagall blickte Harry streng an. „Ich gehe \emph{dorthin,}“ sagte Sie, auf ein Gebäude auf der anderen Straßenseite zeigend, dessen Schild ein kleines hölzernes Fass zeigte, „und kaufe einen Drink, den ich dringend brauche. \emph{Sie} werden sich Ihre Umhänge anpassen lassen \emph{und nichts anderes}. Ich werde \emph{in Kürze} zurückkommen, um nach Ihnen zu sehen und ich \emph{erwarte}, Madam Malkins Geschäft noch intakt vorzufinden und auf keine Weise in Flammen stehend."

Madam Malkin war eine geschäftige alte Frau, die kein Wort sagte, als sie die Narbe auf Harrys Stirn sah und sie schoss einer Assistentin einen scharfen Blick zu, als es so schien, als wollte das Mädchen etwas sagen. Madam Malkin holte eine Ansammlung sich bewegender und windender Stoffstücke heraus, die als eine Art Maßbänder zu fungieren schienen und machte sich daran, den Träger ihrer Kunst in Augenschein zu nehmen.

Neben Harry schien ein blasser Junge mit spitz zulaufendem Gesicht und \emph{supercoolem} weißblondem Haar die letzten Stufen eines ähnlichen Prozesses zu durchlaufen. Eine von Madam Malkins Assistentinnen nahm den weißhaarigen Jungen und den schachbrett-gemusterten Umhang, den er trug unter die Lupe; gelegentlich tippte sie eine Ecke des Umhangs mit ihrem Zauberstab an und der Umhang lockerte oder verengte sich.

„Hallo,“ sagte der Junge. „Auch Hogwarts?"

Harry konnte sich denken, wohin diese Unterhaltung führen würde und in einem Sekundenbruchteil der Frustration entschied er, genug war genug.

„Gütiger Himmel,“ flüsterte Harry, „das ist nicht möglich.“ Er weitete seine Augen. „Ihr... Name, Sir?"*

„Draco Malfoy,“ sagte Draco Malfoy, leicht verwirrt aussehend.

„Sie \emph{sind} es! Draco Malfoy. Ich - ich glaubte nie, mir würde eine solche Ehre zuteil, Sir.“ Harry wünschte, er könnte seine Augen zum Weinen bringen. Die anderen fingen etwa an diesem Punkt üblicherweise zu weinen an.

„Oh,“ sagte Draco Malfoy, ein wenig verwirrt klingend. Dann verzogen sich seine Lippen zu einem überheblichen Lächeln. „Es ist gut, jemanden zu treffen, der weiß, wo sein Platz ist."

Eine der Assistentinnen, die, welche Harry erkannt zu haben schien, gab ein unterdrücktes hustendes Geräusch von sich.

Harry plapperte weiter. „Ich bin hocherfreut Sie kennenzulernen, Mr. Malfoy. Einfach unsagbar erfreut. Und Hogwarts genau im selben Jahr wie Sie zu besuchen! Es lässt mein Herz rasen."

Uups. Dieser letzte Teil klang vielleicht etwas seltsam, als ob er mit Draco flirten würde oder sowas.

„Und \emph{ich} bin erfreut zu sehen, dass ich mit dem Respekt behandelt werde, den man der Familie Malfoy schuldig ist,“ gab der andere Junge zurück, begleited von einem Lächeln, wie es der höchste aller Könige dem niedersten seiner Untertanen zukommen lassen könnte, wenn der Untertan ehrlich, aber arm wäre.

Äh... Verdammt, Harry hatte Schwierigkeiten, sich seinen nächsten Satz zurechtzulegen. Nun, jeder \emph{wollte} Harry Potter die Hand schütteln, also -- „Wenn meine Kleidung fertig ist, Sir, würden Sie mir die Ehre gewähren, meine Hand zu schütteln? Ich könnte mir nichts mehr ersehnen, als Krönung dieses Tages, nein, dieses Monats, viel eher meines gesamten Lebens."

Der Junge mit den weiß-blonden Haaren funkelte ihn daraufhin an. „And was haben \emph{Sie} für die Malfoys getan, einen solchen Gefallen zu verdienen?"

\emph{Oh, ich werde das so dermaßen an der nächsten Person ausprobieren, die mir die Hand schütteln will.} Harry neigte seinen Kopf. „Nein, nein, Sir, ich verstehe. Es tut mir leid, dass ich gefragt habe. Es sollte mich eher geehrt fühlen, Ihnen die Schuhe putzen zu dürfen."

„In der Tat,“ schnappte der andere Junge. Sein strenges Gesicht hellte sich ein wenig auf. „Sagen Sie mir, welchem Haus glauben Sie, werden Sie zugewiesen? Ich bin natürlich für Slytherin bestimmt, wie mein Vater Lucius vor mir. Und in Ihrem Fall, ich nehme an Haus Hufflepuff oder vielleicht Haus Elf."

Harry grinste verlegen. „Professor McGonagall sagte, ich bin die Ravenclaw-hafteste Person, die sie je gesehen oder von der sie auch nur aus Legenden gehört hat, so sehr, dass Rowena selbst mir raten würde, öfter nach draußen zu gehen, was immer \emph{das} heißt und dass ich unzweifelhaft im Ravenclaw-Haus enden werde, wenn der Hut nicht zu laut schreit, als dass der Rest von uns irgendetwas verstehen könnte, Zitat Ende."

„Wow,“ sagte Draco Malfoy und klang ein wenig beeindruckt. Der Junge seufzte irgendwie wehmütig. „Ihre Schmeichelei war großartig oder das dachte ich zumindest - Sie würden sich im Slytherin-Haus auch gut machen. Üblicherweise wird nur meinem Vater diese Art Unterwürfigkeit zuteil. Ich \emph{hoffe}, die anderen Slytherins werden sich bei mir einschmeicheln, jetzt, wo ich in Hogwarts bin... ich vermute, das ist dann wohl ein gutes Zeichen."

Harry hustete. „Entschuldigung, eigentlich habe ich gar keine Ahnung wer du bist."

„\emph{Ach, komm schon!}“ sagte der Junge schwer enttäuscht. „Warum hast du das denn dann gemacht?“ Dracos Augen weiteten sich, als käme ihm ein plötzlicher Verdacht. „Und wie kannst du \emph{nicht} von den Malfoys gehört haben? Und was trägst du da für \emph{Kleidung}? Sind deine Eltern \emph{Muggel?}"

„Zwei meiner Eltern sind tot,“ sagte Harry. Es versetzte seinem Herzen einen Stich. Wenn er es so ausdrückte - „Meine anderen beiden Eltern sind Muggel und sie haben mich aufgezogen."

„\emph{Was?}“ sagte Draco. „Wer \emph{bist} du?"

"Harry Potter, nett dich kennen zu lernen."

„\emph{Harry Potter?}“ keuchte Draco. „\emph{Der} Harry -“ und der Junge unterbrach sich hastig.

Es gab einen kurzen Moment der Stille.

Dann, mit strahlendem Enthusiasmus, „Harry Potter? \emph{Der} Harry Potter? Donnerwetter, ich wollte Sie schon immer mal treffen!"

Die Assistentin, die mit Draco beschäftigt war, gab ein ersticktes Geräusch von sich, machte aber mit ihrer Arbeit weiter und hob Dracos Arme, um vorsichtig seinen gemusterten Umhang zu entfernen.

„Sei still,“ schlug Harry vor.

"Kann ich ein Autogramm von Ihnen haben? Nein, warten Sie, ich möchte zuerst ein Foto mit Ihnen!"

"Sei\emph{still}sei\emph{still}sei\emph{still.}"\\ "Ich bin einfach so \emph{hocherfreut} Sie kennenzulernen!"

"Geh' in Flammen auf und stirb."

"Aber Sie sind Harry Potter, der glorreiche Retter der Zauberwelt! Jedermann's Held, Harry Potter! Ich wollte immer genau so sein, wie Sie, wenn ich erwachsen bin, damit ich -"

Draco brach mitten im Satz ab, sein Gesicht erstarrte in absolutem Entsetzen.

Groß, weißhaarig, auf kühle Weise elegant in einem schwarzen Umhang von feinster Qualität. Eine Hand einen Stock mit silbernem Griff haltend, der den Charakter einer tödlichen Waffe annahm, nur indem er in dieser Hand lag. Seine Augen betrachteten den Raum mit dem leidenschaftslosen Blick eines Scharfrichters; ein Mann, für den das Töten nicht schmerzlich, geschweige denn strengstens verboten war, sondern nur eine Routine-Handlung, wie das Atmen.

Das war der Mann, der, in genau diesem Moment, durch die offene Tür spaziert war.

„Draco,“ sagte der Mann, leise und sehr verärgert, „\emph{was redest} du da?"

In einem Sekundenbruchteil mitfühlender Panik entwarf Harry einen Rettungsplan.

„Lucius Malfoy!“ keuchte Harry Potter. „\emph{Der} Lucius Malfoy?"

Eine von Malkins Assistentinnen musste sich wegdrehen und die Wand anschauen.

Kühle, mörderische Augen betrachteten ihn. „Harry Potter."

"Es ist so, so eine Ehre Sie kennen zu lernen!"

Die dunklen Augen weiteten sich, tödliche Drohung wich schockierter Überraschung.

„Ihr Sohn hat mir \emph{alles} über Sie erzählt,“ sprudelte Harry heraus, kaum noch wissend, was da eigentlich aus seinem Mund kam, stattdessen nur noch so schnell wie möglich redend. „Aber natürlich wusste ich auch schon lange davor alles über Sie, jeder kennt Sie ja, den großen Lucius Malfoy! Den meistgeehrten Würdenträger des gesamten Hauses Slytherin; ich habe schon mit dem Gedanken gespielt, selbst zu versuchen, nach Slytherin zu kommen, nur weil ich hörte, dass Sie als Kind dort waren -"

„\emph{Was reden Sie da, Mr. Potter?}“ kam ein Beinahe-Schrei von außerhalb des Geschäfts und Professor McGonagall platzte eine Sekunde später herein.\\ Es stand solch blankes Entsetzen in ihrem Gesicht, dass Harrys Mund sich automatisch öffnete, um dann doch nichts zu sagen zu haben.

„Professor McGonagall!“ schrie Draco. „Sind Sie es wirklich? Ich habe von meinem Vater so viel über Sie gehört, ich habe mit dem Gedanken gespielt, zu versuchen, nach Gryffindor zu kommen, damit ich -"\\ „\emph{Was?}“ bellten Lucius Malfoy und Professor McGonagall in perfektem Einklang, direkt nebeneinander stehend. Ihre Köpfe drehten sich gleichzeitig, um einander anzusehen und dann schreckten die beiden voreinander zurück, als führten sie einen synchronisierten Tanz auf.

Es gab ein plötzliches Durcheinander als Lucius Draco packte und aus dem Geschäft zog.

Und dann war Stille.

In Professor McGonagalls linker Hand lag ein kleines Trinkglas, dass in der inzwischen vergessenen Eile zur Seite gekippt war und nun langsam Tröpfchen von Alkohol in die kleine Rotwein-Pfütze tropfen ließ, die auf dem Boden entstanden war.

Professor McGonagall schritt vorwärts in das Geschäft, bis sie Madam Malkin gegenüberstand.

„Madam Malkin,“ sagte Professor McGonagall, ihre Stimme ruhig. „Was ist hier geschehen?"

Madam Malkin blickte vier Sekunden lang still zurück, dann drehte sie durch. Sie fiel gegen die Wand, lachte keuchend und das löste ihre beiden Assistentinnen aus ihrer Starre, von denen eine mit ihren Händen und Knien auf den Fußboden fiel und hysterisch zu kichern begann.

Professor McGonagall drehte sich langsam zu Harry um, ihr Gesichtsausdruck frostig. „Ich lasse Sie für sechs Minuten allein. Sechs Minuten, Mr. Potter, bei der Uhr selbst."

„Ich habe nur Witze gemacht,“ protestierte Harry, als die Geräusche hysterischen Gelächters in der Nähe weitergingen.

„\emph{Draco Malfoy sagte vor seinem Vater, er wolle nach Gryffindor!} Witze zu machen \emph{ist nicht genug}, um \emph{das} zu tun!“ Professor McGonagall stoppte, sichtbar atmend. „Welcher Teil von 'lassen Sie sich ihre Umhänge anpassen' klang für Sie nach \emph{bitte belegen Sie das ganze Universum mit einem Verwirrungszauber!}"

"Er war in einem situationsbedingten Kontext, in dem diese Handlungen einen internen Sinn machten -"

"Nein. Erkären Sie's nicht. Ich will nicht wissen, was hier drin passiert ist, niemals. Welche dunklen Kräfte Ihnen auch innewohnen, sie sind \emph{ansteckend} und ich will nicht so enden, wie der arme Draco Malfoy, die arme Madam Malkin oder ihre zwei armen Assistentinnen."

Harry seufzte. Es war offensichtlich, dass Professor McGonagall nicht in der Stimmung war, vernünftige Erklärungen anzuhören. Er sah zu Madam Malkin, immer noch keuchend an der Wand und zu Malkins zwei Assistentinnen, die nun \emph{beide} auf die Knie gefallen waren und schließlich runter auf seinen eigenen von Maßbändern bedeckten Körper.

„Ich bin noch nicht ganz fertig damit, mir meine Kleider anpassen zu lassen,“ sagte Harry freundlich. „Warum gehen Sie nicht zurück und nehmen noch einen Drink?„

* Ich habe hier -- wie wahrscheinlich schon einige Übersetzer vor mir -- einige Probleme mit dem Konzept des \emph{Duzens} und \emph{Siezens}. Es scheint im Englischen nicht zu existieren, die deutsche Übersetzung klingt aber dumm, wenn sich jedermann duzt. Für den Moment nehme ich an, dass Professor McGonagall und alle anderen Lehrer und Autoritätspersonen in Hogwarts (in Übereinstimmung mit Bezeichnungen wie \emph{Mr. Potter}), im Allgemeinen die Schüler siezen und von diesen gesiezt werden. Was Schüler untereinander angeht, werde ich wohl variieren müssen, um den Unterhaltungen keinen merkwürdigen „Tonfall“ zu geben und vielleicht gibt es auch von ersterer Regel noch Ausnahmen, um gewisse besondere... Vertrauensverhältnisse abzubilden, aber ich will ja nicht spoilern.

