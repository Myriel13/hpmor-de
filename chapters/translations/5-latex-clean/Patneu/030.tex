

\hypertarget{gruppenarbeit-teil-1}{% \section{30. Gruppenarbeit, Teil 1}\label{gruppenarbeit-teil-1}}

\textbf{Kapitel 30: Gruppenarbeit, Teil 1\\ }

J. K. Rowling if a man tries to bother you, you can think blue, count two, and look for a red shoe.

--------------------------------------------------------------------------------------------------------------------------------------------

Es war Sonntag, der 3. November und schon bald würden die drei großen Mächte ihres Schuljahres, Harry Potter, Draco Malfoy und Hermine Granger, ihren Kampf um die absolute Vorherrschaft beginnen.

(Harry war leicht verstimmt darüber, wie der Junge-der-überlebt-hat vom Stand der absoluten Vorherrschaft zu einem von drei gleichgestellten Rivalen degradiert worden war, nur indem er einem Wettbewerb beitrat, doch er erwartete, ihn schon bald zurück zu erlangen.)

Das Schlachtfeld war ein dicht mit Bäumen bewachsener Abschnitt des nicht-Verbotenen Waldes, denn Professor Quirrell war der Ansicht, all seine Gegner sehen zu können sei zu langweilig, selbst für ihre allererste Schlacht.

Alle Schüler, die nicht selbst \emph{in} einer der Erstklässler-Armeen waren, kampierten in der Nähe und verfolgten das Geschehen über von Professor Quirrell vorbereitete Bildschirme. Mit Ausnahme von drei Gryffindors im vierten Jahr, die augenblicklich krank und bei Madam Pomfrey ans Bett gefesselt waren. Abgesehen davon waren alle da.

Die Schüler waren nicht in ihre üblichen Schulumhänge gekleidet, sondern in Muggel-Tarnuniformen, die Professor Quirrell irgendwo aufgetrieben und in ausreichender Menge und Auswahl zur Verfügung gestellt hatte, damit sie jedem passten. Der Grund war nicht, dass die Schüler sich um etwaige Flecken oder Risse hätten sorgen müssen, denn dafür gab es immerhin Zauber. Doch wie Professor Quirrell den überraschten Zauberergeborenen erklärt hatte, war schöne, gediegene Kleidung nicht besonders effizient, wenn es darum ging, sich in Wäldern zu verstecken oder zwischen Bäumen Deckung zu suchen.

Und auf der Brust jeder Uniform, ein Flicken mit dem Zeichen der eigenen Armee. Ein \emph{kleiner} Flicken. Wollte man, dass die eigenen Soldaten, etwa, farbige Bänder trugen, damit sie einander von Ferne identifizieren konnten und riskieren, dass der Feind die Bänder in die Finger bekam, so blieb einem das völlig selbst überlassen.

Harry hatte versucht, die Drachen-Armee als Namen zu bekommen.

Draco hatte fast einen Anfall bekommen und gesagt, das würde alle vollends verwirren.

Professor Quirrell hatte entschieden, Draco könne zuerst Anspruch auf den Namen erheben, wenn er es wünschte.

Also kämpfte Harry nun gegen die Drachen-Armee.*

Das war wahrscheinlich kein gutes Zeichen.

Als ihr Zeichen hatte Draco sich, anstelle des zu offensichtlichen feuerspeienden Drachenkopfes, einfach nur das Feuer auserwählt. Elegant, schlicht, tödlich: \emph{Das ist, was bleibt, wenn wir vorüber ziehen.} Sehr Malfoy.

Harry hatte, nachdem er Alternativen wie etwa das 501. Reserve-Bataillon** und Harrys Lakaien der Verdammnis in Erwägung zog, entschieden, seine Armee würde bekannt sein unter der schlichten und würdigen Bezeichnung die Chaos-Legion.

Ihr Zeichen war eine Hand, mit Fingern bereit zu schnipsen.

Man war \emph{einhellig} der Ansicht, das sei kein gutes Zeichen.

Harry hatte Hermine den ernst gemeinten Rat erteilt, dass die Jungen unter ihrem Kommando wahrscheinlich leicht besorgt wären, weil sie ein Mädchen war, das in dem Ruf stand, nett zu sein und dass sie etwas furchteinflößendes nehmen sollte, das ihnen versicherte, wie taff sie war und sie stolz machte, Teil ihrer Armee zu sein, wie das Blut-Kommando oder etwas in der Art.

Hermine nannte ihre Armee das Sunshine-Regiment.

Ihr Zeichen war ein Smiley.

Und in zehn Minuten befänden sie sich im Krieg.

Harry stand auf der hellen Waldlichtung, die ihr zugewiesener Startpunkt war, ein offener Platz mit alten und verrottenden Baumstümpfen, die aus unbekanntem Grund gefällt worden waren, der Boden bedeckt von verwehten Blätterhäufchen und den vertrockneten grauen Überresten von Gras, das den Herausforderungen der Sommerhitze nicht gewachsen gewesen war und die Sonne strahlte von oben herab.

Um ihn versammelt waren die dreiundzwanzig Soldaten, die Professor Quirrell ihm zugewiesen hatte. Natürlich hatte sich beinahe ganz Gryffindor eingeschrieben und mehr als die Hälfte von Slytherin, weniger als die Hälfte von Hufflepuff und eine Handvoll Ravenclaws. In Harrys Armee gab es zwölf Gryffindors und sechs Slytherins, vier Hufflepuffs und einen Ravenclaw neben ihm selbst… nicht dass sich das durch einen Blick auf die Uniformen irgendwie hätte sagen lassen. Kein Rot, kein Grün, kein Gelb, kein Blau. Nur Muggel-Tarnmuster und ein Flicken auf der Brust, mit der Darstellung einer Hand, die Finger bereit zu schnipsen.

Harry betrachtete seine dreiundzwanzig Soldaten, die alle die gleiche Uniform trugen, ohne Zeichen von Gruppenzugehörigkeit, außer jenem kleinen Flicken.

Und siehe, Harry lächelte, denn er begriff, welche Absicht hinter diesem Teil von Professor Quirrells Masterplan stand und Harry würde das auch in vollem Maße für seine \emph{eigenen} Zwecke zu nutzen wissen.

Es gab da eine legendäre Episode der Sozialpsychologie, genannt das Ferienlagerexperiment. Es war durchgeführt worden während der von Fassungslosigkeit geprägten Nachwehen des Zweiten Weltkrieges, mit dem Ziel die Ursachen von Konflikten zwischen Gruppen und entsprechende Gegenmaßnahmen zu ergründen. Die Wissenschaftler hatten ein Sommercamp vorbereitet für 22 Jungen aus 22 verschiedenen Schulen, alle aus stabilen Mittelklasse-Familien ausgewählt. Die erste Phase des Experiments hatte herausfinden sollen, was nötig war, um einen Konflikt zwischen Gruppen \emph{auszulösen.} Man hatte die 22 Jungen in zwei Gruppen zu jeweils 11 aufgeteilt -

- und das hatte völlig ausgereicht.

Die Feindseligkeiten hatten in dem Moment begonnen, als die zwei Gruppen auf die Anwesenheit der jeweils anderen in dem State Park aufmerksam geworden waren, gleich beim ersten Zusammentreffen hagelte es Beleidigungen. Sie hatten sich selbst die Eagles und die Rattlers genannt (zuvor hatten sie keine Namen gebraucht, als sie geglaubt hatten, sie seien die einzigen in dem Park) und daraufhin gegensätzliche Gruppen-Stereotype herausgebildet; die Rattlers hielten sich selbst für rau-und-taff und heftig fluchend, die Eagles wiederum entschieden, sei seien aufrecht-und-anständig.

Der andere Teil des Experiments hatte darin bestanden, zu testen, wie man Gruppenkonflikte lösen konnte. Die Jungen zusammen zu bringen, um sich ein Feuerwerk anzusehen, hatte überhaupt nicht funktioniert. Sie hatten sich nur angebrüllt und voneinander fern gehalten. Was dagegen funktioniert \emph{hatte,} war die Warnung, vor möglichen Vandalen im Park und dass die beiden Gruppen zusammenarbeiten mussten, um ein Problem mit der Wasserversorgung des Parks zu lösen. Eine gemeinsame Aufgabe, ein gemeinsamer Feind.

Harry hegte den starken Verdacht, Professor Quirrell habe dieses Prinzip tatsächlich sehr gut verstanden, als er entschieden hatte, \emph{drei} Armeen pro Jahrgang zu schaffen.

\emph{Drei} Armeen.

Nicht \emph{vier.}

Und definitiv \emph{nicht} nach Häusern getrennt… abgesehen davon, dass Draco neben Mr. Crabbe und Mr. Goyle keine weiteren Slytherins zugewiesen worden waren.

Es waren Sachen wie diese, die Harry überzeugten, dass Professor Quirrell, trotz seiner zur Schau gestellten dunklen Ausstrahlung und seiner vorgeblichen Neutralität im Konflikt zwischen Gut und Böse, insgeheim doch dem Guten den Rücken stärkte, nicht das Harry je wagen würde, das laut zu sagen.

Und Harry hatte entschieden, Professor Quirrells Plan voll auszunutzen, um eine Gruppenidentität nach \emph{seiner} Vorstellung zu schaffen.

Die Rattlers hatten, sobald sie auf die Eagles getroffen waren, von sich selbst gedacht als rau-und-taff und sich entsprechend verhalten.

Die Eagles hatten entschieden, sie seien aufrecht-und-anständig.

Und auf jener hellen Waldlichtung, verstreut zwischen den alten und verrottenden Baumstümpfen, umrissen vom Sonnenlicht, das von oben erstrahlte, befanden sich General Potter und seine dreiundzwanzig Soldaten, aufgestellt in nichts, was auch nur entfernt an eine Formation erinnert hätte. Einige Soldaten standen, andere saßen, manche standen auf einem Bein, nur um anders zu sein.

Immerhin war dies die \emph{Chaos}-Legion.

Und wenn es keinen \emph{Grund} gab, fein säuberlich in Reih` und Glied zu stehen, so hatte Harry verächtlich gemeint, dann würde es auch keine fein säuberlichen Reihen geben.

Harry hatte die Armee in 6 Trupps zu je 4 Soldaten eingeteilt, jeder Trupp kommandiert von einem Trupp-Anzeiger.*** Alle Truppen hatten den strikten Befehl, jeden Befehl zu missachten, der ihnen gegeben wurde, wenn es in dem Moment eine gute Idee zu sein schien, einschließlich diesem… es sei denn Harry oder der Trupp-Anzeiger stellte dem Befehl die Formel "Merlin sagt" voran, in welchem Fall tatsächlich Gehorsam erwartet wurde.

Die wichtigste Angriffstaktik der Chaos-Legion bestand darin, sich aufzuteilen und aus verschiedenen Richtungen heranzustürmen, und aus willkürlich wechselnden Winkeln den zugelassenen Schlafzauber so schnell zu feuern, wie man die magische Stärke wieder aufbauen konnte. Und sah man eine Chance, den Gegner abzulenken oder zu verwirren, so ergriff man sie.

Schnell. Kreativ. Unberechenbar. Ungleichförmig. Befolge nicht nur Befehle, denk nach, ob das was du tust, \emph{jetzt gerade} Sinn macht.

Harry war nicht ganz so sicher, wie er sich gegeben hatte, dass dies tatsächlich das Optimum militärischer Effizienz darstellte… doch ihm war die einmalige Gelegenheit zuteil geworden, die Art und Weise zu ändern, wie einige der Schüler \emph{von sich selbst dachten} und so würde er sie nutzen.

Fünf Minuten bis Kriegsbeginn, nach Harrys Uhr zu urteilen.

General Potter ging (marschierte nicht) hinüber zu seiner gespannt wartenden Luftwaffe, die Besenstiele bereits fest im Griff.

"Alle Geschwader zum Rapport," sagte General Potter. Sie hatten das während ihrer einen Trainingslektion am Samstag geübt.

"Führer Rot bereit," sagte Seamus Finnigan, der keine Ahnung hatte, was es bedeutete.

"Rot Fünf bereit," sagte Dean Thomas, der sein ganzes Leben darauf gewartet hatte, das zu sagen.

"Führer Grün bereit," sagte Theodore Nott eher steif.

"Grün Einundvierzig bereit," sagte Tracey Davis.

"Ich will, dass ihr in der Luft seid, in der Sekunde wenn wir die Glocke hören," sagte General Potter. "Nicht angreifen, ich wiederhole, nicht angreifen. Zieht euch zurück, wenn ihr unter Feuer geratet." (Natürlich zielte man \emph{nicht} mit Schlafzaubern auf die Besen; man feuerte einen Zauber, der auf alles, was er traf, eine Zeit lang einen roten Schimmer warf. Traf man den Besen oder den Reiter, waren sie raus aus dem Krieg.) "Führer Rot und Rot Fünf, fliegt zu Malfoys Armee so schnell ihr könnt, bleibt so hoch wie ihr könnt und sie noch seht, kehrt augenblicklich zurück, sobald ihr sicher wisst, was sie tun. Führer Grün macht das gleiche bei Grangers Armee. Grün Einundvierzig, du fliegst über uns und hältst Ausschau nach allen anrückenden Besen oder Soldaten, du und nur du hast die Erlaubnis, zu feuern. Und denkt daran, ich sagte für nichts davon 'Merlin sagt', aber wir \emph{brauchen} wirklich die Informationen. Für Chaos!"

"Für Chaos!" gaben die vier mit variierendem Enthusiasmus zurück.

Harry erwartete, dass Hermine einen direkten Angriff auf Draco startete, in welchem Falle er seine Truppen in Position bringen würde, um sie zu unterstützen, doch erst nachdem sie schwere Verluste eingefahren und einigen Schaden verursacht hatte. Wenn möglich würde er es als heldenhafte Rettung darstellen; immerhin wäre es der Sache nicht dienlich, wenn Sunshine glaubte, Chaos sei nicht ihr Freund.

Doch nur für den Fall, dass \emph{nicht…} nun, deshalb blieb die Chaos-Legion an Ort und Stelle, bis Führer Grün sich zurück meldete.

Dracos Züge würden in seinem eigenen Interesse liegen. Er würde voraussichtlich seine Armee darauf vorbereiten, sich gegen Hermine zu verteidigen; ihm mochte klar sein oder auch nicht, dass Harry darüber gelogen hatte, mit dem Angriff bis nach dem Ende dieser Schlacht zu warten. Harry hatte trotzdem zwei Besen auf die Drachen-Armee angesetzt, nur für den Fall, sie \emph{versuchten} irgendetwas und nur für den Fall, dass Mr. Goyle oder Mr. Crabbe gut genug waren, um einen Besenstiel vom Himmel zu schießen.

Doch General Granger war die Unvorhersehbare und Harry konnte seinen Zug nicht machen, ehe er nicht wusste, wie sie zog.

--------------------------------------------------------------------------------------------------------------------------------------------

Im Herzen des Waldes, wo schattenhafte Muster, geworfen vom schwankenden Blätterdach der Bäume, über den Grund tanzten, stand General Malfoy, wo die Bäume sich etwas lichteten und betrachtete seine Truppen mit stiller Genugtuung. Sechs Einheiten zu je drei Truppen, eine Einheit aus vier Piloten (der Gregory zugewiesen war) und die Kommando-Einheit, die aus ihm selbst und Vincent bestand. Sie waren nur für kurze Zeit am vorherigen Samstag gedrillt worden, doch Draco war zuversichtlich, dass es ihm gelungen war, die Grundlagen zu vermitteln. Bleibt bei euren Kameraden, haltet ihnen den Rücken frei und vertraut darauf, dass sie den euren decken. Bewegt euch als Einheit. Befolgt Befehle und zeigt keine Angst. Zielen, feuern, in Bewegung bleiben, wieder zielen, wieder feuern.

Die sechs Einheiten waren um Draco herum zu einer Verteidigungslinie angeordnet und blickten wachsam in den Wald hinaus. Rücken-an-Rücken standen sie, die Zauberstäbe gesenkt, bis sie zuschlagen mussten.

Sie ähnelten bereits bemerkenswert den Auroren-Einheiten, deren Training Draco während der Inspektionen seines Vaters beobachtet hatte.

Chaos und Sunshine würden gar nicht wissen, was sie getroffen hatte.

"Achtung," sagte General Malfoy.

Die sechs Einheiten lösten sich auf und schnellten zu Draco herum; die Gesichter seiner Besenreiter wandten sich ihm zu, von dort wo sie standen, die Besen bereits in der Hand.

Draco hatte entschieden, mit dem Salutieren noch zu warten, bis sie ihre erste Schlacht gewonnen hätten und Gryffindors und Hufflepuffs bereitwilliger wären, einem Malfoy zu salutieren.

Doch seine Soldaten standen bereits stramm genug, besonders die Gryffindors und Draco fragte sich, ob die Verzögerung überhaupt nötig gewesen war. Gregory hatte sich unauffällig umgehört und berichtete, dass Dracos Bereitschaft, Harry Potter im Verteidigungs-Unterricht zur Seite zu stehen, als Professor Quirrell Harry lehrte, zu verlieren, Draco als akzeptablen Kommandeur ausgezeichnet hatte. Zumindest wenn man seiner Armee zugewiesen war. \emph{Nicht alle Slytherins sind gleich; es gibt Slytherins und dann gibt es Slytherins,} war was die Gryffindors in Dracos Armee gegenüber ihren Hauskameraden zitierten.

Draco war ehrlich \emph{erstaunt} darüber, wie unglaublich \emph{einfach} das gewesen war. Draco hatte zunächst protestiert, dass ihm keine Slytherins zugewiesen wurden und Professor Quirrell hatte erklärt, wenn er der erste Malfoy sein wolle, der die vollständige politische Kontrolle über das Land erlangte, dann müsse er auch lernen, wie man die anderen drei Viertel der Bevölkerung regierte. Es waren Sachen wie diese, die Draco davon überzeugten, dass Professor Quirrell doch eine ganze Menge mehr Sympathie für die Guten hegte, als er nach außen hin durchblicken ließ.

Die eigentliche Schlacht würde nicht einfach werden, insbesondere wenn Granger die Drachen zuerst angriff. Draco hatte sich mit der Frage geplagt, ob er alle seine Kräfte sofort in einem Präventivschlag gegen Granger einsetzen sollte, war aber besorgt, dass (1) Harry ihn komplett in die Irre geführt haben mochte, bezüglich dessen, was Granger wahrscheinlich tun würde und (2) Harry ihn darüber in die Irre geleitet haben mochte, erst nach Grangers Angriff in die Schlacht einzugreifen.

Allerdings besaß die Drachen-Armee eine Geheimwaffe, genau genommen drei davon, was für den Sieg ausreichen mochte, selbst wenn sie von beiden Armeen zugleich angegriffen würden…

Es war nun beinahe soweit und das bedeutete, es war Zeit für die Ansprache vor der Schlacht, die Draco entworfen und auswendig gelernt hatte.

"Die Schlacht wird bald beginnen," sagte Draco. Seine Stimme war ruhig und präzise. "Denkt an alles, was ich, Mr. Crabbe und Mr. Goyle euch gezeigt haben. Eine Armee ist siegreich, weil sie diszipliniert und tödlich ist. General Potter und die Chaos-Legion werden nicht diszipliniert sein. Granger und das Sunshine-Regiment werden nicht tödlich sein. Wir sind diszipliniert, wir sind tödlich, wir sind Drachen. Die Schlacht wird bald beginnen und wir werden siegen."

--------------------------------------------------------------------------------------------------------------------------------------------

\emph{\emph{(Improvisierte Ansprache von General Potter an die Chaos-Legion, unmittelbar vor ihrer ersten Schlacht, am 3. November 1991 um 2:56 am Nachmittag:)}}

\emph{Meine Truppen, ich werde euch nicht anlügen, unsere Lage heute ist mehr als ernst. Die Drachen-Armee hat noch keine einzige Schlacht verloren. Und Hermine Granger… hat ein sehr gutes Gedächtnis. Die Wahrheit ist,} \emph{die meisten von euch werden wahrscheinlich sterben. Und die Überlebenden werden die Toten beneiden. Doch wir müssen das hier gewinnen. Wir müssen das hier gewinnen, damit eines Tages unsere Kinder} \emph{wieder} \emph{den Geschmack von Schokolade kosten können. Alles steht hier auf dem Spiel. Buchstäblich alles. Wenn wir verlieren, wird das ganze Universum einfach verlöschen wie eine Glühbirne. Und jetzt wird mir gerade klar, dass die meisten von euch gar nicht wissen, was eine Glühbirne ist. Nun, glaubt mir einfach, wenn ich sage, es ist schlimm.} \emph{Doch} \emph{wenn wir schon untergehen, so lasst uns kämpfend untergehen, wie Helden, damit wenn die Finsternis hereinbricht, wir uns sagen können,} \emph{\emph{zumindest hatten wir Spaß.}Fürchtet ihr den Tod? Ich weiß, ich schon. Ich kann die kalten Schauer der Angst spüren, als schütte man mir} \emph{Eiskrem in mein Hemd. Doch ich weiß… dass die Geschichte uns zusieht. Sie hat uns zugesehen, als wir unsere Uniformen anlegten. Sie hat wahrscheinlich Fotos gemacht. Und die Geschichte, meine Truppen, wird von den Siegern geschrieben. Wenn wir das hier gewinnen, können wir unsere eigene Geschichte schreiben. Eine Geschichte in der Hogwarts von vier abtrünnigen Hauselfen gegründet wurde. Wir können alle zwingen, diese Geschichte zu lernen, auch wenn sie nicht wahr ist und wenn sie in unseren Tests nicht die richtige Antwort geben… fallen Sie im Unterricht durch. Ist das nicht wert, dafür zu sterben? Nein, antwortet nicht darauf. Manche Dinge bleiben besser unbekannt. Keiner von uns weiß, weshalb wir hier sind. Keiner von uns weiß, weshalb wir kämpfen. Wir} \emph{erwachten} \emph{einfach in diesen Uniformen, in diesem mysteriösen Wald,} \emph{nur} \emph{in dem Wissen, es} \emph{gibt} \emph{keinen anderen Weg, unsere Namen und Erinnerungen zurück zu erlangen, als den Sieg. Die Schüler in jenen anderen Armeen dort draußen… sie sind genau wie wir. Sie wollen nicht sterben. Sie kämpfen, um einander zu beschützen, die einzigen Freunde, die} \emph{ihnen noch bleiben. Sie kämpfen, weil sie wissen, sie} \emph{haben} \emph{Familien, die sie vermissen werden, auch wenn sie sich jetzt nicht erinnern. Sie kämpfen vielleicht sogar, um die Welt zu retten. Doch wir haben einen besseren Grund, zu kämpfen, als sie es tun. Wir kämpfen, weil wir es lieben. Wir kämpfen zum Vergnügen} \emph{schauerlicher Ungetüme jenseits von Raum und} \emph{Zeit. Wir kämpfen, denn wir sind Chaos. Die letzte Schlacht wird bald beginnen, drum lasst mich euch jetzt sagen, denn später erhalte ich keine Chance dazu, es} \emph{war} \emph{eine Ehre, euer Kommandeur zu sein, wenn auch nur kurz. Danke, danke euch allen. Und denkt daran, euer Ziel ist nicht nur, den} \emph{Feind} \emph{zu schlagen, er soll euch auch fürchten.}

--------------------------------------------------------------------------------------------------------------------------------------------

Ein gewaltiger, dröhnender Gong erschallte über dem Wald.\\ Und das Sunshine-Regiment setzte sich in Marsch.

--------------------------------------------------------------------------------------------------------------------------------------------

Die Spannung stieg und stieg, während Harry und die neunzehn anderen verbleibenden Soldaten auf die Rückmeldung der fliegenden Krieger warteten. Es sollte nicht lange dauern, Besen waren schnell und die Entfernungen in dem Wald nicht groß -

Zwei Besen näherten sich, mit hoher Geschwindigkeit, aus Richtung von Dracos Lager und alle Soldaten spannten sich an. Sie führten nicht die Manöver aus, die der heutige Code für einen \emph{befreundeten} Besen waren.

"\emph{Verteilen und Feuer!}" brüllte General Potter und ließ dann Worten Taten folgen, raste mit Höchstgeschwindigkeit auf die Deckung des Waldes zu und dann, sobald Harry unter den Bäumen war, wirbelte er herum, hob seinen Zauberstab, versuchte den Besen am Himmel auszumachen -

"Entwarnung!" rief eine Stimme. "Sie fliegen zurück!"

Harry zuckte im Geiste mit den Schultern. Es war nicht möglich gewesen, zu verhindern, dass Draco diese Information bekam und er würde nur erfahren, dass sie still hielten.

Und die Chaoten traten langsam aus dem Wald hervor -

"Ankommender Besen aus Richtung Granger!" rief eine andere Stimme. "Ich glaube, es ist Führer Grün, er hat Sturzflug und Rolle gemacht!"

Augenblicke später tauchte Theodore Nott aus dem Himmel herab und kam inmitten der Soldaten zum Stehen.

"Granger hat ihre Streitkräfte zweigeteilt!" rief Nott auf seinem Besen schwebend. Schweiß befleckte seine Uniform und alle Zurückhaltung war aus seiner Stimme gewichen. "Sie greift beide Armeen an! Zwei Besen schützen jede Streitmacht, sie haben mich den halben Weg hierher verfolgt!"

\emph{Sie teilte ihre Armee auf, was in aller Welt -}

Eine große Streitmacht, die ihr Feuer auf eine kleine konzentrierte, konnte diese rapide dezimieren, ohne im Gegenzug selbst viel Schaden zu nehmen. Wenn zwanzig Soldaten zehn Soldaten gegenüber traten, so würden zwanzig Schlafzauber auf die zehn Soldaten gerichtet, bei nur zehn Schlafzaubern in die andere Richtung, wenn also nicht jeder einzelne dieser ersten Schlafzauber sein Ziel traf, würde die kleinere Streitmacht mehr Leute verlieren, als sie mit sich nehmen konnte. \emph{Im} \emph{Kleinen besiegt}**** war der militärische Ausdruck für das, was geschah, wenn man seine Streitkräfte auf diese Weise teilte. Was \emph{dachte} Hermine sich bloß dabei…

Dann begriff Harry.

\emph{Sie spielte fair.}

Das würde ein langes Jahr in Verteidigung werden.

"In Ordnung," sagte Harry laut, damit die Armee es hören konnte. "Wir warten, bis das Rote Geschwader zurückkehrt und dann vernebeln wir Sunshine ein wenig den Tag."

--------------------------------------------------------------------------------------------------------------------------------------------

Draco hörte die Berichte seiner Flieger mit gefasster Miene, aller Schock im Inneren verborgen. Was \emph{dachte} Granger sich bloß dabei?

Dann begriff Draco.

\emph{Es war eine Finte.}

Eine von Sunshines zwei Streitmächten würde die Richtung ändern und beide würden sich zusammenschließen gegen… wen?

--------------------------------------------------------------------------------------------------------------------------------------------

Neville Longbottom marschierte durch den Wald, der näher rückenden Sunny-Streitmacht entgegen, blickte gelegentlich zum Himmel und hielt Ausschau nach Besen. Neben ihm marschierten seine Trupp-Kameraden, Melvin Coote und Lavender Brown von Gryffindor und Allen Flint von Slytherin. Allen Flint war ihr Trupp-Anzeiger, obwohl Harry Neville vorher im Vertrauen gesagt hatte, die Position sei seine, wenn er sie wolle.

Harry hatte Neville eine ganze Menge Dinge im Vertrauen gesagt, angefangen mit "Weißt du, Neville, wenn du so hammermäßig werden willst, wie der imaginäre Neville, der in deinem Kopf lebt, aber nichts machen darf, weil du dich fürchtest, dann solltest du dich wirklich für Professor Quirrells Armeen einschreiben."

Neville war nun \emph{überzeugt} davon, dass der Junge-der-überlebt-hat Gedanken lesen konnte. Es gab einfach keine andere Möglichkeit, wie Harry Potter davon wissen konnte. Neville hatte niemals mit \emph{irgendjemandem} darüber gesprochen oder irgendeinen Hinweis gegeben und \emph{andere} Menschen waren nicht so, nicht dass es Neville jemals aufgefallen wäre.

Und Harrys Versprechen hatte sich erfüllt, das hier \emph{fühlte} sich anders an als Sparring in Verteidigung. Neville hatte gehofft, das Sparring würde alles richten, was mit ihm nicht stimmte und, nun, das hatte es nicht. Auch wenn er im Unterricht ein paar Zauber auf einen anderen Schüler abfeuern konnte, unter Aufsicht von Professor Quirrell, der sicherstellte, dass nichts schief ging, auch wenn er ausweichen und das Feuer erwidern konnte, wenn es \emph{erlaubt} war und alle anderen es \emph{erwarteten} und ihn komisch ansehen würden, wenn er es \emph{nicht} täte, war nichts davon das gleiche, wie sich selbst behaupten zu können.

Doch Teil einer \emph{Armee} zu sein…

Etwas seltsames regte sich in Nevilles Innerem, als er an der Seite seiner Kameraden durch den Wald marschierte, auf ihren Uniformen ein Zeichen von Fingern, bereit zu schnipsen.

Es war ihm erlaubt, zu gehen, wenn er wollte, doch ihm war einfach nach marschieren zumute.

Neben ihm schien Melvin und Lavender und Allen ebenfalls allen nach marschieren zumute zu sein.

Und Neville stimmte leise das Lied von Chaos an.

Die Melodie entsprach dem, was ein Muggel erkannt hätte als John Williams Imperialer Marsch, auch bekannt als "Darth-Vader-Thema" und die Worte, die Harry hinzugefügt hatte, waren sehr eingängig.*****

\emph{Doom doom doom\\ Doom doom doom doom doom doom\\ Doom doom doom\\ Doom doom doom doom doom doom\\ DOOM doom DOOM\\ Doom doom doom-doom-doom doom doom\\ Doom doom-doom-doom doom doom\\ Doom doom doom, doom doom doom.}

Ab der zweiten Zeile stimmten auch die anderen mit ein und schon bald war aus den benachbarten Teilen des Waldes der gleiche gedämpfte Gesang zu vernehmen.

Und Neville marschierte an der Seite seiner Gefährten der Chaos-Legionäre, seltsame Gefühle regten sich in seinem Herzen, Vorstellung wurde Wirklichkeit, während seine Lippen ein schreckliches Lied der Verdammnis entließen.

--------------------------------------------------------------------------------------------------------------------------------------------

Harry starrte auf die Körper, die über den Wald verstreut lagen. Irgendwie war ihm ein wenig mulmig zumute und er musste sich heftig ins Gedächtnis rufen, dass sie nur schliefen. Es waren Mädchen unter den Gefallenen und irgendwie machte es das noch viel schlimmer und er würde darauf achten müssen, das niemals in Gegenwart von Hermine zu erwähnen oder die Auroren würden seine Überreste in einer \emph{winzigen} Teekanne aufsammeln können.

Die Hälfte der Sunshine-Armee hatte sich gegen ganz Chaos keinen großen Kampf geliefert. Die neun Bodentruppen waren unartikuliert schreiend umher gerannt, mit erhobenen Simplen Schilden, kreisförmigen Schirmen um Gesicht und Brust zu schützen. Doch man konnte nicht feuern und zur gleichen Zeit den Schild aufrecht erhalten und Harrys Soldaten hatten einfach auf die Beine gezielt. Alle Sunnies bis auf eine waren gefallen, sobald die "\emph{Somnium!}"-Schreie die Luft erfüllten. Diese Letzte hatte ihren Schild fallen lassen und es geschafft, einen von Harrys Soldaten auszuschalten, bevor sie von der zweiten Welle von Schlafzaubern getroffen wurde (Der Schlaffluch war auch bei mehrfachen Treffern sicher). Die zwei Sunny-Besen waren weit schwieriger herunter zu holen und hatten drei Chaoten auf dem Gewissen, bevor sie von massivem Bodenfeuer eingehüllt wurden.

Hermine war nicht unter den Gefallenen. Draco musste sie erwischt haben und das machte Harry auf ganz und gar unverständliche Weise \emph{wütend;} er war nicht sicher, ob aus einem Beschützerinstinkt für Hermine heraus oder weil er sich betrogen fühlte, dass er es nicht selbst hatte tun können oder vielleicht auch \emph{beides.}

"Okay," sagte Harry und hob die Stimme. "Eines sollte uns allen klar sein, das hier war kein richtiger Kampf. Das war ein Fehler von General Granger bei ihrer ersten Schlacht. Der wirkliche Kampf heute findet gegen die Drachen-Armee statt und er wird nicht annähernd so sein, wie dieser hier. Er wird sehr viel mehr Spaß machen. Rücken wir ab."

--------------------------------------------------------------------------------------------------------------------------------------------

Ein Besen stürzte vom Himmel herab, näherte sich mit erschreckender Geschwindigkeit, schwang auf der Spitze herum, bremste so stark ab, dass man beinahe hören konnte, wie die Luft protestierend aufkreischte und kam direkt neben Draco zum Stehen.

Es war keine gefährliche Angeberei. Gregory Goyle \emph{war} ganz einfach so gut und er verschwendete keine Zeit.

"Potter kommt," sagte Gregory ohne jede Spur seines üblichen künstlich gedehnten Tonfalls. "Sie haben noch immer alle vier Besen, willst du, dass ich sie ausschalte?"

"Nein," sagte Draco scharf. "Über ihrer Armee zu kämpfen, gibt ihnen einen zu großen Vorteil, sie werden vom Boden aus auf dich feuern und sogar du kannst vielleicht nicht allen ausweichen. Warte bis die Streitkräfte aufeinander treffen."

Draco hatte vier Drachen im Austausch für zwölf Sunnies verloren. Offenbar war General Granger \emph{tatsächlich} so unglaublich töricht gewesen, obwohl sie nicht unter den Angreifern gewesen war, daher hatte Draco keine Chance bekommen, sie dafür zu verhöhnen oder sie zu fragen, was in Merlins Namen sie sich nur dabei gedacht hatte.

Der wahre Kampf, das wussten sie alle, würde der gegen Harry Potter sein.

"Seid vorbereitet!" rief Draco seinen Truppen zu. "Haltet euch an eure Kameraden, agiert als Einheit, feuert sobald der Gegner in Reichweite ist!"

Disziplin gegen Chaos.

Es sollte kein großer Kampf werden.

--------------------------------------------------------------------------------------------------------------------------------------------

Immer mehr Adrenalin durchströmte Nevilles Blut, bis er das Gefühl hatte, kaum noch atmen zu können.

"Wir kommen näher," sagte General Potter gerade laut genug, dass seine Stimme die ganze Armee erreichte. "Zeit auszuschwärmen."

Nevilles Kameraden entfernten sich von ihm. Sie würden einander noch immer unterstützen, doch drängte man sich zu dicht zusammen, hatte der Feind es um einiges leichter, einen zu treffen; für die eigenen Kameraden bestimmtes Feuer konnte sein Ziel verfehlen und einen stattdessen selbst treffen. Man war sehr viel schwieriger zu treffen, wenn man sich aufteilte und so schnell bewegte, wie man nur konnte.

Das erste, was General Potter während ihrer Trainingseinheit getan hatte, war sie aufeinander feuern zu lassen, wenn beide Seiten rannten oder still standen und sich Zeit nahmen, zu zielen oder die eine sich bewegte und die andere stillstand - der Gegenzauber des Schlafzaubers war einfach, obwohl man ihn während echter Kämpfe nicht verwenden durfte. General Potter hatte sorgfältig alles aufgezeichnet, was geschah, ein wenig geknobelt und herumgerechnet und dann verkündet, es sei sinnvoller, wenn sie sich nicht darauf konzentrierten, sorgfältig zu zielen, sondern darauf, sich schnell zu bewegen, damit sie nicht getroffen würden.

Es störte Neville noch immer ein wenig, nicht Seite an Seite mit seinen Kameraden zu marschieren, doch die furchterregenden Schachtrufe, die sie gelernt hatten, erschallten bereits über seinen Kopf hinweg und das machte eine Menge wett.

Dieses mal, schwor sich Neville im Stillen, würde seine Stimme definitiv, ganz sicher kein Quieken von sich geben.

"Schilde hochfahren," sagte General Potter, "Energie auf die Frontaldeflektoren."

"\emph{Contego,}" murmelte die Armee und die kreisförmigen Schilde materialisierten sich vor Kopf und Brust.

Ein scharfer Geschmack erfüllte Nevilles Mund. General Potter hätte sie nicht angewiesen, die Schilde zu wirken, wenn sie nicht fast in Reichweite wären. Neville konnte die uniformierten Gestalten der Drachen ausmachen, die sich durch das dichte Laubwerk der Bäume bewegten und die Drachen würden sie ebenfalls sehen -

"\emph{Attacke!}" kam ein Schrei aus der Ferne, die Stimme von Draco Malfoy und General Potter bellte, "\emph{Angriff -}"

All das Adrenalin in Nevilles Blut wurde freigesetzt und seine Beine übernahmen das Kommando, ließen ihn fliegen, schneller als er je zuvor gerannt war, direkt auf den Feind zu und er wusste ohne hinzusehen, dass all seine Kameraden dasselbe taten.

"\emph{Blut für den Blutgott!}" schrie Neville. "\emph{Schädel für den Schädelthron! Ia!} \emph{Shub-Niggurath! Das Tor des Gegners ist seitwärts!}"

Es gab einen lautlosen Einschlag, als ein Schlafzauber seine Energie gegen Nevilles Schild verpuffen ließ. Wenn noch andere Zauber gefeuert worden waren, so hatten sie nicht getroffen.

Neville sah den flüchtigen Ausdruck der Angst auf Wayne Hopkins Gesicht als er neben zwei Gryffindors stand, die Neville nicht erkannte und dann -

- ließ Neville den Simplen Schild fallen und feuerte auf Wayne -

- verfehlte ihn -

- seine rasenden Beine trugen ihn \emph{direkt} an der feindlichen Gruppe vorbei und auf weitere drei Drachen zu, die die Zauberstäbe auf ihn richteten, ihre Münder öffneten sich -

- ohne auch nur nachzudenken tauchte Neville zum Waldboden ab, gerade als die drei Stimmen schrien "\emph{Somnium!}"

Es schmerzte, harte Steine und Zweige bohrten sich in Neville hinein, als er sich abrollte, es war nicht so schlimm, wie von einem Besenstiel zu fallen, doch er war trotzdem ganz schön hart auf dem Boden aufgekommen und dann blieb Neville, einer plötzlichen Eingebung folgend, still liegen und schloss die Augen.

"Aufhören!" schrie eine Stimme. "Schießt nicht auf uns, wir sind Drachen!"

In einem Anflug glorreicher Genugtuung wurde Neville klar, dass er es geschafft hatte, zwischen zwei Gruppen von Drachen zu gelangen, genau als die eine Gruppe auf ihn gefeuert hatte. Harry hatte davon gesprochen, als eine Taktik damit der Gegner zögerte zu feuern, doch offenbar funktionierte sie sogar noch etwas besser.

Und nicht nur das, die Drachen glaubten, sie hätten ihn \emph{erwischt,} da sie Neville hatten fallen sehen, genau als sie gefeuert hatten.

Neville zählte im Kopf bis zwanzig, dann öffnete er einen Spalt weit die Augen.

Die drei Drachen waren ganz in seiner Nähe, ihre Köpfe drehten sich wie rasend, während um sie herum Schreie von "\emph{Somnium!}" und "\emph{Schädel für den Schädelthron!}" die Luft erfüllten. Alle drei hatten jetzt Simple Schilde erhoben.

Nevilles Zauberstab lag noch immer in seiner Hand und es war nicht schwer, ihn auf die Schuhe eines der Jungen zu richten und zu flüstern "\emph{Somnium.}"\\ Neville schloss schnell die Augen und ließ seine Hand erschlaffen als er hörte, wie der Junge zu Boden fiel.

"\emph{Woherkam das?}" schrie Justin Finch-Fletchleys Stimme und Neville hörte Geraschel auf dem belaubten Waldboden, wie von zwei Drachen, die herumwirbelten auf der Suche nach einem Gegner.

"\emph{Schließt die Reihen!}" bellte Malfoys Stimme. "\emph{Alle Mann zu mir, lasst euch nicht auseinander treiben!}"

Nevilles Ohren vernahmen, wie die zwei Drachen tatsächlich über seinen hingestreckten Körper sprangen, als sie sich davon machten.

Neville öffnete die Augen, drückte sich ein wenig schmerzhaft auf die Füße, richtete dann seinen Zauberstab aus und sagte den neuen Zauber auf, den General Potter ihnen allen beigebracht hatte. Sie konnten keine echten Illusionen erschaffen, um den Gegner zu verwirren, doch selbst in ihrem Alter konnten sie -

"\emph{Ventriliquo,}"****** flüsterte Neville, deutete mit dem Zauberstab zur Seite von Justin und dem anderen Jungen und brüllte dann, "\emph{Für Cthulu und die Ehre!}"

Justin und der andere Junge hielten abrupt inne, drehten ihre Schilde dorthin, wo Neville seinen Kampfschrei hatte erklingen lassen, als auch schon mehrere "\emph{Somnium!}"-Schreie die Luft erfüllten und der andere Junge fiel, bevor Neville auch nur zielen konnte.

"\emph{Der letzte gehört mir!}" schrie Neville und dann sprintete er direkt auf Justin zu, der gemein zu ihm gewesen war, bis die älteren Hufflepuffs ihm den Kopf zurecht gerückt hatten. Neville war von seinen Kameraden umgeben und \emph{das} bedeutete -

"\emph{Spezial-Attacke, Chaotischer Sprung!}" heulte Neville mitten im Lauf und fühlte seinen Körper leichter werden, dann noch zweimal leichter, als seine Kameraden die Zauberstäbe auf ihn richteten und leise den Schwebezauber wirkten und Neville hob die linke Hand, schnippte mit den Fingern und stieß sich mit den Beinen so hart wie er konnte vom Boden ab und segelte durch die Luft. Der reine Schock zeichnete Justins Gesicht, als Neville über den Schild des anderen Jungen sprang, seinen Zauberstab auf die unter ihm vorüber ziehende Gestalt richtete und schrie "\emph{Somnium!}"

Weil ihm einfach danach gewesen war, deswegen.

Neville schaffte es nicht ganz, seine Füße richtig zu drehen und pflügte mehr in den Boden, als dass er landete, doch zwei Drittel der anderen Chaos-Legionäre hatten es geschafft, ihm währenddessen mit dem Zauberstab zu folgen und so kam er nicht allzu hart auf.

Und Neville kam keuchend auf die Füße. Er wusste, er sollte in Bewegung bleiben, überall schrien die Leute "Somnium!" -

"\emph{Ich bin Neville, der letzte Spross von Longbottom!}" schrie Neville zum Himmel empor, richtete den Zauberstab geradewegs nach oben, als wolle er den strahlend blauen Himmel selbst herausfordern, wissend dass nach diesem Tag nichts mehr so sein würde wie zuvor. "\emph{Neville von Chaos! Tretet mir entgegen, wenn ihr es wa-}"

(Als Neville im Nachhinein erwachte, wurde ihm erzählt, die Drachen-Armee habe dies als ihr Zeichen zum Gegenangriff aufgefasst.)

--------------------------------------------------------------------------------------------------------------------------------------------

\emph{Das Mädchen neben Harry sackte zu Boden, steckte den Schuss ein, der für ihn bestimmt} \emph{gewesen} \emph{war und} \emph{in der Ferne} \emph{konnte} \emph{er} \emph{Mr. Goyles hämisches Lachen hören, als sein Besen an ihnen vorbei donnerte und dabei so hart durch die Luft schnitt, sie hätte in seinem Schlepptau bersten} \emph{müssen.}

\emph{"\emph{Luminos!}" schrie einer der Jungen neben Harry, der die magische Stärke nicht} \emph{rechtzeitig} \emph{hatte wiederaufbauen können, um es früher zu tun und Mr. Goyle wich aus, ohne mit der Wimper zu zucken.}

\emph{Chaos} \emph{blieben} \emph{jetzt nur noch sechs Soldaten übrig und die Drachen-Armee hatte zwei; das einzige Problem war, dass einer dieser Soldaten unantastbar war und der andere drei Soldaten} \emph{beschäftigt hielt, nur um ihn in seinem Schild zu halten.}

\emph{Sie hatten mehr Soldaten an Mr. Goyle verloren, als an alle anderen Drachen zusammen, er spann und wand sich so schnell durch die Luft, dass niemand} \emph{ihn} \emph{treffen konnte und} \emph{\emph{dabei konnte er noch Leute abschießen.}}

\emph{Harry hatte alle möglichen Arten durchdacht, um Mr. Goyle zu stoppen, doch keine von ihnen war} \emph{\emph{sicher,}} \emph{selbst den Schwebezauber zu benutzen, um ihn zu verlangsamen (er war ein kontinuierlicher Strahl und viel leichter zu zielen) wäre nicht sicher, weil er vom Besenstiel fallen könnte, ihm} \emph{Gegenstände} \emph{in den Weg zu werfen, wäre nicht sicher und es wurde immer schwerer, sich das ins Gedächtnis zu rufen, während Harrys Blut gefror.}

\emph{\emph{Es ist ein Spiel. Du versuchst nicht, ihn}} \emph{umzubringen.} \emph{\emph{Wirf nicht all deine Pläne für die Zukunft weg für ein Spiel…}}

\emph{Harry konnte das Muster} \emph{\emph{sehen,}} \emph{er konnte} \emph{\emph{sehen,}} \emph{wie Mr. Goyle durch die Luft spann, er konnte} \emph{\emph{sehen,}} \emph{wie und wann sie alle feuern mussten, um ein Netz von Schüssen zu spinnen, dem} \emph{Mr. Goyle nicht würde ausweichen können, aber er hatte es seinen Soldaten einfach nicht schnell genug} \emph{\emph{erklären}} \emph{können, sie konnten ihre Schüsse nicht gut genug koordinieren und jetzt hatten sie auch nicht mehr genug Leute dafür -}

\emph{\emph{Ich weigere mich zu verlieren, nicht so, nicht meine gesamte Armee an einen Soldaten!}}

\emph{Mr. Goyles Besen wendete schneller als irgendetwas} \emph{dazu} \emph{hätte in der} \emph{Lage sein sollen und} \emph{bog in} \emph{einen Anflugwinkel auf Harry und seine überlebenden Truppen} \emph{ein, er konnte spüren, wie sich der Junge neben ihm anspannte, bereit sich selbst vor seinen General zu werfen.}

\emph{\emph{ACH, ZUM TEUFEL.}}

\emph{Harrys Zauberstab hob sich, fokussierte auf Mr. Goyle, Harrys Geist visualisierte das Muster,} \emph{Harrys} \emph{Lippen öffneten sich und seine Stimme schrie -}

\emph{"\emph{Luminosluminosluminosluminosluminosluminosluminosluminosluminosluminosluminosluminos-}"}

--------------------------------------------------------------------------------------------------------------------------------------------

Als Harrys Augen sich wieder öffneten, fand er sich liegend in bequemer Position wieder, die Hände, über der Brust gefaltet, hielten seinen Zauberstab, wie ein gefallener Held.

Langsam setzte Harry sich auf. Seine \emph{Magie} schmerzte, ein seltsames Gefühl, doch nicht gänzlich unangenehm, so wie das Brennen und die Trägheit nach einer großen körperlichen Anstrengung.

"Der General ist wach!" rief eine Stimme und Harry blinzelte und richtete seine Aufmerksamkeit in jene Richtung.

Vier seiner Soldaten hielten ihre Zauberstäbe auf eine schimmernde prismatische Halbkugel gerichtet und Harry erkannte, dass die Schlacht noch nicht vorüber war. Richtig… er war nicht von einem Schlaffluch getroffen worden, sondern hatte sich nur erschöpft, wenn er also aufwachte, war er immer noch im Spiel.

Harry vermutete, er würde sich die ein oder andere Standpauke darüber anhören dürfen, seine Magie nicht wegen eines Kinderspiels bis zur Bewusstlosigkeit zu erschöpfen. Doch er hatte Mr. Goyle nicht verletzt, als er die Beherrschung verloren hatte und das war das wichtigste.

Dann rastete eine weitere Erkenntnis ein in Harrys Geist und er blickte auf den Stahlring am kleinen Finger seiner linken Hand hinab und hätte beinahe laut geflucht als er sah, dass der winzige Diamant fehlte und ein Marshmallow auf dem Boden lag, nahe der Stelle, wo er gefallen war.

Er hatte diese Transfiguration seit siebzehn Tagen aufrecht erhalten und würde jetzt wieder von vorn anfangen müssen.

Hätte schlimmer sein können. Er hätte das auch vierzehn Tage später machen können, \emph{nachdem} Professor McGonagall ihm erlaubt hätte, den Stein seines Vaters zu transfigurieren. Das war eine sehr gute Lektion, auf die leichte Tour gelernt.

\emph{Notiz an mich: Immer Ring vom Finger entfernen bevor Magie komplett erschöpft.}

Harry drückte sich hoch, stellte sich dabei ziemlich schwerfällig an. Die eigene Magie aufzubrauchen, erschöpfte zwar die Muskeln nicht, doch zwischen Bäumen herum zu springen ganz entschieden schon.

Er wankte hinüber zu der irisierenden Halbkugel, in der Draco Malfoy sich befand, der seinen Zauberstab erhoben hielt, um den Schild zu erhalten und Harry mit einem kalten Lächeln bedachte.

"Wo ist die fünfte Soldatin?" sagte Harry.

"Ähm…" sagte ein Junge, dessen Name Harry gerade nicht einfiel. "Ich habe einen Schlaffluch auf den Schild gefeuert und er ist abgeprallt und hat Lavender getroffen, ich meine, der Winkel hätte nicht stimmen sollen, aber trotzdem…"

Draco grinste höhnisch in seinem Schild.

"Also, lass mich mal raten," sagte Harry und blickte Draco direkt in die Augen, "diese hübschen kleinen Trios sind die Formation, die das magische Berufsmilitär verwendet? Bestehend aus trainierten Soldaten, die bewegliche Ziele leicht treffen können, wenn ihre Hände ruhig sind und die ihre Verteidigungskräfte kombinieren können, solange sie zusammen bleiben? Anders als \emph{deine} Soldaten?"

Das Grinsen war aus Dracos Gesicht gewichen, der nunmehr hart und grimmig drein blickte.

"Du weißt," sagte Harry leichthin in dem Wissen, dass keiner der anderen die wahre Botschaft zwischen ihnen verstehen würde, "das zeigt nur, dass man immer alles hinterfragen sollte, was man die eigenen Vorbilder machen sieht und fragen sollte, wieso es getan wird und ob es im Kontext auch für einen selbst Sinn macht, es zu tun. Vergiss übrigens nicht, diesen Rat im echten Leben anzuwenden. Und danke für die langsamen, dicht gedrängten Ziele."

Denn Draco hatte diese Lektion bereits erhalten und sie, wie Harry vermutete, aus dem Verdacht heraus verworfen, Harry würde nur versuchen, seine Loyalität weiter von den Traditionen der Reinblüter abzubringen. Was natürlich \emph{stimmte.} Doch dieses Beispiel würde ihm am nächsten Samstag eine exzellente Ausrede bieten, zu behaupten, dass Autoritäten in Frage zu stellen lediglich eine praktische Technik fürs reale Leben darstellte. Und Harry würde ebenfalls die Experimente erwähnen, die er durchgeführt hatte, zuerst mit Individuen und dann mit Gruppen, um zu prüfen, ob seine Ideen, was die Bedeutung der Geschwindigkeit anbelangte, tatsächlich \emph{korrekt} waren und so Draco noch einmal deutlich machen, wie wichtig es war, allzeit nach Chancen Ausschau zu halten, die Methoden in der täglichen Praxis anzuwenden.

"\emph{Noch} haben Sie nicht gewonnen, General Potter!" knurrte Draco. "Vielleicht läuft uns die Zeit davon und Professor Quirrell entscheidet auf Unentschieden."

Ein guter und Besorgnis erregender Punkt. Der Krieg endete erst, wenn Professor Quirrell, nach seiner persönlichen Einschätzung, entschied dass eine der Armeen nach praktischen Maßstäben der realen Welt gewonnen hatte. Es gab keine \emph{formelle} Siegbedingung, denn, so hatte Professor Quirrell erklärt, ansonsten würde Harry herausbekommen, wie er die Regeln austricksen konnte. Da war was dran, musste Harry zugeben.

Und Harry konnte Professor Quirrell nicht vorwerfen, dass er noch kein Ende ausgerufen hatte, denn es war plausibel, dass der letzte Soldat der Drachen-Armee alle fünf Überlebenden der Chaos-Legion ausschalten könnte.

"Na gut," sagte Harry. "Weiß irgendjemand etwas über General Malfoys Schildzauber?"

Es stellte sich heraus, dass es sich bei Dracos Schild um eine Variante des gewöhnlichen \emph{Protego} handelte, die mehrere Nachteile hatte, der wichtigste davon, dass der Schild sich nicht mit demjenigen, der ihn gewirkt hatte, mit bewegen konnte.

Der Vorteil - oder aus Harrys Perspektive Nachteil - war, dass er einfacher zu lernen, einfacher zu wirken und viel einfacher für längere Zeit aufrecht zu erhalten war.

Sie würden den Schild mit Angriffszaubern bearbeiten müssen, um ihn in die Knie zu zwingen.

Und Draco hatte offenbar eine gewisses Maß an Kontrolle darüber, in welchem Reflektionswinkel die Zauber abprallen würden.

Harry kam der Gedanke, sie könnten Wingardium Leviosa verwenden, um schwere Steine auf den Schild zu türmen, bis Draco ihn gegen den Druck nicht mehr aufrecht erhalten konnte… doch dann könnten die Felsen hinterher hinein fallen und Draco treffen und den feindlichen General wirklich zu verletzen, stand heute nicht unter den Zielen auf der Tagesordnung.

"Also," sagte Harry. "Gibt es so etwas wie spezielle Zauber, um Schilde zu brechen?"

Gab es.

Harry fragte, ob irgendeiner seiner Soldaten sie kannte.

Keiner von ihnen.

Draco grinste erneut in seinem Schild.

Harry fragte, ob es irgendeinen Angriffszauber gab, der \emph{nicht} abprallen würde.

Blitzschläge, so schien es, wurden üblicherweise von Schilden absorbiert, anstatt von ihnen abzuprallen.

… Niemand wusste, wie man irgendeinen Zauber wirkte, der mit Blitzen zu tun hatte.

Draco kicherte.

Harry seufzte.

Mit Bedacht legte er seinen Zauberstab zu Boden.

Und Harry verkündete, mit einiger Erschöpfung in der Stimme, er würde den Schild einfach selbst zu Fall bringen müssen, mithilfe einer Methode, die mysteriös bleiben würde und alle anderen sollten auf Draco feuern, sobald der Schild fallen würde.

Die Chaos-Legionäre wirkten nervös.

Draco blieb ruhig, was bedeutete, er beherrschte sich.

Ein dünnes, gefaltetes Laken kam aus Harrys Beutel zum Vorschein.

Harry setzte sich neben den schimmernden Schild und zog sich das Laken über den Kopf, sodass niemand sehen konnte, was er tat - außer Draco natürlich.

Aus Harrys Beutel kamen eine Autobatterie und ein Satz Starthilfekabel.

… er hatte die Muggelwelt immerhin nicht verlassen, um eine neue Ära der magischen Forschung einzuleiten, ohne irgendeine Möglichkeit mitzunehmen, um Elektrizität zu erzeugen.

Kurz darauf hörten die Chaos-Legionäre von unter dem Laken das Geräusch schnipsender Finger, gefolgt von knisternden Lauten. Der Schild begann heller aufzuleuchten und Harrys Stimme sagte, "Lasst euch bitte nicht ablenken, Augen auf General Malfoy."

Die Anstrengung zeigte sich auf Dracos Gesicht, vermischt mit Wut, Verärgerung und Frustration.

Harry lächelte zu ihm hinauf und formte mit den Lippen, \emph{Erzähl's dir später.}

Und das war der Moment, als eine Spirale aus grüner Energie aus dem Wald geschossen kam und in Dracos Schild einschlug, der kreischte wie Stückchen von scharfem Glas, die gegeneinander gerieben wurden und Draco schwankte.

In plötzlicher Panik zog Harry hektisch die Starterkabel von der Batterie ab und verfütterte sie an seinen Beutel, danach folgte die Batterie selbst und dann riss er das Laken herunter, griff nach seinem Zauberstab und erhob sich.

Alle seine Soldaten waren noch da und suchten fieberhaft die Gegend ab.

"\emph{Contego,}" sagte Harry und seine Soldaten folgten dem Beispiel sofort, doch Harry wusste noch nicht einmal, in welche Richtung der Schild zeigen sollte. "Hat irgendwer gesehen, wo das herkam?" Köpfeschütteln. "Und General Malfoy, würde es Ihnen etwas ausmachen, mir zu sagen, ob \emph{Sie} General Granger erwischt haben?"

"Aber ja," sagte Draco ätzend, "es macht mir etwas aus."

\emph{Oh, Hölle.}

Harrys Geist begann zu rechnen, Draco in seinem Schild, Draco nun schon einigermaßen ausgelaugt, Harry ebenfalls erschöpft, Hermine wer-weiß-wo in den Wäldern, Harry und noch vier andere Chaoten übrig…

"Wissen Sie, General Granger," sagte Harry laut, "Sie hätten mit dem Angriff wirklich bis nach meinem Kampf mit General Malfoy warten sollen. Sie hätten vielleicht sogar \emph{alle} Überlebenden erwischen können."

Von irgendwoher erklang das helle Gelächter eines Mädchens.

Harry erstarrte.

\emph{Das war nicht Hermine.}

Und da erhob sich ein grauenvoller, gespenstischer, fröhlicher Gesang, ertönte überall um sie herum.

\emph{"Habt keine Angst, seid nicht verzagt,\\ Uns fürchtet nur, wer Böses wagt…"}

"\emph{Granger hat betrogen!}" platzte es aus Draco in seinem Schild heraus. "Sie hat ihre Soldaten aufgeweckt! Wieso lässt Professor Quirrell -"

"Lass mich raten," sagte Harry, in seinem Bauch rumorte es bereits. Er hasste es wirklich, zu verlieren. "Es war ein sehr leichter Kampf, nicht wahr? Sie fielen wie die Fliegen?"

"Ja," sagte Draco. "Wir haben sie alle mit der ersten Salve erwischt -"

Der Ausdruck entsetzter Erkenntnis sprang von Draco auf die Chaos-Legionäre über.

"Nein," sagte Harry, "haben wir nicht."

Gestalten in Tarnfarben tauchten zwischen den Bäumen auf.

"Verbündete?" sagte Harry.

"Verbündete," sagte Draco.

"Gut," sagte die Stimme von General Granger und eine Spirale aus grüner Energie erstrahlte aus den Wäldern und zersprengte Dracos Schild in Stücke.

--------------------------------------------------------------------------------------------------------------------------------------------

\emph{General Granger blickte über das Schlachtfeld mit dem deutlichen Gefühl der Genugtuung. Sie verfügte nur noch über neun Sunshine-Soldaten, doch das reichte wahrscheinlich aus, um sich um die letzten Überlebenden der feindlichen Streitkräfte zu kümmern, besonders da Parvati, Anthony und Ernie ihre Zauberstäbe bereits auf General Potter gerichtet hielten, den sie befohlen hatte, lebend zu fangen (nun, bei Bewusstsein).}

\emph{Sie wusste, das war Böse, doch sie hatte wirklich wirklich} \emph{\emph{wirklich}} \emph{ihre Schadenfreude genießen wollen.}

\emph{"Da steckt doch ein Trick dahinter, oder?" sagte Harry, die Anstrengung in seiner Stimme hörbar. "Es} \emph{\emph{muss}} \emph{irgendein Trick sein. Du kannst nicht auf einmal ein perfekter General sein. Nicht} \emph{auch noch} \emph{zu allem anderen.} \emph{In dir steckt nicht soviel von einem Syltherin! Du schreibst keine Schauergedichte!} \emph{\emph{Niemand kann in allem so gut sein!}"}

\emph{General Granger blickte sich unter ihren Sunshine-Soldaten um und wandte sich dann wieder Harry zu. Alle beobachteten das hier wahrscheinlich draußen auf den Schirmen.}

\emph{Und General Granger sagte, "Wenn ich nur genug lerne, kann ich alles schaffen."}

\emph{"Ach, komm schon, das ist doch Bu-"}

\emph{"\emph{Somnium.}"}

\emph{Harry sackte mitten im Satz zu Boden.}

\emph{"SUNSHINE GEWINNT," intonierte die gewaltige Stimme von Professor Quirrell, schien von überall und nirgends zu kommen.}

\emph{"Nettigkeit hat triumphiert!" schrie General Granger.}

\emph{"\emph{Hurra!}" riefen die Sunshine-Soldaten. Selbst die Jungen von Gryffindor stimmten} \emph{mit} \emph{ein und das mit Stolz.}

\emph{"Und wie lautet die Moral aus der heutigen Schlacht?" sagte General Granger.}

\emph{"\emph{Wenn wir nur genug lernen, können wir alles schaffen!}„}

\emph{Und die Überlebenden des Sunshine-Regiments marschierten davon zum Feld des Sieges und} \emph{ließen dabei} \emph{ihr Marschlied*******} \emph{erschallen:}

\emph{\emph{Habt keine Angst, seid nicht verzagt,\\ Uns fürchtet nur, wer Böses wagt,\\ Dem zeigen wir ein Heim geschwind,\\ Wo lauter neue Freunde sind,\\ Und fragt man dort, wer euch gesend't,\\ Sagt Grangers Sunshine-Regiment!}}

\emph{* Die} \emph{\emph{Drachen-Armee,}} \emph{im Film auch} \emph{\emph{Drachen-Trupp}} \emph{(engl.:} \emph{\emph{Dragon Army}) ist im Roman} \emph{\emph{Das Große Spiel}} \emph{(engl.:} \emph{\emph{Ender's Game}) von Orson Scott Card eine Gruppe an einer Militärakademie für} \emph{Kinder und Jugendliche, die bis zum Eintreffen des Hauptprotagonisten Ender Wiggin nur aus Außenseitern bestand und nicht eine einzige der dortigen Übungsschlachten gewinnen konnte und daher aufgelöst wurde. Unter Führung von Ender entscheidet der neu gebildete Drachen-Trupp dank seiner Intelligenz und überlegenen Taktiken fortan jede Schlacht für sich.}\\ \emph{** Möglicherweise eine Anspielung auf die} \emph{\emph{501. Legion}} \emph{(auch} \emph{\emph{501. Bataillon}) aus Star Wars, eine Elite-Einheit von Klonkriegern unter Führung des Jedi-Ritters Anakin Skywalker (des späteren Darth Vader) und später Vaders persönliche Leibgarde.} \emph{Weniger wahrscheinlich vielleicht auch auf eine Einheit aus den Romanen der Reihe} \emph{\emph{Falkenberg's Legion}} \emph{von Jerry Pournelle.}\\ \emph{*** Eine bessere Übersetzung hierfür fiel mir bis jetzt nicht ein; der englische Begriff} \emph{\emph{Squad Suggester}} \emph{(\emph{to suggest}} \emph{= vorschlagen/empfehlen) deutet zwar auf eine Führungsperson, deren Anweisungen in Übereinstimmung mit Harrys Taktik allerdings eher den Charakter allgemeiner Leitlinien anstelle von strikt zu befolgenden Befehlen aufweisen} \emph{(Ausnahme: „Merlin sagt“,} \emph{eine Anspielung auf das amerikanische Kinderspiel} \emph{\emph{Simon sagt}).}\\ \emph{****} \emph{engl.:} \emph{\emph{Defeated in detail;}} \emph{ich habe leider keine etablierte Übersetzung dieses Begriffs finden können, also muss es vorerst eine sinngemäße tun.}\\ \emph{***** Das hier rhythmisch wiederholte englische Wort} \emph{\emph{doom}} \emph{bedeutet Verdammnis, Unheil, Verderben oder Verhängnis und wird gesprochen mit einem} \emph{tiefen,} \emph{basslastigen\emph{U}} \emph{in der Mitte. Eine gute Übersetzung bietet sich leider nicht an, doch der beabsichtigte psychologische Effekt auf den Gegner dürfte klar werden, wenn man das Lied einmal selbst kurz anstimmt.}\\ \emph{****** Wer sich nicht mehr erinnert: Das ist der Bauchrede-Zauber, der in den Bonus-Akten von Kapitel 11 bereits kurz erwähnt wurde.}\\ \emph{******* Und für Interessierte auch hier} \emph{wieder} \emph{der Originaltext:}

\emph{Don't be frightened, don't be sad,\\ We'll only hurt you if you're bad,\\ And send you to a home that's true,\\ With new friends to watch over you,\\ Be sure to tell them you were sent\\ By Granger's Sunshine Regiment!}

