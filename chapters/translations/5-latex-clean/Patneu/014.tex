

\hypertarget{das-unverstandene-und-das-unverstuxe4ndliche}{% \section{14. Das Unverstandene und das Unverständliche}\label{das-unverstandene-und-das-unverstuxe4ndliche}}

\textbf{Kapitel 14: Das Unverstandene und das Unverständliche}\\

Melenkurion abatha! Duroc minas mill J. K. Rowling!

--------------------------------------------------------------------------------------------------------------------------------------------

\emph{Es gab mysteriöse Fragen, aber eine mysteriöse Antwort war ein Widerspruch in sich.}

--------------------------------------------------------------------------------------------------------------------------------------------

"Kommen Sie herein," sagte Professor McGonagalls gedämpfte Stimme.

Harry tat es.

Das Büro der Stellvertretenden Schulleiterin war sauber und aufgeräumt; an der Wand, direkt neben dem Schreibtisch, gab es ein Labyrinth kleiner Nischen und allen Formen und Größen, in die meisten waren mehrere Pergament-Rollen gesteckt worden und irgendwie war absolut klar, dass Professor McGonagall genau wusste, wofür jede einzelne Nische gut war, auch wenn niemand sonst das tat. Ein einziges Pergament lag tatsächlich auf dem Schreibtisch, welcher, abgesehen davon, aufgeräumt war. Hinter dem Schreibtisch war eine geschlossene Tür, versperrt von mehreren Schlössern.

Professor McGonagall saß auf einem Stuhl ohne Lehne hinter dem Schreibtisch und sah verwirrt aus -- ihre Augen geweitet, vielleicht mit einem kleinen Hauch von Besorgnis, als sie Harry sah.

"Mr. Potter?" sagte Professor McGonagall. "Worum geht es?"

Harrys Geist wurde leer. Er war von dem Spiel angewiesen worden, hierher zu kommen, er hatte erwartet, dass sie etwas im Sinn hätte…

"Mr. Potter?" sagte Professor McGonagall und begann, leicht verärgert auszusehen.

Glücklicherweise erinnerte Harry panisches Gehirn sich an diesem Punkt, dass er etwas \emph{hatte,} was er mit Professor McGonagall hatte besprechen wollen. Etwas wichtiges und ihre Zeit sicher wert.

"Ähm…" sagte Harry. "Wenn es irgendwelche Zauber gibt, die Sie wirken können, damit uns niemand zuhören kann…"

Professor McGonagall erhob sich aus ihrem Stuhl, schloss fest die äußere Tür und nahm ihren Zauberstab heraus und begann, Zaubersprüche aufzusagen.

An diesem Punkt erkannte Harry, dass er eine unbezahlbare und möglicherweise einmalige Gelegenheit bekam, Professor McGonagall einen Comed-Tee anzubieten und er konnte nicht glauben, dass er das ernsthaft dachte und es wäre schon in Ordnung, weil die Limo nach ein paar Sekunden wieder verschwinden würde und er sagte diesem Teil seiner selbst, er solle \emph{still sein.}

Er tat es und Harry begann im Geiste auszuarbeiten, was er sagen würde. Er hatte nicht geplant, diese Unterhaltung \emph{ganz} so bald zu führen, aber wenn er schon einmal hier war…

Professor McGonagall beendete einen Zauber, der sich sehr viel älter als Latein anhörte und setzte sich dann wieder hin.

"In Ordnung," sagte sie mit leiser Stimme. "Niemand hört zu." Ihr Gesicht war ziemlich angespannt.

\emph{Oh, richtig, sie erwartet, dass ich sie wegen Informationen über die Prophezeiung erpresse.}\\ Äh, darum würde Harry sich ein andermal kümmern.

"Es geht um den Vorfall mit dem Sprechenden Hut," sagte Harry. (Professor McGonagall blinzelte.) "Ähm… ich denke, es gibt einen Extra-Zauber auf dem Sprechenden Hut, etwas worüber der Sprechende Hut selbst nichts weiß, etwas was ausgelöst wird, wenn der Sprechende Hut Slytherin sagt. Ich habe eine Nachricht gehört, von der ich ziemlich sicher bin, dass Ravenclaws sie nicht hören sollten. Sie kam in dem Moment, als der Sprechende Hut von meinem Kopf herunterkam und ich fühlte, dass die Verbindung abbrach. Es klang wie ein Zischen und zugleich wie Englisch," von Professor McGonagall war ein scharfes Einatmen zu vernehmen, "und es sagte: Grüße von Slytherin zu Slytherin: willst du nach meinen Geheimnissen suchen, sprich mit meiner Schlange."

Professor McGonagall saß dort mit offenem Mund, starrte Harry an, als wären ihm soeben zwei Köpfe gewachsen.

"Also…" sagte Professor McGonagall langsam, als könnte sie die Worte aus ihrem eigenen Mund nicht glauben, "haben Sie entschieden, direkt zu mir zu kommen und mir davon zu berichten."

"Nun, ja, natürlich," sagte Harry. Es gab keinen Anlass, ihr zu erzählen, wie lange er tatsächlich gebraucht hatte, um daran zu denken. "Im Gegensatz dazu, sagen wir, selbst danach zu forschen oder es einem der anderen Kinder zu erzählen."

"Ich… verstehe," sagte Professor McGonagall. "Und wenn, möglicherweise, Sie den Eingang zu Salazar Slytherins legendärer Kammer des Schreckens entdecken würden, einen Eingang, den nur Sie und Sie allein öffnen könnten…"

"Ich würde den Eingang verschließen und Ihnen sofort Bericht erstatten, damit ein Team erfahrener magischer Archäologen zusammengestellt werden könnte," sagte Harry sofort. "Dann würde ich den Eingang wieder öffnen und sie würden sehr vorsichtig hineingehen, um sicher zu gehen, dass es dort nichts gefährliches gäbe. Ich ginge vielleicht später hinein, um mich umzusehen oder wenn sie mich bräuchten, um noch etwas anderes zu öffnen, aber erst nachdem der Bereich für gesichert erklärt wurde und sie Fotos davon gemacht hätten, wie alles aussieht, bevor jemand anfängt, in ihrer unbezahlbaren historischen Stätte herumzutrampeln."

Professor McGonagall saß dort mit offenem Mund und starrte ihn an, als habe er sich gerade in eine Katze verwandelt.

"Es ist offensichtlich, wenn man kein Gryffindor ist," sagte Harry freundlich.

"Ich denke," sagte Professor McGonagall mit ziemlich erstickter Stimme, "dass Sie die Seltenheit von gesundem Menschenverstand \emph{weit} unterschätzen, Mr. Potter."

Das klang ziemlich richtig. Obwohl… "Ein Hufflepuff hätte das selbe gesagt."

McGonagall stoppte, wie vor den Kopf gestoßen. "\emph{Das} ist wahr."

"Der Sprechende Hut hat mir Hufflepuff angeboten."

Sie blinzelte in an, als könne sie ihren eigenen Ohren nicht trauen. "Hat er \emph{wirklich?}"

"Ja."

"Mr. Potter," sagte McGonagall, ihre Stimme jetzt gesenkt, "vor fünf Jahrzehnten ist das letzte mal ein Schüler in den Mauern von Hogwarts zu Tode gekommen und ich bin nun ziemlich sicher, dass vor fünf Jahrzehnten zum letzten mal jemand diese Nachricht hörte."

Ein Schauer durchlief Harry. "Dann bin ich mir \emph{äußerst} sicher, dass ich aber auch \emph{gar nichts} in dieser Angelegenheit unternehmen werde, ohne Sie vorher zu Rate zu ziehen, Professor McGonagall." Er hielt inne. "Und dürfte ich vorschlagen, dass Sie die besten Leute zusammen trommeln, die Sie finden können und versuchen herauszufinden, ob es möglich ist, diesen Extra-Zauber vom Sprechenden Hut zu entfernen… und wenn Sie das nicht tun können, legen Sie vielleicht einen \emph{weiteren} Zauber darauf, einen Quietus, der sich kurz aktiviert, genau für den Moment, wenn der Hut vom Kopf eines Schülers entfernt wird, das könnte als Notlösung funktionieren. Und schon, keine toten Schüler mehr." Harry nickte zufrieden.

Professor McGonagall sah sogar noch verblüffter aus, als wäre so etwas unvorstellbar. "Ich kann Ihnen \emph{unmöglich} genug Hauspunkte dafür verleihen, ohne Ravenclaw direkt den Hauspokal zu überreichen."

"Ähm," sagte Harry. "Ähm. Ich würde eher nicht \emph{so} viele Hauspunkte bekommen wollen."

Jetzt warf Professor McGonagall ihm einen seltsamen Blick zu. "Warum nicht?"

Harry hatte ein paar Schwierigkeiten, es in Worte zu fassen. "Weil es einfach zu traurig wäre, verstehen Sie? Wie… wie damals, als ich noch versuchte, in der Muggelwelt zur Schule zu gehen und wann immer es eine Gruppenarbeit gab, würde ich alles allein machen, weil die anderen mich nur bremsen würden. Ich habe nichts dagegen, viele Punkte zu bekommen, sogar mehr als alle anderen, aber wenn ich genug bekomme, um den Gewinn des Hauspokals ganz allein zu entscheiden, dann ist das, als trage ich das ganze Haus Ravenclaw auf meinem Rücken und das ist zu traurig."

"Ich verstehe…" sagte McGonagall zögerlich. Es war offensichtlich, dass es ihr nie in den Sinn gekommen war, so zu denken. "Nehmen Sie an, ich würde ihnen dann nur fünfzig Punkte geben?"

Harry schüttelte erneut den Kopf. "Es ist nicht fair, wenn ich viele Punkte für Erwachsenen-Sachen bekomme, an denen ich teilnehmen kann und die anderen nicht. Wie sollte Terry Boot fünfzig Punkte für das Berichten eines Flüsterns, das er vom Sprechenden Hut gehört hat, verdienen? Es wäre einfach nicht fair."

"Ich verstehe, warum der Sprechende Hut Ihnen Hufflepuff angeboten hat," sagte Professor McGonagall. Sie beäugte ihn mit seltsamem Respekt.

Das schnürte Harry ein wenig die Kehle zu. Er hatte ehrlich gedacht, er Huffelpuffs nicht würdig. Dass der Sprechende Hut nur versucht hatte, ihn nach irgendwo, nur nicht Ravenclaw, zu schieben, in ein Haus, dessen Tugenden er nicht besaß…

Professor McGonagall lächelte jetzt. "Und wenn ich versuchte, Ihnen \emph{zehn} Punkte zu geben…?"

"Werden Sie erklären, woher diese zehn Punkte kamen, wenn irgendjemand fragt? Es könnte eine Menge Slytherins geben und ich meine nicht die Kinder in Hogwarts, die wirklich \emph{wirklich} zornig wären, wenn sie davon wüssten, dass der Zauber vom Sprechenden Hut entfernt wurde und herausfänden, dass ich damit zu tun hatte. Deshalb denke ich, absolute Verschiegenheit wäre hier besser als Mut. Kein Grund mir zu danken, Ma'am, die Tugend ist sich Lohn genug."

"So sei es," sagte Professor McGonagall, "aber ich habe etwas anderes sehr besonderes, das ich Ihnen geben wollte. Ich sehe, dass ich mich sehr in Ihnen getäuscht habe, Mr. Potter. Bitte warten Sie hier."

Sie stand auf, ging zu der verschlossenen Hintertür, schwang ihren Zauberstab und eine Art verschwommener Schleier entstand um sie herum. Harry konnte weder sehen noch hören, was vor sich ging. Wenige Minuten später verschwand der Schleier und Professor McGonagall stand dort, ihm zugewandt, die Tür hinter ihr aussehend, als sei sie nie geöffnet worden.

Und Professor McGonagall hielt in einer Hand eine Halskette, eine dünne goldene Kette, die in der Mitte einen silbernen Ring barg, darin ein Stundenglas. In ihrer anderen Hand lag ein gefaltetes Papier. "Das ist für Sie," sagte sie.

Wow! Er bekam eine Art magischen Gegenstand als eine Quest-Belohnung! Offenbar funktionierte die Sache mit dem Ablehnen finanzieller Belohnungen, bis man einen magischen Gegenstand bekam, im wahren Leben, nicht nur in Computer-Spielen.

Harry nahm seine neue Halskette lächelnd an. "Was ist es?"

Professor McGonagall atmete ein. "Mr. Potter, dies ist ein Gegenstand, der normalerweise nur Kindern überlassen wird, die sich bereits als höchst verantwortungsbewusst erwiesen haben, um ihnen bei schwierigen Stundenplänen zu helfen." McGonagall zögerte, als wolle sie noch etwas anderes hinzufügen. "Ich \emph{muss} betonen, Mr. Potter, dass die wahre Natur dieses Gegenstandes ein \emph{Geheimnis} ist und dass Sie keinem der anderen Schüler davon erzählen oder sie sehen lassen dürfen, wie Sie es benutzen. Wenn das für Sie nicht akzeptabel ist, können Sie es jetzt zurückgeben."

"Ich kann Geheimnisse bewahren," sagte Harry. "Also was tut es?"

"Soweit es die anderen Schüler betrifft, ist es ein Spimster Wicket* und wird verwendet, um eine seltene, nicht ansteckende magische Krankheit namens Spontane Duplizierung zu behandeln. Sie tragen es unter Ihrer Kleidung und da Sie keinen Grund haben, es jemandem zu zeigen, haben Sie auch keinen Grund, es wie ein schreckliches Geheimnis zu behandeln. Spimster Wickets sind nicht interessant. Verstanden, Mr. Potter?"

Harry nickte, sein Lächeln wurde breiter. Er erkannte die Arbeit eines \emph{kompetenten} Slytherins. "Und was tut es \emph{wirklich?}"

"Es ist ein Zeitumkehrer. Jede Umdrehung des Stundenglases schickt Sie eine Stunde in der Zeit zurück. Wenn Sie es also benutzen, um jeden Tag zwei Stunden zurück zu gehen, sollten Sie immer zu selben Zeit zu Bett gehen können."

Harrys Beschluss, seine Ungläubigkeit bezüglich der magischen Welt vorübergehend zu verdrängen, war wie weggepustet.

\emph{Sie geben mir eine Zeitmaschine, um meine Schlafstörung zu behandeln.}

\emph{Sie geben mir eine ZEITMASCHINE um meine SCHLAFSTÖRUNG zu behandeln.}

\emph{SIE} \emph{\textbf{\emph{GEBEN MIR EINE ZEITMASCHINE}}} \emph{UM MEINE} \emph{\textbf{\emph{SCHLAFSTÖRUNG ZU BEHANDELN}}.}

"Ehehehehhheheh…" sagte Harrys Mund. Er hielt die Halskette jetzt von sich weg, als wäre sie eine scharfe Bombe. Nun, nein, nicht als wäre sie eine scharfe Bombe, das beschrieb nicht einmal \emph{annähernd} die Schwere dieser Situation. Harry hielt die Halskette von sich weg, als wäre sie eine Zeitmaschine.

\emph{Sagen Sie, Professor McGonagall, wussten Sie, dass zeit-umgekehrte gewöhnliche Materie sich verhält wie Antimaterie? Denn das tut sie! Wussten Sie, dass ein Kilogramm Antimaterie, die auf ein Kilogramm Materie trifft, in einer Explosion ausgelöscht wird, deren Sprengkraft 43 Millionen Tonnen TNT entspricht? Ist Ihnen klar, dass ich selbst 41 Kilogramm wiege und die sich daraus ergebende Detonation einen GIGANTISCHEN RAUCHENDEN KRATER HINTERLASSEN WÜRDE, WO EINMAL SCHOTTLAND WAR?}

"Entschuldigen Sie," schaffte Harry zu sagen, "aber das klingt wirklich wirklich \emph{wirklich WIRKLICH GEFÄHRLICH!}" Harrys Stimme steigerte sich nicht ganz zu einem Kreischen, er konnte nicht laut genug schreien, um dieser Situation gerecht zu werden, also war es nutzlos, es zu versuchen.

Professor McGonagall betrachtete ihn mit nachsichtigem Wohlwollen. "Ich bin froh, dass Sie das ernst nehmen, Mr. Potter, aber Zeitumkehrer sind nicht \emph{so} gefährlich. Wir würden sie keinen Kindern geben, wenn sie es wären."

"Wirklich," sagte Harry. "Ahahahaha. Natürlich würden Sie Kindern keine Zeitmaschinen geben, wenn sie gefährlich wären, was \emph{habe} ich mir nur gedacht? Also, nur um sicher zu gehen, auf dieses Gerät zu niesen, wird mich \emph{nicht} in's Mittelalter zurückschicken, wo ich Gutenberg mit einem Pferdekarren überfahre und die Aufklärung verhindere? Weil, wissen Sie, ich hasse es, wenn mir das passiert."

McGonagalls Lippen zuckten auf die Art, wie sie es taten, wenn Sie versuchte, nicht zu lächeln. Sie hielt Harry das Papier hin, das sie hielt, aber Harry hielt mit beiden Händen vorsichtig die Halskette von sich weg und starrte das Stundenglas an, um sicher zu sein, dass es sich nicht drehte. "Seien Sie unbesorgt," sagte Professor McGonagall nach einer kurzen Pause, als klar wurde, dass Harry sich nicht bewegen würde, "das kann nicht passieren, Mr. Potter. Der Zeitumkehrer kann nicht verwendet werden, um mehr als sechs Stunden zurückzureisen. Er kann nicht öfter als sechs mal an jedem Tag genutzt werden."

"Oh, gut, sehr gut, das. Und wenn mich jemand anrempelt, wird der Zeitumkehrer \emph{nicht} zerbrechen und es wird \emph{nicht} das ganze Schloss Hogwarts in einer endlosen Schleife von Donnerstagen gefangen sein."

"Nun, sie \emph{können} zerbrechlich sein…" sagte Professor McGonagall. "Und ich denke, ich habe von seltsamen Dingen gehört, die passieren, wenn sie zerbrechen. Aber nichts wie \emph{das!}"

"Vielleicht," sagte Harry, als er wieder sprechen konnte, "sollten Sie Ihre Zeitmaschinen mit einer Art \emph{Schutzhülle} ausstatten, anstatt \emph{das Glas freiliegen zu lassen,} um \emph{sowas zu verhindern.}"

McGonagall sah aus wie vor den Kopf gestoßen. "Das ist eine ausgezeichnete Idee, Mr. Potter. Ich werde das Ministerium darüber informieren."

\emph{Das war's, jetzt ist es offiziell, sie haben es im Parlament verkündet, jeder einzelne in der Zauberwelt ist vollkommen dämlich.}

"Und obwohl ich nur ungern, ganz \emph{PHILOSOPHISCH werde,}" versuchte Harry seine Stimme verzweifelt zu etwas unter einem einem Kreischen herab zu senken, "hat irgendjemand über die \emph{AUSWIRKUNGEN}nachgedacht, die es hätte, sechs Stunden zurück zu gehen und etwas zu tun, das die Zeit verändert, was so ziemlich \emph{ALLE BETROFFENEN MENSCHEN AUSLÖSCHEN} würde und sie \emph{DURCH ANDERE VERSIONEN ERSETZEN} -"

"Oh, man kann die Zeit nicht \emph{verändern!}" unterbrach Professor McGonagall. "Gütiger Himmel, Mr. Potter, denken Sie, diese Gegenstände würden Schülern anvertraut, wenn \emph{das} möglich wäre? Was wenn jemand versuchte, seine Noten zu verändern?"

Harry brauchte einen Moment, um das zu verarbeiten. Seine Hände, die die Stundenglas-Kette mit weiß hervorgetretenen Knöcheln umklammerten, entspannten sich, nur ein wenig. Als hielte er keine Zeitmaschine, nur einen scharfen nuklearen Sprengkopf.

"Also…" sagte Harry langsam. "Scheint es, dass das Universum… irgendwie in sich konsistent ist, obwohl es darin Zeitreisen gibt. Wenn ich und mein zukünftiges Selbst interagieren, werde ich das selbe sehen, wie meine beiden Ichs, obwohl bei meinem ersten Durchgang mein zukünftiges Ich bereits im vollen Bewusstsein von Dingen handelt, die, aus meiner eigenen Perspektive, noch nicht passiert sind…" Harrys Stimme versagte, aufgrund der Unzulänglichkeit des Englischen.

"Korrekt, denke ich," sagte Professor McGonagall. "Obwohl Zauberern geraten wird, zu vermeiden, von ihren vergangenen Selbsten gesehen zu werden. Wenn man zwei Klassen zur selben Zeit besucht und seinen eigenen Weg kreuzt, sollte zum Beispiel die erste Version von einem zu einem bestimmten Zeitpunkt zur Seite treten und die Augen schließen -- Sie haben bereits eine Uhr, gut -- so dass das zukünftige Selbst vorbei kann. Es steht alles in dem Papier."

"Ahahahaa. Und was passiert, wenn jemand diesen Rat \emph{ignoriert?}"

Professor McGonagall schürzte die Lippen. "Ich verstehe, dass es ziemlich befremdlich sein kann."

"Und es erschafft nicht, sagen wir, ein Paradox, das das Universum zerstört."

Sie lächelte nachsichtig. "Mr. Potter, ich denke, ich würde mich erinnern, davon gehört zu haben, wenn \emph{das} jemals passiert wäre."

"\emph{DAS IST NICHT BERUHIGEND! HABT IHR LEUTE NOCH NIE ETWAS VOM ANTHROPISCHEN PRINZIP} \emph{GEHÖRT? UND WELCHER IDIOT HAT ÜBERHAUPT ERST EINS DIESER DINGER GEBAUT?}"

Professor McGonagall lachte tatsächlich. Es war ein angenehmes, fröhliches Geräusch, dass überraschend deplatziert in diesem strengen Gesicht wirkte. "Sie haben einen weiteren 'Sie haben sich in eine Katze verwandelt'-Moment, nicht wahr, Mr. Potter. Das wollen Sie wahrscheinlich nicht hören, aber es ist auf wirklich reizende Weise süß."

"Sich in eine Katze zu verwandeln, ist nicht einmal \emph{ANSATZWEISE} vergleichbar damit. Wissen Sie, bis genau zu diesem Moment hatte ich den schrecklichen, unterdrückten Gedanken irgendwo in meinem Hinterkopf, die einzige verbleibende Erklärung wäre, dass mein ganzes Universum eine Computer-Simulation wie in dem Buch \emph{Simulacron-3}**, aber jetzt ist \emph{sogar das ausgeschlossen,} weil dieses kleine Spielzeug \emph{NICHT TURING-BERECHENBAR IST!} Eine Turingmaschine könnte ein Zurückgehen zu einem definierten Moment in der Vergangenheit simulieren und von dort aus eine andere Zukunft berechnen, eine Orakel-Maschine könnte sich auf das Halte-Verhalten von Maschinen niedrigerer Ordnung stützen, aber was Sie sagen, ist dass Realität sich irgendwie in einem Rutsch in sich selbst konsistent berechnet, mit Hilfe von Informationen, die… noch nicht… passiert sind…"***

Die Erkenntnis traf Harry wie ein Blitzschlag.

Es ergab jetzt alles Sinn. Es ergab \emph{endlich} alles Sinn.

"\emph{ALSO SO FUNKTIONIERT DER COMED-TEE! Natürlich!} Der Zauber \emph{erzwingt} nicht, dass lustige Sachen passieren, er sorgt lässt einen nur \emph{den Drang verspüren, ihn zu trinken,} bevor so wie so lustige Sachen passieren!Ich bin so ein Dummkopf, ich hätte es erkennen sollen, als ich vor Dumbledores zweiter Rede den Drang zum Trinken verspürt, ihn \emph{nicht} getrunken und mich stattdessen an meiner eigenen Spucke verschluckt habe -- den Comed-Tee zu trinken, verursacht nicht die Comedy, die Comedy bringt einen dazu, Comed-Tee zu trinken! Ich sah die Korrelation zwischen beiden Ereignissen und nahm an, der Comed-Tee sei die Ursache und die Comedy die Wirkung, weil ich dachte, die Kausalität würde durch die zeitliche Abfolge beschränkt und Ursache-Wirkungs-Diagramme dürften nicht zyklisch sein, ABER ES MACHT ALLES SINN, SOBALD MAN DIE KAUSAL-PFEILE \emph{IN DER ZEIT ZURÜCK} GEHEND ZEICHNET!"

Die Erkenntnis traf Harry wie ein \emph{zweiter} Blitzschlag.

Dieses mal schaffte er es, still zu bleiben, gab nur ein kleines ersticktes Geräusch von sich, wie ein sterbendes Kätzchen, als er erkannte, wer an diesem Morgen die Notiz auf sein Bett gelegt hatte.

Professor McGonagalls Augen leuchteten. "Nachdem Sie Ihren Abschluss machen oder vielleicht sogar vorher \emph{müssen} Sie wirklich einige dieser Muggeltheorien in Hogwarts lehren, Mr. Potter. Sie klingen wirklich faszinierend, auch wenn sie alle falsch sind."

"Gnahhahhh…"

Professor McGonagall ließ ihm noch ein paar mehr Freundlichkeiten zuteil werden, verlangte noch einige weitere Versprechen, die Harry abnickte, sagte etwas darüber, nicht mit Schlangen zu sprechen, wenn ihn jemand hören könnte und dann fand Harry sich irgendwie vor ihrem Büro wieder, die Tür hinter ihm fest geschlossen.

"Gaahhhrrrraa…" sagte Harry.

Nun ja, das \emph{hatte} ihn umgehauen.

Nicht zuletzt der Umstand, dass wenn der Streich nicht gewesen wäre, er vielleicht niemals überhaupt einen Zeitumkehrer bekommen hätte.

Oder hätte Professor McGonagall ihm den ohnehin gegeben, nur später am Tag, wann immer er sie aufgesucht hätte, um nach seiner Schlafstörung zu fragen oder ihr von der Nachricht des Sprechenden Hutes zu erzählen? Und hätte er sich an diesem Punkt einen Streich spielen wollen, der dazu führte, dass er den Zeitumkehrer \emph{früher} bekam? So dass die einzige \emph{in sich konsistente} Möglichkeit die war, bei welcher der Streich begann, bevor er am Morgen überhaupt aufwachte…?

Harry zog zum ersten mal in seinem Leben in Betracht, dass die Antwort auf seine Frage buchstäblich \emph{undenkbar} war. Dass, weil sein Gehirn nur aus Neuronen bestand, die sich vorwärts in der Zeit bewegten, es nichts gab, was sein Gehirn tun könnte, keine Vorgang den es ausführen könnte, der die Funktionsweise eines Zeitumkehrers nachvollziehen konnte.

Bis zu diesem Moment hatte Harry nach der Mahnung von E. T. Jaynes gelebt, dass wenn man ein Phänomen nicht verstand, dies nur etwas über den eigenen Geisteszustand aussagte, nicht über das Phänomen selbst; dass die eigene Ungewissheit einen nur selbst betraf, nicht das worüber man im Ungewissen war; dass Ungewissheit nur im eigenen Geist existierte, nicht in der Realität; dass eine leere Karte keiner leeren Landschaft entsprach. Es gab mysteriöse Fragen, aber eine mysteriöse Antwort war ein Widerspruch in sich. Ein Phänomen konnte \emph{für} eine bestimmte Person mysteriös sein, aber es konnte keine an sich mysteriösen Phänomene geben. Ein heiliges Mysterium zu verehren, hieß nur, seine eigene Unwissenheit zu verehren.

So hatte Harry die Magie betrachtet und sich geweigert, sich davon einschüchtern zu lassen. Die Menschen hatten keinen Sinn für Geschichte, sie lernten etwas über Chemie, Biologie und Astronomie und dachten, diese Dinge wären schon immer das Suppenfleisch der Wissenschaft gewesen, dass sie \emph{niemals} mysteriös gewesen waren. Die Sterne waren einmal ein Mysterium gewesen. Lord Kelvin hatte einst die Natur des Lebens und der Biologie -- die Kontrolle von Muskeln durch menschlichen Willen und das Erwachsen von Bäumen aus Samen -- als ein Mysterium "unendlich außerhalb" des Reichs der Wissenschaft. (Nicht nur ein wenig außerhalb, wohl gemerkt, sondern \emph{unendlich} außerhalb. Lord Kelvin hatte es sicherlich als enorme emotionale Belastung empfunden, \emph{etwas nicht zu wissen.}) Jedes Rätsel, das je gelöst wurde, war ein Rätsel gewesen von Anbeginn der menschlichen Spezies, bis zu dem Tag, an dem es jemand löste.

Jetzt war er zum ersten mal mit der Aussicht konfrontiert, dass ein Phänomen drohte, \emph{dauerhaft} zu sein. Wenn die Zeit nicht mit nicht-zyklischen Kausal-Beziehungen funktionierte, dann begriff Harry nicht, was mit Ursache und Wirkung gemeint war und wenn Harry keine Ursachen und Wirkungen erkannte, verstand er nicht, woraus die Realität sonst bestand und es war vollkommen denkbar, dass sein menschlicher Geist niemals verstehen \emph{konnte,} weil sein Gehirn aus \emph{altmodischen Einbahnstraßen-Zeit-Neuronen} bestand, was sich nur als armselig kleiner Teilbereich der Realität herausgestellt hatte.

Auf der Haben-Seite stand, dass sich herausstellte, dass es für den Comed-Tee, der vorher allmächtig und völlig unglaublich gewirkt hatte, eine viel einfachere Erklärung gab. Die er \emph{nur deshalb} übersehen hatte, weil die Wahrheit sich völlig außerhalb seines Hypothesenraumes und von allem befunden hatte, was sein Gehirn aufgrund seiner Evolution zu verstehen in der Lage war. Aber jetzt \emph{hatte} er es, wahrscheinlich, tatsächlich herausbekommen. Was irgendwie ermutigend war. Irgendwie.

Harry sah auf seine Armbanduhr hinab. Es war fast 11 Uhr, er war letzte Nacht um 1 Uhr morgens eingeschlafen, also würde er normalerweise heute Nacht um 3 Uhr morgens schlafen gehen. Um also um 10 Uhr abends einzuschlafen und um 7 Uhr morgens aufzuwachen, sollte er insgesamt fünf Stunden zurückgehen. Was bedeutete, wenn er um 6 Uhr morgens herum zurück in seinem Schlafsaal sein wollte, bevor irgendjemand wach war, sollte er sich besser beeilen und…

Selbst im \emph{Rückblick} verstand Harry nicht, wie er auch nur die \emph{Hälfte} der Sachen in dem Streich durchgezogen hatte. Wo war der \emph{Kuchen} hergekommen?

Harry begann, sich ernsthaft vor Zeitreisen zu fürchten.

Andererseits musste er zugeben, dass es \emph{in der Tat} eine einmalige Gelegenheit gewesen war. Ein Streich, den man sich selbst nur einmal im Leben spielen konnte, innerhalb von sechs Stunden nachdem man zum ersten mal von Zeitumkehrern gehört hatte.

Tatsächlich war das sogar \emph{noch} verwirrender, als Harry darüber nachdachte. Die Zeit hatte ihn mit dem fertigen Streich vor \emph{vollendete Tatsachen} gestellt und trotzdem war es ziemlich offensichtlich sein eigenes Werk. Konzept und Ausführung und Schreibstil. Jede Kleinigkeit, sogar die, die er nicht verstand.

Nun, er verschwendete Zeit und der Tag hatte höchstens dreißig Stunden. Harry wusste \emph{einiges} von dem, was er tun musste und der Rest, wie etwa der Kuchen, fiel ihm vielleicht während dessen ein. Es hatte keinen Zweck, es aufzuschieben. Er konnte genau gar nichts tun, während er hier in der \emph{Zukunft} feststeckte.

--------------------------------------------------------------------------------------------------------------------------------------------

Fünf Stunden früher schlich Harry sich in seinen Schlafsaal, mit seinem Umhang als leichte Verkleidung über den Kopf gezogen, nur für den Fall, dass jemand bereits wach war und ihn zur selben Zeit mit dem noch in seinem Bett liegenden Harry sah. Er wollte niemandem sein kleines medizinisches Problem mit Spontaner Duplizierung erklären müssen.

Glücklicherweise schienen alle noch zu schlafen.

Und dort schien auch ein Päckchen, in rotes und grünes Papier gewickelt, mit einer hell-goldenen Schleife, neben seinem Bett zu liegen. Das perfekte klischeehafte Abbild eines Weihnachtsgeschenks obwohl es nicht Weihnachten war.

Harry schlich so sachte wie er konnte hinein, nur für den Fall, dass jemand seinen Quieter herunter geregelt hatte.

Es war ein Umschlag an dem Päckchen, mit einfachem glattem Wax verschlossen, ohne eingeprägtes Siegel.

Harry brach den Umschlag vorsichtig auf und nahm den Brief darin heraus.

Der Brief besagte:

\emph{Dies ist der Unsichtbarkeitsumhang von Ignotus Peverell, von seinen Nachfahren, den Potters, weitergereicht. Anders als andere Umhänge und Zauber, hat er die Macht, jemanden} \emph{\uline{verborgen}} \emph{zu halten, nicht bloß unsichtbar. Dein Vater hat ihn mir kurz vor seinem Tode geliehen, um ihn zu untersuchen und ich gebe zu, dass er mir über die Jahre gute Dienste geleistet hat.}

\emph{Zukünftig werde ich mit Desillusionierung zurecht kommen müssen, fürchte ich. Es ist Zeit, dass der Umhang dir, seinem Erben, zurück gegeben wird. Ich dachte daran, ihn zu einem Weihnachtsgeschenk zu machen, aber ich wollte ihn bereits vorher zurück in deiner Hand wissen. Er scheint zu erwarten, dass du ihn brauchen wirst. Gebrauche ihn klug.}

\emph{Ohne Zweifel denkst du bereits über alle möglichen wundervollen Streiche nach, wie sie dein Vater seinerzeit begangen hat. Wenn das volle Ausmaß seiner Missetaten bekannt würde, würden alle Frauen von Gryffindor sich versammeln, um sein Grab zu schänden. Ich werde nicht versuchen, zu verhindern, dass die Geschichte sich wiederholt, aber lasse GRÖßTE Vorischt walten, dich nicht zu verraten. Sähe Dumbledore die Chance, eines der Heiligtümer des Todes zu besitzen, entließe er es niemals aus seinem Griff, bis zu seinem Tode.}

\emph{Ich wünsche dir Frohe Weihnachten.}

Die Nachricht war nicht unterschrieben.

--------------------------------------------------------------------------------------------------------------------------------------------

"Einen Moment," sagte Harry kurz angebunden, als die anderen Jungen dabei waren, den Ravenclaw-Schlafsaal zu verlassen. "Entschuldigt, ich muss noch etwas mit meinem Koffer erledigen. Ich komme in ein paar Minuten zum Frühstück nach."

Terry Boot blickte Harry finster an. "Du hast besser nicht vor, unsere Sachen zu durchstöbern."

Harry erhob eine Hand. "Ich schwöre, dass ich nicht beabsichtige, irgendwas dergleichen mit irgendeiner eurer Sachen zu tun, dass ich beabsichtige, nur Dinge anzurühren, die mir selbst gehören, dass ich keine Streiche oder anderweitige fragwürdige Absichten gegen irgendeinen von euch im Sinn habe und dass ich nicht erwarte, dass sich diese Absichten ändern, bevor ich zum Frühstück in die Große Halle komme."

Terry runzelte die Stirn. "Warte, ist das -"

"Keine Sorge," sagte Penelope Clearwater, die hier war, um sie zu führen. "Da waren keine Schlupflöcher. Gut formuliert, Potter, du solltest Anwalt werden."

Daraufhin blinzelte Harry Potter. Ah, ja, Ravenclaw-\emph{Vertrauensschülerin.} "Danke," sagte er. "Denke ich."

"Wenn du versuchst die Große Halle zu finden, wirst du dich verlaufen." stellte Penelope im Ton einer simplen, unbestreitbaren Tatsache fest. "Sobald das passiert, frag ein Porträt, wie du in den ersten Stock gelangst. Frag ein anderes Porträt, \emph{sofort} wenn du glaubst, du könntest dich wieder verlaufen haben. \emph{Besonders} wenn es scheint, als würdest du immer höher und höher steigen. Wenn du höher bist, als das ganze Schloss sein sollte, \emph{halte an} und warte auf einen Suchtrupp. Ansonsten sehen wir dich vier Monate später wieder und du wirst fünf Monate älter und mit einem Lendenschurz bekleidet und mit Schnee bedeckt sein und \emph{das auch nur, wenn du im Schloss bleibst.}"

"Verstanden," sagte Harry und schluckte schwer. "Ähm, solltet ihr Schülern das alles nicht gleich erzählen?"

Penelope seufzte. "Was, \emph{alles?} Das würde Wochen dauern. Du lernst es unterwegs." Sie drehte sich zum Gehen um, gefolgt von den anderen Schülern. "Wenn ich dich nicht in dreißig Minuten beim Frühstück sehe, Potter, fangen wir mit der Suche an."

Sobald alle weg waren, befestigte Harry die Notiz an seinem Bett -- er hatte sie und alle anderen Notizen bereits geschrieben, als er in der Keller-Etage seines Koffers gearbeitet hatte, bevor alle anderen aufwachten. Dann griff er vorsichtig in das Quietus-Feld und zog den Unsichtbarkeitsumhang von der schlafenden Form von Harry-1.

Und nur um des Unfugs willen, legte Harry den Umhang in den Beutel von Harry-1, in dem Wissen, dass er auf diese Weise bereits in seinem eigenen wäre.

--------------------------------------------------------------------------------------------------------------------------------------------

"Ich kann dafür sorgen, dass die Nachricht an Cornelion Flubberwalt weitergegeben wird," sagte das Gemälde eines Mannes mit aristokratischer Frisur und einer, eigentlich, völlig normalen Nase. "Aber darf ich fragen, von wem sie \emph{ursprünglich} kam?"

Harry zuckte in kunstvoller Hilflosigkeit mit den Schultern. "Mir wurde gesagt, sie wurde von einer dumpfen Stimme gesprochen, die aus einer Leere in der Luft selbst erklang, einer Kluft, die sich öffnete über einem feurigen Abgrund."

--------------------------------------------------------------------------------------------------------------------------------------------

"Hey!" sagte Hermine in empörtem Tonfall, von ihrem Platz am anderen Ende des Frühstückstisches. "Der Nachtisch ist für \emph{alle} da! Du kannst nicht einfach einen ganzen Kuchen nehmen und ihn in deinen Beutel stopfen!"

"Ich nehme nicht einen Kuchen, ich nehme zwei. Entschuldigt, ihr alle, ich muss eilen!" Harry ignorierte die empörten Aufschreie und verließ die Große Halle. Er musste etwas früher im Kräuterkunde-Unterricht eintreffen.

--------------------------------------------------------------------------------------------------------------------------------------------

Professor Sprout sah ihn scharf an. "Und woher wissen \emph{Sie}, was die Slytherins planen?"

"Ich kann meine Quelle nicht nennen," sagte Harry. "Tatsächlich muss ich Sie bitten, so zu tun, als habe diese Unterhaltung nie stattgefunden. Tun Sie einfach so, als wären Sie zufällig über sie gestolpert, als Sie eine Besorgung machten oder so. Ich werde voraus laufen, sobald der Kräuterkunde-Unterricht endet. Ich denke, ich kann die Slytherins ablenken, bis Sie eintreffen. Ich bin nicht leicht zu erschrecken oder zu mobben und denke, sie werden es nicht wagen, dem Jungen-der-überlebt-hat ernsthaft weh zu tun. Obwohl… ich verlange nicht, dass Sie in den Fluren rennen, aber ich wüsste es zu schätzen, wenn sie unterwegs nicht trödelten."

Professor Sprout betrachtete ihn lange, dann wurden ihre Züge weicher. "Bitte geben Sie auf sich Acht, Harry Potter. Und… danke."

"Seien Sie nur nicht zu spät," sagte Harry. "Und denken Sie daran, wenn Sie dort ankommen, Sie haben nicht erwartet, mich zu sehen und diese Unterhaltung hat nie stattgefunden."

--------------------------------------------------------------------------------------------------------------------------------------------

Es war schrecklich, sich selbst dabei zu beobachten, wie er Neville aus dem Kreis der Slytherins zerrte. Neville hatte recht gehabt, es war zu kräftig gewesen, viel zu kräftig.

"Hallo," sagte Harry Potter kalt. "Ich bin der Junge-der-überlebt-hat."

Acht Jungen im ersten Jahr, ungefähr gleich groß. Einer von ihnen hatte eine Narbe auf seiner Stirn und verhielt sich nicht wie die anderen.

\emph{Ach, schenkte uns doch eine höhere Macht\\ Die Gab' uns zu seh'n, wie's von ander'n vollbracht!\\ Es mög' uns befrei'n von manch' falschem gedacht,\\ Und törichtem Urteil -}****

Professor McGonagall hatte recht. Der Sprechende Hut hatte recht. Es war eindeutig, sobald man es von außen sah.

Etwas stimmte nicht mit Harry Potter.

* Der Begriff kommt in den usprünglichen Romanen offenbar nicht vor. Dort besitzt Hermine zwar einen Zeitumkehrer, zeigt ihn aber normalerweise niemandem und benötigt daher wohl keine Ausrede. Vermutungen, sie sei zur selben Zeit an zwei Orten gewesen, tut sie nur als albern ab, da dies ja nicht möglich sei. Da Spimster Wickets nicht wirklich existieren und im weiteren Verlauf nicht wichtig werden, halte ich an der englischen Bezeichnung fest.\\ ** Eine Anspielung auf den Film \emph{Matrix} wäre wohl populärer gewesen, diesen Film kann der Protagonist Harry aber nicht kennen, weil er erst 1999 gedreht wurde.\\ *** Ich kann nicht wirklich behaupten, genug von Mathematik zu verstehen, um völlig zu begreifen, wovon Harry spricht. Ich wage aber den Versuch einer Erklärung: Offenbar gibt es einen mathematischen Beweis (siehe \emph{Halteproblem}), der besagt, dass es keinen Algorithmus geben kann, mit dem ein Computer berechnen könnte, ob es für jede x-beliebige (andere) Berechnung auch eine Lösung gibt. Ein Computer, der versuchen würde, ein Universum zu simulieren, dessen Zukunft und Vergangenheit sich gegenseitig in einer Schleife beeinflussen können, dessen Ereignisse aber trotzdem in sich konsistent sein (also kein logisches Paradox ergeben) sollen, müsste aber im Voraus berechnen können, ob eine bestimmte Handlung einer Person, die eine Zeitreise macht, irgendwann zu einem nicht paradoxen Ergebnis führen kann oder nicht. Sollte eine solche Handlung \emph{keine} in sich konsistente (nicht paradoxe) Zukunft ergeben, der Computer könnte das aber nicht vorhersehen, würde er sich, beim Versuch diese Zukunft zu berechnen, quasi "aufhängen". Für die Bewohner des simulierten Universums käme das einer Zeitschleife gleich (die Berechnung der Zukunft kann nicht beendet werden und wird endlos fortgesetzt) oder der Zerstörung ihres Universums (wenn der Computer die Simulation abbrechen sollte und damit keine Zukunft existiert). Letzteres könnte entgegen Professor McGonagalls Aussage natürlich von den Bewohnern des simulierten Universums \emph{nicht} bemerkt werden, weil seine Existenz (und die seiner Bewohner) von einem Augenblick zum nächsten einfach enden würde. Sollte jemand, der mehr von Mathematik versteht als ich, einen Fehler entdecken, so möge er es mir bitte erklären.\\ **** Harry zitiert in Gedanken das Gedicht \emph{To a Louse} (dt.: \emph{Für eine Laus}) des schottischen Poeten Robert Burns. Dieser beschreibt darin, wie eine Laus sich im Haar einer vornehmen Dame tummelt und amüsiert sich darüber, wie wenig die Laus über die Bedeutung ihres Wirtes wisse, bevor ihm klar wird, dass wir der Laus als Beute alle gleich wären und die Betrachtung durch die Augen eines anderen unserer Selbstüberschätzung Abhilfe schaffen könnte. Für diejenigen, die der Original-Text interessiert:\\ \emph{Oh wad some power the giftie gie us\\ To see oursel's as others see us!\\ It wad frae monie a blunder free us,\\ And foolish notion -}

