

\hypertarget{die-hypothese-finden}{% \section{17. Die Hypothese finden}\label{die-hypothese-finden}}

\textbf{Kapitel 17: Die Hypothese finden}

Du warst immer schon J. K. Rowling.

Historische Anmerkung: Im römischen Kalender bezeichneten die „Iden“ eines Monats den 15. Tag von März, Mai, Juli und Oktober und den 13. Tag aller anderen Monate.

\later

„\emph{Man beginnt das Muster zu sehen, dem Rythmus der Welt zu lauschen.}“

\later

Donnerstag.

Um genau zu sein, 7:24~Uhr am Donnerstagmorgen.

Harry saß auf seinem Bett, ein Lehrbuch lag schlaff in seinen unbewegten Händen.

Harry war gerade die Idee für einen \emph{wahrhaft brillantes} Experiment gekommen.

Es würde bedeuten, noch eine Stunde länger auf das Frühstück zu warten, aber dafür hatte er ja Müsliriegel. Nein, diese Idee musste definitiv, unbedingt sofort getestet werden, unverzüglich, jetzt.

Harry legte das Lehrbuch beiseite, schwang sich aus dem Bett, raste um das Bett herum, zog die Keller-Etage seines Koffers heraus, rannte die Stufen hinunter und schob Kisten voller Bücher umher. (Er würde irgendwann wirklich auspacken und sich Bücherregale besorgen müssen, aber er war mitten in seinem Lehrbuch-Lesewettstreit mit Hermine und fiel zurück, daher hatte er keine Zeit gehabt.)

Harry fand das Buch, dass er wollte und raste die Treppe wieder hinauf.

Die anderen Jungen machten sich fertig, um zum Frühstück in die Große Halle hinunter zu gehen und in den Tag zu starten.

„Entschuldige, könntest du mir kurz einen Gefallen tun?“ sagte Harry. Er blätterte durch das Inhaltsverzeichnis des Buchs während er sprach, fand die Seite mit den ersten zehntausend Primzahlen, blätterte dorthin und hielt das Buch Anthony Goldstein hin. „Wähle zwei dreistellige Zahlen aus dieser Liste. Sag mir nicht, welche es sind. Multipliziere sie einfach zusammen und sag mir das Ergebnis. Oh, und kannst du Berechnung zweimal machen, um sicher zu gehen? Bitte stell wirklich sicher, dass du die richtige Antwort hast; ich bin nicht sicher, was mit mir oder dem Universum passieren wird, wenn du einen Multiplikationsfehler machst.“

Es sagte eine Menge darüber aus, wie das Leben in diesem Schlafsaal in den letzten paar Tagen verlaufen war, dass Anthony sich nicht einmal damit aufhielt irgendetwas zu fragen in der Art von „Warum flippst du denn auf einmal so aus?“ oder „Das scheint wirklich merkwürdig, warum fragst du sowas?“ oder „Was meinst du mit, du weißt nicht, was mit dem Universum passieren wird?“

Anthony nahm ohne ein Wort das Buch entgegen und zog ein Pergament und eine Schreibfeder hervor. Harry drehte sich um und schloss die Augen, ging sicher, dass er nicht das Geringste sehen konnte, während er ungeduldig vor und zurück tanzte und auf und ab wippte. Er holte einen Block Papier und einen mechanischen Bleistift und war bereit zu schreiben.

„Okay,“ sagte Anthony, „Einhundertundeinundachtzigtausend, Vierhundertundneunundzwanzig.“

Harry schrieb auf 181.429. Er wiederholte, was er gerade aufgeschrieben hatte und Anthony bestätigte es.

Dann rannte Harry wieder hinunter in die Keller-Etage seines Koffers, blickte auf seine Armbanduhr (die Uhr zeigte 4:28~Uhr an, was 7:28 hieß) und schloss dann seine Augen.

Ungefähr dreißig Sekunden später hörte Harry das Geräusch von Schritten, gefolgt vom Geräusch der Keller-Etage seines Koffers, die sich zuschob. (Harry war nicht besorgt, dass er ersticken könnte. Ein automatischer Lufterfrischungs-Zauber war Teil dessen, was man bekam, wenn man willens war, einen wirklich guten Koffer zu kaufen. War Magie nicht wundervoll, man musste sich keine Gedanken über Stromrechnungen machen.)

Und als Harry seine Augen öffnete, sah er genau das, was er zu sehen gehofft hatte, ein gefaltetes Stück Papier lag auf dem Boden, ein Geschenk seines zukünftigen Selbst.

Nennen wir dieses Stück Papier „Papier-2“.

Harry riss ein Stück Papier von seinem Block ab.

Nennen wir das „Papier-1“. Es war, natürlich, das selbe Stück Papier. Man konnte sogar, wenn man genau hinsah, sehen, dass die Abrisskanten übereinstimmten.

Harry ging im Geiste noch einmal den Algorithmus durch, den er befolgen würde.

Wenn Harry Papier-2 öffnete und es leer war, dann würde er auf Papier-1 „101 x 101“ schreiben, es zusammenfalten, eine Stunde lang lernen, in der Zeit zurückgehen, Papier-1 fallen lassen (welches dadurch zu Papier-2 würde) und wieder aus der Keller-Etage hinaufsteigen, um sich seinen Schlafsaal-Kameraden zum Frühstück anzuschließen.

Wenn Harry Papier-2 öffnete und zwei Zahlen darauf geschrieben standen, würde Harry diese Zahlen miteinander multiplizieren.

Wenn ihr Produkt gleich 181.429 war, würde Harry diese zwei Zahlen auf Papier-1 schreiben und Papier-1 in der Zeit zurück schicken.

Wenn nicht, würde Harry der rechten Zahl 2 hinzufügen und das neue Zahlenpaar auf Papier-1 schreiben. Außer, das würde die rechte Zahl größer als 997 machen, in welchem Fall Harry der linken Zahl 2 hinzufügen und für die rechte 101 aufschreiben würde.

Und wenn auf Papier-2 stand 997 x 997, würde Harry Papier-1 leer lassen.

Was bedeutete, dass die einzig mögliche \emph{stabile} Zeitschleife diejenige war, in der Papier-2 die zwei Primfaktoren von 181.429 enthielt.

Wenn das funktionierte, konnte Harry es verwenden, um jede Art von Antwort zu enthüllen, die einfach zu überprüfen, aber schwierig zu finden war. Er hätte nicht \emph{nur} gezeigt, dass P=NP*, sobald man einen Zeitumkehrer besaß, der Trick war sogar \emph{noch vielseitiger} als das. Harry könnte ihn benutzen, um die Kombinationen von Zahlenschlössern zu ermitteln oder Passwörter aller Art. Vielleicht könnte er sogar den Eingang zu Slytherins Kammer des Schreckens finden, wenn Harry eine systematische Methode einfiel, alle Ortsangaben in Hogwarts zu beschreiben. Es wäre ein fantastischer Cheat, sogar nach Harrys Maßstäben.

Harry hob Papier-2 mit zitternder Hand auf und entfaltete es.

Auf Papier-2 stand in leicht zittriger Handschrift:

SPIEL NICHT MIT DER ZEIT HERUM

Harry schrieb „SPIEL NICHT MIT DER ZEIT HERUM“ in leicht zittriger Handschrift auf Papier-1, faltete es ordentlich zusammen und beschloss, keine wahrhaft brillanten Experimente mit der Zeit mehr durchzuführen, bis er wenigstens fünfzehn Jahre alt war.

Nach Harrys bestem Wissen war dies das erschreckendste experimentelle Resultat in der gesamten Geschichte der Wissenschaft gewesen.

In der nächsten Stunde war es Harry einigermaßen schwer gefallen, sich auf das Lesen seines Lehrbuches zu konzentrieren.

So also hatte Harrys Donnerstag begonnen.

\later

Donnerstag.

Um genau zu sein, 3:32~Uhr am Donnerstagnachmittag.

Harry und alle anderen Jungen im ersten Jahr befanden sich draußen auf einem grasbewachsenen Feld mit Madam Hooch und standen neben Hogwarts Vorrat an Besen. Die Mädchen würden getrennt von ihnen das Fliegen lernen. Offenbar wollten Mädchen, aus irgendeinem Grund, nicht in der Anwesenheit von Jungen das Reiten auf Besenstielen lernen.

Harry war schon den ganzen Tag lang ein wenig wackelig auf den Beinen. Er konnte einfach nicht aufhören, sich zu fragen, wie diese \emph{bestimmte} stabile Zeitschleife ausgewählt worden war aus dem, was im Rückblick betrachtet ein ziemlich großer Möglichkeitenraum war.

Außerdem: ernsthaft, \emph{Besenstiele?} Er würde fliegen auf, im Grunde genommen, einem Stück Linie? War das nicht so ziemlich die instabilste Form die man nur finden konnte, abgesehen von dem Versuch, sich auf einer blankpolierten Murmel zu halten? Wer hatte \emph{dieses} Design für ein Fluggerät ausgewählt, aus all den Möglichkeiten? Harry hatte gehofft, es wäre nur eine Redewendung, aber nein, sie standen vor dem, was für alle Welt nach ordinären, hölzernen Küchenbesen aussah. Hatte sich jemand einfach in die Idee von Besenstielen verrannt und hatte sich nichts anderes einfallen lassen können? Es musste so sein. Es konnte unmöglich sein, dass das \emph{optimale} Design um Küchen zu putzen und zu fliegen, zufälligerweise das selbe war, wenn man sie komplett neu entwarf.

Es war ein wolkenloser Tag mit hellblauem Himmel und glänzender Sonne, die nur so darum bettelte, einem in's Auge zu fallen und das Sehen unmöglich zu machen, wenn man versuchte, am Himmel herumzufliegen. Der Boden war richtig schön trocken, roch fast wie gebacken und fühlte sich irgendwie sehr, sehr hart unter Harrys Schuhen an.

Harry sagte sich immer wieder, dass von Elfjährigen als kleinster gemeinsamer Nenner erwartet wurde, das zu lernen und es so schwer nicht sein konnte.

„Streckt eure rechte Hand über dem Besen aus oder die linke, wenn ihr Linkshänder seid,“ rief Madam Hooch. „Und sagt AUF!“

„AUF!“ riefen sie alle.

Der Besenstiel sprang eifrig in Harrys Hand.

Was ihn zunächst einmal an die Spitze der Klasse beförderte. Offensichtlich war „AUF!“ zu sagen sehr viel schwieriger als es aussah und die meisten der Besenstiele rollten auf dem Boden herum oder versuchten sich von ihren Möchtegern-Reitern wegzuschieben.

(Natürlich hätte Harry um Geld gewettet, dass Hermine sich bei ihrem ersten Versuch, früher am Tag, mindestens genau so gut angestellt hatte. Es konnte unmöglich irgendetwas geben, das \emph{er} beim ersten Versuch beherrschte, was Hermine ratlos machte und wenn \emph{doch} und es stellte sich heraus, dass es sich um das \emph{Reiten auf einem Besenstiel} handelte, anstatt irgendetwas intellektuelles, würde es Harry einfach umbringen.)

Es dauerte eine Weile, bevor jeder einen Besenstiel vor sich hatte. Madam Hooch zeigte ihnen, wie man aufstieg und lief dann über das Feld, um Griffe und Haltungen zu korrigieren. Offenbar war es selbst den wenigen Kindern, denen es erlaubt gewesen war, zu Hause zu fliegen, nicht richtig beigebracht worden.

Madam Hooch beobachtete das Feld der Jungen und nickte. „Jetzt, wenn ich in meine Pfeife blase, stoßt ihr euch hart vom Boden ab.“

Harry schluckte schwer, versuchte das mulmige Gefühl in seinem Magen zu unterdrücken.

„Haltet eure Besen gerade, steigt ein paar Fuß weit auf und kommt dann direkt wieder runter, indem ihr euch leicht nach vorn lehnt. Auf mein Pfeifen - drei - zwei—“

Einer der Besen schoss dem Himmel entgegen, begleitet von den Schreien eines Jungen—Schreckens-, keine Freudenschreie. Der Junge drehte sich furchtbar schnell als er höher stieg, sie erhaschten nur kurze Blicke auf sein weißes Gesicht—

Wie in Zeitlupe sprang Harry wieder herunter von seinem eigenen Besenstiel und kramte nach seinem Zauberstab, obwohl er nicht wirklich wusste, was er damit tun wollte, er hatte genau zwei Zauberkunst-Stunden gehabt und in der letzten \emph{war} die Levitation dran gewesen, aber Harry hatte den Zauber nur in einem von drei Fällen erfolgreich wirken können und konnte sicherlich keine ganzen Personen schweben lassen—

\emph{Wenn es irgendeine versteckte Kraft in mir gibt, dann soll sie sich bitte JETZT zeigen!}

„Komm zurück, Junge!“ rief Madam Hooch (was wohl die am wenigsten hilfreiche Anweisung von einer \emph{Fluglehrerin} war, die man sich angesichts eines außer Kontrolle geratenen Besens nur vorstellen konnte und ein vollautomatischer Teil von Harrys Gehirn fügte Madam Hooch seiner Liste von Dummköpfen hinzu).

Und der Junge wurde vom Besenstiel geworfen.

Er schien sich sehr langsam durch die Luft zu bewegen, zunächst.

„\emph{Wingardium Leviosa!}“ schrie Harry.

Der Zauber schlug fehl. Er konnte fühlen, wie er versagte.

Es gab ein BUMPF und ein entferntes splitterndes Geräusch und der Junge lag als Häufchen mit dem Gesicht nach untem im Gras.

Harry steckte seinen Zauberstab ein und rannte mit Höchstgeschwindigkeit vorwärts. Er kam zur gleichen Zeit wie Madam Hooch bei dem Jungen an und Harry griff in seinen Beutel und versuchte sich zu erinnern oh Gott wie war der Name egal er versuchte einfach „Heiler-Pack!“ und es sprang in seine Hand und—

„Handgelenk gebrochen,“ sagte Madam Hooch. „Beruhige dich, Junge, er hat nur ein gebrochenes Handgelenk!“

Er geriet in eine Art mentales Taumeln, als Harrys Geist aus dem Panik-Modus gerissen wurde.

Das Notfall-Heiler-Pack Plus lag offen vor ihm und eine Spritze voll flüssigem Feuer lag in Harrys Hand, welches das Gehirn des Jungen weiter mit Sauerstoff versorgt hätte, wenn er es geschafft hätte, sich den Nacken zu brechen.

„Ah…“ sagte Harry mit eher zittriger Stimme. Sein Herz pochte so laut, dass er sich kaum selbst nach Atem ringen hören konnte. „Gebrochener Knochen… richtig… eine Schiene?“

„Die ist nur für Notfälle,“ schnappte Madam Hooch. „Steck das weg, ihm geht's gut.“ Sie beugte sich über den Jungen, reichte ihm die Hand. „Komm schon, Junge, ist schon gut, hoch mit dir!“

„Sie lassen ihn nicht ernsthaft noch einmal auf einem Besenstiel reiten?“ sagte Harry entsetzt.

Madam Hooch funkelte Harry an. „Natürlich nicht!“ Sie zog den Jungen auf die Füße an seinem gesunden Arm—Harry sah schockiert, dass es \emph{wieder} Neville Longbottom war, was \emph{war} bloß mit ihm? - und wandte sich den zusehenden Kindern zu. „Keiner von euch bewegt sich, während ich diesen Jungen zum Krankenflügel bringe! Ihr lasst diese Besen wo sie sind oder ihr werdet schneller von Hogwarts ausgeschlossen als ihr 'Quidditch' sagen könnt. Na komm, mein Bester.“

Und Madam Hooch ging mit Neville davon, der sein Handgelenk umklammerte und versuchte, sein Schniefen zu unterdrücken.

Als sie außer Hörweite waren, begann einer der Slytherins zu kichern.

Da fingen die anderen an.

Harry wand sich um und betrachtete sie. Es schien ein guter Moment um sich ein paar Gesichter einzuprägen.

Und Harry sah, dass Draco auf ihn zugeschritten kam, begleitet von Mr~Crabbe und Mr~Goyle. Mr~Crabbe grinste nicht, Mr~Goyle ganz entschieden schon. Draco selbst trug ein sehr kontrolliertes Gesicht zur Schau, das hin und wieder zuckte, woraus Harry schloss, dass Draco es zum Brüllen komisch fand, aber keinen politischen Vorteil darin sah, jetzt darüber zu lachen, anstatt nachher in den Slytherin-Kerkern.

„Nun, Potter,“ sagte Draco mit leiser, verhaltener Stimme, immer noch mit diesem kontrollierten, glegentlich zuckenden Gesichtsausdruck, „Ich wollte nur sagen, wenn man sich Notfälle zunutze macht, um Führungsstärke zu demonstrieren, sollte man aussehen, als habe man die Situation unter Kontrolle, nicht als, sagen wir, breche man komplett in Panik aus.“ Mr~Goyle kicherte und Draco schoss ihm einen erstickenden Blick zu. „Aber du hast wahrscheinlich trotzdem ein paar Punkte gemacht. Brauchst du Hilfe dabei, diese Heil-Ausrüstung zu verstauen?“

Harry drehte sich zu dem Heiler-Pack um, was sein Gesicht von Draco abwandte. „Ich denke, ich schaff das schon,“ sagte Harry. Er packte die Spritze zurück an ihren Platz, verschloss den Deckel wieder und erhob sich.

Ernie Macmillan traf ein, gerade als Harry das Pack wieder an seinen Eselsfell-Beutel verfütterte.

„Danke, Harry Potter, im Namen von Hufflepuff,“ sagte Ernie Macmillan förmlich. „Es war ein guter Versuch und ein guter Gedanke.“

Ein guter Gedanke, in der Tat,„ meinte Draco gedehnt. “Warum hatte niemand von Hufflepuff die Zauberstäbe gezückt? Vielleicht, wenn ihr \emph{alle} geholfen hättet, anstatt nur Potter, hättet ihr ihn auffangen können. Ich nahm an, Hufflepuffs sollten zusammen halten?"

Ernie sah aus als wisse er nicht, ob er wütend werden oder vor Scham im Boden versinken solle. „Wir haben nicht rechtzeitig daran gedacht—“

„Ah,“ sagte Draco, „nicht dran \emph{gedacht}, ich nehme an, daher ist es besser, einen Ravenclaw zum Freund zu haben, als alle Hufflepuffs zusammen.“

Oh, zur Hölle, wie sollte Harry das wieder jonglieren… „Das ist nicht hilfreich,“ sagte Harry mäßigend. Und hoffte Draco würde es verstehen als \emph{du kommst meinen Plänen in die Quere, bitte sei still.}

„Hey, was ist das?“ sagte Mr~Goyle. Er beugte sich hinunter zum Gras und hob etwas mit den Ausmaßen einer großen Murmel auf, eine Glaskugel, offenbar gefüllt mit wirbelndem weißen Rauch.

Ernie blinzelte. „Nevilles Erinnermich!“

„Was ist ein Erinnermich?“ fragte Harry.

„Es wird rot, wenn man etwas vergessen hat,“ sagte Ernie. „Es sagt einem allerdings nicht, was man vergessen hat. Gib es bitte her und ich gebe es Neville später wieder.“ Ernie streckte die Hand aus.

Ein plötzliches Grinsen zog über Mr~Goyles Gesicht und er wirbelte herum und rannte weg.

Ernie stand für einen Moment überrascht da und rief dann „Hey!“ und rannte Mr~Goyle nach.

Und Mr~Goyle ergriff einen Besenstiel, hüpfte in einer flüssigen Bewegung darauf und hob ab in die Luft.

Harrys Kinnlade fiel herunter. Hatte Madam Hooch nicht gesagt, man würde ihn dafür \emph{rausschmeißen?}

„\emph{Dieser Idiot!}“ zischte Draco. Er öffnete den Mund, um zu rufen—

„\emph{Hey!}“ rief Ernie. „Das gehört Neville! \emph{Gib es zurück!}“

Die Slytherins begannen zu jubeln und johlen.

Dracos Mund schnappte zu. Harry erhaschte einen kurzen Ausdruck der Unentschlossenheit auf seinem Gesicht.

„Draco,“ sagte Harry leise, „wenn du diesen Idioten nicht zurück auf den Boden beorderst, wird die Lehrerin zurückkommen und—“

„\emph{Komm und hol's dir, Huffelpuff!}“ rief Mr~Goyle und großer Jubel kam von den Slytherins.

„Ich \emph{kann nicht!}“ flüsterte Draco. „Jeder in Slytherin würde denken, dass ich \emph{schwach} bin!“

„Und wenn Mr~Goyle rausgeschmissen wird,“ zischte Harry, „wird dein \emph{Vater} denken, dass du ein Schwachkopf bist!“

Dracos Gesicht verzog sich qualvoll.

In diesem Moment—

„Hey, \emph{Slytherschleim,}“ rief Ernie, „hat dir noch niemand gesagt, dass Hufflepuffs zusammenhalten? \emph{Zauberstäbe raus, Hufflepuff!}“

Und plötzlich waren eine ganze Menge Zauberstäbe auf Mr~Goyle gerichtet.

Drei Sekunden später—

„\emph{Zauberstäbe raus, Slytherin!}“ sagten etwa fünf verschiedene Slytherins.

Und es wurden eine ganze Menge Zauberstäbe in Richtung Hufflepuff gerichtet.

Zwei Sekunden später—

„\emph{Zauberstäbe raus, Gryffindor!}“

„\emph{Tu irgendwas, Potter!}“ flüsterte Draco. „\emph{Ich kann nicht derjenige sein, der das beendet, das musst du sein! Ich schulde dir auch was dafür, aber denk dir was aus, sollst du denn nicht brillant sein?}“

In etwa fünfeinhalb Sekunden, wurde Harry klar, würde irgendjemand den Simplen Sumerischen Schlag-Zauber wirken, wenn es vorbei war und die Lehrer damit fertig waren, Leute rauszuwerfen, würden die einzig verbleibenden Jungen in diesem Jahrgang Ravenclaws sein.

„\emph{Zauberstäbe raus, Ravenclaw!}“ rief Michael Corner, der sich von dem Desaster offenbar ausgeschlossen fühlte.

„\emph{GREGORY GOYLE!}“ schrie Harry. „\emph{Ich fordere dich zu einem Wettstreit um den Besitz von Nevilles Erinnermich heraus!}“

Es gab eine plötzliche Pause

„Oh, wirklich?“ sagte Draco so laut und gedehnt wie Harry es je gehört hatte. „Das klingt interessant. Was für ein Wettstreit, Potter?“

Ähm…

„Wettstreit“ war das weiteste, was Harrys Inspiration zustande gebracht hatte. Was für ein Wettstreit, er konnte nicht „Schach“ sagen, weil Draco nicht annehmen könnte, ohne dass es seltsam aussah, er konnte nicht „Armdrücken“ sagen, weil Mr~Goyle ihn zerquetschen würde—

„Wie wär's damit?“ sagte Harry laut. „Gregory Goyle und ich stehen voneinander entfernt und niemand anders darf einem von uns nahe kommen. Wir benutzen unsere Zauberstäbe nicht und auch niemand sonst. Ich bewege mich nicht vom Fleck und er ebenfalls nicht. Und wenn ich Nevilles Erinnermich in die Hände bekomme, dann gibt Gregory Goyle alle Ansprüche auf das Erinnermich, das er hält, auf und gibt es mir.“

Es gab eine weitere Pause als die erleichterten Blicke der Leute in Verwirrung umschlugen.

„Hah, Potter!“ sagte Draco laut. „Ich würde gern sehen, wie du \emph{das} anstellst! Mr~Goyle aktzeptiert!“

„Dann los!“ sagte Harry.

„Potter, \emph{was?}“ flüsterte Draco, was er irgendwie hinbekam, ohne die Lippen zu bewegen.

Harry wusste nicht, wie er darauf antworten sollte, ohne seine zu bewegen.

Die Leute steckten ihre Zauberstäbe weg und Mr~Goyle stieß elegant zum Boden herab und sah ziemlich verwirrt aus. Einige Hufflepuffs setzten sich in Mr~Goyles Richtung in Bewegung, doch Harry warf ihnen einen verzweifelt flehenden Blick zu und sie machten kehrt.

Harry lief auf Mr~Goyle zu und stoppte als er noch ein paar Schritte entfernt war, weit genug, dass sie sich nicht erreichen konnten.

Betont langsam steckte Harry seinen Zauberstab weg.

Alle anderen wichen zurück.

Harry schluckte. Er hatte eine grobe Vorstellung davon, was er tun \emph{wollte}, aber es musste so geschehen, dass niemand verstand, \emph{was} er getan hatte—

„Alles klar,“ sagte Harry laut. „Und jetzt…“ Er holte tief Luft und hob eine Hand, die Finger zum Schnippen bereit. Alle, die von den Kuchen gehört hatten, keuchten auf, was praktisch jeder war. „\emph{Ich rufe den Wahnsinn von Hogwarts an!} \emph{Schwachkopf! Schwabbelspeck! Krimskrams! Quiek!}“ Und Harry schnippte mit den Fingern.

Viele Leute wichen zurück.

Und nichts passierte.

Harry ließ zu, dass die Stille sich eine Weile lang ausdehnte, entwickelte, bis…

„Ähm,“ sagte jemand. „War's das?“

Harry sah den Jungen an, der gesprochen hatte. „Sieh vor dich. Siehst du die kahle Stelle am Boden, ohne jedes Gras darauf?“

„Ähm, ja,“ sagte der Junge, ein Gryffindor (Dean irgendwas?).

„Heb sie aus.“

Jetzt wurden Harry eine Menge seltsamer Blicke zugeworfen.

„Äh, warum?“ sagte Dean irgendwas.

„Mach's einfach,“ sagte Terry Boot mit matter Stimme. „Macht keinen Sinn, nach dem Warum zu fragen, vertrau mir.“

Dean irgendwas kniete sich hin und begann, den Dreck wegzuschaufeln.

Nach etwa einer Minute stand Dean wieder auf. „Da ist nichts,“ sagte Dean.

Huh. Harry hatte geplant gehabt, in der Zeit zurück zu gehen und eine Schatzkarte zu vergraben, die zu einer anderen Schatzkarte führen würde, die zu Nevilles Erinnermich führen würde, das er dort platzieren würde, nachdem er es von Mr~Goyle zurückbekam…

Dann wurde Harry klar, dass es einen sehr viel einfacheren Weg gab, der keine ganz so große Gefahr für das Geheimnis um die Zeitumkehrer war.

„Danke, Dean!“ sagte Harry laut. „Ernie, würdest du auf dem Boden suchen, wo Neville gefallen ist und schauen, ob du Nevilles Erinnermich finden kannst?“

Die Leute blickten noch verwirrter drein.

„Mach's einfach,“ sagte Terry Boot. „Er wird's weiter versuchen, bis irgendwas klappt und das Erschreckende ist, dass—“

„\emph{Merlin!}“ keuchte Ernie. Er hielt Nevilles Erinnermich hoch. „Es ist hier! Genau wo er gefallen ist!“

„Was?“ schrie Mr~Goyle. Er blickte nach unten und sah…

… dass er immer noch Nevilles Erinnermich hielt.

Es gab eine ziemlich lange Pause.

„Äh,“ sagte Dean irgendwas, „das ist nicht möglich, oder?“

„Es ist eine Handlungslücke,“ sagte Harry. „Ich habe mich selbst seltsam genug gemacht, um das Universum für einen Moment abzulenken und es hat vergessen, dass Goyle das Erinnermich bereits aufgehoben hat.“

„Nein, warte, ich meine das ist \emph{vollkommen} unmöglich—“

„Entschuldige, stehen wir alle hier herum und warten darauf, auf Besenstielen zu fliegen? Ja, tun wir. Also sei still. Jedenfalls, sobald ich Nevilles Erinnermich in Händen halte ist der Wettstreit vorbei und Gregory Goyle muss alle Ansprüche auf das Erinnermich, das er hält, aufgeben und es mir übergeben. Das waren die Bedingungen, nicht wahr?“ Harry streckte eine Hand aus und winkte Ernie heran. „Roll es einfach hier rüber, da mir ja niemand nahe kommen soll, okay?“

„Moment mal!“ rief ein Slytherin—Blaise Zaibini, Harry würde diesen Namen nicht so schnell vergessen. „Woher wissen wir, dass das Nevilles Erinnermich ist? Du hättest einfach ein \emph{anderes} Erinnermich dort fallen lassen können—“

„Slytherin ist stark in diesem da,“ sagte Harry lächelnd. „Doch ihr habt mein Wort, dass das, welches Ernie hält, Nevilles ist. Kein Kommentar über das, welches Gregory Goyle hält.“

Zabini fuhr zu Draco herum. „\emph{Malfoy!} Du wirst ihn doch damit nicht einfach durchkommen lassen—“

„Sei still, du,“ grollte Mr~Crabbe, der hinter Draco stand. „Mr~Malfoy brauch' \emph{dich} nicht, ihm zu sagen, was'er tun soll!“

\emph{Braver} Lakai.

„Ich habe mit Draco, vom Noblen und Uralten Haus Malfoy gewettet,“ sagte Harry. „Nicht mit dir, Zabini. Ich habe getan, wovon Mr~Malfoy sagte, er wolle es mich tun sehen und was die Beurteilung der Wette angeht, so überlasse ich dies Mr~Malfoy.“ Harry neigte seinen Kopf in Dracos Richtung und zog leicht die Augenbrauen hoch. Das sollte Draco erlauben, angemessen sein Gesicht zu wahren.

Es gab eine Pause.

„Du schwörst, das \emph{ist} tatsächlich Nevilles Erinnermich?“ sagte Draco.

„Ja,“ sagte Harry. „Dieses wird an Neville zurückgehen und war ursprünglich seines. Und das, welches Gregory Goyle hält, geht an mich.“

Draco nickte entschieden. „Dann werde ich das Wort des Noblen Hauses Potter nicht in Frage stellen, unerheblich wie seltsam dies alles war. Und das Noble und Uralte Haus Malfoy hält sein Wort ebenso. Mr~Goyle, übergib Mr~Potter das—“

„Hey!“ sagte Zabini. „Er hat \emph{noch} nicht gewonnen, er hält es noch nicht in Händen—“

„Fang, Harry!“ sagte Ernie und er warf das Erinnermich.

Harry schnappte das Erinnermich mit Leichtigkeit aus der Luft, in der Hinsicht hatte er immer gute Reflexe gehabt. „Da,“ sagte Harry, „Ich gewinne…“

Harry verstummte. Alle Gespräche endeten.

Das Erinnermich glühte hellrot in seiner Hand, strahlend wie eine Miniatur-Sonne, die selbst am hellichten Tag Schatten auf den Boden warf.

\later

Donnerstag.

Um genau zu sein, 5:09~Uhr am Donnerstagnachmittag in Professor McGonagalls Büro, nach der Flugstunde. (Mit einer zwischendrin eingeschobenen Extra-Stunde für Harry.)

Professor McGonagall saß auf ihrem Stuhl. Harry auf dem heißen Stuhl vor ihrem Schreibtisch.

„Professor,“ sagte Harry angespannt, „Die Slytherins hatten ihre Zauberstäbe auf Hufflepuff gerichtet, die Gryffindors richteten die Zauberstäbe auf Slytherin, irgendein \emph{Schwachkopf} rief Zauberstäbe raus in Ravenclaw und ich hatte vielleicht fünf Sekunden, bevor das ganze Ding in die Luft geht! Es war alles, was mir eingefallen ist!“

Professor McGonagalls Gesichtsausdruck war verkniffen und aufgebracht. „\emph{Sie dürfen den Zeitumkehrer nicht in dieser Weise einsetzen, Mr~Potter!} Verstehen Sie das Konzept der Geheimhaltung nicht?“

„Sie \emph{wissen} nicht, wie ich es gemacht habe! Sie denken nur, dass ich wirklich seltsame Sachen tun kann, indem ich mit den Fingern schnippe! Ich habe schon andere seltsame Sachen gemacht, die man nicht einmal mit Zeitumkehrern tun kann und ich werde noch \emph{mehr} solcher Sachen machen und \emph{dieser} Fall wird nicht einmal herausstechen! Ich \emph{musste es tun,} Professor!“

„Sie mussten es \emph{nicht} tun!“ schnappte Professor McGonagall. „Alles, was Sie tun mussten, war, diesen \emph{namenlosen Slytherin} zurück auf den Boden zu holen und dafür zu sorgen ,dass die Zauberstäbe weggesteckt wurden! Sie hätten ihn zu einer Partie Zauberschnippschnapp** herausfordern können, aber nein, Sie mussten den Zeitumkehrer in ungeheuerlicher und unnötiger Weise verwenden!“

„Es war alles, was mir eingefallen ist! Ich weiß nicht einmal, was Zauberschnippschnapp \emph{ist}, sie hätten keine Partie Schach akzeptiert und wenn ich Armdrücken genommen hätte, hätte ich verloren!“

„\emph{Dann hätten Sie Armdrücken wählen sollen!}“

Harry blinzelte. „Aber dann hätte ich \emph{verloren—} “

Harry hielt inne.

Professor McGonagall sah jetzt \emph{sehr} verärgert aus.

„Es tut mir leid, Professor McGonagall,“ sagte Harry kleinlaut. „Ich habe ehrlich nicht daran gedacht und Sie haben recht, das hätte ich, es wäre brillant gewesen wenn doch, aber es ist mir einfach überhaupt nicht in den Sinn gekommen…“

Harrys Stimme erstarb. Es war ihm plötzlich bewusst, dass er eine \emph{Menge} anderer Optionen gehabt hatte. Er hätte \emph{Draco} bitten können etwas vorzuschlagen, er hätte die Menge fragen können… seine Verwendung des Zeitumkehrers \emph{war} ungeheuerlich und unnötig gewesen. Es hatte einen gigantischen Möglichkeitenraum gegeben, warum hatte er \emph{diese} gewählt?

Weil er einen Weg gesehen hatte, zu \emph{gewinnen}. Ein unwichtiges Schmuckstück zu gewinnen, dass die Lehrer Mr~Goyle ohnehin wieder weggenommen hätten.

Der Drang zu gewinnen. Das war es, was ihn gepackt hatte.

„Es tut mir leid,“ sagte Harry erneut. „Für meinen Stolz und meine Dummheit.“

Professor McGonagall fuhr sich mit der Hand über die Stirn. Ihr Ärger schien sich zum Teil zu verflüchtigen. Aber ihre Stimme erklang sehr hart. „Noch eine solche Vorstellung, Mr~Potter, und Sie werden diesen Zeitumkehrer zurückgeben. Habe ich mich klar ausgedrückt?“

„Ja,“ sagte Harry. „Ich verstehe und es tut mir leid.“

„Dann, Mr~Potter, wird es Ihnen für den Moment gestattet, den Zeitumkehrer zu behalten. Und in Anbetracht der Ausmaße des Debakels, welches Sie, in der Tat, abgewendet haben, werde ich keine Punkte von Ravenclaw abziehen.“

\emph{Außerdem könnten Sie ohnehin nicht erklären, warum Sie die Punkte} \emph{abgezogen haben.} Aber Harry war nicht dumm genug, das laut zu sagen.

„Was noch wichtiger ist, warum ist das Erinnermich so losgegangen?“ sagte Harry. „Bedeutet das, ein Vergessenszauber wurde auf mich gewirkt?“

„Das verwirrt mich ebenso,“ sagte Professor McGonagall langsam. „Wenn es so simpel wäre, würde ich denken, dass die Gerichte Erinnermichs benutzten und das tun sie nicht. Ich werde in der Sache nachforschen, Mr~Potter.“ Sie seufzte. „Sie können nun gehen.“

Harry setzte an, sich von seinem Stuhl zu erheben, dann zögerte er. „Ähm, entschuldigen Sie, ich wollte Ihnen noch etwas anderes mitteilen—“

Man konnte das Zucken kaum erkennen. „Worum geht es, Mr~Potter?“

„Es ist wegen Professor Quirrell—“

„Ich bin sicher, Mr~Potter, dass es nichts von Bedeutung ist.“ Professor McGonagall sprach die Worte sehr hastig. „Sicher haben Sie mitbekommen, wie der Schulleiter die Schüler angewiesen hat, dass sie uns nicht mit irgendwelchen unwichtigen Beschwerden über den Verteidigungs-Professor behelligen sollen?“

Harry war ziemlich verwirrt. „Aber das könnte sehr wichtig \emph{sein}, gestern verspürte ich ein plötzliches Gefühl des Unheils, als—“

„Mr~Potter! Auch verspüre ich ein Gefühl des Unheils! Und mein Gefühl des Unheils sagt mir, dass \emph{Sie diesen Satz nicht beenden sollten!}“

Harry stand der Mund offen. Professor McGonagall hatte es geschafft; Harry war sprachlos.

„Mr~Potter,“ sagte Professor McGonagall, „wenn Sie irgendetwas interessant erscheinendes über Professor Quirrell herausgefunden haben, zögern Sie bitte nicht, es weder mir noch irgendjemand anderem mitzuteilen. Nun denke ich, haben Sie genug meiner wertvollen Zeit in Anspruch genommen—“

„\emph{Das passt nicht zu Ihnen!}“ platzte Harry heraus. „Es tut mir leid, aber das erscheint einfach \emph{unglaublich} verantwortungslos! Nach dem, was ich gehört habe, liegt eine Art Fluch auf der Verteidigungs-Stelle und wenn Sie bereits \emph{wissen}, dass etwas schief gehen wird, würde ich annehmen, dass Sie so wachsam wie möglich wären—“

„\emph{Schief} gehen, Mr~Potter? \emph{Ich hoffe doch nicht.}“ Professor McGonagalls Gesicht war ausdruckslos. „Nachdem Professor Blake letzten Februar mit nicht weniger als drei Slytherins im fünften Jahr in einem Schrank erwischt wurde und ein Jahr davor Professor Summers als Lehrerin so vollkommen versagt hat, dass ihre Schüler dachten, ein Irrwicht sei ein Möbelstück, wäre es \emph{katastrophal}, wenn mir jetzt ein Problem mit dem außerordentlich kompetenten Professor Quirrell bekannt würde und ich wage zu sagen, die meisten unserer Schüler würden ihre Verteidigungs-Z.A.G.s und U.T.Z.s nicht bestehen.“

„Ich verstehe,“ sagte Harry langsam und nahm das alles auf. „Also, mit anderen Worten, was immer mit Professor Quirrell nicht stimmt, Sie wollen es auf gar keinen Fall vor Ende des Schuljahres wissen. Und da es momentan September ist, könnte er, soweit es Sie betrifft, live im Fernsehen den Premierminister ermorden und damit davon kommen.“

Professor McGonagall starrte ihn an ohne zu blinzeln. „Ich bin sicher, dass niemals eine solche Äußerung von mir zu vernehmen sein dürfte, Mr~Potter. In Hogwarts streben wir danach vorausschauend \emph{allem} zu begegnen, was den Bildungserfolg unserer Schüler gefährden könnte.“

\emph{So wie Ravenclaw-Erstklässler, die ihren Mund nicht halten können.} „Ich glaube, ich verstehe vollkommen, Professor McGonagall.“

„Oh, das bezweifle ich, Mr~Potter. Das bezweifle ich sehr.“ Professor McGonagall lehnte sich vor, ihr Gesicht spannte sich erneut an. „Da Sie und ich bereits sehr viel sensiblere Angelegenheiten als diese besprochen haben, werde ich offen sprechen. Sie und Sie allein haben von diesem mysteriösen Gefühl des Unheils berichtet. Sie und nur Sie allein sind ein Chaos-Magnet, wie mir noch nie einer untergekommen ist. Nach unserem kleinen Einkaufsbummel in der Winkelgasse und \emph{dann} dem Sprechenden Hut und dann dem \emph{heutigen} kleinen Auftritt, kann ich bereits voraussehen, dass es mir bestimmt sein wird, im Büro des Schulleiters einer lächerlichen Geschichte über Professor Quirrell zu lauschen, in welcher Sie und Sie allein eine Hauptrolle spielen und nach der es keine andere Wahl gibt, als ihn zu entlassen. Ich habe mich bereits damit abgefunden, Mr~Potter. Und sollte sich dieses bedauernswerte Ereignis noch vor den Iden des Mai ereignen, werde ich Sie vor den Toren von Hogwarts an Ihren eigenen Gedärmen aufhängen und Ihnen dabei Feuerkäfer in die Nase gießen. Verstehen Sie mich \emph{jetzt} vollkommen?“

Harry nickte, seine Augen weit aufgerissen. Dann, nach einer Sekunde, „Was passiert, wenn ich dafür sorgen kann, dass es genau am letzten Tag des Schuljahres passiert?“

„\emph{Raus aus meinem Büro!}“

\later

Donnerstag.

Irgendwas musste es mit Donnerstagen in Hogwarts auf sich haben.

Es war 5:32~Uhr am Donnerstagnachmittag und Harry stand neben Professor Flitwick vor dem großen steinernen Wasserspeier, der den Eingang zum Büro des Schulleiters bewachte.

Harry hatte es gerade rechtzeitig von Professor McGonagalls Büro zu den Ravenclaw-Studierzimmern zurück geschafft, um von einem der Schüler gesagt zu bekommen, er solle sich in Professor Flitwicks Büro melden und dort hatte Harry herausgefunden, dass Dumbledore ihn sprechen wollte.

Harry hatte Professor Flitwick besorgt gefragt, ob der Schulleiter gesagt habe, worum es ging.

Professor Flitwick hatte nur hilflos mit den Achseln gezuckt.

Offenbar hatte Dumbledore gesagt, dass Harry viel zu jung sei, die Worte der Macht und des Wahnsinns auszurufen.

\emph{Schwachkopf, Schwabbelspeck, Krimskrams, Quiek?} hatte Harry gedacht, aber nicht laut gesagt.

„Machen Sie sich bitte nicht zu viele Sorgen, Mr~Potter,“ quiekte Professor Flitwick von irgendwo nahe Harrys Schulterhöhe. (Harry war dankbar für Professor Flitwicks gigantischen Rauschebart, es war schwer sich an einen Professor zu gewöhnen, der nicht nur kleiner war als er, sondern auch mit höherer Stimme sprach.) „Schulleiter Dumbledore mag etwas seltsam wirken, oder auch sehr seltsam, oder sogar extrem seltsam, aber er hat noch nie einen Schüler im mindesten verletzt und ich glaube, das wird er auch niemals.“ Professor Flitwick warf Harry ein ermutigendes Lächeln zu. „Denken Sie einfach immer daran und Sie geraten sicher nicht in Panik!“

Das war nicht hilfreich.

„Viel Glück!“ quiekte Professor Flitwick und lehnte sich zu dem Wasserspeier und sagte etwas, das zu hören Harry irgendwie vollkommen misslang. (Natürlich, das Passwort wäre nicht besonders gut, wenn man hören konnte, wie es jemand sagte.) Und der steinerne Wasserspeier schritt mit einer sehr natürlichen und unauffälligen Bewegung beiseite, was Harry ziemlich schockierend fand, da der Wasserspeier trotzdem die ganze Zeit wie ein solides, unbewegliches Stück Stein wirkte.

Hinter dem Wasserspeier befand sich eine langsam rotierende Wendeltreppe. Es lag etwas verstörend hypnotisches darin und noch verstörender war die Tatsache, dass das \emph{Rotieren} der Stufen einen eigentlich nirgendwo hin befördern sollte.

„Hoch mit Ihnen!“ quiekte Flitwick.

Harry stieg ziemlich nervös auf die Spirale und stellte fest, dass er sich, aus einem Grund, den zu visualisieren sein Gehirn völlig versäumte, aufwärts bewegte.

Der Wasserspeier rumpelte hinter ihm zurück an seinen Platz und die Wendeltreppe drehte sich weiter und Harry stieg immer höher und nach einigen ziemlich schwindelerregenden Momenten fand Harry sich vor einer Eichentür mit einem Greifen-Türklopfer aus Messing wieder.

Harry streckte die Hand aus und drehte den Türknauf.

Die Tür schwang auf.

Und Harry erblickte den interessantesten Raum, den er je im Leben gesehen hatte.

Da waren winzige Mechanismen aus Metall, die sirrten oder tickten oder langsam ihre Form veränderten oder kleine Wölkchen aus Rauch ausstießen. Es gab dutzende von mysteriösen Flüssigkeiten in dutzenden seltsam geformter Behältnisse, die alle blubberten, kochten, hervor quollen, ihre Farbe veränderten oder interessante Formen bildeten, die wenn man sie ansah nach einer halben Sekunde verschwanden. Es gab Dinge, die wie Uhren mit vielen Händen aussahen, beschrieben mit Ziffern oder in nicht zu identifizierenden Sprachen. Da waren ein Armband, dass einen linsenförmigen Kristall trug, der in eintausend Farben glitzerte und ein Vogel, der auf einer goldenen Plattform hockte und ein hölzerner Becher, der mit etwas gefüllt war, das aussah wie Blut und eine Statue eines Falken, überzogen mit schwarzem Lack. Die Wand war komplett mit Bildern schlafender Menschen behangen und der Sprechende Hut hing lässig auf einem Kleiderständer, an dem sich auch zwei Regenschirme und drei rote Pantoffel für linke Füße befanden.

Mitten in all dem Chaos befand sich ein aufgeräumter schwarzer Schreibtisch aus Eichenholz. Vor dem Schreibtisch befand sich ein eichener Stuhl. Und hinter dem Schreibtisch war ein gut gepolsterter Thron, darin befand sich Albus Percival Wulfric Brian Dumbledore, geschmückt mit einem langen, silbernen Bart, einem Hut wie ein zerquetschter Pilz und etwas, das für Muggelaugen aussah wie drei Lagen leuchtend pinker Pyjamas.

Dumbledore lächelte und seine Augen leuchteten vor wahnsinniger Intensität.

Einigermaßen beklommen setzte sich Harry vor den Schreibtisch. Die Tür schwang hinter ihm mit einem lauten \emph{Wumm} zu.

„Hallo, Harry,“ sagte Dumbledore.

„Hallo, Schulleiter,“ antwortete Harry. Also sprachen sie sich schon mit Vornamen an? Würde Dumbledore jetzt sagen, er solle ihn-

„Bitte, Harry!“ sagte Dumbledore. „Schulleiter klingt so förmlich. Nenne mich doch einfach kurz Schuh.“***

„Sicher doch, Schuh,“ sagte Harry.

Es gab eine kurze Pause.

„Weißt du,“ sagte Dumbledore, „dass du die erste Person bist, die mich dabei jemals beim Wort genommen hat?“

„Ah…“ sagte Harry. Er versuchte seine Stimme ruhig zu halten, trotz des plötzlichen flauen Gefühls in seinem Magen. „Es tut mir leid, ich, ah, Schulleiter, Sie sagten mir, ich solle es tun, also habe ich—“

„Schuh, bitte!“ sagte Dumbledore heiter. „Und es gibt keinen Grund so besorgt zu sein, ich werde dich nicht aus einem Fenster werfen, nur weil du einen Fehler machst. Ich werde dir vorher viele Warnungen zukommen lassen, wenn du etwas falsch machst! Außerdem spielt es keine Rolle, wie Menschen mit einem reden, sondern was sie von einem halten.“

\emph{Er hat noch nie einen Schüler verletzt, denken Sie einfach immer daran und Sie geraten sicher nicht in Panik.}

Dumbledore zog eine kleine Metallschachtel hervor und öffnete sie, zum Vorschein kamen kleine gelbe Klumpen. „Limonen-Brausekugel?“ sagte der Schulleiter.

„Äh, nein danke, Schuh,“ sagte Harry. \emph{Zählte es als einen Schüler verletzen, wenn man ihm LSD unterjubelte oder fiel das in die Kategorie harmloser Spaß?} „Sie, ähm, sagten etwas darüber, dass ich noch zu jung sei, die Worte der Macht und des Wahnsinns auszurufen?“

„Was du mit größter Sicherheit bist!“ sagte Dumbledore. „Glücklicherweise sind die Worte der Macht und des Wahnsinns vor siebenhundert Jahren verloren gegangen und niemand hat mehr die leiseste Ahnung wie sie lauten. Es war nur eine kleine Anmerkung!“

„Ah…“ sagte Harry. Er war sich bewusst, dass sein Mund offen stand. „Warum haben Sie mich dann hierher bestellt?“

„\emph{Warum?}“ wiederholte Dumbledore. „Ah, Harry, wenn ich mich den ganzen Tag lang fragen würde, \emph{warum} ich Dinge tue, würde ich niemals auch nur eine Sache fertig bringen! Ich bin ein ziemlich beschäftigter Mensch, weißt du.“

Harry nickte lächelnd. „Ja, es war eine sehr beeindruckende Liste. Schulleiter von Hogwarts, Großmeister des Zaubergamots und Ganz Hohes Tier der Internationalen Vereinigung der Zauberer. Entschuldigen Sie die Frage, aber kann man mehr als sechs Stunden bekommen, wenn man mehr als einen Zeitumkehrer benutzt? Weil es ziemlich beeindruckend ist, wenn Sie all das in nur dreißig Stunden am Tag hinbekommen.“

Es gab eine weitere kurze Pause, während der Harry weiterhin lächelte. Er war ein wenig beklommen, tatsächlich sogar sehr, aber sobald klar war, dass Dumbledore mit voller Absicht mit ihm spielte, \emph{weigerte sich etwas in ihm absolut,} einfach da zu sitzen und es hinzunehmen, wie ein wehrloser Trottel.

„Ich fürchte, die Zeit mag es nicht zu sehr gedehnt zu werden,“ sagte Dumbledore nach der kurzen Pause, „und doch scheinen wir selbst ein wenig zu groß für sie zu sein, daher haben wir stets zu kämpfen, unser Leben in die Zeit hinein zu zwängen.“

„In der Tat,“ sagte Harry mit feierlichem Ernst. „Daher ist es am besten, direkt zur Sache zu kommen.“

Für einen Moment fragte sich Harry, ob er zu weit gegangen war.

Dann gluckste Dumbledore. „Direkt auf den Punkt soll es sein.“ Der Schulleiter lehnte sich vor, neigte seinen zerquetschten Pilzhut und strich mit seinem Bart über den Schreibtisch. „Harry, an diesem Montag hast du etwas getan, was selbst mit einem Zeitumkehrer unmöglich sein sollte. Oder eher, unmöglich mit \emph{nur} einem Zeitumkehrer. Woher kamen diese zwei Kuchen, frage ich mich?“

Ein Adrenalinschub durchfuhr Harry. Er hatte das mit Hilfe des Unsichtbarkeitsumhanges getan, dem, der ihm in einem Weihnachts-Päckchen zusammen mit einer Nachricht zugekommen war und die Nachricht hatte besagt: \emph{Sähe Dumbledore die Chance, eines der Heiligtümer des Todes zu besitzen, entließe er es niemals aus seinem Griff…}

„Ein naheliegender Gedanke,“ sprach Dumbledore weiter, „wäre, da keiner der anwesenden Erstklässler einen solchen Zauber wirken konnte, dass noch jemand anwesend war und doch ungesehen blieb. Und wenn niemand ihn sehen konnte, nun, dann wäre es für ihn sehr einfach, die Kuchen zu werfen. Man könnte weiter vermuten, da du einen Zeitumkehrer hast, dass du der Unsichtbare warst und da der Desillusionierungszauber deine derzeitigen Fähigkeiten weit übersteigt, dass du einen Unsichtbarkeitsumhang hattest.“ Dumbledore lächelte verschwörerisch. „Wie schlage ich mich soweit, Harry?“

Harry war erstarrt. Er hatte das Gefühl, dass eine direkte Lüge ganz und gar unklug wäre und möglicherweise auch überhaupt nicht hilfreich und er wusste nicht, was er sonst hätte sagen sollen.

Dumbledore winkte freundlich ab. „Keine Sorge, Harry, du hast nichts falsch gemacht. Unsichtbarkeitsumhänge sind nicht gegen die Vorschriften—ich nehme an, sie sind selten genug, dass niemals jemand daran gedacht hat, sie auf die Liste zu setzen. Aber tatsächlich habe ich mich etwas ganz anderes gefragt.“

„Oh?“ sagte Harry im normalsten Tonfall, den er zustande brachte.

Dumbledores Augen leuchteten vor Begeisterung. „Weißt du, Harry, nachdem man einige Abenteuer durchlebt hat, bekommt man allmählich ein Gespür für solche Dinge. Man beginnt das Muster zu sehen, dem Rythmus der Welt zu lauschen. Man ahnt mit der Zeit so einiges \emph{bevor} der Moment der Offenbarung kommt. Du bist der Junge-der-überlebt-hat und irgendwie hat ein Unsichtbarkeitsumhang den Weg in deine Hände gefunden, nur vier Tage, nachdem du das magische Britannien für dich entdeckt hast. Solche Umhänge werden nicht in der Winkelgasse verkauft, aber es gibt \emph{einen}, der seinen eigenen Weg zu dem ihm bestimmten Träger finden könnte. Und so kann ich nicht umhin mich zu fragen, ob du möglicherweise nicht nur \emph{einen} Unsichtbarkeitsumhang gefunden hast, sondern \emph{den} Unsichtbarkeitsumhang, eines der drei Heiligtümer des Todes, dem nachgesagt wird, es sei in der Lage, den Träger auch vor dem Blick des Todes selbst zu verbergen.“ Dumbledores Augen glänzten begierig. „Kann ich ihn sehen, Harry?“

Harry schluckte. Sein Körper war jetzt vollkommen geflutet mit Adrenalin und es war absolut aussichtslos, dies war der mächtigste Zauberer der Welt und es war unmöglich, dass er es zur Tür hinaus schaffen würde und selbst wenn doch gab es keinen Ort in Hogwarts, an dem er sich verstecken könnte, er würde den Umhang verlieren, der unter den Potters seit wer weiß wie langer Zeit weitergereicht worden war—

Langsam lehnte sich Dumbledore in seinem hohen Stuhl zurück. Das Leuchten war aus seinen Augen verschwunden und er blickte verwirrt und ein wenig besorgt drein. „Harry,“ sagte Dumbledore, „wenn du nicht willst, kannst du einfach nein sagen.“

„Kann ich?“ krächzte Harry.

„Ja, Harry,“ sagte Dumbledore. Seine Stimme klang nun traurig und besorgt. „Es scheint, dass du Angst vor mir hast, Harry. Darf ich fragen, womit ich mir dein Misstrauen verdient habe?“

Harry schluckte. „Haben Sie eine Möglichkeit einen bindenden magischen Eid zu schwören, dass Sie meinen Umhang nicht an sich nehmen werden?“

Dumbledore schüttelte langsam den Kopf. „Unbrechbare Schwüre benutzt man nicht so leichtfertig. Und davon abgesehen, Harry, wenn du den Zauber nicht bereits kennen würdest, hättest du nur mein Wort, dass der Zauber bindend ist. Doch sicherlich ist dir klar, dass ich deine Erlaubnis nicht \emph{brauche}, um den Umhang zu sehen. Ich bin mächtig genug, ihn an mich zu bringen, Eselsfell-Beutel oder nicht.“ Dumbledores Gesicht war sehr ernst. „Doch das werde ich nicht tun. Der Umhang gehört dir, Harry. Ich werde ihn dir nicht abnehmen. Ihn nicht einmal für einen Moment ansehen, es sei denn, du entscheidest, ihn mir zu zeigen. Dies ist ein Versprechen und ein Eid. Sollte ich es für nötig befinden, dir seinen Einsatz auf dem Schulgelände zu untersagen, werde ich dich anweisen, dein Verlies bei Gringotts aufzusuchen und ihn dort zu deponieren.“

„Ah…“ sagte Harry. Er schluckte schwer, versuchte den Adrenalin-Fluss zu beruhigen und vernünftig zu denken. Er nahm den Eselsfell-Beutel von seinem Gürtel. „Wenn Sie meine Erlaubnis \emph{wirklich} nicht brauchen… dann haben Sie sie.“ Harry streckte Dumbledore den Beutel hin und biss sich hart auf seine Lippe, um sich selbst ein Signal zu schicken, für den Fall, dass später ein Vergessenszauber bei ihm angewendet werden sollte.

Der alte Zauberer griff in den Beutel und ohne einen entsprechenden Befehl zog er den Unsichtbarkeitsumhang heraus.

„Ah,“ hauchte Dumbledore. „Ich hatte recht…“ Er ließ das schwarz schimmernde Samtgewebe durch seine Hand gleiten. „Jahrhunderte alt und noch immer so perfekt, wie an dem Tag, als er gefertigt wurde. Wir haben viele unserer Künste über die Jahre verloren und nun kann nicht einmal ich selbst ein solches Stück erschaffen, niemand kann es. Ich kann seine Macht fühlen, wie sie in meinem Geiste widerhallt, wie ein auf ewig gesungenes Lied, welches niemand mehr hört…“ Der Zauberer sah von dem Umhang auf. „Verkaufe ihn nicht,“ sagte er, „übergib ihn in niemandes Besitz. Überlege zweimal, bevor du ihn irgendjemandem zeigst und wäge dreimal ab, bevor du ihn als Heiligtum des Todes enthüllst. Behandle ihn mit Respekt, da dies in der Tat ein machtvoller Gegenstand ist.“

Für einen Moment wurde Dumbledores Gesichtsausdruck wehmütig…

… und dann gab er Harry den Umhang zurück.

Harry steckte ihn zurück in seinen Beutel.

Dumbledores Gesicht wurde wieder ernst. „Darf ich fragen, Harry, was dich veranlasst hat, mir so zu misstrauen?“

Plötzlich schämte Harry sich ziemlich.

„Es war eine Notiz bei dem Umhang,“ sagte Harry kleinlaut. „Sie besagt, Sie würden versuchen, mir den Umhang wegzunehmen, wenn Sie davon wüssten. Ich weiß allerdings nicht, wer die Nachricht hinterlassen hat, wirklich nicht.“

„Ich… verstehe,“ sagte Dumbledore langsam. „Nun, Harry, ich werde die Motive desjenigen, wer immer dir die Nachricht hinterlassen hat, nicht in Zweifel ziehen. Vielleicht hatte er selbst nichts als die besten Absichten? Er gab dir immerhin den Umhang.“

Harry nickte, beeindruckt von Dumbledores Großzügigkeit und verlegen über den scharfen Kontrast zu seinem eigenen Auftreten.

Der alte Zauberer fuhr fort. „Doch du und ich sind beide Spielfiguren der selben Farbe, denke ich. Der Junge, der Voldemort schlussendlich besiegt hat und der alte Mann, der ihn lange genug aufhielt, damit du den Tag zu retten vermochtest. Ich werde dir deine Vorsicht nicht vorwerfen, Harry, wir alle müssen versuchen, so klug wie möglich zu sein. Ich erbitte nur, dass du zweimal überlegst und dreimal abwägst, wenn dir das nächste mal jemand rät, mir zu misstrauen.“

„Es tut mir leid,“ sagte Harry. In diesem Moment fühlte er sich elend, er hatte im Prinzip gerade Gandalf zurückgewiesen und Dumbledores Freundlichkeit machte es nur noch schlimmer. „Ich hätte Ihnen nicht misstrauen sollen.“

„Leider, Harry, in dieser Welt…“ Der alte Zauberer schüttelte den Kopf. „Kann ich nicht einmal sagen, dass du unvernünftig warst. Du kanntest mich nicht. Und es ist wahr, dass es manche in Hogwarts gibt, denen du besser nicht vertrauen solltest. Vielleicht sogar manche, die du Freunde nennst.“

Harry schluckte. Das klang ziemlich unheilvoll. „Zum Beispiel?“

Dumbledore erhob sich von seinem Stuhl und nahm eines seiner Instrumente in Augenschein, eine Uhr mit acht Armen verschiedener Länge.

Nach einigen Augenblicken sprach der alte Zauberer wieder. „Er erscheint dir wahrscheinlich höchst charmant,“ sagte Dumbledore. „Höflich—zu dir zumindest. Redegewandt, vielleicht sogar schmeichelnd. Immer zur Stelle mit einer helfenden Hand, einem Gefallen, einem guten Rat—“

„Oh, \emph{Draco Malfoy!}“ sagte Harry, ziemlich erleichtert, dass es nicht jemand wie Hermine war. „Oh nein, nein nein nein, das haben Sie falsch verstanden, er verändert nicht mich, ich verändere ihn.“

Dumbledore erstarrte, während er das Zifferblatt beäugte. „Du tust \emph{was?}“

„Ich hole Draco Malfoy von der Dunklen Seite herüber,“ sagte Harry. „Sie wissen schon, mache ihn zu einem von den Guten.“

Dumbledore straffte sich und wandte sich Harry zu. Er trug einen der überraschtesten Gesichtsausdrücke zur Schau, die Harry jemals an irgendwem gesehen hatte, ganz zu schweigen davon, an jemanden mit einem langen, silbernen Bart. „Bist du sicher,“ sagte der alte Zauberer nach einem Moment, „dass er bereit ist, gerettet zu werden? Ich fürchte, dass was immer du Gutes in ihm zu sehen glaubst, nur Wunschdenken ist—oder schlimmer noch, Trug, ein Köder—“

„Ähm, unwahrscheinlich,“ sagte Harry. „Ich meine, wenn er versucht, so zu tun, als sei er einer von den Guten, ist er unglaublich schlecht darin. Es ist nicht so, dass Draco zu mir gekommen und ganz charmant gewesen wäre und ich entschieden hätte, er müsse irgendwo in seinem tiefsten Inneren einen guten Kern haben. Ich habe mir genau ihn ausgesucht, um ihn zur Umkehr zu bewegen, eben weil er der Erbe des Hauses Malfoy ist und wenn man eine Person aussuchen sollte, die zu retten sich lohnt, ist es offensichtlich er.“

Dumbledores linkes Auge zuckte. „Du hast vor, in Draco Malfoys Herz die Samen der Liebe und Freundlichkeit zu sähen, weil du erwartest, dass Malfoys Erbe dir von Nutzen sein wird?“

„Nicht nur \emph{mir!}“ sagte Harry empört. „Dem ganzen magischen Britannien, wenn das funktioniert! \emph{Und} er wird auch selbst ein glücklicheres und geistig gesünderes Leben haben! Sehen Sie, ich habe nicht die Zeit \emph{jeden} zur Abkehr von der Dunklen Seite zu bewegen und ich muss mich fragen, wo für das Licht am schnellsten der größte Vorteil herauszuholen ist—“

Dumbledore brach in Gelächter aus. Lachte viel stärker als Harry erwartet hätte, heulte fast. Es schien auf positive Art \emph{unwürdevoll}. Ein uralter, mächtiger Zauberer sollte auf tief dröhnende Weise glucksen, nicht vor Lachen keine Luft mehr bekommen. Harry war einmal vor Lachen buchstäblich von seinem Stuhl gefallen, als er den Film \emph{Die Marx Brothers im Krieg}**** gesehen hatte und so sehr lachte Dumbledore gerade.

„\emph{So} lustig ist es auch wieder nicht,“ sagte Harry nach einer Weile. Er zweifelte bereits wieder an Dumbledores geistiger Verfassung.

Dumbledore brachte sich mit sichtbarer Anstrengung wieder unter Kontrolle. „Ah, Harry, eines der Symptome der Krankheit namens Weisheit ist, dass man anfängt über Dinge zu lachen, die niemand anders komisch findet, denn wenn man weise ist, Harry, begreift man die Witze erst!“ Der alte Zauberer wischte sich Tränen aus den Augen. „Oh, mei. Oh, mei. Oft wird böser Wille Böses vereiteln*****, in der Tat so ist es.“

Harrys Gehirn brauchte einen Moment, die bekannten Worte einzuordnen… „Hey, das ist ein \emph{Tolkien}-Zitat! \emph{Gandalf} sagt das!“

„Theoden, eigentlich,“ sagte Dumbledore.

„Sie sind ein \emph{Muggelgeborener?}“ sagte Harry geschockt.

„Ich fürchte nicht,“ sagte Dumbledore und lächelte erneut. „Ich wurde siebzig Jahre vor der Veröffentlichung dieses Buches geboren, liebes Kind. Doch es scheint, dass meine muggelgeborenen Schüler in bestimmter Hinsicht ähnlich denken. Ich habe nicht weniger als zwanzig Kopien von \emph{Der Herr der Ringe} und drei Sätze von Tolkiens gesammelten Werken zusammengetragen und ich schätze jede einzelne von ihnen.“ Dumbledore zog seinen Zauberstab, reckte ihn empor und warf sich in Pose. „\emph{Du kannst nicht vorbei!} Wie sieht das aus?“

„Ah,“ sagte Harry, wobei sein Gehirn drohte, vollkommen auszusetzen, „ich denke, Ihnen fehlt ein Balrog.“ Und die pinken Pyjamas und der zerquetschte Pilzhut waren ebenfalls nicht hilfreich.

„Ich verstehe.“ Dumbledore seufzte und schob verdrießlich den Zauberstab in seinen Gürtel. „Ich fürchte in letzter Zeit hat es nicht allzu viele Balrogs in meinem Leben gegeben. Heutzutage gibt es für mich nur noch Besprechungen im Zaubergamot, wo ich verzweifelt darauf achten muss, dass ja keine Arbeit je erledigt wird und förmliche Abendessen, bei denen ausländische Politiker darum wetteifern, wer wohl der starrsinnigste Narr sein mag. Und Leuten gegenüber mysteriös zu erscheinen, Dinge zu wissen, die ich nicht wissen kann, kryptische Äußerungen zu machen, die erst im Nachhinein verstanden werden können und all die anderen kleinen Dinge, mit denen sich mächtige Zauberer die Zeit vertreiben, nachdem sie dem Teil des Musters, der ihnen das Helden-Dasein ermöglicht, den Rücken gekehrt haben. Apropos, Harry, ich habe da etwas, das ich dir geben wollte, etwas, das deinem Vater gehörte.“

„Haben Sie?“ sagte Harry. „Meine Güte, wer hätte das gedacht.“

„Ja, in der Tat,“ sagte Dumbledore. „Ich nehme an, es ist etwas vorhersehbar, nicht wahr?“ Sein Gesicht wurde ernst. „Nichts desto trotz…“

Dumbledore ging zu seinem Schreibtisch zurück und setzte sich, wobei er einen der Schubläden herauszog. Er griff mit beiden Armen hinein und zog, leicht angestrengt, ein ziemlich großes und schwer aussehendes Objekt aus dem Schubkasten, welches er dann mit einem gewaltigen Wumm auf seinem eichenen Schreibtisch platzierte.

„Dies,“ sagte Dumbledore, „war der Stein deines Vaters.“

Harry starrte ihn an. Er war hellgrau, ausgeblichen, unregelmäßig geformt, mit scharfen Kanten und in so ziemlich jeder Hinsicht ein unauffälliger, gewöhnlicher, großer, alter Stein. Dumbledore hatte ihn so platziert, dass er auf der breitest-verfügbaren Diagonale des Schreibtisches lag, dennoch wackelte er noch immer unsicher darauf herum.

Harry blickte auf. „Das ist ein Scherz, oder?“

„Ist es nicht,“ sagte Dumbledore, schüttelte den Kopf und blickte sehr ernst. „Ich holte dies aus den Ruinen von James und Lilys Heim in Godric's Hollow, wo ich auch dich fand und ich habe ihn von damals bis heute verwahrt, bis zu dem Tag, an dem ich in dir übergeben kann.“

In der Ansammlung von Hypothesen, die Harrys Modell der Welt ausmachten, gewann diejenige, dass Dumbledore verrückt war, rapide an Wahrscheinlichkeit. Aber es \emph{gab} noch immer einen erhebliches Ausmaß an Wahrscheinlichkeit für andere Alternativen… „Ähm, ist es ein \emph{magischer} Stein?“

„Nicht so weit ich weiß,“ sagte Dumbledore. „Aber ich rate dir mit größtmöglichem Nachdruck, in jederzeit nahe bei dir zu tragen.“

Alles klar. Dumbledore war \emph{wahrscheinlich} verrückt, aber wenn \emph{nicht…} nun, es wäre wirklich \emph{zu peinlich} in Schwierigkeiten zu geraten, weil man den Rat des unergründlichen alten Zauberers nicht beherzigt hatte. Das musste auf der Top 100 Liste der offensichtlichen Fehler so etwa Nr. 4 sein.

Harry trat näher und legte die Hände auf den Stein, versuchte einen Winkel zu finden, von dem aus er ihn anheben konnte, ohne sich zu schneiden. „Dann stecke ich ihn in meinen Beutel.“

Dumbledore runzelte die Stirn. „Das könnte nicht nah genug bei dir sein. Und was wenn dein Eselsfell-Beutel verloren geht oder gestohlen wird?“

„Sie denken, ich sollte einfach überall wo ich hin gehe einen großen Stein bei mir tragen?“

Dumbledore blickte Harry ernst an. „Das könnte sich als weise herausstellen.“

„Ah…“ sagte Harry. Er sah ziemlich schwer aus. „Ich könnte mir denken, die anderen Schüler werden mir vielleicht Fragen darüber stellen.“

„Sag ihnen, ich hätte dir gesagt, du sollst es tun,“ sagte Dumbledore. „Niemand wird das in Frage stellen, da sie alle denken, dass ich verrückt bin.“ Sein Gesichtsausdruck war noch immer vollkommen ernst.

„Ähm, um ehrlich zu sein, wenn sie rumlaufen und ihren Schülern sagen, sie sollen schwere Steine mit sich rumschleppen, kann ich irgendwie verstehen, wieso die Leute das denken.“

„Ah, Harry,“ sagte Dumbledore. Der alte Zauberer machte mit einer Hand eine ausladende Geste, die all die mysteriösen Gerätschaften im Raum einzuschließen schien. „Wenn wir jung sind, glauben wir, dass wir alles wüssten und daher glauben wir, dass wenn wir keine Erklärung für etwas sehen, auch keine existiert. Wenn wir älter sind, erkennen wir, dass das gesamte Universum nach einem Rythmus und aus einem Grund funktioniert, selbst dann, wenn wir ihn nicht kennen. Es ist nur unsere eigene Unwissenheit, die uns als Verrücktheit erscheint.“

„Die Realität gehorcht stets den Regeln,“ sagte Harry, „auch wenn wir die Regeln nicht kennen.“

„Exakt, Harry,“ sagte Dumbledore. „Das zu verstehen—und ich sehe, \emph{dass} du es verstehst—ist der Kern aller Weisheit.“

„Also… \emph{warum} genau muss ich diesen Stein tragen?“

„Ich kann mir, ehrlich gesagt, keinen Grund denken,“ sagte Dumbledore.

„… können Sie nicht.“

Dumbledore nickte. „Aber nur weil ich mir keinen Grund denken kann, heißt nicht, dass es keinen Grund \emph{gibt.}“

Die Gerätschaften tickten weiter.

„Okay,“ sagte Harry, „ich weiß nicht einmal, ob ich das wirklich sagen sollte, aber das ist einfach nicht der korrekte Weg mit unserer, zugegebenen, Unwissenheit was die Funktionsweise des Universums betrifft umzugehen.“

„Ist es nicht?“ sagte der alte Zauberer und wirkte überrascht und enttäuscht.

Harry hatte das Gefühl, diese Unterhaltung würde sich nicht zu seinen Gunsten entwickeln, doch er fuhr trotzdem fort. „Nein. Ich bin nicht einmal sicher ob diese Art Fehlschluss einen offiziellen Namen hat, aber wenn ich mir einen ausdenken sollte, wäre es 'die Hypothese bevorzugen' oder etwas in der Art. Wie kann ich das verständlich machen… ähm… nehmen Sie an, Sie hätten eine Million Schachteln und nur eine der Schachteln enthielte einen Diamanten. Und Sie hätten eine Kiste voller Diamanten-Detektoren und jeder Diamanten-Detektor ginge immer in Gegenwart eines Diamanten los und die Hälfte der Zeit bei Schachteln, die keinen Diamanten enthalten. Würden Sie mit zwanzig Detektoren alle Schachteln überprüfen, hätten Sie, statistisch gesehen, einen falschen und einen echten Kandidaten übrig. Und dann bräuchten Sie nur noch einen oder zwei weitere Detektoren, bevor Sie den einen echten Kandidaten heraushätten. Der Punkt ist, dass wenn es viele mögliche Antworten gibt, Sie die meisten Belege brauchen, nur um die richtige Hypothese zu \emph{finden}, unter Millionen von Möglichkeiten—um überhaupt erst auf sie aufmerksam zu werden. Die Menge an Beweisen, die man braucht, um zwischen zwei oder drei plausiblen Kandidaten zu wählen, ist im Vergleich dazu viel geringer. Wenn Sie also einfach ohne Beweise vorgreifen und einer bestimmten Möglichkeit ihre Aufmerksamkeit widmen, überspringen Sie den Großteil der Arbeit. So als würden Sie in einer Stadt leben, in der es eine Million Menschen gibt und dort hat es einen Mord gegeben und ein Ermittler sagt, nun, wir haben überhaupt keine Anhaltspunkte, also haben wir schon die Möglichkeit in Betracht gezogen, dass Mortimer Snodgrass es getan hat?“

„Hat er?“ sagte Dumbledore.

„Nein,“ sagte Harry. „Doch stellt sich später heraus, dass der Mörder schwarze Haare hatte und Mortimer hat schwarze Haare, denken alle, ah, sieht aus als wäre es wohl Mortimer gewesen. Deshalb ist es unfair Mortimer gegenüber, dass die Polizei \emph{ihm ihre Aufmerksamkeit widmet,} ohne bereits gute Gründe zu haben, ihn zu verdächtigen. Wenn es viele Möglichkeiten gibt, ist die meiste Arbeit nötig, nur um die richtige Antwort zu \emph{finden}—auf sie aufmerksam zu werden. Man braucht keine \emph{Beweise} oder offizielle Belege, wie Wissenschaftler oder Gerichte sie verlangen, aber man braucht irgendeine Art \emph{Hinweis} und dieser Hinweis muss diese spezielle Möglichkeit von den Millionen anderer unterscheiden. Ansonsten kann man die richtige Antwort nicht einfach so aus der Luft greifen. Man kann nicht einmal eine Möglichkeit, über die es wert ist nachzudenken, einfach so aus der Luft greifen. Und es muss noch eine Million anderer Dinge geben, die ich tun könnte, anstatt den Stein meines Vaters mit mir herumzutragen. Nur weil ich nichts über das Universum weiß, heißt das nicht, dass ich nicht wüsste, wie ich mich angesichts meiner Unsicherheit verhalten sollte. Die Regeln des Nachdenkens mit Wahrscheinlichkeiten stehen nicht weniger fest, als die Gesetze der guten alten Logik und was Sie gerade getan haben, ist \emph{nicht erlaubt.}“ Harry hielt inne. „\emph{Es sei denn,} natürlich, Sie hätten einen \emph{Hinweis}, den Sie nicht erwähnen.“

„Ah,“ sagte Dumbledore. Er tippte sich nachdenklich gegen die Wange. „Ein interessantes Argument, sicherlich, aber versagt es nicht an dem Punkt, wo du einen Vergleich ziehst zwischen einer Stadt voll potenzieller Mörder, von denen nur einer den Mord begangen hat und dem Wählen einer Handlungsweise aus einer Vielzahl von Möglichkeiten, wenn viele mögliche Handlungsweisen sich als weise herausstellen könnten? Ich sage nicht, dass den Stein deines Vaters zu tragen, die allerbeste mögliche Handlungsweise ist, nur dass es weiser ist, es zu tun als nicht.“

Dumbledore griff noch einmal in die selbe Schreibtisch-Schublade wie vorher, schien diesmal darin nach etwas zu kramen - zumindest schien sich sein Arm zu bewegen. „Ich möchte anmerken,“ sagte Dumbledore während Harry noch auszuknobeln versuchte, wie er auf diese völlig unerwartete Erwiderung reagieren sollte, „dass es ein verbreiteter Irrglaube unter Ravenclaws ist, dass alle schlauen Kinder dorthin sortiert werden und keine für die anderen Häuser übrig bleiben. Das stimmt so nicht; nach Ravenclaw sortiert zu werden zeigt an, dass man von Wissensdurst getrieben wird, was nicht ganz das selbe ist, wie intelligent zu sein.“ Der Zauberer lächelte, als er sich über den Schubladen beugte. „Nichtsdestotrotz scheinst du ziemlich intelligent zu \emph{sein}. Weniger wie ein junger Held und mehr wie ein junger mysteriöser uralter Zauberer. Ich glaube, ich habe bei dir vielleicht den falschen Ansatz gewählt, Harry, und dass du fähig sein könntest, Dinge zu verstehen, die nur wenige andere zu begreifen in der Lage wären. Daher sollte ich es wagen und dir noch ein bestimmtes \emph{anderes} Erbstück anvertrauen.“

„Sie meinen doch nicht…“ keuchte Harry. „Mein Vater… \emph{besaß einen weiteren Stein?}“

„Entschuldige bitte,“ sagte Dumbledore, „ich \emph{bin} immer noch älter und mysteriöser als du und wenn es um irgendwelche Offenbarungen geht, werde ich derjenige sein, der sie macht, danke sehr… oh, wo \emph{ist} das Ding!“ Dumbledore griff tiefer in die Schreibtisch-Schublade hinein und noch tiefer. Sein Kopf und seine Schultern und sein ganzer Oberkörper verschwanden darin, bis nur noch seine Hüften und Beine herausschauten, als würde der Schreibtisch ihn verschlingen.

Harry konnte nicht anders als sich zu fragen, wieviele Sachen darin sein mochten und wie das komplette Inventar wohl aussehen würde.

Schließlich tauchte Dumbledore wieder aus dem Schubladen auf mit dem Gegenstand seiner Suche, welchen er neben dem Stein auf dem Schreibtisch platzierte.

Es war ein gebrauchtes, eselsohriges Lehrbuch mit ausgeleiertem Einband: \emph{Zaubertränke für Fortgeschrittene}****** von Libatius Borage. Das Abbild eines rauchenden Glasfläschchens befand sich auf dem Einband.

„Dies,“ intonierte Dumbledore, „war das Zaubertränke-Lehrbuch deiner Mutter in ihrem fünften Schuljahr.“

„Welches ich allzeit bei mir tragen soll,“ sagte Harry.

„\emph{Welches ein schreckliches Geheimnis verbirgt.} Ein Geheimnis, dessen Enthüllung sich als so verhängnisvoll herausstellen könnte, dass ich dich bitten muss zu schwören - und ich muss darauf bestehen, dass du es aufrichtig schwörst, Harry, was immer du auch von all dem halten magst - es niemals irgendjemand oder irgendetwas anderem zu erzählen.“

Harry bedachte das Zaubertränke-Lehrbuch seiner Mutter aus ihrem fünften Schuljahr, welches, offenbar, ein schreckliches Geheimnis verbarg.

Das Problem war, dass Harry Eide wie diese sehr ernst \emph{nahm}. Jeder Schwur war ein Unbrechbarer Schwur, wenn er von der richtigen Person kam.

Und…

„Ich fühle mich durstig,“ sagte Harry, „und das ist überhaupt kein gutes Zeichen.“

Dumbledore versäumte vollkommen irgendwelche Fragen über diese kryptische Aussage zu stellen. „\emph{Schwörst} du, Harry?“ sagte Dumbledore.„ Seine Augen blickten eindringlich in Harrys. “Sonst kann ich es dir nicht sagen."

„Ja,“ sagte Harry. „Ich schwöre.“ Das war das Problem dabei ein Ravenclaw zu sein. Man konnte ein solches Angebot nicht ablehnen oder die Neugier verschlang einen bei lebendigem Leib und alle anderen wussten das.

„Und ich schwöre meinerseits,“ sagte Dumbledore, „dass alles, was ich dir erzählen werde, die Wahrheit ist.“

Dumbledore öffnete das Buch, scheinbar wahllos und Harry lehnte sich vor um es zu sehen.

„Siehst du diese Notizen,“ sagte Dumbledore mit so leiser Stimme, es war fast ein Flüstern, „die an den Rändern des Buches geschrieben stehen?“

Harry schielte ein wenig. Die verblichenen Seiten schienen etwas zu beschreiben, dass sich \emph{Trank der Majestät eines Adlers}******* nannte, viele der Zutaten waren Gegenstände, die Harry überhaupt nicht erkannte und deren Namen offenbar nicht aus dem Englischen stammten. An den Rand gekritzelt stand eine handschriftliche Notiz, die lautete \emph{Ich frage mich, was passiert, wenn man hier Thestral-Blut statt Blaubeeren verwenden würde?} und direkt darunter war eine Antwort in anderer Handschrift, \emph{Man würde wochenlang krank sein und vielleicht sterben.}

„Ich sehe sie, sagte Harry. “Was ist mit ihnen?"

Dumbledore deutete auf das zweite Gekritzel. „Diejenigen in dieser Handschrift,“ sagte er, noch immer mit der leisen Stimme, „wurden von deiner Mutter geschrieben. Und diejenigen in \emph{dieser} Handschrift,“ er bewegte seinen Finger, deutete auf das erste Gekritzel, „wurden von mir geschrieben. Ich habe mich unsichtbar gemacht und in ihren Schlafsaal geschlichen, während sie geschlafen hat. Lily dachte, eine ihrer Freundinnen hätte sie geschrieben und sie hatten die tollsten Streitereien darüber.“

Das war der exakte Moment, an dem Harry klar wurde, dass der Schulleiter von Hogwarts, tatsächlich, wahnsinnig \emph{war}.

Dumbledore sah ihn mit ernstem Gesichtsausdruck an. „Verstehst du die Implikationen dessen, was ich dir gerade erzählt habe, Harry?“

„Ähhh… sagte Harry. Seine Stimme schien ihm im Hals festzustecken. “Entschuldigung… ich… nicht wirklich…"

„Ah, nun ja,“ sagte Dumbledore seufzend. „Ich nehme an, dann hat deine Cleverness wohl doch ihre Grenzen. Sollen wir einfach so tun als hätte ich nichts gesagt?“

Harry erhob sich von seinem Stuhl, ein festgefrorenes Lächeln auf dem Gesicht. „Natürlich,“ sagte Harry. „Wissen Sie, es wird schon langsam spät und ich bin etwas hungrig, also sollte ich wohl wirklich besser runter zum Abendessen gehen“ und Harry sauste zur Tür.

Der Türknauf weigerte sich beharrlich sich zu drehen.

„Du verletzt mich, Harry,“ erklang Dumbledores Stimme mit leisen Tönen, die von direkt hinter ihm stammten. „Ist dir nicht wenigstens klar, dass was ich dir erzählt habe ein Zeichen des Vertrauens ist?“

Harry drehte sich langsam um.

Vor ihm stand ein sehr mächtiger und sehr wahnsinniger Zauberer mit einem langen silbernen Bart, einem Hut wie ein gigantischer zerquetschter Pilz und was für Muggelaugen aussah wie drei Schichten leuchtend pinker Pyjamas.

Hinter ihm befand sich eine Tür, die im Augenblick nicht zu funktionieren schien.

Dumbledore sah ziemlich traurig und erschöpft aus, als wolle er sich auf einen Zauberer-Stab lehnen, den er nicht hatte. „Also wirklich,“ sagte Dumbledore, „da versucht man einmal etwas anderes als dem selben Muster zu folgen wie seit einhundertundzehn Jahren und alle laufen gleich davon.“ Der alte Zauberer schüttelte sorgenvoll den Kopf. „Ich hatte besseres von dir erhofft, Harry Potter. Ich hatte gehört, dass deine eigenen Freunde dich ebenfalls für verrückt halten. Ich weiß, sie liegen falsch. Wirst du nicht das selbe von mir glauben?“

„Bitte öffnen Sie die Tür,“ sagte Harry mit zitternder Stimme. „Wenn Sie wollen, dass ich Ihnen jemals wieder vertraue, öffnen Sie die Tür.“

Das Geräusch einer sich öffnenden Tür erklang hinter ihm.

„Ich hatte vor, dir noch mehr Dinge zu erzählen,“ sagte Dumbledore, „und wenn du jetzt gehst, wirst du nicht erfahren welche.“

Manchmal \emph{hasste} Harry es wirklich, ein Ravenclaw zu sein.

\emph{Er hat noch niemals einen Schüler verletzt,} sagte Harrys innere Gryffindor-Seite. \emph{Denk einfach immer daran und du gerätst sicher nicht in Panik. Du wirst doch nicht weg rennen, nur weil die Dinge interessant werden, oder?}

\emph{Du kannst dich nicht einfach vom Schulleiter abwenden!} sagte sein Hufflepuff-Teil. \emph{Was wenn er anfängt, Hauspunkte abzuziehen? Er könnte dein Schulleben sehr unangenehm machen, wenn er entscheidet, dass er dich nicht mag!}

Und ein Teil seiner selbst, den Harry nicht besonders mochte aber nicht ganz zum Schweigen bringen konnte, dachte an die potenziellen Vorteile, einer der wenigen Freunde dieses verrückten alten Zauberers zu sein, der gleichzeitig auch noch Schulleiter, Großmeister und Ganz Hohes Tier war. Und unglücklicherweise schien sein innerer Slytherin sehr viel besser als Draco darin zu sein, Leute auf die Dunkle Seite hinüber zu ziehen, weil er Dinge sagte, wie \emph{armer Kerl, er sieht aus als würde er jemanden zum Reden brauchen, nicht wahr?} und \emph{du würdest doch nicht wollen, dass so ein mächtiger Mann am Ende jemand weniger tugendhaftem als dir vertraut, oder?} und \emph{ich frage mich, was für unglaubliche Geheimnisse Dumbledore dir verraten könnte, wenn, du weißt schon, ihr Freunde würdet und sogar ich wette er hat eine wiiirklich interessante Büchersammlung.}

\emph{Ihr seid eine Bande Geistesgestörter,} dachte Harry der ganzen Versammlung zu, aber er war einstimmig von jedem einzelnen Bestandteil seiner selbst überstimmt worden.

Harry drehte sich um, machte einen Schritt auf die offene Tür zu und betont schloss er sie wieder. Es war ein Opfer, das ihn nichts kostete, angesichts dessen, dass er sowieso bleiben würde und Dumbledore seine Bewegungsfreiheit ohnehin kontrollieren konnte, aber vielleicht würde es Dumbledore beeindrucken.

Als Harry sich wieder umdrehte sah er, dass der mächtige verrückte Zauberer wieder freundlich lächelte. Das war gut, vielleicht.

„Bitte tun Sie das nicht wieder,“ sagte Harry. „Ich mag es nicht, gefangen zu sein.“

„Das \emph{tut} mir Leid, Harry,“ sagte Dumbledore in aufrichtig entschuldigendem Tonfall. „Aber es wäre furchtbar unklug gewesen, dich ohne den Stein deines Vaters gehen zu lassen.“

„Natürlich,“ sagte Harry. „Es war unvernünftig von mir zu erwarten, dass die Tür sich öffnet bevor ich die Quest-Gegenstände in mein Inventar gepackt habe.“

Dumbledore lächelte und nickte.

Harry ging hinüber zum Schreibtisch, drehte seinen Eselsfell-Beutel zur Vorderseite seines Gürtels und schaffte es, mit einigem Aufwand, den Stein mit seinen Elfjährigen-Armen anzuheben und an ihn zu verfüttern.

Er konnte tatsächlich fühlen wie das Gewicht sich langsam verminderte als die sich weitende Öffnung den Stein in sich aufnahm und der Rülpser der folgte war ziemlich laut und hatte einen deutlich vorwurfsvollen Unterton.

Das Zaubertränke-Lehrbuch seiner Mutter aus ihrem fünften Jahr (das tatsächlich ein ziemlich schreckliches Geheimnis verbarg) folgte kurz darauf.

Und dann machte Harrys innerer Slytherin einen gerissenen Vorschlag, sich beim Schulleiter beliebt zu machen, welcher, unglücklicherweise, auf perfekte Weise so abgegeben worden war, dass er die Mehrheit der Ravenclaw-Fraktion für sich gewann.

„Also,“ sagte Harry. „Ähm. Ich nehme nicht an, dass Sie mir, da ich schon mal hier bin, vielleicht eine kleine Tour durch Ihr Büro geben möchten? Ich bin ein wenig neugierig was einige von diesen Dingen hier tun“ und das war für ihn die Untertreibung des Monats September.

Dumbledore sah ihn an und nickte dann mit leichtem Grinsen. „Ich bin geschmeichelt von deinem Interesse,“ sagte Dumbledore, „aber ich fürchte es gibt nicht viel zu sagen.“ Dumbledore machte einen Schritt näher auf die Wand zu und deutete auf ein Gemälde eines schlafenden Mannes. „Dies sind Porträts vergangener Schulleiter von Hogwarts.“ Er wandte sich um und deutete auf seinen Schreibtisch. „Dies ist mein Schreibtisch.“ Er zeigte auf seinen Stuhl. „Das ist mein Stuhl—“

„Entschuldigen Sie,“ sagte Harry, „eigentlich hatte ich die dort gemeint.“ Harry deutete auf einen kleinen Würfel, der leise flüsterte „blurps… blurps… blurps“.

„Oh, die kleinen fitzeligen Dinger?“ sagte Dumbledore. „Die kamen zusammen mit dem Schulleiter-Büro und ich habe absolut keine Ahnung was die meisten von ihnen tun. Obwohl \emph{diese} Uhr mit den acht Armen die Anzahl der, sagen wir mal Nieser, von linkshändigen Hexen innerhalb der französischen Grenzen zählt; du würdest nicht glauben was für ein Aufwand es war, das herauszufinden. Und \emph{dieses} hier mit den goldenen Wabbel-Dingern ist meine eigene Erfindung und Minerva wird nie im Leben herausfinden, was es tut.“

Dumbledore schritt hinüber zu dem Kleiderständer, während Harry das noch verarbeitete. „Hier haben wir natürlich den Sprechenden Hut, ich glaube ihr beide kennt euch bereits. Er hat mir gesagt, dass er niemals wieder auf deinem Kopf platziert werden solle unter egal welchen Umständen. Du bist erst der vierzehnte Schüler der Geschichte über den er das gesagt hat, Baba Yaga war noch eine und von den anderen zwölf erzähle ich dir, wenn du älter bist. Das ist ein Regenschirm. Das ist ein weiterer Regenschirm.“ Dumbledore machte noch ein paar Schritte und drehte sich um, er grinste jetzt ziemlich breit. „Und natürlich wollen die meisten Leute, die in mein Büro kommen, Fawkes sehen.“

Dumbledore stand neben dem Vogel auf der goldenen Plattform.

Harry kam heran, ziemlich verwirrt. „Das ist Fawkes?“

„Fawkes ist ein Phoenix,“ sagte Dumbledore. „Sehr seltene, sehr mächtige magische Kreaturen.“

„Ah…“ sagte Harry. Er senkte den Kopf und starrte in die winzigen schwarzen Knopfaugen, die nicht das kleinste Anzeichen von Macht oder Intelligenz aufwiesen.

„Ahhh…“ sagte Harry erneut.

Er war ziemlich sicher, dass er die Form des Vogels erkannte. Sie war ziemlich schwer zu verwechseln.

„Ähmm…“

\emph{Sag was intelligentes!} brüllte Harrys Geist auf sich selbst ein. \emph{Steh nicht einfach da und kling wie ein sabbernder Idiot!}

\emph{Na, was zum Teufel soll ich denn sagen?} feuerte Harrys Geist zurück.

\emph{Irgendwas!}

\emph{Du meinst, irgendwas außer „Fawkes ist ein Hühnchen“—}

\emph{Ja! Alles außer das!}

„Also, ah, welche Art von Magie besitzen Phoenixe denn?“

„Ihre Tränen haben die Kraft zu heilen,“ sagte Dumbledore. „Sie sind Wesen des Feuers und reisen zwischen allen Orten so leicht, wie Feuer sich an einem Ort auslöscht um sich an einem anderen neu zu entzünden. Die ungeheure Belastung der ihnen innewohnenden Magie lässt ihre Körper schnell altern und doch sind sie so nah an der Unsterblichkeit, wie irgendein lebendes Wesen in dieser Welt nur sein kann, da wann immer ihre Körper versagen, sie sich in einem Ausbruch von Flammen opfern und zurück bleibt ein Küken oder manchmal ein Ei.“ Dumbledore kam näher und betrachtete stirnrunzelnd das Hühnchen. „Hm… Sieht schon ein wenig blass aus, finde ich.“

Als Harrys Geist diese Aussage voll verarbeitet hatte, stand das Hühnchen bereits in Flammen.

Der Schnabel des Hühnchens öffnete sich, aber es hatte nicht einmal die Zeit für ein einzelnes Krächzen, bevor es versengt und verkohlt wurde. Das Feuer war kurz, intensiv und vollkommen in sich geschlossen; es gab keinerlei Brandgeruch.

Und dann erstarb das Feuer, nur Sekunden nachdem es begonnen hatte, zurück blieb nur ein erbärmliches Häufchen Asche auf der goldenen Plattform.

Schau nicht so entsetzt, Harry!„ sagte Dumbledore. “Fawkes ist nichts passiert.„ Dumbledores Hand tauchte in eine Tasche und dann fuhr die gleiche Hand durch die Asche und förderte ein kleines gelbliches Ei zutage. “Siehst du, hier ist ein Ei!"

„Oh… wow… umwerfend…“

„Aber jetzt sollten wir wirklich weitermachen,“ sagte Dumbledore. Er ließ das Ei in der Asche des Hühnchens liegen, kehrte zu seinem Thron zurück und setzte sich. „Es ist fast Zeit zum Abendessen und wir wollen doch nicht unsere Zeitumkehrer benutzen müssen.“

In der Regierung von Harry fand ein gewaltsamer Machtkampf statt. Slytherin und Hufflepuff hatten die Seiten gewechselt, nachdem sie gesehen hatten, wie der Schulleiter von Hogwarts ein Hühnchen in Brand steckte.

„Ja, weitermachen,“ sagten Harrys Lippen. „Und dann Abendessen.“

\emph{Du klingst schon wieder wie ein sabbernder Idiot} bemerkte Harrys Innerer Kritiker.

„Nun,“ sagte Dumbledore. „Ich fürchte ich habe ein Geständnis zu machen, Harry. Ein Geständnis und eine Entschuldigung.“

„Entschuldigungen sind gut“ \emph{das macht nicht einmal Sinn! Was rede ich denn da?}

Der alte Zauberer seufzte tief. „Du magst vielleicht nicht mehr so denken, wenn du erfährst, was ich zu sagen habe. Ich fürchte, Harry, dass ich dich dein ganzes Leben lang manipuliert habe. Ich war es, der dich der Obhut deiner bösen Stiefeltern übergeben hat—“

„Meine Stiefeltern sind nicht böse!“ entfuhr es Harry. „Meine \emph{Eltern}, meine ich!“

„Sind sie nicht?“ sagte Dumbledore überrascht und enttäuscht. „Nicht einmal ein wenig böse? Das passt nicht in's Muster…“

Harrys innerer Slytherin schrie geistig so laut er konnte, \emph{SEI STILL DU IDIOT, ER WIRD DICH IHNEN WEGNEHMEN!}

„Nein, nein,“ sagte Harry, die Lippen zu einer scheußlichen Grimasse erstarrt, „ich wollte nur Ihre Gefühle nicht verletzen, sie sind eigentlich sehr böse…“

„Sind Sie?“ Dumbledore lehnte sich vor, betrachtete ihn eindringlich. „Was tun Sie denn?“

\emph{Rede schnell} „sie, ah, ich muss das Geschirr und die Wäsche machen und sie lassen mich nicht viele Bücher lesen und—“

„Ah, gut, das ist gut zu hören,“ sagte Dumbledore und lehnte sich wieder zurück. Er lächelte auf traurige Weise. „Dann entschuldige ich mich \emph{dafür}. Nun, wo war ich? Ah, ja. Es tut mir leid, dir sagen zu müssen, Harry, dass ich buchstäblich für alles schlechte verantwortlich bin, das dir je im Leben widerfahren ist. Ich weiß, dass dich das wahrscheinlich sehr wütend machen wird.“

„Ja, ich bin sehr wütend!“ sagte Harry. „Grrr!“

Harrys Interner Kritiker verlieh ihm prompt den Preis für die Schlechteste Schauspielerische Leistung aller Zeiten.

„Und ich wollte nur, dass du weißt,“ sagte Dumbledore, „ich wollte dir so früh wie möglich sagen, falls später einem von uns etwas zustößt, dass es mir wirklich, wirklich leid tut. Für alles was bereits geschehen ist und alles, was noch kommen wird.“

Feucht glitzerte es in den Augen des alten Zauberers.

„Und ich bin sehr wütend!“ sagte Harry. „So wütend, dass ich sofort verschwinden will, es sei denn Sie haben noch irgendetwas anderes zu sagen!“

\emph{GEH einfach, bevor er dich in Brand steckt!} kreischten Slytherin, Hufflepuff und Gryffindor.

„Ich verstehe,“ sagte Dumbeldore. „Dann eine letzte Sache noch, Harry. Du solltest dich \emph{nicht} an der verbotenen Tür in dem Korridor im dritten Stock versuchen. Es gibt keine Möglichkeit, wie du durch all die Fallen gelangen könntest und ich würde ungern hören, dass du verletzt wurdest. Obwohl ich bezweifle, dass du auch nur die erste Tür würdest öffnen können, da sie verschlossen ist und du den Zauber \emph{Alohomora} nicht kennst—“

Harry fuhr herum und schoss mit Höchstgeschwindigkeit auf den Ausgang zu, der Türknauf drehte sich bereitwillig in seiner Hand und dann raste er die Wendeltreppe hinunter noch während sie rotierte, fast über seine eigenen Füße stolpernd, nur einen Augenblick später war er unten und der Wasserspeier schritt beiseite und Harry kam aus dem Treppenaufgang geschossen, wie eine Kanonenkugel.

\later

Harry Potter.

Irgendwas musste es auf sich haben mit Harry Potter.

Immerhin war es für jeden Donnerstag und doch schienen solche Sachen niemand anderem zu passieren.

Es war 6:21~Uhr am Donnerstagnachmittag als Harry Potter, wie eine Kanonenkugel aus dem Treppenaufgang geschossen kommend und auf Höchstgeschwindigkeit beschleunigend, direkt in Minerva McGonagall hinein rannte, als sie gerade auf ihrem Weg zum Büro des Schulleiters um eine Ecke bog.

Glücklicherweise hatte sich keiner von ihnen besonders weh getan. Wie Harry ein wenig früher am Tag erklärt worden war - als er sich geweigert hatte auch nur noch einmal in die Nähe eines Besenstiels zu gehen - benötigte man beim Quidditch solide eiserne Klatscher, um auch nur einigermaßen eine Chance zu haben, die Spieler zu verletzen, da Zauberer dazu neigten, sehr viel weniger anfällig für Zusammenprälle zu sein als Muggel.

Harry und Professor McGonagall fanden sich beide auf dem Boden wieder und die Pergamente, die sie getragen hatte, flatterten quer durch den Korridor.

Es gab eine furchtbare, furchtbare Pause.

„Harry Potter,“ keuchte Professor McGonagall genau neben Harry, wo sie auf dem Boden lag. Ihre Stimme steigerte sich beinahe zu einem Kreischen. „\emph{Was hatten Sie im Büro des Schulleiters zu suchen?}“

„Gar nichts!“ quiekte Harry.

„\emph{Haben Sie über den Verteidigungs-Professor gesprochen?}“

„Nein! Dumbledore hat mich dort hoch bestellt und er hat mir diesen großen Stein gegeben und gesagt, er gehörte meinem Vater und ich sollte ihn überall bei mir tragen!“

Es gab eine weitere schreckliche Pause.

„Ich verstehe,“ sagte Professor McGonagall, ihre Stimme etwas ruhiger. Sie stand auf, klopfte sich ab und warf den verstreuten Pergamenten einen bösen Blick zu, die zu einem ordentlichen Stapel zusammen sprangen und rückwärts gegen die Wand des Korridors wuselten, als wollten sie ihrem Blick entkommen. „Mein Beileid, Mr~Potter und entschuldigen Sie, dass ich an Ihnen gezweifelt habe.“

„Professor McGonagall,“ sagte Harry. Seine Stimme zitterte. Er drückte sich vom Boden hoch, stand und sah auf zu ihrem vertrauenerweckenden, \emph{vernünftigen} Gesicht. „Professor McGonagall…“

„Ja, Mr~Potter?“

„Denken Sie, ich sollte?“ sagte Harry kleinlaut. „Den Stein meines Vaters überall bei mir tragen?“

Professor McGonagall seufzte. „Das ist eine Sache zwischen Ihnen und dem Schulleiter fürchte ich.“ Sie zögerte. „Ich kann nur sagen, dass den Schulleiter vollkommen zu ignorieren fast niemals weise ist. Es tut mir \emph{leid} von Ihrem Dilemma zu hören, Mr~Potter und wenn es etwas gibt, das ich tun \emph{kann}, um Ihnen zu helfen mit was immer Sie zu tun entscheiden—“

„Ähm,“ sagte Harry. „Tatsächlich habe ich gedacht, dass sobald ich weiß wie, ich den Stein in einen Ring transfigurieren******** und an meinem Finger tragen könnte. Wenn Sie mir beibringen könnten, wie man eine Transfiguration aufrecht erhält—“

„Es ist gut, dass Sie mich zuerst gefragt haben,“ sagte Professor McGonagall, ihr Gesichtsausdruck wurde strenger. „Wenn Sie die Kontrolle über die Transfiguration verlieren würden, würde deren Umkehrung Ihnen den Finger abtrennen und wahrscheinlich Ihre Hand zerfetzen. Und in Ihrem Alter ist selbst ein Ring ein zu großes Ziel für Sie, um es unbegrenzt zu transfigurieren, ohne dass es erheblich an Ihrer Magie zehrt. Aber ich kann einen Ring für Sie schmieden lassen, mit einer Einfassung für ein Juwel, ein \emph{kleines} Juwel, dass mit Ihrer Haut in Kontakt bleibt und Sie können das Aufrechterhalten mit einem sicheren Objekt, wie einem Marshmallow üben. Wenn Sie es einen ganzen Monat lang schaffen, selbst im Schlaf, werde ich Ihnen das Transfigurieren des, ah, Steins Ihres Vaters gestatten…“ Professor McGonagalls Stimme verklang. „Hat der Schulleiter \emph{wirklich—} “

„Ja. Ah… ähm…“

Professor McGonagall seufzte. „Das ist etwas seltsam, selbst für ihn.“ Sie bückte sich und hob den Stapel Pergamente auf. „Das tut mir leid, Mr~Potter. Ich entschuldige mich erneut, dass ich Ihnen misstraut habe. Aber nun muss ich den Schulleiter aufsuchen.“

„Ah… viel Glück, denke ich. Äh…“

„Danke, Mr~Potter.“

„Ähm…“

Professor McGonagall ging hinüber zu dem Wasserspeier, sprach unhörbar das Passwort und schritt voran auf die rotierende Wendeltreppe. Sie geriet nach oben außer Sicht und der Wasserspeier bewegte sich zurück—

„\emph{Professor McGonagall, der Schulleiter hat ein Hühnchen in Brand gesteckt!}“

„Er hat \emph{wa—}“

* Harry bezieht sich hier auf das \emph{P-NP-Problem}, eines der \emph{Millenium-Probleme} und wichtigsten ungelösten Probleme der Informatik. Einige Ausführungen über die Bedeutung einer Lösung des Problems macht Harry selbst. Wer sich weiter dafür interessiert, kann mit dem gleichnamigen Wikipedia-Artikel beginnen, es ist für das Verständnis der Geschichte aber nicht relevant.

** Auch \emph{Snape explodiert} (engl.: \emph{Exploding Snap}), laut Harry-Potter-Wiki ein Kartenspiel, dessen Regeln denen von Mau-Mau ähneln.

*** Ja, hat er wirklich gerade gesagt (engl.: \emph{Heh}, kurz für \emph{Headmaster}).

**** engl.: \emph{Duck Soup,} eine Slapstick-Komödie aus den 1930er Jahren.

***** engl.: \emph{Oft evil will shall evil mar.} Ein Zitat aus \emph{Der Herr der Ringe—Die Zwei Türme,} offenbar nur im Roman vorhanden; ausgesprochen, nachdem Pippin durch eine Unachtsamkeit des Feindes unbeabsichtigt Kenntis von Saurons Plänen erlangt.

****** engl.: \emph{Intermediate Potion-Making,} meine Übersetzung stimmt nicht damit überein; aus dem Kontext schloss ich, dass der Autor eher das Buch \emph{Advanced Potion-Making} aus Band 6 der ursprünglichen Romane gemeint haben dürfte, daher habe ich dessen deutsche Bezeichnung gewählt.

******* engl.: \emph{potion of eagle's splendour,} kommt offenbar in den ursprünglichen Romanen nicht vor und ist nicht relevant für die Geschichte.

******** Für diejenigen, die Professor McGonagalls Fach noch als \emph{Verwandlung} kennen: Ich habe beschlossen in diesem Fall von der Übersetzung der Original-Romane (mit den Begriffen \emph{Verwandlung} und \emph{verwandeln}) abzuweichen und stattdessen die näher am englischen Original liegenden Begriffe \emph{Transfiguration} und \emph{transfigurieren} zu verwenden, zumindest in soweit die Rede von \emph{freier Transfiguration} nach der Definition von Professor McGonagall ist, da sie zumindest meiner Ansicht nach „wissenschaftlicher“ klingen und der gesteigerten Bedeutung für die weitere Handlung, die der Autor dem Fach zukommen lässt, besser Ausdruck verleihen. Die früheren Kapitel werde ich rückwirkend entsprechend anpassen.

