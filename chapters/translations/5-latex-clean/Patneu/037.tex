

\hypertarget{zwischenspiel-grenzen-uxfcberschreiten}{% \section{37. Zwischenspiel: Grenzen überschreiten}\label{zwischenspiel-grenzen-uxfcberschreiten}}

\textbf{Kapitel 37: Zwischenspiel: Grenzen überschreiten}

Es war beinahe Mitternacht.

Lange auf zu bleiben stellte für Harry kein Problem dar. Er hatte einfach den Zeitumkehrer nicht verwendet. Harry folgte einer eigenen Tradition, seinen Schlafzyklus so zu abzustimmen, dass er ganz sicher wach sein würde, wenn der Weihnachtsabend in den Weihnachtsmorgen überging; denn wenn er auch niemals jung genug gewesen sein mochte, an den Weihnachtsmann zu \emph{glauben,} so war er doch zumindest einmal jung genug gewesen, um zu zweifeln.

Immerhin wäre es doch nett gewesen, wenn \emph{tatsächlich} eine mysteriöse Gestalt existierte, die des Nachts ins Haus kam und Geschenke brachte…

Ein Schauer lief Harry über den Rücken.

Etwas wie die Ahnung drohenden Unheils.

Ein schleichender Schrecken.

Ein Gefühl der Verdammnis.

Kerzengerade saß Harry in seinem Bett.

Er blickte zum Fenster.

"\emph{Professor Quirrell?}" entfuhr es Harry leise.

Professor Quirrell vollführte eine leichte Aufwärtsgeste und Harrys Fenster schien sich in seinem Rahmen zusammen zu falten. Sofort trieb ein kalter Hauch von Winter durch den Spalt ins Zimmer, zusammen mit ein paar verstreuten Schneeflocken aus einem Himmel, an dem vereinzelt graue Wolken inmitten der Schwärze und der Sterne hingen.

"Keine Sorge, Mr~Potter," sagte der Verteidigungsprofessor in gemessenem Ton. "Ich habe ihren Eltern einen Schlafzauber auferlegt; bis zu meiner Abreise werden sie nicht erwachen."

"Es sollte doch niemand wissen wo ich bin!" sagte Harry noch immer leise. "Selbst Eulen sollten meine Post nach Hogwarts liefern, nicht hierher!" Dem hatte Harry nur von ganzem Herzen beipflichten können; es wäre einfach dämlich, könnte ein Todesser den Krieg zu jeder Zeit gewinnen, indem er ihm einfach eine magisch ausgelöste Handgranate schickte.

Professor Quirrell grinste an seinem Platz dort hinter jenem Fenster. "Oh, da mache ich mir keine Sorgen, Mr~Potter. Gegen Ortungszauber \emph{sind} Sie sehr gut geschützt und wahrscheinlich würde es keinem Blutreinheits-Verfechter einfallen, einfach ein Telefonbuch zu konsultieren." Sein Grinsen wurde breiter. "Außerdem war ein beträchtlicher Aufwand erforderlich, die Schutzzauber zu durchqueren, die der Schulleiter um Ihr Haus gelegt hat - obwohl natürlich ein jeder, der Ihre Adresse kennt, einfach draußen warten und Sie angreifen könnte, sobald Sie das nächste Mal das Haus verlassen."

Harry starrte Professor Quirrell eine Weile an. "Was \emph{machen} Sie hier?" sagte Harry schließlich.

Das Lächeln wich aus Professor Quirrells Gesicht. "Ich komme, mich zu entschuldigen, Mr~Potter," sagte der Verteidigungsprofessor leise. "Ich hätte mit Ihnen nicht so streng ins Gericht gehen dürfen, wie ich es -"

"Nicht," sagte Harry. Er blickte auf die Decke hinab, die er um seinen Pyjama geschlungen hatte. "Lassen Sie es einfach gut sein."

"Habe ich Sie so sehr gekränkt?" sagte Professor Quirrells leise Stimme.

"Nein," sagte Harry. "Doch das werden Sie, wenn Sie sich jetzt entschuldigen."

"Ich verstehe," sagte Professor Quirrell und umgehend bekam seine Stimme einen ernsten Klang. "So ich Sie denn als Ebenbürtigen behandeln soll, Mr~Potter, dann muss ich Ihnen sagen, dass Sie in schlimmstem Maße die Umgangsformen verletzt haben, die befreundete Slytherins untereinander pflegen. Falls sie momentan das Spiel nicht gegen einen anderen spielen, so dürfen Sie sich \emph{niemals} in solcher Art in seine Pläne einmischen, nicht ohne \emph{vorher} zu fragen. Denn Sie können nicht wissen, wie diese Pläne letztlich tatsächlich aussehen oder welcher Einsatz wohl auf dem Spiel stehen mag. Durch eine solche Aktion gäben Sie sich als jemandes Feind zu erkennen, Mr~Potter."

"Es tut mir leid," sagte Harry im selben leisen Tonfall, den zuvor Professor Quirrell angeschlagen hatte.

"Entschuldigung akzeptiert," sagte Professor Quirrell.

"Aber," sagte Harry, noch immer leise, "was Politik angeht, müssen wir beide uns bei Gelegenheit wirklich noch einmal unterhalten."

Professor Quirrell seufzte. "Ich weiß, wie sehr Sie Herablassung verachten, Mr~Potter -"

Das war mal eine Untertreibung.

"Doch noch herablassender wäre es," sagte Professor Quirrell, "wenn ich es Ihnen nicht ganz deutlich sagen würde. Es mangelt Ihnen schlicht an Lebenserfahrung, Mr~Potter."

"Und stimmt Ihnen somit ein jeder mit ausreichender Lebenserfahrung zu?" erwiderte Harry ruhig.

"Was nützt schon Lebenserfahrung jemandem, der Quidditch spielt?" sagte Professor Quirrell und zuckte mit den Schultern. "Ich glaube, Sie werden Ihre Ansicht bezeiten noch ändern, wenn ein jedes Vertrauen, dass Sie gesetzt haben, enttäuscht wurde und Sie zynisch geworden sind."

Der Verteidigungsprofessor sagte es, als sei es das Selbstverständlichste auf der Welt, wie er da so vor dem Hintergrund des schwarzen, wolkenbefleckten Nachthimmels und der Sterne stand und zwei winzige Schneeflocken in der beißenden Winterluft hinter ihm vorbeitrieben.

"Dabei fällt mir ein," sagte Harry. "Frohe Weihnachten."

"So ist es wohl," sagte Professor Quirrell. "Immerhin, so dies denn letztlich \emph{keine} Entschuldigung wird, dann muss es wohl ein Weihnachtsgeschenk sein. Tatsächlich das allererste, das ich je überreicht habe."

Harry hatte bisher noch nicht einmal angefangen, Latein zu erlernen, damit er das Forschungstagebuch von Roger Bacon zu lesen vermochte und er wagte kaum, die Frage zu stellen.

"Ziehen Sie Ihren Wintermantel an," sagte Professor Quirrell, "oder nehmen Sie einen Wärmetrank, so Sie einen haben und dann treffen Sie mich draußen, unter den Sternen. Ich werde versuchen, ob ich es diesmal ein wenig länger aufrecht erhalten kann."

Es dauerte einen Moment, bis die Worte zu Harry durchgedrungen waren, dann schoss er auch schon in Richtung seines Wandschranks.

Diesmal hielt Professor Quirrell den Sternenlicht-Zauber mehr als eine Stunde lang aufrecht, obwohl sich das Gesicht des Verteidigungsprofessors vor Anstrengung verzog und er sich für eine Weile setzen musste. Nur einmal setzte Harry zu einem Einwand an und wurde sofort zum Schweigen gebracht.

Und so überschritten sie die Grenze zwischen Weihnachtsabend und Weihnachtstag in jeder zeitlosen Leere, in der das irdische Treiben bedeutungslos war, dieser einzig wahren, immerwährenden Stillen Nacht.

Und wie versprochen, so blieben Harrys Eltern die ganze Zeit lang in tiefem Schlaf, bis Harry sicher wieder in seinem Zimmer und der Verteidigungsprofessor verschwunden war.

