

\hypertarget{glauben-an-den-glauben}{% \section{23. Glauben an den Glauben}\label{glauben-an-den-glauben}}

\textbf{Kapitel 23: Glauben an den Glauben\\ }

Everybody wants a rock to wind a piece of string around J. K. Rowling.

--------------------------------------------------------------------------------------------------------------------------------------------

"Und dann noch Janet, die ein Squib war," sagte das Porträt einer kleinen jungen Frau mit gold-getrimmtem Hut.

Draco schrieb es auf. Das waren nur achtundzwanzig, doch es war Zeit zurückzukehren und Harry zu treffen.

Er hatte andere Porträts um Hilfe bei der Übersetzung bitten müssen, das Englische hatte sich sehr verändert, doch die ältesten Porträts hatten Zauber für Erstklässler beschrieben, die sich verdammt genau nach denen anhörten, die sie jetzt benutzten. Draco hatte etwa die Hälfte von ihnen erkannt und die andere Hälfte klang auch nicht viel mächtiger.

Das saure Gefühl in seinem Magen war mit jeder Antwort stärker geworden, bis er es schließlich nicht mehr ausgehalten und sich daran gemacht hatte, stattdessen anderen Porträts Harry Potters seltsame Fragen über Squib-Ehen zu stellen. Die ersten fünf Porträts hatten überhaupt niemanden gekannt und schließlich hatte er diese Porträts gebeten, \emph{ihre} Bekanntschaften zu bitten, \emph{ihre} Bekanntschaften zu fragen und so ein paar Leute gefunden, die tatsächlich zugaben, mit Squibs befreundet zu sein.

(Der Slytherin-Erstklässler hatte erklärt, er arbeite an einem wichtigen Projekt mit einem Ravenclaw und der Ravenclaw habe ihm gesagt, sie bräuchten diese Informationen und hatte sich dann aus dem Staub gemacht, ohne zu erklären wieso. Das hatte ihm viele mitfühlende Blicke eingebracht.)

Dracos Füße waren schwer als er durch die Korridore von Hogwarts ging. Er hätte rennen sollen, doch er konnte die Energie einfach nicht aufbringen. Er dachte immer wieder, dass er nichts davon wissen wollte, er nichts damit zu tun haben wollte, er nicht die Verantwortung dafür tragen wollte, es einfach Harry Potter tun lassen, wenn die Magie verschwand, sollte Harry Potter sich darum kümmern…

Doch Draco wusste, das war nicht richtig.

Kühl die Verliese von Slytherin, grau die steinernen Mauern, Draco mochte die Atmosphäre normalerweise, doch jetzt erinnerte sie ihn zu sehr an das Verschwinden.

Seine Hand auf dem Türknauf, war Harry Potter bereits dort und wartete in seinem Kapuzenmantel.

"Die antiken Erstklässler-Zauber," sagte Harry Potter. "Was hast du erfahren?"

"Sie sind nicht stärker als die Zauber, die wir jetzt benutzen."

Harry Potters Faust fuhr schwer auf ein Schreibpult nieder. "Verdammt. Alles klar. Mein eigenes Experiment war ein Fehlschlag, Draco. Es existiert etwas namens Interdikt von Merlin -"

Draco schlug sich gegen die Stirn, als es ihm einfiel.

"- das jeden davon abhält, wissen über mächtige Zauber aus Büchern zu erlangen, selbst wenn man die Aufzeichnungen eines mächtigen Zauberers findet und liest, ergeben sie für einen keinen Sinn, es muss von einem lebenden Geist zum anderen gehen. Ich konnte keine mächtigen Zauber finden, für die wir die Anweisungen hätten, die wir aber nicht wirken könnten. Doch wenn man sie nicht aus alten Büchern beziehen kann, warum sollte sich dann jemand die Mühe machen, sie mündlich weiter zu verbreiten, nachdem sie nicht mehr funktionieren? Hast du die Daten über die Squib-Paare?"

Draco wollte ihm das Pergament reichen -

Doch Harry Potter hob abwehrend die Hand. "Gesetz der Wissenschaft, Draco. Zuerst erzähle ich dir von der Theorie und der Vorhersage. Dann zeigst du mir die Daten. So weißt du, dass ich mir nicht einfach eine passende Theorie ausdenke; du weißt, dass die Theorie tatsächlich die Daten \emph{vorhergesagt} hat. Ich werde es dir ohnehin erklären müssen, also muss ich es tun, \emph{bevor} du mir die Daten zeigst. So lautet die Regel. Also leg deinen Mantel an und setzen wir uns."

Harry Potter setzte sich an ein Pult, auf dem zerrissene Papierfetzen zurechtgelegt waren. Draco nahm den Mantel aus seiner Schultasche, zog ihn an und setzte sich Harry gegenüber, wobei er einen verwirrten Blick auf die Papier-Schnipsel warf. Sie waren in zwei Reihen angeordnet und die Reihen waren etwa zwanzig Schnipsel lang.

"Das Geheimnis des Blutes," sagte Harry Potter, "ist etwas namens Desoxyribonukleinsäure. Du wirst diesen Begriff niemandem gegenüber erwähnen, der kein Wissenschaftler ist. Desoxyribonukleinsäure ist das Rezept, das deinem Körper vorgibt, wie er zu wachsen hat, zwei Beine, zwei Arme, klein oder groß, ob du braune Augen hast oder grüne. Es ist etwas materielles, man kann es \emph{sehen,} wenn man ein Mikroskop hat, das wie ein Teleskop ist, nur dass es Dinge zeigt, die sehr klein sind, anstatt sehr weit weg. Und dieses Rezept enthält zwei Kopien von allem, falls eine Kopie defekt ist. Stell dir zwei lange Reihen von Papier-Stückchen vor. An jedem Punkt in der Reihe gibt es zwei Stückchen Papier und wenn du Kinder hast, wählt dein Körper an jedem Punkt der Reihe ein Stückchen Papier zufällig aus und der Körper der Mutter macht dasselbe und so bekommt das Kind auch zwei Stückchen Papier an jedem Punkt in der Reihe. Zwei Kopien von allem, eine von deiner Mutter, eine von deinem Vater und wenn du Kinder hast, bekommen sie ein Stückchen Papier von dir an jeden Punkt."

Während Harry sprach, fuhren seine Finger über die paarweisen Stückchen Papier, deuteten auf den einen Teil des Paares, als er sagte "von deiner Mutter", auf den anderen, als er sagte "von deinem Vater". Und als er davon sprach, ein Stückchen Papier zufällig auszuwählen, zog seine Hand einen Knut aus seinem Umhang und warf ihn; Harry besah sich die Münze und deutete dann auf das obere Stückchen Papier. Alles ohne in seiner Ansprache inne zu halten.

"Wenn es nun etwa darum geht, klein oder groß zu sein, gibt es \emph{viele} Punkte im Rezept, die \emph{kleine} Unterschiede machen. Wenn also ein großer Vater eine kleine Mutter heiratet, bekommt das Kind einige Stückchen Papier, die sagen 'groß' und einige, die sagen 'klein' und üblicherweise wird das Kind dann mittlere Größe haben. Aber nicht immer. Durch Glück könnte das Kind viele Stückchen kriegen, die 'groß' sagen und nicht viele Papierchen, die 'klein' sagen und ziemlich groß werden. Man könnte auch einen großen Vater mit fünf Papierchen haben, die 'groß' sagen und eine große Mutter mit fünf Papierchen, die 'groß' sagen und durch unfassbares Glück bekommt das Kind \emph{alle zehn} Papierchen, die 'groß' sagen und wird größer als sie beide. Verstehst du? Blut ist keine perfekte Flüssigkeit, sie vermischt sich nicht perfekt. Desoxyribonukleinsäure besteht aus vielen kleinen Teilen, wie ein Glas voll mit Kieseln statt einem Glas voll Wasser. Deswegen liegt ein Kind nicht immer genau mittig zwischen seinen Eltern."

Draco lauschte mit offenem Mund. Wie in Merlins Namen hatten die Muggel all das herausgefunden? Konnten sie das Rezept \emph{sehen?}

"Jetzt," sagte Harry Potter, "nimm einmal an, es gibt, wie bei der Größe, eine Menge kleiner Punkte im Rezept, an denen man ein Stückchen Papier haben kann, dass 'magisch' oder 'nicht magisch' sagt. Wenn du genug Papier-Stückchen hast, die 'magisch' sagen, bist du ein Zauberer, wenn du \emph{viele} Papier-Stückchen hast, bist du ein mächtiger Zauberer, hast du zu wenige, bis du ein Muggel und dazwischen ein Squib. Wenn nun zwei Squibs heiraten, sollten die Kinder in den meisten Fällen ebenfalls Squibs sein, doch ab und an wird ein Kind Glück haben und die meisten der magischen Papierchen seines Vaters \emph{und} die meisten der magischen Papierchen seiner Mutter bekommen und stark genug sein, um ein Zauberer zu werden. Doch wahrscheinlich kein sehr mächtiger. Wenn man mit vielen mächtigen Zauberern anfinge und sie nur einander heirateten, würden sie mächtig bleiben. Doch wenn sie anfingen, Muggelgeborene zu heiraten, die kaum magisch sind oder Squibs… verstehst du? Das Blut würde sich nicht perfekt vermischen, es wäre ein Glas voller Kiesel, kein Glas voll Wasser, denn genau so funktioniert Blut. Es gäbe noch immer ab und an mächtige Zauberer, wenn sie durch Glück viele magische Papierchen bekommen. Doch sie wären nicht so mächtig, wie die mächtigsten Zauberer aus früheren Zeiten."

Draco nickte langsam. Er hatte es noch nie zuvor auf diese Art erklärt gehört. Es lag eine überraschende Schönheit darin, wie perfekt es zusammenpasste.

"\emph{Aber,}" sagte Harry. "Das ist nur \emph{eine} Hypothese. Angenommen, es gibt stattdessen nur einen \emph{einzigen} Punkt im Rezept, der dich zu einem Zauberer macht. Nur \emph{einen} Punkt, an dem ein Stückchen Papier 'magisch' oder 'nicht magisch' sagen kann. Und es gibt zwei Kopien von allem, immer. Dann gibt es nur drei Möglichkeiten. Beide Kopien können 'magisch' sagen. Eine Kopie kann 'magisch' sagen und eine Kopie 'nicht magisch'. Oder beide Kopien können 'nicht magisch' sagen. Zauberer, Squibs und Muggel. Zwei Kopien und du kannst Zauber wirken, eine Kopie und du kannst noch immer Zaubertränke oder magische Gerätschaften verwenden und null Kopien bedeuten, es könnte dir schwer fallen, Magie auch nur direkt anzuschauen. Muggelgeborene würden nicht wirklich von Muggeln geboren, sie würden von zwei Squibs geboren, zwei Eltern mit je einer magischen Kopie, die in der Muggelwelt aufgewachsen sind. Nun stell dir vor, eine Hexe heiratet einen Squib. Jedes Kind wird ein Papierchen von der Mutter bekommen, das 'magisch' sagt, immer, es spielt keine Rolle welches Stückchen zufällig gewählt wird, beide sagen 'magisch'. Doch als würde man eine Münze werfen, wird das Kind in der Hälfte der Fälle ein Papierchen, das 'magisch' sagt, vom Vater erhalten und in der anderen Hälfte das Papierchen des Vaters, das 'nicht magisch' sagt. Wenn eine Hexe einen Squib heiratet, wird das Ergebnis kein Haufen schwacher zaubernder Kinder sein. Die Hälfte der Kinder werden genauso mächtige Zauberer und Hexen sein, wie ihre Mutter und die andere Hälfte werden Squibs sein. Denn wenn es nur \emph{einen} Punkt im Rezept gibt, der dich zum Zauberer macht, dann ist die Magie nicht wie ein Glas voller Kiesel, die sich vermischen können. Sondern wie ein einzelner magischer Kiesel, ein Stein der Magier."*

Harry arrangierte drei Paare von Papierchen nebeneinander. Auf ein Paar schrieb er 'magisch' und 'magisch'. Bei einem weiteren schrieb er nur auf das obere Papierchen 'magisch'. Und das dritte Paar ließ er leer.

"In welchem Fall," sagte Harry, "man entweder zwei Steine hat oder nicht. Entweder ist man ein Zauberer oder nicht. Mächtige Zauberer würden dann durch härteres Studium und mehr Übung zu dem werden, was sie sind. Und wenn Zauberer \emph{an sich} weniger mächtig werden, nicht weil Zauber verloren gehen, sondern man sie nicht wirken kann… dann essen sie vielleicht die falsche Nahrung oder sowas. Doch wenn es seit achthundert Jahren stetig schlimmer wurde, dann könnte das bedeuten, dass die Magie selbst aus der Welt verschwindet."

Harry arrangierte ein weiteres Paar Papierchen Seite an Seite und nahm die Schreibfeder zur Hand. Bald darauf hatte jedes Paar ein Stückchen Papier auf dem 'magisch' stand und ein weiteres leeres Papierchen.

"Und das bringt mich zu der Vorhersage," sagte Harry. "Was passiert, wenn zwei Squibs heiraten. Wirf eine Münze zweimal. Sie kann Kopf und Kopf, Kopf und Zahl, Zahl und Kopf oder Zahl und Zahl zeigen. Ein Viertel der Zeit bekommt man also zweimal Kopf, ein Viertel der Zeit zweimal Zahl und die Hälfte der Zeit einmal Kopf und einmal Zahl. Genauso wenn zwei Squibs heiraten. Ein Viertel der Kinder würden magisch und magisch bekommen und Zauberer sein. Ein Viertel würde nicht magisch und nicht magisch bekommen und Muggel sein. Die andere Hälfte wären Squibs. Es ist ein sehr altes und klassisches Muster. Es wurde vom unvergessenen Gregor Mendel entdeckt und war der erste jemals entdeckte Hinweis darauf, wie das Rezept funktioniert. Jeder, der irgendetwas über Blut-Wissenschaft weiß, würde dieses Muster sofort wiedererkennen. Es wäre nicht exakt, genauso wenig, wie du, wenn du ein Paar Münzen vierzigmal wirfst, immer exakt zehn Paare mit zweimal Kopf bekommen wirst. Doch wenn es sieben oder dreizehn Zauberer aus vierzig Kindern sind, deutet es stark darauf hin. Das ist der Test, den ich dich habe machen lassen. Jetzt schauen wir uns deine Daten an."

Und bevor Draco auch nur darüber nachdenken konnte, hatte Harry Potter ihm das Pergament aus der Hand genommen.

Dracos Kehle war sehr trocken.

Achtundzwanzig Kinder.

Er war sich nicht sicher, was die exakte Anzahl betraf, doch er war ziemlich sicher, dass etwa ein Viertel Zauberer gewesen waren.

"Sechs Zauberer aus achtundzwanzig Kindern," sagte Harry Potter nach einem Augenblick. "Nun, das wäre das. Und Erstklässler haben vor acht Jahrhunderten auch die selben Zauber auf dem gleichen Level gewirkt wie wir. Dein Test und mein Test stimmen beide überein."

Eine lange Stille entstand im Klassenraum.

"Was nun?" flüsterte Draco.

Er war noch nie zuvor so angsterfüllt gewesen.

"Es ist noch nicht endgültig," sagte Harry Potter. "Mein Experiment war schlug fehl, erinnerst du dich? Du musst für mich einen weiteren Test entwerfen, Draco."

"Ich, ich…" sagte Draco. Seine Stimme versagte. "Ich schaffe das nicht Harry, das ist zu viel für mich."

Harrys Blick war unerbittlich. "Doch du kannst, weil du es musst. Ich habe mir auch selbst Gedanken darüber gemacht, nachdem ich auf das Interdikt von Merlin gestoßen bin. Draco, gibt es irgendeine Möglichkeit, die Stärke der Magie direkt zu beobachten? Etwas, das nichts mit dem Blut von Zauberern oder den Zaubern die wir lernen, zu tun hat?"

Dracos Geist war vollkommen leer.

"Alles was die Magie beeinflusst, beeinflusst auch Zauberer," sagte Harry. "Doch dann können wir nicht feststellen, ob es an den Zauberern oder der Magie liegt. Was wird von Magie beeinflusst, das \emph{kein} Zauberer ist?"

"Magische Geschöpfe, offensichtlich," sagte Draco ohne nachzudenken.

Harry Potter lächelte langsam. "Draco, das ist \emph{brillant.}"

\emph{Das ist die Art dumme Frage, die man überhaupt nur stellen würde, wenn man von Muggeln aufgezogen worden ist.}

Dann wurde das saure Gefühl in Dracos Magen sogar noch schlimmer, als ihm klar wurde, was es bedeuten würde, wenn magische Geschöpfe schwächer \emph{wurden.} Dann konnten sie sicher sagen, dass die Magie verschwand und ein Teil von Draco war bereits sicher, dass es genau das war, was sie feststellen würden. Er wollte das nicht sehen, er wollte es nicht wissen…

Harry Potter war schon halb aus der Tür. "\emph{Komm schon,} Draco! Es gibt ein Porträt nicht weit von hier, wir werden sie einfach nach jemand altem fragen und es gleich herausfinden! Wir tragen die Kapuzen, wenn uns jemand sieht, können wir einfach weglaufen! Na los!"

--------------------------------------------------------------------------------------------------------------------------------------------

Danach dauerte es nicht lange.

Es war ein breites Porträt, doch die drei Personen darin wirkten ziemlich zusammengedrängt. Es gab einen Mann in mittleren Jahren aus dem zwölften Jahrhundert, gekleidet in wallende schwarze Gewänder; der mit einer traurig drein blickenden jungen Frau aus dem vierzehnten Jahrhundert sprach, deren Haar sich unablässig auf ihrem Kopf zu kräuseln schien, als würde sie durch einen Zauber statisch aufgeladen und sie wiederum sprach mit einem ehrwürdigen, verhutzelten alten Mann aus dem siebzehnten Jahrhundert mit einer mächtigen goldenen Fliege und ihn konnten sie verstehen.

Sie hatten nach Dementoren gefragt.

Sie hatten nach Phoenixen gefragt.

Sie hatten nach Drachen und Trollen und Hauselfen gefragt.

Harry hatte stirnrunzelnd erklärt, dass die Geschöpfe, die am meisten Magie benötigten, einfach komplett aussterben könnten und hatte nach den mächtigsten bekannten magischen Geschöpfen gefragt.

Nichts auf der Liste kam ihnen unbekannt vor, mit Ausnahme einer Spezies dunkler Geschöpfe namens Gedankenschinder, zu welchen der Übersetzer anmerkte, sie seien letztlich von Harold Shea ausgerottet worden und diese klangen nicht halb so furchterregend wie Dementoren.

Magische Geschöpfe waren gegenwärtig offenbar so mächtig wie eh und je.

Das saure Gefühl in Dracos Magengegend legte sich und jetzt war er einfach nur noch verwirrt.

"Harry," sagte Draco mitten in einer Übersetzung des alten Mannes über die Liste der elf Kräfte eines Augentyrannen, "was bedeutet das?"

Harry hob einen Finger und der alte Mann brachte die Liste zu Ende.

Dann dankte Harry allen Porträts für ihre Hilfe - Draco, ziemlich automatisch, ebenso und deutlich eleganter - und sie kehrten in das Klassenzimmer zurück.

Und Harry zog das ursprüngliche Pergament mit den Hypothesen hervor und begann zu kritzeln.

\emph{\uline{Beobachtung:}}

\emph{Die Zauberei ist nicht so mächtig, wie damals als Hogwarts gegründet wurde.}

\emph{\uline{Hypothesen:}}

\emph{\emph{{1. Die Magie selbst verschwindet.\\ 2. Zauberer paaren sich mit Muggeln und Squibs.\\ 3. Das Wissen um mächtige Zauber geht verloren.\\ 4. Zauberer essen als Kinder die falsche Nahrung oder etwas anderes außer Blut lässt sie schwächer werden.\\ 5. Muggel-Technologie beeinträchtigt die Magie. (Seit 800 Jahren?)\\ 6. Stärkere Zauberer haben weniger Kinder. (Draco = Einzelkind? Prüfen ob 3 mächtige Zauberer, Quirrell / Dumbledore / Dunkler Lord, }}\emph{irgendwelche Kinder hatten.)}}

\emph{\uline{Tests:}}

\emph{A. Gibt es Zauber, die wir kennen, aber nicht wirken können (1 oder 2) oder sind die verlorenen Zauber nicht mehr bekannt (3)? \uline{Ergebnis:} Nicht aussagekräftig durch Interdikt von Merlin. Kein bekannter nicht wirkbarer Zauber, aber könnte einfach nicht weitergegeben worden sein.}

\emph{B. Wirkten antike Erstklässler dieselbe Art von Zaubern, mit der selben Stärke, wie heute? (Schwacher Beleg für 1 über 2, aber Blut könnte auch nur mächtige Magie verlieren.) \uline{Ergebnis:} Zauber für Erstklässler damals auf selbem Level wie heute.}

\emph{C. Zusätzlicher Test durch wissenschaftliches Wissen über Blut, der zwischen 1 und 2 unterscheidet, Erklärung später. \uline{Ergebnis:} Es gibt nur einen Punkt im Rezept, der einen zum Zauberer macht und entweder hat man zwei Papierchen, die 'magisch' sagen oder man hat sie nicht.}

\emph{D. Verlieren magische Geschöpfe ihre Kräfte? Unterscheidet 1 von (2 oder 3). \uline{Ergebnis:} Magische Geschöpfe scheinen so stark zu sein, wie eh und je.}

"A ist fehlgeschlagen," sagte Harry Potter. "B ist ein schwacher Beleg für 1 über 2. C falsifiziert 2. D falsifiziert 1. 4 war unwahrscheinlich und B spricht auch gegen 4. 5 war unwahrscheinlich und D spricht dagegen. 6 ist zusammen mit 2 falsifiziert. Bleibt noch 3. Interdikt von Merlin hin oder her, ich konnte keinen tatsächlich bekannten Zauber finden, der nicht gewirkt werden könnte. Also, wenn man alles zusammen nimmt, sieht es ganz so aus, als geht das Wissen verloren."

Und die Falle schnappte zu.

Sobald die Panik verschwand, sobald Draco klar wurde, dass die Magie \emph{nicht} verblasste, brauchte es ganze fünf Sekunden, um zu begreifen.

Draco schob sich von dem Schreibpult zurück und erhob sich so heftig, dass der Stuhl umkippte und mit einem kratzenden Geräusch über den Fußboden schlitterte.

"Dann war also alles nur ein dummer Trick."

Harry Potter starrte ihn einen Moment lang an, saß immer noch da. Als er sprach, war seine Stimme ein Flüstern. "Es war ein fairer Test, Draco. Wäre er anders ausgegangen, hätte ich es akzeptiert. Das ist nichts, bei dem ich jemals betrügen würde. Niemals. Ich habe mir deine Daten nicht angesehen, bevor ich meine Vorhersagen machte. Ich erzählte dir direkt davon, als das Interdikt von Merlin das erste Experiment ungültig machte -"

"Oh," sagte Draco und Zorn schlich sich in seine Stimme, "du wusstest nicht, wie das ganze ausgehen würde?"

"Ich \emph{wusste} nichts, das du nicht wusstest," sagte Harry, noch immer leise. "Ich gebe zu, ich hatte einen Verdacht. Hermine Granger war zu mächtig, sie hätte kaum magisch sein sollen und das war sie nicht, wie kann eine Muggelgeborene die beste Zauberin in Hogwarts sein? Und sie bekommt auch noch die besten Noten für ihre Aufsätze, es ist zu viel des Zufalls, dass ein Mädchen sowohl magisch \emph{als auch} akademisch die Stärkste ist, es sei denn aus ein und dem selben Grund. Hermine Grangers Existenz deutete darauf hin, dass es nur eine einzige Sache gibt, die einen zu einem Zauberer macht, etwas dass man entweder hat oder nicht und dass unsere Macht davon abhängt, wie viel wir wissen und wie hart wir trainieren. Und es gab auch keine verschiedenen Klassen für Reinblüter und Muggelgeborene und so weiter. Die Welt sah auf zu vielerlei Art nicht so aus, wie sie es würde, wenn du recht hättest. Aber Draco, ich habe nichts gesehen, das du nicht auch sehen konntest. Ich habe keine Tests durchgeführt, von denen ich dir nicht erzählt habe. Ich habe nicht betrogen, Draco. Ich wollte, dass wir die Antwort zusammen herausfinden. Und ich dachte niemals daran, dass die Magie aus der Welt verschwinden könnte, bis du es sagtest. Es war auch für mich ein erschreckender Gedanke."

"Wie auch immer," sagte Draco. Es fiel ihm sehr schwer, seine Stimme im Zaum zu halten und Harry nicht einfach anzuschreien. "Du behauptest also, du wirst nicht losziehen und allen anderen davon erzählen."

"Nicht ohne es vorher mit dir zu besprechen," sagte Harry. Er öffnete die Hände zu einer flehenden Geste. "Draco, ich sage das so nett, wie ich nur kann, aber \emph{es hat sich herausgestellt,} \emph{dass} \emph{die Welt einfach nicht so} \emph{ist.}"

"Fein. Dann sind du und ich fertig miteinander. Ich werde einfach verschwinden und vergessen, dass irgendwas davon je passiert ist."

Draco fuhr herum, fühlte ein Brennen in seiner Kehle, den Geschmack des Verrats und zwar als ihm klar wurde, dass er Harry Potter wirklich gemocht \emph{hatte} und dieser Gedanke bremste ihn nicht einen Moment lang, als er mit großen Schritten auf die Tür des Klassenzimmers zuhielt.

Und Harry Potters Stimme erklang, nun lauter und besorgt:

"Draco… du \emph{kannst} nicht vergessen. Verstehst du nicht? Das war dein Opfer."

Draco stoppte mitten im Schritt und schwang herum. "\emph{Wovon} redest du?"

Doch eine eisige Kälte lief Draco bereits den Rücken hinunter.

Er wusste es, noch bevor Harry Potter es aussprach.

"Ein Wissenschaftler zu werden. Du hast etwas in Frage gestellt, woran du glaubst, nicht nur eine kleine Sache, sondern etwas von immenser Bedeutung für dich. Du hast Experimente durchgeführt, Daten gesammelt und der Ausgang bewies, die Überzeugung war falsch. Du hast die Ergebnisse gesehen und verstanden, was sie bedeuteten." Harry Potters Stimme geriet ins Wanken. "Denk daran, Draco, eine \emph{wahre} Überzeugung kannst du so nicht opfern, da die Experimente sie bestätigen werden, anstatt sie zu falsifizieren. Dein Opfer, um ein Wissenschaftler zu werden, war deine \emph{falsche} Überzeugung, dass Zaubererblut sich vermischte und schwächer wurde."

"\emph{Das ist nicht wahr!}" sagte Draco. "Ich habe die Überzeugung nicht geopfert. Ich glaube das immer noch!" Seine Stimme wurde lauter und die Kälte wurde schlimmer.

Harry Potter schüttelte den Kopf. Seine Stimme war ein Flüstern. "Draco… tut mir leid, Draco, du glaubst es \emph{nicht,} nicht mehr." Harrys Stimme hob sich wieder. "Ich werde es dir beweisen. Stell dir vor, jemand erzählt dir, er hat einen Drachen in seinem Haus. Du sagst ihm, du willst ihn sehen. Er sagt, es ist ein unsichtbarer Drache. Du sagst fein, dann höre ich zu, wie er sich bewegt. Er sagt, es ist ein unhörbarer Drache. Du sagst, du wirfst etwas Mehl in die Luft, um den Umriss des Drachen zu sehen. Er sagt, der Drache ist durchlässig für Mehl. Und das verräterische ist, dass er weiß, im \emph{Voraus,} welche experimentellen Resultate er wird weg erklären müssen. Er \emph{weiß,} dass alles so ausgehen wird, als ob es keinen Drachen gäbe, er weiß im \emph{Voraus,} wie er sich herausreden muss. Also vielleicht \emph{sagt} er, da ist ein Drache. Vielleicht \emph{glaubt} er, dass er glaubt, da sei ein Drache, das nennt man an den Glauben glauben. Aber er glaubt es nicht wirklich. Man kann sich selbst täuschen über das, was man glaubt, die meisten Leute bemerken niemals, dass es einen Unterschied macht, ob man etwas glaubt oder denkt, dass es gut ist, das zu glauben." Harry Potter hatte sich jetzt von dem Schreibpult erhoben und einige Schritte auf Draco zu gemacht. "Und Draco, du glaubst nicht mehr an die Blutreinheitslehre, ich werde dir zeigen, dass du es nicht tust. Wenn die Blutreinheitslehre wahr ist, dann macht Hermine Granger keinen Sinn, also was könnte die Erklärung für sie sein? Vielleicht ist sie ein verwaistes Zauberer-Kind, dass von Muggeln aufgezogen wurde, so wie ich? Ich könnte zu Granger gehen und sie bitten, mir Bilder von ihren Eltern zu zeigen, um zu sehen, ob sie ihnen ähnlich sieht. Würdest du erwarten, dass sie anders aussehen? Sollen wir den Test machen?"

"Sie hätten sie zu Verwandten gegeben," sagte Draco mit zitternder Stimme. "Sie würden trotzdem gleich aussehen."

"Siehst du. Du weißt bereits genau, welches experimentelle Resultat du entschuldigen musst. Wenn du immer noch an die Blutreinheitslehre glauben würdest, würdest du sagen, sicher, sehen wir nach, ich wette sie sieht nicht wie ihre Eltern aus, sie ist zu mächtig um wirklich eine Muggelgeborene zu sein -"

"Sie \emph{hätten} sie zu Verwandten gegeben!"

"Wissenschaftler können Tests durchführen, um sicher festzustellen, ob jemand wirklich das Kind eines Vaters ist. Granger würde es wahrscheinlich machen, wenn ich ihrer Familie genug bezahlte. \emph{Sie} hätte vor den Ergebnissen keine Angst. Was also erwartest du, was jener Test zeigen wird? Sag mir, wir machen den Test und dann werden wir das. Doch du weißt bereits, was der Test aussagen wird. Du wirst es immer wissen. Du wirst es niemals vergessen können. Du magst dir \emph{wünschen,} du würdest an die Blutreinheitslehre glauben, doch du wirst immer genau das \emph{zu sehen erwarten, was geschehen würde,} wenn es nur eine Sache gibt, die einen zu einem Zauberer macht. Das war dein Opfer, um ein Wissenschaftler zu werden."

Dracos Atem ging stoßweise. "Weißt du, \emph{was du getan hast?}" Draco stürmte vorwärts und ergriff Harry am Kragen seines Umhangs. Seine Stimme hob sich zu einem Schrei, klang unerträglich laut in der Stille des geschlossenen Klassenzimmers. "\emph{Weißt du, was du getan hast?}"

Harrys Stimme zitterte. "Du hattest eine Überzeugung. Sie war falsch. Ich habe dir geholfen, das zu erkennen. Die Wahrheit steht bereits fest, sie anzunehmen macht sie nicht schlimmer -"

Die Finger von Dracos rechter Hand verkrampften sich zu einer Faust und diese Hand sackte herab und fuhr unaufhaltsam wieder empor und schlug Harry Potter so hart gegen den Unterkiefer, dass sein Körper rücklings gegen ein Schreibpult und dann krachend zu Boden stürzte.

"\emph{Idiot!}" schrie Draco. "\emph{Idiot! Idiot!}"

"Draco," flüsterte Harry, der auf dem Boden lag, "Draco, es tut mir leid, ich habe erwartet, dass wir noch Monate hätten, bevor das passiert, ich habe nicht erwartet, dass du als Wissenschaftler so schnell erwachst, ich dachte, ich hätte mehr Zeit, dich vorzubereiten, dir die Techniken beizubringen, durch die es weniger schmerzt, zuzugeben, dass man falsch liegt -"

"Was ist mit Vater?" sagte Draco. Seine Stimme bebte vor Zorn. "Wolltest du \emph{ihn} auch vorbereiten oder war es dir einfach \emph{egal,} was danach passiert?"

"Du kannst es \emph{ihm} nicht sagen!" sagte Harry, seine Stimme klang alarmiert. "Er ist kein Wissenschaftler! Du hast es versprochen, Draco!"

Für einen Moment kam der Gedanke, dass sein Vater es nicht wüsste, als Erleichterung.

Und dann stieg die wahre Wut in ihm hoch.

"Also hast du geplant, dass ich ihn anlügen und ihm sagen soll, ich glaube immer noch," sagte Draco mit zitternder Stimme. "Ich werde ihn immer anlügen müssen und wenn ich jetzt erwachsen werde, kann ich kein Todesser sein und ich werde ihm nicht einmal erzählen können, warum nicht."

"Wenn dein Vater dich wirklich liebt," flüsterte Harry auf dem Boden, "wird er dich noch immer lieben, selbst wenn du kein Todesser wirst und es klingt als liebt dein Vater dich \emph{wirklich,} Draco -"

"\emph{Dein} Stiefvater ist ein Wissenschaftler," sagte Draco. Die Worte stechend wie Messer. "Wenn \emph{du} kein Wissenschaftler würdest, würde er dich immer noch lieben. Doch du wärst nicht \emph{mehr ganz so besonders} für ihn."

Harry zuckte zusammen. Der Junge öffnete den Mund, als wolle er sagen 'Es tut mir leid' und schloss ihn dann wieder, schien es sich anders überlegt zu haben, was entweder sehr schlau oder großes Glück von ihm war, denn Draco hätte sonst vielleicht versucht, ihn umzubringen.

"Du hättest mich warnen sollen," sagte Draco. Seine Stimme schwoll an. "\emph{Du hättest mich warnen sollen!}"

"Das… das habe ich… jedes mal, wenn ich dir von der Macht erzählte, erzählte ich dir von ihrem Preis. Ich sagte, du würdest zugeben müssen, dass du unrecht hast. Ich sagte, dies würde der schwierigste Pfad für dich sein. Dass dies das Opfer sei, das jeder bringen müsse, um ein Wissenschaftler zu sein. Ich sagte, was wenn das Experiment das eine sagt und deine Familie und Freunde etwas anderes -"

"\emph{Du nennst das eine Warnung?}" Draco schrie jetzt. "\emph{Das nennst du eine Warnung? Wenn wir ein Ritual durchführen, das ein dauerhaftes Opfer verlangt?}"

"Ich… ich…" Der Junge auf dem Boden schluckte. "Ich schätze, vielleicht war es nicht klar. Es tut mir leid. Doch das, was durch die Wahrheit zerstört werden kann, sollte es auch."

Ihn zu schlagen, war nicht genug.

"Bei einer Sache liegst du falsch," sagte Draco, mit tödlicher Stimme. "Granger ist nicht die stärkste Schülerin in Hogwarts. Sie bekommt nur die besten Noten. Du wirst gleich den Unterschied erfahren."

Plötzlicher Schock zeigte sich auf Harrys Gesicht und er versuchte, sich schnell auf die Füße zu rollen -

Es war bereits zu spät für ihn.

"\emph{Expelliarmus!}"

Harrys Zauberstab flog quer durch den Raum.

"\emph{Gom jabbar!}"

Ein pulsierendes Etwas aus tintenhafter Schwärze ergriff Harrys linke Hand.

"Das ist ein Folter-Zauber," sagte Draco. "Er wird genutzt, um Informationen aus Leuten herauszubekommen. Ich werde ihn einfach aktiv lassen und die Tür hinter mir schließen, wenn ich gehe. Vielleicht lasse ich den Schließzauber nach ein paar Stunden abklingen. Vielleicht hält er auch, bis du hier drin stirbst. Viel Spaß."

Draco wich geschmeidig zurück, den Zauberstab noch immer auf Harry gerichtet. Dracos Hand sank herab, las seine Schultasche auf, ohne von seinem Ziel zu weichen.

Der Schmerz zeigte sich bereits auf Harry Potters Gesicht als er sprach. "Malfoys stehen über den Gesetzen zur Zauberei Minderjähriger, nehme ich an? Es ist nicht, weil dein Blut stärker ist. Sondern weil du bereits geübt hast. Am Anfang warst du so schwach wie jeder von uns. Ist meine Vorhersage falsch?"

Dracos Hand verkrampfte sich um seinen Zauberstab, doch er ließ nicht von seinem Ziel ab.

"Nur dass du's weißt," sagte Harry durch zusammengebissene Zähne, "wenn du mir gesagt hättest, ich läge falsch, hätte ich zugehört. \emph{Ich} werde \emph{dich} niemals foltern, wenn du mir zeigst, dass ich falsch liege. Und das \emph{wirst} du. Eines Tages. Du bist als Wissenschaftler erwacht und selbst wenn du niemals lernst, deine Macht zu nutzen, wirst du immer," Harry keuchte, "nach Wegen, suchen, deine Überzeugungen, zu testen -"

Draco wich nun weniger geschmeidig zurück, etwas hastiger und er musste sich anstrengen, seinen Zauberstab auf Harry gerichtet zu halten als er hinter sich griff, um die Tür zu öffnen und aus dem Klassenraum trat.

Dann schloss Draco die Tür wieder.

Er wirkte den mächtigsten Schließzauber, den er kannte.

Draco wartete, bis er Harrys ersten Schrei hörte, bevor er den \emph{Quietus} wirkte.

Und dann ging er davon.

--------------------------------------------------------------------------------------------------------------------------------------------

\emph{"\emph{Aaahhhhh! Finite Incantatem! Aaaahhh!}"}

\emph{Man hatte Harrys linke Hand in einen Topf voll siedend heißem Öl gesteckt und darin liegen lassen. Er hatte alles, was er hatte, in den} \emph{\emph{Finite Incantatem}} \emph{gesteckt und er funktionierte immer noch nicht.}

\emph{Manche Flüche brauchten spezifische Gegenzauber oder man konnte sie nicht rückgängig machen oder vielleicht war Draco auch einfach so viel stärker.}

\emph{"\emph{Aaaaahhhh!}"}

\emph{Harrys Hand fing jetzt wirklich an zu schmerzen und das behinderte seine Versuche, kreativ zu denken.}

\emph{Doch ein paar Schreie später wurde Harry klar, was er tun musste.}

\emph{Sein Beutel befand sich, unglücklicherweise, auf der falschen Seite seines Körpers und er musste sich ziemlich verdrehen, um hineinzugreifen, besonders da seine andere Hand wild herumfuchtelte, in einem reflexhaften, unaufhaltsamen Versuch, die Quelle des Schmerzes abzuschütteln. Als er es schließlich schaffte, hatte sein anderer Arm es schon wieder geschafft, seinen Zauberstab weg zu schleudern.}

\emph{"Heil-\emph{ahhhhh}-Ausrüstung! Heil-Ausrüstung!"}

\emph{Auf dem Fußboden war das grüne Licht zu dämmrig, um etwas zu erkennen.}

\emph{Harry konnte nicht stehen. Er konnte nicht kriechen. Er rollte sich über den Boden, dorthin wo er seinen Zauberstab vermutete und dort war er nicht und er schaffte es, sich mit einer Hand weit genug aufzurichten, um seinen Zauberstab zu erkennen und er rollte dorthin und bekam den Zauberstab zu fassen und rollte zurück, dorthin wo die Heil-Ausrüstung offen lag. Außerdem gab es einiges an Geschrei und ein wenig Erbrechen.}

\emph{Harry brauchte acht Versuche, bevor er einen} \emph{\emph{Lumos}} \emph{zustande brachte.}

\emph{Und dann, nun, das Paket war nicht dafür gemacht, einhändig geöffnet zu werden und zwar weil alle Zauberer Idioten waren, deswegen. Harry musste seine Zähne benutzen und daher dauerte es eine Weile, bis Harry es schließlich schaffte, den Betäubungsstoff um seine linke Hand zu wickeln.}

\emph{Als endlich jedes Gefühl aus seiner linken Hand gewichen war, versagte Harrys Geist und er lag bewegungslos auf dem Fußboden und weinte eine Zeit lang.}

\emph{\emph{N}\emph{a}\emph{,}} \emph{sprach Harrys Geist leise zu sich selbst, als er sich genug erholt hatte, um wieder in Worten zu denken.} \emph{\emph{War es das wert?}}

\emph{Langsam griff Harrys funktionierende Hand hinauf zu einem Schreibpult.}

\emph{Harry zog sich selbst auf die Füße.}

\emph{Nahm einen tiefen Atemzug.}

\emph{Atmete aus.}

\emph{Lächelte.}

\emph{Es war kein großartiges Lächeln, doch nichtsdestotrotz ein Lächeln.}

\emph{\emph{Danke, Professor Quirrell, ohne Sie hätte ich nicht verlieren können.}}

\emph{Er hatte Draco noch nicht zur Umkehr bewegt, noch nicht mal annähernd. Entgegen dem, was Draco selbst nun glauben mochte, er war noch immer das Kind eines Todessers, durch und durch. Immer noch ein Junge, der in dem Glauben aufgewachsen war, "Vergewaltigung" sei etwas, das die coolen älteren Kids machten. Aber was für ein Anfang.}

\emph{Harry konnte nicht behaupten, dass alles genau nach Plan gelaufen war. Es war alles gelaufen,} \emph{genau} \emph{wie vollkommen spontan ausgedacht.} \emph{Nach\emph{Plan}} \emph{hätte} \emph{dies} \emph{noch} \emph{bis in den Dezember hinein} \emph{nicht passieren sollen, nachdem Harry Draco die Techniken gelehrt hätte, die Beweise nicht zu leugnen, wenn er sie sah.}

\emph{Doch er hatte den Ausdruck der Angst auf Dracos Gesicht gesehen, erkannt dass Draco} \emph{\emph{bereits}\emph{jetzt}} \emph{eine alternative Hypothese ernst nahm und den Moment genutzt. Ein Fall von wahrer Neugier konnte} \emph{was} \emph{Rationalität} \emph{anging} \emph{die gleiche Kraft zur Besserung entfalten, wie ein Fall von wahrer Liebe im Film.}

\emph{Im Rückblick hatte Harry Stunden dafür eingeplant, die wichtigste Entdeckung in der Geschichte der Magie zu machen und Monate dafür, die noch unentwickelten geistigen} \emph{Barrieren} \emph{eines elfjährigen Jungen zu stürmen. Das mochte darauf hindeuten, dass Harry ein großes Defizit aufwies, wenn es darum ging, die Zeit zur Erfüllung von Aufgaben abzuschätzen.}

\emph{Würde Harry für das, was er getan hatte, zur Wissenschafts-Hölle fahren? Harry war nicht sicher. Er hatte es so eingefädelt, dass Draco} \emph{sichgeistig} \emph{auf die Möglichkeit konzentrierte, dass die Magie verschwand, hatte sichergestellt, dass Draco den Teil des Experiments ausführen würde, der zunächst in diese Richtung zu deuten schien. Er hatte bis nach der Erklärung der Genetik gewartet, um Draco auf die Erkenntnis über magische Geschöpfe zu stoßen (obwohl Harry eher an antike Artefakte, wie den Sprechenden Hut, gedacht hatte, den niemand mehr duplizieren konnte, doch der weiterhin funktionierte). Doch Harry hatte hatte die tatsächlichen Belege nicht verschönert, hatte nicht die Bedeutung irgendwelcher Ergebnisse verfälscht. Als das Interdikt von Merlin den Test entkräftet hatte, der die Entscheidung hätte bringen sollen, hatte er Draco direkt davon erzählt.}

\emph{Und dann war da noch der Teil} \emph{\emph{danach…}}

\emph{Doch er hatte Draco nicht wirklich} \emph{\emph{belogen.}} \emph{Draco hatte es geglaubt und} \emph{\emph{dadurch würde es wahr.}}

\emph{Das Ende war, zugegeben, nicht lustig gewesen.}

\emph{Harry wandte sich um und stolperte zur Tür.}

\emph{Zeit, Dracos Schließzauber zu überprüfen.}

\emph{Der erste Schritt war der Versuch, einfach den Türknauf zu drehen. Draco hatte vielleicht geblufft.}

\emph{Draco hatte nicht geblufft.}

\emph{"\emph{Finite Incantatem.}" Harrys Stimme erklang ziemlich heiser und er konnte spüren, dass der Zauber nicht gegriffen hatte.}

\emph{Also versuchte Harry es erneut und diesmal fühlte es sich richtig an. Doch ein neuerliches Drehen des Türknaufes zeigte, dass es nicht funktioniert hatte. Keine Überraschung.}

\emph{Zeit für die großen Geschütze. Harry nahm einen tiefen Atemzug. Dieser Zauber war einer der mächtigsten, die er bisher gelernt hatte.}

\emph{"\emph{Alohomora!}"}

\emph{Harry schwankte ein wenig, nachdem er ihn ausgesprochen hatte.}

\emph{Und die Tür des Klassenraumes öffnete sich noch immer nicht.}

\emph{Das versetzte Harry einen Schock. Harry hatte natürlich nicht vorgehabt, auch nur in die Nähe von Dumbledores verbotenem Korridor zu gehen. Doch ein Zauber, um magische Schlösser zu öffnen, hatte trotzdem nützlich geklungen und daher hatte Harry ihn gelernt. Sollte Dumbledores verbotener Korridor Leute anlocken, die so dumm waren, nicht zu bemerken, dass die Sicherheitsvorkehrungen lascher waren als das, wozu selbst Draco Malfoy in der Lage war?}

\emph{Die Furcht schlich sich in Harry zurück. Der Zettel in der Heil-Ausrüstung hatte besagt, der Betäubungsstoff könne nur für dreißig Minuten sicher verwendet werden. Danach würde er sich automatisch ablösen und für 24 Stunden nicht wieder zu gebrauchen sein. Jetzt war es gerade 6:51 Uhr abends. Er hatte den Betäubungsstoff vor etwa fünf Minuten aufgelegt.}

\emph{Also trat Harry einen Schritt zurück und nahm die Tür in Augenschein. Es war eine solide Platte aus dunklem Eichenholz, durchbrochen nur von dem Türknauf aus Messing.}

\emph{Harry kannte keine explosiven oder zertrümmernden oder Schneide-Zauber und Sprengstoffe zu transfigurieren hätte die Regel gegen das Transfiguieren von brennbaren Gegenständen verletzt. Säure war eine Flüssigkeit und hätte Dämpfe freigesetzt…}

\emph{Doch das war kein Hindernis für einen} \emph{\emph{kreativen Denker.}}

\emph{Harry legte seinen Zauberstab gegen eines der Türscharniere aus Messing und konzentrierte sich auf die Form von Baumwolle als reine Abstraktion abseits jeder materiellen Baumwolle und auch auf das reine Material abseits des Musters, das es zu einem Türscharnier machte und brachte die beiden Konzepte zusammen, prägte der Form die Substanz auf. Eine Stunde Übung in Transfiguration pro Tag, einen Monat lang, hatten Harry in die Lage versetzt, ein Objekt von unter fünf Kubikzentimetern in etwas unter einer Minute zu transfigurieren.}

\emph{Nach zwei Minuten hatte sich das Scharnier nicht im mindesten verändert.}

\emph{Wer immer Dracos Schließzauber entworfen hatte, hatte auch daran gedacht. Oder die Tür war Teil von Hogwarts und das Schloss war immun.}

\emph{Ein kurzer Blick offenbarte, dass die Mauern aus solidem Stein bestanden. Ebenso der Fußboden. Und die Decke. Man konnte keinen Teil eines größeren Ganzen einzeln transfigurieren; Harry hätte versuchen müssen, die ganze Mauer zu transfigurieren, was Stunden oder gar Tage kontinuierlicher Anstrengungen benötigt hätte, wenn er es überhaupt geschafft hätte und wenn die Mauer kein Kontinuum mit dem Rest des ganzen Schlosses bildete…}

\emph{Harrys Zeitumkehrer würde sich nicht vor 9 Uhr abends öffnen. Danach könnte er nach 6 Uhr abends zurückgehen, bevor die Tür verschlossen wurde.}

\emph{Wie lange würde der Folter-Zauber anhalten?}

\emph{Harry schluckte schwer. Tränen stiegen ihm erneut in die Augen.}

\emph{Sein brillant-kreativer Verstand hatte gerade den genialen Vorschlag unterbreitet, dass Harry seine Hand mit der in seinem Werkzeug-Set in seinem Beutel verstauten Metallsäge abschneiden könnte, was, offensichtlich, weh tun würde, doch erheblich weniger weh tun könnte als Dracos Schmerz-Zauber, da die Nerven nicht mehr da wären und er hatte auch Stauschläuche in seiner Heil-Ausrüstung.}

\emph{Und das war offensichtlich eine lächerlich dämliche Idee, die Harry den Rest seines ganzen Lebens bereuen würde.}

\emph{Doch Harry wusste nicht, ob er zwei Stunden unter Folter überstehen konnte.}

\emph{Er wollte aus diesem Klassenzimmer} \emph{\emph{raus,}} \emph{er wollte} \emph{\emph{sofort}} \emph{aus diesem Klassenzimmer raus, er wollte nicht zwei Stunden schreiend hier drin warten, bis er den Zeitumkehrer benutzen konnte,} \emph{er\emph{musste raus}} \emph{und jemanden finden, der den Folter-Zauber von seiner Hand entfernte…}

\emph{\emph{Denk nach!}} \emph{Harry schrie auf sein Hirn ein.} \emph{\emph{Denk nach! Denk nach!}}

--------------------------------------------------------------------------------------------------------------------------------------------

Der Slytherin-Schlafsaal war größtenteils verlassen. Die Leute waren beim Abendessen. Aus irgendeinem Grund war Draco selbst nicht besonders hungrig.

Draco machte die Tür zu seinem privaten Zimmer zu, schloss sie ab, wirkte einen Schließzauber darauf und einen \emph{Quietus,} setzte sich auf sein Bett und begann zu weinen.

Es war nicht fair.

Es war nicht fair.

Es war das erste mal, dass Draco wirklich \emph{verloren} hatte, Vater hatte ihn gewarnt, dass wirklich zu verlieren, das erste mal wenn es geschah, weh tun würde, doch er hatte \emph{so viel} verloren, es war nicht fair, es war nicht fair, dass er gleich beim allerersten mal \emph{alles} verlieren sollte.

Irgendwo in den Verliesen schrie ein Junge, den Draco wirklich gemocht hatte, vor Schmerzen. Draco hatte noch nie zuvor jemandem weh getan, den er gemocht hatte. Leute zu bestrafen, die es verdient hatten, sollte Spaß machen, doch hierbei fühlte er sich innerlich nur elend. Vater hatte ihn nicht davor gewarnt und Draco fragte sich, ob dies eine harte Lektion war, die jeder lernen musste, wenn er erwachsen wurde oder ob er einfach schwach war.

Draco wünschte, es wäre Pansy, die schrie. Das hätte sich besser angefühlt.

Und das schlimmste war, zu wissen, dass es vielleicht ein Fehler gewesen war, Harry Potter weh zu tun.

Wer sonst bliebe Draco jetzt noch? Dumbledore? Nach dem, was er getan hatte? Draco hätte sich eher bei lebendigem Leib verbrennen lassen.

Draco würde zu Harry Potter zurückkehren müssen, da er nirgendwo sonst hin konnte. Und wenn Harry Potter sagte, er wolle ihn nicht, dann wäre Draco ein Nichts, nur ein erbärmlicher kleiner Junge, der niemals ein Todesser sein konnte, niemals Dumbledores Seite beitreten konnte, niemals etwas über Wissenschaft lernen konnte.

Die Falle war perfekt gestellt gewesen, perfekt ausgeführt. Vater hatte Draco wieder und wieder gewarnt, dass das, was man Dunklen Ritualen opferte, niemals wiederkehrte. Doch Vater hatte nicht gewusst, dass die verfluchten Muggel Rituale ersonnen hatten, die keine Zauberstäbe brauchten, zu denen man verführt werden konnte, ohne es zu wissen und das war nur eines der schrecklichen Geheimnisse, um die Wissenschaftler wussten und die Harry Potter mit sich gebracht hatte.

Daraufhin begann Draco noch stärker zu weinen.

Er wollte das nicht, er \emph{wollte das nicht,} doch es gab kein zurück. Es war zu spät. Er war bereits ein Wissenschaftler.

Draco wusste, er sollte zurück gehen und Harry Potter befreien und sich entschuldigen. Es wäre klug gewesen, das zu tun.\\ Stattdessen blieb Draco auf seinem Bett und schluchzte.

Er hatte Harry Potter bereits weh getan. Es mochte das einzige mal sein, dass Draco die Gelegenheit bekam, ihm weh zu tun und er würde sich den Rest seines Lebens an dieser Erinnerung festhalten müssen.

Sollte er weiter schreien.

--------------------------------------------------------------------------------------------------------------------------------------------

Harry ließ die Überreste seiner Metallsäge zu Boden fallen. Die Messing-Scharniere hatten sich als undurchdringlich erwiesen, waren nicht einmal angekratzt und in Harry keimte der Verdacht, dass selbst der verzweifelte Versuch, Säure oder Sprengstoffe zu transfigurieren, diese Tür nicht zu öffnen vermocht hätte. Auf der Haben-Seite stand, dass der Versuch die Metallsäge zerstört hatte.

Seine Armbanduhr sagte, es war 7:02 Uhr abends, weniger als fünfzehn Minuten verblieben und Harry versuchte sich zu erinnern, ob sich irgendwelche anderen scharfen Gegenstände in seinem Beutel befanden, die zerstört werden mussten und fühlte einen erneuten Schwall von Tränen aufsteigen. Könnte er nur, wenn sein Zeitumkehrer sich öffnete, zurückgehen und \emph{verhindern -}

Und da wurde Harry klar, wie \emph{dumm} er war.

Es war nicht das erste mal, dass er in einem Zimmer eingeschlossen war.

Professor McGonagall hatte ihm bereits gesagt, wie man das richtig machte.

… sie hatte ihm auch gesagt, er solle seinen Zeitumkehrer nicht für solche Dinge benutzen.

Würde Professor McGonagall erkennen, dass diese Gelegenheit \emph{wirklich} eine spezielle Ausnahme rechtfertigte? Oder ihm einfach endgültig den Zeitumkehrer wegnehmen?

Harry sammelte all seine Sachen zusammen, alle Beweise, steckte sie in seinen Beutel. Ein \emph{Ratzeputz} kümmerte sich um das Erbrochene auf dem Boden, doch nicht um den Schweiß, der seinen Umhang durchtränkt hatte. Er ließ die umgestürzten Schreibpulte, wie sie waren, es war nicht wichtig genug, es mit nur einer Hand zu tun.

Als er fertig war, blickte Harry auf seine Armbanduhr hinab. 7:04 abends.

Und dann wartete Harry. Sekunden verstrichen, fühlten sich an wie Jahre.

Um 7:07 Uhr abends öffnete sich die Tür.

Professor Flitwicks rauschebärtiges Gesicht sah ziemlich besorgt aus. "Alles in Ordnung, Harry?" sagte die quiekende Stimme des Ravenclaw-Hauslehrers. "Ich habe eine Nachricht bekommen, du wärst hier drin eingeschlossen worden -"

* Der Autor spricht hier im Original von einem \emph{sorcerer's stone.} Nach dem für die US-amerikanische Ausgabe geänderten Titel des ersten Harry-Potter-Bandes \emph{Harry Potter and the Sorcerer's Stone} (siehe hierzu Wikipedia) könnte dieser Begriff auch \emph{Stein der Weisen} bedeuten. Da \emph{der} Stein der Weisen aus dem ersten Band im britischen Original aber als \emph{Philosopher's Stone} bezeichnet wird und der Autor diese Bezeichnung im Weiteren ebenfalls verwendet, ist davon auszugehen, dass \emph{sorcerer's stone} keine amerikanisierte Bezeichnung für diesen ist und Harry stattdessen tatsächlich einen \emph{Stein der Magier} oder \emph{Magier-Stein} (der einen zum Magier macht oder auch nicht) in Fortsetzung seiner Kiesel-Analogie meint.

