

\hypertarget{machiavellische-intelligenz-hypothese}{% \section{24. Machiavellische Intelligenz-Hypothese}\label{machiavellische-intelligenz-hypothese}}

\textbf{Kapitel 24: Machiavellische Intelligenz-Hypothese\\ }

J. K. Rowling coils and strikes, unseen; Orca circles, hard and lean.

--------------------------------------------------------------------------------------------------------------------------------------------

\emph{3. Akt:}

Draco wartete in einer kleinen Nische mit Fenster, die er in der Nähe der Großen Halle entdeckt hatte, mit rumorendem Magen.

Es würde ein Preis zu zahlen sein und er würde nicht gering ausfallen. Das hatte Draco gewusst, sobald er aufgewacht war und ihm klar wurde, dass er es nicht wagte, zum Frühstück in die Große Halle zu gehen, weil er Harry Potter dort begegnen könnte und Draco wusste nicht, was daraufhin geschehen würde.

Schritte näherten sich.

„So, da wär'n wir,“ sagte Vincents Stimme, „De' Boss hat keine gute Laune heut', also pass auf, was'de sagst.“

Draco würde dem Idioten bei lebendigem Leib die Haut abziehen und den gepellten Körper zurückschicken, zusammen mit der Bitte um einen intelligenteren Diener, wie etwa eine tote Wüstenrennmaus.

Ein Paar Schritte entfernte sich, das andere Paar kam näher.

Das Rumoren in Dracos Bauch wurde schlimmer.

Harry Potter kam in Sicht. Sein Gesicht war gefährlich neutral, doch sein blau-getrimmter Umhang wirkte merkwürdig schief, als sei er nicht ganz richtig angezogen worden -

„\emph{Deine Hand,}“ sagte Draco ohne nachzudenken.

Harry hob den linken Arm, als wolle er sich selbst betrachten.

Die Hand hing leblos von ihm herab, wie etwas Abgestorbenes.

„Madam Pomfrey sagt, es ist nicht von Dauer,“ sagte Harry leise. „Sie sagte, sie sollte sich bis morgen zum Unterrichtsbeginn größtenteils erholt haben.“

Einen kurzen Moment lang erleichterte ihn die Nachricht.

Und dann begriff Draco.

„Du bist zu Madam Pomfrey gegangen,“ flüsterte Draco.

„Natürlich bin ich das,“ sagte Harry Potter, als spreche er das offensichtliche aus. „Meine Hand funktionierte nicht.“

Langsam dämmerte es Draco, welch \emph{vollkommener} Narr er gewesen war, viel schlimmer als die älteren Syltherins, die er zusammen gestaucht hatte.

Er war einfach davon ausgegangen, dass niemand die Behörden informieren würde, wenn ein Malfoy ihm etwas antat. Dass niemand jemals die Aufmerksamkeit von Lucius Malfoy würde erregen wollen.

Doch Harry Potter war kein verängstigter kleiner Hufflepuff, der versuchte, sich aus dem Spiel herauszuhalten. Er spielte es bereits und Vaters Blick war bereits auf ihn gerichtet.

„Was hat Madam Pomfrey sonst noch gesagt?“ sagte Draco, das Herz schlug ihm bis zum Hals.

„Professor Flitwick sagte, der Zauber, der auf meine Hand gewirkt wurde, sei ein dunkler Folter-Fluch gewesen und das sei eine extrem ernste Sache und dass es absolut inakzeptabel sei, den Täter nicht preiszugeben.“

Eine lange Pause entstand.

„Und dann?“ sagte Draco mit zitternder Stimme.

Harry Potter lächelte dünn, „Ich entschuldigte mich vielmals, was Professor Flitwick einen \emph{sehr} strengen Blick entlockte und dann sagte ich Professor Flitwick, dass das ganze, in der Tat, eine extrem ernste, geheime, \emph{heikle} Angelegenheit sei und dass ich den Schulleiter bereits über das Projekt in Kenntnis gesetzt hätte.“

Draco keuchte. „Nein! Das wird Flitwick nicht hinnehmen! Er wird mit Dumbledore sprechen!“

„In der Tat,“ sagte Harry Potter. „Ich wurde prompt ins Büro des Schulleiters gezerrt.“

Draco zitterte jetzt. Wenn Dumbledore Harry Potter vor den Zaubergamot brachte, freiwillig oder sonstwie und der Junge-der-überlebt-hat unter Veritaserum aussagte, dass Draco ihn gefoltert hatte… zu viele Menschen liebten Harry Potter, Vater könnte diese Abstimmung \emph{verlieren…}

Vater mochte Dumbledore überzeugen können, das nicht zu tun, doch das würde \emph{kosten.} Einen schrecklichen Preis. Das Spiel hatte jetzt Regeln, man konnte nicht mehr willkürlich jemanden bedrohen. Doch Draco war Dumbledore aus freien Stücken in die Arme gelaufen. Und Draco war ein sehr wertvolles Pfand.

Obwohl, da Draco nun kein Todesser mehr sein konnte, er nicht so wertvoll war, wie Vater glaubte.

Der Gedanke zerriss sein Herz wie ein Schneidezauber.

„Was dann?“ flüsterte Draco.

„Dumbledore schloss sofort darauf, dass du es warst. Er wusste, wir hatten uns miteinander bekannt gemacht.“

Das schlimmst-mögliche Szenario. Wenn Dumbledore nicht erraten hätte, wer es getan hatte, hätte er vielleicht keine Legilimentik riskiert, nur um es herauszubekommen… doch wenn Dumbledore es \emph{wusste…}

„Und?“ presste Draco hervor.

„Wir haben ein bisschen geplaudert.“

„Und?“

Harry Potter grinste. „Und ich erklärte ihm, dass es in seinem besten Interesse sei, nichts zu unternehmen.“

Dracos Geist lief gegen eine Ziegelmauer und zerbarst. Er starrte Harry Potter einfach nur mit offenem hängendem Mund an, wie ein Idiot.

So lange brauchte es, bis Draco begriff.

Harry kannte Dumbledores mysteriöses Geheimnis, das Snape als Druckmittel genutzt hatte.

Draco konnte es jetzt genau vor sich sehen. Dumbledore mit ernstem Blick, seine Ungeduld verbergend, als er Harry erklärte, wie ernst diese Angelegenheit sei.

Und Harry, der Dumbledore höflich vermittelte, den Mund zu halten, wenn er wusste, was gut für ihn war.

Vater hatte Draco vor solchen Leuten gewarnt, Menschen, die einen ruinieren konnten und trotzdem noch so sympathisch waren, dass es einem schwer fiel, sie richtig zu hassen.

„Woraufhin,“ sagte Harry, „der Schulleiter Professor Flitwick mitteilte, dass dies, in der Tat, eine ernste und heikle Angelegenheit sei, von der er bereits unterrichtet worden sei und er nicht glaube, dass es zu diesem Zeitpunkt weder mir noch irgendwem sonst helfe, darauf zu beharren. Professor Flitwick wollte etwas darüber sagen, dass die üblichen Ränkespiele des Schulleiters viel zu weit gegangen seien und an diesem Punkt musste ich unterbrechen und erklären, dass es meine \emph{eigene} Idee gewesen war und nichts, wozu der Schulleiter mich gedrängt hätte, also wirbelte Professor Flitwick herum und fing an, \emph{mich} zu belehren und der Schulleiter unterbrach \emph{ihn} und sagte, dass ich als der Junge-der-überlebt-hat dazu verdammt sei, in verrückte und gefährliche Abenteuer zu geraten, also sei es sicherer, wenn ich sie absichtlich suche, als wenn sie mir zufällig zustoßen würden und daraufhin rang Professor Flitwick seine kleinen Hände und begann uns \emph{beide} mit seiner hohen Stimme darüber anzukreischen, dass es ihm völlig egal sei, was wir da zusammen auskochten, doch dass es nie wieder zu passieren habe, solange ich im Haus Ravenclaw sei oder er würde mich hinauswerfen und ich könne nach Gryffindor gehen, wo all dieses \emph{Gedumbledore} hingehöre -“

Harry machte es Draco \emph{wirklich} schwer, ihn zu hassen.

„Jedenfalls,“ sagte Harry, „wollte ich nicht aus Ravenclaw rausgeworfen werden, also versprach ich Professor Flitwick, dass nichts in der Art noch einmal vorkommen würde und falls doch, ich ihm einfach erzählen würde, wer es getan hat.“

Harrys Augen hätten kalt sein sollen. Waren sie nicht. Sein Tonfall hätte daraus eine tödliche Drohung machen sollen. Tat er nicht.

Und Draco erkannte die Frage, die offensichtlich hätte sein sollen und sie tötete augenblicklich die Stimmung.

„Warum… hast du es nicht?“

Harry trat ans Fenster hinüber, in den kleinen Sonnenstrahl, der in die Nische herein schien und wandte den Blick nach draußen, zu den grünen Ländereien von Hogwarts. Die Helligkeit strahlte ihn an, seinen Umhang, sein Gesicht.

„Warum ich es nicht getan habe?“ sagte Harry. Seine Stimme gefasst. „Ich schätze, weil ich einfach nicht wütend auf dich sein konnte. Ich wusste, ich hatte dich zuerst verletzt. Ich werde nicht einmal sagen, wir sind quitt, denn was ich dir angetan habe, war schlimmer als das, was du mit mir gemacht hast.“

Es war als laufe er in eine weitere Ziegelmauer. Soweit es Draco betraf, hätte Harry zu jenem Zeitpunkt auch altertümliches Griechisch sprechen können.

Dracos Geist suchte nach bekannten Mustern und versagte vollkommen. Die Aussage war ein Zugeständnis, das nicht in Harrys bestem Interesse gewesen war. Es war nicht einmal, was Harry sagen sollte, um Draco zu einem loyaleren Diener zu machen, nun da Harry Macht über ihn hatte. Dazu sollte Harry betonen, wie großzügig er gewesen war, nicht wie sehr er Draco verletzt hatte.

„Trotzdem,“ sagte Harry und jetzt war seine Stimme leiser, fast ein Flüstern, „tu das bitte nicht noch einmal, Draco. Es schmerzte und ich bin nicht sicher, ob ich dir ein zweites mal vergeben könnte. Ich bin nicht sicher, ob ich es wollen könnte.“

Draco verstand es einfach nicht.

Versuchte Harry, sein \emph{Freund} zu sein?

Harry Potter konnte unmöglich so dumm sein zu glauben, das sei immer noch möglich, nach dem was er getan hatte.

Man konnte jemandes Freund und Verbündeter sein, wie Draco es mit Harry versucht hatte oder man konnte sein Leben zerstören und ihm keine Wahl lassen. Nicht beides.

Doch dann verstand Draco nicht, was Harry \emph{sonst} vorhaben könnte.

Und da kam Draco ein seltsamer Gedanke, etwas wovon Harry gestern gesprochen hatte.

Und der Gedanke war: \emph{Teste es.}

\emph{Du bist als Wissenschaftler erwacht,}hatte Harry gesagt,\emph{und selbst wenn du niemals lernst, deine Macht zu nutzen, wirst du immer, nach Wegen, suchen, deine Überzeugungen, zu testen…}Jene rätselhaften Worte, gesprochen keuchend vor Schmerz, waren Draco immer wieder durch den Kopf gegangen.

Wenn Harry \emph{vorgab,} ein reuiger Freund zu sein, der versehentlich jemanden verletzt hatte…

„Du hast \emph{geplant,} was du mir angetan hast!“ sagte Draco und schaffte es, den Hauch eines Vorwurfs in seine Stimme zu legen. „Du hast es nicht getan, weil du wütend wurdest, du hast es getan, weil du es \emph{wolltest!}“

\emph{Du Narr,} würde Harry Potter sagen, \emph{natürlich habe ich es geplant und jetzt gehörst du mir -}

Harry wandte sich wieder Draco zu. „Was gestern passiert ist, war \emph{nicht} der Plan,“ sagte Harry, scheinbar mit einem Kloß im Hals. „Der \emph{Plan} war, dass ich dir beibringen würde, warum es immer besser ist, die Wahrheit zu kennen und dann hätten wir gemeinsam versucht, die Wahrheit über Blut herauszufinden und was immer die Antwort wäre, wir würden sie akzeptieren. Gestern habe ich… die Dinge überstürzt.“

„Immer besser, die Wahrheit zu kennen,“ sagte Draco kalt. „Als hättest du mir einen \emph{Gefallen} getan.“

Harry nickte, warf Draco völlig aus der Bahn und sagte, „Was wenn Lucius die selbe Idee kommt wie mir, dass das Problem ist, dass stärkere Zauberer weniger Kinder haben? Er könnte ein Programm starten, die stärksten Reinblüter dafür zu bezahlen, mehr Kinder zu bekommen. Tatsächlich wäre es, wenn die Blutreinheitslehre richtig \emph{wäre,} genau das, was Lucius tun \emph{sollte} -- das Problem von \emph{seiner} Seite aus angehen, wo er sofort Dinge in die Wege leiten kann. Im Augenblick, Draco, bist du Lucius einziger Freund, der versuchen würde, ihn davon abzuhalten, den unnötigen Aufwand zu betreiben, weil du der einzige bist, der \emph{wirklich} die Wahrheit kennt und die echten Ergebnisse vorhersagen kann.“

Draco kam der Gedanke, dass Harry Potter an einem so seltsamen Ort aufgewachsen war, dass er nun im Prinzip eher ein magisches Geschöpf war als ein Zauberer. Draco konnte einfach nicht sagen, was Harry als nächstes sagen oder tun würde.

„\emph{Warum?}“ sagte Draco. Den Schmerz und Verrat in seine Stimme zu legen, war überhaupt nicht schwer. „Warum hast du mir das \emph{angetan?} Was \emph{war} dein Plan?“

„Nun,“ sagte Harry, „du bist Lucius Erbe und ob du es glaubst oder nicht, Dumbledore denkt, ich gehöre zu ihm. Wir könnten also aufwachsen und ihre Kämpfe miteinander ausfechten. Oder wir könnten etwas anderes tun.“

Langsam nahm Dracos Geist das auf. „Du willst einen Kampf bis zum Ende zwischen ihnen provozieren und dann die Macht ergreifen, wenn sie beide erschöpft sind.“ Draco fühlte kaltes Grauen in seiner Brust. Er \emph{würde} das verhindern müssen, was es ihn auch kostete -

Doch Harry schüttelte den Kopf. „Bei den Sternen am Himmel, \emph{nein!}“

„Nein…?“

„Dabei würdest du nicht mitmachen und ich ebenso wenig,“ sagte Harry. „Das ist \emph{unsere} Welt, wir wollen sie nicht zerstören. Doch stell dir vor, sagen wir, Lucius dächte, die Verschwörung sei dein Werkzeug und du wärst auf seiner Seite, Dumbledore dächte, die Verschwörung sei mein Werkzeug und ich sei auf seiner Seite, Lucius dächte, du hättest mich umgedreht und Dumbledore würde glauben, die Verschwörung gehöre mir, Dumbledore dächte, ich hätte dich umgedreht und Lucius würde glauben, die Verschwörung gehöre dir und so würden sie uns beide behilflich sein, aber nur auf eine Weise, dass der andere es nicht mitbekommen würde.“

Draco musste seine Sprachlosigkeit nicht einmal vortäuschen.

Vater hatte ihn einmal mitgenommen, um ein Stück namens \emph{Die Tragödie von Light}* anzusehen, über einen \emph{unglaublich} cleveren Slytherin namens Light, der ausgezogen war, die Welt vom Bösen zu befreien, mit Hilfe eines antiken Ringes, der jeden töten konnte, dessen Namen und Gesicht er kannte und dessen Gegner ein weiterer unglaublich cleverer Slytherin war, ein Schurke namens Lawliet, der eine Verkleidung trug, um sein wahres Gesicht zu verbergen und Draco hatte an allen richtigen Stellen gerufen und gejubelt, besonders im Mittelteil und als das Stück traurig geendet hatte und Draco sehr enttäuscht gewesen war, hatte Vater behutsam angemerkt, die 'Tragödie' im Titel sei passend.

Danach hatte Vater Draco gefragt, ob er verstände, wieso sie in dieses Stück gegangen seien.

Draco hatte gemeint, um ihm beizubringen, so listig zu sein wie Light und Lawliet, wenn er erwachsen wurde.

Vater hatte Draco gesagt, dass er unmöglich noch falscher liegen könnte und darauf hingewiesen, dass obwohl Lawliet klugerweise sein Gesicht verhüllt hatte, es keinen guten Grund gab, Light seinen \emph{Namen} mitzuteilen. Vater war dann fortgefahren, fast jeden Teil des Stückes auseinander zu nehmen, während Draco mit immer größer werdenden Augen lauschte. Und schließlich hatte Vater gesagt, dass Stücke wie dieses \emph{immer} unrealistisch seien, denn hätte der Bühnenautor gewusst, was jemand der \emph{tatsächlich} so schlau war wie Light, \emph{wirklich} tun würde, hätte der Bühnenautor selbst versucht, die Weltherrschaft an sich zu reißen, anstatt nur Stücke darüber zu schreiben.

Daraufhin hatte Vater ihm von der Regel der Drei erzählt, die lautete, dass jede Verschwörung, für deren Funktionieren das Eintreten von mehr als drei verschiedenen Dingen nötig war, im echten Leben niemals funktionieren würde.

Vater hatte \emph{weiterhin} erklärt, da nur ein Narr sich an einem Plan versuchen würde, der \emph{so kompliziert wie möglich} war, sei die echte Grenze zwei.

Draco fehlten die Worte, um die schier kolossale Undurchführbarkeit von Harrys Masterplan zu beschreiben.

Doch es war \emph{genau} die Art von Fehler, die man machen würde, wenn man keine Mentoren hatte und sich für clever hielt und das Schmieden von Verschwörungen aus Theaterstücken gelernt hatte.

„Also,“ sagte Harry, „was hältst du von dem Plan?“

„Er ist clever…“ sagte Draco langsam. \emph{Brillant!} zu rufen und vor Bewunderung nach Luft zu schnappen, hätte zu verdächtig ausgesehen. „Harry, kann ich eine Frage stellen?“

„Sicher,“ sagte Harry.

„Warum hast du Granger einen teuren Beutel gekauft?“

„Um freundlich zu erscheinen,“ sagte Harry sofort. „Obwohl ich erwarte, dass sie außerdem ziemlich verlegen sein wird, in den nächsten paar Monaten irgendwelche kleinen Gefallen abzulehnen, um die ich sie bitte.“

Und da wurde Draco klar, dass Harry tatsächlich \emph{versuchte,} sein Freund zu sein.

Harrys Schachzug gegen Granger \emph{war} schlau gewesen, vielleicht sogar brillant. Dem eigenen Gegner vorzutäuschen, man sei unverdächtig \emph{und} auf freundliche Weise dafür zu sorgen, dass er in deiner Schuld stand, damit man ihn dahin bringen konnte, wo man ihn brauchte, \emph{indem man ihn} \emph{einfach fragte.} Draco hätte das nicht durchziehen können, sein Ziel wäre zu misstrauisch gewesen, doch der Junge-der-überlebt-hat \emph{konnte es.} Also war der erste Teil von Harrys Plan, seinem Feind ein teures Geschenk zu machen, daran hätte Draco nicht gedacht, doch es konnte \emph{funktionieren…}

Wenn du Harrys Feind warst, wären seine Pläne zunächst schwer zu durchschauen, vielleicht sogar dämlich, doch seine Logik ergab \emph{Sinn,} du würdest verstehen, dass er dich verletzen wollte.

Wie Harry sich jetzt Draco gegenüber verhielt, machte \emph{keinen} Sinn.

Denn wenn du Harrys \emph{Freund} warst, dann versuchte er, mit dir Freundschaft zu schließen, auf die befremdliche, unverständliche Art und Weise, wie es ihm von Muggeln beigebracht worden war, selbst wenn das bedeutete, dein ganzes Leben zu zerstören.

Die Stille dehnte sich aus.

„Ich weiß, ich habe unsere Freundschaft furchtbar missbraucht,“ sagte Harry schließlich. „Doch bitte erkenne, Draco, dass ich schlussendlich nur wollte, dass wir gemeinsam die Wahrheit herausfinden. Kannst du das verzeihen?“

Eine Gabelung zweier Wege, doch mit nur einem Pfad, auf dem er später einfach umkehren konnte, wenn Draco seine Meinung änderte…

„Ich denke, ich verstehe, was du zu tun versucht hast,“ log Draco, „also ja.“

Harrys Augen leuchteten auf. „Ich bin froh, das zu hören, Draco,“ sagte er sanft.

Die zwei Schüler standen in der Nische, Harry noch immer in den einsamen Sonnenstrahl getaucht, Draco im Schatten.

Und Draco wurde mit einem Anflug von Entsetzen und Verzweiflung bewusst, dass obwohl es in der Tat ein furchterregendes Schicksal war, Harrys Freund zu sein, Harry nun über so viele verschiedene Möglichkeiten verfügte, Draco zu drohen, dass sein Feind zu sein sogar noch \emph{schlimmer} wäre.

Wahrscheinlich.

Vielleicht.

Nun, er konnte später immer noch sein Feind werden…

Er war verdammt.

„Also,“ sagte Draco. „Was jetzt?“

„Lernen wir nächsten Samstag wieder?“

„Wird besser nicht wie beim letzten mal -“

„Keine Sorge, wird es nicht,“ sagte Harry. „Noch ein paar \emph{solcher} Samstage und du wärst \emph{mir} voraus.“

Harry lachte. Draco nicht.

„Oh und bevor du gehst,“ sagte Harry und grinste verlegen. „Ich weiß, das ist ein schlechter Zeitpunkt, aber ich wollte dich eigentlich bei etwas um Rat fragen.“

„Okay,“ sagte Draco, noch immer leicht abgelenkt von der letzten Aussage.

Harrys Blick wurde eindringlich. „Diesen Beutel für Granger zu kaufen, hat das meiste Gold aufgebraucht, das ich aus meinem Verlies bei Gringotts stehlen konnte -“

Was.

„- und McGonagall hat den Schlüssel zum Verlies oder jetzt vielleicht auch Dumbledore. Und ich wollte gerade einen Plan in die Wege leiten, der einiges kosten könnte, also habe ich mich gefragt, ob du weißt, wie ich herankomme an -“

„Ich leihe dir das Geld,“ sagte Dracos Mund völlig automatisch.

Harry sah überrascht aus, doch auf angenehme Weise. „Draco, das musst du nicht -“

„Wie viel?“

Harry nannte die Summe und Draco konnte den Schock auf seinem Gesicht nicht ganz verbergen. Das war fast das gesamte Taschengeld, das Vater Draco für das ganze Schuljahr gegeben hatte, Draco hätte nur noch ein paar Galleonen übrig -

Dann gab sich Draco geistig einen Tritt. Alles, was er tun musste, war Vater zu schreiben und zu erklären, das er das Geld nicht mehr hatte, weil er es geschafft hatte, \emph{es Harry Potter zu leihen} und Vater würde ihm eine besondere Glückwunschkarte schicken, geschrieben mit goldener Tinte, einen riesigen Schokofrosch, der für zwei ganze Wochen reichen würde und zehn mal so viele Galleonen, nur für den Fall, dass Harry Potter noch einen Kredit brauchte.

„Es ist viel zu viel, nicht wahr,“ sagte Harry. „Tut mir leid, ich hätte nicht fragen sollen -“

„Verzeihung, ich \emph{bin} immerhin ein Malfoy, wie du weißt,“ sagte Draco. „Ich war nur überrascht, dass du so viel \emph{wolltest.}“

„Keine Sorge,“ sagte Harry fröhlich. „Es berührt nicht die Interessen deiner Familie, ich bin nur etwas böse.“

Draco nickte. „Dann, kein Problem. Willst du es gleich haben?“

„Sicher,“ sagte Harry.

Als sie die Nische verließen und die Verliese ansteuerten, musste Draco einfach fragen, „Also \emph{kannst} du mir sagen, für welchen Plan das ist?“

„Rita Kimmkorn.“

Draco stieß innerlich ein paar sehr schlimmer Worte aus, doch es war viel zu spät, um nein zu sagen.

--------------------------------------------------------------------------------------------------------------------------------------------

Als sie die Verliese erreicht hatten, hatte Draco begonnen, seine Gedanken wieder zu ordnen.

Es \emph{fiel} ihm schwer, Harry Potter zu hassen. Harry \emph{hatte} versucht, freundlich zu sein, er war nur einfach wahnsinnig.

Und das würde Dracos Rache nicht aufhalten oder auch nur verlangsamen.

„Also,“ sagte Draco, nachdem er sich umgesehen hatte, um sicherzustellen, dass niemand in der Nähe war. Ihre Stimmen waren natürlich beide nur ein Rauschen, doch es war niemals schlecht, auf Nummer sicher zu gehen. „Ich habe nachgedacht. Wenn wir neue Rekruten in die Verschwörung einführen, werden sie \emph{glauben} müssen, dass wir Gleichgestellte sind. Ansonsten braucht nur \emph{einer} von ihnen den Plan an Vater auszuplaudern. Das war dir natürlich auch schon klar, nicht wahr?“

„Natürlich,“ sagte Harry.

„\emph{Werden} wir Gleichgestellte sein?“ sagte Draco.

„Ich fürchte nicht,“ sagte Harry. Es war klar, dass er versuchte, es schonend zu sagen und ebenso, dass er einiges an Herablassung zu unterdrücken versuchte und es ihm nicht ganz gelingen wollte. „Tut mir leid, Draco, aber du weißt im Augenblick noch nicht einmal, was das Wort \emph{Bayes} in \emph{Verschwörung von Bayes} bedeutet. Du wirst noch monatelang studieren müssen, bevor wir irgendjemand anders aufnehmen, nur damit du eine \emph{gute Fassade} aufrecht erhalten kannst.“

„Weil ich nicht genug Wissenschaft kenne,“ sagte Draco und bemühte sich sorgsam, neutral zu klingen.

Woraufhin Harry den Kopf schüttelte. „Das Problem ist nicht, dass du nichts von spezifischen wissenschaftlichen Konzepten weißt, wie etwa Desoxyribonukleinsäure. \emph{Das} würde dich nicht abhalten, mir gleichgestellt zu sein. Das Problem ist, dass du nicht geübt bist in den Methoden der Rationalität, den \emph{tieferen} Geheimnissen dahinter, wie all diese Entdeckungen überhaupt gemacht wurden. Ich \emph{versuche} sie dir beizubringen, doch sie sind viel schwerer zu lernen. Denk daran, was wir gestern gemacht haben, Draco. Ja, du hast einiges der Arbeit gemacht. Doch ich war der einzige, der die Kontrolle hatte. Du hast einige der Fragen beantwortet. Ich habe sie alle gestellt. Du hast beim Schieben geholfen, ich allein habe gesteuert. Und ohne die Methoden der Rationalität, Draco, kannst du unmöglich die Verschwörung dorthin führen, wo sie hingehört.“

„Ich verstehe,“ sagte Draco, seine Stimme klang enttäuscht.

Harrys Stimme versuchte noch sanfter zu werden. „Ich werde versuchen, deine Kompetenz zu respektieren, Draco, etwa bei sozialem Zeugs. Doch du musst auch meine Kompetenz anerkennen und es ist einfach \emph{unmöglich,} dass du mir gleichgestellt bist, wenn es um die Führung der Verschwörung geht. Du bist erst seit \emph{einem Tag} ein Wissenschaftler, du kennst \emph{ein} Geheimnis der Desoxyribonukleinsäure und bist in \emph{keiner} der Methoden der Rationalität geübt.“

„Ich verstehe,“ sagte Draco.

Und das tat er.

\emph{Soziales Zeugs,} hatte Harry gesagt. Die Kontrolle der Verschwörung an sich zu reißen, würde wahrscheinlich nicht einmal schwierig. Und danach würde er Harry Potter töten, nur um sicherzugehen -

Die Erinnerung, wie elend er sich letzte Nacht innerlich gefühlt hatte, in dem Wissen, dass Harry schrie, stieg in Draco hoch.

Draco dachte noch ein paar schlimme Worte.

Na schön. Er würde Harry \emph{nicht} töten. Harry war von Muggeln aufgezogen worden, es war nicht seine Schuld, dass er verrückt war.

Stattdessen würde Harry weiterleben, nur damit Draco ihm sagen könnte, es sei alles zu seinem eigenen Wohl gewesen, wirklich, er sollte ihm wirklich dankbar sein -

Und plötzlich durchzuckte Draco angenehm überraschend die Erkenntnis, dass es tatsächlich zu Harrys eigenem Besten \emph{war.} Wenn Harry seinen Plan ausführte, Dumbledore und Vater für dumm zu verkaufen, würde er \emph{sterben.}

Das machte es \emph{perfekt.}

Draco würde Harry alle seine Träume wegnehmen, so wie Harry es mit ihm getan hatte.

Draco würde Harry sagen, es sei zu seinem eigenen Besten gewesen und es wäre die absolute Wahrheit.

Draco würde die Verschwörung und die Macht der Wissenschaft in Händen halten, um die Zauberwelt zu reinigen und Vater würde so stolz auf ihn sein als sei er ein Todesser.

Harry Potters üble Pläne würden vereitelt und die Mächte des Guten würden triumphieren.

Die perfekte Rache.

Es sei denn…

\emph{Gib einfach vor, dass du vorgibst, dass du ein Wissenschaftler bist,}hatte Harry ihm gesagt.

Draco fehlten die Worte, um genau zu beschreiben, was mit Harrys Geist nicht stimmte -

(da Draco noch nie den Ausdruck \emph{Rekursionstiefe} gehört hatte)

- doch er konnte sich denken, welche Art von Plänen daraus erwuchsen.

… es sei denn, das war genau das, wovon Harry \emph{wollte,} dass Draco es tat, als Teil eines noch \emph{größeren} Plans, dem Draco \emph{genau in die Hände} spielen würde, indem er versuchte, diesen hier zu vereiteln, Harry mochte sogar \emph{wissen,} dass sein Plan undurchführbar war, er mochte keinen anderen Zweck erfüllen, \emph{außer} Draco dazu zu verführen, ihn zu verhindern -

Nein. In der Richtung lag der \emph{Irrsinn.} Es \emph{musste} eine Grenze geben, selbst der Dunkle Lord hatte nicht \emph{so} verdreht gedacht. Solche Dinge passierten im wahren Leben nicht, nur in Vaters dummen Gute-Nacht-Geschichten über dämlicheBetonköpfe, die immer die Pläne des Helden vorantrieben, indem sie versuchten ihn aufzuhalten.

--------------------------------------------------------------------------------------------------------------------------------------------

Und an Dracos Seite ging Harry mit einem Lächeln auf dem Gesicht, während er an die evolutionären Ursprünge der menschlichen Intelligenz dachte.

Am Anfang, bevor Menschen wirklich verstanden, wie Evolution funktionierte, hatten sie verrückte Dinge gedacht, wie \emph{die menschliche Intelligenz hat sich entwickelt, damit wir bessere Werkzeuge erfinden konnten.}

Der Grund, warum das verrückt war, war der, dass nur eine einzige Person in einem Stamm ein Werkzeug erfinden musste, dann würde es jeder andere benutzen und es würde sich verbreiten zu anderen Stämmen und noch von ihren Nachfahren, hundert Jahre später, genutzt werden. Vom Standpunkt des wissenschaftlichen Fortschritts aus gesehen, war das toll, doch aus evolutionärer Sicht bedeutete es, dass die Person, die etwas erfand, keinen großen Überlebens\emph{vorteil} bekam, nicht so viel \emph{mehr} Kinder hatte als alle anderen. Nur \emph{relative} Überlebensvorteile konnten die relative Häufigkeit eines Gens in einer Population erhöhen und eine einsame Mutation an einen Punkt bringen, an dem sie universell wurde und jeder sie hatte. Und brillante Erfindungen waren einfach nicht häufig genug, um die Art von dauerhaftem Selektionsdruck zu erzeugen, die es brauchte, um eine Mutation zu fördern. Es war eine naheliegende Vermutung, wenn man sich Menschen mit ihren Gewehren und Panzern und Atomwaffen ansah und sie mit Schimpansen verglich, dass die Intelligenz zum Erschaffen von Technologie diente. Eine naheliegende Vermutung, aber falsch.

Bevor Menschen wirklich verstanden, wie Evolution funktionierte, hatten sie verrückte Dinge gedacht, wie \emph{das Klima veränderte sich und Stämme mussten auswandern und Menschen mussten schlauer werden, um all die neuen Probleme zu lösen.}

Doch menschliche Wesen hatten ein viermal so großes Gehirn wie ein Schimpanse, 20\% der metabolischen Energie wurden dazu verwendet, das Gehirn zu versorgen. Menschen waren \emph{lächerlich} viel schlauer, als jede andere Spezies. So etwas geschah nicht, nur weil die Umwelt etwas schwierigere Anforderungen stellte. Dann würden die Organismen nur ein wenig schlauer werden, um sie zu lösen. Um schlussendlich dieses gigantische, übergroße Gehirn hervorzubringen, musste es irgendeinen \emph{außer Kontrolle geratenen} evolutionären Prozess gegeben haben, etwas das immer weiter und weiter Druck ausübte, ohne Ende.

Und heutige Wissenschaftler hatten eine ziemlich genaue Ahnung, was dieser außer Kontrolle geratene evolutionäre Prozess gewesen war.

Harry hatte einmal ein berühmtes Buch namens \emph{Unsere haarigen Vettern}** gelesen. Das Buch hatte beschrieben, wie ein erwachsener Schimpanse namens Luit den alternden Alpha, Yeroen, herausgefordert hatte, mit Hilfe eines jungen, vor kurzem erwachsen gewordenen, Schimpansen namens Nikkie. Nikkie hatte sich nicht direkt in die Kämpfe zwischen Luit und Yeroen eingemischt, hatte jedoch Yeroens andere Unterstützer im Stamm davon abgehalten, ihm zu Hilfe zu kommen, sie abgelenkt, wann immer sich eine Auseinandersetzung zwischen Luit und Yeroen anbahnte. Und beizeiten hatte Luit gewonnen und war der neue Alpha geworden, mit Nikkie als dem Zweitmächtigsten…

… obwohl es nicht sehr lange gedauert hatte, bis Nikkie eine Allianz mit dem besiegten Yeroen gebildet und Luit gestürzt hatte und der \emph{neue} neue Alpha geworden war.

Da wusste man wirklich zu schätzen, was Millionen von Jahren, in denen Hominiden versucht hatten, sich \emph{gegenseitig} zu übertreffen -- ein evolutionäres Wettrüsten ohne Ende -- zustande gebracht hatten, was gesteigerte geistige Kapazität anging.

Denn, mal ehrlich, ein Mensch hätte das wirklich kommen sehen.

--------------------------------------------------------------------------------------------------------------------------------------------

Und an Harrys Seite ging Draco, unterdrückte ein Lächeln als er an seine Rache dachte.

Eines Tages, vielleicht erst nach Jahren, doch eines Tages würde Harry Potter erfahren, was es bedeutete, einen Malfoy zu unterschätzen.

Draco war an einem einzigen Tag als Wissenschaftler erwacht. Harry hatte gemeint, es hätte noch Monate dauern sollen.

Doch natürlich war man, als ein Malfoy, ein mächtigerer Wissenschaftler als jeder, der keiner war.

Also würde Draco alle von Harry Potters Methoden der Rationalität erlernen und dann, wenn die Zeit reif war -

* engl.: \emph{The Tragedy of Light}\\ ** engl.: \emph{Chimpanzee Politics;} vollständiger Titel \emph{Unsere haarigen Vettern. Neueste Erfahrungen mit Schimpansen} (engl.: \emph{Chimpanzee Politics: Power and Sex Among Apes})

