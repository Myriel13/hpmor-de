

\hypertarget{zwischenspiel-persuxf6nliches-finanzmanagement}{% \section{32. Zwischenspiel: Persönliches Finanzmanagement}\label{zwischenspiel-persuxf6nliches-finanzmanagement}}

\textbf{Kapitel 32: Zwischenspiel: Persönliches Finanzmanagement\\ }

"Aber Schulleiter," wandte Harry ein und etwas von seiner Verzweiflung schlich sich in seine Stimme, "meine gesamten Vermögenswerte in einem undiversifizierten Verlies voller Goldmünzen herumliegen zu lassen - das ist verrückt, Schulleiter! Es ist als ob, keine Ahnung, man mit Transfiguration experimentiert, ohne eine anerkannte Autoritätsperson zu konsultieren! So etwas macht man einfach nicht mit Geld!"

Aus dem runzligen Gesicht des alten Zauberers - unter einem festlichen Feiertagshut wie ein katastrophaler Auffahrunfall zwischen Wagen aus grünem und rotem Stoff - traf Harry ein ernster, trauriger Blick.

"Es tut mir leid, Harry," sagte Dumbledore, "und ich muss mich entschuldigen, doch dir die Kontrolle über deine eigenen Finanzen zu gestatten, würde dir erheblich zu viel unabhängigen Handlungsspielraum einräumen."

Harrys Mund öffnete sich, doch kein Ton drang heraus. Er war, buchstäblich, sprachlos.

"Ich werde dir erlauben, fünf Galleonen für Weihnachtsgeschenke abzuheben," sagte Dumbledore, "was mehr ist, als irgendein Junge deines Alters ausgeben sollte, doch wie ich glaube, keine Gefahr darstellt -"

"\emph{Ich kann nicht glauben, dass Sie das gerade gesagt haben!}" platzte es aus Harry heraus. "Sie \emph{geben zu,} derart manipulativ zu sein?"

"Manipulativ?" sagte der alte Zauberer und lächelte schwach. "Nein, manipulativ wäre es, würde ich es \emph{nicht} zugeben oder wenn ich tiefere Absichten abseits des Offensichtlichen hätte. Ich sage es ganz geradeheraus, Harry. Du bist noch nicht bereit, das Spiel zu spielen und es wäre töricht, dir tausende von Galleonen zu gewähren, mit denen du das Spielfeld durcheinander bringst."

--------------------------------------------------------------------------------------------------------------------------------------------

Das lebhafte Treiben in der Winkelgasse hatte sich mit dem Näherrücken von Weihnachten noch einmal verdoppelt und verhundertfacht und alle Läden und Geschäfte waren eingehüllt in leuchtende Zaubereien, die blitzten und funkelten als laufe die Stimmung Gefahr, außer Kontrolle zu geraten wie ein Flächenbrand und die gesamte Gegend in einen fröhlichen Festtagskrater zu verwandeln. Die Straßen waren mit Hexen und Zauberern in festlicher und \emph{lauter} Aufmachung so überfüllt, dass es fast ein ebensolcher Angriff auf die Augen wie auf die Ohren war und ob der verblüffenden Vielfalt der Einkäufer wurde klar, dass die Winkelgasse als internationale Attraktion galt. Man sah Hexen, gehüllt in gigantische Stoffschwaden, wie eingewickelte Mumien und Zauberer mit formellen Zylinderhüten in Morgenmänteln* und junge Knirpse, kaum dem Kleinkindalter entwachsen, wurden von ihren Eltern an der Hand durch das magische Wunderland geführt und durften kreischen nach Herzenslust. Es war die Jahreszeit des Frohsinns.

Und inmitten all der Lichter und der Freude, ein Flecken schwärzester Nacht; eine kalte, finstere Aura, die eine paar wenige, kostbare Schritte Platz selbst gegen all das Gedränge verteidigte.

"Nein," sagte Professor Quirrell mit einem Ausdruck grimmigen Abscheus auf dem Gesicht, als habe er gerade etwas gegessen, das nicht nur fürchterlich schmeckte, sondern obendrein auch noch moralisch anstößig war. Es war die Sorte grimmiger Miene, die ein normaler Mensch ziehen mochte, nachdem er in eine Fleischpastete gebissen und festgestellt hatte, dass sie verdorben und aus kleinen Kätzchen gemacht war.

"Ach, \emph{kommen} Sie," sagte Harry. "Sie müssen doch \emph{ein paar} Ideen haben."

"Mr. Potter," sagte Professor Quirrell, seine Lippen zu einem schmalen Strich verengt, "ich habe zugestimmt, auf diesem Ausflug Ihr erwachsener Begleiter zu sein. Ich habe mich nicht bereit erklärt, Sie bei der Auswahl Ihrer Weihnachtsgeschenke zu beraten. Ich beteilige mich nicht an Weihnachten, Mr. Potter."

"Wie wär's mit Newtnachten?" sagte Harry heiter. "Isaac Newton \emph{wurde} tatsächlich am 25. Dezember geboren, anders als manch andere historische Figuren, die ich nennen könnte."

Das vermochte Professor Quirrell nicht zu beeindrucken.

"Schauen Sie," sagte Harry, "es tut mir ja leid, aber ich muss für Fred und George \emph{irgendwas} besonderes machen und ich habe keine Ahnung, was für Möglichkeiten ich habe."

Professor Quirrell summte nachdenklich vor sich hin. "Sie könnten fragen, welche Familienmitglieder sie am wenigsten mögen und dann einen Attentäter anheuern. Ich kenne da jemanden von einer bestimmten Exilregierung, der sein Handwerk wirklich versteht und Sie bekämen noch einen Nachlass auf mehrere Weasleys."

"\emph{Dieses} Weihnachten," sagte Harry und verfiel in eine tiefere Stimmlage, "machen Sie Ihren Freunden das Geschenk… des \emph{Todes.}"

Das entlockte Professor Quirrell ein Lächeln. Es erreichte sogar seine Augen.

"Nun," sagte Harry, "zumindest haben Sie nicht vorgeschlagen, ihnen eine Hausratte zu besorgen -" Harrys Mund schnappte zu und er bereute die Worte fast ebenso schnell, wie sie seinen Mund verlassen hatten.

"Verzeihung?" sagte Professor Quirrell.

"Ach nichts," sagte Harry sofort, "lange, dumme Geschichte." Und sie zu erzählen, schien irgendwie falsch, vielleicht weil Harry fürchtete, dass Professor Quirrell gelacht hätte, selbst wenn Bill Weasley \emph{nicht} geheilt und alles wieder in Ordnung gebracht worden wäre…

Und wo war Professor Quirrell nur \emph{gewesen,} dass er die Geschichte noch nie gehört hatte? Harry hatte den Eindruck gewonnen, dass jeder im magischen Britannien davon wusste.

"Sehen Sie mal," sagte Harry, "ich versuche \emph{ihre Loyalität mir gegenüber zu festigen,} verstehen Sie? Die Weasley-Zwillinge zu meinen Lakaien zu machen? Wie das alte Sprichwort sagt: Ein Freund ist nicht jemand, den benutzt man nur einmal und entsorgt ihn dann, ein Freund ist jemand, den benutzt man immer wieder und wieder. Fred und George sind die zwei nützlichsten Freunde, die ich in Hogwarts habe, Professor Quirrel und ich habe vor, sie immer wieder und wieder zu gebrauchen. Wenn Sie mir also helfen könnten, hierbei zu denken wie ein Slytherin und etwas zu finden, für das sie \emph{wirklich} dankbar wären…" ließ Harry den Satz einladend ausklingen.

Man musste diese Dinge nur richtig verpacken.

Sie gingen ein ganzes Stück weiter, bevor Professor Quirrell wieder sprach, die Stimme praktisch triefend vor Widerwillen. "Die Weasley-Zwillinge verwenden Zauberstäbe aus zweiter Hand, Mr. Potter. Sie würden an ihre Großzügigkeit erinnert, mit jedem Zauber, den sie wirken."

Harry klatschte unwillkürlich aufgeregt in die Hände. Er musste einfach nur das Geld bei Ollivander hinterlegen und Mr. Ollivander anweisen, es nie zurückzuerstatten - nein, besser noch, es an Lucius Malfoy zu schicken, wenn die Weasley-Zwillinge nicht vor Beginn ihres nächsten Schuljahres bei ihm auftauchten. "Das ist \emph{brillant,} Professor!"

Professor Quirrell wirkte nicht, als wüsste er das Kompliment zu schätzen. "Ich nehme an, ich kann Weihnachten in \emph{diesem} Sinne tolerieren, Mr. Potter, doch nur gerade eben so." Dann lächelte er schwach. "Natürlich wird Sie das vierzehn Galleonen kosten und Sie haben nur fünf."

"\emph{Fünf} Galleonen," sagte Harry und schniefte empört. "Was glaubt der Schulleiter eigentlich, mit wem er es zu tun hat?"

"Ich denke," sagte Professor Quirrell, "es kam ihm einfach nicht in den Sinn, die Konsequenzen zu fürchten, wenn Sie Ihre Genialität dem Auftreiben von Geldmitteln widmen würden. Obwohl Sie weise waren, zu verlieren, anstatt eine direkte Drohung auszusprechen. Nur aus Neugier, Mr. Potter, was \emph{hätten} Sie getan, wenn ich mich nicht gelangweilt abgewandt hätte, während Sie in einem Anfall kindischen Trotzes Knuts im Wert von fünf Galleonen abzählten?"

"Nun, der einfachste Weg wäre gewesen, Geld von Draco Malfoy zu leihen," sagte Harry.

Professor Quirrell ließ ein kurzes Kichern vernehmen. "Im Ernst, Mr. Potter."

\emph{Zur Kenntnis genommen.} "Wahrscheinlich hätte ich ein paar Promi-Auftritte gegeben. Ich würde nicht auf irgendetwas ökonomisch disruptives zurückgreifen, nur für ein bisschen Taschengeld." Harry hatte es überprüft, man \emph{würde} ihm erlauben, den Zeitumkehrer zu behalten, während er über die Feiertage nach Hause fuhr, damit sein Schlafzyklus nicht aus dem Takt geriet. Doch es war \emph{ebenso} möglich, dass irgendjemand Ausschau hielt nach magischen Tageshändlern.** Der Gold-und-Silber-Trick hätte einiges an Vorarbeit auf der Muggelseite benötigt, außerdem Startkapital und die Kobolde wären nach dem ersten Durchlauf vielleicht misstrauisch geworden. Und eine echte Bank zu gründen, würde \emph{einiges} an Arbeit mit sich bringen… Harry hatte es noch nicht \emph{wirklich} geschafft, irgendwelche Methoden auszuarbeiten, um Geld zu machen, die schnell \emph{und} zuverlässig \emph{und} sicher waren, daher war er froh gewesen, als sich Professor Quirrell als so einfach zu täuschen herausgestellt hatte.

"Ich hoffe doch, diese fünf Galleonen werden sich als ausreichend erweisen, da Sie sich solche Mühe gemacht haben, sie zu zählen," sagte Professor Quirrell. "Ich bezweifle, dass der Schulleiter sehr erpicht darauf sein wird, mir ein zweites mal den Schlüssel zu Ihrem Verlies anzuvertrauen, sobald er entdeckt, dass ich zum Narren gehalten wurde."

"Ich bin sicher, Sie haben Ihr bestes gegeben," sagte Harry in tiefer Dankbarkeit.

"Benötigen Sie vielleicht Hilfe dabei, einen sicheren Ort zu finden, um all diese Knuts aufzubewahren, Mr. Potter?"

"Nun, sozusagen," sagte Harry. "Kennen Sie irgendwelche guten Investitionsmöglichkeiten, Professor Quirrell?"

Und die beiden setzten, in ihrer winzigen Sphäre der Stille und Isolation, ihren Weg durch die leuchtenden und wimmelnden Massen fort und sah man genau hin, so konnte man erkennen, wie auf ihrem Pfad sich die grünen Äste herabbogen und Blumen verdorrten und Kinderspielzeuge, die freudige Melodien spielten, zu tieferen und deutlich düsteren Noten wechselten.

Harry bemerkte es \emph{durchaus,} doch er sagte nichts, lächelte nur leicht in sich hinein.

Jeder hatte seine eigene Art, die Feiertage zu begehen und der Grinch war ebenso sehr ein Teil von Weihnachten wie Santa.

* Diese Aufmachung ist offenbar eine Anspielung auf die Titelfigur der \emph{Chrestomanci}-Buchreihe von Diana Wynne Jones.\\ ** Ein \emph{Tageshändler} oder auch \emph{Daytrader} impliziert nicht, was man mit Bezug auf einen Zeitumkehrer möglicherweise vermuten könnte, sondern es handelt sich um einen Begriff aus dem Börsen-Bereich und bezeichnet dort einen Händler (auch \emph{Trader} genannt), der versucht durch Spekulationen auf kurzfristige Kursschwankungen, etwa von Währungen, Rohstoffen oder Aktien, Gewinne zu erzielen. Kauf und Verkauf der entsprechenden Wertpapiere finden hierbei in der Regel innerhalb eines Handelstages statt, daher der Name. Offensichtlich dürfte die Fähigkeit, zu wissen was in wenigen Stunden geschehen wird, bei dieser Tätigkeit einen enormen Vorteil bieten.

