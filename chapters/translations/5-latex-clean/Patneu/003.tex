

\hypertarget{die-realituxe4t-mit-ihren-alternativen-vergleichen}{% \section{3. Die Realität mit ihren Alternativen vergleichen}\label{die-realituxe4t-mit-ihren-alternativen-vergleichen}}

\textbf{Kapitel 3: Die Realität mit ihren Alternativen vergleichen}

Wenn J. K. Rowling euch nach dieser Geschichte fragt, wisst ihr von nichts.

\later

\emph{"Aber dann ist die Frage - wer?"}

\later

"Gütiger Gott," sagte der Barmann, Harry anstarrend," ist das - kann das sein -?"

Harry lehnte sich gegen die Bar des Tropfenden Kessels so gut er konnte, obwohl sie etwa bis zu den Spitzen seiner Augenbrauen aufragte. Eine \emph{solche} Frage verdiente sein absolut bestes.

"Bin ich - könnte ich sein - vielleicht - man kann nie wissen - ob ich's \emph{nicht} bin - aber dann ist die Frage -- \emph{wer?}"

"Du meine Güte," flüsterte der alte Barmann. "Harry Potter… welche Ehre."

Harry blinzelte, dann sammelte er sich. "Nun, ja, Sie sind sehr scharfsinnig; die meisten Leute kriegen das nicht so schnell mit -"

"Das ist genug," sagte Professor McGonagall. Ihre Hand schloss sich um Harrys Schulter. "Belästige den Jungen nicht, Tom, das ist alles neu für ihn."

"Aber er ist es?" trillerte eine alte Frau. " Es ist Harry Potter?" Mit einem kratzenden Geräusch erhob sie sich von ihrem Stuhl.

"Doris -" sagte McGonagall warnend. Der zornige Blick den sie durch den Raum schoss, hätte genug sein sollen, um jeden einzuschüchtern.

"Ich möchte nur seine Hand schütteln," flüsterte die Frau. Sie verbeugte sich tief und streckte eine faltige Hand aus, die Harry, sich verwirrt und unbehaglicher als jemals in seinem Leben fühlend, vorsichtig schüttelte. Tränen fielen aus den Augen der Frau auf ihre umklammernden Hände. "Mein Enkel war ein Auror," flüsterte sie ihm zu. "Ist neunundsiebzig gestorben. Danke, Harry Potter. Dem Himmel sei Dank für Sie."

"Gern geschehen," sagte Harry automatisch, drehte dann seinen Kopf und warf Professor McGonagall einen verängstigten, flehenden Blick zu.

Professor McGonagall stampfte mit dem Fuß auf, gerade als der allgemeine Ansturm zu beginnen drohte. Es versursachte ein Geräusch, das für Harry dem Ausdruck "Krachen der Verdammnis" eine neue Bedeutung verlieh und alle schienen festzufrieren.

"Wir haben es eilig," sagte Professor McGonagall mit einer Stimme, die absolut, vollkommen normal klang.

Sie verließen die Bar ohne Schwierigkeiten.

"Professor?" sagte Harry, sobald sie im Hinterhof waren. Er hatte fragen wollen, was vor sich ging, aber hörte sich merkwürdigerweise eine ganz andere Frage stellen. "Wer war der blasse Mann in der Ecke? Der Mann mit dem zuckenden Auge?"

"Hm?" sagte Professor McGonagall und klang etwas überrascht; vielleicht hatte sie diese Frage auch nicht erwartet. "Das war Professor Quirrell. Er wird dieses Jahr Verteidigung gegen die dunklen Künste in Hogwarts unterrichten."

"Ich hatte das merkwürdige Gefühl, dass ich ihn kannte…" Harry rieb sich die Stirn. "Und dass ich ihm nicht die Hand hätte schütteln sollen." Als ob man jemanden getroffen hätte, der einmal ein Freund gewesen war, bevor irgendetwas drastisch schief gegangen war… das war es nicht ganz, aber Harry fand keine Worte dafür. "Und was \emph{war…} das alles?"

Professor McGonagall bedachte ihn mit einem merkwürdigen Blick. "Mr. Potter… wissen Sie… wie \emph{viel} hat man Ihnen darüber erzählt… wie Ihre Eltern starben?"

Harry gab einen festen Blick zurück. "Meine Eltern sind am Leben und wohlauf und sie haben es immer abgelehnt, darüber zu sprechen, wie meine \emph{biologischen} Eltern starben. Woraus ich schließe, dass es unschön war."

"Eine bewundernswerte Loyalität," sagte Professor McGonagall. Ihre Stimme wurde tiefer. "Obwohl es mich etwas verletzt, es Sie so sagen zu hören. Lily und James waren meine Freunde."

Harry wandte den Blick ab, plötzlich beschämt. "Es tut mir leid," sagte er mit kleiner Stimme. "Aber ich \emph{habe} eine Mum und einen Dad. Und ich weiß, dass ich mich selbst nur unglücklich machen würde, indem ich die Realität mit… etwas Perfektem vergleiche, was ich in meiner Vorstellung erschaffen habe.

"Das ist erstaunlich weise von Ihnen," sagte Professor McGonagall leise. "Aber Ihre \emph{biologischen} Eltern sind in der Tat sehr gut gestorben, um Sie zu beschützen.

\emph{Um mich zu beschützen?}

Etwas Seltsames umklammerte Harrys Herz. "Was… \emph{ist} passiert?"

Professor McGonagall seufzte. Ihr Zauberstab tippte gegen Harrys Stirn und seine Sicht verschwamm für einen Moment. "Eine Art Verkleidung," sagte sie, "damit das nicht noch einmal passiert, nicht bis Sie bereit sind." Dann schnellte ihr zauberstab erneut heraus und tippte dreimal gegen eine Ziegelmauer…

… die sich zu einem Loch vertiefte und sich zu einem großen Torbogen dehnte und ausbreitete und zitterte, der eine lange Reihe von Geschäften offenbarte, die auf Schildern mit Kesseln und Drachenlebern warben.

Harry blinzelte nicht. Es war nicht, als ob sich irgendjemand in eine Katze verwandelte.

Und sie gingen vorwärts, zusammen, in die Zauberwelt.

Es gab Händler, die Sprungstiefel verhökerten ("Mit echtem Flubber hergestellt!") und "Messer +3! Gabeln +2! Löffel mit einem +4 Bonus!" Es gab Brillen, die alles, was man ansah, grün aussehen ließen und eine Reihe von gemütlichen Sesseln mit Schleudersitzen für Notfälle.

Harrys Kopf drehte und drehte sich, als ob er versuchte sich von seinen Schultern zu schrauben. Es war, als würde man durch den Abschnitt über magische Gegenstände eines Regelbuches von \emph{Dungeons and Dragons für Fortgeschrittene} spazieren (er spielte das Spiel nicht, las aber gern die Regelbücher). Harry wollte verzweifelt auch nicht einen einzigen der zum Verkauf stehenden Gegenstände verpassen, falls es einer der drei benötigten für die Unendliche-Wünsche-Schleife sein sollte.

Dann erspähte Harry etwas, dass ihn, ohne überhaupt darüber nachzudenken, von der Stellvertretenden Schulleiterin wegsteuern und direkt auf ein Geschäft zulaufen ließ, eine Fassade aus blauen Ziegelsteinen mit bronzener Verkleidung. Er wurde erst in die Realität zurückgeholt, als Professor McGonagall genau vor ihn trat.

"Mr. Potter?" sagte sie.

Harry blinzelte, erkannte dann, was er gerade getan hatte. "Es tut mir leid! Ich vergaß für einen Moment, dass ich mit Ihnen anstatt mit meiner Familie zusammen war." Harry deutete auf das Schaufenster, in welchem flammende Buchstaben, die blendend hell und doch entfernt schienen, die Worte \emph{Bigbam's Brilliante Bücher} bildeten. "Wenn man an einem Buchladen vorbeigeht, in dem man noch nicht war, muss man hereingehen und sich umsehen. Das ist Familientradition."

"Das ist das Ravenclaw-hafteste, das ich je gehört habe."

"Was?"

"Nichts. Mr. Potter, unser erster Schritt ist es, Gringotts zu besuchen, die Bank der Zauberwelt. Der Verließ Ihrer \emph{biologischen} Familie befindet sich dort, mit dem Erbe, welches Ihre \emph{biologischen} Eltern Ihnen hinterlassen haben und Sie werden Geld für Ihre Schulausstattung benötigen." Sie seufzte. "Und, ich nehme an, ein bestimmte Menge Taschengeld für Bücher wäre auch zu rechtfertigen. Obwohl Sie sich vielleicht für eine Weile zurückhalten möchten. Hogwarts hat eine ziemlich große Bibliothek über magische Themen. Und der Turm in dem Sie, wie ich stark vermute, leben werden hat selbst eine noch umfassendere Bibliothek. Jedes Buch, welches Sie jetzt kauften, wäre wahrscheinlich ein Duplikat."

Harry nickte und sie gingen weiter.

"Verstehen Sie mich nicht falsch, es ist eine \emph{großartige} Ablenkung," sagte Harry, während sein Kopf sich weiterdrehte, "wahrscheinliche die beste Ablenkung, die man mir jemals geboten hat, aber glauben Sie nicht, dass ich unsere noch ausstehende Diskussion vergessen habe."

Professor McGonagall seufzte. "Ihre Eltern - oder zumindest Ihre Mutter - mögen sehr weise gewesen sein, es Ihnen nicht zu sagen."

"Also wünschten Sie, ich könnte in seliger Unwissenheit verbleiben? Es gibt eine gewisse Schwachstelle in diesem Plan, Professor McGonagall."

"Ich nehme an, es wäre ziemlich zwecklos," sagte die Hexe knapp, "wenn jedermann auf der Straße Ihnen die Geschichte erzählen könnte. Nun gut."

Und sie erzählte ihm von Ihm-dessen-Name-nicht-genannt-werden-darf, dem Dunklen Lord, Voldemort.

"Voldemort?" flüsterte Harry. Es hätte komisch sein sollen, aber das war es nicht. Der Name brannte mit einem kalten Gefühl, Ruchlosigkeit, diamantener Klarheit, ein Hammer aus purem Titan, herabstoßend auf einen Amboss aus nachgiebigem Fleisch. Ein Schauer überlief Harry als er auch nur das Wort formulierte und er beschloss hier und jetzt sicherere Begriffe, wie Du-weißt-schon-wer zu verwenden.

Der Dunkle Lord hatte über das Britannien der Zauberer getobt, wie ein wildernder Wolf, am Gefüge ihrer täglichen Leben zerrend und reißend. Andere Länder hatten ihre Hände gerungen, aber gezögert einzugreifen, ob aus erbärmlicher Selbstsüchtigkeit oder simpler Furcht, dass wer immer die ersten unter ihnen wären, dem Dunklen Lord entgegenzutreten, ihr Frieden das nächste Ziel seines Terrors würde.

\emph{(Der Zuschauer-Effekt,} dachte Harry, an Latané und Darleys Experiment denkend, das gezeigt hatte, dass man wahrscheinlicher Hilfe bekäme, wenn man einen epileptischen Anfall vor einer anstatt drei Personen hätte. \emph{Zerstreuung der Verantwortung, jeder hoffend, dass jemand anders als erster handle.)}

Die Todesser waren dem Erwachen des Dunklen Lords und seiner Führung gefolgt, Aasgeier, an den Wunden pickend oder Schlangen, beißend und schwächend. Die Todesser waren nicht so schrecklich wie der Dunkle Lord, aber sie waren schrecklich und sie waren viele. Und die Todesser besaßen mehr als Zauberstäbe; es lag Reichtum in diesen maskierten Reihen und politische Macht und Geheimnisse für Erpressungen genutzt, um eine Gesellschaft zu lähmen, die versuchte, sich zu verteidigen.

Ein alter und resprektierter Journalist, Yermy Wibble, forderte höhere Steuern und Einberufungen. Er rief, es wäre absurd für die Vielen, in Angst vor den Wenigen zu kauern. Seine Haut, nur seine Haut, wurde am nächsten Tag an die Wand der Redaktion genagelt aufgefunden, neben den Häuten seiner Frau und zweier Töchter. Jeder wünschte, es möge etwas mehr getan werden und niemand wagte es, als erster den Vorschlag zu machen. Wer immer am meisten herausstach, wurde zum nächsten Exempel.

Bis die Namen von James und Lily Potter an die Spitze dieser Liste rückten.

Und diese zwei wären gestorben mit ihren Zauberstäben in den Händen und hätten nichts bereut, weil sie Helden \emph{waren}; aber sie hatten ein kleines Kind, ihren Sohn, Harry Potter.

Tränen stiegen Harry in die Augen. Er wischte sie weg, ärgerlich oder vielleicht verzweifelt, \emph{Ich kannte diese Leute nicht, nicht wirklich, sie sind} jetzt \emph{nicht meine Eltern, es wäre sinnlos für sie solche Trauer zu fühlen -}

Nachdem Harry nicht mehr in den Umhang der Hexe schluchzte, sah er auf und fühlte sich ein kleines bisschen besser, als er auch in Professor McGonagalls Augen Tränen sah.

"Also was ist passiert?" sagte Harry, mit bebender Stimme.

"Der Dunkle Lord kam nach Gordic's Hollow," sagte Professor McGonagall mit einem Flüstern. "Sie hätten verborgen sein sollen, aber wurden verraten. Der Dunkle Lord tötete James, er tötete Lily und am Ende kam er zu Ihnen, zu Ihrem Kinderbett. Er wirkte den Tödlichen Fluch auf Sie und dort endete es. Der Tödliche Fluch besteht aus purem Hass und schlägt direkt gegen die Seele, trennt sie vom Körper ab. Er kann nicht geblockt werden und wen immer er trifft, stirbt. Aber Sie überlebten. Sie sind die einzige Person, die jemals überlebt hat. Der Tödliche Fluch prallte zurück und streckte den Dunklen Lord nieder, nur die verbrannte Ruine seines Körpers und eine Narbe auf Ihrer Stirn zurücklassend. Deshalb, Harry Potter, möchten die Leute die Narbe auf Ihrer Stirn sehen und Ihnen die Hand schütteln."

Der Sturm des Weinens, der durch Harry hindurch geschwemmt war, hatte alle seine Tränen verbraucht; er konnte nicht wieder weinen, er war fertig.

(Und irgendwo in Harrys Hinterkopf war ein kleiner, kleiner Hinweis auf etwas Verwirrendes, ein Gefühl, das irgendetwas an der Geschichte nicht stimmte; es hätte Teil von Harrys Fähigkeiten sein sollen, diesen kleinen Hinweis zu bemerken, aber er war abgelenkt. Es ist eine traurige Wahrheit, dass immer dann, wenn man seine Künste als Rationalist am dringensten bräuchte, man sie am wahrscheinlichsten vergisst.)

Harry löste sich von Professor McGonagalls Seite. "Ich werde - darüber nachdenken müssen," sagte er und versuchte seine Stimme unter Kontrolle zu halten. Er starrte auf seine Schuhe. "Ähm. Sie können sie weiterhin meine Eltern nennen, wenn Sie möchten, Sie müssen nicht 'biologische Eltern' oder so etwas sagen. Ich schätze, es gibt keinen Grund, warum ich nicht zwei Mütter und zwei Väter haben kann."

Es kam kein Geräusch von Professor McGonagall.

Und sie liefen still zusammen, bis sie vor einem großen, weißen Gebäude ankamen, mit breiten bronzenen Türen und den eingemeißelten Worten \emph{Gringotts Bank} darüber.

