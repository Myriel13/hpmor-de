

\hypertarget{bestuxe4tigungsfehler}{% \section{8. Bestätigungsfehler}\label{bestuxe4tigungsfehler}}

\textbf{Kapitel 8: Bestätigungsfehler}

All diese Welten gehören J. K. Rowling, außer Europa. Versucht dort keine Fanfics.

\later

Ein alarmierter Rezensent fragte, ob wenn Luna eine Seherin sei, das bedeute, dies würde ein HPDM bottom!Draco mpreg* Fanfic. Ich bedaure, dass FFN mir keine höhere Schriftgröße erlaubt, in der ich \textbf{NEIN} sagen kann. Es war mir ehrlich nicht in den Sinn gekommen, Luna könne eine \emph{echte} Seherin sein—ich werde entscheiden müssen, ob ich das weiter verwenden will oder nicht—aber ich denke, wir können alle sicher annehmen, dass, wenn Luna eine Seherin \emph{ist}, sie etwas in der Richtung sagte, wie „das Licht pflanzt einen Samen in der Finsternis“ und Xenophilius dies, wie immer, auf eher falsche Weise interpretiert hat.

\later

„\emph{Erlaube mir, dich zu warnen, dass meine Genialität auf die Probe zu stellen, ein gefährliches Vorhaben ist und dein Leben erheblich surrealer machen könnte.}“

\later

Niemand hatte um Hilfe gebeten, das war das Problem. Sie waren nur am Reden, Essen oder Löcher-in-die-Luft-starren, während ihre Eltern Klatsch austauschten. Aus welch seltsamem Grund auch immer hatte sich niemand hingesetzt und ein Buch gelesen, was bedeutete, sie konnte sich nicht einfach daneben setzen und ihr eigenes Buch herausholen. Und auch als sie kühn die Initiative ergriffen hatte, indem sie sich hinsetzte und anfing, ein drittes mal \emph{Eine Geschichte von Hogwarts} durchzulesen, schien niemand geneigt, sich neben sie zu setzen.

Abgesehen davon, Leuten mit ihren Hausaufgaben zu helfen oder mit irgendetwas anderem, was sie brauchten, wusste sie nicht wirklich, wie man auf Leute zuging. Sie \emph{fühlte} sich nicht wie eine schüchterne Person. Sie dachte von sich als die Art Mädchen, das die Dinge anpackte. Und doch, irgendwie, erschien es ihr, wenn es keine Frage gab wie „ich weiß nicht mehr, wie schriftliche Division geht“, einfach zu \emph{unangenehm}, einfach auf jemanden zuzugehen und zu sagen… was? Sie hatte nie herausfinden können was. Und es schien keine Standard-Anleitung zu geben, was lächerlich war. Die ganze Angelegenheit des Leute-kennen-lernens schien ihr nie vernünftig zu sein. Warum musste \emph{sie}sich um alles kümmern, wenn zwei Leute daran beteiligt waren? Warum halfen die Erwachsenen nie dabei? Sie wünschte, irgendein anderes Mädchen würde einfach auf \emph{sie} zukommen und sagen, „Hermine, der Lehrer hat gesagt, wir sollen Freunde sein.“

Doch es sollte klar sein, dass Hermine Granger, die am ersten Schultag allein in einem der wenigen Abteile, die leer geblieben waren, im letzten Waggon des Zuges saß und die Abteiltür offen ließ, nur für den Fall, dass irgendjemand aus irgendeinem Grund mit ihr reden wollte, \emph{nicht} traurig, einsam, schwermütig, deprimiert, verzweifelt oder besessen von ihren Problemen war. Stattdessen las sie noch einmal, zum dritten mal, \emph{Eine Geschichte von Hogwarts} und genoss es wirklich, mit nur einem leichten Hauch von Verärgerung in ihrem Hinterkopf über die allgemeine Unvernünftigkeit der Welt.

Das Geräusch einer sich öffnenden Zug-Zwischentür war zu hören und dann kamen Schritte und ein merkwürdig schlitterndes Geräusch den Gang des Zuges hinunter. Hermine legte \emph{Eine Geschichte von Hogwarts} zur Seite und stand auf und streckte ihren Kopf nach draußen—nur für den Fall, dass jemand Hilfe brauchte—und sah einen kleinen Jungen in einem Zaubererumhang, wahrscheinlich im ersten oder zweiten Jahr, seiner Größe nach zu urteilen, der ziemlich dumm aussah mit einem Schal um seinen Kopf gebunden. Ein kleiner Koffer stand neben ihm auf dem Gang. Im Moment als sie ihn sah, klopfte er an die Tür eines anderen, geschlossenen Abteils und er sagte, mit von dem Schal nur leicht gedämpfter Stimme, „Entschuldigt, kann ich eine kurze Frage stellen?“

Sie hörte die Antwort aus dem Abteil nicht, aber nachdem der Junge die Tür öffnete, glaubte sie ihn sagen gehört zu haben—außer sie hatte sich irgendwie verhört—„Kennt irgendjemand hier die sechs Quarks oder weiß, wo ich eine Erstklässlerin namens Hermine Granger finden kann?“

Nachdem der Junge die Abteiltür geschlossen hatte, sagte Hermine, „Kann ich dir mit irgendwas helfen?“

Das Gesicht mit dem Schal drehte sich zu ihr um und die Stimme sagte, „Nicht, wenn du mir nicht die sechs Quarks nennen oder mir sagen kannst, wo ich Hermine Granger finde.“

„Up, Down, Strange, Charm, Truth, Beauty und warum suchst du nach ihr?“

Es war aus dieser Entfernung schwer zu sagen, aber sie glaubte den Jungen unter seinem Schal breit grinsen zu sehen. „Ah, also \emph{bist du} eine Erstklässlerin namens Hermine Granger,“ sagte die junge, gedämpfte Stimme. „Und auch noch im Zug nach Hogwarts.“ Der Junge begann sich ihr und ihrem Abteil zu nähern und sein Koffer schlitterte hinter ihm her. „Technisch gesehen brauchte ich nur nach dir zu \emph{suchen}, aber man darf wohl annehmen, dass ich mit dir reden oder dich in meine Gruppe einladen oder einen magischen Schlüsselgegenstand von dir kriegen oder rausfinden soll, dass Hogwarts über den Ruinen eines antiken Tempels erbaut wurde oder sowas. SC oder NSC, das ist die Frage?“

Hermine öffnete den Mund, um darauf zu antworten, aber sie wusste nicht, was sie antworten \emph{könnte} auf… \emph{was auch immer} sie da gerade gehört hatte, während der Junge zu ihr herüberging, in das Abteil hinein sah, zufrieden nickte und sich auf die Sitzbank gegenüber ihrer eigenen setzte. Sein Koffer wieselte hinter ihm her, wuchs zu dreifacher Größe heran und schmiegte sich auf seltsam beunruhigende Weise an ihren eigenen an.

„Bitte, setz dich,“ sagte der Junge „und würdest du bitte die Tür hinter dir schließen. Keine Sorge, ich beiße niemandem, der mich nicht zuerst beißt.“ Er begann bereits, den Schal von seinem Kopf abzuwickeln.

Die Unterstellung, dass dieser Junge dachte, sie hätte \emph{Angst} vor ihm, veranlasste ihre Hand die Tür zuschlittern und unnötig heftig in die Wand knallen zu lassen. Sie fuhr herum und sah ein junges Gesicht mit hellen, lachenden grünen Augen und einer zornigen roten Narbe auf der Stirn, die etwas in ihrem Hinterkopf anklingen ließ, aber im Moment hatte sie über wichtigere Dinge nachzudenken. „Ich habe nicht gesagt, ich wäre Hermine Granger!“

„\emph{Ich} sagte nicht, dass du \emph{sagtest}, du wärest Hermine Granger, ich sagte nur, du wärst Hermine Granger. Wenn du fragst, woher ich das weiß, ich weiß alles. Guten Abend, Ladies und Gentlemen, mein Name ist Harry James Potter-Evans-Verres oder kurz Harry Potter, ich weiß, das sagt \emph{dir} wahrscheinlich nicht viel—“

Hermines Verstand stellte endlich die Verbindung her. Die Narbe auf der Stirn, die Form eines Blitzes. „Harry Potter! Du bist in \emph{Geschichte der modernen Magie} und \emph{Aufstieg und Niedergang der dunklen Künste} und der \emph{Großen Chronik der Zauberer des zwanzigsten Jahrhunderts.}“ Es war tatsächlich das erste mal in ihrem Leben, dass sie jemanden aus einem \emph{Buch getroffen} hatte und das war ein ziemlich merkwürdiges Gefühl.

Der Junge blinzelte dreimal. Ich stehe in \emph{Büchern?} Moment, natürlich stehe ich in Büchern… was für ein seltsamer Gedanke."

„Meine Güte, wusstest du das nicht?“ sagte Hermine. „Ich hätte alles über mich rausgefunden, was ich kann, wenn ich es wäre.“

Der Junge sprach eher trocken. „Miss~Granger, es ist erst weniger als 72 Stunden her, seit ich in der Winkelgasse war und meinen Anspruch auf Ruhm entdeckt habe. Ich habe die letzten zwei Tage damit verbracht, wissenschaftliche Bücher zu kaufen. \emph{Glaub mir,} ich habe vor, alles herauszufinden, was ich kann.“ Der Junge zögerte. „Was \emph{sagen} die Bücher über mich?“

Hermine Grangers Verstand vollführte eine Rückblende, sie hatte nicht erwartet, über \emph{diese} Bücher abgefragt zu werden, also hatte sie sie nur einmal gelesen, aber es war erst einen Monat her, also war ihr das Material noch frisch im Gedächtnis. „Du bist der einzige, der den Tödlichen Fluch überlebt hat, deshalb wirst du der Junge-der-überlebt-hat genannt. Du wurdest am 31. Juli 1980 als Kind von James Potter und Lily Potter, ehemals Lily Evans, geboren. Am 31. Oktober 1981 griff der Dunkle Lord Er-dessen-Name-nicht-genannt-werden-darf, obwohl ich nicht weiß, wieso nicht, dein Zuhause an. Du wurdest lebend mit der Narbe auf deiner Stirn in den Ruinen deines Elternhauses neben den verbrannten Überresten von Du-weißt-schon-wers Körper gefunden. Großmeister Albus Percival Wulfric Brian Dumbledore hat dich irgendwohin weggeschickt, niemand weiß wohin. Der \emph{Aufstieg und Niedergang der dunklen Künste} behauptet, du hättest aufgrund der Liebe deiner Mutter überlebt und dass deine Narbe alle magischen Fähigkeiten von Du-weiß-schon-wem enthält und das die Zentauren dich fürchten, aber die \emph{Große Chronik der Zauberer des zwanzigsten Jahrhunderts} erwähnt nichts dergleichen und die \emph{Geschichte der modernen Magie} warnt, dass es eine Menge verrückte Theorien über dich gibt.“

Der Mund des Jungen hing weit offen. „Wurde dir gesagt, du sollst im Zug nach Hogwarts auf Harry Potter warten oder sowas?“

„Nein,“ sagte Hermine. „Wer hat dir von \emph{mir} erzählt?“

„Professor McGonagall und ich glaube, ich verstehe warum. Hast du ein eidetisches Gedächtnis, Hermine?“

Hermine schüttelte den Kopf. „Es ist nicht fotografisch, ich habe mir immer gewünscht, das wäre es, aber ich musste meine Schulbücher fünf mal durchlesen, um sie mir alle zu merken.“

„Wirklich,“ sagte der Junge mit etwas erstickter Stimme. „Ich hoffe, du hast nichts dagegen, wenn ich das überprüfe—es ist nicht so, dass ich dir nicht glaube, aber wie sagt man, 'Vertrauen ist gut, Kontrolle ist besser'. Kein Grund, mir Gedanken zu machen, wenn ich einfach das Experiment machen kann.“

Hermine lächelte ziemlich selbstzufrieden. Sie liebte Tests ja so. „Leg los.“

Der Junge steckte eine Hand in einen Beutel an seiner Seite und sagte „Zaubertränke und Zauberbräue von Arsenius Bunsen“. Als er seine Hand wieder herauszog, hielt sie das Buch, das er genannt hatte.

Augenblicklich wollte Hermine einen dieser Beutel, mehr als sie je zuvor irgendwas gewollt hatte.

Der Junge öffnete das Buch irgendwo in der Mitte und sah nach unten. „Wenn du \emph{beißendes Öl} herstellen würdest—“

„Ich kann diese Seite von hier aus sehen, weißt du!“

Der Junge kippte das Buch, so dass sie es nicht mehr sehen konnte und blätterte erneut durch die Seiten. „Wenn du einen Trank des \emph{Spinnen-Kletterns} brauen würdest, was wäre die nächste Zutat nach dem Hinzufügen der Acromantula-Seide?“

„Nach dem Hineingeben der Seide, warte bis der Trank exakt die Farbe des wolkenlosen Morgenhimmels, 8 Grad vom Horizont entfernt und 8 Minuten bevor die Spitze der Sonne zum ersten mal sichtbar wird, angenommen hat. Rühre achtmal entgegengesetzt und einmal im Uhrzeigersinn, dann füge acht Quäntchen Einhorn-Popel hinzu.“

Der Junge ließ das Buch scharf zuschnappen und legte es in seinen Beutel zurück, der es mit einem kleinen rülpsenden Geräusch schluckte. „Gut gut gut \emph{gut} gut gut. Ich würde Ihnen gern einen Antrag machen, Miss~Granger.“

„Einen Antrag?“ sagte Hermine misstrauisch. Mädchen solltem dem eigentlich nicht nachgeben.

Es geschah auch an diesem Punkt, dass Hermine sich der anderen Sache—nun, einer der anderen Sachen—bewusst wurde, die an dem Jungen seltsam waren. Wie es schien klangen Leute, die \emph{in} Büchern waren, auch \emph{wie} ein Buch, wenn sie redeten. Das war eine ziemlich überraschende Entdeckung.

Der Junge griff in seinen Beutel und sagte, „Dose mit Limo“ und zog einen hellgrünen Zylinder heraus. Er streckte seine Hand aus und sagte, „Kann ich dir etwas zu trinken anbieten?“

Hermine nahm die Limonade höflich an. Tatsächlich \emph{fühlte} sie sich jetzt etwas durstig. „Vielen Dank,“ sagte Hermine, als sie den Deckel öffnete. „War das dein Vorschlag?“

Der Junge hustete. „Nein,“ sagte er. Genau als Hermine anfing zu trinken, sagte er, „ich hätte gern, dass du mir hilfst, das Universum zu beherrschen.“

Hermine trank zu Ende und setzte die Dose ab. „Nein danke, ich bin nicht böse.“

Der Junge sah sie überrascht an, als hätte er eine andere Antwort erwartet. „Nun, ich habe das eher rhetorisch gemeint,“ sagte er. „Mehr im Sinne des Bacon'schen Vorhabens**, weißt du, nicht politischer Macht. 'Alles zu bewirken, was möglich ist'*** und so weiter. Ich will experimentelle Studien über Zauber durchführen, die zu Grunde liegenden Regeln herausfinden, die Magie ins Reich der Wissenschaft holen, die Zauber- und Muggelwelten vereinen, den Lebensstandard des gesamten Planeten erhöhen, die menschliche Gesellschaft um Jahrhunderte voranbringen, das Geheimnis der Unsterblichkeit entdecken, das Sonnensystem kolonisieren, die Galaxis erkunden und was am wichtigsten ist, herausfinden, was zum Kuckuck hier eigentlich vor sich geht, weil all das himmelschreiend unmöglich ist.“

Das klang schon interessanter. „Und?“

Der Junge starrte sie ungläubig an. „\emph{Und? Reicht} das nicht?“

„Und was willst du von mir?“ sagte Hermine.

„Ich will natürlich, dass du mir bei der Forschung hilfst. Mit deinem enzyklopädischen Wissen und meiner Intelligenz und Rationalität, werden wir das Bacon'sche Projekt in null komma nichts abgeschlossen haben, wobei ich mit 'null komma nichts' wahrscheinlich mindestens fünfunddreißg Jahre meine.“

Hermine begann diesen Jungen lästig zu finden. „Ich habe dich noch nichts intelligentes machen sehen. Vielleicht lasse ich \emph{dich} bei \emph{meiner} Forschung helfen.“

Es herrschte Stille in dem Abteil.

„Du forderst mich also auf, meine Intelligenz zu demonstrieren,“ sagte der Junge nach einer langen Pause.

Hermine nickte.

„Ich warne dich, dass meine Genialität auf die Probe zu stellen, ein gefährliches Vorhaben ist und dein Leben erheblich surrealer machen könnte.“

„Ich bin noch nicht beeindruckt,“ sagte Hermine. Unbemerkt stieg das grüne Getränk einmal mehr an ihre Lippen.

„Nun, vielleicht wird dich \emph{das} beeindrucken,“ sagte der Junge. Er lehnte sich vor und sah sie konzentriert an. „Ich habe bereits ein bisschen experimentiert und herausgefunden, dass ich den Zauberstab nicht brauche, ich kann alles, was ich will, passieren lassen, in dem ich nur mit den Fingern schnipse.“

Er sagte es, als Hermine gerade am Schlucken war und sie würgte und hustete und spuckte die hellgrüne Flüssigkeit aus.

Auf ihren brandneuen, noch ungetragenen Hexenumhang, am ersten Schultag.

Hermine schrie tatsächlich. Es war ein hochgestochenes Geräusch, das in dem geschlossenen Abteil wie ein Fliegeralarm klang. „\emph{Iih! Meine Sachen!}“

„Keine Panik!“ sagte der Junge. „Ich krieg das hin. Schau her!“ Er hob eine Hand und schnippte mit den Fingern.

„Du—“ Dann sah sie an sich herunter.

Die grüne Flüssigkeit war immer noch da, aber noch während sie hinsah, fing sie an zu verschwinden und zu verblassen und nach ein paar Augenblicken, war es, als hätte sie niemals etwas auf sich verschüttet.

Hermine starrte den Jungen an, der ein ziemlich selbstzufriedenes Lächeln aufgesetzt hatte.

Ungesagte, zauberstablose Magie! In \emph{seinem} Alter? Wenn er die Schulbücher erst vor \emph{drei Tagen} bekommen hatte?

Dann erinnerte sie sich, was sie gelesen hatte und sie keuchte und wich vor ihm zurück. \emph{Alle magischen Kräfte des Dunklen Lords! In seiner Narbe!}

Sie sprang hastig auf die Füße. „Ich, ich, ich muss auf die Toilette, warte hier, okay—“ Sie musste einen Erwachsenen finden, sie musste es ihnen sagen—

Das Lächeln des Jungen verblasste. „Es war nur ein Trick, Hermine. Tut mir leid, ich wollte dir keine Angst machen.“

Ihre Hand stoppte auf dem Türgriff. „Ein \emph{Trick?}“

„Ja,“ sagte der Junge. „Du hast mich aufgefordert, meine Intelligenz zu demonstrieren. Also habe ich etwas anscheinend unmögliches gemacht, was immer ein guter Weg ist, um anzugeben. Ich kann nicht \emph{wirklich} alles tun, indem ich nur mit den Fingern schnipse.“ Der Junge hielt inne. „Zumindest \emph{glaube} ich das nicht, ich hab's nie wirklich experimentell gestestet.“ Der Junge hob die Hand und schnippte wieder mit den Fingern. „Nö, keine Banane.“

Hermine war so verwirrt, wie noch nie in ihrem Leben.

Der Junge schmunzelte nun wieder beim Anblick ihres Gesichts. „Ich habe dich \emph{gewarnt}, dass meine Genialität auf die Probe zu stellen, dein Leben sehr viel surrealer macht. Merk dir das für's nächste mal, wenn ich dich vor etwas warne.“

„Aber, aber,“ stammelte Hermine. „Was hast du dann \emph{gemacht?}“

Der Blick des Jungen gewann eine bedächtige, abwägende Qualität, die sie nie zuvor von jemandem in ihrem Alter gesehen hatte. „Du denkst, du hast das Zeug, um selbst eine Wissenschaftlerin zu sein, mit oder ohne meine Hilfe? Dann lass uns sehen, wie \emph{du} ein verwirrendes Phänomen untersuchst.“

„Ich…“ Hermines Kopf wurde einen Moment lang leer. Sie liebte Tests, aber \emph{so} einen hatte sie noch nie erlebt. Hastig versuchte sie sich an alles zu erinnern, was sie darüber gelesen hatte, was Wissenschaftler tun sollten. Sie zermarterte sich das Gehirn, in ihrem Geist ratterten Zahnräder und schließlich spuckte er die Anweisungen aus, um eine wissenschaftliche Untersuchung durchzuführen:

\emph{Schritt 1: Entwirf eine Hypothese.

Schritt 2: Führe ein Experiment durch, um sie zu überprüfen.

Schritt 3: Messe die Ergebnisse.

Schritt 4: Erstelle eine Präsentation.}

Schritt 1 war, eine Hypothese aufzustellen. Das bedeutete, sich zu überlegen, was gerade passiert sein \emph{könnte}. „Alles klar. Meine Hypothese ist, dass du einen Zauber auf meinen Umhang gewirkt hast, um alles, was darauf geschüttet wird, verschwinden zu lassen.“

„Alles klar,“ sagte der Junge, „ist das deine Antwort?“

Der Schock ließ nach und Hermines Verstand begann, richtig zu arbeiten. „Warte, das kann nicht richtig sein. Ich habe dich nicht deinen Zauberstab berühren oder irgendwelche Zaubersprüche aufsagen sehen, also wie hättest du einen Zauber wirken können?“

Der Junge wartete, sein Gesicht nichtssagend.

„Aber angenommen, alle Umhänge kommen bereits \emph{aus dem Laden} mit einem Zauber auf ihnen, der sie sauber hält, was für sie durchaus nützlich wäre. Das hast du herausgefunden, indem du vorher etwas auf \emph{dich selbst} verschüttet hast.“

Jetzt hoben sich die Augenbrauen des Jungen. „Ist \emph{das} deine Antwort?“

„Nein, ich habe Schritt 2 noch nicht durchgeführt, 'Führe ein Experiment durch, um deine Hypothese zu überprüfen.'“

Der Junge schloss seinen Mund wieder und begann zu lächeln.

Hermine sah die Dose an, die sie automatisch in den Getränkehalter am Fenster gestellt hatte. Sie nahm sie hoch, blickte hinein und sah, dass sie etwa zu einem Drittel voll war.

„Nun,“ sagte Hermine, „das Experiment, das ich durchführen will, sie es auf meinen Umhang zu gießen und zu sehen, was passiert und meine Vorhersage ist, dass der Fleck verschwinden wird. Nur, wenn es \emph{nicht} funktioniert, ist mein Umhang dreckig und das will ich nicht.“

„Mach es mit meinem,“ sagte der Junge, „so musst du dir keine Sorgen machen, dass dein Umhang dreckig wird.“

„Aber—“ sagte Hermine. Irgendetwas war\emph{falsch} an diesem Gedanken, aber sie wusste nicht genau, wie sie es ausdrücken sollte.

„Ich habe Ersatz-Umhänge in meinem Koffer,“ sagte der Junge.

„Aber du kannst dich nirgendwo umziehen,“ warf Hermine ein. Dann dachte sie besser darüber nach. „Obwohl ich annehme, ich könnte rausgehen und die Tür schließen—“

„Ich habe auch einen Platz zum Umziehen in meinem Koffer.“

Hermine sah seinen Koffer an, der, wie sie zu vermuten begann, etwas besonderer als ihr eigener war.

„Alles klar,“ sagte Hermine, „wenn du das sagst“ und sie goss eher zögerlich ein bisschen grüne Limo auf einen Zipfel des Umhangs des Jungen. Dann starrte sie ihn an, versuchte sich zu erinnern, wie lange die ursprüngliche Flüssigkeit gebraucht hatte, um zu verschwinden…

Und der grüne Fleck verschwand!

Hermine stieß einen erleichterten Seufzer aus, nicht zuletzt, weil das bedeutete, dass sie es nicht mit der geballten magischen Macht des Dunklen Lords zu tun hatte.

Nun, Schritt 3 war, die Ergebnisse zu messen, aber in diesem Fall hieß das nur, zu sehen, dass der Fleck verschwunden war. Und sie nahm an, sie konnte Schritt 4, mit der Präsentation, wahrscheinlich überspringen. „Meine Antwort ist, dass die Umhänge verzaubert sind, um sich selbst sauber zu halten.“

„Nicht ganz,“ sagte der Junge.

Hermine fühlte einen Stich der Enttäuschung. Sie wünschte, sie \emph{würde} ihn nicht fühlen, der Junge war kein Lehrer, aber es war trotzdem ein Test und sie hatte eine Frage falsch beantwortet und das fühlte sich immer an wie ein Schlag in die Magengrube.

(Es sagte fast alles, was man über Hermine Granger wissen musste, dass sie sich davon niemals hatte aufhalten lassen oder sie das auch nur ihre Liebe zu Tests beeinflussen ließ.)

„Das Traurige ist,“ sagte der Junge, „du hast wahrscheinlich alles gemacht, was das Buch dir gesagt hat, das du tun sollst. Du hast eine Vorhersage gemacht, die unterscheiden würde, ob der Umhang verzaubert oder nicht verzaubert ist und sie überprüft und die Null-Hypothese verworfen, dass die Umhänge nicht verzaubert sind. Aber wenn du nicht die wirklich, wirklich besten Bücher gelesen hast, bringen sie dir nicht ganz bei, wie man \emph{richtig} wissenschaftlich arbeitet. Gut genug, um \emph{wirklich} die richtige Antwort zu bekommen und nicht bloß eine weitere Veröffentlichung herauszuwerfen, worüber Dad sich immer beschwert. Also lass mich versuchen, zu erklären—ohne die Antwort zu verraten—was du dieses mal falsch gemacht hast und ich gebe dir noch eine Chance.“

Sie fing an, dem Jungen seinen ach-so-überlegenen Tonfall übelzunehmen, aber das war zweitrangig, im Vergleich dazu herauszufinden, was sie falsch gemacht hatte. „Alles klar.“

Der Gesichtsausdruck des Jungen wurde härter. „Dieses Spiel basiert auf einem berühmten Experiment, bekannt als die 2–4–6-Aufgabe und so funktioniert es. Ich habe eine \emph{Regel}—die ich kenne, du aber nicht—die auf einige Gruppen von drei Zahlen zutrifft, aber auf andere nicht. 2–4–6 ist ein Beispiel einer Dreiergruppe, auf die die Regel zutrifft. Eigentlich… lass mich die Regel aufschreiben, damit du weißt, dass es eine feste Regel ist, sie zusammenfalten und dir geben. Bitte sieh nicht hin, weil ich von vorhin schließe, dass du verkehrt herum lesen kannst.“

Der Junge sagte „Papier“ und „mechanischer Bleistift“ zu seinem Beutel und sie schloss fest die Augen, während er schrieb.

„So,“ sagte der Junge und hielt ein eng gefaltetes Stück Papier. „Steck das in deine Tasche“ und sie tat es.

„Jetzt die Spielregeln,“ sagte der Junge, „du gibst mir eine Dreiergruppe von Zahlen und ich sage 'Ja', wenn die drei Zahlen ein Beispiel für die Regel sind und 'Nein', wenn sie es nicht sind. Ich bin die Natur, die Regel ist eins meiner Gesetze und du erforscht mich. Du weißt bereits, dass 2–4–6 ein 'Ja' bekommt. Wenn du alle weiteren experimentellen Tests durchgeführt hast, die du willst—mich nach so vielen Dreiergruppen gefragt hast, wie du für nötig hältst—hörst du auf und errätst die Regel und dann kannst du das Stück Papier entfalten und sehen, wie du dich geschlagen hast. Verstehst du das Spiel?“

„Natürlich tue ich das,“ sagte Hermine.

„Leg los.“

„4–6–8“ sagte Hermine.

„Ja,“ sagte der Junge

„10–12–14“, sagte Hermine.

„Ja,“ sagte der Junge.

Hermine versuchte, sich etwas weitergehendes auszudenken, da es schien, als hätte sie bereits alle Tests gemacht, die sie brauchte und doch konnte es nicht so einfach sein, oder?

„1–3–5.“

„Ja.“

„Minus 3, minus 1, plus 1.“

„Ja.“

Hermine fiel nichts anderes mehr ein, was sie tun könnte. „Die Regel ist, dass die Zahlen jedes mal um zwei höher werden müssen.“

„Jetzt nimm an, ich sage dir,“ sagte der Junge, „dass dieser Test schwieriger ist, als er aussieht und nur 20\% der Erwachsenen ihn richtig machen.“

Hermine runzelte die Stirn. Was hatte sie übersehen? Dann, plötzlich, fiel ihr ein Test ein, den sie noch machen musste.

„2–5–8!“ sagte sie triumphierend.

„Ja.“

„10–20–30!“

„Ja.“

„Die richtige Antwort ist, dass die Zahlen jedesmal um die \emph{gleiche} Menge ansteigen müssen. Es müssen nicht 2 sein.“

„Nun gut,“ sagte der Junge, „nimm das Stück Papier heraus und sieh dir an, wie du dich geschlagen hast.“

Hermine nahm das Papier aus ihrer Tasche und entfaltete es.

\emph{Drei reelle Zahlen in aufsteigender Reihenfolge, von der niedrigsten zur höchsten.}

Hermines Kinnlade klappte herunter. Sie hatte das deutliche Gefühl, dass ihr etwas furchtbar unfaires angetan worden war, dass der Junge ein dreckiger, schäbiger, schummelnder Lügner war, aber als sie sich zurück erinnerte, fielen ihr keine falschen Antworten ein, die er gegeben hatte.

„Was du gerade entdeckt hast, wird 'Bestätigungsfehler' genannt,“ sagte der Junge. „Du hattest eine Regel in deinem Kopf und hast dir immer weiter Dreiergruppen ausgedacht, die die Regel 'Ja' sagen lassen sollten. Aber du hast nicht versucht, irgendwelche Dreiergruppen zu testen, die die Regel 'Nein' sagen lassen sollten. Tatsächlich hast du nicht ein \emph{einziges} 'Nein' bekommen, also hätte genau so gut 'egal welche drei Zahlen' die Regel sein können. Es ist in etwa so, als wenn Leute sich Experimente ausdenken, die ihre Hypothese bestätigen könnten, anstatt zu versuchen, sich Experimente auszudenken, die sie widerlegen könnten—das ist nicht ganz genau der selbe Fehler, aber es ist nah dran. Du musst lernen, die negative Seite der Dinge zu betrachten, in die Dunkelheit zu starren. Wenn dieses Experiment durchgeführt wird, geben nur 20\% der Erwachsenen die richtige Antwort. Und viele der anderen erfinden fantastisch komplizierte Hypothesen und setzen großes Vertrauen in ihre falschen Antworten, weil sie so viele Experimente gemacht haben und alles war, wie sie erwarteten.“

„Nun,“ sagte der Junge, „willst du es nochmal mit dem ursprünglichen Problem probieren?“

Ihr Blick war jetzt sehr konzentriert, als wäre dies der \emph{wahre} Test.

Hermine schloss die Augen und versuchte, sich zu konzentrieren. Sie schwitzte unter ihrem Umhang. Sie hatte das merkwürdige Gefühl, dass dies das schwierigste war, worüber sie je bei einem Test hatte nachdenken sollen oder vielleicht sogar das \emph{erste} mal, dass sie in einem Test nachdenken sollte.

Was für ein anderes Experiment konnte sie machen? Sie hatte einen Schokofrosch, könnte sie etwas davon auf den Umhang reiben und sehen, ob \emph{das} verschwand? Aber das schien immer noch nicht die Art von verdrehtem negativem Denken zu sein, nach der der Junge verlangte. Als ob sie immer noch nach einem 'Ja' fragte, ob der Schokofrosch-Fleck verschwand, anstatt nach einem 'Nein' zu fragen.

Also… ihrer Hypothese nach… wann sollte die Limo… \emph{nicht} verschwinden?

„Ich habe ein Experiment zum Durchführen,“ sagte Hermine. „Ich will etwas Limo auf den Boden schütten, um zu sehen, ob sie \emph{nicht} verschwindet. Hast du ein paar Papiertücher in deinem Beutel, damit ich die Pfütze aufwischen kann, wenn es nicht funktioniert?“

„Ich habe Servietten,“ sagte der Junge. Sein Gesicht war immer noch neutral.

Hermine nahm die Dose und schüttete ein kleines bisschen Limo auf den Boden.

Ein paar Sekunden später verschwand sie.

Dann traf sie die Erkenntnis und sie wollte sich selbst einen Tritt verpassen. „Natürlich! \emph{Du} hast mir diese Dose gegeben! Es ist nicht der Umhang, der verzaubert ist, es war die ganze Zeit die Limo!“

Der Junge stand auf und verbeugte sich feierlich. Er trug jetzt ein breites Grinsen zur Schau. „Dann… darf ich dir bei deiner Forschung behilflich sein, Hermine Granger?“

„Ich, ah…“ Hermine fühlte immer noch den Schub der Euphorie, aber sie war nicht ganz sicher, wie sie \emph{das} beantworten sollte.

Sie wurden von einem schwachen, zaghaften, leisen, eher \emph{zurückhaltenden} Klopfen an der Tür abgelenkt.

Der Junge drehte sich um und sah aus dem Fenster und sagte, „ich trage meinen Schal nicht, also kannst du das übernehmen?“

An diesem Punkt erkannte Hermine warum der Junge—nein, der Junge-der-überlebt-hat, Harry Potter—den Schal überhaupt getragen hatte und fühlte sich etwas dumm, dass sie es nicht früher erkannt hatte. Es war eigentlich irgendwie merkwürdig, weil sie angenommen hätte, Harry Potter würde sich stolz der Welt präsentieren und ihr kam der Gedanke, dass er tatsächlich schüchterner sein könnte, als er schien.

Als Hermine die Tür aufzog, wurde sie von einem zitternden kleinen Jungen begrüßt, der ganz genau so aussah, wie er klopfte.

„Entschuldigt,“ sagte der Junge mit winziger Stimme, „ich bin Neville Longbottom. Ich suche nach meiner Kröte, ich, ich kann sie nirgendwo in diesem Waggon finden… habt ihr meine Kröte gesehen?“

„Nein,“ sagte Hermine und dann gab ihre Hilfsbereitschaft Vollgas. „Hast du in allen anderen Abteilen nachgesehen?“

„Ja,“ flüsterte der Junge.

„Dann werden wir einfach alle anderen Waggons durchsuchen müssen,“ sagte Hermine forsch. „Ich werde dir helfen. Mein Name ist übrigens Hermine Granger.“

Der Junge sah aus, als wolle er vor Dankbarkeit in Ohnmacht fallen.

„Warte mal,“ ertönte die Stimme des \emph{anderen} Jungen—Harry Potter. „Ich bin nicht sicher, dass das die beste Vorgehensweise ist.“

Jetzt sah Neville aus, als würde er losheulen und Hermine fuhr verärgert herum. Wenn Harry Potter die Sorte Mensch war, die einen kleinen Jungen im Stich lassen würde, nur weil er nicht unterbrochen werden wollte… „Was? Warum \emph{nicht?}“

„Nun,“ sagte Harry Potter, „Es wird eine Weile dauern, den ganzen Zug per Hand abzusuchen und wir könnten die Kröte trotzdem übersehen und wenn wir sie, bis wir in Hogwarts sind, nicht finden würden, wäre er in Schwierigkeiten. Also würde es sehr viel mehr Sinn machen, wenn er direkt in den vordersten Waggon ginge, wo die Vertrauensschüler sind und einen Vertrauensschüler um Hilfe bitten würde. Das war das erste, was ich gemacht habe, als ich nach dir gesucht habe, Hermine, obwohl sie es nicht wirklich wussten. Aber sie haben vielleicht Zauber oder magische Gegenstände, die es sehr viel einfacher machen würden, eine Kröte zu finden. Wir sind nur Erstklässler.“

Das… \emph{machte} sehr viel Sinn.

„Denkst du, du schaffst es allein zum Waggon der Vertrauensschüler?“ fragte Harry Potter. „Ich habe ein paar Gründe, mein Gesicht nicht zu oft zu zeigen.“

Plötzlich keuchte Neville und trat einen Schritt zurück. „Ich erinnere mich an diese Stimme! Du bist einer der Lords des Chaos! \emph{Du bist der, der mir Schokolade gegeben hat!}“

Was? Was was \emph{was?}

Harry Potter wandte sich vom Fenster ab und richtete sich dramatisch auf. „\emph{Niemals!}“ sagte er, die Stimme voller Empörung. „Sehe ich wie so ein Schurke aus, der einem Kind Süßigkeiten geben würde?“

Nevilles Augen weiteten sich. „\emph{Du bist} Harry Potter? \emph{Der} Harry Potter? \emph{Du?}“

„Nein, nur \emph{ein} Harry Potter, es gibt drei von mir in diesem Zug—“

Neville kreischte kurz und rannte weg. Man hörte kurz das Trippeln hastiger Schritte und dann das Geräusch einer Waggontür, die sich öffnete und wieder schloss.

Hermine ließ sich schwer auf ihre Bank fallen. Harry Potter schloss die Tür und setzte sich dann neben sie.

„Kannst du mir bitte erklären, was vor sich geht?“ sagte Hermine mit schwacher Stimme. Sie fragte sich, ob mit Harry Potter rumzuhängen, bedeutete, ständig verwirrt zu sein.

„Oh, na ja, was passiert ist, ist, dass Fred und George und ich diesen armen kleinen Jungen am Bahnhof gesehen haben—die Frau neben ihm war ein Stück weggegangen und er sah richtig verängstigt aus, als wäre er sicher, von Todessern angegriffen zu werden oder so. Nun gibt es eine Redensart, dass die Angst vor etwas oft schlimmer ist, als die Sache selbst, also kam mir der Gedanke, dass das ein Bursche war, der tatsächlich davon profitieren könnte, seinen schlimmsten Alptraum wahr werden zu sehen und dass es nicht so schlimm war, wie er befürchtete—“

Hermine saß dort mit weit aufhängendem Mund.

„- und Fred und George fiel dieser Zauber ein, um die Schals über unseren Gesichtern finster und verschwommen zu machen, als wären wir untote Könige und das wären unsere Grabtücher—“

Ihr gefiel überhaupt nicht, wo das alles hinführte.

„- und nachdem wir ihm all die Süßigkeiten gegeben hatten, die ich gekauft hatte, machten wir so 'Geben wir ihm noch ein bisschen Geld! Ha ha ha! Nimm ein paar Knuts, Junge! Nimm 'nen silbernen Sickel!' und tanzten um ihn herum und lachten boshaft und so weiter. Ich glaube, es gab zunächst ein paar Leute in der Menge, die eingreifen wollten, aber Zuschaueruntätigkeit hielt sie zumindest so lange davon ab, bis sie merkten, was wir machten und dann, denke ich, waren sie alle zu verwirrt, um irgendwas zu tun. Schließlich sagte er mit seinem winzigen leisen Flüstern 'geht weg' und wir drei schrieen und rannten weg, irgendwas darüber kreischend, dass das Licht uns verbrenne. Hoffentlich wird er in Zukunft nicht mehr so große Angst davor haben, gemobbt zu werden. Das nennt man, nebenbei, Desensibilisierungs-Therapie.“

Okay, sie hatte \emph{nicht} richtig geraten, wo das hinführte.

Das brennende Feuer der Empörung, das einer von Hermines Hauptantrieben war, erwachte stotternd zum Leben, obwohl ein Teil von ihr irgendwie \emph{schon} verstand, was sie zu tun versucht hatten. „Das ist furchtbar! \emph{Du bist} furchtbar! Der arme Junge! Was du gemacht hast, war \emph{gemein!}“

„Ich denke, das Wort, nach dem du suchst, ist \emph{unterhaltsam} und in jedem Fall stellst du die falsche Frage. Die Frage ist, hat es mehr genutzt als geschadet oder mehr geschadet als genutzt? Wenn du irgendwelche Argumente zu \emph{dieser} Frage beizusteuern hast, höre ich sie mir gerne an, aber ich werde mich keinen weiteren Vorwürfen aussetzen, bis der erledigt ist. Ich stimme sicherlich zu, dass das, was ich gemacht habe, ganz schrecklich und mobbend und gemein \emph{aussah}, weil es einen ängstlichen kleinen Jungen beinhaltet und so, aber das ist wohl kaum die Hauptsache, oder? Das nennt sich nebenbei \emph{Konsequentialismus}, es bedeutet, dass, ob eine Handlung richtig oder falsch ist, nicht davon abhängt, ob sie schlecht oder gemein oder so \emph{aussieht}, die einzige Frage ist, wie sie sich am Ende auswirkt—was die Konsequenzen sind.“

Hermine öffnete den Mund, um etwas absolut \emph{ätzendes} zu sagen, aber unglücklicherweise schien sie den Teil versäumt zu haben, wo sie sich etwas einfallen ließ, was sie sagen sollte, bevor sie ihren Mund öffnete. Alles was ihr einfiel, war, „Was, wenn er \emph{Alpträume} kriegt?“

„Ganz ehrlich, ich glaube nicht, dass er unsere Hilfe brauchte, um Alpträume zu haben und wenn er stattdessen \emph{darüber} Alpträume hat, dann werden es Alpträume über schreckliche Monster sein, die einem Schokolade geben und das war doch irgendwie der \emph{Sinn.}“

Hermines Gehirn bekam jedesmal vor Verwirrung einen Schluckauf, wenn sie versuchte, ordentlich zornig zu werden. „Ist dein Leben immer so eigenartig?“ sagte sie endlich.

Harry Potters Gesicht glühte vor Stolz. „Ich \emph{mache} es so eigenartig. Du siehst hier das Ergebnis einer Menge harter Arbeit und Hirnschmalz.“

„Also…“ sagte Hermine und verklang verlegen.

„Also,“ sagte Harry Potter, „wie viel Wissenschaft kennst du genau? Ich kann Infinitesimalrechnung und ich weiß ein bisschen was über Bayes'sche Wahrscheinlichkeitstheorie und Entscheidungstheorie und eine Menge über Kognitionswissenschaft und ich habe die \emph{Feynman-Vorlesungen} gelesen (oder zumindest Band 1) und \emph{Unsichere Entscheidungen: Heuristiken und Fehlschlüsse}**** und \emph{Sprache im Denken und Handeln}***** und \emph{Einfluss: Wie und warum sich Menschen überzeugen lassen} und \emph{Rationale Entscheidungen in einer unsicheren Welt}****** und \emph{Gödel, Escher, Bach} und \emph{Ein Schritt weiter hinaus}******* und—“

Die folgenden Tests und Gegen-Tests gingen für mehrere Minuten weiter, bis sie von einem weiteren zaghaften Klopfen an der Tür unterbrochen wurden. „Komm rein,“ sagten sie und Harry Potter fast zur selben Zeit und sie glitt zur Seite und offenbarte Neville Longbottom.

Neville weinte jetzt \emph{tatsächlich}. „Ich bin zum vordersten Waggon gegangen und habe einen V-Vertrauensschüler gefunden, aber er s-sagte mir, Vertrauensschüler sollten nicht mit kleinen Sachen wie v-verlorenen Kröten belästigt werden.“

Der Gesichtsausdruck des Jungen-der-überlebt-hatte veränderte sich. Seine Lippen wurden zu einer schmalen Linie. Seine Stimme, als er sprach, war kalt und grimmig. „Was waren seine Farben? Grün und silber?“

„N-nein, sein Abzeichen war r-rot und golden.“

„\emph{Rot und Gold!}“ platzte Hermine heraus. „Aber das sind \emph{Gryffindors} Farben!“

Harry Potter \emph{zischte} daraufhin, ein erschreckendes Geräusch, das von einer echten Schlange hätte stammen können und beide, sie und Neville, zusammenzucken ließ. „Ich \emph{nehme an,}“ spuckte Harry Potter, „dass die Kröte eines Erstklässlers zu finden, nicht \emph{heldenhaft} genug ist, um eines \emph{Gryffindor}-Vertrauensschülers würdig zu sein. Komm schon, Neville, diesmal werde \emph{ich} mit dir kommen, wir werden sehen, ob der Junge-der-überlebt-hat mehr Aufmerksamkeit bekommt. Zuerst werden wir einen Vertrauensschüler suchen, der einen Zauber kennen sollte und wenn das nicht funktioniert, suchen wir einen Vertrauensschüler, der keine Angst hat, sich die Hände schmutzig zu machen und wenn \emph{das} nicht funktioniert, fange ich an meine Fans zu rekrutieren und wenn es sein muss, nehmen wir diesen Zug Schraube für Schraube auseinander.“

Der Junge-der-überlebt-hat stand auf und nahm Neville an der Hand und Hermine erkannte mit einem plötzlichen Hirn-Schluckauf, dass sie fast gleich groß waren, obwohl ein Teil von ihr darauf bestanden hatte, dass Harry Potter einen Fuß größer wäre und Neville mindestens sechs Zoll kleiner.

„\emph{Bleib!}“ fuhr Harry sie an—nein, Moment, seinen \emph{Koffer}—und er schloss die Tür fest hinter sich, als er ging.

Sie hätte wahrscheinlich mit ihnen gehen sollen, aber in nur einem kurzen Moment war Harry Potter so angsteinflößend geworden, dass sie tatsächlich eher froh war, nicht daran gedacht zu haben, es vorzuschlagen.

Hermines Verstand war jetzt so durcheinander, dass sie nicht einmal glaubte, „Ein Hogwarts von Geschichte“ vernünftig lesen zu können. Sie fühlte sich, als wäre sie gerade von einer Dampfwalze überrollt und zu einem Pfannkuchen zerquetscht worden. Sie war nicht sicher, was sie dachte oder was sie fühlte oder warum. Sie saß nur am Fenster und starrte auf die vorbeiziehende Landschaft.

Nun, sie wusste zumindest, warum sie sich innerlich ein wenig traurig fühlte.

Vielleicht war Gryffindor doch nicht ganz so wunderbar, wie sie geglaubt hatte.

* Ich gehe hier mal davon aus, dass der Autor das angesprochene Konzept bewusst nur mit Abkürzungen erwähnt, damit Leser, die bereits wissen oder vermuten, um was es geht, es verstehen und andere es nicht gegen ihren Willen herausfinden müssen. Da das Konzept, wie der Autor deutlich macht, in dieser Geschichte \uline{nicht} verwendet wird und es daher für das Verständnis nicht notwendig ist, behalte ich die Abkürzungen so bei. Wer dennoch wissen möchte, was sie bedeuten, möge selbst eine Suchmaschine seiner Wahl verwenden. Als letztes noch: Da diese Geschichte ansonsten bisher für Leser ab 12 Jahren eingestuft ist, möchte ich so junge Leser bitten, nicht danach zu suchen! Eure Eltern werden die Fragen, die ihr ihnen danach vielleicht stellt, wahrscheinlich (noch) nicht von euch hören wollen und ich will von ihnen nicht gefragt werden, warum ihr solche Fragen stellt. Wartet, bis ihr älter seid!

** engl.: \emph{Baconian project;} hiermit meint Harry wohl das Streben nach „Macht“ durch Erkenntnisgewinn über das Universum mit Hilfe der Idolenlehre (engl.: \emph{Baconian method}), eines von Francis Bacon in seinem Werk \emph{Novum organum scientiarum} beschriebenen Vorläufers der wissenschaftlichen Methode.

*** engl.: \emph{'The effecting of all things possible';} ein Zitat aus Francis Bacons Werk \emph{Nova Atlantis,} in welchem er einen utopischen Staat entwirft, in dem er unter anderem seinen Vorstellungen über modernes wissenschaftliches Arbeiten Ausdruck verleiht und über die Grenzen des Fortschritts/des Möglichen, etwa wo diese liegen und ob es sie überhaupt gibt, spekuliert. Leider konnte ich keine etablierte Übersetzung des Zitates finden und besitze das Werk auch nicht, daher hoffe ich, es so gut wie möglich sinngemäß wiedergegeben zu haben.

**** engl.: \emph{Judgment Under Uncertainty: Heuristics and Biases;} offenbar nicht ins Deutsche übersetzt.

***** engl.: \emph{Language in Thought and Action}

****** engl.: \emph{Rational Choice in an Uncertain World;} offenbar nicht ins Deutsche übersetzt.

******* engl.: \emph{A Step Farther Out;} offenbar nicht ins Deutsche übersetzt.

