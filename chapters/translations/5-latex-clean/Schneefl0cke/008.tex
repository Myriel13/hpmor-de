

\hypertarget{positive-voreingenommenheit}{% \section{9. Positive Voreingenommenheit}\label{positive-voreingenommenheit}}

Kapitel 8: Positive Voreingenommenheit

\emph{\hfill\break "Erlaube mir, dich zu warnen: Meinen Einfallsreichtum herauszufordern, ist ein gefährliches Unterfangen und kann dazu führen, dass dein Leben noch viel surrealer wird."}

Keiner hatte um Hilfe gebeten, das war das Problem.

Sie waren einfach herumgelaufen und hatten geredet, gegessen oder in die Luft gestarrt, während ihre Eltern Klatsch und Tratsch austauschten.

Aus irgendeinem seltsamen Grund hatte sich niemand hingesetzt und ein Buch gelesen, was bedeutete, dass sie sich nicht einfach neben sie setzen und ihr eigenes Buch herausnehmen konnte.

Und selbst als sie mutig die Initiative ergriffen und sich hingesetzt hatte, um ihr Buch 'Die Geschichte von Hogwarts' zu lesen, schien niemand geneigt, sich neben sie zu setzen.

Abgesehen davon, dass sie den Leuten bei ihren Hausaufgaben oder allem anderen, was sie brauchten, half, wusste sie wirklich nicht, wie man Leute kennenlernt.

Sie hatte nicht das Gefühl, dass sie eine schüchterne Person war. Sie hielt sich selbst für ein Mädchen, das die Dinge in die Hand nimmt, und doch war es ihr irgendwie zu peinlich, auf jemanden zuzugehen ohne das es einen Hinweis gab wie:

"\emph{Ich weiß nicht mehr, wie man die große Zahlen dividiert}".

"Was sollte Sie sagen? Sie hatte noch nie herausfinden können, was.

Und es schien kein Standard-Informationsblatt zu geben, was lächerlich war. Die ganze Sache mit dem Kennenlernen von Leuten war ihr noch nie sinnvoll erschienen.

Warum musste sie die ganze Verantwortung selbst übernehmen, wenn zwei Menschen beteiligt waren? Warum halfen Erwachsene nie? Sie wünschte sich, ein anderes Mädchen würde einfach auf sie zugehen und sagen:

"\emph{Hermine, der Lehrer hat mir gesagt, ich soll mich mit dir anfreunden}."

Aber um es ganz klar zu sagen: Hermine Granger, die am ersten Schultag allein in einem der wenigen leeren Abteile saß, im letzten Waggon des Zuges, mit offener Abteiltür, nur für den Fall, dass jemand aus irgendeinem Grund mit ihr reden wollte, war nicht traurig, einsam, düster, deprimiert, verzweifelt oder von ihren Problemen besessen.

Sie las vielmehr zum dritten Mal \emph{'Die Geschichte von Hogwarts'} und genoss es, mit nur einem schwachen Anflug von Ärger über die allgemeine Unvernunft der Welt im Hinterkopf.

Es gab das Geräusch einer sich öffnenden Zwischentür und dann Schritte und ein seltsames schleichendes Geräusch auf dem Gang des Zuges.

Hermine legte \emph{'Die Geschichte von Hogwarts'} beiseite und stand auf und steckte ihren Kopf nach draußen - nur für den Fall, dass jemand Hilfe brauchte - und sah einen kleinen Jungen in einem Zauberergewand, der Größe nach wahrscheinlich im ersten oder zweiten Jahr, und er sah ziemlich albern aus mit einem Schal um den Kopf gewickelt.

Ein kleiner Koffer stand neben ihm auf dem Boden. Noch während sie ihn sah, klopfte er an die Tür eines anderen, geschlossenen Abteils und sagte mit einer durch den Schal nur leicht gedämpften Stimme:

"Verzeihung, kann ich eine kurze Frage stellen?"

Sie hörte die Antwort nicht aus dem Abteil heraus, aber nachdem der Junge die Tür geöffnet hatte, glaubte sie, ihn sagen zu hören - es sei denn, sie hatte sich irgendwie verhört -

\emph{"Kennt hier jemand die sechs Quarks oder wo ich eine Erstklässlerin namens Hermine Granger finden kann?"}

Nachdem der Junge die Abteiltür geschlossen hatte, sagte Hermine:

"Kann ich dir bei etwas helfen?"

Das verkniffene Gesicht drehte sich zu ihr um, und die Stimme sagte: "Nicht, wenn du nicht die sechs Quarks benennen kannst oder mir sagst, wo ich Hermine Granger finde."

"Top, Down, Strange, Charme, Bottom, Top, und warum suchst du sie?"

Es war aus dieser Entfernung schwer zu erkennen, aber sie glaubte, den Jungen unter seinem Schal breit grinsen zu sehen.

"Ah, du bist also eine Erstklässlerin namens Hermine Granger", sagte die junge, gedämpfte Stimme. "Im Zug nach Hogwarts, und nicht weniger."

Der Junge begann, auf sie und ihr Abteil zuzugehen, und sein Koffer schlitterte hinter ihm her.

"Technisch gesehen hätte ich dich nur suchen müssen, aber es ist wahrscheinlich, dass ich mit dir reden soll oder dich einladen soll, meiner Gruppe beizutreten oder einen wichtigen magischen Gegenstand von dir zu bekommen oder herauszufinden, dass Hogwarts auf den Ruinen eines alten Tempels gebaut wurde oder so. PC oder NPC, das ist hier die Frage?"

Hermine öffnete den Mund, um darauf zu antworten, aber dann fiel ihr keine mögliche Antwort ein auf … was auch immer es war, das sie gerade gehört hatte, selbst als der Junge zu ihr hinüberging, in das Abteil schaute, zufrieden nickte und sich auf die Bank gegenüber von ihrer eigenen setzte.

Sein Koffer huschte hinter ihm her, wuchs auf das Dreifache seines früheren Durchmessers an und schmiegte sich auf eine seltsam beunruhigende Weise an ihren eigenen.

"Bitte, setz dich", sagte der Junge, "und schließ bitte die Tür hinter dir, wenn du willst.

Keine Sorge, ich beiße niemanden, der mich nicht zuerst beißt."

Er war bereits dabei, den Schal um seinen Kopf abzuwickeln. Die Unterstellung, dass dieser Junge dachte, sie hätte Angst vor ihm, ließ ihre Hand die Tür zuschieben und sie mit unnötiger Wucht gegen die Wand stoßen. Sie wirbelte herum und sah ein junges Gesicht mit hellen, lachenden grünen Augen und einer wütenden rot-dunklen Narbe auf der Stirn, die sie im Hinterkopf an etwas erinnerte, aber im Moment hatte sie Wichtigeres zu tun.

"Ich habe nicht gesagt, dass ich Hermine Granger bin!"

"Ich habe nicht gesagt, dass du Hermine Granger bist, ich habe nur gesagt, dass du Hermine Granger bist. Wenn du fragst, woher ich das weiß, dann deshalb, weil ich alles weiß. Guten Abend meine Damen und Herren, mein Name ist Harry James Potter-Evans-Verres oder kurz Harry Potter, ich weiß, dass dir das wahrscheinlich erst einmal nichts sagt -"

Hermines Verstand stellte endlich die Verbindung her. Die Narbe auf seiner Stirn, die die Form eines Blitzes hatte.

"Harry Potter! Du bist in "\emph{Moderne magische Geschichte}" und "\emph{Aufstieg und Fall der dunklen} \emph{Künste}" und "\emph{Große zauberhafte Ereignisse des zwanzigsten Jahrhunderts}".

Es war tatsächlich das erste Mal in ihrem ganzen Leben, dass sie jemandem aus einem Buch begegnete, und es war ein ziemlich seltsames Gefühl. Der Junge blinzelte dreimal.

"Ich bin in Büchern? Warte, natürlich bin ich in Büchern … was für ein seltsamer Gedanke."

"Meine Güte, wusstest du das nicht?", sagte Hermine. "Ich hätte alles herausgefunden, was ich konnte, wenn ich es gewesen wäre."

Der Junge sprach eher trocken.

"Miss Granger, es ist noch keine 72 Stunden her, dass ich in der Winkelgasse war und meinen Anspruch auf Ruhm entdeckt habe. Ich habe die letzten zwei Tage damit verbracht, wissenschaftliche Bücher zu kaufen. Glaub mir, ich habe die Absicht, alles herauszufinden, was ich kann."

Der Junge zögerte.

"Was steht in den Büchern über mich?"

Hermine Grangers Gedanken blitzten zurück, sie hatte nicht gewusst, dass sie mit diesen Büchern getestet werden würde, also hatte sie sie nur einmal gelesen, aber das war erst einen Monat her, also war der Stoff noch frisch in ihrem Kopf.

"Du bist der Einzige, der den Tötungsfluch überlebt hat, deshalb wirst du der Junge-der-überlebt hat genannt.

Du wurdest am 31. Juli 1980 als Sohn von James Potter und Lily Potter, ehemals Lily Evans, geboren.

Am 31. Oktober 1981 griff der Dunkle Lord, der nicht genannt werden darf, obwohl ich nicht weiß, warum, dein Zuhause an.

Du wurdest lebend mit der Narbe auf der Stirn in den Ruinen deines Elternhauses in der Nähe der verbrannten Überreste von Du-weißt-schon-wem gefunden.

Der größte Zauberer Albus Percival Wulfric Brian Dumbledore hat dich irgendwo hingeschickt, niemand weiß wohin.

In \emph{Aufstieg und Fall der Dunklen Künste} wird behauptet, dass du wegen der Liebe deiner Mutter überlebt hast und dass deine Narbe die gesamte magische Kraft des Dunklen Lords enthält und dass die Zentauren dich fürchten, aber die Großen Zaubererereignisse des zwanzigsten Jahrhunderts erwähnen nichts dergleichen und die \emph{Moderne Magische Geschichte} warnt davor, dass es viele verrückte Theorien über dich gibt."

Dem Jungen blieb der Mund offen stehen.

"Wurde dir gesagt, dass du im Zug nach Hogwarts auf Harry Potter warten sollst, oder so etwas in der Art?"

"Nein", sagte Hermine.

"Wer hat dir von mir erzählt?"

"Professor McGonagall und ich glaube, ich weiß auch warum. Hast du ein eidetisches Gedächtnis, Hermine?"

Hermine schüttelte den Kopf.

"Es ist nicht fotografisch, ich habe es mir immer gewünscht, aber ich musste meine Schulbücher fünfmal durchlesen, um sie mir alle einzuprägen."

"Wirklich", sagte der Junge mit leicht verstellter Stimme.

"Ich hoffe, es macht dir nichts aus, wenn ich das teste - es ist nicht so, dass ich dir nicht glaube, aber wie das Sprichwort sagt: 'Vertraue ist gut, Kontrolle ist besser'.

Es hat keinen Sinn, sich zu wundern, wenn ich einfach das Experiment machen kann."

Hermine lächelte, ziemlich selbstgefällig.

Sie liebte Tests so sehr.

"Nur zu."

Der Junge steckte eine Hand in einen Beutel an seiner Seite und sagte:

\emph{"Magische Entwürfe und Zaubertränke} von Arsenius Jigger"

.

Als er seine Hand zurückzog, hielt sie das Buch, das er genannt hatte. Sofort wünschte sich Hermine einen dieser Beutel mehr, als sie sich je etwas gewünscht hatte.

Der Junge schlug das Buch irgendwo in der Mitte auf und blickte nach unten.

"Wenn du Öl aus Schärfe machen würdest -"

"Ich kann die Seite von hier aus sehen, weißt du!"

Der Junge kippte das Buch so, dass sie es nicht mehr sehen konnte, und blätterte wieder um.

"Wenn du einen Trank aus Spinnenschärfe brauen würdest, was wäre die nächste Zutat, die du nach der Acromantulaseide hinzufügst?"

"Nachdem Sie die Seide hineingetan haben, warten Sie, bis der Trank genau den Farbton des wolkenlosen Morgenhimmels angenommen hat, 8 Grad vom Horizont entfernt und 8 Minuten bevor die Spitze der Sonne zum ersten Mal sichtbar wird.

Rühren Sie achtmal im Uhrzeigersinn und einmal dagegen, und fügen Sie dann acht Einhornhaare hinzu."

Der Junge klappte das Buch mit einem scharfen Schnappen zu und steckte es zurück in seinen Beutel, der es mit einem kleinen Rülpsgeräusch verschluckte.

"Nun gut, nun gut, nun gut. Ich würde Ihnen gerne einen Vorschlag machen, Miss Granger."

"Einen Vorschlag?" sagte Hermine misstrauisch.

Mädchen sollten sich so etwas nicht anhören. An diesem Punkt wurde Hermine auch die andere Sache - nun, eine der Sachen - klar, die an dem Jungen seltsam war.

Offenbar klangen Leute, die in Büchern standen, tatsächlich wie ein Buch, wenn sie sprachen. Das war eine ziemlich überraschende Entdeckung.

Der Junge griff in seine Tasche und sagte:

"Limonade" und holte einen hellgrünen Zylinder heraus.

Er hielt sie ihr hin und sagte:

"Kann ich dir etwas zu trinken anbieten?"

Hermine nahm das kohlensäurehaltige Getränk höflich an.

Tatsächlich fühlte sie sich inzwischen ziemlich durstig.

"Vielen Dank", sagte Hermine, als sie den Deckel aufsetzte. "War das dein Vorschlag?"

Der Junge hustete.

"Nein", sagte er.

Gerade als Hermine anfing zu trinken, sagte er:

"Ich hätte gerne, dass du mir hilfst, das Universum zu übernehmen."

Hermine trank ihren Drink aus und ließ die Dose sinken.

"Nein danke, ich bin nicht böse."

Der Junge sah sie überrascht an, als hätte er eine andere Antwort erwartet.

"Nun, das war ein bisschen rhetorisch gemeint", sagte er.

"Im Sinne des Baconschen Projekts, weißt du, nicht der politischen Macht.

'\emph{Das Bewirken aller möglichen Dinge}' und so weiter. Ich möchte experimentelle Studien über Zaubersprüche durchführen, die zugrundeliegenden Gesetze herausfinden, die Magie in den Bereich der Wissenschaft bringen, die Zauberer- und die Muggelwelt verschmelzen, den Lebensstandard des gesamten Planeten anheben, die Menschheit um Jahrhunderte voranbringen, das Geheimnis der Unsterblichkeit entdecken, das Sonnensystem kolonisieren, die Galaxie erforschen und, was am wichtigsten ist, herausfinden, was zum Teufel hier wirklich vor sich geht, denn all das ist ganz offensichtlich unmöglich."

Das klang schon etwas interessanter.

"Und?"

Der Junge starrte sie ungläubig an.

"Und? Das ist nicht genug?"

"Und was willst du von mir?", fragte Hermine.

"Ich möchte natürlich, dass du mir bei den Nachforschungen hilfst. Mit deinem enzyklopädischen Gedächtnis, gepaart mit meiner Intelligenz und Rationalität, werden wir das Baconsche Projekt in kürzester Zeit fertig haben, wobei ich mit 'kürzester Zeit' wahrscheinlich mindestens fünfunddreißig Jahre meine."

Hermine fing an, diesen Jungen nervig zu finden.

"Ich habe dich noch nie etwas Intelligentes machen sehen. Vielleicht darfst du mir bei meinen Nachforschungen helfen."

In dem Abteil herrschte eine gewisse Stille.

"Du verlangst also von mir, dass ich meine Intelligenz demonstriere", sagte der Junge nach einer langen Pause.

Hermine nickte.

"Ich warne dich, meinen Scharfsinn herauszufordern ist ein gefährliches Unterfangen und macht dein Leben noch viel surrealer."

"Ich bin noch nicht beeindruckt", sagte Hermine.

Unbemerkt stieg das grüne Getränk wieder an ihre Lippen.

"Nun, vielleicht wird dich das beeindrucken", sagte der Junge.

Er beugte sich vor und sah sie eindringlich an.

"Ich habe schon ein bisschen experimentiert und herausgefunden, dass ich den Zauberstab nicht brauche, ich kann alles, was ich will, einfach mit einem Fingerschnippen geschehen lassen."

Es kam gerade, als Hermine mitten im Schlucken war, und sie würgte und hustete und stieß die hellgrüne Flüssigkeit aus.

Auf ihre nagelneuen, nie getragenen Hexenumhang, am allerersten Schultag. Hermine schrie!

Es war ein hoher Ton, der in dem geschlossenen Abteil wie eine Fliegeralarm-Sirene klang.

"Igitt! Meine Klamotten!!!"

"Keine Panik!", sagte der Junge. "Ich kann das für dich reparieren. Schau einfach zu!"

Er hob eine Hand und schnippte mit den Fingern.

"Du wirst -"

Dann schaute sie an sich herunter. Die grüne Flüssigkeit war immer noch da, aber noch während sie zusah, begann sie zu verschwinden und zu verblassen, und innerhalb weniger Augenblicke war es so, als hätte sie nie etwas über sich verschüttet.

Hermine starrte den Jungen an, der ein ziemlich selbstgefälliges Lächeln aufsetzte. Wortlose, zauberstablose Magie!

In seinem Alter? Wenn er die Schulbücher erst vor drei Tagen bekommen hatte? Dann erinnerte sie sich daran, was sie gelesen hatte, und sie keuchte und wich vor ihm zurück.

Die ganze magische Kraft des Dunklen Lords! In seiner Narbe! Sie erhob sich hastig auf ihre Füße.

"Ich, ich, ich muss auf die Toilette, warte hier, ja?", sie musste einen Erwachsenen finden, dem sie es sagen musste - Das Lächeln des Jungen verblasste.

"Es war nur ein Trick, Hermine. Es tut mir leid, ich wollte dich nicht erschrecken."

Ihre Hand blieb an der Türklinke stehen.

"Ein Trick?"

"Ja", sagte der Junge. "Du hast mich gebeten, meine Intelligenz zu demonstrieren.

Also habe ich etwas scheinbar Unmögliches getan, was immer eine gute Möglichkeit ist, damit anzugeben.

Ich kann nicht wirklich etwas tun, indem ich nur mit den Fingern schnippe."

Der Junge hielt inne.

"Zumindest glaube ich nicht, dass ich es kann, ich habe es nie experimentell getestet."

Der Junge hob seine Hand und schnippte erneut mit den Fingern.

"Nö, keine Banane."

Hermine war so verwirrt, wie sie es noch nie in ihrem Leben gewesen war. Der Junge lächelte nun wieder über ihren Gesichtsausdruck.

"Ich habe dich gewarnt, dass es dein Leben unwirklich macht, wenn du meinen Einfallsreichtum herausforderst. Merke dir das, wenn ich dich das nächste Mal vor etwas warne."

"Aber, aber", stammelte Hermine. "Was hast du dann getan?"

Der Blick des Jungen nahm eine messende, abwägende Qualität an, die sie noch nie von jemandem in ihrem Alter gesehen hatte.

"Du denkst, du hast das Zeug zu einer eigenständigen Wissenschaftlerin, mit oder ohne meine Hilfe? Dann lass uns mal sehen, wie du ein verwirrendes Phänomen untersuchst."

"Ich.."

Hermines Gedanken waren für einen Moment leer. Sie liebte Tests, aber so einen Test hatte sie noch nie erlebt.

Verzweifelt versuchte sie, nach irgendetwas zu suchen, das sie darüber gelesen hatte, was Wissenschaftler tun sollten.

Ihr Verstand übersprang Gänge, schleifte gegen sich selbst und spuckte die Anweisungen für ein wissenschaftliches Untersuchungsprojekt aus:

Schritt 1: Bilden Sie eine Hypothese.

Schritt 2: Führe ein Experiment durch, um deine Hypothese zu testen.

Schritt 3: Messen Sie die Ergebnisse.

Schritt 4: Erstellen Sie ein Papp-Poster.

Schritt 1 war, eine Hypothese zu bilden. Das bedeutete, dass man sich etwas ausdenken sollte, was gerade passiert sein könnte.

"Also gut. Meine Hypothese ist, dass du einen Zauber auf meine Robe gewirkt hast, der alles, was darauf verschüttet wurde, verschwinden lässt."

"In Ordnung", sagte der Junge, "ist das deine Antwort?"

Der Schock ließ langsam nach, und Hermines Verstand begann, richtig zu arbeiten.

"Warte, das kann nicht richtig sein. Ich habe nicht gesehen, dass du deinen Zauberstab berührt hast oder irgendwelche Zaubersprüche gesagt hast, also wie könntest du einen Zauberspruch gewirkt haben?"

Der Junge wartete, sein Gesicht war neutral.

"Aber nehmen wir mal an, dass alle Roben aus dem Laden bereits mit einem Zauber versehen sind, der sie sauber hält, was eine nützliche Art von Zauber wäre. Das hast du herausgefunden, als du vorhin etwas auf dich verschüttet hast."

Jetzt hoben sich die Augenbrauen des Jungen.

"Ist das deine Antwort?"

"Nein, ich habe Schritt 2 nicht gemacht: 'Mach ein Experiment, um deine Hypothese zu testen.'"

Der Junge schloss seinen Mund wieder und begann zu lächeln. Hermine schaute auf die Getränkedose, die sie automatisch in den Becherhalter am Fenster gestellt hatte.

Sie nahm sie hoch und spähte hinein und stellte fest, dass sie etwa zu einem Drittel gefüllt war.

"Nun", sagte Hermine, "das Experiment, das ich machen will, ist, es auf meine Roben zu schütten und zu sehen, was passiert, und meine Vorhersage ist, dass der Fleck verschwinden wird.

Nur wenn es nicht funktioniert, werden meine Roben befleckt sein, und das will ich nicht."

"Mach es mit meinen", sagte der Junge, "dann musst du dir keine Sorgen machen, dass deine Roben fleckig werden."

"Aber -"

sagte Hermine. Irgendetwas stimmte mit diesem Gedanken nicht, aber sie wusste nicht, wie sie es genau sagen sollte.

"Ich habe Ersatzroben in meinem Koffer", sagte der Junge.

"Aber du kannst dich nirgends umziehen", wandte Hermine ein.

Dann besann sie sich eines Besseren.

"Obwohl ich wohl gehen und die Tür schließen könnte -"

"Ich habe auch einen Platz zum Umziehen in meinem Koffer."

Hermine schaute auf seinen Koffer, der, wie sie zu ahnen begann, etwas spezieller war als ihr eigener.

"Na gut", sagte Hermine, "wenn du meinst", und sie schüttete etwas grüne Limonade auf eine Ecke des Umhangs des Jungen.

Dann starrte sie darauf und versuchte sich zu erinnern, wie lange die ursprüngliche Flüssigkeit gebraucht hatte, um zu verschwinden.

.. Und der grüne Fleck verschwand! Hermine stieß einen Seufzer der Erleichterung aus, nicht zuletzt, weil dies bedeutete, dass sie es nicht mit der gesamten magischen Kraft des Dunklen Lords zu tun hatte.

Nun, Schritt 3 war das Messen der Ergebnisse, aber in diesem Fall ging es nur darum zu sehen, dass der Fleck verschwunden war.

Und sie nahm an, dass sie Schritt 4, wegen des Pappposters, wahrscheinlich überspringen konnte.

"Meine Antwort ist, dass die Roben verzaubert sind, um sich selbst sauber zu halten."

"Nicht ganz", sagte der Junge.

Hermine spürte einen Stich der Enttäuschung. Sie wünschte wirklich, sie hätte sich nicht so gefühlt, der Junge war kein Lehrer, aber es war immer noch ein Test und sie hatte eine Frage falsch beantwortet und das fühlte sich immer wie ein kleiner Schlag in den Magen an.

(Es sagte fast alles aus, was man über Hermine Granger wissen musste, dass sie sich davon nie hatte abhalten lassen, oder es sogar mit ihrer Liebe zu Prüfungen in Konflikt gebracht hatte.)

"Das Traurige ist", sagte der Junge, "dass du wahrscheinlich alles getan hast, was das Buch dir gesagt hat. Du hast eine Vorhersage gemacht, die zwischen einem verzauberten und einem nicht verzauberten Gewand unterscheiden würde, und du hast sie getestet und die Nullhypothese, dass das Gewand nicht verzaubert war, verworfen.

Aber wenn du nicht die allerbesten Bücher lesen, werden sie dir nicht beibringen, wie man Wissenschaft richtig betreibt.

Gut genug, um wirklich die richtige Antwort zu bekommen, meine ich, und nicht nur eine weitere Publikation herauszuhauen, worüber sich Papa immer beschwert.

Also lass mich versuchen zu erklären - ohne die Antwort zu verraten - was du diesmal falsch gemacht hast, und ich gebe dir noch eine Chance."

Sie begann, sich über den ach so überlegenen Ton des Jungen zu ärgern, obwohl er nur ein weiterer Elfjähriger wie sie war, aber das war zweitrangig, wenn es darum ging herauszufinden, was sie falsch gemacht hatte.

"In Ordnung."

Der Ausdruck des Jungen wurde noch intensiver.

"Dies ist ein Spiel, das auf einem berühmten Experiment namens \emph{2-4-6-Aufgabe} basiert, und es funktioniert folgendermaßen.

Ich habe eine Regel - die mir bekannt ist, aber dir nicht -, die auf einige Dreiergruppen von Zahlen passt, aber nicht auf andere.

\emph{2-4-6} ist ein Beispiel für eine Dreiergruppe, die zu dieser Regel passt. Tatsächlich… lass mich die Regel aufschreiben, nur damit du weißt, dass es eine feste Regel ist, und sie zusammenfalten, um sie dir zu geben.

Bitte schau nicht hin, denn ich schließe von vorhin, dass du verkehrt herum lesen kannst."

Der Junge sagte "Papier" und "Druckbleistift" zu seiner Tasche, und sie schloss die Augen fest, während er schrieb.

"Da", sagte der Junge und hielt ein fest gefaltetes Stück Papier in der Hand.

"Steck das in deine Tasche", und das tat sie.

"Nun, die Art und Weise, wie dieses Spiel funktioniert", sagte der Junge,

"ist, dass du mir einen Drilling aus drei Zahlen gibst, und ich werde dir sagen 'Ja', wenn die drei Zahlen ein Fall der Regel sind, und 'Nein', wenn sie es nicht sind.

Ich bin die Natur, die Regel ist eines meiner Gesetze, und du untersuchst mich. Du weißt bereits, dass \emph{2-4-6} ein 'Ja' erhält.

Wenn du alle weiteren experimentellen Tests durchgeführt hast, die du willst - mich so viele Drillinge gefragt hast, wie du für nötig hältst -, hörst du auf und errätst die Regel, und dann kannst du das Blatt Papier aufklappen und sehen, wie du abgeschnitten hast.

Verstehst du das Spiel?"

"Natürlich tue ich das", sagte Hermine. "Los."

"\emph{4-6-8}", sagte Hermine.

"Ja", sagte der Junge.

"\emph{10-12-14}", sagte Hermine.

"Ja", sagte der Junge.

Hermine versuchte, ihre Gedanken ein wenig weiter zu spinnen, denn es schien, als hätte sie bereits alle Tests gemacht, die sie brauchte, und doch konnte es nicht so einfach sein, oder?

\emph{"1-3-5.}"

"Ja."

"\emph{Minus 3, minus 1, plus 1}."

"Ja."

Hermine fiel nichts mehr ein.

"Die Regel lautet, dass die Zahlen jedes Mal um zwei zunehmen müssen."

"Und wenn ich dir jetzt sage", sagte der Junge,

"dass dieser Test schwieriger ist, als er aussieht, und dass nur 20 \% der Erwachsenen ihn richtig machen."

Hermine runzelte die Stirn.

Was hatte sie übersehen? Dann dachte sie plötzlich an einen Test, den sie noch machen musste.

"\emph{2-5-8!"}, sagte sie triumphierend.

"Ja."

"\emph{10-20-30!}"

"Ja."

"Die richtige Antwort ist, dass die Zahlen jedes Mal um den gleichen Betrag steigen müssen. Es muss nicht 2 sein."

"Sehr gut", sagte der Junge, "nimm das Papier heraus und schau, wie du es gemacht hast."

Hermine nahm das Papier aus ihrer Tasche und faltete es auf.

\emph{Drei echte Zahlen in aufsteigender Reihenfolge, von der niedrigsten zur höchsten.}

Hermine fiel die Kinnlade herunter. Sie hatte das deutliche Gefühl, dass ihr etwas furchtbar Ungerechtes angetan worden war, dass der Junge ein dreckiger, mieser, betrügerischer Lügner war, aber als sie ihre Gedanken zurückwarf, konnte sie sich keine falschen Antworten ausdenken, die er gegeben hatte.

"Was du gerade entdeckt hast, nennt man 'positive Voreingenommenheit'", sagte der Junge.

"Du hattest eine Regel im Kopf und hast immer wieder an Triolen gedacht, die die Regel 'Ja' sagen lassen sollten.

Aber du hast nicht versucht, irgendwelche Triolen zu testen, die die Regel '\emph{Nein}' sagen lassen sollten.

Tatsächlich hast du kein einziges '\emph{Nein}' erhalten, also hätte '\emph{beliebige drei Zahlen}' genauso gut die Regel sein können.

Es ist in etwa so, wie wenn Leute sich Experimente vorstellen, die ihre Hypothesen bestätigen könnten, anstatt zu versuchen, sich Experimente vorzustellen, die sie falsifizieren könnten - das ist nicht ganz genau derselbe Fehler, aber es ist nahe dran.

Man muss lernen, auf die negative Seite der Dinge zu schauen, in die Dunkelheit zu starren. Wenn dieses Experiment durchgeführt wird, bekommen nur 20\% der Erwachsenen die richtige Antwort.

Und viele der anderen erfinden fantastisch komplizierte Hypothesen und setzen großes Vertrauen in ihre falschen Antworten, weil sie so viele Experimente gemacht haben und alles so herauskam, wie sie es erwartet haben."

"Nun", sagte der Junge, "willst du es noch einmal mit dem ursprünglichen Problem versuchen?"

Seine Augen waren jetzt ganz konzentriert, als ob dies der eigentliche Test wäre.

Hermine schloss die Augen und versuchte, sich zu konzentrieren. Sie schwitzte unter ihrem Umhang. Sie hatte das seltsame Gefühl, dass dies die schwierigste Aufgabe war, die sie je in einem Test lösen musste, oder vielleicht sogar das erste Mal, dass sie überhaupt in einem Test denken musste.

Welches andere Experiment konnte sie machen? Sie hatte einen Schokoladenfrosch, könnte sie versuchen, etwas davon auf die Roben zu reiben und sehen, ob er verschwindet? Aber das schien immer noch nicht die Art von verdrehtem negativem Denken zu sein, nach dem der Junge fragte.

Als würde sie immer noch nach einem

"Ja" fragen, wenn der Schokofrosch-Fleck verschwinden würde, anstatt nach einem "Nein"

zu fragen.

Also … bei ihrer Hypothese … wann sollte der Fleck … nicht verschwinden?

"Ich muss ein Experiment machen", sagte Hermine.

"Ich will etwas Limonade auf den Boden schütten und sehen, ob sie nicht verschwindet. Hast du ein paar Papiertücher in deiner Tasche, damit ich das Verschüttete aufwischen kann, falls es nicht funktioniert?"

"Ich habe Servietten", sagte der Junge.

Sein Gesicht sah immer noch neutral aus. Hermine nahm die Dose und schüttete ein kleines bisschen Limo auf den Boden.

Ein paar Sekunden später war es verschwunden. Dann traf sie die Erkenntnis und sie fühlte sich wie ein Tritt in den Hintern.

"Natürlich! Du hast mir die Dose gegeben! Es ist nicht das Gewand, das verzaubert ist, es war die ganze Zeit die Limonade!"

Der Junge stand auf und verbeugte sich feierlich vor ihr. Er grinste jetzt breit.

"Dann … darf ich dir bei deinen Nachforschungen helfen, Hermine Granger?"

"Ich, ah…"

Hermine spürte immer noch den Rausch der Euphorie, aber sie war sich nicht ganz sicher, wie sie darauf antworten sollte.

Sie wurden von einem schwachen, zaghaften, schwachen, eher widerwilligen Klopfen an der Tür unterbrochen.

Der Junge drehte sich um, schaute aus dem Fenster und sagte:

"Ich habe meinen Schal nicht an, kannst du ihn holen?"

In diesem Moment wurde Hermine klar, warum der Junge - nein, der Junge-der-lebte, Harry Potter - den Schal überhaupt über dem Kopf getragen hatte, und sie fühlte sich ein bisschen dumm, weil sie es nicht früher bemerkt hatte.

Es war tatsächlich irgendwie seltsam, denn sie hätte gedacht, dass Harry Potter sich der Welt stolz zeigen würde; und ihr kam der Gedanke, dass er vielleicht tatsächlich schüchterner war, als er schien.

Als Hermine die Tür aufzog, wurde sie von einem zitternden Jungen begrüßt, der genauso aussah, wie er geklopft hatte.

"Entschuldigen Sie", sagte der Junge mit winziger Stimme,

"ich bin Neville Longbottom. Ich suche meine Lieblingskröte, ich kann sie im Zug nirgends finden … haben Sie meine Kröte gesehen?"

"Nein", sagte Hermine, und dann kam ihre Hilfsbereitschaft voll zum Tragen.

"Hast du auch in allen anderen Abteilen nachgesehen?"

"Ja", flüsterte der Junge.

"Dann müssen wir nur noch die anderen Wagen überprüfen", sagte Hermine zügig.

"Ich werde dir helfen. Mein Name ist übrigens Hermine Granger."

Der Junge sah aus, als könnte er vor Dankbarkeit in Ohnmacht fallen.

"Warte mal", kam die Stimme des anderen Jungen - Harry Potter.

"Ich bin mir nicht sicher, ob das die beste Art ist, es zu tun."

Bei diesem Satz sah Neville aus, als würde er weinen, und Hermine drehte sich verärgert um.

Wenn Harry Potter zu der Sorte Mensch gehörte, die einen kleinen Jungen im Stich lassen würde, nur weil er nicht unterbrochen werden wollte…

"Was? Warum nicht?!"

"Nun", sagte Harry Potter, "es wird eine Weile dauern, den ganzen Zug von Hand zu durchsuchen, und wir könnten die Kröte sowieso übersehen, und wenn wir sie nicht gefunden hätten, bis wir in Hogwarts sind, wäre er in Schwierigkeiten.

Was also viel mehr Sinn machen würde, wäre, wenn er direkt zum vorderen Waggon geht, wo die Vertrauensschüler sind, und einen Vertrauensschüler um Hilfe bittet.

Das war das Erste, was ich getan habe, als ich dich gesucht habe, Hermine, obwohl sie es nicht wussten.

Aber vielleicht haben sie Zaubersprüche oder magische Gegenstände, die es viel einfacher machen würden, eine Kröte zu finden. Wir sind doch erst im ersten Jahr."

Das… machte sehr viel Sinn.

"Meinst du, du schaffst es allein zum Wagen der Vertrauensschüler?", fragte Harry Potter.

"Ich habe so meine Gründe, warum ich mich nicht allzu oft blicken lassen will."

Plötzlich keuchte Neville und wich einen Schritt zurück.

"Ich erinnere mich an diese Stimme! Du bist einer der Lords von Chaos! Du bist derjenige, der mir Schokolade geschenkt hat!"

Was? Was, was, was?

Harry Potter drehte seinen Kopf vom Fenster weg und erhob sich dramatisch.

"Ich? Nie!", sagte er mit einer Stimme voller Empörung.

"Sehe ich aus wie ein Bösewicht, der einem Kind Süßigkeiten schenken würde?"

Nevilles Augen weiteten sich.

"Du bist Harry Potter? Der Harry Potter? Du?"

"Nein, nur ein Harry Potter, es gibt drei von mir in diesem Zug -"

Neville stieß einen kleinen Schrei aus und rannte davon.

Es gab ein kurzes Getrappel hektischer Schritte und dann das Geräusch einer sich öffnenden und schließenden Waggontür.

Hermine setzte sich hart auf ihre Bank. Harry Potter schloss die Tür und setzte sich dann neben sie.

"Kannst du mir bitte erklären, was hier los ist?" sagte Hermine mit schwacher Stimme. Sie fragte sich, ob das Zusammensein mit Harry Potter bedeutete, immer so verwirrend war.

"Oh, nun, was passiert ist, ist, dass Fred und George und ich diesen armen kleinen Jungen am Bahnhof gesehen haben - die Frau neben ihm war für eine Weile weggegangen, und er sah wirklich verängstigt aus, als wäre er sicher, dass er von Todessern oder so angegriffen werden würde.

Nun, es gibt ein Sprichwort, das besagt, dass die Angst oft schlimmer ist als die Sache selbst, also kam mir der Gedanke, dass dies ein Junge ist, der tatsächlich davon profitieren könnte, zu sehen, dass sein schlimmster Albtraum wahr wird und dass es nicht so schlimm ist, wie er befürchtet hat -"

Hermine saß mit offenem Mund da.

"- und Fred und George haben sich diesen Zauber ausgedacht, der die Tücher über unseren Gesichtern verdunkelt und verschwimmen lässt, als wären wir untote Könige und das wären unsere Leichentücher -"

Es gefiel ihr ganz und gar nicht, worauf das hinauslief.

"- und nachdem wir fertig waren, ihm all die Süßigkeiten zu geben, die ich gekauft hatte, sagten wir: 'Lasst uns ihm etwas Geld geben!

Ha ha ha! Nimm ein paar Knuts, Junge! Nimm einen Sickel!' und tanzten um ihn herum und lachten böse und so weiter.

Ich glaube, es gab einige Leute in der Menge, die sich zuerst einmischen wollten, aber die Apathie der Zuschauer \emph{('Bystander Syndrom', anm. des Übersetzers)} hielt sie davon ab, zumindest bis sie sahen, was wir taten, und dann waren sie alle zu verwirrt, um etwas zu tun.

Schließlich sagte er in diesem winzig kleinen Flüsterton '\emph{geht weg}', also schrien wir drei alle und rannten weg, wobei wir etwas darüber schrien, dass das Licht uns verbrennen würde.

Hoffentlich hat er in Zukunft nicht mehr so viel Angst, schikaniert zu werden. Das nennt man übrigens Desensibilisierungstherapie."

Okay, sie hatte nicht richtig vermutet, worauf das hinauslaufen würde.

Das brennende Feuer der Empörung, das einer von Hermines Hauptantriebsmotoren war, sprudelte zum Leben, auch wenn ein Teil von ihr irgendwie sah, was sie versucht hatten.

"Das ist furchtbar! Du bist furchtbar! Der arme Junge! Was du getan hast, war gemein!"

"Ich glaube, das Wort, das du suchst, ist 'genial', und auf jeden Fall stellst du die falsche Frage.

Die Frage ist: Hat es mehr Gutes getan als Schaden angerichtet, oder mehr Schaden als Gutes? Wenn du irgendwelche Argumente zu dieser Frage beizutragen haben, bin ich froh, sie zu hören, aber ich werde keine anderen Kritiken zur Kenntnis nehmen, bis diese Frage geklärt ist.

Ich stimme sicherlich zu, dass das, was ich getan habe, schrecklich und schikanös und gemein aussieht, da es sich um einen verängstigten kleinen Jungen handelt und so weiter, aber das ist wohl kaum der entscheidende Punkt, oder? Das nennt man übrigens Konsequentialismus, es bedeutet, dass die Frage, ob eine Handlung richtig oder falsch ist, nicht davon abhängt, ob sie schlecht oder gemein aussieht oder so etwas, die einzige Frage ist, wie sie am Ende ausfällt - was die Konsequenzen sind."

Hermine öffnete den Mund, um etwas ganz Verheerendes zu sagen, aber leider schien sie den Teil vernachlässigt zu haben, in dem sie an etwas dachte, das sie sagen wollte, bevor sie den Mund öffnete. Alles, was ihr einfiel, war:

"Was ist, wenn er Albträume hat?"

"Ehrlich gesagt glaube ich nicht, dass er unsere Hilfe brauchte, um Albträume zu haben, und wenn er stattdessen Albträume davon hat, dann werden es Albträume sein, in denen es um schreckliche Monster geht, die einem Schokolade geben, und das war ja irgendwie der Sinn der Sache."

Hermines Gehirn hickste immer wieder verwirrt, wenn sie versuchte, sich richtig zu ärgern.

"Ist dein Leben immer so sonderbar?", fragte sie schließlich.

Harry Potters Gesicht glänzte vor Stolz.

"Ich mache es so sonderbar. Du siehst hier das Produkt einer Menge harter Arbeit und Ellbogenschmalz."

"Also …" sagte Hermine und brach unbeholfen ab.

"Also", sagte Harry Potter,

"wie viel Wissenschaft kennst du genau? Ich kann Mathematik und ich kenne ein bisschen Bayessche Wahrscheinlichkeitstheorie und Entscheidungstheorie und eine Menge Kognitionswissenschaft, und ich habe die \emph{Feynman-Vorlesungen} gelesen (oder zumindest Band 1) und J\emph{udgment Under Uncertainty: Heuristics and Biases} und \emph{Language in Thought and Action} und \emph{Influence: Science and Practice} und \emph{Rational Choice in an Uncertain World} und \emph{Godel, Escher, Bach} und \emph{A Step Farther Out} und -"

Das anschließende Quiz und Gegenquiz dauerte mehrere Minuten, bevor es durch ein weiteres zaghaftes Klopfen an der Tür unterbrochen wurde.

"Herein", sagten sie und Harry Potter fast gleichzeitig, und sie glitt zurück, um Neville Longbottom zu enthüllen. Neville weinte jetzt tatsächlich.

"Ich bin zum vorderen Wagen gegangen und habe einen Vertrauensschüler gefunden, aber er hat mir gesagt, dass Vertrauensschüler nicht wegen Kleinigkeiten wie fehlenden Kröten belästigt werden dürfen."

Das Gesicht des Jungen-der-lebte veränderte sich. Seine Lippen zogen sich zu einer dünnen Linie zusammen. Seine Stimme, wenn er sprach, war kalt und grimmig.

"Was waren seine Farben? Grün und Silber?"

"N-nein, sein Abzeichen war r-rot und gold."

"Rot und Gold!?", platzte Hermine heraus. "Aber das sind doch die Farben von Gryffindor!"

Harry Potter zischte daraufhin, ein beängstigendes Geräusch, das von einer lebenden Schlange hätte kommen können und sowohl sie als auch Neville zusammenzucken ließ.

"Ich nehme an", spuckte Harry Potter, „dass das Finden der Kröte eines Erstklässlers nicht heldenhaft genug ist, um eines Gryffindor- Vertrauensschüler würdig zu sein.

Komm schon, Neville, diesmal komme ich mit, mal sehen, ob der Junge-der-lebte mehr Aufmerksamkeit bekommt.

Zuerst suchen wir einen Vertrauensschüler, der einen Zauberspruch kennen sollte, und wenn das nicht klappt, suchen wir einen Vertrauensschüler, der keine Angst hat, sich die Hände schmutzig

zu machen, und wenn das nicht klappt, fange ich an, meine Fans zu rekrutieren und wenn es sein muss, nehmen wir den ganzen Zug Schraube für Schraube auseinander.“

Der Junge-der-lebte stand auf und nahm Nevilles Hand in seine, und Hermine erkannte mit einer plötzlichen Erkenntnis, dass sie fast gleich groß waren, obwohl ein Teil von ihr darauf bestanden hatte, dass Harry Potter einen Fuß größer war und Neville mindestens sechs Zoll kleiner.

"Bleib!"

schnauzte Harry Potter sie an - nein, warte, seinen Koffer - und er schloss die Tür fest hinter sich, als er ging.

Wahrscheinlich hätte sie mitgehen sollen, aber in nur einem kurzen Moment war Harry Potter so unheimlich geworden, dass sie eigentlich ganz froh war, dass sie nicht daran gedacht hatte, es vorzuschlagen. Hermines Gedanken waren jetzt so durcheinander, dass sie nicht einmal mehr glaubte, '\emph{Die Geschichte von Hogwarts'} lesen zu können.

Sie fühlte sich, als wäre sie gerade von einer Dampfwalze überrollt und in einen Pfannkuchen verwandelt worden.

Sie war sich nicht sicher, was sie dachte oder was sie fühlte oder warum. Sie saß einfach am Fenster und starrte auf die sich bewegende Szenerie.

Nun, sie wusste zumindest, warum sie sich innerlich ein wenig traurig fühlte.

\emph{Vielleicht war Gryffindor doch nicht so wunderbar, wie sie gedacht hatte.}

