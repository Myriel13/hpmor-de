

\hypertarget{rollen-teil-1}{% \section{90. Rollen, Teil 1}\label{rollen-teil-1}}

\textbf{\uline{Rollen, Teil 1}}

Ein einfacher Innervate-Zauber des Schulleiters hatte Fred Weasley geweckt, gefolgt von einem vorläufigen Heilzauber für einen gebrochenen Arm und angeknackste Rippen. Harrys Stimme hatte dem Schulleiter aus der Ferne von der verwandelten Säure im Kopf des Trolls erzählt (Dumbledore hatte über die Terrassenkante hinuntergeschaut und eine Geste gemacht, bevor er zurückkehrte) und dann davon, dass der Verstand der Weasley-Zwillinge manipuliert worden war, wobei er ein separates Gespräch führte, an das sich Harry erinnerte, das er aber nicht verarbeiten konnte.

Harry stand immer noch über Hermines Körper, er hatte sich nicht von der Stelle bewegt und dachte so schnell er konnte durch das Gefühl der Dissoziation und der fragmentierten Zeit, ob es irgendetwas gab, das er jetzt tun sollte, irgendeine Gelegenheit, die unwiderruflich verstrich. Irgendeine Möglichkeit, die Menge an magischer Allmacht zu reduzieren, die später benötigt werden würde. Ein temporaler Bakeffekt, um diesen Moment für spätere Zeitreisen zu markieren, falls er eines Tages einen Weg finden würde, weiter als sechs Stunden zurückzureisen. Es gab Theorien über Zeitreisen im Rahmen der Allgemeinen Relativitätstheorie (die viel weniger plausibel erschienen war, bevor Harry auf Zeitumkehrer gestoßen war), und diese Theorien besagten, dass man nicht in die Zeit vor dem Bau der Zeitmaschine zurückgehen konnte - eine relativistische Zeitmaschine behielt einen kontinuierlichen Weg durch die Zeit bei, sie teleportierte nichts. Aber Harry sah nichts Hilfreiches, was er mit den Zaubersprüchen in seinem Lexikon tun konnte, Dumbledore war nicht sehr kooperativ, und außerdem war dies einige Minuten nach der kritischen Stelle in der Zeit

„Harry“, flüsterte der Schulleiter und legte seine Hand auf Harrys Schulter. Er war von dort, wo er über den Weasley-Zwillingen stand, verschwunden und neben Harry aufgetaucht; George Weasley hatte sich von dort, wo er saß, aufgesetzt um neben seinem Bruder zu knien, und Fred lag nun mit aufgerissenen Augen und zuckte beim Atmen. „Harry, du musst von diesem Ort verschwinden.“

„Warte“, sagte Harrys Stimme. „Ich versuche zu überlegen, ob ich noch etwas tun kann.“

Die Stimme des alten Zauberers klang hilflos. „Harry - ich weiß, du glaubst nicht an Seelen - aber ob Hermine dich jetzt beobachtet oder nicht, ich glaube nicht, dass sie sich wünschen würde, dass du so bist.“

\emph{… nein, das war offensichtlich.}

Harry richtete seinen Zauberstab auf Hermines Körper—

„Harry! Was machst du—“

—und wirkte alle Karft die er noch besaß—„Frigideiro! Unterkühlung“, sagte Harry aus der Ferne, während er taumelte. Es war einer der Zaubersprüche gewesen, mit denen er und Hermine vor einer Ewigkeit experimentiert hatten, also konnte er ihn genau kontrollieren, obwohl es eine Menge Kraft gebraucht hatte, um so viel Masse zu beeinflussen. Hermines Körper sollte jetzt auf fast genau fünf Grad Celsius sein. „Es sind schon Menschen aus kaltem Wasser wiederbelebt worden, nachdem sie mehr als dreißig Minuten nicht geatmet hatten. Die Kälte schützt dich vor Hirnschäden, weißt du, sie verlangsamt alles. Es gibt ein Sprichwort von Muggelärzten, man ist erst tot, wenn man warm und tot ist - ich glaube, sie kühlen den Patienten sogar während einiger Operationen, wenn sie das Herz von jemandem für eine Weile anhalten müssen.“

Fred und George begannen zu schluchzen. Dumbledores Gesicht war bereits tränenverschmiert. „Es tut mir leid“, flüsterte er. „Harry, es tut mir so leid, aber du musst damit aufhören.“ Der Schulleiter packte Harry an den Schultern und zog an ihm.

Harry ließ sich von Hermines Körper wegdrehen, ging vorwärts, als der Schulleiter ihn von dem Blut wegdrückte. Der Abkühlungszauber würde ihm Zeit verschaffen. Mindestens Stunden, vielleicht Tage, wenn er es schaffte, den Zauber weiter auf Hermine anzuwenden, oder wenn sie ihren Körper irgendwo kalt lagerten. Jetzt hatte er Zeit zum Nachdenken.

…

Minerva hatte Albus' Gesicht gesehen und gewusst, dass etwas nicht stimmte; sie hatte Zeit gehabt, sich zu fragen, was passiert war und sogar, wer gestorben war; ihre Gedanken blitzten zu Alastor, zu Augusta, zu Arthur und Molly, allesamt die wahrscheinlichsten Ziele zu Beginn von Voldemorts zweitem Aufstieg. Sie hatte gedacht, dass sie sich gestählt hatte, sie hatte gedacht, dass sie auf das Schlimmste vorbereitet war. Dann sprach Albus, und der ganze Stahl verließ sie.

\emph{Nicht Hermine—} \textbf{\emph{nein}} \emph{—}

Albus gab ihr eine kurze Zeit zum Weinen; und dann erzählte er ihr, dass Harry Potter, der Miss~Granger hatte sterben sehen, sich vor den Abstellraum der Krankenstation gesetzt hatte, wo Miss~Grangers Überreste aufbewahrt wurden, und sich weigerte, sich von der Stelle zu bewegen, und jedem, der ihn ansprach, sagte, er solle weggehen, damit er nachdenken könne. Das Einzige, was dem Jungen eine Reaktion entlockt hatte, war, als Fawkes versucht hatte, ihm etwas vorzusingen; Harry Potter hatte den Phönix angeschrien, das nicht zu tun, seine Gefühle seien echt, er wolle nicht, dass Magie versuche, sie zu heilen, als seien sie eine Krankheit. Danach hatte Fawkes sich geweigert, wieder zu singen. Albus dachte, dass sie jetzt vielleicht die beste Chance hatte, Harry Potter zu erreichen. Also musste sie sich zusammenreißen und ihr Gesicht in Ordnung bringen; für private Trauer würde später Zeit sein, wenn ihre überlebenden Kinder sie nicht mehr brauchten.

Minerva McGonagall riss die verrenkten Teile ihrer selbst zusammen, wischte sich ein letztes Mal über die Augen und legte ihre Hand auf den Türknauf der Krankenstation, deren hinterer Abstellraum nun schon zum zweiten Mal in diesem Jahrhundert und zum fünften Mal seit Bestehen des Schlosses Hogwarts als Ruhestätte eines vielversprechenden jungen Schülers genutzt wurde. Sie öffnete die Tür.

Die Augen von Harry Potter starrten sie an. Der Junge saß auf dem Boden vor der Tür zum hinteren Lagerraum und hielt seinen Zauberstab in seinem Schoß. Ob diese Augen trauerten, ob sie leer waren, ob sie gar gebrochen waren, das konnte man nicht erkennen, wenn man das Gesicht des Jungen betrachtete. Es waren keine getrockneten Tränen auf diesen Wangen.

„Warum sind Sie hier, Professor McGonagall?“ fragte Harry Potter.

„Ich habe dem Schulleiter gesagt, dass ich gerne eine Weile allein sein möchte.“

Ihr fiel nichts ein, was sie hätte sagen können. Aber sie wusste nicht, was sie sagen sollte, es gab nichts, was sie sich vorstellen konnte zu sagen, was die Sache besser machen würde. Sie hatte nicht vorausgeplant, bevor sie den Raum betreten hatte, es ging ihr nicht gut. „Worüber denkst du nach?“ sagte Minerva.

Es war der einzige Satz, der ihr in den Sinn kam. Albus hatte ihr erzählt, dass Harry immer wieder gesagt hatte, dass er nachdachte; und sie musste Harry zum Reden bringen, irgendwie.

Harry starrte halb zu ihr und halb an ihr vorbei, eine Spannung trat in sein Gesicht, während sie den Atem anhielt. Es dauerte eine Weile, bevor Harry sprach. „Ich versuche zu überlegen, ob es irgendetwas gibt, das ich jetzt tun sollte“, sagte Harry Potter. „Es ist aber schwer. Mein Verstand stellt sich immer wieder vor, wie die Vergangenheit hätte anders verlaufen können, wenn ich schneller nachgedacht hätte, und ich kann nicht ausschließen, dass irgendwo da drin eine wichtige Erkenntnis steckt.“

„Mr~Potter—“ , sagte sie zögernd. „Harry, ich glaube nicht, dass es gesund für dich ist, so - zu denken—“

„Da bin ich anderer Meinung. Es ist nicht das Denken, das Menschen umbringt.“ Die Worte wurden in einem gleichmäßigen Monoton gesprochen, als würden sie Zeilen aus einem Buch rezitieren.

„Harry“, sagte sie, kaum dass sie es ausgesprochen hatte, „es gibt nichts, was du hättest tun können—“

Etwas flackerte in Harrys Gesichtsausdruck auf. Seine Augen schienen sich zum ersten Mal auf sie zu konzentrieren. „Nichts, was ich hätte tun können?“ Harrys Stimme erhob sich bei dem letzten Wort. „Nichts, was ich hätte TUN können? Ich habe den Überblick verloren, auf wie viele verschiedene Arten ich sie hätte retten können! Wenn ich darum gebeten hätte, dass wir alle Kommunikationsspiegel bekommen! Wenn ich darauf bestanden hätte, dass Hermine aus Hogwarts genommen und in eine Schule gesteckt wird, die nicht wahnsinnig ist! Wenn ich mich sofort rausgeschlichen hätte, anstatt zu versuchen, mit normalen Idioten zu streiten! Hätte ich mich früher an den Patronus erinnert! Hätte ich früher an mögliche Notfälle gedacht und mich auf den Patronus vorbereitet! Selbst in allerletzter Minute wäre es vielleicht noch nicht zu spät gewesen! Ich tötete den Troll und drehte mich zu ihr um und sie war immer noch LEBENDIG und ich kniete einfach neben ihr und hörte mir ihre letzten Worte an wie ein IDIOT, anstatt den Patronus noch einmal zu wirken und Dumbledore zu rufen, damit er Fawkes schickt! Oder wenn ich das ganze Problem aus einem anderen Blickwinkel angegangen wäre - wenn ich einen Schüler mit einem Zeitumkehrer gesucht hätte, um eine Nachricht in die Vergangenheit zu schicken, bevor ich herausfinde, dass ihr etwas zugestoßen ist, anstatt mit einem Ergebnis zu enden, das nicht mehr geändert werden kann - ich habe den Schulleiter gebeten, zurückzugehen und Hermine zu retten und dann alles zu fälschen, die Leiche zu fälschen, jedermanns Erinnerungen zu bearbeiten, aber Dumbledore hat gesagt, dass er so etwas schon einmal versucht hat und es nicht funktioniert hat und er stattdessen einen anderen Freund verloren hat. Oder wenn ich - wenn ich nur mitgegangen wäre - wenn, in dieser Nacht—“ Harry presste seine Hände über sein Gesicht, und als er sie wieder wegnahm, war sein Gesicht wieder ruhig und gelassen. „Jedenfalls“, sagte Harry Potter, nun wieder in einem monotonen Tonfall, „will ich diesen Fehler nicht wiederholen, also werde ich bis zum Abendessen darüber nachdenken, ob es irgendetwas gibt, was ich tun sollte. Wenn mir bis dahin nichts eingefallen ist, werde ich zum Abendessen gehen und essen. Und jetzt geh bitte weg.“

Sie war sich jetzt bewusst, dass ihr wieder Tränen über die Wangen liefen.

„Harry - Harry, du musst mir doch glauben, dass das nicht deine Schuld ist!“

„Natürlich ist es meine Schuld. Es gibt hier niemanden sonst, der für irgendetwas verantwortlich sein könnte.“

„Nein! Du-weißt-schon-wer hat Hermine getötet!“ Sie war sich kaum bewusst, was sie da sagte, dass sie den Raum nicht gegen mögliche Zuhörer abgeschirmt hatte. „Nicht du! Egal, was du noch hättest tun können, nicht du hast sie getötet, es war Voldemort! Wenn du das nicht glauben kannst, wirst du verrückt, Harry!“

„So funktioniert Verantwortung nicht, Professor.“ Harrys Stimme war geduldig, als würde er einem Kind Dinge erklären, von denen er sicher war, dass es sie nicht verstehen würde. Er sah sie nicht mehr an, sondern starrte nur noch auf die Wand rechts neben ihr. „Wenn man eine Fehleranalyse durchführt, hat es keinen Sinn, einen Fehler einem Teil des Systems zuzuordnen, den man hinterher nicht mehr ändern kann, das ist, als würde man von einer Klippe springen und die Schwerkraft dafür verantwortlich machen. Die Schwerkraft wird sich beim nächsten Mal nicht ändern. Es hat keinen Sinn zu versuchen, die Verantwortung auf Leute zu schieben, die ihre Handlungen nicht ändern werden. Wenn man es einmal aus dieser Perspektive betrachtet, erkennt man, dass Schuldzuweisungen nie etwas bringen, es sei denn, man gibt sich selbst die Schuld, denn man ist der Einzige, dessen Handlungen man ändern kann, indem man ihm die Schuld zuschiebt. Deshalb hat Dumbledore auch sein Zimmer voller kaputter Zauberstäbe. Wenigstens diesen Teil versteht er.“

Ein entfernter Teil ihres Verstandes machte sich eine Notiz, bis zu einem viel späteren Zeitpunkt zu warten und dann den Schulleiter scharf darauf anzusprechen, was er beeindruckbaren kleinen Kindern zeigte. Vielleicht würde sie ihn dieses Mal sogar anschreien. Sie hatte sowieso daran gedacht, ihn anzuschreien, wegen Miss~Granger - „Sie sind nicht verantwortlich“, sagte sie, obwohl ihre Stimme zitterte. „Es sind die Professoren - wir sind für die Sicherheit der Schüler verantwortlich, nicht du.“

Harrys Augen flackerten wieder zu ihr. „Du bist verantwortlich?“ Es lag eine Anspannung in der Stimme. „Du willst, dass ich dich zur Verantwortung ziehe, Professor McGonagall?“

Sie hob ihr Kinn und nickte. Das wäre bei weitem besser, als wenn Harry sich selbst die Schuld geben würde.

Der Junge stemmte sich von seinem Platz auf dem Boden hoch und machte einen Schritt nach vorne. „Also gut“, sagte Harry in einem monotonen Tonfall. „Ich habe versucht, das Vernünftigste zu tun, als ich sah, dass Hermine fehlte und dass keiner der Professoren etwas wusste. Ich bat einen Schüler aus dem siebten Jahr, mit mir auf einem Besen zu gehen und mich zu beschützen, während wir nach Hermine suchten. Ich habe um Hilfe gebeten. Ich habe um Hilfe \emph{gebettelt}. Und niemand hat mir geholfen. Weil du allen den absoluten Befehl gegeben hast, an einem Ort zu bleiben, sonst werden sie von der Schule verwiesen, \emph{keine Ausrede}n. Egal, was Dumbledore sonst noch falsch macht, er betrachtet seine Schüler wenigstens als Menschen, nicht als Tiere, die man in einen Pferch treiben muss, damit sie nicht rauslaufen. Du wusstest, dass du nicht gut im militärischen Denken bist, deine erste Idee war, uns durch die Gänge laufen zu lassen, du wusstest, dass einige Schüler dort besser in Strategie und Taktik sind als du, und trotzdem hast du uns ohne Ermessensspielraum in einem Raum festgenagelt. Als dann etwas passierte, was du nicht vorhergesehen hast und es absolut sinnvoll gewesen wäre, einen Siebtklässler auf einem schnellen Besen loszuschicken, um Hermine Granger zu suchen, wussten die Schüler, dass du das nicht verstehen oder verzeihen würdest. Sie hatten keine Angst vor dem Troll, sie hatten Angst vor dir. Die Disziplin, die Konformität, die Feigheit, die du ihnen eingeflößt hast, hat mich gerade lange genug aufgehalten, damit Hermine sterben konnte. Nicht, dass ich hätte versuchen sollen, normale Leute um Hilfe zu bitten, natürlich, und ich werde mich ändern und beim nächsten Mal weniger dumm sein. Aber wenn ich so dumm wäre, jemandem die Verantwortung zuzuschieben, der nicht ich ist, dann würde ich das sagen.“

Tränen liefen ihr über die Wangen.

„Das würde ich dir sagen, wenn ich denken würde, dass du für irgendetwas verantwortlich sein könntest. Aber normale Menschen wählen nicht aufgrund von Konsequenzen, sie spielen nur Rollen. Du hast ein Bild im Kopf von einem strengen Zuchtmeister von dir selbst und tun, tust dieses Bild tun würde, ob es nun Sinn macht oder nicht. Ein strenger Zuchtmeister würde die Schüler in ihre Zimmer zurückschicken, auch wenn ein Troll durch die Flure streift. Ein strenger Zuchtmeister würde den Schülern befehlen, die Halle nicht zu verlassen, sonst gibt es einen Schulverweis. Und das kleine Bild von Professor McGonagall, das du in deinem Kopf hast, kann nicht aus Erfahrungen lernen oder sich ändern, also hat diese Unterhaltung keinen Sinn. Leute wie du sind für nichts verantwortlich, Leute wie ich schon, und wenn wir versagen, gibt es niemanden, dem wir die Schuld geben können.“ Der Junge schritt vorwärts, um direkt vor ihr zu stehen. Seine Hand wanderte unter seine Robe und holte die goldene Kugel hervor, die die vom Ministerium ausgegebene Schutzhülle seines Zeitdrehers war. Er sprach mit toter, ebener Stimme ohne jede Betonung. „Das hätte Hermine retten können, wenn ich in der Lage gewesen wäre, es zu benutzen. Aber du dachtest, es wäre deine Aufgabe, mich auszuschalten und mir in die Quere zu kommen. In Hogwarts ist seit fünfzig Jahren niemand mehr gestorben, das hast du gesagt, als du es abgeschlossen hast, weißt du noch? Ich hätte noch mal fragen sollen, nachdem Bellatrix Black aus Askaban geholt wurde, oder nachdem Hermine ein Mordversuch angehängt wurde. Aber ich hab's vergessen, weil ich dumm war. Bitte schließe es jetzt auf, bevor einer meiner anderen Freunde stirbt.“

Unfähig zu sprechen, holte sie ihren Zauberstab hervor und löste damit den Zeitzauber, den sie in das Schloss der Hülle gewirkt hatte. Harry Potter klappte die goldene Schale auf, betrachtete die winzige gläserne Sanduhr in ihren Kreisen, nickte und klappte das Gehäuse zu. „Danke. Und jetzt geh weg.“ Die Stimme des Jungen brach wieder. „Ich muss nachdenken.“

Sie schloss die Tür hinter sich, ein schrecklicher und immer noch weitgehend gedämpfter Laut entkam ihrer Kehle—

Albus schimmerte neben ihr ins Dasein und nahm eine kurze grelle Farbe an, als die Desillusionierung nachließ. Sie erschrak nicht, ganz und gar nicht. „Ich habe dir doch gesagt, du sollst damit aufhören“, sagte Minerva. Ihre Stimme klang dumpf in ihren eigenen Ohren. „Das war privat.“

Albus fuchtelte mit den Fingern an der Tür hinter ihr herum. „Ich hatte Angst, dass Mr~Potter dir etwas antun könnte.“ Der Schulleiter hielt inne, dann sagte er leise: „Ich bin sehr überrascht, dass du dir das hast gefallen lassen.“

„Ich hätte nur 'Mr~Potter' sagen müssen, und er hätte aufgehört.“

Ihre Stimme war fast zu einem Flüstern gesunken.

„Nur das, und er hätte aufgehört. Und dann hätte er niemanden mehr gehabt, zu dem er diese schrecklichen Dinge hätte sagen können, niemanden mehr.“

„Ich fand die Bemerkungen von Mr~Potter völlig unfair und unverdient“, sagte Albus.

„Wenn du es gewesen wärst, Albus, hättest du nicht gedroht, jeden, der den Raum verlässt, hinauszuwerfen. Kannst du mir ernsthaft etwas anderes erzählen?“

Albus' Brauen hoben sich.

„Deine Rolle in diesem Desaster war winzig, deine Entscheidungen damals durchaus vernünftig, und es ist nur Harry Potters perfekte Rückschau, die ihn etwas anderes denken lässt. Sicherlich bist du klüger, als dir selbst die Schuld daran zu geben, Minerva.“

Sie wusste genau, dass Albus ein Bild von Hermine in seinem schrecklichen Zimmer aufstellen würde, dass es einen Ehrenplatz einnehmen würde. Albus würde sich selbst die Schuld geben, da war sie sich sicher, auch wenn er zu der Zeit gar nicht in Hogwarts gewesen war. Aber sie sollte das nicht.

\emph{Du glaubst also auch nicht, dass es die Mühe wert ist, mich dafür verantwortlich zu machen …}

Sie sackte gegen die nächstgelegene Wand und versuchte, die Tränen nicht wieder aufsteigen zu lassen; sie hatte Albus noch nie weinen sehen, außer dreimal.

„Du hast immer an deine Schüler geglaubt, wie ich es nie getan habe. Sie hätten keine Angst vor dir gehabt. Sie hätten gewusst, dass du sie verstehen würdest.“

„Minerva…“

„Ich bin nicht geeignet, deine Nachfolge als Schulleiterin anzutreten. Das wissen wir beide.“

„Du irrst dich“, sagte Albus leise. „Wenn die Zeit gekommen ist, wirst du die fünfundvierzigste Schulleiterin von Hogwarts sein, und du wirst deine Sache ausgezeichnet machen.“

Sie schüttelte den Kopf.

„Was nun, Albus? Wenn er nicht auf mich hören will, wer dann?“

…

Es war vielleicht eine halbe Stunde später. Der Junge bewachte immer noch die Tür, hinter der die Leiche seiner besten Freundin lag, und hielt Wache. Er starrte nach unten, auf seinen Zauberstab, wie er in seinen Händen lag. Manchmal verzog sich sein Gesicht in Gedanken, ein anderes Mal entspannte es sich. Obwohl sich die Tür nicht öffnete und kein Geräusch zu hören war, schaute der Junge auf. Er verzog sein Gesicht. Seine Stimme, als er sprach, war dumpf. „Ich will keine Gesellschaft.“

Die Tür öffnete sich. Der Verteidigungsprofessor von Hogwarts betrat den Raum, schloss die Tür hinter sich und nahm vorsichtig eine Position in einer Ecke zwischen zwei Wänden ein, so weit weg von dem Jungen, wie es der Raum zuließ. Ein scharfes Gefühl der Katastrophe war in der Luft zwischen den beiden aufgestiegen und hing dort unveränderlich.

„Warum bist du hier?“, fragte der Junge.

Der Mann legte den Kopf leicht schief. Bleiche Augen musterten den Jungen, als wäre er ein Exemplar von einem fernen Planeten und entsprechend gefährlich.

„Ich bin gekommen, um mich zu entschuldigen, Mr~Potter“, sagte der Mann leise.

„Für was entschuldigen?“, fragte der Junge. „Warum, was hättest du tun können, um Hermines Tod zu verhindern?“

„Ich hätte daran denken sollen, die Anwesenheit von dir, Mr~Longbottom und Miss~Granger zu überprüfen, die alle offensichtliche nächste Ziele waren“, sagte der Verteidigungsprofessor ohne Zögern. „Mr~Hagrid war geistig nicht in der Lage, das Schülerkontingent zu kommandieren. Ich hätte die Bitte der stellvertretenden Schulleiterin um Ruhe ignorieren und sie auffordern sollen, Professor Flitwick zurückzulassen, der die Schüler besser vor jeder Bedrohung hätte schützen können und der die Kommunikation über Patronus hätte aufrechterhalten können.“

„Korrekt.“ Die Stimme des Jungen war rasiermesserscharf. „Ich hatte vergessen, dass es in Hogwarts noch jemanden gibt, der für die Dinge verantwortlich sein könnte. Warum bist du also nicht darauf gekommen, Professor? Weil ich nicht glaube, dass du dumm warst.“

Es gab eine Pause, und die Finger des Jungen wurden weiß an seinem Zauberstab.

„Du hast zu der Zeit auch nicht daran gedacht, Mr~Potter.“

In der Stimme des Verteidigungsprofessors lag eine gewisse Müdigkeit.

„Ich bin schlauer als du. Ich denke schneller als du. Ich habe mehr Erfahrung als du. Aber der Abstand zwischen uns beiden ist nicht derselbe wie der zwischen uns und gewöhnlichen Menschen. Wenn du etwas übersehen kannst, dann kann ich das auch.“ Die Lippen des Mannes verzogen sich. „Siehst du, ich habe sofort gefolgert, dass der Troll nur eine Ablenkung von einer anderen Sache war, und an sich nicht von großer Bedeutung. Solange niemand die Schüler sinnlos durch die Hallen irren ließ oder die jungen Slytherins unvorsichtig in eben jene Kerker schickte, in denen der Troll gesichtet worden war.“

Der Junge schien sich nicht zu beruhigen.

„Ich nehme an, das ist plausibel.“

„Auf jeden Fall“, sagte der Mann, „wenn es jemanden gibt, der für Miss~Grangers Tod verantwortlich gemacht werden kann, dann bin ich es, nicht du. Ich bin es, nicht du, der hätte…“

„Ich habe mitbekommen, dass du mit Professor McGonagall gesprochen hast und dass sie dir ein Skript gegeben hat, dem du folgen sollst.“

Der Junge gab sich keine Mühe, die Bitterkeit aus seiner Stimme zu halten.

„Wenn du mir etwas zu sagen hast, Professor, dann sag es ohne die Masken.“

Es gab eine Pause.

„Wie du willst“, sagte der Verteidigungsprofessor emotionslos.

Die blassen Augen blieben scharf und eindringlich.

„Ich bedaure, dass das Mädchen tot ist. Sie war eine gute Schülerin in meiner Verteidigungsklasse und hätte später eine Verbündete für dich sein können. Ich würde dich gerne über Ihren Verlust hinwegtrösten, aber ich weiß nicht, wie ich das anstellen soll. Wenn ich die Verantwortlichen finde, werde ich sie natürlich töten. Du kannst gerne mitmachen, wenn es die Umstände erlauben.“

„Wie rührend“, sagte der Junge, seine Stimme kühl.

„Du behauptest also nicht, Hermine gemocht zu haben?“

„Ich vermute, ihr Charme hat bei mir nicht gewirkt. Ich gehe nicht mehr so leicht solche Bindungen ein.“

Der Junge nickte.

„Danke, dass du ehrlich bist. Ist das alles, Professor?“

Es gab eine Pause.

„Das Schloss ist jetzt vernarbt“, sagte der Mann, der in der Ecke stand.

„Was?“

„Als ein gewisses altes Gerät in meinem Besitz mich darüber informierte, dass Miss~Granger kurz vor dem Tod stand, habe ich den verfluchten Feuerzauber gewirkt, von dem ich einmal gesprochen habe. Ich brannte einige Wände und Böden durch, damit mein Besen einen direkteren Weg nehmen konnte.“

Der Mann sprach immer noch tonlos.

„Hogwarts heilt solche Wunden nicht leicht, wenn überhaupt. Ich nehme an, es wird nötig sein, die Löcher mit kleineren Beschwörungen zu flicken. Ich bedaure das jetzt, da ich ohnehin zu spät dran war.“

„Ah“, sagte der Junge.

Er schloss kurz die Augen.

„Du wolltest sie doch retten. Du wolltest es so sehr, dass du dich tatsächlich angestrengt hast.“

Ein kurzes, trockenes Lächeln des Mannes.

„Ich danke dir dafür, Professor. Aber ich würde jetzt gerne bis zum Abendessen in Ruhe gelassen werden. Gerade du wirst das verstehen. Ist das alles?“

„Nicht ganz“, sagte der Mann.

Ein Hauch von sardonischer Trockenheit kehrte nun in seine Stimme zurück.

„Aufgrund der jüngsten Erfahrungen habe ich die Befürchtung, dass du jetzt etwas äußerst Dummes vorhast.“

„Was zum Beispiel?“, sagte der Junge.

„Ich bin mir nicht ganz sicher. Vielleicht hast du beschlossen, dass ein Universum ohne Miss~Granger wertlos ist und für die Beleidigungen, die es dir zugefügt hat, zerstört werden sollte.“

Der Junge lächelte ohne jeden Humor.

„Deine eigenen Probleme zeigen sich, Professor. Ich stehe nicht wirklich auf so etwas. Hast du das irgendwann einmal getan?“

„Nicht besonders. Ich habe keine große Vorliebe für das Universum, aber ich lebe dort.“

Es gab eine Pause.

„Was hast du vor, Mr~Potter?“, fragte der Mann in der Ecke. „Du hast einen bedeutenden Entschluss gefasst, auch wenn du versuchst, ihn vor mir zu verbergen. Was hast du jetzt vor?“

Der Junge schüttelte den Kopf.

„Ich denke immer noch nach und möchte dazu gerne in Ruhe gelassen werden.“

„Ich erinnere mich an ein Angebot, das du mir einmal gemacht hast, vor einigen Monaten“, sagte der Verteidigungsprofessor. „Willst du jemanden Intelligentes zum Reden haben? Ich werde verstehen, wenn du nicht angenehm bist.“

Der Junge schüttelte wieder den Kopf.

„Nein, danke.“

„Nun, dann“, sagte der Verteidigungsprofessor. „Wie wäre es mit jemandem, der mächtig ist und nicht besonders an \emph{naive Skrupel} gebunden ist?“

Es gab ein Zögern, und dann schüttelte der Junge erneut den Kopf.

„Jemand, der sich mit vielen geheimen Überlieferungen auskennt und mit Magie, die manche als unnatürlich ansehen könnten?“

Die Augen des Jungen verengten sich leicht, so unmerklich, dass ein anderer es nicht bemerkt hätte—

„Ich verstehe“, sagte der Professor der Verteidigung. „Dann frag mich ruhig danach. Ich gebe dir mein Wort, daß ich den anderen nichts davon erzählen werde.“

Der Junge brauchte eine Weile, um zu sprechen, und als er es tat, war es mit gebrochener Stimme. „Ich will Hermine zurückholen. Denn es gibt kein Leben nach dem Tod, und ich werde nicht zulassen, dass sie - einfach nicht—“

Der Junge presste seine Hände über sein Gesicht, und als er sie zurückzog, wirkte er wieder so leidenschaftslos wie der Mann, der in der Ecke stand.

Die Augen des Verteidigungsprofessors waren abstrakt und leicht verwirrt.

„Wie?“, fragte der Mann schließlich.

„Wie auch immer ich es muss.“

Wieder gab es eine Pause.

„Ungeachtet der Risiken“, sagte der Mann in der Ecke. „Ungeachtet dessen, wie gefährlich die Magie ist, die man braucht, um es zu erreichen.“

„Ja.“

Die Augen des Verteidigungsprofessors waren nachdenklich.

„Aber welche allgemeine Vorgehensweise schwebte dir denn vor? Ich nehme an, ihren Leichnam in einen Inferius zu verwandeln, ist nicht das, was du—“

„Wäre sie in der Lage zu denken?“, fragte der Junge. „Würde ihr Körper trotzdem verwesen?“

„Nein, und ja.“

„Dann nein.“

„Was wäre mit dem Auferstehungsstein von Cadmus Peverell, wenn man ihn für dich beschaffen könnte?“

Der Junge schüttelte den Kopf.

„Ich will nicht, dass eine Illusion von Hermine aus meinen Erinnerungen geholt wird. Ich möchte, dass sie ihr Leben leben kann—“ die Stimme des Jungen brach. „Ich habe mich noch nicht für einen objektiven Angriffspunkt entschieden. Wenn ich das Problem mit brachialer Gewalt lösen muss, indem ich mir genug Macht und Wissen aneigne, um es einfach geschehen zu lassen, werde ich das tun.“

Wieder eine Pause.

„Und um das zu tun“, sagte der Mann in der Ecke, „wirst du dein Lieblingswerkzeug benutzen, die Wissenschaft.“

„Natürlich.“

Der Verteidigungsprofessor atmete aus, fast wie ein Seufzer.

„Ich nehme an, das macht Sinn.“

„Bist du bereit zu helfen oder nicht?“, fragte der Junge.

„Welche Hilfe suchst du?“

„Magie. Woher kommt sie?“

„Ich weiß es nicht“, sagte der Mann.

„Und sonst auch niemand?“

„Oh, die Situation ist noch viel schlimmer, Mr~Potter. Es gibt kaum einen Esoteriker, der die Natur der Magie nicht enträtselt hat, und jeder von ihnen glaubt etwas anderes.“

„Woher kommen die neuen Zaubersprüche? Ich lese immer wieder von jemandem, der einen Zauberspruch erfunden hat, um irgendetwas zu tun, aber es wird nicht erwähnt, wie.“

Ein Achselzucken der gewandeten Schultern.

„Woher kommen neue Bücher, Mr~Potter? Wer viele Bücher liest, wird manchmal fähig, sie auch selbst zu schreiben. Wie das geht? Keiner weiß es.“

„Es gibt Bücher darüber, wie man schreibt—“

„Wenn man sie liest, wird man kein berühmter Dramatiker. Nachdem alle solchen Ratschläge berücksichtigt sind, bleibt nur noch das Geheimnis. Die Erfindung neuer Zaubersprüche ist ein ähnliches Mysterium in reinerer Form.“

Der Mann legte den Kopf schief.

„Solche Unternehmungen sind gefährlich. Es heißt, man sollte entweder keine Kinder haben oder warten, bis sie erwachsen sind. Es gibt einen Grund, warum so viele Erfinder aus Gryffindor zu kommen scheinen, und nicht aus Ravenclaw, wie man vielleicht erwarten würde.“

„Und die mächtigeren Arten von Zaubern?“, fragte der Junge.

„Ein legendärer Zauberer könnte in seinem Leben ein Opferritual erfinden und das Wissen an seine Erben weitergeben. Der Versuch, fünf solche zu erfinden, wäre Selbstmord. Deshalb sind die Zauberer von wahrer Macht diejenigen, die sich uraltes Wissen angeeignet haben.“

Der Junge nickte distanziert.

„So viel also zur direkten Lösung. Es wäre schön gewesen, einfach einen Zauber für 'Tote erwecken', 'Gott werden' oder 'Tu was ich wünsche' zu erfinden. Weißt du etwas über Atlantis?“

„Nur das, was jeder Gelehrte weiß“, sagte der Mann trocken. „Wenn du etwas über die achtzehn Standardtheorien erfahren willst - starr mich nicht an, Mr~Potter. Wenn es so einfach wäre, hätte ich es schon viele Jahre früher getan.“

„Ich verstehe. Tut mir leid.“

Eine Zeit lang herrschte Schweigen. Der Blick des Verteidigungsprofessors ruhte auf dem Jungen, der Junge starrte scheinbar ins Leere.

„Es gibt ein paar Zaubersprüche, die ich lernen will. Zaubersprüche, die ich schon früher hätte anwenden können, wenn ich daran gedacht hätte, sie vorher zu lernen.“ Die Stimme des Jungen war kalt. „Zaubersprüche, die ich brauchen werde, wenn diese Art von Dingen weiter passiert. Die meisten kann ich wohl einfach nachschlagen. Bei einigen erwarte ich, dass ich es nicht kann.“

Der Verteidigungsprofessor legte den Kopf schief.

„Ich werde dir fast jede Magie beibringen, die du wissen willst, Mr~Potter. Ich habe zwar einige Grenzen, aber du kannst immer fragen. Aber was genau suchst du? Dir fehlt die rohe Kraft für den Tötungsfluch und die meisten anderen Zaubersprüche, die als verboten gelten—“

„Der Zauber des verfluchten Feuers. Ich nehme nicht an, dass es ein Opferritual ist, das sogar ein Kind anwenden könnte, wenn es sich trauen würde?“

Die Lippen des Verteidigungsprofessors zuckten.

„Es erfordert die dauerhafte Opferung eines Bluttropfens; dein Körper wäre um diesen Bluttropfen leichter, von diesem Tag an. So etwas möchte man nicht oft tun, Mr~Potter. Damit das verfluchte Feuer sich nicht gegen dich wendet und dich verzehrt, ist Willensstärke gefragt; die übliche Praxis ist es, den eigenen Willen zuerst in kleineren Prüfungen zu testen. Und obwohl es kein primäres Element des Rituals ist, fürchte ich, dass es mehr Magie erfordert, als du für die nächsten paar Jahre besitzen wirst.“

„Schade“, sagte der Junge. „Es wäre schön gewesen, den Gesichtsausdruck des Feindes zu sehen, wenn er das nächste Mal versucht, einen Troll zu benutzen.“

Der Verteidigungsprofessor legte den Kopf schief, seine Lippen zuckten wieder.

„Was ist mit den Gedächtniszaubern? Die Weasley-Zwillinge haben sich seltsam verhalten und der Schulleiter sagte, er glaube, dass ihr Kopf verändert worden ist. Das scheint einer der Lieblingstricks des Feindes zu sein.“

„Regel Nummer acht“, sagte der Verteidigungsprofessor. „Jede Technik, die gut genug ist, um mich einmal zu besiegen, ist gut genug, um sie selbst zu lernen.“

Der Junge lächelte humorlos.

„Und ich habe einmal von einer Erwachsenen gehört, die Obliviate gewirkt hat, während sie fast völlig ausgelaugt war, also muss es nicht allzu viel Magie erfordern, es zu wirken. Es gilt nicht einmal als unverzeihlich, obwohl ich mir nicht vorstellen kann, warum nicht. Wenn ich Mr~Hagrid dazu bringen könnte, sich an eine andere Reihenfolge zu erinnern—“

„So einfach ist das nicht“, sagte der Verteidigungsprofessor. „Du bist nicht stark genug, um den Falsche-Erinnerung-Zauber zu benutzen, und selbst eine einfache Obliviation wird deine derzeitige Ausdauer überfordern.Es ist eine gefährliche Kunst, die ohne Genehmigung des Ministeriums nicht angewendet werden darf, und ich würde dich davor warnen, sie unter Umständen anzuwenden, bei denen es unangenehm wäre, versehentlich zehn Jahre des Lebens von jemandem zu löschen. Ich wünschte, ich könnte dir versprechen, dass ich mir einen dieser streng gehüteten Wälzer aus der Mysteriumsabteilung besorge und ihn dir unter einem getarnten Umschlag zukommen lasse. Aber was ich dir tatsächlich sagen muss, ist, dass du den Standard-Einführungstext im nord-nordwestlichen Regal der Hauptbibliothek von Hogwarts findest, abgelegt unter E.“

„Im Ernst“, sagte der Junge flach.

„In der Tat.“

„Ich danke dir für diese Hinweise, Professor.“

„Deine Kreativität ist sehr viel praktischer geworden, Mr~Potter, seit ich dich kenne.“

„Danke für das Kompliment.“

Der Junge blickte nicht von seinem Platz auf, wo er wieder auf den Zauberstab zwischen seinen Händen hinunterstarrte.

„Ich würde jetzt gerne wieder nachdenken. Bitte erkläre ihnen in meinem Namen, was passiert, wenn ich gestört werde.“

Die Tür zum Lagerraum klickte auf, und Professor Quirrell trat heraus. Sein Gesicht hatte einen toten, emotionslosen Ausdruck; sie hätte sagen können, dass es sie an Severus erinnerte, obwohl Severus nie ganz so ausgesehen hatte. Schon als die Tür wieder zuknallte, hatte Minerva eine wortlose Schweigebarriere aufgerichtet. Die Worte sprudelten nur so aus ihr heraus: „Wie ist es gelaufen - du warst eine Weile da drin - redet Harry jetzt?“

Professor Quirrell schritt zügig durch den Raum zur hinteren Wand in der Nähe des Eingangs und sah sie wieder an. Die Emotionslosigkeit glitt von seinem Gesicht, als würde er eine Maske abnehmen, und zurück blieb jemand sehr Grimmiges.

„Ich habe mit Mr~Potter so gesprochen, wie er es von mir erwartet hat, und habe es vermieden, Dinge zu sagen, die ihn verärgern würden. Ich glaube nicht, dass es ihn tröstete. Ich glaube, ich habe den Dreh dafür nicht raus.“

„Danke - es ist gut, dass er überhaupt gesprochen hat—“

Sie zögerte.

„Was hat Mr~Potter gesagt?“

„Ich fürchte, ich habe ihm versprochen, nicht darüber zu sprechen. Und jetzt… Ich glaube, ich muss die Hogwarts-Bibliothek besuchen.“

„Die Bibliothek?“

„Ja“, sagte Professor Quirrell.

Eine uncharakteristische Anspannung war in seine Stimme gekommen.

„Ich beabsichtige, die Sicherheit in der Verbotenen Abteilung mit einigen Vorkehrungen zu verstärken, die ich mir selbst ausgedacht habe. Die derzeitigen sind ein Witz. Und Mr~Potter muss unter \emph{allen Umständen} von der verbotenen Abteilung ferngehalten werden.“

Sie starrte den Verteidigungsprofessor an, ihr Herz schlug ihr plötzlich bis zum Hals.

Professor Quirrell sprach weiter.

„Sie werden dem Jungen nicht sagen, dass ich Ihnen so viel gesagt habe. Sie werden Flitwick und Vector bestätigen, dass der Junge durch die üblichen Ausflüchte abgelenkt werden soll, wenn er altkluge Fragen zur Zaubererfindung stellt. Und obwohl es nicht mein Fachgebiet ist, stellvertretende Schulleiterin, wenn es irgendeinen Weg gibt, den Sie sich vorstellen können, um den Jungen davon zu überzeugen, nicht weiter in seinem Kummer und Wahnsinn zu versinken - irgendeinen Weg, um die Schlüsse, zu denen er kommt, rückgängig zu machen -, dann schlage ich vor, dass Sie ihn \emph{sofort} ergreifen.“

