

\hypertarget{querdenken}{% \section{16. Querdenken}\label{querdenken}}

Querdenken

\emph{Ich bin kein Psychopath, ich bin nur sehr kreativ.}

Schon als er am Mittwoch das Klassenzimmer für Verteidigung betrat, wusste Harry, dass dieses Fach anders sein würde.\\ Zunächst einmal war es das größte Klassenzimmer, das er bisher in Hogwarts gesehen hatte, ähnlich einem großen Universitätsklassenzimmer, mit übereinander angeordneten Reihen von Tischen vor einer gigantischen, flachen Bühne aus weißem Marmor.\\ Das Klassenzimmer befand sich hoch oben im Schloss - im fünften Stock - und Harry wusste, dass das die einzige Erklärung dafür war, wo ein Raum wie dieser Platz finden sollte.\\ Es wurde klar, dass Hogwarts einfach keine Geometrie besaß, weder euklidisch noch sonst wie; es gab Verbindungen, keine Richtungen.\\ Anders als in einer Universitätshalle gab es keine Reihen von Klappsitzen; stattdessen gab es ganz gewöhnliche Hogwarts-Holzschreibtische und Holzstühle, die in einer Kurve über jede Ebene des Klassenzimmers aufgereiht waren.\\ Nur dass auf jedem Schreibtisch ein flaches, weißes, rechteckiges, geheimnisvolles Objekt stand. In der Mitte der gigantischen Plattform, auf einem kleinen erhöhten Podest aus dunklerem Marmor, stand ein einsames Lehrerpult.\\ An dem saß Quirrell zusammengesunken in seinem Stuhl, den Kopf nach hinten geworfen, und sabberte leicht über seine Roben.

\emph{An was erinnert mich das jetzt…?}\\ Harry war so früh in der Stunde angekommen, dass noch keine anderen Schüler da waren.

(Die englische Sprache war mangelhaft, wenn es darum ging, Zeitreisen zu beschreiben; insbesondere fehlten dem Englischen alle Worte, die ausdrücken konnten, wie bequem es war.)

Quirrell schien im Moment nicht… \emph{funktionstüchtig}… zu sein, und Harry hatte sowieso keine besondere Lust, sich Quirrell zu nähern.\\ Harry suchte sich einen Schreibtisch aus, kletterte hinauf, setzte sich und holte das Lehrbuch für Verteidigung heraus.\\ Er war etwa zu sieben Achteln durch - eigentlich hatte er vor, das Buch vor dieser Stunde zu beenden, aber er war im Verzug und hatte den Zeitdreher heute schon zweimal benutzt.\\ Bald waren Geräusche zu hören, als sich das Klassenzimmer zu füllen begann. Harry ignorierte sie.

"Potter? Was machst du denn hier?"\\ Diese Stimme gehörte nicht hierher.

Harry sah auf.\\ "Draco? Was machst du denn hier, …\emph{oh mein Gott, du hast Minsions}."

Einer der Jungs, die hinter Draco standen, schien für einen Elfjährigen ziemlich viele Muskeln zu haben, und der andere stand in einer verdächtig ausbalanciert wirkenden Haltung da.\\ Der weißblondhaarige Junge lächelte ziemlich süffisant und gestikulierte hinter ihm.\\ "Potter, ich stelle dir Mr. Crabbe vor", seine Hand wanderte von den Muskeln zum Gleichgewicht,\\ "Mr. Goyle. Vincent, Gregory, das ist Harry Potter."\\ Mr. Goyle legte den Kopf schief und warf Harry einen Blick zu, der wahrscheinlich etwas bedeuten sollte, am Ende aber nur schielte.

Mr. Crabbe sagte: "Es freut mich, Sie kennenzulernen", in einem Ton, der so klang, als wollte er seine Stimme so weit wie möglich senken.\\ Ein flüchtiger Ausdruck der Bestürzung ging über Dracos Gesicht, wurde aber schnell durch sein überlegenes Grinsen ersetzt.

"Du hast Minions!" wiederholte Harry. "Woher bekomme ich auch welche?!"

Dracos Grinsen wurde noch breiter.\\ "Ich fürchte, Potter, der erste Schritt ist, in Slytherin einsortiert zu werden -"

"Was? Das ist nicht fair!"

"- und dann, dass eure Familien eine Abmachung aus der Zeit vor eurer Geburt haben."

Harry sah Mr. Crabbe und Mr. Goyle an. Sie schienen sich beide sehr zu bemühen, sich aufzurichten. Das heißt, sie lehnten sich nach vorne, kauerten sich über die Schultern, streckten ihre Hälse heraus und starrten ihn an.

"Ähm … warte mal", sagte Harry. "Das wurde vor Jahren arrangiert?"

"Ganz genau, Potter. Ich fürchte, du hast kein Glück."

Mr. Goyle holte einen Zahnstocher hervor und begann, immer noch drohend, seine Zähne zu putzen.

"Und", sagte Harry, "Lucius hat darauf bestanden, dass du nicht mit deinen Leibwächtern aufwachsen sollst und sie erst am ersten Schultag kennenlernen sollst."

Das wischte das Grinsen aus Dracos Gesicht.\\ "Ja, Potter, wir wissen alle, dass du brillant bist, die ganze Schule weiß es inzwischen, du kannst aufhören, damit anzugeben -"

"Also hat man ihnen ihr ganzes Leben lang erzählt, dass sie deine Lakaien sein werden, und sie haben Jahre damit verbracht, sich vorzustellen, wie Lakaien zu sein haben -"

Draco zuckte zusammen.

"- und was noch schlimmer ist, sie kennen sich und haben geübt -"

"Der Boss hat gesagt, du sollst die Klappe halten", grummelte Mr. Crabbe.\\ Mr. Goyle biss auf seinen Zahnstocher, den er zwischen den Zähnen hielt, und benutzte eine Hand, um die Knöchel der anderen zu knacken.

"Ich habe euch doch gesagt, ihr sollt das nicht vor Harry Potter machen!"\\ Die beiden schauten etwas verlegen und Mr. Goyle steckte den Zahnstocher schnell wieder in eine Tasche seines Umhangs. Aber in dem Moment, in dem Draco sich von ihnen abwandte, um Harry wieder anzusehen, fingen sie wieder an zu lauern.

"Ich entschuldige mich", sagte Draco steif, "für die Beleidigung, die dir diese Schwachköpfe zugefügt haben."

Harry warf einen bedeutungsvollen Blick auf Mr. Crabbe und Mr. Goyle.\\ "Ich würde sagen, du bist ein wenig zu hart zu ihnen, Draco. Ich denke, sie verhalten sich genau so, wie ich es von meinen Minions erwarten würde. Ich meine, wenn ich Minions hätte."

Draco fiel die Kinnlade runter.\\ "Hey, Gregory, du glaubst doch nicht, dass er versucht, uns vom Boss wegzulocken, oder?"\\ "Ich bin sicher, Mr. Potter wäre nicht so dumm."

"Oh, ich würde nicht im Traum daran denken", sagte Harry sanft.\\ "Es ist nur etwas, das man im Hinterkopf behalten sollte, wenn der jetzige Arbeitgeber einem nicht wohlgesonnen zu sein scheint. Außerdem schadet es nie, andere Angebote zu haben, während man über seine Arbeitsbedingungen verhandelt, oder?"

"Was macht er denn in Ravenclaw?"\\ "Das kann ich mir nicht vorstellen, Mr. Crabbe."

"Ihr beide haltet die Klappe", sagte Draco durch knirschende Zähne. "Das ist ein Befehl."\\ Mit sichtlicher Anstrengung richtete er seine Aufmerksamkeit wieder auf Harry.

"Wie auch immer, was machst du im Slytherin-Verteidigungskurs?"\\ Harry runzelte die Stirn.\\ "Warte mal."\\ Seine Hand fuhr in seine Tasche.\\ "Stundenplan."\\ Er blickte auf das Pergament.\\ "Verteidigung, 14:30 Uhr, und im Moment ist es …"\\ Harry schaute auf seine mechanische Uhr, die 11:23 Uhr anzeigte.\\ "14:23 Uhr, es sei denn, ich habe die Zeit aus den Augen verloren. Habe ich das?"

Wenn ja, dann wusste Harry, wie er zu der Stunde kam, in der er eigentlich sein sollte.\\ Gott, er liebte seinen Zeitumkehrer und eines Tages, wenn er alt genug war, würden sie heiraten.

"Nein, das klingt richtig", sagte Draco und sah verwirrt aus.\\ Sein Blick schweifte über den Rest des Auditoriums, das sich mit grüngeschmückten Roben füllte und…\\ "Gryffindorks!?", spuckte Draco. "Was machen die denn hier?"

"Hm", sagte Harry. "Professor Quirrell hat gesagt… ich habe seine genauen Worte vergessen… dass er einige der Hogwarts-Lehrkonventionen ignorieren würde. Vielleicht hat er einfach alle seine Klassen zusammengelegt."

"Hm", sagte Draco. "Du bist der erste Ravenclaw hier drin."

"Ja. Bin früher gekommen."

"Was machst du dann ganz hinten in der letzten Reihe?"

Harry blinzelte.\\ "Keine Ahnung, schien ein guter Platz zu sein?"

Draco gab einen spöttischen Laut von sich.\\ "Du könntest nicht weiter weg von der Lehrerin sitzen, wenn du es versuchen würdest."\\ Der blondhaarige Junge lehnte sich etwas näher heran.\\ "Jedenfalls, stimmt es, was du zu Derrick und seinen Leuten gesagt hast?"

"Wer ist Derrick?"

"Du hast ihn mit einem Kuchen beworfen."

"Eigentlich mit zwei Torten.\\ Was soll ich zu ihm gesagt haben?"

"Dass er nichts Gerissenes oder Ehrgeiziges getan hat und dass er eine Schande für Salazar Slytherin ist."\\ Draco starrte Harry eindringlich an.

"Das … klingt ungefähr richtig", sagte Harry. "Ich glaube, es war eher so: '\emph{Ist das eine Art unglaublich cleverer Plan, der dir einen Vorteil für die Zukunft verschafft, oder ist es wirklich so eine Schande für das Andenken von Salazar Slytherin, wie es aussieht}' oder so ähnlich.\\ Ich erinnere mich nicht mehr an die genauen Worte."

"Du bringst alle durcheinander", sagte der blondhaarige Junge.

"Hm?" sagte Harry in ehrlicher Verwirrung.

"Warrington sagte, dass eine lange Zeit unter dem Sprechenden Hut zu verbringen eines der Warnzeichen für einen großen dunklen Zauberer ist.\\ Alle haben darüber geredet und sich gefragt, ob sie anfangen sollten, sich bei dir einzuschleimen, nur für den Fall. Dann hast du einen Haufen Hufflepuffs beschützt, um Himmels willen. Dann hast du Derrick gesagt, er sei eine Schande für das Andenken von Salazar Slytherin!\\ Was soll man davon halten?"

"Dass der Sprechende Hut beschlossen hat, mich in das Haus '\emph{Slytherin}' zu stecken!\\ \emph{War nur ein Scherz! Ravenclaw!}' und ich mich entsprechend verhalten habe."

Mr. Crabbe und Mr. Goyle kicherten beide, was Mr. Goyle veranlasste, sich schnell eine Hand vor den Mund zu schlagen.\\ "Wir sollten uns lieber hinsetzen", sagte Draco.\\ Er zögerte, richtete sich ein wenig auf und sprach ein wenig förmlicher.\\ "Aber ich möchte unser letztes Gespräch fortsetzen und ich akzeptiere deine Bedingungen."

Harry nickte.\\ "Würde es dir sehr viel ausmachen, wenn ich bis Samstagnachmittag warten würde? Ich stecke gerade in einem kleinen Wettbewerb."

"Einem Wettbewerb?"

"Um zu sehen, ob ich alle meine Schulbücher schneller lesen kann als Hermine Granger."

"Granger", wiederholte Draco. Seine Augen verengten sich.\\ "Das Schlammblut, das denkt, sie sei Merlin? Wenn du versuchst, sie vorzuführen, dann wünscht dir ganz Slytherin viel Glück, Potter, und ich werde dich bis Samstag nicht mehr belästigen."

Draco neigte respektvoll den Kopf und schlenderte davon, verfolgt von seinen Lakaien.

\emph{Oh, das wird ein Riesenspaß das zu jonglieren, das merke ich jetzt schon.}

Das Klassenzimmer füllte sich nun schnell mit allen vier Farben: grün, rot, gelb und blau.\\ Draco und seine beiden Freunde schienen mitten im Versuch zu sein, drei zusammenhängende Plätze in der ersten Reihe zu ergattern - die natürlich schon besetzt waren.\\ Mr. Crabbe und Mr. Goyle drohten energisch, aber es schien keine große Wirkung zu haben. Harry beugte sich über sein Verteidigungslehrbuch und las weiter.

Um 14:35 Uhr, als die meisten Plätze besetzt waren und niemand mehr hereinzukommen schien, ruckte Professor Quirrell plötzlich in seinem Stuhl und setzte sich aufrecht hin, und sein Gesicht erschien auf allen flachen, weißen, rechteckigen Gegenständen, die auf den Tischen der Schüler standen.\\ Harry war überrascht, sowohl von dem plötzlichen Auftauchen von Professor Quirrells Gesicht als auch von der Ähnlichkeit mit dem Muggelfernsehen.\\ Es hatte etwas Nostalgisches und zugleich Trauriges an sich, es wirkte so sehr wie ein Stück Heimat und war es doch nicht wirklich…

"Guten Tag, meine jungen Lehrlinge", sagte Professor Quirrell. Seine Stimme schien aus dem Pultbildschirm zu kommen und direkt zu Harry zu sprechen.\\ "Willkommen zu Ihrer ersten Lektion in Kampfmagie, wie die Gründer von Hogwarts es genannt hätten; oder, wie es im späten zwanzigsten Jahrhundert heißt, Verteidigung gegen die dunklen Künste."

Es gab ein gewisses hektisches Gekrabbel, als die überraschten Schüler nach ihrem Pergament oder ihren Heften griffen.

"Nein", sagte Professor Quirrell. "Machen Sie sich nicht die Mühe, aufzuschreiben, wie dieses Fach einmal hieß. Keine dieser sinnlosen Fragen wird in irgendeiner meiner Lektionen in eure Noten einfließen. Das ist ein Versprechen."

Viele Schüler setzten sich daraufhin aufrecht hin und sahen ziemlich schockiert aus.\\ Professor Quirrell lächelte dünn.

"Diejenigen von euch, die ihre Zeit mit der Lektüre ihrer nutzlosen Erstjahres-Verteidigungslehrbücher verschwendet haben -"\\ Jemand gab einen erstickten Laut von sich.\\ Harry fragte sich, ob es Hermine war.\\ "- haben vielleicht den Eindruck gewonnen, dass dieses Fach zwar Verteidigung gegen die dunklen Künste heißt, es aber in Wirklichkeit darum geht, wie man sich gegen Albtraumfalter verteidigt, die leicht schlechte Träume verursachen, oder gegen Säureschnecken, die sich durch einen zwei Zentimeter dicken Holzbalken hindurch auflösen können, wenn diese einen ganzen Tag Zeit haben."

Professor Quirrell stand auf und schob seinen Stuhl vom Schreibtisch zurück.\\ Der Bildschirm auf Harrys Pult verfolgte jede seiner Bewegungen. Professor Quirrell schritt auf den vorderen Teil des Klassenzimmers zu und brüllte:

"\textbf{Der Ungarische Hornschwanz ist größer als ein Dutzend Männer!}\\ \textbf{Er spuckt Feuer so schnell und so genau, dass er einen Schnatz im Flug schmelzen kann!}

\textbf{Ein einziger Todesfluch bringt ihn zu Fall!"}

Die Schüler schnappten nach Luft.

\textbf{"Der Bergtroll ist gefährlicher als der Ungarische Hornschwanz! Er ist stark genug, um sich durch Stahl zu beißen! Sein Fell ist widerstandsfähig genug, um betäubenden Verhexungen und schneidenden Zaubern zu widerstehen! Sein Geruchssinn ist so ausgeprägt, dass er aus der Ferne erkennen kann, ob seine Beute Teil eines Rudels ist oder allein und verletzlich!}

\textbf{Am furchteinflößendsten von allem ist, dass der Troll unter den magischen Kreaturen einzigartig ist, weil er ständig eine Form der Verwandlung an sich selbst aufrechterhält - er verwandelt sich immer in seinen eigenen Körper.}\\ \textbf{\hfill\break Wenn es Ihnen irgendwie gelingt, ihm einen Arm abzureißen, wächst ihm innerhalb von Sekunden ein neuer! Feuer und Säure erzeugen Narbengewebe, das die Regenerationskräfte eines Trolls vorübergehend verwirren kann - für ein oder zwei Stunden!\\ Sie sind schlau genug, um Keulen als Werkzeuge zu benutzen!}\\ \textbf{Der Bergtroll ist die drittperfekteste Tötungsmaschine der gesamten Natur!}\\

\textbf{Ein einziger Todesfluch bringt ihn zu Fall!}"

Die Schüler schauten ziemlich geschockt. Professor Quirrell lächelte eher grimmig.

"Eure traurige Entschuldigung für ein Verteidigungslehrbuch aus dem dritten Jahr schlägt euch vor, den Bergtroll dem Sonnenlicht auszusetzen, was ihn einfrieren wird.\\ Das, meine jungen Lehrlinge, ist die Art von nutzlosem Wissen, die ihr in meinem Unterricht nie finden werdet.\\ Man begegnet Bergtrollen nicht bei Tageslicht!\\ Die Idee, dass man Sonnenlicht benutzen sollte, um sie aufzuhalten, ist das Ergebnis von dummen Lehrbuchautoren, die versuchen, ihre Beherrschung von kleingeistigem Wissen auf Kosten der Praktikabilität zu zeigen.\\ Nur weil es eine lächerlich obskure Methode gibt, mit Bergtrollen umzugehen, heißt das nicht, dass man sie tatsächlich anwenden sollte!\\

\textbf{Der Tötungsfluch ist unblockbar, unaufhaltsam und funktioniert jedes einzelne Mal bei allem, was ein Gehirn hat.}

Wenn Sie als erwachsener Zauberer nicht in der Lage sind, den Tötungsfluch zu benutzen, dann können Sie einfach weg apparieren!\\ Das Gleiche gilt, wenn Sie der zweitperfektesten Tötungsmaschine gegenüberstehen, einem Dementor.

\textbf{Dann appariert man einfach weg!}"

"Es sei denn natürlich", sagte Professor Quirrell, seine Stimme nun tiefer und härter,\\ "man steht unter dem Einfluss eines Anti-Apparationsfluchs.

Nein, es gibt genau ein Monster, das euch bedrohen kann, sobald ihr erwachsen seid. Das gefährlichste Monster auf der ganzen Welt, so gefährlich, dass nichts anderes in die Nähe kommt.\\ \textbf{\hfill\break Der dunkle Zauberer. Das ist das Einzige, das euch dann noch bedrohen kann.}"

Professor Quirrells Lippen waren zu einer dünnen Linie verzogen.

"Ich werde euch widerwillig so viel Wissen beibringen, dass ihr die vom Ministerium vorgeschriebenen Teile eurer Abschlussprüfung bestehen könnt.\\ Da eure genaue Note in diesen Abschnitten keinen Unterschied für euer zukünftiges Leben machen wird, kann jeder, der mehr als eine gute Note will, gerne seine eigene Zeit mit dem Studium unseres erbärmlichen Lehrbuchs verschwenden.

Der Titel dieses Fachs lautet nicht \emph{Verteidigung gegen kleine Schädlinge.} Ihr seid hier, um zu lernen, wie ihr euch gegen die Dunklen Künste verteidigt.

Das heißt, um es ganz klar zu sagen, sich gegen dunkle Zauberer zu verteidigen. Leute mit Zauberstäben, die euch verletzen wollen und es wahrscheinlich schaffen werden, wenn ihr sie nicht zuerst verletzt!\\

Es gibt keine Verteidigung ohne Angriff! Es gibt keine Verteidigung ohne Kampf!

Diese Realität wird von den fetten, überbezahlten, von Auroren bewachten Politikern, die euren Lehrplan verordnet haben, als zu hart für Elfjährige angesehen.\\ \textbf{In den Abgrund mit diesen Narren!} Ihr seid hier für das Fach, das seit achthundert Jahren in Hogwarts gelehrt wird!\\

\textbf{Willkommen in eurem ersten Jahr der Kampfmagie!}"

Harry begann zu applaudieren. Er konnte nicht anders, es war zu anregend.\\ Als Harry zu klatschen begann, gab es einige vereinzelte Reaktionen aus Gryffindor und noch mehr aus Slytherin, aber die meisten Schüler schienen einfach zu verblüfft, um zu reagieren.\\ Professor Quirrell machte eine schneidende Geste, und der Applaus erstarb augenblicklich.

"Ich danke Ihnen sehr", sagte Professor Quirrell.\\ "Nun zu den praktischen Dingen. Ich habe alle meine Kampfklassen des ersten Jahres zu einer zusammengefasst, was es mir erlaubt, Ihnen doppelt so viel Unterrichtszeit zu bieten wie normalerweise -"\\ Es gab entsetzte Aufschreie.\\ "- ein erhöhtes Pensum, das ich dadurch ausgleichen werde, dass ich keine Hausaufgaben auftrage."\\ Die Schreie des Entsetzens verstummten abrupt.\\ "Ja, Sie haben mich richtig verstanden. Ich werde euch beibringen, wie man kämpft, und nicht, wie ihr bis Montag zwölf Zentimeter über Kämpfen schreiben musst."

Harry wünschte sich verzweifelt, er hätte sich neben Hermine gesetzt, damit er ihren Gesichtsausdruck jetzt sehen könnte, aber andererseits war er sich ziemlich sicher, dass er sich das genau einbildete.\\ Außerdem war Harry verliebt. \emph{Es würde eine Dreierhochzeit werden: er, der Zeitumkehrer und Professor Quirrell.}

"Für diejenigen von euch, die es wünschen, habe ich ein paar Aktivitäten nach der Schule arrangiert, die ihr sicher sehr interessant und lehrreich finden werdet.\\ Wollt ihr der Welt eure eigenen Fähigkeiten zeigen, anstatt vierzehn anderen Leuten beim Quidditchspielen zuzusehen? Mehr als sieben Leute können in einer Armee kämpfen."\\ \emph{\hfill\break Verdammt gut.}

"Diese und andere außerschulische Aktivitäten bringen euch auch Quirrell-Punkte ein. Was sind Quirrell-Punkte, fragt ihr? Das Hauspunktesystem passt nicht zu meinen Bedürfnissen, weil es Hauspunkte zu selten macht.\\ Ich ziehe es vor, meine Schüler häufiger wissen zu lassen, wie sie sich schlagen. Und bei den seltenen Gelegenheiten, bei denen ich euch einen schriftlichen Test anbiete, wird er sich selbst bewerten, und wenn ihr zu viele zusammenhängende Fragen falsch beantwortet, zeigt euer Test die Namen der Schüler an, die diese Fragen richtig beantwortet haben, und diese Schüler können Quirrell-Punkte verdienen, indem sie euch helfen."\\ .\\ \emph{..wow. Warum benutzen die anderen Professoren nicht so ein System?}

"Wozu sind Quirrell-Punkte gut, fragt ihr euch? Für den Anfang sind zehn Quirrell-Punkte einen Hauspunkt wert.\\ Aber sie bringen euch auch andere Gefallen ein. Wollt ihr eure Prüfung zu einer ungewöhnlichen Zeit ablegen? Gibt es eine bestimmte Sitzung, die Sie am liebsten ausfallen lassen würden? Sie werden feststellen, dass ich für Schüler, die genügend Quirrell-Punkte angesammelt haben, sehr flexibel sein kann.\\ Quirrell-Punkte werden die Generäle der Armeen kontrollieren. Und zu Weihnachten - kurz vor den Weihnachtsferien - werde ich jemandem einen Wunsch erfüllen.\\ Jede schulbezogene Leistung, die in meiner Macht, meinem Einfluss oder vor allem meinem Einfallsreichtum liegt.\\ Ja, ich war in Slytherin und ich biete an, einen listigen Plan für euch auszuarbeiten,\\ wenn es das ist, was nötig ist, um den Wunsch zu erfüllen.\\ Dieser Wunsch geht an denjenigen, der in allen 7 Jahren die meisten Quirrell-Punkte gesammelt hat."

\emph{Das wäre dann wohl Harry.}

"Lasst jetzt eure Bücher und losen Gegenstände an euren Schreibtischen - sie werden sicher sein, die Bildschirme werden sie für euch bewachen - und kommt auf diese Plattform herunter.\\ Es ist Zeit, ein Spiel zu spielen, das "\emph{Wer ist der gefährlichste Schüler im Klassenzimmer?}" heißt.

Harry drehte seinen Zauberstab in der rechten Hand und sagte: "Ma-ha-su!" Es gab ein weiteres hochfrequentes "\emph{Bing}" von der schwebenden blauen Kugel, die Professor Quirrell Harry als Ziel zugewiesen hatte.\\ Dieses spezielle Geräusch bedeutete einen perfekten Treffer, den Harry bei neun seiner letzten zehn Versuche erhalten hatte.\\ Irgendwo hatte Professor Quirrell einen Zauber ausgegraben, der unglaublich leicht auszusprechen war, eine lächerlich einfache Zauberstabbewegung hatte und dazu neigte, dort zu treffen, wo man gerade hinschaute.\\ Professor Quirrell hatte verächtlich verkündet, dass echte Kampfmagie viel schwieriger sei als das.

Dass die Verhexung im tatsächlichen Kampf völlig nutzlos sei. Dass es sich um einen kaum geordneten Zauberstoß handelte, dessen einziger wirklicher Sinn das Zielen Üben war, und dass er, wenn er traf, einen Schmerz erzeugte, der kurzzeitig einem harten Schlag auf die Nase gleichkam.\\ Dass der einzige Zweck dieses Tests darin bestand, zu sehen, wer ein schneller Lerner war, da Professor Quirrell sicher war, dass niemand zuvor auf diese Verhexung oder etwas Ähnliches gestoßen war.

Harry interessierte sich für nichts davon. "Ma-ha-su!?"\\ Ein roter Energieblitz schoss aus seinem Zauberstab und traf das Ziel, und die blaue Kugel machte wieder das "\emph{Bing}", was bedeutete, dass der Zauberspruch tatsächlich funktioniert hatte.\\ Harry fühlte sich zum ersten Mal, seit er nach Hogwarts gekommen war, wie ein richtiger Zauberer. Er wünschte sich, die Zielscheibe würde ausweichen, wie die kleinen Kugeln, die Ben Kenobi für das Training von Luke benutzt hatte, aber aus irgendeinem Grund hatte Professor Quirrell stattdessen alle Schüler und Zielscheiben in einer ordentlichen Reihenfolge aufgestellt, die sicherstellte, dass sie nicht aufeinander schießen würden.\\ Also senkte Harry seinen Zauberstab, hüpfte nach rechts, schnappte nach seinem Zauberstab, drehte ihn und rief: "Ma-ha-su!" Es gab ein leiseres "\emph{dong}", was bedeutete, dass er es fast richtig gemacht hatte. Harry steckte seinen Zauberstab in seine Tasche, hüpfte zurück nach links, zog und feuerte einen weiteren roten Energieblitz ab.\\ Das hohe \emph{Bing}, das dabei entstand, war mit Abstand eines der befriedigendsten Geräusche, die er je in seinem Leben gehört hatte.\\ Harry wollte aus vollem Halse triumphierend schreien.

ICH KANN ZAUBERN! FÜRCHTET MICH, GESETZE DER PHYSIK, ICH KOMME, UM EUCH ZU VERLETZEN!

"Ma-ha-su!" Harrys Stimme war laut, aber kaum wahrnehmbar über dem ständigen Gesang ähnlicher Rufe, die von der Plattform des Klassenzimmers herüberschallten.

"Genug", sagte Professor Quirrells verstärkte Stimme.

(Sie klang nicht laut. Sie klang wie eine normale Lautstärke, die direkt hinter der linken Schulter kam, egal, wo man relativ zu Professor Quirrell stand.)

"Wie ich sehe, hat es jetzt jeder von euch mindestens einmal geschafft."\\ Die Zielkugeln färbten sich rot und begannen, zur Decke hinauf zu schweben.\\ Professor Quirrell stand auf dem erhöhten Podest in der Mitte der Plattform und stützte sich mit einer Hand leicht auf das Lehrerpult.

"Ich habe euch gesagt", sagte Professor Quirrell,\\ "dass wir ein Spiel spielen werden, das '\emph{Wer ist der gefährlichste Schüler im Klassenzimmer'} heißt.\\ Es gibt einen Schüler in diesem Klassenzimmer, der die sumerische Einfacher-Schlag-Verhexung schneller beherrscht als jeder andere -"

\emph{Oh, blah, blah, blah.}

"- und daraufhin sieben anderen Schülern geholfen hat. Dafür hat sie die ersten 7 Quirrell-Punkte verdient, die in eurem Jahrgang vergeben werden.\\ Komm heraus, Hermine Granger. Es ist Zeit für die nächste Phase des Spiels."

Hermine Granger schritt vorwärts, mit einem gemischten Ausdruck von Triumph und Besorgnis auf ihrem Gesicht.\\ Die Ravenclaws sahen ihr stolz nach, die Slytherins mit hasserfüllten Blicken und Harry mit offenem Ärger. Harry hatte sich dieses Mal gut geschlagen.\\ Wahrscheinlich war er sogar in der oberen Hälfte der Klasse, jetzt, wo alle einen ebenso unbekannten Zauberspruch vor sich hatten und Harry die gesamte Magische Theorie von Adalbert Schwafel durchgelesen hatte.\\ Und trotzdem war Hermine immer noch besser. Irgendwo in seinem Hinterkopf war die Befürchtung, dass Hermine einfach schlauer war als er.

Aber im Moment stützte Harry seine Hoffnungen auf die bekannten Tatsachen, dass\\ (a) Hermine viel mehr als die Standardlehrbücher gelesen hatte und\\ (b) Adalbert Schwafel ein uninspirierter Trottel war, der die Magische Theorie geschrieben hatte, um einer Schulbehörde zu gefallen, die nicht viel von Elfjährigen hielt.

Hermine erreichte das zentrale Podium und trat nach oben.\\ "Hermine Granger hat einen völlig unbekannten Zauberspruch in zwei Minuten gemeistert, fast eine ganze Minute schneller als die Zweitplatzierte."

Professor Quirrell drehte sich langsam auf seinem Platz um und schaute alle Schüler an, die sie beobachteten.

"Könnte Miss Grangers Intelligenz sie zur gefährlichsten Schülerin im Klassenzimmer machen? Und? Was denkt ihr?"\\ Keiner schien im Moment etwas zu denken.\\ Selbst Harry war sich nicht sicher, was er sagen sollte.\\ "Lasst es uns herausfinden, oder?", sagte Professor Quirrell.\\ Er wandte sich wieder an Hermine und gestikulierte in Richtung der weiteren Klasse.\\ "Wählen Sie einen beliebigen Schüler aus und zaubern Sie die Verhexung Einfacher Schlag auf ihn."

Hermine erstarrte, wo sie stand.

"Komm jetzt", sagte Professor Quirrell sanft.\\ "Sie haben diesen Zauberspruch schon über fünfzig Mal perfekt gewirkt.\\ Er ist nicht dauerhaft schädlich oder auch nur besonders schmerzhaft. Er tut so weh wie ein harter Schlag und hält nur ein paar Sekunden an."\\ Professor Quirrells Stimme wurde härter.\\ "Das ist ein direkter Befehl von Ihrem Professor, Miss Granger. Wählen Sie ein Ziel und feuern Sie eine Einfache Schlagverhexung."

Hermines Gesicht verzog sich vor Entsetzen und ihr Zauberstab zitterte in ihrer Hand.\\ Harrys eigene Finger umklammerten seinen eigenen Zauberstab aus Mitgefühl fest. Obwohl er sehen konnte, was Professor Quirrell zu tun versuchte.\\ Auch wenn er sehen konnte, worauf Professor Quirrell hinauswollte.

"Wenn Sie Ihren Zauberstab nicht heben und schießen, Miss Granger, verlieren Sie einen Quirrell-Punkt."

Harry starrte Hermine an und wollte, dass sie in seine Richtung schaute.\\ Seine rechte Hand klopfte sanft auf seine eigene Brust. Nimm mich, ich habe keine Angst … Hermines Zauberstab zuckte in ihrer Hand; dann entspannte sich ihr Gesicht, und sie ließ ihren Zauberstab an ihre Seite sinken.

"Nein", sagte Hermine Granger. Ihre Stimme war ruhig, und obwohl sie nicht laut war, hörte sie jeder in der Stille.

"Dann muss ich dir einen Punkt abziehen", sagte Professor Quirrell. "Das ist ein Test, und du hast ihn nicht bestanden."

Das erreichte sie.\\ Harry konnte es sehen. Aber sie hielt ihre Schultern gerade. Professor Quirrells Stimme war mitfühlend und schien den ganzen Raum zu erfüllen.

"Es reicht nicht immer aus, Dinge zu wissen, Miss Granger. Wenn Sie nicht in der Lage sind, Gewalt zuzufügen und zu empfangen, die so weh tut wie wenn man sich den Zeh stößt, dann können Sie sich nicht verteidigen und Sie werden die Verteidigungsklasse nicht bestehen.\\ Bitte setzen Sie sich wieder zu Ihren Klassenkameraden."

Hermine ging zurück zum Ravenclaw-Haufen.\\ Ihr Gesicht sah friedlich aus und Harry wollte aus irgendeinem seltsamen Grund anfangen zu klatschen.\\ \emph{Auch wenn Professor Quirrell recht gehabt hatte.}

"Also", sagte Professor Quirrell.\\ "Es wird deutlich, dass Hermine Granger nicht die gefährlichste Schülerin im Klassenzimmer ist.\\ Was glauben Sie, wer könnte hier tatsächlich die gefährlichste Person sein? - Außer mir, natürlich."

Ohne zu überlegen, drehte sich Harry zu den Slytherins um.

"Draco, aus dem edlen und sehr alten Haus Malfoy", sagte Professor Quirrell.\\ "Es scheint, dass viele deiner Mitschüler in deine Richtung blicken.\\ Komm heraus, wenn du willst."

Draco tat dies und ging mit einem gewissen Stolz in seiner Haltung.\\ Er trat auf das Podium und sah mit einem Lächeln zu Professor Quirrell auf.

"Mr. Malfoy", sagte Professor Quirrell. "Feuer."

Harry hätte versucht, ihn zu stoppen, wenn noch Zeit gewesen wäre, aber in einer einzigen geschmeidigen Bewegung drehte sich Draco auf dem Ravenclaw-Kontingent, hob seinen Zauberstab und sagte "Mahasu!", als wäre das alles eine Silbe und Hermine sagte "\textbf{\emph{Au}}!"\\ und das war's.

"Gut getroffen", sagte Professor Quirrell.\\ "Zwei Quirrell-Punkte für dich. Aber sagen Sie mir, warum haben Sie es auf Miss Granger abgesehen?"

Es gab eine Pause.\\ Schließlich sagte Draco:\\ "Weil sie am meisten auffiel."

Professor Quirrells Lippen verzogen sich zu einem dünnen Lächeln.\\ "Und das ist der wahre Grund, warum Draco Malfoy so gefährlich ist. Hätte er einen anderen ausgewählt, würde sich das Kind eher darüber ärgern, dass es ausgesucht wurde, und Mr.\\ Malfoy würde sich eher einen Feind machen. Und obwohl Mr. Malfoy vielleicht eine andere Begründung für die Auswahl gegeben hätte, hätte ihm das nichts gebracht, außer einige von euch zu verprellen, während andere ihm bereits zujubeln, ob er nun etwas sagt oder nicht.\\ Was bedeutet, dass Mr. Malfoy gefährlich ist, weil er weiß, wen er angreift und wen nicht, wie er sich Verbündete macht und vermeidet, sich Feinde zu machen.\\ 2 weitere Quirrell-Punkte für Sie, Mr. Malfoy. Und da Sie eine beispielhafte Tugend von Slytherin demonstriert haben, denke ich, dass auch das Haus Salazar einen Punkt verdient hat.\\ Sie dürfen sich wieder Ihren Freunden anschließen."

Draco verbeugte sich leicht und ging zurück zum Slytherin-Kontingent.\\ Ein wenig Klatschen kam von den grüngesäumten Roben, aber Professor Quirrell machte eine schneidende Geste und es wurde wieder still.

"Es scheint, dass unser Spiel vorbei ist", sagte Professor Quirrell.\\ "Und doch gibt es in diesem Klassenzimmer einen einzigen Schüler, der gefährlicher ist als der Spross von Malfoy."

Und aus irgendeinem Grund schienen jetzt sehr viele Leute auf…\\ "Harry Potter. Komm heraus."

Das verhieß nichts Gutes. Widerwillig ging Harry auf Professor Quirrell zu, der auf seinem erhöhten Podest stand, immer noch leicht an das Lehrerpult gelehnt.\\ Die Nervosität, ins Rampenlicht gestellt zu werden, schien Harrys Verstand zu schärfen, während er sich dem Podium näherte, und sein Verstand wühlte sich durch Möglichkeiten, was Professor Quirrell denken könnte, um Harrys Gefährlichkeit zu demonstrieren.\\ Würde man ihn bitten, einen Zauber zu sprechen? Einen Dunklen Lord zu besiegen? Seine angebliche Immunität gegen den Tötungsfluch zu demonstrieren? Sicherlich war Professor Quirrell zu klug für so etwas… Harry blieb kurz vor dem Podium stehen, und Professor Quirrell forderte ihn nicht auf, näher zu kommen.

"Die Ironie ist", sagte Professor Quirrell,\\ "dass ihr alle die richtige Person aus den völlig falschen Gründen angeschaut habt.\\ Ihr denkt", Professor Quirrells Lippen verzogen sich,\\ "dass Harry Potter den Dunklen Lord besiegt hat und deshalb sehr gefährlich sein muss.\\ Bah. Er war ein Jahr alt.\\ Welche Laune des Schicksals auch immer den Dunklen Lord getötet hat, hatte wahrscheinlich wenig mit Mr. Potters Fähigkeiten als Kämpfer zu tun.\\ Aber nachdem ich Gerüchte über einen Ravenclaw gehört hatte, der es mit fünf älteren Slytherins aufgenommen hatte, befragte ich mehrere Augenzeugen und kam zu dem Schluss, dass Harry Potter mein gefährlichster Schüler sein würde."

Ein Adrenalinstoß ergoss sich in Harrys System und ließ ihn aufrechter stehen.\\ Er wusste nicht, zu welchem Schluss Professor Quirrell gekommen war, aber das konnte nichts Gutes bedeuten.

"Ah, Professor Quirrell -"\\ begann Harry zu sagen. Professor Quirrell sah amüsiert aus.

"Sie denken, dass ich eine falsche Antwort gegeben habe, nicht wahr, Mr. Potter? Sie werden lernen, etwas Besseres von mir zu erwarten."

Professor Quirrell richtete sich von dort auf, wo er sich auf den Schreibtisch gelehnt hatte.\\ "Mr. Potter, alle Dinge haben ihre gewohnten Verwendungen. Nennen Sie mir zehn ungewohnte Verwendungen von Gegenständen in diesem Raum für den Kampf!"

Einen Moment lang war Harry sprachlos vor lauter Schock, verstanden worden zu sein.\\ Und dann begannen die Ideen zu sprudeln.

"Es gibt Schreibtische, die schwer genug sind, um bei einem Sturz aus großer Höhe tödlich zu sein.\\ Es gibt Stühle mit Metallbeinen, die jemanden aufspießen könnten, wenn sie hart genug gestoßen werden.\\ Die Luft in diesem Klassenzimmer wäre tödlich, da Menschen im Vakuum sterben, und sie kann als Träger für Giftgase dienen."

Harry musste kurz innehalten, um Luft zu holen, und in diese Pause hinein sagte Professor Quirrell:\\ "Das sind drei. Sie brauchen zehn. Der Rest der Klasse denkt, dass Sie bereits den gesamten Inhalt des Klassenzimmers verbraucht haben."

"Ha!\\ Der Boden kann entfernt werden, um eine Stachelgrube zu schaffen, in die man fallen kann, die Decke kann auf jemanden einstürzen, die Wände können als Rohmaterial für die Verwandlung in jede Menge tödliche Dinge dienen - Messer zum Beispiel."

"Das sind sechs. Aber Sie kratzen doch sicher schon am Boden des Fasses?"

"Ich habe noch nicht mal angefangen! Sehen Sie sich nur die vielen Leute an!\\ Dass ein Gryffindor den Feind für dich angreift, ist natürlich ein normaler Einsatz -"

"Den zähle ich nicht mit." grinste Quirrel

"- aber ihr Blut kann auch benutzt werden, um jemanden zu ertränken.\\ Ravenclaws sind für ihre Gehirne bekannt, aber ihre inneren Organe könnten auf dem Schwarzmarkt für genug Geld verkauft werden, um einen Attentäter anzuheuern.\\ Slytherins sind nicht nur als Attentäter nützlich, sie können auch mit ausreichender Geschwindigkeit geworfen werden, um einen Gegner zu zerquetschen.\\ Und Hufflepuffs sind nicht nur harte Arbeiter, sondern haben auch Knochen, die man herausnehmen, schärfen und benutzen kann, um jemanden zu erstechen."

Inzwischen starrte der Rest der Klasse Harry mit einem gewissen Entsetzen an.\\ Selbst die Slytherins sahen schockiert aus.

"Das sind zehn, obwohl ich großzügig bin, wenn ich den aus Ravenclaw mitzähle.\\ Und jetzt, als Extrapunkt, einen Quirrell-Punkt für jeden Einsatz von Gegenständen in diesem Raum, die Sie noch nicht benannt haben."\\ Professor Quirrell bedachte Harry mit einem kameradschaftlichen Lächeln.\\ "Der Rest deiner Klasse denkt, dass du jetzt in Schwierigkeiten steckst, da du alles bis auf die Zielscheiben benannt hast und du keine Ahnung hast, was man mit diesen anstellen kann."

"Pah!\\ Ich habe alle Personen benannt, aber nicht meinen Umhang, mit dem man einen Gegner ersticken kann, wenn man ihn oft genug um den Kopf wickelt, oder Hermine Grangers Umhang, den man in Streifen reißen und zu einem Seil binden kann, um jemanden zu erhängen, oder Draco Malfoys Umhang, mit dem man ein Feuer entfachen kann -"

"3 Punkte", sagte Professor Quirrell, "jetzt keine Kleidung mehr."

"Mein Zauberstab kann durch die Augenhöhle in das Gehirn eines Feindes gestoßen werden", und jemand gab einen entsetzten, würgenden Laut von sich.

"4 Punkte, keine Zauberstäbe mehr."

"Meine Armbanduhr kann jemanden ersticken, wenn sie ihm in den Hals gestopft wird -"

"5 Punkte, und es reicht."

"Hmpf", sagte Harry. "Zehn Quirrell-Punkte gegen einen Hauspunkt, richtig? Sie hätten mich weitermachen lassen sollen, bis ich den Hauspokal gewonnen habe, ich habe noch nicht einmal mit den ungewohnten Verwendungen von allem, was ich in meinen Taschen habe, angefangen"\\ oder dem Beutel selbst und er konnte nicht über den Zeitumkehrer oder den Unsichtbarkeitsumhang sprechen, aber es musste etwas geben, was er über diese roten Kugeln sagen konnte…

"Genug, Mr. Potter. Glaubt ihr alle, ihr wisst, was Mr. Potter zum gefährlichsten Schüler der Klasse macht?"

Es gab ein leises Gemurmel der Zustimmung.

"Sprechen Sie es bitte laut aus. Terry Boot, was macht Ihren Mitschüler so gefährlich?"

"Ah … ähm … er ist kreativ?"

\textbf{"Falsch!", brüllte Professor Quirrell, und seine Faust schlug mit einem verstärkten Geräusch, das alle aufschrecken ließ, scharf auf seinen Schreibtisch.}

"Alle Ideen von Mr. Potter waren mehr als unbrauchbar!"\\ Harry war überrascht.\\ "Den Boden entfernen, um eine Stachelfalle zu schaffen? Lächerlich!\\ Im Kampf hat man nicht diese Art von Vorbereitungszeit, und wenn man sie hätte, gäbe es hundert bessere Verwendungsmöglichkeiten!\\ Material von den Wänden verwandeln? Mr. Potter kann keine Verwandlung durchführen! Mr. Potter hatte genau eine Idee, die er sofort anwenden konnte, genau jetzt, ohne lange Vorbereitung oder einen kooperativen Gegner oder Magie, die er nicht kennt.\\ Diese Idee war, seinen Zauberstab durch die Augenhöhle seines Feindes zu rammen. Was seinen Zauberstab eher zerbrechen als seinen Gegner töten würde!\\ Kurz gesagt, Mr. Potter, ich fürchte, Ihre Vorschläge waren durchweg schrecklich."

"Was?" sagte Harry entrüstet.\\ "Sie haben nach ungewöhnlichen Ideen gefragt, nicht nach praktischen! Ich habe über den Tellerrand geschaut!\\ Wie würden Sie etwas in diesem Klassenzimmer benutzen, um jemanden zu töten?"

Professor Quirrells Gesichtsausdruck war missbilligend, \emph{aber in seinen Augen lag ein Lächeln.}

"Mr. Potter, ich habe nie gesagt, dass Sie \emph{töten} sollen. Es gibt eine Zeit und einen Ort, um seinen Feind lebendig zu fangen, und in einem Hogwarts-Klassenzimmer ist normalerweise einer dieser Orte.\\ Aber um Ihre Frage zu beantworten: Schlagen Sie ihnen mit der Kante eines Stuhls auf den Hals."

Die Slytherins lachten etwas, aber sie lachten mit Harry, nicht über ihn.\\ Alle anderen schauten eher entsetzt.

"Aber Mr. Potter hat jetzt demonstriert, warum er der gefährlichste Schüler im Klassenzimmer ist.\\ Ich habe nach ungewohnten Verwendungen von Gegenständen in diesem Raum für den Kampf gefragt. Mr. Potter hätte vorschlagen können, ein Pult zu benutzen, um einen Fluch abzublocken, oder einen Stuhl, um einen entgegenkommenden Feind zum stolpern zu bringen, oder Stoff um seinen Arm zu wickeln, um einen improvisierten Schild zu schaffen.\\ \emph{Stattdessen war jede einzelne Verwendung, die Mr. Potter nannte, eher offensiv als defensiv und entweder tödlich oder potenziell tödlich.}"

\emph{Was? Das kann nicht wahr sein.\\ } \emph{Moment, das konnte doch nicht wahr sein…}

Harry hatte ein plötzliches Schwindelgefühl, als er versuchte, sich daran zu erinnern, was genau er vorgeschlagen hatte, sicher musste es ein Gegenbeispiel geben…

"Und das", sagte Professor Quirrell,\\ "ist der Grund, warum Mr. Potters Ideen so seltsam und nutzlos waren - weil er weit ins Unpraktische hineinreichen musste, um seinem Anspruch gerecht zu werden, \emph{den Feind zu töten.}\\ Für ihn war jede Idee, die dem nicht entsprach, nicht erwägenswert. Dies spiegelt eine Eigenschaft wider, die wir als \emph{Tötungsabsicht} bezeichnen könnten.

\emph{Ich habe sie. Harry Potter hat sie,}\\ deshalb konnte er fünf ältere Slytherins niederstrecken. Draco Malfoy hat sie nicht. Noch nicht. Mr. Malfoy würde kaum vor dem Gerede über gewöhnlichen Mord zurückschrecken, aber selbst er war schockiert - ja, das waren Sie, Mr. Malfoy, ich habe Ihr Gesicht beobachtet - als Mr. Potter beschrieb, wie man die Körper seiner Mitschüler als Rohmaterial verwendet.

Es gibt Zensoren in eurem Kopf, die euch vor solchen Gedanken zurückschrecken lassen. Mr. Potter denkt nur daran, den Feind zu töten, er wird zu jedem Mittel greifen, um dies zu tun, er zuckt nicht zurück, seine Zensoren sind ausgeschaltet.\\ Obwohl sein jugendliches Genie so undiszipliniert und unpraktisch ist, dass es nutzlos ist, m\emph{acht seine Absicht zu töten Harry Potter zum gefährlichsten Schüler im Klassenzimmer.}

Ein letzter Punkt für ihn - nein, machen wir das zu einem Punkt für Ravenclaw - für dieses unabdingbare Requisit eines wahren Kampfzauberers."

Harrys Mund klaffte in sprachlosem Schock auf, während er verzweifelt nach etwas suchte, das er dazu sagen konnte.

\emph{Das ist so ganz und gar nicht das, worum es mir geht!}

Aber er konnte sehen, dass die anderen Schüler begannen, es zu glauben.\\ Harrys Verstand blätterte durch mögliche Dementis und fand nichts, das der autoritären Stimme von Professor Quirrell standhalten konnte.\\ Das Beste, was Harry einfiel, war:\\ \emph{"Ich bin kein Psychopath, ich bin nur sehr kreativ"}, und das klang irgendwie bedrohlich.\\ Er musste etwas Unerwartetes sagen, etwas, das die Leute dazu bringen würde, innezuhalten und umzudenken -

"Und jetzt", sagte Professor Quirrell. "Mr. Potter. Feuer."\\ Natürlich passierte nichts.\\ "Ah, nun", sagte Professor Quirrell. Er seufzte.\\ "Ich nehme an, wir müssen alle irgendwo anfangen. Mr. Potter, wählen Sie einen beliebigen Schüler für eine einfache Schlagverhexung aus.\\ Sie werden das tun, bevor ich Ihre Klasse für heute entlasse. Wenn Sie das nicht tun, werde ich anfangen, Hauspunkte abzuziehen, und zwar so lange, bis Sie es tun."

Harry hob vorsichtig seinen Zauberstab.\\ So viel musste er tun, sonst würde Professor Quirrell sofort anfangen, Hauspunkte abzuziehen.Langsam, wie auf einem Bratenteller, drehte sich Harry zu den Slytherins um.\\ Und Harrys Augen trafen die von Draco. Draco Malfoy sah nicht im Geringsten ängstlich aus. Der blondhaarige Junge gab kein sichtbares Zeichen der Zustimmung, wie Harry es bei Hermine getan hatte, aber das konnte man auch kaum von ihm erwarten.\\ Die anderen Slytherins würden das eher merkwürdig finden.

"Warum das Zögern?", fragte Professor Quirrell. "Sicherlich gibt es nur eine offensichtliche Wahl."

"Ja", sagte Harry. "Nur eine offensichtliche Wahl."\\ Harry drehte den Zauberstab und sagte:\\ \textbf{"Ma-ha-su!"}

Es herrschte völlige Stille im Klassenzimmer. Harry schüttelte seinen linken Arm und versuchte, das anhaltende Brennen loszuwerden. Es herrschte noch mehr Stille. Schließlich seufzte Professor Quirrell.

"Ja, ziemlich genial, aber es gab eine Lektion zu erteilen und Sie sind ihr ausgewichen.\\ Einen Punkt abzug von Ravenclaw für die Zurschaustellung Ihrer eigenen Cleverness auf Kosten des eigentlichen Ziels. Die Klasse ist entlassen."

Und bevor noch jemand etwas sagen konnte, rief Harry aus:\\ "War nur ein Scherz! RAVENCLAW!"

Danach herrschte für einen kurzen Moment Stille, ein Geräusch, als ob die Leute nachdächten, und dann begann das Gemurmel und steigerte sich schnell zu einer brüllenden Unterhaltung.\\ Harry drehte sich zu Professor Quirrell um, die beiden mussten sich unterhalten - Quirrell war in sich zusammengesackt und stapfte zu seinem Stuhl zurück.

\emph{Nein. Nicht akzeptabel. Sie mussten wirklich reden. Vergiss die Zombie-Nummer, Professor Quirrell würde wahrscheinlich aufwachen, wenn Harry ihn ein paar Mal anstupste.}

Harry machte sich auf den Weg nach vorne -

\textbf{SCHLECHTE IDEE !!! -}

Harry schwankte und blieb stehen, weil ihm schwindelig war. Und dann fiel eine Schar Ravenclaws über ihn her und die Diskussionen begannen.

