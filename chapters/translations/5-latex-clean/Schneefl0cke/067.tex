

\hypertarget{selbstverwirklichung-teil-3}{% \section{68. Selbstverwirklichung, Teil 3}\label{selbstverwirklichung-teil-3}}

\textbf{\uline{Selbstverwirklichung, Teil 3}}

Hermine fühlte sich gerade nicht sehr nett, auch nicht gut, in ihr brannte ein heißer Ball aus Wut und sie fragte sich, ob das so etwas wie Harrys Dunkelheit war (obwohl es wahrscheinlich nicht einmal annähernd so war) und sie hätte sich wegen eines dummen kleinen Spiels nicht so fühlen sollen, aber -

\emph{ihre ganze Armee}. \textbf{\emph{Zwei Soldaten hatten ihre ganze Armee besiegt}}.

Das hatte man ihr gesagt, nachdem sie aufgewacht war. \emph{Das war ein bisschen zu viel.}

"Nun", sagte Professor Quirrell.

Aus der Nähe sah der Verteidigungsprofessor nicht mehr ganz so gesund aus wie beim letzten Mal, als sie in seinem Büro gewesen war; seine Haut sah blasser aus, und er bewegte sich etwas langsamer. Seine Miene war so streng wie immer und sein Blick so durchdringend; seine Finger klopften scharf auf seinen Schreibtisch, \emph{tap-tap.}

"Ich würde vermuten, dass von Ihnen dreien nur Mr~Malfoy erraten hat, warum ich Sie hierher gebeten habe."

"Hat das etwas mit dem Adel und den ältesten Häusern zu tun?", fragte Harry von nebenan und klang verwirrt. "Ich habe doch nicht gegen irgendein verrücktes Gesetz verstoßen, als ich auf Daphne geschossen habe, oder?"

"Nicht ganz", sagte der Mann mit schwerer Ironie. "Da Miss~Greengrass sich nicht auf die korrekten Duellformulare berufen hat, ist sie nicht berechtigt, zu verlangen, dass Ihnen der Name des Hauses aberkannt wird. Obwohl ich natürlich ein formelles Duell nicht erlaubt hätte. Kriege halten sich nicht an solche Regeln."

Der Verteidigungsprofessor beugte sich vor und stützte sein Kinn auf die gespreizten Hände, als ob ihn das aufrechte Sitzen schon ermüdet hätte. Seine Augen blickten sie an, scharf und gefährlich.

"General Malfoy. Warum habe ich Sie herbestellt?"

"General Potter gegen uns beide ist kein fairer Kampf mehr", sagte Draco Malfoy mit ruhiger Stimme.

"Was?", platzte Hermine heraus. "Wir hätten sie fast gehabt, wenn Daphne nicht ohnmächtig geworden wäre -"

"Miss~Greengrass ist nicht aus magischer Erschöpfung in Ohnmacht gefallen", sagte Professor Quirrell trocken. "Mr~Potter hat ihr mit einer Schlafverhexung in den Rücken geschossen, während Ihre Soldaten durch den Anblick ihres Generals, der gegen eine Wand fliegt, abgelenkt waren. Aber trotzdem herzlichen Glückwunsch, Miss~Granger, dass Sie mit nur vierundzwanzig Sonnenschein-Soldaten zwei Chaotische Legionäre fast besiegt haben."

Das Blut, das in ihren Wangen flammte, wurde noch ein wenig heißer.

"Das - das war nur - wenn ich nur gewusst hätte, dass er eine Rüstung trägt -"

Professor Quirrell starrte sie mit zusammengelegten Fingern an.

"Natürlich gibt es Möglichkeiten, wie Sie hätten gewinnen können, Miss~Granger. Die gibt es immer, in jeder verlorenen Schlacht. Die Welt um uns herum strotzt vor Möglichkeiten, explodiert vor Möglichkeiten, die fast alle Leute ignorieren, weil es von ihnen verlangen würde, eine Denkgewohnheit zu verletzen; in jeder Schlacht warten tausend Hufflepuff-Knochen darauf, zu Speeren geschliffen zu werden. Hätten Sie daran gedacht, ein massenhaftes Finite Incantatem nach allgemeinen Prinzipien zu versuchen, hätten Sie Mr~Potters Kettenhemd und alles andere, was er trug, außer seiner Unterwäsche zerschlagen, was mich vermuten lässt, dass Mr~Potter seine eigene Verwundbarkeit nicht ganz erkannt hat. Oder Sie hätten Ihre Soldaten Mr~Potter und Mr~Longbottom ausschwärmen lassen und ihnen die Zauberstäbe physisch aus den Händen reißen können. Mr~Malfoys eigene Reaktion war nicht das, was ich als durchdacht bezeichnen würde, aber zumindest hat er seine tausend Alternativen nicht völlig ignoriert." Ein sardonisches Lächeln. "Aber Sie, Miss~Granger, hatten das Pech, sich daran zu erinnern, wie man den Betäubungs-Fluch wirkt, und deshalb haben Sie Ihr ausgezeichnetes Gedächtnis nicht nach einem Dutzend einfacherer Zauber durchsucht, die sich als wirksam erwiesen hätten. Und Sie haben alle Hoffnungen Ihrer Armee auf Ihre eigene Person gesetzt, so dass sie den Mut verloren, als Sie fielen. Danach setzten sie ihre vergeblichen Schlafverhexungen fort, beherrscht von den Kampfgewohnheiten, die ihnen antrainiert worden waren, unfähig, das Muster zu durchbrechen, wie es Mr~Malfoy tat. Ich kann nicht ganz nachvollziehen, was den Leuten durch den Kopf geht, wenn sie immer wieder dieselbe gescheiterte Strategie wiederholen, aber anscheinend ist es eine erstaunlich seltene Erkenntnis, dass man etwas anderes versuchen kann. Und so wurde das Sonnenschein Regiment von zwei Soldaten ausgelöscht."

Der Verteidigungsprofessor grinste schadenfroh.

"Man erkennt gewisse Ähnlichkeiten dazu, wie fünfzig Todesser das ganze magische Britannien beherrschten und wie unser geliebtes Ministerium weiter regiert."

Der Verteidigungsprofessor seufzte.

"Nichtsdestotrotz, Miss~Granger, bleibt die Tatsache, dass dies nicht die erste derartige Niederlage für Sie ist. In der letzten Schlacht haben Sie und Mr~Malfoy Ihre Kräfte vereint, und dennoch wurden Sie bis zum Stillstand gebracht, so dass Sie und Mr~Malfoy Mr~Potter auf das Dach verfolgen mussten. Die Chaos Legion hat nun zweimal hintereinander eine militärische Stärke demonstriert, die beiden anderen Armeen zusammen entspricht. Das lässt mir keine andere Wahl. General Potter, Sie werden acht Soldaten aus Ihrer Armee auswählen, darunter mindestens einen Chaos-Leutnant, die auf die Drachenarmee und das Sonnenscheinregiment aufgeteilt werden -"

"Was?" platzte Hermine wieder heraus, sie blickte zu den anderen Generälen hinüber und sah, dass Harry genauso geschockt aussah wie sie, während Draco Malfoy nur resigniert aussah.

"General Potter ist stärker als ihr beide zusammen", sagte Professor Quirrell mit ruhiger Präzision. "Euer Wettstreit ist vorbei, er hat gewonnen, und es ist an der Zeit, die drei Armeen neu auszubalancieren, um ihn erneut herauszufordern."

"Professor Quirrell!", sagte Harry. "Ich habe nicht -"

"Das ist meine Entscheidung als Professor für Kampfmagie an der Hogwarts-Schule für Hexerei und Zauberei, und sie ist nicht verhandelbar."

Die Worte waren immer noch präzise, aber der Blick in Professor Quirrells Augen ließ Hermine das Blut in den Adern gefrieren, auch wenn er Harry und nicht sie anschaute.

"Und ich finde es verdächtig, Mr~Potter, dass Sie in dem Moment, in dem Sie Miss~Granger und Mr~Malfoy isolieren und zwingen wollten, Sie auf das Dach zu jagen, genau so viel von ihrer vereinten Kraft vernichten konnten, wie Sie wollten. Das ist in der Tat das Leistungsniveau, das ich seit Beginn dieses Jahres von Ihnen erwartet habe, und ich stelle verärgert fest, dass Sie sich die ganze Zeit über in meinen Klassen zurückgehalten haben! Ich habe gesehen, was Sie wirklich leisten können, Mr~Potter. Sie sind weit über den Punkt hinaus, an dem Mr~Malfoy oder Miss~Granger Sie auf gleicher Augenhöhe bekämpfen können, und es wird Ihnen nicht erlaubt sein, etwas anderes zu behaupten. Dies, Mr~Potter, sage ich Ihnen in meiner Eigenschaft als Ihr Professor: Damit Sie Ihr volles Potenzial ausschöpfen können, müssen Sie Ihre Fähigkeiten voll ausschöpfen und dürfen sich aus keinem Grund zurückhalten - schon gar nicht aus kindischen Ängsten, was Ihre Freunde denken könnten!"

Sie verließ das Büro des Verteidigungsprofessors mit einer größeren Armee und weniger Würde und fühlte sich wie ein trauriger kleiner Käfer, der gerade zerquetscht worden war, und versuchte sehr, sehr hart, nicht zu weinen.

"Ich habe mich nicht zurückgehalten!" sagte Harry, sobald sie um die erste Ecke von Professor Quirrells Büro bogen, in dem Moment, in dem die Holztür hinter den Steinmauern außer Sichtweite verschwand.

"Ich habe mich nicht verstellt, ich habe nie einen von euch gewinnen lassen!"

Sie antwortete nicht, konnte nicht antworten, es würde alles losbrechen, wenn sie versuchte, ein Wort zu sagen.

"Wirklich?", sagte Draco Malfoy. Der Drachengeneral hatte immer noch diesen Hauch von Resignation. "Weil Quirrell recht hat, weißt du, es ist verdächtig, dass du fast jeden in unseren beiden Armeen schlagen konntest, sobald du uns auf das Dach jagen wolltest. Und hast du damals nicht etwas davon gesagt, Potter, dass wir dich besiegen müssten, wenn du wirklich kämpfen würdest?"

Das brennende Gefühl kroch ihre Kehle hinauf, und wenn es ihre Augen erreichte, brach sie in Tränen aus, und von da an war sie für sie beide nur noch ein weinendes kleines Mädchen.

"Das -" Harrys Stimme sagte eindringlich, sie sah ihn nicht an, aber seine Stimme klang so, als hätte er den Kopf zu ihr gedreht.

"Das war - ich habe mich dieses Mal viel mehr angestrengt, es gab einen wichtigen Grund, ich musste es tun, also habe ich einen ganzen Haufen Tricks angewandt, die ich mir aufgespart hatte - und -"

\emph{Sie hatte sich immer sehr angestrengt, jedes Mal.}

"- und ich, ich ließ eine Seite von mir heraus, die ich normalerweise nicht für so etwas wie den Verteidigungskurs benutzen würde -"

\emph{Wenn sie also jemals in die Nähe eines Sieges gegen Harry käme, wenn es wirklich darauf ankäme, könnte er einfach seine dunkle Seite zeigen und sie zerquetschen, war es das? … natürlich war es das. Sie konnte Harry nicht einmal in die Augen sehen, wenn er so dunkel war, wie hatte sie jemals gedacht, dass sie ihn wirklich besiegen könnte?}

Der Korridor gabelte sich, und Harry Potter und Draco Malfoy gingen nach links zu einer Treppe, die in den zweiten Stock führte, und sie ging stattdessen nach rechts, sie wusste nicht einmal, wohin dieser Gang führte, aber im Moment würde sie sich lieber im Schloss verirren.

"Entschuldige, Draco", sagte Harrys Stimme, und dann hörte sie hinter sich das Getrappel von Schritten.

"Lass mich in Ruhe", sagte sie, es klang streng, aber dann musste sie den Mund schließen, die Lippen fest aufeinanderpressen und den Atem anhalten, damit nicht alles herauskam. Der Junge kam einfach weiter und rannte um sie herum und stellte sich vor sie, \emph{weil er dumm war, deshalb,} und Harry sagte, seine Stimme

war jetzt ein hohes und verzweifeltes Flüstern:

"Ich bin nicht weggelaufen, als du mich in all meinen Klassen geschlagen hast, außer beim Besenreiten!"

\emph{Er verstand es nicht, und er würde es nie verstehen, Harry Potter würde es nie verstehen, denn egal welchen Wettbewerb er verlor, er würde immer noch der Junge-der-lebte sein, wenn du Harry Potter warst und Hermine Granger dich schlug, dann bedeutete das, dass jeder von dir erwartete, dass du dich der Herausforderung stellen würdest, wenn du Hermine Granger warst und Harry Potter dich schlug, dann bedeutete das, dass du einfach niemand warst.}

"Das ist nicht fair", sagte sie, ihre Stimme zitterte, aber sie weinte nicht, noch nicht, "ich sollte nicht gegen deine dunkle Seite kämpfen müssen, ich bin nur - ich bin erst-"

\emph{Ich bin erst zwölf,} das war es, was sie dachte.

"Ich habe meine dunkle Seite nur einmal benutzt und das war - als ich musste!"

"Also hast du heute meine ganze Armee besiegt, nur mit deinem normalen Ich?"

Sie weinte immer noch nicht, und sie fragte sich, wie ihr Gesicht jetzt aussah, ob sie wie eine wütende Hermine oder eine traurige aussah.

"Ich -" sagte Harry. Seine Stimme wurde etwas leiser, "Ich habe nicht… wirklich erwartet zu gewinnen, dieses Mal, ich weiß, ich habe gesagt, ich sei unbesiegbar, aber das war nur, um dir Angst zu machen, ich dachte wirklich nur, wir würden dich ein bisschen aufhalten -"

Sie begann wieder zu gehen, ging direkt an ihm vorbei, und als sie vorbeiging, verengte sich Harrys Gesicht, als würde er gleich weinen.

"Hat Professor Quirrell recht?", kam ein hohes, verzweifeltes Flüstern von hinter ihr. "Wenn ich dich zum Freund habe, werde ich dann immer Angst haben, mich zu bessern, weil ich weiß, dass es deine Gefühle verletzen wird? Das ist nicht fair, Hermine!"

Sie holte Luft, hielt sie an und rannte, ihre Füße trappelten so schnell sie konnten über den Stein, sie rannte so schnell sie sich traute, mit verschwommener Sicht, sie rannte, damit niemand sie hörte, und dieses Mal folgte Harry ihr nicht.

…

Minerva war gerade dabei, das am Montag fällige Pergament für Verwandlung durchzugehen, und hatte gerade ein Pergament aus dem fünften Jahr mit einem Fehler, der möglicherweise jemanden hätte töten können, auf minus zweihundert Punkte herabgestuft. In ihrem ersten Jahr als Professorin hatte sie sich über die Dummheit der älteren Schüler empört, jetzt war sie einfach nur noch resigniert. Manche haben nicht nur nie gelernt, sie haben auch nie gemerkt, dass sie hoffnungslos waren, sie blieben aufgeweckt und eifrig und versuchten es immer weiter. Manchmal glaubten sie einem, wenn man ihnen, bevor sie Hogwarts verließen, sagte, dass sie nie etwas Ungewöhnliches ausprobieren, die freie Verwandlung aufgeben und die Kunst nur durch etablierte Zaubersprüche anwenden sollten; und manchmal… taten sie es nicht.

Sie war gerade dabei, eine besonders verworrene Antwort zu enträtseln, als ein Klopfen an der Tür ihre Gedanken unterbrach; und es war nicht ihre Sprechstunde, aber es hatte nur eine sehr kurze Zeit als Leiterin des Hauses Gryffindor gedauert, bis sie gelernt hatte, ihr Urteil auszusetzen. Man konnte hinterher immer Hauspunkte abziehen.

"Kommen Sie herein", sagte sie mit klarer Stimme.

Das junge Mädchen, das ihr Büro betrat, hatte offensichtlich geweint und sich danach das Gesicht gewaschen, in der Hoffnung, man würde es nicht sehen -

"Miss~Granger!", sagte Professor McGonagall. Sie hatte einen Moment gebraucht, um das Gesicht mit den geröteten Augen und den aufgeplusterten Wangen zu erkennen.

"Was ist passiert?"

"Professor", sagte das junge Mädchen mit schwankender Stimme, "Sie sagten, wenn ich mir jemals Sorgen mache oder mich wegen irgendetwas unwohl fühle, sollte ich sofort zu Ihnen kommen -"

"Ja", sagte Professor McGonagall, "was ist passiert?"

Das Mädchen begann zu erklären -

….

Hermine blieb stehen, und die Treppe drehte sich um sie herum, eine sich drehende Spirale, die sie eigentlich gar nicht weiterbringen sollte, sondern sie stattdessen ständig nach oben trug. Hermine dachte, dass es wie der Zauber der Endlosen Treppe aussah, der 1733 von dem Zauberer Arram Sabeti erfunden worden war, der auf dem Gipfel des Mount Everest gelebt hatte, als noch keine Muggel ihn besteigen durften. \emph{Nur konnte das nicht stimmen, denn Hogwarts war viel älter - vielleicht war der Zauber neu erfunden worden?}

Sie hätte Angst haben müssen, hätte nervös sein müssen wegen ihres zweiten Treffens mit dem Schulleiter. \emph{In der Tat war sie ängstlich und nervös vor ihrem zweiten Treffen mit dem Schulleiter.} Nur Hermine Granger hatte nachgedacht; sie hatte viel nachgedacht, nachdem sie nicht mehr hatte weiterlaufen können und mit brennenden Lungen gegen die Wand gerutscht war, nachgedacht, während sie sich zu einem Ball zusammengerollt hatte, mit dem Rücken an der kalten Steinwand, die Beine angezogen und weinend.

\emph{Selbst wenn sie gegen Harry Potter verlor, würde sie niemals, niemals gegen Draco Malfoy verlieren, das war einfach absolut inakzeptabel, und Professor Quirrell hatte General Malfoy dafür gelobt, dass er seine tausend Alternativen nicht ignorierte;} und so hatte Hermine, nachdem sie sich ausgeweint hatte, an vierzehn andere Zauber gedacht, die sie gegen Harry und Neville hätte ausprobieren sollen, und dann hatte sie angefangen, sich zu fragen, ob sie vielleicht den gleichen Fehler bei anderen Dingen machte; und so war sie bei Professor McGonagall gelandet. Nicht um Hilfe zu bitten, denn im Moment hatte Hermine keine Pläne, bei denen sie um Hilfe bitten konnte, sondern nur, um Professor McGonagall alles zu erzählen, denn als sie darüber nachgedacht hatte, war ihr das wie eine der tausend Alternativen erschienen, von denen Professor Quirrell gesprochen hatte. Und sie hatte Professor McGonagall davon erzählt, wie Harry Potter sich verändert hatte, seit dem Tag, an dem der Phönix auf seiner Schulter gewesen war, und davon, wie die Leute sie immer mehr nur noch als etwas von Harry zu sehen schienen, und wie es schien, dass Harry sich in ihrem Schuljahr immer weiter von allen anderen entfernte und manchmal mit einer traurigen Miene herumlief, als würde er etwas verlieren, und sie wusste nicht mehr, was sie tun sollte.

Und Professor McGonagall hatte ihr gesagt, dass sie mit dem Schulleiter sprechen müssten. Und Hermine hatte sich besorgt gefühlt, aber dann war ihr der Gedanke gekommen, dass Harry Potter keine Angst vor dem Schulleiter gehabt hätte. Harry Potter wäre einfach nach vorne gestürmt und hätte getan, was auch immer er vorhatte. Vielleicht (der Gedanke war ihr gekommen) war es den Versuch wert, so zu sein, keine Angst zu haben, einfach zu tun, was auch immer, und zu sehen, was mit ihr geschah, es konnte nicht wirklich schlimmer sein.

Die Endlose Treppe hörte auf, sich zu drehen. Die große Eichentür vor ihnen mit dem messingfarbenen Greifenklopfer öffnete sich, ohne dass sie sie berührten. Hinter einem schwarzen Eichenschreibtisch mit Dutzenden von Schubladen, die in alle Richtungen zeigten und so aussahen, als wären Schubladen in andere Schubladen eingesetzt, saß der silberbärtige Schulleiter von Hogwarts auf seinem Thron, Albus Percival Wulfric Brian Dumbledore, in dessen sanft funkelnde Augen Hermine etwa drei Sekunden lang blickte, bevor sie von all den anderen Dingen im Raum abgelenkt wurde.

Einige Zeit später - sie war sich nicht sicher, wie lange, aber es war, während sie zum dritten Mal versuchte, die Anzahl der Dinge im Raum zu zählen und immer noch nicht die gleiche Antwort bekam, obwohl ihre Erinnerung darauf bestand, dass nichts hinzugefügt oder entfernt worden war - räusperte sich der Schulleiter und sagte: "Miss~Granger?"

Hermines Kopf ruckte herum, und sie spürte ein wenig Hitze in ihren Wangen; aber Dumbledore schien überhaupt nicht verärgert über sie zu sein, nur gelassen und mit einem fragenden Blick in diesen milden, halbverglasten Augen.

"Hermine", sagte Professor McGonagall, die Stimme der älteren Hexe war sanft und ihre Hand ruhte beruhigend auf Hermines Schulter, "bitte sagen Sie dem Schulleiter, was Sie mir über Harry erzählt haben."

Hermine begann zu sprechen, trotz ihrer neu gewonnenen Entschlossenheit stolperte ihre Stimme immer noch ein wenig vor Nervosität, als sie beschrieb, wie Harry sich in den letzten Wochen verändert hatte, seit Fawkes auf seiner Schulter gewesen war.

Als sie fertig war, gab es eine Pause, und dann seufzte der Schulleiter.

"Es tut mir leid, Hermine Granger", sagte Dumbledore. Die blauen Augen waren noch trauriger geworden, als sie sprach. "Das ist … bedauerlich, aber ich kann nicht sagen, dass es unerwartet ist. Das ist die Bürde eines Helden, wie Sie sehen."

"Eines Helden?!", sagte Hermine.

Sie blickte nervös zu Professor McGonagall auf und sah, dass sich das Gesicht der Verklärungsprofessorin verkniffen hatte, obwohl ihre Hand immer noch beruhigend auf Hermines Schulter drückte.

"Ja", sagte Dumbledore. "Ich war selbst einmal ein Held, bevor ich ein geheimnisvoller alter Zauberer war, in den Tagen, als ich mich Grindelwald entgegenstellte. Sie haben die Geschichtsbücher gelesen, Miss~Granger?"

Hermine nickte.

"Nun", sagte Dumbledore, "das ist es, was Helden tun müssen, Miss~Granger, sie haben ihre Aufgaben und sie müssen stark werden, um sie zu erfüllen, und das ist es, was Sie sehen, was mit Harry passiert. Wenn es irgendetwas gibt, das getan werden kann, um seinen Weg zu erleichtern, dann werden Sie diejenige sein, die es tut, und nicht ich. Denn ich bin nicht Harrys Freund, leider, sondern nur sein geheimnisvoller alter Zauberer."

"Ich -", sagte Hermine. "Ich bin mir nicht sicher - ich möchte immer noch -" Ihre Stimme brach ab, es schien zu schrecklich, um es laut auszusprechen.

Dumbledore schloss die Augen, und als er sie wieder öffnete, sah er ein wenig älter aus als zuvor.

"Niemand kann Sie aufhalten, Miss~Granger, wenn Sie aufhören wollen, Harrys Freund zu sein. Und was das für ihn bedeuten würde, das wissen Sie vielleicht besser als ich."

"Das - scheint nicht fair zu sein", sagte Hermine, ihre Stimme zitterte. "Dass ich Harrys Freund sein muss, weil er sonst niemanden hat? Das scheint nicht fair zu sein."

"Eine Freundin zu sein ist nichts, wozu man gezwungen werden kann, Miss~Granger." Die blauen Augen schienen direkt durch sie hindurchzuschauen.

"Die Gefühle sind da, oder sie sind nicht da. Wenn sie da sind, können Sie sie akzeptieren oder verleugnen. Sie sind Harrys Freund - und die Entscheidung, das zu verleugnen, würde ihn furchtbar verletzen, vielleicht unheilbar. Aber Miss~Granger, was würde Sie zu solchen Extremen treiben?"

Sie konnte keine Worte finden. Sie war noch nie in der Lage gewesen, Worte zu finden.

"Wenn man Harry zu nahe kommt - man wird verschluckt, und niemand sieht einen mehr, man ist nur noch etwas von ihm, jeder denkt, die ganze Welt dreht sich um ihn und…" Ihr fehlten die Worte.

Der alte Zauberer nickte langsam. "Es ist in der Tat eine ungerechte Welt, in der wir leben, Miss~Granger. Alle Welt weiß jetzt, dass ich es war, der Grindelwald besiegt hat, und nur wenige erinnern sich an Elizabeth Beckett, die gestorben ist, um den Weg zu öffnen, damit ich hindurchgehen konnte. Und doch erinnert man sich an sie. Harry Potter ist der Held dieses Stücks, Miss~Granger; die Welt dreht sich um ihn. Ihm ist Großes bestimmt; und ich glaube, dass man sich mit der Zeit an den Namen Albus Dumbledore als Harry Potters mysteriösen alten Zauberer erinnern wird, mehr als an alles andere, was ich getan habe. Und vielleicht wird man sich an den Namen Hermine Granger als seine Gefährtin erinnern, wenn du dich zu deiner Zeit dessen würdig erweist. Denn das sage ich dir wahrlich: Niemals wirst du allein mehr Ruhm finden, als in Harry Potters Gesellschaft."

Hermine schüttelte rasch den Kopf.

"Aber das ist doch nicht -"

Sie hatte gewusst, dass sie es nicht erklären konnte.

"Es geht nicht um Ruhm, es geht darum - etwas zu sein, das jemand anderem gehört!"

"Du glaubst also, du wärst lieber selbst der Held?"

Der alte Zauberer seufzte.

"Miss~Granger, ich bin ein Held gewesen und ein Anführer; und ich wäre tausendmal glücklicher gewesen, wenn ich zu jemandem wie Harry Potter hätte gehören können. Jemand, der aus härterem Stoff als ich gemacht ist, um die harten Entscheidungen zu treffen, und dennoch würdig ist, mich zu führen. Ich dachte einmal, ich würde so einen Mann kennen, aber ich habe mich geirrt… Miss~Granger, Sie haben überhaupt keine Ahnung, wie glücklich diejenigen sind, die wie Sie sind, im Vergleich zu den Helden."

Das heiße, brennende Gefühl kroch wieder in ihrer Kehle hoch, zusammen mit der Hilflosigkeit, sie verstand nicht, warum Professor McGonagall sie hierher gebracht hatte, wenn der Schulleiter ihr nicht helfen wollte, und einem Blick in Professor McGonagalls Gesicht nach zu urteilen, sah es so aus, als wäre sich Professor McGonagall auch jetzt nicht sicher, ob es eine gute Idee gewesen war.

"Ich will kein Held sein", sagte Hermine Granger, "ich will nicht die Gefährtin eines Helden sein, ich will einfach nur ich sein."

(Ein paar Sekunden später kam ihr der Gedanke, dass sie vielleicht tatsächlich eine Heldin sein wollte, aber sie beschloss, ihre Aussage nicht zu ändern.)

"Ah", sagte der alte Zauberer. "Das ist eine große Aufgabe, Miss~Granger."

Dumbledore erhob sich von seinem Thron, trat hinter seinem Schreibtisch hervor und zeigte auf ein Symbol an der Wand, das so allgegenwärtig war, dass Hermines Augen es glatt übersehen hatten; ein verblasstes Schild, auf dem das Wappen von Hogwarts eingraviert war, der Löwe und die Schlange und der Dachs und der Rabe, und in Latein eingravierte Worte, deren Sinn sie nie verstanden hatte.

"Ein Hufflepuff würde sagen", sagte Dumbledore, tippte mit dem Finger auf den verblichenen Dachs und ließ Hermine wegen des Sakrilegs zusammenzucken (falls es das Original war), "dass die Leute nicht zu dem werden, was sie sein sollen, weil sie zu faul sind, sich die ganze Arbeit zu machen. Ein Ravenclaw", er klopfte auf den Raben, "würde die Worte wiederholen, von denen die Weisen wissen, dass sie viel älter sind als Sokrates: \emph{Erkenne dich selbst}, und sagen, dass die Menschen durch Unwissenheit und Gedankenlosigkeit nicht das werden, was sie werden sollen. Und Salazar Slytherin", Dumbledore runzelte die Stirn, während er mit dem Finger auf die verblichene Schlange tippte, "er sagte, dass wir zu dem werden, was wir sein sollen, indem wir unseren Wünschen folgen, wohin sie auch führen. Vielleicht würde er sagen, dass Menschen daran scheitern, sie selbst zu werden, weil sie sich weigern, das zu tun, was notwendig ist, um ihre Ambitionen zu erreichen. Aber dann stellt man fest, dass fast alle dunklen Zauberer, die aus Hogwarts kamen, Slytherins waren. Sind sie zu dem geworden, was sie werden sollten? Ich glaube nicht." Dumbledores Finger tippte auf den Löwen, dann wandte er sich ihr zu. "Sagen Sie mir, Miss~Granger, was würde ein Gryffindor sagen? Ich brauche nicht zu fragen, ob der Sprechende Hut Ihnen dieses Haus angeboten hat."

Es schien keine schwere Frage zu sein.

"Ein Gryffindor würde sagen, dass die Leute nicht zu dem werden, was sie sein sollten, weil sie Angst haben."

"Die meisten Menschen haben Angst, Miss~Granger", sagte der alte Zauberer. "Sie leben ihr ganzes Leben eingekreist von lähmender Angst, die alles abschneidet, was sie erreichen könnten, was sie werden könnten. Angst, etwas Falsches zu sagen oder zu tun, Angst, ihren bloßen Besitz zu verlieren, Angst vor dem Tod und vor allem die Angst davor, was andere Menschen von ihnen denken werden. Solche Angst ist etwas sehr Schreckliches, Miss~Granger, und es ist furchtbar wichtig, das zu wissen. Aber es ist nicht das, was Godric Gryffindor gesagt hätte. Menschen werden zu dem, was sie sein sollen, Miss~Granger, indem sie tun, was richtig ist." Die Stimme des alten Zauberers war sanft. "Also sagen Sie mir, Miss~Granger, was erscheint Ihnen als die richtige Wahl? Denn das ist es, was Sie wirklich sind, und wohin dieser Weg auch führt, das ist es, was Sie werden sollen."

Es gab eine lange Lücke, gefüllt mit den Geräuschen von Dingen, die nicht gezählt werden konnten. Sie dachte darüber nach, denn sie war eine Ravenclaw.

"Ich glaube nicht, dass es richtig ist", sagte Hermine langsam, "dass jemand so im Schatten eines anderen leben muss …"

"Viele Dinge auf der Welt sind nicht richtig", sagte der alte Zauberer, "die Frage ist, was man dagegen tun sollte. Hermine Granger, ich werde weniger subtil sein, als es für einen geheimnisvollen alten Zauberer üblich ist, und dir ganz offen sagen, dass du dir nicht vorstellen kannst, wie schlimm es werden könnte, wenn sich die Ereignisse um Harry Potter zum Schlechten wenden. Seine Suche ist eine Angelegenheit, vor der du nicht einmal im Traum davonlaufen dürftest, wenn du sie kennen würdest."

"Welche Aufgabe?", fragte Hermine. Ihre Stimme zitterte, denn es war ganz klar, nach welcher Antwort der Schulleiter suchte, und sie wollte sie nicht geben. "Was ist damals mit Harry passiert, warum hatte er Fawkes auf der Schulter?"

"Er ist erwachsen geworden", sagte der alte Zauberer.

Seine Augen blinzelten mehrmals unter der Halbmondbrille, und sein Gesicht sah plötzlich sehr faltig aus.

"Sehen Sie, Miss~Granger, Menschen werden nicht mit der Zeit erwachsen, Menschen werden erwachsen, wenn sie in erwachsene Situationen gebracht werden. Genau das ist an jenem Samstag mit Harry Potter passiert. Ihm wurde gesagt - Sie dürfen diese Information mit niemandem teilen, verstehen Sie - ihm wurde gesagt, dass er gegen jemanden kämpfen müsse. Ich kann Ihnen nicht sagen, gegen wen. Ich kann Ihnen nicht sagen, warum. Aber das ist es, was ihm passiert ist, und deshalb braucht er seine Freunde."

Es gab eine Pause.

"Bellatrix Black?!" sagte Hermine. Sie hätte nicht schockierter sein können, wenn jemand ein elektrisches Kabel in ihr Ohr gesteckt hätte. "Du willst Harry gegen Bellatrix Black? kämpfen lassen?"

"Nein", sagte der alte Zauberer. "Nicht gegen sie. Ich kann dir nicht sagen, wer oder warum."

Sie dachte noch einmal darüber nach.

"Gibt es eine Möglichkeit, dass ich mit Harry mithalten kann?", fragte Hermine.

"Ich meine, ich sage nicht, dass ich das tun werde, aber - wenn er Freunde braucht, können wir dann gleichwertige Freunde sein? Kann ich auch ein Held sein?"

"Ah", sagte der alte Zauberer und lächelte.

"Das können nur Sie entscheiden, Miss~Granger."

"Aber Sie werden mir nicht so helfen, wie Sie Harry helfen."

Der alte Zauberer schüttelte den Kopf. "Ich habe ihm wenig genug geholfen, Miss~Granger. Und wenn Sie mich um eine Aufgabe bitten -"

Der alte Zauberer lächelte wieder, eher schief.

"Miss~Granger, Sie sind in Ihrem ersten Jahr in Hogwarts. Seien Sie nicht zu eifrig, erwachsen zu werden; dafür wird später noch genug Zeit sein."

"Ich bin zwölf. Harry ist elf."

"Harry Potter ist etwas Besonderes", sagte der alte Zauberer. "Das wissen Sie doch, Miss~Granger."

Die blauen Augen waren plötzlich durchdringend unter der Halbmondbrille, und sie fühlte sich an den Tag des Dementors erinnert, als Dumbledores Stimme in ihrem Kopf gesagt hatte, dass er von Harrys dunkler Seite wusste.

Hermine hob ihre Hand und berührte Professor McGonagalls Hand, die die ganze Zeit fest auf ihrer Schulter gelegen hatte, und Hermine sagte, sie war überrascht, dass ihre Stimme nicht brach:

"Ich würde jetzt gerne gehen, bitte."

"Natürlich", sagte Professor McGonagall, und Hermine spürte die Hand auf ihrer Schulter, die sie sanft umdrehte, um sich der Eichentür zuzuwenden.

"Hast du dich schon für deinen Weg entschieden, Hermine Granger?", sagte Albus Dumbledores Stimme hinter ihr, selbst als sich die Tür langsam knarrend öffnete und den Blick auf die Verzauberung der Endlosen Treppe freigab.

Sie nickte.

"Und?"

"Ich werde", sagte sie, ihre Stimme stockte, "ich werde, ich werde -"

Sie schluckte.

"Ich werde tun - was richtig ist -"

Sie sagte nichts mehr, sie konnte es nicht, und dann begann sich die Endlose Treppe wieder um sie zu drehen. Weder sie noch Professor McGonagall sprachen auf dem Weg nach unten. Als die Wasserspeier aus dem Fließenden Stein aus dem Weg traten und die beiden auf die Gänge von Hogwarts hinaustraten, sprach Professor McGonagall endlich, und sie sagte flüsternd:

"Es tut mir so schrecklich leid, Miss~Granger. Ich hätte nicht gedacht, dass der Schulleiter solche Dinge zu Ihnen sagen würde. Ich glaube, er hat wirklich vergessen, wie es ist, ein Kind zu sein."

Hermine blickte wieder zu ihr auf und sah, dass Professor McGonagall aussah, als würde sie gleich in Tränen ausbrechen … nur nicht wirklich, aber es war eine Anspannung in ihrem Gesicht, die so aussah.

"Wenn ich auch ein Held sein will", sagte Hermine, "gibt es irgendetwas, was Sie tun können, um zu helfen?"

Professor McGonagall schüttelte schnell den Kopf und sagte: "Miss~Granger, ich bin mir nicht sicher, ob der Schulleiter damit falsch liegt. Du bist erst zwölf."

"Okay", sagte Hermine. Sie gingen ein Stück vorwärts. "Entschuldigung", sagte Hermine, "ist es okay, wenn ich alleine zum Ravenclaw-Turm zurücklaufe? Es tut mir leid, es ist nicht Ihre Schuld oder so, ich möchte im Moment einfach nur allein sein."

"Natürlich, Miss~Granger", sagte Professor McGonagall, ihre Stimme klang ein wenig heiser, und Hermine hörte, wie ihre Schritte aufhörten und sie sich dann hinter ihr umdrehte.

Hermine Granger ging weg.

Sie stieg eine Treppe hinauf und dann noch eine, wobei sie sich fragte, ob es noch jemanden in Hogwarts gab, der ihr eine Chance geben würde, eine Heldin zu sein.

Professor Flitwick würde dasselbe sagen wie Professor McGonagall, und selbst wenn er es nicht tat, konnte er wahrscheinlich nicht helfen, Hermine wusste nicht, wer helfen könnte.

Nun, Professor Quirrell würde etwas Gescheites einfallen, wenn sie genug Quirrell-Punkte verbrauchte, aber sie hatte das Gefühl, dass es eine schlechte Idee wäre, ihn zu fragen - dass der Verteidigungsprofessor niemandem helfen konnte, die Art von Held zu werden, die es wert war, zu werden, und dass er nicht einmal den Unterschied verstehen würde.

\emph{Sie hatte den Ravenclaw-Turm schon fast erreicht, als sie das Aufblitzen von Gold sah.}

