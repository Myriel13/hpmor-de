

\hypertarget{selbstverwirklichung-teil-5}{% \section{70. Selbstverwirklichung, Teil 5}\label{selbstverwirklichung-teil-5}}

\textbf{\uline{Selbstverwirklichung, Teil 5}}

Selbst wenn man drei Jahrzehnte lang stellvertretende Schulleiterin und davor Professor für Verwandlung gewesen war, sah man Albus Dumbledore nur selten völlig unvorbereitet.

"… Susan Bones, Lavender Brown und Daphne Greengrass", beendete Minerva. "Ich sollte auch anmerken, Albus, dass Miss Grangers Bericht über Ihre scheinbar nicht unterstützende Haltung - ich glaube, ihr Ausdruck war '\emph{er sagte, ich solle froh sein, nur ein Handlanger zu sein'} - eine Menge Interesse bei den älteren Mädchen geweckt hat. Mehrere von ihnen kamen zu mir, um zu fragen, ob Miss Grangers Anschuldigungen wahr seien, da sie gesagt hatte, dass ich dabei war."

Der alte Zauberer lehnte sich in seinem riesigen Stuhl zurück und starrte sie immer noch an, wobei seine Augen unter der Halbmondbrille etwas abstrakt wirkten.

"Das hat mich in eine Art Dilemma gebracht, Albus", sagte Professor McGonagall. Ihr Gesicht blieb ganz neutral, dafür sorgte sie. "Ich weiß jetzt, dass Sie nicht wirklich die Absicht hatten, das Mädchen zu entmutigen. Ganz im Gegenteil. Aber Sie und Severus haben mir oft gesagt, dass ich, um ein Geheimnis zu wahren, kein Zeichen geben darf, das sich von der Reaktion eines wirklich Unwissenden unterscheidet. So blieb mir nichts anderes übrig, als zu bestätigen, dass Miss Grangers Schilderung korrekt war, und den angemessenen Grad von Besorgnis vorzutäuschen, mit einem leichten Unterton von Beleidigung. Hätte ich nämlich nicht gewusst, dass Sie Miss Granger absichtlich manipulieren, wäre ich ziemlich verärgert gewesen."

"Ich … verstehe", sagte der alte Zauberer langsam. Seine Hände spielten abwesend mit seinem silbernen Bart, kleine schnelle Gesten.

"Zum Glück", fuhr Professor McGonagall fort, "sind die Professoren Sinistra und Vector bisher die einzigen beiden Fakultätsmitglieder, die Miss Grangers Knöpfe tragen."

"Knöpfe?", wiederholte der alte Zauberer. Minerva zog eine kleine silberne Scheibe mit den Initialen S.P.H.E.W. hervor, legte sie auf Albus' Schreibtisch und tippte sie kurz mit dem Finger an. Und die Stimmen von Hermine Granger, Padma Patil, Parvati Patil, Lavender Brown, Susan Bones, Hannah Abbott, Daphne Greengrass und Tracey Davis riefen unisono: "\emph{Wir sind nicht mit dem Zweitbesten zufrieden im Leben, es ist Zeit, einer Hexe eine Quest zu geben}!"

\emph{('We won't settle for second best, it's time to give a witch a quest!', Anm. des Übersetzers)}

"Miss Granger verkauft sie für zwei Sickles und erzählt mir, dass sie bisher fünfzig davon verkauft hat. Ich glaube, dass Nymphadora Tonks aus dem siebten Jahr Hufflepuff sie für sie verzaubert. Um meinen Bericht abzuschließen", sagte Professor McGonagall zügig, "unsere acht frisch gebackenen Heldinnen haben um Erlaubnis gebeten, vor dem Eingang Ihres Büros eine Protestaktion durchzuführen."

"Ich hoffe", sagte Albus und runzelte die Stirn, "Du hast ihnen erklärt, dass -"

"Ich habe ihnen erklärt, dass Mittwoch um 19 Uhr in Ordnung wäre", sagte Minerva. Sie nahm den Knopf vom Schreibtisch des Schulleiters zurück, schenkte Albus ein honigsüßes Lächeln und wandte sich der Tür zu.

"Minerva?", fragte der alte Zauberer hinter ihr. "Minerva?!"

Die Eichentür schloss sich fest hinter ihr.

…

Zwischen den kurzen Steinwänden, die den Vorraum zum Büro des Schulleiters abgrenzten, war nicht viel Platz, und obwohl viele Leute dem Protest hatten beiwohnen wollen, hatten nicht viele kommen dürfen. Nur Professor Sinistra und Professor Vector, die die Knöpfe trugen, und die Vertrauensschüler Penelope Clearwater und Rose Brown und Jacqueline Preece, die auch Knöpfe trugen. Dahinter Professor McGonagall und Professor Sprout und Professor Flitwick, die die Knöpfe nicht trugen und die ganze Angelegenheit begutachteten. Harry Potter und der Schulleiter von Hogwarts waren da, und die Jungen Präfekten Vertrauensschüler Weasley und Oliver Beatson, die alle die Knöpfe trugen, um Solidarität zu zeigen. Und natürlich die acht Gründungsmitglieder von S.P.H.E.W., die mit ihren Schildern eine Streikpostenkette neben den Wasserspeiern bildeten. Auf Hermines eigenem Schild, das an einem massiven Holzgriff befestigt war, der in ihren Händen immer schwerer zu werden schien, je mehr Sekunden vergingen, stand \textbf{KEIN HANDLANGER}. Und Professor Quirrell, der mit dem Rücken an der entfernten Steinwand lehnte und mit unleserlichen Augen zusah. Der Verteidigungsprofessor hatte einen ihrer Knöpfe bekommen, obwohl sie ihm nie einen verkauft hatte; und er trug ihn nicht, sondern warf ihn müßig mit einer Hand hin und her.

Diese ganze Idee war ihr vor vier Tagen viel besser vorgekommen, als die Feuer ihrer Empörung noch frisch und heiß gebrannt hatten und sie mit der Aussicht konfrontiert gewesen war, das alles vier Tage später zu tun, statt jetzt. Aber sie musste weitermachen, denn das war es, was Helden taten, sie machten weiter, und auch, weil es unendlich zu schrecklich erschienen war, allen zu sagen, dass sie es abbrach. Hermine fragte sich, wie viel Heldentum aus solchen Gründen weitergegangen war. In den meisten Büchern stand nicht "\emph{Und dann weigerten sie sich, aufzugeben, egal wie vernünftig es gewesen wäre, denn das wäre zu peinlich gewesen}"; aber ein großer Teil der Geschichte machte auf diese Weise viel mehr Sinn.

Um 19:15 Uhr, hatte Professor McGonagall ihr gesagt, würde Schulleiter Dumbledore herunterkommen und ein paar Minuten mit ihnen sprechen. Professor McGonagall hatte gesagt, dass sie keine Angst haben sollten - der Schulleiter war tief im Inneren ein guter Mensch, und sie hatten ordnungsgemäß die Genehmigung der Schule für den Protest erhalten. Aber Hermine war sich sehr bewusst, dass sie, auch wenn sie es mit unterschriebener Erlaubnis tat, immer noch der Autorität trotzen würde.

Nachdem sie beschlossen hatte, eine Heldin zu sein, hatte Hermine das Naheliegendste getan und war in die Hogwarts-Bibliothek gegangen, um sich Bücher darüber zu holen, wie man eine Heldin ist. Dann hatte sie die Bücher wieder in die Regale zurückgestellt, weil es offensichtlich war, dass keiner der Autoren selbst ein Held gewesen war. Stattdessen hatte sie einfach fünfmal hintereinander, bis sie jedes Wort auswendig kannte, die dreißig Zoll von Godric Gryffindor gelesen, die allesamt seine Autobiografie und seine Lebensratschläge waren. (Oder zumindest die englische Übersetzung; Latein konnte sie noch nicht lesen.) Godric Gryffindors Autobiografie war viel komprimierter als die Bücher, die Hermine zu lesen gewohnt war, er benutzte einen Satz, um Dinge zu sagen, die allein schon dreißig Zoll hätten einnehmen müssen, und dann kam noch ein weiterer Satz danach… Aber nach dem, was sie gelesen hatte, war klar, dass es zwar nicht darum ging, der Autorität zu trotzen, aber man konnte kein Held sein, wenn man zu viel Angst hatte, es zu tun. Und Hermine Granger wusste inzwischen, wie andere sie sahen, und sie wusste, was andere Leute dachten, dass sie nicht tun konnte.

Hermine hievte ihr Streikpostenschild ein wenig höher und konzentrierte sich darauf, langsam und rhythmisch zu atmen, anstatt zu hyperventilieren, bis sie umfiel.

"Wirklich?", sagte Miss Preece in einem Ton der unverhohlenen Faszination. "Sie durften nicht wählen?"

"In der Tat", sagte Professor Sinistra. (Das Haar der Astronomie-Professorin war immer noch dunkel und ihr dunkles Gesicht nur leicht faltig; Hermine hätte ihr Alter auf etwa siebzig geschätzt, außer -) "Ich kann mich noch gut an den Jubel meiner Mutter erinnern, als das Gesetz über die Gleichberechtigung von Frauen verkündet wurde, obwohl sie sich eigentlich nicht dafür interessiert hat." (Was bedeutete, dass Professor Sinistra 1918 in der Nähe ihrer Muggelfamilie gewesen war.) "Und das war nicht das Schlimmste. Nur ein paar Jahrhunderte früher -"

Dreißig Sekunden später starrten alle Nicht-Muggelgeborenen, sowohl männliche als auch weibliche, Professor Sinistra mit völlig schockierten Mienen an. Hannah hatte ihr Schild fallen gelassen.

"Und das war auch nicht das Schlimmste, bei weitem nicht", schloss Professor Sinistra. "Aber Sie sehen, wohin so etwas potenziell führen kann."

"Merlin bewahre uns", sagte Penelope Clearwater mit erstickter Stimme. "Sie meinen, so würden uns die Menschen behandeln, wenn wir keine Zauberstäbe hätten, um uns zu verteidigen?"

"Hey!", sagte einer der jungen Vertrauensschüler. "Das ist nicht -"

Ein kurzes, sardonisches Lachen ertönte aus der Richtung von Professor Quirrell. Als Hermine den Kopf drehte, um nachzusehen, sah sie, dass der Verteidigungsprofessor immer noch untätig mit dem Knopf herumspielte, ohne sich die Mühe zu machen, zu den anderen aufzublicken, während er sagte: "So ist die menschliche Natur, Miss Clearwater. Seien Sie versichert, dass Sie nicht freundlicher wären, wenn Hexen Zauberstäbe hätten und Männer nicht."

"Das glaube ich kaum!", schnauzte Professor Sinistra.

Ein kaltes Glucksen. "Ich vermute, dass es in den stolzesten Reinblüterfamilien öfter vorkommt, als man zu glauben wagt. Irgendeine einsame Hexe erspäht einen gutaussehenden Muggel; und denkt, wie leicht es wäre, dem Mann einen Liebestrank unterzuschieben, und von ihm allein und vollkommen angebetet zu werden. Und da sie weiß, dass er ihr keinen Widerstand leisten kann, ist es nur natürlich, dass sie sich von ihm nimmt, was immer sie will -"

"Professor Quirrell!", sagte Professor McGonagall scharf.

"Es tut mir leid", sagte Professor Quirrell milde, seine Augen immer noch auf den Knopf in seiner Hand gerichtet, "tun wir alle immer noch so, als ob es passieren würde? Dann entschuldige ich mich."

Professor Sinistra schnappte zurück: "Und ich nehme an, dass Zauberer nicht -"

"Es sind Kinder anwesend, Professoren!" Wieder Professor McGonagall.

"Manche schon", sagte Professor Quirrell gleichmütig, als würde er über das Wetter diskutieren. "Obwohl ich persönlich es nicht tue."

Eine Zeit lang herrschte Schweigen. Hermine hob ihr Schild wieder hoch - es war ihr auf die Schulter gerutscht, während sie zuhörte. Daran hatte sie nie gedacht, nicht einmal ein bisschen, und jetzt versuchte sie, nicht daran zu denken, und ihr Magen fühlte sich etwas mulmig an. Sie schaute in Harry Potters Richtung, ohne recht zu wissen, warum sie es tat; und sie sah, dass Harrys Gesicht vollkommen ruhig war. Ein Schauer lief ihr über den Rücken, bevor sie wegschaute, nicht schnell genug, um das kleine Nicken zu verpassen, das Harry ihr schenkte, als ob sie sich über etwas einig wären.

"Um fair zu sein", sagte Professor Sinistra nach einer Weile, "seit ich meinen Hogwarts-Brief erhalten habe, kann ich mich nicht daran erinnern, auf irgendwelche Vorurteile gestoßen zu sein, weil ich eine Frau bin, oder farbig. Nein, jetzt sind es Vorurteile, weil ich ein Muggelgeborener bin. Ich glaube, Miss Granger hat gesagt, dass sie bisher nur bei Helden ein Problem gefunden hat."

Hermine brauchte einen Moment, um zu erkennen, dass ihr die Frage gestellt worden war, und dann sagte sie "Ja", in einem Ton, der ein wenig quietschte.

Diese ganze Sache hatte sich ein bisschen mehr aufgebläht, als sie sich vorgestellt hatte, als sie damit angefangen hatte.

"Was genau haben Sie überprüft, Miss Granger?", sagte Professor Vector. Sie sah älter aus als Professor Sinistra, ihr Haar begann, ein wenig zu ergrauen; Hermine war Professor Vector noch nie persönlich begegnet, bis der Arithmetikprofessor sie um einen Knopf gebeten hatte.

"Ähm", sagte Hermine, ihre Stimme etwas hoch, "ich habe in den Geschichtsbüchern nachgeschaut und es hat genauso viele weibliche Zaubereiminister gegeben wie männliche. Dann habe ich mir die Obersten Mugwumps angesehen und da gab es ein paar mehr Zauberer als Hexen, aber nicht viele. Aber wenn man sich Leute wie berühmte Jäger von dunklen Zauberern ansieht, oder Leute, die Invasionen von dunklen Kreaturen aufgehalten haben, oder Leute, die dunkle Lords gestürzt haben -"

"Und natürlich die Dunklen Zauberer selbst", sagte Professor Quirrell. Jetzt hatte der Verteidigungsprofessor aufgeschaut. "Sie können das zu Ihrer Liste hinzufügen, Miss Granger. Von allen mutmaßlichen Todessern kennen wir nur zwei Hexen, Bellatrix Black und Alecto Carrow. Und ich wage zu behaupten, dass die meisten Zauberer sich schwer tun würden, eine einzige Dunkle Lady außer Baba Yaga zu nennen."

Hermine starrte ihn nur an. Das konnte nicht sein -

"Professor Quirrell", sagte Professor Vector, "was genau wollen Sie damit andeuten?"

Der Verteidigungsprofessor hob den Knopf an, so dass der goldene Schriftzug S.P.H.E.W. ihnen gegenüberstand, und sagte: "Helden", dann drehte er den Knopf so, dass er seine silberne Rückseite zeigte, und sagte: "Dunkle Lords. Es sind ähnliche Karrierewege, die von ähnlichen Menschen beschritten werden, und man kann sich kaum fragen, warum sich junge Hexen von dem einen Weg abwenden, ohne dessen Spiegelbild zu betrachten."

"Oh, jetzt verstehe ich!", sagte Tracey Davis und meldete sich so plötzlich zu Wort, dass Hermine einen kleinen Schreck bekam. "Du schließt dich unserem Protest an, weil du dir Sorgen machst, dass nicht genug Mädchen zu dunklen Hexen werden!" Dann kicherte Tracey, was Hermine zu diesem Zeitpunkt nicht geschafft hätte, wenn man ihr eine Million Pfund Sterling bezahlt hätte.

Es lag ein halbes Lächeln auf Professor Quirrells Gesicht, als er antwortete: "Nicht wirklich, Miss Davis. In Wahrheit interessiert mich diese Art von Dingen nicht im Geringsten. Aber es ist müßig, die Hexen zu den Zaubereiministern und anderen gewöhnlichen Leuten zu zählen, die ein gewöhnliches Leben führen, wo doch Grindelwald und Dumbledore und Er-der-nicht-genannt-werden-muss alle Männer waren." Die Finger des Verteidigungsprofessors drehten müßig an dem Knopf, drehten ihn immer wieder um. "Andererseits machen nur die wenigsten Leute etwas Interessantes aus ihrem Leben. Was macht es für Sie aus, ob sie überwiegend Hexen oder überwiegend Zauberer sind, solange Sie nicht zu ihnen gehören? Und ich vermute, Sie werden nicht zu den Interessanten gehören, Miss Davis; denn obwohl Sie ehrgeizig sind, haben Sie keinen Ehrgeiz."

"Das ist nicht wahr!", sagte Tracey entrüstet. "Und was soll das überhaupt heißen?"

Professor Quirrell richtete sich von dort auf, wo er sich an die Wand gelehnt hatte. "Sie wurden nach Slytherin sortiert, Miss Davis, und ich erwarte, dass Sie jede Gelegenheit zum Aufstieg ergreifen werden, die Ihnen in die Hände fällt. Aber es gibt keinen großen Ehrgeiz, der Sie antreibt, und Sie werden Ihre Gelegenheiten nicht wahrnehmen. Bestenfalls werden Sie sich bis zum Zaubereiminister oder einer anderen hohen, unbedeutenden Position empor hangeln, ohne jemals die Grenzen Ihrer Existenz zu überschreiten." Dann wandte sich Professor Quirrells Blick von Tracey ab, er sah sie an, die blassblauen Augen starrten sie mit einer furchtbaren Intensität an - "Sagen Sie mir, Miss Granger. Haben Sie eine Ambition?"

"Professor -", quietschte die hohe, strenge Stimme von Professor Flitwick, und dann brach die Stimme ihres Hausoberhauptes ab, und aus dem Augenwinkel sah Hermine, dass Harry seine Hand auf Professor Flitwicks Schulter gelegt hatte und den Kopf schüttelte, sein Gesicht sah sehr erwachsen aus. Hermine fühlte sich wie ein Reh, das im Scheinwerferlicht steht.

"Was hat Sie dazu getrieben, Ihre Grenzen zu überschreiten, Miss Granger?", fragte der Verteidigungsprofessor und blickte sie immer noch direkt an. "Warum reicht es nicht mehr aus, gute Noten im Unterricht zu bekommen? Ist es wahre Größe, die Sie suchen? Sind Sie mit irgendeinem Aspekt der Welt unzufrieden, den Sie nach Ihrem Willen umgestalten müssen? Oder ist das alles nur ein Kinderspiel für Sie? Ich wäre sehr enttäuscht, wenn es nur darum ginge, Harry Potter Konkurrenz zu machen."

"Ich -", sagte Hermine, ihre Stimme war so hoch, dass sie eine Art Piepsen von sich gab, aber dann fiel ihr nicht ein, was sie noch sagen sollte.

"Sie können sich einen Moment Zeit zum Nachdenken nehmen, wenn Sie möchten", sagte Professor Quirrell. "Stellen Sie sich vor, es ist eine Hausaufgabe, die am Donnerstag fällig ist. Wie ich höre, sind Sie darin recht eloquent."

Die Gargoyles traten zur Seite, der Fließende Stein rumpelte wie Fels, während er sich wie Fleisch bewegte. Die riesigen hässlichen Gestalten warteten nur kurz, totengraue Augen starrten in stummer Wachsamkeit hinaus. Dann klappten die großen Wasserspeier ihre Flügel wieder ein und traten in ihre früheren Positionen zurück. Der Fließende Stein veränderte sein Äußeres überhaupt nicht, als er von der Beweglichkeit in die Bewegungslosigkeit zurückkehrte, und die kurze Lücke im Stein von Hogwarts war wieder fest. Und vor ihnen allen, in leuchtend violetten Roben, die wahrscheinlich nur für Muggelgeborene scheußlich aussahen, stand die hoch aufragende Gestalt von Albus Percival Wulfric Brian Dumbledore, dem Schulleiter von Hogwarts, dem Obersten Hexenmeister des Zaubergamot, der Oberste Mugwump der Internationalen Konföderation der Zauberer, der Bezwinger des Dunklen Lords Grindelwald und Beschützer Großbritanniens, der Wiederentdecker der sagenumwobenen Zwölf Gebräuche des Drachenblutes, der mächtigste lebende Zauberer; und er sah sie an, Hermine Jean Granger, Generalin des kürzlich erweiterten Sonnenscheinregiments, die im ersten Jahr der Hogwarts-Klassen die besten Noten bekam und sich selbst zur Heldin erklärt hatte.

\emph{Sogar ihr Name war kürzer als seiner.}

Der Schulleiter lächelte sie wohlwollend an, seine faltigen Augen funkelten fröhlich unter ihren Halbkreisen aus Glas, und sagte: "Hallo, Miss Granger."

Das Seltsame war, dass es nicht annähernd so beängstigend war wie das Gespräch mit Professor Quirrell. "Hallo, Schulleiter Dumbledore", sagte Hermine mit nur einem leichten Zittern in der Stimme.

"Miss Granger", sagte Dumbledore, nun mit ernsterem Blick, "ich glaube, wir beide haben ein kleines Missverständnis gehabt. Ich wollte nicht andeuten, dass Sie kein Held sein können oder sollten. Ich wollte auch nicht andeuten, dass Hexen im Allgemeinen keine Helden sein sollten. Nur, dass du… ein bisschen jung bist, um an solche Dinge zu denken."

Hermine, die sich nicht helfen konnte, blickte zu Professor McGonagall und sah, dass Professor McGonagall ihr ein aufmunterndes Lächeln schenkte - oder sie schenkte den beiden jedenfalls eine Art Lächeln -, also blickte Hermine wieder zum Schulleiter und sagte, das kleine Zittern in ihrer Stimme nun etwas größer: "Seit Sie vor vierzig Jahren Schulleiter wurden, gab es elf Schüler, die in Hogwarts ihren Abschluss gemacht haben und zu Helden wurden, ich meine Leute wie Lupe Cazaril und so weiter, und zehn davon waren Jungen. Cimorene Linderwall war die einzige Hexe."

"Hm", sagte der Schulleiter. Es lag ein nachdenklicher Ausdruck auf seinem Gesicht; er schien zumindest darüber nachzudenken. "Miss Granger, ich war noch nie einer, der solche Zahlen ausrechnet. Oft ist es viel zu einfach, zu zählen, als zu verstehen. Aus Hogwarts sind viele gute Menschen hervorgegangen, sowohl Hexen als auch Zauberer; die als Helden Berühmten sind nur eine Art von guten Menschen, und vielleicht nicht die höchste. Sie haben Alice Longbottom oder Lily Potter nicht in Ihre Rechnung einbezogen… Aber lassen wir das beiseite. Sagen Sie mir, Miss Granger, haben Sie gezählt, wie viele Helden in den 40 Jahren vor mir aus Hogwarts kamen? Denn in dieser Zeit kann ich mich nur an drei erinnern, die heute als Helden bezeichnet werden; und unter diesen drei waren überhaupt keine Hexen."

"Ich will damit nicht sagen, dass es nur an Ihnen liegt!" sagte Hermine. "Ich denke nur, dass viele Leute, wie auch die Schulleiter vor Ihnen, vielleicht sogar deine ganze Gesellschaft und alles, Mädchen entmutigen."

Der alte Zauberer seufzte. Seine halbgläsernen Augen sahen nur sie an, als wären sie die einzigen beiden anwesenden Menschen. "Miss Granger, vielleicht kann man Hexen davon abhalten, Zauberlehrerin zu werden, oder Quidditch-Spielerin oder sogar Aurorin. Aber keine Helden. Wenn jemand dazu bestimmt ist, ein Held zu sein, dann wird er ein Held sein. Sie werden durch Feuer gehen und durch Eis schwimmen. Dementoren werden sie nicht aufhalten, auch nicht der Tod von Freunden, und auch nicht Entmutigung."

"Nun", sagte Hermine und hielt inne, um mit den Worten zu ringen. "Nun, ich meine … was ist, wenn das eigentlich nicht stimmt? Ich meine, mir scheint, wenn man will, dass mehr Hexen Helden werden, sollte man ihnen das Heldentum beibringen."

"Viele Jungen und Mädchen sind in ihren Träumen Helden", sagte Dumbledore leise. Er sah keines der anderen Mädchen an, nur sie. "In der wachen Welt sind es noch weniger. Viele haben sich behauptet und sich der Dunkelheit gestellt, als diese sie holen kam. Wenige kommen wegen der Dunkelheit und zwingen sie, sich ihnen zu stellen. Es ist ein hartes Leben, manchmal einsam, oft kurz. Ich habe niemandem gesagt, er solle sich dieser Berufung verweigern, aber ich würde auch nicht wünschen, ihre Zahl zu erhöhen."

Hermine zögerte; da war etwas in dem gezeichneten Gesicht, das sie aufhielt, wie ein Hinweis auf all die Emotionen, die nicht gezeigt wurden, Jahre und Jahre davon… \emph{Wenn es mehr Helden gäbe, wäre ihr Leben vielleicht nicht so einsam, oder so kurz.} Sie brachte es nicht über sich, das zu sagen, nicht zu ihm.

"Aber der Punkt ist strittig", sagte der alte Zauberer. Er lächelte, ein bisschen reumütig, wie sie fand. "Miss Granger, man kann Heldentum nicht lehren, wie man Zauberei lehren würde. Sie können nicht zwölf Zoll Pergament darüber schreiben, wie man weitermacht, wenn alle Hoffnung verloren scheint. Man kann Schülern nicht beibringen, wann man aufsteht und dem Schulleiter sagt, dass er Unrecht getan hat. Helden werden geboren, nicht gelehrt. Und aus welchem Grund auch immer, werden mehr von ihnen als Jungen geboren als als Mädchen." Der Schulleiter zuckte mit den Schultern, als wollte er sagen, dass er nichts dagegen tun könne.

"Ähm", sagte Hermine. Sie konnte nicht anders, sie blickte hinter sich. Professor Sinistra sah ein wenig entrüstet aus. Und es stimmte nicht, dass alle sie anstarrten, als wäre sie gerade dumm gewesen, so wie sie es sich vorzustellen begann, während sie Dumbledore zuhörte. Hermine drehte sich wieder zu Dumbledore um, holte tief Luft und sagte: "Nun, vielleicht werden Leute, die Helden werden wollen, Helden sein, egal was passiert. Aber ich wüsste nicht, wie man das wirklich wissen könnte, abgesehen davon, dass man es hinterher einfach sagt. Und als ich Ihnen gesagt habe, dass ich ein Held werden will, waren Sie nicht sehr ermutigend."

"Mr. Potter", sagte der Schulleiter milde. Seine Augen verließen ihre nicht. "Bitte schildern Sie Miss Granger Ihren Eindruck von unserer eigenen ersten Begegnung. Würden Sie sagen, dass ich ermutigend war? Sprechen Sie die Wahrheit."

Es gab eine Pause.

"Mr. Potter?", sagte Professor Vectors Stimme von hinter ihr und klang verwirrt.

"Ähm", sagte Harrys Stimme von weiter hinten und klang äußerst zögernd. "Ähm … nun, in meinem Fall hat der Schulleiter ein Huhn in Brand gesetzt."

"Er hat was?!" platzte Hermine heraus, nur waren da mehrere andere Leute, die ungefähr zur gleichen Zeit etwas ausriefen, sodass sie nicht sicher war, dass jemand sie hörte.

Dumbledore starrte sie weiter an und sah vollkommen ernst aus.

"Ich wusste nichts von Fawkes", sagte Harrys Stimme schnell, "also erzählte er mir, dass Fawkes ein Phönix sei, während er auf ein Huhn auf Fawkes' Ständer zeigte, damit ich dachte, das sei Fawkes, und dann zündete er das Huhn an - und außerdem gab er mir diesen großen Stein und sagte mir, er habe meinem Vater gehört und ich solle ihn überallhin mitnehmen -"

"Aber das ist doch verrückt!" platzte Susan heraus.

Plötzlich herrschte Stille. Der Schulleiter drehte langsam seinen Kopf und starrte Susan an.

"Ich -", sagte Susan. "Ich meine - ich -"

Der Schulleiter beugte sich hinunter, bis er dem jungen Mädchen direkt gegenüberstand.

"Ich habe nicht -", sagte Susan.

Dumbledore legte einen Finger an seine Lippen und zwirbelte sie, wobei er ein \emph{bweeble-bweeble-bweeble-}Geräusch machte.

Der Schulleiter richtete sich wieder auf und sagte: "Nun, meine guten Heldinnen, es war angenehm, mit euch zu sprechen, aber leider bleibt an diesem Tag noch viel zu tun. Seid jedoch versichert, dass ich für jeden unergründlich bin, nicht nur für Hexen."

Die Gargoyles traten zur Seite, der Fließende Stein rumpelte wie Fels, als er sich bewegte. Die riesigen hässlichen Gestalten warteten kurz mit toten grauen Augen, die in stummer Wachsamkeit hinausstarrten, als Albus Percival Wulfric Brian Dumbledore, so wohlwollend lächelnd wie beim ersten Mal, als er aus seinem Büro auftauchte, zurück in den Zauber der Endlosen Treppe trat. Dann klappten die großen Wasserspeier ihre Flügel wieder ein und traten in ihre früheren Positionen zurück, wobei nur noch ein letztes kurzes "\emph{Bwa-ha-ha!}" ertönte, bevor sich die Lücke schloss.

Es herrschte eine lange Stille.

"Er hat wirklich ein Huhn angezündet?", sagte Hannah.

….

Die acht hatten auch danach noch weiter protestiert, aber um ehrlich zu sein, hatten sie sich nicht mehr getraut. Nach einigen vorsichtigen Fragen von Professor Flitwick war festgestellt worden, dass Harry Potter das brennende Huhn nicht gerochen hatte. Was bedeutete, dass es sich wahrscheinlich um einen Kieselstein oder so etwas handelte, der in ein Huhn verwandelt und dann mit einem Begrenzungszauber eingeschlossen wurde, um sicherzustellen, dass kein Rauch in die Luft entweicht - sowohl Professor Flitwick als auch Professor McGonagall hatten sehr nachdrücklich darauf hingewiesen, dass niemand so etwas ohne ihre Aufsicht versuchen durfte.

Aber trotzdem…

\emph{Aber trotzdem..}. \emph{was}?

Hermine wusste nicht einmal, \emph{aber trotzdem was}.

\emph{Aber trotzdem.}

Nach vielen Blicken, die zwischen den Mädchen ausgetauscht worden waren, von denen keine die Erste sein wollte, die es sagte, hatte Hermine den Protest für beendet erklärt, und die Erwachsenen und die Jungen waren abgedriftet.

"Du denkst doch nicht, dass wir Dumbledore gegenüber unfair waren, oder?", sagte Susan, als die Heldinnen zum Geräusch von acht Paar Füßen, die auf dem Steinpflaster der Korridore von Hogwarts herumtrampelten, weggingen. "Ich meine, wenn er für alle verrückt ist und nicht nur bei Hexen, dann ist das doch keine Diskriminierung, oder?"

"Ich habe keine Lust mehr, gegen den Schulleiter zu protestieren", sagte Hannah schwach. Das Hufflepuff-Mädchen schien ein wenig unsicher auf den Beinen zu sein. "Es ist mir egal, was Professor McGonagall sagt, dass er es uns nicht übel nimmt, es ist einfach zu viel für meine Nerven."

Lavender schnaubte. "Ich schätze, du wirst in nächster Zeit keine Armeen von Inferi erschlagen -"

"Hör auf damit!" sagte Hermine scharf. "Wir müssen doch alle lernen, Heldinnen zu sein, oder? Es ist in Ordnung, wenn es jemand nicht sofort weiß."

"Der Schulleiter glaubt, dass man es nicht lernen kann", sagte Padma. Das Gesicht des Ravenclaw-Mädchens war nachdenklich, ihre Schritte gemessen, als sie durch den Korridor schritt. "Der Schulleiter hält das nicht einmal für eine gute Idee."

Daphne schritt mit geradem Rücken und kerzengeradem Kopf und sah in ihren Hogwarts-Roben mehr wie eine anständige junge Dame aus, als Hermine es in ihrem besten Gesellschaftskleid hätte tun können. "Der Schulleiter", sagte Daphne mit präziser Stimme, wobei ihre Schuhe harte, scharfe Tackelgeräusche auf dem Stein machten, "denkt, dass wir alle ein Haufen dummer Mädchen sind, die Spielchen spielen, und dass Hermine vielleicht eines Tages eine gute Handlangerin sein wird, aber der Rest von uns hoffnungslos ist."

"Hat er recht?", sagte Parvati. Das Gesicht des Gryffindor-Mädchens war sehr ernst und ließ sie viel mehr wie ihren Zwilling aussehen, als sie es normalerweise tat. "Ich meine, es muss gefragt werden -"

"Nein!", spuckte Tracey. Das Slytherin-Mädchen stakste durch den Flur und sah aus, als wäre sie bereit, jemanden zu töten, wie ein weiblicher Snape in Miniatur.

Von allen Mädchen war Tracey diejenige, die Hermine am wenigsten kannte. Hermine hatte schon einmal mit Lavender gesprochen, aber sie hatte Tracey noch nie wirklich gesehen, außer bei einem Kampf mit dem Zauberstab, bis die Slytherin von ihrem Sofa aufgesprungen war, um sich freiwillig zu melden.

"Dem werden wir es zeigen! Wir werden es allen zeigen!"

"Okay", sagte Susan, "das war definitiv böse -"

"Nein", sagte Lavender, "das ist eigentlich ein Motto der Chaos Legion. Nur hat sie das wahnsinnige Lachen nicht gemacht."

"Das stimmt", sagte Tracey, ihre Stimme tief und grimmig. "Diesmal werde ich nicht lachen." Das Mädchen pirschte weiter durch den Korridor, als würde sie von dramatischer Musik begleitet, die nur sie hören konnte.

(\emph{Hermine begann sich Sorgen zu machen, was genau die beeinflussbaren Jugendlichen der Chaos Legion von Harry Potter lernten}.)

"Aber - ich meine -" sagte Parvati. Sie hatte immer noch einen nachdenklichen Ausdruck im Gesicht. "Ich meine, du kannst doch verstehen, warum der Schulleiter uns für dumme Mädchen hält, oder? Was hat der Protest vor dem Büro des Schulleiters damit zu tun, Heldinnen zu werden?"

"Huh", sagte Lavender, die nun selbst nachdenklich aussah. "Das ist wahr. Wir sollten etwas Heldenhaftes tun."

"Ähm -", sagte Hannah, was sehr gut Hermines eigene Gefühle zu diesem Thema ausdrückte.

"Nun", sagte Parvati, "ist denn schon jeder durch Dumbledores verbotenen Korridor im dritten Stock gegangen? Ich meine, jeder in Gryffindor ist da schon durchgegangen -"

"Moment mal!" sagte Hermine verzweifelt. "Ich will nicht, dass ihr etwas Gefährliches tut!"

Es gab eine Pause, während alle Hermine ansahen, \emph{die viel zu spät begriff, warum Dumbledore nicht wollte, dass jemand anderes ein Held wurde.}

"Ich glaube nicht, dass man eine Heldin werden kann, wenn man nie etwas Gefährliches tut", bemerkte Lavender vernünftig.

"Außerdem", sagte Padma mit einem nachdenklichen Gesichtsausdruck. "Jeder weiß doch, dass in Hogwarts nie etwas wirklich Schlimmes passiert, oder? Den Schülern, meine ich, nicht den Verteidigungsprofessoren. Wir haben all diese uralten Schutzzauber und so weiter."

"Ähm -" Sagte Hannah wieder.

"Ja", sagte Parvati, "das Schlimmste, was passieren kann, ist, dass wir ein paar Dutzend Hauspunkte verlieren oder so, und wir sind zwei aus jedem Haus, also wird das alles ausgeglichen."

"Das ist ja genial, Hermine!", sagte Daphne in einem Ton großer Verwunderung. "So wie du es eingerichtet hast, können wir mit allem durchkommen! Und ich habe deinen schlauen Plan bis jetzt nicht einmal bemerkt!"

"\textbf{Äh} -", sagten Hermine, Hannah und Susan.

"Richtig!", sagte Parvati. "Jetzt ist es also an der Zeit, dass wir zu echten Heldinnen werden. Wir werden gegen die Dunkelheit antreten -"

"Und bringen sie dazu, sich uns zu stellen -" sagte Lavender.

"Und sie lehren, sich zu fürchten", sagte Tracey Davis grimmig.

