

\hypertarget{frontaluxfcberschreitungen}{% \section{41. Frontalüberschreitungen}\label{frontaluxfcberschreitungen}}

\textbf{\uline{Frontalüberschreitung}}

Der beißende Januarwind heulte um die riesigen, leeren Steinmauern, die die materiellen Grenzen des Schlosses Hogwarts abgrenzten, und flüsterte und pfiff in seltsamen Tonlagen, während er an geschlossenen Fenstern und Steintürmchen vorbeiflog. Der jüngste Schnee war größtenteils weggeblasen, aber gelegentliche Flecken geschmolzenen und wieder gefrorenen Eises klebten noch an der Steinwand und reflektierten das Sonnenlicht. Aus der Ferne musste es so aussehen, als würde Hogwarts mit Hunderten von Augen blinzeln.

Eine plötzliche Böe ließ Draco zusammenzucken und versuchte, seinen Körper noch näher an den Stein zu drücken, der sich wie Eis anfühlte und nach Eis roch. Irgendein völlig sinnloser Instinkt schien davon überzeugt zu sein, dass er gleich von der Außenmauer von Hogwarts geweht werden würde, und dass die beste Art, dies zu verhindern, darin bestand, in einem hilflosen Reflex zusammenzuzucken und sich möglicherweise zu übergeben. Draco bemühte sich sehr, nicht an die sechs Stockwerke leerer Luft unter ihm zu denken und sich stattdessen darauf zu konzentrieren, \emph{wie er Harry Potter töten würde.}

"Wissen Sie, Mr. Malfoy", sagte das junge Mädchen neben ihm mit gesprächiger Stimme,\\ "wenn mir ein Seher gesagt hätte, dass ich eines Tages an den Fingerspitzen an der Seite eines Schlosses hängen würde und versuchen würde, nicht nach unten zu schauen oder daran zu denken, wie laut Mama schreien würde, wenn sie mich sieht, hätte ich keine Ahnung gehabt, wie es passieren würde, außer dass es Harry Potters Schuld wäre."

\textbf{Vorher}:\\ Die beiden verbündeten Generäle traten gemeinsam über Longbottoms Leiche, ihre Stiefel schlugen in fast perfekter Synchronität auf dem Boden auf. Nur ein einziger Soldat stand jetzt zwischen ihnen und Harry, ein Slytherin-Junge namens Samuel Clamons, dessen Hand weiß um seinen Zauberstab geballt war, den er nach oben hielt, um seinen Prismatischen Schild zu stützen. Der Junge atmete schnell, aber sein Gesicht zeigte dieselbe kalte Entschlossenheit, die in den Augen seines Generals, Harry Potter, leuchtete, der hinter dem Schild am toten Ende des Korridors neben einem offenen Fenster stand und die Hände geheimnisvoll hinter seinem Rücken hielt.

\emph{Die Schlacht war lächerlich schwierig gewesen, denn der Feind war zwei zu eins in der Überzahl.}

Es hätte einfach sein sollen, die Drachenarmee und das Sonnenschein Regiment waren in den Trainingseinheiten leicht miteinander verschmolzen, sie hatten lange genug gegeneinander gekämpft, um sich wirklich gut zu kennen. Die Moral war hoch, beide Armeen wussten, dass sie dieses Mal nicht nur kämpften, um für sich selbst zu gewinnen, sondern für eine Welt frei von Verrätern. Trotz der überraschten Proteste beider Generäle hatten die Soldaten der vereinigten Armee darauf bestanden, sich Dramiones Sonnche-Dragoment zu nennen, und für ihre Insignien Aufnäher mit einem lächelnden, von Flammen umschlungenen Gesicht hergestellt.

Aber Harrys Soldaten hatten alle ihre eigenen Abzeichen geschwärzt - es sah nicht wie Farbe aus, eher so, als hätten sie diesen Teil ihrer Uniformen verbrannt - \emph{und sie hatten sich mit einer verzweifelten Wut durch die oberen Stockwerke von Hogwarts gekämpft.}

Die kalte Wut, die Draco manchmal in Harry sah, schien auf seine Soldaten überzuspringen, und sie hatten gekämpft, als wäre es kein Spiel gewesen. Und Harry hatte seine ganze Trickkiste ausgeleert, es lagen winzige Metallkugeln (Granger hatte sie als "Kugellager" identifiziert) auf Böden und Treppen, die sie unpassierbar machten, bis sie beseitigt waren, nur hatte Harrys Armee bereits koordinierte Schwebezauber geübt und sie konnten ihre eigenen Leute direkt über die Hindernisse fliegen, die sie geschaffen hatten… Man konnte keine Geräte von außen ins Spiel bringen, aber man konnte alles, was man wollte, während des Spiels verwandeln, so lange es sicher war.

\emph{Und das war einfach nicht fair, wenn man gegen einen Jungen kämpfte, der von Wissenschaftlern aufgezogen wurde, die über Dinge wie Kugellager und Skateboards und Bungee-Seilchen Bescheid wussten.}

Und so war es dazu gekommen. Die Überlebenden der verbündeten Streitkräfte hatten die letzten Reste von Harry Potters Armee in einem Sackgassen-Korridor in die Enge getrieben.\\ Weasley und Vincent hatten sich gleichzeitig auf Longbottom gestürzt, sie bewegten sich zusammen, als hätten sie wochenlang statt stundenlang geübt, und irgendwie hatte Longbottom es geschafft, sie beide zu verhexen, bevor er selbst fiel. Und jetzt waren es Draco und Granger und Padma und Samuel und Harry, und so wie Samuel aussah, konnte seinen Schild nicht mehr viel länger halten.

Draco hatte seinen Zauberstab bereits auf Harry gerichtet und wartete darauf, dass der Schild von selbst fallen würde; es gab keinen Grund, vorher ein Brecherfluch zu verschwenden.\\ Padma richtete ihren eigenen Zauberstab auf Samuel, Granger richtete ihren auf Harry… Harry verbarg immer noch seine Hände hinter seinem Rücken, anstatt seinen Zauberstab zu zücken; und er sah sie mit einem Gesicht an, das aus Eis hätte geschnitzt sein können.

\emph{Vielleicht war es ein Bluff.}\\ \emph{War es wahrscheinlich nicht.}

Es herrschte eine kurze, angespannte Stille.

Und dann sprach Harry. "Ich bin jetzt der Schurke", sagte der Junge kalt, "und wenn ihr glaubt, dass Schurken so leicht zu erledigen sind, solltet ihr noch einmal nachdenken. Schlagt mich, wenn ich ernsthaft kämpfe, und ich bleibe geschlagen; aber verliert, und wir werden das Ganze beim nächsten Mal wiederholen."

Der Junge streckte seine Hände vor, und Draco sah, dass Harry seltsame Handschuhe trug, mit einem merkwürdigen gräulichen Material an den Fingerspitzen und Schnallen, die die Handschuhe fest an seine Handgelenke klammerten.

Neben Draco keuchte der Sonnenschein-General entsetzt auf, und Draco feuerte, ohne nach dem Grund zu fragen, einen Schildbrecher Fluch ab.

Samuel taumelte, er stieß einen Schrei aus, als er taumelte, aber er hielt die Mauer; und wenn Padma oder Granger jetzt feuerten, würden sie ihre eigenen Kräfte so sehr erschöpfen, dass sie vielleicht einfach verlieren würden.

"Harry!?", rief Granger. "Das kann doch nicht dein Ernst sein!"

Harry war bereits in Bewegung.\\ Als er sich aus dem offenen Fenster schwang, sagte seine Stimme:\\ "Folgt mir, wenn ihr euch traut."

Der eisige Wind heulte um sie herum.\\ Dracos Arme begannen bereits, sich müde anzufühlen.\\ … Es hatte sich herausgestellt, dass Harry Granger gestern sorgfältig demonstriert hatte, wie man die Handschuhe, die er gerade trug, verwandelte, wobei etwas verwendet wurde, das man "Gecko-Haare" nannte; und wie man verwandelte Flicken aus demselben Material an die Zehen ihrer Schuhe klebte; und Harry und Granger hatten in einem unschuldigen kindlichen Spiel versucht, ein wenig an den Wänden und der Decke herumzuklettern. Und dass Harry Granger, ebenfalls gestern, mit genau zwei Dosen Federfalltrank versorgt hatte, die sie in ihrem Beutel mit sich herumtrug,\\ \emph{"nur für den Fall"}.Nicht, dass Padma ihnen überhaupt gefolgt wäre. Sie war nicht verrückt.

Draco löste vorsichtig seine rechte Hand, streckte sie so weit wie möglich aus und klatschte sie wieder auf den Stein. Neben ihm tat Granger das Gleiche. Sie hatten den Federfall-Trank bereits geschluckt. Es war eine Umgehung der Spielregeln, aber der Trank würde nicht aktiviert werden, wenn nicht einer von ihnen tatsächlich fiel, und solange sie nicht fielen, benutzten sie den Gegenstand nicht. Professor Quirrell beobachtete sie.

\emph{Die beiden waren vollkommen, vollkommen, vollkommen sicher.}\\ \textbf{\emph{Harry Potter hingegen war im Begriff zu sterben.}\\ }\strut \\ "Ich frage mich, warum Harry das tut", sagte General Granger in einem nachdenklichen Ton, während sie langsam die Fingerspitzen einer Hand mit einem langgezogenen, klebrigen Geräusch von der Wand löste. Ihre Hand ploppte fast so schnell wieder zurück, wie sie hochgehoben worden war. "Das werde ich ihn fragen müssen, nachdem ich ihn getötet habe."

\emph{Es war erstaunlich, wie viele Gemeinsamkeiten sich zwischen den beiden herausstellten.}

Draco war im Moment nicht wirklich zum Reden zumute, aber er schaffte es, mit zusammengebissenen Zähnen zu sagen: "Es könnte Rache sein. Für das Date."

"Wirklich", sagte Granger. "Nach all dieser Zeit."\\ \emph{Stock. Plopp.}\\ "Wie süß von ihm", sagte Granger.\\ \emph{Stock. Plopp.}\\ "Ich werde mich wohl auf eine romantische Art bei ihm bedanken", sagte Granger.\\ \emph{Stock. Plopp.}\\ "Was hat er gegen dich?", sagte Granger.\\ \emph{Stock. Plopp.}

Der eisige Wind heulte um sie herum. Man hätte meinen können, es würde sich sicherer anfühlen, wieder Boden unter den Füßen zu haben. Aber wenn dieser Boden ein schräges, mit groben Latten gedecktes Dach war, auf dem viel mehr Eis lag als auf den Steinmauern, und man mit hoher Geschwindigkeit darüber rannte… Dann würde man sich gewaltig irren.

"Luminos!", rief Draco.\\ "Luminos!", rief Granger.\\ "Luminos!", hat Draco geschrien.\\ "Luminos!", rief Granger.

Die ferne Gestalt wich aus und rannte, und kein einziger Schuss traf, aber sie holten auf.\\ \emph{\hfill\break Bis Granger ausrutschte.}

Es war unvermeidlich, im Nachhinein betrachtet, im wirklichen Leben konnte man nicht mit hoher Geschwindigkeit über eisige Dachschrägen rennen.\\ \emph{Und auch unvermeidlich, denn es geschah ohne den geringsten Gedanken,}\\ Draco wirbelte herum und griff nach Grangers rechtem Arm, und er fing sie auf, nur war sie schon zu weit aus dem Gleichgewicht, sie fiel und zog Draco mit sich, es ging alles so schnell -\\ Es gab einen harten, schmerzhaften Aufprall, nicht nur Dracos Gewicht schlug auf dem Dach auf, sondern auch etwas von Grangers Gewicht, und wenn sie nur ein kleines bisschen näher an der Kante aufgeschlagen wäre, hätten sie es schaffen können, aber stattdessen kippte ihr Körper wieder und ihre Beine rutschten ab und ihre andere Hand griff verzweifelt zu. ..

Und so kam es, dass Draco sich mit einem weißen Griff an Grangers Arm festhielt, während ihre andere Hand verzweifelt nach der Kante des Daches griff und die Zehen von Dracos Schuhen sich in die Kante eines Dachziegels gruben.

"Hermine!" Harrys Stimme kreischte aus der Ferne.

"Draco", flüsterte Grangers Stimme, und Draco blickte nach unten.\\ Das könnte ein Fehler gewesen sein. Unter ihr war viel Luft, nichts als Luft, sie befanden sich auf der Kante eines Daches, das aus der Hauptsteinmauer von Hogwarts herausgeragt hatte.

"Er wird kommen und mir helfen", flüsterte das Mädchen, "aber zuerst wird er uns beide luminieren, das darf er nicht. Du musst mich los lassen."

\emph{Es hätte die einfachste Sache der Welt sein sollen.}\\ \emph{Sie war nur ein Schlammblut, nur ein Schlammblut,} \textbf{\emph{nur ein Schlammblut!}}\emph{\hfill\break Sie würde nicht mal verletzt werden!}

… Dracos Gehirn hörte in diesem Moment auf nichts, was Draco ihm sagte.

"Tu es", flüsterte Hermine Granger, ihre Augen loderten ohne eine einzige Spur von Angst,\\ "tu es, Draco, tu es, du kannst ihn selbst besiegen, wir müssen gewinnen, Draco!"

Da war ein Geräusch von jemandem, der rannte, und es kam näher.\\ \emph{\hfill\break Oh, sei vernünftig…}\\ Die Stimme in Dracos Kopf hörte sich furchtbar nach Harry Potter-Lehrstunden an.\\ \emph{… willst du dein Gehirn dein Leben bestimmen lassen?}

\textbf{Nachspiel}, \textbf{1:}\\ Es kostete Daphne Greengrass einige Mühe, ruhig zu bleiben, als Millicent Bulstrode die Geschichte im Gemeinschaftsraum der Slytherin-Mädchen erzählte (ein gemütlicher, kühler Ort in den Kerkern, die unter dem Hogwarts-See verlaufen, mit Fischen, die an jedem Fenster vorbeischwimmen, und Sofas, auf die man sich legen konnte, wenn man wollte).\\ Vor allem, weil es nach Daphnes Meinung schon ohne all die Verbesserungen von Millicent eine ganz gute Geschichte war.

"Und was dann?", keuchten Flora und Hestia Carrow.\\ "General Granger sah zu ihm auf", sagte Millicent dramatisch, "und sie sagte: 'Draco! Ihr müsst mich loslassen! Mach dir keine Sorgen um mich, Draco, ich verspreche, dass es mir gut gehen wird!\\ Und was glaubst du, hat Malfoy dann getan?"

"Er sagte 'Niemals!'", rief Charlotte Wiland, "und hielt sich noch fester!\\ Alle zuhörenden Mädchen außer Pansy Parkinson nickten. "

Nö!", sagte Millicent. "Er ließ sie fallen. Und dann ist er aufgesprungen und hat General Potter erschossen. Das Ende."

Es gab eine fassungslose Pause.

"Das kann er nicht tun!", sagte Charlotte.

"Sie ist ein Schlammblut", sagte Pansy und klang verwirrt. "Natürlich hat er losgelassen!"

"Dann hätte Malfoy sie gar nicht erst packen dürfen!", sagte Charlotte.\\ "Aber wenn er sie einmal gepackt hat, musste er sie festhalten! Besonders im Angesicht des nahenden sicheren Untergangs!"

Tracey Davis, die neben Daphne saß, nickte zustimmend.

"Ich verstehe nicht, warum", sagte Pansy.

"Das liegt daran, dass du nicht den kleinsten Funken Romantik in dir hast", sagte Tracey.\\ "Außerdem kannst du nicht einfach Mädchen fallen lassen. Ein Junge, der ein Mädchen so fallen lässt… er würde jeden fallen lassen. Er würde dich fallen lassen, Pansy."

"Was meinst du damit, mich fallen lassen?" sagte Pansy.

Daphne konnte nicht mehr widerstehen.\\ "Weißt du", sagte Daphne düster, "du frühstückst eines Tages an unserem Tisch, und das nächste, was du weißt, ist, dass Malfoy dich loslässt und du vom Dach von Hogwarts fällst! Das war's!"

"Ja!", sagte Charlotte. "Er ist ein Hexenschinder!"

"Weißt du, warum Atlantis gefallen ist?", sagte Tracey. "'Weil jemand wie Malfoy es fallen gelassen hat, deshalb!"

Daphne senkte ihre Stimme.\\ "In der Tat … was, wenn Malfoy derjenige ist, der Hermine, ich meine General Granger, überhaupt erst zum Ausrutschen gebracht hat? Was, wenn er darauf aus ist, alle Muggelgeborenen stolpern und fallen zu lassen?"

"Du meinst - ?", keuchte Tracey.

"Richtig!" sagte Daphne dramatisch. "Was, wenn Malfoy - der Erbe von Slipperin? ist?"

"Der nächste Drop Lord!", sagte Tracey.

Das war ein viel zu guter Spruch, um ihn für sich zu behalten, und so war er bei Einbruch der Dunkelheit in ganz Hogwarts zu lesen, und am nächsten Morgen war er die Schlagzeile des Klitterers.

\textbf{Nachspiel}, \textbf{2}:\\ Hermine sorgte dafür, dass sie an diesem Abend schön früh in ihrem üblichen Klassenzimmer war, damit sie allein auf einem Stuhl saß und friedlich ein Buch las, wenn Harry dort ankam.\\ \emph{Wenn es eine Möglichkeit gab, dass eine Tür entschuldigend aufknarrte, dann war es diese Tür.}

"Ähm", sagte die Stimme von Harry Potter.

Hermine las weiter.

"Es, ähm, tut mir irgendwie leid, ich wollte nicht, dass du tatsächlich vom Dach fällst oder so …"

\emph{Es war eigentlich eine recht unterhaltsame Erfahrung gewesen.}

"Ich, ah… ich habe nicht viel Erfahrung im Entschuldigen, ich falle auf die Knie, wenn du willst, oder kaufe dir etwas Teures, Hermine ich weiß nicht, wie ich mich bei dir dafür entschuldigen soll, was kann ich tun, sag es mir einfach?"

Sie las schweigend weiter in dem Buch.

Es war auch nicht so, dass sie eine Idee hatte, wie Harry sich entschuldigen könnte. Im Moment verspürte sie nur eine seltsame Neugier, was wohl passieren würde, wenn sie ihr Buch noch eine Weile weiterlas.

