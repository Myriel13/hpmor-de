

\hypertarget{die-kardinalsuxfcnde}{% \section{38. Die Kardinalsünde}\label{die-kardinalsuxfcnde}}

\textbf{\uline{Die Kardinalsünde}}

Hell die Sonne, hell die Luft, hell die Schüler und hell ihre Eltern, sauber der gepflasterte Boden von Bahnsteig 9.75, die Wintersonne hängt tief am Himmel um 9:45 Uhr am Morgen des 5. Januar 1992. Einige der jüngeren Schüler trugen Schals und Fäustlinge, aber die meisten trugen einfach ihre Roben; sie waren schließlich Zauberer.

Nachdem Harry sich von der Landeplattform entfernt hatte, zog er seinen Schal und seinen Mantel aus, öffnete ein Fach seines Koffers und verstaute seine Wintersachen. Einen langen Moment lang stand er da und ließ sich die Januarluft um die Nase wehen, nur um zu sehen, wie es war.

Harry nahm seine Zaubererroben heraus und zog sie an. Und schließlich zog Harry seinen Zauberstab; und er konnte nicht anders, als an die Eltern zu denken, die er gerade erst zum Abschied geküsst hatte, an die Welt, deren Probleme er hinter sich ließ…

Mit einem seltsamen Gefühl der Schuld für das Unvermeidliche sagte Harry: "Thermos." Die Wärme strömte durch ihn hindurch. Und der Junge-der-lebte war wieder da.

Harry gähnte und streckte sich, er fühlte sich mehr lethargisch als alles andere am Ende seines Urlaubs. Ihm war heute Morgen nicht danach, seine Lehrbücher zu lesen oder gar ernsthafte Science-Fiction zu lesen; was er brauchte, war etwas völlig Frivoles, um seine Aufmerksamkeit zu beschäftigen … Nun, das würde nicht schwer zu bekommen sein, wenn er bereit war, sich von vier Knuts zu trennen. Außerdem, wenn der Tagesprophet korrupt und der Klitterer die einzige konkurrierende Zeitung war, könnte es dort ein paar unterdrückte echte Nachrichten geben.

Harry stapfte zurück zu demselben Zeitungsstand wie beim letzten Mal und fragte sich, ob der Klitterer die Schlagzeile, die er zuvor gesehen hatte, übertreffen konnte. Der Verkäufer begann zu lächeln, als Harry sich näherte, und dann veränderte sich das Gesicht des Mannes plötzlich, als er die Narbe erblickte.

"Harry Potter?", keuchte der Verkäufer.

"Nein, Mr~Durian", sagte Harry und schaute kurz auf das Namensschild des Mannes,

"nur eine verblüffende Imitation -"

Und dann stockte Harrys Stimme in seiner Kehle, als er die obere Falte des Klitterers erblickte.

BESOFFENER SEHER PLAUDERT GEHEIMNISSE AUS:

DUNKLER LORD KEHRT ZURÜCK

, Einen Augenblick lang versuchte Harry, sein Gesicht zu verziehen, bevor ihm klar wurde, dass es in gewisser Weise genauso aufschlussreich sein könnte, nicht schockiert zu sein -

"Entschuldigung", sagte Harry.

Seine Stimme klang ein wenig erschrocken, und er wusste nicht einmal, ob das zu freizügig war, oder wie seine normale Reaktion aussehen würde, wenn er nichts wüsste.

\emph{Er hatte zu viel Zeit in der Nähe von Slytherins verbracht, er hatte vergessen, wie man Geheimnisse vor normalen Menschen bewahrt.}

4 Knuts schlugen auf den Tresen. "Ein Exemplar des Klitterers, bitte."

"Oh, keine Sorge, Mr~Potter!", sagte der Verkäufer hastig und fuchtelte mit den Händen.

"Es ist - schon gut, nur -"

Eine Zeitung flog durch die Luft und traf Harrys Finger, und er entfaltete sie.

BESOFFENER SEHER PLAUDERT GEHEIMNISSE AUS:

DUNKLER LORD KEHRT ZURÜCK

UM DRACO MALFOY ZU HEIRATEN

"Es ist kostenlos", sagte der Verkäufer, "für Sie, ich meine -"

"Nein", sagte Harry, "ich wollte sowieso eins kaufen."

Der Verkäufer nahm die Münzen, und Harry las weiter.

"Donnerwetter", sagte Harry eine halbe Minute später, "da bekommt man eine Seherin auf sechs Schluck Scotch und sie spuckt alle möglichen geheimen Sachen aus. Ich meine, wer hätte gedacht, dass Sirius Black und Peter Pettigrew insgeheim dieselbe Person sind?"

"Ich nicht", sagte der Verkäufer.

"Sie haben sogar ein Bild von den beiden zusammen, also wissen wir, wer es ist, der heimlich dieselbe Person ist."

"Jep", sagte der Verkäufer. "Ziemlich clevere Verkleidung, nicht wahr?"

"Und ich bin insgeheim fünfundsechzig Jahre alt."

"Sie sehen nicht halb so alt aus", sagte der Verkäufer freundlich.

"Und ich bin mit Hermine Granger verlobt, und Bellatrix Black, und Luna Lovegood, und ach ja, auch mit Draco Malfoy…"

"Das wird eine interessante Hochzeit werden", sagte der Verkäufer.

Harry blickte von der Zeitung auf und sagte mit angenehmer Stimme:

"Weißt du, zuerst habe ich gehört, dass Luna Lovegood verrückt sei, und ich habe mich gefragt, ob sie es wirklich ist oder ob sie sich nur etwas ausdenkt und die ganze Zeit vor sich hin kichert.

Als ich dann meine zweite Klitterer-Schlagzeile las, beschloss ich, dass sie nicht verrückt sein konnte, ich meine, es kann nicht einfach sein, sich so etwas auszudenken, man kann es nicht aus Versehen tun. Und wissen Sie jetzt, was ich denke? Ich denke, sie muss doch verrückt sein. Wenn normale Leute versuchen, etwas zu erfinden, kommt es nicht so raus. Irgendetwas muss in deinem Kopf wirklich schief laufen, bevor so etwas herauskommt, wenn du anfängst, dir Dinge auszudenken!"

Der Verkäufer starrte Harry an.

"Im Ernst", sagte Harry. "Wer liest dieses Zeug?"

"Du", sagte der Verkäufer.

Harry schlenderte davon, um seine Zeitung zu lesen. Er setzte sich nicht an denselben Tisch in der Nähe, an den er sich mit Draco gesetzt hatte, als er sich das erste Mal darauf vorbereitet hatte, diesen Zug zu besteigen. Das erschien ihm wie eine Versuchung, die Geschichte zu wiederholen. Es war nicht nur so, dass seine erste Woche in Hogwarts, nach dem Klitterer zu urteilen, vierundfünfzig Jahre lang gewesen war. Es war, dass sein Leben, nach Harrys bescheidener Meinung, keine neuen Fäden der Komplexität brauchte. Also suchte sich Harry einen kleinen Eisenstuhl irgendwo anders, weit weg vom Hauptgedränge und dem gelegentlichen dumpfen Knacken von Eltern, die mit ihren Kindern apparierten, und setzte sich hin und las den Klitterer, um zu sehen, ob er irgendwelche unterdrückten Neuigkeiten enthielt. Und neben der offensichtlichen Verrücktheit (der Himmel möge ihnen allen helfen, wenn irgendetwas davon echt wäre) gab es eine ganze Menge abfälligen romantischen Klatsch; aber nichts, was wirklich so wichtig wäre, wenn es wahr wäre.

Harry las gerade über das vom Ministerium vorgeschlagene Heiratsgesetz, das alle Ehen verbieten sollte, als -

"Harry Potter", sagte eine seidige Stimme, die einen Adrenalinstoß durch Harrys Blut jagte.

Harry sah auf.

"Lucius Malfoy", sagte Harry, seine Stimme war müde.

\emph{Das nächste Mal würde er das Klügste tun und draußen im Muggelteil von King's Cross bis 10:55 Uhr warten.}

Lucius neigte höflich den Kopf und ließ sein langes weißes Haar über seine Schultern wehen. Der Mann trug immer noch denselben Stock, schwarz lackiert mit einem silbernen Schlangenkopf als Griff; und irgendetwas an seinem Griff sagte stillschweigend, dass dies eine Waffe von tödlicher Macht ist, nicht dass ich schwach bin und mich darauf stütze. Sein Gesicht war ausdruckslos. Zwei Männer flankierten ihn, ihre Augen tasteten ihn ständig ab, ihre Zauberstäbe bereits tief in den Händen. Die beiden bewegten sich wie ein einziger Organismus mit vier Beinen und vier Armen, der Senior Crabbe-and-Goyle, und Harry glaubte zu erraten, wer welcher war, aber das war eigentlich egal. Sie waren lediglich Lucius' Anhängsel, so sicher, als wären sie die beiden rechten Zehen an seinem linken Fuß gewesen.

"Entschuldigen Sie die Störung, Mr~Potter", sagte die weiche, seidige Stimme.

"Aber Sie haben auf keine meiner Eulen geantwortet; und ich dachte, dies könnte meine einzige Gelegenheit sein, Sie zu treffen."

"Ich habe keine Ihrer Eulen erhalten", sagte Harry ruhig. "Dumbledore hat sie abgefangen, nehme ich an. Aber ich hätte ihnen auch nicht geantwortet, außer durch Draco. Wenn ich mich direkt mit Ihnen befassen würde, ohne Dracos Wissen, würde das unsere Freundschaft verletzen."

\emph{Bitte geh weg, bitte geh weg.}

.. Die grauen Augen funkelten ihn an.

"Das ist also deine Pose?", fragte der ältere Malfoy. "Nun ja. Ich werde ein wenig mitspielen. Was war Ihre Absicht, Ihren guten Freund, meinen Sohn, in ein öffentliches Bündnis mit diesem Mädchen zu manövrieren?"

"Oh", sagte Harry leichthin, "das ist doch offensichtlich, oder? Wenn Draco mit Granger zusammenarbeitet, wird ihm klar, dass Muggelgeborene doch Menschen sind. Bwa. Ha. Ha."

Eine dünne Spur eines Lächelns wanderte über Lucius' Lippen.

"Ja, das klingt tatsächlich wie einer von Dumbledores Plänen. Was es aber nicht ist."

"In der Tat", sagte Harry. "Es ist Teil meines Spiels mit Draco und kein Werk von Dumbledore, und mehr will ich dazu nicht sagen."

"Lassen wir die Spielchen", sagte der ältere Malfoy, wobei sich die grauen Augen plötzlich verhärteten. "Wenn mein Verdacht stimmt, würden Sie ohnehin kaum in Dumbledores Auftrag handeln, Mr~Potter."

Es entstand eine kleine Pause.

"Sie wissen es also", sagte Harry, seine Stimme war kalt. "Sagen Sie mir. An welchem Punkt genau haben Sie es erkannt?"

"Als ich Ihre Antwort auf Professor Quirrells kleine Rede las", sagte der weißhaarige Mann und kicherte grimmig. "Ich war zuerst verwirrt, denn es schien nicht in Ihrem eigenen Interesse zu sein; ich brauchte Tage, um zu verstehen, wessen Interesse damit gedient ist, und dann wurde mir endlich alles klar. Und es ist auch offensichtlich, dass Sie schwach sind, in mancher Hinsicht, wenn nicht in anderen."

"Sehr klug von Ihnen", sagte Harry, immer noch kalt. "Aber vielleicht verkennen Sie meine Interessen."

"Vielleicht tue ich das."

Ein Hauch von Stahl kam in die seidige Stimme.

"In der Tat, genau das ist es, was ich befürchte. Sie treiben seltsame Spiele mit meinem Sohn, zu einem Zweck, den ich nicht erraten kann. Das ist kein freundlicher Akt, und Sie können nur erwarten, dass ich mir Sorgen mache!"

Lucius stützte sich jetzt mit beiden Händen auf seinen Stock, und diese beiden Hände waren weiß, und seine Leibwächter hatten sich plötzlich angespannt.

Irgendein Instinkt in Harry behauptete, dass es eine sehr schlechte Idee wäre, seine Angst zu zeigen, um Lucius sehen zu lassen, dass er sich einschüchtern ließ. Sie befanden sich ohnehin in einem öffentlichen Bahnhof -

"Ich finde es interessant", sagte Harry und legte Stahl in seine Stimme,

"dass du denkst, ich könnte davon profitieren, Draco zu schaden. Aber das ist irrelevant, Lucius. Er ist mein Freund, und ich verrate meine Freunde nicht."

"\emph{Was}?", flüsterte Lucius. Sein Gesicht zeigte blankes Entsetzen. Dann -

"Gesellschaft", sagte einer der Lakaien, und Harry dachte anhand der Stimme, dass es der ältere Crabbe sein musste. Lucius richtete sich auf und drehte sich um, dann stieß er ein missbilligendes Zischen aus.

Neville kam auf ihn zu, er sah verängstigt, aber entschlossen aus, im Schlepptau eine große Frau, die überhaupt nicht verängstigt aussah.

"Madam Longbottom", sagte Lucius eisig.

"Mr~Malfoy", erwiderte die Frau ebenso eisig. "Sind Sie unserem Harry Potter ein Dorn im Auge?"

Das bellende Lachen, das von Lucius kam, wirkte seltsam bitter.

"Oh, ich glaube eher nicht. Sie sind gekommen, um ihn vor mir zu beschützen, nicht wahr?"

Der weißhaarige Kopf bewegte sich zu Neville.

"Und das wäre Mr~Potters treuer Leutnant, der letzte Spross der Familie Longbottom, Neville, selbsternannt vom Chaos. Wie sonderbar sich die Welt doch dreht. Manchmal denke ich, es muss alles verrückt sein."

Harry wusste gar nicht, was er darauf antworten sollte, und Neville sah verwirrt und verängstigt aus.

"Ich bezweifle, dass es die Welt ist, die verrückt ist", sagte Madam Longbottom. Ihre Stimme nahm einen hämischen Ton an. "Sie scheinen in schlechter Stimmung zu sein, Mr~Malfoy. Hat Sie die Rede unseres lieben Professor Quirrell ein paar Verbündete gekostet?"

"Es war eine geschickte Verleumdung meiner Fähigkeiten", sagte Lucius kalt,

"allerdings nur wirksam bei den Narren, die glauben, dass ich wirklich ein Todesser war."

"Was?", platzte es aus Neville heraus.

"\emph{Ich stand unter dem Imperius, junger Mann}", sagte Lucius und klang nun müde.

"Der Dunkle Lord hätte ohne die Unterstützung des Hauses Malfoy kaum mit der Rekrutierung unter den reinblütigen Familien beginnen können. Ich habe widersprochen, und er hat sich einfach meiner versichert. Seine eigenen Todesser wussten es erst danach, daher das falsche Zeichen, das ich trage; da ich aber nicht wirklich zugestimmt habe, bindet es mich nicht. Einige der Todesser glauben immer noch, dass ich der Erste unter ihnen war, und um des Friedens willen habe ich sie das glauben lassen, um sie unter Kontrolle zu halten. Aber ich war nicht so dumm, diesen unglückseligen Abenteurer meiner Wahl zu unterstützen -"

"Ignorier ihn", sagte Madam Longbottom, die Anweisung sowohl an Harry als auch an Neville gerichtet. "Er muss den Rest seines Lebens damit verbringen, sich zu verstellen, aus Angst vor einer Aussage unter Veritaserum." Sagte sie mit hämischer Genugtuung.

Lucius wandte sich abweisend von ihr ab und wandte sich wieder Harry zu.

"Würden Sie diese Schlampe bitten, sich zu entfernen, Mr~Potter?"

"Ich denke nicht", sagte Harry mit trockener Stimme. "Ich ziehe es vor, mich mit dem Teil des Hauses Malfoy zu beschäftigen, der in meinem Alter ist."

Dann gab es eine lange Pause.

Die grauen Augen suchten ihn ab.

"Natürlich …", sagte Lucius langsam. "Ich komme mir jetzt wirklich wie ein Narr vor.

Die ganze Zeit hast du nur so getan, als hättest du keine Ahnung, worüber wir reden."

Harry begegnete dem Blick, sagte aber nichts.

Lucius hob seinen Stock ein paar Zentimeter und schlug ihn hart auf den Boden. Die Welt verschwand in einem blassen Dunst, alle Geräusche verstummten, es gab nichts im Universum außer Harry und Lucius Malfoy und dem schlangenköpfigen Stock.

"Mein Sohn ist mein Herz", sagte der ältere Malfoy,

"das letzte Wertvolle, das mir in dieser Welt geblieben ist, und das sage ich dir im Geiste der Freundschaft: Wenn er zu Schaden käme, würde ich mein Leben der Rache überlassen.

Aber so lange mein Sohn nicht zu Schaden kommt, wünsche ich dir viel Glück bei deinen Unternehmungen. Und da du nichts weiter von mir verlangt hast, werde ich auch nichts weiter von dir verlangen."

Dann verschwand der blasse Schleier und zeigte eine empörte Madam Longbottom, die von dem älteren Crabbe am Weiterkommen gehindert wurde; ihren Zauberstab hielt sie jetzt in der Hand.

"Wie kannst du es wagen!", zischte sie. Lucius' dunkle Roben wirbelten um ihn herum, und sein weißes Haar, als er sich an den älteren Goyle wandte.

"Wir kehren zurück nach Malfoy Manor."

Es gab drei Knallgeräusche, und sie waren weg. Es folgte eine Stille.

"Du lieber Himmel", sagte Madam Longbottom. "Was sollte das denn?"

Harry zuckte hilflos mit den Schultern.

Dann sah er Neville an. Neville stand der Schweiß auf der Stirn.

"Vielen Dank, Neville", sagte Harry.

"Deine Hilfe war sehr willkommen, Neville. Und jetzt, Neville, denke ich, solltest du dich hinsetzen."

"Ja, General", sagte Neville, und anstatt zu einem der anderen Stühle neben Harry hinüberzugehen, ließ er sich auf dem Pflaster halb in eine sitzende Position zusammenfallen.

"Du hast viele Veränderungen in meinem Enkel bewirkt", sagte Madam Longbottom.

"Mit einigen bin ich einverstanden, mit anderen nicht."

"Schicken Sie mir die Liste, was was ist", sagte Harry. "Ich werde sehen, was ich tun kann."

Neville stöhnte, sagte aber nichts. Madam Longbottom gab ein Kichern von sich.

"Das werde ich, junger Mann, danke."

Ihre Stimme wurde leiser.

"Mr~Potter … die Rede, die Professor Quirrell gehalten hat, ist etwas, das unsere Nation schon lange hören musste. Ich kann deinen Kommentar dazu nicht nachvollziehen."

"Ich werde mir Ihre Meinung zu Herzen nehmen", sagte Harry milde.

"Ich hoffe sehr, dass du das tust", sagte Madam Longbottom und wandte sich wieder an ihren Enkel. "Muss ich noch -"

"Es ist in Ordnung, wenn du gehst, Oma", sagte Neville. "Diesmal komme ich auch allein zurecht."

"Dem stimme ich zu", sagte sie und knallte und verschwand wie eine Seifenblase.

Die beiden Jungen saßen einen Moment lang still da. Neville sprach zuerst, seine Stimme war müde.

"Du wirst versuchen, alle Änderungen, die sie gutheißt, zu beheben, richtig?"

"Nicht alle", sagte Harry unschuldig. "Ich will nur sichergehen, dass ich dich nicht korrumpiere."

Draco sah sehr besorgt aus. Sein Kopf huschte immer wieder herum, obwohl Draco darauf bestanden hatte, dass sie in Harrys Koffer hinuntergingen und einen echten Schweigezauber und nicht nur die schallverschleiernde Barriere verwendeten.

"Was hast du zu Vater gesagt?", platzte Draco heraus, in dem Moment, als der Schweigezauber wirkte und die Geräusche von Bahnsteig 9 3/4 verschwanden.

"Ich… hör mal, kannst du mir sagen, was er zu dir gesagt hat, bevor er dich abgesetzt hat?", fragte Harry.

"Dass ich es ihm sofort sagen soll, wenn du mich zu bedrohen scheinst", sagte Draco.

"Dass ich es ihm sofort sagen soll, wenn ich irgendetwas tue, was eine Bedrohung für dich darstellen könnte! \emph{Vater hält dich für gefährlich}, Harry, was immer du heute zu ihm gesagt hast, es hat ihn erschreckt! \textbf{Es ist keine gute Idee, Vater zu erschrecken!}"

\emph{Oh, verdammt…}

"Worüber habt ihr gesprochen?", fragte Draco.

Harry lehnte sich müde in dem kleinen Klappstuhl zurück, der auf dem Boden des Kellers seines Koffers stand.

"Weißt du, Draco, so wie die Grundfrage der Rationalität lautet: '\emph{Was glaube ich zu wissen und wie glaube ich es zu wissen?}', gibt es auch eine Kardinalsünde, eine Denkweise, die das Gegenteil davon ist. Wie die alten griechischen Philosophen. Sie hatten keine Ahnung, was vor sich ging, also gingen sie herum und sagten Dinge wie \emph{"Alles ist Wasser"} oder \emph{"Alles ist Feuer",} und sie fragten sich nie: \emph{"Moment mal, selbst wenn alles Wasser ist, wie könnte ich das wissen?} Sie haben sich nicht gefragt, ob sie Beweise haben, die diese Möglichkeit von allen anderen Möglichkeiten, die man sich vorstellen kann, unterscheiden, Beweise, auf die sie sehr wahrscheinlich nicht stoßen würden, wenn die Theorie nicht wahr wäre -"

"Harry", sagte Draco, seine Stimme war angestrengt, "worüber hast du mit Vater gesprochen?"

"Ich weiß es eigentlich nicht", sagte Harry, "deshalb ist es sehr wichtig, dass ich mir nicht einfach etwas ausdenke -"

Harry hatte Draco noch nie in einer so hohen Tonlage vor Entsetzen aufschreien hören.

