

\hypertarget{die-realituxe4t-mit-ihren-alternativen-vergleichen}{% \section{4. Die Realität mit ihren Alternativen vergleichen}\label{die-realituxe4t-mit-ihren-alternativen-vergleichen}}

*****\\ "Aber die Frage ist dann - wer?" ~\\ *****

"Großer Gott", sagte der Barmann und blickte Harry an, "ist das - kann das sein -?"

Harry lehnte sich, so gut er konnte, an die Theke des Tropfenden Kessels, obwohl er nur bis zu den Spitzen seiner Augenbrauen reichte. Eine Frage wie diese verdiente sein Bestes.

"Bin ich - könnte ich sein - vielleicht - man weiß ja nie - wenn ich es nicht bin - aber dann ist die Frage - wer?"

"Du meine Güte", flüsterte der alte Barmann. "Harry Potter… was für eine Ehre."

Harry blinzelte, dann raffte er sich auf. "Nun, ja, Sie sind ziemlich scharfsinnig; die meisten Leute erkennen das nicht so schnell -"

"Das reicht", sagte Professor McGonagall. Ihre Hand legte sich fester auf Harrys Schulter. "Belästigen Sie den Jungen nicht, Tom, das ist alles neu für ihn."

"Aber er ist es?", zitterte eine alte Frau. "Es ist Harry Potter?" Mit einem schabenden Geräusch erhob sie sich von ihrem Stuhl.

"Doris -" sagte McGonagall warnend. Der Blick, den sie durch den Raum schoss, hätte ausreichen müssen, um jeden einzuschüchtern.

"Ich will ihm nur die Hand schütteln", flüsterte die Frau. Sie beugte sich herunter und streckte eine faltige Hand aus, die Harry, der sich verwirrt und unbehaglicher fühlte als jemals zuvor in seinem Leben, vorsichtig schüttelte. Tränen fielen aus den Augen der Frau auf ihre verschränkten Hände. "Mein Enkel war ein Auror", flüsterte sie ihm zu. "Starb in den Siebzigern. Ich danke dir, Harry Potter. Dem Himmel sei Dank für dich."

"Gern geschehen", sagte Harry automatisch, dann drehte er den Kopf und warf Professor McGonagall einen ängstlichen, flehenden Blick zu. Professor McGonagall knallte mit dem Fuß auf den Boden, gerade als der allgemeine Ansturm beginnen sollte. Sie machte ein Geräusch, das Harry eine neue Bezeichnung für den Ausdruck "Geräusch des Verderbens" gab, und alle erstarrten auf der Stelle.

"Wir haben es eilig", sagte Professor McGonagall mit einer Stimme, die vollkommen, vollkommen normal klang. Sie verließen die Bar ohne Probleme.

"Professor?" sagte Harry, als sie im Innenhof waren. Er wollte eigentlich fragen, was los war, fand sich aber seltsamerweise dabei wieder, stattdessen eine ganz andere Frage zu stellen.

"Wer war der bleiche Mann da an der Ecke? Der Mann mit dem zuckenden Auge?"

"Hm?", sagte Professor McGonagall und klang ein wenig überrascht; vielleicht hatte sie diese Frage auch nicht erwartet.\\ "Das war Professor Quirinus Quirrell. Er wird dieses Jahr in Hogwarts Verteidigung gegen die dunklen Künste unterrichten."

"Ich hatte das seltsame Gefühl, dass ich ihn kenne …" Harry rieb sich die Stirn. "Und dass ich ihm nicht die Hand schütteln sollte."

Jemanden zu treffen, der einmal ein Freund gewesen war, bevor etwas drastisch schief gelaufen war … das war es eigentlich gar nicht, aber Harry fand keine Worte.

"Und was war … das alles?"

Professor McGonagall warf ihm einen seltsamen Blick zu.

"Mr. Potter… wissen Sie… wie viel man Ihnen erzählt hat… darüber, wie Ihre Eltern gestorben sind?" Harry erwiderte einen festen Blick.

"Meine Eltern leben noch, und sie haben sich immer geweigert, darüber zu sprechen, wie meine genetischen Eltern gestorben sind. Woraus ich schließe, dass es nicht gut war."

"Eine bewundernswerte Loyalität", sagte Professor McGonagall. Ihre Stimme wurde leiser. "Auch wenn es ein wenig weh tut, Sie das so sagen zu hören. Lily und James waren Freunde von mir."

Harry sah weg, plötzlich beschämt. "Es tut mir leid", sagte er mit leiser Stimme. "Aber ich habe eine Mum und einen Dad. Und ich weiß, dass ich mich nur unglücklich machen würde, wenn ich diese Realität mit … etwas Perfektem vergleichen würde, das ich mir in meiner Vorstellung aufgebaut habe."

"Das ist erstaunlich weise von Ihnen", sagte Professor McGonagall leise. "Aber Ihre genetischen Eltern sind in der Tat sehr gut gestorben, Sie haben dich beschützt."

Mich beschützt? Etwas Seltsames klammerte sich an Harrys Herz.

"Was… ist passiert?"

Professor McGonagall seufzte. Ihr Zauberstab berührte Harrys Stirn, und seine Sicht verschwamm für einen Moment.

"Eine Art Verkleidung", sagte sie, "damit so etwas nicht wieder vorkommt, nicht bevor du bereit bist."

Dann holte sie ihren Zauberstab wieder hervor und klopfte dreimal auf eine Backsteinmauer… … die sich zu einem Loch aushöhlte und sich erweiterte und ausdehnte und zu einem riesigen Torbogen zitterte, der eine lange Reihe von Geschäften mit Schildern enthüllte, die für Kessel und Drachenleber warben. Harry hat nicht geblinzelt. Es war ja nicht so, als würde sich jemand in eine Katze verwandeln. Und sie gingen vorwärts, gemeinsam, in die Welt der Zauberer.

Da gab es Händler, die Hüpfstiefel anpriesen ("Hergestellt mit echtem Flubber!") und "Messer +3! Gabeln +2! Löffel mit einem Bonus von +4!"\\ Es gab Brillen, die alles, was man ansah, grün färbten, und eine Reihe von bequemen Sesseln mit Schleudersitzen für Notfälle.

Harrys Kopf drehte und drehte sich, als wollte er sich von seinem Hals winden. Es war, als würde man durch den Abschnitt über magische Gegenstände in einem Regelbuch für Fortgeschrittenes Dungeons and Dragons gehen (er spielte das Spiel nicht, aber er las gerne die Regelbücher). Harry wollte auf keinen Fall einen einzigen Gegenstand verpassen, der zum Verkauf stand, für den Fall, dass es einer der drei war, die man brauchte, um den Zyklus der unendlichen Wunschzauber zu vervollständigen.

Dann entdeckte Harry etwas, das ihn, ganz ohne nachzudenken, von der stellvertretenden Schulleiterin ablenkte und direkt auf den Laden zusteuerte, eine Fassade aus blauen Ziegeln mit bronze-metallischen Verzierungen. Er wurde erst wieder in die Realität zurückgeholt, als Professor McGonagall direkt vor ihn trat.

"Mr. Potter?", fragte sie.

Harry blinzelte, dann wurde ihm klar, was er gerade getan hatte.

"Es tut mir leid! Ich habe für einen Moment vergessen, dass ich bei Ihnen bin und nicht bei meiner Familie."

Harry gestikulierte zum Schaufenster, in dem in feurigen Buchstaben, die durchdringend hell und doch fern leuchteten, "Bigbam's brilliante Bücher" geschrieben stand.

"Wenn man an einem Buchladen vorbeikommt, den man noch nie besucht hat, muss man hineingehen und sich umsehen. Das ist die Familienregel."

"Das ist das Ravenclawischste, was ich je gehört habe."

"Was?"

"Nichts. Mr. Potter, unser erster Schritt ist ein Besuch bei Gringotts, der Bank der Zaubererwelt. Ihr genetischer Familientresor ist dort, mit dem Erbe, das Ihre genetischen Eltern Ihnen hinterlassen haben, und Sie werden Geld für Schulsachen brauchen."\\ Sie seufzte.\\ "Und, ich nehme an, ein gewisses Taschengeld für Bücher könnte man auch entschuldigen. Obwohl Sie sich vielleicht eine Zeit lang zurückhalten sollten. Hogwarts hat eine ziemlich große Bibliothek zu magischen Themen. Und der Turm, in dem Sie, wie ich stark vermute, wohnen werden, hat eine noch umfangreichere eigene Bibliothek. Jedes Buch, das Sie jetzt kaufen, wäre wahrscheinlich ein Duplikat."

Harry nickte, und sie gingen weiter.

"Verstehen Sie mich nicht falsch, es ist eine großartige Ablenkung", sagte Harry, während er den Kopf weiterdrehte, "wahrscheinlich die beste Ablenkung, die je jemand an mir versucht hat, aber glauben Sie nicht, dass ich unsere anstehende Diskussion vergessen habe."

Professor McGonagall seufzte. "Deine Eltern - oder zumindest deine Mutter - waren vielleicht sehr klug, dir das nicht zu sagen."

"Sie wünschen also, dass ich weiterhin in seliger Unwissenheit leben kann? Dieser Plan hat einen gewissen Makel, Professor McGonagall."

"Ich nehme an, es wäre ziemlich sinnlos", sagte die Hexe knapp, "wenn jeder auf der Straße die Geschichte erzählen könnte. Nun gut."

Und sie erzählte ihm von Er-der-nicht-genannt-werden-darf, dem Dunklen Lord, Voldemort.

"Voldemort?" flüsterte Harry. Es hätte lustig sein sollen, aber das war es nicht. Der Name brannte mit einem kalten Gefühl, Rücksichtslosigkeit, diamantener Klarheit, ein Hammer aus reinem Titan, der auf einen Amboss aus nachgebendem Fleisch niedergeht.\\ Ein Schauer lief Harry über den Rücken, als er das Wort aussprach, und er beschloss in diesem Moment, sicherere Begriffe wie Du-weißt-schon-wer zu verwenden. Der Dunkle Lord war über das zaubernde Britannien hergefallen wie ein wilder Wolf, hatte an der Struktur ihres Alltagslebens gerissen und zerrissen.\\ Andere Länder hatten die Hände gerungen, aber gezögert, einzugreifen, sei es aus apathischem Egoismus oder aus schlichter Angst, denn wer von ihnen sich dem Dunklen Lord als Erster entgegenstellte, dessen Frieden würde das nächste Ziel seines Terrors sein.

(Der Zuschauer-Effekt, dachte Harry und dachte an das Experiment von Latane und Darley, das gezeigt hatte, dass man eher Hilfe bekam, wenn man vor einer Person einen epileptischen Anfall hatte, als vor drei Personen. Eine Streuung ~der Verantwortung, jeder hofft, dass ein anderer zuerst eingreift).

Die Todesser waren im Kielwasser des Dunklen Lords und in seiner Vorhut gefolgt, Aasgeier, die an Wunden rissen, oder Schlangen, die bissen und schwächten. Die Todesser waren nicht so schrecklich wie der Dunkle Lord, aber sie waren schrecklich. Und sie waren viele. Und die Todesser verfügten über mehr als nur Zauberstäbe; es gab Reichtum in diesen maskierten Reihen und politische Macht und erpresserische Geheimnisse, die eine Gesellschaft lähmten, die versuchte, sich zu schützen.

Ein alter und angesehener Journalist, Yermy Wibble, forderte höhere Steuern und eine Wehrpflicht. Er schrie, dass es absurd sei, dass sich die Vielen in Angst vor den Wenigen ducken sollten. Seine Haut, nur seine Haut, war am nächsten Morgen an die Wand der Redaktion genagelt gefunden worden, neben den Häuten seiner Frau und seiner beiden Töchter. Jeder wünschte sich, dass noch etwas getan werden sollte, und niemand wagte es, die Führung zu übernehmen und es vorzuschlagen. Wer\\ am meisten auffiel, wurde das nächste Beispiel.

Bis die Namen von James und Lily Potter an die Spitze der Liste stiegen. Und diese beiden wären mit ihren Zauberstäben in der Hand gestorben ~und hätten ihre Wahl nicht bereuet, denn sie waren Helden; aber sie ~hatten ein kleines Kind, ihren Sohn, Harry Potter.

Tränen traten in Harrys Augen. Er wischte sie weg, aus Wut oder vielleicht aus Verzweiflung, ich kannte diese Leute nicht, nicht wirklich, sie sind jetzt nicht meine Eltern, es wäre sinnlos, so traurig um sie zu sein -

Als Harry fertig war, in die Hexenroben zu schluchzen, sah er auf und fühlte sich ein wenig besser, als er auch Tränen in Professor McGonagalls Augen sah.

"Also, was ist passiert?" sagte Harry, seine Stimme zitterte.

"Der Dunkle Lord kam nach Godric's Hollow", sagte Professor McGonagall im Flüsterton. "Ein Zauber hätte dich verstecken sollen, aber ihr wurdet verraten. Der Dunkle Lord tötete James, und er tötete Lily, und am Ende kam er zu dir, zu deinem Kinderbett. Er hat den Tötungsfluch auf dich gewirkt und das war's dann. Der Tötungsfluch besteht aus purem Hass und trifft direkt die Seele, trennt sie vom Körper. Er kann nicht geblockt werden, und wen er trifft, der stirbt. Aber du hast überlebt. Du bist der einzige Mensch, der je überlebt hat. Der tödliche Fluch prallte ab und traf den Dunklen Lord. Er hinterließ nur seinen verbrannten Körper und eine Narbe auf deiner Stirn. Das war das Ende des Terrors. Und wir waren frei. Das, Harry Potter, ist der Grund, warum die Leute die Narbe auf deiner Stirn sehen wollen und warum sie deine Hand schütteln wollen."

Der Sturm der Trauer, der Harry durchflutet hatte, hatte alle seine Tränen verbraucht; er konnte nicht mehr weinen, er war fertig.

(Und irgendwo in seinem Hinterkopf war ein kleiner, kleiner Ton der Verwirrung, ein Gefühl, dass an dieser Geschichte etwas nicht stimmte; und es hätte zu Harrys Kunst gehören sollen, diesen winzigen Ton zu bemerken, aber er war abgelenkt. Denn es ist eine traurige Regel, dass, wenn man seine Kunst als Rationalist am meisten braucht, man sie am ehesten vergisst.)

Harry löste sich von Professor McGonagalls Seite.

"Ich werde - darüber nachdenken müssen", sagte er und versuchte, seine Stimme unter Kontrolle zu halten. Er starrte auf seine Schuhe.

"Ähm. Sie können sie ruhig meine Eltern nennen, wenn Sie wollen, Sie müssen ja nicht 'genetische Eltern' oder so sagen. Ich denke, es gibt keinen Grund, warum ich nicht zwei Mütter und zwei Väter haben kann."

Es gab keinen Ton von Professor McGonagall. Und sie gingen gemeinsam schweigend weiter, bis sie vor ein großes weißes Gebäude mit riesigen Bronzetüren kamen, über denen die geschnitzten Worte "Gringotts Bank" standen.

