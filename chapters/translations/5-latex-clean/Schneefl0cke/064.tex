

\hypertarget{ansteckende-luxfcgen}{% \section{65. Ansteckende Lügen}\label{ansteckende-luxfcgen}}

\textbf{\uline{Ansteckende Lügen}}

Hermine Granger hatte einmal irgendwo gelesen, dass einer der Schlüssel, um schlank zu bleiben, darin bestand, auf das Essen zu achten, das man aß, sich selbst beim Essen zu bemerken, damit man mit der Mahlzeit zufrieden war. Heute Morgen hatte sie sich Toast gemacht, und Butter auf den Toast getan, und Zimt auf die Butter, und das hätte eigentlich ausreichen müssen, um sie dazu zu bringen, dieses Mal die Güte zu bemerken, die vor ihr lag…

Ohne den Zimt oder die Butter zu bemerken, ohne das Essen zu bemerken oder dass sie aß, schluckte Hermine einen weiteren Bissen Toast herunter und sagte: "Kannst du versuchen, das noch einmal zu erklären? Ich bin immer noch völlig verblüfft."

"Es ist ziemlich einfach, wenn man wie ein Slytherin der Lichtseite denkt", sagte der Junge, von dem alle anderen in der Schule, außer ihnen beiden, jetzt glaubten, dass er ihre wahre Liebe sei. Harry Potters Löffel rührte geistesabwesend sein Frühstücksmüsli um; er hatte heute Morgen nicht viele Bissen davon genommen, nicht dass Hermine es gesehen hätte. "Jede gute Sache auf der Welt bringt ihre eigene Opposition ins Dasein. Phönixe sind da keine Ausnahme."

Hermine nahm unbemerkt einen weiteren Bissen von ihrem mit Butter und Zimt bestrichenen Toast und sagte: "Wie kann jemand nicht verstehen, dass Fawkes dich für gut genug hält, um auf deiner Schulter herumzureiten? Das würde er mit einem dunklen Zauberer nicht tun! Das würde er einfach nicht!" Und sie hatte niemanden wegen Fawkes' Berührung auf ihrer eigenen Wange angeschrien, weil sie wusste, dass es nicht richtig gewesen wäre - dass man, wenn ein Phönix einen berührte, nicht damit prahlen sollte, dafür war ein Phönix nicht da. Aber sie hatte wirklich gehofft, dass es die Gerüchte über Harry Potter, der böse wurde, und Hermine Granger, die ihm folgte, zerstören würde. Und das hatte es nicht. Und sie konnte wirklich nicht verstehen, warum nicht.

Harry aß einen weiteren Bissen von seinem Müsli, seine Augen gingen jetzt in die Ferne, trafen nicht mehr ihre eigenen. "Sieh es doch mal so: Du schwänzt eines Tages die Schule, lügst und sagst deinem Lehrer, du seist krank. Die Lehrerin sagt, du sollst ein ärztliches Attest mitbringen, also fälschst du eins. Die Lehrerin sagt, dass sie den Arzt anrufen wird, um das zu überprüfen, also musst du ihr eine falsche Nummer für den Arzt geben und einen Freund dazu bringen, so zu tun, als wäre er der Arzt, wenn sie anruft -"

"Du hast was?"

Harry schaute von seinem Müsli auf, und jetzt lächelte er. "Ich sage nicht, dass ich das wirklich getan habe, Hermine…" Dann fiel sein Blick abrupt zurück auf sein Müsli. "Nein. Nur ein Beispiel. Lügen verbreiten sich, das will ich damit sagen. Man muss mehr Lügen erzählen, um sie zu vertuschen, über jede Tatsache lügen, die mit der ersten Lüge zusammenhängt. Und wenn man immer weiter lügt und immer weiter versucht, es zu vertuschen, muss man früher oder später sogar anfangen, über die allgemeinen Gesetze des Denkens zu lügen. Zum Beispiel verkauft dir jemand irgendeine Art von Alternativmedizin, die nicht funktioniert, und jede experimentelle Doppelblindstudie wird bestätigen, dass sie nicht funktioniert. Wenn also jemand die Lüge weiter verteidigen will, muss er dich dazu bringen, nicht an die experimentelle Methode zu glauben. Zum Beispiel ist die experimentelle Methode nur für rein wissenschaftliche Arten der Medizin, nicht für erstaunliche alternative Medizin wie diese. Oder ein guter und tugendhafter Mensch sollte so fest wie möglich glauben, egal was die Beweise sagen. Oder die Wahrheit existiert nicht und es gibt so etwas wie eine objektive Realität nicht. Viele solche Aussagen sind nicht nur falsch, sie sind anti-epistemologisch, sie sind systematisch falsch. Bei jeder Rationalitätsregel, die dir sagt, wie du die Wahrheit finden kannst, gibt es jemanden da draußen, der will, dass du das Gegenteil glaubst. Wenn du einmal eine Lüge erzählst, ist die Wahrheit immer dein Feind; und es gibt eine Menge Leute da draußen, die Lügen erzählen -" Harrys Stimme stockte.

"Was hat das mit Fawkes zu tun?", fragte sie. Harry zog den Löffel aus seinem Müsli und zeigte in Richtung des Haupttisches. "Der Schulleiter hat einen Phönix, richtig? Und er ist Oberster Hexenmeister des Zaubergamot? Also hat er politische Gegner, wie Lucius. Glaubst du, dass diese Opposition einfach aufgibt, weil Dumbledore einen Phönix hat und die nicht? Glaubst du, die geben zu, dass Fawkes auch nur ein Beweis dafür ist, dass Dumbledore ein guter Mensch ist? Nein, natürlich nicht. Die müssen irgendwas erfinden, um zu sagen, dass Fawkes… unwichtig ist. Zum Beispiel, dass Phönixe nur Leuten folgen, die direkt auf jeden losgehen, den sie für böse halten, also bedeutet ein Phönix nur, dass man ein Idiot oder ein gefährlicher Fanatiker ist. Oder, Phönixe folgen nur Leuten, die reine Gryffindor sind, also haben sie nicht die Tugenden der anderen Häuser. Oder es zeigt nur, für wie mutig ein magisches Tier einen hält, sonst nichts, und es wäre nicht fair, Politiker danach zu beurteilen. Sie müssen etwas sagen, um den Phönix zu verleugnen. Ich wette, Lucius musste sich nicht einmal etwas Neues ausdenken. Ich wette, das war alles schon einmal gesagt worden, vor Jahrhunderten, seit dem ersten Mal, als jemand einen Phönix auf seiner Schulter reiten ließ, und jemand anderes wollte, dass die Leute das nicht als Beweis in Betracht zogen. Ich wette, zu der Zeit, als Fawkes auftauchte, war es bereits Allgemeinwissen, es wäre nur seltsam erschienen, zu berücksichtigen, wen ein Phönix mochte oder nicht mochte. Das wäre so, als würde eine Muggel-Zeitung politische Kandidaten auf ihren wissenschaftlichen Kenntnisstand testen. Für jede Kraft des Guten, die es in diesem Universum gibt, gibt es jemanden, der davon profitiert, dass die Menschen sie missachten oder in eine enge Box sperren, wo sie nicht an sie herankommt."

"Aber -" sagte Hermine. "Okay, ich verstehe, warum Lucius Malfoy nicht will, dass jemand glaubt, dass Fawkes eine Rolle spielt, aber warum glaubt jemand, der kein Bösewicht ist, daran?"

Harry Potter zuckte ein wenig mit den Schultern. Sein Löffel fiel zurück in sein Müsli und rührte ohne Pause weiter. "Warum ist jede Art von Zynismus für die Leute attraktiv? Weil es wie ein Zeichen von Reife erscheint, von Kultiviertheit, als hätte man alles gesehen und wüsste es besser. Oder weil es sich so anfühlt, als würde man sich selbst hochpushen, wenn man etwas niedermacht. Oder weil sie selbst keinen Phönix haben und ihr politischer Instinkt ihnen sagt, dass es keinen Vorteil bringt, nette Dinge über Phönixe zu sagen. Oder weil es sich zynisch anfühlt, eine geheime Wahrheit zu kennen, die das gemeine Volk nicht weiß …" Harry Potter blickte in die Richtung des Haupttisches, und seine Stimme sank, bis sie fast ein Flüstern war. "Ich denke, vielleicht ist es das, was er falsch versteht - dass er über alles andere zynisch ist, aber nicht über den Zynismus selbst."

Ohne nachzudenken, schaute Hermine selbst in Richtung des Lehrertisches, aber der Platz des Verteidigungsprofessors war immer noch leer, wie schon am Montag und Dienstag; die stellvertretende Schulleiterin hatte vorhin verkündet, dass Professor Quirrells Unterricht für heute ausfallen würde. Nachdem Harry ein paar Bissen Sirupkuchen gegessen und dann den Tisch verlassen hatte, schaute Hermine zu Anthony und Padma, die zufällig in der Nähe gegessen hatten, aber sicher nicht lauschten oder so.

Anthony und Padma sahen sie wieder an. Padma sagte zögernd: "Geht es nur mir so, oder hat Harry Potter in den letzten Tagen angefangen, wie eine kompliziertere Art von Buch zu reden? Ich meine, ich habe ihm noch nicht sehr lange zugehört -"

"Es liegt nicht nur an dir", sagte Anthony.

Hermine sagte nichts, aber sie machte sich zunehmend Sorgen. Was auch immer mit Harry Potter am Tag des Phönix geschehen war, es hatte ihn verändert; es war jetzt etwas Neues in ihm. Nicht kalt, aber hart. Manchmal ertappte sie ihn dabei, wie er aus dem Fenster auf nichts Sichtbares starrte, mit einem Ausdruck grimmiger Entschlossenheit auf seinem Gesicht. In der Kräuterkundestunde am Montag war eine Venus-Feuerfalle außer Kontrolle geraten; und Harry hatte Terry aus dem Weg geräumt, als Professor Sprout einen Flammen-Frost-Zauber ausrief; und als Harry aufgestanden war, war er einfach wieder an seinen Platz gegangen, als ob nichts Interessantes passiert wäre. Und als sie in der Verwandlungsprüfung am Montag ausnahmsweise mal besser abgeschnitten hatte als Harry, hatte Harry sie angelächelt, als wollte er ihr gratulieren, statt die Zähne zusammenzubeißen; und… \emph{das hatte sie sehr gestört}. Sie hatte das Gefühl, dass Harry… \emph{… sich von ihr wegbewegte…}

"Er wirkt auf einmal viel älter", sagte Anthony. "Nicht wie ein richtiger Erwachsener, ich kann mir Harry nicht als Erwachsenen vorstellen, aber es ist, als ob er sich plötzlich in eine Version des vierten Jahres verwandelt hat… von was auch immer er ist."

"Nun", sagte Padma. Sie tupfte zierlich einen Scone mit Schokoladengeschmack mit einem Zuckerguss mit Scone-Geschmack ab. "Ich denke, Drache und Sonnenschein sollten sich bei der nächsten Schlacht besser verbünden, oder Mr~Harry Potter wird uns zerschmettern. Letztes Mal haben wir uns verbündet, und selbst da hätte das Chaos fast gewonnen -"

"Ja", sagte Anthony. "Sie haben recht, Miss~Patil. Hermine, sag dem Drachengeneral, dass wir uns mit ihm treffen wollen -"

"Nein!", sagte Hermine. "Wir sollten uns nicht gegen General Potter zusammentun müssen, nur um eine Chance zu haben. Das macht keinen Sinn, besonders jetzt, wo niemand mehr Muggelsachen benutzen kann. Es sind immer noch vierundzwanzig Soldaten in jeder Armee."

Weder Padma noch Anthony sagten etwas dazu.

….

\emph{Klopf-klopf, klopf-klopf.}

"Kommen Sie herein, Mr~Potter", sagte die Stimme. Die Tür knarrte auf, und Harry Potter schlüpfte durch die Öffnung in ihr Büro; er schob die Tür mit einer Hand hinter sich zu und setzte sich wortlos in den gepolsterten Stuhl, der nun vor ihrem Schreibtisch stand. Sie hatte diesen Stuhl so oft verwandelt, dass er manchmal spontan seine Form änderte, um ihre Stimmung widerzuspiegeln, ohne dass sie einen Zauberstab bewegte oder eine Beschwörung aussprach oder auch nur eine bewusste Absicht hatte. In diesem Moment war der Stuhl tief gepolstert, so dass Harry, als er sich setzte, in ihm versank, als würde der Stuhl ihn umarmen.

Harry schien es nicht zu bemerken. Es lag eine stille Entschlossenheit über dem Jungen; seine Augen hatten sich fest mit den ihren verbunden und ließen keinen Moment lang nach.

"Sie haben mich gerufen?", fragte der Junge.

"Das habe ich", sagte Professor McGonagall. "Ich habe zwei gute Nachrichten für Sie, Mr~Potter. Erstens - haben Sie Mr~Rubeus Hagrid schon kennengelernt? Den Hausverwalter? Er war ein alter Freund Ihrer Eltern."

Harry zögerte.

Dann: "Mr~Hagrid hat ein wenig mit mir gesprochen, nachdem ich hierher gekommen bin", sagte Harry. "Ich glaube, es war am Dienstag in meiner ersten Schulwoche. Er hat aber nicht gesagt, dass er meine Eltern kennt. Damals dachte ich, er wollte sich nur dem Jungen-der-lebte vorstellen… Hatte er irgendeine versteckte Absicht? Er schien nicht der Typ zu sein, der …"

"Ah …", sagte sie. Sie brauchte einen Moment, um ihre Gedanken zu sortieren. "Es ist eine lange Geschichte, Mr~Potter, aber Mr~Hagrid wurde fälschlicherweise beschuldigt, einen Schüler ermordet zu haben, und zwar vor fünf Jahrzehnten. Mr~Hagrid wurde der Zauberstab zerbrochen und er wurde der Schule verwiesen. Später, als Professor Dumbledore Schulleiter wurde, gab er Mr~Hagrid einen Platz hier als Hüter des Geländes und der Schlüssel."

Harrys Augen beobachteten sie aufmerksam. "Sie sagten, dass vor fünf Jahrzehnten das letzte Mal ein Schüler in Hogwarts gestorben ist, und Sie waren sich sicher, dass vor fünf Jahrzehnten das letzte Mal jemand die geheime Botschaft des Sprechenden Hutes gehört hat."

Sie spürte ein leichtes Frösteln - \emph{selbst der Schulleiter oder Severus hätten diese Verbindung wohl nicht so schnell hergestellt} - und sagte: "Ja, Mr~Potter. Jemand hat die Kammer des Schreckens geöffnet, aber das wurde nicht geglaubt, und Mr~Hagrid wurde für den daraus resultierenden Tod verantwortlich gemacht. Der Schulleiter hat jedoch die zusätzliche Verzauberung auf dem Sprechenden Hut gefunden, und er hat sie einem speziellen Gremium desZaubergamot gezeigt. Infolgedessen wurde Mr~Hagrids Strafe aufgehoben - gerade heute Morgen, um genau zu sein - und er darf sich einen neuen Zauberstab zulegen." Sie zögerte. "Wir … haben Mr~Hagrid noch nicht davon in Kenntnis gesetzt, Mr~Potter. Wir haben gewartet, bis die Tat vollbracht war, um ihm nach so langer Zeit keine falschen Hoffnungen zu machen. Mr~Potter… wir haben uns gefragt, ob wir Mr~Hagrid sagen können, dass du es warst, der ihm geholfen hat…?"

Sie sah den abwägenden Blick in seinen Augen -

"Ich erinnere mich, dass Mr~Hagrid dich im Arm gehalten hat, als du ein Baby warst", sagte sie. "Ich glaube, er würde sich sehr freuen, das zu erfahren."

Sie konnte es in Harrys Gesicht sehen, den Moment, in dem er entschied, dass Rubeus ihm nichts nützen würde.

Harry schüttelte den Kopf. "Schlimm genug, dass jemand auf einen Parselmund in der diesjährigen Schülerschar schließen könnte", sagte Harry. "Ich denke, es wäre klüger, das Ganze so geheim wie möglich zu halten."

Sie erinnerte sich an James und Lily, die nie gezögert hatten, die Freundschaft zu erwidern, die der riesige, gutherzige Mann ihnen angeboten hatte, obwohl James der Spross eines wohlhabenden Hauses war oder Lily eine angehende Zauberlehrerin und Rubeus ein bloßer Halbriese, dessen Zauberstab zerbrochen war… "Weil Sie nicht erwarten, dass er sich als nützlich erweisen wird, Mr~Potter?"

Es herrschte Schweigen. Sie hatte nicht vorgehabt, das laut zu sagen. Traurigkeit zog über Harrys Gesicht.

"Wahrscheinlich", sagte Harry leise. "Aber ich glaube nicht, dass er und ich uns gut verstehen würden, oder?"

Etwas schien ihr im Hals stecken zu bleiben.

"Apropos Leute ausnutzen", sagte Harry. "Es scheint, als würde ich bald in einen Krieg mit einem Dunklen Lord hineingezogen werden. Während ich also in Ihrem Büro bin, möchte ich darum bitten, dass mein Schlafzyklus auf dreißig Stunden pro Tag verlängert wird. Neville Longbottom will anfangen, das Duellieren zu üben, ein älterer Hufflepuff hat angeboten, ihn zu unterrichten, und sie haben mich eingeladen, mitzumachen. Außerdem gibt es noch andere Dinge, die ich lernen möchte - und wenn Sie oder der Schulleiter meinen, dass ich irgendetwas Bestimmtes lernen sollte, um ein mächtiger Zauberer zu werden, wenn ich groß bin, lassen Sie es mich wissen. Bitte weisen Sie Madam Pomfrey an, den entsprechenden Zaubertrank zu verabreichen, oder was auch immer sie zu tun hat -"

"Mr~Potter!"

Harrys Augen blickten direkt in ihre eigenen. "Ja, Minerva? Ich weiß, dass es nicht deine Idee war, aber ich würde gerne den Gebrauch überleben, den der Schulleiter von mir macht. Bitte sei mir dabei kein Hindernis."

Das zerbrach Sie fast.

"Harry", flüsterte sie mit leiser Stimme, "Kinder sollten nicht so denken müssen!"

"Du hast recht, das sollten sie nicht", sagte Harry. "Aber viele Kinder müssen viel zu früh erwachsen werden, nicht nur ich; und die meisten dieser Kinder würden wahrscheinlich in fünf Sekunden mit mir tauschen. Ich werde mich nicht selbst bemitleiden, Professor McGonagall, nicht wenn es da draußen Leute gibt, die in echten Schwierigkeiten stecken, und ich gehöre nicht dazu."

Sie schluckte hart und sagte: "Mr~Potter, bei dreißig Stunden am Tag wirst du - älter, du alterst schneller -" \emph{Wie Albus}.

"Und in meinem fünften Jahr werde ich ungefähr im gleichen physiologischen Alter sein wie Hermine", sagte Harry. "Scheint ja nicht so schlimm zu sein." Auf Harrys Gesicht war jetzt ein schiefes Lächeln zu sehen. "Ehrlich gesagt, würde ich das wahrscheinlich auch wollen, wenn es keinen Dunklen Lord gäbe. Zauberer leben eine Weile, und entweder Zauberer oder Muggel werden das im Laufe des nächsten Jahrhunderts wahrscheinlich noch weiter hinausschieben. Es gibt keinen Grund, nicht so viele Stunden in einen Tag zu packen, wie ich kann. Ich habe Dinge vor, die ich erledigen will, und es wäre gut, wenn sie schnell erledigt wären."

Es gab eine lange Pause. "In Ordnung", sagte Minerva. Es kam fast wie ein Flüstern heraus. Sie erhob ihre Stimme. "In Ordnung, Mr~Potter, ich werde den Schulleiter fragen, und wenn er zustimmt, wird es getan."

Harrys Augen verengten sich für einen Moment. "Ich verstehe. Dann erinnern Sie den Schulleiter bitte daran, dass Godric Gryffindor in seinen letzten Worten sagte, \emph{wenn es für ihn das Richtige gewesen wäre, dann würde er niemandem mehr raten, sich falsch zu entscheiden, nicht einmal dem jüngsten Schüler in Hogwarts}."

Und sie wusste mit einem hohlen Gefühl, dass jede Chance, dass Albus das aufhalten konnte, dass irgendetwas von dem, was er gesagt hatte, sich gerade in Nichts aufgelöst hatte. Das war es, was Albus ihr gesagt hatte, als sie eingewendet hatte, dass Cameron Edward zu jung war, und dann, als sie eingewendet hatte, dass Peter Pevensie zu jung war, und schließlich hatte sie es aufgegeben, zu widersprechen.

"Wer hat Ihnen das gesagt, Mr~Potter?" \emph{Nicht Albus - sicher würde Albus das nie zu einem Schüler sagen} -

"Ich habe in letzter Zeit viel gelesen", sagte Harry. Sein Körper begann sich von dem ihn umschließenden Stuhl zu erheben, dann hielt er inne. "Darf ich nach der zweiten guten Nachricht fragen?"

"Oh", sagte sie. "Ah - Professor Quirrell ist aufgewacht und sagt, Sie dürfen -"

…

Der Krankenflügel von Hogwarts war ein heller, offener Raum, der auf allen vier Seiten von Dachfenstern erhellt wurde, obwohl er mitten im Schloss zu liegen schien. Weiße Betten in langen Reihen dehnten sich aus, von denen im Moment nur drei belegt waren. Ein älterer Junge und ein älteres Mädchen auf gegenüberliegenden Seiten, beide lagen bewegungslos mit geschlossenen Augen, wahrscheinlich bewusstlos und mit einem Zauber belegt, während irgendein Heilzauber oder -trank ihre Körper auf unangenehme Weise umgestaltete; und der dritte Insasse hatte den Vorhang um sein Bett gezogen, was vermutlich eine gute Sache war. Madam Pomfrey hatte ihn mit einem harten Stoß weitergeschoben und ihm gesagt, er solle nicht gaffen, und Harry hatte sich scharf daran erinnern müssen, dass einige Leute immer noch nicht wussten, wer der Junge-der-lebte war - entweder das, oder Madam Pomfreys Identität war mit ihrer absoluten Herrschaft über ihr eigenes Krankenhaus verbunden, \emph{und so weiter, was auch immer}.

Hinter den Bettenreihen befanden sich fünf Türen, die in die Privaträume führten, in denen sie die Patienten aufbewahrten, die nicht Stunden, sondern Tage bleiben würden, deren Zustand aber eine Verlegung nach St. Mungos nicht rechtfertigte. Fensterlos, ohne Dach, unbeleuchtet bis auf eine einzige rauchlose Fackel an einer der massiven Steinwände; das war der Raum hinter der mittleren Tür. Harry hatte sich gefragt, ob Professoren Hogwarts bitten konnten, sich zu verändern; oder ob die Krankenstation immer einen solchen Raum zur Verfügung hatte, für Leute, die das Licht nicht mochten.

In der Mitte des Raumes, zwischen zwei gleich großen Bettgestellen, die aus demselben grauen Marmor wie die Wände geschnitzt zu sein schienen, stand ein weißes Krankenhausbett, das im Licht der nicht rauchenden Fackel vage ockerfarben aussah; und in diesem Bett, ein weißes Laken um die Oberschenkel hochgezogen und mit einem Krankenhauskittel bekleidet, saß Professor Quirrell, den Rücken leicht gegen das Kopfteil des Bettes gelehnt. Es hatte etwas Beängstigendes, Professor Quirrell in einem von Madam Pomfreys Betten zu sehen, auch wenn der Verteidigungsprofessor unverletzt schien. Auch wenn er wusste, dass Professor Quirrell seine eigene scheinbare Niederlage gegen Severus' Hände absichtlich arrangiert hatte, um sich selbst einen Vorwand zu geben, um aus Askaban wieder zu Kräften zu kommen. Harry hatte noch nie jemanden in einem Krankenhausbett sterben sehen, aber er hatte zu viele Filme gesehen. Es war eine Andeutung von Sterblichkeit, \emph{und der Verteidigungsprofessor sollte nicht sterblich sein}.

Madam Pomfrey hatte Harry gesagt, dass es ihm absolut verboten sei, ihren Patienten zu belästigen. Harry hatte gesagt: "Ich verstehe", was technisch gesehen nichts über Gehorsam aussagte. Die strenge alte Heilerin hatte sich dann umgedreht und begonnen, zu Professor Quirrell zu sagen, dass er sich auf keinen Fall überanstrengen oder… aufregen dürfe… Madam Pomfrey war ins Stocken geraten, hatte sich eilig umgedreht und war aus dem Raum geflohen.

"Nicht schlecht", bemerkte Harry, nachdem sich die Tür hinter der flüchtenden Arzthelferin geschlossen hatte. "Das muss ich auch noch lernen, irgendwann."

Professor Quirrell lächelte ein Lächeln, das absolut keinen Humor enthielt, und sagte, wobei seine Stimme ein gutes Stück trockener klang als sonst: "Danke für deine Kunstkritik, Mr~Potter."

Harry starrte in die blassblauen Augen und fand, dass Professor Quirrell … \emph{… älter} aussah. Es war subtil, vielleicht war es nur Harrys Einbildung, vielleicht war es die schlechte Beleuchtung. Aber das Haar über Quirinus Quirrells Stirn mochte sich ein wenig zurückgebildet haben, was übrig geblieben war, mochte dünner und grauer geworden sein, ein Fortschreiten der Kahlheit, die bereits an seinem Hinterkopf sichtbar war. Das Gesicht mochte ein wenig eingefallen sein. Die blassblauen Augen waren scharf und intensiv geblieben.

"Ich bin froh", sagte Harry leise, "Sie bei scheinbar guter Gesundheit zu sehen."

"Der Schein kann natürlich trügen", sagte Professor Quirrell. Er schnippte mit den Fingern, und als seine Hand die Geste beendete, hielt er seinen Zauberstab. "Würden Sie glauben, dass diese Frau glaubt, sie hätte ihn mir abgenommen?" Dann sprach der Verteidigungsprofessor sechs Beschwörungsformeln; sechs von den dreißig, die er benutzt hatte, um ihre wichtigen Gespräche in Marys Restaurant zu sichern.

Harry hob die Augenbrauen, leise fragend.

"Das ist alles, was ich im Moment zustande bringe", sagte der Verteidigungsprofessor. "Ich nehme an, dass es sich als ausreichend erweisen wird. Dennoch gibt es ein Sprichwort: \emph{Wenn du nicht willst, dass man etwas hört, dann sag es nicht}. Betrachte es es in vollem Umfang als zutreffend. Wie ich höre, willst du mich sprechen?"

"Ja", sagte Harry. Er hielt inne, sammelte seine Gedanken. "Hat der Schulleiter oder sonst jemand Ihnen gesagt, dass wir nicht mehr zum Mittagessen gehen können?"

"Etwas in dieser Richtung", sagte der Verteidigungsprofessor. Und ohne seinen Gesichtsausdruck zu verändern: "Natürlich tat es mir furchtbar leid, das zu hören."

"Eigentlich ist es noch viel extremer", sagte Harry. "Ich bin auf unbestimmte Zeit an Hogwarts und sein Gelände gebunden. Ich kann nicht ohne einen Wächter und einen guten Grund weggehen. Ich werde den Sommer über nicht nach Hause gehen, und vielleicht nie wieder. Ich hatte gehofft… mit Ihnen darüber zu sprechen."

Es gab eine Pause. Der Verteidigungsprofessor atmete wie ein kurzer Seufzer aus und sagte: "Wir werden uns einfach auf die bekannte Tatsache verlassen müssen, dass die stellvertretende Schulleiterin jeden persönlich ermorden wird, der versucht, mich anzuzeigen. Mr~Potter, ich beabsichtige, dieses Gespräch so zu führen, dass wir es schnell beenden können, ist das klar?"

Harry nickte, und - im Licht der einzigen Taschenlampe, das zum rötlichen Ende des optischen Spektrums hin schattiert war, reflektierten die grünen Schuppen der Schlange nicht sehr stark, und die blau-weiße Bänderung kaum mehr. Dunkel erschien die Schlange in diesem Licht. Die Augen, die zuvor wie graue Gruben gewirkt hatten, reflektierten nun das Taschenlampenlicht und schienen heller als der Rest der Schlange.

"\textbf{\emph{So}}", zischte die giftige Kreatur. "\textbf{\emph{Was wolltest du ssssagen}}?"

Und Harry zischte: "\emph{Sssschulmeissster denkt, dass der ehemalige Herr dieser Frau derjenige ist, der sie aus dem Gefängisss gestohlen hat.}"

Diesmal hatte Harry nachgedacht, und zwar gründlich, bevor er beschlossen hatte, Professor Quirrell nur zu verraten, dass der Schulleiter das glaubte; und nichts über die Prophezeiung zu sagen, die Voldemort auf Harrys Eltern angesetzt hatte, und auch nicht, dass der Schulleiter den Orden des Phönix neu konstituieren würde…. es war ein Risiko, ein erhebliches Risiko, aber Harry brauchte einen Verbündeten in dieser Sache.

"\textbf{\emph{Er glaubt das er am Leben isssst}}?!", sagte die Schlange schließlich. Die geteilte, zweizinkige Zunge flackerte schnell von einer Seite zur anderen, ein sardonisches Schlangenlachen. "\textbf{\emph{Irgendwie bin ich nicht überasssscht.}}"

"\emph{Ja}", zischte Harry trocken, "\emph{sehr amüsant, da bin ich ssicher. Nur dass ich jetzt für die nächsten sechs Jahre in Hogwarts festsitze, zur ssSicherheit! Ich habe beschlossen, dass ich in der Tat nach Macht streben werde; und dafür ist die Gefangenschaft nicht hilfreich. Ich muss den Schulmeister davon überzeugen, dass der Dunkle Lord noch nicht erwacht ist, dass das Entkommen das Werk einer anderen Macht war -}"

Wieder das schnelle Flackern der Schlangenzunge; das schlangenhafte Lachen war diesmal stärker, trockener. "\textbf{\emph{Amateurhafte Narrheit}}."

"\emph{Pardon}?", zischte Harry.

"\textbf{\emph{Du siehst einen Fehler, denkst an das Rückgängigmachen, an das Zurücksetzen der Zeit auf den Anfang. Doch nicht einmal mit Sanduhren kann man die Zeit rückgängig machen. Man muss stattdessen vorwärts gehen. Du denkst daran, andere zu überzeugen, dass sie einen Fehler gemacht haben. Es ist viel einfacher, sie zu überzeugen, dass sie recht haben. Also überleg mal, Junge: Welche neue Begebenheit würde Schulmeissster dazu bringen, zu entscheiden, dass du wieder ssssicher bist, ssimultan deine anderen Agendass vorantreiben}}?"

Harry starrte die Schlange verwirrt an. Sein Verstand versuchte, das Rätsel zu begreifen und zu enträtseln -

"\textbf{\emph{Isst es nicht offensichtlichs?}}", zischte die Schlange. Wieder flackerte ein sardonisches Lachen über die Zunge. "\textbf{\emph{Um dich zu befreien, um die Macht in Britannien zu erlangen, musst du erneut den Dunklen Lord besiegen}}."

Im rötlich-orange flackernden Fackellicht schwankte eine grüne Schlange über einem weißen Krankenhausbett, während der Junge in die Glut ihrer Augen starrte.

"\emph{Sssso}", sagte Harry schließlich. "\emph{Lasss uns klar sssein was du verssschlägst. Du ssssagst, dass wir einen Schausssspieler den Dunklen Lord darsssstellen lasssen.}"

"\textbf{\emph{Ssowas in der Art. Die Frau, die wir gerettet haben, wird kooperieren, ssollte am überzeugendsten sein, wenn ssie an seiner Seite sseht.}}" Noch mehr sardonisches Zungenschnalzen. "\textbf{\emph{Du wirsssst aus Hogwarts an einen öffentlichen Ort entführt, viele Zeugen, Mündel halten sich schützend vor dich. Der Dunkle Lord verkündet, dass er endlich seine physische Form wiedererlangt hat, nachdem er jahrelang als Geist umhergezogen ist; er sagt, dass er eine noch größere Macht erlangt hat, nicht einmal du kannsssst ihn jetzt aufhalten. Er bietet Dir ein Duell an. Du setzen den Patronusss ein, der Dunkle Lord lacht dich aus und sagt, er sei kein Lebensfresser. Tötungsfluch auf dich, du blockst, Beobachter sehen Dunklen Lord explodieren -}}"

"\emph{Zaubert Todessssfluch?}" Harry zischte ungläubig. "\emph{Auf mich? Schon wieder? zweites Mal? Niemand wird glauben, dass der Dunkle Lord möglicherweise so dumm sein könnte} -"

"\textbf{\emph{Du und ich sind die einzigen beiden Menschen im Land, die das bemerken würden}}", zischte die Schlange. "\textbf{\emph{Verlass dich auf mich, Junge.}}"

"\emph{Was ist, wenn es einen Dritten gibt, irgendwie}?"

Die Schlange wiegte sich nachdenklich. "\textbf{\emph{Ich könnte ein anderesss Drehbuch für das Spiel schreiben, wenn du willst. Welches ssSzenario auch immer, ssollte die Möglichkeit offen lassen, dass der Dunkle Lord noch einmal zurückkehren könnte - das Volk muss denken, dass es immer noch darauf angewiesen ist, dass du es beschützt."}}

Harry starrte in die rot flackernden Gruben der Schlangenaugen.

"\textbf{\emph{Nun}}?", zischte die schwankende Gestalt.

Der offensichtliche Gedanke war, dass es genau dieselbe Art von Dummheit wäre, sich ein zweites Mal auf die Intrigen und Täuschungen des Verteidigungsprofessors einzulassen, eine noch kompliziertere Lüge zu spinnen, um den ersten Fehler zu vertuschen, und eine weitere tödliche Schwachstelle zu schaffen, falls irgendjemand jemals die Wahrheit herausfinden würde, wie der vermeintliche Dunkle Lord, der den Tötungsfluch erneut anwendet.

\emph{Es bedurfte nicht einmal seiner Hufflepuff-Seite, um darauf hinzuweisen,} dachte Harry mit seiner eigenen mentalen Stimme. \emph{Aber es stellte sich auch die Frage, ob die angemessene Moral, um aus der letzten Erfahrung zu lernen, darin bestand, immer sofort Nein zu dem Verteidigungsprofessor zu sagen, oder..}

"Werde darüber nachdenken", zischte Harry. "Will nicht gleich eine Antwort geben, diesmal, will zuerst Risiken und Vorteile aufzählen -"

"\textbf{\emph{Verstanden}}", zischte die Schlange. "\textbf{\emph{Aber merke dir eins, Junge, andere Ereignisse gehen ohne dich weiter. Zögern ist immer leicht, selten nützlich}}." D

er Junge trat aus dem Privatzimmer in die Hauptkrankenstation und fuhr sich nervös mit den Fingern durch sein unordentliches schwarzes Haar, während er an den weißen Betten vorbeiging, die belegt und unbesetzt waren. Kurz darauf verließ der Junge den Hogwarts-Krankenraum ganz und ging mit einem verwirrten Nicken an Madam Pomfrey vorbei nach draußen. Der Junge ging hinaus in einen Korridor, dann in einen größeren Korridor, und blieb dann stehen und lehnte sich an die Wand.

Die Sache war die… … er wollte wirklich nicht die nächsten sechs Jahre in Hogwarts festsitzen; und wenn man darüber nachdachte… … der Vorfall mit der Rettung von Bellatrix aus Askaban war nicht nur für Harry mit Kosten verbunden. Andere Leute würden sich Sorgen machen, in Angst vor der Rückkehr des Dunklen Lords leben, unbekannte Ressourcen aufwenden, um unbekannte Vorsichtsmaßnahmen zu treffen.

Harry könnte verlangen, dass sie das Drehbuch so schreiben, dass es nicht plausibel erscheint, dass der Dunkle Lord ein drittes Mal zurückkehren würde. Und dann würden sich die Leute entspannen, es wäre alles vorbei.

\emph{Es sei denn, es gäbe tatsächlich einen Dunklen Lord, vor dem man sich fürchten müsste. Es hatte eine Prophezeiung gegeben.}

Der Junge, der an der Wand lehnte, stieß einen leisen Seufzer aus und begann wieder zu gehen. Harry hatte es fast vergessen, aber er war dazu gekommen, Professor Quirrell das Kartenspiel zu zeigen, das er am Sonntagabend vom "\emph{Weihnachtsmann}" bekommen hatte, und in dem der Herzkönig angeblich ein Portschlüssel war, der ihn zum Hexeninstitut von Salem in Amerika bringen würde. Natürlich hatte Harry Professor Quirrell weder gesagt, wer ihm die Karte geschickt hatte, noch was sie bewirken sollte, bevor er Professor Quirrell gefragt hatte, ob es möglich sei zu sagen, wohin der Portschlüssel ihn schicken würde. Der Verteidigungsprofessor hatte sich in seine menschliche Gestalt zurückverwandelt und untersuchte den Herzkönig, indem er ein paar Mal mit seinem Zauberstab darauf tippte. Und laut Professor Quirrell… … würde der Portschlüssel den Benutzer irgendwo nach London schicken, aber näher konnte er es nicht bestimmen. Harry hatte Professor Quirrell den Zettel gezeigt, der dem Kartenspiel beilag, und nichts von den früheren Notizen gesagt. Professor Quirrell hatte ihn mit einem Blick aufgenommen, ein trockenes Kichern von sich gegeben und bemerkt, dass, wenn man den Zettel genau las, nicht explizit stand, dass der Portschlüssel ihn zum Hexeninstitut von Salem bringen würde. Man musste lernen, auf diese Art von Feinheiten zu achten, sagte Professor Quirrell, wenn man ein mächtiger Zauberer werden wollte, wenn man erwachsen war; oder, in der Tat, wenn man überhaupt erwachsen werden wollte.

Der Junge seufzte wieder, als er sich zum Unterricht schleppte. Er begann sich zu fragen, ob alle anderen Zaubererschulen auch so waren, oder ob nur Hogwarts ein Problem hatte.

