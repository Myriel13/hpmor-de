

\hypertarget{gruppenarbeit-teil-2}{% \section{31. Gruppenarbeit Teil 2}\label{gruppenarbeit-teil-2}}

\textbf{\uline{Gruppenarbeit, Teil 2}}

\textbf{Nachspiel}:

Harry schritt in seinem Büro hin und her, das sich wunderbar zum herumschreiten eignete, und eine andere Verwendung hatte es nicht, soweit er es beurteilen konnte.

\emph{Wie?}

\textbf{\emph{Wie?!}}

\emph{Hermine hätte diese Schlacht nicht gewinnen dürfen! Nicht beim ersten Versuch, nicht, wenn sie von Natur aus überhaupt nicht gewalttätig war, automatisch eine große Militärkommandantin zu sein, war selbst für sie zu viel.}

\emph{\hfill\break Hatte sie über diese Taktik in einem Militärgeschichtsbuch gelesen? Aber es war nicht nur diese eine Taktik gewesen, sie hatte ihre Streitkräfte perfekt positioniert, um jeden Rückzug zu blockieren, ihre Truppen waren besser koordiniert gewesen als seine oder Dracos… Hatte Professor Quirrell sein Versprechen gebrochen, ihr nicht zu helfen? Hatte er ihr das Tagebuch von General Tacticus gegeben oder so?}

Harry fehlte hier etwas, etwas wirklich Wichtiges, und seine Gedanken drehten sich immer wieder im Kreis, und er konnte es immer noch nicht herausfinden.

Schließlich seufzte Harry. Er kam bei dieser Sache nicht weiter und er musste vor dem nächsten Kampf den Schildbrecher Fluch von Hermine oder jemand anderem lernen - Professor Quirrell hatte Harry erklärt, seine Stimme amüsiert, aber mit einem scharfen Unterton der Warnung, dass \emph{"keine magischen Gegenstände außer denen, die ich dir gebe"} Muggeltechnologie einschloss, egal wie sehr das keine Magie war.

Außerdem musste Harry sich überlegen, wie er Mr. Goyle das nächste Mal zur Strecke bringen konnte… Kämpfe erzielten eine Menge Quirrell-Punkte, wenn man ein General war, und Harry musste sich beeilen, wenn er Professor Quirrells Weihnachtswunsch erfüllen wollte.

In seinem Privatraum in Slytherin starrte Draco Malfoy ins Leere, als wäre die Wand vor seinem Schreibtisch die faszinierendste Fläche der Welt.

\emph{Wie?}

\textbf{\emph{Wie?!}}

Im Nachhinein betrachtet war es eine offensichtliche Idee gewesen, und eine gerissene Intrigen, aber Granger sollte nicht gerissen sein!

Sie war so sehr Hufflepuff das Sie nicht auf ihn im Unterricht geschossen hatte! Hatte Professor Quirrell sie trotz seines Versprechens beraten, oder…

Und dann tat Draco endlich das, was er schon viel früher hätte tun sollen. Was er nach der ersten Begegnung mit Granger hätte tun sollen.

Das, was Harry Potter ihm gesagt und beigebracht hatte, und doch hatte Harry Draco auch gewarnt, dass es Zeit brauchen würde, bis sein Gehirn begriff, dass die Methoden auf das wirkliche Leben anwendbar waren, und das hatte Draco bis heute nicht verstanden.

Er hätte jeden einzelnen seiner Fehler vermeiden können, wenn er einfach die Dinge angewandt hätte, die Harry ihm bereits gesagt hatte - Draco sagte laut:

"Ich merke, dass ich verwirrt bin."

\emph{Deine Stärke als Rationalist ist deine Fähigkeit, von der Fiktion mehr verwirrt zu sein als von der Realität.}

.. Draco war verwirrt. Deshalb war etwas, was er glaubte, \emph{Fiktion}.

Granger hätte dazu nicht in der Lage sein dürfen.

\emph{Deshalb hatte sie es wohl nicht getan.}

\emph{Ich verspreche, General Granger nicht auf eine Weise zu helfen, von der Sie beide nichts wissen.}

Mit plötzlicher entsetzter Erkenntnis fegte Draco Papiere beiseite und jagte durch das Durcheinander auf seinem Schreibtisch, bis er es fand.

Und da war es. Direkt in der Liste der Leute und Ausrüstung, die jeder der drei Armeen zugeteilt waren.

\emph{Verflucht sei Professor Quirrell!}

Draco hatte es gelesen, und er hatte es immer noch nicht gesehen -

Das nachmittägliche Sonnenlicht ergoss sich in das Büro des Sonnenscheinregiments und beleuchtete General Granger in ihrem Stuhl, als würde sie mit einer goldenen Aura glühen.

"Was glauben Sie, wie lange wird Malfoy brauchen, um es herauszufinden?", fragte General Granger.

"Nicht lange", sagte Oberst Blaise Zabini.

"Vielleicht weiß er es schon. Wie lange wird Potter brauchen, um es herauszufinden?"

"Ewig", sagte General Granger,

"es sei denn, Malfoy sagt es ihm, oder einer seiner eigenen Soldaten findet es heraus.

Harry Potter denkt einfach nicht so."

"Wirklich?", sagte Hauptmann Ernie Macmillan und blickte von einem der Ecktische auf, an dem er von Hauptmann Ron Weasley beim Schach erdrückt wurde.

(Sie hatten natürlich alle anderen Stühle zurückgebracht, nachdem Malfoy gegangen war.)

"Ich meine, es scheint mir ziemlich offensichtlich zu sein. Wer würde schon versuchen ganz alleine auf all die Ideen zu kommen?"

"Harry", sagte Hermine, genau zu dem Zeitpunkt, als Zabini sagte: "Malfoy."

"Malfoy denkt, er sei viel besser als alle anderen", sagte Zabini.

"Und Harry… sieht die meisten anderen Leute nicht so", sagte Hermine. Eigentlich war es irgendwie traurig. Harry war sehr, sehr allein aufgewachsen. Es war nicht so, dass er herumlief und in Worten dachte, dass nur Genies eine Daseinsberechtigung hätten. Es kam ihm einfach nicht in den Sinn, dass irgendjemand außer Hermine irgendwelche guten Ideen haben könnte.

"Jedenfalls", sagte Hermine. "Captain Goldstein und Weasley, ihr seid dafür zuständig, euch strategische Ideen für unsere nächste Schlacht auszudenken.

Hauptleute Macmillan und Susan - pardon, ich meine Macmillan und Bones - lassen sich ein paar Taktiken einfallen, die wir anwenden können, außerdem jedes Training, von dem ihr meint, dass wir es versuchen sollten. Oh, und Glückwunsch zu Ihrem Marschlied, Hauptmann Goldstein, ich denke, das war ein großes Plus für den Esprit de Corps."

"Und was machst du und Oberst Zabini?", fragte Susan.

Hermine erhob sich von ihrem Stuhl und streckte sich.

"Ich werde versuchen, herauszufinden, was Harry Potter denkt, und Oberst Zabini wird versuchen, herauszufinden, was Draco Malfoy tun könnte, und wir beide werden uns wieder zu euch gesellen, wenn wir etwas herausgefunden haben. Ich werde spazieren gehen, während ich nachdenke. Zabini, willst du mitkommen?"

"Ja, General", sagte Zabini steif.

Es war nicht als Befehl gemeint gewesen. Hermine seufzte ein wenig vor sich hin. Daran musste sie sich erst einmal gewöhnen, und obwohl Zabinis erste Idee durchaus funktioniert hatte, war sie sich nicht ganz sicher, ob Professor Quirrells zitierte Mischung aus positiven und negativen Anreizen ausreichen würde, um den Slytherin bis Dezember, wenn Verräter zum ersten Mal erlaubt sein würden, ganz auf ihrer Seite zu haben… Sie hatte auch noch keine Ahnung, was sie mit Professor Quirrells Weihnachtswunsch anfangen sollte.

\emph{\hfill\break Vielleicht würde sie einfach Mandy fragen, ob sie sich etwas wünschte, wenn die Zeit gekommen war.}

