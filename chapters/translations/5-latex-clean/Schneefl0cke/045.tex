

\hypertarget{humanismus-teil-4}{% \section{46. Humanismus Teil 4}\label{humanismus-teil-4}}

\textbf{\uline{Humanismus, Teil 4}}

Der letzte Zipfel der Sonne sank unter den Horizont, das rote Licht verblasste in den Baumkronen, nur der blaue Himmel beleuchtete die sechs Menschen, die auf dem wintertrockenen und schneegefleckten Gras standen, in der Nähe eines leeren Käfigs, auf dessen Boden ein leerer, zerfledderter Mantel lag.

Harry fühlte sich… nun ja, wieder normal. Gesund. Der Zauber hatte den Tag und seine Schäden nicht ungeschehen gemacht, hatte die Verletzungen nicht so gemacht, als wären sie nie da gewesen, aber seine Verletzungen waren… \emph{verbunden, gelindert?} Es war schwer zu beschreiben.

Auch Dumbledore sah gesünder aus, wenn auch nicht völlig genesen.\\ Der Kopf des alten Zauberers drehte sich für einen Moment, blickte Professor Quirrell in die Augen und sah dann wieder zu Harry.\\ "Harry", sagte Dumbledore, "bist du kurz davor, vor Erschöpfung zusammenzubrechen und möglicherweise zu sterben?"

"Nein, merkwürdigerweise nicht", sagte Harry.\\ "Das hat etwas aus mir herausgeholt, aber viel weniger, als ich dachte."\\ \emph{Oder vielleicht hat es auch etwas zurückgegeben und etwas genommen …}\\ "Ehrlich gesagt hatte ich erwartet, dass mein Körper jetzt mit einem dumpfen Schlag auf dem Boden aufschlagen würde."

\emph{Es gab ein deutliches Körper-schlägt-auf-den-Boden-mit-einem-Knall-Geräusch.}

"Danke, dass Sie sich darum gekümmert haben, Quirinus", sagte Dumbledore zu Professor Quirrell, der nun über und hinter den bewusstlosen Körpern der drei Auroren stand.

"Ich gestehe, ich fühle mich immer noch ein wenig kribbelig. Aber ich werde mich selbst um die Gedächtniszauber kümmern."

Professor Quirrell legte den Kopf schief und sah dann Harry an.\\ "Ich werde eine Menge nutzloser Ungläubigkeit auslassen", sagte Professor Quirrell,\\ "Bemerkungen, dass Merlin selbst das nicht geschafft hat, und so weiter.\\ Lass uns gleich zur wichtigen Frage übergehen. Was die süßen, schlängelnden Schlangen war das?"

"Der Patronus-Zauber", sagte Harry. "Version 2.0."

"Ich freue mich zu sehen, dass du wieder dein übliches Ich bist", sagte Dumbledore.\\ "Aber du gehst nirgendwo hin, junger Ravenclaw, bis du mir sagst, was genau dieser warme und glückliche Gedanke war."

"Hm …", sagte Harry. Er tippte mit einem Finger nachdenklich auf seine Wange.\\ "Ich frage mich, ob ich das sollte?"

Professor Quirrell grinste plötzlich.

"Bitte?", sagte der Schulleiter. "Bitte mit Zucker obendrauf?"

Harry verspürte einen Impuls und entschied sich, ihm zu folgen. Es war gefährlich, aber eine bessere Gelegenheit würde es vielleicht bis zum Ende der Zeit nicht mehr geben.

"Drei Limonaden", sagte Harry zu seinem Beutel und blickte dann zu dem Verteidigungsprofessor und dem Schulleiter von Hogwarts auf.\\ "Meine Herren", sagte Harry, "ich habe diese Limonaden bei meinem ersten Besuch auf Bahnsteig Neun und drei Viertel gekauft, an dem Tag, an dem ich in Hogwarts eingetreten bin. Ich habe sie für besondere Anlässe aufbewahrt; sie sind mit einem kleinen Zauber versehen, der sicherstellt, dass sie zur richtigen Zeit getrunken werden. Das ist der letzte meiner Vorräte, aber ich glaube nicht, dass es jemals eine schönere Gelegenheit geben wird. Sollen wir?"

Dumbledore nahm eine Limonadendose von Harry, und Harry warf Professor Quirrell eine weitere zu. Die beiden älteren Männer murmelten jeweils einen identischen Zauberspruch über die Dose und runzelten kurz die Stirn über das Ergebnis. Harry seinerseits kippte einfach den Deckel auf und trank. Der Verteidigungsprofessor und der Schulleiter von Hogwarts taten es ihm höflich nach.

Harry sagte: "Ich dachte an meine absolute Ablehnung des Todes als Teil der natürlichen Ordnung."

\emph{Es war vielleicht nicht die richtige Art von warmem Gefühl, die man brauchte, um einen Patronus-Zauber zu wirken, aber es kam trotzdem in Harrys Top 10.}

Die Blicke, die er vom Verteidigungsprofessor und vom Schulleiter erntete, machten Harry kurz nervös, als die verschüttete Limolasste; aber dann sahen sich die beiden gegenseitig an und entschieden offenbar, dass sie nicht damit durchkommen würden, Harry in der Gegenwart des anderen etwas wirklich Schlimmes anzutun.

"Mr. Potter", sagte Professor Quirrell, "selbst ich weiß, dass die Dinge so nicht funktionieren sollten."

"In der Tat", sagte Dumbledore. "Erklär es."

Harry öffnete den Mund, und dann, als die Erkenntnis ihn traf, klappte er ihn schnell wieder zu.\\ \emph{Godric hatte es niemandem erzählt, auch Rowena nicht, wenn sie es gewusst hätte; es hätte eine ganze Reihe von Zauberern geben können, die es herausgefunden und ihren Mund gehalten hätten. Man konnte nicht vergessen, wenn man wusste, dass es das war, was man versuchte; wenn man einmal begriffen hatte, wie es funktionierte,} \textbf{\emph{würde die Tierform des Patronus-Zaubers nie wieder für einen funktionieren}} \emph{- und die meisten Zauberer hatten nicht die richtige Erziehung, um sich gegen Dementoren zu wenden und sie zu vernichten -}

"Ähm, entschuldige bitte", sagte Harry.\\ "Aber mir ist gerade klar geworden, dass es eine unglaublich schlechte Idee wäre, euch das zu erklären, bis ihr ein paar Dinge selbst geklärt habt."

"Ist das die Wahrheit, Harry?" sagte Dumbledore langsam.\\ "Oder tust du nur so, als wärst du weise -"

"Schulleiter!?", sagte Professor Quirrell und klang aufrichtig schockiert.\\ "Mr. Potter hat Ihnen gesagt, dass über diesen Zauberspruch nicht mit denen gesprochen wird, die ihn nicht sprechen können! Man bedrängt einen Zauberer nicht in solchen Angelegenheiten!"

"Wenn ich Ihnen sagen würde -" begann Harry.

"Nein", sagte Professor Quirrell und klang dabei ziemlich streng.\\ "Sie sagen uns nicht, warum, Mr. Potter, Sie sagen uns nur, dass wir es nicht wissen sollen. Wenn Sie sich eine Andeutung ausdenken wollen, dann tun Sie das vorsichtig, in aller Ruhe, nicht mitten in der Unterhaltung."

Harry nickte.

"Aber", sagte der Schulleiter. "Aber, aber was soll ich dem Ministerium sagen? Man kann doch nicht einfach einen Dementor verlieren!"

"Sagen Sie ihnen, dass ich ihn gegessen habe", sagte Professor Quirrell und brachte Harry dazu, sich an der Limonade zu verschlucken, die er gedankenlos an seine Lippen gehoben hatte.\\ "Es macht mir nichts aus. Sollen wir zurückgehen, Mr. Potter?"

Die beiden machten sich auf den Weg zurück nach Hogwarts und ließen Albus Dumbledore zurück, der verzweifelt auf den leeren Käfig starrte und die drei schlafenden Auroren, die auf ihren Erinnerungszauber warteten.

\textbf{Nachspiel, Harry Potter und Professor Quirrell:}\\ Sie gingen eine Weile, bevor Professor Quirrell sprach, und alle Hintergrundgeräusche wurden stumm, als er es tat.

"Du bist außergewöhnlich gut darin, Dinge zu töten, mein Schüler", sagte Professor Quirrell.

"Danke", sagte Harry aufrichtig.

"Ich bin nicht neugierig", sagte Professor Quirrell, "aber für den unwahrscheinlichen Fall, dass es nur der Schulleiter war, dem Sie das Geheimnis nicht anvertraut haben …?"

Harry dachte darüber nach. Professor Quirrell konnte schon den Tierpatronus-Zauber nicht zaubern. Aber man konnte ein Geheimnis nicht ungeschehen machen, und Harry lernte schnell genug, um zu erkennen, dass er zumindest eine Weile nachdenken sollte, bevor er dieses auf die Welt losließ.

Harry schüttelte den Kopf, und Professor Quirrell nickte zustimmend.\\ "Aus reiner Neugier, Professor Quirrell", sagte Harry, "wenn es Teil eines bösen Plans gewesen wäre, den Dementor nach Hogwarts zu bringen, was wäre dann sein Ziel gewesen?"

"Dumbledore zu ermorden, während er geschwächt war", sagte Professor Quirrell, ohne auch nur zu zögern.\\ "Hm. Der Schulleiter hat Ihnen gesagt, dass er mich verdächtigt hat?"

Harry sagte eine Sekunde lang nichts, während er versuchte, sich eine Antwort zu überlegen, und gab dann auf, als ihm klar wurde, dass er bereits geantwortet hatte.

"Interessant…" sagte Professor Quirrell. "Mr. Potter, es ist nicht auszuschließen, dass heute ein Komplott am Werk war. Dass Ihr Zauberstab so nahe am Käfig des Dementors gelandet ist, könnte ein Unfall gewesen sein. Oder einer der Auroren könnte imperisiert, konfundiert oder legilimiert worden sein, um einen Einfluss auszuüben. Flitwick und ich sollten als Verdächtige nicht ausgeschlossen werden, nach lhrer Rechnung. Man beachte, dass Professor Snape heute alle seine Vorlesungen abgesagt hat, und ich vermute, dass er mächtig genug ist, um sich selbst zu desillusionieren; die Auroren haben schon früh Erkennungszauber gewirkt, aber sie haben sie nicht unmittelbar vor Ihrem Zug wiederholt. Aber am ehesten, Mr. Potter, könnte die Tat von Dumbledore selbst ausgeheckt worden sein; und wenn er es getan hat, dann könnte er auch im Voraus Schritte unternehmen, um Ihren Verdacht auf andere zu lenken."

Sie gingen ein paar Schritte weiter.

"Aber warum sollte er das tun?" fragte Harry.

Der Verteidigungsprofessor schwieg einen Moment und sagte dann:\\ "Mr. Potter, welche Schritte haben Sie unternommen, um den Charakter des Schulleiters zu untersuchen?"

"Nicht viele", sagte Harry.\\ Ihm war es erst kürzlich klar geworden…\\ "Nicht annähernd genug."

"Dann stelle ich fest", sagte Professor Quirrell, "dass man nicht alles über einen Menschen herausfindet, wenn man nur seine Freunde fragt."

Nun war es an Harry, ein paar Schritte schweigend auf dem leicht ausgetretenen Feldweg zu gehen, der zurück nach Hogwarts führte.

Er hätte es eigentlich schon besser wissen müssen. Bestätigungsfehler war der Fachausdruck; er bedeutete unter anderem, dass man bei der Wahl seiner Informationsquellen eine auffällige Tendenz hatte, Informationsquellen zu wählen, die mit seiner aktuellen Meinung übereinstimmten.

"Danke", sagte Harry. "Eigentlich … habe ich es vorhin nicht gesagt, oder? Ich danke Ihnen für alles. Sollte Sie jemals wieder ein Dementor bedrohen oder auch nur leicht verärgern, lassen Sie es mich wissen, und ich werde ihn Mister Glowy Person\\ vorstellen. Ich mag es nicht, wenn Dementoren meine Freunde leicht verärgern."

Das brachte ihm einen unverständlichen Blick von Professor Quirrell ein.\\ "Du hast den Dementor zerstört, weil er mich bedroht hat?"

"Ähm", sagte Harry, "ich hatte mich schon vorher irgendwie dazu entschlossen, aber ja, das allein wäre schon Grund genug gewesen."

"Ich verstehe", sagte Professor Quirrell.\\ "Und was hättest du gegen die Bedrohung für mich getan, wenn dein Zauber zur Zerstörung des Dementors nicht funktioniert hätte?"

"Plan B", sagte Harry. "Den Dementor in dichtes Metall mit einem hohen Schmelzpunkt einhüllen, wahrscheinlich Wolfram, ihn in einen aktiven Vulkan werfen und hoffen, dass er im Erdmantel landet. Ah, der ganze Planet ist unter seiner Oberfläche mit geschmolzener Lava gefüllt -"

"Ja", sagte Professor Quirrell. "Ich weiß."

Der Verteidigungsprofessor trug ein sehr seltsames Lächeln auf den Lippen.\\ "Daran hätte ich eigentlich selbst denken müssen, alles in allem. Sagen Sie mir, Mr. Potter, wenn Sie etwas dort verlieren wollten, wo es niemand jemals wiederfinden würde, wo würden Sie es hinlegen?"

Harry dachte über diese Frage nach.

"Ich nehme an, ich sollte nicht fragen, was Sie gefunden haben, das verloren gehen muss -"

"Durchaus", sagte Professor Quirrell, wie Harry es erwartet hatte; und dann:\\ "Vielleicht erfährst du es, wenn du älter bist", was Harry nicht getan hatte.

"Nun", sagte Harry, "außer zu versuchen, es in den geschmolzenen Kern des Planeten zu bekommen, könnte man es in festem Gestein einen Kilometer unter der Erde an einem zufällig ausgewählten Ort vergraben - vielleicht teleportiert man es hinein, wenn es eine Möglichkeit gibt, das blind zu tun, oder man bohrt ein Loch und repariert es hinterher; das Wichtigste wäre, keine Spuren zu hinterlassen, die dorthin führen, so dass es nur ein anonymer Kubikmeter irgendwo in der Erdkruste ist.\\ Man könnte ihn in den Marianengraben werfen, das ist die tiefste Stelle des Ozeans auf dem Planeten - oder man wählt irgendeinen anderen Ozeangraben, damit es nicht so auffällt. Wenn man es luftdicht und unsichtbar machen könnte, dann könnte man es in die Stratosphäre werfen. Oder idealerweise würde man es in den Weltraum schießen, mit einer Tarnung gegen Entdeckung und einem zufällig fluktuierenden Beschleunigungsfaktor, der es aus dem Sonnensystem herausbringen würde.\\ Und danach würden Sie sich natürlich selbst die eigene Erinnerung löschen, so dass nicht einmal Sie genau wüssten, wo es ist."

Der Verteidigungsprofessor lachte, und es klang noch seltsamer als sein Lächeln.

"Professor Quirrell?" sagte Harry.

"Alles ausgezeichnete Vorschläge", sagte Professor Quirrell.\\ "Aber sagen Sie mir, Mr. Potter, warum genau diese fünf?"

"Hm?", sagte Harry. "Sie schienen mir einfach die naheliegendsten Ideen zu sein."

"Ach?", sagte Professor Quirrell.\\ „Aber es gibt ein interessantes Muster in ihnen, sehen Sie. Man könnte sagen, es klingt wie eine Art Rätsel. \emph{(„Something like a Riddle“, anm. des Übersetzers :-) )}\\ Ich muss zugeben, Mr. Potter, dass es zwar seine Höhen und Tiefen hatte, aber im Großen und Ganzen war dies ein überraschend guter Tag."

Und sie gingen weiter den Weg hinunter, der zu den Toren von Hogwarts führte, in einigem Abstand voneinander; denn Harry hielt sich, ohne darüber nachzudenken, automatisch weit genug von dem Verteidigungsprofessor entfernt, um nicht dieses Gefühl des Unheils auszulösen, das aus irgendeinem Grund im Moment ungewöhnlich stark zu sein schien.

\textbf{Nachspiel, Daphne Greengrass:}\\ Hermine hatte sich geweigert, irgendwelche Fragen zu beantworten, und sobald sie den Spalt, der zu den Slytherin-Kerkern führte, passiert hatten, waren Daphne und Tracey sofort losgelaufen, so schnell sie nur konnten. Gerüchte verbreiteten sich in Hogwarts schnell, also mussten sie sofort in die Kerker gehen, wenn sie die Ersten sein wollten, die allen die Geschichte erzählten.

"Denkt dran", sagte Daphne, "platzt nicht gleich mit dem Kuss heraus, sobald wir reinkommen, okay? Es funktioniert besser, wenn wir die ganze Geschichte der Reihe nach erzählen."

Tracey nickte aufgeregt.

Und kaum waren sie in den Slytherin-Gemeinschaftsraum geplatzt, holte Tracey Davis tief Luft und rief:\\ "\textbf{Alle! Harry Potter konnte den Patronus-Zauber nicht wirken und der Dementor hat ihn fast gefressen und Professor Quirrell hat ihn gerettet, aber dann war Potter ganz böse, bis Granger ihn mit einem Kuss zurückgebracht hat! Das ist sicher wahre Liebe!}"

Das hätte Sie kommen sehen müssen, dachte Daphne.

Die Nachricht löste nicht die erwartete Reaktion aus. Die meisten Mädchen warfen einen Blick hinüber und blieben dann in ihren Sofas sitzen, oder die Jungen lasen einfach in ihren Stühlen weiter.

"Ja", sagte Pansy säuerlich, von wo aus sie mit Gregorys Füßen in ihrem Schoß saß, sich zurücklehnte und etwas las, was ein Malbuch zu sein schien,\\ "Millicent hat es uns schon erzählt."

\emph{Wie -}

"Warum hast du ihn nicht zuerst geküsst, Tracey?", sagten Flora und Hestia Carrow von ihren eigenen Stühlen aus.\\ "Jetzt wird Potter ein Schlammblutmädchen heiraten! Du hättest seine wahre Liebe sein können und in ein reiches Adelshaus kommen können und alles, wenn du ihn einfach zuerst geküsst hättest!"

Traceys Gesicht war ein Bild in fassungsloser Erkenntnis.

"Was?!", kreischte Daphne. "So funktioniert Liebe nicht!"

"Natürlich tut sie das", erklärte Millicent von dort aus, wo sie gerade eine Art Zauberspruch übte, während sie aus dem Fenster auf das wirbelnde Wasser des Hogwarts-Sees blickte.\\ "Der erste Kuss bekommt den Prinzen."

"Es war nicht ihr erster Kuss!", rief Daphne. "Hermine war bereits seine wahre Liebe! Deshalb konnte sie ihn zurückbringen!"

Da wurde Daphne klar, was sie gerade gesagt hatte, und sie zuckte innerlich zusammen, aber wie heißt es so schön: Man musste wirklich aufpassen was man sagte.

"Moment mal, was?", sagte Gregory und schwang seine Füße von Pansys Schoß.\\ "Was ist das? Miss Bulstrode hat den Teil nicht erzählt."

Alle anderen sahen nun auch Daphne an.\\ "Oh, ja", sagte Daphne, "Harry hat sie weggeschubst und geschrien: '\emph{Ich habe dir gesagt, nicht Küssen!'} Dann schrie Harry, als würde er sterben, und Fawkes fing an, ihm etwas vorzusingen - ich bin mir nicht sicher, was davon zuerst passiert ist -"

"Das hört sich für mich nicht nach wahrer Liebe an", sagten die Carrow-Zwillinge.\\ "Das klingt, als hätte ihn die falsche Person geküsst."

"Ich hätte es sein sollen", flüsterte Tracey.\\ Ihr Gesicht war immer noch fassungslos.\\ "Ich hätte seine wahre Liebe sein sollen. Harry Potter war mein General.\\ Ich hätte, ich hätte mit Granger um ihn kämpfen sollen -"

Daphne drehte sich wütend auf Tracey.\\ "Du? Harry von Hermine wegnehmen?"

"Ja!", sagte Tracey. "Ich!"

"Du spinnst", stellte Daphne mit Überzeugung fest. "Selbst wenn du ihn zuerst geküsst hättest, weißt du, was das aus dir machen würde? Das traurige, kleine, verliebte Mädchen, das am Ende des zweiten Aktes stirbt."

"Das nimmst du zurück!", rief Tracey.

In der Zwischenzeit hatte Gregory den Raum durchquert, wo Vincent gerade seine Hausaufgaben machte.\\ "Mr. Crabbe", sagte Gregory mit leiser Stimme, "ich glaube, Mr. Malfoy muss davon erfahren."

\textbf{Nachspiel, Hermine Granger:}\\ Hermine starrte auf das mit Wachs versiegelte Papier, auf dessen Oberfläche einfach die Zahl 42 eingraviert war.\\

\emph{Ich habe herausgefunden, warum wir den Patronus-Zauber nicht wirken konnten, Hermine, es hat nichts damit zu tun, dass wir nicht glücklich genug waren.\\ Aber ich kann's dir nicht sagen. Ich kann es nicht mal dem Schulleiter sagen. Es muss noch geheimer sein als die Teilverwandlung. Jedenfalls im Moment. Aber falls du mal Dementoren bekämpfen musst, das Geheimnis steht hier, kryptisch, so dass, wenn jemand nicht weiß, dass es um Dementoren und den Patronus-Zauber geht, er nicht weiß, was es bedeutet…}

Sie hatte Harry erzählt, dass sie ihn sterben sah, dass ihre Eltern starben, dass all ihre Freunde starben, dass alle starben. Sie hatte ihm nicht von ihrer Angst erzählt, alleine zu sterben, irgendwie war das immer noch zu schmerzhaft. Harry hatte ihr erzählt, wie er sich an den Tod seiner Eltern erinnerte, und dass er es lustig fand.

\emph{Es gibt kein Licht an dem Ort, an den dich der Dementor bringt, Hermine. Keine Wärme. Keine Fürsorge. Es ist ein Ort, an dem man nicht mal Glück empfinden kann. Dort gibt es Schmerz und Angst. Und die können dich immer noch antreiben.\\ Man kann hassen und sich daran erfreuen, das zu zerstören, was man hasst. Man kann lachen, wenn man andere Menschen leiden sieht. Aber man kann nie glücklich sein, man kann sich nicht einmal daran erinnern, was es ist, das nicht mehr da ist.\\ .. Ich glaube nicht, dass ich jemals erklären kann, wovor du mich gerettet hast. Normalerweise schäme ich mich dafür, Menschen in Schwierigkeiten zu bringen, normalerweise kann ich es nicht ausstehen, wenn Menschen für mich Opfer bringen, aber dieses eine Mal sage ich, dass, egal was es dich am Ende kostet, mich geküsst zu haben, du keine Sekunde daran zweifeln darfst, dass es das Richtige war.}

Hermine hatte nicht bemerkt, wie wenig der Dementor sie berührt hatte, wie klein und oberflächlich die Dunkelheit gewesen war, in die er sie gebracht hatte. Sie hatte alle sterben sehen, und das hatte trotzdem weh tun können.

Hermine steckte das Papier zurück in ihre Tasche, wie es sich für ein gutes Mädchen gehörte. Sie hatte es aber unbedingt lesen wollen. Sie hatte Angst vor Dementoren.

\textbf{Nachspiel, Minerva McGonagall:}\\ Sie fühlte sich wie erstarrt; sie hätte nicht so schockiert sein dürfen, sie hätte Harry nicht so schwer treffen dürfen, aber nach dem, was er durchgemacht hatte… Sie hatte den Jungen vor ihr nach irgendwelchen Anzeichen von Dementation abgesucht und sie nicht gefunden. Aber irgendetwas an der Ruhe, mit der er eine so ahnungsvolle Frage gestellt hatte, schien zutiefst beunruhigend.

"Mr. Potter, ich kann unmöglich ohne die Erlaubnis des Schulleiters über solche Dinge sprechen!"

Der Junge in ihrem Büro nahm dies auf, ohne seine Miene zu verändern.\\ "Ich würde es vorziehen, den Schulleiter in dieser Angelegenheit nicht zu stören", sagte Harry Potter ruhig. "Ich bestehe sogar darauf, ihn nicht zu stören, und Sie haben ja versprochen, dass unser Gespräch vertraulich behandelt wird. Also lassen Sie es mich so ausdrücken. Ich weiß, dass es in der Tat eine Prophezeiung gab. Ich weiß, dass Sie derjenige sind, der sie ursprünglich von Professor Trelawney gehört hat.\\ Ich weiß, dass die Prophezeiung das Kind von James und Lily als jemanden bezeichnete, der für den Dunklen Lord gefährlich ist. Und ich weiß, wer ich bin, ja, jeder weiß jetzt, wer ich bin, also enthüllen Sie nichts Neues oder Gefährliches, wenn Sie mir nur das sagen: Wie lautete der genaue Wortlaut, der mich, das Kind von James und Lily, identifizierte?"

Trelawneys hohle Stimme hallte in ihren Gedanken wider -\\

GEBOREN VON DENEN, DIE IHM DREI MAL BEKÄMPFT HABEN, GEBOREN WENN DER SIEBTE MONAT STIRBT…

"Harry", sagte Professor McGonagall, "das kann ich dir unmöglich sagen!"\\ Es kühlte sie bis auf die Knochen, dass Harry schon so viel wusste, sie konnte sich nicht vorstellen, wie Harry es erfahren hatte - Der Junge sah sie mit seltsamen, traurigen Augen an.

"Können Sie nicht ohne die Erlaubnis des Schulleiters niesen, Professor McGonagall? Denn ich verspreche Ihnen, dass ich einen guten Grund habe, zu fragen, und einen guten Grund, die Frage geheim zu halten."

"Bitte nicht, Harry", flüsterte sie.

"Also gut", sagte Harry. "Eine einfache Frage. Bitte. Wurde die Familie Potter namentlich erwähnt? Steht in der Prophezeiung wörtlich 'Potter'?"

Sie starrte Harry eine Weile lang an. Sie hätte nicht sagen können, warum oder woher sie das Gefühl hatte, dass dies ein kritischer Punkt war, dass sie die Bitte nicht leichtfertig ablehnen konnte und auch nicht leichtfertig darauf eingehen konnte -\\ "Nein", sagte sie schließlich. "Bitte, Harry, frag nicht weiter."

Der Junge lächelte, ein wenig traurig, wie es schien, und sagte:\\ "Danke, Minerva. Du bist eine gute Frau und wahrhaftig edel."

Und während ihr noch der Mund vor lauter Schreck offen stand, stand Harry Potter auf und verließ das Büro; und erst da wurde ihr klar, dass Harry ihre Ablehnung als Antwort aufgefasst hatte, und zwar als die wahre Antwort -

Harry schloss die Tür hinter sich.

Die Logik hatte sich mit einer seltsamen diamantenen Klarheit präsentiert. Harry hätte nicht sagen können, ob sie ihm während Fawkes' Gesang gekommen war, oder vielleicht sogar schon vorher.

Lord Voldemort hatte James Potter getötet. Er hatte es vorgezogen, das Leben von Lily Potter zu verschonen. Er hatte also seinen Angriff fortgesetzt, mit dem einzigen Ziel, ihr kleines Kind zu töten. Dunkle Lords hatten normalerweise keine Angst vor Kleinkindern. Es gab also eine Prophezeiung, die besagte, dass Harry Potter für Lord Voldemort gefährlich sei, und Lord Voldemort hatte diese Prophezeiung gekannt.

"\emph{Ich gebe dir diese seltene Chance zu fliehen. Aber ich werde mich nicht bemühen, dich zu bezwingen, und dein Tod hier wird dein Kind nicht retten.\\ Tritt beiseite, törichte Frau, wenn du überhaupt noch einen Funken Verstand in dir hast!}"

War es eine Laune gewesen, ihr diese Chance zu geben? Aber dann hätte Lord Voldemort nicht versucht, sie zu überreden. Hätte die Prophezeiung Lord Voldemort davor gewarnt, Lily Potter zu töten? Dann hätte Lord Voldemort sich die Mühe gemacht, sie zu unterwerfen.

Lord Voldemort hatte eine leichte Neigung gehabt, Lily Potter nicht zu töten. Die Neigung war stärker gewesen als eine Laune, aber nicht so stark wie eine Warnung.\\ Nehmen wir also an, dass jemand, den Lord Voldemort als weniger wichtigen Verbündeten oder Diener betrachtete, nützlich, aber nicht unentbehrlich, den Dunklen Lord angefleht hatte, Lilys Leben zu verschonen. Lilys Leben, aber nicht das von James. Diese Person hatte gewusst, dass Lord Voldemort das Haus der Potters angreifen würde. Er kannte sowohl die Prophezeiung als auch die Tatsache, dass der Dunkle Lord sie kannte. Sonst hätte er nicht um Lilys Leben gebettelt.

Laut Professor McGonagall wussten außer ihr selbst nur noch Albus Dumbledore und Severus Snape von der Prophezeiung.

Severus Snape, der Lily geliebt hatte, bevor sie Lily Potter war, und James hasste. Severus hatte also von der Prophezeiung erfahren und sie dem Dunklen Lord erzählt.\\ Was er getan hatte, weil die Prophezeiung die Potters nicht mit Namen beschrieben hatte. Es war ein Rätsel gewesen. Und Severus hatte dieses Rätsel erst zu spät gelöst. Aber wenn Severus der Erste war, der die Prophezeiung hörte und bereit war, sie dem Dunklen Lord mitzuteilen, warum hätte er dann auch Dumbledore oder Professor McGonagall davon erzählen sollen? Also hatten Dumbledore oder Professor McGonagall sie zuerst gehört. Der Schulleiter von Hogwarts hatte keinen offensichtlichen Grund, dem Professor für Verwandlung von einer äußerst sensiblen und entscheidenden Prophezeiung zu erzählen. Aber der Professor für Verwandlung hatte allen Grund, es dem Schulleiter zu sagen. Es schien also wahrscheinlich, dass Professor McGonagall die erste war, die es erfahren hatte.

Die vorherigen Wahrscheinlichkeiten sagten, dass es Professor Trelawney gewesen war, der ansässige Seher von Hogwarts. Seher waren selten, wenn man also die meisten Sekunden zusammenzählte, die Professor McGonagall im Laufe ihres Lebens in der Gegenwart eines Sehers verbracht hatte, waren die meisten dieser Seher-Sekunden Trelawney-Sekunden.

Professor McGonagall hatte es Dumbledore erzählt und hätte auch sonst niemandem ohne Erlaubnis von der Prophezeiung erzählt. Daher war es Albus Dumbledore, der dafür gesorgt hatte, dass Severus Snape irgendwie von der Prophezeiung erfuhr.\\ Und Dumbledore selbst hatte das Rätsel erfolgreich gelöst, sonst hätte er Severus, der Lily einst geliebt hatte, nicht als Vermittler ausgewählt.

Dumbledore hatte absichtlich dafür gesorgt, dass Lord Voldemort von der Prophezeiung erfuhr, in der Hoffnung, ihn in den Tod zu locken.

Vielleicht hatte Dumbledore dafür gesorgt, dass Severus nur einen Teil der Prophezeiung erfuhr, oder es gab andere Prophezeiungen, an denen Severus unschuldig geblieben war… irgendwie hatte Dumbledore gewusst, dass ein sofortiger Angriff auf die Potters immer noch zu Lord Voldemorts sofortiger Niederlage führen würde, obwohl Lord Voldemort selbst nicht daran geglaubt hatte.

Oder vielleicht war das nur ein Glücksfall von Dumbledores Wahnsinn gewesen, seinem Hang zu bizarren Komplotten… Severus hatte sich danach in den Dienst von Dumbledore gestellt; vielleicht würden die Todesser Severus nicht wohlwollend betrachten, wenn Dumbledore seine Rolle bei ihrer Niederlage enthüllen würde.

Dumbledore hatte versucht, dafür zu sorgen, dass Harrys Mutter verschont wurde. Aber dieser Teil seines Plans war gescheitert. Und er hatte James Potter wissentlich zum Tode verurteilt.

\emph{Dumbledore war für den Tod von Harrys Eltern verantwortlich.}

Wenn die ganze Kette der Logik korrekt war. konnte Harry nicht mit Fug und Recht behaupten, dass die erfolgreiche Beendigung des Zaubererkrieges nicht als mildernde Umstände zählte.\\

\emph{Aber irgendwie …. störte es ihn trotzdem….} \textbf{\emph{sehr}}\emph{.}

Und es war Zeit und überfällig, Draco Malfoy zu fragen, was die andere Seite dieses Krieges über den Charakter von Albus Percival Wulfric Brian Dumbledore zu sagen hatte.

