

\hypertarget{einfuxfchrung}{% \section{1. Einführung}\label{einfuxfchrung}}

Allgemeines:

Diese Übersetzung ist nicht mein Werk. Das englische Original kann unter folgenden Links gefunden werden:

\url{http://www.hpmor.com/}\strut

\url{https://www.fanfiction.net/s/5782108/1/Harry_Potter_and_the_Methods_of_Rationality}\strut

der Author ist „Eliezer Yudkowsky“, und hat die Geschichte unter dem Pseudonym „Less Wrong“ veröffentlicht. Die Einverständnis des Autors habe ich per E-Mail erhalten.

Ich beanspruche keinerlei Copyright-Rechte oder ähnliches für diese Übersetzung. Sie ist frei verfügbar, auch gegen Veränderungen und Anpassungen habe ich nichts einzuwenden.

Im Deutschen gibt es bis jetzt angefangene Übersetzungen, z.B. hier

\url{https://www.fanfiktion.de/s/5c793dfe000a402030774dc7/1/Harry-Potter-und-die-Methoden-der-Rationalitaet-Ubersetzung-HPMOR} von der Füchsin,

\url{https://www.fanfiktion.de/s/4cb8beb50000203e067007d0/1/Harry-Potter-und-die-Methoden-des-rationalen-Denkens} von Jost oder auf Youtube.

Alle Überstzungen die bis jetzt im Netz von der FanFiction vorhanden sind, sind bis jetzt unvollständig. Ich werde in etwa alle zwei Tage ein Kapitel hier veröffentlichen, mit der Übersetzung aller Kapitel bin ich fertig.

Die in der Geschichte erwähnte Wissenschaft ist real und nicht erfunden.

Zur Motivation:

Ich bin kein großer Autor, zumindest nicht von Geschichten die ich veröffentlichen möchte. Aber ich fand es schade, dass von meiner liebsten Fan Fiction keine Deutsche Übersetzung zu finden war. Ich selbst habe berufsbedingt zwar kein Problem mit dem englischen, aber ich möchte die Geschichte doch einem größerem Publikum zugänglich machen. Selbst wenn Sie inzwischen ein wenig in die Jahre gekommen ist, hoffe ich doch ein paar neue Fans davon erreichen zu können. Die Geschichte ist von mir *zu Ende und fertig* übersetzt, und ich werde regelmäßig neue Kapitel hier veröffentlichen damit es endlich eine vollständige Deutsche Fassung gibt. Meinen Acount auf der Seite hier habe ich ausschließlich zu diesem Zweck erstellt.

Da ich noch keine wirklich Ahnung habe wie man Texte hier richtig formatiert (z.B. etwas fett machen oder unterstreichen etc.), und auch der eine oder andere Rechtschreibfehler enthalten sein wird, bitte ~ich das im Voraus zu entschuldigen.

Zur Geschichte selbst:

Die Geschichte kann als eine Art Paralleluniversum verstanden werden. Die meisten Charaktere und Ereignisse verhalten sich bzw. passieren genau oder ähnlich wie im Buch. Einige wenige Dinge wurden jedoch verändert, sodass die Geschichte einen anderen Verlauf nimmt und relativ zügig in eine ganz andere Bahn gelenkt wird. Lasst euch einfach überraschen!

Zur Übersetzung:

Es gibt einige Wortwitze, Anspielungen, Lieder oder Gedichte die nur im englischen funktionieren. In solchen Fällen werde ich das englische Original hintenan stellen.

Und nun wünsche ich viel Spaß beim Lesen!

