

\hypertarget{koordinationsprobleme-teil-1}{% \section{33. Koordinationsprobleme, Teil 1}\label{koordinationsprobleme-teil-1}}

\textbf{\uline{Koordinationsprobleme, Teil 1}}

\emph{Zitat Original Autor: Die in diesem Kapitel verwendete Version der Entscheidungstheorie ist nicht die akademisch dominante. Sie basiert auf etwas, das sich "zeitlose Entscheidungstheorie" nennt und unter anderem von Gary Drescher, Wei Dai, Vladimir Nesov und, nun ja… mir entwickelt wird. (hustet ein paar Mal).}

\emph{Das Erschreckende war, wie schnell die ganze Sache außer Kontrolle geraten war.}

"Albus", sagte Minerva und versuchte nicht einmal, die Sorge aus ihrer Stimme zu halten, als die beiden die Große Halle betraten, "es muss etwas geschehen."

Die Atmosphäre in Hogwarts vor der Weihnachtszeit war normalerweise hell und fröhlich. Die Große Halle war bereits in Grün und Rot geschmückt worden, nach einem Slytherin und einem Gryffindor, deren Weihnachtshochzeit zu einem Symbol der Freundschaft über die Häuser und Loyalitäten hinweg geworden war, eine Tradition, die fast so alt war wie Hogwarts selbst und die sich sogar bis in die Muggelländer verbreitet hatte.

Die Schüler, die jetzt zu Abend aßen, warfen sich nervöse Blicke über die Schultern oder schickten böse Blicke an andere Tische, oder an manchen Tischen wurde heftig gestritten. Man hätte die Atmosphäre vielleicht als angespannt bezeichnen können, aber der Ausdruck, der Minerva in den Sinn kam, war \emph{fünfter Grad der Vorsicht.}

Man nehme eine Schule, unterteilt in vier Häuser… Nun füge in jedes Jahr drei Armeen im Krieg hinzu. Und die Parteilichkeit von Drache und Sonnenschein und Chaos hatte sich über die Erstklässler hinaus verbreitet; sie waren zu den Armeen für diejenigen geworden, die keine Armeen hatten.

Schüler trugen Armbinden mit Insignien des Feuers oder des Lächelns oder der erhobenen Hand und verhexten sich gegenseitig in den Korridoren.

Alle drei Erstklässler hatten ihnen gesagt, sie sollten aufhören - sogar Draco Malfoy hatte sie angehört und dann grimmig genickt -, aber ihre vermeintlichen Gefolgsleute hatten nicht auf sie gehört.

Dumbledore blickte mit einem distanzierten Blick auf die Tische hinaus.

"In jeder Stadt", zitierte der alte Zauberer leise, "ist die Bevölkerung seit langem in die Blaue und die Grüne Fraktion gespalten… Und sie kämpfen gegen ihre Gegner, ohne zu wissen, zu welchem Zweck, sie gefährden sich selbst … So wächst in ihnen eine Feindseligkeit gegen ihre Mitmenschen heran, die keine Ursache hat und zu keiner Zeit aufhört oder verschwindet, denn sie gibt weder den Banden der Ehe noch der Verwandtschaft noch der Freundschaft Raum, und der Fall ist derselbe, auch wenn die, die sich in Bezug auf diese Farben unterscheiden, Brüder oder andere Verwandte sind. Ich für meinen Teil bin nicht in der Lage, dies etwas anderes zu nennen als eine Krankheit der Seele …"

"Es tut mir leid", sagte Minerva, "ich weiß nicht -"

"Procopius", sagte Dumbledore. "Die haben ihre Wagenrennen sehr ernst genommen, im Römischen Reich. Ja, Minerva, ich stimme zu, dass etwas getan werden muss."

"Bald", sagte Minerva, und ihre Stimme senkte sich noch weiter.

"Albus, ich denke, es muss vor Samstag geschehen."

Am Sonntag würden die meisten Schüler Hogwarts verlassen, um die Feiertage bei ihren Familien zu verbringen; am Samstag war dann die letzte Schlacht der drei Erstklässlerarmeen, die über die Vergabe von Professor Quirrells dreifach verfluchtem Weihnachtswunsch entscheiden würde.

Dumbledore blickte zu ihr hinüber und musterte sie ernsthaft.

"Du befürchtest, dass es dann zur Explosion kommt und jemand verletzt wird."

Minerva nickte.

"Und dass man Professor Quirrell die Schuld geben wird."

Minerva nickte wieder, ihr Gesicht war angespannt. Sie hatte sich schon längst damit abgefunden, dass Verteidigungsprofessoren gefeuert wurden.

"Albus", sagte Minerva, "wir können Professor Quirrell jetzt nicht verlieren, das können wir nicht! Wenn er nur bis Januar bleibt, werden unsere Fünftklässler ihre Z.A.G. bestehen, wenn er bis März bleibt, werden unsere Siebtklässler ihre U.T.Z bestehen, er macht Jahre der Vernachlässigung in Monaten wieder gut, eine ganze Generation wird aufwachsen, die sich trotz des Fluchs des Dunklen Lords verteidigen kann - du musst den Kampf beenden, Albus! Verbiete die Armeen sofort!"

"Ich bin mir nicht sicher, ob der Verteidigungsprofessor das wohlwollend aufnehmen würde", sagte Dumbledore und blickte zum Haupttisch hinüber, wo Quirrell in seine Suppe sabberte.

"Er scheint sehr an seinen Armeen zu hängen, obwohl ich, als ich zustimmte, dachte, dass es in jedem Jahr vier sein würden."

Der alte Zauberer seufzte.

"Ein kluger Mann, wahrscheinlich mit den besten Absichten; aber vielleicht nicht klug genug, fürchte ich. Und die Armeen zu verbieten, könnte auch die Explosion auslösen."

"Aber dann Albus, was wirst du tun?"

Der alte Zauberer schenkte ihr ein wohlwollendes Lächeln.

"Nun, ich werde natürlich ein Komplott schmieden. Das ist die neue Mode in Hogwarts."

Und sie waren zu nahe an den Haupttisch gekommen, als dass Minerva noch etwas hätte sagen können.

\emph{Das Erschreckende war, wie schnell die ganze Sache außer Kontrolle geraten war.} Die erste Schlacht im Dezember war… chaotisch gewesen, zumindest hatte Draco das gehört. Die zweite Schlacht war derangiert gewesen. Und die nächste würde noch schlimmer werden, es sei denn, den dreien zusammen gelang ein letzter verzweifelter Versuch, sie aufzuhalten.

"Professor Quirrell, das ist Irrsinn", sagte Draco barsch. "Das ist nicht mehr Slytherin, das ist einfach …" Draco fand keine Worte mehr. Er fuchtelte hilflos mit den Händen. "Man kannst unmöglich richtige Pläne schmieden, bei all dem, was hier los ist. Bei der letzten Schlacht hat einer meiner Soldaten seinen eigenen Selbstmord vorgetäuscht. Wir haben Hufflepuffs, die versuchen zu intrigieren, und sie denken, sie können es, aber sie können es nicht. Die Dinge passieren jetzt einfach zufällig, es hat nichts damit zu tun, wer der Klügste ist oder welche Armee am besten kämpft, es ist…" Er konnte es nicht einmal beschreiben.

"Ich stimme Mr~Malfoy zu", sagte Granger in dem Tonfall von jemandem, der nicht erwartet hatte, sich selbst jemals diese Worte sagen zu hören. "Verräter zuzulassen, funktioniert nicht, Professor Quirrell."

Draco hatte versucht, jedem in seiner Armee zu verbieten, sich zu verschwören, außer ihm, und das hatte die Verschwörungen nur in den Untergrund getrieben, niemand wollte außen vor bleiben, wenn die Soldaten in anderen Armeen sich verschwören konnten. Nachdem er ihre letzte Schlacht kläglich verloren hatten, hatte er schließlich nachgegeben und sein Dekret widerrufen; aber da hatten seine Soldaten bereits begonnen, ihre eigenen persönlichen Pläne in die Tat umzusetzen, ohne irgendeine Art von zentraler Koordination.

Nachdem ihm alle Pläne mitgeteilt worden waren, oder das, was seine Soldaten behaupteten, ihre Pläne zu sein, hatte Draco versucht, einen Plan zu entwerfen, um die letzte Schlacht zu gewinnen. Es hatte wesentlich mehr als drei verschiedene Dinge gebraucht, um richtig zu sein, und Draco hatte Incendio auf dem Papier und Everto benutzt, um die Asche verschwinden zu lassen, denn wenn Vater es gesehen hätte, wäre er enterbt worden.

Professor Quirrells Augenlider waren halb geschlossen, sein Kinn ruhte auf seinen Händen, während er sich nach vorne auf seinen Schreibtisch lehnte.

"Und Sie, Mr~Potter?", fragte der Verteidigungsprofessor. "Sind Sie ebenfalls damit einverstanden?"

"Wir bräuchten nur Franz Ferdinand zu erschießen, und wir könnten den Ersten Weltkrieg beginnen", sagte Harry. "Es herrscht das totale Chaos. Ich bin voll dafür."

"Harry!", sagte Draco völlig schockiert. Er bemerkte erst eine Sekunde später, dass er es genau zur gleichen Zeit und in genau dem gleichen Ton der Empörung gesagt hatte wie Granger. Granger warf ihm einen erschrockenen Blick zu, und Draco hielt sein Gesicht sorgfältig neutral.

\emph{Ups}.

"Das stimmt!", sagte Harry. "Ich hintergehe euch! Euch beide! Schon wieder! Ha ha!"

Professor Quirrell lächelte dünn, obwohl seine Augen immer noch halb geschlossen waren. "Und warum ist das so, Mr~Potter?"

"Weil ich glaube, dass ich mit dem Chaos besser umgehen kann als Miss~Granger oder Mr~Malfoy", sagte der Verräter. "Unser Krieg ist ein Nullsummenspiel, und es kommt nicht darauf an, ob er im absoluten Sinne leicht oder schwer ist, sondern nur, wer besser oder schlechter abschneidet."

\emph{Harry Potter lernte viel zu schnell.}

Professor Quirrells Augen bewegten sich unter ihren Lidern, um Draco und dann Granger zu betrachten.

"In Wahrheit, Mr~Malfoy, Miss~Granger, könnte ich einfach nicht mit mir selbst leben, wenn ich das große Debakel vor seinem Höhepunkt beenden würde.

Einer Ihrer Soldaten ist sogar ein Vierfachagent geworden."

"Vierfach?!", sagte Granger. "Aber es gibt doch nur drei Seiten in diesem Krieg!"

"Ja", sagte Professor Quirrell, "das könnte man meinen, oder? Ich bin mir nicht sicher, ob es jemals in der Geschichte einen Vierfachagenten gegeben hat, oder eine Armee mit einem so hohen Anteil an echten und vermeintlichen Verrätern.

Wir erforschen neue Gefilde, Miss~Granger, und wir können jetzt nicht mehr umkehren."

Draco verließ das Büro des Verteidigungsprofessors mit hart aufeinander knirschenden Zähnen, und Granger blickte noch verärgerter neben ihm her.

"Ich kann nicht glauben, dass du das getan hast, Harry!", sagte Granger.

"Tut mir leid", sagte Harry, wobei es überhaupt nicht nach Entschuldigung klang, seine Lippen waren zu einem fröhlichen, bösen Lächeln geschwungen.

"Denk daran, Hermine, es ist nur ein Spiel, und warum sollten Generäle wie wir die Einzigen sein, die sich verschwören dürfen? Und außerdem, was wollt ihr zwei denn machen? Sich gegen mich verbünden?"

Draco tauschte einen Blick mit Granger und wusste, dass sein eigenes Gesicht genauso angespannt war wie ihres. Harry hatte sich immer offener und schadenfroher auf Dracos Weigerung verlassen, mit einem Schlammblut gemeinsame Sache zu machen; \emph{und Draco wurde es allmählich leid, dass das gegen ihn verwendet wurde.}

\emph{Wenn das noch länger so weiterging, würde er sich mit Granger verbünden, nur um Harry Potter zu zermalmen, und um zu sehen, wie sehr dem Sohn eines Schlammbluts das gefiel.}

\emph{Das Erschreckende daran war, wie schnell die ganze Sache außer Kontrolle geraten war.}

Hermine starrte auf das Pergament, das Zabini ihr gegeben hatte, und fühlte sich absolut und völlig hilflos. Da waren Namen, und Linien, die die Namen mit anderen Namen verbanden, und einige der Linien waren in verschiedenen Farben und…

"Sagen Sie mir", sagte General Granger, "gibt es jemanden in meiner Armee, der kein Spion ist?"

Die beiden befanden sich nicht im Büro, sondern in einem anderen, verlassenen Klassenzimmer, und sie waren allein; denn, so hatte Oberst Zabini gesagt, es war inzwischen fast sicher, dass mindestens einer der Hauptleute ein Verräter war.

Wahrscheinlich Hauptmann Goldstein, aber Zabini wusste es nicht genau. Ihre Frage hatte ein ironisches Lächeln auf das Gesicht des jungen Slytherins gezaubert.

Blaise Zabini schien sie immer ein wenig zu verachten, aber er schien sie nicht aktiv abzulehnen; nichts im Vergleich zu der Verachtung, die er für Draco Malfoy hegte, oder dem Groll, den er für Harry Potter entwickelt hatte.

Zuerst hatte sie sich Sorgen gemacht, dass Zabini sie verraten könnte, aber der Junge schien verzweifelt zu zeigen, dass die anderen beiden Generäle nicht besser waren als er; und Hermine dachte, dass Zabini zwar wahrscheinlich froh wäre, sie an irgendjemand anderen zu verraten, aber er würde Malfoy oder Harry niemals gewinnen lassen.

"Die meisten Ihrer Soldaten sind immer noch loyal zu Ihnen, da bin ich mir ziemlich sicher", sagte Zabini. "Es ist nur so, dass niemand aus dem Spaß herausgelassen werden will." Der verächtliche Blick auf dem Gesicht des Slytherins machte deutlich, was er von Leuten hielt, die Verschwörungen nicht ernst nahmen.

"Also denken sie, sie könnten Doppelagenten sein und heimlich für unsere Seite arbeiten, während sie vorgeben, uns zu verraten."

"Und das würde auch für jeden in den anderen Armeen gelten, der sagt, er wolle unser Spion sein", sagte Hermine vorsichtig.

Der junge Slytherin zuckte mit den Schultern.

"Ich glaube, ich kann gut einschätzen, wer Malfoy wirklich verraten will, ich bin mir nicht sicher, ob irgendjemand Potter wirklich an dich verraten will. Aber Nott ist ein sicherer Tipp, um Potter an Malfoy zu verraten, und da ich Entwhistle angeblich im Auftrag von Malfoy an ihn herantreten ließ und Entwhistle wirklich an uns berichtet, ist das fast so gut -"

Hermine schloss für einen Moment die Augen. "Wir werden verlieren, nicht wahr?"

"Sieh mal", sagte Zabini geduldig, "du bist im Moment in Führung, was Quirrell-Punkte angeht. Wir müssen nur diesen letzten Kampf nicht komplett verlieren, dann haast du genug Quirrell-Punkte, um den Weihnachtswunsch zu gewinnen."

Professor Quirrell hatte angekündigt, dass das letzte Gefecht nach einem formalen Punktesystem ablaufen würde, was man von ihm verlangt hatte, um spätere Schuldzuweisungen zu vermeiden. Jedes Mal, wenn man jemanden erschoss, bekam der General der eigenen Armee zwei Quirrell-Punkte. Ein Gong würde durch das Kampfgebiet läuten (sie wussten noch nicht, wo sie kämpfen würden, obwohl Hermine wieder auf den Wald hoffte, wo Sonnenschein gut abschnitt) und sein Ton würde verraten, welche Armee die Punkte gewonnen hatte.

Und wenn jemand vortäuschen würde, getroffen worden zu sein, würde der Gong trotzdem ertönen, und dann würde später, nach keiner festgelegten Zeit, ein doppelter Gong ertönen, um den Rückzug zu begrüßen. Und wenn man den Namen einer Armee rief, "Für Sonnenschein!" oder "Für Chaos!" oder "Für Drache!", wechselte man die Zugehörigkeit zu dieser Armee.

.. Selbst Hermine hatte den Fehler in diesem Regelwerk erkennen können. Aber Professor Quirrell hatte weiter verkündet, dass, wenn man ursprünglich Sonnenschein zugewiesen worden war, niemand auf einen im Namen von Sonnenschein schießen konnte - oder besser gesagt, sie konnten es, aber dann verlor Sonnenschein einen einzigen Quirrell-Punkt, symbolisiert durch einen dreifachen Gong.

Das hinderte einen daran, seine eigenen Soldaten für Punkte zu erschießen, und riet davon ab, Selbstmord zu begehen, bevor der Feind einen erwischte, aber man konnte immer noch Spione erschießen, wenn man es musste.

Im Moment hatte Hermine zweihundertvierundvierzig Quirrell-Punkte, und Malfoy hatte zweihundertneunzehn, und Harry hatte zweihunderteinundzwanzig; und es gab vierundzwanzig Soldaten in jeder Armee.

"Also kämpfen wir vorsichtig", sagte Hermine, "und versuchen einfach, nicht allzu sehr zu verlieren."

"Nein", sagte Zabini. Das Gesicht des jungen Slytherins war jetzt ernst. "Das Problem ist, dass sowohl Malfoy als auch Potter wissen, dass sie nur gewinnen können, wenn sie sich zusammenschließen und uns zerquetschen und dann alleine kämpfen. Ich denke also, wir sollten Folgendes tun -"

Hermine verließ das Klassenzimmer in einer Art Benommenheit.

Zabinis Plan war nicht der offensichtliche gewesen, er war seltsam und kompliziert und vielschichtig und die Art von Sache, von der sie erwartet hätte, dass Harry sie sich ausdenkt, nicht Zabini. Es fühlte sich falsch an, dass nur sie in der Lage war, so einen Plan zu verstehen. Junge Mädchen sollten nicht in der Lage sein, solche Pläne zu verstehen. Der Hut hätte sie nach Slytherin sortiert, wenn er gesehen hätte, dass sie solche Pläne verstehen konnte.

\emph{.. Das Erstaunliche war, wie schnell er das Chaos hatte eskalieren lassen, sobald er es absichtlich getan hatte.}

Harry saß in seinem Büro; er hatte die Befugnis erhalten, Möbel bei den Hauselfen zu bestellen, also hatte er einen Thron bestellt und Vorhänge in einem schwarz-scharlachroten Muster. Scharlachrotes Licht wie Blut, vermischt mit Schatten, ergoss sich über den Boden.

\emph{Irgendetwas in Harry fühlte sich an, als wäre er endlich nach Hause gekommen.}

Vor ihm standen die vier Leutnants des Chaos, seine treuesten Gefolgsleute, von denen einer ein Verräter war. \emph{Das hier. So sollte das Leben sein.}

"Wir sind versammelt", sagte Harry.

"Lasst das Chaos regieren", riefen seine vier Leutnants im Chor.

"Mein Luftkissenfahrzeug ist voll mit Aalen", sagte Harry.

"Ich kaufe diese Platte nicht, sie ist zerkratzt", sagten seine vier Leutnants im Chor.

"Ganz mimsy waren die Borogroves."

"Und die Klosterratten wühlen im Müll!"

Damit waren die Formalitäten abgeschlossen.

"Was macht die Verwirrung?" sagte Harry in einem trockenen Flüsterton wie Imperator Palpatine.

"Es läuft gut, General Chaos", sagte Neville in dem Ton, den er immer für militärische Angelegenheiten verwendete, ein Ton, der so tief war, dass der Junge oft innehalten und husten musste. Der chaotische Leutnant war ordentlich in seine schwarze Schulrobe gekleidet, die mit dem Gelb des Hufflepuff-Hauses getrimmt war, und sein Haar war gescheitelt und gekämmt im üblichen Look für einen ernsten Jungen. Harry hatte die Inkongruenz besser gefallen als jeder der Umhänge, die sie ausprobiert hatten.

"Unsere Legionäre haben seit gestern Abend fünf neue Verschwörungen begonnen."

Harry lächelte böse. "Hat einer von ihnen eine Chance, zu funktionieren?"

"Das glaube ich nicht", sagte Neville von Chaos. "Hier ist der Bericht."

"Ausgezeichnet", sagte Harry und lachte schallend, als er Neville das Pergament aus der Hand nahm, wobei er sein Bestes gab, damit es so klang, als würde er an Staub ersticken. Das brachte die Gesamtzahl auf sechzig.

\emph{Soll Draco doch versuchen, damit umzugehen. Soll er's doch versuchen.}

Und was Blaise Zabini angeht… Harry lachte wieder, und diesmal gab er sich nicht einmal Mühe, böse zu klingen. Er musste sich wirklich mal das Haustier von jemandem ausleihen, damit er eine Katze zum Streicheln hatte, während er das hier machte.

"Kann die Legion jetzt aufhören, Verschwörungen zu machen?", sagte Finnigan of Chaos. "Ich meine, haben wir nicht schon genug -"

"Nein", sagte Harry mit Nachdruck. "Wir können nie genug Verschwörungen haben."

Professor Quirrell hatte es perfekt ausgedrückt. Sie waren dabei, die Grenzen weiter zu verschieben, als sie vielleicht jemals verschoben worden waren; und Harry hätte nicht mit sich selbst leben können, wenn er jetzt umgedreht hätte.

Es klopfte an der Tür.

"Das wird der Drachengeneral sein", sagte Harry und lächelte mit böser Vorahnung.

"Er kommt genau dann an, wann ich es erwartet habe. Führt ihn herein und euch selbst hinaus."

Und die vier Leutnants des Chaos schlurften hinaus und warfen Draco finstere Blicke zu, als der feindliche General Harrys geheimes Versteck betrat.

\emph{Wenn er das nicht tun durfte, wenn er älter war, würde Harry einfach für immer elf bleiben.}

Die Sonne tropfte durch die roten Vorhänge und schickte Blutstrahlen über den Boden hinter Harry Potters gepolstertem Stuhl in Erwachsenengröße, den er mit goldenem und silbernem Glitzer überzogen hatte und darauf bestand, ihn als seinen Thron zu bezeichnen.

(Draco fühlte sich allmählich viel sicherer, dass er das Richtige getan hatte, als er beschloss, Harry Potter zu stürzen, bevor dieser die Weltherrschaft übernehmen konnte. Draco konnte sich nicht einmal vorstellen, wie es sein würde, unter seiner Herrschaft zu leben.)

"Guten Abend, Drachengeneral", sagte Harry Potter in einem kühlen Flüsterton.

"Sie sind genau so eingetroffen, wie ich es erwartet habe."

\emph{Das war nicht überraschend, wenn man bedenkt, dass Draco und Harry den Zeitpunkt des Treffens im Voraus vereinbart hatten.}

Und es war auch nicht Abend, aber inzwischen wusste Draco es besser, als etwas zu sagen.

"General Potter", sagte Draco mit so viel Würde, wie ihm möglich war,

"Sie wissen, dass unsere beiden Armeen zusammenarbeiten müssen, damit einer von uns Professor Quirrells Wunsch erfüllen kann, oder?"

"Ja", zischte Harry, als würde der Junge ihn für ein Parselmaul halten.

"Wir müssen zusammenarbeiten, um Sonnenschein zu vernichten, und erst dann können wir es untereinander ausfechten. Aber wenn einer von uns den anderen vorher verrät, könnte er sich im späteren Kampf einen Vorteil verschaffen. Und der General von Sonnenschein, der das alles weiß, wird versuchen, jeden von uns in dem Glauben zu lassen, dass der andere ihn verraten hat. Und Sie und ich, die das wissen, werden versucht sein, den anderen zu verraten und so zu tun, als sei es Grangers List.

Und Granger weiß das auch."

Draco nickte. So viel war klar.

"Und … wir beide wollen nur gewinnen, und es gibt niemanden, der einen von uns bestrafen würde, wenn wir überlaufen …"

"Genau", sagte Harry Potter, sein Gesicht wurde nun ernst.

"Wir sind mit einem echten Gefangenendilemma konfrontiert."

Das Gefangenendilemma lief nach Harrys Lehren folgendermaßen ab: Zwei Gefangene waren in getrennten Zellen eingesperrt worden.

Es gab Beweise gegen jeden Gefangenen, aber nur geringe Beweise, genug für eine Gefängnisstrafe von zwei Jahren für jeden.

Jeder Gefangene konnte sich entscheiden, überzulaufen, den anderen zu verraten, vor Gericht gegen ihn auszusagen; und das würde ein Jahr von der eigenen Gefängnisstrafe abziehen, aber zwei Jahre zu der des anderen hinzufügen.

Oder ein Gefangener könnte kooperieren und schweigen. Wenn also beide Gefangenen überliefen und jeder gegen den anderen aussagte, würden sie jeweils drei Jahre verbüßen; wenn beide kooperierten oder schwiegen, würden sie jeweils zwei Jahre verbüßen; aber wenn einer überlief und der andere kooperierte, würde der Überläufer ein Jahr und der Kooperierende vier Jahre verbüßen.

Und beide Gefangenen mussten ihre Entscheidung treffen, ohne die Wahl des anderen zu kennen, und keiner von ihnen würde eine Chance bekommen, seine Entscheidung nachträglich zu ändern.

Draco hatte bemerkt, dass, wenn die beiden Gefangenen während des Zaubererkrieges Todesser gewesen wären, der Dunkle Lord jeden Verräter getötet hätte.

Harry hatte genickt und gesagt, das sei eine Möglichkeit, das Gefangenendilemma zu lösen - und tatsächlich würden beide Todesser wollen, dass es einen Dunklen Lord gibt, aus genau diesem Grund.

(Draco hatte Harry gebeten, innezuhalten und ihn eine Weile darüber nachdenken zu lassen, bevor sie fortfuhren. Es hatte eine Menge darüber erklärt, warum Vater und seine Freunde zugestimmt hatten, unter einem Dunklen Lord zu dienen, der oft nicht nett zu ihnen war…)

Tatsächlich, so hatte Harry gesagt, war das so ziemlich der Grund, warum die Menschen Regierungen hatten - man war besser dran, wenn man jemand anderen bestahl, so wie jeder Gefangene individuell besser dran war, wenn er im Gefangenendilemma überlief. Aber wenn jeder so denken würde, würde das Land ins Chaos stürzen und jeder wäre schlechter dran, so wie es passieren würde, wenn beide Gefangenen überlaufen würden. Also ließen sich die Menschen von Regierungen regieren, so wie sich die Todesser vom Dunklen Lord hatten regieren lassen.

(Draco hatte Harry gebeten, wieder aufzuhören. Draco war immer davon ausgegangen, dass ehrgeizige Zauberer sich an die Macht setzten, weil sie herrschen wollten, und die Menschen ließen sich beherrschen, weil sie ängstliche kleine Hufflepuffs waren. Und das schien, wenn man darüber nachdachte, immer noch wahr zu sein; aber Harrys Perspektive war faszinierend, auch wenn sie falsch war.)

Aber, so hatte Harry danach fortgesetzt, die Angst davor, von einem Dritten bestraft zu werden, war nicht der einzige mögliche Grund, beim Gefangenendilemma zu kooperieren.

Angenommen, so hatte Harry gesagt, man würde das Spiel gegen eine magisch erzeugte identische Kopie von sich selbst spielen.

Draco hatte gesagt, wenn es zwei Dracos gäbe, würde natürlich keiner von beiden wollen, dass dem anderen etwas Schlimmes zustößt, ganz zu schweigen davon, dass kein Malfoy sich als Verräter zu erkennen geben würde.

Harry hatte wieder genickt und gesagt, dass dies eine weitere Lösung für das Gefangenendilemma sei - die Leute könnten kooperieren, weil sie sich umeinander sorgten, oder weil sie ein Gefühl für Ehre hätten, oder weil sie ihren Ruf bewahren wollten. Tatsächlich, so hatte Harry gesagt, war es ziemlich schwierig, ein echtes Gefangenendilemma zu konstruieren - im wirklichen Leben sorgten sich die Leute normalerweise um die andere Person, oder um ihre Ehre oder um ihren Ruf oder um die Strafe eines Dunklen Lords oder um etwas anderes als die Gefängnisstrafe.

Aber nehmen wir an, die Kopie wäre von jemandem gewesen, der völlig egoistisch war - (Pansy Parkinson war das Beispiel gewesen, das sie benutzt hatten) - so dass jede Pansy sich nur darum kümmerte, was mit ihr geschah und nicht mit der anderen Pansy. Angenommen, das war alles, was Pansy interessierte… und dass es keinen Dunklen Lord gab… und dass Pansy sich keine Sorgen um ihren Ruf machte… und dass Pansy entweder kein Ehrgefühl hatte oder sich der anderen Gefangenen gegenüber nicht verpflichtet fühlte.

… wäre es dann rational, wenn Pansy kooperieren oder überlaufen würde? Einige Leute, sagte Harry, behaupteten, dass es rational wäre, wenn Pansy gegen ihre Kopie überlief, aber Harry und jemand namens Douglas Hofstadter waren der Meinung, dass diese Leute falsch lagen. Denn, so hatte Harry gesagt, wenn Pansy überlief - nicht willkürlich, sondern aus Gründen, die ihr rational erschienen - dann würde die andere Pansy genau so denken. Zwei identische Kopien würden sich nicht unterschiedlich entscheiden. Pansy musste also wählen zwischen einer Welt, in der beide Pansys kooperierten, oder einer Welt, in der beide Pansys überliefen, und sie war besser dran, wenn beide Kopien kooperierten.

Und wenn Harry gedacht hätte, dass "rationale" Menschen im Gefangenendilemma überlaufen, dann hätte er nichts getan, um diese Art von "Rationalität" zu verbreiten, denn ein Land oder eine Verschwörung voller "rationaler" Menschen würde sich im Chaos auflösen. Man würden seinen Feinden von der 'Rationalität' erzählen. Das hatte sich damals alles vernünftig angehört, aber jetzt kam Draco der Gedanke, dass…

"Du hast gesagt", sagte Draco, "dass die rationale Lösung des Gefangenendilemmas darin besteht, zu kooperieren. Aber du würdest natürlich wollen, dass ich das glaube, nicht wahr?"

Und wenn Draco zur Kooperation überredet würde, würde Harry einfach sagen:

\emph{"Ha ha, wieder verraten!"} und ihn hinterher darüber auslachen.

"Ich würde dir den Unterricht nicht vorgaukeln", sagte Harry ernst.

"Aber ich muss dich daran erinnern, Draco, dass ich nicht gesagt habe, dass du einfach automatisch kooperieren sollst. Nicht bei einem echten Gefangenendilemma wie diesem. Ich habe gesagt, dass du bei deinen Entscheidungen nicht denken sollst, als würdest du nur für dich selbst wählen oder als würdest du für alle wählen.

Du solltest so denken, als würdest du für alle Menschen wählen, die dir so ähnlich sind, dass sie wahrscheinlich das Gleiche tun werden wie du, aus den gleichen Gründen.

Und du sollst auch die Vorhersagen derjenigen wählen, die du gut genug kennst, um Sie genau vorherzusagen, so dass du niemals bereuen müsstest, dass du aufgrund der korrekten Vorhersagen, die andere Leute über dich machen, rational bist - erinner mich daran, irgendwann einmal das Newcomb-Problem zu erklären.

Die Frage, die du und ich uns also stellen müssen, Draco, ist folgende: Sind wir uns ähnlich genug, dass wir wahrscheinlich das Gleiche tun werden, was auch immer es ist, und unsere Entscheidungen größtenteils auf die gleiche Weise treffen? Oder kennen wir uns gut genug, um uns gegenseitig vorherzusagen, so dass ich vorhersagen kann, ob du kooperieren oder abtrünnig werden wirst, und du kannst vorhersagen, dass ich mich entschlossen habe, dasselbe zu tun, wovon ich vorhersage, dass du es tun wirst, weil ich weiß, dass du vorhersagen kannst, dass ich mich so entscheide?"

… und Draco konnte nicht anders, als zu denken, dass, da er sich anstrengen musste, nur um die Hälfte davon zu verstehen, die Antwort offensichtlich '\emph{Nein}' war.

"Ja", sagte Draco.

Es gab eine Pause.

"Ich verstehe", sagte Harry und klang enttäuscht. "Oh, na ja. Dann müssen wir uns wohl einen anderen Weg einfallen lassen."

\emph{Draco hatte nicht gedacht, dass das funktionieren würde.}

Draco und Harry sprachen darüber hin und her. Sie hatten sich beide schon viel früher darauf geeinigt, dass das, was sie auf dem Schlachtfeld taten, im wirklichen Leben nicht als gebrochenes Versprechen zählen würde - obwohl Draco ein wenig wütend darüber war, was Harry in Professor Quirrells Büro getan hatte, und das auch sagte. Aber wenn die beiden sich nicht auf Ehre oder Freundschaft verlassen konnten, blieb die Frage, wie sie ihre Armeen dazu bringen konnten, zusammenzuarbeiten, um Sonnenschein zu besiegen, trotz allem, was Granger versuchen würde, sie zu trennen.

Professor Quirrells Regeln machten es nicht verlockend, Sonnenschein die Soldaten der anderen Armee töten zu lassen - das erhöhte nur die Hürde, die man selbst überwinden musste -, aber es verleitete jede Seite dazu, Kills zu stehlen, anstatt wie eine einzelne Armee zu handeln, oder einige der Soldaten der anderen Seite in den Wirren der Schlacht zu erschießen.

… Hermine ging zurück nach Ravenclaw, ohne wirklich darauf zu achten, wohin sie ging, ihr Geist war mit Krieg und Verrat und anderen altersunangemessenen Konzepten beschäftigt, und sie bog um eine Ecke und stieß direkt mit einem Erwachsenen zusammen.

"Tut mir leid", sagte sie automatisch, und dann, ganz ohne nachzudenken,

"Iiih!"

"Keine Sorge, Miss~Granger", sagte das fröhliche Lächeln, das sich unter den funkelnden Augen und über dem silbernen Bart des Schulleiters von HOGWARTS abzeichnete.

"Es sei Ihnen ganz und gar verziehen."

Ihr Blick war hilflos auf das freundliche Gesicht des mächtigsten Zauberers der Welt geheftet, der auch der Oberste Hexenmeister war, der auch der Oberste Mugwump war, der vor Jahren durch den Stress des Kampfes gegen den Dunklen Lord wahnsinnig geworden war, und zahlreiche andere Fakten, die ihr nacheinander in den Sinn kamen, während ihre Kehle immer wieder kleine peinliche Quiekser von sich gab.

"In der Tat, Miss~Granger", sagte Albus Percival Wulfric Brian Dumbledore,

"es ist ein ziemlicher Glücksfall, dass wir uns über den Weg gelaufen sind.

Ich habe mich nämlich gerade neugierig gefragt, was Sie drei wohl vorhaben, um Ihre Wünsche zu erfragen …"

Der Samstag dämmerte hell und klar, und die Schüler sprachen mit gedämpften Stimmen, als ob der erste, der schrie, die Explosion auslösen könnte.

Draco hatte gehofft, dass sie wieder in den oberen Etagen von Hogwarts kämpfen würden. Professor Quirrell hatte gesagt, dass echte Kämpfe eher in Städten als in Wäldern stattfänden, und Kämpfe in Schulräumen und Korridoren sollten das simulieren, mit Bändern, um die erlaubten Bereiche zu markieren. Die Drachenarmee hatte sich bei diesen Kämpfen gut geschlagen. Stattdessen hatte sich Professor Quirrell, genau wie Draco es befürchtet hatte, für diesen Kampf etwas Besonderes einfallen lassen.

Das Schlachtfeld war der Hogwarts-See. Und auch nicht in Booten. Sie kämpften unter Wasser. Der Riesenkrake war vorübergehend gelähmt worden; es waren Zauber eingesetzt worden, um die Grindylows fernzuhalten; Professor Quirrell hatte mit dem Meervolk gesprochen; und alle Soldaten hatten Unterwassertränke erhalten, die es ihnen erlaubten, zu atmen, klar zu sehen, miteinander zu reden und nicht ganz so schnell zu schwimmen wie ein schneller Spaziergang, indem sie mit den Beinen traten.

In der Mitte des Schlachtfeldes hing eine riesige silberne Kugel, die wie ein kleiner Unterwassermond leuchtete. Sie würde helfen, einen Orientierungssinn zu vermitteln - anfangs. Der Mond würde sich im Laufe des Kampfes langsam verfinstern, und wenn es ganz dunkel geworden war, würde der Kampf enden, wenn er es nicht schon getan hatte.

Krieg im Wasser. Man konnte keinen Umkreis verteidigen, Angreifer konnten aus jeder Richtung auf einen zukommen, und selbst mit dem Trank konnte man in der Dunkelheit des Sees nicht sehr weit sehen. Und wenn man zu weit weg vom Geschehen schwamm, fing man nach einer Weile an zu leuchten und war leicht zu jagen - normalerweise würde Professor Quirrell eine Armee, die sich zerstreut und wegläuft, anstatt zu kämpfen, einfach für besiegt erklären; aber heute arbeiteten sie mit einem Punktesystem.

Natürlich hatte man noch etwas Zeit, bevor man zu glühen begann, wenn man Assassine spielen wollte.

Die Drachenarmee war zu Beginn des Spiels tief ins Wasser gesetzt worden; darüber und in der Ferne leuchtete der ferne Unterwassermond.

Das trübe Wasser wurde größtenteils von Lumos-Zaubern beleuchtet, obwohl seine Soldaten die Lichter löschen würden, sobald sie ein Manöver begannen.

Es hatte keinen Sinn, sich vom Feind sehen zu lassen, bevor man ihn gesehen hatte. Draco trat ein paar Mal mit den Beinen, was ihn in eine höhere Position brachte, von der aus er auf die im Wasser schwebenden Soldaten hinunterblicken konnte.

Die Gespräche erstarben fast augenblicklich unter Dracos eisigem Blick, seine Soldaten blickten mit erfreuten Mienen der Angst und Sorge zu ihm auf.

"Hört mir ganz genau zu", sagte General Malfoy. Seine Stimme kam etwas leiser, ein wenig blasig und krächzend daher, aber der Ton kam deutlich rüber.

"Es gibt nur einen Weg, wie wir das hier gewinnen können. Wir müssen zusammen mit Chaos auf Sonnenschein marschieren und Sonnenschein besiegen. Dann kämpfen wir es mit Potter aus und gewinnen. Das muss passieren, verstanden? Egal, was sonst noch passiert, dieser Teil muss so ablaufen -"

Und Draco erklärte den Plan, den er und Harry sich ausgedacht hatten.

Erstaunte Blicke wurden unter den Soldaten ausgetauscht.

"- und wenn einer eurer Pläne dem in die Quere kommt", beendete Draco, "nachdem wir aus dem Wasser sind, werde ich euch in Brand setzen."

Ein nervöser Chor von "Ja" ertönte.

"Und alle, die geheime Befehle haben, sorgen dafür, dass ihr sie buchstabengetreu ausführt", sagte Draco.

\emph{Etwa die Hälfte seiner Soldaten nickte offen, und Draco markierte sie für den Tod, nachdem er die Macht übernommen hatte.}

Natürlich waren alle privaten Befehle gefälscht, wie zum Beispiel, dass ein Drache angewiesen wurde, einem anderen Drachen einen falschen Verräterauftrag zu geben, und dass der zweite Drache im stillen Vertrauen angewiesen wurde, alles zu melden, was der erste Drache sagte.

Draco hatte jedem Drachen gesagt, dass der ganze Krieg von dieser einen Sache abhängen könnte, und dass er hoffte, dass sie verstanden, dass das wichtiger war als die Pläne, die sie zuvor gemacht hatten. Mit etwas Glück würde das alle Idioten bei Laune halten und vielleicht auch ein paar Spione aufscheuchen, wenn die Berichte nicht mit den Anweisungen übereinstimmten.

Dracos wirklicher Plan, um gegen das Chaos zu gewinnen… nun, er war einfacher als der, den er verbrannt hatte, aber Vater hätte ihn trotzdem nicht gemocht.

Trotz aller Bemühungen war Draco nicht in der Lage gewesen, sich etwas Besseres einfallen zu lassen. Es war ein Komplott, das sich gegen niemanden außer Harry Potter hätte richten können. Tatsächlich war es ursprünglich Harrys Plan gewesen, wie der Verräter behauptete, obwohl Draco das vermutet hatte, ohne dass es ihm gesagt wurde. Draco und der Verräter hatten ihn nur ein wenig abgeändert…

Harry atmete tief ein und spürte, wie das Wasser harmlos in seiner Lunge gluckerte.

Sie hatten im Wald gekämpft und er war nicht dazu gekommen, es zu sagen. Sie hatten in den Gängen von Hogwarts gekämpft und er war nicht dazu gekommen, es zu sagen. Sie hatten in der Luft gekämpft, Besen an jeden Soldaten ausgegeben, und es hatte immer noch keinen Sinn gemacht, es zu sagen. Harry hatte geglaubt, er würde diese Worte nie sagen können, nicht solange er noch jung genug war, um sie wirklich zu hören… Die Chaoslegionäre sahen Harry verwundert an, während ihr General mit den Füßen nach oben in Richtung des fernen Lichts der Oberfläche schwamm und mit dem Kopf nach unten in die trüben Tiefen.

"\textbf{Warum steht ihr auf dem Kopf?}", rief der junge Kommandant seiner Armee zu und begann zu erklären, wie man kämpft, nachdem man die privilegierte Ausrichtung der Schwerkraft aufgegeben hat.

Eine hohle, dröhnende Glocke hallte durch das Wasser, und im selben Moment stürzten Zabini und Anthony und fünf weitere Soldaten nach unten, in die trüben Tiefen des Sees. Parvati Patil, die einzige Gryffindor in der Gruppe, drehte kurz den Kopf zurück und winkte allen aufmunternd zu, als sie abtauchte; und nach einem Moment taten Scott und Matt dasselbe. Der Rest ging einfach unter und verschwand. General Granger schluckte einen Kloß im Hals hinunter, als sie ihnen nachsah.

Sie riskierte alles damit, ihre Armee zu teilen, anstatt einfach zu versuchen, so viele feindliche Soldaten wie möglich mitzunehmen.

Die Sache, die sie begreifen musste, Zabini hatte ihr gesagt, war, dass keine Armee sich bewegen würde, bevor sie nicht einen Plan hatte, der sie den Sieg erwarten ließ.

Sonnenschein konnte nicht nur planen, selbst zu gewinnen, sie mussten die beiden anderen Armeen glauben lassen, sie würden gewinnen, bis es zu spät war.

Ernie und Ron sahen immer noch aus, als stünden sie unter Schock. Susan blickte den verschwindenden Soldaten mit einem berechnenden Blick nach. Ihre Armee, das, was davon übrig war, sah einfach nur verwirrt aus, Lichtspuren tupften auf ihre Uniformen, als sie alle knapp unter der sonnenbeschienenen Oberfläche des Sees trieben.

"Was jetzt?", fragte Ron.

"Jetzt warten wir", sagte Hermine, laut genug, dass alle Soldaten es hören konnten.

Es fühlte sich seltsam an, mit dem Mund voller Wasser zu sprechen, sie hatte ständig das Gefühl, irgendeine schreckliche Unhöflichkeit am Esstisch zu begehen und dabei zu sein, sich vollzusabbern.

"Wir alle, die hier übrig sind, werden gezappt werden, aber das wäre sowieso passiert, wenn Drache und Chaos sich gegen uns verbünden würden.

Wir müssen nur so viele wie möglich von ihnen mitnehmen."

"Ich habe einen Plan", sagte einer ihrer Sonnenschein Soldaten… Hannah, ihre Stimme war anfangs etwas schwer zu erkennen gewesen.

"Es ist ziemlich kompliziert, aber ich weiß, wie wir Drache und Chaos dazu bringen können, sich gegenseitig zu bekämpfen -"

"Ich auch!", sagte Fay. "Ich habe auch einen Plan! Schau, Neville Longbottom ist heimlich auf unserer Seite -"

"Du hast mit Neville geredet?", fragte Ernie.

"Das ist nicht richtig, ich war derjenige, der -"

Daphne Greengrass und ein paar andere Slytherins, die nicht mit Zabini gegangen waren, kicherten hilflos, als die Schreie "\emph{Nein, warte, ich war derjenige, der Longbottom geholt hat"} von einem Soldaten nach dem anderen ausbrachen.

Hermine sah sie alle nur müde an.

"Okay", sagte Hermine, als sich alles gelegt hatte, "hat es jeder verstanden? All eure Verschwörungen wurden von der Chaoslegion gefälscht, oder vielleicht auch von Drache. Jeder, der Harry oder Malfoy wirklich verraten wollte, ging direkt zu mir oder Zabini, nicht zu euch. Vergleicht doch einfach mal all eure geheimen Plots, dann werdet ihr es selbst sehen."

Sie war vielleicht nicht so gut im Intrigen schieden wie Zabini, aber sie konnte immer verstehen, was alle ihre Offiziere ihr sagten, deshalb hatte Professor Quirrell sie zum General gemacht.

"Also macht euch nicht die Mühe, irgendwelche Intrigen zu schmieden, wenn die anderen Armeen hier ankommen. Kämpft einfach, okay? Bitte?"

"Aber", sagte Ernie mit einem Schock im Gesicht, "Neville ist in Hufflepuff! Willst du damit sagen, dass er uns angelogen hat?"

Daphne lachte so sehr und so hilflos, dass die Ausdünstungen sie kopfüber ins Wasser geworfen hatten.

"Ich bin mir nicht sicher, was Longbottom ist", sagte Ron düster, "aber ich glaube nicht, dass er noch ein Hufflepuff ist. Nicht jetzt, wo Harry Potter ihn erwischt hat."

"Weißt du", sagte Susan, "ich habe ihn das gefragt, und Neville hat mir gesagt, er sei ein Chaos-Hufflepuff geworden."

"Wie auch immer", sagte Hermine mit lauter Stimme.

"Zabini hat jeden mitgenommen, von dem wir dachten, er sei ein Spion, also können wir uns jetzt nicht mehr ganz so streng beobachten, hoffe ich."

"Anthony war ein Spion?", schrie Ron.

"Parvati war eine Spionin?", keuchte Hannah.

"Parvati war auf jeden Fall eine Spionin", sagte Daphne. "Sie kaufte im Spionage-Schuhladen ein und trug Spionage-Lippenstift, und eines Tages wird sie einen netten Spionage-Ehemann heiraten und eine Menge kleiner Spione haben."

Und dann hallte ein Gongton durch das Wasser, der anzeigte, dass Sonnenschein gerade zwei Punkte erzielt hatte. Kurz darauf folgte der dreifache Gong, dass Drache einen Punkt verloren hatte. Verräter durften keine Generäle töten, nicht nach dem Desaster der ersten Schlacht im Dezember, als alle drei Generäle in der ersten Minute erschossen worden waren. Aber mit etwas Glück…

"Ach", sagte Hermine. "Es klingt, als würde Mr~Crabbe ein kleines Nickerchen machen."

Wie zwei Fischschwärme schwammen die Armeen dahin.

Neville Longbottom strampelte mit seinen Füßen in langsamen, gemessenen Bewegungen. Tauchen, immer tauchen, egal, in welche Richtung man sich gerade bewegte. Man wollte dem Feind das kleinste Profil zeigen, ihn mit dem Kopf oder den Füßen präsentieren. Man tauchte also immer, nach unten und mit dem Kopf voran, und der Feind war immer unten.

Wie jeder Chaoslegionär in der Armee drehte sich Nevilles Kopf ständig, während er schwamm, schaute nach oben, unten, herum, zu jeder Seite.

Er hielt nicht nur nach Sonnenschein-Soldaten Ausschau, sondern auch nach jedem Anzeichen dafür, dass ein Chaoslegionär seinen Stab gezogen hatte und im Begriff war, sie zu verraten.

Normalerweise warteten Verräter bis zum Schlachtgetümmel, um ihren Zug zu machen, aber dieser frühe Gong hatte sie alle in Alarmbereitschaft versetzt.

… die Wahrheit war, dass Neville darüber traurig war. Im November war er ein Soldat in einer vereinten Armee gewesen, in der alle an einem Strang zogen und sich gegenseitig halfen, und jetzt beobachteten sie sich alle ständig gegenseitig auf die ersten Anzeichen von Verrat. Es mochte für General Chaos mehr Spaß machen, aber für Neville war es nicht annähernd so viel Spaß. Die Richtung, die früher als \emph{"oben"}

bekannt war, wurde immer heller, je näher sie der Oberfläche und dem Sonnenschein kamen.

"Zauberstäbe raus", sagte General Chaos. Nevilles Truppe zog ihre Zauberstäbe und richtete sie geradeaus auf den Feind, während ihre Köpfe schneller umher scannten.

Wenn es Sonnen-Verräter gab, war es an der Zeit, dass sie zuschlugen. Der andere Fischschwarm, die Drachenarmee, tat das Gleiche.

"\textbf{Jetzt}!", rief die ferne Stimme des Drachengenerals. "\textbf{Jetzt}!", rief General Chaos.

"\textbf{Für Sonnenschein!}", schrien alle Soldaten beider Armeen und stürmten nach unten.

"Was?", sagte Minerva unwillkürlich, als sie die Bildschirme neben dem See beobachtete, ein Schrei, der an vielen anderen Orten widerhallte; ganz Hogwarts beobachtete diese Schlacht, wie sie die erste beobachtet hatte.

Professor Quirrell lachte trocken. "Ich habe Sie gewarnt, Schulleiter. Es ist unmöglich, Regeln zu haben, ohne dass Mr~Potter sie ausnutzt."

Für lange kostbare Sekunden, als die siebenundvierzig Soldaten ihre eigenen siebzehn angriffen, war Hermines Verstand leer.

\emph{Warum}…

Dann setzte sich alles zusammen. Jedes Mal, wenn ein Soldat, der ursprünglich aus Sonnenschein stammte, von jemandem erschossen wurde, der den Namen Sonnenschein rief, würde sie einen Quirrell-Punkt verlieren. Wenn zwei Sonnenschein-Soldaten von einer der beiden Armeen erschossen wurden, würden beide feindlichen Armeen zwei Punkte näher dran sein, sie zu überholen, es war der gleiche Gewinn, nur geteilt. Und wenn jemand einen anderen Soldaten erschoss, der nicht im Namen von Sonnenschein war, würde dieser Gong nicht in dem Durcheinander verloren gehen…

Hermine war plötzlich sehr froh, dass Zabini nicht den offensichtlichen Plan verfolgt hatte, Ärger zwischen den anderen beiden Armeen zu verursachen, während sie Sonnenschein angriffen. Es war trotzdem entmutigend, dieses Gefühl, dass die eigenen Chancen schwinden, dass einem die Hoffnung genommen wird. Die meisten von Hermines Soldaten schauten immer noch verwirrt, aber einige hatten Ausdrücke von dämmerndem Entsetzen, als sie es verstanden.

"Es ist alles in Ordnung", sagte Susan Bones fest. Die Köpfe drehten sich um und sahen den Sonnenschein Hauptmann an.

"Unsere Aufgabe ist es, so viele von ihnen mitzunehmen, wie wir können. Und denk daran, Zabini hat alle Spione mitgenommen! Wir müssen nicht so auf der Hut sein wie sie!"

Das Mädchen lächelte trotzig und provozierte damit ein erwiderndes Lächeln von vielen der anderen Soldaten, sogar von Hermine selbst.

"Es kann so sein wie im November. Wir müssen nur den Kopf hochhalten, unser Bestes geben und uns gegenseitig vertrauen -"

Daphne erschoss Sie.

"\textbf{Blut für den Blutgott}!", schrie Neville von Chaos, obwohl es, da er unter Wasser war, eher wie \emph{'Blubbled für den blubbled glub!}' herauskam. Captain Weasley wirbelte herum, hob seinen Zauberstab in Richtung Neville und feuerte.

Aber Neville schwamm abwärts auf ihn zu, den Zauberstab geradeaus gerichtet, und das bedeutete, dass der Einfache Schild Nevilles gesamtes Profil verdecken konnte; wenn jetzt jemand auf ihn schoss, dann nicht der Sonnen-Ron.

Ein grimmig entschlossener Blick ging über Captain Weasleys Gesicht, und er schoss pfeilschnell auf Neville zu, wobei er das Wort Contego aussprach, obwohl das Schild im Wasser nicht zu sehen war. Die beiden feindlichen Champions schossen aufeinander zu wie Pfeile aus Bögen, jeder mit dem Ziel, den anderen in der Mitte zu spalten. Sie hatten sich schon viele Male duelliert, aber dieses Mal würde sich alles auszahlen.

"\textbf{Regenbögen und Einhörner}!", brüllte der Sonnenschein Captain.

"\textbf{Die Schwarze Ziege mit den tausend Jungen!}"

"\textbf{Mach deine Hausaufgaben!}"

Näher und noch näher stürmten die beiden Champions, keiner wollte ausweichen, der erste, der sich umdrehte, würde eine verwundbare Breitseite präsentieren und erschossen werden, doch wenn keiner der beiden die Nerven verlor, würden sie direkt ineinander krachen… Sie fielen geradewegs zu Boden, während der Feind sich aufrichtete, um ihn zu treffen, der Hammer senkte sich, um auf den Amboss zu treffen, in einem Weg, den keiner von beiden verlassen wollte…

"Spezialangriff, Chaotische Drehung!"

Neville sah den Ausdruck des Entsetzens auf Captain Weasleys Gesicht, als der Schwebezauber ihn erwischte. Sie hatten ihn getestet, bevor der Kampf begonnen hatte; und genau wie Harry vermutet hatte, wurde \emph{Wingardium Leviosa} zu einer ganz neuen Art von Waffe, sobald alle unter Wasser schwammen.

"Verflucht seist du, Longbottom!?", kreischte Ron Weasley, "Kannst du nicht einmal ohne deine blöden Spezialattacken kämpfen -" und schon war der Sonnenschein-Kapitän auf die Seite gedreht und Neville schoss ihm ins Bein.

"Ich kämpfe nicht fair", sagte Neville zu der schlafenden Gestalt, "\emph{ich kämpfe wie Harry Potter.}"

Granger: 237 / Malfoy: 217 / Potter: 220 Es tat immer noch jedes Mal weh, wenn er auf Hermine schießen musste.

Harry konnte es kaum ertragen, den Ausdruck des Friedens zu betrachten, der über ihr schlafendes Gesicht gekommen war, die Arme, die nun ziellos umherschwebten, während die Kurven des Sonnenlichts über ihre Tarnuniform und die Wolke ihres kastanienbraunen Haars strichen. Und wenn Harry versucht hätte, sich davor zu drücken, derjenige zu sein, der sie erschießt … nicht nur Draco hätte gewusst, was das bedeutete, Hermine wäre beleidigt gewesen.

\emph{Sie ist nicht tot,} sagte Harry zu seinem Gehirn, als seine stampfenden Füße ihn wegstießen, sie ruht sich nur aus. \emph{IDIOT}.

\emph{Bist du sicher?} sagte sein Gehirn. \emph{Was, wenn sie eine Tote-Hermine ist? Könnten wir zurückgehen und nachsehen?}

Harry warf einen kurzen Blick zurück.

\emph{Siehst du, es geht ihr gut, da kommen Blasen aus ihrem Mund.}

\emph{Könnte ihr letzter Atemzug gewesen sein, der entweicht.}

\emph{Oh, sei still. Warum bist du überhaupt so paranoid-beschützend?}

\emph{Äh, der erste richtige Freund, den wir in unserem ganzen Leben hatten? Hey, weißt du noch, was mit unserem Lieblingsstein passiert ist?}

\emph{Er war nicht einmal lebendig, geschweige denn empfindungsfähig, das ist das erbärmlichste Kindheitstrauma aller Zeiten} -

Die beiden Armeen trennten sich schnell und wurden wieder zu zwei Fischschwärmen. General Granger hatte siebzehn Punkte verloren und drei Chaoten und zwei Drachen mitgenommen; und ein Chaot und zwei Drachen waren als Verräter erschossen worden. Sie hatte also netto sieben Punkte verloren, Harry einen, Draco zwei; damit hatte Sonnenschein zwanzig Punkte Vorsprung auf Drache und siebzehn Punkte auf Chaos.

Chaos konnte immer noch leicht gewinnen, wenn sie alle zwanzig verbliebenen Drachen auslöschten. Der Joker waren natürlich die sieben verbliebenen Sonnenschein-Soldaten… … wenn man sie so nennen konnte. Die beiden Schwärme schwammen unruhig nebeneinander. Die Soldaten warteten auf den Befehl, ihre wahre Zugehörigkeit zu verkünden und anzugreifen.

"Alle, die sie haben", sagte Harry laut, "erinnern sich an die Sonderbefehle eins bis drei. Und vergesst nicht, dass es Merlins Befehl auf Drei ist. „

Die vertrauenswürdigen zwei Drittel der Armee nickten nicht, und das andere Drittel schaute nur verwirrt.“

Die Drachentruppen begannen, auf Draco zuzusteuern, und die Chaotentruppen wirbelten herum und begannen sofort, die Drachen zu verfolgen - Draco fluchte laut, als er die Glocke eines chaotischen Sieges hörte, da hatte jemand seinen Einfachen Schild nicht richtig ausgerichtet - und dann waren die Drachentruppen in Unterstützungsreichweite zueinander und die Chaoten zogen sich in die düstere Ferne zurück. Irgendwie hatten die Drachen trotz ihrer zahlenmäßigen Überlegenheit dreimal gegen die Chaoslegion getroffen und die Chaoslegion hatten viermal zurückgeschlagen, und er hatte gehört, wie ein Drachenspion hingerichtet wurde.

Entweder hatte sich Harry Potter sehr schnell eine Menge sehr guter Ideen einfallen lassen, oder er hatte aus irgendeinem unvorstellbaren Grund schon sehr viel Zeit damit verbracht, herauszufinden, wie man unter Wasser kämpfen konnte.

Das funktionierte nicht und Draco musste die Dinge neu überdenken. Es sah so aus, als hätten alle auch beim Schwimmen Probleme mit dem Zielen, der Kampf könnte so lange dauern, dass die Zeit stehen bliebe… der ferne Unterwassermond war nur noch halb voll, das war nicht gut… er musste schnell umdenken…

"Was ist los?", fragte Padma Patil, als sie mit ihrer Truppe zu Draco hinüberschwamm. Padma war seine Stellvertreterin; sie war klug und mächtig, und besser noch, sie hasste Granger und sah Harry als Rivalen, was sie vertrauenswürdig machte. Die Arbeit mit Padma ließ ihn die Wahrheit des alten Sprichworts erkennen, dass Ravenclaw die Schwester von Slytherin war; Draco war überrascht gewesen, als sein Vater ihm gesagt hatte, es sei ein akzeptables Haus für seine zukünftige Frau, aber jetzt sah er den Sinn darin.

"Warte, bis wir alle hier sind", sagte Draco. Die Wahrheit war, dass er erst einmal zu Atem kommen musste. Das war das Problem, wenn man der General und der mächtigste Zauberer war, man musste ständig Magie anwenden.

Als nächstes kam Zabini, der eine Truppe von zwei Sonnen und vier Drachen befehligte, von denen einer Gregory war, der ein Auge auf Zabini hatte.

Draco hat Zabini nicht getraut. Und weder Draco noch Zabini vertrauten den Sonnen genug, um sie zur Mehrheit einer Einheit zu machen; sie sollten entweder Draco direkt oder Granger gegenüber loyal sein, die durch das Versprechen getäuscht worden war, dass die Drachen am Ende verraten würden, nachdem beide Streitkräfte dezimiert worden waren, genauso wie Harrys vertrauenswürdigere Chaoten durch das Versprechen, dass sie falsche Schlafverhexungen abfeuern und später zur Unterstützung des Chaos wechseln würden, dazu gebracht werden sollten, nicht auf die Sonnen zu schießen; aber es war möglich, dass einige der Sonnen dem Chaos treu waren und keine echten Schlafverhexungen abfeuerten, und dass dies der Grund war, warum Drache nicht so gewann, wie es ihr zahlenmäßiger Vorteil hätte zulassen sollen.

.. Die nächste Einheit, die sich näherte, war dezimiert, drei Soldaten hielten Zauberstäbe auf zwei andere Soldaten, die mit leeren Händen schwammen.

Draco knirschte mit den Zähnen. Noch mehr Probleme mit Verrätern. Er musste mit Professor Quirrell darüber reden, dass es wenigstens eine Möglichkeit gab, Verräter zu bestrafen, Zustände wie diese waren unrealistisch, im richtigen Leben folterte man seine Verräter zu Tode.

"General Malfoy!", rief der Kommandant der Problemeinheit, als sie heranschwamm, ein Ravenclaw-Junge namens Terry.

"Wir wissen nicht, was wir tun sollen - Cesi hat Bogdan erschossen, aber Cesi sagt, dass Kellah ihm gesagt hat, dass Bogdan Specter erschossen hat -"

"Das habe ich nicht!", sagte Kellah.

"Doch, hast du!", kreischte Cesi.

"General Malfoy, sie ist die Spionin, ich hätte sie -"

"Somnium", sagte Draco.

Es gab die dreifache Glocke eines Ein-Punkt-Verlustes von Drache, und dann begann Kellahs schlaffer Körper im Wasser davon zu treiben. Draco hatte zu diesem Zeitpunkt schon das Wort "\emph{Rekursion}" gehört, und er erkannte einen Harry-Potter-Plot, wenn er einen sah.

(Leider hatte Draco noch nie etwas von Autoimmunerkrankungen gehört, und der Gedanke, dass ein cleverer Virus seinen Angriff damit beginnen würde, Symptome einer Autoimmunerkrankung zu erzeugen, um den Körper dazu zu bringen, seinem eigenen Immunsystem zu misstrauen, kam ihm nicht ohne weiteres in den Sinn…)

"Allgemeiner Befehl!?", sagte Draco und erhob seine Stimme.

"Niemand darf Spione erschießen, außer mir, Gregory, Padma und Terry.

Wenn jemand etwas Verdächtiges sieht, kommt er zu einem von uns."

Und dann - da ertönte die Glocke von Sonnenschein, die zwei Punkte erzielte.

"Was?", sagten Draco und Zabini etwa gleichzeitig; ihre Köpfe schwenkten herum. Niemand schien getroffen worden zu sein, und alle Sonnenschein-Soldaten waren anwesend und anwesend.

(Außer Parvati, die von irgendeinem noch unbekannten Verräter in Padmas Truppe angeschossen worden war; und natürlich hatte Padma wieder auf Parvati geschossen, für den Fall, dass sie nur simuliert hatte, also war sie es nicht…)

"Ein Sonnen-Verräter im Chaos?", fragte Zabini und klang verwirrt.

"Aber alle, von denen ich wusste, dass sie während des Angriffs von Chaos auf Sonnenschein zuschlagen sollten -"

"Nein!", sagte Padma in einem Ton der plötzlichen Erkenntnis.

"Das war Chaos, das einen Spion exekutiert hat!"

"Was?", sagte Zabini. "Aber warum -"

Und Draco bekam es mit.

\emph{Verdammt noch mal!}

"Weil Potter denkt, er sei sicher, weil er Sonnenschein schlägt, aber nicht, weil er uns schlägt! Also will er keinen einzigen Punkt verlieren, wenn er einen Verräter hinrichtet! Allgemeiner Befehl! Wenn Du einen Verräter hinrichten musst, rufe zuerst Sonnenschein! Und vergesst nicht, danach wieder zu Drache zu wechseln -"

Granger: 253 / Malfoy: 252 / Potter: 252 Longbottoms Körper trieb chaotisch durch das Wasser, Arme und Beine wirr durcheinander.

Nachdem Draco endlich einen Treffer gelandet hatte, hatten sie alle noch einmal auf ihn geschossen, nur um sicherzugehen. In der Nähe befand sich Harry Potter, der nun von einer prismatischen Kugel geschützt wurde und sie alle grimmig ansah, während der letzte Splitter der Mondsichel langsam abnahm, irgendwo weit weg.

Wenn Longbottom es geschafft hätte, einen weiteren Soldaten zu erschießen (Draco wusste, dass Harry dachte), wenn die beiden Chaoten es geschafft hätten, nur ein wenig länger durchzuhalten, hätten sie vielleicht gewonnen…

Nachdem Draco seine Streitkräfte neu formiert hatte und wieder losgezogen war, hatte die darauf folgende Schlacht und die Hinrichtung von Spionen im Namen von Sonnenschein dazu geführt, dass Sonnenschein genau einen Punkt vor Drache und Chaos lag. Nachdem Harry damit angefangen hatte, war Draco keine andere Wahl geblieben, als dem Beispiel zu folgen. Aber jetzt waren sie General Chaos, den Überlebenden der Drache-Armee und dem letzten verbliebenen Verräter von Sonnen drei zu eins unterlegen: Draco, und Padma, und Zabini.

Und Draco, der kein Narr war, hatte Padma befohlen, Zabinis Zauberstab zu nehmen, nachdem Longbottom auf Gregory geschossen hatte und seinerseits zu Draco gefallen war. Der Junge hatte ihm einen beleidigten Blick zugeworfen, Draco gesagt, dass er ihm das schuldig sei, und ihn ausgehändigt.

Damit blieben Draco und Padma übrig, um General Chaos zur Strecke zu bringen.

"Ich nehme nicht an, dass Sie sich ergeben wollen?", sagte Draco und lächelte so böse wie jedes Lächeln, das er jemals Harry Potter zugewandt hatte.

"Schlafen, vor der Kapitulation!", rief General Chaos.

"Nur damit du es weißt", sagte Draco, "Zabini hat nicht wirklich eine ältere Schwester, die du vor den Gryffindor-Tyrannen retten kannst. Aber Zabini hat eine Mutter, die Muggelgeborene wie Granger nicht gutheißt, und ich habe ihr ein paar Notizen geschrieben und Zabini ein paar Gefallen angeboten - nichts, was mit meinem Vater zu tun hat, nur Dinge, die ich in der Schule tun kann. Und übrigens, Zabinis Mutter billigt den Jungen, der gelebt hat, auch nicht. Nur für den Fall, dass du immer noch glaubst, Zabini sei wirklich auf deiner Seite."

Harrys Gesicht wurde noch grimmiger.

Draco hob seinen Zauberstab und begann rhythmisch zu atmen, um Kraft für eine Schildbrecher Verhexung zu sammeln.

Grangers Prismatische Kugel war jetzt fast so stark wie die von Draco, und Harrys war nicht viel schwächer, \emph{wo nahmen die beiden nur die Zeit her?}

"Lagann!", sprach Draco und setzte alles ein, was er hatte, und die grüne Spirale loderte auf und Harrys Schild zerbrach, und fast im selben Moment -

"Somnium!", sagte Padma.

Granger: 253 / Malfoy: 252 / Potter: 254 Harry stieß einen langen Atemzug der Erleichterung aus, und das nicht nur, weil er die Prismatische Kugel nicht mehr halten musste. Seine Hand zitterte, als er seinen Zauberstab senkte.

"Weißt du", sagte Harry, "ich war einen Moment lang ziemlich besorgt."

Spezialauftrag zwei: Wenn ein Sonnen-Verräter nicht wirklich auf dich zu schießen scheint, täusche gelegentlich vor, getroffen zu werden.

Zielen Sie lieber auf Drachen als auf Sonnen, aber schießen Sie ruhig auf Sonnen, wenn Sie nicht auf Drachen schießen können.

Sonderauftrag 3: Merlin sagt, schießen Sie nicht auf Blaise Zabini oder einen der Patil-Zwillinge.

Mit einem breiten Grinsen streifte Parvati Patil den verwandelten Aufnäher von den Insignien ihrer Uniform ab und ließ ihn im Wasser davonschwimmen.

"Gryffindors für Chaos", sagte sie und reichte Zabini seinen Zauberstab zurück.

"Vielen Dank", sagte Harry und verbeugte sich schwungvoll vor dem Gryffindor-Mädchen.

"Und ich danke Ihnen auch", verbeugte er sich vor Zabini.

"Wissen Sie, als Sie mit diesem Plan zu mir kamen, habe ich mich gefragt, ob Sie brillant oder verrückt sind, und ich habe beschlossen, dass Sie beides sind.

Und übrigens", sagte Harry und drehte sich nun um, als wolle er Dracos Körper ansprechen, "Zabini hat einen Cousin -"

"Somnium", sagte Zabinis Stimme.

Granger: 255 / Malfoy: 252 / Potter: 254 Und Harry Potters Körper schwebte davon,

sein Ausdruck des Schocks und des Entsetzens entspannte sich schnell in den Schlaf.

"Wenn ich es mir recht überlege", sagte Parvati fröhlich, "machen wir Gryffindors für Sonnenschein." Sie begann zu lachen, so erfreut wie noch nie in ihrem Leben, sie hatte endlich ihre Zwillingsschwester ermorden und ersetzen können, und das wollte sie schon seit Ewigkeiten tun, und das war perfekt gewesen, es war alles perfekt gewesen - - und dann drehte sich ihr Zauberstab blitzschnell herum, gerade als Zabinis Zauberstab sich drehte, um auf sie zu zeigen.

"Warte!", sagte Zabini. "Nicht schießen, keinen Widerstand leisten. Das ist ein Befehl."

"Was?", sagte Parvati.

"Tut mir leid", sagte Zabini und sah nicht ganz aufrichtig entschuldigend aus,

"aber ich kann mir nicht ganz sicher sein, dass du für Sonnenschein bist.

Also befehle ich dir, dich von mir erschießen zu lassen."

"Moment mal!?", sagte Parvati. "Wir sind dem Chaos nur einen Punkt voraus!

Wenn du mich jetzt erschießt -"

"Ich erschieße dich natürlich im Namen der Drachen", sagte Zabini, der jetzt ein wenig überlegen klang.

"Nur weil wir sie ausgetrickst haben, heißt das nicht, dass es bei mir auch funktioniert." Parvati starrte ihn an, ihre Augen verengten sich.

"General Malfoy hat gesagt, deine Mutter mag Hermine nicht."

"Das nehme ich an", sagte Zabini, immer noch mit diesem überlegenen Grinsen.

"Aber einige von uns sind eher bereit als Draco Malfoy, ein Elternteil zu ärgern."

"Und Harry Potter sagte, du hast einen Cousin -"

"Nö", sagte Zabini.

Parvati starrte ihn an und versuchte nachzudenken, aber sie war nicht wirklich gut im Plotten; Zabini hatte gesagt, der Plan sei gewesen, den Punktestand von Chaos und Drache heimlich so ausgeglichen wie möglich zu halten, damit sie Sonnenscheins Namen benutzen würden, um ihre Verräter hinzurichten, anstatt auch nur einen einzigen Punkt zu verlieren, und das hatte funktioniert… aber… sie hatte das Gefühl, etwas zu übersehen, sie war keine Slytherin…

"Warum erschieße ich dich nicht im Namen des Drachen?", fragte Parvati.

"Weil ich im Rang über dir stehe", sagte Zabini.

Parvati hatte ein ungutes Gefühl dabei. Sie starrte ihn einen langen Moment lang an.

Und dann - "Somni-", begann sie zu sagen, und merkte dann, dass sie nicht für Drache gesagt hatte, und unterbrach sich hektisch -

Granger: 255 / Malfoy: 254 / Potter: 254

"Hey, Leute", sagte Blaise Zabinis Gesicht auf den Bildschirmen und sah ziemlich amüsiert aus, "ich schätze, es liegt alles an mir."

Überall am Seeufer hielten die Leute den Atem an.

Sonnenschein lag mit genau einem Punkt Vorsprung vor Drache und Chaos. Blaise Zabini konnte sich entweder im Namen von Drache oder Chaos erschießen, oder die Dinge einfach so lassen, wie sie waren.

Eine Reihe von Glockenspielen zeigte an, dass die letzte Minute der Zeit ablief. Und der Slytherin lächelte ein seltsames, verdrehtes Lächeln und spielte beiläufig mit seinem Zauberstab, das dunkle Holz kaum sichtbar im dunklen Wasser.

"Wisst ihr", sagte Blaise Zabinis Stimme, im Tonfall von jemandem, der die Worte schon eine Weile einstudiert hatte, "es ist nur ein Spiel, wirklich. Und Spiele sollen doch Spaß machen. Wie wäre es also, wenn ich einfach tue, worauf ich Lust habe?"

