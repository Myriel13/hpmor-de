

\hypertarget{reduktionismus}{% \section{28. Reduktionismus}\label{reduktionismus}}

\textbf{\uline{Reduktionismus}}

„Okay“, sagte Harry und schluckte. „Okay, Hermine, es ist genug, du kannst aufhören.“

Die weiße Zuckerpille vor Hermine hatte immer noch nicht ihre Form oder Farbe verändert, obwohl sie sich so sehr konzentrierte, wie Harry es noch nie gesehen hatte, die Augen zusammengekniffen, Schweißperlen auf der Stirn, die Hand zitterte, als sie den Zauberstab umklammerte—

„Hermine, hör auf! Es wird nicht funktionieren, Hermine, ich glaube nicht, dass wir Dinge erschaffen können, die noch nicht existieren!“

Langsam lockerte Hermines Hand ihren Griff um den Zauberstab.

„Ich dachte, ich hätte es gespürt“, sagte sie im leisen Flüsterton.

„Ich dachte, ich hätte gespürt, wie er anfing zu verwandeln, nur für eine Sekunde.“

Ein Kloß steckte in Harrys Hals.

„Du hast es dir wahrscheinlich nur eingebildet. Zu sehr gehofft.“

„Wahrscheinlich habe ich das“, sagte sie.

Sie sah aus, als wollte sie weinen. Langsam nahm Harry seinen Druckbleistift in die Hand und griff nach dem Blatt Papier, auf dem alle Punkte durchgestrichen waren, und zog einen Strich durch den Punkt, auf dem 'ALZHEIMER MEDIZIN' stand.

Sie hätten niemandem eine verwandelte Pille geben können. Aber Verwandlung, zumindest die Art, die sie beherrschten, verzauberte die Ziele nicht - es würde keinen normalen Besen in einen fliegenden verwandeln.

Wenn Hermine also überhaupt in der Lage gewesen wäre, eine Pille herzustellen, wäre es eine nichtmagische Pille gewesen, eine, die aus gewöhnlichen materiellen Gründen funktionierte.

Sie hätten heimlich Pillen für ein Muggel-Wissenschaftslabor herstellen können, sie die Pillen studieren lassen und versuchen können, sie nachzubauen, bevor die Verwandlung nachließ.

.. niemand in beiden Welten hätte wissen müssen, dass Magie im Spiel war, es wäre einfach ein weiterer wissenschaftlicher Durchbruch gewesen.

.. Es war auch nicht die Art von Sache, an die ein Zauberer denken würde. Sie respektierten bloße Muster von Atomen nicht so sehr, sie hielten unverzauberte materielle Dinge nicht für Objekte der Macht. Wenn es nicht magisch war, war es uninteressant.

Zuvor hatte Harry ganz heimlich - er hatte es nicht einmal Hermine erzählt - versucht, Nanotechnologie a la Eric Drexler zu verwandeln.

(Er hatte natürlich versucht, eine Desktop-Nanofabrik herzustellen, keine winzigen selbstreplizierenden Assembler, Harry war nicht verrückt.)

Es wäre auf einen Schlag göttliche Macht gewesen, wenn es funktioniert hätte.

„Das war's für heute, oder?“, sagte Hermine.

Sie war in ihrem Stuhl nach hinten gesackt und hatte den Kopf gegen die Lehne gelehnt; und ihr Gesicht zeigte ihre Müdigkeit, was sehr ungewöhnlich für Hermine war. Sie tat gerne so, als wäre sie grenzenlos, zumindest wenn Harry in der Nähe war.

„Noch eine“, sagte Harry vorsichtig, "aber die ist klein, außerdem könnte sie tatsächlich funktionieren. Ich habe es mir für den Schluss aufgehoben, weil ich gehofft habe, dass wir mit einem guten Gefühl abschließen können. Das ist echtes Zeug, nicht wie Phaser. Sie haben es bereits im Labor hergestellt, nicht wie das Alzheimer Heilmittel.

Und es ist eine generische Substanz, nicht spezifisch wie die verlorenen Bücher, von denen du versucht hast, Kopien zu verwandeln.

Ich habe ein Diagramm der Molekularstruktur angefertigt, um es dir zu zeigen. Wir wollen es nur länger machen, als es jemals zuvor gemacht wurde, und mit allen Röhren in einer Linie, und die Endpunkte in einen Diamanten eingebettet."

Harry holte ein Blatt Millimeterpapier hervor.

Hermine setzte sich wieder auf, nahm es, studierte es und runzelte die Stirn.

„Das sind alles Kohlenstoffatome? Und Harry, wie lautet der Name? Ich kann es nicht verwandeln, wenn ich nicht weiß, wie es heißt.“

Harry machte ein angewidertes Gesicht. Er hatte immer noch Schwierigkeiten, sich an so etwas zu gewöhnen, es sollte egal sein, wie etwas hieß, wenn man wusste, was es war.

„Man nennt sie Buckytubes oder Kohlenstoff-Nanoröhrchen. Es ist eine Art Material, das erst dieses Jahr entdeckt wurde. Es ist etwa hundertmal stärker als Stahl und hat nur ein Sechstel des Gewichts.“

Hermine blickte von dem Millimeterpapier auf, ihr Gesicht war überrascht.

„Das ist echt?“

„Ja“, sagte Harry, „es ist nur schwer auf Muggelart zu machen. Wenn wir genug von dem Zeug bekämen, könnten wir damit einen Weltraumfahrstuhl bauen, der bis in eine geosynchrone Umlaufbahn oder höher reicht, und was das Delta-V angeht, ist das der halbe Weg zu jedem Ort im Sonnensystem. Außerdem könnten wir Solarenergiesatelliten wie Konfetti auswerfen.“

Hermine runzelte wieder die Stirn. „Ist dieses Zeug sicher?“

„Ich wüsste nicht, warum es das nicht sein sollte“, sagte Harry. „Ein Buckytube ist im Grunde nur ein Graphitblatt, das in ein rundes Rohr eingewickelt ist, und Graphit ist das gleiche Zeug, das in Bleistiften verwendet wird—“

„Ich weiß, was Graphit ist, Harry“, sagte Hermine.

Sie strich sich geistesabwesend die Haare zurück und starrte mit gerunzelten Augenbrauen auf das Blatt Papier.

Harry griff in eine Tasche seines Umhangs und holte einen weißen Faden hervor, der mit zwei kleinen grauen Plastikringen an beiden Enden verbunden war.

Dort, wo der Faden mit den beiden Ringen zusammentraf, hatte er ein paar Tropfen Sekundenkleber hinzugefügt, um das Ganze zu einem einzigen Objekt zu machen, das als Ganzes verwandelt werden konnte.

Cyanacrylat funktionierte, wenn Harry sich richtig erinnerte, durch kovalente Bindungen, und das war so nahe an einem „festen Objekt“, wie man es in einer Welt, die letztlich aus winzigen einzelnen Atomen bestand, überhaupt bekommen konnte.

„Wenn du bereit bist“, sagte Harry, „versuch, dies in einen Satz ausgerichteter Buckytube-Fasern zu transformieren, die in zwei massive Diamantringe eingebettet sind.“

„In Ordnung…“ sagte Hermine langsam. „Harry, ich habe das Gefühl, ich habe gerade etwas übersehen.“

Harry zuckte hilflos mit den Schultern.

\emph{Vielleicht bist du nur müde.}

Er wusste es aber besser, als es laut auszusprechen. Hermine legte ihren Zauberstab an einen Plastikring und starrte ihn eine Weile lang an.

Zwei kleine Kreise aus glitzerndem Diamant lagen auf dem Tisch, verbunden durch einen langen schwarzen Faden.

„Es hat sich verändert“, sagte Hermine. Sie klang, als ob sie versuchte, enthusiastisch zu sein, aber ihr war die Energie ausgegangen.

„Und was jetzt?“

Harry fühlte sich durch den Mangel an Leidenschaft seiner Forschungspartnerin ein wenig deflationiert, tat aber sein Bestes, um es nicht zu zeigen; vielleicht würde der gleiche Prozess in umgekehrter Richtung funktionieren, um sie aufzumuntern.

„Jetzt teste ich, ob es das Gewicht hält.“

Es gab ein A-Gestell, das Harry für ein früheres Experiment mit Diamantstäben zusammengebastelt hatte - man konnte mit Hilfe von Verwandlung leicht massive Diamantobjekte herstellen, sie würden nur nicht halten.

Bei dem früheren Experiment war gemessen worden, ob man durch Verwandlung eines langen Diamantstabes in einen kürzeren Diamantstab ein hängendes schweres Gewicht anheben konnte, während es sich zusammenzog, d.h., ob man gegen die Spannung verwandeln konnte, was man in der Tat konnte.

Harry schlang vorsichtig einen Kreis des glitzernden Diamanten über den dicken Metallhaken am oberen Ende der Vorrichtung, dann befestigte er einen dicken Metallbügel am unteren Ring und begann dann, Gewichte an dem Bügel zu befestigen.

(Harry hatte die Weasley-Zwillinge gebeten, den Apparat für ihn zu verwandeln, und die Weasley-Zwillinge hatten ihn ungläubig angeschaut, als könnten sie sich nicht vorstellen, für welche Art von Streich er das wohl brauchen könnte, aber sie hatten keine Fragen gestellt. Und ihre Verwandlungen dauerten nach ihren Angaben etwa drei Stunden, also hatten Harry und Hermine noch eine Weile Zeit).

„Einhundert Kilogramm“, sagte Harry etwa eine Minute später. „Ich glaube nicht, dass ein so dünnes Stahlseil das halten würde. Es müsste noch viel höher gehen, aber mehr Gewicht habe ich nicht.“

Es herrschte eine weitere Stille.

Harry richtete sich auf und ging zurück zu ihrem Tisch, setzte sich auf seinen Stuhl und machte feierlich ein Häkchen neben „\emph{Buckytubes}“.

„Da“, sagte Harry. „Das hat funktioniert.“

„Aber es ist nicht wirklich nützlich, Harry, oder?“ sagte Hermine von dort aus, wo sie saß und ihren Kopf in die Hände gestützt hatte.

„Ich meine, selbst wenn wir sie einem Wissenschaftler geben würden, könnten sie nicht lernen, wie man viele Buckytubes herstellt, wenn sie unsere studieren.“

„Vielleicht können sie etwas lernen“, sagte Harry. „Hermine, sieh es dir an, dieser winzige Faden, der das ganze Gewicht hält, wir haben gerade etwas gemacht, das kein Muggel-Labor herstellen könnte—“

„Aber jede andere Hexe könnte es herstellen“, sagte Hermine. Ihre Erschöpfung machte sich jetzt in ihrer Stimme bemerkbar. „Harry, ich glaube, das klappt nicht.“

„Du meinst unsere Beziehung? Toll! Lass uns Schluss machen.“

Das entlockte ihr ein leichtes Grinsen. „Ich meine unsere Forschung.“

„Oh, Hermine, wie \textbf{kannst} du nur?“

„Du bist süß, wenn du böse bist“, sagte sie. „Aber Harry, das ist verrückt, ich bin zwölf, du bist elf, es ist albern zu glauben, dass wir etwas entdecken werden, was noch niemand herausgefunden hat.“

„Willst du wirklich sagen, wir sollen aufgeben, die Geheimnisse der Magie zu enträtseln, nachdem wir es weniger als einen Monat lang versucht haben?“ sagte Harry und versuchte, einen Ton der Herausforderung in seine Stimme zu legen. Ehrlich gesagt fühlte er ein wenig die gleiche Müdigkeit wie Hermine. Keine der guten Ideen funktionierte jemals. Er hatte nur eine erwähnenswerte Entdeckung gemacht, das Mendelsche Muster, und er konnte Hermine nicht davon erzählen, ohne sein Versprechen an Draco zu brechen.

„Nein“, sagte Hermine. Ihr junges Gesicht wirkte sehr ernst und erwachsen.

„Ich will damit sagen, dass wir jetzt erst einmal die ganze Magie studieren sollten, die Zauberer bereits kennen, damit wir nach unserem Abschluss in Hogwarts so etwas machen können.“

„Ähm…“ sagte Harry. „Hermine, ich sage es nur ungern, aber stell dir vor, wir hätten beschlossen, mit dem Forschen bis später zu warten, und das erste, was wir nach unserem Abschluss versuchen, wäre, ein Alzheimer-Heilmittel zu verwandeln, und es würde funktionieren. Wir würden uns… Ich denke nicht, dass das Wort \emph{“dumm"} angemessen beschreiben würde, wie wir uns fühlen würden.

Was, wenn es noch so etwas gibt und es funktioniert?"

„Das ist nicht fair, Harry!“ sagte Hermine. Ihre Stimme zitterte, als wäre sie kurz davor, in Tränen auszubrechen. „Das kannst du den Leuten nicht zumuten! Es ist nicht unsere Aufgabe, so etwas zu tun, wir sind Kinder!“

Einen Moment lang fragte sich Harry, was passieren würde, wenn jemand Hermine sagen würde, dass sie gegen einen unsterblichen Dunklen Lord kämpfen müsste, ob sie sich in einen der weinerlichen, selbstmitleidigen Helden verwandeln würde, von denen Harry es nie ertragen konnte, in seinen Büchern zu lesen.

„Wie auch immer“, sagte Hermine. Ihre Stimme zitterte. „Ich will das nicht mehr machen. Ich glaube nicht, dass Kinder Dinge tun können, die Erwachsene nicht können, das gibt es nur in Geschichten.“

Im Klassenzimmer herrschte Stille. Hermine begann, ein wenig ängstlich auszusehen, und Harry wusste, dass sein eigener Ausdruck kälter geworden war. Es hätte vielleicht nicht so wehgetan, wenn Harry nicht schon derselbe Gedanke gekommen wäre - dass, obwohl dreißig für einen wissenschaftlichen Revolutionär alt sein mochte und zwanzig ungefähr richtig war, es zwar Leute gab, die mit siebzehn promoviert hatten, und vierzehnjährige Erben, die große Könige oder Generäle gewesen waren, aber es gab nicht wirklich jemanden, der es mit elf in die Geschichtsbücher geschafft hatte.

„In Ordnung“, sagte Harry. „Finde heraus, wie man etwas tun kann, was ein Erwachsener nicht kann. Ist das deine Herausforderung?“

„So habe ich das nicht gemeint“, sagte Hermine, ihre Stimme kam in einem erschrockenen Flüstern heraus.

Mühsam riss Harry seinen Blick von Hermine los. „Ich bin nicht böse auf dich“, sagte Harry. Seine Stimme war kalt, obwohl er sich bemühte. "Ich bin wütend auf, ich weiß nicht, auf alles. Aber ich bin nicht bereit zu verlieren, Hermine. Verlieren ist nicht immer das Richtige.

Ich werde herausfinden, wie ich etwas tun kann, was ein erwachsener Zauberer nicht tun kann, und dann melde ich mich bei dir. Wie wär's damit?"

Wieder herrschte Schweigen.

„Okay“, sagte Hermine, ihre Stimme schwankte ein wenig.

Sie erhob sich von ihrem Stuhl und ging hinüber zur Tür des verlassenen Klassenzimmers, in dem sie gearbeitet hatten. Ihre Hand legte sich auf den Türknauf. „Wir sind doch noch Freunde, oder? Und wenn du dir nichts einfallen lässt—“

Ihre Stimme stockte.

„Dann werden wir zusammen lernen“, sagte Harry. Seine Stimme war jetzt noch kälter.

„Ähm, tschüss dann“, sagte Hermine, ging schnell aus dem Zimmer und schloss die Tür hinter sich.

Manchmal hasste Harry es, eine dunkle Seite zu haben, selbst wenn er in ihr war. Und der Teil von ihm, der genau dasselbe gedacht hatte wie Hermine, dass nämlich Kinder nicht tun konnten, was Erwachsene nicht konnten, sagte all die Dinge, vor denen Hermine zu viel Angst gehabt hatte, um sie auszusprechen, wie zum Beispiel:

\emph{Das ist eine verdammt schwierige Herausforderung, die du dir gerade geschnappt hast, und Junge, dieses Mal wirst du mit Eiern im Gesicht enden, und so weißt du wenigstens, dass du versagt hast.}

Und der Teil von ihm, der es nicht genoss zu verlieren, antwortete mit sehr kalter Stimme:

\emph{"Schön, du kannst die Klappe halten und zusehen.}

Es war fast Mittagszeit und Harry war es egal. Er hatte sich nicht einmal die Mühe gemacht, einen Snack-Riegel aus seinem Beutel zu holen.

Sein Magen konnte ein wenig Hunger vertragen.

\emph{Die Zaubererwelt war winzig, sie dachten nicht wie Wissenschaftler, sie kannten keine Wissenschaft, sie hinterfragten nicht, womit sie aufgewachsen waren, sie hatten keine Schutzhüllen an ihren Zeitmaschinen angebracht, sie spielten Quidditch, ganz magisches Britannien war kleiner als eine kleine Muggelstadt, die größte Zaubererschule bildete nur bis zum Alter von siebzehn Jahren aus, dumm war, das mit elf Jahren nicht in Frage zu stellen, dumm war, anzunehmen, dass Zauberer wussten, was sie taten und bereits alle tief hängenden Früchte ausgeschöpft hatten, die ein wissenschaftlicher Polymath sehen würde.}

Schritt Eins war gewesen, eine Liste aller magischen Beschränkungen zu erstellen, an die Harry sich erinnern konnte, all die Dinge, die man angeblich nicht tun konnte.

Schritt 2: Markieren der Beschränkungen, die aus wissenschaftlicher Sicht am wenigsten Sinn machten.

Schritt 3: Priorisieren Sie die Einschränkungen, die ein Zauberer wahrscheinlich nicht in Frage stellen würde, wenn er die Wissenschaft nicht kennt.

Schritt Vier: Überlegen Sie sich, wie man sie angreifen kann.

Hermine fühlte sich immer noch ein wenig zittrig, als sie sich neben Mandy an den Ravenclaw-Tisch setzte. Hermines Mittagessen bestand aus zwei Früchten (Tomatenscheiben und geschälte Mandarinen), drei Gemüsesorten (Karotten, Karotten und noch mehr Karotten), einem Fleisch (gebratene Diricawl-Keulen, deren ungesunden Belag sie vorsichtig entfernen würde) und einem kleinen Stück Schokoladenkuchen, das sie sich durch das Essen der anderen Dinge verdienen würde.

Es war nicht so schlimm gewesen wie der Zaubertrankunterricht, von dem sie manchmal immer noch Albträume hatte.

Aber diesmal hatte sie es geschafft, und sie hatte sich wie sein Ziel gefühlt. Nur für einen Moment, bevor die schrecklich kalte Dunkelheit wegschaute und sagte, dass sie nicht böse auf sie sei, weil sie ihr keine Angst machen wollte.

Und sie hatte immer noch dieses Gefühl, als hätte sie vorhin etwas verpasst, etwas wirklich Wichtiges.

\emph{Aber sie hatten keine der Regeln der Verwandlung verletzt… oder doch? Sie hatten keine Flüssigkeiten hergestellt, keine Gase, sie hatten keine Befehle des Verteidigungsprofessors befolgt.

.. Die Pille! Das war etwas, das gegessen werden sollte!…na ja, nein, niemand würde einfach so eine herumliegende Pille essen, sie hatte ja auch nicht funktioniert, sie hätten sie ja auch einfach verschwinden lassen können, aber sie würde Harry trotzdem davon erzählen müssen und dafür sorgen, dass sie es vor Professor McGonagall nicht erwähnten, falls sie nie wieder Verwandlung studieren durften.}

.. Hermine bekam langsam ein richtig schlechtes Gefühl im Magen. Sie schob ihren Teller vom Tisch zurück, so konnte sie nicht zu Mittag essen.

Sie schloss die Augen und begann, im Geiste die Regeln der Verwandlung zu rezitieren.

„\emph{Ich werde niemals etwas in Flüssigkeit oder Gas verwandeln.}“

„\emph{Ich werde nie etwas verwandeln, das wie Essen aussieht oder in einen menschlichen Körper passt.}“

Nein, sie hätten wirklich nicht versuchen sollen, die Pille zu verwandeln, oder sie hätten es zumindest merken sollen… sie war so in Harrys brillante Idee vertieft gewesen, dass sie nicht daran gedacht hatte… Das ungute Gefühl in Hermines Magen wurde immer schlimmer.

Sie hatte das Gefühl, dass etwas am Rande des Erkennens schwebte, eine Wahrnehmung, die im Begriff war, sich zu zeigen, eine junge Frau, die im Begriff war, eine alte Frau zu werden, eine Vase, die im Begriff war, zwei Gesichter zu bekommen… Und sie fuhr fort, sich an die Regeln der Verwandlung zu erinnern.

Harrys Fingerknöchel waren schon weiß geworden, als er aufhörte, die Luft vor seinem Zauberstab in eine Büroklammer zu verwandeln.

Es wäre natürlich nicht sicher gewesen, die Büroklammer in Gas zu verwandeln, aber Harry sah keinen Grund, warum es andersherum unsicher sein sollte. Es sollte einfach \emph{nicht} möglich sein. Aber warum nicht? Luft war eine ebenso reale Substanz wie alles andere… Nun, vielleicht machte diese Einschränkung doch Sinn. Luft war ungeordnet, alle Moleküle veränderten ständig ihr Verhältnis zueinander.

Vielleicht konnte man einer Substanz nur dann eine neue Form aufzwingen, wenn die Substanz lange genug stillstand, um sie zu beherrschen, obwohl auch die Atome in Festkörpern ständig in Schwingung waren…

Je mehr Harry versagte, je kälter er sich fühlte, desto klarer schien alles zu werden.

Also gut. Der Nächste auf der Liste. Man konnte nur ganze Objekte als Ganzes verwandeln. Man konnte nicht ein halbes Streichholz in eine Nadel verwandeln, man musste das ganze Ding verwandeln.

Als Harry von Draco in diesem Klassenzimmer gefangen worden war, konnte er deshalb nicht einfach einen dünnen zylindrischen Querschnitt der Wände in Schwamm verwandeln und ein Stück Stein ausstanzen, das groß genug war, damit er durch das Loch passte.

Er hätte der ganzen Wand eine neue Form aufzwingen müssen, und vielleicht einem ganzen Abschnitt von Hogwarts, nur um diesen kleinen Querschnitt zu verändern.

\emph{Und das war lächerlich!} Dinge bestehen aus Atomen. Aus vielen kleinen, winzigen Pünktchen. Es gab keine Kontiguität, keine Festigkeit, nur elektromagnetische Kräfte, die die kleinen Punkte miteinander in Beziehung hielten.

.. Mandy Brocklehurst hielt mit ihrer Gabel auf dem Weg zu ihrem Mund inne. „Huh“, sagte sie zu Su Li, die auf dem nun leeren Platz neben ihr saß, „was ist in Hermine gefahren?“

Harry hätte am liebsten seinen Radiergummi umgebracht.

Er hatte versucht, einen einzelnen Punkt auf dem rosa Rechteck in Stahl zu verwandeln, abgesehen vom Rest des Radiergummis, und der Radierer kooperierte nicht.

Es musste eine konzeptionelle Einschränkung sein, keine reale. Das musste so sein. Dinge wurden aus Atomen gemacht, und jedes Atom war ein winziges separates Ding. Atome wurden durch einen Quantennebel gemeinsamer Elektronen zusammengehalten, für kovalente Bindungen, oder manchmal auch nur durch Magnetismus im Nahbereich, für ionische Bindungen oder van der Waals-Kräfte. Wenn es darauf ankam, waren die Protonen und Neutronen im Inneren der Kerne winzige separate Dinge.

Die Quarks im Inneren der Protonen und Neutronen waren winzige separate Dinge! Es gab einfach nichts in der Realität, in der Welt da draußen, das der Vorstellung der Menschen von festen Objekten entsprach.

Es waren alles nur kleine Pünktchen. Und die freie Verwandlung war am Anfang nur im Kopf, nicht wahr? Keine Worte, keine Gesten. Nur das reine Konzept der Form, das strikt von der Substanz getrennt gehalten wird, der Substanz auferlegt wird, getrennt von ihrer Form gedacht wird.

Das und der Zauberstab und das, was einen zu einem Zauberer machte. Die Zauberer konnten keine Teile von Dingen umwandeln, konnten nur das umwandeln, was ihr Verstand als Ganzes wahrnahm, weil sie nicht in ihren Knochen wussten, dass tief unten alles nur Atome waren.

Harry hatte sich so sehr auf dieses Wissen konzentriert, wie er nur konnte, auf die wahre Tatsache, dass der Radiergummi nur eine Ansammlung von Atomen war, dass alles nur Ansammlungen von Atomen war, und dass die Atome des kleinen Flecks, den er zu verformen versuchte, eine ebenso gültige Ansammlung bildeten wie jede andere Ansammlung, an die er denken wollte.

Und Harry war immer noch nicht in der Lage gewesen, diesen einen Teil des Radiergummis zu verändern, die Verwandlung ging nirgendwo hin.

\textbf{\emph{Das. war. Lächerlich.}}

Harrys Fingerknöchel wurden wieder weiß an seinem Zauberstab. Er war es leid, Versuchsergebnisse zu bekommen, die keinen Sinn ergaben.

Vielleicht verhinderte die Tatsache, dass ein Teil seines Verstandes immer noch in Begriffen von Objekten dachte, dass die Verwandlung durchging. Er hatte an eine Ansammlung von Atomen gedacht, die ein Radiergummi war. Er hatte an eine Ansammlung gedacht, die ein kleines Pflaster war.

\emph{Zeit, einen Zahn zuzulegen.}

Harry drückte seinen Zauberstab fester gegen das winzige Stückchen Radiergummi und versuchte, die Illusion zu durchschauen, die Nichtwissenschaftler für die Realität hielten, die Welt der Tische und Stühle, der Luft und Radiergummis und Menschen.

Wenn Sie durch einen Park gingen, war die immersive Welt, die Sie umgab, etwas, das in Ihrem eigenen Gehirn als ein Muster von feuernden Neuronen existierte.

Die Empfindung eines strahlend blauen Himmels war nicht etwas hoch über Ihnen, sondern etwas in Ihrem visuellen Kortex, und Ihr visueller Kortex befand sich im hinteren Teil Ihres Gehirns.

Alle Empfindungen dieser hellen Welt spielten sich in Wirklichkeit in der stillen Knochenhöhle ab, die Sie Ihren Schädel nannten, dem Ort, an dem Sie lebten und den Sie niemals verließen.

Wenn Sie jemandem wirklich Hallo sagen wollten, der tatsächlichen Person, würden Sie ihm nicht die Hand schütteln, sondern sanft an den Schädel klopfen und sagen: „\emph{Wie geht es Ihnen da drinnen?}“ Das war es, was die Leute waren, das war es, wo sie wirklich lebten.

Und das Bild des Parks, durch den Sie zu gehen glaubten, war etwas, das in Ihrem Gehirn visualisiert wurde, als es die Signale verarbeitete, die von Ihren Augen und Ihrer Netzhaut nach unten gesendet wurden.

Es war keine Lüge, wie die Buddhisten dachten, es gab nicht etwas furchtbar Mystisches und Unerwartetes hinter dem Schleier der Maya, was hinter der Illusion des Parks lag, war nur der tatsächliche Park, aber es war alles noch Illusion.

Harry saß nicht im Klassenzimmer. Er schaute nicht auf den Radiergummi. Harry befand sich im Inneren seines Schädels. Er erlebte ein verarbeitetes Bild, das sein Gehirn aus den Signalen seiner Netzhaut dekodiert hatte.

Der echte Radiergummi war irgendwo anders, irgendwo, das nicht das Bild war. Und der echte Radiergummi entsprach nicht dem Bild, das Harrys Gehirn von ihm hatte.

Die Vorstellung des Radiergummis als festes Objekt existierte nur in seinem eigenen Gehirn, im parietalen Kortex, der seinen Sinn für Form und Raum verarbeitete.

Der echte Radiergummi war eine Ansammlung von Atomen, die durch elektromagnetische Kräfte zusammengehalten wurden und gemeinsame kovalente Elektronen besaßen, während in der Nähe Luftmoleküle voneinander abprallten und an den Radiergummi-Molekülen abprallten.

Der echte Radiergummi war weit weg, und Harry konnte ihn in seinem Schädel nie ganz berühren, konnte sich nur Vorstellungen davon machen.

Aber sein Zauberstab hatte die Macht, er konnte die Dinge da draußen in der Realität verändern, es waren nur Harrys eigene Vorurteile, die ihn einschränkten.

Irgendwo hinter dem Schleier berührte die Wahrheit hinter Harrys Konzept von

„\emph{meinem Zauberstab}“ die Ansammlung von Atomen, die Harrys Verstand als „\emph{einen Fleck auf dem Radiergummi}“ betrachtete, und wenn dieser Zauberstab die Ansammlung von Atomen verändern konnte, die Harry als „\emph{den ganzen Radiergummi}“ betrachtete, gab es absolut keinen Grund, warum er nicht auch die andere Ansammlung verändern konnte.

\textbf{\emph{.. Die Verwandlung klappte immer noch nicht.}}

Harry biss die Zähne zusammen und legte noch einen Zahn zu.

Die Vorstellung, die Harrys Verstand von dem Radiergummi als einem einzelnen Objekt hatte, war offensichtlicher Blödsinn.

Es war eine Karte, die nicht zum Territorium passte und auch nicht passen konnte. Menschen modellierten die Welt mit Hilfe geschichteter Organisationsebenen, sie hatten getrennte Gedanken darüber, wie Länder funktionierten, wie Menschen funktionierten, wie Organe funktionierten, wie Zellen funktionierten, wie Moleküle funktionierten, wie Quarks funktionierten.

Wenn Harrys Gehirn über den Radiergummi nachdenken musste, dachte es über die Regeln nach, die für Radiergummis galten, wie z. B. „\emph{Radiergummis können Bleistiftstriche beseitigen}“.

Nur wenn Harrys Gehirn vorhersagen musste, was auf der unteren chemischen Ebene passieren würde, nur dann würde Harrys Gehirn anfangen - als ob es eine separate Tatsache wäre - über Gummimoleküle nachzudenken.

Aber das war alles im Verstand. Harrys Verstand mochte separate Überzeugungen über Regeln haben, die für Radiergummis gelten, aber es gab kein separates physikalisches Gesetz, das für Radiergummis galt.

Harrys Verstand modellierte die Realität mit mehreren Organisationsebenen, mit unterschiedlichen Überzeugungen über jede Ebene.

Aber das war alles in der Karte, das wahre Territorium war nicht so, die Realität selbst hatte nur eine einzige Organisationsebene, die Quarks, es war ein einheitlicher Prozess auf niedriger Ebene, der mathematisch einfachen Regeln gehorchte.

Zumindest hatte Harry das geglaubt, bevor er von der Magie erfuhr, aber der Radiergummi war nicht magisch. Und selbst wenn der Radiergummi magisch gewesen wäre, war die Vorstellung, dass es wirklich einen einzigen festen Radiergummi geben könnte, unmöglich.

Dinge wie Radiergummis konnten keine Grundelemente der Realität sein, sie waren zu groß und kompliziert, um Atome zu sein, sie mussten aus Teilen bestehen.

Man konnte keine Dinge haben, die grundlegend kompliziert waren. Der implizite Glaube, den Harrys Gehirn an den Radiergummi als ein einzelnes Objekt hatte, war nicht nur falsch, es war eine \emph{Karte-Territorium-Verwechslung}, der Radiergummi existierte nur als ein separates Konzept in Harrys mehrstufigem Modell der Welt, nicht als ein separates Element der einstufigen Realität.

\textbf{\emph{… die Verwandlung fand immer noch nicht statt.}}

Harry atmete schwer, misslungene Verwandlung war fast so ermüdend wie erfolgreiche, aber verdammt, er würde jetzt nicht aufgeben.

\emph{Na schön, scheiß auf diesen Müll aus dem 19. Jahrhundert.}

Die Realität bestand nicht aus Atomen, sie war kein Satz winziger Billardkugeln, die herumhüpften.

Das war nur eine weitere Lüge. Die Vorstellung von Atomen als kleinen Punkten war nur eine weitere bequeme Halluzination, an die sich die Menschen klammerten, weil sie sich nicht mit der unmenschlich fremden Form der zugrunde liegenden Realität auseinandersetzen wollten.

Kein Wunder also, dass seine Versuche, auf dieser Basis zu verwandeln, nicht funktionierten. Wenn er Macht wollte, musste er seine Menschlichkeit aufgeben und seine Gedanken zwingen, der wahren Mathematik der Quantenmechanik zu entsprechen.

Es gab keine Teilchen, es gab nur Amplitudenwolken in einem Vielteilchen-Konfigurationsraum, und das, was sich sein Gehirn liebevoll als Radiergummi vorstellte, war nichts anderes als ein gigantischer Faktor in einer Wellenfunktion, der zufällig faktorisiert wurde, er hatte keine separate Existenz, genauso wenig wie es einen bestimmten festen Faktor 3 gab, der in der Zahl 6 versteckt war,

Wenn sein Zauberstab in der Lage war, Faktoren in einer annähernd faktorisierbaren Wellenfunktion zu verändern, dann sollte er verdammt noch mal auch in der Lage sein, den etwas kleineren Faktor zu verändern, den Harrys Gehirn als einen Fleck auf dem Radiergummi visualisierte—

Hermine rannte durch die Gänge, die Schuhe stampften hart auf den Stein, der Atem kam hechelnd, der Schock des Adrenalins raste immer noch durch ihr Blut. Wie das Bild einer jungen Frau, die sich in eine alte Krone verwandelt, wie die Tasse, die zwei Gesichter bekommt.

\emph{Was hatten sie getan?}

\textbf{\emph{Was hatten sie getan?}}

Sie kam zum Klassenzimmer und ihre Finger rutschten zuerst auf der Türklinke ab, zu verschwitzt, sie griff fester zu und die Tür öffnete sich - - in einem einzigen Wahrnehmungsblitz sah sie, wie Harry auf ein kleines rosa Rechteck auf dem Tisch vor ihm starrte - - wie ein paar Schritte entfernt der winzige schwarze Faden, fast unsichtbar aus dieser Entfernung, all dieses Gewicht trug—

\textbf{„Harry, raus aus dem Klassenzimmer!“}

Purer Schock ging über Harrys Gesicht, und er stand so schnell auf, dass er fast umfiel,

er hielt nur an, um das kleine rosa Rechteck vom Tisch zu greifen, und er rannte zur Tür hinaus, sie war schon zur Seite getreten, ihren Zauberstab schon in der Hand, als sie auf den Faden zeigte—

„Finite Incantatem!“

Und Hermine knallte die Tür wieder zu, gerade als von drinnen das gigantische Krachen von hundert Kilogramm fallendem Metall kam.

Sie keuchte, schnappte nach Luft, sie war den ganzen Weg hierher gerannt, ohne anzuhalten, sie war schweißgebadet und ihre Beine und Oberschenkel brannten wie lebende Flammen, sie hätte Harrys Fragen für alle Galleonen der Welt nicht beantworten können. Hermine blinzelte und bemerkte, dass sie zu fallen begonnen hatte, und Harry hatte sie aufgefangen und ließ sie sanft auf den Boden sinken.

„… gesund…“, schaffte sie es zu flüstern.

„Was?“, sagte Harry und sah blasser aus, als sie ihn je gesehen hatte.

„… bist du, fühlst du dich, gesund…“

Harry sah noch erschrockener aus, als ihm die Frage gestellt wurde.

„Ich, ich glaube nicht, dass ich irgendwelche Symptome habe—“

Hermine schloss für einen Moment die Augen.

„Gut“, flüsterte sie. „Fang an, zu atmen…“

Das dauerte eine Weile. Harry sah immer noch ängstlich aus.

\emph{Das war auch gut so, vielleicht würde es ihm eine Lehre sein.}

Hermine griff in den Beutel, den Harry ihr gekauft hatte, flüsterte „\emph{Wasser}“ durch ihre ausgedörrte Kehle, nahm die Flasche heraus und trank in großen Schlucken. Und dann dauerte es noch eine Weile, bis sie wieder sprechen konnte.

„Wir haben die Regeln gebrochen, Harry“, sagte sie mit heiserer Stimme.

„Wir haben die Regeln gebrochen.“

„Ich…“ Harry schluckte. „Ich weiß immer noch nicht, wie, ich habe nachgedacht, aber—“

„Ich habe gefragt, ob die Verwandlung sicher ist, und du hast mir geantwortet!“

Es gab eine Pause.

„Das war's?“ sagte Harry.

Sie hätte schreien können.

„Harry, verstehst du das nicht?“, sagte sie. "Es ist aus winzigen Fasern gemacht, \emph{was, wenn es sich auflöst,} wer weiß, was alles schief gehen kann, wir haben Professor McGonagall nicht gefragt!

Verstehst du nicht, was wir gemacht haben? Wir haben mit Verwandlung experimentiert. \textbf{Wir haben mit Verwandlung experimentiert!}"

Es gab eine weitere Pause.

„Richtig…“ sagte Harry langsam. "Das ist wahrscheinlich eines der Dinge, bei denen sie sich nicht einmal die Mühe machen zu sagen, dass man das nicht tun soll, weil es zu offensichtlich ist.

Man sollte keine brillanten neuen Ideen für Verwandlung alleine in einem unbenutzten Klassenzimmer testen, ohne die Professoren zu fragen."

„Du hättest uns umbringen können, Harry!“ Hermine wusste, dass es nicht fair war, sie hatte den Fehler auch gemacht, aber sie war trotzdem wütend auf ihn, er klang immer so zuversichtlich und das hatte sie gedankenlos mitgerissen.

„Wir hätten Professor McGonagalls Rekord verderben können!“

„Ja“, sagte Harry, „lass uns ihr nichts davon erzählen, ja?“

„Wir müssen aufhören“, sagte Hermine.

„Wir müssen damit aufhören, sonst werden wir noch verletzt. Wir sind zu jung, Harry, wir können das nicht tun, noch nicht.“

Ein schwaches Grinsen ging über Harrys Gesicht.

„Ähm, da liegst du irgendwie falsch.“

Und er hielt ihr ein kleines rosa Rechteck hin, einen Radiergummi mit einem hellen Metallfleck darauf.

Hermine starrte es verwirrt an.

„Quantenmechanik war nicht genug“, sagte Harry. "Ich musste bis hinunter zur zeitlosen Physik gehen, bevor es klappte. Ich musste den Zauberstab so sehen, dass er eine Beziehung zwischen getrennten vergangenen und zukünftigen Realitäten erzwingt, anstatt irgendetwas in der Zeit zu verändern - aber ich habe es geschafft, Hermine, ich habe über die Illusion von Objekten hinaus gesehen, und ich wette, es gibt keinen einzigen anderen Zauberer auf der Welt, der das hätte tun können.

Selbst wenn einige Muggelgeborene über zeitlose Formulierungen der Quantenmechanik Bescheid wüssten, wäre es nur ein seltsamer Glaube über seltsames, fernes Quantenzeug, sie würden nicht sehen, dass es Realität ist, akzeptieren, dass die Welt, die sie kennen, nur eine Halluzination ist.

Ich habe einen Teil des Radiergummis verwandelt, ohne das ganze Ding zu verändern."

Hermine hob wieder ihren Zauberstab und richtete ihn auf den Radiergummi.

Für einen Moment ging Ärger über Harrys Gesicht, aber er machte keine Anstalten, sie aufzuhalten.

„Finite Incantatem“, sagte Hermine. „Frag bei Professor McGonagall nach, bevor du es noch einmal versuchst.“

Harry nickte, obwohl sein Gesicht immer noch ein wenig angespannt war.

„Und wir müssen trotzdem aufhören“, sagte Hermine.

„Warum?!“, sagte Harry. „Verstehst du nicht, was das bedeutet, Hermine? Zauberer können nicht alles wissen! Es gibt zu wenige von ihnen, noch weniger, die sich mit Wissenschaft auskennen, sie haben die tief hängenden Früchte nicht ausgeschöpft—“

„Es ist nicht sicher“, sagte Hermine. „Wenn wir neue Dinge herausfinden können, ist es noch weniger sicher! Wir sind zu jung! Wir haben schon einen großen Fehler gemacht, beim nächsten Mal könnten wir einfach sterben!“ Da zuckte Hermine zusammen.

Harry wandte den Blick von ihr ab und begann, langsam und tief zu atmen.

„Bitte versuch nicht, es allein zu tun, Harry“, sagte Hermine mit zitternder Stimme.

„Bitte.“

\emph{Bitte lass mich nicht entscheiden müssen, ob ich es Professor Flitwick sagen soll}.

Es gab eine lange Pause.

„Du willst also, dass wir lernen“, sagte Harry.

Sie merkte, dass er sich bemühte, den Ärger aus seiner Stimme zu halten.

„Nur lernen.“

Hermine war sich nicht sicher, ob sie etwas sagen sollte, aber…

„So wie du, ähm, zeitlose Physik studiert hast, richtig?“

Harry sah sie wieder an. „Das, was du gemacht hast“, sagte Hermine, ihre Stimme zaghaft,

„das war nicht wegen unserer Experimente, richtig? Du konntest es tun, weil du viele Bücher gelesen hast.“

Harry öffnete den Mund, dann schloss er ihn wieder.

Auf seinem Gesicht lag ein frustrierter Ausdruck.

„Also gut“, sagte Harry. „Wie wäre es damit. Wir lernen, und wenn mir etwas einfällt, das wirklich einen Versuch wert zu sein scheint, probieren wir es aus, nachdem ich einen Professor gefragt habe.“

„Okay“, sagte Hermine. Sie fiel nicht vor Erleichterung um, aber nur, weil sie sich bereits gesetzt hatte.

„Sollen wir essen gehen?“ Sagte Harry vorsichtig.

Hermine nickte. Ja. Mittagessen hörte sich gut an. Diesmal aber wirklich. Vorsichtig begann sie, sich vom Steinboden abzustoßen und zuckte zusammen, als ihr Körper sie anschrie - Harry richtete seinen Zauberstab auf sie und sagte „Wingardium Leviosa“

.

Hermine blinzelte, als das enorme Gewicht auf ihren Beinen auf etwas Erträgliches abnahm. Ein Lächeln huschte über Harrys Gesicht.

„Man kann etwas anheben, ohne es komplett schweben lassen zu können“, sagte er. „Erinnerst du dich an das Experiment?“

Hermine lächelte hilflos zurück, obwohl sie dachte, dass sie eigentlich immer noch wütend sein müsste. Und sie begann, zurück zur Großen Halle zu gehen, wobei sie sich bemerkenswert und wunderbar leichtfüßig fühlte, während Harry seinen Zauberstab vorsichtig auf sie gerichtet hielt.

Er schaffte es nur fünf Minuten lang, aber es war der Gedanke, der zählte.

Minerva sah Dumbledore an. Dumbledore blickte fragend zu ihr zurück.

„Hast du irgendetwas davon verstanden?“, fragte der Schulleiter und klang verwirrt.

Es war das vollkommenste Kauderwelsch, das Minerva je gehört hatte. Es war ihr ein wenig peinlich, den Schulleiter herbeigerufen zu haben, um es zu hören, aber sie hatte ausdrückliche Anweisungen erhalten.

„Ich fürchte nicht“, sagte Professor McGonagall hochnäsig.

„Also“, sagte Dumbledore. Der silberne Bart schwang von ihr weg, der funkelnde Blick des alten Zauberers war wieder woanders. „Du vermutest, dass du etwas tun kannst, was andere Zauberer nicht können, etwas, das wir für unmöglich halten.“

Sie standen zu dritt im privaten Arbeitsraum des Schulleiters für Verwandlung, wo der leuchtende Phönix von Dumbledores Patronus ihr befohlen hatte, Harry zu bringen, kurz nachdem ihr eigener Patronus ihn erreicht hatte.

Licht schien durch die Oberlichter und beleuchtete das große siebenzackige alchemistische Diagramm, das in der Mitte des kreisförmigen Raumes gezeichnet war, und zeigte, dass es ein wenig verstaubt war, was Minerva traurig machte.

Verwandlungsforschung war eine von Dumbledores großen Freuden, und sie wusste, wie sehr er in letzter Zeit unter Zeitdruck stand, aber nicht, dass er so unter Druck stand.

Und jetzt wollte Harry Potter noch mehr von der Zeit des Schulleiters verschwenden. Aber das konnte sie Harry nicht verübeln.

Er hatte das Richtige getan, als er zu ihr gekommen war, um ihr zu sagen, dass er eine Idee hatte, wie man etwas in Verwandlung machen könnte, das derzeit für unmöglich gehalten wurde, und sie selbst hatte genau das getan, was ihr aufgetragen worden war: Sie hatte Harry befohlen, still zu sein und nichts mit ihr zu besprechen, bis sie sich mit dem Schulleiter beraten hatte und sie mit dem Umzug an einen sicheren Ort fertig waren.

Hätte Harry zu Beginn gesagt, was genau er zu tun gedachte, hätte sie sich nicht die Mühe gemacht.

„Hören Sie, ich weiß, es ist schwer zu erklären“, sagte Harry und klang dabei ein wenig verlegen.

„Es läuft darauf hinaus, dass das, was ihr glaubt, im Widerspruch zu dem steht, was die Wissenschaftler glauben, und zwar in einem Fall, in dem ich wirklich erwarten würde, dass Wissenschaftler mehr wissen als Zauberer.“

Minerva hätte laut geseufzt, wenn Dumbledore die ganze Sache nicht sehr ernst genommen hätte.

Harrys Idee entstammte schlichter Unwissenheit, nichts weiter. Wenn man die Hälfte einer Metallkugel in Glas verwandelte, hatte die ganze Kugel eine andere Form.

Einen Teil zu verändern, bedeutete, das Ganze zu verändern, und das bedeutete, die ganze Form zu entfernen und durch eine andere zu ersetzen.

Was würde es überhaupt bedeuten, nur die Hälfte einer Metallkugel zu verwandeln? Dass die Metallkugel als Ganzes die gleiche Form hat wie vorher, aber die Hälfte dieser Kugel nun eine andere Form hat?

„Mr~Potter“, sagte Professor McGonagall, „was Sie tun wollen, ist nicht nur unmöglich, es ist unlogisch. Wenn Sie die Hälfte von etwas verändern, haben Sie das Ganze verändert.“

„In der Tat“, sagte Dumbledore.

„Aber Harry ist der Held, also ist er vielleicht in der Lage, Dinge zu tun, die logisch unmöglich sind.“

\emph{Minerva hätte mit den Augen gerollt, wenn sie nicht schon längst taub geworden wäre.}

„Angenommen, es wäre möglich“, sagte Dumbledore, „fällt dir ein Grund ein, warum sich die Ergebnisse in irgendeiner Weise von der gewöhnlichen Verwandlung unterscheiden sollten?“

Minerva runzelte die Stirn. Die Tatsache, dass das Konzept buchstäblich unvorstellbar war, bereitete ihr einige Schwierigkeiten, aber sie versuchte, es für bare Münze zu nehmen.

Eine Verwandlung, die nur der Hälfte einer Metallkugel auferlegt wurde…

„Es geschehen seltsame Dinge an der Schnittstelle?“, fragte Minerva.

„Aber das sollte nicht anders sein, als wenn man das Objekt als Ganzes in eine Form mit zwei verschiedenen Teilen verwandelt…“

Dumbledore nickte.

„Das ist auch mein eigener Gedanke. Und Harry, wenn deine Theorie richtig ist, bedeutet das, dass das, was du tun willst, genau wie jede andere Verwandlung ist, nur auf einen Teil des Gegenstandes angewandt und nicht auf das Ganze? Überhaupt keine Veränderungen?“

„Ja“, sagte Harry fest. „Das ist der ganze Punkt.“

Dumbledore schaute sie wieder an.

„Minerva, kannst du dir irgendeinen Grund vorstellen, warum das gefährlich sein sollte?“

„Nein“, sagte Minerva, nachdem sie die Suche in ihrem Gedächtnis beendet hatte.

„Ich auch nicht“, sagte der Schulleiter.

„Also gut, da dies in jeder Hinsicht genau analog zur gewöhnlichen Verwandlung sein sollte und uns auch kein Grund einfällt, warum es gefährlich sein sollte, denke ich, dass die zweite Stufe der Vorsicht ausreicht.“

Minerva war überrascht, aber sie erhob keinen Einspruch.

Dumbledore hatte sehr viel mehr Erfahrung in Verwandlung, und er hatte buchstäblich Tausende von neuen Verwandlungen ausprobiert, ohne jemals einen zu niedrigen Grad der Vorsicht zu wählen.

\emph{Er hatte Verwandlung im Kampf eingesetzt und war noch am Leben.}

Wenn der Schulleiter der Meinung war, dass der zweite Grad genug war, dann war es genug.

Dass Harry mit Sicherheit scheitern würde, war natürlich völlig irrelevant.

Die beiden begannen, die Schutzzauber und Erkennungsnetze einzurichten.

Das wichtigste Netz war dasjenige, das überprüfte, ob kein verwandeltes Material in die Luft gelangt war.

Harry würde sicherheitshalber in einer separaten Hülle mit eigener Luftzufuhr eingeschlossen werden, nur sein Zauberstab durfte den Schild verlassen, und die Schnittstelle war dicht.

Sie befanden sich in Hogwarts, also konnten sie kein Material, das Anzeichen von Selbstentzündung aufwies, automatisch herausapparieren, aber sie konnten es fast genauso schnell aus einem Dachfenster werfen, die Fenster waren genau aus diesem Grund alle nach außen geklappt.

Harry selbst würde beim ersten Anzeichen von Ärger durch ein anderes Dachfenster

hinausgeschleudert.

Harry sah ihnen bei der Arbeit zu, sein Gesicht sah ein wenig ängstlich aus.

„Keine Sorge“, sagte Professor McGonagall mitten in ihrer laufenden Beschreibung, „das wird mit ziemlicher Sicherheit nicht nötig sein, Mr~Potter. Wenn wir erwarten würden, dass etwas schief geht, würden Sie es nicht versuchen dürfen. Es sind nur gewöhnliche Vorsichtsmaßnahmen für jede Verwandlung, die noch nie jemand versucht hat.“

Harry schluckte und nickte.

Und ein paar Minuten später war Harry in den Sicherheitsstuhl geschnallt und stützte seinen Zauberstab gegen eine Metallkugel - eine, die nach seinen derzeitigen Testergebnissen zu groß für ihn sein sollte, um sie in weniger als dreißig Minuten zu verformen.

Und ein paar Minuten später lehnte Minerva an der Wand und fühlte sich ohnmächtig. Auf der Kugel, auf der Harrys Zauberstab gelegen hatte, war ein kleiner Glasfleck.

Harry sagte nicht „\emph{Ich hab's ja gesagt}“, aber der selbstgefällige Blick auf seinem schwitzenden Gesicht sagte es für ihn.

Dumbledore war dabei, die Kugel mit analytischen Zaubern zu belegen und sah dabei immer faszinierter aus. Dreißig Jahre waren aus seinem Gesicht geschmolzen.

„Faszinierend“, sagte Dumbledore. „Es ist genau so, wie er behauptet hat. Er hat einfach einen Teil des Subjekts verwandelt, ohne das Ganze zu verwandeln. Du sagst, es ist wirklich nur eine konzeptionelle Einschränkung, Harry?“

„Ja“, sagte Harry, „aber eine tiefe, nur zu wissen, dass es eine konzeptionelle Einschränkung sein musste, reichte nicht aus. Ich musste den Teil meines Verstandes unterdrücken, der den Fehler machte, und stattdessen an die zugrunde liegende Realität denken, die die Wissenschaftler herausgefunden haben.“

„Wahrlich faszinierend“, sagte Dumbledore. „Ich nehme an, dass jeder andere Zauberer, um das Gleiche zu tun, monatelanges Studium benötigen würde, wenn er es überhaupt könnte? Und darf ich dich bitten, ein paar andere Gegenstände teilweise zu verwandeln?“

„Wahrscheinlich ja und natürlich“, sagte Harry.

Eine halbe Stunde später fühlte sich Minerva ebenso verwirrt, aber erheblich beruhigt, was die Sicherheit betraf. Es war das Gleiche, abgesehen davon, dass es logisch unmöglich war.

„Ich glaube, das ist genug, Schulleiter“, sagte Minerva schließlich.

„Ich vermute, dass die partielle Verwandlung anstrengender ist als die gewöhnliche Art.“

„Mit der Übung wird es weniger“, sagte der erschöpfte und blasse Junge mit unsicherer Stimme,

„aber ja, da haben Sie recht.“

Der Prozess, Harry aus der Sicherheitsstation zu holen, dauerte noch eine Minute, und dann eskortierte Minerva ihn zu einem viel bequemeren Stuhl, und Dumbledore brachte eine Eislimonade.

„Herzlichen Glückwunsch, Mr~Potter!“, sagte Professor McGonagall, und meinte es auch so. Sie hätte fast alles dagegen gewettet, dass das funktioniert.

„Ich gratuliere in der Tat“, sagte Dumbledore. „Selbst ich habe vor meinem vierzehnten Lebensjahr keine originellen Entdeckungen in der Verwandlung gemacht. Seit den Tagen von Dorotea Senjak hat kein Genie mehr so früh geblüht.“

„Danke“, sagte Harry und klang ein wenig überrascht.

„Nichtsdestotrotz“, sagte Dumbledore nachdenklich, „denke ich, dass es höchst weise wäre, dieses glückliche Ereignis geheim zu halten, zumindest für den Moment. Harry, hast du deine Idee mit irgendeiner anderen Person besprochen, bevor du mit Professor McGonagall gesprochen hast?“

Es herrschte Schweigen.

„Ähm…“ sagte Harry. „Ich möchte niemanden an die Inquisition ausliefern, aber ich habe einem anderen Schüler erzählt—“

Das Wort explodierte fast von Professor McGonagalls Lippen.

„Was? Sie haben eine völlig neue Form der Verwandlung mit einem Schüler besprochen, bevor Sie eine anerkannte Autorität konsultiert haben? Haben Sie eine Ahnung, wie unverantwortlich das war?“

„Es tut mir leid“, sagte Harry. „Das war mir nicht klar.“

Der Junge sah angemessen erschrocken aus, und Minerva spürte, wie sich etwas in ihr entspannte.

Wenigstens verstand Harry, wie töricht er sich verhalten hatte.

„Sie müssen Miss~Granger zur Verschwiegenheit verpflichten“, sagte Dumbledore mit ernster Stimme. „Und sagen Sie es niemandem sonst, es sei denn, es gibt einen äußerst triftigen Grund dafür, und auch sie haben geschworen.“

„Ah…warum?“ sagte Harry.

Minerva fragte sich das Gleiche. Wieder einmal dachte der Schulleiter zu weit voraus, als dass sie mithalten konnte.

„Weil du etwas tun kannst, was dir sonst niemand zutraut“, sagte Dumbledore.

„Etwas völlig Unerwartetes. Das könnte sich als dein entscheidender Vorteil erweisen, Harry, und wir müssen ihn bewahren. Bitte, vertrauen Sie mir dabei.“

Professor McGonagall nickte, ihr festes Gesicht zeigte nichts von ihrer inneren Verwirrung.

„Bitte tun Sie das, Mr~Potter“, sagte sie.

„In Ordnung…“ sagte Harry langsam.

„Sobald wir mit der Untersuchung der Materialien fertig sind“, fügte Dumbledore hinzu,

„darfst du die Technik üben, nur von Glas zu Stahl und von Stahl zu Glas, wobei Miss~Granger als Aufpasserin fungieren wird. Sollte einer von euch auch nur das geringste Anzeichen einer Verwandlungskrankheit bemerken, meldet das natürlich sofort einem Professor.“

Kurz bevor Harry den Arbeitsraum verließ, die Hand am Türgriff, drehte sich der Junge noch einmal um und sagte: „Wo wir schon mal hier sind, ist einem von Ihnen etwas an Professor Snape aufgefallen, das anders ist?“

„Anders?“, fragte der Schulleiter.

Minerva ließ sich ihr schiefes Lächeln nicht anmerken. Natürlich hatte der Junge Angst vor dem

„\emph{bösen Zaubertränkemeister}“, denn er konnte ja nicht wissen, warum Severus zu trauen war.

Es wäre gelinde gesagt merkwürdig gewesen, Harry zu erklären, dass Severus immer noch in seine Mutter verliebt war.

„Ich meine, hat sich sein Verhalten in letzter Zeit in irgendeiner Weise verändert?“, fragte Harry.

„Nicht, dass ich gesehen hätte…“, sagte der Schulleiter langsam. „Warum fragst du?“

Harry schüttelte den Kopf.

„Ich möchte Ihren eigenen Beobachtungen nicht vorgreifen. Aber vielleicht sollten Sie die Augen offen halten?“

Das ließ Minerva auf eine Weise zittern, wie es kein offener Vorwurf an Severus hätte tun können.

Harry verbeugte sich respektvoll vor den beiden und verabschiedete sich.

„Albus“, sagte Minerva, nachdem der Junge gegangen war,

„woher wusstest du, dass man Harry ernst nehmen sollte? Ich hätte seine Idee für schlichtweg unmöglich gehalten!“

Das Gesicht des alten Zauberers wurde ernst.

„Aus demselben Grund, aus dem es geheim gehalten werden muss, Minerva. Aus demselben Grund, aus dem ich dir gesagt habe, dass du zu mir kommen sollst, wenn Harry eine solche Behauptung aufstellt.Denn es ist eine Macht, die Voldemort nicht kennt.“

Die Worte brauchten ein paar Sekunden, um in ihr zu sinken. Und dann lief ihr ein kalter Schauer über den Rücken, wie immer, wenn sie sich erinnerte.

\emph{Es hatte als gewöhnliches Vorstellungsgespräch begonnen, Sybill Trelawney bewarb sich um die Stelle als Professorin für Wahrsagerei.}

\emph{\hfill\break }\textbf{\emph{DERJENIGE, DER DIE MACHT HAT, DEN DUNKLEN LORD ZU BESIEGEN, NAHT, GEBOREN VON DENEN, DIE SICH IHM DREIMAL WIDERSETZT HABEN, GEBOREN, WENN DER SIEBTE MONAT STIRBT, UND DER DUNKLE LORD WIRD IHN ALS SEINESGLEICHEN KENNZEICHNEN, ABER ER WIRD EINE MACHT HABEN, DIE DER DUNKLE LORD NICHT KENNT, UND EINER VON BEIDEN MUSS ALLES BIS AUF EINEN REST DES ANDEREN VERNICHTEN, DENN DIESE BEIDEN UNTERSCHIEDLICHEN GEISTER KÖNNEN NICHT IN DERSELBEN WELT EXISTIEREN.}}

Diese furchtbaren Worte, gesprochen mit dieser schrecklich dröhnenden Stimme, schienen nicht zu so etwas wie einer teilweisen Verwandlung zu passen.

„Vielleicht nicht“, sagte Dumbledore, nachdem Minerva versucht hatte, es zu erklären.

„Ich gestehe, ich hatte auf etwas gehofft, das uns helfen würde, Voldemorts Horkrux zu finden, wo immer er ihn auch versteckt haben mag. Aber…“

Der alte Zauberer zuckte mit den Schultern.

"Prophezeiungen sind heikle Dinge, Minerva, und es ist am besten, kein Risiko einzugehen.

Die kleinste Sache kann sich als entscheidend erweisen, wenn sie unerwartet bleibt."

„Und was meinte er wohl mit Severus?“, fragte Minerva.

„Da habe ich keine Ahnung“, seufzte Dumbledore.

"Es sei denn, Harry macht einen Schachzug gegen Severus und dachte, dass eine offene Frage ernst genommen werden könnte, wo eine direkte Anschuldigung abgetan werden würde.

Und wenn das tatsächlich der Fall war, hat Harry richtig gefolgert, dass ich nicht darauf vertrauen würde, dass es so ist. Lass uns einfach unvoreingenommen weiter beobachten, wie er es verlangt."

\textbf{Nachspiel, 1:}

„Ähm, Hermine?“ sagte Harry mit ganz leiser Stimme.

„Ich denke, ich schulde dir eine wirklich, wirklich, wirklich große Entschuldigung.“

\textbf{Nachspiel, 2:}

Alissa Cornfoots Augen waren leicht glasig, als sie auf den Zaubertränkemeister starrte, der ihrer Klasse einen strengen Vortrag hielt, eine winzige bronzene Bohne hochhielt und etwas über schreiende Pfützen aus Menschenfleisch sagte.

Seit Beginn dieses Jahres hatte sie Probleme, in Zaubertränke zuzuhören. Ständig starrte sie ihren furchtbaren, gemeinen, schmierigen Professor an und \emph{phantasierte über besonderes Nachsitzen.}

Wahrscheinlich stimmte wirklich etwas nicht mit ihr, aber sie konnte einfach nicht damit aufhören—

„Au!“ sagte Alissa dann. Snape hatte gerade die bronzene Bohne zielsicher an Alissas Stirn geschnippt.

„Miss~Cornfoot“, sagte der Meister der Zaubertränke mit schneidender Stimme,

„das ist ein heikler Trank, und wenn Sie nicht aufpassen, verletzen Sie nicht nur sich selbst, sondern auch Ihre Mitschüler. Kommen Sie nach dem Unterricht zu mir.“

Die letzten vier Worte halfen ihr nicht, aber sie strengte sich an und schaffte es, die Stunde zu überstehen, ohne jemanden zu verletzen.

Nach dem Unterricht näherte sich Alissa dem Pult. Ein Teil von ihr wollte demütig dastehen, mit schamhaftem Gesicht und reumütig hinter dem Rücken verschränkten Händen, nur für den Fall, aber ein leiser Instinkt sagte ihr, dass das keine gute Idee sein könnte.

Also stand sie stattdessen einfach da, mit neutralem Gesicht, in einer Haltung, die für eine junge Dame sehr angemessen war, und sagte: „Professor?“

„Miss~Cornfoot“, sagte Snape, ohne von den Blättern aufzublicken, die er sortierte,

„ich erwidere Ihre Zuneigung nicht, ich fange an, Ihre Blicke als störend zu empfinden, und Sie werden von nun an Ihre Augen zurückhalten. Ist das klar?“

„Ja“, sagte Alissa mit einem erstickten Quieken, und Snape entließ sie, woraufhin sie mit flammenden Wangen wie geschmolzene Lava aus dem Klassenzimmer floh.

