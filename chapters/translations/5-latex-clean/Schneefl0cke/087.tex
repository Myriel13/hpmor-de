

\hypertarget{zeitdruck-teil-1}{% \section{88. Zeitdruck, Teil 1}\label{zeitdruck-teil-1}}

\textbf{\uline{Zeitdruck, Teil 1}}

16. April 1992. 12:07 Uhr. Mittagszeit.

Harry stapfte hinüber zum größtenteils menschenleeren Gryffindor-Tisch und stellte mit einem Blick fest, dass es heute zum Mittagessen Breen und Roopo-Bälle gab. Die Gespräche in der Umgebung, die Harry ebenfalls hören konnte, hatten mit Quidditch zu tun; eine akustische Umgebung, die er etwas schlechter bewertet würde als das Geräusch rostiger Kettensägen, aber besser als das, was am Ravenclaw-Tisch immer noch über Hermine gequatscht wurde. Zumindest im Haus Gryffindor hatte man anfangs weniger Sympathie für Draco Malfoy und eher den Wunsch, dass alle bestimmte unglückliche Tatsachen einfach vergessen werden sollten; und wenn das schon nicht der richtige Grund für das Schweigen war, so war es doch wenigstens Schweigen. Dean und Seamus und Lavender waren alle über die Feiertage weg, aber das ließ zumindest…

"Was war das für ein Aufruhr am Lehrertisch?" sagte Harry zu der Gruppe um die Weasley-Zwillinge, während er begann, sich seinen eigenen Teller zu beladen. "Es sah so aus, als würde es gerade enden, als ich hereinkam."

"Unsere geliebte, aber ungeschickte Professor Trelawney -"

"hat anscheinend eine ganze Suppenterrine auf sich fallen lassen -"

"Ganz zu schweigen von Mr~Hagrid."

Ein kurzer Blick auf den Lehrertisch bestätigte, dass die Professorin für Wahrsagerei verzweifelt mit ihrem Zauberstab herumfuchtelte, während der Halbriese seine Kleidung abtupfte. Niemand sonst schien ihm viel Aufmerksamkeit zu schenken, nicht einmal Professor McGonagall. Professor Flitwick stand wie immer auf seinem Stuhl, der Schulleiter schien wieder abwesend zu sein (er war die meisten Tage der Ferien weg gewesen), die Professoren Sprout und Sinistra und Vector aßen in ihrer üblichen Gruppierung, und -

"Wisst ihr", sagte Harry, während er den Kopf abwandte, um die Deckenillusion eines klaren blauen Himmels anzustarren, "das macht mir manchmal immer noch Angst."

"Was denn?", fragte Fred oder George.

Der mächtige und rätselhafte Verteidigungsprofessor "ruhte" oder was-auch-immer mit ihm los war, seine Hände griffen fummelnd und zögernd nach einem Hühnerbein, das ihm auf dem Teller zu entgleiten schien.

"Äh, nichts", sagte Harry. "Ich bin noch nicht ganz an Hogwarts gewöhnt."

Harry aß weiter in mäßigem Schweigen, während verschiedene Weasleys über irgendeine bizarre bewusstseinsverändernde Substanz namens Chudley-Canons diskutierten.

"Was für tiefgründige, mysteriöse Gedanken hast du?", fragte eine jung aussehende Hexe mit kurzen Haaren, die in der Nähe saß. "Ich meine, nur aus Neugier. Übrigens, ich bin Brienne."

Sie schaute ihn mit einem dieser Blicke an, von denen Harry fest beschlossen hatte, sie einfach zu ignorieren, bis er älter war.

"Also", sagte Harry, "kennst du diese wirklich einfachen Programme der Künstlichen Intelligenz wie ELIZA, die so programmiert sind, dass sie Wörter in syntaktischen englischen Sätzen verwenden, nur dass sie kein Verständnis dafür haben, was die Wörter bedeuten?"

"Natürlich", sagte die Hexe. "Ich habe ein Dutzend von ihnen in meinem Koffer"

"Nun, ich bin mir ziemlich sicher, dass mein Verständnis von Mädchen irgendwo auf diesem Niveau liegt."

Ein plötzliches Schweigen kam. Es dauerte ein paar Sekunden, bis Harry begriff, dass nicht die ganze Große Halle ihn anstarrte, und dann drehte er den Kopf herum, um nachzusehen. Die Gestalt, die soeben in die Große Halle getorkelt war, schien Mr~Filch zu sein, Hogwarts' Alibi-Hausmeister, der zusammen mit seiner Raubkatze Mrs~Norris eine unbedeutende Zufallsbegegnung darstellte, an der Harry oft vorbeiging, wenn er seinen epischen Umhang trug. (Harry hatte sich einmal mit den Weasley-Zwillingen beraten, um diesem verdienten Ziel einen Streich zu spielen, woraufhin Fred oder George leise darauf hingewiesen hatten, dass man Mr~Filch nie einen Zauberstab benutzen sah, was wirklich seltsam war, wenn man bedenkt, wie viele Zaubersprüche in dieser Position nützlich wären, und man sich fragte, warum Dumbledore dem Mann eine Stelle in Hogwarts gegeben hatte, woraufhin Harry die Klappe gehalten hatte.)

In diesem Moment war Mr~Filchs braune Kleidung unordentlich und schweißgetränkt, seine Schultern hoben sich sichtlich beim Atmen, und seine allgegenwärtige Katze fehlte.

"Troll -", keuchte Mr~Filch. "In den Kerkern -"

Minerva McGonagall stand so schnell vom Tisch auf, dass ihr Stuhl hinter ihr auf den Boden fiel. "Argus!?", rief sie. "Was ist mit dir passiert?"

Argus Filch taumelte aus den riesigen Türen nach vorne, sein Oberkörper war mit kleinen karminroten Punkten übersät, als hätte ihm jemand Steaksoße ins Gesicht gespritzt.

"Troll - grau - doppelt so groß wie ich - es - es -" Argus Filch bedeckte sein Gesicht mit den Händen. "Er hat Mrs~Norris gefressen - hat sie ganz aufgefressen, mit nur einem Bissen -"

Minerva spürte einen Stich der Bestürzung in ihrem anderen Ich, sie hatte die andere Katze nicht besonders gemocht, aber die beiden waren immer noch Katzen gewesen. Ein Aufruhr ging von der Großen Halle aus. Severus stand vom Haupttisch auf, irgendwie tat er das, ohne sichtbare Aufmerksamkeit auf sich zu ziehen, und schritt ohne ein weiteres Wort aus den riesigen Türen.

\emph{Natürlich}, dachte Minerva, \emph{der Korridor im dritten Stock - das könnte ein Ablenkungsmanöver sein} - Sie überließ all diese Angelegenheiten gedanklich Severus' Obhut, zog ihren Zauberstab, hob ihn hoch und ließ fünf scharfe Sprünge violetten Feuers los. Es herrschte fassungslose Stille, bis auf Argus' unterbrochenes Schluchzen.

"Es scheint, dass eine gefährliche Kreatur in Hogwarts frei herumläuft", sagte sie zu den Lehrkräften am Haupttisch. "Ich werde Sie alle bitten, bei der Durchsuchung der Hallen zu helfen." Dann wandte sie sich an die fassungslosen und zuschauenden Schüler und erhob ihre Stimme. "Vertrauensschüler - führt eure Häuser sofort zurück zu den Schlafsälen!"

Percy Weasley sprang vom Gryffindor-Tisch auf. "Folgt mir!", sagte er mit hoher Stimme. "Bleibt zusammen, Erstklässler! Nein, nicht ihr -", aber in diesem Moment erhoben die anderen Vertrauensschüler ihre eigenen Stimmen, als ein erneutes Geplapper aufkam.

Dann sprach eine klare, kühle Stimme unter dem plötzlichen Ansturm. "Stellvertretende Schulleiterin."

Sie drehte sich um. Der Verteidigungsprofessor wischte sich gerade seelenruhig die Hände an einer Serviette ab, als er sich vom Tisch erhob. "Mit Verlaub", sagte der Mann unbekannter Identität, "Sie sind kein Experte in Sachen Kampftaktik, gnädige Frau. In dieser Situation wäre es klüger, wenn Sie -"

"Ich bitte um Entschuldigung, Professor", sagte Professor McGonagall, während sie sich in Richtung der großen Türen drehte. Filius und Pomona hatten sich bereits erhoben, um ihr zu folgen, und Rubeus Hagrid überragte sie alle, als der Halbriese aufstand. Sie hatte ähnliche Erfahrungen schon zu oft gemacht. "Die traurige Erfahrung hat mich gelehrt, dass es bei Gelegenheiten wie diesen nicht ratsam ist, den Ratschlägen des derzeitigen Verteidigungsprofessors zu folgen. In der Tat halte ich es für klug, dass wir beide gemeinsam nach dem Troll suchen, damit Sie nicht für irgendwelche unvorhergesehenen Ereignisse, die während dieser Zeit geschehen, verdächtigt werden können."

Ohne zu zögern, schwang sich der Verteidigungsprofessor auf den Gryffindor-Tisch und klatschte mit einem Geräusch, das wie ein Bodenknacken klang, in die Hände. "Michelle Morgan aus dem Haus Gryffindor, zweite Kommandantin von Pinninis Armee", sagte der Verteidigungsprofessor ruhig in die entstandene Stille hinein. "Bitte sprechen Sie zu Ihrem Hausoberhaupt."

Michelle Morgan kletterte auf ihre Bank und sprach, wobei die kleine Hexe viel selbstbewusster klang, als Minerva sie zu Beginn des Jahres in Erinnerung hatte. "Schüler, die durch die Gänge laufen, würden sich verteilen und wären nicht zu verteidigen. Alle Schüler sollten in der Großen Halle bleiben und eine Gruppe in der Mitte bilden… nicht umgeben von Tischen, ein Troll würde direkt über die Tische springen… mit dem Umkreis, der von Siebtklässlern verteidigt wird. Nur von den Armeen, egal wie gut sie sich duellieren können, damit sie sich nicht gegenseitig in die Schusslinie geraten." Michelle zögerte. "Es tut mir leid, Mr~Hagrid, aber - es wäre nicht sicher für Sie, Sie sollten bei den Schülern zurückbleiben. Und Professor Trelawney sollte sich auch nicht allein einem Troll stellen", Michelle klang bei diesem Teil viel weniger entschuldigend, "aber wenn sie sich mit Professor Quirrell zusammentut, können die beiden zusammen eine zusätzliche vertrauenswürdige und effektive Kampfeinheit bilden. Damit ist meine Analyse abgeschlossen, Professor."

"Angemessen", sagte der Verteidigungsprofessor. "Zwanzig Quirrell-Punkte für Sie. Professor Sie vernachlässigen den noch einfacheren Punkt, dass \emph{zu Hause} nicht \emph{sicher} bedeutet und ein Troll stark genug ist, eine Porträttür aus den Angeln zu reißen -"

"Genug", schnauzte Minerva. "Ich danke Ihnen, Miss~Morgan." Sie blickte zu den beobachtenden Tischen. "Schüler, ihr werdet tun, was sie gesagt hat." Sie wandte sich wieder an den Haupttisch. "Professor Trelawney, Sie werden den Verteidigungsprofessor begleiten -"

"Ah", sagte Sybill zögernd. Unter ihrem übertriebenen Make-up und dem Durcheinander von Tüchern sah die Frau ziemlich blass aus. "Ich fürchte - mir geht es heute nicht ganz gut - ich fühle mich in der Tat ziemlich schwach -"

"Du musst nicht gegen den Troll kämpfen", sagte Minerva scharf, deren Geduld wie immer im Umgang mit der Frau strapaziert wurde. "Bleib einfach bei dem Verteidigungsprofessor und lass ihn nicht einen Augenblick aus den Augen, Du musst hinterher bezeugen können, dass du die ganze Zeit bei ihm warst." Sie wandte sich an Rubeus. "Rubeus, ich überlasse Ihnen hier die Verantwortung. Passen Sie auf sie auf."

Der riesige Mann richtete sich daraufhin auf, verlor seinen mürrischen Blick

und nickte ihr stolz zu. Dann sah Minerva die Schüler an und erhob ihre Stimme. "Es sollte völlig selbstverständlich sein, dass jeder, der die Große Halle aus irgendeinem Grund verlässt, von der Schule verwiesen wird. Es werden keine Ausreden akzeptiert. Habt ihr mich verstanden?"

Die Weasley-Zwillinge, mit denen sie direkten Blickkontakt aufgenommen hatte, nickten respektvoll. Ohne ein weiteres Wort drehte sie sich um und marschierte mit den anderen Professoren hinter ihr in Richtung Hallentüren. Auf der anderen Seite des Raumes, unbemerkt an der Wand, zeigte eine Uhr 12:14 Uhr.

… und er merkte es immer noch nicht.

\emph{Tick}.

Als Harry mit zusammengekniffenen Augen dorthin starrte, wo die Professoren hinausgegangen waren, und sich fragte, was eigentlich vor sich ging und was es zu bedeuten hatte, als die Schüler sich zu einer besser zu verteidigenden Masse zusammenfanden und Zauberstäbe schnippten, um die Tische aus dem Weg zu schweben, begriff Harry immer noch nicht.

\emph{Tick}.

"Hätten sich die Professoren nicht alle in Paaren aufstellen sollen?", sagte ein älterer Gryffindor-Schüler, dessen Namen Harry nicht kannte. "Ich meine - es wäre langsamer, aber es wäre sicherer, denke ich -"

\emph{Tick}.

Jemand anderes antwortete darauf und erhob die Stimme, aber Harry verstand nicht viel davon, das Wesentliche war, dass Bergtrolle hochgradig magieresistent und unglaublich stark waren und sich regenerieren konnten, aber sie waren trotzdem laut, also wenn man sie kommen hörte, sollte es für einen Hogwarts-Professor nicht so schwer sein, sie in Vadims Unzerbrechliches Irgendwas einzuwickeln.

\emph{Tick}.

Und Harry hatte es immer noch nicht gemerkt.

\emph{Tick}.

Die Geräusche in der Menge waren gedämpft, die Leute unterhielten sich mit leiser Stimme miteinander, während sie sich umschauten und auf das Geräusch einer zuschlagenden Tür oder eines wütenden Brüllens lauschten.

\emph{Tick}.

Einige Schüler spekulierten im Flüsterton darüber, was der Verteidigungsprofessor wohl erreichen wollte, indem er einen Troll einschmuggelte, und ob er wütend war, dass Professor McGonagall seinen Ablenkungsversuch mitbekommen hatte, und wovon er ablenken wollte.

\emph{Tick}.

Und der Gedanke kam Harry immer noch nicht, erst als alle Schüler eine Masse von vielleicht hundert Körpern gebildet hatten, die von stolz grimmig dreinblickenden Siebtklässlern mit nach außen gerichteten Zauberstäben patrouilliert wurden, und jemand vorschlug, eine Personenzählung vorzunehmen, und jemand anderes sarkastisch erwiderte, dass das an einem anderen Tag vielleicht Sinn gemacht hätte, aber im Moment seien praktisch alle in den Frühlingsferien und niemand wisse wirklich, wie viele Schüler im Raum sein sollten, geschweige denn, ob welche fehlten.

\emph{Tick}.

Das war, als Harry sich fragte, wo Hermine war.

\emph{Tick}.

Harry schaute hinüber, wo sich die Ravenclaws versammelt hatten, er sah Hermine nicht, aber dann standen alle so dicht gedrängt, dass man nicht erwarten konnte, kleinere Schüler in der Menge zu sehen, inmitten der Oberstufenschüler.

\emph{Tick}.

Harry schaute dann zu den Hufflepuffs hinüber, um zu sehen, ob er Neville entdecken konnte, und obwohl Neville hinter einem viel größeren Schüler stand, gelang es Harrys visueller Verarbeitung, ihn fast sofort zu entdecken. Hermine war auch nicht bei den Hufflepuffs, nicht dass Harry es sehen konnte - und sie würde sicher nicht bei den Slytherins sein -

\emph{Tick}.

Harry drängte sich durch die dicht gedrängte Menge, trat neben oder um ältere Schüler herum und duckte sich in einem Fall einfach zwischen ihren Beinen hindurch, bis er zwischen den Ravenclaws stand und definitiv feststellen konnte, dass, nein, keine Hermine.

\emph{Tick}.

"Hermine Granger!" Sagte Harry laut. "Bist du hier?"

Niemand antwortete.

\emph{Tick}.

Irgendwo in seinem Hinterkopf stieg ein Gefühl des Entsetzens auf, während andere Teile von ihm versuchten, genau zu entscheiden, wie sehr er in Panik geraten sollte. Die erste Verteidigungsstunde des Jahres war ziemlich verschwommen in Harrys Gedächtnis, aber er erinnerte sich entfernt an etwas über Trolle, die in der Lage waren, Beute zu verfolgen, die allein und unverteidigt war.

\emph{Tick}.

Eine andere Gedankenspur suchte krampfhaft nach unausgesprochenen Möglichkeiten, was genau konnte er tun? Es war noch nicht 15 Uhr, also konnte er jetzt nichts mit seinem Zeitumkehrer erreichen. Selbst wenn er sich aus dem Zimmer schleichen könnte - es musste einen Weg geben, seinen Umhang unbemerkt anzulegen, irgendeine Art von Ablenkung, die er nutzen konnte - er hatte keine Ahnung, wo Hermine war, und Hogwarts war riesig.

\emph{Tick}.

Ein anderer Teil seines Verstandes versuchte, Möglichkeiten zu modellieren. Nach dem, was der andere Schüler gesagt hatte, waren Trolle keine leisen Raubtiere, sie waren laut - \emph{Hermine wird keine Ahnung haben, dass es ein Troll ist, also wird sie dem Lärm nachgehen. Sie ist eine Heldin, nicht wahr?} - aber Hermine hatte jetzt einen Unsichtbarkeitsumhang und einen Besen in ihrer Tasche. Harry hatte auf diesem Teil für sie und Neville bestanden, und Professor McGonagall hatte ihm gesagt, dass es erledigt war. Das sollte ausreichen, um Hermine entkommen zu lassen, auch wenn sie auf einem Besen lausig war. Alles, was sie tun musste, war, auf einen Teil des Daches zu gelangen, es war ein klarer Tag und Sonnenlicht sollte irgendwie schlecht für Trolle sein, Harry erinnerte sich an diesen Teil und deshalb würde Hermine sich genau daran erinnern. Und sicher, auch wenn Hermine sich wieder beweisen wollte, konnte sie unmöglich so dumm sein, einen Bergtroll anzugreifen.

\emph{Tick}.

Das würde sie nicht.

\emph{Tick}.

Das war einfach nicht ihre Art.

\emph{Tick}.

Und dann fiel Harry ein, dass schon mal jemand versucht hatte, Hermine Granger einen Mord anzuhängen, mit Hilfe von Gedächtniszaubern. Und zwar innerhalb von Hogwarts, ohne einen Alarm auszulösen. Und er hatte dafür gesorgt, dass Draco so langsam starb, dass die Alarmanlage erst 6 Stunden später ausgelöst wurde, wenn niemand mit einem Zeitdreher nachsehen konnte. Und dass derjenige, der clever genug war, einen Troll an den alten Zaubern von Hogwarts vorbeizuschleusen, ohne dass der Schulleiter kam, um die seltsame Kreatur zu untersuchen, auch clever genug war, den naheliegenden Schritt zu tun, Hermines magische Gegenstände zu verhexen…

\emph{Tick}.

Ein Teil von ihm fühlte so etwas wie langsam aufsteigende Panik, als sich die Perspektive verschob, ein Necker-Würfel, der seine Orientierung änderte, \emph{was zum Teufel hatte Harry sich dabei gedacht, Hermine und Neville in Hogwarts zu lassen, nur weil sie ein paar dumme Schmuckstücke bekommen hatten, das würde niemanden aufhalten, der sie töten wollte!}

\emph{Tick}.

Ein anderer Teil seines Verstandes leistete Widerstand, diese Möglichkeit war nicht sicher, sie war komplex und die Wahrscheinlichkeit konnte leicht unter 50\% liegen. Es war leicht, sich vorzustellen, dass er vor allen Leuten in eine riesige Panik geriet und Hermine dann von den Waschräumen außerhalb der Großen Halle zurückkam. Oder wenn der Troll am Ende gar nicht in ihre Nähe kam… wie in der Geschichte von dem Jungen, der Wolf rief, würde ihm beim nächsten Mal niemand glauben, wenn sie wirklich in Schwierigkeiten war; es könnte Rufschädigung bedeuten, die er später für etwas anderes brauchen würde…

\emph{Tick}.

Harry erkannte eine Instanz des Schemas der Angst vor Peinlichkeit, das die meisten Menschen davon abhielt, unter unsicheren Bedingungen jemals etwas zu tun, und er zerdrückte den Impuls hart. Schon damals war es seltsam, wie viel Willenskraft es brauchte, um den Entschluss aufzubringen, vor allen Leuten laut zu schreien, wenn er Hermine nur nicht in der Menge gesehen hatte würde es peinlich werden…

\emph{Tick}.

Harry holte tief Luft und rief so laut er konnte: "Hermine Granger! Bist du hier?"

Die Schüler drehten sich alle um und sahen ihn an. Dann drehten sich einige von ihnen um, um sich selbst umzusehen. Der Lärm im Raum wurde leiser, einige Gespräche verstummten.

"Hat irgendjemand Hermine Granger gesehen seit - seit etwa zehn Uhr dreißig heute oder so? Hat jemand eine Ahnung, wo sie sein könnte?"

Das Hintergrundgeplapper verstummte weiter. Niemand erhob die Stimme, um ihm etwas zuzurufen, schon gar nicht: "\emph{Keine Sorge, Harry, ich bin hier.}

"Oh, Merlin", sagte jemand aus der Nähe, und dann setzte das Hintergrundgeplapper wieder ein und nahm einen neuen, aufgeregten Ton an.

Harry starrte auf seine Hände, blendete das Gejammer aus und versuchte zu denken, denken, \textbf{DENKEN} -

\emph{Tick}. \emph{Tick}. \emph{Tick}.

Susan Bones und ein rothaariger Junge mit einem ramponiert aussehenden Zauberstab schoben sich beide gleichzeitig durch die Menge zu Harry.

"Wir müssen den Professoren irgendwie Bescheid sagen -"

"Wir müssen sie suchen -"

"Sie finden!"

Susan wandte sich an den anderen Jungen. "Wie sollen wir das anstellen, Captain Weasley?"

"Wir gehen los und suchen nach ihr!" schnappte Ron Weasley zurück.

"Bist du verrückt? Es gibt bereits Professoren, die die Gänge durchsuchen, wie kommst du darauf, dass wir eine bessere Chance haben als sie, General Granger zu begegnen? Wir werden nur von dem Troll gefressen! Und dann von der Schule verwiesen!"

Es war seltsam, wie manchmal das Hören schlechter Ideen die richtige Idee im Gegensatz dazu offensichtlich machte.

"Also gut, Leute! Hört zu!" Die Leute drehten sich um und schauten. "RUHE! ALLE! RUHE!" Harrys Kehle schmerzte danach, aber er hatte die Aufmerksamkeit aller. "Ich habe einen Besen", sagte Harry so laut, wie er es mit seiner immer noch schmerzenden Kehle schaffte. Er erinnerte sich an Askaban und an den Besen, auf dem nur zwei Personen Platz hatten, deshalb hatte er um einen gebeten hatte, der drei Personen tragen konnte. "Es ist ein 3-Sitzer. Ich brauche einen Siebtklässler aus den Armeen, der mit mir kommt. Wir werden so schnell wie möglich durch die Gänge fliegen und nach Hermine Granger suchen, sie abholen und sofort zurückkommen. Wer kommt mit mir?"

Dann wurde es ganz still in der Großen Halle. Die Schüler blickten sich unruhig an. Die jüngeren Schüler blickten erwartungsvoll zu den älteren Schülern, während diese wiederum zu den Schülern blickten, die die Umgrenzung bewachten. Die meisten von ihnen starrten geradeaus und richteten ihre Zauberstäbe aus nur für den Fall, dass der Troll in diesem Moment durch eine Wand brechen würde. Keiner bewegte sich. Keiner sprach.

Harry Potter sprach wieder. "Wir werden nicht gegen den Troll kämpfen. Wenn wir ihn sehen, werden wir einfach wegfliegen, und es gibt keine Möglichkeit, dass er mit uns auf einem Besen mithalten kann. Ich übernehme die Verantwortung dafür, es mit der Verwaltung zu klären. Bitte."

Die Leute schauten weiter zu den anderen. Harry starrte auf die schweigende Menge, das Dutzend Siebtklässler, die streng nach außen blickten, und spürte, wie ihn die Kälte überkam. Irgendwo in seinem Hinterkopf lachte Professor Quirrell höhnisch und spottete über die Idee, dass gewöhnliche Dummköpfe jemals aus eigenem Willen etwas Nützliches tun würden, ohne dass ein Zauberstab auf ihren Kopf gerichtet war…

\emph{Tick}.

Das Standardmittel gegen die Apathie der Umstehenden war, sich auf eine einzelne Person zu konzentrieren.

"Also gut", sagte Harry und versuchte, die befehlende Stimme des Jungen-der-lebte zu behalten, der nicht an Gehorsam zweifelte. "Miss~Morgan, kommen Sie mit mir, sofort. Wir haben keine Zeit zu verlieren."

Die Hexe, die er genannt hatte, drehte sich von der Stelle um, an der sie unentwegt auf die Umgebung gestarrt hatte, und ihr Gesichtsausdruck war für die eine Sekunde erschrocken, bevor sich ihr Gesicht verschloss.

"Die stellvertretende Schulleiterin hat uns allen befohlen, hier zu bleiben, Mr~Potter."

Es kostete Harry Mühe, seine Zähne zu öffnen.

"Professor Quirrell hat das nicht gesagt und du auch nicht. Professor McGonagall ist keine Taktikerin, sie hat nicht daran gedacht, zu überprüfen, ob wir vermisste Schüler haben, und sie hielt es für eine gute Idee, die Schüler durch die Gänge marschieren zu lassen. Aber Professor McGonagall versteht, wenn man sie auf ihre Fehler hinweist, du hast gesehen, wie sie auf dich und Professor Quirrell gehört hat, und ich bin sicher, sie würde nicht wollen, dass wir die Tatsache einfach ignorieren, dass Hermine Granger da draußen ist, allein -"

\emph{Tick}.

"Ich würde erwarten, dass die Professorin sagt, dass sie nicht möchte, dass noch mehr Schüler in den Fluren herumlaufen. Die Professorin sagte, wenn jemand aus irgendeinem Grund geht, wird er von der Schule verwiesen. Vielleicht brauchst du dir keine Sorgen zu machen, weil du der Junge-der-lebte bist, aber der Rest von uns schon!"

\emph{Tick}.

Irgendwo in seinem Hinterkopf lachte Professor Quirrell gerade über ihn. Von einem normalen Menschen zu erwarten, dass er ohne perfekte strategische Klarheit handelte, ohne den Fokus der Verantwortung auf sich selbst zu richten, wenn er eine gute Ausrede hatte, nichts zu tun…

"Das Leben einer Schülerin steht auf dem Spiel", sagte Harry mit ruhiger Stimme. "Sie könnte in diesem Moment gegen den Troll kämpfen. Nur aus Neugierde, sagt dir das überhaupt etwas?"

\emph{Tick}.

Miss~Morgans Gesicht verzog sich.

"Du - du bist der Junge-der-lebte! Geh einfach allein los und schnippe mit den Fingern, wenn du ihr helfen willst!"

\emph{Tick}.

Harry war sich kaum noch bewusst, was er sagte.

"Das ist nur Cleverness und Tricks, so eine Macht habe ich im echten Leben nicht, ein junges Mädchen braucht deine Hilfe, bist du nun ein Gryffindor oder nicht?"

"Warum sagst du mir das?!", rief Miss~Morgan. "Ich habe hier nicht das Sagen! Das hat Mr~Hagrid!"

Es entstand eine peinliche Pause, die den ganzen Raum durchdrang. Harry drehte sich und blickte zu dem riesigen Halbriesen auf, der die Schülerschar überragte, während sich auch alle anderen Köpfe zu ihm drehten.

"Mr~Hagrid", sagte Harry und versuchte, seine Stimme beherrschend zu halten. "Sie müssen diese Expedition genehmigen und zwar sofort."

Rubeus Hagrid sah zwiespältig aus, obwohl das schwer zu beurteilen war mit seinem riesigen Kopf, der so von seinem ungeschorenen Bart und seinen Locken umgeben war; nur seine Augen sahen lebendig aus, eingebettet in all das Haar.

"Eh …", sagte der Halbriese. "Man hat mir gesagt, ich soll euch alle beschützen -"

"Toll, können wir jetzt auch Hermine Granger in Sicherheit bringen? Du weißt schon, die Schülerin, der ein Mord angehängt wurde, den sie nicht begangen hat und die jemanden braucht, der ihr hilft?"

Der Halbriese schreckte auf, als Harry die Worte sprach. Harry starrte den riesigen Mann an, verzweifelt darauf bedacht, dass er die Andeutung aufnahm, und hoffte, dass die Worte ihn nicht an irgendjemand anderen verraten hatten - er konnte nicht nur ein Muskel sein, sicherlich waren James und Lily mit diesem Mann aus mehr als nur Mitleid befreundet -

"Reingelegt?", rief eine anonyme Stimme, von irgendwo dort drüben, wo sich die Slytherins versammelt hatten. "Ha, bist du immer noch dabei? Es würde ihr recht geschehen, wenn sie gefressen werden würde."

Es gab einige Lacher, auch wenn von anderswo Empörungsschreie kamen. Das Gesicht des Halbriesen straffte sich.

"Du bleibst hier, Junge", sagte Mr~Hagrid in einem dröhnenden Ton, der wahrscheinlich sanft gemeint war. "Ich werde selbst nach ihr sehen. Die Wahrheit ist, dass Trolle ein bisschen trickreich sein können - man muss sie an einem Knöchel fangen und sie genau richtig baumeln lassen, sonst reißen sie einen in Stücke…"

"Können Sie auf einem Besen reiten, Mr~Hagrid?"

"Eh -" Rubeus Hagrid runzelte die Stirn. "Nein."

"Dann können Sie nicht schnell genug suchen. Sechstklässler! Ich rufe alle Sechstklässler! Gibt es hier irgendwelche Sechstklässler, die keine nutzlosen Feiglinge sind?"

Stille.

"Fünfte Klasse"?

Mr~Hagrid, sagen Sie ihnen, dass sie mich begleiten dürfen und mich beschützen sollen! Ich versuche, vernünftig zu sein, verdammt noch mal!?"

Der Halbriese rang die Hände mit einem gequälten Gesichtsausdruck.

"Eh - ich -"

Irgendetwas schnappte in Harry zu und er begann, direkt auf die Türen zur Großen Halle zuzusteuern und schob jeden beiseite, der ihm nicht aus dem Weg ging, als wären es teigige Statuen. \emph{Irgendwo in seinem Kopf bewegte er sich durch einen leeren Raum voller mechanischer Puppen, von deren bedeutungslosen, lippenbewegten Geräuschen er abgelenkt worden war} -

eine riesige Gestalt stellte sich ihm in den Weg. Harry blickte auf.

"Das kann ich nicht zulassen, Harry Potter, nicht ausgerechnet du. In diesem Schloss gehen seltsame Dinge vor sich, und jemand könnte hinter Miss~Granger her sein - oder hinter dir."

Rubeus Hagrids Stimme war bedauernd, aber fest, und seine riesigen Hände lagen wie Gabelstapler an seiner Seite. "Ich kann dich nicht da rausgehen lassen, Harry Potter."

"Stupor!"

Der rote Blitz krachte in die Seite von Hagrids Kopf und ließ den riesigen Mann zusammenzucken. Sein Kopf ruckte schneller herum, als etwas so Großes sich hätte bewegen können, und brüllte: "Was glaubst du, was du da tust!?" auf die junge Gestalt von Susan Bones.

"Entschuldigung!", schrie sie. "Incendium! Glisseo!"

Die Hände des riesigen Mannes, der jetzt gegen das Feuer in seinem Bart klatschte, schafften es nicht ganz, sich zu fangen, als er zu Boden stürzte, aber das war zu diesem Zeitpunkt schon egal, denn Harry war an ihm vorbei und -

Neville Longbottom trat vor ihn, verzweifelt, aber entschlossen aussehend, den Zauberstab des Hufflepuff-Jungen bereits in der Hand. Harrys Hand griff reflexartig nach seinem Zauberstab, er schaffte es gerade noch, sich zu beherrschen, bevor Neville auf ihn schießen konnte, und starrte seinen Leutnant an, als wäre die Welt verrückt geworden.

"Harry!" Neville platzte heraus. "Harry, Mr~Hagrid hat recht, das darfst du nicht, das könnte alles eine Falle sein, sie könnten hinter dir her sein -"

Alle Muskeln von Neville versteiften sich und er kippte steif wie ein Brett zu Boden. Ein blass aussehender Ron Weasley trat hinter Neville hervor, seinen eigenen Zauberstab im Anschlag, und sagte: "Geh."

"Ron, du Verrückter, was machst du -", kam eine Stimme, die aus der Ferne als Miss~Clearwaters Freund identifiziert werden konnte, aber Harry war bereits zur Tür geeilt, ohne zurückzublicken, selbst als Rons Stimme und Susans Stimme sich wieder in Beschwörung erhoben. Es gab ein großes empörtes Gebrüll, und unbekannte Stimmen begannen zu schreien. Dann war Harry durch, seine Hand griff in seinen Beutel und seine Stimme sagte "Besenl", als hinter ihm die großen Türen wieder zu schwingen begannen.

Harry rannte weiter durch die Eingangshalle, selbst als der lange, dreiteilige Besen und seine Steigbügel aus dem Beutel herauszukommen begannen, wiederholte eine Reihe von Schimpfwörtern in seinem Kopf und dachte, \emph{das passiert, wenn man versucht, mit dem Teil seines Verstandes vernünftig zu sein, der nicht versucht, ein Suchmuster zu entwerfen, um Orte abzudecken, wo Hermine sein könnte}. Die Bibliothek befand sich im dritten Stock und praktisch auf der anderen Seite des Schlosses… Harry hatte die große Marmortreppe schon fast erreicht, als er den Besen in der Hand hatte und "\emph{Hoch}!", - er war in der Luft und beschleunigte hinauf in den zweiten Stock - "Gah!" schrie Harry und schaffte es gerade noch, seinen Besen in der Luft zu drehen, damit er nicht eine der menschlichen Gestalten aufspießte, die oben an der Treppe lauerten. Es gab einen grauenvollen Moment, in dem er versuchte, nicht vom Besen zu fallen, die Drehungen durchzuführen, die ihn in den Steigbügeln halten würden, obwohl er wirklich nahe am Boden war und fast keinen Spielraum hatte, und dann - "Fred? George?"

"Wir wissen nicht, wie wir sie finden können!", platzte einer der Weasley-Zwillinge heraus und verschränkte verzweifelt die Hände. "Wir haben uns rausgeschlichen, weil wir dachten, wir könnten Miss~Granger finden - es muss einen schnellen Weg geben, jemanden innerhalb des Hogwarts-Schlosses zu finden, da sind wir beide uns sicher - aber wir können nicht herausfinden, was es ist!"

Harry starrte die beiden an, von dort aus, wo er kopfüber am Besenstiel hing, wohin ihn sein verzweifeltes Manöver gebracht hatte, und ganz aus Reflex sagte er:

"Und warum wart ihr so sicher, dass ihr sie finden könnt?"

"Wir wissen es nicht!", rief der andere Weasley-Zwilling.

"Habt ihr schon einmal Leute in Hogwarts finden können?"

"Ja! Wir -" und der Weasley-Zwilling, der gerade sprach, hielt abrupt inne, beide Rothaarigen starrten mit leerem Blick in die Ferne.

Es gab ein donnerndes Krachen, als ob zwei riesige Türen von jemandem, der sehr, sehr stark war, aufgestoßen wurden. Harry wirbelte in der Luft herum, um den Weasley-Zwillingen die beiden offenen Steigbügelpositionen auf dem Besenstiel zu präsentieren, er sagte nichts, es gab keinen Grund für sie, ihre Positionen zu verraten, wenn sie es nicht mussten. Die Zeit schien zu langsam zu vergehen, als die Weasley-Zwillinge in die Steigbügel kletterten, Harrys Herz klopfte heftig, trotz seiner mentalen Berechnung, dass Mr~Hagrid, wenn er rannte, nicht einmal den Fuß der Treppe rechtzeitig erreichen würde. Der Steinboden unter ihnen verschwamm und die Wände schienen ein hörbares Zischen von sich zu geben (obwohl das nur der Wind in ihren Ohren war), als sie vorbeifuhren; Harry erinnerte sich daran, dass er auf einem längeren Besen für drei Personen saß, gerade noch rechtzeitig, um für die nächste Kurve abzubremsen. Und jetzt waren alle Besensitze besetzt, aber wenn sie Hermine tatsächlich fanden - Harry konnte den Unsichtbarkeitsumhang anlegen, das sollte ihn vor dem Troll verbergen, und das würde einen Sitzplatz für Hermine frei machen - Harry duckte sich heftig, bevor ein plötzlicher Torbogen ihm den Kopf abriss.

"Wir haben Jesse gefunden!", platzte der hinter Harry sitzende Weasley-Zwilling heraus. "Ich weiß, dass wir das haben! Damals mussten wir ihm sagen, dass Filch hinter ihm her ist!"

"Wie?" sagte Harry, der Großteil seines Gehirns damit beschäftigt, nicht bei einem schrecklichen Flugunfall zu sterben. Er hätte zur Sicherheit langsamer werden sollen, aber in ihm stieg eine Spannung auf, ein quellenloses Grauen. Er konnte nicht langsamer werden, etwas Schreckliches würde passieren, wenn er langsamer wurde…

"Wir -", sagte der weiter unten sitzende Weasley-Zwilling. "Wir können uns nicht erinnern!"

Eine weitere scharfe Kurve mit, so schätzte Harry, etwa 0,3 \% der Lichtgeschwindigkeit, und sie gingen durch einen kurvigen Korridor, den Harry immer nahm, um von der Großen Halle zur Bibliothek zu gelangen, nur dass es nicht der kürzeste Weg war, wenn man auf einem Besen saß, er hätte stattdessen den langen geraden Westkorridor nehmen sollen - Der Teil seines Gehirns, der nicht steuerte, holte die Realität ein.

"Jemand hat an eurem Verstand herumgepfuscht!" brüllte Harry, während er so schnell durch den kurvigen Korridor schlängelte, dass der hintere Weasley manchmal leicht gegen die Wand knallte, weil die Länge des Besens mit Harrys unangepassten Flugkünsten kollidierte.

"Was?", riefen Fred und George.

"Wer auch immer Hermine erwischt hat, hat auch mit eurem Verstand gespielt!"

Der Besen drehte sich und schoss neben einer Wendeltreppe nach oben, alle drei drückten sich gegen den Besen, damit sie durch den Spalt in der Decke in den dritten Stock gelangen konnten, und dann waren sie vor der Bibliothek, der Besen bremste mit einem Kreischen ab, obwohl es nichts gab, woran er sich hätte abstoßen können. Harry warf den Weasley-Zwillingen einen kurzen Blick zu, damit sie sich nicht vom Fleck rührten, während er vom Besen stieg und die Türen der Bibliothek aufstieß, wobei er seinen Atem kontrollierte, während er seinen Kopf hineinsteckte.

Hermine Granger war nicht da.

Madam Pince, die an ihrem Schreibtisch ein Sandwich aß, blickte mit einem plötzlichen Blick auf. "Die Bibliothek ist geschlossen!"

"Haben Sie Hermine Granger gesehen?" sagte Harry.

"Ich sagte, die Bibliothek ist geschlossen, Junge! Mittagspause!"

"Das ist äußerst wichtig. Hast du Hermine Granger gesehen oder hast du eine Ahnung, wo sie sein könnte?"

"Nein, und jetzt verschwinde!"

"Haben Sie eine Möglichkeit, Professor McGonagall im Notfall schnell zu erreichen?"

"Hä?", sagte die Bibliothekarin erschrocken. Sie erhob sich hinter ihrem Schreibtisch. "Was ist -"

"Ja oder nein. Bitte antworten Sie sofort."

"Ah - da ist der Kamin -"

"Sie ist nicht in ihrem Büro", sagte Harry. "Haben Sie eine andere Möglichkeit, sie zu erreichen. Ja oder nein."

"Junger Mann, ich bestehe darauf, dass Sie -"

Harrys Gehirn meldete dies als \emph{"Ich spreche wieder mit NSCs"}, und er drehte sich auf dem Absatz um und rannte zurück zum Besen.

"Halt!", rief Madam Pince und platzte zu spät aus der Tür, als Harry und die Weasley-Zwillinge wieder losschossen, aus dem Blickfeld der Bibliothekarin.

Der Druck in Harrys Kopf stieg immer noch an, wie eine physische Hand, die seine Brust zusammendrückte, er musste Hermine finden und er hatte keine andere Ahnung, wo sie sein könnte, es sei denn, es waren die Hexenschlafsäle im Ravenclaw-Turm und die konnte er nicht betreten. Ganz Hogwarts zu durchsuchen grenzte an eine mathematische Unmöglichkeit, es gab wahrscheinlich keinen durchgehenden Fluchtweg, der alle Räume zumindest einmal betrat - warum hatte er nicht daran gedacht, zu verlangen, dass Hermine und Neville und er einen Satz dieser hübschen kleinen Spiegel bekamen, die die Auroren zur Kommunikation benutzten -

Die Erkenntnis, dass er dumm war, traf Harry wie ein Schlag in den Magen. Er brauchte keine Spiegel, um eine Nachricht zu senden, er hatte seit Januar keine Spiegel mehr gebraucht. Harry bremste den Besen mitten in einem Gang ab, seinen Zauberstab schon in der Hand, der treibende Wille, Hermine Granger zu beschützen, stieg wie eine Sonne aus silbernem Feuer in seinem Kopf auf und floss seinen Arm hinunter, als er "\textbf{EXPECTO PATRONUM!}" schrie und der strahlend weiße Mensch wie eine Nova in die Existenz einbrach, die Stimmen der Weasley-Zwillinge schrien laut vor Schreck.

"Sag Hermine Granger - dass in Hogwarts ein Troll frei herumläuft - er könnte Jagd auf sie machen - sie muss ins direkte Sonnenlicht, sofort!"

Die silberne Gestalt drehte sich um, als wollte sie sich entfernen, und verschwand dann.

"Bei Merlins Unterhosen", hauchten Fred und George.

Der silberne Umriss sprengte zurück in die Welt und sagte in der seltsamen Außenversion von Harrys eigener Stimme: "Hermine Granger sagt", die Stimme der flammenden Gestalt wurde höher, "\emph{AHHHHHHHHH}!"

Die Zeit schien zu zerbrechen, als würde sich alles sehr schnell und langsam zugleich bewegen. Ein verzweifelter Impuls, den Besen zu beschleunigen, mit maximaler Geschwindigkeit zu fliegen, nur wusste Harry nicht, wo -

"Wenn du weißt, wo sie ist", rief Harry der lodernden humanoiden Gestalt zu und starrte in sie hinein, als wäre sie eine Sonne, "dann bring mich zu ihr!"

Der silberne Schein bewegte sich und Harry beschleunigte hinter ihm her, die Weasley-Zwillinge stießen hinter ihm hohe Schreie aus, als er wie eine Kanonenkugel durch die Luft schoss, er bewegte sich schneller als der Verstand, er konzentrierte sich nicht auf die Wände, die an ihm vorbeizogen, oder darauf, wie schnell er sich bewegte, folgte einfach dem silbernen Licht durch Korridore und flog Treppen hinauf und blitzte durch Türen, die Fred oder George mit verzweifelten Beschwörungsformeln aufschrieen, um sie zu öffnen, und das alles dauerte immer noch zu lange, irgendwo tief in seinem Innern fühlte Harry sich, als würde er durch Honig versinken, während Fenster und Porträts vorbeischossen. Der Besen kreischte durch eine letzte Kurve, die einen der Weasley-Zwillinge nicht ganz so hart gegen die Wand schlug, wie ein Klatscher es tun würde, und dann folgten sie dem leuchtenden Patronus durch einen offenen Raum in der Decke, der nach oben schoss und in weniger als einem Atemzug ein Stockwerk und dann das nächste überflog. Sein Patronus kam zum Stillstand (Harry bremste daraufhin stark ab), als sie gerade die Ebene eines weiten, offenen Raumes erreichten, der sich ausbreitete, bis er aus der Decke herauskam und sich in eine Außenterrasse verwandelte, eine Fläche aus gefliestem Marmor, die zur Luft und zum Himmel hin offen war -

