

\hypertarget{machiavellistische-intelligenzhypothese}{% \section{24. Machiavellistische Intelligenzhypothese}\label{machiavellistische-intelligenzhypothese}}

\textbf{\uline{Machiavellistische Intelligenzhypothese}}

Draco wartete in einer kleinen Fensternische, die er in der Nähe der Großen Halle gefunden hatte, mit knurrendem Magen. Es würde einen Preis geben, und er würde nicht gering sein. Draco hatte das gewusst, sobald er aufgewacht war und festgestellt hatte, dass er sich nicht in die Große Halle zum Frühstück traute, weil er dort Harry Potter sehen könnte, und Draco wusste nicht, was danach passieren würde. Schritte näherten sich.

"Da bist du ja", sagte Vincents Stimme. "Der Boss ist heute nicht gut gelaunt, also pass lieber auf, was du tust."

Draco wollte diesen Idioten bei lebendigem Leib häuten und den gehäuteten Körper mit der Bitte um einen intelligenteren Diener zurückschicken, wie eine tote Rennmaus. Eine Reihe von Schritten ging los, und die andere Reihe von Schritten kam näher. Das Aufstoßen in Dracos Magen wurde schlimmer. Harry Potter kam in Sicht. Sein Gesicht war sorgfältig neutral, aber seine blauen Roben sahen seltsam schief aus, als wären sie nicht richtig angezogen worden -

"Deine Hand", sagte Draco, ohne überhaupt darüber nachzudenken.

Harry hob seinen linken Arm, als wolle er ihn selbst betrachten. Die Hand baumelte schlaff davon, wie etwas Totes.

"Madam Pomfrey sagte, es ist nicht dauerhaft", sagte Harry leise. "Sie sagte, es sollte sich größtenteils erholen, wenn morgen der Unterricht beginnt."

Für einen Moment war die Nachricht eine Erleichterung. Und dann wurde es Draco klar. "Du bist zu Madam Pomfrey gegangen", flüsterte Draco.

"Natürlich habe ich das", sagte Harry Potter, als würde er das Offensichtliche aussprechen. "Meine Hand funktionierte nicht."

Langsam dämmerte es Draco, was für ein absoluter Narr er gewesen war, viel schlimmer als die älteren Slytherins, die er verprügelt hatte. Er hatte einfach angenommen, dass niemand zu den Behörden gehen würde, wenn ein Malfoy ihnen etwas antat. Dass niemand ein Auge von Lucius Malfoy auf sich haben wollte, niemals. Aber Harry Potter war kein ängstlicher kleiner Hufflepuff, der sich aus dem Spiel heraushalten wollte. Er war schon dabei, es zu spielen, und Vaters Auge war schon auf ihn gerichtet.

"Was hat Madam Pomfrey noch gesagt?", fragte Draco, das Herz in der Kehle.

"Professor Flitwick sagte, dass der Zauber, der auf meine Hand gewirkt wurde, ein dunkler Folterzauber und eine äußerst ernste Angelegenheit war, und dass es absolut inakzeptabel sei, sich zu weigern, zu sagen, wer es getan hat."

Es gab eine lange Pause. "Und dann?" sagte Draco mit zitternder Stimme.

Harry Potter lächelte leicht. "Ich habe mich zutiefst entschuldigt, was Professor Flitwick zu einem sehr strengen Blick veranlasste, und dann habe ich ihm gesagt, dass die ganze Sache in der Tat eine äußerst ernste, geheime und heikle Angelegenheit sei und dass ich den Schulleiter bereits über das Projekt informiert habe."

Draco schnappte nach Luft. "Nein! Flitwick wird das nicht einfach so hinnehmen! Er wird sich bei Dumbledore erkundigen!"

"In der Tat", sagte Harry Potter. "Ich wurde prompt ins Büro des Schulleiters geschleppt."

Draco zitterte jetzt. Wenn Dumbledore Harry Potter vor den Zaubergamot brachte, freiwillig oder nicht, und den Jungen-der-lebte unter Veritaserum aussagen ließ, dass Draco ihn gefoltert hatte… zu viele Leute liebten Harry Potter, Vater könnte diese Stimme verlieren… Vater könnte Dumbledore vielleicht überzeugen, das nicht zu tun, aber das würde kosten. Furchtbar viel. Das Spiel hatte jetzt Regeln, man konnte nicht mehr wahllos jemanden bedrohen. Aber Draco hatte sich aus freien Stücken in Dumbledores Hände begeben. Und Draco war eine sehr wertvolle Geisel. Aber da Draco kein Todesser mehr sein konnte, war er nicht so wertvoll, wie Vater dachte. Der Gedanke zerrte an seinem Herzen wie ein Schneidezauber. "Was dann?", flüsterte Draco.

"Dumbledore hat sofort gefolgert, dass du es warst. Er wusste, dass wir uns zusammengetan haben."

Das schlimmstmögliche Szenario. Hätte Dumbledore nicht geahnt, wer es war, hätte er es vielleicht nicht riskiert, Legilimenz einzusetzen, nur um es herauszufinden… aber wenn Dumbledore es wusste…

"Und?" Draco zwang das Wort heraus.

"Wir haben uns ein wenig unterhalten."

"Und?"

Harry Potter grinste. "Und ich habe ihm erklärt, dass es in seinem besten Interesse wäre, nichts zu tun."

Dracos Gedanken rannten gegen eine Backsteinmauer und plätscherten vor sich hin. Er starrte Harry Potter einfach nur an, mit schlaff hängendem Mund wie ein Narr. Es dauerte so lange, bis Draco sich erinnerte. Harry kannte Dumbledores mysteriöses Geheimnis, das Snape als seinen Halt benutzte. Draco konnte es gerade noch sehen. Dumbledore mit strengem Blick, der seinen Eifer verbarg, als er Harry erklärte, was für eine furchtbar ernste Angelegenheit dies war. \emph{Und Harry sagte Dumbledore höflich, er solle den Mund halten, wenn er wisse, was gut für ihn sei}. Vater hatte Draco vor solchen Leuten gewarnt, Leute, die einen ruinieren konnten und trotzdem so sympathisch waren, dass es schwer war, sie richtig zu hassen.

"Daraufhin", sagte Harry, "sagte der Schulleiter zu Professor Flitwick, dass es sich in der Tat um eine geheime und heikle Angelegenheit handele, über die er bereits informiert worden sei, und dass er nicht glaube, dass es mir oder irgendjemandem helfen würde, jetzt darauf einzugehen. Professor Flitwick fing an, etwas darüber zu sagen, dass die üblichen Intrigen des Schulleiters viel zu weit gingen, und ich musste ihn an dieser Stelle unterbrechen und erklären, dass es meine eigene Idee gewesen sei und nichts, wozu mich der Schulleiter gezwungen hätte, woraufhin Professor Flitwick herumwirbelte und anfing, mich zu belehren, woraufhin der Schulleiter ihn unterbrach und sagte, dass ich als Junge-der-lebte dazu verdammt sei, seltsame und gefährliche Abenteuer zu erleben, und dass es daher sicherer sei, wenn ich mich absichtlich darauf einlasse, anstatt darauf zu warten, dass sie zufällig passieren,

Und das war der Moment, in dem Professor Flitwick seine kleinen Hände in die Höhe warf und uns beide mit hoher Stimme anschrie, dass es ihm egal sei, was wir zusammen ausheckten, aber dass das nie wieder passieren dürfe, solange ich im Haus Ravenclaw sei, sonst würde er mich rauswerfen lassen und ich könnte nach Gryffindor gehen, wo dieses ganze Dumbledoring hingehörte -"

Harry machte es Draco sehr schwer, ihn zu hassen.

"Jedenfalls", sagte Harry, "wollte ich nicht aus Ravenclaw rausgeworfen werden, also versprach ich Professor Flitwick, dass so etwas nie wieder passieren würde, und wenn doch, würde ich ihm einfach sagen, wer es getan hat." Harrys Augen hätten kalt sein müssen. Waren sie aber nicht. Die Stimme hätte eine tödliche Drohung sein müssen.

War sie aber nicht. Und Draco sah die Frage, die offensichtlich hätte sein sollen, und die die Stimmung in einem Augenblick tötete.

"Warum… hast du nicht?"

Harry ging zum Fenster hinüber, in den kleinen Sonnenstrahl, der in die Nische schien, und drehte seinen Kopf nach außen, in Richtung des grünen Geländes von Hogwarts. Die Helligkeit schien auf ihn, auf seine Roben, auf sein Gesicht.

"Warum habe ich das nicht?" sagte Harry. Seine Stimme stockte. "Ich schätze, weil ich einfach nicht wütend auf dich werden konnte. Ich wusste, ich würde dich zuerst verletzen. Ich will es nicht einmal fair nennen, denn was ich dir angetan habe, war schlimmer als das, was du mir angetan hast."

Es war, als würde man gegen eine weitere Ziegelmauer rennen. Harry hätte in diesem Moment archaisches Griechisch sprechen können, so sehr Draco ihn auch verstand. Dracos Verstand suchte nach Mustern und stieß auf nichts. Die Aussage war ein Zugeständnis, das nicht in Harrys bestem Interesse lag. Es war nicht einmal das, was Harry sagen sollte, um Draco zu einem loyaleren Diener zu machen, jetzt, wo Harry Macht über ihn hatte. Dafür sollte Harry betonen, wie freundlich er gewesen war, nicht wie sehr er Draco verletzt hatte.

"Trotzdem", sagte Harry, und jetzt war seine Stimme tiefer, fast ein Flüstern, "bitte tu das nicht wieder, Draco. Es tat weh, und ich bin nicht sicher, ob ich dir ein zweites Mal verzeihen könnte. Ich bin nicht sicher, ob ich es wollen könnte."

Draco hat es einfach nicht verstanden. Wollte Harry etwa mit ihm befreundet sein? Harry Potter konnte nicht so dumm sein, zu glauben, dass das noch möglich war, nach allem, was er getan hatte. Man konnte jemandes Freund und Verbündeter sein, wie Draco es bei Harry versucht hatte, oder man konnte sein Leben zerstören und ihm keine andere Wahl lassen. Nicht beides. Aber dann verstand Draco nicht, was Harry sonst noch versucht haben könnte. Und dann kam Draco ein seltsamer Gedanke, etwas, von dem Harry gestern immer wieder gesprochen hatte. Und der Gedanke war: \emph{Teste es.} \emph{Du bist jetzt als Wissenschaftler erwacht,} hatte Harry gesagt, \emph{und selbst wenn du nie lernst, deine Kraft zu benutzen, wirst du immer nach Wegen suchen, um deine Überzeugungen zu testen.}..

Diese unheilvollen Worte, gesprochen in qualvollen Atemzügen, waren Draco immer wieder durch den Kopf gegangen. Wenn Harry den reumütigen Freund mimte, der versehentlich jemanden verletzt hatte…

"Du hast geplant, was du mit mir gemacht hast!" sagte Draco und schaffte es, einen Ton der Anklage in seine Stimme zu legen. "Du hast es nicht getan, weil du wütend warst, du hast es getan, weil du es wolltest!"

\emph{Dummkopf}, würde Harry Potter sagen, \emph{natürlich habe ich es geplant, und jetzt gehörst du mir} -

Harry wandte sich wieder Draco zu. "Was gestern passiert ist, war nicht der Plan", sagte Harry, seine Stimme schien ihm im Hals stecken zu bleiben. "Der Plan war, dass ich dir beibringen würde, warum es immer besser ist, die Wahrheit zu kennen, und dann würden wir gemeinsam versuchen, die Wahrheit über das Blut herauszufinden, und was auch immer die Antwort wäre, wir würden sie akzeptieren. Gestern habe ich… die Dinge überstürzt."

"\emph{Es ist immer besser, die Wahrheit zu kennen}", sagte Draco kalt. "Als ob du mir einen Gefallen getan hättest."

Harry nickte, was Draco völlig aus dem Konzept brachte, und sagte: "Was, wenn Lucius auf die gleiche Idee kommt wie ich, dass das Problem darin besteht, dass stärkere Zauberer weniger Kinder haben? Er könnte ein Programm starten, um die stärksten Reinblüter dafür zu bezahlen, mehr Kinder zu bekommen. Tatsächlich, wenn der Blutpurismus richtig wäre, ist es genau das, was Lucius tun sollte - das Problem auf seiner Seite angehen, wo er sofort etwas bewirken kann. Im Moment, Draco, bist du der einzige Freund, den Lucius hat, der versuchen würde, ihn davon abzuhalten, die Mühe zu verschwenden, denn du bist der Einzige, der die wirkliche Wahrheit kennt und die wirklichen Ergebnisse vorhersagen kann."

Draco kam der Gedanke, dass Harry Potter an einem Ort aufgewachsen war, der so fremdartig war, dass er nun tatsächlich eher ein magisches Wesen als ein Zauberer war. Draco konnte einfach nicht erraten, was Harry als nächstes sagen oder tun würde.

"Warum?" sagte Draco. Es war gar nicht so schwer, Schmerz und Verrat in seine Stimme zu legen. "Warum hast du mir das angetan? Was war dein Plan?"

"Nun", sagte Harry, "du bist Lucius' Erbe, und ob du es glaubst oder nicht, Dumbledore denkt, ich gehöre zu ihm. Wir könnten also erwachsen werden und ihre Kämpfe miteinander austragen. Oder wir könnten etwas anderes tun."

Langsam setzte sich Dracos Verstand in Bewegung.

"Du willst einen Kampf bis zum Ende zwischen ihnen provozieren und dann die Macht an dich reißen, wenn sie beide erschöpft sind." Draco spürte kaltes Grauen in seiner Brust. Er würde versuchen müssen, das zu verhindern, koste es, was es wolle - aber Harry schüttelte den Kopf.

"Sterne über dem Kopf, nein!"

"Nein…?"

"Du würdest das nicht mitmachen und ich auch nicht", sagte Harry. "Das ist unsere Welt, wir wollen sie nicht kaputt machen. Aber stell dir vor, sagen wir, Lucius dachte, die Verschwörung sei dein Werkzeug und du wärst auf seiner Seite, Dumbledore dachte, die Verschwörung sei mein Werkzeug und ich wäre auf seiner Seite, Lucius dachte, du hättest mich verwandelt und Dumbledore glaubte, die Verschwörung sei meine, Dumbledore dachte, ich hätte dich verwandelt und Lucius glaubte, die Verschwörung sei deine, und so halfen sie uns beide, aber nur auf eine Weise, die der andere nicht bemerken würde."

Draco musste nicht vortäuschen, sprachlos zu sein. Vater hatte ihn einmal in ein Theaterstück mitgenommen, das \emph{"Die Tragödie des Lichts"} hieß und in dem es um diesen unglaublich klugen Slytherin namens \emph{Licht} ging, der sich aufmachte, die Welt vom Bösen zu reinigen, indem er einen uralten Ring benutzte, der jeden töten konnte, dessen Namen und Gesicht er kannte, und der von einem anderen unglaublich klugen Slytherin bekämpft wurde, einem Schurken namens \emph{Lawliet}, der eine Verkleidung trug, um sein wahres Gesicht zu verbergen; und Draco hatte an den richtigen Stellen geschrien und gejubelt, besonders in der Mitte; und dann war das Stück traurig zu Ende gegangen und Draco war sehr enttäuscht gewesen und Vater hatte ihn sanft darauf hingewiesen, dass das Wort '\emph{Tragödie}' genau dort im Titel stand. Danach hatte Vater Draco gefragt, ob er verstehe, warum sie sich dieses Stück angesehen hätten. Draco hatte gesagt, es sei, um ihn zu lehren, so schlau wie \emph{Licht} und \emph{Lawliet} zu sein, wenn er erwachsen sei. Vater hatte gesagt, dass Draco unmöglich falscher liegen könnte, und darauf hingewiesen, dass \emph{Lawliet} zwar sein Gesicht geschickt verborgen hatte, es aber keinen guten Grund für ihn gab, \emph{Licht} seinen Namen zu sagen. Vater hatte dann fortgefahren, fast jeden Teil des Stücks zu demontieren, während Draco mit immer größer werdenden Augen zuhörte. Und Vater hatte damit geendet, dass Stücke wie dieses immer unrealistisch seien, denn wenn der Dramatiker gewusst hätte, was jemand, der tatsächlich so schlau ist wie \emph{Licht}, tun würde, hätte er versucht, selbst die Welt zu übernehmen, anstatt nur Stücke darüber zu schreiben.

Damals hatte Vater Draco von der Dreierregel erzählt, die besagte, dass jede Handlung, bei der mehr als drei verschiedene Dinge passieren mussten, im wirklichen Leben niemals funktionieren würde. Vater hatte weiter erklärt, dass, da nur ein Narr eine möglichst komplizierte Handlung versuchen würde, die wirkliche Grenze bei zwei läge. Draco konnte nicht einmal Worte finden, um die schiere, gigantische Unausführbarkeit von Harrys Masterplan zu beschreiben. Aber es war genau die Art von Fehler, die man machen würde, wenn man keine Mentoren hatte und dachte, man sei schlau und hätte das Intrigen spinnen durch das Ansehen von Theaterstücken gelernt.

"Also", sagte Harry, "was hältst du von dem Plan?"

"Er ist clever…" Sagte Draco langsam. \emph{Brillant}! zu schreien und vor Ehrfurcht zu keuchen hätte zu verdächtig ausgesehen.

"Harry, kann ich eine Frage stellen?"

"Klar", sagte Harry.

"Warum hast du Granger ein teures Täschchen gekauft?"

"Damit Sie nicht denkt ich wäre ihr böse", sagte Harry sofort. "Obwohl ich davon ausgehe, dass sie sich auch unangenehm fühlen wird, wenn sie in den nächsten Monaten irgendwelche kleinen Bitten von mir ablehnt."

Und das war der Moment, in dem Draco erkannte, dass Harry tatsächlich versuchte, sein Freund zu sein. Harrys Schachzug gegen Granger war klug gewesen, vielleicht sogar brillant. Den Feind dazu bringen, dass er keinen Verdacht schöpft, und ihn auf eine freundliche Art und Weise in die Pflicht nehmen, so dass man ihn in Position manövrieren konnte, indem man ihn einfach fragte. Draco hätte damit nicht durchkommen können, sein Ziel wäre zu misstrauisch gewesen, aber der Junge-der-lebte konnte es. Der erste Schritt von Harrys Plan war also, seinem Feind ein teures Geschenk zu machen, Draco wäre das nicht eingefallen, aber es könnte funktionieren…

Wenn man Harrys Feind wäre, wären seine Intrigen anfangs vielleicht schwer zu durchschauen, sie könnten sogar dumm sein, aber seine Überlegungen würden Sinn ergeben, wenn man sie erst einmal verstanden hat, man würde begreifen, dass er versucht, einem zu schaden. Die Art und Weise, wie Harry sich im Moment Draco gegenüber verhielt, machte keinen Sinn. Denn wenn du Harrys Freund warst, dann versuchte er, mit dir befreundet zu sein, auf die fremde, unverständliche Art, zu der er von Muggeln erzogen worden war, selbst wenn das bedeutete, dein ganzes Leben zu zerstören.

Die Stille dehnte sich. "Ich weiß, dass ich unsere Freundschaft furchtbar missbraucht habe", sagte Harry schließlich. "Aber bitte begreife, Draco, dass ich am Ende nur wollte, dass wir beide zusammen die Wahrheit finden. Ist das etwas, das du mir verzeihen kannst?"

\emph{Eine Gabelung mit zwei Wegen, aber mit nur einem Weg, auf den man später leicht zurückgehen konnte, falls Draco seine Meinung änderte…} "Ich denke, ich verstehe, was du vorhattest", log Draco, "also ja."

Harrys Augen leuchteten auf. "Ich bin froh, das zu hören, Draco", sagte er leise.

Die beiden Schüler standen in der Nische, Harry immer noch in den einsamen Sonnenstrahl getaucht, Draco im Schatten. Und Draco erkannte mit einem Anflug von Entsetzen und Verzweiflung, dass es zwar in der Tat ein schreckliches Schicksal war, Harrys Freund zu sein, aber Harry hatte jetzt so viele verschiedene Möglichkeiten, Draco zu bedrohen, dass es noch schlimmer wäre, sein Feind zu sein.

\emph{Wahrscheinlich. Ja, vielleicht. Na ja, er könnte später immer noch zum Feind werden… Er war dem Untergang geweiht.}

"Also", sagte Draco. "Und was jetzt?" "Lernen wir nächsten Samstag wieder?" "Hoffentlich läuft es nicht so wie beim letzten Mal -"

"Keine Sorge, das wird es nicht", sagte Harry. "Noch ein paar solche Samstage und du wärst mir voraus." Harry lachte. Draco tat es nicht. "Oh, und bevor du gehst", sagte Harry und grinste verlegen. "Ich weiß, es ist ein schlechter Zeitpunkt, aber ich wollte dich eigentlich um einen Rat bitten."

"Okay", sagte Draco, immer noch ein wenig abgelenkt von der letzten Aussage. Harrys Augen wurden aufmerksam. "Der Kauf dieses Beutels für Granger hat das meiste Gold verbraucht, das ich aus meinem Gringotts-Tresor stehlen konnte -"

"Was?

"- und McGonagall hat den Tresorschlüssel, oder Dumbledore hat ihn jetzt, vielleicht. Und ich war gerade dabei, einen Plan zu schmieden, für den ich etwas Geld brauche, also habe ich mich gefragt, ob du weißt, wie ich Zugang dazu bekommen kann -"

"Ich werde dir das Geld leihen", sagte Dracos Mund in einem schier existenziellen Reflex.

Harry sah verblüfft aus, aber auf eine erfreute Art. "Draco, das musst du nicht -"

"Wie viel?"

Harry nannte den Betrag und Draco konnte nicht ganz verhindern, dass sich der Schock in seinem Gesicht zeigte. Das war fast alles Taschengeld, das Vater Draco für das ganze Jahr gegeben hatte, Draco würde nur noch ein paar Galleonen übrig haben - Dann trat Draco mental gegen sein eigenes Schienbein. Er musste Vater nur schreiben und erklären, dass das Geld weg war, weil er es geschafft hatte, es Harry Potter zu leihen, und Vater würde ihm einen besonderen Glückwunschbrief mit goldener Tinte, einen riesigen Schokoladenfrosch, der zwei Wochen zum Essen brauchen würde, und zehnmal so viele Galleonen schicken, nur für den Fall, dass Harry Potter ein weiteres Darlehen brauchte.

"Das ist viel zu viel, nicht wahr?", sagte Harry. "Es tut mir leid, ich hätte nicht fragen sollen -"

"Verzeihung, ich bin ein Malfoy, weißt du", sagte Draco. "Ich war nur überrascht, dass du so viel willst."

"Keine Sorge", sagte Harry Potter fröhlich. "Es ist nichts, was die Interessen deiner Familie bedroht, ich bin einfach nur böse damit."

Draco nickte. "Dann ist das kein Problem. Willst du es gleich holen gehen?"

"Klar", sagte Harry.

Als sie die Nische verließen und sich auf den Weg zu den Kerkern machten, konnte Draco nicht anders, als zu fragen: "Kannst du mir sagen, für welche Intrige das ist?"

"Rita Kimmkorn."

Draco dachte sich ein paar sehr böse Worte, aber es war viel zu spät, um nein zu sagen. Als sie die Kerker erreicht hatten, hatte Draco wieder angefangen, seine Gedanken zu ordnen. Es fiel ihm schwer, Harry Potter zu hassen. Harry hatte versucht, freundlich zu sein, er war einfach wahnsinnig.

\emph{Und das würde Dracos Rache nicht aufhalten oder gar verlangsamen.}

"Also", sagte Draco, nachdem er sich umgesehen hatte, um sicherzugehen, dass niemand in der Nähe war. Ihre Stimmen würden natürlich beide verschwommen sein, aber es schadete nie, besonders sicher zu sein.

"Ich habe nachgedacht. Wenn wir neue Rekruten in die Verschwörung bringen, müssen sie denken, dass wir gleichberechtigt sind. Sonst bräuchte es nur einen von ihnen, um die Verschwörung auffliegen zu lassen. Das hast du doch schon geklärt, oder?"

"Natürlich", sagte Harry.

"Werden wir gleichberechtigt sein?", fragte Draco.

"Ich fürchte nicht", sagte Harry. Es war klar, dass er versuchte, sanft zu klingen, und auch klar, dass er versuchte, eine gehörige Portion Herablassung zu unterdrücken, was ihm nicht ganz gelang.

"Es tut mir leid, Draco, aber du weißt noch nicht einmal, was das Wort Bayesianer in Bayesianische Verschwörung bedeutet. Du wirst monatelang lernen müssen, bevor wir jemand anderen aufnehmen, nur damit du eine gute Fassade aufbauen kannst."

"Weil ich nicht genug von der Wissenschaft verstehe", sagte Draco und achtete darauf, seine Stimme neutral zu halten.

Harry schüttelte daraufhin den Kopf. "Das Problem ist nicht, dass du keine Ahnung von bestimmten wissenschaftlichen Dingen wie Desoxyribonukleinsäure hast. Das würde dich nicht davon abhalten, mir ebenbürtig zu sein. Das Problem ist, dass du nicht in den Methoden der Rationalität ausgebildet bist, dem tieferen, geheimen Wissen, wie all diese Entdeckungen überhaupt zustande gekommen sind. Ich werde versuchen, sie dir beizubringen, aber sie sind viel schwerer zu erlernen. Denk daran, was wir gestern getan haben, Draco. Ja, du hast einen Teil der Arbeit erledigt. Aber ich hatte als Einziger die Kontrolle. Du hast einige der Fragen beantwortet. Ich habe sie alle gestellt. Du hast beim Schieben geholfen. Ich habe selbst gelenkt. Und ohne die Methoden der Vernunft, Draco, kannst du die Verschwörung unmöglich dorthin lenken, wo sie hin soll."

"Ich verstehe", sagte Draco, und seine Stimme klang enttäuscht.

Harrys Stimme versuchte, sich noch mehr zu beruhigen. "Ich werde versuchen, deinen Sachverstand zu respektieren, Draco, wenn es um Menschendinge geht. Aber du musst auch mein Fachwissen respektieren, und es gibt einfach keine Möglichkeit, dass du mir ebenbürtig sein könntest, wenn es darum geht, die Verschwörung zu steuern. Du bist erst seit einem Tag Wissenschaftler, Du kennst ein Geheimnis über Desoxyribonukleinsäure, und du bist in keiner der Methoden der Rationalität ausgebildet."

"Ich verstehe", sagte Draco. Und das tat er. \emph{Menschendinge}, hatte Harry gesagt. Die Kontrolle über die Verschwörung zu erlangen, wäre wahrscheinlich nicht einmal schwierig. Und danach würde er Harry Potter töten, nur um sicherzugehen - Die Erinnerung stieg in Draco auf, wie krank er sich letzte Nacht innerlich gefühlt hatte, weil er wusste, dass Harry schrie. Draco dachte noch ein paar böse Worte.

Gut, er würde Harry nicht töten. Harry war von Muggeln aufgezogen worden, es war nicht seine Schuld, dass er geisteskrank war. Stattdessen würde Harry weiterleben, nur damit Draco ihm sagen konnte, dass alles nur zu Harrys eigenem Besten gewesen war, er sollte wirklich dankbar sein -

und mit einem plötzlichen Zucken überraschter Freude erkannte Draco, dass es tatsächlich zu Harrys eigenem Besten war. Wenn Harry versuchte, seinen Plan, Dumbledore und Vater für dumm zu verkaufen, auszuführen, würde er sterben. Das machte es perfekt. Draco würde Harry alle seine Träume nehmen, so wie Harry es mit ihm getan hatte. Draco würde Harry sagen, es sei zu seinem Besten gewesen. Und das würde absolut stimmen. Draco würde die Verschwörung und die Macht der Wissenschaft nutzen, um die Zaubererwelt zu läutern, und Vater würde so stolz auf ihn sein, als wäre er ein Todesser gewesen. Harry Potters böse Machenschaften würden vereitelt und die Kräfte des Rechts würden siegen. Die perfekte Rache.

Es sei denn…

\emph{Tu einfach so, als wärst du ein Todesser der so tut als wäre er Wissenschaftler},

hatte Harry ihm gesagt. Draco hatte keine Worte, um genau zu beschreiben, was mit Harrys Verstand nicht stimmte -

(da Draco den Begriff "Rekursionstiefe" noch nie gehört hatte)

- aber er konnte erahnen, welche Art von Verschwörungen damit verbunden waren. … es sei denn, all das war genau das, was Harry von Draco wollte, als Teil eines noch größeren Plans, in den Draco direkt hineinspielen würde, indem er versuchte, diesen zu vereiteln, Harry könnte sogar wissen, dass sein Plan undurchführbar war, er könnte keinen anderen Zweck haben, als Draco zu locken, ihn zu vereiteln -

Nein. In dieser Richtung lag der Wahnsinn. Es musste eine Grenze geben. Der Dunkle Lord selbst war nicht so verdreht gewesen. So etwas gab es nicht im wirklichen Leben, nur in Vaters albernen Gute-Nacht-Geschichten über dumme Wasserspeier, die jedes Mal, wenn sie versuchten, den Helden aufzuhalten, dessen Pläne unterstützten.

Und neben Draco ging Harry mit einem Lächeln im Gesicht entlang und dachte über die evolutionären Ursprünge der menschlichen Intelligenz nach. Am Anfang, bevor die Menschen ganz verstanden hatten, wie die Evolution funktionierte, waren sie auf verrückte Ideen gekommen, wie zum Beispiel, dass sich die menschliche Intelligenz entwickelte, damit wir bessere Werkzeuge erfinden konnten. Der Grund, warum das verrückt war, war, dass nur eine Person im Stamm ein Werkzeug erfinden musste, und dann würden alle anderen es benutzen, und es würde sich auf andere Stämme ausbreiten und noch hundert Jahre später von deren Nachkommen benutzt werden. Das war großartig aus der Perspektive des wissenschaftlichen Fortschritts, aber in evolutionärer Hinsicht bedeutete es, dass die Person, die etwas erfand, keinen großen Fitnessvorteil hatte, nicht viel mehr Kinder hatte als alle anderen.

Nur relative Fitnessvorteile konnten die relative Häufigkeit eines Gens in der Population erhöhen und eine einsame Mutation so weit treiben, dass sie universell war und jeder sie hatte. Und brillante Erfindungen waren einfach nicht häufig genug, um die Art von beständigem Selektionsdruck zu erzeugen, der nötig war, um eine Mutation zu fördern.

Wenn man sich die Menschen mit ihren Gewehren, Panzern und Atomwaffen ansah und sie mit Schimpansen verglich, war es eine natürliche Vermutung, dass die Intelligenz vorhanden war, um die Technologie zu entwickeln. Eine natürliche Vermutung, aber falsch. Bevor die Menschen ganz verstanden hatten, wie die Evolution funktionierte, hatten sie verrückte Ideen wie die, dass sich das Klima verändert, dass Stämme auswandern müssen und dass die Menschen klüger werden müssen, um all die neuen Probleme zu lösen.

Aber der Mensch hatte ein viermal so großes Gehirn wie ein Schimpanse. 20\% der metabolischen Energie eines Menschen ging in die Ernährung des Gehirns. Der Mensch war lächerlich viel schlauer als jede andere Spezies. So etwas passierte nicht, weil die Umwelt die Schwierigkeit ihrer Probleme ein wenig steigerte. Dann wurden die Organismen einfach ein bisschen schlauer, um sie zu lösen. Um zu diesem gigantischen, überdimensionalen Gehirn zu gelangen, war eine Art unkontrollierbarer Evolutionsprozess nötig, etwas, das grenzenlos schieben und schieben konnte. Und die heutigen Wissenschaftler hatten eine ziemlich gute Vermutung, was dieser außer Kontrolle geratene evolutionäre Prozess gewesen war.

Harry hatte einmal ein berühmtes Buch mit dem Titel \emph{'Der Affe in uns'} gelesen. Das Buch beschrieb, wie ein erwachsener Schimpanse namens Luit den alternden Alpha, Yeroen, mit der Hilfe eines jungen, kürzlich gereiften Schimpansen namens Nikkie konfrontiert hatte. Nikkie hatte nicht direkt in die Kämpfe zwischen Luit und Yeroen eingegriffen, sondern hatte Yeroens andere Unterstützer im Stamm davon abgehalten, ihm zu Hilfe zu kommen und sie abgelenkt, wann immer sich eine Konfrontation zwischen Luit und Yeroen entwickelte. Und mit der Zeit hatte Luit gewonnen und war der neue Alpha geworden, mit Nikkie als zweitmächtigstem…

… obwohl es nicht lange gedauert hatte, bis Nikkie sich mit dem besiegten Yeroen verbündete, Luit stürzte und der neue Alpha wurde.

Da weiß man wirklich zu schätzen, was Millionen von Jahren, in denen Hominiden versucht haben, sich gegenseitig zu überlisten - ein evolutionäres Wettrüsten ohne Grenzen - in Form von erhöhter geistiger Kapazität hervorgebracht hatten.

\emph{Denn, wissen Sie, ein Mensch hätte das absolut kommen sehen.}

Und neben Harry lief Draco her und unterdrückte ein Lächeln, als er als er über seine Rache nachdachte.

\emph{Eines Tages, vielleicht in Jahren, aber eines Tages, würde Harry Potter lernen. was es bedeutet, einen Malfoy zu unterschätzen.}

Draco war an einem einzigen Tag als Wissenschaftler erwacht. Harry hatte gesagt dass das eigentlich erst in Monaten passieren sollte. Aber wenn man ein Malfoy war, war man natürlich ein mächtigerer Wissenschaftler sein als jeder andere.

Also würde Draco alle Methoden von Harry Potter lernen und dann, wenn die Zeit reif war…

