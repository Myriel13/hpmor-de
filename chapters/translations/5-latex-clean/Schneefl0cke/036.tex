

\hypertarget{das-uxfcberschreiten-der-grenze}{% \section{37. Das Überschreiten der Grenze}\label{das-uxfcberschreiten-der-grenze}}

\textbf{\uline{Das Überschreiten der Grenze}}

Es war fast Mitternacht. Lange aufzubleiben war für Harry einfach genug. Er hatte nur noch nicht den Zeitumkehrer benutzt. Harry folgte der Tradition, seinen Schlafzyklus so zu timen, dass er wach war, wenn aus dem Weihnachtsabend der Weihnachtstag wurde; denn obwohl er nie jung genug gewesen war, um an den Weihnachtsmann zu glauben, war er doch einmal jung genug gewesen, um zu zweifeln.

\emph{Es wäre schön gewesen, wenn es eine geheimnisvolle Gestalt gegeben hätte, die in der Nacht in dein Haus kam und dir Geschenke brachte.}

.. Da lief Harry ein Schauer über den Rücken. Eine Andeutung von etwas Schrecklichem, das sich näherte. Ein schleichender Schrecken. Ein Gefühl des Unheils. Harry setzte sich kerzengerade im Bett auf. Er schaute zum Fenster.

"Professor Quirrell?" kreischte Harry ganz leise.

Professor Quirrell machte eine leichte Hebebewegung, und Harrys Fenster schien sich in seinen Rahmen zu falten. Sofort wehte ein kalter Winterwind durch den Spalt in den Raum, zusammen mit ein paar wenigen Schneeflocken von einem Himmel, der mit grauen Nachtwolken übersät war, inmitten von Schwarz und Sternen.

"Fürchten Sie sich nicht, Mr. Potter", sagte der Verteidigungsprofessor mit normaler Stimme.

"Ich habe Ihre Eltern in den Schlaf gezaubert; sie werden nicht erwachen, bevor ich gegangen bin."

"Niemand soll wissen, wo ich bin!", sagte Harry und kreischte dabei immer noch ein wenig.

"Selbst Eulen sollen meine Post nach Hogwarts bringen, nicht hierher!"

Harry hatte dem bereitwillig zugestimmt; es wäre dumm, wenn ein Todesser jederzeit den ganzen Krieg gewinnen könnte, nur indem er ihm eine magisch ausgelöste Handgranate per Eule zukommen lässt.

Professor Quirrell grinste, von wo aus er im Hinterhof jenseits des Fensters stand.

"Oh, ich sollte mir keine Sorgen machen, Mr. Potter. Sie sind gut gegen Ortungszauber geschützt, und kein Blutpurist käme auf die Idee, ein Telefonbuch zu konsultieren."

Sein Grinsen wurde breiter.

"Und es hat schon einige Mühe gekostet, die Schutzzauber zu überwinden, die der Schulleiter um dieses Haus gelegt hat - obwohl natürlich jeder, der Ihre Adresse kennt, einfach draußen warten und Sie angreifen könnte, wenn Sie das nächste Mal gehen."

Harry starrte Professor Quirrell eine Weile lang an.

"Was machen Sie hier?" sagte Harry schließlich. Das Lächeln verließ Professor Quirrells Gesicht.

"Ich bin gekommen, um mich zu entschuldigen, Mr. Potter", sagte der Verteidigungsprofessor leise.

"Ich hätte nicht so barsch mit Ihnen sprechen sollen, wie ich -"

"Nicht doch", sagte Harry. Er blickte auf die Decke hinunter, die er um seinen Schlafanzug geschlungen hatte. "Tu das nicht."

"Habe ich Sie so sehr beleidigt?", fragte Professor Quirrells ruhige Stimme.

"Nein", sagte Harry. "Aber das werden Sie, wenn Sie sich entschuldigen."

"Ich verstehe", sagte Professor Quirrell, und in einem Augenblick wurde seine Stimme streng.

"Wenn ich Sie also gleichberechtigt behandeln soll, Mr. Potter, dann muss ich sagen, dass Sie die Etikette, die zwischen befreundeten Slytherins gilt, schwerwiegend verletzt haben. Wenn Sie nicht gerade gegen jemanden spielen, dürfen Sie sich nicht so in seine Pläne einmischen, nicht ohne ihn vorher zu fragen. Denn du weißt weder, was sie wirklich vorhaben, noch, welchen Einsatz sie verlieren könnten. Es würde Sie als ihren Feind kennzeichnen, Mr. Potter."

"Es tut mir leid", sagte Harry in demselben ruhigen Ton, den Professor Quirrell verwendet hatte.

"Entschuldigung angenommen", sagte Professor Quirrell.

"Aber", sagte Harry, immer noch leise, "Sie und ich müssen wirklich weiter über Politik sprechen, irgendwann."

Professor Quirrell seufzte. "Ich weiß, Sie mögen keine Herablassung, Mr. Potter -"

\emph{Das war ein bisschen untertrieben.}

"Aber es wäre noch herablassender", sagte Professor Quirrell, "wenn ich es nicht deutlich sagen würde. Ihnen fehlt eine gewisse Lebenserfahrung, Mr. Potter."

"Und ist jeder, der genügend Lebenserfahrung hat, mit Ihnen einer Meinung? sagte Harry ruhig.

"Was nützt jemandem, der Quidditch spielt, Lebenserfahrung?", sagte Professor Quirrell und zuckte mit den Schultern.

"Ich denke, Sie werden Ihre Meinung mit der Zeit ändern, nachdem jedes Vertrauen, das Sie in andere setzten, versagt hat und Sie zynisch geworden sind."

Der Verteidigungsprofessor sagte es so, als wäre es die gewöhnlichste Aussage der Welt, eingerahmt von der Schwärze und den Sternen und dem wolkenverhangenen Himmel, während ein oder zwei winzige Schneeflocken in der beißenden Winterluft an ihm vorbei wehten.

"Da fällt mir ein", sagte Harry. "Fröhliche Weihnachten."

"Das nehme ich an", sagte Professor Quirrell. "Denn wenn es keine Entschuldigung ist, dann muss es ein Weihnachtsgeschenk sein. Das allererste, das ich jemals gegeben habe, um genau zu sein."

Harry hatte noch nicht einmal angefangen, Latein zu lernen, um das experimentelle Tagebuch von Roger Bacon lesen zu können; und er wagte kaum, den Mund aufzumachen, um zu fragen.

"Ziehen Sie Ihren Wintermantel an", sagte Professor Quirrell, "oder nehmen Sie einen wärmenden Trank, wenn Sie einen haben; und treffen Sie mich draußen, unter den Sternen. Ich werde sehen, ob ich es diesmal etwas länger aushalten kann."

Harry brauchte einen Moment, um die Worte zu verarbeiten, und dann eilte er zum Kleiderschrank.

\emph{Professor Quirrell hielt den Sternenzauber mehr als eine Stunde lang aufrecht, obwohl das Gesicht des Verteidigungsprofessors immer angespannter wurde und er sich nach einer Weile hinsetzen} \emph{musste.}

Harry protestierte nur einmal und wurde zum Schweigen gebracht.

\emph{Sie überquerten die Grenze von Heiligabend zum Weihnachtstag in jener zeitlosen Leere, in der irdische Rotationen nichts bedeuteten, der einzig wahren, immerwährenden Stillen Nacht.}

\emph{\hfill\break Und wie versprochen, schliefen Harrys Eltern tief und fest, bis Harry wieder sicher in seinem Zimmer war und der Verteidigungsprofessor gegangen war.}

