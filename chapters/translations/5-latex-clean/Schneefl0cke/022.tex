

\hypertarget{der-glaube-an-den-glauben}{% \section{23. Der Glaube an den Glauben}\label{der-glaube-an-den-glauben}}

\textbf{\uline{Der Glaube an den Glauben}}

\emph{Anmerkung des Übersetzers: das vorherige, dieses und die nächsten Kapitel sind sehr theoretisch und befassen sich damit wie die Magie in der Fanfiktion funktioniert. Für viele mit einem Naturwissenschaftlichen Hintergrund sind diese Ideen und Gedanken sehr Interessant, solltet Ihr Sie allerdings langweilig finden könnt ihr Sie zumindest überfliegen, da die Theorie für die Geschichte keinen großen Einfluss hat.}

„Und dann war Janet ein Squib“, sagte das Porträt einer kleinen jungen Frau mit einem goldumrandeten Hut.\\ Draco schrieb es auf. Das war erst achtundzwanzig, aber es war Zeit, zurückzugehen und Harry zu treffen.\\ Er hatte andere Porträts um Hilfe beim Übersetzen bitten müssen, die englische Sprache hatte sich sehr verändert, aber die ältesten Porträts hatten Zaubersprüche aus dem ersten Jahr beschrieben, die sich sehr ähnlich anhörten wie die, die sie jetzt hatten.\\ Draco hatte etwa die Hälfte von ihnen erkannt und die andere Hälfte klang auch nicht mächtiger. Das mulmige Gefühl in seinem Magen war mit jeder Antwort größer geworden, bis er es schließlich nicht mehr aushielt und stattdessen anderen Porträts Harry Potters seltsame Frage über Squib-Ehen stellte.\\ Die ersten fünf Porträts kannten niemanden, und schließlich hatte er diese Porträts gebeten, ihre Bekannten zu fragen, um ihre Bekannten zu fragen, und so gelang es ihm, einige Leute zu finden, die tatsächlich zugeben würden, mit Squibs befreundet zu sein.

(Der Slytherin aus dem ersten Jahr hatte erklärt, dass er mit einem Ravenclaw an einem wichtigen Projekt arbeitete, und der Ravenclaw hatte ihm gesagt, dass sie diese Information brauchten, und war dann abgehauen, ohne zu sagen warum.\\ Das hatte ihm viele mitleidige Blicke eingebracht.)

Dracos Füße waren schwer, als er durch die Korridore von Hogwarts ging.\\ Er hätte rennen sollen, aber er schien die Energie nicht aufbringen zu können. Er dachte immer wieder daran, dass er nichts davon wissen wollte, dass er in nichts davon verwickelt sein wollte, dass er nicht wollte, dass dies seine Verantwortung war, dass Harry Potter es einfach tun sollte, dass Harry Potter sich darum kümmern sollte, wenn die Magie schwand... Aber Draco wusste, dass das nicht richtig war.

Die Kerker von Slytherin, die grauen Steinwände, Draco mochte normalerweise die Atmosphäre, aber jetzt schien sie zu sehr zu verblassen. Seine Hand auf dem Türknauf, Harry Potter bereits drinnen und wartend, mit seinem verhüllten Umhang.

„Die alten Zaubersprüche der Erstklässler“, sagte Harry Potter. „Was hast du gefunden?"

„Sie sind nicht mächtiger als die Zaubersprüche, die wir jetzt benutzen."

Harry Potter schlug mit der Faust auf den Schreibtisch, hart.\\ „Verdammt noch mal. Also gut. Mein eigenes Experiment war ein Fehlschlag, Draco. Es gibt da etwas, das nennt sich das Interdikt von Merlin -"

Draco schlug sich an die Stirn, als es ihm klar wurde.

„- das jeden daran hindert, Wissen über mächtige Zaubersprüche aus Büchern zu erhalten, selbst wenn man die Notizen eines mächtigen Zauberers findet und liest, werden sie keinen Sinn ergeben, es muss von einem lebenden Geist zum anderen gehen.\\ Ich konnte deshalb keine mächtigen Zaubersprüche finden, für die wir die Anweisungen hatten, die wir aber nicht wirken konnten.\\ Aber wenn man sie nicht aus alten Büchern herausbekommt, warum sollte sich jemand die Mühe machen, sie mündlich weiterzugeben, wenn sie nicht mehr funktionieren? Hast du die Daten über die Squib-Paare bekommen?"

Draco begann, das Pergament zu überreichen - doch Harry Potter hielt eine Hand auf.

„Gesetz der Wissenschaft, Draco. Erst erzähle ich dir die Theorie und die Vorhersage. Dann zeigst du mir die Daten.\\ So weißt du, dass ich mir nicht nur eine Theorie ausdenke, die passen soll; du weißt, dass die Theorie die Daten tatsächlich im Voraus vorausgesagt hat.\\ Ich muss dir das sowieso erklären, also muss ich es erklären, bevor du mir die Daten zeigst. Das ist die Regel.\\ Also zieh deinen Umhang an und lass uns hinsetzen.„

Harry Potter setzte sich an einen Schreibtisch, auf dessen Oberfläche zerrissene Papierschnipsel verteilt waren. Draco holte seinen Umhang aus seiner Büchertasche, zog ihn an und setzte sich Harry gegenüber auf die andere Seite, wobei er die Papierschnipsel verwirrt betrachtete. Sie waren in zwei Reihen angeordnet und die Reihen waren etwa zwanzig Fetzen lang.

„Das Geheimnis des Blutes“, sagte Harry Potter mit einem intensiven Gesichtsausdruck, „ist etwas, das man \emph{Desoxyribonukleinsäure} nennt.\\ Diesen Namen sagt man nicht vor jemandem, der kein Wissenschaftler ist. Desoxyribonukleinsäure ist das Rezept, das deinem Körper sagt, wie er wachsen soll, zwei Beine, zwei Arme, klein oder groß, ob du braune oder grüne Augen hast.\\ Es ist ein materielles Ding, man kann es sehen, wenn man Mikroskope hat, die wie Teleskope sind, nur dass sie auf Dinge schauen, die sehr klein sind und nicht sehr weit weg.\\ Und dieses Rezept hat immer zwei Kopien von allem, für den Fall, dass eine Kopie kaputt ist. Stell dir zwei lange Reihen von Papierstücken vor.\\ An jedem Platz in der Reihe gibt es zwei Stücke Papier, und wenn du Kinder bekommst, wählt der Körper zufällig ein Stück Papier von jedem Platz in der Reihe aus, und der Körper der Mutter wird dasselbe tun, und so bekommt das Kind auch zwei Stücke Papier an jedem Platz in der Reihe.\\ Zwei Kopien von allem, eine von der Mutter, eine vom Vater, und wenn man Kinder hat, bekommen sie an jedem Ort zufällig ein Stück Papier von einem selbst."

Während Harry sprach, fuhren seine Finger über die gepaarten Papierschnipsel und zeigten auf den einen Teil des Paares, wenn er\\ „von deiner Mutter„\\ sagte, auf den anderen, wenn er\\ „von deinem Vater“ sagte.

Und als Harry davon sprach, ein Stück Papier wahllos auszuwählen, zog seine Hand einen Knut aus seinem Umhang und flippte ihn; Harry schaute auf die Münze und zeigte dann auf das obere Stück Papier. Das alles ohne eine Pause in der Rede.

„Nun, wenn es darum geht, ob man klein oder groß ist, gibt es eine Menge Stellen im Rezept, die kleine Unterschiede machen.\\ Wenn also ein großer Vater eine kleine Mutter heiratet, bekommt das Kind einige Zettel, auf denen 'groß' steht, und einige Zettel, auf denen 'klein' steht, und normalerweise endet das Kind in der Mitte.\\ Aber nicht immer. Mit etwas Glück bekommt das Kind viele Zettel, auf denen\\ „groß“ steht, und nicht viele Zettel, auf denen „klein“ steht, und wächst dann ziemlich groß auf.\\ Sie könnten einen großen Vater mit fünf Zetteln haben, auf denen „groß“ steht, und eine große Mutter mit fünf Zetteln, auf denen „groß“ steht, und durch erstaunliches Glück bekommt das Kind alle zehn Zettel, auf denen „groß“ steht, und endet größer als beide.

Verstehst du? Blut ist keine perfekte Flüssigkeit, es vermischt sich nicht perfekt. Desoxyribonukleinsäure besteht aus vielen kleinen Stücken, wie ein Glas mit Kieselsteinen anstelle eines Glases mit Wasser. Deshalb ist ein Kind auch nicht immer genau in der Mitte der Eltern.„

Draco hörte mit offenem Mund zu. \emph{Wie in Merlins Namen hatten die Muggel das alles herausgefunden? Konnten sie das Rezept sehen?}

„Nun“, sagte Harry Potter, „nehmen wir an, dass es, genau wie bei der Größe, viele kleine Stellen im Rezept gibt, an denen ein Zettel steht, auf dem '\emph{magisch}' oder '\emph{nicht magisch'} steht.\\ Wenn du genug Zettel hast, auf denen '\emph{magisch}' steht, bist du ein Zauberer, wenn du viele Zettel hast, bist du ein mächtiger Zauberer, wenn du zu wenige hast, bist du ein Muggel, und dazwischen bist du ein Squib.\\ Wenn dann zwei Squibs heiraten, sollten die Kinder meistens auch Squibs sein, aber hin und wieder hat ein Kind Glück und bekommt die meisten magischen Papiere des Vaters und die meisten magischen Papiere der Mutter und ist stark genug, um ein Zauberer zu sein.\\ Aber wahrscheinlich kein sehr mächtiger. Wenn man mit einer Menge mächtiger Zauberer anfängt und sie nur untereinander heiraten, würden sie mächtig bleiben.\\ Aber wenn sie anfangen würden, Muggelgeborene zu heiraten, die kaum magisch sind, oder Squibs... Siehst du? Das Blut würde sich nicht perfekt vermischen, es wäre ein Glas mit Kieselsteinen, nicht ein Glas mit Wasser, denn so funktioniert Blut nun mal.\\ Es würde immer noch ab und zu mächtige Zauberer geben, wenn sie durch Glück viele magische Papiere bekommen haben.\\ Aber sie würden nicht so mächtig sein wie die mächtigsten Zauberer von früher.„

Draco nickte langsam. So hatte er es noch nie erklärt bekommen. Es war überraschend schön, wie genau es passte.

„Aber“, sagte Harry. „Das ist nur eine Hypothese. Nehmen wir an, dass es stattdessen nur eine einzige Stelle im Rezept gibt, die einen zum Zauberer macht.\\ Nur eine einzige Stelle, an der auf einem Stück Papier '\emph{magisch}' oder \emph{'nicht magisch'} steht. Und von allem gibt es immer zwei Kopien. Dann gibt es also nur drei Möglichkeiten.\\ Auf beiden Kopien kann „\emph{magisch}“ stehen.\\ Auf einer Kopie kann „\emph{magisch}“ stehen und auf einer Kopie „\emph{nicht magisch"}.\\ Oder beide Kopien können \emph{"nicht magisch„} sagen.

Zauberer, Squibs und Muggel. Zwei Kopien und man kann zaubern, eine Kopie und man kann immer noch Tränke oder magische Geräte benutzen, und null Kopien bedeutet, dass man vielleicht sogar Schwierigkeiten hat, Magie direkt zu sehen.\\ Muggelgeborene würden nicht wirklich von Muggeln geboren werden, sondern von zwei Squibs, zwei Eltern mit je einer magischen Kopie, die in der Muggelwelt aufgewachsen sind.\\ Jetzt stell dir vor, eine Hexe heiratet einen Squib.

Jedes Kind bekommt von der Mutter ein Papier mit der Aufschrift „\emph{Magie}“, immer, egal welches Stück zufällig ausgewählt wird, auf beiden steht „\emph{Magie}“.\\ Aber wie beim Werfen einer Münze bekommt das Kind in der Hälfte der Fälle einen Zettel vom Vater, auf dem „\emph{magisch}“ steht, und in der anderen Hälfte der Fälle bekommt das Kind den Zettel des Vaters, auf dem \emph{"nicht magisch„} steht.

Wenn eine Hexe einen Squib heiratet, wird das Ergebnis nicht eine Menge schwacher Zaubererkinder sein.\\ Die Hälfte der Kinder werden Zauberer und Hexen sein, die genauso mächtig sind wie ihre Mutter, und die andere Hälfte der Kinder werden Squibs sein.\\ Denn wenn es nur eine Stelle im Rezept gibt, die einen zum Zauberer\\ macht, dann ist Magie nicht wie ein Glas mit Kieselsteinen, die sich mischen lassen.\\ Sie ist wie ein einzelner magischer Kieselstein, ein Zauberstein.“

Harry ordnete drei Paare von Papieren nebeneinander an.\\ Auf ein Paar schrieb er '\emph{Magie}' und '\emph{Zauberei}'.\\ Auf ein anderes Paar schrieb er nur auf das obere Papier '\emph{Magie}' und lies das andere leer.

Und das dritte Paar ließ er vollständig leer.

„In diesem Fall“, sagte Harry, „hast du entweder zwei Steine oder nicht.\\ Entweder du bist ein Zauberer oder nicht. Mächtige Zauberer würden so werden, indem sie härter lernen und mehr üben.\\ Und wenn Zauberer von Natur aus weniger mächtig werden, nicht weil Zaubersprüche verloren gehen, sondern weil die Leute sie nicht zaubern können... dann essen sie vielleicht die falschen Lebensmittel oder so. Aber wenn es über achthundert Jahre immer schlimmer geworden ist, dann könnte das bedeuten, dass die Magie selbst aus der Welt verschwindet.„

Harry ordnete zwei weitere Paare von Papieren nebeneinander an und nahm einen Federkiel heraus. Bald hatte jedes Paar ein Stück Papier, auf dem\\ „\emph{Magie}“ stand, und das andere Papier war leer.

„Und das bringt mich zu der Vorhersage“, sagte Harry.\\ „Was passiert, wenn zwei Squibs heiraten. Wirf zweimal eine Münze.\\ Es kann Kopf und Kopf,\\ Kopf und Zahl,\\ Zahl und Kopf oder\\ Zahl und Zahl herauskommen.

In einem Viertel der Fälle erhält man also zwei Köpfe, in einem Viertel der Fälle zwei Zahlen und in der Hälfte der Fälle einen Kopf und eine Zahl.\\ Dasselbe gilt, wenn zwei Squibs heiraten. Ein Viertel der Kinder würde mit Magie und Zauberei aufwachsen und Zauberer sein.\\ Ein Viertel würde nicht-magisch und nicht-magisch aufwachsen und Muggel sein. Die andere Hälfte würde Squibs sein.

Es ist ein sehr altes und sehr klassisches Muster. Es wurde von Gregor Mendel entdeckt, der nicht vergessen ist, und es war der erste Hinweis, der jemals aufgedeckt wurde, wie das Rezept funktioniert.\\ Jeder, der etwas über Blutkunde weiß, würde dieses Muster sofort erkennen. Es wäre nicht exakt, genauso wenig wie man, wenn man eine Münze zweimal vierzig Mal wirft, immer genau zehn Paare von zwei Köpfen erhält.\\ Aber wenn es sieben oder dreizehn Zauberer von vierzig Kindern sind, ist das ein starker Indikator.\\ Das ist der Test, den ich dich machen ließ. Jetzt lass uns deine Daten sehen.„

Und bevor Draco auch nur denken konnte, hatte Harry Potter das Pergament aus Dracos Hand genommen. Dracos Kehle war sehr trocken. Achtundzwanzig Kinder. Er war sich der genauen Zahl nicht sicher, aber er war sich ziemlich sicher, dass etwa ein Viertel davon Zauberer gewesen waren.

„Sechs Zauberer von achtundzwanzig Kindern“, sagte Harry Potter nach einem Moment. „Nun, das war's dann wohl. Und die Erstklässler haben vor acht Jahrhunderten die gleichen Zaubersprüche auf dem gleichen Energieniveau gewirkt.\\ Dein Test und mein Test kamen beide auf die gleiche Weise heraus.„

Es herrschte eine lange Stille im Klassenzimmer.

„Was nun?“ Draco flüsterte. Er war noch nie so erschrocken gewesen.

„Es ist noch nicht endgültig“, sagte Harry Potter.\\ „Mein Experiment ist fehlgeschlagen, schon vergessen? Ich brauche dich, um einen weiteren Test zu entwerfen, Draco.„

„ICH, ICH...“ sagte Draco. Seine Stimme brach. „Ich kann das nicht tun, Harry, das ist zu viel für mich."

Harrys Blick war grimmig.\\ „Doch, du kannst, weil du es musst. Ich habe auch schon darüber nachgedacht, nachdem ich von dem Interdikt von Merlin erfahren habe.\\ Draco, gibt es eine Möglichkeit, die Stärke der Magie direkt zu beobachten? Einen Weg, der nichts mit dem Blut von Zauberern oder den Zaubersprüchen zu tun hat, die wir lernen?„

Dracos Verstand war einfach leer.

„Alles, was die Magie beeinflusst, beeinflusst auch Zauberer“, sagte Harry.\\ „Aber dann können wir nicht sagen, ob es die Zauberer sind oder die Magie.\\ Auf was wirkt die Magie, das kein Zauberer ist?„

„Magische Kreaturen, offensichtlich“, sagte Draco, ohne darüber nachzudenken.

Harry Potter lächelte langsam. „Draco, das ist brillant."

Es war die Art von dummer Frage, die man nur stellen würde, wenn man von Muggeln erzogen worden wäre. Dann wurde das Übelkeitsgefühl in Dracos Magen noch schlimmer, als ihm klar wurde, was es bedeuten würde, wenn die magischen Wesen schwächer werden würden.\\ Dann wüssten sie mit Sicherheit, dass die Magie schwächer wurde, und ein Teil von Draco war sich bereits sicher, dass sie genau das herausfinden würden.\\ Er wollte das nicht sehen, er wollte es nicht wissen... Harry Potter war schon auf halbem Weg zur Tür.

„Komm schon, Draco! Nicht weit von hier gibt es ein Porträt, wir bitten sie einfach, einen Alten zu holen und finden es sofort heraus!\\ Wir sind getarnt, wenn uns jemand sieht, können wir einfach wegrennen! Los geht's!„

Danach dauerte es nicht mehr lange.\\ Es war ein breites Porträt, aber die drei Leute darin sahen ziemlich gedrängt aus. Da war ein Mann mittleren Alters aus dem zwölften Jahrhundert, der in schwarze Stoffbahnen gekleidet war; der sprach mit einer traurig aussehenden jungen Frau aus dem vierzehnten Jahrhundert, deren Haare ständig um ihren Kopf zu kräuseln schienen, als wäre sie durch einen statischen Zauber aufgeladen worden; und sie sprach mit einem würdevollen, verhutzelten alten Mann aus dem siebzehnten Jahrhundert mit einer soliden goldenen Fliege; und ihn konnten sie verstehen.

Sie hatten nach Dementoren gefragt. Sie hatten nach Phönixen gefragt. Sie hatten nach Drachen und Trollen und Hauselfen gefragt.\\ Harry hatte die Stirn gerunzelt, hatte darauf hingewiesen, dass die Kreaturen, die am meisten Magie benötigten, gerade völlig aussterben könnten, und hatte nach den mächtigsten magischen Kreaturen gefragt, die es gab.\\ Es gab nichts Unbekanntes auf der Liste, abgesehen von einer Spezies dunkler Kreaturen namens Gedankenfresser, von denen der Übersetzer anmerkte, dass sie von Harold Shea ausgerottet worden waren, und die klangen nicht halb so furchterregend wie Dementoren.\\ Magische Kreaturen waren jetzt so mächtig wie eh und je, wie es schien. Die Übelkeit in Dracos Magen ließ nach, und jetzt fühlte er sich nur noch verwirrt.

„Harry“, sagte Draco, als der alte Mann gerade eine Liste aller elf Kräfte der Augen eines Betrachters übersetzte, „was bedeutet das?„

Harry hielt einen Finger hoch und der alte Mann beendete die Liste.\\ Dann bedankte sich Harry bei den Porträts für ihre Hilfe - Draco tat das, quasi automatisch, ebenfalls und noch freundlicher - und sie gingen zurück ins Klassenzimmer.\\ Und Harry holte das Original-Pergament mit den Hypothesen heraus und begann zu kritzeln.

\textbf{Beobachtung}:\\ Die Zauberei ist heute nicht mehr so mächtig wie bei der Gründung von Hogwarts.

\textbf{Hypothesen}:\\ 1. Die Magie selbst verblasst.\\ 2. Die Zauberer kreuzen sich mit Muggeln und Squibs.\\ 3. Das Wissen, mächtige Zaubersprüche zu wirken, geht verloren.\\ 4. Zauberer essen als Kinder das Falsche, oder etwas anderes als Blut lässt sie schwächer aufwachsen.\\ 5. Die Muggeltechnologie stört die Magie. (Seit 800 Jahren?)\\ 6. Stärkere Zauberer haben weniger Kinder.\\ (Draco = Einzelkind? Prüfen Sie, ob 3 mächtige Zauberer, Quirrell / Dumbledore / Dunkler Lord, Kinder hatten.)

\textbf{Tests}:\\ \textbf{A}. Gibt es Zaubersprüche, die wir kennen, aber nicht wirken können (1 oder 2) oder sind die verlorenen Zaubersprüche nicht mehr bekannt (3)?\\ \textbf{Ergebnis}: Nicht schlüssig aufgrund des Interdikts von Merlin.\\ Kein bekannter unbeherrschbarer Zauber, könnte aber einfach nicht weitergegeben worden sein.

\textbf{B.} Wirkten die alten Erstklässler die gleiche Art von Zaubern mit der gleichen Kraft wie heute?\\ (Schwacher Beweis für 1 über 2, aber Blut könnte auch nur ein Verlust von mächtigen Zaubern sein).\\ \textbf{Ergebnis}: Gleiches Niveau von Erstjahreszaubern damals wie heute.

\textbf{C.} Zusätzlicher Test zur Unterscheidung von 1 und 2 mit Hilfe von wissenschaftlichen Erkenntnissen über Blut, wird später erklärt.\\ \textbf{Ergebnis}: Es gibt nur eine Stelle im Rezept, die einen zum Zauberer macht, und entweder man hat zwei Papiere, auf denen „Magie“ steht, oder man hat sie nicht.

\textbf{D}. Verlieren magische Geschöpfe ihre Kräfte? Unterscheidet 1 von (2 oder 3). \textbf{Ergebnis}: Magische Wesen scheinen so stark zu sein, wie sie es immer waren.

„A ist gescheitert“, sagte Harry Potter.\\ "B ist ein schwacher Beweis für 1 gegenüber 2.\\ C verfälscht 2.\\ D falsifiziert 1.\\ 4 war unwahrscheinlich und\\ B argumentiert auch gegen 4.\\ 5 war unwahrscheinlich und D argumentiert dagegen.\\ 6 ist zusammen mit 2 falsifiziert.\\ Damit bleibt 3. Interdikt von Merlin hin oder her, ich habe eigentlich keinen bekannten Zauber gefunden, der nicht gewirkt werden konnte.\\ Wenn man also alles zusammenzählt, sieht es so aus, als würde Wissen verloren gehen."

\emph{Und die Falle schnappte zu.\\ }\strut \\ Es dauerte keine fünf Sekunden, bis die Panik verflogen war und Draco begriff, dass die Magie nicht verschwunden war. Draco schob sich vom Schreibtisch weg und stand so heftig auf, dass sein Stuhl mit einem schabenden Geräusch über den Boden rutschte und umkippte.

„Es war also alles nur ein dummer Trick."

Harry Potter starrte ihn einen Moment lang an, immer noch sitzend.\\ Als er sprach, war seine Stimme leise.

„Es war ein fairer Test, Draco. Wenn es anders ausgegangen wäre, hätte ich es akzeptiert. Das ist nichts, bei dem ich jemals schummeln würde. Niemals. Ich habe mir deine Daten nicht angesehen, bevor ich meine Vorhersagen gemacht habe.\\ Ich habe dir im Voraus gesagt, als das Interdikt von Merlin das erste Experiment für ungültig erklärte -„

„Oh“, sagte Draco, die Wut begann in seine Stimme zu kommen, „du wusstest nicht, wie die ganze Sache ausgehen würde?„

„Ich wusste nichts, was du nicht wusstest“, sagte Harry, immer noch leise.\\ „Ich gebe zu, dass ich einen Verdacht hatte. Hermine Granger war zu mächtig, sie hätte kaum zaubern dürfen und sie war es nicht, wie kann eine Muggelgeborene die beste Zauberin in Hogwarts sein? Und sie bekommt auch die besten Noten in ihren Aufsätzen, es ist zu viel Zufall, dass ein Mädchen sowohl magisch als auch akademisch am stärksten ist, es sei denn, es gibt eine einzige Ursache.\\ Hermine Grangers Existenz deutet darauf hin, dass es nur eine Sache gibt, die einen zum Zauberer macht, etwas, das man entweder hat oder nicht hat, und die Machtunterschiede kommen daher, wie viel man weiß und wie viel man übt.

Und es gab keine verschiedenen Klassen für Reinblüter und Muggelgeborene, und so weiter. Es gab zu viele Möglichkeiten, dass die Welt nicht so aussah, wie sie aussehen würde, wenn du Recht hättest.\\ Aber Draco, ich habe nichts gesehen, was du nicht auch sehen konntest. Ich habe keine Tests durchgeführt, von denen ich dir nichts erzählt habe.\\ Ich habe nicht geschummelt, Draco. Ich wollte, dass wir die Antwort gemeinsam herausfinden. Und ich habe nie daran gedacht, dass die Magie aus der Welt verschwinden könnte, bis du es gesagt hast. Es war auch für mich eine beängstigende Vorstellung.„

„Wie auch immer“, sagte Draco. Er bemühte sich sehr, seine Stimme zu kontrollieren und Harry nicht einfach anzuschreien. „Du behauptest, dass du nicht weglaufen und jemand anderem davon erzählen wirst.„

„Nicht ohne dich vorher zu fragen“, sagte Harry. Er öffnete seine Hände in einer flehenden Geste. „Draco, ich bin so nett, wie ich kann, aber die Welt ist einfach nicht so, wie du sie dir vorgestellt hast."

„Gut. Dann ist es aus zwischen uns. Ich werde einfach weggehen und vergessen, dass das alles je passiert ist."\\ Draco wirbelte herum, spürte das Brennen in seiner Kehle, das Gefühl des Verrats, und da wurde ihm klar, dass er Harry Potter wirklich gemocht hatte, und dieser Gedanke bremste ihn keinen Moment, als er auf die Tür des Klassenzimmers zuging.

Und Harry Potters Stimme kam, jetzt lauter und besorgt:\\ „Draco... Du kannst nicht vergessen. Verstehst du denn nicht? Das war dein Opfer."

Draco blieb auf halbem Weg stehen und drehte sich um.\\ „Wovon redest du?"\\ Draco lief es jetzt schon eiskalt den Rücken herunter. Er wusste es schon, bevor Harry Potter es aussprach.

„Um ein Wissenschaftler zu werden. Du hast einen deiner Glaubenssätze in Frage gestellt, nicht nur einen kleinen Glaubenssatz, sondern etwas, das für dich große Bedeutung hatte.\\ Du hast Experimente gemacht, Daten gesammelt, und das Ergebnis hat bewiesen, dass der Glaube falsch war. Du hast die Ergebnisse gesehen und verstanden, was sie bedeuten."\\ Harry Potters Stimme schwankte.\\ „Vergiss nicht, Draco, du kannst einen \emph{wahren} Glauben nicht auf diese Weise opfern, denn die Experimente werden ihn bestätigen, anstatt ihn zu falsifizieren.\\ Dein Opfer, um ein Wissenschaftler zu werden, war dein \emph{falscher} Glaube, dass sich Zaubererblut vermischt und schwächer wird.„

„Das ist nicht wahr!“, sagte Draco. „Ich habe den Glauben nicht geopfert. Ich glaube immer noch daran!"\\ Seine Stimme wurde lauter und das Frösteln wurde immer schlimmer.

Harry Potter schüttelte den Kopf. Seine Stimme kam im Flüsterton.\\ „Draco... Es tut mir leid, Draco, du glaubst es nicht, nicht mehr."\\ Harrys Stimme erhob sich wieder.\\ „Ich werde es dir beweisen.

Stell dir vor, jemand erzählt dir, dass er einen Drachen in seinem Haus hält.\\ Du sagst ihnen, dass du ihn sehen willst. Sie sagen, es sei ein unsichtbarer Drache. Du sagst, gut, du wirst hören, wie er sich bewegt.\\ Sie sagen, es sei ein unhörbarer Drache. Du sagst, du wirst etwas Mehl in die Luft werfen und die Umrisse des Drachens sehen.\\ Sie sagen, der Drache ist für Mehl durchlässig. Und das Bezeichnende daran ist, dass sie im Voraus genau wissen, welche experimentellen Ergebnisse sie weg erklären müssen.\\ Sie wissen, dass alles so herauskommen wird, wie es herauskommt, wenn es keinen Drachen gibt, sie wissen im Voraus genau, welche Ausreden sie machen müssen.\\ Also sagen sie dass es einen Drachen gibt. Vielleicht glauben sie, dass sie glauben, dass es einen Drachen gibt, das nennt man \emph{Glauben-im-Glauben}.\\ Aber sie glauben es nicht wirklich. Man kann sich in dem, was man glaubt, irren, die meisten Menschen merken gar nicht, dass es einen Unterschied gibt zwischen dem \emph{Glauben an etwas} und dem \emph{Glauben, dass es gut ist, es zu glauben}."

Harry Potter hatte sich nun vom Schreibtisch erhoben und ging ein paar Schritte auf Draco zu.\\ „Und Draco, dass du nicht mehr an den Blutpurismus glaubst, und ich werde dir zeigen, dass du es nicht tust. Wenn der Blutpurismus wahr ist, dann macht Hermine Granger keinen Sinn, was könnte sie also erklären? Vielleicht ist sie eine Zauberer-Waise, die von Muggeln aufgezogen wurde, so wie ich? Ich könnte zu Granger gehen und darum bitten, Bilder von ihren Eltern zu sehen, um zu sehen, ob sie ihnen ähnlich sieht. Würdest du erwarten, dass sie anders aussieht? Sollen wir diesen Test durchführen?„

„Sie hätten sie bei Verwandten untergebracht“, sagte Draco, wobei seine Stimme zitterte. „Sie werden immer noch gleich aussehen."

„Siehst du. Du weißt bereits, welches Versuchsergebnis du zu entschuldigen hast.\\ Wenn du noch an den Blutpurismus glauben würdest, würdest du sagen, klar, schauen wir mal, ich wette, sie wird nicht wie ihre Eltern aussehen, sie ist zu mächtig, um eine echte Muggelgeborene zu sein -"

„Sie hätten sie bei Verwandten untergebracht!"

„Wissenschaftler können Tests machen, um sicher zu gehen, ob jemand das wahre Kind eines Vaters ist. Granger würde es wahrscheinlich tun, wenn ich ihrer Familie genug zahle. Sie würde keine Angst vor den Ergebnissen haben.\\ Also, was erwartest du, was dieser Test zeigen wird? Sag mir ich soll ihn durchführen und wir werden es tun.\\ Aber du weißt bereits, was der Test sagen wird. Du wirst es immer wissen. Du wirst es nie vergessen können.\\ Du magst dir wünschen, an Blutpurismus zu glauben, aber du wirst immer erwarten, dass genau das passiert, was passieren würde, wenn es nur eine Sache gäbe, die dich zu einem Zauberer macht.\\ \emph{Das war dein Opfer, ein Wissenschaftler zu werden.}"

Dracos Atem ging rasend schnell.\\ „Ist dir klar, was du getan hast?!"\\ Draco stürmte vor und packte Harry am Kragen seiner Robe.\\ Seine Stimme erhob sich zu einem Schrei, sie klang unerträglich laut in dem geschlossenen Klassenzimmer und der Stille.\\ \textbf{„Ist dir klar, was du getan hast?!"}

Harrys Stimme war zittrig.\\ „Du hattest einen Glauben. Der Glaube war falsch. Ich habe dir geholfen, das zu erkennen. Was wahr ist, ist schon so, es sich einzugestehen, macht es nicht schlimmer -„

Die Finger von Dracos rechter Hand ballten sich zu einer Faust, und diese Hand fuhr unaufhaltsam nach oben und schlug Harry Potter so hart gegen den Kiefer, dass sein Körper erst gegen den Schreibtisch und dann auf den Boden krachte.

„Idiot!“, schrie Draco. „Idiot! Idiot!„

„Draco“, flüsterte Harry vom Boden aus,\\ „Draco, es tut mir leid, ich dachte nicht, dass das monatelang passieren würde, ich habe nicht erwartet, dass du so schnell als Wissenschaftler erwachst, ich dachte, ich hätte länger Zeit, dich vorzubereiten, dir die Techniken beizubringen, die es weniger schmerzhaft machen, zuzugeben, dass du falsch liegst -„

„Was ist mit Vater?“ sagte Draco. Seine Stimme zitterte vor Wut.\\ „Wolltest du ihn vorbereiten oder war es dir einfach egal, was danach passiert?„

„Du kannst es ihm nicht sagen!“ sagte Harry, seine Stimme erhob sich alarmiert.\\ „Er ist kein Wissenschaftler! Du hast es versprochen, Draco!„

Für einen Moment war der Gedanke, dass Vater es nicht weiß, eine Erleichterung.\\ Doch dann stieg die wahre Wut auf.\\ „Du hast also geplant, dass ich ihn anlüge und ihm sage, dass ich immer noch glaube“, sagte Draco mit zitternder Stimme.\\ „Ich werde ihn immer anlügen müssen, und jetzt, wo ich erwachsen bin, kann ich kein Todesser sein, und ich werde ihm nicht einmal sagen können, warum nicht.„

„Wenn dein Vater dich wirklich liebt“, flüsterte Harry vom Boden aus,\\ „wird er dich auch dann noch lieben, wenn du kein Todesser wirst, und es klingt, als würde dein Vater dich wirklich lieben, Draco -„

„Dein Stiefvater ist ein Wissenschaftler“, sagte Draco.\\ Die Worte kamen heraus wie beißende Messer.\\ „Wenn du kein Wissenschaftler werden würdest, würde er dich trotzdem lieben.\\ \emph{Aber du wärst ein bisschen weniger besonders für ihn}."

Harry zuckte zusammen. Der Junge öffnete den Mund, als wolle er\\ \emph{„Es tut mir leid„} sagen, und schloss ihn dann wieder, anscheinend hatte er es sich anders überlegt, was entweder sehr klug von ihm war oder ein großes Glück, denn Draco hätte vielleicht versucht, ihn umzubringen.

„Du hättest mich warnen sollen“, sagte Draco. Seine Stimme erhob sich. „Du hättest mich warnen müssen!"

„Ich... ich habe... jedes Mal, wenn ich dir von der Macht erzählt habe, habe ich dir auch von dem Preis erzählt. Ich sagte, du musst zugeben, dass du dich geirrt hast. Ich sagte, dass dies der schwerste Weg für dich sein würde.\\ Dass dies das Opfer sei, das jeder bringen müsse, um Wissenschaftler zu werden. Ich sagte, was ist, wenn das Experiment das eine sagt und deine Familie und Freunde das andere -„

„\textbf{Das nennst du eine Warnung?!}“ Draco schrie jetzt.\\ \textbf{„Das nennst du eine Warnung? Wenn wir ein Ritual durchführen, das ein permanentes Opfer erfordert?"}

„ICH... ICH..."\\ Der Junge am Boden schluckte.\\ „Ich schätze, das war vielleicht nicht deutlich genug. Es tut mir leid. Aber das, was durch die Wahrheit zerstört werden \emph{kann}, sollte auch zerstört werden.„

Ihn zu schlagen war nicht genug.\\ „In einem Punkt irrst du dich“, sagte Draco, seine Stimme war tödlich.\\ „Granger ist nicht die stärkste Schülerin in Hogwarts. Sie hat nur die besten Noten im Unterricht. Du wirst gleich den Unterschied herausfinden."

Plötzlicher Schock zeigte sich in Harrys Gesicht, und er versuchte, sich schnell auf die Füße zu rollen - es war bereits zu spät für ihn.

„Expelliarmus!"\\ Harrys Zauberstab flog quer durch den Raum.\\ „Gom jabbar!„

Ein Puls aus tintenschwarzer Schwärze schlug auf Harrys linke Hand.

„Das ist ein Folterzauber“, sagte Draco.\\ „Er ist dazu da, um Informationen aus Leuten herauszubekommen.\\ Ich werde ihn einfach an dir lassen und die Tür hinter mir abschließen, wenn ich gehe. Vielleicht stelle ich den Sperrzauber so ein, dass er nach ein paar Stunden nachlässt.\\ Vielleicht lässt er auch nicht nach, bis du hier drin stirbst. Viel Spaß."\\ Draco bewegte sich geschmeidig rückwärts, den Zauberstab immer noch auf Harry gerichtet. Dracos Hand tauchte nach unten, griff nach seiner Büchertasche, ohne dass sein Ziel wankte. Der Schmerz zeigte sich bereits in Harry Potters Gesicht, als er sprach.

„Die Malfoys stehen über den Gesetzen für minderjährige Zauberer, nehme ich an? Es liegt nicht daran, dass dein Blut stärker ist. Sondern weil du schon geübt hast. Am Anfang warst du so schwach wie jeder von uns. Ist meine Vorhersage falsch?„

Dracos Hand wurde weiß um seinen Zauberstab, aber sein Ziel blieb fest.

„Nur damit du es weißt“, sagte Harry durch knirschende Zähne,\\ „wenn du mir gesagt hättest, dass ich falsch liege, hätte ich auf dich gehört.\\ Ich werde dich nicht foltern, wenn du mir zeigst, dass ich falsch liege. Und so wirst du auch. Eines Tages. Du bist jetzt als Wissenschaftler erwacht, und selbst wenn du nie lernst, deine Macht zu benutzen, wirst du immer“, Harry keuchte,\\ „nach Wegen suchen, deine Überzeugungen zu testen -"

Dracos Rückzug war jetzt weniger geschmeidig, etwas schneller, und er musste sich anstrengen, um seinen Zauberstab auf Harry zu halten, als dieser nach hinten griff, um die Tür zu öffnen und aus dem Klassenzimmer zu treten.\\ Dann schloss Draco die Tür wieder. Er wirkte den mächtigsten Schließzauber, den er kannte. Draco wartete, bis er Harrys ersten Schrei hörte, bevor er den Quietus zauberte.

\emph{Und dann ging er weg.}

\textbf{"Aaahhhh! Finite Incantatem! Aaaahhh!}"\\ Harrys linke Hand war in einen Topf mit kochendem Speiseöl gesteckt und dort gelassen worden.\\ Er hatte alles, was er hatte, in das \emph{Finite Incantatem} gesteckt und es funktionierte immer noch nicht.\\ Manche Verhexungen erforderten bestimmte Gegenzauber oder man konnte sie nicht rückgängig machen, oder vielleicht lag es einfach daran, dass Draco so viel stärker war.

„\textbf{Aaaaahhhh}!"\\ Harrys Hand begann jetzt wirklich zu schmerzen, und das störte seine Versuche, kreativ zu denken.\\ Aber ein paar Schreie später wurde Harry klar, was er tun musste. Leider befand sich sein Beutel auf der falschen Seite seines Körpers, und er musste sich ziemlich verrenken, um ihn zu erreichen, vor allem, weil sein anderer Arm reflexartig herumfuchtelte und er unaufhaltsam versuchte, die Quelle des Schmerzes wegzuschleudern.\\ Bis er es geschafft hatte, hatte sein anderer Arm es geschafft, seinen Zauberstab wieder wegzuschleudern.

„Medizinischer \textbf{ahhhhh}-Kit! Sanitätskoffer!"\\ Auf dem Boden war das grüne Licht zu schummrig, um es zu erkennen.\\ Harry konnte nicht stehen. Er konnte nicht krabbeln. Er rollte über den Boden dorthin, wo er seinen Zauberstab vermutete, und er war nicht da, und mit einer Hand schaffte er es, sich hoch genug zu heben, um seinen Zauberstab zu sehen, und er rollte dorthin und holte den Zauberstab und rollte zurück, wo der Verbandskasten geöffnet war.

Es gab auch eine ganze Menge Geschrei und ein bisschen Erbrechen.

Es brauchte acht Versuche, bis Harry Lumos zaubern konnte.\\ Und dann, nun ja, das Paket war nicht dafür gedacht, einhändig geöffnet zu werden, \emph{weil alle Zauberer Idioten waren, deshalb.}\\ Harry musste seine Zähne benutzen und so dauerte es eine Weile, bis Harry es endlich schaffte, das Taubheitstuch über seine linke Hand zu wickeln.\\ Als alles Gefühl in seiner linken Hand endlich verschwunden war, ließ Harry seinen Verstand los, lag regungslos auf dem Boden und weinte eine Zeit lang.

\emph{Nun}, sagte Harrys Geist leise in sich hinein, als er sich genug erholt hatte, um wieder in Worten zu denken. \emph{War es das wert?}\\ Langsam griff Harrys funktionstüchtige Hand nach einem Schreibtisch. Harry zog sich auf seine Füße.\\ Nahm einen tiefen Atemzug. Atmete aus. Lächelte. Es war kein großes Lächeln, aber es war trotzdem ein Lächeln.

\emph{Danke, Professor Quirrell, ohne dich hätte ich nicht verloren.}

Er hatte Draco noch nicht erlöst, nicht mal annähernd.\\ Im Gegensatz zu dem, was Draco selbst jetzt glauben mochte, war Draco immer noch das Kind eines Todessers, durch und durch.\\ Ein Junge, der aufgewachsen war und dachte, „\emph{Vergewaltigung„} sei etwas, was die coolen Älteren machten.\\ Aber es war ein verdammt guter Anfang.\\ Harry konnte nicht behaupten, dass alles so gelaufen war, wie er es geplant hatte.

Es war alles so, wie er es sich auf der Stelle ausgedacht hatte. Der Plan sah vor, dass dies erst im Dezember oder so geschehen sollte, nachdem Harry Draco die Technik beigebracht hatte, die Beweise nicht zu leugnen, wenn er sie sah.\\ Aber er hatte den ängstlichen Blick auf Dracos Gesicht gesehen, erkannt, dass Draco bereits eine alternative Hypothese ernst nahm, und den Moment genutzt.\\ Ein Fall von wahrer Neugierde hatte in der Rationalität die gleiche erlösende Kraft wie ein Fall von wahrer Liebe in Filmen.

Rückblickend betrachtet, hatte Harry sich Stunden Zeit gelassen, um die wichtigste Entdeckung in der Geschichte der Magie zu machen, und Monate, um die unentwickelten mentalen Barrieren eines elfjährigen Jungen zu durchbrechen.

\emph{Das könnte darauf hindeuten, dass Harry eine Art großes kognitives Defizit in Bezug} \emph{auf das Abschätzen von Aufgabenerledigungszeiten hatte.}

Kam Harry in die Hölle der Wissenschaft für das, was er getan hatte? Harry war sich nicht sicher. Er hatte es geschafft, Draco auf die Möglichkeit aufmerksam zu machen, dass die Magie schwindet, und dafür gesorgt, dass Draco den Teil des Experiments durchführte, der auf den ersten Blick in diese Richtung wies.\\ Er hatte gewartet, bis er die Genetik erklärt hatte, um Draco dazu zu bringen, sich über magische Wesen Gedanken zu machen

(obwohl Harry an uralte Artefakte wie den Sprechenden Hut gedacht hatte, den niemand mehr duplizieren konnte, der aber weiterhin funktionierte).

Aber Harry hatte eigentlich keine Beweise übertrieben, hatte die Bedeutung der Ergebnisse nicht verfälscht. Als das Interdikt von Merlin den Test ungültig gemacht hatte, der endgültig hätte sein sollen, hatte er es Draco von Anfang an gesagt.\\ Und dann war da noch der Teil danach... Aber er hatte Draco nicht wirklich angelogen. \emph{Draco hatte es geglaubt, und das würde es wahr machen.}

Das Ende war, zugegebenermaßen, nicht lustig gewesen. Harry drehte sich um und wankte zur Tür. Zeit, Dracos Sperrzauber zu testen. Der erste Schritt war, einfach zu versuchen, den Türknauf zu drehen. Draco könnte geblufft haben.

Draco hatte nicht geblufft.

„Finite Incantatem.“ Harrys Stimme klang ziemlich heiser, und er konnte spüren, dass der Zauber nicht gewirkt hatte.\\ Also versuchte Harry es noch einmal, und dieses Mal fühlte es sich richtig an. Aber eine weitere Drehung am Türknauf zeigte, dass es nicht funktioniert hatte.\\ Das war keine Überraschung. Zeit, die großen Geschütze aufzufahren. Harry holte tief Luft. Dieser Zauber war einer der mächtigsten, die er bis jetzt gelernt hatte.

„Alohomora!“ Harry taumelte ein wenig, nachdem er es gesagt hatte. Die Tür des Klassenzimmers öffnete sich immer noch nicht.\\ Das schockierte Harry. Harry hatte natürlich nicht vorgehabt, in die Nähe von Dumbledores verbotenem Korridor zu gehen. Aber ein Zauber zum Öffnen von magischen Schlössern war Harry ohnehin als nützlich erschienen, und so hatte er ihn gelernt. War Dumbledores verbotener Korridor dazu gedacht, die Leute so dumm zu ködern, dass sie nicht bemerkten, dass die Sicherheitsvorkehrungen schlimmer waren als das, was Draco Malfoy anbringen konnte?\\ Die Angst kroch zurück in Harrys System.

Auf dem Schild im Medizinkasten hatte gestanden, dass das Betäubungstuch nur bis zu dreißig Minuten lang sicher verwendet werden konnte.\\ Danach würde er sich automatisch ablösen und wäre 24 Stunden lang nicht wieder verwendbar. Im Moment war es 18:51 Uhr.\\ Er hatte das Tuch vor etwa fünf Minuten angelegt. Also trat Harry einen Schritt zurück und betrachtete die Tür.\\ Es war eine solide Platte aus dunklem Eichenholz, unterbrochen nur durch den messingfarbenen Metalltürknauf.\\ Harry kannte keine Spreng-, Schneid- oder Zertrümmerungszauber, und Sprengstoff zu verwandeln hätte gegen die Regel verstoßen, dass man keine Dinge verwandeln darf, die verbrannt werden sollen.\\ Säure war eine Flüssigkeit und hätte Dämpfe erzeugt... Aber das war kein Hindernis für einen kreativen Denker.\\ Harry legte seinen Zauberstab an eines der Messingscharniere der Tür und konzentrierte sich auf die Form der Baumwolle als reine Abstraktion, abgesehen von der materiellen Baumwolle, und ebenso auf das reine Material, abgesehen von dem Muster, das es zu einem Messingscharnier machte, und brachte die beiden Konzepte zusammen, indem er der Substanz eine Form aufzwang.

Eine Stunde Verwandlungsübungen jeden Tag für einen Monat hatte Harry an den Punkt gebracht, an dem er einen Gegenstand von fünf Kubikzentimetern in knapp einer Minute verformen konnte.\\ Nach zwei Minuten hatte sich das Scharnier überhaupt nicht verändert. Wer auch immer Dracos Verriegelungszauber entworfen hatte, er hatte auch an das gedacht.\\ Oder die Tür war ein Teil von Hogwarts und das Schloss war immun. Ein Blick zeigte, dass die Wände aus massivem Stein waren.\\ Genauso wie der Boden. Genauso wie die Decke. Man konnte einen Teil von etwas, das ein festes Ganzes war, nicht separat verwandeln; Harry hätte versuchen müssen, die ganze Wand zu verwandeln, was Stunden oder vielleicht Tage ununterbrochener Anstrengung erfordert hätte, wenn er es überhaupt hätte tun können, und wenn die Wand nicht mit dem Rest des ganzen Schlosses zusammenhing.

.. Harrys Zeitumkehrer würde sich erst um 21 Uhr öffnen. Danach konnte er bis 18 Uhr zurückgehen, bevor die Tür verschlossen wurde.\\ Wie lange würde der Folterzauber dauern? Harry schluckte schwer. Tränen stiegen ihm wieder in die Augen.

Sein brillanter, kreativer Verstand hatte gerade den genialen Vorschlag gemacht, \emph{dass Harry sich die Hand mit der Metallsäge aus dem Werkzeugset in seiner Tasche abschneiden könnte, was natürlich weh tun würde, aber viel weniger weh tun könnte als Dracos Schmerzzauber, da die Nerven weg wären; und er hatte Aderpressen im Heilerkoffer.}

Und das war offensichtlich eine abscheulich dumme Idee, die Harry den Rest seines Lebens bereuen würde.\\ Aber Harry wusste nicht, ob er zwei Stunden lang unter der Folter durchhalten würde. Er wollte raus aus diesem Klassenzimmer, er wollte sofort raus aus diesem Klassenzimmer, er wollte nicht zwei Stunden schreiend hier drin warten, bis er den Zeitumkehrer benutzen konnte, er musste raus und jemanden finden, der den Folterzauber von seiner Hand nahm...

\textbf{\emph{Denk nach}}! schrie Harry sein Gehirn an. \textbf{\emph{Denk nach! Denk nach!}}

Der Slytherin-Schlafsaal war fast leer. Die Leute waren beim Abendessen. Aus irgendeinem Grund fühlte sich Draco selbst nicht sehr hungrig.\\ Draco schloss die Tür zu seinem Privatzimmer, verriegelte sie, brachte sie zum Schweigen, setzte sich auf sein Bett und begann zu weinen.

\emph{Es war nicht fair. Es war nicht fair.}

Es war das erste Mal, dass Draco wirklich verloren hatte, Vater hatte ihn gewarnt, dass richtiges Verlieren beim ersten Mal weh tun würde, aber er hatte schon so viel verloren, es war nicht fair, es war nicht fair, dass er gleich beim ersten Mal alles verlor.\\ Irgendwo in den Kerkern schrie ein Junge, den Draco eigentlich mochte, vor Schmerz. Draco hatte noch nie jemanden verletzt, den er mochte.\\ Leute zu bestrafen, die es verdient hatten, sollte eigentlich Spaß machen, aber das hier fühlte sich einfach innerlich krank an. Vater hatte ihn nicht davor gewarnt, und Draco fragte sich, ob das eine harte Lektion war, die jeder lernen musste, wenn er erwachsen wurde, oder ob Draco einfach nur schwach war.

Draco wünschte, es wäre Pansy, die schrie. Das hätte sich besser angefühlt. Und das Schlimmste war, zu wissen, dass es ein Fehler gewesen sein könnte, Harry Potter zu verletzen. Wer war jetzt noch für Draco da? Dumbledore? Nach dem, was er getan hatte? \emph{Eher wäre Draco lebendig verbrannt}\\ \emph{.}\\ Draco würde zu Harry Potter zurückkehren müssen, weil er nirgendwo anders hinkonnte. Und wenn Harry Potter sagte, er wolle ihn nicht, dann wäre Draco ein Nichts, nur ein erbärmlicher kleiner Junge, der nie ein Todesser sein könnte, nie Dumbledores Fraktion beitreten, nie Wissenschaft lernen.

Die Falle war perfekt gestellt, perfekt ausgeführt worden. Vater hatte Draco immer wieder gewarnt, dass man das, w\emph{as man den dunklen Ritualen geopfert hat, nicht zurückgewinnen kann.}\\ Aber Vater hatte nicht gewusst, dass die verfluchten Muggel Rituale erfunden hatten, für die man keinen Zauberstab brauchte, Rituale, zu denen man überlistet werden konnte, ohne es zu wissen, und das war nur eines der schrecklichen Geheimnisse, die die Wissenschaftler kannten und die Harry Potter mitgebracht hatte.

Draco fing daraufhin an, noch heftiger zu weinen. Er wollte das nicht, er wollte das nicht, aber es gab kein Zurück mehr.\\ Es war zu spät. Er war bereits ein Wissenschaftler. Draco wusste, er hätte zurückgehen und Harry Potter befreien und sich entschuldigen sollen.\\ Es wäre das Klügste gewesen, was er hätte tun können. Stattdessen blieb Draco in seinem Bett und schluchzte.

Er hatte Harry Potter bereits verletzt. Es war vielleicht das einzige Mal, dass Draco ihm wehtun konnte, und er würde für den Rest seines Lebens an dieser einen Erinnerung festhalten müssen.\\

\hfill\break \textbf{Soll er doch weiter schreien.}

Harry ließ die Reste seiner Metallsäge auf den Boden fallen. Die Messingscharniere hatten sich als unempfindlich erwiesen, nicht einmal einen Kratzer hatten sie abbekommen, und Harry begann zu vermuten, dass selbst die Verzweiflungstat, zu versuchen, Säure oder Sprengstoff zu verwandeln, nicht ausgereicht hätte, um diese Tür zu öffnen.

Positiv war, dass der Versuch die Bügelsäge zerstört hatte. Seine Uhr zeigte 19:02 Uhr an, weniger als fünfzehn Minuten waren noch übrig, und Harry versuchte sich daran zu erinnern, ob es noch andere scharfe Dinge in seiner Tasche gab, die zerstört werden mussten, und fühlte, wie ein weiterer Anfall von Tränen aufkam.\\ Wenn er doch nur, wenn sein Zeitdreher sich öffnete, zurückgehen und verhindern könnte - Und da wurde Harry klar, dass er albern war.

Es war nicht das erste Mal, dass er in einem Raum eingesperrt war. Professor McGonagall hatte ihm schon gesagt, wie man das richtig macht. Sie hatte ihm auch gesagt, dass er den Zeitumkehrer nicht für so etwas benutzen sollte. Würde Professor McGonagall erkennen, dass dieser Anlass wirklich eine besondere Ausnahme rechtfertigte? Oder würde sie den Zeitumkehrer einfach ganz wegnehmen? Harry sammelte alle seine Sachen, alle Beweise, in seinem Beutel ein.

Ein \emph{Scourgify} kümmerte sich um das Erbrochene auf dem Boden, allerdings nicht um den Schweiß, der seine Roben durchtränkt hatte.\\ Die umgestürzten Tische ließ er stehen, es war nicht wichtig genug, um es wert zu sein, es mit einer Hand zu tun. Als er fertig war, schaute Harry auf seine Uhr.

19:04 Uhr. Und dann wartete Harry. Sekunden vergingen, die sich wie Jahre anfühlten. Um 19:07 Uhr öffnete sich die Tür. Professor Flitwicks pausbäckiges Gesicht sah ziemlich besorgt aus.\\ „Ist alles in Ordnung, Harry?“, fragte die piepsige Stimme des Hausoberhauptes von Ravenclaw. „Ich habe eine Nachricht bekommen, dass du hier eingeschlossen wurdest -“

