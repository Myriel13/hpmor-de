

\hypertarget{selbstverwirklichung-teil-8}{% \section{73. Selbstverwirklichung, Teil 8}\label{selbstverwirklichung-teil-8}}

\textbf{\uline{Selbstverwirklichung, Teil 8}}

\emph{Der rote Feuerstrahl traf Hannah voll ins Gesicht, schleuderte sie mit dem Kopf voran gegen die Steinmauer, wo ihr bleiches Gesicht noch einen Augenblick zu verweilen schien, umrahmt von fliegenden braun-goldenen Haarsträhnen, bevor sie in einem Haufen Roben zu Boden sank, als die dritte und letzte Salve der flammenden grünen Spiralen den Schildzauber ihres Feindes zu Fall brachte.}

Die Märztage zogen vorbei, gefüllt mit Vorlesungen und Lernen und Hausaufgaben, Frühstück und Mittagessen und Abendessen.

\emph{Der Gryffindor-Junge starrte die acht an, Spannung in jeder Linie seines Körpers, sein Gesicht wirkte tonlos; und dann lösten seine Hände ihren geballten Griff um das Revers des Slytherin-Jungen, und er ging davon, ohne dass jemand ein Wort sagte.}

(Nun, Lavender hätte fast ein Wort gesagt - ihr Mund öffnete sich gerade in Empörung, vielleicht weil sie keine Chance bekommen hatte, ihre Rede zu deklamieren - aber zum Glück bemerkte Hermine es und machte die Geste, die bedeutete: SCHEIßE.)

\emph{Dann war da natürlich noch das Schlafen. Man würde das Schlafen nicht vergessen wollen, nur weil es so normal schien.}

"Innvervate!", sagte die junge Stimme von Susan Bones, und Hermines Augen flogen auf und ihre Lippen zogen mit einem Keuchen Luft ein, ihre Lungen fühlten sich schwer an, als würde ein riesiges Gewicht auf ihrer Brust ruhen. Neben ihr saß Hannah bereits aufrecht, hielt ihren Kopf in den Händen und zog eine Grimasse. Daphne hatte sie gewarnt, dass dies ein "harter" Kampf werden würde, was in Hermine, und eigentlich in allen von ihnen, eine gewisse Beklommenheit hervorrief. Außer vielleicht Susan, die einfach zur verabredeten Zeit aufgetaucht war, neben ihnen herging, ohne zu sprechen, und gegen den Tyrannen aus dem siebten Jahr kämpfte, bis sie das letzte Mädchen war, das noch stand. Vielleicht hatte sich der Gryffindor geweigert, gegen die letzte Tochter von Bones zu kämpfen, oder vielleicht hatte Susan einfach nur Glück gehabt; auf jeden Fall hatte Hermine, als sie versucht hatte, sich wieder aufzusetzen, festgestellt, dass sich ihre Brust schwer anfühlte, weil tatsächlich ein ziemlich großer Körper auf ihr gelegen hatte. Und man wollte die Magie auch nicht vergessen, auch wenn der eigentliche Moment des Zauberns nur einen sehr kleinen Teil des Tages ausmachte. Das war schließlich der ganze Sinn von Hogwarts.

"Okay, wie wäre es, wenn wir alle auf Skateboards herumfahren?", sagte Lavender. "Wir könnten schneller überall hinkommen als zu Fuß. Und wir würden auf Skateboards richtig geil aussehen, Muggelgeräte sind vielleicht nicht so schnell wie Besen, aber sie sehen cooler aus - wir sollten darüber abstimmen."

Was die verbleibenden Bruchteile der Zeit anging, so füllten sie diese je nach Veranlagung: Klatsch und Tratsch über Oberstufenromanzen oder Bücher und Lernsessions. Hermine streckte eine zitternde Hand aus, um ihr Exemplar von 'Die Geschichte von Hogwarts' von dort aufzufangen, wo es heruntergefallen war, das immer tröstende Buch nur einen Schritt von der Stelle entfernt, an der sie selbst auf dem Boden gelandet war, nachdem die rotgewandete Oberstufenschülerin sie gegen eine Wand "gestoßen" hatte. Und dann war die ältere Gryffindor-Hexe weggegangen, ohne einen Blick zurück, nur mit einem geflüsterten "Salazars -" und einem Wort, das sie mehr verletzte als alles, was die Slytherins über Schlammblüter sagten, "\emph{Schlammblüter}" war nur ein seltsames Zaubererwort, aber Hermine kannte das Wort, das die Gryffindor gesagt hatte. Sie konnte sich nicht daran gewöhnen, sie konnte sich einfach nicht daran gewöhnen, gehasst zu werden. Es tat immer noch genauso weh, jedes Mal, wenn es passierte, und irgendwie tat es noch mehr weh, wenn es von den Gryffindors kam, die eigentlich die Guten sein sollten.

Harry hatte, wie befohlen, acht seiner Soldaten auf die anderen Armeen aufgeteilt; er hatte freiwillig zwei Chaotische Leutnants abgegeben, Dean Thomas zur Drachenarmee geschickt und dann Seamus Finnigan gegen Blaise Zabini eingetauscht, von dem Harry gesagt hatte, dass er in Sonnenschein "nicht ausgelastet" war. Lavender hatte sich entschieden, sich dem Großteil von SPHEW in Sonnenschein anzuschließen; Tracey hatte beschlossen, bei Chaos zu bleiben.

"Du kannst also deinen Zauber auf General Potter anwenden?", fragte Lavender, während Hermine die beiden so gut wie möglich ignorierte. "Ich muss sagen, Traces, ich glaube, unser Sonnenschein-General hat ihn inzwischen ziemlich gut im Griff - du hättest mehr Glück dabei, Hermine davon zu überzeugen, dass ihr drei eine dieser, du weißt schon, Absprachen treffen solltet -"

Niemand hatte bisher herausgefunden, was Draco Malfoy vorhatte.

"Sicher?", sagte Harry Potter und klang dabei etwas zögernd.

"Du weißt doch, dass ein Rationalist sich nie über etwas sicher ist, Hermine, nicht einmal darüber, dass zwei und zwei vier ergibt. Ich kann nicht wirklich Malfoys Gedanken lesen, und selbst wenn ich es könnte, könnte ich nicht sicher sein, dass er nicht ein perfekter Okklumentiker ist. Ich kann nur sagen, dass es nach dem, was ich von Malfoy gesehen habe, viel plausibler ist, als Daphne Greengrass denkt, dass er tatsächlich versucht, den Slytherins einen besseren Weg zu zeigen. Wir sollten… wir sollten wirklich versuchen, uns dem anzuschließen, Hermine."

(Nun, Harry schien Draco Malfoy für einen guten Kerl zu halten. Aber das Problem war, dass Harry auch dazu neigte, Leuten wie Professor Quirrell zu vertrauen.)

….

"Professor Quirrell", sagte Harry, "ich mache mir Sorgen wegen des Hasses, den das Haus Slytherin auf Hermine Granger zu entwickeln scheint."

Sie saßen im Büro des Verteidigungsprofessors, Harry saß weit hinten am Lehrerpult (und das Gefühl des drohenden Unheils war selbst dann noch spürbar), das leere Bücherregal umrahmte immer noch Professor Quirrells kahl werdenden Kopf. Die Tasse, die auf Harrys Oberschenkel balancierte, war mit Professor Quirrells obskurem, wahrscheinlich teurem chinesischen Tee gefüllt, und es sagte etwas über die Art und Weise aus, wie Harry in letzter Zeit gedacht hatte, dass er eine bewusste Entscheidung treffen musste, um ihn zu trinken.

"Und das betrifft mich aus welchem Grund?", sagte Professor Quirrell und nippte an seinem Tee.

"Ja, nun", sagte Harry, "ich werde einfach ignorieren, dass - oh, hören Sie auf damit, Professor Quirrell, Sie planen mindestens seit dem ersten Freitag dieses Jahres, den Ruf des Hauses Slytherin wiederherzustellen."

Es hätte ein winziger Anflug eines Lächelns sein können, an den Rändern dieser dünnen, blassen Lippen; und dann wiederum hätte es auch nicht sein können.

"Ich denke, das Haus Slytherin wird am Ende gut genug abschneiden, Mr. Potter, unabhängig vom Schicksal eines einzelnen Mädchens. Aber ich stimme zu, dass die gegenwärtigen Aussichten für Ihren kleine Freundin nicht günstig sind. Die Tyrannen beider Häuser, viele von ihnen mit mächtigen und gut vernetzten Familien, sehen in Miss Granger eine Bedrohung für ihren Ruf und eine Schande für ihren Stolz. So mächtig das Motiv auch ist, ihr zu schaden, es verblasst im Vergleich zum rohen Neid der Gryffindors, die sehen, wie eine Außenseiterin die Lorbeeren des Heldentums erringt, von denen sie seit ihrer Kindheit träumen."

Jetzt war das Lächeln auf Professor Quirrells Lippen eindeutig, wenn auch leicht. "Und dann sind da noch diejenigen aus dem Hause Slytherin, die hören, dass Salazar Slytherins Geist sie verlassen hat, um ein Schlammblut zu begünstigen. Ich frage mich, ob Sie sich überhaupt vorstellen können, Mr. Potter, wie solche wie sie reagieren würden? Die, die es nicht glauben, würden Miss Granger für diese Beleidigung fröhlich umbringen. Und was die Slytherins angeht, die sich tief in ihrem Inneren fragen, ob es vielleicht doch wahr sein könnte … ihre innere Panik ist kaum zu ermessen."

Professor Quirrell nippte gleichmütig an seinem Tee.

"Wenn Sie erfahrener sind, Mr. Potter, werden Sie solche Konsequenzen schon im Vorfeld Ihrer Verschwörungen erkennen. So wie es aussieht, sind Sie mit Ihrer vorsätzlichen Ignoranz gegenüber der menschlichen Natur, die Sie als unangenehm empfinden, schlecht bedient."

Harry nippte an seinem eigenen Tee.

"Ah…", sagte Harry. "Professor Quirrell … Hilfe?"

"Ich habe Miss Granger bereits meine Hilfe angeboten", sagte Professor Quirrell, "als ich ahnte, was sich entwickeln würde. Meine Schülerin hat mir in höflichen Worten gesagt, ich solle mich aus ihren Angelegenheiten heraushalten. Sie würde dir auch nichts anderes sagen, nehme ich an. Da ich in dieser Angelegenheit weder wirklich etwas zu gewinnen noch zu verlieren habe, habe ich kaum die Absicht, den Punkt zu bedrängen."

Der Verteidigungsprofessor zuckte mit den Schultern, seine Teetasse fest im exakt richtigen höflichen Griff, so dass die Oberfläche der Flüssigkeit nicht einmal kräuselte, als Professor Quirrell sich in seinem Stuhl zurücklehnte.

"Mach dir nicht zu viele Sorgen, Mr. Potter. Die Emotionen kochen hoch in der Nähe von Miss Granger, aber sie ist in weniger Gefahr, als du vielleicht denkst. Wenn du älter bist, wirst du lernen, dass das Erste und Wichtigste, was ein normaler Mensch tut, nichts ist."

…

Der Umschlag, den das Slytherin-System Daphne beim Mittagessen überreicht hatte, war wie immer unsigniert; das Pergament darin nannte eine Zeit und einen Ort und sagte schlicht: "Schwer."

Das war es nicht, was Daphne beunruhigt hatte. Was Daphne beunruhigt hatte, war, dass Millicent an diesem Tag beim Mittagessen weder in ihre noch in Traceys Richtung geschaut zu haben schien. Sie hatte nur geradeaus auf ihren Teller gestarrt und gegessen. Millicent hatte nur ein einziges Mal, das Daphne sah, aufgeschaut, in Richtung des Hufflepuff-Tisches, und dann schnell wieder nach unten geschaut; allerdings war Daphne zu weit weg, um den Ausdruck auf Millicents Gesicht zu sehen, da Millicent sich weit weg von ihr und Tracey gesetzt hatte. Daphne hatte während des Mittagessens darüber nachgedacht, mit einem flauen Gefühl im Magen, wie sie es noch nie zuvor verspürt hatte, und das sie dazu veranlasst hatte, nach der Hälfte ihres ersten Tellers mit dem Essen aufzuhören.

\emph{Was ich sehe, muss eintreten … dagegen sieht es wahrscheinlich aus, von Lethifolds gefressen zu werden wie eine Teeparty} … Es war keine bewusste Entscheidung, die Daphne traf, nichts, wie es Slytherins tun sollten, kein Abwägen der Vorteile für sie selbst. Stattdessen - Daphne erzählte Hannah und Susan und allen anderen, dass ihr Informant sie gewarnt hatte, dass der nächste Tyrann es vor allem auf Hufflepuffs abgesehen hatte, und dass der Tyrann vorhatte, den Zorn der Lehrer zu riskieren, um entweder Hannah oder Susan wirklich zu verletzen, so richtig ernsthaft, und dass die beiden sich da raushalten mussten. Hannah hatte zugestimmt, sich rauszuhalten. Susan hatte -

"Was macht Sie hier?" schrie General Granger, obwohl es eine Art Schrei und Flüstern zugleich war.

Susans rundes Gesicht veränderte sich nicht, als hätte das Hufflepuff-Mädchen plötzlich die Art von erfahrener Leere entwickelt, die Daphnes eigene Mutter benutzte.

"Bin ich wirklich hier?" sagte Susan ruhig.

"Du hast gesagt, du würdest nicht kommen!"

"Habe ich das gesagt?", fragte Susan. Sie schwenkte ihren Zauberstab lässig in einer Hand und lehnte sich an die Steinwand des Korridors, in dem sie warteten, wobei sich ihr rotbraunes Haar irgendwie perfekt gegen den gelben Saum ihres Hexengewandes absetzte.

"Ich frage mich, warum. Vielleicht wollte ich nicht, dass Hannah auf seltsame Ideen kommt. Hufflepuff-Loyalität, du weißt schon."

"Wenn du nicht gehst", sagte der Sonnenschein-General, "rufe ich einen Missionsabbruch aus, und wir gehen alle zurück in unsere Lernsäle, Miss Bones!"

"Hey!", sagte Lavender. "Wir haben nicht abgestimmt…"

"Das ist mir recht", sagte Susan, die mit festem Blick auf das andere Ende des Korridors blickte, wo er in den gekachelten Flur überging, in dem sie den Tyrannen erwarten sollten. "Dann bleibe ich eben selbst hier."

"Warum -", sagte Daphne.

Ihr Herz saß ihr im Hals. \emph{Wenn ich versuche, es zu ändern, wenn irgendjemand versucht, es zu ändern, werden wirklich schreckliche, furchtbare, nicht gute, extrem schlechte Dinge passieren. Und dann wird es trotzdem passieren…}

"Warum tust du das?"

"Das passt nicht zu mir", sagte Susan. "Ich weiß. Aber -" Susan zuckte mit den Schultern. "Menschen benehmen sich nicht immer wie sie selbst, weißt du."

Sie flehten. Sie bettelten. Susan sagte gar nichts mehr, sie sah nur noch zu, wartete. Daphne weinte fast, sie fragte sich immer wieder, ob sie das verursacht hatte, ob der Versuch, das Schicksal zu ändern, das Ganze noch schlimmer machte -

"Daphne", sagte Hermine, ihre Stimme klang viel höher als sonst, "geh und hol einen Lehrer. Lauft."

Daphne drehte sich auf den Fersen und begann, in die andere Richtung des steinigen Korridors zu rennen, und dann wurde es ihr klar, und sie drehte sich um, wo alle anderen Mädchen außer Susan ihr beim Gehen zusahen, und Daphne, die sich fühlte, als müsste sie sich gleich übergeben, sagte:

"Ich kann nicht…"

"Was?", sagte Hermine.

"Ich glaube, es wird immer schlimmer, wenn man versucht, dagegen anzukämpfen", sagte Daphne. \emph{So funktionierte es manchmal in Theaterstücken.}

Hermine starrte sie an, und dann sagte Hermine: "Padma."

Das andere Ravenclaw-Mädchen riss sich einfach los, ohne zu widersprechen.

Daphne sah ihr nach, weil sie wusste, dass Padma nicht so gut laufen konnte wie sie, und sich nun fragte, ob das vielleicht der einzige Grund sein würde, warum Hilfe zu spät kommen würde…

"Die Feinde sind hier", sagte Susan lakonisch. "Hm, sie haben eine Geisel."

Sie wirbelten alle herum und sahen - drei ältere Tyrannen, Daphnes Augen erkannten Reese Belka, der ein Top-Leutnant in einer der Armeen der Siebtklässler war, und Randolph Lee, der die Nummer zwei im Hogwarts-Duellierclub war, und am schlimmsten von allen, Robert Jugson III, im sechsten Jahr, dessen Vater mit ziemlicher Sicherheit ein Todesser war.

Alle drei waren von Schildzaubern umgeben, blaue Nebel, die unter der Oberfläche in andersfarbigen Bändern glühten und darüber gelegentlich Facetten zeigten, mehrschichtige Schilde, als hätten die drei gedacht, dass sie gegen ernsthafte Duellanten kämpften und dementsprechend Energie aufgewendet.

Und hinter ihnen, gefesselt und gestützt von glühenden Seilen, war Hannah Abbott.

Ihre Augen waren groß und panisch und ihr Mund bewegte sich, obwohl sie durch den Quietus, den sie vorhin aufgestellt hatten, nichts hören konnten. Dann machte Jugson eine unwillkürliche Geste mit seinem Zauberstab, und die glühenden Seile schleuderten Hannah auf sie zu, es gab einen kleinen Knall, als Hannahs Körper durch die Quieting-Barriere flog,

Susans Zauberstab war sofort auf Hannah gerichtet, und Susans Stimme murmelte "Wingardium Leviosa" -

"Lauft!", schrie Hannah, als sie sanft auf den Boden gesenkt wurde. Doch der Korridor hinter und vor ihnen war nun mit einem leuchtenden grauen Feld versperrt, ein Barrierezauber, den Daphne nicht erkannte.

"Muss ich erklären, worum es hier geht?" sagte Lee mit falscher Jovialität. Der Duellant im siebten Jahr zeigte ein Lächeln, das nicht bis zu seinen Augen reichte. "Nun, nur für den Fall, ihr kleinen Unannehmlichkeiten, und das schließt Sie ein, Miss Greengrass, Sie haben schon genug Ärger gemacht und genug Lügen erzählt. Wir haben Ihre kleine Freundin mitgebracht, nur um sicherzugehen, dass jeder weiß, dass wir Sie alle erwischt haben - obwohl ich vermute, dass sich das andere Ravenclaw-Mädchen irgendwo hinter einer Ecke versteckt oder an der Decke festhält? Na ja, was soll's. Das ist dein -"

"Genug geredet", sagte Robert Jugson III, "Zeit für Schmerzen", und hob seinen Zauberstab. "Cluthe!"

Gleichzeitig zeigte Susan auf ihren Zauberstab und sagte "Prismatis!", und fast augenblicklich bildete sich eine kleine Regenbogenkugel in der Luft, die Miniaturbarriere war so verdichtet und hell, dass sie selbst dann noch intakt blieb, als Jugsons Verhexung sie traf und in Richtung Belka abprallte, deren Zauberstab aufblitzte, um den dunklen Blitz wegzuschlagen; und einen Moment später war das vielfarbige Leuchten verschwunden. Daphnes Augen weiteten sich für einen Moment; sie hatte nie daran gedacht, eine prismatische Sphäre so zu benutzen -

"Jugsy, Schatz?", sagte Belka. Ihre Lippen verbreiterten sich zu einem bösartigen Lächeln. "Ich dachte, wir hätten das besprochen. Erst besiegen wir sie, dann spielen wir."

"B-bitte", sagte Hermine Granger mit stockender Stimme, "lass sie gehen - ich, ich, ich verspreche, ich werde -"

"Oh, wirklich", sagte Lee in einem genervten Ton. "Willst du etwa anbieten, dich selbst auszuliefern, wenn wir die anderen gehen lassen? Jetzt haben wir euch alle."

Da lächelte Jugson. "Das könnte lustig werden", sagte der Todesser im sechsten Jahr leise und drohend. "Wie wäre es, wenn du meine Schuhe leckst, Schlammblut, und einer deiner Freunde darf gehen? Such dir den aus, der dir am besten gefällt, und lass die anderen in Ruhe."

"Nö", sagte die junge Stimme von Susan Bones, "das wird nicht passieren", und mit einer blitzschnellen Bewegung sprang das Hufflepuff-Mädchen nach links, gerade als ein roter Betäuber aus Belkas Zauberstab hervorbrach, Daphne konnte die Bewegung kaum sehen, als Susan gegen die Korridorwand zu prallen schien und dann von ihr abprallte, als wäre sie ein Gummiball und ihre Beine in Jugsons Gesicht krachten, Sie durchschlug den Schild nicht, aber der Sechstklässler taumelte durch den Aufprall nach hinten und Susan folgte ihm nach unten und ihr Fuß stampfte auf den Arm des Zauberstabs des Jungen, der wiederum vom Schild abgewehrt wurde, "Elmekia! " schrie Lee und Parvati rief "Prismatis! " und die Regenbogenwand bildete sich, aber der feurig blaue Strahl ging durch sie hindurch, als wäre sie gar nicht da, der Bolzen verfehlte Susan um Zentimeter, es gab einen Bewegungswirbel, dem Daphne nicht folgen konnte, in dem Belka die Füße unter den Füßen weggeschlagen wurden, aber die ältere Hexe rollte sich gerade wieder auf die Beine und dann - Daphne sah es kommen, und ihre Lippen begannen "Pris-" zu murmeln, aber es war schon zu spät. Sie hatte ihren Zauberstab erhoben, als könnte sie sie abwehren, und es gab einen weißen Blitz, als die Verhexungen auf das magische Holz trafen, aber dann krampften Susans Beine und schickten sie gegen eine Korridorwand.

Ihr Kopf schlug mit einem seltsamen Knacken auf, und dann fiel Susan zu Boden und blieb regungslos mit dem Kopf in einem seltsam anmutenden Winkel liegen, ihren Zauberstab immer noch in einer ausgestreckten Hand.

Einen Moment lang herrschte eisige Stille. Parvati krabbelte zu Susan hinüber, drückte mit dem Daumen auf den Pulspunkt an Susans Handgelenk und dann - dann erhob sich Parvati langsam, zitternd, mit großen Augen -

"Vitalis revelio", sagte Lee gerade, als Parvati den Mund öffnete und Susans Körper von einem warmen, roten Glühen umgeben war. Jetzt grinste der Siebtklässler wirklich. "Wahrscheinlich nur ein gebrochenes Schlüsselbein, würde ich sagen. Aber netter Versuch."

"Merlin, die sind knifflig", sagte Jugson. "Ihr habt mich eine Sekunde lang getäuscht, meine Lieben."

Die Siebtklässlerin lächelte überhaupt nicht.

"Tonare!", schrie Daphne, hob ihren Zauberstab über den Kopf und konzentrierte sich stärker als je zuvor in ihrem Leben. "Rava calvaria! Lucis -"

Sie sah nicht einmal den Fluch, der sie erwischte.

Hermine spürte den Ruck der Innervation, der sie wach machte, und aus einer intuitiven Strategie heraus rollte sie sich nicht sofort auf die Füße; es war ein völlig aussichtsloser Kampf gewesen, und sie wusste nicht, was sie tun konnte, aber irgendein Instinkt sagte ihr, dass auf die Füße zu springen es nicht gut war. Nur einen Spalt öffnete Hermine ihre Augen, und die dünnen Lichtstrahlen, die in sie eindrangen, zeigten Parvati, die von allen drei Tyrannen zurückwich, das letzte Mädchen, das noch stand, das Hermine sehen konnte. Und ihre Augen zeigten auch, dass Tracey nicht weit von ihr entfernt gefallen war, und Hermines Zauberstab war immer noch in ihrer Hand; und so, in der verzweifelten Hoffnung, dass das Slytherin-Mädchen mehr Verstand zeigen würde, als sie es normalerweise tat, machte Hermine die Bewegungen des Zauberstabs so subtil, wie sie konnte, und bewegte kaum ihre Lippen, flüsterte: "Innervate." Hermine spürte, dass der Zauber wirkte, aber Tracey bewegte sich nicht. Hermine hoffte, dass es daran lag, dass Tracey gerissen war und wartete, um… \emph{Was sollten sie tun?} Hermine wusste es nicht, und die Panik, die in den Momenten des Kampfes gewartet hatte, begann sie innerlich aufzufressen, jetzt, wo sie still war, jetzt, wo sie versuchte zu denken, jetzt, wo sie sehen konnte, dass alles absolut hoffnungslos war. In diesem Moment hörte Hermine einen dumpfen Schlag, und obwohl es jetzt außerhalb ihres Blickfeldes war, wusste sie, dass Parvati gefallen war.

Ein Moment der Stille kam und verging.

"Und was jetzt?", fragte die Stimme des unheimlichen Jungen.

"Jetzt wecken wir das Schlammblut auf", sagte die präzise Stimme des schaurig-schönen Jungen, "und finden heraus, wer wirklich dahintersteckt, nicht der Geist von Salazar Slytherin."

"Nein, meine Lieben", sagte die Stimme des schaurig-schönen Mädchens, "zuerst fesseln wir sie alle ganz fest -"

Und dann gab es ein Geräusch wie Blitz und Donner, und Hermines Augen weiteten sich vor Schreck, bevor sie sich aufhalten konnte, und in ihrem erweiterten Blickfeld sah sie, wie der schaurige Junge zuckte, als gelbe Energiebögen wie riesige, flammende Würmer über ihn krochen. Sein Zauberstab flog ihm aus der Hand, als er zuckend zu Boden sackte, und einen Moment später lag er still.

"Schlafen jetzt alle anderen?", sagte eine Stimme. "Gut." Susan Bones erhob sich vom Boden in der Nähe der Stelle, an der der schaurige Junge gestanden hatte, den Hals immer noch seltsam gebeugt. Dann rollte sie ihren Kopf um die Schultern, eine lässige, lockere Bewegung, und ihr Kopf war wieder gerade. Die rundliche Erstklässlerin stand mit einer Hand in die Hüfte gestemmt vor den beiden verbliebenen Tyrannen. Sie grinste. Und umgeben von facettenreichem blauen Dunst.

"Vielsaft!", spuckte das Rüpelmädchen.

"Polyfluis Reverso!", brüllte der verbliebene Rüpel.

Etwas wie die Form eines verspiegelten Schals spuckte aus seinem Zauberstab - Ging widerstandslos durch den Dunst, der Susan umgab - Einen Augenblick lang leuchtete sie in einer seltsamen Spiegelfarbe, wie ein Spiegelbild ihrer selbst - Und dann verblasste das Leuchten. Das junge Mädchen stand immer noch da, die Hand in die Hüfte gestemmt.

"Falsch, Und das ist auch eine Wahrheit", sagte Susan. "Falls es dir noch niemand gesagt hat -" In ihrer kleinen Hand erhob sich ein Zauberstab, verschwommen durch den blauen Dunst, der ihn umgab. "Man legt sich nicht mit den '\emph{Hufflepuffs}' an" und mit einem grauen Blitz, der so hell war, dass er Hermines halb geschlossene Augen schmerzte, begann der eigentliche Kampf.

Er dauerte eine Weile an. Ein Teil der Decke wurde geschmolzen. Die Tyrannin versuchte, einen Waffenstillstand auszurufen, dass sie gehen und Jugson mitnehmen würden, und Susan brüllte die Silben eines Fluchs, den Hermine als Abi-Dalzim's Schreckliches Schmelzen erkannte und der in sieben Ländern verboten war.

Schließlich lag das Slytherin Mädchen bewusstlos und nicht mehr zu wecken auf dem Boden, und der letzte Junge war geflohen und hatte die Körper seiner Gefährten zurückgelassen, und Susan war an eine Wand gelehnt, schweißüberströmt und ihre versengten Roben mit nassen Flecken durchtränkt, schnappte nach Luft und klammerte sich mit der linken Hand an ihre rechte Schulter. Nach einer Weile richtete sich Susan auf und drehte sich um, um nach hinten zu schauen, wo ihre Mithexen auf dem Boden schliefen. Nun, sie hätten auf dem Boden schlafen sollen. Lavender saß bereits mit Augen so groß wie Wassermelonen auf.

"Das…", sagte Lavender.

"War…", sagte Tracey.

"Was?", sagte Hermine.

"Ich meine, was?", sagte Parvati.

"Cool!" sagte Lavender.

"Oh, verdammt", sagte Susan Bones.

Ihr Gesicht hatte unter dem Schweiß schon ein wenig blass ausgesehen, und jetzt wurde es noch blasser und sah fast beängstigend weiß aus.

"Ah … kann ich euch davon überzeugen, dass ihr das alles nur halluziniert habt?"

Es gab einen schnellen Austausch von Blicken. Hermine schaute Parvati an, Parvati schaute Lavender an, Lavender wechselte kurz einen Blick mit Tracey. Die vier sahen wieder zu Susan und schüttelten den Kopf.

"Oh, verdammt", sagte Susan wieder. "Ich bin in ein paar Minuten wieder da, aber ich muss jetzt wirklich los, bitte sagt nichts darüber, tschüss!"

Und Susan rannte überraschend schnell auf den Flur hinaus, bevor irgendjemand ein weiteres Wort sagen konnte.

"Nein, im Ernst, was bei Merlins Unterhose?!", sagte Parvati.

"Innervate", sagte Hermine und richtete ihren Zauberstab auf Daphne, deren Körper sie vorher nicht hatte sehen können; und Lavender richtete ihren Zauberstab auf Hannahs Körper und sagte dasselbe.

Hannahs Augen öffneten sich und sie versuchte verzweifelt, sich auf die Füße zu rollen, brach aber auf halbem Weg zu Boden.

"Es ist okay, Hannah!", sagte Lavender. "Wir haben gewonnen."

"Wir haben was?!", sagte Hannah von ihrem kleinen Häufchen auf dem Boden.

Daphne hatte sich nicht gerührt, aber Hermine konnte sehen, wie sich ihr Brustkorb hob und senkte, und der Atemrhythmus sah normal genug aus. "Ich glaube, es geht ihr gut", sagte Hermine, "aber -" Sie brauchte einen Moment, um zu schlucken, ihr Mund war immer noch trocken. Das war alles sehr, sehr, sehr aus dem Ruder gelaufen. "Ich denke, wir sollten Daphne zu Madam Pomfrey bringen …"

"Sicher, sicher, geb mir nur eine Sekunde hier, dann geht's bestimmt wieder", sagte Parvati.

"Entschuldigt mal", sagte Hannah in einem Ton, der höflich, aber bestimmt war. "Wie haben wir gewonnen? Und warum sieht die Decke so geschmolzen aus?"

Es gab eine Pause.

"Susan hat es geschafft", sagte Tracey.

"Ja", sagte Parvati mit leicht zittriger Stimme, als sie aufstand und begann, ihre rotgeschmückten Roben abzubürsten, "es hat sich herausgestellt, dass Susan Bones die Erbin von Hufflepuff ist und dass sie den lange verschollenen Eingang zu Helga Hufflepuffs Kammer der harten Arbeit und Übung geöffnet hat."

"Huh?", sagte Hannah, die über sich selbst tastete, als ob sie sich vergewissern wollte, dass alle ihre Körperteile noch vorhanden waren. "Ich dachte, das wäre nur etwas, was Professor Sprout sagt, um uns eine wichtige moralische Lektion zu erteilen - Susan ist was?"

Langsam fing Hermine an, sich ein bisschen mehr zusammenzureißen. Es waren nicht wirklich mehr als dreißig Sekunden extremen Schreckens gewesen, zumindest nicht die Teile, bei denen sie bei Bewusstsein gewesen war.

"Eigentlich", sagte Hermine vorsichtig, als ihr Verstand wieder zu arbeiten begann, "bin ich mir ziemlich sicher, dass das nur etwas ist, was Professor Sprout sagt, es war nicht in '\emph{Die Geschichte von Hogwarts'} oder irgendwo anders, wo ich es gelesen habe -"

"Sie ist eine Doppelhexe!?", rief Tracey, ihre Stimme so hoch, dass es knackte. "Das ist sie! Sie ist eine von ihnen! Das ist sie schon die ganze Zeit!"

"Was?", schrie Parvati und drehte sich um, um Tracey anzuschauen. "Das ist das Verrückteste -"

"Natürlich!", sagte Lavender, die nun ganz auf den Beinen war und vor Aufregung auf und ab zu hüpfen begann. "Ich hätte es merken müssen!"

"Susan ist eine was?", sagte Hermine.

"Eine Doppelhexe!", sagte Tracey.

"Siehst du", sagte Lavender und sprach sehr schnell, "es gab schon immer Geschichten, über diese Kinder, die als Superzauberer geboren werden, die Zaubersprüche sprechen können, die sonst niemand kann, und es gibt eine ganze geheime Schule, die in Hogwarts versteckt ist, mit Klassen, die nur sie sehen und besuchen können -"

"Das sind doch nur Geschichten!?", schrie Parvati. "So funktioniert das wirkliche Leben nicht! Ich meine, klar, ich habe diese Bücher auch gelesen -"

"Einen Moment, bitte", sagte Hermine. Vielleicht war ihr Verstand doch ein wenig langsam. "Du meinst, obwohl du schon auf eine magische Schule gehen darfst und alles, willst du immer noch auf eine doppelte magische Schule gehen?"

Lavender sah sie verwirrt an.

"Was?", sagte Lavender. "Wer würde nicht gerne super extra magische Kräfte haben wollen? Es wäre wie dieses ganze erstaunliche Schicksal und alles! Das würde bedeuten, dass du etwas Besonderes bist!"

Hannah nickte dazu und sah von dort auf, wo sie an Daphnes Seite gekrochen war und das Mädchen auf gebrochene Knochen untersuchte.

"Ich wünschte, ich wäre eine Doppelhexe", sagte Hannah, und dann, ein wenig trauriger klingend, "obwohl ich nicht glaube, dass es so etwas gibt, wirklich … was genau hast du Susan tun sehen? Ich meine, bist du sicher, dass du nicht nur Dinge gesehen hast, nachdem du betäubt wurdest?"

Hermine konnte an diesem Punkt wirklich, wirklich keine Worte finden.

"Oh, nein", sagte Tracey. Das Slytherin-Mädchen wirbelte herum und blickte auf den Eingang zum Korridor, ihre Roben flatterten um sie herum. "Oh nein! Wir müssen hier weg! Wir müssen weg, bevor Susan mit jemandem zurückkommt, der uns mit einem Super-Gedächtniszauber belegen kann!"

"Susan würde das nicht tun!", sagte Parvati. "Ich meine, wenn es überhaupt einen -"

"WAS GEHT HIER VOR?", brüllte eine hohe, quietschende Stimme, als Professor Flitwick in den teilweise geschmolzenen Korridor stürmte wie ein kleines, gefährlich komprimiertes Paket reiner akademischer Wut, eine aschfahle Padma keuchend hinter sich her.

….

"Was ist passiert?!" platzte Susan zu dem Mädchen, \emph{das genau wie sie aussah}, abgesehen von den versengten, schweißnassen Roben.

"Oh, gute Frage!", sagte die andere Susan Bones, während sie sich schnell dessen entledigte, was von ihrer geliehenen Kleidung noch übrig war. Einen Moment später begann das Mädchen, sich wieder in ihre gewohnte Form von Nymphadora Tonks zu verwandeln. "Tut mir leid, aber mir ist selbst nichts eingefallen, also hast du etwa drei Minuten Zeit, dir eine Antwort darauf auszudenken -"

…

Wie Daphne Greengrass hinterher säuerlich feststellte, bestand der Fehler in Hermines schlauem Plan, dafür zu sorgen, dass die Hauspunkte gleichmäßig von allen vier Häusern abgezogen wurden, wenn sie erwischt wurden, darin, dass er bei Nachsitzen nicht funktionierte.

Sie hatten sich alle darauf geeinigt, den Mund über Susans mysteriöse Kräfte zu halten - sogar Tracey, nachdem Susan gedroht hatte, sie mit einem Super-Gedächtniszauber zu belegen, wenn sie es nicht versprechen würde. Leider entdeckten sie beim Abendessen, dass jemand vergessen hatte, den Tyrannen von ihrer Abmachung zu erzählen, und auch, dass Susan Bones ihre Seele schrecklichen, verbotenen Kräften geopfert hatte, die nun den Rumpf ihres Körpers bewohnten, und dass sie deshalb alle nachsitzen mussten.

"Hermine?" sagte Harry Potter neben ihr am Esstisch, seine Stimme sehr zaghaft. "Nimm es mir bitte nicht übel, und ich verstehe, wenn du sagst, dass es mich nichts angeht, aber ich glaube, das Ganze gerät langsam außer Kontrolle."

Hermine fuhr fort, das Stück Schokoladenkuchen auf ihrem Teller zu einem nahtlosen Brei aus Kuchen und Glasur zu zerdrücken.

"Ja", sagte Hermine, ihre Stimme war vielleicht ein wenig säuerlich, "das habe ich zu Professor Flitwick gesagt, als ich mich bei ihm entschuldigt habe, dass ich weiß, dass die Dinge außer Kontrolle geraten sind, und er hat geschrien: \emph{Wirklich, Miss Granger? Meinen Sie?}", und zwar so laut, dass meine Ohren Feuer fingen. Ich meine, meine Ohren haben Feuer gefangen. Professor Flitwick musste sie wieder auslöschen."

Harry hatte sich die Hand an die Stirn gelegt.

"Entschuldige", sagte Harry. Sein Gesicht war vollkommen gerade. "Manchmal habe ich immer noch Schwierigkeiten, mich an so etwas zu gewöhnen. Hey, Hermine, erinnerst du dich noch daran, als wir jung und naiv waren und die Welt noch für einen relativ verständlichen Ort hielten?"

Hermine legte ihre Gabel ab und sah ihn einen Moment lang an.

"Wünschst du dir manchmal, ein Muggel zu sein, Harry?"

"Huh?", sagte Harry. "Nun, natürlich nicht! Ich meine, selbst wenn ich ein Muggel wäre, hätte ich wahrscheinlich irgendwann versucht, die Welt zu beherr…-",

Hermine warf ihm einen Blick zu und der Junge verschluckte hastig das Wort und sagte: "Ich meine natürlich optimieren, du weißt, dass ich das wirklich meine, Hermine! Mein Punkt ist, es ist nicht so, dass sich meine Ziele auf die eine oder andere Weise ändern würden. Aber mit Magie wird es viel einfacher sein, Dinge zu erledigen, als wenn ich nur mit den Fähigkeiten eines Muggels arbeiten müsste. Wenn man es logisch betrachtet, ist das der Grund, warum ich nach Hogwarts gehe, anstatt das alles einfach zu ignorieren und eine Karriere in der Nanotechnologie anzustreben."

Hermine hatte ihre Schokoladenkuchensoße fertiggestellt und begann, ihre Möhren darin einzutauchen und zu essen.

"Warum fragst du?", sagte Harry. "Wünschst du dir, wieder in der Muggelwelt zu sein?"

"Nicht wirklich", sagte Hermine, während sie sowohl in die Karotte als auch in die Schokolade knabberte. "Ich hatte nur, na ja, ein komisches Gefühl, weil ich eine Hexe sein wollte… Wolltest du auch ein Zauberer sein, als du klein warst?"

"Natürlich", sagte Harry prompt. "Ich wollte auch übersinnliche Kräfte und Superkräfte und adamantiumverstärkte Knochen und mein eigenes fliegendes Schloss, und manchmal war ich traurig, dass ich mich vielleicht damit begnügen musste, ein berühmter Wissenschaftler und Astronaut zu sein."

Hermine nickte. "Weißt du", sagte sie leise, "ich glaube, die Hexen und Zauberer, die hier aufwachsen, wissen die Magie nicht richtig zu schätzen …"

"Nun, natürlich tun sie das nicht", sagte Harry, "das ist es, was uns einen Vorteil verschafft. Ist das nicht offensichtlich? Ich meine ernsthaft, das war mir innerhalb von fünf Minuten, als ich in die Winkelgasse kam klar, es ist verdammt offensichtlich."

Auf dem Gesicht des Jungen lag ein verwirrter Ausdruck, als könne er nicht verstehen, warum sie auf etwas so Alltägliches aufmerksam wurde.

