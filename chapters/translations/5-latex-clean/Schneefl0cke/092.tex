

\hypertarget{rollen-teil-4}{% \section{93. Rollen, Teil 4}\label{rollen-teil-4}}

\textbf{\uline{Rollen, Teil 4}}

Harry war in die Große Halle gegangen, hatte sich nur einmal umgesehen, genug Kalorien zu sich genommen, um sich zu versorgen, war hinausgegangen, hatte seinen Umhang wieder angelegt und eine kleine zufällige Ecke gefunden, in der er essen konnte. Als er die Schüler an ihren Tischen sah -

\emph{Abscheu zu empfinden, wenn man andere Menschen ansieht, ist kein gutes Zeichen,} sagte Hufflepuff. \emph{Es ist nicht vernünftig, ihnen etwas vorzuwerfen, weil sie nicht die Gelegenheit hatten, das zu lernen, was du gelernt hast.}

\emph{Untätigkeit in Notfällen hat nichts damit zu tun, dass Menschen egoistisch sind.}

\emph{Das war Normalitätsvoreingenommenheit, wie bei diesem Flugzeugabsturz in Teneriffa, wo ein paar Leute rausgerannt sind und entkommen sind, aber die meisten Leute saßen einfach auf ihren Sitzen und haben sich nicht bewegt, während ihr Flugzeug buchstäblich in Flammen stand.}

\emph{Schauen dir an, wie lange es dauerte, bis Sie sich wirklich bewegten.}

\emph{Es hat keinen Sinn, zu hassen,} sagte Gryffindor. \emph{Es wird nur deinen Altruismus beschädigen.}

\emph{Versuch eine Trainingsmethode zu finden, mit der du es das nächste Mal verhindern kannst,} sagte Ravenclaw.

\emph{Ich trage die experimentelle Vorhersage ein}, sagte Slytherin, \emph{dass wir immer genau das gleiche werden was unter der Hypothese vorhergesagt wird, und zwar dass Menschen nicht gerettet werden können, nicht gelehrt werden können und uns nie bei etwas Wichtigem helfen werden. Außerdem brauchen wir einen Weg, um zu notieren, wie oft ich Recht habe und hatte.}

Harry ignorierte die Stimmen in seinem Kopf und aß einfach Toastscheiben so schnell er konnte. Das war zwar keine richtige Ernährung im Allgemeinen, aber einmalige Ausnahmen würden nicht schaden, solange er den Vitaminbedarf am nächsten Tag nachholte.

Mitten im Bissen flog aus dem Nichts die silbern leuchtende Silhouette eines Phönix heran und sagte mit der Stimme eines müden alten Mannes: "Bitte nimm deinen Umhang ab, Harry, ich habe dir einen Brief zu überbringen."

Harry hustete kurz, verschluckte sich an dem Toast, der in den falschen Hals gegangen war, stand auf, nahm den Unsichtbarkeitsumhang ab, sagte laut: "Sag Dumbledore, dass es mir gut geht", setzte sich dann hin und aß weiter seinen Toast.

Der Toast war schon vertilgt, als Albus Dumbledore zu Harrys Ecke kam und ein gefaltetes Blatt Papier in der Hand hielt; echtes Papier, mit Linien, kein Zaubererpergament.

"Ist das -" sagte Harry.

"Von deinem Vater und von deiner Mutter", sagte der alte Zauberer. Wortlos überreichte Dumbledore die gefalteten Blätter, und wortlos nahm Harry sie entgegen. Der alte Zauberer zögerte, dann sagte er leise: "Der Verteidigungsprofessor hat mir gesagt, ich solle mich zurückhalten, und ich habe selbst dasselbe gedacht, als ich Zeit zum Nachdenken hatte. Ich habe immer zu lange gebraucht, um die Tugenden des Schweigens zu lernen. Aber wenn ich mich täusche, brauchst du nur ein Wort zu sagen -"

"Du irrst dich nicht", sagte Harry.

Er blickte auf die gefalteten, linierten Papiere hinunter und spürte das Übelkeitsgefühl in seinem Bauch, wie sein Körper eine starke pessimistische Vorhersage anzeigte. Seine Eltern würden ihn nicht wirklich verleugnen, und es gab nicht viel, was sie ihm antun konnten (ein Teil von ihm hatte immer noch auf eine sehr viszerale Weise Angst davor, dass ihm die Fernsehprivilegien weggenommen würden, egal wie wenig Sinn das jetzt machte). Aber er war aus der Rolle herausgetreten, die Eltern von Kindern erwarteten, die nach ihrer inneren Überzeugung auf der unteren Stufe der Hackordnung standen. Es wäre dumm, etwas anderes zu erwarten als völlige Entrüstung, völlige rechtschaffene Wut, wenn man sich jemandem gegenüber so verhielt, der dachte, er sei dominant über einen.

"Nachdem du es gelesen hast", sagte der Schulleiter, "glaube ich, dass du sofort in die Große Halle kommen solltest, Harry. Es gibt eine Ankündigung, die du hören möchtest."

"Ich interessiere mich nicht für Beerdigungen -"

"Nein. Das nicht. Bitte, Harry, komm, sobald du mit dem Lesen fertig bist, und zwar ohne deinen Umhang. Wirst du?"

"Ja."

Der alte Zauberer ging.

Harry musste sich zwingen, den Brief zu öffnen. Das Wichtigste war, seine verletzlichen Freunde und Verwandten aus der Gefahrenzone herauszuhalten, es mochte ein Klischee sein, aber soweit Harry sagen konnte, war die Logik gültig. Beschädigte Beziehungen konnten später wieder repariert werden.

Im ersten Brief stand in einer Handschrift, die Harry nur mit Mühe lesen konnte: \emph{Sohn, egal was du in Büchern gelesen hast, uns aus der Gefahrenzone zu halten ist nicht so wichtig wie Erwachsene zu haben, die dir helfen können, wenn du in Schwierigkeiten bist. Du hast entschieden, ohne uns ein Wort zu sagen, dass wir dich wegen deiner "dunklen Seite" im Stich lassen würden. Der Geist von Shakespeare weiß, dass ich in diesem letzten Jahr Dinge gesehen habe, von denen ich mit meiner Schulweisheit nicht zu träumen gewagt hätte - manchmal frage ich mich, ob deine Mum sich nur über mich lustig macht und die Behörden dich weggenommen haben, als ich anfing zu glauben, du würdest Magie benutzen - also kann ich nicht leugnen, dass es möglich ist, dass du es geschafft hast, etwas zu entwickeln… Ich bin mir nicht sicher, wie ich es nennen soll, aber "dunkle Seite" scheint verfrüht, wenn wir nicht wissen, was los ist. Bist du sicher, dass es kein aufkeimendes telepathisches Talent ist und du nur die Gedanken der anderen Zauberer um dich herum aufnimmst? Ihre Gedanken könnten einem Kind, das in einer gesünderen Zivilisation aufgewachsen ist, böse erscheinen. Das sind unbegründete Spekulationen, das gebe ich zu, aber du solltest auch keine voreiligen Schlüsse ziehen.}

\emph{Die zwei wichtigsten Dinge, die ich dir zu sagen habe, sind folgende. Erstens, mein Sohn, habe ich volles Vertrauen in deine Fähigkeit, auf der hellen Seite der Macht zu bleiben, solange du dich dafür entscheidest, und ich habe volles Vertrauen, dass du dich dafür entscheiden wirst. Wenn dir ein böser Geist furchtbare Vorschläge ins Ohr flüstert, ignoriere sie einfach. Es ist mir ein Bedürfnis zu betonen, dass du besondere Vorsicht walten lassen solltest, um diesen bösen Geist zu ignorieren, auch wenn er dir scheinbar wunderbare kreative Ideen vorschlägt, und ich hoffe, ich muss dich nicht an den Vorfall mit dem Wissenschaftsprojekt erinnern, der, das gebe ich zu, viel mehr Sinn machen würde, wenn du mit dämonischer Besessenheit zu kämpfen hattest. Das Zweite, was ich dir sagen möchte, ist, dass du nicht befürchten musst, dass Mum oder ich dich wegen deiner "dunklen Seite" im Stich lassen werden. Wir haben vielleicht nicht erwartet, dass du magische Kräfte bekommst oder eine Affinität für schwarze Magie entwickelst, aber wir haben erwartet, dass du ein Teenager wirst. Was, wenn man es aus der Perspektive deines armen Vaters betrachtet, schon eine hinreichend beunruhigende Aussicht für ein Kind ist, das im Alter von neun Jahren an der Beschwörung von insgesamt fünf Feuerwehrautos beteiligt war.}

\emph{Kinder werden erwachsen.}

\emph{Ich werde dich nicht anlügen und sagen, dass du dich uns mit 20 genauso nah fühlen wirst wie jetzt. Aber deine Mutter und ich werden uns dir genauso nahe fühlen, wenn wir alt und grau sind und die Roboter im Pflegeheim nerven. Kinder werden immer erwachsen und entfernen sich von ihren Eltern, und die Eltern folgen ihnen immer von hinten und geben hilfreiche Ratschläge. Kinder werden erwachsen, und ihre Persönlichkeiten ändern sich, und sie tun Dinge, von denen ihre Eltern wünschen, dass sie sie nicht tun würden, und sie verhalten sich respektlos gegenüber ihren Eltern und lassen sie aus ihren magischen Schulen hinauswerfen, und die Eltern lieben sie trotzdem weiter. Das ist der Weg der Natur. Für den Fall, dass du noch nicht in die Pubertät gekommen bist und deine Teenagerjahre verhältnismäßig schlimmer sind als das was jetzt passiert, behalten wir uns das Recht vor, dieses Gefühl zu überdenken. Egal, was passiert, denk daran, dass wir dich lieben und immer lieben werden, egal was passiert. Ich weiß nicht, ob unsere Liebe nach euren Regeln eine magische Kraft hat, aber wenn ja, zöger nicht, sie in Anspruch zu nehmen.}

\emph{Mit all dem gesagt… Harry, was du da getan hast, ist nicht akzeptabel. Ich denke, das weißt du. Und ich weiß auch, dass es nicht der richtige Zeitpunkt ist, dich darüber zu belehren. Aber du musst uns schreiben und uns sagen, was los ist. Ich kann sehr gut verstehen, warum du uns sofort aus deiner Schule entfernen lassen willst, und ich weiß, dass wir dich zu nichts zwingen können, aber bitte, Harry, sei vernünftig und begreife, wie entsetzt wir sein müssen. Ich würde dir gerne sagen, dass es dir absolut verboten ist, mit irgendwelcher Magie herumzuspielen, die die Erwachsenen um dich herum als das geringste bisschen unsicher ansehen, aber soweit ich weiß, geben die Lehrer an deiner Schule jeden Montag Unterricht in fortgeschrittener Nekromantie. Bitte, bitte sei so vorsichtig, wie es deine Situation erlaubt, was auch immer deine Situation sein mag. Trotz deiner sehr eiligen Zusammenfassung haben wir nicht die geringste Ahnung, was vor sich geht, und ich hoffe, dass du uns so viel wie möglich schreiben wirst. Es ist klar, dass du zumindest in gewisser Weise erwachsen wirst, und ich werde versuchen, mich nicht wie der Kinderbuch-Elternteil zu verhalten, der alles nur noch schlimmer macht - obwohl ich hoffe, dass du verstehst, wie schwer das ist - und deine Mum hat mir eine Reihe von beängstigenden Dingen darüber gesagt, dass die Zauberei geheim bleibt und dass ich dich in Schwierigkeiten bringen könnte, wenn ich Wellen schlage.}

\emph{Ich kann dir nicht sagen, dass du alles vermeiden sollst, was unsicher ist, denn deine Schule ist eine Irrenanstalt und dein Schulleiter wird dich nicht gehen lassen. Ich kann dir nicht sagen, dass du keine Verantwortung für alles übernehmen sollst, was um dich herum passiert, denn soweit ich weiß, gibt es andere Kinder, die in Schwierigkeiten}

\emph{stecken. Aber denk daran, dass es nicht deine moralische Verantwortung ist, irgendwelche Erwachsenen zu beschützen, ihre Aufgabe ist es, dich zu beschützen, und jeder gute Erwachsene würde dem zustimmen. Bitte schreib und erzähl uns mehr, sobald du kannst. Wir beide sind verzweifelt und wollen helfen. Wenn es irgendetwas gibt, das wir tun können, lass es uns bitte sofort wissen. Es gibt nichts, was uns passieren kann, was schlimmer wäre, als zu erfahren, dass Dir etwas zugestoßen ist.}

\emph{In Liebe, Dad.}

Auf der letzten Seite stand nur:

\emph{Du hast mir versprochen, dass du dich nicht durch Magie von mir trennen lässt. Ich habe dich nicht zu einem Jungen erzogen, der ein Versprechen an seine Mum bricht. Du musst sicher zurückkommen, weil du es versprochen hast.}

\emph{In Liebe, Mum.}

Langsam ließ Harry die Briefe sinken und begann, auf die Große Halle zuzugehen. Seine Hände zitterten, sein ganzer Körper bebte, und es schien ihn sehr viel Mühe zu kosten, nicht zu weinen; was er wortlos wusste, dass er es nicht tun durfte. Er hatte den ganzen Tag über nicht geweint. Und er wollte nicht weinen. Weinen war dasselbe wie das Eingestehen einer Niederlage. \emph{Und das hier war noch nicht vorbei}. Also würde er nicht weinen.

Das Essen, das an diesem Abend in der Großen Halle serviert wurde, war einfach: Toast und Butter und Marmelade, Wasser und Orangensaft, Haferflocken und andere einfache Kost, ohne Nachtisch. Einige Schüler hatten einfache schwarze Roben ohne ihre Hausfarben angezogen. Es hätte Anlass zum Streit geben müssen, aber stattdessen herrschte eine Stille, das Geräusch von Menschen, die essen, ohne zu reden. Es brauchte zwei Seiten, um eine Debatte zu führen, und eine der Seiten war an diesem Abend nicht besonders am Debattieren interessiert.

Die stellvertretende Schulleiterin Minerva McGonagall saß am Lehrertisch und aß nicht. Sie hätte es tun sollen. Vielleicht würde sie es in Kürze tun. Aber sie konnte sich nicht dazu zwingen, es jetzt zu tun.

\emph{Für eine Gryffindor gab es nur einen Weg.}

Minerva hatte nur kurze Zeit gebraucht, um sich daran zu erinnern, als ihr nach dem Drängen des Verteidigungsprofessors nichts mehr einfiel, was sie an schlauen Plänen ausprobieren konnte. Das war nicht die Art eines Gryffindors; oder vielleicht sollte sie nur sagen, dass es nicht ihre Art war, denn Albus schien sich tatsächlich im Ränkespiel zu versuchen…. und doch, wenn sie an seine Geschichte zurückdachte, gab es keine Intrigen im Moment der Krise, keine Cleverness und Spiele in letzter Not. Für Albus Dumbledore, wie für sie, war die Regel in extremis, zu entscheiden, was das Richtige war, und es zu tun, egal was es einen selbst kostete. Selbst wenn es bedeutete, seine Grenzen zu überschreiten, seine Rolle zu ändern oder sein Bild von sich selbst loszulassen. Das war der letzte Ausweg von Gryffindor.

Durch einen Seiteneingang der Großen Halle sah sie Harry Potter leise hineinschlüpfen.

\emph{Es war so weit.}

Professor Minerva McGonagall erhob sich von ihrem Stuhl, richtete die abgenutzte Spitze ihres Hutes und ging langsam zum Rednerpult vor dem Haupttisch. Die Geräusche in der Großen Halle, die bereits gedämpft waren, verstummten völlig, als sich alle Schüler zu ihr umdrehten.

"Ihr habt es inzwischen alle gehört", sagte sie, wobei ihre Stimme nicht ganz fest war. \emph{Dass Hermine Granger tot ist.} Sie sprach diese Worte nicht laut aus, denn sie hatten es alle gehört.

"Irgendwie ist ein Troll in das Schloss Hogwarts eingedrungen, ohne dass unsere alten Zauber Alarm geschlagen haben. Irgendwie ist es diesem Troll gelungen, eine Schülerin zu verletzen, ohne Alarm auszulösen, bis hin zu ihrem Tod. Ermittlungen sind im Gange, um herauszufinden, wie es dazu kommen konnte. Der Oberste Rat tagt, um zu entscheiden, wie Hogwarts reagieren wird. Zu gegebener Zeit wird der Gerechtigkeit Genüge getan werden. In der Zwischenzeit gibt es eine andere Angelegenheit der Gerechtigkeit, die sofort erledigt werden muss. George Weasley, Fred Weasley, kommt bitte nach vorne und tretet vor uns alle."

Die Weasley-Zwillinge tauschten einen Blick aus, wo sie am Gryffindor-Tisch saßen, dann standen sie auf und gingen auf sie zu, langsam, zögernd; und Minerva erkannte in diesem Moment, dass die Weasley-Zwillinge dachten, dass sie von der Schule verwiesen werden sollten. Sie dachten ernsthaft, dass sie sie ausschließen würde. Das war es, was das Bild von Professor McGonagall, das in ihrem Kopf lebte, angerichtet hatte.

Die Weasley-Zwillinge gingen zum Rednerpult hinüber und sahen mit ängstlichen, aber entschlossenen Gesichtern zu ihr auf; und sie spürte, wie etwas in ihrem Herzen noch ein wenig weiter brach.

"Ich werde euch nicht von der Schule verweisen", sagte sie und wurde durch den überraschten Blick auf ihren Gesichtern weiter betrübt.

"Fred Weasley, George Weasley, dreht euch um und wendet euch euren Mitschülern zu, damit sie euch sehen können."

Immer noch überrascht blickend, taten die Weasley-Zwillinge dies.

Sie nahm all den Stahl in ihrem Herzen zusammen und sagte, was richtig war.

"Ich schäme mich", sagte Minerva McGonagall, "für die Ereignisse des heutigen Tages. Ich schäme mich, dass es nur zwei von euch gab. Ich schäme mich für das, was ich Gryffindor angetan habe. Von allen Häusern hätte es Gryffindor sein müssen, um zu helfen, als Hermine Granger in Not war, als Harry Potter die Mutigen zur Hilfe rief. Es stimmt, ein Siebtklässler hätte einen Bergtroll zurückhalten können, während er nach Miss~Granger suchte. Und ihr hättet glauben sollen, dass die Leiterin des Hauses Gryffindor", ihre Stimme brach, "an euch geglaubt hätte. Wenn ihr nicht gehorcht habt um das Richtige zu tun, bei Ereignissen, die sie nicht vorhergesehen hat. Und der Grund, warum ihr das nicht geglaubt habt ist, dass ich es euch nie gezeigt habe. Ich habe nicht an euch geglaubt. Ich habe nicht an die Tugenden von Gryffindor selbst geglaubt. Ich habe versucht, euren Trotz auszumerzen, statt euren Mut zur Weisheit zu erziehen. Was immer der Sprechende Hut in mir gesehen hat, das ihn veranlasst hat, mich in Gryffindor zu platzieren, ich habe es verraten. Ich habe dem Schulleiter meinen Rücktritt als stellvertretende Schulleiterin und als Leiterin des Hauses Gryffindor angeboten."

Es gab Schreie des Entsetzens und der Bestürzung, nicht nur vom Gryffindortisch, als Harrys Herz in seiner Brust erstarrte. Harry musste nach vorne laufen, etwas sagen, er hatte das nicht gewollt - Minerva holte noch einmal Luft und fuhr fort.

"Der Schulleiter hat es jedoch abgelehnt, meinen Rücktritt zu akzeptieren", sagte sie. "Also werde ich weiter dienen und versuchen, das, was ich angerichtet habe, rückgängig zu machen. Irgendwie muss ich einen Weg finden, meinen Schülern beizubringen, wie man das Richtige tut. Nicht das, was sicher ist, nicht das, was einfach ist, nicht das, was man uns sagt. Wenn ich euch nur beibringen kann, eure Aufsätze pünktlich abzugeben, gibt es vielleicht kein Haus Gryffindor. Dieser Weg wird für mich schwieriger sein, und vielleicht für uns alle. Aber ich weiß jetzt, dass ich vorher nur den leichten Weg gegangen bin."

Sie trat vom Rednerpult herunter und ging dorthin, wo die Weasley-Zwillinge standen.

"Fred Weasley, George Weasley", sagte sie. "Ihr beide habt nicht immer das Richtige getan. Der Weg der Weisheit liegt nicht in der schamlosen und unnötigen Missachtung von Autoritäten. Und doch habt ihr heute bewiesen, dass ihr das letzte aus unserem Haus seid, was meine Fehler überlebt hat. Weil es das Richtige war, habt ihr einem drohenden Schulverweis getrotzt und euer Leben riskiert, um euch einem Bergtroll zu stellen. Für euren erstaunlichen Mut, der eurem Haus zur Ehre gereicht, verleihe ich jedem von euch zweihundert Punkte für Gryffindor."

Wieder der schockierte Blick auf ihren Gesichtern, wieder der Schmerz wie ein Messer durch ihr Herz. Sie drehte sich zu den anderen Schülern um.

"Ich werde keine Punkte für Ravenclaw vergeben", sagte sie. "Ich vermute, dass Mr~Potter sie nicht haben möchte. Wenn ich mich irre, kann er mich korrigieren und so viele Hauspunkte nehmen, wie er will. Aber was auch immer es wert ist, Mr~Potter, ich bin", ihre Stimme stockte, "es tut mir leid -"

"Stopp!" schrie Harry auf, und dann noch einmal: "Stopp!"

Das Wort blieb ihm in der Kehle stecken.

"Das müssen Sie nicht, Professor."

Irgendetwas in ihm drehte sich, drohte ihn aufzuspalten, wie die Hände eines Riesen, die an ihm zerrten, um ihn in Stücke zu reißen.

"Und, und Sie sollten Susan Bones nicht vergessen, und Ron Weasley - sie haben auch geholfen, sie sollten auch Hauspunkte bekommen -"

"Miss~Bones und der junge Weasley?", fragte Professor McGonagall.

"Davon hat Rubeus nichts gesagt - was haben sie getan?"

"Miss~Bones hat versucht, Mr~Hagrid zu betäuben, als er versuchte, mich aufzuhalten, und Mr~Weasley hat auf Neville geschossen, als dieser versuchte, mich aufzuhalten. Sie sollten beide Punkte bekommen, und, und Neville auch", Harry hatte vorher nicht daran gedacht, sich vorzustellen, wie Neville sich jetzt fühlen musste, aber in dem Moment, in dem er nachdachte, wusste er es, "weil Neville versucht hat, etwas zu tun, auch wenn es nicht das Richtige war, das Richtige zu tun ist die zweite Lektion, man kann damit anfangen zu üben, nachdem man gelernt hat, überhaupt etwas zu tun -"

"Zehn Punkte für Hufflepuff, Miss~Bones", sagte Professor McGonagall, wobei ihre Stimme in der Mitte brach. "Zehn Punkte für Gryffindor, Ron Weasley, deine Familie hat sich an diesem Tag überaus verdient gemacht. Und zehn Punkte nach Hufflepuff für Neville Longbottom, weil er Mr~Potter die Stirn geboten und getan hat, was er für richtig hielt -"

"Nein!", schrie eine junge Stimme vom Hufflepuff-Tisch, gefolgt von einem würgenden Laut. Harry schaute dorthin, schaute dann schnell wieder zu Professor McGonagall und sagte, so ruhig er konnte:

"Neville hat recht, eigentlich kann man nicht buchstäblich null Punkte für den Teil vergeben, in dem man die Aktion richtig macht, das sendet auch die falsche Botschaft, aber er war auf halbem Weg, also könnten es stattdessen fünf Punkte sein."

Professor McGonagall sah einen Moment lang so aus, als wüsste sie nicht, was sie sagen sollte; aber dann ging ihr Blick zu Nevilles Platz am Tisch und sie sagte: "Wie Sie wünschen, Mr~Potter. Was gibt es, Miss~Bones?"

Harry schaute und sah, dass Susan Bones nach vorne getreten war und sich über ihre eigenen Augen wischte, und das Hufflepuff-Mädchen sagte: "Eigentlich - Professor McGonagall - hat General Potter es nicht gesehen - aber Captain Weasley und ich waren nicht die Einzigen, die versucht haben, Mr~Hagrid in die Quere zu kommen, nachdem er hinausgelaufen war. Bevor einige der älteren Schüler uns aufhielten. Aber wir haben es geschafft, Mr~Hagrid eine Minute aufzuhalten, so dass General Potter entkommen konnte."

"Sie müssen ihnen auch Punkte geben", sagte Ron Weasley vom Gryffindor-Tisch. "Sonst nehme ich keine."

"Wer noch?", fragte Professor McGonagall, ihre Stimme war etwas unsicher.

Sieben weitere Kinder standen auf.

\emph{Wie war das noch mal Slytherin? Du sagtest, dass nichts klappen würde? Das Menschen sich nicht ändern werden?} sagte Hufflepuff.

Etwas in Harry brach, so dass er all seine Kraft aufwenden musste, um sich zusammenzureißen.

Als alles gesagt und alles getan war, ging Minerva zu der Stelle, an der Harry Potter stand. Obwohl es nicht ihre größte Fähigkeit war, legte sie einen Schutzzauber um die beiden, um die Sicht zu trüben, und dämpfte die Geräusche mit einem weiteren Gedanken. "Sie, das hätten Sie nicht tun müssen -" sagte Harry Potter. "Sie hätten das nicht sagen sollen -" Er klang, als würde er sich verschlucken. "P-Professor, alles, was ich zu Ihnen gesagt habe, war verletzend und hasserfüllt und falsch -"

"Das wusste ich schon, Harry", sagte sie. "Trotzdem wollte ich es besser machen."

Da war ein Gefühl der Leichtigkeit in ihrer Brust, so wie man es vielleicht nach einem Sprung von einer Klippe erlebt, wenn die Beine den Körper nicht mehr aufrecht halten müssen. Sie war sich nicht sicher, ob sie das schaffen konnte, sie kannte den Weg nicht; und doch schien es zum ersten Mal möglich, dass Hogwarts nicht zu einem traurigen Gespenst seines früheren Selbst werden würde, wenn sie seine Schulleiterin wurde.

Harry starrte sie an, dann gab er ein seltsames Geräusch von sich, das klang, als sei es aus seiner Kehle gepresst worden, und verbarg sein Gesicht in den Händen. Also kniete sie sich hin und umarmte ihn. Es könnte schiefgehen, aber es könnte auch richtig gehen, und sie würde sich von dieser Ungewissheit nicht aufhalten lassen; es war an der Zeit, dass sie den Mut eines Gryffindors lernte, um ihn ihrerseits zu lehren.

"Ich hatte mal eine Schwester", flüsterte sie. Nur das, und sonst nichts.

\emph{Nur um sicher zu gehen,} sagte ein Teil von Harry, während der Rest von ihm schluchzend in Professor McGonagalls Arme sank, d\emph{as bedeutet nicht, dass wir Hermines Tod akzeptiert haben, oder?}

\textbf{\emph{NEIN}}\emph{,} sagte der ganze Rest von ihm, jeder Teil seines Verstandes in einstimmiger Übereinstimmung, Wärme und Kälte und ein verborgener Ort aus Stahl.

\textbf{\emph{Niemals, niemals, für immer.}}

Und ein uralter Zauberer blickte auf sie beide, die Hexe und den weinenden jungen Zauberer. Albus Dumbledore lächelte mit einem seltsam traurigen Ausdruck in den Augen, wie jemand, der einen weiteren Schritt in Richtung eines vorhersehbaren Ziels gemacht hat.

Der Verteidigungsprofessor beobachtete sie beide, die Frau und den weinenden Jungen. Seine Augen waren sehr kalt, und sehr berechnend. Er glaubte nicht, dass dies ausreichen würde.

Erst am nächsten Morgen wurde entdeckt, dass Hermine Grangers Leiche fehlte.

