

\hypertarget{tabubruxfcche-teil-2-der-huxf6rner-effekt}{% \section{80. Tabubrüche, Teil 2, Der Hörner-Effekt}\label{tabubruxfcche-teil-2-der-huxf6rner-effekt}}

\textbf{\uline{Tabubrüche, Teil 2, Der Hörner-Effekt}}

Die älteste Halle des Zaubergamot ist kühl und dunkel, mit konzentrischen Halbkreisen aus Stein, die sich von der untersten Mitte aus erheben, und einfachen Holzbänken, die auf diese erhöhten Halbkreise gesetzt sind. Es gibt keine Lichtquelle, aber die Kammer ist gut beleuchtet, ohne erkennbare Ursache oder Grund; es ist einfach eine brachiale Tatsache, dass die Halle gut beleuchtet ist. Die Wände wie der Boden sind aus Stein, dunklem Stein, einer eleganten und geheimnisvollen Verbindung von Gestein, die sehr schön anzusehen ist, mit einer glatten Textur, die unter ihrer Oberfläche zu fließen und sich zu bewegen scheint. Dies ist die Älteste Halle, der älteste Ort der Zauberei, der bis in die heutige Zeit überdauert hat; jeder andere Ort der Macht wurde in dem einen oder anderen Krieg zerstört. Dies ist die Halle des Zaubergamot, die am ältesten ist, weil die Kriege mit dem Bau dieses Ortes endeten. Dies ist die Halle des Zaubergamot; es gibt ältere Orte, aber sie sind versteckt. Die Legende besagt, dass die Wände aus dunklem Stein von Merlin herbeigezaubert, erschaffen, gewollt wurden, als er die mächtigsten Zauberer der Welt versammelte und sie dazu brachte, ihn als ihr Oberhaupt zu akzeptieren. Und als (so die Legende weiter) die Seher weiterhin voraussagten, dass noch nicht genug getan worden war, um das Ende der Welt und ihrer Magie zu verhindern, dann (so die Geschichte) opferte Merlin sein Leben, seine Zauberkunst und seine Zeit, um das Interdikt von Merlin in Kraft zu setzen. Es war keine Tat, die umsonst war, denn ein Ort wie dieser konnte von keiner Macht, die der Zaubererwelt noch bekannt war, wieder aufgerichtet werden. Und auch nicht zerstört werden, denn diese Mauern aus dunklem Stein würden unversehrt und vielleicht auch ungewärmt durch das Herz einer nuklearen Explosion gehen. Es ist schade, dass niemand mehr weiß, wie man sie herstellt.

Im höchsten der ansteigenden Halbkreise des Zaubergamot, auf der obersten Ebene aus dunklem Stein, befindet sich ein Podium. Auf diesem Podium steht ein alter Mann, mit sorgenvollem Gesicht und einem silbernen Bart, der ihm bis unter die Taille reicht; das ist Albus Percival Wulfric Brian Dumbledore. In der rechten Hand hält er einen Zauberstab, auf seiner Schulter thront ein Feuervogel. Seine linke Hand hält einen kurzen Stab, dünn und gesichtslos und aus demselben dunklen Stein geschmiedet wie die Wände, und das ist die Linie des ungebrochenen Merlin, das Gerät des Obersten Hexenmeisters. Karen Dutton vermachte die Linie Albus Dumbledore am letzten Tag ihres Lebens, nur wenige Stunden, nachdem er halbtot von seinem Sieg über Grindelwald zurückgekehrt war, mit einem hell flammenden Phönix an seiner Seite. Sie wiederum erhielt die Linie von dem Perfektionisten Nicodemus Capernaum, wobei jeder Zauberer sie an seinen auserwählten Nachfolger weitergab, zurück und zurück in ununterbrochener Kette bis zu dem Tag, an dem Merlin sein Leben niederlegte. Das ist (falls Sie sich wundern), wie das Land des magischen Britanniens es geschafft hat, Cornelius Fudge zu seinem Minister zu wählen und dennoch mit Albus Dumbledore als oberstem Zauberer zu enden. Nicht per Gesetz (denn geschriebenes Recht kann umgeschrieben werden), sondern nach uralter Tradition wählt das Zaubergamot nicht aus, wer den Vorsitz über seine Narren führen soll. Seit dem Tag von Merlins Opferung ist es die wichtigste Pflicht eines jeden Obersten Hexenmeisters, höchste Vorsicht bei der Auswahl von Menschen walten zu lassen, die sowohl gut sind als auch in der Lage, gute Nachfolger zu erkennen. Man würde erwarten, dass diese Kette des Lichts irgendwann im Laufe der Jahrhunderte einen Schritt verpasst; dass sie zumindest einmal in die Irre geht und dann nie wieder zurückkehrt. Aber das ist nicht der Fall. Die Linie von Merlin geht weiter, ungebrochen.

Zumindest sagen das die Mitglieder von Dumbledores Fraktion. Lord Malfoy würde Ihnen was anderes erzählen. Und in Asien erzählt man sich ganz andere Geschichten, was die britische Version nicht unbedingt falsch macht.

Auf der untersten Plattform der Alten Halle steht ein hochlehniger, beiniger und bewaffneter Stuhl ohne Kissen, eher aus dunklem Metall als aus dunklem Stein, den Merlin nicht dort platziert hat. Das Gebäude des Ministeriums, das um diesen Ort herum entstanden ist, ist holzgetäfelt und goldenn, hell und feuerbeleuchtet, erfüllt von geschäftigem Treiben. Dieser Ort ist anders. Es ist das steinerne Herz des magischen Britanniens, und es ist weder goldgetüncht noch holzgetäfelt, weder feuerbeleuchtet noch hell. In diesen Raum strömen feierlich Hexen und Zauberer in pflaumenfarbenen Roben, die mit einem silbernen Z bestickt sind. Sie tragen eine ernste Miene, die zeigt, dass sie sich bewusst sind, dass sie furchtbar, furchtbar wichtig sind. Schließlich treffen sie sich in der Allerältesten Halle. Sie sind die Lords und Ladies des Zaubergamot und betrachten sich als das größte Volk des größten magischen Landes der Welt. Kleinere Leute sind vor ihnen auf die Knie gefallen und haben sie angefleht; sie sind mächtig, sie sind wohlhabend, sie sind edel; \emph{sind sie nicht großartig?}

Albus Dumbledore kennt jeden in diesem Raum mit Namen. Er hat viele von ihnen unterrichtet, obwohl zu wenige von ihnen gelernt haben. Manche sind seine Verbündeten, manche seine Gegner, den Rest umwirbt er mit dem vorsichtigen Tanz ihrer Neutralität. Für ihn sind sie alle Menschen. Wenn man den derzeitigen Verteidigungsprofessor von Hogwarts nach seiner Meinung über die Lords und Ladies fragen würde, würde er sagen, dass zwar viele von ihnen ehrgeizig sind, aber nur wenige ein Ziel haben. Er würde anmerken, dass das Zaubergamot genau das ist, wo so jemand landen würde - dass es genau die Art von Gelegenheit ist, die man ergreifen würde, wenn man nichts Besseres zu tun hätte. Solche Leute sind selten interessant, aber sie sind oft nützlich; Figuren, die man manipulieren kann, Punkte, die man erzielen kann, von den wahren Spielern des Spiels.

Nicht inmitten der aufsteigenden Halbkreise, sondern abseits unter einem erhöhten Bogen für die Zuschauer, neben einer Hexe mit spitzem Hut, deren Gesicht von Besorgnis gezeichnet ist, sitzt ein Junge in den formellsten schwarzen Gewändern, die er besitzt. Seine Augen sind grün und abstrakt, und er wirft kaum einen Blick auf die Lords und Ladies, während sie hereinwuseln. Für ihn sind sie nur eine Ansammlung von rauschenden, pflaumenfarbenen Roben, die die Holzbänke schmücken, ein visueller Hintergrund für die Szene in der Ältestenhalle. Wenn es hier einen Feind gibt oder etwas, das es zu manipulieren gilt, dann ist es lediglich "das Zaubergamot".

Die wohlhabenden Eliten des magischen Britanniens haben kollektive Macht, aber keine individuelle Handlungsmacht; ihre Ziele sind zu fremd und trivial, als dass sie eine persönliche Rolle in der Geschichte spielen könnten. Ab jetzt, in der Gegenwart, mag der Junge die pflaumenfarbenen Roben weder, noch mag er sie nicht, weil sein Gehirn ihnen nicht genug Wichtigkeit zugesteht, um Subjekt eines moralischen Urteils zu sein. Er ist ein Spieler, und sie sind eine Tapete.

\emph{Diese Ansicht wird sich bald ändern.}

Harry sah sich in der Halle des Zaubergamot um; sie sah ziemlich alt und historisch aus, und es bestand kein Zweifel daran, dass Hermine ihn stundenlang über diesen Ort hätte belehren können. Die pflaumenfarbenen Roben kamen nicht mehr an, und Harrys Taschenuhr, die jede halbe Stunde um drei Minuten vorrückte, sagte, dass die Verhandlung bald beginnen würde. Professor McGonagall saß neben ihm, und ihre Augen verließen ihn nie länger als zwanzig Sekunden am Stück. Harry hatte an diesem Morgen den Tagespropheten gelesen. Die Schlagzeile hatte gelautet:

\textbf{"VERRÜCKTE MUGGELGEBORENE VERSUCHT}

\textbf{ADELSHAUS AUSZULÖSCHEN"}

und der Rest der Zeitung war dasselbe gewesen. Als Harry neun Jahre alt war, hatte die IRA eine britische Kaserne in die Luft gesprengt, und er hatte im Fernsehen gesehen, wie alle Politiker darum wetteiferten, wer sich am lautesten empören konnte. Und Harry kam der Gedanke - schon damals, als er noch nicht viel über Psychologie wusste -, dass es so aussah, als würde jeder darum wetteifern, wer am wütendsten sein konnte, und niemand hätte sagen dürfen, dass jemand zu wütend war, selbst wenn er gerade die atomare Bombardierung Irlands vorgeschlagen hätte. Schon damals war ihm eine wesentliche Leere in der Empörung der Politiker aufgefallen - obwohl er in diesem Alter nicht die Worte hatte, um sie zu beschreiben - ein Gefühl, dass sie versuchten, billige Punkte zu machen, indem sie das gleiche sichere Ziel wie alle anderen anvisierten. Harry hatte dieses Gefühl der Hohlheit bei politischer Empörung schon immer gehabt, aber es war seltsam, wie viel offensichtlicher es schien, wenn man im Tagespropheten ein Dutzend Artikel las, die auf Hermine Granger einschlugen. Der Leitartikel, geschrieben von einem Namen, den Harry nicht kannte, hatte gefordert, das Mindestalter für Askaban herabzusetzen, nur damit das verdrehte Schlammblut, das die Ehre Schottlands mit ihrem wilden, unprovozierten Angriff auf den letzten Erben eines der ältesten Häuser im heiligen Refugium von Hogwarts beschmutzt hatte, zu den Dementoren geschickt werden konnte, die die einzige Strafe waren, die der Schwere ihres unsäglichen Verbrechens angemessen war. Nur das würde ausreichen, um alle anderen fremden, untermenschlichen Rohlinge abzuschrecken, die in ihrem verdrehten Wahnsinn ebenfalls glaubten, sie könnten sich der majestätischen, unvermeidlichen und gnadenlosen Geißelung durch das Zaubergamot entziehen, die den ehrenhaften Adel von etcetera etcetera etcetera bedrohte. Der nächste Artikel hatte das Gleiche in weniger beredten Worten gesagt.

Vorhin hatte Albus Dumbledore zu ihm gesagt: "Ich werde nicht versuchen, dich von diesem Prozess abzuhalten."\\ Die Stimme des alten Zauberers war ruhig und unnachgiebig.\\ "Ich kann mir gut vorstellen, wie das ablaufen würde. Aber ich möchte, dass du mich im Gegenzug mit der gleichen Höflichkeit behandelst. Die Politik des Zaubergamot ist heikel, und davon weißt du nichts. Wage irgendeine Dummheit und es wird auf Hermine Grangers Kosten sein; und du wirst dich für den Rest deiner Tage an diese Dummheit erinnern, Harry James Potter-Evans-Verres."

"Ich verstehe", sagte Harry. "Ich weiß. Nur - wenn Sie vorhaben, in letzter Minute, wenn alles verloren scheint, ein Kaninchen aus dem Hut zu zaubern und den Tag zu retten, dann sagen Sie es mir bitte jetzt, anstatt mich sitzen zu lassen und mir Sorgen zu machen -"

"Das würde ich dir nicht antun", sagte der alte Zauberer, und eine schreckliche Müdigkeit schien ihn zu durchdringen, als er sich zum Gehen wandte. "Noch weniger Hermine. Aber ich habe keine Kaninchen in meinem Hut, Harry. Wir können nur sehen, was Lucius Malfoy will."

Es gab ein kleines scharfes Klopfen, ein einzelnes kurzes Geräusch, das irgendwie den ganzen Raum zum Schweigen brachte und Harrys Kopf ruckartig nach oben schnellen ließ. Hoch oben hatte Dumbledore gerade mit dem dunklen Stab, den er in der linken Hand hielt, auf sein Podium getippt.

"Die neunzigste Sitzung des zweihundertachtundzwanzigsten Zaubergamot wird auf Antrag von Lord Lucius Malfoy einberufen", sagte der alte Zauberer tonlos.

Sofort erhob sich weit seitlich des Podiums, aber ebenfalls im obersten Kreis, ein hochgewachsener Mann mit einer langen weißen Mähne, die von seinem Kopf über die Schultern seiner pflaumenfarbenen Roben herabfiel.

"Ich präsentiere einen Zeugen zur Befragung unter Veritaserum", sagte Lucius Malfoy, sein kühler Tonfall war im ganzen Raum deutlich zu hören, gleichmäßig kontrolliert mit nur einem leichten Unterton von rechtschaffener Wut. "Lasst Hermine, die erste Granger, vorführen."

"Ich bitte euch alle, daran zu denken, dass sie eine Erstklässlerin von Hogwarts ist", sagte Dumbledore. "Ich werde keinen Missbrauch dieser Zeugin dulden -"

Jemand in den Bänken sagte deutlich hörbar "\emph{Pfah}!" und ein angewidertes Schnauben machte sich breit, sogar ein oder zwei Spötter.

Harry starrte auf die pflaumenfarbenen Roben, seine Augen verengten sich. Und mit der wachsenden Wut kam noch etwas anderes, ein wachsendes Gefühl der Beunruhigung, von etwas schrecklich Verzerrtem, als ob die Realität selbst gestört würde. Harry wusste das, irgendwie, aber er konnte nicht herausfinden, was schief war, oder warum sein Verstand dachte, dass es schlimmer wurde…

"Ordnung!" Dumbledore brüllte. Er klopfte zweimal mit dem Steinstab gegen das Podium und erzeugte zwei weitere kleine Klicks, die alle Geräusche übertönten. "Ich werde hier für Ordnung sorgen!"

Die Tür, durch die die Zeugin hereingebracht wurde, befand sich direkt unter Harrys eigenem Sitz, so dass Harry erst, als die gesamte Gruppe vollständig in die steinerne Halle eingetreten war, sah - \emph{- ein Aurorentrio -} - Hermine stand mit dem Rücken zu Harry, als sie herausgebracht wurde, er konnte ihr Gesicht nicht sehen - \emph{- gefolgt von einem silbern glänzenden Spatz und einem rennenden, mondbeschienenen Eichhörnchen -}

\emph{- und die Quelle des schrecklichen Unrechts, halb verborgen unter einem zerfledderten Mantel.}\\

Harry schoss auf die Füße, bevor er auch nur denken konnte, und nur Professor McGonagalls plötzlicher, verzweifelter Griff nach seinem Handgelenk hielt seine Hand davon ab, nach seinem Zauberstab zu greifen; und die Professorin für Verwandlung flüsterte verzweifelt:\\ "Harry, es ist alles in Ordnung, da ist ein Patronus -"

Es dauerte ein paar Sekunden, bis Harry sich an sich selbst erinnerte. Es dauerte ein paar Sekunden, bis der Teil von ihm, der verstand, dass Hermine nicht direkt einem Dementor ausgesetzt gewesen war, seine anderen Teile zu so etwas wie Vernunft brachte - \emph{Aber Tierpatronus sind nicht perfekt,} sagte eine andere Stimme in seinem Kopf. \emph{Oder Dumbledore würde die Gestalt eines nackten Mannes nicht als schmerzhaft empfinden. Du hast gespürt, wie er sich näherte, tierischer Patronus hin oder her…}

Langsam setzte sich Harry Potter wieder hin, als Professor McGonagall mit ihrem Griff an seinem Handgelenk nach unten zog.

\textbf{\emph{Aber da hatte er dem magischen Land bereits den Krieg erklärt, und der Gedanke, dass andere Leute ihn einen Dunklen Lord nannten, schien so oder so nicht mehr wichtig zu sein.}}

Hermine's Gesicht wurde für ihn sichtbar, als sie sich auf den Stuhl setzte. Sie war nicht aufrecht und trotzig, wie sie es vor Snape gewesen war, sie weinte nicht, wie sie es getan hatte, als die Auroren sie verhafteten. Sie saß einfach da mit einem Blick des leeren Entsetzens, während sich dunkle Metallketten aus dem Stuhl schlängelten und ihre Arme und Beine fesselten. Harry konnte es nicht ertragen. Ohne zu denken, versuchte er, in sich selbst zu fliehen, in seine dunkle Seite zu fliehen, die kalte Wut wie einen Schild über sich zu ziehen. Es dauerte zu lange, er hatte seit Askaban nicht mehr versucht, ganz in seine dunkle Seite zu gehen. Und dann, als sein Blut kalt war, schaute er wieder auf, sah Hermine wieder auf dem Stuhl und stellte fest, dass seine dunkle Seite mit dieser Art von Schmerz nichts anzufangen wusste, sie durchbohrte die Kälte wie ein Messer und tat nicht im Geringsten weniger weh.

"Na, wenn das nicht Harry Potter ist!", kam eine hohe, helle Frauenstimme, kränklich süß und nachsichtig.

Langsam drehte Harry seinen Kopf vom Stuhl weg und sah eine lächelnde Frau, die so viel Make-up trug, dass ihre Haut fast rosa aussah, neben einem Mann sitzen, den Harry von Fotos als Minister Cornelius Fudge erkannte.\\ "Haben Sie etwas zu sagen, Mr. Potter?", erkundigte sich die Frau, so fröhlich, als ob dies keine Verhandlung wäre.

Auch die anderen Leute sahen ihn jetzt an. Harry konnte nicht sprechen, alle Worte in seinem Kopf wären dumm gewesen, wenn er sie laut ausgesprochen hätte. Er fand nichts zu sagen, was Neville nicht auch hätte sagen können. Dumbledore hatte Harry gewarnt, dass der Junge-der-lebte, wenn jemand wollte, dass der Junge-der-lebte sprach, so tun sollte, als wäre er in seinem Alter -\\ "Der Schulleiter hat gesagt, ich sollte nicht reden", sagte der Junge, nicht ganz in der Lage, die Schärfe aus seiner Stimme zu halten.

"Oh, aber du hast unsere Erlaubnis zu reden!", sagte die Frau strahlend. "Ich bin sicher, das Zaubergamot freut sich immer, von dem Jungen-der-lebte zu hören!"

Neben ihr nickte Minister Cornelius Fudge.\\ Das Gesicht der Frau war aufgedunsen und übergewichtig, sichtlich blass unter der Schminke. Fast unweigerlich kam ihr ein bestimmtes Wort in den Sinn, und dieses Wort war \emph{Kröte}.

\emph{Was}, sagte Harrys logischer Teil, i\emph{n keiner Weise mit Moral korrelieren sollte. Nur in Disney-Filmen waren hässliche Menschen eher böse und umgekehrt; und diese Filme wurden wahrscheinlich von Autoren geschrieben, die nie hässlich gewesen waren. Er würde ihr eine Chance geben, jeder in diesem Raum verdiente eine Chance…}

"Weil ich den Dunklen Lord losgeworden bin?", sagte der Junge und zeigte auf den Dementor, der hinter Hermines Stuhl schwebte. "Es gibt etwas in diesem Raum, das dunkler ist."

Das Gesicht der Frau verengte sich und wurde ein wenig ernst.\\ "Mir ist klar, dass ein kleiner Junge wie du dich vielleicht vor ihnen fürchtet, Mr. Potter, aber die Dementoren sind dem Zaubereiministerium gegenüber recht gehorsam. Und sie sind natürlich notwendig, um zu bewachen -"

"Ein zwölfjähriges Mädchen?", rief der Junge. "Das sind die dunkelsten Kreaturen auf der ganzen Welt, ich konnte es hier sogar durch den Patronus spüren - das Grauen, das immer näher kommt - es ist furchtbar böse und es - es würde jeden in diesem Raum fressen, wenn es könnte! Man sollte es nicht in die Nähe eines Kindes lassen, niemals! Nicht zu mir, nicht zu ihr, zu niemandem. Ihr solltet dafür stimmen, es wegzuschicken!"

"Darüber werden wir sicher nicht abstimmen", schnauzte die Krötenfrau.

"Das reicht, Madam Umbridge, Mr. Potter", kam Dumbledores strenge Stimme von hoch oben.\\ Und dann, nach einer kurzen Pause, fuhr der alte Zauberer fort:\\ "Obwohl der Junge natürlich in jeder Hinsicht recht hat."

Einige der Mitglieder des Zaubergamot schauten beschämt auf die Ermahnung des Jungen-der-lebte, und ein paar andere nickten heftig zu den Worten des alten Zauberers. Aber es waren zu wenige. Harry konnte es sehen. Sie waren zu wenige.

Dann wurde das Veritaserum gebracht, und Hermine sah für einen kurzen Moment so aus, als würde sie gleich schluchzen, sie sah Harry an - nein, Professor McGonagall - und Professor McGonagall murmelte Worte, die Harry aus seinem Blickwinkel nicht verstehen konnte. Dann schluckte Hermine drei Tropfen Veritaserum und ihr Gesicht wurde schlaff.

"Gawain Robards", sagte die sanfte Stimme von Lucius Malfoy. "Ihre Redlichkeit ist uns allen bekannt. Wenn Sie uns die Ehre erweisen würden?"

Einer der drei Auroren trat vor.\\ Nach den ersten Fragen sah Harry weg und starrte mit den Fingern in den Ohren zur Seite, während Hermines Gehirn den Inhalt des Falsche-Erinnerung-Zaubers wiedergab. Er konnte die drogengedämpfte Angst in Hermines Stimme nicht ertragen, als sie die falschen Erinnerungen erzählte, und seine dunkle Seite konnte es auch nicht ertragen, und er hatte den Inhalt bereits zusammengefasst gehört. Harrys Gedanken blitzten zurück zu einem weiteren Tag des Schreckens, und obwohl Harry kurz davor gewesen war, Lord Voldemorts Fortbestehen als Senilität eines alten Zauberers abzutun, erschien es ihm plötzlich schrecklich und eindeutig plausibel, dass das Wesen, das Hermine mit einem Gedächtniszauber belegt hatte, derselbe Geist war, der sich Bellatrix Black zunutze gemacht hatte. Die beiden Ereignisse hatten eine gewisse Signatur gemeinsam. Um zu entscheiden, dass dies geschehen sollte, um zu planen, dass dies geschah - \emph{dazu brauchte es mehr als das Böse, es} \emph{brauchte Leere.}

Harry blickte einen Moment auf und sah, dass die pflaumenfarbenen Roben\\ beobachteten, einfach nur beobachteten. Einige Zeit später, nachdem alle Sterne am Nachthimmel kalt und dunkel geworden waren und das letzte Licht im Universum zu Glut verpufft und schwarz geworden war, endete die Befragung von Hermine.

"Wenn es den Anwesenden gefällt", sagte die Stimme von Lord Malfoy, "möchte ich die Aussage meines Sohnes Draco, bezeugt unter zwei Tropfen Veritaserum, jetzt vorlesen lassen."

\emph{Bis zu diesem Kampf hatte ich nichts gegen Granger ausgeheckt.}\\ \emph{Aber nach diesem Tag fühlte ich mich wirklich beleidigt, ich hatte ihr all die Male geholfen…} -

Das Geräusch, das aus Hermines Kehle kam, war, als wäre sie gerade unter einem fallenden Stein zerquetscht worden, so gewaltig, dass sie weder weinen noch atmen konnte, nur ein kleines, trauriges Keuchen.

"Verzeihung", sagte eine Hexe von der, wie es schien, Malfoy zugewandten Seite des Raumes. "Aber Lord Malfoy, warum sollte Ihr Sohn diesem Schlammblutmädchen helfen?"

"Mein Sohn", sagte Lucius Malfoy mit schwerer Stimme, "scheint gewissen fehlgeleiteten Ideen Gehör geschenkt zu haben. Er ist jung - und er hat gelernt, nun, da wir alle als Land gesehen haben, was solche Torheit an Vergeltung bringt."

Ein paar Schritte weiter auf den Besucherbänken kritzelte ein Mann, der eine Zeitungsmütze und ein Abzeichen trug, das ihn als Mitarbeiter des Tagespropheten auswies, eifrig mit einem langen Federkiel. Die wenigen Leute, die Dumbledore vorhin zugenickt hatten, hatten ziemlich kranke Gesichter. Eine Hexe in pflaumenfarbenen Roben stand ganz bewusst von der scheinbar Dumbledores Seite des Raumes auf und machte sich auf den Weg hinüber zur Malfoy-Seite.

Der Auror las weiter, seine Stimme war monoton.\\ \emph{Ich war so müde vom Zaubern all dieser Sperrzauber, dass ich schwach war, als ich den letzten gewirkt habe. Ich dachte, ich wäre stärker als Granger, aber ich war mir nicht sicher, also testete ich es empirisch, indem ich sie zu einem Duell herausforderte,} \emph{deshalb d-d-das und auch, weil ich, wenn ich gewonnen hätte, vorhatte, sie am nächsten Tag noch einmal zu schlagen, wo jeder sie sehen konnte. Blödes Veritaserum. Aber das wusste sie nicht, als sie versuchte, mich zu töten! Und ich war wirklich beleidigt von dem, was sie getan hatte, ich hatte ihr vorher wirklich geholfen und ich hatte damals nichts gegen sie geplant, nur sie ging vor allen Leuten auf mich los!}

Als alle Zeugenaussagen beendet waren, begannen die Beratungen des Zaubergamot.

Wenn man sie so nennen konnte. Es schien, dass viele Mitglieder des Zaubergamot der festen Meinung waren, dass Mord schlecht sei. Die pflaumenfarbenen Roben auf Dumbledores Seite des Raumes schwiegen, die vermeintlichen Mächte des Guten sparten ihr politisches Kapital für gewinnbarere Schlachten.

Und Harry konnte, als stünde Professor Quirrell neben ihm, eine trockene Stimme in seinem Kopf hören, die ihm erklärte, dass es wohl kaum zum eigenen Vorteil der Politiker gewesen wäre, gerade jetzt zu sprechen. Aber es gab einen Zauberer im Raum, dessen Status hoch genug war, dass er, wie es schien, seine Vorsicht vor Gesichtsverlust überwunden hatte; ein Zauberer allein, dessen Status hoch genug war, dass er ein Wort der Vernunft sprechen konnte und ungeschoren davonkam. Er allein sprach, um Hermine zu verteidigen, der Mann mit dem Phönix, der hell auf seiner Schulter flammte. Nur Albus Dumbledore sprach.

Der Oberste Hexenmeister ging nicht auf die Möglichkeit ein, dass Hermine Granger völlig unschuldig war. Das, hatte der Schulleiter Harry erklärt, würde nicht geglaubt werden, würde es nur noch schlimmer machen. Aber Albus Dumbledore sagte in einer sanften Ermahnung nach der anderen, dass die Täterin ein Mädchen im ersten Jahr in Hogwarts sei; dass viele in ihrer Jugend törichte Dinge getan hätten; dass ein Erstklässler in Hogwarts einfach zu jung sei, um die Konsequenzen ihrer Taten zu begreifen. Er selbst (sagte der Oberste Hexenmeister leise) hatte in seiner Kindheit gewisse törichte Dinge versucht, als er viel älter war als sie. Albus Dumbledore sagte, dass Hermine Granger von allen Lehrkräften in Hogwarts geliebt worden war und vier Hufflepuff-Mädchen bei ihren Zauberkunst-Hausaufgaben geholfen und im Laufe des Schuljahres einhundertdrei Punkte für Ravenclaw erzielt hatte. Albus Dumbledore sagte, dass niemand, der Hermine Granger kannte, etwas anderes als schockiert von diesen Ereignissen war. Dass sie alle das Entsetzen in ihrer Stimme gehört hatten, als sie ihre Aussage erzählte. Und wenn ein ungewöhnlicher Wahnsinn vorübergehend von ihr Besitz ergriffen hatte, dann - seine Stimme erhob sich zu einem strengen Befehl - verdiente sie von ihnen nichts außer Mitleid und die Aufmerksamkeit eines Heilers. Und schließlich erinnerte Albus Dumbledore den Zaubergamot unter Protestrufen daran, dass es sich bei der Anklage um versuchten Mord und nicht um Mord handelte. Albus Dumbledore sagte, über einen aufkommenden Sturm von Einwänden hinweg, dass niemandem ein bleibender Schaden entstanden sei. Und Albus Dumbledore flehte sie an, nicht selbst Schlimmeres zu tun als alles, was bisher geschehen war -

"Genug!" brüllte Lucius Malfoy, und ein Handzeichen beendete die Beratungen. Der weißhaarige Mann stand groß und furchterregend da, seinen silbernen Stock hoch in einer Hand haltend wie einen Hammer, der gleich fallen wird. "Für das, was diese verrückte Frau versucht hat, meinem Sohn anzutun - für die Blutschuld, die sie schuldet, weil sie versucht hat, die Linie eines edlen und sehr alten Hauses zu beenden - sage ich, dass sie -"

"Askaban!", brüllte ein Mann mit einem vernarbten Gesicht, der zu Lord Malfoys rechter Hand saß. "Schickt das verrückte Schlammblut nach Askaban!"

"Askaban!", rief ein weiterer pflaumenfarbener Robe, und dann noch einer, und noch einer -

Ein Klicken des Stabes in Dumbledores Hand ließ den Raum verstummen. "Ihr seid von Sinnen!", sagte der alte Zauberer streng. "Und Ihr Vorschlag ist barbarisch, unter der Würde dieser Versammlung. Es gibt Dinge, die wir nicht tun. Lord Malfoy?"

Lucius Malfoy hatte dem mit teilnahmsloser Miene zugehört.\\ "Nun", sagte Lord Malfoy nach ein paar Augenblicken. Ein kalter Schimmer erhellte seine Augen. "Ich hatte nicht vor, euch zu fragen. Aber wenn das der Wille des Zaubergamot ist - dann soll sie so bezahlen, wie jeder an ihrer Stelle bezahlen würde. Es soll Askaban sein."

Ein großer Jubelschrei der Wut erhob sich -

"Seid ihr alle wahnsinnig?!", rief Albus Dumbledore. "Sie ist zu jung! Ihr Geist würde es nicht aushalten! Seit drei Jahrhunderten ist so etwas in England nicht mehr geschehen!"

"Was werden die anderen Länder von uns denken?", sagte die scharfe Stimme einer Frau, die Harry als Nevilles Großmutter erkannte.

"Werden Sie Askaban bewachen, wenn sie dort ist, Lord Malfoy?", sagte eine strenge alte Hexe, die Harry nicht kannte. "Denn ich fürchte, meine Auroren werden sich weigern, es zu bewachen, wenn kleine Kinder darin festgehalten werden."

"Die Beratungen sind beendet", sagte Lucius Malfoy kalt. "Aber wenn Sie nicht in der Lage sind, Auroren zu finden, die dem Votum des Zaubergamot gehorchen, Madam Bones, können Sie den Posten abgeben; wir können leicht einen anderen finden, der an Ihrer Stelle dient. Der Wille dieser Halle ist eindeutig. Für die Ungeheuerlichkeit ihrer Verbrechen soll das Mädchen als Erwachsene vor Gericht gestellt und entsprechend bestraft werden; zehn Jahre in Askaban, das ist die Strafe für versuchten Mord."

Als der alte Zauberer wieder sprach, war seine Stimme leiser.\\ "Gibt es keine Alternative dazu, Lucius? Wir können uns in meine Gemächer zurückziehen, um es zu besprechen, wenn es nötig ist."

Der große Mann mit den langen weißen Haaren drehte sich um und sah zu dem alten Zauberer, der am Podium stand, und die beiden starrten sich einen langen Moment lang an. Als Lucius Malfoy wieder sprach, schien seine Stimme leicht zu zittern, als ob die strenge Kontrolle über sie versagen würde.\\ "Blut verlangt nach Vergeltung, das Blut meiner Familie. Um keinen Preis werde ich die Blutschuld, die ich meinem Sohn schulde, verkaufen. Du würdest das nicht verstehen, der du nie Liebe oder ein eigenes Kind hattest. Und ich glaube, dass mein Sohn, wenn er unter uns wäre, lieber für das Blut seiner Mutter büßen würde als für sein eigenes. Gestehe dein eigenes Verbrechen dem Zaubergamot, so wie du es mir gestanden hast, und ich werde -"

"Denk nicht einmal daran, Albus", sagte die strenge alte Hexe, die zuvor gesprochen hatte.

Der alte Zauberer stand auf dem Podium.

Der alte Zauberer stand am Podium, sein Gesicht verdrehte sich, drehte sich auf -

"Hör auf", sagte die alte Hexe. "Du kennst die Antwort, die du geben musst, Albus. Sie wird sich nicht dadurch ändern, dass du dich darüber quälst."

Der alte Zauberer sprach.\\ "Nein", sagte Albus Dumbledore.

"Und du, Malfoy", fuhr die strenge alte Hexe fort, "ich nehme an, alles, was du die ganze Zeit wolltest, war, seinen Ruf zu ruinieren -"

"Wohl kaum", sagte Lucius Malfoy, seine Lippen verzogen sich nun zu einem bitteren Lächeln.\\ "Nein, ich habe hier nichts anderes im Sinn als die Rache meines Sohnes. Ich wollte dem Zaubergamot nur die Wahrheit hinter dem vorgetäuschten Heldentum dieses alten Mannes und seinem Lob für das Mädchen zeigen - dass er kaum daran denken würde, sich zu opfern, um sie zu retten."

"Eine Grausamkeit, die in der Tat eines Todessers würdig ist", sagte Augusta Longbottom. "Nicht, dass ich damit irgendetwas andeuten will, natürlich."

"Grausamkeit?", sagte Lucius Malfoy, das bittere Lächeln immer noch auf seinem Gesicht. "Ich glaube nicht. Ich wusste, wie seine Antwort lauten würde. Ich habe euch immer gewarnt, dass er nur seine vorgetäuschte Rolle spielt. Wenn ihr an sein Zögern glaubt, seid ihr umso dümmer. Denkt daran, dass seine Antwort die gleiche war."\\ Der Mann erhob seine Stimme.\\ "Lasst uns abstimmen, meine Freunde. Ich denke, ein Handzeichen wird dafür ausreichen. Ich kann mir nicht vorstellen, dass es viele sein werden, die sich mit Mördern verbünden wollen."\\ Die Stimme wurde kalt, beim letzten Ton, das Versprechen darin sehr deutlich.

"Sieh dir das Mädchen an", sagte Albus Dumbledore. "Siehst du das Grauen, das du begehst! Sie ist -" Die Stimme des alten Zauberers brach. "Sie hat Angst -"

Das Veritaserum musste nachgelassen haben, denn Hermine Grangers Gesicht verzog sich unter der Schlaffheit, ihre Glieder zitterten sichtbar unter den Ketten, als wollte sie rennen, von diesem Stuhl weglaufen, wurde aber von Gewichten niedergedrückt, die größer waren als die verzauberten Metallglieder, die sie fesselten. Dann gab es eine krampfhafte Anstrengung und Hermines Hals bewegte sich, ihr Kopf verdrehte sich, genug, um ihre Augen in eine Linie zu bringen - Sie sah Harry Potter an und obwohl sie nicht sprach, war es absolut klar, was sie sagte.

\emph{Harry hilf mir bitte -}

Und in der ältesten Halle des Zaubergamot erklang eine eisige Stimme, eine Sprache von der Farbe flüssigen Stickstoffs, zu hoch gegriffen, als dass sie aus einer zu jungen Kehle käme, und diese Stimme sagte: "Lucius Malfoy."

In den alten und geheiligten Hallen des Zaubergamot sahen sich die Menschen um, und es dauerte zu lange, bis sie fanden, was sie suchten. Sie mochte hoch in der Tonlage sein, sie mochte zu leise für die Worte sein, die gesprochen wurden; und dennoch hätte man nicht erwartet, diese Stimme von einem Kind zu hören. Erst als Lord Malfoy etwas erwiderte, wurde den Leuten klar, wo sie hinschauen sollten.

"Harry Potter", sagte Lucius Malfoy. Er neigte nicht den Kopf.

Die Köpfe drehten sich, die Augen bewegten sich, und die Leute konzentrierten sich auf den unordentlichen Jungen, der neben der weinenden älteren Hexe stand. Der Junge stand nur brusthoch mit seinen Schuhen, gekleidet in kurze, formelle schwarze Roben. Doch wenn man nicht gerade scharfe Augen hatte, hätte man von der anderen Seite der Halle aus nicht die berühmte und tödliche Narbe unter seinem unordentlichen Haar sehen können.

"Diese Torheit steht dir nicht, Lucius", sagte der Junge. "Zwölfjährige Mädchen laufen nicht herum und begehen Morde. Du bist ein Slytherin und ein intelligenter. Du weißt, dass das ein Komplott ist. Hermine Granger wurde mit Gewalt auf dieses Spielbrett gesetzt, von welcher Hand auch immer, die hinter diesem Komplott steckt. Es war sicher beabsichtigt, dass du so handelst, wie du jetzt handelst - nur dass Draco Malfoy tot sein sollte und du jenseits aller Vernunft. Aber er ist am Leben und du bist zurechnungsfähig. Warum kooperierst du mit der dir zugedachten Rolle in einem Komplott, das darauf abzielt, das Leben deines Sohnes zu nehmen?"

In Lucius schien ein Sturm zu toben, das Gesicht unter dem wallenden weißen Haar drohte aufzubrechen und etwas Unberechenbares auszuspucken. Der Lord Malfoy schien fast zu sprechen, einmal und dann noch zweimal, und schluckte drei ungehörte Sätze hinunter, bevor sich seine Lippen für die Wahrheit teilten.

"Eine Verschwörung, sagst du?" sagte Lord Malfoy schließlich. Sein Gesicht zuckte, kaum kontrolliert. "Und wessen Komplott wäre das dann?"

"Wenn ich das wüsste", sagte der Junge, "dann hätte ich es schon viel früher gesagt. Aber jeder, der einmal Hermine Grangers Klassenkameradin war, kann dir sagen, dass sie eine höchst unwahrscheinliche Mörderin ist. Sie hilft den Hufflepuffs tatsächlich bei den Hausaufgaben. Das war kein natürliches Ereignis, Lord Malfoy."

"Komplott - oder kein Komplott -" Lucius' Stimme zitterte. "Dieser Schlammblut-Dreck hat meinen Sohn angefasst und dafür werde ich sie auslöschen. Das solltest du genau wissen, Harry Potter."

"Es ist fraglich", sagte der Junge, "um es milde auszudrücken, ob Hermine Granger tatsächlich diesen Blutkühlungszauber gewirkt hat. Ich kenne die genauen Umstände nicht und weiß auch nicht, welche Zaubersprüche involviert waren, aber eine einfache List hätte nicht ausgereicht, um sie dazu zu bringen, es zu tun. Sie hat nicht aus eigenem Willen gehandelt, vielleicht hat sie gar nicht gehandelt. Dein Rachefeldzug ist fehlgeleitet, Lord Malfoy, und zwar absichtlich. Es ist kein 12-jähriges Mädchen, das deinen Zorn verdient."

"Und was kümmert dich ihr Schicksal?" Die Stimme von Lucius Malfoy erhob sich. "Was hast du mit ihr zu tun?"

"Sie ist meine Freundin", sagte der Junge, "so wie Draco mein Freund ist. Es ist möglich, dass dieser Schlag gegen mich gerichtet war, und gar nicht gegen das Haus Malfoy."

Wieder zuckten die Muskeln in Lucius' Gesicht.\\ "Und jetzt lügst du mich an - so wie du meinen Sohn belogen hast!"

"Ob du es glaubst oder nicht", sagte der Junge leise, "ich habe nie etwas anderes gewollt, als dass Draco die Wahrheit erfährt -"

"Genug!", schrie der Lord Malfoy. "Genug von deinen Lügen! Genug von deinen Spielchen! Du verstehst nicht -- du würdest nie verstehen - \emph{was es bedeutet, dass er mein Sohn ist!} Diese Rache wird mir nicht verwehrt bleiben! Nie wieder! Nie wieder! Für das Blut, das dieses Mädchen dem Hause Malfoy schuldet, wandert sie nach Askaban. Und wenn ich jemals eine andere Hand bei der Arbeit finde - selbst wenn es deine eigene ist - soll auch diese Hand abgeschnitten werden!"\\ Lucius Malfoy hob seinen tödlichen Silberstock wie auf Kommando, die Zähne zusammengebissen und die Lippen zu einem Knurren zusammengezogen, wie ein Wolf, der einem Drachen gegenübersteht.\\ "Und wenn du nichts Besseres zu sagen hast als das - sei still, Harry Potter!"

Harrys Blut hämmerte sogar unter dem Eis seiner dunklen Seite, der Angst um Hermine, dem Teil von ihm, der Lucius angreifen und ihn dort, wo er stand, für seine Unverschämtheit und seine Dummheit vernichten wollte - aber Harry hatte nicht die Macht, er hatte nicht einmal eine einzige Stimme im Zaubergamot - \emph{Draco hatte gesagt, dass Lucius Angst vor ihm hatte, aus irgendeinem unbekannten Grund}. Und Harry konnte es in der Fratze sehen, zu der Lord Malfoys Gesicht geworden war, gezeichnet und angespannt, dass es ihn all seinen Mut kostete, Harry zu sagen, er solle den Mund halten.

Also sagte Harry, seine Stimme kühl und tödlich, und hoffte inständig, dass es etwas bedeutete: "Du wirst dir meine Feindschaft verdienen, wenn du das tust, Lucius…"

Jemand in den unteren Reihen der offensichtlich blutpuristischen Seite des Zaubergamot, der eher auf den Jungen als auf Lord Malfoy hinabschaute, lachte ungläubig. Andere pflaumenfarbene Roben begannen ebenfalls zu lachen.

Lord Malfoy starrte ihn mit harter Würde an, als sich das Lachen ausbreitete.\\ "Wenn du die Feindschaft des Hauses Malfoy haben willst, sollst du sie bekommen, Kind.

"Also wirklich", sagte die Frau mit dem viel zu rosafarbenen Make-up, "ich denke, das ging jetzt lange genug, meinen Sie nicht auch, Lord Malfoy? Der Junge wird seinen Unterricht verpassen."

"In der Tat wird er das", sagte Lucius Malfoy und erhob dann wieder seine Stimme. "Ich rufe zur Abstimmung auf! Durch Handzeichen möge das Zaubergamot die Blutschuld anerkennen, die das edle und uralte Haus Malfoy für die versuchte Ermordung seines letzten Sprosses und die Beendigung seiner Linie durch Hermine, die erste Granger, hat!"

Eine Hand nach der anderen schoss in die Höhe, und der Sekretär, der im unteren Kreis saß, begann, Markierungen auf Pergament zu machen, um sie zu zählen, aber es war offensichtlich, in welche Richtung die Mehrheit gegangen war.

Und Harry schrie in seinem Kopf, ein verzweifelter Hilferuf an jeden Teil von ihm, der einen Ausweg, eine Strategie, eine Idee anbieten würde.

Aber da war nichts, da war nichts, er hatte seine letzten Karten ausgespielt und verloren. Und dann stürzte sich Harry mit einer letzten krampfhaften Verzweiflung in seine dunkle Seite, drängte sich in seine dunkle Seite, griff nach ihrer tödlichen Klarheit, bot seiner dunklen Seite alles an, wenn sie nur dieses Problem für ihn lösen würde; und endlich kam die tödliche Ruhe über ihn, das wahre Eis antwortete endlich auf seinen Ruf. Jenseits aller Panik und Verzweiflung begann sein Verstand, jede Tatsache in seinem Besitz zu durchforsten, sich an alles zu erinnern, was er über Lucius Malfoy, über das Zaubergamot, über die Gesetze des magischen Britanniens wusste; seine Augen blickten auf die Stuhlreihen, auf jede Person und jedes Ding in seiner Sichtweite, auf der Suche nach jeder Möglichkeit, die er ergreifen konnte -

