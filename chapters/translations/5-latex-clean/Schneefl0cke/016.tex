

\hypertarget{privilegierung-der-hypothese}{% \section{17. Privilegierung der Hypothese}\label{privilegierung-der-hypothese}}

\textbf{Privilegierung der Hypothese}\\

\hfill\break \emph{Man beginnt, das Muster zu sehen, den Rhythmus der Welt zu hören}

\textbf{Donnerstag}.\\ Wenn man es genau nehmen wollte, 7:24 Uhr am Donnerstagmorgen. Harry saß auf seinem Bett, ein Lehrbuch lag schlaff in seinen reglosen Händen.\\ Harry hatte gerade eine Idee für einen wirklich brillanten experimentellen Versuch gehabt. Es würde bedeuten, eine Stunde länger auf das Frühstück zu warten, aber dafür hatte er ja Müsliriegel.\\ Nein, diese Idee musste unbedingt sofort getestet werden, sofort, jetzt.\\ Harry legte das Lehrbuch beiseite, sprang aus dem Bett, rannte um sein Bett herum, riss die Treppe seines Koffers heraus, rannte die Treppe hinunter und begann, Kisten mit Büchern umzuräumen.

(Er hätte wirklich irgendwann auspacken und Bücherregale besorgen müssen, aber er steckte mitten in seinem Lehrbuch-Lesewettbewerb mit Hermine und war in Rückstand geraten, also hatte er keine Zeit gehabt.)

Harry fand das Buch, das er wollte, und rannte wieder nach oben. Die anderen Jungen machten sich gerade bereit, zum Frühstück in die Große Halle zu gehen und den Tag zu beginnen.

„Entschuldigung, kannst du etwas für mich tun?“, sagte Harry. Er blätterte durch den Index des Buches, während er sprach, fand die Seite mit den ersten zehntausend Primzahlen, blätterte zu dieser Seite und schob das Buch Anthony Goldstein zu.

„Wähle zwei dreistellige Zahlen aus dieser Liste. Sag mir nicht, wie sie lauten. Multipliziere sie einfach miteinander und sage mir das Produkt.\\ Oh, und kannst du die Rechnung zweimal machen, um es zu überprüfen? Bitte stell wirklich sicher, dass du die richtige Antwort hast, ich weiß nicht, was mit mir oder dem Universum passiert, wenn du einen Multiplikationsfehler machst."

Es sagte viel darüber aus, wie das Leben in den letzten Tagen gewesen war, dass Anthony sich nicht einmal die Mühe machte, etwas zu sagen wie\\ „\emph{Warum bist du plötzlich ausgeflippt?„} oder\\ „\emph{Das scheint wirklich seltsam zu sein, was sind deine Gründe für die Frage?}“ oder\\ „\emph{Was meinst du damit, du bist dir nicht sicher, was mit dem Universum passieren wird?„}

Anthony nahm das Buch wortlos entgegen und holte ein Pergament und einen Federkiel heraus.\\ Harry wirbelte herum und schloss die Augen, um nichts zu sehen, tanzte hin und her und hüpfte vor Ungeduld auf und ab.\\ Er holte einen Block Papier und einen Druckbleistift und machte sich bereit zu schreiben.

„Okay“, sagte Anthony, „einhunderteinundachtzigtausendvierhundertneunundzwanzig.“

Harry schrieb 181.429 auf. Er wiederholte, was er gerade aufgeschrieben hatte, und Anthony bestätigte es. Dann rannte Harry wieder hinunter in seinen Koffer, schaute auf seine Uhr\\ (die Uhr zeigte 4:28, was 7:28 bedeutete)\\ und schloss dann die Augen.\\ Ungefähr dreißig Sekunden später hörte Harry das Geräusch von Schritten, gefolgt von dem Geräusch, dass die Kellerebene des Koffers zuglitt.

(Harry machte sich keine Sorgen, zu ersticken. Ein automatischer Lufterfrischungszauber war Teil dessen, was man bekam, wenn man bereit war, einen wirklich guten Koffer zu kaufen.\\ War Magie nicht wunderbar, sie musste sich nicht um Stromrechnungen kümmern.)

Und als Harry seine Augen öffnete, sah er genau das, was er zu sehen gehofft hatte: ein gefaltetes Stück Papier, das auf dem Boden lag, das Geschenk seines zukünftigen Ichs.

Nennen Sie dieses Stück Papier „Papier-2“.\\ Harry riss ein Stück Papier von seinem Block ab. Nennen wir es „Papier-1“.

\emph{Es war natürlich das gleiche Stück Papier.}\\ Man konnte sogar sehen, wenn man genau hinsah, dass die ausgefransten Ränder übereinstimmten.\\ Harry ging im Geiste den Algorithmus durch, dem er folgen würde.

1. Wenn Harry Papier-2 öffnete und es leer war, dann würde er\\ \emph{„101 x 101"}\\ auf Papier-1 schreiben, es zusammenfalten, eine Stunde lang lernen, in der Zeit zurückgehen, Papier-1 an sich selbst geben\\ (das dadurch zu Papier-2 werden würde)\\ und sich auf den Weg nach oben aus dem Koffer machen, um mit seinen Schlafsaal-Kollegen zu frühstücken.

2. Wenn Harry Papier-2 öffnete und darauf zwei Zahlen geschrieben waren, würde er diese Zahlen miteinander multiplizieren.

3. Wenn das Produkt 181.429 ergibt, würde Harry diese beiden Zahlen auf Papier-1 notieren und Papier-1 in der Zeit an sich selbst zurückschicken.

4. Andernfalls würde Harry zu der Zahl auf der rechten Seite 2 addieren und das neue Zahlenpaar auf Papier-1 notieren.

5. Es sei denn, die Zahl auf der rechten Seite wäre größer als 997. In diesem Fall würde Harry 2 zu der Zahl auf der linken Seite addieren und auf der rechten Seite 101 notieren.

6. Und wenn auf Papier-2 997 x 997 stand, würde Harry Papier-1 leer lassen.

\textbf{\emph{Das bedeutete, dass die einzig mögliche stabile Zeitschleife diejenige war, in der Papier-2 die beiden Primfaktoren von 181.429 enthielt.}}

\emph{Wenn dies funktionierte, konnte Harry damit jede Art von Antwort wiederherstellen, die leicht zu überprüfen, aber schwer zu finden war.}

Er hätte nicht nur gezeigt, dass P=NP ist, wenn man einen Zeitumkehrer hat, dieser Trick war allgemeiner als das.\\ Harry könnte damit die Kombinationen von Kombinationsschlössern finden, oder Passwörter jeder Art.\\ Vielleicht sogar den Eingang zu Slytherins Kammer des Schreckens finden, wenn Harry einen systematischen Weg finden könnte, alle Orte in Hogwarts zu beschreiben.

\emph{Das wäre selbst nach Harrys Maßstäben ein genialer Trick.}

Harry nahm Papier-2 in seine zitternde Hand und entfaltete es.

Auf Papier-2 stand in leicht zittriger Handschrift: SPIEL NICHT MIT DER ZEIT\\ Harry schrieb in leicht zittriger Handschrift

„SPIEL NICHT MIT DER ZEIT„

auf Papier-1, faltete es ordentlich zusammen und beschloss, keine wirklich brillanten Experimente mit der Zeit mehr durchzuführen, bis er mindestens fünfzehn Jahre alt war.\\ Soweit Harry wusste, war das das erschreckendste Versuchsergebnis in der gesamten Geschichte der Wissenschaft gewesen.

Es war für Harry etwas schwierig gewesen, sich in der nächsten Stunde auf das Lesen seines Lehrbuchs zu konzentrieren.\\ So hatte Harrys Donnerstag begonnen.

\textbf{Donnerstag}.\\ Wenn man genau sein wollte, um 15:32 Uhr am Donnerstagnachmittag.\\ Harry und alle anderen Jungen des ersten Jahres waren mit Madam Hooch draußen auf einer grasbewachsenen Wiese und standen neben dem Hogwarts-Vorrat an Besen.\\ Die Mädchen würden das Fliegen separat lernen. Offenbar wollten Mädchen aus irgendeinem Grund nicht in Gegenwart von Jungen das Fliegen auf Besen lernen.\\ Harry war den ganzen Tag über etwas wackelig auf den Beinen gewesen. Er konnte einfach nicht aufhören, sich zu fragen, wie diese spezielle stabile Zeitschleife aus einem, im Nachhinein betrachtet, ziemlich großen Raum von Möglichkeiten ausgewählt worden war.

Außerdem: ernsthaft, Besen? Er sollte im Grunde auf einer Linie fliegen? War das nicht so ziemlich die instabilste Form, die man überhaupt finden konnte, abgesehen von dem Versuch, sich an einem Punkt festzuhalten? Wer hatte dieses Design für ein Fluggerät ausgewählt, unter all den Möglichkeiten? Harry hatte gehofft, dass es sich nur um eine Redewendung handelte, aber nein, sie standen vor etwas, das um alles in der Welt wie ein gewöhnlicher Küchenbesen aus Holz aussah.\\ Hatte sich jemand nur an der Idee der Besen festgebissen und nichts anderes in Betracht gezogen? Es musste so sein.\\ Es gab keine Möglichkeit, dass die optimalen Designs für das Reinigen von Küchen und das Fliegen zufällig zusammenfielen, wenn man sie von Grund auf ausarbeitete.

Es war ein klarer Tag mit strahlend blauem Himmel und einer brillanten Sonne, die geradezu darum bettelte, einem in die Augen zu geraten und es unmöglich zu machen, etwas zu sehen, wenn man versuchte, am Himmel herumzufliegen.\\ Der Boden war schön trocken, roch förmlich nach Backofen und fühlte sich irgendwie sehr, sehr hart unter Harrys Schuhen an.\\ Harry erinnerte sich immer wieder daran, dass alle Elfjährigen dies lernen sollten und es nicht so schwer sein konnte.

„Streck die rechte Hand über den Besen aus, oder die linke Hand, wenn du Linkshänder bist“, rief Madam Hooch.

„Und sagt: AUF!„

„AUF!“, riefen alle. Der Besen sprang eifrig in Harrys Hand. Was ihn ausnahmsweise an die Spitze der Klasse brachte.\\ Offenbar war es viel schwieriger, „\emph{AUF}!“ zu sagen, als es aussah, und die meisten Besen rollten auf dem Boden herum oder versuchten, sich von ihren Möchtegern-Reitern zu entfernen.

(Natürlich hätte Harry Geld darauf gewettet, dass Hermine es mindestens genauso gut gemacht hätte, als sie früher am Tag selbst an der Reihe war, es zu versuchen. Es konnte unmöglich etwas geben, was er beim ersten Versuch meistern konnte, was Hermine verblüffen würde, und wenn es doch etwas gäbe und es sich als Besenreiten statt als etwas Intellektuelles herausstellen würde, würde Harry einfach sterben.)

Es dauerte eine Weile, bis jeder einen Besen vor sich stehen hatte. Madam Hooch zeigte ihnen, wie man aufsteigt, und ging dann um das Feld herum, um Griffe und Haltungen zu korrigieren.\\ Offenbar hatte man selbst den wenigen Kindern, die zu Hause fliegen durften, nicht beigebracht, wie man es richtig macht.\\ Madam Hooch betrachtete das Feld der Jungen und nickte.

„Wenn ich jetzt in meine Pfeife blase, stoßt ihr vom Boden ab, und zwar kräftig."\\ Harry schluckte schwer und versuchte, das mulmige Gefühl in seinem Magen zu unterdrücken.\\ „Haltet eure Besen ruhig, steigt ein paar Meter hoch und kommt dann gerade wieder herunter, indem ihr euch leicht nach vorne lehnt.\\ Auf meinen Pfiff - drei - zwei -„

Einer der Besen schoss in die Höhe, begleitet von den Schreien eines kleinen Jungen - aus Entsetzen, nicht aus Freude.\\ Der Junge drehte sich mit einer furchtbaren Geschwindigkeit, während er aufstieg, sie bekamen nur einen flüchtigen Blick auf sein weißes Gesicht - wie in Zeitlupe sprang Harry von seinem eigenen Besen zurück und kramte nach seinem Zauberstab, obwohl er nicht wirklich wusste, was er damit vorhatte, er hatte genau zwei Stunden Zauberkunst gehabt und die letzte war der Schwebezauber gewesen, aber Harry hatte den Zauber nur ein Mal von drei erfolgreich anwenden können und er konnte sicherlich keine ganzen Menschen schweben lassen -

\emph{Wenn es irgendeine verborgene Kraft in mir gibt, dann soll sie sich JETZT offenbaren!}

„Komm zurück, Junge!“, rief Madam Hooch

(was die denkbar schlechteste Anweisung für den Umgang mit einem außer Kontrolle geratenen Besen sein musste, von einer Fluglehrerin, und ein \emph{vollautomatischer Teil von Harrys Gehirn fügte Madam Hooch seiner Liste der Dummköpfe hinzu}).

Und der Junge wurde vom Besen geworfen. Er schien sich zunächst sehr langsam durch die Luft zu bewegen.

„Wingardium Leviosa!“, schrie Harry. Der Zauberspruch versagte. Er konnte es spüren. Es gab einen dumpfen Aufprall und ein entferntes Knacken, und der Junge lag mit dem Gesicht nach unten in einem Haufen auf dem Gras.\\ Harry steckte seinen Zauberstab in die Scheide und rannte mit voller Geschwindigkeit nach vorne. Er kam gleichzeitig mit Madam Hooch an der Seite des Jungen an, und Harry griff in seinen Beutel und versuchte, sich zu erinnern, \emph{oh Gott, wie hieß das noch gleich}, egal, er versuchte es einfach mit\\ „\emph{Heilerpackung}“, und es tauchte in seiner Hand auf und -

„Gebrochenes Handgelenk“, sagte Madam Hooch. „Beruhige dich, Junge, er hat nur ein gebrochenes Handgelenk!„

Es gab eine Art mentalen Ruck, als Harrys Verstand aus dem Panikmodus schnappte.\\ Das Notfall Heiler Pack Plus lag geöffnet vor ihm, und in Harrys Hand befand sich eine Spritze mit flüssigem Feuer, die das Gehirn des Jungen mit Sauerstoff versorgt hätte, wenn er es geschafft hätte, sich das Genick zu brechen.

„Ah...“ Sagte Harry mit einer etwas schwankenden Stimme. Sein Herz pochte so laut, dass er fast nicht hören konnte, wie er nach Luft schnappte.\\ „Gebrochener Knochen... richtig... Tourniquet setzen?„

„Das ist nur für Notfälle“, schnauzte Madam Hooch. „Legen Sie es weg, es geht ihm gut."\\ Sie beugte sich über den Jungen und bot ihm eine Hand an.\\ „Komm schon, Junge, es ist alles in Ordnung, hoch mit dir!„

„Du willst ihn doch nicht ernsthaft wieder auf dem Besen reiten lassen?“ sagte Harry entsetzt.

Madam Hooch warf Harry einen finsteren Blick zu. „Natürlich nicht!"\\ Sie zog den Jungen mit seinem guten Arm auf die Füße - Harry sah mit Schrecken, dass es wieder Neville Longbottom war, \emph{was war mit ihm los}? - und sie wandte sich an alle zuschauenden Kinder.\\ „Keiner von euch bewegt sich, während ich den Jungen in den Krankenflügel bringe! Ihr lasst die Besen, wo sie sind, oder ihr seid aus Hogwarts raus, bevor ihr 'Quidditch' sagen könnt.\\ Komm, meine Lieber.„

Und Madam Hooch ging mit Neville davon, der sich das Handgelenk umklammerte und versuchte, seine Tränen zu kontrollieren.\\ Als sie außer Hörweite waren, begann einer der Slytherins zu kichern. Das rief die anderen auf den Plan.\\ Harry drehte sich um und sah sie an. Es schien ein guter Zeitpunkt zu sein, sich einige Gesichter einzuprägen.\\ Und Harry sah, dass Draco auf ihn zu schlenderte, begleitet von Mr. Crabbe und Mr. Goyle. Mr. Crabbe lächelte nicht. Mr. Goyle schon. Draco selbst trug ein sehr beherrschtes Gesicht, das gelegentlich zuckte, woraus Harry schloss, dass Draco es zwar urkomisch fand, aber keinen politischen Vorteil darin sah, jetzt darüber zu lachen, statt nachher in den Slytherin-Kerkern.

„Nun, Potter“, sagte Draco mit leiser Stimme, die nicht trug, immer noch mit diesem sehr kontrollierten Gesicht, das gelegentlich zuckte, „ich wollte nur sagen, wenn man Notfälle ausnutzt, um Führungsstärke zu demonstrieren, sollte man so aussehen, als hätte man die Situation völlig unter Kontrolle, anstatt, sagen wir, in völlige Panik zu verfallen."

Mr. Goyle kicherte, und Draco warf ihm einen beschwichtigenden Blick zu. „Aber du hast wahrscheinlich trotzdem ein paar Punkte gesammelt. Brauchst du Hilfe beim Verstauen des Heilerkoffers?„\\ Harry drehte sich um und schaute auf das Heilerpaket, was dazu führte, dass er sein eigenes Gesicht von Draco abwandte.

„Ich denke, ich komme klar“, sagte Harry. Er legte die Spritze zurück an ihren Platz, verriegelte sie wieder und stand auf.\\ Ernie Macmillan kam gerade an, als Harry das Päckchen wieder in seinen Beutel steckte.

„Ich danke dir, Harry Potter, im Namen von Hufflepuff“, sagte Ernie Macmillan formell.\\ „Es war ein guter Versuch und ein guter Gedanke.„

„In der Tat ein guter Gedanke“, murmelte Draco. „Warum hat niemand in Hufflepuff seinen Zauberstab gezückt? Wenn ihr alle geholfen hättet, statt nur Potter, hättet ihr ihn vielleicht erwischen können. Ich dachte, Hufflepuffs sollten zusammenhalten?"

Ernie sah aus, als wäre er hin- und hergerissen zwischen Wut und dem Wunsch, vor Scham zu sterben.\\ "Wir haben nicht rechtzeitig daran gedacht -„

„Ah“, sagte Draco, „\emph{daran haben wir nicht gedacht,} deshalb ist es wohl besser, einen Ravenclaw als Freund zu haben als ganz Hufflepuff.„\\ \emph{Oh, verdammt, wie sollte Harry das wieder gerade rücken...}

„Du bist nicht hilfreich“, sagte Harry in einem milden Ton. In der Hoffnung, Draco würde das so interpretieren, dass du dich in meine Pläne einmischst, \emph{sei bitte still.}

„Hey, was ist das?“, sagte Mr. Goyle. Er bückte sich ins Gras und hob etwas auf, das etwa so groß war wie eine große Murmel, eine Glaskugel, die mit einem wirbelnden weißen Nebel gefüllt zu sein schien.\\ Ernie blinzelte.\\ „Nevilles Erinnermich!„

„Was ist ein Erinnermich?“, fragte Harry.

„Er wird rot, wenn man etwas vergessen hat“, sagte Ernie.

„Er sagt dir aber nicht, was du vergessen hast. Gib ihn bitte her, und ich gebe ihn Neville später zurück.“ Ernie hielt ihm die Hand hin.

Ein plötzliches Grinsen huschte über Mr. Goyles Gesicht, er drehte sich um und rannte davon. Ernie blieb einen Moment lang überrascht stehen, dann rief er\\ „Hey!„\\ und rannte hinter Mr. Goyle her. Und Mr. Goyle schnappte sich einen Besen, hüpfte mit einer geschmeidigen Bewegung darauf und erhob sich in die Luft.\\ Harrys Kinnlade fiel herunter. Hatte Madam Hooch nicht gesagt, dass er dafür von der Schule fliegen würde?

„Dieser Idiot!“ zischte Draco. Er öffnete den Mund, um zu schreien -

„Hey!“, rief Ernie. „Das ist Nevilles! Gib es zurück!„

Die Slytherins begannen zu jubeln und zu johlen. Dracos Mund schnappte zu. Harry bemerkte den plötzlichen Blick der Unentschlossenheit auf seinem Gesicht.

„Draco“, sagte Harry in leisem Ton, „wenn du diesen Idioten nicht zurück auf den Boden befiehlst, wird der Lehrer zurückkommen und -„

„Komm und hol's dir, Hufflepuffle!“, rief Mr. Goyle, und ein großer Jubel ging von den Slytherins aus.\\ „Ich kann nicht!??“, flüsterte Draco. „Jeder in Slytherin würde denken, ich sei schwach!„

„Und wenn Mr. Goyle rausgeworfen wird“, zischte Harry, „wird dein Vater denken, dass du ein Schwachkopf bist!„

Dracos Gesicht verzog sich vor Schmerz. In diesem Moment -\\ „Hey, Slytherslime“, rief Ernie,\\ „hat dir nie jemand gesagt, dass Hufflepuffs zusammenhalten? Zauberstäbe raus, Hufflepuff!„\\ Und plötzlich waren eine ganze Menge Zauberstäbe in Mr. Goyles Richtung gerichtet. Drei Sekunden später -\\ „Zauberstäbe raus, Slytherin!“ sagten etwa fünf verschiedene Slytherins. Und eine ganze Reihe von Zauberstäben zeigte in Richtung Hufflepuff.\\ 2 Sekunden später: „Zauberstäbe raus, Gryffindor!„

„Tu was, Potter!??“, flüsterte Draco. „Ich kann nicht derjenige sein, der das aufhält, das musst du sein! Ich bin dir einen Gefallen schuldig, denk dir einfach etwas aus, solltest du nicht brillant sein?„

In etwa fünfeinhalb Sekunden, so wurde Harry klar, würde jemand die sumerische Einfacher-Schlag-Verhexung zaubern und bis es vorbei war und die Lehrer mit den Rauswürfen fertig waren, würden die einzigen Jungen in seinem Jahrgang Ravenclaws sein.

„Zauberstäbe raus, Ravenclaw!“, rief Michael Corner, der sich anscheinend von dem Desaster ausgeschlossen fühlte.

„GREGORY GOYLE!“, schrie Harry.\\ „Ich fordere dich zu einem Wettstreit um den Besitz von Nevilles Erinnermich heraus!„\\ Es gab eine plötzliche Pause.

„Oh, wirklich?“, sagte Draco in dem lautesten Tonfall, den Harry je gehört hatte. „Das klingt interessant. Was für ein Wettbewerb, Potter?„

Äh... „\emph{Wettbewerb}“ war alles, was Harrys Inspiration hervorgebracht hatte.\\ Was für ein Wettbewerb, er konnte nicht „Schach“ sagen, weil Draco das nicht akzeptieren könnte, ohne dass es seltsam aussähe, er konnte nicht „\emph{Armdrücken}“ sagen, weil Mr. Goyle ihn zerquetschen würde -

„Wie wäre es hiermit?“ sagte Harry laut.\\ „Gregory Goyle und ich stehen getrennt voneinander, und niemand sonst darf sich einem von uns beiden nähern. Wir benutzen unsere Zauberstäbe nicht und auch sonst niemand. Ich bewege mich nicht von der Stelle und er auch nicht.\\ Und wenn ich Nevilles Erinnermich in die Hände bekomme, dann verzichtet Gregory Goyle auf jeden Anspruch auf das Erinnermich, das er in der Hand hält, und gibt es mir.„

Es gab eine weitere Pause, in der sich die Blicke der Leute von Erleichterung zu Verwirrung wandelten.

„Hah, Potter!“, sagte Draco laut. „Das würde ich gerne sehen! Mr. Goyle nimmt an!„

„Es geht los!“, sagte Harry.

„Potter, was?“, flüsterte Draco, was er irgendwie tat, ohne seine Lippen zu bewegen. Harry wusste nicht, wie er antworten sollte, ohne seine zu bewegen.\\ Die Leute steckten ihre Zauberstäbe weg, und Mr. Goyle sank anmutig zu Boden und sah ziemlich verwirrt aus.\\ Einige Hufflepuffs kamen auf Mr. Goyle zu, aber Harry warf ihnen einen\\ verzweifelten, flehenden Blick zu und sie wichen zurück.\\ Harry ging auf Mr. Goyle zu und blieb stehen, als er ein paar Schritte entfernt war, weit genug entfernt, dass sie sich nicht erreichen konnten.\\ Langsam und bedächtig zog Harry seinen Zauberstab aus der Tasche. Alle anderen wichen zurück. Harry schluckte.\\ Er wusste in groben Zügen, was er tun wollte, aber es musste so gemacht werden, dass niemand verstand, was er getan hatte -\\ „Also gut“, sagte Harry laut. „Und jetzt...„\\ Er holte tief Luft und hob eine Hand, die Finger bereit zum Schnippen. Jeder, der von den Torten gehört hatte, was praktisch jeder war, keuchte auf.\\ „Ich rufe den Wahnsinn von Hogwarts herbei! Schwachkopf! Schwabbelspeck! Krimskrams! Quiek!“\\ Und Harry schnippte mit den Fingern.\\ Eine Menge Leute zuckten zusammen. Und nichts passierte. Harry ließ die Stille eine Weile andauern, sich entwickelnd, bis.\\ ..\\ „Ähm“, sagte jemand. „War's das?"\\ Harry sah den Jungen an, der gesprochen hatte.\\ „Sieh mal vor dich hin. Siehst du den Fleck Erde, der kahl aussieht, ohne Gras darauf?„

„Ähm, ja“, sagte der Junge, ein Gryffindor (Dean irgendwas?).

„Grab es aus.“ Jetzt erntete Harry eine Menge seltsamer Blicke.

„Ähm, warum?“, sagte Dean irgendwas.

„Tu es einfach“, sagte Terry Boot mit müder Stimme. „Es hat keinen Sinn, nach dem Warum zu fragen, glaube mir.„

Dean irgendwas kniete sich hin und begann, den Schmutz wegzuschaufeln.\\ Nach einer Minute oder so, stand Dean wieder auf.\\ „Da ist nichts“, sagte Dean.

Aha. Harry hatte vorgehabt, in die Vergangenheit zu gehen und eine Schatzkarte zu vergraben, die zu einer anderen Schatzkarte führen würde, die zu Nevilles Erinnermich führen würde, die er dort hinlegen würde, nachdem er sie von Mr. Goyle zurückbekommen hatte... Dann wurde Harry klar, dass es einen viel einfacheren Weg gab, der das Geheimnis des Zeitumkehrers nicht ganz so sehr gefährdete.

„Danke, Dean!"\\ sagte Harry laut.\\ „Ernie, würdest du dich auf dem Boden umsehen, wo Neville gefallen ist, und schauen, ob du Nevilles Erinnermich finden kannst?„\\ Die Leute schauten noch verwirrter.

„Tu es einfach“, sagte Terry Boot. „Er wird es so lange versuchen, bis etwas funktioniert, und das Unheimliche ist, dass -„

„Merlin!“, keuchte Ernie. Er hielt Nevilles Erinnermich hoch. „Er ist hier! Genau da, wo er hingefallen ist!„

„Was?“, rief Mr. Goyle. Er schaute nach unten und sah......dass er immer noch Nevilles Erinnermich in der Hand hielt.\\ Es gab eine ziemlich lange Pause.\\ „Ähm“, sagte Dean etwas, „das ist doch nicht möglich, oder?„

„Es ist ein Handlungsloch“, sagte Harry. „Ich habe mich so komisch verhalten, dass ich das Universum für einen Moment abgelenkt habe und es hat vergessen, dass Goyle das Erinnermich bereits abgeholt hat."

„Nein, warte, ich meine, das ist absolut nicht möglich -"

„Entschuldigung, stehen wir hier alle herum und warten darauf, auf Besen zu fliegen? Ja, das tun wir. Also haltet die Klappe. Sobald ich Nevilles Erinnermich in die Finger kriege, ist der Wettbewerb vorbei und Gregory Goyle muss den Anspruch auf das Erinnermich aufgeben und es mir geben. Das waren die Bedingungen, schon vergessen?"\\ Harry streckte eine Hand aus und winkte Ernie zu sich.

„Rolle es einfach hier rüber, da niemand in meine Nähe kommen darf, okay?„

„Moment mal!“, rief ein Slytherin - Blaise Zabini, den Namen würde Harry wohl nie vergessen.\\ „Woher sollen wir wissen, dass das Nevilles Erinnermich ist? Du könntest einfach einen anderen Erinnermich dort abgelegt haben -„

„Der Slytherin ist stark in diesem hier“, sagte Harry und lächelte.\\ „Aber du hast mein Wort, dass der, den Ernie in der Hand hält, der von Neville ist. Kein Kommentar über den, den Gregory Goyle in der Hand hält."

Zabini drehte sich zu Draco.\\ „Malfoy! Du wirst ihn doch nicht einfach so davonkommen lassen -„

„Halt die Klappe!“, polterte Mr. Crabbe, der hinter Draco stand.\\ „Mr. Malfoy braucht dich nicht, um ihm zu sagen, was er tun soll!„\\ \emph{Guter Minion.}

„Ich habe mit Draco gewettet, aus dem edlen und sehr alten Haus Malfoy“, sagte Harry.\\ „Nicht mit dir, Zabini. Ich habe getan, was Mr. Malfoy von mir verlangt hat, und was die Beurteilung der Wette angeht, überlasse ich das Mr. Malfoy.„

Harry neigte den Kopf in Richtung Draco und hob die Augenbrauen leicht an. Das sollte Draco erlauben, genug Gesicht zu wahren. Es entstand eine Pause.

„Du versprichst, dass das tatsächlich Nevilles Erinnermich ist?“ Sagte Draco.

„Ja“, sagte Harry. „Das ist derjenige, der an Neville zurückgehen wird und es war ursprünglich seins. Und der, den Gregory Goyle hält, geht an mich."

Draco nickte und sah entschlossen aus.\\ „Ich werde also das Wort des edlen Hauses Potter nicht anzweifeln, egal wie seltsam das alles war.\\ Und das edle und uralte Haus Malfoy hält ebenfalls sein Wort. Mr. Goyle, geben Sie das Mr. Potter -„

„Hey!“ sagte Zabini. „Er hat noch nicht gewonnen, er hat noch nicht -„

„Fang, Harry!“, sagte Ernie, und er warf den Erinnermich zu.\\ Harry schnappte den Erinnermich mit Leichtigkeit aus der Luft, er hatte schon immer gute Reflexe gehabt, was das angeht.

„Da“, sagte Harry, „ich gewinne...„\\ Harry brach ab. Alle Gespräche verstummten. Der Erinnermich glühte hellrot in seiner Hand, strahlte wie eine Miniatursonne, die am helllichten Tag Schatten auf den Boden warf.

\textbf{Donnerstag}.\\ Wenn man es genau nehmen wollte, 17:09 Uhr am Donnerstagnachmittag, in Professor McGonagalls Büro, nach dem Flugunterricht.

(Mit einer extra Stunde für Harry, die dazwischen geschoben wurde.)

Professor McGonagall sitzt auf ihrem Hocker.\\ Harry auf dem Stuhl vor ihrem Schreibtisch.\\ „Professor“, sagte Harry angespannt, „Slytherin richtete seine Zauberstäbe auf Hufflepuff, Gryffindor richtete seine Zauberstäbe auf Slytherin, irgendein Idiot rief Zauberstäbe in Ravenclaw, und ich hatte vielleicht fünf Sekunden Zeit, um zu verhindern, dass das Ganze in die Luft fliegt!\\ Das war alles, woran ich denken konnte!"

Professor McGonagalls Gesicht war verkniffen und wütend.\\ „Sie dürfen den Zeitumkehrernicht auf diese Weise benutzen, Mr. Potter! Ist das Konzept der Geheimhaltung nicht etwas, das Sie verstehen?"

„Sie wissen nicht, wie ich es gemacht habe!\\ Sie denken nur, dass ich wirklich seltsame Dinge tun kann, indem ich mit den Fingern schnippe! Ich habe schon andere verrückte Sachen gemacht, die man nicht einmal mit Zeitumkehrern machen kann, und ich werde noch mehr solche Sachen machen, und dieser Fall wird nicht einmal auffallen!\\ Ich musste es tun, Professor!„

„Du musstest es nicht tun!“, schnauzte Professor McGonagall. „Alles, was Sie hätten tun müssen, war, diesen anonymen Slytherin wieder auf den Boden zu bringen und die Zauberstäbe wegzulegen!\\ Du hättest ihn zu einer Partie Snape explodiert herausfordern können, aber nein, du musstest den Zeitumkehrer auf eine schamlose und unnötige Weise benutzen!"

„Das war alles, was mir einfiel! Ich weiß nicht einmal, was \emph{Snape explodiert} ist, ein Schachspiel hätten sie nicht akzeptiert und wenn ich Armdrücken gewählt hätte, hätte ich verloren!"

\textbf{„Dann hättest du Armdrücken wählen sollen!"}

Harry blinzelte. „Aber dann hätte ich verloren -„

Harry hielt inne. Professor McGonagall sah sehr wütend aus.

„Es tut mir leid, Professor McGonagall“, sagte Harry mit leiser Stimme.\\ „Daran habe ich wirklich nicht gedacht, und Sie haben recht, ich hätte es tun sollen, es wäre brillant gewesen, wenn ich es getan hätte, aber ich habe einfach überhaupt nicht daran gedacht...„

Harrys Stimme verstummte. Ihm wurde plötzlich klar, dass er eine Menge anderer Möglichkeiten gehabt hätte.\\ Er hätte Draco bitten können, etwas vorzuschlagen, er hätte die Menge fragen können... sein Einsatz des Zeitumkehrers war schamlos und unnötig gewesen.\\ Es hatte einen riesigen Raum von Möglichkeiten gegeben, warum hatte er sich diese ausgesucht?

\emph{Weil er einen Weg gesehen hatte, zu gewinnen.}

Den Besitz eines unwichtigen Schmuckstücks zu gewinnen, das die Lehrer Mr. Goyle sowieso abgenommen hätten. Die Absicht, zu gewinnen. Das war es, was ihn geritten hatte.

„Es tut mir Leid“, sagte Harry wieder. „Für meinen Stolz und meine Dummheit."

Professor McGonagall wischte sich mit einer Hand über die Stirn. Etwas von ihrem Zorn schien sich zu verflüchtigen. Aber ihre Stimme kam immer noch sehr hart heraus.

„Noch so ein Auftritt, Mr. Potter, und Sie werden diesen Zeitumkehrer zurückgeben. Habe ich mich klar ausgedrückt?„

„Ja“, sagte Harry. „Ich verstehe und es tut mir leid."

„Dann, Mr. Potter, werden Sie den Zeitumkehrer vorerst behalten dürfen.\\ Und in Anbetracht des Ausmaßes des Debakels, das Sie tatsächlich abgewendet haben, werde ich Ravenclaw keine Punkte abziehen.„

\emph{Außerdem könntest du nicht erklären, warum man die Punkte abgezogen hatte.}\\ Aber Harry war nicht dumm genug, das laut auszusprechen.

„Viel wichtiger ist, warum ist der Erinnermich so losgegangen?“ sagte Harry.

„Bedeutet es, dass ich Obliviated wurde?„

„Das ist mir auch ein Rätsel“, sagte Professor McGonagall langsam.\\ „Wenn es so einfach wäre, würden die Gerichte Erinnermichs verwenden, aber das tun sie nicht.\\ Ich werde es mir ansehen, Mr. Potter."\\ Sie seufzte.\\ „Sie können jetzt gehen."

Harry begann, sich von seinem Stuhl zu erheben, hielt dann aber inne.\\ „Ähm, Entschuldigung, ich wollte Ihnen noch etwas sagen -"

Man konnte das Zusammenzucken kaum sehen.

„Was ist es, Mr. Potter?"

„Es geht um Professor Quirrell -"

„Ich bin sicher, Mr. Potter, dass es nichts von Bedeutung ist."\\ Professor McGonagall sprach die Worte in großer Eile.\\ „Sie haben doch sicher gehört, wie der Schulleiter den Schülern gesagt hat, dass Sie uns nicht mit irgendwelchen unwichtigen Beschwerden über den Verteidigungsprofessor belästigen sollen?"

Harry war ziemlich verwirrt.

„Aber das könnte wichtig sein, gestern hatte ich dieses plötzliche Gefühl des Unheils, als -"

„Mr. Potter! Ich habe auch ein Gefühl des Unheils! Und mein Untergangssinn sagt mir, dass Sie diesen Satz nicht beenden dürfen!„

Harrys Mund klaffte auf.\\ Professor McGonagall hatte es geschafft; Harry war sprachlos.

„Mr. Potter“, sagte Professor McGonagall, „wenn Sie etwas Interessantes über Professor Quirrell herausgefunden haben, können Sie es gerne mit jemand anderem teilen. Ich denke, Sie haben jetzt genug von meiner wertvollen Zeit in Anspruch genommen -„

„Das sieht Ihnen gar nicht ähnlich!“ platzte Harry heraus.\\ „Es tut mir leid, aber das scheint einfach unglaublich unverantwortlich zu sein!\\ Nach dem, was ich gehört habe, liegt auf der Position des Verteidigers eine Art Fluch, und wenn man schon weiß, dass etwas schiefgehen wird, sollte man meinen, dass ihr alle auf der Hut sein müsst -"

„Schiefgehen, Mr. Potter? Ich hoffe nicht."\\ Das Gesicht von Professor McGonagall war ausdruckslos.\\ „Nachdem Professor Blake im Februar letzten Jahres mit nicht weniger als \emph{drei Slytherin Mädchen} aus dem fünften Schuljahr in einem Schrank erwischt wurde und ein Jahr davor Professor Summers als Pädagogin so völlig versagt hat, dass ihre Schüler einen Irrwicht für eine Art Möbelstück hielten, wäre es katastrophal, wenn mir jetzt irgendein Problem mit dem außerordentlich kompetenten Professor Quirrell zu Ohren käme, und ich wage zu behaupten, dass die meisten unserer Schüler bei ihren Verteidigungs Z. A.G. s und U. T.Z. s durchfallen würden.„

„Ich verstehe“, sagte Harry langsam und nahm alles in sich auf.\\ „Mit anderen Worten, was auch immer mit Professor Quirrell los ist, Sie wollen es unbedingt bis zum Ende des Schuljahres nicht wissen.\\ Und da wir gerade September haben, könnte er den Premierminister live im Fernsehen ermorden und damit durchkommen, soweit es Sie betrifft."

Professor McGonagall starrte ihn unverwandt an.\\ „Ich bin mir sicher, dass ich eine solche Aussage niemals gutheißen könnte, Mr. Potter. In Hogwarts sind wir bestrebt, proaktiv gegen alles vorzugehen, was den Bildungserfolg unserer Schüler bedroht."

\emph{Wie zum Beispiel Ravenclaws im ersten Jahr, die ihren Mund nicht halten können.}

„Ich glaube, ich verstehe Sie vollkommen, Professor McGonagall."

„Oh, das bezweifle ich, Mr. Potter. Ich bezweifle das sehr."\\ Professor McGonagall beugte sich vor, ihr Gesicht straffte sich wieder.\\ „Da Sie und ich bereits weitaus sensiblere Angelegenheiten als diese besprochen haben, werde ich offen sprechen.\\ Sie, und nur Sie, haben von diesem mysteriösen Gefühl des Untergangs berichtet. Sie, und nur Sie, sind ein Chaosmagnet, wie ich ihn noch nie gesehen habe.\\ Nach unserem kleinen Einkaufsbummel in der Winkelgasse, dann dem Sprechenden Hut und der heutigen kleinen Episode kann ich mir gut vorstellen, dass ich im Büro des Schulleiters sitzen und mir eine urkomische Geschichte über Professor Quirrell anhören werde, in der du und nur du die Hauptrolle spielst, woraufhin es keine andere Wahl geben wird, als ihn zu feuern.\\ Ich habe mich bereits damit abgefunden, Mr. Potter. Und wenn dieses traurige Ereignis früher als Mitte Mai stattfindet, werde ich Sie an den Toren von Hogwarts mit Ihren eigenen Eingeweiden aufhängen und Ihnen Feuerkäfer in die Nase schütten.\\ Hast du mich jetzt vollkommen verstanden?"

Harry nickte, seine Augen waren ganz groß. Dann, nach einer Sekunde:\\ „Was kriege ich, wenn ich es am letzten Tag des Schuljahres schaffe?"

„Raus aus meinem Büro!„

\textbf{Donnerstag}.\\ Irgendetwas musste es mit Donnerstagen in Hogwarts auf sich haben. Es war 17:32 Uhr am Donnerstagnachmittag, und Harry stand neben Professor Flitwick vor dem großen steinernen Wasserspeier, der den Eingang zum Büro des Schulleiters bewachte.\\ Kaum hatte er es von Professor McGonagalls Büro zurück in die Ravenclaw-Lernräume geschafft, als einer der Schüler ihm sagte, er solle sich in Professor Flitwicks Büro melden, und dort hatte Harry erfahren, dass Dumbledore ihn sprechen wollte.\\ Harry, der sich ziemlich beunruhigt fühlte, hatte Professor Flitwick gefragt, ob der Schulleiter gesagt habe, worum es gehe.\\ Professor Flitwick hatte hilflos mit den Schultern gezuckt. Offenbar hatte Dumbledore gesagt, dass Harry viel zu jung sei, um die Worte der Macht und des Wahnsinns zu beschwören.\\ \emph{Stachelschwein?} hatte Harry gedacht, aber nicht laut ausgesprochen.

„Bitte machen Sie sich nicht zu viele Sorgen, Mr. Potter“, quietschte Professor Flitwick von irgendwo in Harrys Schulterhöhe.

(Harry war dankbar für Professor Flitwicks riesigen geschwollenen Bart, es war schwer, sich an einen Professor zu gewöhnen, der nicht nur kleiner war als er, sondern auch mit höherer Stimme sprach.)

„Schulleiter Dumbledore mag ein wenig seltsam erscheinen, oder sehr seltsam, oder sogar extrem seltsam, aber er hat noch nie einem Schüler etwas zuleide getan, und ich glaube nicht, dass er das jemals tun wird."\\ Professor Flitwick schenkte Harry ein ermutigendes Lächeln.\\ „Behalten Sie das einfach immer im Hinterkopf, dann werden Sie sicher nicht in Panik geraten!„

\emph{Das war nicht gerade hilfreich.}

„Viel Glück!“, quietschte Professor Flitwick, beugte sich zu dem Wasserspeier hinüber und sagte etwas, das Harry irgendwie gar nicht hören konnte.

(Natürlich würde ein Passwort nicht viel nützen, wenn man hören könnte, wie jemand es sagt.)

Und der steinerne Wasserspeier ging mit einer sehr natürlichen und gewöhnlichen Bewegung zur Seite, die Harry ziemlich schockierend fand, da der Wasserspeier die ganze Zeit noch wie ein fester, unbeweglicher Stein aussah.\\ Hinter dem Wasserspeier befand sich eine Reihe von sich langsam drehenden Wendeltreppen. Es hatte etwas beunruhigend Hypnotisches an sich, und noch beunruhigender war, dass das Drehen der Spirale einen eigentlich nirgendwohin bringen sollte.

„Rauf mit dir!“, quietschte Flitwick. Harry trat etwas nervös auf die Spirale und fand sich aus irgendeinem Grund, den sich sein Gehirn überhaupt nicht vorstellen konnte, aufwärts bewegt.\\ Der Wasserspeier polterte hinter ihm zurück, und die Wendeltreppe drehte sich weiter und Harry kam immer weiter nach oben, und nach einer ziemlich schwindelerregenden Zeit fand sich Harry vor einer Eichentür mit einem Greifenklopfer aus Messing wieder.\\ Harry streckte die Hand aus und drehte den Türknauf. Die Tür schwang auf. Und Harry sah den interessantesten Raum, den er je in seinem Leben gesehen hatte.

Es gab winzige Metallmechanismen, die surrten oder tickten oder langsam ihre Form veränderten oder kleine Rauchwölkchen ausstießen.\\ Es gab Dutzende von mysteriösen Flüssigkeiten in Dutzenden von seltsam geformten Behältern, die alle blubberten, kochten, tropften, ihre Farbe änderten oder interessante Formen annahmen, die eine halbe Sekunde, nachdem man sie gesehen hatte, wieder verschwanden.\\ Es gab Dinge, die wie Uhren mit vielen Zeigern aussahen, beschriftet mit Zahlen oder in unerkennbaren Sprachen.\\ Es gab ein Armband mit einem linsenförmigen Kristall, der in tausend Farben funkelte, und einen Vogel, der auf einer goldenen Plattform thronte, und einen hölzernen Becher, der mit etwas gefüllt war, das wie Blut aussah, und eine Statue eines Falken, die mit schwarzer Emaille verkrustet war.\\ An der Wand hingen lauter Bilder von schlafenden Menschen, und der Sprechende Hut stand lässig auf einem Hutständer, der auch zwei Regenschirme und drei rote Pantoffeln für die linken Füße enthielt.\\ Inmitten des ganzen Chaos stand ein sauberer schwarzer Eichenschreibtisch. Vor dem Schreibtisch stand ein eichener Hocker.\\ Und hinter dem Schreibtisch stand ein gut gepolsterter Thron, auf dem Albus Percival Wulfric Brian Dumbledore saß, geschmückt mit einem langen silbernen Bart, einem Hut, der wie ein zerquetschter Riesenpilz aussah, und etwas, das für Muggelaugen wie ein dreilagiger hellrosa Pyjama aussah.

Dumbledore lächelte, und seine hellen Augen funkelten mit einer wahnsinnigen Intensität. Mit einigem Bangen setzte sich Harry vor den Schreibtisch. Die Tür schwang mit einem lauten Knall hinter ihm zu.

„Hallo, Harry“, sagte Dumbledore.\\ „Hallo, Schulleiter“, erwiderte Harry.\\ Sie waren also beim Vornamen? Würde Dumbledore jetzt sagen, dass er ihn -\\ „Bitte, Harry!“, sagte Dumbledore.\\ „Schulleiter klingt so förmlich. Nennen Sie mich einfach kurz \emph{Bro}.„\\ „Das werde ich tun, \emph{Bro}“, sagte Harry.

Es entstand eine kleine Pause.\\ „Weißt du“, sagte Dumbledore,\\ „du bist der erste Mensch, der mich jemals beim Wort genommen hat.„

„Ah...“ sagte Harry. Er versuchte, seine Stimme zu kontrollieren, trotz des plötzlichen flauen Gefühls in seinem Magen.\\ „Es tut mir leid, ich, äh, Schulleiter, Sie haben mich gebeten, es zu tun, also habe ich -„

„Ha, bitte!“, sagte Dumbledore fröhlich.\\ „Es gibt keinen Grund, sich Sorgen zu machen, ich werde dich nicht aus dem Fenster werfen, nur weil du einen Fehler machst.\\ Ich warne dich vorher ausgiebig, wenn du etwas falsch machst! Außerdem kommt es nicht darauf an, wie die Leute mit dir reden, sondern was sie von dir denken.„

\emph{Er hat noch nie einem Schüler etwas zuleide getan, vergiss das nicht, dann wirst du sicher nicht in Panik geraten.}

Dumbledore zog ein kleines Metallkästchen hervor, klappte es auf und zeigte einige kleine gelbe Klumpen. „Zitronenbonbon?“, fragte der Schulleiter.

„Äh, nein, danke“, sagte Harry.\\ \emph{Zählt es als Verletzung, wenn man einem Schüler LSD unterjubelt, oder fällt das in die Kategorie harmloser Spaß?}\\ „Sie, ähm, sagten etwas darüber, dass ich zu jung sei, um die Worte der Macht und des Wahnsinns zu beschwören?„

„Das bist du ganz sicher!“ sagte Dumbledore.\\ „Zum Glück sind die Worte der Macht und des Wahnsinns vor sieben Jahrhunderten verloren gegangen und niemand hat mehr die geringste Ahnung, was sie sind. Es war nur eine kleine Bemerkung.„

„Ah...“ sagte Harry. Er war sich bewusst, dass ihm der Mund offen stand.\\ "Warum haben Sie mich dann hierher gerufen?„

„Warum?“ wiederholte Dumbledore.\\ „Ach, Harry, wenn ich den ganzen Tag herumlaufen und fragen würde, warum ich etwas tue, hätte ich nie die Zeit, auch nur eine einzige Sache zu erledigen!\\ Ich bin ein sehr beschäftigter Mensch, weißt du."

Harry nickte und lächelte.\\ „Ja, es ist eine sehr beeindruckende Liste.\\ Schulleiter von Hogwarts, Oberster Zauberer des Zaubergamots und Oberster Mugwump der Internationalen Konföderation der Zauberer.\\ Verzeihen Sie die Frage, aber ich frage mich, ob es möglich ist, mehr als sechs Stunden zu bekommen, wenn man mehr als einen Zeitumkehrer benutzt? Denn es ist ziemlich beeindruckend, wenn man das alles mit nur dreißig Stunden am Tag schafft.„

Wieder gab es eine kleine Pause, in der Harry weiter lächelte.\\ Er war ein wenig ängstlich, eigentlich sehr ängstlich, aber nachdem klar geworden war, dass Dumbledore sich absichtlich mit ihm anlegte, weigerte sich etwas in ihm absolut, das wie ein wehrloser Klumpen hinzunehmen.

„Ich fürchte, die Zeit mag es nicht, zu sehr gedehnt zu werden“, sagte Dumbledore nach der kleinen Pause, „und doch scheinen wir selbst ein wenig zu groß für sie zu sein, und so ist es ein ständiger Kampf, unser Leben in die Zeit einzupassen.„

„In der Tat“, sagte Harry mit ernster Ernsthaftigkeit.\\ „Deshalb ist es am besten, wenn wir schnell zu unserem Punkt kommen."

Einen Moment lang fragte sich Harry, ob er zu weit gegangen war.\\ Dann gluckste Dumbledore. „Direkt zur Sache, so sei es."\\ Der Schulleiter beugte sich vor, neigte seinen zerdrückten Pilzhut und strich seinen Bart gegen den Schreibtisch.\\ „Harry, an diesem Montag hast du etwas getan, das selbst mit einem Zeitumkehrer unmöglich gewesen wäre.\\ Oder besser gesagt, unmöglich mit nur einem Zeitumkehrer. Ich frage mich, woher die beiden Kuchen kommen?„

Ein Adrenalinstoß schoss durch Harry.\\ Er hatte das mit dem Unsichtbarkeitsumhang gemacht, der ihm zu Weihnachten zusammen mit einem Zettel geschenkt worden war, und auf diesem Zettel hatte gestanden:\\ \emph{Wenn Dumbledore eine Chance sähe, einen der Heiligtümer des Todes zu besitzen, würde er ihn} \emph{niemals aus den Augen lassen.}...

„Ein naheliegender Gedanke“, fuhr Dumbledore fort, „ist, dass, da keiner der anwesenden Erstklässler in der Lage war, einen solchen Zauber zu sprechen, jemand anderes anwesend war, und doch ungesehen.\\ Und wenn niemand dich sehen konnte, wäre es doch ein Leichtes für dich, die Torten zu werfen. Man könnte weiter vermuten, dass du, da du einen Zeitumkehrer hattest, der Unsichtbare warst; und dass du, da der Zauber der Desillusionierung weit über dein derzeitigen Fähigkeiten hinausgeht, einen Unsichtbarkeitsumhang hattest."\\ Dumbledore lächelte verschwörerisch.\\ „Bin ich so weit auf der richtigen Spur, Harry?"

Harry war wie erstarrt. Er hatte das Gefühl, dass eine unverblümte Lüge keineswegs klug und möglicherweise auch nicht im Geringsten hilfreich wäre, und ihm fiel nichts anderes ein.

Dumbledore winkte mit einer freundlichen Hand.\\ „Keine Sorge, Harry, du hast nichts falsch gemacht.\\ Unsichtbarkeitsmäntel sind nicht gegen die Regeln - ich nehme an, sie sind selten genug, dass niemand dazu gekommen ist, sie auf die Liste zu setzen. Aber eigentlich habe ich mich etwas ganz anderes gefragt.„

„Oh?“ sagte Harry mit der normalsten Stimme, die er zustande brachte.

Dumbledores Augen leuchteten vor Begeisterung.\\ „Siehst du, Harry, nachdem du ein paar Abenteuer erlebt hast, neigst du dazu, den Dreh rauszukriegen.\\ Man fängt an, das Muster zu erkennen, den Rhythmus der Welt zu hören. Man fängt an, einen Verdacht zu hegen, bevor der Moment der Enthüllung kommt.\\ Du bist der Junge, der gelebt hat, und irgendwie gelangte ein Unsichtbarkeitsmantel in deine Hände, nur vier Tage nachdem du unser magisches Britannien entdeckt hast.\\ Solche Umhänge stehen in der Winkelgasse nicht zum Verkauf, aber es gibt einen, der vielleicht seinen Weg zu einem auserwählten Träger findet.\\ Und so komme ich nicht umhin, mich zu fragen, ob du durch einen seltsamen Zufall nicht nur einen Unsichtbarkeitsumhang, sondern \emph{den Umhang der Unsichtbarkeit} gefunden haben, einen der drei Heiligtümer des Todes, der den Träger vor dem Blick des Todes selbst verbergen soll."\\ Dumbledores Blick war hell und begierig.\\ \emph{„Darf ich es sehen, Harry?„}

Harry schluckte. Der Adrenalinspiegel in seinem Körper war jetzt voll im Gange und es war völlig nutzlos, dies war der mächtigste Zauberer der Welt und es gab keine Chance, dass er es aus der Tür schaffte und es gab keinen Platz in Hogwarts, wo er sich verstecken konnte, wenn er es doch tat, er war dabei, den Umhang zu verlieren, der seit wer weiß wie langer Zeit durch die Potters weitergegeben worden war -

Langsam lehnte sich Dumbledore in seinen hohen Stuhl zurück.\\ Das helle Licht war aus seinen Augen verschwunden, und er sah verwirrt und ein wenig traurig aus.\\ „Harry“, sagte Dumbledore, „wenn du nicht willst, kannst du einfach nein sagen.„

„Ich kann?“ krächzte Harry.

„Ja, Harry“, sagte Dumbledore. Seine Stimme klang jetzt traurig und besorgt. „Es scheint, dass du Angst vor mir hast, Harry. Darf ich fragen, was ich getan habe, um dein Misstrauen zu verdienen?"

Harry schluckte. „Gibt es eine Möglichkeit, einen verbindlichen magischen Eid zu schwören, dass du meinen Mantel nicht nehmen wirst?"

Dumbledore schüttelte langsam den Kopf.\\ „Unbrechbare Schwüre sollte man nicht so leichtfertig einsetzen. Und außerdem, Harry, wenn du den Zauber nicht schon kennen würdest, hättest du nur mein Wort, dass der Spruch bindend ist.\\ Aber du weißt sicher, dass ich deine Erlaubnis nicht brauche, um den Umhang zu sehen. Ich bin mächtig genug, ihn selbst hervorzuholen, Beutel hin oder her."\\ Dumbledores Gesicht war sehr ernst.\\ „Aber das werde ich nicht tun. Der Umhang gehört dir, Harry. Ich werde ihn dir nicht wegnehmen. Nicht einmal, um ihn auch nur einen Moment lang anzusehen, es sei denn, du willst ihn mir zeigen.\\ Das ist ein Versprechen und ein Schwur. Sollte ich dir verbieten müssen, ihn auf dem Schulgelände zu benutzen, verlange ich, dass du zu deinem Tresor bei Gringotts gehst und ihn dort aufbewahrst.„

„Ah...“ sagte Harry. Er schluckte schwer und versuchte, die Flut von Adrenalin zu beruhigen und vernünftig zu denken.\\ Er nahm den Beutel von seinem Gürtel.\\ „Wenn du wirklich nicht meine Erlaubnis brauchst... dann hast du sie.„

Harry hielt Dumbledore den Beutel hin und \emph{biss sich fest auf die Lippe,} um sich selbst das Signal zu geben, falls seine Erinnerung gelöscht werden würde.

Der alte Zauberer griff in den Beutel und holte, ohne ein Wort zu sagen, den Unsichtbarkeitsumhang hervor.

„Ah“, hauchte Dumbledore. „Ich hatte recht..."\\ Er ließ das schwarz schimmernde Samtgewebe durch seine Hand gleiten.

„Jahrhunderte alt und immer noch so perfekt wie an dem Tag, an dem es gemacht wurde. Wir haben im Laufe der Jahre viel von unserer Kunst verloren, und jetzt kann selbst ich so etwas nicht mehr herstellen, niemand kann das.\\ Ich spüre seine Macht wie ein Echo in meinem Geist, wie ein Lied, das ewig gesungen wird, ohne dass es jemand hört...„\\ Der Zauberer blickte von dem Mantel auf.\\ „Verkaufe ihn nicht“, sagte er, „gebe ihn niemandem zum Besitz.\\ Überlege es dir zweimal, bevor du es jemandem zeigst, und überlege es dir noch dreimal, bevor du verrätst, dass es ein Heiligtum des Todes ist.\\ Behandle es mit Respekt, denn es ist in der Tat ein Ding der Macht."

Für einen Moment wurde Dumbledores Gesicht wehmütig......und dann reichte er Harry den Umhang zurück. Harry steckte ihn zurück in seinen Beutel. Dumbledores Miene war wieder ernst.\\ „Darf ich dich noch einmal fragen, Harry, wie du dazu gekommen bist, mir so zu misstrauen?„

Plötzlich fühlte sich Harry ziemlich beschämt.\\ „Da war ein Zettel mit dem Umhang“, sagte Harry mit leiser Stimme.\\ „Darin stand, dass du versuchen würdest, mir den Mantel wegzunehmen, wenn du es wüsstest.\\ Ich weiß allerdings nicht, wer den Zettel hinterlassen hat, wirklich nicht.„

„Ich... verstehe“, sagte Dumbledore langsam.\\ „Nun, Harry, ich will die Motive desjenigen, der dir den Zettel hinterlassen hat, nicht anzweifeln.\\ Wer weiß, ob sie nicht selbst die besten Absichten hatten? Immerhin haben sie dir den Umhang gegeben."

Harry nickte, beeindruckt von Dumbledores Wohltätigkeit und beschämt über den scharfen Kontrast zu seiner eigenen Einstellung.

Der alte Zauberer fuhr fort.\\ „Aber du und ich sind beide Spielfiguren der gleichen Farbe, denke ich.\\ Der Junge, der Voldemort schließlich besiegte, und der alte Mann, der ihn lange genug aufhielt, damit du den Tag retten konntest.\\ Ich werde dir deine Vorsicht nicht übel nehmen, Harry, wir müssen alle unser Bestes tun, um weise zu sein.\\ Ich werde dich nur bitten, zweimal und dreimal nachzudenken, wenn dir das nächste Mal jemand sagt, du sollst mir misstrauen.„

„Es tut mir leid“, sagte Harry.\\ Er fühlte sich in diesem Moment elend, er hatte gerade Gandalf im Wesentlichen verraten, und Dumbledores Freundlichkeit ließ ihn sich nur noch schlechter fühlen.\\ „Ich hätte dir nicht misstrauen dürfen."

„Leider, Harry, in dieser Welt..."\\ Der alte Zauberer schüttelte den Kopf.\\ „Ich kann nicht einmal sagen, dass du unklug warst. Du hast mich nicht gekannt. Und in Wahrheit gibt es einige in Hogwarts, denen du besser nicht trauen solltest. Vielleicht sogar einigen, die du Freunde nennst."

Harry schluckte. Das klang ziemlich unheilvoll.\\ „Wer zum Beispiel?„

Dumbledore erhob sich von seinem Stuhl und begann, eines seiner Instrumente zu untersuchen, ein Zifferblatt mit acht Zeigern unterschiedlicher Länge.\\ Nach ein paar Augenblicken sprach der alte Zauberer wieder.\\ „Er wirkt auf dich wahrscheinlich recht charmant“, sagte Dumbledore.\\ „Höflich - zumindest dir gegenüber. Wohlgesprochen, vielleicht sogar bewundernd. Immer bereit mit einer helfenden Hand, einem Gefallen, einem Ratschlag -„

„Oh, Draco Malfoy!“ sagte Harry und war ziemlich erleichtert, dass es nicht Hermine oder so war.\\ „Oh nein, nein, nein, nein, Sie haben das ganz falsch verstanden, er bekehrt nicht mich, sondern ich ihn."

Dumbledore erstarrte, wo er auf das Zifferblatt starrte. „Sie tust was?!„

„Ich werde Draco Malfoy von der dunklen Seite abwenden“, sagte Harry.\\ „Sie wissen schon, ihn zu einem guten Menschen machen.„

Dumbledore richtete sich auf und drehte sich zu Harry um.\\ Er trug einen der erstauntesten Gesichtsausdrücke, die Harry je bei jemandem gesehen hatte, geschweige denn bei jemandem mit einem langen Silberbart.

„Bist du sicher“, sagte der alte Zauberer nach einem Moment, „dass er bereit ist, erlöst zu werden? Ich fürchte, dass das Gute, das du in ihm zu sehen glaubst, nur Wunschdenken ist - oder schlimmer noch, ein Köder.„

„Äh, unwahrscheinlich“, sagte Harry.\\ „Ich meine, wenn er versucht, sich als guter Kerl zu tarnen, ist er unglaublich schlecht darin. Es geht nicht darum, dass Draco auf mich zukommt und ganz charmant ist und ich dann beschließe, dass er tief im Inneren einen verborgenen Kern von Güte haben muss. Ich habe ihn speziell für die Erlösung ausgewählt, weil er der Erbe des Hauses Malfoy ist und wenn man eine Person für die Erlösung auswählen müsste, wäre es offensichtlich er."

Dumbledores linkes Auge zuckte.\\ „Du beabsichtigst, die Saat der Liebe und Güte in Draco Malfoys Herz zu säen, weil du erwartest, dass der Erbe des Hauses Malfoy sich als wertvoll für dich erweist?„

„Nicht nur für mich!“ sagte Harry entrüstet.\\ „Für das ganze magische Britannien, wenn das klappt! Und er selbst wird ein glücklicheres und geistig gesünderes Leben haben!\\ Sehen Sie, ich habe nicht genug Zeit, um alle von der Dunklen Seite abzuwenden, und ich muss fragen, wo das Licht am schnellsten den größten Vorteil erlangen kann -„

Dumbledore begann zu lachen.\\ Er lachte viel heftiger, als Harry es erwartet hätte, fast heulend. Es wirkte geradezu würdelos. Ein alter und mächtiger Zauberer sollte in tiefen, dröhnenden Tönen kichern, nicht so sehr lachen, dass er nach Atem rang.\\ Harry war einmal buchstäblich von seinem Stuhl gefallen, als er den Marx Brothers-Film \emph{Duck Soup} gesehen hatte, und so sehr lachte Dumbledore jetzt.

„So lustig ist das nicht“, sagte Harry nach einer Weile. Er begann sich wieder Sorgen um Dumbledores Verstand zu machen.

Dumbledore bekam sich mit sichtlicher Mühe wieder unter Kontrolle.\\ „Ach, Harry, ein Symptom der Krankheit, die man Weisheit nennt, ist, dass man anfängt, über Dinge zu lachen, die niemand sonst lustig findet, denn wenn man weise ist, Harry, fängt man an, die Witze zu verstehen!"\\ Der alte Zauberer wischte sich die Tränen aus den Augen.\\ „Ah, ich. Ach, ich. Oft wird das Böse, dem Bösen selbst schaden."

Harrys Gehirn brauchte einen Moment, um die bekannten Worte einzuordnen...\\ „Hey, das ist ein Tolkien-Zitat! Gandalf sagt das!„

„Theoden, eigentlich“, sagte Dumbledore.

„Du bist ein Muggelgeborener?“ sagte Harry schockiert.

„Ich fürchte nicht“, sagte Dumbledore und lächelte wieder.\\ „Ich wurde siebzig Jahre vor der Veröffentlichung dieses Buches geboren, liebes Kind.\\ Aber es scheint, dass meine muggelstämmigen Schüler in gewissen Dingen ähnlich denken. Ich habe nicht weniger als zwanzig Exemplare von \emph{Der Herr der Ringe} und drei Sätze von Tolkiens gesamten gesammelten Werken angesammelt, und ich schätze jedes einzelne davon."\\ Dumbledore zog seinen Zauberstab, hielt ihn hoch und nahm eine Pose ein.\\ „\textbf{Du kannst nicht vorbei!} Wie sieht das aus?„

„Ah“, sagte Harry in etwas, das einer kompletten Hirnabschaltung nahekommt, „ich glaube, da fehlt ein Balrog."

\emph{Und der rosa Pyjama und der zerquetschte Pilzhut waren nicht im Geringsten hilfreich.}

„Ich verstehe."\\ Dumbledore seufzte und steckte den Zauberstab mürrisch in seinen Gürtel.\\ „Ich fürchte, in letzter Zeit gab es nur noch wenige Balrogs in meinem Leben.\\ Heutzutage geht es nur noch um Sitzungen des Zaubergamot, bei denen ich verzweifelt versuchen muss, zu verhindern, dass irgendeine Arbeit erledigt wird, und um formelle Abendessen, bei denen ausländische Politiker darum wetteifern, wer der hartnäckigste Narr sein kann.\\ Und geheimnisvoll zu sein bei Leuten, Dinge zu wissen, die ich nicht wissen kann, kryptische Aussagen zu machen, die nur im Nachhinein verstanden werden können, und all die anderen kleinen Arten, mit denen sich mächtige Zauberer amüsieren, nachdem sie den Teil des Musters verlassen haben, der es ihnen erlaubt, Helden zu sein.\\ Wo wir gerade dabei sind, Harry, ich habe dir etwas zu geben, etwas, das deinem Vater gehörte.„

„Wirklich?“, sagte Harry.\\ „Donnerwetter, wer hätte das gedacht.„

„Ja, in der Tat“, sagte Dumbledore.\\ „Ich nehme an, es ist ein wenig vorhersehbar, nicht wahr?"\\ Sein Gesicht wurde feierlich.\\ „Nichtsdestotrotz...„\\ Dumbledore ging zurück zu seinem Schreibtisch und setzte sich, wobei er eine der Schubladen herauszog.\\ Er griff mit beiden Armen hinein und zog unter leichter Anstrengung einen ziemlich großen und schwer aussehenden Gegenstand aus der Schublade, den er mit einem gewaltigen Knall auf seinem eichenen Schreibtisch ablegte.\\ „Das“, sagte Dumbledore, „war der Stein deines Vaters."

Harry starrte ihn an. Er war hellgrau, verfärbt, unregelmäßig geformt, scharfkantig und ein ganz gewöhnlicher, großer Stein.\\ Dumbledore hatte ihn so abgelegt, dass er auf dem breitesten verfügbaren Querschnitt ruhte, aber er wackelte immer noch unstabil auf seinem Schreibtisch.\\ Harry schaute auf.\\ „Das ist ein Scherz, oder?„

„Ist es nicht“, sagte Dumbledore, schüttelte den Kopf und sah sehr ernst aus.\\ „Ich habe ihn aus den Ruinen von James und Lilys Haus in Godric's Hollow mitgenommen, wo ich auch dich gefunden habe; und ich habe ihn von da an bis heute aufbewahrt, bis zu dem Tag, an dem ich ihn dir geben konnte."

In dem Gemisch von Hypothesen, das Harry als Weltmodell diente, stieg die Wahrscheinlichkeit von Dumbledores Unzurechnungsfähigkeit rapide an.\\ Aber es gab immer noch eine beträchtliche Menge an Wahrscheinlichkeiten für andere Alternativen...

„Ähm, ist es ein magischer Stein?„

„Nicht, soweit ich weiß“, sagte Dumbledore.\\ „Aber ich rate dir dringend, ihn immer bei dir zu tragen."

\emph{Also gut.\\ Dumbledore war wahrscheinlich verrückt, aber wenn er es nicht war... nun, es wäre einfach zu} \emph{peinlich, Ärger zu bekommen, wenn man den Rat des undurchschaubaren alten Zauberers ignorierte.}

\emph{Das musste so ungefähr Platz 4 auf der Liste der Top 100 offensichtlichen Versagensarten sein.}

Harry trat vor und legte seine Hände auf den Stein, um einen Winkel zu finden, aus dem er ihn anheben konnte, ohne sich zu schneiden.

„Ich stecke ihn dann in meinen Beutel."

Dumbledore runzelte die Stirn.\\ „Das ist vielleicht nicht nah genug an deiner Person.\\ Und was ist, wenn dein Beutel verloren geht oder gestohlen wird?"

„Sie meinen, ich soll einfach einen großen Stein überall mit hinnehmen?"

Dumbledore warf Harry einen ernsten Blick zu.\\ „Das könnte sich als klug erweisen.„

„Ah...“ sagte Harry.\\ Er sah ziemlich schwer aus.\\ „Ich denke, die anderen Schüler würden dazu neigen, mir Fragen zu stellen.„

„Sag ihnen, dass ich es dir befohlen habe“, sagte Dumbledore.\\ „Das wird niemand in Frage stellen, denn sie halten mich alle für verrückt."\\ Sein Gesicht war immer noch vollkommen ernst.

„Äh, um ehrlich zu sein, wenn Sie herumgehen und Ihren Schülern befehlen, große Steine zu tragen, kann ich irgendwie verstehen, warum die Leute das denken.„

„Ah, Harry“, sagte Dumbledore.\\ Der alte Zauberer machte eine Geste, eine Bewegung mit einer Hand, die all die geheimnisvollen Instrumente im Raum zu erfassen schien.

„Wenn wir jung sind, glauben wir, dass wir alles wissen, und so glauben wir, dass, wenn wir keine Erklärung für etwas sehen, dann gibt es auch keine Erklärung.\\ Wenn wir älter sind, erkennen wir, dass das ganze Universum nach einem Rhythmus und einem Grund funktioniert, auch wenn wir es selbst nicht wissen.\\ Es ist nur unsere eigene Unwissenheit, die uns als Irrsinn erscheint.„

„Die Wirklichkeit ist immer gesetzmäßig“, sagte Harry, „auch wenn wir das Gesetz nicht kennen.„

„Ganz genau, Harry“, sagte Dumbledore.\\ „Dies zu verstehen - und ich sehe, dass du es verstehst - ist die Essenz der Weisheit."

„Also... warum genau muss ich diesen Stein tragen?„

„Mir fällt eigentlich kein Grund ein“, sagte Dumbledore.

„... es fällt.... ihnen keiner ein....“

Dumbledore nickte.\\ „Aber nur weil mir kein Grund einfällt, heißt das nicht, dass es keinen Grund gibt.“ Die Instrumente tickten weiter.

„Okay“, sagte Harry,\\ „ich bin mir nicht einmal sicher, ob ich das sagen sollte, aber das ist einfach nicht die richtige Art, mit unserer zugegebenen Unwissenheit darüber umzugehen, wie das Universum funktioniert.„

„Ist es nicht?“, sagte der alte Zauberer und sah überrascht und enttäuscht aus.\\ Harry hatte das Gefühl, dass dieses Gespräch nicht zu seinen Gunsten verlaufen würde, aber er fuhr trotzdem fort.

„Nein. Ich weiß nicht einmal, ob dieser Trugschluss einen offiziellen Namen hat, aber wenn ich mir selbst einen ausdenken müsste, dann wäre es \emph{'Privilegierung der Hypothese}' oder so ähnlich.\\ Wie kann ich das formal ausdrücken... ähm...

nehmen wir an, Sie hätten eine Million Schachteln, und nur eine der Schachteln enthielte einen Diamanten.\\ Und Sie hätten eine Kiste voller Diamantendetektoren, und jeder Diamantendetektor löste immer aus, wenn ein Diamant vorhanden ist, und löste die Hälfte der Zeit bei Kisten aus, in denen kein Diamant ist.\\ Wenn Sie zwanzig Detektoren über alle Kisten laufen ließen, hätten Sie im Durchschnitt einen falschen Kandidaten und einen wahren Kandidaten übrig.\\ Und dann bräuchte man nur noch ein oder zwei weitere Detektoren, bis man den einen richtigen Kandidaten übrig hätte.

Der Punkt ist, dass, wenn es viele mögliche Antworten gibt, die meisten Beweise, die Sie brauchen, nur dazu dienen, die wahre Hypothese aus Millionen von Möglichkeiten zu finden - und sie Ihnen überhaupt erst zur Kenntnis zu bringen.\\ Die Menge an Beweisen, die Sie benötigen, um zwischen zwei oder drei plausiblen Kandidaten zu entscheiden, ist im Vergleich dazu viel kleiner.

Wenn Sie also einfach ohne Beweise vorpreschen und eine bestimmte Möglichkeit in den Mittelpunkt Ihrer Aufmerksamkeit rücken, überspringen Sie den größten Teil der Arbeit.

Zum Beispiel leben Sie in einer Millionenstadt, und es gibt einen Mord, und ein Detektiv sagt, \emph{na ja, wir haben überhaupt keine Beweise, also haben wir die Möglichkeit in Betracht gezogen, dass Mortimer Snodgrass es getan hat?„}

„Hat er?“, fragte Dumbledore.

„Nein“, sagte Harry.\\ „Aber später stellt sich heraus, dass der Mörder schwarzes Haar hatte, und Mortimer hat schwarzes Haar, also sagen alle, ah, sieht aus, als hätte Mortimer es doch getan.\\ Es ist also unfair gegenüber Mortimer, wenn die Polizei auf ihn aufmerksam macht, ohne bereits gute Gründe zu haben, ihn zu verdächtigen.

Wenn es viele Möglichkeiten gibt, besteht die meiste Arbeit darin, die wahre Antwort ausfindig zu machen - und damit zu beginnen, ihr Aufmerksamkeit zu schenken.

Sie brauchen keine Beweise oder die Art von offiziellen Beweisen, die Wissenschaftler oder Gerichte verlangen, aber Sie brauchen eine Art von Hinweis, und dieser Hinweis muss diese spezielle Möglichkeit von den Millionen anderer unterscheiden.\\ Andernfalls kann man die richtige Antwort nicht einfach aus dem Nichts herauszupflücken. Sie können nicht einmal eine Möglichkeit, über die es sich lohnt nachzudenken, aus der Luft pflücken.

Und es muss eine Million anderer Dinge geben, die ich tun könnte, außer den Stein meines Vaters herumzutragen.\\ Nur weil ich unwissend über das Universum bin, heißt das nicht, dass ich unsicher bin, wie ich in der Gegenwart meiner Ungewissheit denken soll.\\ Die Gesetze für das Denken mit Wahrscheinlichkeiten sind nicht weniger eisern als die Gesetze, die die altmodische Logik regieren, und was Sie gerade getan haben, ist nicht erlaubt."

Harry hielt inne.\\ „Es sei denn, Sie haben einen Hinweis, den Sie nicht erwähnen.„

„Ah“, sagte Dumbledore. Er tippte sich an die Wange und sah nachdenklich aus.\\ „Ein interessantes Argument, gewiss, aber bricht es nicht an dem Punkt zusammen, an dem du eine Analogie herstellst zwischen einer Million potenzieller Mörder, von denen nur einer den Mord begangen hat,\\ \emph{und dem Ergreifen einer von vielen möglichen Handlungsweisen, wenn viele mögliche Handlungsweisen alle klug sein könnten?}\\ Ich sage nicht, dass das Tragen des Steins deines Vaters die einzig mögliche Handlungsweise ist, sondern nur, dass es klüger ist, es zu tun als nicht.„

Dumbledore griff erneut in dieselbe Schreibtischschublade, auf die er vorhin zugegriffen hatte, diesmal schien er darin herumzuwühlen - zumindest schien sich sein Arm zu bewegen.

„Ich möchte anmerken“, sagte Dumbledore, während Harry noch überlegte, wie er auf diese völlig unerwartete Erwiderung antworten sollte,\\ „dass es ein weit verbreiteter Irrglaube der Ravenclaws ist, dass alle klugen Kinder dorthin sortiert werden und keine für andere Häuser übrig bleiben.\\ Dem ist nicht so; nach Ravenclaw sortiert zu sein, zeigt, dass man von dem Wunsch getrieben wird, Dinge zu wissen, was keineswegs dieselbe Eigenschaft ist wie intelligent zu sein."\\ Der Zauberer lächelte, als er sich über die Schublade beugte.\\ „Nichtsdestotrotz scheinst du ziemlich intelligent zu sein. Weniger wie ein gewöhnlicher junger Held und mehr wie ein junger geheimnisvoller alter Zauberer.\\ Ich denke, dass ich bei dir vielleicht den falschen Ansatz gewählt habe, Harry, und dass du vielleicht in der Lage bist, Dinge zu verstehen, die nur wenige andere begreifen können.\\ Also werde ich kühn sein und dir ein gewisses anderes Erbstück anbieten.„

„Sie meinen doch nicht etwa...“, keuchte Harry. „Mein Vater... besaß einen anderen Stein?„

„Entschuldige mal“, sagte Dumbledore,\\ „ich bin immer noch älter und geheimnisvoller als du, und wenn es irgendwelche Enthüllungen zu machen gibt, dann werde ich das Enthüllen übernehmen, danke... oh, wo ist das Ding!„\\ Dumbledore griff weiter in die Schreibtischschublade, und noch weiter. Sein Kopf, seine Schultern und sein ganzer Oberkörper verschwanden darin, bis nur noch seine Hüften und Beine herausschauten, als ob die Schreibtischschublade ihn verschlingen würde.

Harry konnte nicht anders, als sich zu fragen, wie viel Zeug da drin war und wie das komplette Inventar aussehen würde.\\ Schließlich erhob sich Dumbledore wieder aus der Schublade und hielt das Ziel seiner Suche in der Hand, das er neben dem Stein auf dem Schreibtisch abstellte.

Es war ein gebrauchtes, ausgefranstes und abgenutztes Lehrbuch: Zaubertränke für Fortgeschrittene von Libatius Borage.\\ Auf dem Einband war das Bild einer rauchenden Phiole zu sehen.

„Das“, sagte Dumbledore, „war das Zaubertrankbuch deiner Mutter im fünften Jahr.„

„Das ich immer bei mir tragen muss“, sagte Harry.

„Es birgt ein schreckliches Geheimnis. Ein Geheimnis, dessen Enthüllung sich als so verhängnisvoll erweisen könnte, dass ich dich bitten muss zu schwören - und ich verlange von dir, dass du es ernsthaft schwörst, Harry, was immer du auch von all dem halten magst - niemals irgendjemandem oder irgendetwas anderes zu erzählen.„

Harry betrachtete das Zaubertränke-Lehrbuch seiner Mutter aus dem fünften Jahr, das offenbar ein schreckliches Geheimnis enthielt.\\ Das Problem war, dass Harry solche Schwüre sehr ernst nahm. Jeder Schwur war ein unbrechbarer Schwur, wenn er von der richtigen Sorte Mensch geleistet wurde.\\ Und...\\ „Ich fühle mich durstig“, sagte Harry, „und das ist überhaupt kein gutes Zeichen.„

Dumbledore unterließ es völlig, nach dieser kryptischen Aussage zu fragen.\\ „Schwörst du, Harry?“, fragte Dumbledore. Seine Augen blickten aufmerksam in die von Harry.\\ „Sonst kann ich es dir nicht sagen.„

„Ja“, sagte Harry. „Ich schwöre.„\\ Das war das Problem, wenn man ein Ravenclaw war. Man konnte so ein Angebot nicht ablehnen, sonst hätte einen die Neugier bei lebendigem Leib aufgefressen, und alle anderen wussten es.

„Und ich schwöre im Gegenzug“, sagte Dumbledore,\\ "dass das, was ich dir jetzt sage, die Wahrheit ist.„\\ Dumbledore öffnete das Buch, scheinbar wahllos, und Harry beugte sich vor, um zu sehen.

„Siehst du diese Notizen“, sagte Dumbledore mit einer Stimme, die so leise war, dass sie fast ein Flüstern war,\\ „die an den Rändern des Buches geschrieben sind?„

Harry blinzelte leicht.\\ Die vergilbten Seiten schienen etwas zu beschreiben, das man einen Trank der Adlerpracht nannte, wobei viele der Zutaten Gegenstände waren, die Harry überhaupt nicht kannte und deren Namen nicht aus dem Englischen zu stammen schienen.\\ Am Rand war eine handschriftliche Anmerkung hingekritzelt, die besagte:\\ \emph{Ich frage mich, was passieren würde, wenn man hier Thestralblut anstelle von Blaubeeren verwenden würde.}\\ .. und direkt darunter stand eine Antwort in anderer Handschrift:\\ \emph{Man würde wochenlang krank werden und vielleicht sterben.}

„Ich sehe sie“, sagte Harry. „Was ist mit ihnen?„

Dumbledore zeigte auf das zweite Gekritzel.\\ „Die in dieser Handschrift“, sagte er, immer noch mit dieser tiefen Stimme,\\ „wurden von deiner Mutter geschrieben.\\ Und die in dieser Handschrift“, er deutete mit dem Finger auf das erste Gekritzel,\\ „stammen von mir.\\ Ich habe mich unsichtbar gemacht und mich in ihren Schlafsaal geschlichen, während sie schlief. Lily dachte, einer ihrer Freunde würde sie schreiben und sie hatten die lustigsten Streitereien."

Das war genau der Punkt, an dem Harry erkannte, dass der Schulleiter von Hogwarts tatsächlich verrückt war.

Dumbledore sah ihn mit einem ernsten Gesichtsausdruck an.\\ „Verstehst du die Tragweite dessen, was ich dir gerade gesagt habe, Harry?"

„Ehhh..."\\ sagte Harry. Seine Stimme schien wie festgeklebt zu sein.\\ „Tut mir leid... Ich... nicht wirklich...„

„Ach so“, sagte Dumbledore und seufzte.\\ „Ich nehme an, deine Klugheit hat also doch Grenzen. Sollen wir alle so tun, als hätte ich nichts gesagt?„

Harry erhob sich von seinem Stuhl und setzte ein starres Lächeln auf.\\ „Natürlich“, sagte Harry.\\ „Wissen Sie, es ist schon ziemlich spät am Tag und ich bin ein bisschen hungrig, also sollte ich eigentlich runter zum Essen gehen“, und Harry machte sich auf den Weg zur Tür.

Der Türknauf ließ sich überhaupt nicht drehen.

„Du hast mich verletzt, Harry“, sagte Dumbledores Stimme in leisem Ton, der direkt von hinten kam.\\ „Siehst du nicht wenigstens ein, dass das, was ich dir gesagt habe, ein Zeichen des Vertrauens ist?„

Harry drehte sich langsam um.\\ Vor ihm stand ein sehr mächtiger und sehr verrückter Zauberer mit einem langen silbernen Bart, einem Hut, der wie ein zerquetschter Riesenpilz aussah, und der etwas trug, was für Muggelaugen wie drei Lagen eines leuchtend rosa Pyjamas aussah.\\ Hinter ihm war eine Tür, die im Moment nicht zu funktionieren schien. Dumbledore sah ziemlich traurig und müde aus, als ob er sich auf einen Zauberstab stützen wollte, den er nicht hatte.

„Wirklich“, sagte Dumbledore,\\ „wenn man etwas Neues ausprobiert, anstatt hundertzehn Jahre lang immer demselben Muster zu folgen, fangen die Leute an wegzulaufen."\\ Der alte Zauberer schüttelte bedauernd den Kopf.\\ „Ich hatte mir mehr von dir erhofft, Harry Potter. Ich hatte gehört, dass auch deine eigenen Freunde dich für verrückt halten.\\ Ich weiß, dass sie sich irren. Wirst du nicht dasselbe von mir glauben?„

„Bitte öffnen Sie die Tür“, sagte Harry, und seine Stimme zitterte.\\ „Wenn du willst, dass ich dir jemals wieder vertraue, mach die Tür auf.„

Hinter ihm hörte er, wie sich eine Tür öffnete.

„Es gab noch mehr Dinge, die ich dir sagen wollte“, sagte Dumbledore,\\ „und wenn du jetzt gehst, wirst du nicht erfahren, was sie waren."

Manchmal hasste Harry es absolut, ein Ravenclaw zu sein.

\emph{Er hat noch nie einen Schüler verletzt,} sagte Harrys Gryffindor-Seite.\\ \emph{Denk immer daran und du wirst sicher nicht in Panik geraten.}

\emph{Du wirst doch nicht weglaufen, nur weil die Dinge interessant werden, oder? Du kannst nicht einfach vor dem Schulleiter weglaufen!} sagte der Hufflepuff-Teil.\\ \emph{Was ist, wenn er anfängt, Hauspunkte abzuziehen? Er könnte dir das Schulleben sehr schwer machen, wenn er entscheidet, dass er dich nicht mag!}

Und ein Teil von ihm, den Harry nicht besonders mochte, aber nicht ganz zum Schweigen bringen konnte, grübelte über die möglichen Vorteile nach, einer der wenigen Freunde dieses verrückten alten Zauberers zu sein, der zufällig auch noch Schulleiter, Oberster Hexenmeister und oberster Mugwump war.

Und leider schien sein innerer Slytherin viel besser als Draco darin zu sein, Leute auf die dunkle Seite zu ziehen, denn er sagte Dinge wie:\\ „\emph{Armer Kerl, er sieht aus, als bräuchte er jemanden zum Reden, nicht wahr?"}\\ und\\ \emph{"Du würdest doch nicht wollen, dass ein so mächtiger Mann jemandem vertraut, der weniger tugendhaft ist, oder?"}\\ und\\ \emph{„Ich frage mich, was für unglaubliche Geheimnisse Dumbledore dir erzählen könnte, wenn du dich mit ihm anfreundest, und ich wette sogar, er hat eine wirklich interessante Büchersammlung.}

\emph{Ihr seid alle ein Haufen Verrückter,} dachte Harry über die ganze Versammlung, aber er wurde einstimmig von jedem Teil seiner selbst überstimmt.\\ Harry drehte sich um, machte einen Schritt auf die offene Tür zu, streckte die Hand aus und schloss sie absichtlich wieder.\\ Es war ein unnötiges Opfer, da er sowieso hier bleiben würde, Dumbledore konnte seine Bewegungen trotzdem kontrollieren, aber vielleicht würde es Dumbledore beeindrucken.\\ Als Harry sich wieder umdrehte, sah er, dass der mächtige, wahnsinnige Zauberer wieder lächelte und freundlich aussah.\\ Das war gut, vielleicht.

„Bitte tun Sie das nicht wieder“, sagte Harry. „Ich mag es nicht, in der Falle zu sitzen.„

„Das tut mir leid, Harry“, sagte Dumbledore in einem Ton, der nach aufrichtiger Entschuldigung klang.\\ „Aber es wäre furchtbar unklug gewesen, dich ohne den Felsen deines Vaters gehen zu lassen.„

„Natürlich“, sagte Harry.\\ „Es war unvernünftig von mir, zu erwarten, dass sich die Tür öffnet, bevor ich die Questgegenstände in mein Inventar gelegt habe.„

Dumbledore lächelte und nickte.\\ Harry ging zum Schreibtisch hinüber, drehte seinen Beutel vorne an seinem Gürtel um und schaffte es mit einiger Anstrengung, den Stein in seinen elfjährigen Armen hochzuheben und hineinzulegen.\\ Er konnte tatsächlich spüren, wie das Gewicht langsam abnahm, als der Zauber mit der Lippenvergrößerung den Stein verschlang, und das darauf folgende Rülpsen war ziemlich laut und hatte einen deutlich klagenden Klang.

Das Zaubertränke-Lehrbuch seiner Mutter aus dem fünften Jahr (das ein Geheimnis enthielt, das in der Tat ziemlich schrecklich war) folgte kurz darauf.

Und dann machte Harrys innerer Slytherin einen schlitzohrigen Vorschlag, um sich beim Schulleiter einzuschmeicheln, der leider perfekt auf die Unterstützung der Ravenclaw-Mehrheitsfraktion abgestimmt war.

„Also“, sagte Harry.\\ „Ähm. Da ich schon mal hier bin, nehme ich nicht an, dass Sie mir eine kleine Führung durch Ihr Büro geben wollen? Ich bin ein bisschen neugierig, was es mit einigen dieser Dinge auf sich hat“, \emph{und das war seine Untertreibung für den Monat September.}

Dumbledore blickte ihn an und nickte dann mit einem leichten Grinsen.\\ „Ich fühle mich durch dein Interesse geschmeichelt“, sagte Dumbledore,\\ „aber ich fürchte, da gibt es nicht viel zu sagen."\\ Dumbledore trat einen Schritt näher an die Wand heran und zeigte auf ein Gemälde eines schlafenden Mannes.

„Das sind Porträts von früheren Schulleitern von Hogwarts."\\ Er drehte sich um und zeigte auf seinen Schreibtisch.\\ „Das ist mein Schreibtisch."\\ Er deutete auf seinen Stuhl.\\ „Das ist mein Stuhl -„

„Entschuldigen Sie“, sagte Harry, „eigentlich habe ich mich über die hier gewundert.„\\ Harry zeigte auf einen kleinen Würfel, der leise\\ „blopp... blopp... blopp“ flüsterte.

„Oh, die kleinen fummeligen Dinger?“, sagte Dumbledore.\\ „Sie kamen mit dem Büro des Schulleiters und ich habe absolut keine Ahnung, was die meisten von ihnen tun.\\ Obwohl dieses Zifferblatt mit den acht Zeigern die Anzahl der, nennen wir sie mal Nieser, von linkshändigen Hexen innerhalb der Grenzen Frankreichs zählt, du würdest nicht glauben, wie viel Arbeit es gekostet hat, das herauszufinden.\\ Und das hier mit den goldenen Wibblern ist meine eigene Erfindung und Minerva wird nie und nimmer herausfinden, was es macht."

Dumbledore ging einen Schritt hinüber zur Hutablage, während Harry dies noch verarbeitete.

„Hier haben wir natürlich den Sortierhut, ich glaube, ihr beide kennt euch schon. Er hat mir gesagt, dass er unter keinen Umständen mehr auf deinen Kopf gesetzt werden darf.\\ Du bist erst der vierzehnte Schüler in der Geschichte, über den das gesagt wird, Baba Yaga war eine andere und von den anderen zwölf werde ich dir erzählen, wenn du älter bist.\\ Das ist ein Regenschirm. Das ist ein anderer Regenschirm."\\ Dumbledore machte noch ein paar Schritte und drehte sich um, jetzt breit lächelnd.\\ „Und natürlich wollen die meisten Leute, die in mein Büro kommen, Fawkes sehen."\\ Dumbledore stand nun neben dem Vogel auf dem goldenen Podest.

Harry kam etwas verwirrt herüber. „Das ist Fawkes?„

„Fawkes ist ein Phönix“, sagte Dumbledore. „Ein sehr seltenes, sehr mächtiges magisches Geschöpf."

„Ah..."\\ sagte Harry. Er senkte den Kopf und starrte in die winzigen, glänzenden schwarzen Augen, die nicht das geringste Anzeichen von Kraft oder Intelligenz zeigten.

„Ahhh..."\\ sagte Harry wieder. Er war sich ziemlich sicher, dass er die Form des Vogels erkannte.\\ Es war ziemlich schwer zu übersehen.\\ "Hmm..."

\textbf{\emph{Sag etwas Intelligentes!}} brüllte Harrys Verstand vor sich hin.\\ \textbf{\emph{Steh nicht einfach da und klinge wie ein schwafelnder Idiot!}}\\ \emph{Was zum Teufel soll ich denn sagen?} feuerte Harrys Verstand zurück.\\ \textbf{\emph{Irgendetwas!}}\\ \emph{Du meinst, irgendwas außer „Fawkes ist ein Huhn"}\\ \textbf{\emph{- Ja! Irgendwas anderes als das!}}

„Also, ah, welche Art von Magie können Phönixe denn?„

„Ihre Tränen haben die Kraft zu heilen“, sagte Dumbledore.\\ „Sie sind Geschöpfe des Feuers und bewegen sich zwischen allen Orten so leicht, wie Feuer an einem Ort erlöschen und an einem anderen entfacht werden kann.\\ Die enorme Belastung durch ihre angeborene Magie lässt ihre Körper schnell altern, und doch sind sie so nahe an der Unsterblichkeit wie kein anderes Wesen auf dieser Welt, denn wann immer ihr Körper sie im Stich lässt, verbrennt er in einem Feuerstoß und hinterlässt ein Junges oder manchmal auch ein Ei."

Dumbledore trat näher und begutachtete das Huhn, wobei er die Stirn runzelte.

„Hm... sieht ein bisschen alt aus, würde ich sagen.„\\ Als sich diese Aussage in Harrys Kopf festsetzte, stand das Huhn bereits in Flammen.\\ Der Schnabel des Huhns öffnete sich, aber es hatte nicht einmal Zeit für ein einziges Krächzen, bevor es zu verkohlen begann.\\ Die Flamme war kurz, intensiv und völlig eigenständig; es gab keinen Brandgeruch. Und dann erlosch das Feuer nur Sekunden, nachdem es begonnen hatte, und hinterließ einen winzigen, erbärmlichen Haufen Asche auf der goldenen Plattform.

„Schau nicht so entsetzt, Harry!“, sagte Dumbledore. „Fawkes ist nicht verletzt worden."

\emph{Dumbledores Hand tauchte in eine Tasche, und dann durchwühlte dieselbe Hand die Asche und förderte ein kleines gelbliches Ei zutage.}

„Schau, hier ist ein Ei!"

„Oh... wow... erstaunlich...„

„Aber jetzt sollten wir wirklich weitermachen“, sagte Dumbledore.\\ Er ließ das Ei in der Asche des Huhns zurück, kehrte zu seinem Thron zurück und setzte sich hin.\\ „Es ist schließlich fast Zeit für das Abendessen, und wir wollen doch nicht unsere Zeitumkehrer benutzen müssen.„

In der Regierung von Harry war ein heftiger Machtkampf im Gange.\\ Slytherin und Hufflepuff hatten die Seiten gewechselt, nachdem sie gesehen hatten, wie der Schulleiter von Hogwarts ein Huhn in Brand setzte.

„Ja, Sachen“, kam es über Harrys Lippen. „Und dann das Abendessen.„\\ \textbf{\emph{Du klingst schon wieder wie ein schwafelnder Schwachkopf,}} bemerkte Harrys innerer Kritiker.

„Nun“, sagte Dumbledore. „Ich fürchte, ich muss dir ein Geständnis machen, Harry. Ein Geständnis und eine Entschuldigung.„

„Entschuldigungen sind gut“,\\ \textbf{\emph{das ergibt nicht mal einen Sinn! Was rede ich denn da?}}

Der alte Zauberer seufzte tief.\\ „Du denkst vielleicht nicht mehr so, nachdem du verstanden hast, was ich zu sagen habe.\\ Ich fürchte, Harry, dass ich dich dein ganzes Leben lang manipuliert habe. Ich war es, der dich in die Obhut deiner bösen Stiefeltern gegeben hat.„

„Meine Stiefeltern sind nicht böse!“, platzte Harry heraus. „Meine Eltern, meine ich!"

„Sind sie nicht?"\\ sagte Dumbledore und sah überrascht und enttäuscht aus.\\ „Nicht einmal ein bisschen böse? Das passt nicht ins Schema...„

Harrys innerer Slytherin schrie aus vollem Halse:\\ „\textbf{\emph{Halt die Klappe, du Idiot, er wird dich ihnen wegnehmen!}} „\\ Nein, nein“, sagte Harry, die Lippen zu einer grässlichen Grimasse gefroren,\\ „ich wollte nur Ihre Gefühle schonen, sie sind eigentlich sehr böse..."

„Sind sie das?"\\ Dumbledore beugte sich vor und starrte ihn aufmerksam an.\\ „Was machen sie denn?"

Er sprach schnell:\\ „Sie, ah, ich muss abwaschen und Probleme lösen und sie lassen mich nicht viele Bücher lesen und -„

„Ah, gut, das ist gut zu hören“, sagte Dumbledore und lehnte sich wieder zurück.\\ Er lächelte auf eine traurige Art und Weise.\\ „Dann entschuldige ich mich dafür. Also, wo war ich? Ah, ja.\\ Es tut mir leid, Harry, dass ich für so gut wie alles Schlechte verantwortlich bin, das dir je widerfahren ist.\\ Ich weiß, dass dich das wahrscheinlich sehr wütend machen wird.„

„Ja, ich bin sehr wütend!“, sagte Harry. „\textbf{Grrr}!„\\ Harrys innerer Kritiker verlieh ihm prompt den Preis für die schlechteste Schauspielerei in der Geschichte aller Zeiten.

„Und ich wollte dich nur wissen lassen“, sagte Dumbledore,\\ „ich wollte dir so früh wie möglich sagen, für den Fall, dass einem von uns später etwas zustößt, dass es mir wirklich, wirklich leid tut. Für alles, was bereits geschehen ist, und alles, was noch geschehen wird.„\\ In den Augen des alten Zauberers glitzerte es feucht.

„Und ich bin sehr wütend!“, sagte Harry.\\ „So wütend, dass ich auf der Stelle gehen möchte, es sei denn, Sie haben noch etwas zu sagen!„\\ \textbf{\emph{Geh einfach, bevor er dich in Brand steckt!}} schrien Slytherin, Hufflepuff und Gryffindor.

„Ich verstehe“, sagte Dumbledore.\\ „Dann noch eine letzte Sache, Harry.\\ Du darfst nicht versuchen, die verbotene Tür im Korridor des dritten Stocks zu öffnen.\\ Es ist unmöglich, dass du durch all die Fallen kommst, und ich möchte nicht hören, dass du dich beim Versuch verletzt.\\ Ich bezweifle, dass du auch nur die erste Tür öffnen könntest, da sie verschlossen ist und du den Zauberspruch \emph{Alohomora} nicht kennst -„

Harry wirbelte herum und raste mit Höchstgeschwindigkeit zum Ausgang, der Türknauf drehte sich in seiner Hand und dann rannte er die Wendeltreppe hinunter, noch während sie sich drehte, seine Füße stolperten fast über sich selbst, in einem Augenblick war er unten und der Wasserspeier ging zur Seite und Harry schoss wie eine Kanonenkugel aus dem Treppenhaus.

\textbf{Donnerstag}.\\ Irgendetwas musste es mit Harry Potter auf sich haben. Es war schließlich Donnerstag für alle, und doch schien so etwas niemandem sonst zu passieren.\\ Es war 18:21 Uhr am Donnerstagnachmittag, als Harry Potter, wie eine Kanonenkugel aus dem Treppenhaus schießend und mit Höchstgeschwindigkeit beschleunigend, direkt in Minerva McGonagall rannte, als sie auf dem Weg zum Büro des Schulleiters um eine Ecke bog.\\ Glücklicherweise wurde keiner von ihnen schwer verletzt. Wie man Harry etwas früher am Tag erklärt hatte\\ - damals weigerte er sich noch, auch nur in die Nähe eines Besens zu gehen -,\\ brauchte man beim Quidditch solide Klatscher aus Eisen, um eine vernünftige Chance zu haben, die Spieler zu verletzen, da Zauberer viel widerstandsfähiger gegen Stöße waren als Muggel.

Harry und Professor McGonagall landeten beide auf dem Boden, und die Pergamente, die sie bei sich trug, verteilten sich im ganzen Korridor.\\ Es gab eine schreckliche, schreckliche Pause.

„Harry Potter“, hauchte Professor McGonagall von dort, wo sie direkt neben Harry auf dem Boden lag. Ihre Stimme erhob sich fast zu einem Schrei.\\ „Was hast du im Büro des Schulleiters gemacht?„

„Nichts!“, quietschte Harry.

\textbf{„Hast du über den Verteidigungsprofessor gesprochen?"}

„Nein! Dumbledore hat mich dorthin gerufen und er hat mir diesen großen Stein gegeben und gesagt, er gehöre meinem Vater und ich solle ihn überall hin mitnehmen!„

Wieder gab es eine schreckliche Pause.

„Ich verstehe“, sagte Professor McGonagall, ihre Stimme wurde etwas ruhiger. Sie stand auf, bürstete sich ab und blickte auf die verstreuten Pergamente, die sich zu einem ordentlichen Stapel auftürmten und zurück an die Korridorwand huschten, als wollten sie sich vor ihrem Blick verstecken.

„Mein Beileid, Mr. Potter, und ich entschuldige mich, dass ich an Ihnen gezweifelt habe.„

„Professor McGonagall“, sagte Harry.\\ Seine Stimme war schwankend. Er stieß sich vom Boden ab, stand auf und sah zu ihrem vertrauenswürdigen, vernünftigen Gesicht auf.\\ „Professor McGonagall..."

„Ja, Mr. Potter?„

„Meinen Sie, ich sollte?“ sagte Harry mit leiser Stimme.\\ „Den Stein meines Vaters überallhin mitnehmen?"

Professor McGonagall seufzte.\\ „Das ist eine Sache zwischen Ihnen und dem Schulleiter, fürchte ich."\\ Sie zögerte.\\ „Ich will sagen, dass es fast nie klug ist, den Schulleiter völlig zu ignorieren. Es tut mir leid, von Ihrem Dilemma zu hören, Mr. Potter, und wenn es irgendeine Möglichkeit gibt, dass ich Ihnen helfen kann, was auch immer Sie zu tun gedenken -„

„Ähm“, sagte Harry.\\ „Eigentlich habe ich mir gedacht, dass ich, sobald ich weiß, wie es geht, den Stein in einen Ring verwandeln und ihn an meinem Finger tragen könnte.\\ Wenn Sie mir beibringen könnten, wie man eine Verwandlung aufrecht erhält -„

„Es ist gut, dass Sie mich zuerst gefragt haben“, sagte Professor McGonagall, ihr Gesicht wurde ein wenig streng.\\ „Wenn Sie die Kontrolle über die Verwandlung verlieren, würde die Umkehrung Ihren Finger abschneiden und wahrscheinlich Ihre Hand in zwei Hälften reißen.\\ Und in Ihrem Alter ist selbst ein Ring ein zu großes Ziel, als dass Sie ihn auf Dauer tragen könnten, ohne dass er Ihre Magie ernsthaft beeinträchtigen würde.\\ Aber ich kann einen Ring für Sie schmieden lassen, mit einer Fassung für ein Juwel, \emph{ein kleines Juwel,} das mit Ihrer Haut in Berührung kommt, und Sie können üben, ein sicheres Objekt, wie einen Marshmallow, zu halten.\\ Wenn Sie das erfolgreich durchhalten, sogar im Schlaf, einen ganzen Monat lang, werde ich erlauben, den Stein Ihres Vaters zu verwandeln..."\\ Professor McGonagalls Stimme verstummte.\\ „Hat der Schulleiter wirklich -"

„Ja. Ah... ähm …"

Professor McGonagall seufzte.\\ „Das ist selbst für ihn ein bisschen seltsam."\\ Sie bückte sich und hob den Stapel Pergamente auf.\\ „Das tut mir leid, Mr. Potter. Ich entschuldige mich nochmals dafür, dass ich Ihnen misstraut habe. Aber jetzt bin ich selbst an der Reihe, den Schulleiter zu sehen."

„Ah... Viel Glück, denke ich. Ähm..."

„Ich danke Ihnen, Mr. Potter."

„Ähm..."

Professor McGonagall ging zu dem Wasserspeier hinüber, sprach unhörbar das Passwort und trat hindurch in die sich drehende Wendeltreppe. Sie begann, außer Sichtweite zu steigen, und der Wasserspeier fuhr zurück -

\textbf{„Professor McGonagall, der Schulleiter hat ein Huhn angezündet!"}

\textbf{„Er hat was?!"}

