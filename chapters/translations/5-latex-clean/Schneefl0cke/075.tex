

\hypertarget{zwischenspiel-mit-dem-beichtvater-irrationale-beharrlichkeit}{% \section{76. Zwischenspiel mit dem Beichtvater: Irrationale Beharrlichkeit}\label{zwischenspiel-mit-dem-beichtvater-irrationale-beharrlichkeit}}

\textbf{\uline{Zwischenspiel mit dem Beichtvater: Irrationale Beharrlichkeit}}

Rianne Felthorne stieg die Treppe aus aufgerautem Stein und grobem Mörtel hinab, wobei sie ein Lumos durch die dunklen Abstände zwischen den Fackeln leitete. Sie kam zu einer leeren, von vielen dunklen Öffnungen durchbrochenen Felshöhle, die von einer einzigen altertümlichen Fackel beleuchtet wurde, die beim Eintreten brannte.

Es war noch niemand da, und nach langen Minuten des nervösen Stehens begann sie einen Zauber, um ein gepolstertes Sofa zu verwandeln, das groß genug war, dass zwei Leute darauf sitzen \emph{oder vielleicht sogar liegen konnten}. Ein Holzschemel wäre einfacher gewesen, das hätte sie in fünfzehn Sekunden schaffen können, aber - nun ja - selbst als das Sofa vollständig herbeigezaubert war, war Professor Snape immer noch nicht eingetroffen, und sie setzte sich auf die linke Seite des Sofas, wobei ihr Puls in der Kehle hämmerte. Irgendwie wurde sie nur noch nervöser, nicht weniger, je länger die Verzögerung andauerte. Sie wusste, dass dies das letzte Mal war. Das letzte Mal, bevor all diese Erinnerungen verschwanden und Rianne Felthorne sich in einer mysteriösen Höhle wiederfand und sich fragte, was hier vor sich ging. Es hatte etwas an sich, das sich wie Sterben anfühlte. In den Büchern stand, dass eine richtig durchgeführte Erinnerungslöschung nicht schädlich war, die Leute vergaßen die ganze Zeit Dinge. Die Leute träumten und wachten dann auf, ohne sich an ihre Träume zu erinnern. \emph{'Obliviate'} beinhaltete nicht einmal so viel Diskontinuität, nur einen kurzen Moment der Desorientierung; es war so, als würde man von einem lauten Geräusch abgelenkt werden und einen Gedanken verlieren, an den man sich danach nicht mehr erinnern konnte. So stand es in den Büchern, und deshalb waren Gedächtniszauber vom Ministerium für alle autorisierten Regierungszwecke voll zugelassen.

Aber trotzdem, diese Gedanken, die Gedanken, die sie jetzt gerade dachte; \emph{bald würde sie niemand mehr haben.} \emph{Wenn sie in die Zukunft blickte, gab es niemanden, der die Gedanken vervollständigen konnte, die sie noch nicht zu Ende gedacht hatte. Selbst wenn sie es schaffte, in der nächsten Minute alle losen Enden in ihren Gedanken zu verknüpfen, würde danach nichts mehr davon übrig sein. War es nicht genau das, worüber man sich Gedanken machen würde, wenn man in der nächsten Minute sterben würde?}

Da ertönte das Geräusch gedämpfter Schritte… Severus Snape tauchte in der Höhle auf. Sein Blick wanderte zu ihr, die auf dem Sofa saß, und ein seltsamer Ausdruck ging über sein Gesicht; seltsam, weil er weder sardonisch noch wütend oder kalt war.

"Danke, Miss~Felthorne", sagte Snape leise, "das war sehr rücksichtsvoll von Ihnen." Der Meister der Zaubertränke zückte seinen Zauberstab und führte die üblichen Geheimhaltungszauber aus, dann ging er auf sie zu und setzte sich schwerfällig neben sie auf das verwandelte Sofa. Ihr Puls pochte jetzt aus einem ganz anderen Grund. Sie drehte sich langsam zu Professor Snape um und sah, dass er den Kopf gegen das Sofa gelehnt hatte und die Augen geschlossen waren. Er schlief jedoch nicht. Sein Gesicht wirkte angespannt, unentspannt, den Schmerz tragend.

Sie wusste - sie war sich plötzlich sicher -, dass sie diesen Anblick nur sehen durfte, weil sie sich danach nicht mehr daran erinnern würde; und dass niemand vor ihr ihn je hatte sehen dürfen. Das verzweifelte Gespräch, das in Rianne Felthornes Kopf ablief, hörte sich ungefähr so an:

\emph{- Ich könnte mich einfach hinüberbeugen und ihn küssen,}

\emph{- du bist völlig von Sinnen,}

\emph{- seine Augen sind geschlossen, ich wette, er würde mich nicht rechtzeitig aufhalten,

- ich wette, es würde Jahre dauern, bis jemand deine Leiche findet -}

Aber dann öffnete Professor Snape seine Augen (zu ihrer inneren Enttäuschung und Erleichterung) und sagte mit normalerer Stimme:

"Ihre Bezahlung, Miss~Felthorne."

Aus seiner Robe nahm er einen Rubin, geschliffen nach Gringotts-Standard, und hielt ihn ihr entgegen.

"Fünfzig Karat. Ich habe nichts dagegen, wenn Sie das überprüfen."

Sie streckte eine zitternde Hand aus und hoffte, dass Snape ihr den Rubin in die Finger drücken würde, dass sie eine lebendige Berührung seiner Haut an ihrer spüren würde - aber stattdessen hob Snape seine Hand leicht an und ließ den Rubin in ihre Hand fallen, dann lehnte er sich zurück gegen die Couch.

"Du wirst dich daran erinnern, dass du ihn auf dem Boden dieser Höhle gefunden hast, wo du auf Erkundungstour warst", sagte Snape. "Und da niemand außer dir das wirklich glauben wird, wirst du sich daran erinnern, dass du dachtest, es wäre weniger lästig, wenn du das Geld in einem separaten Fach in Gringotts deponieren würdest."

Eine Zeit lang war nur das schwache Knistern der Fackel zu hören.

"Warum -" sagte Rianne Felthorne. \emph{Er weiß, dass ich mich nicht erinnern werde.} "Warum hast du es getan? Ich meine - du sagtest, ich solle dir sagen, wo Tyrannen sein würden und wer sie sein würden, aber nicht, ob Granger dort sein würde. Und ich weiß, so wie der Zeitumkehrer funktioniert, kann man, wenn man Granger dort sein lassen will, nicht sagen, ob es schon passiert ist. Also habe ich herausgefunden, dass wir diejenigen waren, die ihr sagten, wohin sie gehen soll. Das waren wir, nicht wahr?"

Snape nickte wortlos. Er hatte die Augen wieder geschlossen.

"Aber", sagte Rianne, "ich habe nicht verstanden, warum du ihr geholfen hast. Und jetzt - nach dem, was du Granger in der Großen Halle angetan hast - verstehe ich es noch weniger."

Rianne hatte sich selbst nie für besonders nett gehalten. Sie hatte von der Kontroverse um den Sonnenschein General wenig mitbekommen. Aber Granger im Kampf gegen Tyrannen zu helfen, hatte etwas … nun, sie hatte sich angewöhnt, das als die gute Seite zu betrachten und sich selbst als auf der guten Seite stehend zu sehen. Und sie hatte festgestellt, dass es ihr sogar gefiel. Es war schwer, das einfach gehen zu lassen.

"Warum hast du ihr geholfen?"

Snape schüttelte den Kopf, sein Gesicht straffte sich.

"Ist -" sagte Rianne zögernd. "Ich meine - wo wir schon mal hier sind - gibt es irgendetwas, worüber du sprechen möchtest?"

Es gab etwas, das sie sagen wollte, aber sie konnte die Worte nicht über ihre eigenen Lippen bringen.

"Mir fällt da eine Sache ein", sagte Snape nach einer Pause. "Wenn du interessiert bist."

Snapes Augen waren immer noch geschlossen, also konnte sie nicht einfach nicken.

Ihre Stimme brach fast, als sie sich zwang, "Ja" zu sagen.

"Es gibt einen bestimmten Jungen in Ihrer Klasse, der Sie mag, Miss~Felthorne", sagte Snape hinter seinen geschlossenen Augen. "Ich werde seinen Namen nicht sagen. Aber er beobachtet dich jedes Mal, wenn du durch den Raum gehst, wenn er denkt, dass du nicht hinsiehst. Er träumt von dir und wünscht sich, dich zu besitzen, aber er hat dich noch nie auch nur um einen Kuss gebeten."

Ihr Herz fing an, noch heftiger zu hämmern.

"Bitte sag mir die ehrliche Wahrheit. Was hältst du vom Verhalten dieses Jungen?"

"Nun -", sagte sie. Sie stolperte über ihre Worte. "Ich denke - niemals auch nur um einen Kuss zu bitten - wäre -" \emph{Traurig. Einfach zu erbärmlich}. "Schwäche", sagte sie, und ihre Stimme zitterte.

"Ich stimme zu", sagte Snape. "Aber nehmen wir an, der Junge hätte dir geholfen. Würdest du denken, dass du ihm einen Kuss schuldest, wenn er dich darum bittet?"

Sie atmete scharf ein.

"Oder würdest du denken", fuhr Snape fort, die Augen immer noch geschlossen, "dass er nur lästig war?"

Die Worte stachen wie ein Messer in sie hinein und sie konnte nicht anders, als laut zu keuchen.

Snapes Augen flogen auf und sein Blick traf den ihren auf der anderen Seite des Sofas. Dann begann der Meister der Zaubertränke zu lachen, ein kleines, trauriges Kichern.

"Nein, nicht du, Miss~Felthorne!" sagte Snape. "Nicht du! Wir reden hier wirklich von einem Jungen. Einer, der in deine Zaubertränkeklasse geht."

"Oh", sagte sie. Sie versuchte, sich daran zu erinnern, was Snape zuvor gesagt hatte, und fühlte sich jetzt ziemlich entnervt, als sie an einen Jungen dachte, der sie beobachtete, der sie immer schweigend beobachtete.

"Nun, ähm, in diesem Fall. Das ist irgendwie gruselig, ehrlich gesagt. Wer ist es?"

Der Meister der Zaubertränke schüttelte den Kopf. "Das spielt keine Rolle", sagte Snape. "Nur so aus Neugierde: Was würdest du davon halten, wenn dieser Junge Jahre später immer noch in dich verliebt wäre?"

"Ähm", sagte sie, etwas verwirrt, "das wäre unsinnig und total erbärmlich."

Die Fackel knisterte ein wenig in der Höhle.

"Es ist seltsam", sagte Snape leise. "Ich hatte im Laufe meiner Tage zwei Mentoren. Beide waren außerordentlich scharfsinnig, und keiner von beiden hat mir je die Dinge gesagt, die ich nicht gesehen habe. Es ist klar genug, warum der erste nichts gesagt hat, aber der zweite …"

Snapes Gesicht straffte sich.

"Ich nehme an, ich müsste naiv sein, um zu fragen, warum er geschwiegen hat."

Die Stille dehnte sich, während Rianne krampfhaft versuchte, sich etwas einfallen zu lassen, was sie sagen könnte.

"Es ist eine seltsame Sache", sagte Snape, seine Stimme noch leiser, "nach nur zweiunddreißig Jahren zurückzublicken und sich zu fragen, wann dein Leben jenseits aller Rettung ruiniert wurde. War es bestimmt, als der Sprechende Hut '\emph{Slytherin}!' für mich rief? Es scheint ungerecht, da ich keine Wahl hatte; der Sprechende Hut sprach in dem Moment, als er meinen Kopf berührte. Dennoch kann ich nicht behaupten, dass er mich unrechtmäßig benannt hat. Ich habe das Wissen nie um seiner selbst willen geschätzt. Ich war der einen Person, die ich Freund nannte, nicht loyal. Ich war nie jemand für gerechte Wut, weder damals noch heute. Tapferkeit? Es ist nicht mutig, ein Leben zu riskieren, das bereits ruiniert ist. Meine kleinen Ängste haben mich immer beherrscht, und ich bin nie von einem der Wege abgewichen, die ich beschritten habe, wegen dieser kleinen Ängste. Nein, der Sprechende Hut hätte mich niemals in \emph{ihr} Haus stecken können. Vielleicht stand mein endgültiger Verlust schon damals fest. Ist das gerecht, frage ich, selbst wenn der Sprechende Hut die Wahrheit sagt? Ist es gerecht, dass einige Kinder mehr Mut besitzen als andere und dass so über das Leben eines Menschen geurteilt wird?"

Rianne Felthorne wurde allmählich klar, dass sie nicht die geringste Ahnung davon hatte, wer ihr Zaubertränkemeister im Inneren war, und \emph{leider halfen ihr all diese dunklen, verborgenen Tiefen nicht bei ihrem Problem.}

"Aber nein", sagte Snape.

"Ich weiß, wo es zum letzten Mal schiefgegangen ist. Ich könnte genau auf den Tag und die Stunde zeigen, an dem ich meine letzte Chance verpasst habe. Miss~Felthorne, hat der Sprechende Hut dir Ravenclaw angeboten?"

"J-ja", sagte sie, ohne nachzudenken.

"Warst du jemals gut in Rätseln?"

"Ja", sagte sie wieder, denn was immer Professor Snape sagen wollte, sie würde es nicht hören, wenn sie nein sagte.

"Ich bin schrecklich in Rätseln", sagte Snape mit entfernter Stimme. "Man hat mir einmal ein Rätsel aufgegeben, das ich lösen sollte, und ich habe nicht einmal den einfachsten Teil verstanden, bis es zu spät war. Ich habe nicht einmal erkannt, dass das Rätsel für mich bestimmt war, bis es zu spät war. Ich dachte, ich hätte es nur zufällig mitgehört, aber in Wahrheit war ich es, der mitgehört wurde. Also verkaufte ich das Rätsel an einen anderen, und das war der Zeitpunkt, an dem die Trümmer meines Lebens nicht mehr zu retten waren."

Snapes Stimme war immer noch distanziert und klang mehr abstrakt als traurig.

"Und selbst jetzt verstehe ich nichts von Bedeutung. Sag mir, Miss~Felthorne, nehmen wir an, ein Mann trüge ein Messer bei sich, er stolperte über ein Baby und stach sich selbst. Würden Sie sagen, dass das Baby", Snapes Stimme senkte sich, als würde er eine noch tiefere Stimme imitieren, "\emph{DIE MACHT hat, ihn zu vernichten}?"

"Ähm … nein?", sagte sie zögernd.

"Was bedeutet es dann, die Macht zu haben, jemanden auszulöschen?"

Rianne dachte über das Rätsel nach.

(Sie wünschte sich, nicht zum ersten Mal in ihrem Leben, sie hätte Ravenclaw gewählt und sich mit dem Missfallen ihrer Eltern ins Verderben gestürzt; aber der Sprechende Hut hatte ihr nie Gryffindor angeboten.)

"Nun …" sagte Rianne. Es fiel ihr schwer, ihre Gedanken in Worte zu fassen. "

"Ich wünschte", sagte Severus Snape in einem Flüsterton, der so leise war, dass sie ihn kaum hören konnte, "dass alles anders gewesen wäre …"

Der Meister der Zaubertränke erhob sich vom Sofa, das Gewicht seiner Anwesenheit verschwand neben ihr. Er drehte sich um, zog seinen Zauberstab aus seinem Umhang und richtete ihn auf sie.

"Warte -", sagte sie. "Bevor das -"

Irgendwie war es unglaublich schwer, den ersten Schritt von der Fantasie zur Realität zu machen, vom Vorstellen zum Tun. Selbst wenn es nur ein Schritt war und nie weiter gehen würde.

Die Kluft erstreckte sich wie die Entfernung zwischen zwei Bergen. Der Sprechende Hut hatte ihr nie Gryffindor angeboten… .\emph{..war es fair, dass so über das Leben einer Frau geurteilt wurde? Wenn man es jetzt nicht sagen kann, wenn man sich hinterher nicht einmal mehr daran erinnern wird - wenn von diesem Moment an nichts mehr so sein wird, wenn es so wäre als würde man sterben -, wann wird man es dann jemals sagen, zu irgendjemandem wenn nicht jetzt?}

"Kann ich zuerst einen Kuss bekommen?", fragte Rianne Felthorne.

Snapes schwarze Augen studierten sie so intensiv, dass ihr die Röte bis zur Brust stieg, und sie fragte sich, ob er genau wusste, dass sie immer noch schwach war und es kein Kuss war, den sie wirklich gewollt hatte.

"Warum nicht", sagte der Meister der Zaubertränke leise, \emph{und er beugte seinen Kopf über das Sofa und küsste sie.}

Es war nicht so, wie sie es sich vorgestellt hatte. In ihren Fantasien waren Snapes Küsse heftig, von ihr ergriffen, aber das hier war - eigentlich war es nur unangenehm. Snapes Lippen drückten zu fest auf ihre, zwangen sie gegen ihre Zähne zurück, und der Winkel stimmte nicht und ihre Nasen waren irgendwie gebogen und seine Lippen waren zu fest und - erst als der Meister der Zaubertränke sich wieder aufrichtete und seinen Zauberstab erneut hob, wurde es ihr klar.

"Das war -", sagte sie mit verwunderter Stimme und sah zu ihm auf. "Das war - war es - dein erster -"

Rianne Felthorne blinzelte über die steinerne Höhle, die sie entdeckt hatte, und hielt immer noch den außergewöhnlichen Rubin in der Hand, den sie im Dreck in einer Ecke gefunden hatte. Es war ein unglaublicher Glücksfall, und sie wusste nicht, warum sie sich beim Anblick des Rubins so traurig fühlte, als hätte sie etwas vergessen, etwas, das ihr sehr sehr wichtig gewesen war.

