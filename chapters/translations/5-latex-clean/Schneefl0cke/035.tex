

\hypertarget{statusunterschiede}{% \section{36. Statusunterschiede}\label{statusunterschiede}}

\textbf{\uline{Statusunterschiede}}

Zerreißende Orientierungslosigkeit, so fühlte es sich an, wenn man von Bahnsteig Neun und Dreiviertel auf den Rest der Erde hinausging, die Welt, die Harry einst für die einzig wahre gehalten hatte. Die Menschen trugen legere Hemden und Hosen, statt der würdevolleren Roben von Zauberern und Hexen. Verstreute Müllteile hier und da um die Bänke herum. Ein vergessener Geruch, die Dämpfe von verbranntem Benzin, rau und scharf in der Luft. Die Atmosphäre des Bahnhofs King's Cross, weniger hell und fröhlich als in Hogwarts oder derWinkelgasse; die Menschen schienen kleiner, ängstlicher und hätten ihre Probleme wahrscheinlich gerne gegen einen dunklen Zauberer eingetauscht, um zu kämpfen.

Harry wollte „\emph{Scourgify}“ für den Schmutz und „\emph{Everto}“ für den Müll zaubern, und wenn er den Zauberspruch gekannt hätte, einen „\emph{Kopfblasenzauber}“, damit er die Luft nicht einatmen musste.

Aber er konnte seinen Zauberstab nicht benutzen, an diesem Ort…

\emph{So muss es sich anfühlen haben, von einem Land der Ersten Welt in ein Land der Dritten Welt zu kommen, wurde Harry klar.}

Nur dass es die Nullte Welt war, die Harry verlassen hatte, die Welt der Zauberer, der Reinigungszauber und der Hauselfen; wo man zwischen den Künsten der Heiler und der eigenen Magie hundertundsiebzig Jahre alt werden konnte, bevor das Alter einen wirklich einholte.

Und das nichtmagische London, die Muggelwelt, in die Harry vorübergehend zurückgekehrt war. Hier würden Mum und Dad den Rest ihres Lebens verbringen, es sei denn, die Technologie überholte die Lebensqualität der Zauberer oder etwas Tieferes in der Welt änderte sich.

Ohne darüber nachzudenken, drehte sich Harrys Kopf und seine Augen huschten hinter ihn, um den hölzernen Koffer zu sehen, der hinter ihm herhuschte, unbemerkt von allen Muggeln, wobei die krallenartigen Tentakel eine schnelle Bestätigung dafür boten, dass er sich das alles nicht nur eingebildet hatte… Und dann war da noch der andere Grund für das enge Gefühl in seiner Brust.

\emph{Seine Eltern wussten es nicht. Sie wussten nichts. Sie wussten nicht…}

„Harry?“, rief eine dünne, blonde Frau, deren perfekt glatte und makellose Haut sie ein gutes Stück jünger als dreiunddreißig aussehen ließ; und Harry erkannte mit einem Schreck, dass es Magie war, er hatte die Zeichen vorher nicht erkannt, aber jetzt konnte er sie sehen. Und welche Art von Trank auch immer so lange anhielt, er musste furchtbar gefährlich gewesen sein, denn die meisten Hexen taten sich das nicht an, sie waren nicht so verzweifelt… Es sammelte sich Wasser in Harrys Augen.

„Harry?“, rief ein älter aussehender Mann mit einem Bauchansatz, gekleidet mit ostentativer akademischer Nachlässigkeit in einer schwarzen Weste über einem dunkelgraugrünen Hemd, jemand, der überall, wo er hinkam, ein Professor war, der sicherlich einer der brillantesten Zauberer seiner Generation gewesen wäre, wenn er mit zwei Kopien dieses Gens geboren worden wäre, statt mit null.

.. Harry hob seine Hand und winkte ihnen zu. Er konnte nicht sprechen. Er konnte überhaupt nicht sprechen. Sie kamen zu ihm herüber, nicht rennend, sondern in einem gleichmäßigen, würdevollen Gang; so schnell ging Professor Michael Verres-Evans, und Mrs~Petunia Evans-Verres hatte nicht vor, noch schneller zu gehen. Das Lächeln auf dem Gesicht seines Vaters war nicht sehr breit, aber es war zumindest so breit, wie Harry es je gesehen hatte, breiter als wenn ein neues Stipendium hereinkam oder wenn einer seiner Studenten eine Stelle bekam, und ein breiteres Lächeln konnte man sich nicht wünschen. Mum blinzelte angestrengt, und sie versuchte zu lächeln, was ihr aber nicht besonders gut gelang.

„So!“, sagte sein Vater, als er heranschritt. „Schon irgendwelche revolutionären Entdeckungen gemacht?“

\emph{Natürlich dachte Papa, er mache einen Scherz.}

Es hatte nicht so sehr wehgetan, als seine Eltern nicht an ihn geglaubt hatten, damals, als auch sonst niemand an ihn geglaubt hatte, damals, als Harry noch nicht wusste, wie es sich anfühlte, von Leuten wie Schulleiter Dumbledore und Professor Quirrell ernst genommen zu werden.

Und das war der Moment, in dem Harry begriff, dass der Junge, der lebte, nur im magischen Britannien existierte, dass es im Muggel-London keine solche Person gab, sondern nur einen süßen kleinen elfjährigen Jungen, der zu Weihnachten nach Hause ging.

„Entschuldigung“, sagte Harry, seine Stimme zitterte, „ich werde jetzt zusammenbrechen und weinen, das heißt nicht, dass in der Schule etwas nicht in Ordnung war.“

Harry begann, sich vorwärts zu bewegen, und hielt dann inne, hin- und hergerissen zwischen der Umarmung seines Vaters und der Umarmung seiner Mutter, er wollte nicht, dass sich einer von beiden gekränkt fühlte oder dass Harry sie mehr liebte als den anderen—

„Du“, sagte sein Vater, „bist ein sehr dummer Junge, Mr~Verres“, und er nahm Harry sanft bei den Schultern und schob ihn in die Arme seiner Mutter, die sich hinkniete, Tränen liefen bereits über ihre Wange.

„Hallo, Mum“, sagte Harry mit schwankender Stimme, „ich bin wieder da.“

Und er umarmte sie, inmitten der lauten mechanischen Geräusche und dem Geruch von verbranntem Benzin; und Harry fing an zu weinen, weil er wusste, dass nichts zurückgehen konnte, am wenigsten er selbst.

Der Himmel war völlig dunkel, und die Sterne kamen zum Vorschein, als sie sich durch den Weihnachtsverkehr in die Universitätsstadt Oxford schlängelten und in der Einfahrt des kleinen, schäbig aussehenden alten Hauses parkten, das ihre Familie benutzte, um den Regen von ihren Büchern fernzuhalten.

Als sie den kurzen Weg zur Haustür hinaufgingen, kamen sie an einer Reihe von Blumentöpfen mit kleinen, schummrigen elektrischen Lichtern vorbei (schummrig, weil sie sich tagsüber mit Solarstrom aufladen mussten), und die Lichter leuchteten auf, als sie vorbeikamen. Die Schwierigkeit bestand darin, Bewegungssensoren zu finden, die wasserdicht waren und genau in der richtigen Entfernung auslösten.

\emph{.. In Hogwarts gab es richtige Fackeln wie diese.}

Und dann öffnete sich die Haustür und Harry trat in ihr Wohnzimmer und blinzelte heftig.

Jeder Zentimeter Wandfläche wird von einem Bücherregal eingenommen. Jedes Bücherregal hat sechs Fächer, die fast bis zur Decke reichen. Einige Regale sind bis zum Rand mit Hardcover-Büchern bestückt: Naturwissenschaften, Mathematik, Geschichte und alles andere.

Andere Regale haben zwei Lagen Taschenbuch-Science-Fiction, wobei die hintere Lage der Bücher auf alten Taschentuchkartons oder Kanthölzern steht, so dass man die hintere Lage der Bücher über die vorderen Bücher sehen kann.

Und es ist immer noch nicht genug. Die Bücher quellen über auf den Tischen und den Sofas und bilden kleine Haufen unter den Fenstern… Der Haushalt der Verres war genau so, wie er ihn verlassen hatte, nur mit mehr Büchern, was auch genau so war, wie er ihn verlassen hatte.

Und ein Weihnachtsbaum, nackt und ungeschmückt, nur zwei Tage vor Heiligabend, was Harry kurz aus der Fassung brachte, bevor er mit einem warmen Gefühl in der Brust feststellte, dass seine Eltern natürlich gewartet hatten.

„Wir haben das Bett aus deinem Zimmer genommen, um Platz für weitere Bücherregale zu schaffen“, sagte sein Vater. „Du kannst doch in deinem Koffer schlafen, oder?“

„Du kannst in meinem Kofferraum schlafen“, sagte Harry.

„Da fällt mir ein“, sagte sein Vater. „Was haben sie am Ende mit deinem Schlafzyklus gemacht?“

„Magie“, sagte Harry und machte sich auf den Weg zu der Tür, die zu seinem Schlafzimmer führte, nur für den Fall, dass Dad keinen Scherz machte…

„Das ist keine Erklärung!“, sagte Professor Verres-Evans, gerade als Harry rief:

„Du hast den ganzen freien Platz in meinem Bücherregal verbraucht?!“

Harry hatte den 23.

Dezember damit verbracht, Muggelsachen einzukaufen, die er nicht einfach verwandeln konnte; sein Vater war beschäftigt gewesen und hatte gesagt, dass Harry zu Fuß gehen oder den Bus nehmen müsse, was Harry ganz recht gewesen war.

Einige der Leute im Baumarkt hatten Harry fragende Blicke zugeworfen, aber er hatte mit unschuldiger Stimme gesagt, dass sein Vater in der Nähe einkaufte und sehr beschäftigt war und ihn geschickt hatte, um einige Dinge zu besorgen (er hielt eine Liste in sorgfältig erwachsen aussehender, halb unleserlicher Handschrift hoch); und schließlich war Geld Geld.

Sie hatten alle zusammen den Weihnachtsbaum geschmückt, und Harry hatte eine kleine tanzende Fee auf die Spitze gesetzt (zwei Sickles, fünf Knuts bei Gambol \& Japes).

Gringotts hatte Galleonen bereitwillig in Papiergeld umgetauscht, aber sie schienen keine einfache Möglichkeit zu haben, größere Mengen Gold in steuerfreies, unverdächtiges Muggelgeld auf einem nummerierten Schweizer Bankkonto zu verwandeln. Das hatte Harrys Plan, den größten Teil des selbst gestohlenen Geldes in eine vernünftige Mischung aus 60 \% internationalen Indexfonds und 40 \% Berkshire Hathaway zu verwandeln, ziemlich durchkreuzt. Für den Moment hatte Harry sein Vermögen noch ein wenig weiter diversifiziert, indem er sich spät nachts unsichtbar und zeitversetzt hinausschlich und hundert goldene Galleonen im Hinterhof vergrub.

\emph{Das hatte er sowieso schon immer mal machen wollen.}

Einen Teil des 24. Dezembers hatte der Professor damit verbracht, Harrys Bücher zu lesen und Fragen zu stellen. Die meisten der Experimente, die sein Vater vorgeschlagen hatte, waren unpraktisch, zumindest für den Moment; von denen, die übrig blieben, hatte Harry schon viele gemacht.

(„Ja, Dad, ich habe überprüft, was passiert, wenn man Hermine eine veränderte Aussprache gibt und sie nicht weiß, ob sie verändert wurde, das war das allererste Experiment, das ich gemacht habe, Dad!“)

Die letzte Frage, die Harrys Vater gestellt hatte, als er mit einem Ausdruck fassungsloser Abscheu von den Zaubertränken aufschaute, war, ob das alles einen Sinn habe, wenn man ein Zauberer sei; und Harry hatte mit Nein geantwortet.

Daraufhin hatte sein Vater erklärt, dass Magie \emph{unwissenschaftlich} sei. Harry war immer noch ein wenig schockiert über die Vorstellung, auf einen Ausschnitt der Realität zu zeigen und ihn als \emph{unwissenschaftlich} zu bezeichnen. Dad schien zu denken, dass der Konflikt zwischen seinen Intuitionen und dem Universum bedeutete, dass das Universum ein Problem hatte.

(Andererseits gab es eine Menge Physiker, die dachten, dass die Quantenmechanik seltsam sei, anstatt dass die Quantenmechanik normal sei und sie seltsam.)

Harry hatte seiner Mutter den Heilerkasten gezeigt, den er gekauft hatte, um ihn im Haus aufzubewahren, obwohl die meisten der Tränke bei Dad nicht funktionieren würden.

Mum hatte das Set so angestarrt, dass Harry sich fragte, ob Mums Schwester jemals so etwas für Opa Edwin und Oma Elaine gekauft hatte. Und als Mum immer noch nicht geantwortet hatte, hatte Harry hastig gesagt, dass sie wohl einfach nie daran gedacht hatte. Und dann war er schließlich aus dem Zimmer geflüchtet.

\emph{Lily Evans hatte wahrscheinlich nicht daran gedacht, das war das Traurige daran.}

Harry wusste, dass andere Menschen dazu neigten, nicht an schmerzhafte Themen zu denken, so wie sie dazu neigten, ihre Hände nicht absichtlich auf glühende Herdplatten zu legen; und Harry begann zu vermuten, dass die meisten Muggelgeborenen schnell eine Tendenz entwickelten, nicht an ihre Familie zu denken, die sowieso alle sterben würden, bevor sie ihr erstes Jahrhundert erreichten. Nicht, dass Harry die Absicht gehabt hätte, das zuzulassen, natürlich nicht.

Und dann war es schon spät am 24. Dezember und sie fuhren zu ihrem Heiligabend-Dinner.

Das Haus war riesig, nicht nach Hogwarts-Maßstäben, aber sicherlich nach den Maßstäben dessen, was man bekommen konnte, wenn der Vater ein angesehener Professor war und in Oxford leben wollte. Zwei Stockwerke aus Backstein, die in der untergehenden Sonne glänzten, mit Fenstern über Fenstern und einem hohen Fenster, das viel weiter nach oben ging, als Glas gehen sollte, das war ein riesiges Wohnzimmer… Harry holte tief Luft und läutete an der Tür. Aus der Ferne ertönte der Ruf: „\emph{Schatz, kannst du rangehen?}“ Darauf folgte ein langsames Getrappel sich nähernder Schritte.

Dann öffnete sich die Tür und enthüllte einen freundlichen Mann mit dicken, rosigen Wangen und schütterem Haar in einem blauen Button-Down-Hemd, das an den Nähten leicht spannte.

„Dr~Granger?“ sagte Harrys Vater zügig, bevor Harry überhaupt etwas sagen konnte.

„Ich bin Michael, und das sind Petunia und unser Sohn Harry. Das Essen ist in dem magischen Koffer“, und Dad machte eine vage Geste hinter sich - nicht ganz in Richtung der Truhe, wie es schien.

„Ja, bitte, kommen Sie herein“, sagte Leo Granger. Er trat vor und nahm die Weinflasche aus den ausgestreckten Händen des Professors, mit einem gemurmelten „Danke“, dann trat er zurück und winkte ins Wohnzimmer. „Nehmen Sie Platz. Und„, er drehte den Kopf nach unten, um Harry anzusprechen, “alle Spielsachen sind unten im Keller, ich bin mir sicher, Hermy wird in Kürze unten sein, es ist die erste Tür rechts", und zeigte auf einen Korridor. Harry schaute ihn nur einen Moment lang an, in dem Bewusstsein, dass er seine Eltern daran hinderte, hereinzukommen.

„Spielzeug?“, sagte Harry mit heller, hoher Stimme und großen Augen. „Ich liebe Spielzeug!“

Seine Mutter atmete hinter ihm ein, und Harry schritt ins Haus, wobei er darauf achtete, nicht zu sehr zu strampeln, während er ging.

Das Wohnzimmer war genauso groß, wie es von außen ausgesehen hatte, mit einer riesigen gewölbten Decke, an der ein gigantischer Kronleuchter baumelte, und einem Weihnachtsbaum, der mörderisch schwer durch die Tür zu manövrieren gewesen sein musste. Die unteren Ebenen des Baumes waren sorgfältig und liebevoll in ordentlichen Mustern aus Rot und Grün und Gold geschmückt, mit einer neu entdeckten Einstreuung von Blau und Bronze; die Höhen, die nur ein Erwachsener erreichen konnte, waren achtlos und wahllos mit Lichterketten und Lametta-Kränzen drapiert. Ein Flur erstreckte sich, bis er in den Schränken einer Küche endete, und eine Holztreppe mit poliertem Metallgeländer zog sich hinauf in einen zweiten Stock.

„Donnerwetter!“ sagte Harry. „Das ist ein großes Haus! Ich hoffe, ich verlaufe mich hier nicht!“

Dr~Roberta Granger fühlte sich ziemlich nervös, als das Abendessen näher rückte. Der Truthahn und der Braten, ihre eigenen Beiträge zu dem gemeinsamen Projekt, kochten im Ofen vor sich hin; die anderen Gerichte sollten von ihren Gästen, der Familie Verres, die einen Jungen namens Harry adoptiert hatte, mitgebracht werden. Der in der Zaubererwelt als der Junge-der-lebte bekannt war. Und der auch der einzige Junge war, den Hermine jemals „\emph{süß}“ genannt oder überhaupt bemerkt hatte, wirklich.

Die Verreses hatten gesagt, dass Hermine das einzige Kind in Harrys Altersgruppe war, dessen Existenz ihr Sohn jemals in irgendeiner Weise anerkannt hatte. Und es war vielleicht ein bisschen voreilig, aber beide Paare hatten den leisen Verdacht, dass in ein paar Jahren die Hochzeitsglocken läuten könnten. Während der erste Weihnachtsfeiertag wie immer mit der Familie ihres Mannes verbracht werden sollte, hatten sie beschlossen, Heiligabend mit den möglichen zukünftigen Schwiegereltern ihrer Tochter zu verbringen. Als es an der Tür klingelte, während sie gerade dabei war, den Truthahn zu braten, erhob sie ihre Stimme und rief: „Schatz, kannst du rangehen?“.

Es gab ein kurzes Aufstöhnen eines Stuhls und seines Insassen, und dann ertönten die schweren Schritte ihres Mannes und die Tür schwang auf.

„Dr~Granger?“, sagte die forsche Stimme eines älteren Mannes. „Ich bin Michael, und das sind Petunia und unser Sohn Harry. Das Essen ist in dem magischen Koffer.“

„Ja, bitte, kommen Sie herein“, sagte ihr Mann, gefolgt von einem gemurmelten

„Danke“, das darauf hindeutete, dass eine Art Geschenk angenommen worden war, und

„Nehmen Sie Platz.“

Dann wechselte Leos Stimme zu einem Ton künstlicher Begeisterung und sagte:

„Und alle Spielsachen sind unten im Keller, ich bin mir sicher, dass Hermy in Kürze unten sein wird, es ist die erste Tür auf der rechten Seite.“

Es gab eine kurze Pause.

Dann sagte die helle Stimme eines kleinen Jungen: „Spielzeug? Ich liebe Spielzeug!“

Es gab das Geräusch von Schritten, die das Haus betraten, und dann sagte dieselbe helle Stimme:

„Donnerwetter! Das ist ein großes Haus! Ich hoffe, ich verlaufe mich hier nicht!“

\emph{Roberta klappte den Ofen zu und lächelte.}

Sie hatte sich ein wenig Sorgen gemacht über die Art und Weise, wie Hermines Briefe den Jungen-der-lebte beschrieben hatten - obwohl ihre Tochter natürlich nichts gesagt hatte, was darauf hindeutete, dass Harry Potter gefährlich war; nichts im Vergleich zu den dunklen Andeutungen in den Büchern, die Roberta während ihres Ausflugs in die Winkelgasse gekauft hatte, angeblich für Hermine. Ihre Tochter hatte überhaupt nicht viel gesagt, nur dass Harry redete, als käme er aus einem Buch, und dass Hermine härter lernte als je zuvor in ihrem Leben, nur um ihm im Unterricht voraus zu sein. Aber so wie es sich anhörte, war Harry Potter ein ganz normaler elfjähriger Junge. Sie kam gerade zur Haustür, als ihre Tochter hektisch die Treppe hinunterpolterte, und zwar mit einer Geschwindigkeit, die überhaupt nicht sicher aussah - Hermine hatte behauptet, dass Hexen widerstandsfähiger gegen Stürze seien, aber Roberta war sich nicht ganz sicher, ob sie das glaubte - Roberta warf einen ersten Blick auf Professor und Mrs~Verres, die beide ziemlich nervös aussahen, gerade als der Junge mit der legendären Narbe auf der Stirn sich zu ihrer Tochter umdrehte und, jetzt mit leiserer Stimme, sagte:

„Guten Tag an diesem schönsten aller Abende, Miss~Granger.“

Er streckte die Hand zurück, als würde er seine Eltern auf einem Silbertablett anbieten.

„Ich stelle Ihnen meinen Vater, Professor Michael Verres-Evans, und meine Mutter, Mrs~Petunia Evans-Verres, vor.“

Und während Roberta der Mund offen stand, drehte sich der Junge wieder zu seinen Eltern um und sagte, nun wieder mit dieser hellen Stimme:

„Mum, Dad, das ist Hermine! Sie ist wirklich klug!“

„Harry!“, zischte ihre Tochter. „Hör auf damit!“

Der Junge drehte sich wieder um und betrachtete Hermine.

„Ich fürchte, Miss~Granger“, sagte der Junge ernst, „dass Sie und ich in die labyrinthischen Nischen des Kellers verbannt worden sind. Überlassen wir sie ihren erwachsenen Gesprächen, die zweifellos weit über unseren eigenen kindlichen Intellekt hinausgehen würden, und nehmen wir unsere laufende Diskussion über die Auswirkungen des Hume'schen Projektivismus auf die Verwandlung wieder auf.“

„Entschuldigen Sie uns bitte“, sagte ihre Tochter in sehr festem Ton, packte den Jungen am linken Ärmel und zerrte ihn in den Flur - Roberta drehte sich hilflos, um sie zu verfolgen, als sie an ihr vorbeigingen, der Junge winkte ihr aufmunternd zu -, dann zog Hermine den Jungen in den Kellereingang und schlug die Tür hinter sich zu.

„Ich, ah, ich entschuldige mich für…“, sagte sie mit stockender Stimme.

„Es tut mir leid“, sagte der Professor und lächelte liebevoll, „Harry kann in solchen Dingen ein bisschen empfindlich sein. Aber ich nehme an, er hat Recht damit, dass wir an ihrer Unterhaltung nicht interessiert sind.“

\emph{Ist er gefährlich?} wollte Roberta fragen, aber sie schwieg und versuchte, an subtilere Fragen zu denken. Ihr Mann neben ihr kicherte, als ob er das, was sie gerade gesehen hatten, eher lustig als beängstigend gefunden hätte.

\emph{Der schrecklichste Dunkle Lord der Geschichte hatte versucht, den Jungen zu töten, und die verbrannte Hülle seiner Leiche war neben dem Kinderbett gefunden worden.}

\textbf{\emph{Ihr möglicher zukünftiger Schwiegersohn.}}

Roberta hatte zunehmend Bedenken gehabt, ihre Tochter der Hexerei zu überlassen - vor allem, nachdem sie die Bücher gelesen, die Daten zusammengefügt und erkannt hatte, dass ihre magische Mutter wahrscheinlich auf dem Höhepunkt von Grindelwalds Terror getötet worden war und nicht bei ihrer Geburt gestorben war, wie ihr Vater immer behauptet hatte. Aber Professor McGonagall hatte nach ihrem ersten Besuch weitere Besuche gemacht, um „zu sehen, wie es Miss~Granger geht“

; und Roberta konnte nicht anders, als zu denken, dass, wenn Hermine sagte, dass ihre Eltern wegen ihrer Hexenkarriere lästig waren, etwas getan werden würde, um sie in Ordnung zu bringen…

Roberta setzte ihr bestes Lächeln auf und tat, was sie konnte, um etwas vorgetäuschte Weihnachtsstimmung zu verbreiten.

Der Esszimmertisch war viel länger, als sechs Personen - \emph{äh, vier Personen und zwei Kinder} - wirklich brauchten, aber alles war mit einer Tischdecke aus feinem weißen Leinen drapiert, und das Geschirr war unnötigerweise auf schicke Servierplatten ausgelagert worden, die wenigstens aus Edelstahl und nicht aus echtem Silber waren. Harry hatte ein wenig Mühe, sich auf den Truthahn zu konzentrieren. Das Gespräch hatte sich natürlich um Hogwarts gedreht; und es war Harry klar gewesen, dass seine Eltern hofften, Hermine würde etwas ausplappern und mehr über Harrys Schulleben erzählen, als Harry ihnen erzählt hatte.

Und entweder hatte Hermine das erkannt, oder sie ging einfach automatisch allem aus dem Weg, was sich als lästig erweisen könnte. Also ging es Harry gut. Aber leider hatte Harry den Fehler gemacht, seinen Eltern mit allen möglichen Fakten über Hermine zu eulen, die sie ihren eigenen Eltern nicht erzählt hatte. Zum Beispiel, dass sie bei ihren außerschulischen Aktivitäten General einer Armee war. Hermines Mutter hatte sehr beunruhigt ausgesehen, und Harry hatte sie schnell unterbrochen und sein Bestes getan, um zu erklären, dass alle Zaubersprüche Betäubungszauber waren, Professor Quirrell immer aufpasste und die Existenz von magischer Heilung bedeutete, dass viele Dinge viel weniger gefährlich waren, als sie sich anhörten, woraufhin Hermine ihn hart unter den Tisch getreten hatte. Zum Glück hatte Harrys Vater, der, wie Harry zugeben musste, in manchen Dingen besser war als er, mit fester professoraler Autorität verkündet, dass er sich überhaupt keine Sorgen gemacht hatte, da er sich nicht vorstellen konnte, dass Kinder das tun durften, wenn es gefährlich war.

Das war aber nicht der Grund, warum Harry Schwierigkeiten hatte, das Abendessen zu genießen. …das Problem mit dem Selbstmitleid war, dass es nie lange dauerte, bis man jemand anderen fand, dem es noch schlechter ging. Dr~Leo Granger hatte irgendwann einmal gefragt, ob die nette Lehrerin, die Hermine zu mögen schien, Professor McGonagall, ihr viele Punkte in der Schule gegeben hatte. Hermine hatte das bejaht, mit einem scheinbar echten Lächeln. Harry hatte es mit einiger Mühe geschafft, sich davon abzuhalten, eisig darauf hinzuweisen, dass Professor McGonagall niemals irgendeinen Hogwarts-Schüler bevorzugen würde und dass Hermine viele Punkte bekam, weil sie sich jeden einzelnen verdient hatte. An einer anderen Stelle hatte Leo Granger am Tisch seine Meinung geäußert, dass Hermine sehr klug sei und auf die medizinische Fakultät hätte gehen und Zahnarzt werden können, wenn da nicht die ganze Hexensache wäre.

Hermine hatte wieder gelächelt, und ein kurzer Blick hatte Harry davon abgehalten, vorzuschlagen, dass Hermine auch eine international berühmte Wissenschaftlerin hätte werden können, und zu fragen, ob den Grangers dieser Gedanke gekommen wäre, wenn sie einen Sohn statt einer Tochter gehabt hätten, \emph{oder ob es so oder so inakzeptabel war, dass ihre Nachkommen es besser machten als sie.}

Aber Harry erreichte schnell seinen Siedepunkt. Und er lernte immer mehr zu schätzen, dass sein eigener Vater immer alles getan hatte, um Harrys Entwicklung als Wunderkind zu unterstützen, ihn immer ermutigt hatte, nach Höherem zu streben, und nie eine seiner Leistungen herabgesetzt hatte, auch wenn ein Wunderkind immer noch ein Kind war.

\emph{War das die Art von Haushalt, in dem er hätte landen können, wenn Mum Vernon Dursley geheiratet hätte?}

Harry tat jedoch, was er konnte.

„Und sie schlägt dich wirklich in all deinen Fächern außer Besenreiten und Verwandlung?“, fragte Professor Michael Verres-Evans.

„Ja“, sagte Harry mit erzwungener Ruhe, während er sich einen weiteren Bissen vom Heiligabend-Truthahn abschnitt. „Mit soliden Vorsprüngen, in den meisten von ihnen.“

Es gab andere Umstände, unter denen Harry das nur ungern zugegeben hätte, weshalb er bis jetzt nicht dazu gekommen war, es seinem Vater zu sagen.

„Hermine war schon immer recht gut in der Schule“, sagte Dr~Leo Granger in einem zufriedenen Ton.

„Harry nimmt an Wettkämpfen auf nationaler Ebene teil!“, sagte Professor Michael Verres-Evans.

„Oh je!“, sagte Petunia.

Hermine kicherte, und das trug nicht dazu bei, dass Harry sich in ihrer Situation besser fühlte. Es schien Hermine nicht zu stören, und das störte Harry.

„Es ist mir nicht peinlich, gegen sie zu verlieren, Dad“, sagte Harry. \emph{In diesem Moment war es das nicht}. „Habe ich schon erwähnt, dass sie alle ihre Schulbücher vor dem ersten Unterrichtstag auswendig gelernt hat? Und ja, ich habe sie getestet.“

„Ist das, äh, üblich für sie?“ sagte Professor Verres-Evans zu den Grangers.

„Oh ja, Hermine lernt immer alles auswendig“, sagte Dr~Roberta Granger mit einem fröhlichen Lächeln. „Sie kennt jedes Rezept in all meinen Kochbüchern auswendig. Ich vermisse sie jedes Mal, wenn ich Essen koche.“

Nach dem Gesichtsausdruck seines Vaters zu urteilen, fühlte Dad zumindest etwas von dem, was Harry fühlte.

„Mach dir keine Sorgen, Dad“, sagte Harry, „sie bekommt jetzt den ganzen fortgeschrittenen Stoff, den sie nehmen kann. Ihre Lehrer in Hogwarts wissen, dass sie klug ist, im Gegensatz zu ihren Eltern!“

Seine Stimme hatte sich bei den letzten drei Worten erhoben, und selbst als sich alle Gesichter umdrehten, um ihn anzustarren, und Hermine ihn wieder trat, wusste Harry, dass er es vermasselt hatte, aber es war zu viel, einfach viel zu viel.

„Natürlich wissen wir, dass sie klug ist“, sagte Leo Granger und begann, beleidigt auf das Kind zu blicken, das die Frechheit besessen hatte, an ihrem Esstisch seine Stimme zu erheben.

„Ihr habt nicht die geringste Ahnung“, sagte Harry, wobei das Eis nun in seine Stimme sickerte.

„Du denkst, sie liest viele Bücher und das ist niedlich, stimmt's? Du siehst ein perfektes Zeugnis und findest es gut, dass sie sich im Unterricht gut macht. Ihre Tochter ist die talentierteste Hexe ihrer Generation und der hellste Stern von Hogwarts, und eines Tages, Dr~und Dr~Granger, wird die Tatsache, dass Sie ihre Eltern waren, der einzige Grund sein, dass sich die Geschichte an Sie erinnert!“

Hermine, die ruhig von ihrem Platz aufgestanden und um den Tisch herumgegangen war, wählte diesen Moment, um Harry an der Schulter zu packen und ihn aus seinem Stuhl zu ziehen.

Harry ließ sich ziehen, aber als Hermine ihn wegzog, sagte er mit noch lauterer Stimme:

„Es ist gut möglich, dass in tausend Jahren die Tatsache, dass Hermine Grangers Eltern Zahnärzte waren, der einzige Grund sein wird, warum sich jemand an die Zahnmedizin erinnert!“

Roberta starrte mit einem geduldigen Blick auf ihrem jungen Gesicht dorthin, wo ihre Tochter gerade den Jungen-der-lebte aus dem Zimmer geschleppt hatte.

„Es tut mir furchtbar leid“, sagte Professor Verres mit einem amüsierten Lächeln.

„Aber bitte machen Sie sich keine Sorgen, Harry redet immer so. Sind sie nicht schon wie ein Ehepaar?“

\emph{Das Erschreckende daran war, dass sie es waren.}

Harry hatte mit einem ziemlich strengen Vortrag von Hermine gerechnet. Aber nachdem Hermine sie in den Kellerzugang gezogen und die Tür hinter ihnen geschlossen hatte, hatte sie sich umgedreht - und lächelte, aufrichtig, soweit Harry das beurteilen konnte.

„Bitte nicht, Harry“, sagte sie mit sanfter Stimme. „Auch wenn es sehr nett von dir ist. Es ist alles in Ordnung.“

Harry sah sie nur an.

„Wie hältst du das nur aus?“, fragte er. Er musste seine Stimme leise halten, sie wollten nicht, dass die Eltern es hörten, aber sie stieg in der Tonlage, wenn auch nicht in der Lautstärke

. „Wie kannst du das aushalten?!“

Hermine zuckte mit den Schultern und sagte:

„Weil Eltern so sein sollten?“

„Nein“, sagte Harry, seine Stimme tief und intensiv,

„das ist es nicht, mein Vater setzt mich nie herab - nun, er tut es, aber nie so—“

Hermine hielt einen einzelnen Finger hoch, und Harry wartete, beobachtete, wie sie nach Worten suchte. Es dauerte eine Weile, bis sie sagte:

"Harry… Professor McGonagall und Professor Flitwick mögen mich, weil ich die talentierteste Hexe meiner Generation und der hellste Stern von Hogwarts bin. Und Mum und Dad wissen das nicht, und du wirst es ihnen auch nie sagen können, aber sie lieben mich trotzdem. Was bedeutet, dass alles so ist, wie es sein sollte, in Hogwarts und zu Hause. Und da sie meine Eltern sind, Mr.

Potter, dürfen Sie nicht widersprechen."

Sie lächelte wieder ihr geheimnisvolles Lächeln vom Abendessen und sah Harry sehr liebevoll an.

„Ist das klar, Mr~Potter?“

Harry nickte heftig.

„Gut“, sagte Hermine, beugte sich vor und \emph{küsste ihn auf die Wange.}

Das Gespräch hatte gerade erst wieder begonnen, als ein entfernter, hoher Schrei zu ihnen zurückschwebte:

\textbf{„Hey! Nicht küssen!“}

Die beiden Väter brachen in Gelächter aus, während sich die beiden Mütter mit identischen Blicken des Entsetzens von ihren Stühlen erhoben und in Richtung Keller stürmten.

Als die Kinder zurückgebracht worden waren, sagte Hermine in eisigem Ton, dass sie \emph{Harry nie wieder küssen würd}e, und Harry sagte mit empörter Stimme, \emph{dass die Sonne zu kalter, toter Schlacke verbrennen würde, bevor er sie nahe genug herankommen ließe, um es zu versuchen.}

Was bedeutete, dass alles so war, wie es sein sollte, und sie setzten sich alle wieder hin, um ihr Weihnachtsessen zu beenden.

