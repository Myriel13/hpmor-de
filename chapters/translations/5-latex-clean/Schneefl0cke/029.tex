

\hypertarget{gruppenarbeit-teil-1}{% \section{30. Gruppenarbeit Teil 1}\label{gruppenarbeit-teil-1}}

\textbf{\uline{Gruppenarbeit, Teil 1}}

Es war Sonntag, der 3. November, und bald würden die drei großen Mächte des Schuljahres, Harry Potter, Draco Malfoy und Hermine Granger, ihren Kampf um die Vorherrschaft beginnen.

(Harry ärgerte sich ein wenig darüber, dass der Junge-der-lebte allein durch die Teilnahme an dem Wettbewerb von der obersten Dominanz zu einem von drei gleichwertigen Rivalen degradiert worden war, aber er rechnete damit, dass er sie bald zurückerhalten würde.)

Das Schlachtfeld war ein Abschnitt des nicht verbotenen Waldes, der dicht mit Bäumen bewachsen war, denn Professor Quirrell war der Meinung, dass es selbst für die allererste Schlacht zu langweilig war, alle feindlichen Kräfte sehen zu können.

Alle Schüler, die nicht gerade in einer Erstklässlerarmee waren, lagerten in der Nähe und sahen auf Bildschirmen zu, die Professor Quirrell aufgestellt hatte.

Bis auf drei Gryffindors aus dem vierten Jahr, die gerade krank waren und von Madam Pomfrey an die Betten gefesselt wurden. Abgesehen davon waren alle da. Die Schüler waren gekleidet, nicht in ihre normalen Schulroben, sondern in Muggel-Tarnuniformen, die Professor Quirrell irgendwo besorgt und in ausreichender Menge und Vielfalt bereitgestellt hatte, damit sie allen passten. Es war nicht so, dass sich die Schüler Sorgen um Flecken und Risse gemacht hätten, dafür waren Zaubersprüche da. Aber wie Professor Quirrell den überraschten Zaubergeborenen erklärt hatte, war schöne, gediegene Kleidung nicht dazu geeignet, sich in Wäldern zu verstecken oder um Bäume herumzuwuseln.

Und auf der Brust jeder Uniform ein Aufnäher mit dem Namen und den Insignien der eigenen Armee. Ein kleiner Aufnäher. Wenn man wollte, dass die Soldaten, sagen wir, farbige Bänder tragen, damit sie sich auf Distanz identifizieren konnten, und riskierten, dass der Feind die Bänder in die Finger bekam, war das deine Sache.

Harry hatte versucht, den Namen \emph{Drachenarmee} zu bekommen. Draco hatte einen Anfall bekommen und gesagt, das würde alle völlig verwirren.

Professor Quirrell hatte entschieden, dass Draco den Namen für sich beanspruchen könne, wenn er wolle.

Jetzt kämpfte Harry also gegen die Drachenarmee. Das war wahrscheinlich kein gutes Zeichen. Für ihr Abzeichen hatte Draco statt des allzu offensichtlichen feuerspeienden Drachenkopfs einfach das \emph{Feuer} gewählt.

Elegant, unaufdringlich, tödlich: Das ist, was übrig bleibt, nachdem wir hier waren.

\emph{Typisch Malfoy.}

Nachdem Harry Alternativen wie \emph{501st Provisional Battalion} und \emph{Harry's Minions of Doom} in Betracht gezogen hatte, entschied er, dass seine Armee unter der einfachen und würdevollen Bezeichnung \emph{Chaos Legion} bekannt sein würde.

Ihr Abzeichen war eine \emph{Hand, deren Finger zum Schnippen bereit} \emph{waren}.

Man war sich einig, dass dies kein gutes Zeichen war.

Harry hatte Hermine ernsthaft geraten, dass die Jungen, die unter ihr dienten, wahrscheinlich nervös waren, weil sie ein Mädchen war, das den Ruf hatte, nett zu sein, und dass sie etwas Furchteinflößendes wählen sollte, das sie von ihrer Härte überzeugen und sie stolz machen würde, Teil ihrer Armee zu sein, wie die \emph{Blood Commandos} oder so etwas.

Hermine hatte ihre Armee das \emph{Sonnenschein-Regiment} genannt. Ihr Abzeichen war ein \emph{Smiley-Gesicht}.

\emph{Und in zehn Minuten würden sie im Krieg sein.}

Harry stand auf der hellen Waldlichtung, die ihnen als Startplatz zugewiesen worden war, einer offenen Fläche mit alten und verrottenden Baumstümpfen, die für irgendeinen unbekannten Zweck weggeräumt worden waren, dem Boden, der mit einer kleinen Ansammlung von verwehten Blättern und den vertrockneten grauen Überresten von Gras bedeckt war, das den Test der Sommerhitze nicht bestanden hatte, und der Sonne, die strahlend von oben herab schien.

Um ihn herum standen die dreiundzwanzig Soldaten, die Professor Quirrell ihm zugeteilt hatte. Fast ganz Gryffindor hatte sich natürlich gemeldet, und mehr als die Hälfte von Slytherin, und weniger als die Hälfte von Hufflepuff, und eine Handvoll Ravenclaw. In Harrys Armee waren zwölf Gryffindors und sechs Slytherins und vier Hufflepuffs und ein Ravenclaw außer ihm selbst… nicht, dass man das beim Anblick der Uniformen hätte erkennen können. Kein Rot, kein Grün, kein Gelb, kein Blau.

Nur Muggel-Tarnmuster und ein Aufnäher auf der Brust mit der Darstellung einer Hand, die mit den Fingern schnippt.

Harry blickte auf seine dreiundzwanzig Soldaten, die alle die gleichen Uniformen trugen und außer diesem einen Aufnäher keine Zeichen der Gruppenidentität aufwiesen.

Und Harry lächelte, denn er verstand, worum es bei diesem Teil von Professor Quirrells Masterplan ging; und Harry nutzte ihn auch für seine eigenen Zwecke aus.

Es gab eine legendäre Episode in der Sozialpsychologie, die man das Räuberhöhlen-Experiment nannte.

Es war in den verwirrten Nachwehen des Zweiten Weltkriegs eingerichtet worden, mit der Absicht, die Ursachen und Abhilfen von Konflikten zwischen Gruppen zu untersuchen.

Die Wissenschaftler hatten ein Sommerlager für 22 Jungen aus 22 verschiedenen Schulen eingerichtet und sie so ausgewählt, dass sie alle aus stabilen Mittelschichtfamilien stammten.

In der ersten Phase des Experiments sollte untersucht werden, was nötig ist, um einen Konflikt zwischen Gruppen auszulösen. Die 22 Jungen waren in zwei Gruppen zu je 11 Personen aufgeteilt worden - \emph{und das hatte völlig ausgereicht.}

Die Feindseligkeit hatte von dem Moment an begonnen, als die beiden Gruppen von der Existenz der jeweils anderen Gruppe im Staatspark erfahren hatten, und bei der ersten Begegnung waren Beleidigungen gefallen.

Sie nannten sich selbst die \emph{Eagles} und die \emph{Rattlers} (sie brauchten keine Namen für sich selbst, als sie dachten, sie wären die einzigen im Park) und entwickelten gegensätzliche Gruppenstereotypen, wobei die \emph{Rattlers} sich für rau und hart hielten und viel fluchten, während die \emph{Eagles} sich dementsprechend für aufrecht und anständig hielten.

Der andere Teil des Experiments bestand darin, zu testen, wie Gruppenkonflikte gelöst werden können.

Die Jungs zusammenzubringen, um ein Feuerwerk zu sehen, hatte überhaupt nicht funktioniert. Sie hatten sich nur gegenseitig angeschrien und waren auseinander gegangen.

Was funktioniert hatte, war die Warnung, dass es Vandalen im Park geben könnte, und dass die beiden Gruppen zusammenarbeiten mussten, um eine Störung im Wassersystem des Parks zu beheben.

\emph{Eine gemeinsame Aufgabe, ein gemeinsamer Feind.} Harry hatte den starken Verdacht, dass Professor Quirrell dieses Prinzip sehr gut verstanden hatte, als er beschlossen hatte, drei Armeen pro Jahr zu schaffen.

Drei Armeen. Nicht vier. Und definitiv nicht nach Häusern getrennt… außer, dass Draco außer Mr~Crabbe und Mr~Goyle keine Slytherins zugeteilt worden waren. Es waren Dinge wie diese, die Harry die Gewissheit gaben, dass Professor Quirrell trotz seiner affektierten dunklen Atmosphäre und seiner vorgetäuschten Neutralität im Konflikt zwischen Gut und Böse insgeheim auf der Seite des Guten stand, nicht dass Harry es jemals gewagt hätte, das laut auszusprechen.

Und Harry hatte beschlossen, Professor Quirrells Plan, eine Gruppenidentität auf seine Weise zu definieren, voll auszunutzen.

Die Rattler hatten, nachdem sie die Eagles getroffen hatten, angefangen, sich selbst als rau und hart zu betrachten, und sie hatten sich dementsprechend verhalten.

Die Eagles hielten sich für gut und anständig. Und auf dieser hellen Waldlichtung, verstreut um die alten und verrottenden Baumstümpfe, die in der von oben herab strahlenden Sonne umrissen wurden, waren General Potter und seine dreiundzwanzig Soldaten in nichts angeordnet, was auch nur im Entferntesten einer Formation ähnelte.

Einige Soldaten standen, einige Soldaten saßen, einige standen auf einem Bein, nur um anders zu sein. Immerhin war es \emph{die Chaos Legion}. Und wenn es keinen Grund gab, in ordentlichen kleinen Reihen zu stehen, hatte Harry verächtlich gesagt, dann gab es auch keine ordentlichen kleinen Reihen.

Harry hatte die Armee in 6 Trupps zu je 4 Soldaten eingeteilt, jeder Trupp wurde von einem Truppführer befehligt. Alle Truppen hatten den strikten Befehl, jeden Befehl zu missachten, der ihnen gegeben wurde, auch wenn er in dem Moment eine gute Idee zu sein schien, einschließlich dieses Befehls.

.. es sei denn, Harry oder der Squad Anführer setzten dem Befehl den Zusatz

„\emph{Merlin sagt}“ voran, in diesem Fall sollte man tatsächlich gehorchen.

Der Hauptangriff der Chaos-Legion bestand darin, sich aufzuteilen und aus mehreren Richtungen heranzurennen, wobei man zufällig die Vektoren wechselte und den zugelassenen Schlafzauber so schnell abfeuerte, wie man die magische Kraft wieder aufbauen konnte.

Und wenn man eine Chance sah, den Feind abzulenken oder zu verwirren, nahm man sie wahr.

\emph{Schnell. Kreativ. Unberechenbar. Nicht-homogen. Nicht nur Befehle befolgen, sondern darüber nachdenken, ob das, was man gerade tut, sinnvoll ist.}

Harry war sich nicht ganz so sicher, wie er vorgab, dass dies das Optimum an militärischer Effizienz war

… \emph{aber er hatte eine einmalige Chance bekommen, die Denkweise einiger Schüler zu ändern, und die wollte er nutzen.}

Fünf Minuten bis zum Kriegsbeginn, laut Harrys Uhr. General Potter ging (nicht marschierte) hinüber zu dem Ort, an dem seine Luftwaffe angespannt wartete, die Besen bereits fest in den Händen haltend.

„Alle Flügel melden sich“, sagte General Potter. Sie hatten dies während ihrer einzigen Trainingseinheit am Samstag geprobt.

„Red Leader bereit“, sagte Seamus Finnigan, der keine Ahnung hatte, was das bedeutete.

„Red Five stand by“, sagte Dean Thomas, der sein ganzes Leben darauf gewartet hatte, es zu sagen.

„Green Leader bereit“, sagte Theodore Nott ziemlich steif.

„Green Forty-One bereit“, sagte Tracey Davis.

„Ich will, dass Ihr sofort in der Luft seid, wenn wir die Glocke hören“, sagte General Potter. „Nicht angreifen, ich wiederhole, nicht angreifen. Weicht aus, wenn ihr unter Beschuss geratet.“

(Natürlich zielte man nicht mit Schlafzaubern auf Besen; man feuerte einen Zauber ab, der alles, was er traf, vorübergehend rot glühen ließ. Wenn man den Besen oder den Reiter traf, waren sie außer Gefecht gesetzt.)

"Red Leader und Red Five, fliegt so schnell wie möglich auf Malfoys Armee zu, bleibt so hoch wie möglich, während ihr sie noch seht, und kehrt zurück, sobald ihr sicher wisst, was sie vorhaben.

Green Leader, tun Sie dasselbe für Grangers Armee. Green Fourty-One fliege über uns und halte Ausschau nach sich nähernden Besenn oder Soldaten, du und nur du bist autorisiert zu schießen.

Und denkt daran, ich habe nicht gesagt, \emph{'Merlin sagt'} für irgendetwas davon, aber wir brauchen die Informationen wirklich. \textbf{Für das Chaos!}"

„\textbf{Für das Chaos!}“, echoten die vier mit unterschiedlichem Grad an Begeisterung.

Harry erwartete, dass Hermine sofort einen Angriff auf Draco starten würde, in diesem Fall würde er seine Truppen in Position bringen und sie unterstützen, aber erst, nachdem sie schwere Verluste erlitten und einigen Schaden angerichtet hatte.

Er würde es, wenn möglich, als heldenhafte Rettung darstellen; es wäre nicht gut, wenn Sonnenschein denken würde, dass Chaos nicht ihr Freund war.

Aber nur für den Fall, dass sie es nicht tat… nun, das war der Grund, warum die Chaos Legion hier blieb, bis Green Leader sich zurückmeldete.

Dracos Schritte würden in seinem eigenen Interesse liegen. Er würde vorhersehbar seine Armee bereit machen, um sich gegen Hermine zu verteidigen; er würde vielleicht oder vielleicht auch nicht merken, dass Harry gelogen hatte, indem er mit dem Angriff wartete, bis die Schlacht vorbei war.

Harry hatte immer noch zwei Besen auf die Drachenarmee gesetzt, nur für den Fall, dass sie etwas tun würden, und nur für den Fall, dass Draco oder Mr~Goyle oder Mr~Crabbe gut genug waren, einen Besen vom Himmel zu schießen. Aber General Granger war die Unberechenbare, und Harry konnte sich nicht bewegen, bevor er nicht wusste, wie sie sich bewegte.

Im Herzen des Waldes, wo Schattenmuster auf dem Boden tanzten, während hoch oben die Blätterdächer schwankten, stand General Malfoy dort, wo die Bäume relativ spärlich waren, und blickte mit ruhiger Zufriedenheit auf seine Truppen.

Sechs Einheiten zu je drei Mann, die vierköpfige Lufteinheit (der Gregory zugeteilt war) und die Kommando Einheit, die aus ihm und Vincent bestand.

Sie hatten am vergangenen Samstag nur kurz geübt, aber Draco war zuversichtlich, dass er es geschafft hatte, die Grundlagen zu erklären.

\emph{Bleib bei deinen Kameraden, halte ihnen den Rücken frei und vertraue darauf, dass sie auf dich aufpassen. Bewegt euch als ein Körper. Gehorche den Befehlen und zeige keine Angst. Zielen, feuern, bewegen, wieder zielen, wieder feuern.}

Die sechs Einheiten formierten sich in einem defensiven Perimeter um Draco und blickten wachsam in den Wald hinaus.

Rücken an Rücken standen sie, die Zauberstäbe tief gegriffen, bis sie zuschlagen mussten. Sie sahen den Auroren-Einheiten, deren Training Draco bei den Inspektionen seines Vaters beobachtet hatte, schon verblüffend ähnlich.

\emph{Chaos und Sonnenschein würden nicht wissen, was sie getroffen hatte.}

„\textbf{Achtung!}“, sagte General Malfoy. Die sechs Einheiten entfalteten sich und drehten sich auf Draco zu; die Gesichter seiner Besenreiter wandten sich von dort ab, wo sie mit den Besen in der Hand standen.

Draco hatte beschlossen, mit dem Einfordern von Grüßen zu warten, bis sie ihre erste Schlacht gewonnen hatten, wenn Gryffindors und Hufflepuffs eher bereit sein würden, einem Malfoy zu salutieren.

Aber seine Soldaten standen schon so aufrecht, vor allem die Gryffindors, dass Draco sich fragte, ob er überhaupt zu warten brauchte.

Gregory hatte im Stillen zugehört und zurückgemeldet, dass Dracos freiwillige Bereitschaft, Harry Potter im Verteidigungsunterricht beizustehen, damals, als Professor Quirrell Harry beigebracht hatte, wie man verliert, Draco als akzeptablen Kommandanten gekennzeichnet hatte.

Zumindest, wenn man zufällig seiner Armee zugeteilt wurde. Nicht alle Slytherins sind gleich; es gibt Slytherins, und dann gibt es Slytherins, war das, was die Gryffindors in Dracos Armee zu ihren Hauskameraden sagten.

Draco war ehrlich gesagt verblüfft, wie unglaublich einfach das gewesen war. Draco hatte anfangs protestiert, weil ihm keine Slytherins zugeteilt worden waren, und Professor Quirrell hatte ihm gesagt, dass er, wenn er der erste Malfoy sein wollte, der die vollständige politische Kontrolle über das Land erlangte, lernen musste, wie man die anderen drei Viertel der Bevölkerung regiert.

Es waren Dinge wie diese, die Draco darin bestärkten, dass Professor Quirrell viel mehr Sympathie für die Guten hatte, als er zugeben wollte.

Der eigentliche Kampf würde nicht einfach werden, besonders wenn Granger die Drachen zuerst angreifen würde.

Draco hatte darüber gegrübelt, ob er alle seine Kräfte sofort in einem Präventivschlag gegen Granger einsetzen sollte, aber er hatte sich Sorgen gemacht, dass (1) Harry ihn völlig in die Irre geführt hatte, was Granger wahrscheinlich tun würde, und (2) Harry ihn in die Irre geführt hatte, dass er bis nach Grangers Angriff warten sollte, um sich dem Kampf anzuschließen.

Allerdings hatte die Drachenarmee eine Geheimwaffe, sogar drei davon, die ausreichen könnten, um zu gewinnen, selbst wenn sie von beiden Armeen gleichzeitig angegriffen würden.

.. Es war fast soweit, und das bedeutete, dass es Zeit für die Rede vor der Schlacht war, die Draco verfasst und auswendig gelernt hatte.

„Die Schlacht wird gleich beginnen“, sagte Draco. Seine Stimme war ruhig und präzise. "Denkt an alles, was ich und Mr~Crabbe und Mr~Goyle euch gezeigt haben. Eine Armee gewinnt, weil sie diszipliniert und tödlich ist.

General Potter und die Chaoslegion werden nicht diszipliniert sein.

Granger und das Sonnenschein Regiment werden nicht tödlich sein.

Wir sind diszipliniert, wir sind tödlich, wir sind Drachen. Die Schlacht wird beginnen und wir werden sie gewinnen.„

(Ex tempore Ansprache von General Potter an die Chaos Legion, unmittelbar vor ihrer ersten Schlacht, am 3. November 1991, um 14:56~Uhr:)

Meine Truppen, ich werde euch nicht anlügen, unsere heutige Situation ist sehr düster. Die Drachenarmee hat noch nie eine einzige Schlacht verloren. Und Hermine Granger… hat ein sehr gutes Gedächtnis.

Die Wahrheit ist, die meisten von euch werden wahrscheinlich sterben. Und die Überlebenden werden die Toten beneiden.

Aber wir müssen das hier gewinnen. Wir müssen gewinnen, damit unsere Kinder eines Tages wieder Schokolade essen können.

Hier steht alles auf dem Spiel. Buchstäblich alles. Wenn wir verlieren, geht das ganze Universum aus wie eine Glühbirne.

Und jetzt weiß ich, dass die meisten von euch nicht wissen, was eine Glühbirne ist. Nun, glaubt mir, es ist schlimm.

Aber wenn wir untergehen müssen, lasst uns kämpfend untergehen, wie Helden, so dass wir, wenn die Dunkelheit näher kommt, denken können, dass wir wenigstens Spaß hatten.

Habt ihr Angst vorm Sterben? Ich weiß, dass ich es habe. Ich spüre diese kalten Schauer der Angst, als würde jemand Eiscreme in mein Hemd pumpen.

Aber ich weiß… dass die Geschichte uns beobachtet. Sie hat uns beobachtet, als wir unsere Uniformen angezogen haben. Sie hat wahrscheinlich Fotos gemacht. Und Geschichte, meine Truppen, wird von den Siegern geschrieben.

Wenn wir das hier gewinnen, können wir unsere eigene Geschichte schreiben. Eine Geschichte, in der Hogwarts von 4 abtrünnigen Hauselfen gegründet wurde.

Wir können alle zwingen, diese Geschichte zu lernen, auch wenn sie nicht wahr ist. Und wenn sie bei unseren Tests nicht richtig antworten, fallen sie durch.

Ist es das nicht wert, dafür zu sterben?

Nein, antworte nicht darauf. Manche Dinge bleiben besser unbekannt.

Keiner von uns weiß, warum wir hier sind. Keiner von uns weiß, warum wir kämpfen. Wir sind einfach in diesen Uniformen in diesem mysteriösen Wald aufgewacht und wussten nur, dass es keinen Weg gab, unsere Namen und Erinnerungen zurückzubekommen, außer dem Sieg.

Die Studenten in den anderen Armeen da draußen… sie sind genau wie wir. Sie wollen nicht sterben. Sie kämpfen, um sich gegenseitig zu beschützen, die einzigen Freunde, die sie noch haben. Sie kämpfen, weil sie wissen, dass sie Familien haben, die sie vermissen werden, auch wenn sie sich jetzt nicht erinnern können.

Sie kämpfen vielleicht sogar, um die Welt zu retten. Aber wir haben einen besseren Grund zu kämpfen, als sie es tun.

\emph{Wir kämpfen, weil es uns Spaß macht}.

Wir kämpfen, um uralte Monstrositäten jenseits von Raum und Zeit zu amüsieren.

Wir kämpfen, weil wir das Chaos sind. Bald beginnt die letzte Schlacht, also lasst mich jetzt sagen, weil ich später keine Gelegenheit mehr haben werde, dass es eine Ehre war, euer Kommandant zu sein, wie kurz auch immer.

Ich danke euch, ich danke euch allen. Und vergesst nicht: Euer Ziel ist es nicht nur, den Feind auszuschalten, sondern ihm Angst zu machen.“

Ein großer dröhnender Gong hallte über den Wald. Und das Sonnenschein Regiment begann zu marschieren.

Die Spannung stieg und stieg, während Harry und die neunzehn anderen Soldaten, die übrig geblieben waren, darauf warteten, dass sich die Luftkrieger zurückmeldeten.

Es sollte nicht lange dauern, Besen waren schnell und die Entfernungen im Wald waren nicht groß - Zwei Besen näherten sich mit hoher Geschwindigkeit aus der Richtung von Dracos Lager, und alle Soldaten spannten sich an.

Sie führten nicht die Manöver aus, die heute der Code für einen freundlichen Besen waren.

„Verteilen und feuern!“, brüllte General Potter, und dann ließ er den Worten Taten folgen und huschte mit Höchstgeschwindigkeit in Richtung des Waldes; und sobald Harry zwischen den Bäumen war, drehte er sich zurück, hob seinen Zauberstab und versuchte, den Besen am Himmel zu suchen - „Frei!“, rief eine Stimme.

„Sie sind auf dem Rückweg!“

Harry zuckte innerlich mit den Schultern. Es hatte keine Möglichkeit gegeben, Draco daran zu hindern, diese Information zu erhalten, und er würde nur erfahren, dass sie stillstanden.

Und die Chaoten tauchten langsam aus dem Wald auf—

„Besen nähert sich aus Grangers Richtung!“, schrie eine andere Stimme.

„Ich glaube, es ist der Green Leader, er hat die Rolle gemacht!“

Augenblicke später tauchte Theodore Nott aus dem Himmel und tauchte inmitten der Soldaten auf.

„Granger hat ihre Streitkräfte in zwei Hälften geteilt!“, schrie Nott, während er auf seinem Besen schwebte. Schweiß befleckte seine Uniform, und die ganze Zurückhaltung war aus seiner Stimme verschwunden.

„Sie greift beide Armeen an! Zwei Besen decken jede Streitmacht ab, sie haben mich auf halbem Weg hierher verfolgt!“

\emph{Ihre Armee geteilt, was um alles in der Welt} - \emph{eine große Streitmacht, die das Feuer auf eine kleine Streitmacht konzentriert, kann diese Streitmacht schnell dezimieren, ohne dass sie dafür viel Schaden nimmt.}

Wenn zwanzig Soldaten zehn Soldaten gegenüberstanden, würden zwanzig Schlafzauber auf die zehn Soldaten gerichtet sein, während nur zehn Schlafzauber in die andere Richtung gingen; wenn also nicht jeder dieser ersten Schlafzauber sein Ziel traf, würde die kleinere Truppe mehr Leute verlieren, als sie mitnehmen konnte.

\emph{Defeat in detail} war der militärische Ausdruck für das, was passierte, wenn man seine Kräfte so aufteilte.

\emph{Was könnte Hermine sich nur dabei denken.}.. Dann wurde Harry klar.

\emph{Sie ist nur fair. Es würde ein langes Jahr im Verteidigungsunterricht werden.}

„Also gut“, sagte Harry laut, so dass die Armee es hören konnte.

„Wir warten, bis der rote Flügel sich meldet, und dann gehen wir uns ein bisschen Sonnenschein holen.“

Draco hörte sich die Berichte der Flieger mit ruhiger Miene an, sein ganzer Schock war darin verborgen.

\emph{Was dachte sich Granger dabei?}

Dann wurde es Draco klar.

\emph{Es ist eine Finte. Eine von Sonnenscheins beiden Kräften würde die Richtung ändern, und beide würden sich auf…wen stürzen?}

Neville Longbottom marschierte durch den Wald auf die sich nähernde Sonnen-Truppe zu, wobei er ab und zu einen Blick in den Himmel warf, um nach Besen Ausschau zu halten.

Neben ihm marschierten seine Truppkameraden, Melvin Coote und Lavender Brown aus Gryffindor und Allen Flint aus Slytherin.

Allen Flint war ihr Anführer, obwohl Harry zuerst zu Neville gesagt hatte, unter vier Augen, dass die Position ihm zustand, wenn er sie wollte.

Harry hatte unter vier Augen ziemlich viel zu Neville gesagt, angefangen mit:

„\emph{Weißt du, Neville, wenn du so fantastisch werden willst wie der imaginäre Neville,} \emph{der in deinem Kopf lebt, aber nichts tun darf, weil du Angst hast, dann solltest du dich wirklich für Professor Quirrells Armeen anmelden.}“

Neville war sich nun sicher, dass der Junge-der-lebte Gedanken lesen konnte.

Es gab einfach keine andere Möglichkeit, wie Harry Potter das hätte wissen können. Neville hatte nie mit jemandem darüber gesprochen oder Anzeichen dafür gegeben; und andere Leute waren nicht so, nicht dass Neville es je bemerkt hätte.

Und Harrys Versprechen war wahr geworden, das fühlte sich anders an als das Sparring im Verteidigungsunterricht.

Neville hatte gehofft, dass Sparring alles in Ordnung bringen würde, was mit ihm nicht stimmte, und, nun ja, das hatte es nicht.

Selbst wenn er in der Klasse ein paar Zauber auf einen anderen Schüler abfeuern konnte, während Professor Quirrell zusah, um sicherzugehen, dass nichts schief ging, selbst wenn er ausweichen und zurückfeuern konnte, wenn es erlaubt war und alle anderen es erwarteten und sie ihn komisch anstarrten, wenn er es nicht tat, war nichts davon dasselbe, wie für sich selbst einstehen zu können.

Aber Teil einer Armee zu sein… \emph{Irgendetwas Seltsames regte sich in Neville,} während er neben seinen Kameraden durch den Wald marschierte, auf ihren Uniformen ein Abzeichen, das zum Schnippen bereit war.

Er durfte gehen, wenn er wollte, aber ihm war einfach nach Marschieren zumute. Neben ihm schienen Melvin und Lavender und Allen auch Lust zu haben, zu marschieren. Und Neville begann leise, das Lied des Chaos zu singen. Die Melodie war das, was ein Muggel als \emph{John Williams' Imperial March'} identifiziert hätte, auch bekannt als „\emph{Darth Vader's Theme}“; und die Worte, die Harry hinzugefügt hatte, waren leicht zu merken.

\emph{Doom doom doom doom doom doom doom doom doom doom doom doom doom doom doom doom doom doom doom doom doom doom doom doom doom doom doom doom doom doom doom doom doom doom doom doom doom doom doom doom doom doom doom doom doom doom doom doom doom doom doom doom doom doom doom doom doom doom doom doom.}

Bei der zweiten Reihe stimmten die anderen mit ein, und bald konnte man den gleichen leisen Gesang aus den nahe gelegenen Teilen des Waldes hören.

Und Neville marschierte an der Seite seiner Chaos-Legionäre, seltsame Gefühle regten sich in seinem Herzen, Fantasie wurde Wirklichkeit, als von seinen Lippen ein furchterregendes Lied des Untergangs drang.

Harry starrte auf die Leichen, die im Wald verstreut lagen. Irgendetwas in ihm fühlte sich etwas mulmig an, und er musste sich hart daran erinnern, dass sie nur schliefen.

Es waren Mädchen unter den Gefallenen, und das machte es irgendwie noch viel schlimmer, und er würde aufpassen müssen, das niemals vor Hermine zu erwähnen, sonst würden die Auroren seine Überreste in eine kleine Teekanne gestopft finden.

Die halbe Sonnenschein-Armee hatte sich nicht besonders gut gegen das ganze Chaos gewehrt. Die neun Bodensoldaten waren mit erhobenen einfachen Schilden, kreisrunden Schirmen, die ihre Gesichter und Brust schützen sollten, schreiend hereingerannt.

Aber man konnte nicht gleichzeitig schießen und den Schild halten, und Harrys Soldaten hatten einfach auf die Beine gezielt.

Alle bis auf einen der Sonnen waren umgefallen, sobald die Schreie von „Somnium!“

die Luft erfüllten.

Diese letzte hatte ihren Schild fallen lassen und schaffte es, einen von Harrys Soldaten auszuschalten, bevor sie von der zweiten Welle von Schlafzaubern getroffen wurde (die Schlafverhexung war sicher bei mehreren Treffern).

Die beiden Besen von Sonnenschein waren viel schwieriger auszuschalten gewesen und hatten drei Chaoten erledigt, bevor sie vom massiven Bodenfeuer erfasst wurden.

Hermine war nicht unter den Gefallenen. Draco musste sie erwischt haben und das machte Harry auf einer völlig unverständlichen Ebene wütend, er war sich nicht sicher, ob er sich beschützend gegenüber Hermine fühlte oder betrogen, dass er nicht derjenige gewesen war, der es getan hatte, oder vielleicht beides.

„In Ordnung“, sagte Harry und erhob seine Stimme. „Damit eins klar ist: Das war kein echter Kampf. Das war General Granger, die in ihrem ersten Kampf einen Fehler gemacht hat. Der eigentliche Kampf heute ist mit der Drachenarmee und es wird nicht so sein wie hier. \emph{Es wird viel mehr Spaß machen.} Lasst uns loslegen.“

Ein Besen fiel vom Himmel, näherte sich erschreckend schnell, drehte sich auf seinem Ende und bremste so stark ab, dass man fast die Luft vor Protest schreien hören konnte, und kam direkt neben Draco zum Stehen.

Das war keine gefährliche Angeberei. Gregory Goyle war einfach so gut und er verschwendete keine Zeit.

„Potter kommt“, sagte Gregory mit keiner Spur seines üblichen falschen Tonfalls.

„Sie haben noch alle ihre vier Besen, soll ich sie ausschalten?“

„Nein“, sagte Draco scharf. „Über ihre Armee hinweg zu kämpfen, gibt ihnen einen zu großen Vorteil, sie werden vom Boden aus auf dich feuern und selbst du könntest nicht in der Lage sein, dem Ganzen auszuweichen. Wartet, bis die Streitkräfte angreifen.“

Draco hatte vier Drachen im Tausch gegen zwölf Sonnen verloren.

Anscheinend war General Granger tatsächlich so unglaublich dumm gewesen, obwohl sie nicht unter den Angreifern gewesen war, also hatte Draco keine Gelegenheit gehabt, sie zu verspotten oder zu fragen, was in Merlins Namen sie sich dabei gedacht hatte. Der wahre Kampf, das wussten sie alle, würde mit Harry Potter stattfinden.

„Macht euch bereit!“, brüllte Draco seine Truppen an.

„Bleibt mit euren Kameraden zusammen, handelt als Einheit, feuert, sobald der Feind in Reichweite ist!“

\emph{Disziplin gegen Chaos.} \emph{Es würde kein großer Kampf werden.}

Das Adrenalin pumpte und pumpte in Nevilles Blut, bis er das Gefühl hatte, kaum noch atmen zu können.

„Wir kommen näher“, sagte General Potter mit einer Stimme, die kaum laut genug war, um sie an die ganze Armee zu übertragen.

„Zeit, sich zu verteilen.“

Nevilles Kameraden entfernten sich von ihm. Sie würden sich immer noch gegenseitig unterstützen, aber wenn man sich zusammenrottete, würde der Feind es viel leichter haben, einen zu treffen; Feuer, das auf einen der Kameraden gerichtet war, könnte daneben gehen und stattdessen dich treffen.

Sie wären viel schwerer zu treffen, wenn Sie sich ausbreiten und so schnell wie möglich bewegen würden.

Das erste, was General Potter während ihrer Trainingseinheit getan hatte, war, sie dazu zu bringen, aufeinander zu feuern, wenn beide Seiten schnell liefen, oder beide still standen und sich Zeit zum Zielen nahmen, oder einer sich bewegte und einer still stand - der umgekehrte Zauber zur Schlafverhexung war einfach, obwohl man ihn während der Kämpfe nicht benutzen durfte.

General Potter hatte alles, was geschah, sorgfältig aufgezeichnet, etwas gerechnet und dann verkündet, dass es sinnvoller sei, wenn sie sich nicht darauf konzentrierten, langsamer zu werden, um sorgfältig zu zielen, sondern sich schnell zu

bewegen, um nicht getroffen zu werden.

Es störte Neville immer noch ein wenig, nicht Seite an Seite mit seinen Kameraden zu marschieren, aber die furchteinflößenden Schlachtrufe, die sie gelernt hatten, dröhnten bereits in seinem Kopf und das machte einiges wieder wett.

Diesmal, so schwor sich Neville im Stillen, würde seine Stimme ganz bestimmt nicht quietschen.

„Schilde hoch“, sagte General Potter, „Energie auf die vorderen Deflektoren.“

„Contego“, murmelte die Armee, und die kreisrunden Schilde entstanden vor ihren Köpfen und Brustkörpern.

Ein scharfer Geschmack erfüllte Nevilles Mund. General Potter hätte ihnen nicht befohlen, Schilde zu werfen, wenn sie nicht fast in Reichweite wären.

Neville konnte die uniformierten Gestalten der Drachen sehen, die sich durch die dichten Schirme der Bäume bewegten, und die Drachen würden sie ebenfalls sehen—

„\textbf{Angriff}!“, kam ein Schrei aus der Ferne, die Stimme von Draco Malfoy, und General Potter brüllte:

„\textbf{Angriff}—“

Das gesamte Adrenalin in Nevilles Blut wurde freigesetzt und seine Beine übernahmen die Kontrolle.

Er lief schneller, als er jemals zuvor gelaufen war, direkt auf den Feind zu und wusste, ohne nachzusehen, dass alle seine Kameraden dasselbe taten.

„\textbf{Blut für den Blutgott!}“, schrie Neville.

"\textbf{Schädel für den Schädelthron!}

\textbf{Iäh! Shub-Niggurath!}

\textbf{Die Ziege mit den tausend Jungen!}"

Es gab einen lautlosen Aufprall, als sich ein Schlafzauber gegen Nevilles Schild verirrte.

Wenn noch andere Zauber abgefeuert worden waren, hatten sie nicht getroffen. Neville sah den kurzen Ausdruck von Angst auf Wayne Hopkins' Gesicht, als er neben zwei Gryffindors stand, die Neville nicht erkannte, und dann—

—Neville ließ den Einfachen Schild fallen und feuerte auf Wayne—

—\emph{verfehlte}—

—seine rasenden Beine gingen direkt an der feindlichen Gruppierung vorbei und auf drei weitere Drachen zu, deren Zauberstäbe auf ihn zukamen, ihre Münder öffneten sich—

—ohne überhaupt darüber nachzudenken, tauchte Neville auf den Waldboden hinunter, gerade als drei Stimmen „Somnium!“ riefen. Es tat weh, harte Steine und harte Zweige gruben sich in Neville, als er sich abrollte, es war nicht so schlimm wie der Sturz von seinem Besen, aber er war immer noch ziemlich hart auf dem Boden aufgeschlagen, und dann lag Neville mit plötzlicher Einsicht still und schloss die Augen.

„Hör auf damit!“, schrie eine Stimme. „Erschießt uns nicht, wir sind Drachen!“

Mit einem Anflug von glorreicher Genugtuung erkannte Neville, dass er es geschafft hatte, zwischen zwei Gruppen von Drachen zu geraten, gerade als die eine Gruppe auf ihn geschossen hatte.

Harry hatte davon gesprochen, dass dies eine Taktik sei, um den Feind in Angst und Schrecken zu versetzen, aber anscheinend funktionierte es noch ein bisschen besser als das.

Und nicht nur das, die Drachen glaubten, sie hätten ihn erwischt, denn sie sahen Neville fallen, als sie gerade schossen.

Neville zählte in seinem Kopf bis zwanzig, dann öffnete er die Augen einen Spalt. Die drei Drachen waren ganz in seiner Nähe, die Köpfe drehten sich schnell, als Schreie von „\textbf{Somnium}!“ und „\textbf{Schädel für den Schädelthron!} “ die Luft um sie herum erfüllten.

Alle drei hatten jetzt einfache Schilde oben.

Nevilles Zauberstab war immer noch in seiner Hand, und es kostete ihn nicht viel Mühe, ihn auf die Stiefel des einen Jungen zu richten und „\emph{Somnium}“ zu flüstern.

Neville schloss schnell die Augen und entspannte seine Hand, als er hörte, wie der Junge zu Boden fiel.

„Wo kommt das her?“, schrie Justin Finch-Fletchleys Stimme, und Neville hörte Rascheln auf dem belaubten Waldboden, wie von zwei Drachen, die sich auf der Suche nach einem Feind umdrehten.

„\textbf{Reformiert die Reihen!}“, brüllte Malfoys Stimme.

„\textbf{Alle zu mir, lasst euch nicht verstreuen!}“

Nevilles Ohren hörten, wie die beiden Drachen tatsächlich über seinen liegenden Körper sprangen, als sie losrannten.

Neville öffnete die Augen, drückte sich etwas mühsam auf die Beine, dann richtete er seinen Zauberstab und sagte den neuen Zauber, den General Potter ihnen allen beigebracht hatte.

Sie konnten zwar keine echten Illusionszauber, um den Feind zu verwirren, aber selbst in ihrem Alter konnten sie das - „\emph{Ventriliquo}“, flüsterte Neville, zeigte mit dem Zauberstab auf eine Seite von Justin und dem anderen Jungen und rief dann:

„\textbf{Für Cthulhu und Ruhm!}“

Justin und der andere Junge blieben abrupt stehen und drehten ihre Schilde in Richtung der Stelle, an die Neville seinen Schlachtruf gerichtet hatte, und das war der Moment, in dem mehrere „\textbf{Somnium}!“-Rufe die Luft erfüllten und der andere Junge zu Boden ging, bevor Neville mit dem Zielen fertig war.

„\textbf{Der letzte gehört mir!}“, schrie Neville, und dann sprintete er direkt auf Justin zu, der gemein zu ihm gewesen war, bis die älteren Hufflepuffs ihn zurechtwiesen.

Neville war von seinen Kameraden umringt, und das bedeutete—

„\textbf{Spezialangriff, Chaotischer Sprung!}“, brüllte Neville, während er rannte, und fühlte, wie sein Körper leichter wurde, dann noch einmal leichter, als seine Kameraden ihre Zauberstäbe auf ihn richteten und leise den Schwebezauber wirkten, und Neville hob seine linke Hand und schnippte mit den Fingern, dann stieß er sich mit den Beinen so stark vom Boden ab, wie er konnte, und schwebte durch die Luft.

Schierer Schock malte Justins Gesicht, als Neville über den Schild des anderen Jungen hinwegging und seinen Zauberstab auf die unter ihm vorbeiziehende Gestalt richtete und „\textbf{Somnium!“} rief.

Weil ihm danach zumute war, das war der Grund.

Neville bekam seine Füße nicht ganz richtig herum und pflügte bei der Landung eher in den Boden, aber zwei von drei der anderen Chaoslegionäre hatten es geschafft, ihre Zauberstäbe durchgehend auf ihn zu richten, und die Landung war nicht zu hart.

Neville kam keuchend auf die Beine. Er wusste, dass er sich bewegen sollte, überall riefen die Leute „\textbf{Somnium}!“—

„\textbf{Ich bin Neville, der letzte Spross von Longbottom!}“, schrie Neville in den Himmel über ihm, hielt seinen Zauberstab gerade nach oben, als wolle er den flammend blauen Himmel selbst herausfordern, wissend, dass nach diesem Tag nichts mehr so sein würde wie vorher.

„\textbf{Neville von Chaos! Stellt euch mir, wenn …}“

(Als Neville danach aufwachte, erfuhr er, dass die Drachenarmee dies als ihr Stichwort für einen Gegenangriff genommen hatte).

Das Mädchen neben Harry sackte zu Boden, als es den für ihn bestimmten Schuss abbekam, und er konnte Mr~Goyles schadenfrohes Lachen aus der Ferne hören, als sein Besen an ihnen vorbeisauste und die Luft so stark zerschnitt, dass sie in seinem Kielwasser hätte zerschellen müssen.

„Luminos!“, schrie einer der Jungen neben Harry, der die magische Kraft nicht schnell \textbf{genug} hatte wieder aufbauen können, und Mr~Goyle wich ihm ohne Pause aus. Das Chaos hatte jetzt nur noch sechs Soldaten, und die Drachenarmee hatte zwei, und das einzige Problem war, dass einer dieser Soldaten unbesiegbar war, und der andere verbrauchte drei Soldaten, nur um ihn mit seinem Schild zu decken.

Sie hatten mehr Soldaten an Mr~Goyle verloren als an alle anderen Drachen zusammen, er wich so schnell durch die Luft aus, dass ihn niemand treffen konnte, \emph{und er konnte dabei auf Leute schießen.}

Harry hatte an alle möglichen Möglichkeiten gedacht, Mr~Goyle aufzuhalten, aber keine davon war sicher, selbst die Verwendung des Schwebezaubers, um ihn zu verlangsamen (es war ein kontinuierlicher Strahl und viel einfacher zu zielen), wäre nicht sicher, weil er vom Besen fallen könnte, Dinge in seinen Weg zu werfen wäre nicht sicher, und es wurde immer schwieriger, sich daran zu erinnern, während Harrys Blut gefror.

\emph{Es ist ein Spiel. Du versuchst nicht, ihn zu töten. Werf nicht alle deine Zukunftspläne für ein Spiel weg…}

Harry konnte das Muster sehen, er konnte sehen, wie Mr~Goyle weben würde, er konnte sehen, wie und wann sie alle feuern mussten, um ein Netz von Schüssen zu erzeugen, dem Mr~Goyle nicht ausweichen konnte, aber er hatte es seinen Soldaten nicht schnell genug erklären können, sie konnten ihre Schüsse nicht gut genug koordinieren, und jetzt hatten sie nicht mehr genug Leute übrig, um es zu tun—

\emph{ich weigere mich zu verlieren, nicht so, nicht meine ganze Armee gegen einen einzigen Soldaten!}

Mr~Goyles Besen drehte sich schneller, als irgendetwas hätte drehen können, und begann, sich auf Harry und seine überlebenden Truppen zuzubewegen, er konnte spüren, wie sich der Junge neben ihm anspannte und sich bereit machte, sich vor seinen General zu werfen.

\textbf{\emph{VERDAMMT}}! Harrys Zauberstab hob sich und konzentrierte sich auf Mr~Goyle, Harrys Geist visualisierte das Muster und Harrys Lippen öffneten sich und seine Stimme schrie - „Luminosluminosluminosluminosluminosluminosluminosluminosluminosluminosluminosluminosluminosluminosluminosluminosluminosluminosluminosluminos—“

Als Harrys Augen sich wieder öffneten, fand er sich in einer bequemen Position mit über der Brust gefalteten Händen wieder, seinen Zauberstab wie ein gefallener Held haltend. Langsam setzte sich Harry auf. Seine Magie schmerzte, eine seltsame, aber nicht ganz unangenehme Empfindung, ähnlich wie das Brennen und die Lethargie, die auf hartes körperliches Training folgten.

„Der General ist wach!“, rief eine Stimme, und Harry blinzelte und konzentrierte sich in diese Richtung. Vier seiner Soldaten hielten ihre Zauberstäbe als eine schimmernde prismatische Halbkugel, und Harry wurde klar, dass die Schlacht noch nicht vorbei war.

Richtig… er war nicht von einer Schlafverhexung getroffen worden, er hatte sich nur erschöpft, und als er aufwachte, war er immer noch im Spiel. Harry vermutete, dass er von irgendjemandem einen Vortrag darüber bekommen würde, dass er seine Magie nicht wegen eines Kinderspiels bis zur Bewusstlosigkeit ausreizen sollte. Aber er hatte Mr~Goyle nicht verletzt, als er seine Beherrschung verloren hatte, und das war das Wichtigste. Dann fiel Harry eine weitere Verwicklung ein, und er schaute auf den Stahlring am kleinen Finger seiner linken Hand hinunter und fluchte fast laut, als er sah, dass der winzige Diamant fehlte und ein Marshmallow in der Nähe der Stelle, an der er gefallen war, auf dem Boden lag. Er hatte diese Verwandlung siebzehn Tage lang ausgehalten und würde nun wieder von vorne anfangen müssen.

\emph{Es hätte schlimmer kommen können.}

Er hätte das auch vierzehn Tage später machen können, nachdem Professor McGonagall ihm erlaubt hatte, den Stein seines Vaters zu verwandeln. Das war eine sehr gute Lektion, um sie auf die einfache Art zu lernen.

\emph{Notiz an mich selbst: Immer den Ring vom Finger nehmen, bevor man die Magie vollständig aufbraucht.}

Harry drückte sich hoch und hatte dabei ziemliche Mühe. Seine Magie zu verbrauchen, erschöpfte die Muskeln nicht, aber zwischen Bäumen herumrennen, schon. Er taumelte zu der schillernden Halbkugel hinüber, in der Draco Malfoy stand, der seinen Zauberstab in die Höhe hielt, um den Schild zu stützen, und Harry kalt anlächelte.

„Wo ist der fünfte Soldat?“, fragte Harry.

„Ähm…“, sagte ein Junge, an dessen Namen sich Harry im Moment nicht erinnern konnte.

„Ich habe eine Schlafverhexung auf den Schild abgefeuert und sie ist abgeprallt und hat Lavender getroffen, ich meine, der Winkel hätte nicht stimmen dürfen, aber das hat er geschafft…“

Draco grinste innerhalb des Schildes.

„Lass mich raten“, sagte Harry und sah Draco direkt in die Augen, „diese netten kleinen Trios sind die Formation, die von professionellen magischen Streitkräften benutzt wird? Bestehend aus ausgebildeten Soldaten, die leicht bewegliche Ziele treffen können, wenn ihre eigenen Hände ruhig sind, und die ihre Verteidigungskräfte kombinieren können, solange sie zusammenbleiben? Im Gegensatz zu deinen Soldaten?“

Das Grinsen war aus Dracos Gesicht verschwunden, das nun hart und grimmig war.

„Weißt du“, sagte Harry leichthin, wohl wissend, dass keiner der anderen die wirkliche Botschaft, die zwischen ihnen ausgetauscht wurde, verstehen würde, „das zeigt nur, dass man immer alles hinterfragen sollte, was man seine Vorbilder tun sieht, und sich fragen, warum es getan wird und ob es in dem Zusammenhang Sinn macht, dass man es auch tut. Vergess übrigens nicht, diesen Ratschlag auf das wirkliche Leben anzuwenden. Und danke für die sich langsam bewegenden gebündelten Ziele.“

Denn Draco hatte diese Belehrung bereits erhalten und, so vermutete Harry, sie aus dem Verdacht heraus abgelehnt, dass Harry versuchte, seine Loyalitäten weiter von der Reinbluttradition weg zu verlagern.

\emph{Was Harry natürlich tat}.

Aber dieses Beispiel würde eine ausgezeichnete Ausrede abgeben, um am nächsten Samstag zu behaupten, dass das Infragestellen von Autoritäten eine rein praktische Technik für das wirkliche Leben sei. Und Harry würde auch die Experimente erwähnen, die er zuerst mit Einzelpersonen und dann mit Gruppen durchgeführt hatte, um zu überprüfen, ob seine Ideen über die Wichtigkeit von Schnelligkeit tatsächlich richtig waren, um den Punkt zu untermauern, dass Draco immer nach Gelegenheiten Ausschau halten musste, die Methoden in der täglichen Praxis anzuwenden.

„Sie haben noch nicht gewonnen, General Potter!“, knurrte Draco. „Vielleicht läuft uns die Zeit davon und Professor Quirrell wird es als unentschieden werten.“

Ein berechtigter und besorgniserregender Punkt. Der Krieg endete erst, wenn Professor Quirrell in seinem persönlichen Urteil entschied, dass eine Armee nach praktischen Maßstäben der realen Welt gewonnen hatte. Es gab keine formale Siegesbedingung, hatte Professor Quirrell erklärt, \emph{denn dann würde Harry herausfinden, wie man die Regeln ausnutzen konnte.}

Harry musste zugeben, dass dies eine fairer Annahme war. Und Harry konnte es Professor Quirrell nicht verübeln, dass er kein Ende ausrief, denn es war plausibel, dass der letzte Soldat der Drachenarmee alle fünf Überlebenden der Chaoslegion ausschalten konnte.

„In Ordnung“, sagte Harry. „Weiß jemand etwas über den Schildzauber von General Malfoy?“

Es stellte sich heraus, dass Dracos Schild eine Version des Standard-Protegos war, die mehrere Nachteile hatte, von denen der wichtigste war, dass der Schild sich nicht mit dem Zaubernden bewegen konnte. Der Vorteil - oder aus Harrys Sicht der Nachteil - war, dass er leichter zu erlernen, leichter zu wirken und viel leichter über längere Zeit aufrecht zu erhalten war. Sie würden den Schild mit Angriffszaubern beschießen müssen, um ihn zu Fall zu bringen. Und Draco konnte anscheinend eine gewisse Kontrolle über den Reflexionswinkel ausüben, an dem die Zauber abprallen würden.

Harry kam der Gedanke, dass sie Wingardium Leviosa benutzen könnten, um schwere Felsen auf dem Schild aufzutürmen, bis Draco dem Druck nicht mehr standhalten konnte…aber dann könnten die Felsen nachher herunterfallen und Draco treffen, und den feindlichen General wirklich zu verletzen, gehörte nicht zu den heutigen Zielen.

„Also“, sagte Harry. „Gibt es so etwas wie spezielle Schilddurchdringungszauber?“

Die gab es. Harry fragte, ob einer seiner Soldaten sie kannte.

\emph{Keiner kannte einen.}

Draco grinste wieder, in seinem Schild. Harry fragte, ob es irgendeinen Angriffszauber gäbe, der nicht abprallen würde. Blitze, so schien es, wurden normalerweise von Schilden absorbiert, anstatt an ihnen abzuprallen.

\emph{… niemand wusste, wie man irgendeinen Blitzzauber spricht}.

Draco kicherte wieder.

Harry seufzte. Er legte seinen Zauberstab ganz bewusst auf den Boden. Und Harry verkündete mit einer gewissen Ermüdung in der Stimme, dass er einfach weitermachen und den Schild selbst abbauen würde, mit einer Methode, die geheimnisvoll bleiben würde; und alle anderen sollten auf Draco schießen, sobald sein Schild unten war.

Die Chaos-Legionäre sahen nervös aus.

Draco sah ruhig aus, das heißt, kontrolliert.

Eine dünne, gefaltete Decke kam aus Harrys Tasche. Harry setzte sich neben den schimmernden Schild und zog die Decke über seinen Kopf, damit niemand sehen konnte, was er tat - außer Draco natürlich. Aus Harrys Tasche kamen eine Autobatterie und ein Überbrückungskabel.

\emph{… es war ja nicht so, dass er die Muggelwelt verlassen wollte, um eine neue Ära der magischen Forschung zu beginnen, und keine Möglichkeit zur Stromerzeugung mitgenommen hätte.}

Kurz darauf hörten die Chaos Legionäre das Geräusch von schnippenden Fingern, gefolgt von einem knisternden Geräusch unter der Decke. Der Schild begann heller zu leuchten, und Harrys Stimme sagte:

„Lasst euch bitte nicht ablenken, achtet auf General Malfoy.“

Die Anspannung war auf Dracos Gesicht zu sehen, zusammen mit der Wut, dem Ärger und der Frustration. Harry lächelte zu ihm auf und sagte:

„Ich erzähl's dir später.“

Und das war der Moment, in dem eine Spirale aus grüner Energie aus dem Wald schoss und in Dracos Schild einschlug, das kreischte, als würden scharfe Glasscherben aneinander gerieben, und Draco taumelte. In plötzlicher, verzweifelter Panik nahm Harry die Überbrückungskabel von der Batterie und steckte sie in den Beutel, dann steckte er die Batterie selbst in den Beutel, und dann riss er die Decke ab, griff nach seinem Zauberstab und stand auf.

Alle seine Soldaten waren noch da und blickten sich hektisch um.

„Contego“, sagte Harry, und seine Soldaten folgten ihm, aber Harry wusste nicht einmal, in welche Richtung der Schild zeigen sollte.

„Hat jemand gesehen, woher das kam?“

Kopfschütteln.

„Und General Malfoy, würden Sie mir bitte sagen, ob Sie General Granger erwischt haben?“

„Aber ja“, sagte Draco säuerlich, „es macht mir was aus.“

\emph{Oh, verdammt.}

Harrys Verstand begann zu rechnen, \emph{Draco innerhalb des Schildes, Draco nun} \emph{einigermaßen erschöpft, Harry ebenfalls erschöpft, Hermine im Wald wer-weiß-wo, Harry und vier andere Chaoten übrig…}

„Wissen Sie, General Granger“, sagte Harry laut, „Sie hätten mit dem Angriff wirklich warten sollen, bis ich gegen General Malfoy gekämpft habe. Dann hätten Sie vielleicht alle Überlebenden erwischen können.“

Von irgendwoher kam das schrille Lachen eines Mädchens. Harry erstarrte.

\emph{Das war nicht Hermine.}

Und in diesem Moment ertönte ein unheimlicher, fröhlicher Gesang, der aus allen Ecken kam.

„\emph{Habt keine Angst und seid nicht traurig, wir tun euch nur weh, wenn ihr böse seid…}“

\emph{(Don't be frightened, don't be sad,We'll only hurt you if you're bad…, anmk. Des Übersetzers)}

„Granger hat geschummelt!“, platzte Draco im Inneren des Schildes heraus. „Sie hat ihre Soldaten aufgeweckt! Warum hat Professor Quirrell nicht—“

„Lass mich raten“, sagte Harry, während sich die Übelkeit bereits in seinem Magen zusammenbraute.

\emph{Er hasste es wirklich zu verlieren.}

„Es war eine sehr leichte Schlacht, richtig? Sie sind umgefallen wie die Fliegen?“

„Ja“, sagte Draco. „Wir haben sie alle mit dem ersten Schuss erwischt—“

Der Blick der entsetzten Erkenntnis breitete sich von Draco zu den Chaos-Legionären aus.

„Nein“, sagte Harry, „haben wir nicht.“

Getarnte Gestalten tauchten zwischen den Bäumen auf.

„Verbündete?“ sagte Harry.

„Verbündete“, sagte Draco.

„Gut“, sagte General Grangers Stimme, und eine Spirale aus grüner Energie loderte aus dem Wald und zerschmetterte Dracos Schild zu Splittern.

General Granger überblickte das Schlachtfeld mit einem eindeutigen Gefühl der Zufriedenheit. Sie hatte nur noch neun Sonnenschein Soldaten, aber das war wahrscheinlich genug, um mit dem letzten Überlebenden der feindlichen Streitkräfte fertig zu werden, vor allem, wenn Parvati und Anthony und Ernie bereits ihre Zauberstäbe auf General Potter richteten, den sie lebendig (naja, bei Bewusstsein) hatte abführen lassen. Es war schlimm, das wusste sie, aber sie hatte sich wirklich freuen wollen.

„Es gibt einen Trick, nicht wahr?“, sagte Harry, die Anspannung war in seiner Stimme zu hören. „Es muss einen Trick geben.Man kann sich nicht einfach in einen perfekten General verwandeln. Nicht zusätzlich zu allem anderen. Du bist nicht so gerissen! Du schreibst keine gruseligen Gedichte! Niemand ist in allem so gut!?“

General Granger blickte sich bei ihren Sonnenschein Soldaten um und sah dann wieder zu Harry. Wahrscheinlich sahen das alle auf den Bildschirmen draußen. Und General Granger sagte:

„Ich kann alles, wenn ich nur hart genug lerne.“

„Oh, das ist einfach nur…“

„Somnium.“

Harry sackte mitten im Satz auf den Boden.

\textbf{„Sonnenschein gewinnt“}

, intonierte die riesige Stimme von Professor Quirrell, die von überall und nirgends zu kommen schien.

„\textbf{Die Nettigkeit hat gesiegt!}“, rief General Granger.

„\textbf{Hurra}!“, riefen die Sonnenschein-Soldaten.

Sogar die Gryffindor-Jungs sagten es, und sie sagten es mit Stolz.

„Und was ist die Moral von der heutigen Schlacht?“, sagte General Granger.

\textbf{„Wir können alles schaffen, wenn wir nur fleißig genug lernen!“}

Und die Überlebenden des Sunshine Regiments marschierten los in Richtung Siegesfeld und sangen dabei ihr Marschlied:

Habt keine Angst und seid nicht traurig,

wir tun euch nur weh, wenn ihr böse seid,

wir schicken euch in ein richtiges Zuhause,

mit neuen Freunden, die auf euch aufpassen,

und sagt ihnen, dass ihr von Grangers Sonnenschein-Regiment geschickt wurdet!

\emph{(Don't be frightened, don't be sad,

We'll only hurt you if you're bad,

And send you to a home that's true,

With new friends to watch over you,

Be sure to tell them you were sent

By Granger's Sunshine Regiment!}

\emph{Anm. d. Ü.)}

