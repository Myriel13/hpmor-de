

\hypertarget{verwirrung-bemerken}{% \section{26. Verwirrung bemerken}\label{verwirrung-bemerken}}

\textbf{\uline{Verwirrung bemerken}}

Professor Quirrells Sprechstunde fand donnerstags von 11:40 bis 11:55~Uhr statt. Das galt für \emph{alle} seine Schüler in \emph{allen} Jahrgängen. Es kostete einen Quirrell-Punkt, nur um an die Tür zu klopfen, und wenn er der Meinung war, dass dein Grund seine Zeit nicht wert war, würdest du weitere fünfzig verlieren.

Harry klopfte an die Tür. Es gab eine Pause. Dann sagte eine bissige Stimme:

„Ich nehme an, Sie können genauso gut hereinkommen, Mr~Potter.“

Und bevor Harry den Türknauf berühren konnte, knallte die Tür auf und schlug mit einem scharfen Knall gegen die Wand, der sich anhörte, als wäre etwas im Holz oder im Stein oder in beidem zerbrochen.

Professor Quirrell lehnte sich in seinem Stuhl zurück und las in einem verdächtig alt aussehenden Buch, gebunden in nachtblauem Leder mit silbernen Runen auf dem Rücken.

Seine Augen hatten sich nicht von den Seiten entfernt.

„Ich bin nicht in guter Stimmung, Mr~Potter. Und wenn ich nicht gut gelaunt bin, ist es nicht angenehm, in meiner Nähe zu sein. Erledigen Sie in Ihrem eigenen Interesse schnell Ihr Geschäft und gehen Sie.“

Eine kalte Kälte sickerte aus dem Raum, als ob er etwas enthielte, das Dunkelheit ausstrahlte, so wie Lampen Licht ausstrahlen, und das nicht vollständig abgedunkelt war. Harry war ein wenig verblüfft. Nicht gut gelaunt zu sein, schien nicht ganz zu genügen. Was konnte Professor Quirrell nur so sehr beunruhigen…?

Nun, man ließ seine Freunde nicht einfach im Stich, wenn sie schlecht gelaunt waren. Vorsichtig trat Harry in den Raum.

„Kann ich irgendetwas tun, um zu helfen—“

„Nein“, sagte Professor Quirrell und blickte immer noch nicht von seinem Buch auf.

„Ich meine, wenn Sie sich mit Idioten herumgeschlagen haben und jemanden brauchen, der vernünftig ist und mit dem Sie reden können…“

Es gab eine überraschend lange Pause. Professor Quirrell schlug das Buch zu und es verschwand mit einem leisen, flüsternden Geräusch. Dann sah er auf, und Harry zuckte zusammen.

„Ich nehme an, ein intelligentes Gespräch wäre mir an dieser Stelle angenehm“, sagte Professor Quirrell in demselben bissigen Ton, der Harry eingeladen hatte, einzutreten.

„Sie werden es wahrscheinlich nicht so finden, seien Sie gewarnt.“

Harry holte tief Luft.

„Ich verspreche, dass es mir nichts ausmacht, wenn Sie mich anschnauzen. Was ist passiert?“

Die Kälte im Raum schien sich zu vertiefen.

„Ein Gryffindor aus dem sechsten Jahr hat einen Fluch auf einen meiner vielversprechenderen Schüler, einen Slytherin aus dem sechsten Jahr, gezaubert.“

Harry schluckte. „Was…für ein Fluch?“

Die Wut auf Professor Quirrells Gesicht war nicht mehr zu halten.

„Warum machen Sie sich die Mühe, so eine unwichtige Frage zu stellen, Mr~Potter? Unser Freund, der Gryffindor aus dem sechsten Schuljahr, hat es nicht für wichtig gehalten!“

„Ist das Ihr Ernst?!“ sagte Harry, bevor er sich zurückhalten konnte.

"Nein, ich bin heute ohne besonderen Grund in einer schrecklichen Stimmung. Ja, ich meine es ernst, du Narr! Er wusste es nicht. Er wusste es tatsächlich nicht. Ich habe es nicht geglaubt, bis die Auroren es unter Veritaserum bestätigt haben.

\textbf{Er ist im sechsten Jahr in Hogwarts und hat einen hochgradigen dunklen Fluch ausgesprochen, ohne zu wissen, was er bewirkt!}"

„Sie meinen doch nicht etwa“, sagte Harry, „dass er sich darüber geirrt hat, was er bewirkt, dass er irgendwie die falsche Zauberbeschreibung gelesen hat—“

„\textbf{Er wusste nur, dass er auf einen Feind gerichtet werden sollte}. \textbf{Er wusste, dass das alles war, was er wusste.}“

Und das hatte gereicht, um den Zauber zu sprechen.

„Ich verstehe nicht, wie etwas mit einem so kleinen Gehirn aufrecht gehen kann.“

„In der Tat, Mr~Potter“, sagte Professor Quirrell. Es gab eine Pause. Professor Quirrell beugte sich vor und nahm das silberne Tintenfass von seinem Schreibtisch, drehte es in seinen Händen und starrte es an, als ob er sich fragen würde, wie er ein Tintenfass zu Tode quälen könnte.

„Wurde der Slytherin im sechsten Jahr ernsthaft verletzt?“, fragte Harry.

„Ja.“

„Wurde der Gryffindor im sechsten Jahr von Muggeln aufgezogen?“

„Ja.“

„Weigert sich Dumbledore, ihn rauszuwerfen, weil der arme Junge es nicht besser wusste?“

Professor Quirrells Hände wurden auf dem Tintenfass bleich.

„Haben Sie ein Argument, Mr~Potter, oder stellen Sie nur das Offensichtliche fest?“

„Professor Quirrell“, sagte Harry ernst, "alle Muggelschüler in Hogwarts brauchen eine \emph{Sicherheitsvorlesung}, in der ihnen die Dinge gesagt werden, die so lächerlich offensichtlich sind, dass kein Zauberergeborener jemals auf die Idee kommen würde, sie zu erwähnen.

Sprich keine Flüche, wenn du nicht weißt, was sie bewirken, wenn du etwas Gefährliches entdeckst, erzähl der Welt nichts davon, braue keine hochgradigen Zaubertränke ohne Aufsicht in einem Badezimmer, der Grund, warum es Gesetze für die Magie von Minderjährigen gibt, all die Grundlagen."

„Warum?“, fragte Professor Quirrell. „Lass die Dummen sterben, bevor sie sich fortpflanzen.“

„Wenn es Ihnen nichts ausmacht, ein paar Slytherins aus dem sechsten Jahr mit ihnen zu verlieren.“

Das Tintenfass fing in Professor Quirrells Händen Feuer und verbrannte mit einer schrecklichen Langsamkeit, grässliche schwarz-orange Flammen rissen an dem Metall und schienen winzige Bissen davon zu nehmen, das Silber verdrehte sich, während es schmolz, als ob es versuchte zu entkommen und es nicht schaffte.

Es war ein blechernes Kreischen zu hören, als würde das Metall schreien.

„Ich nehme an, Sie haben Recht“, sagte Professor Quirrell mit einem resignierten Lächeln.

„Ich werde eine Vorlesung entwerfen, die sicherstellt, dass Muggelgeborene, die zu dumm zum Leben sind, niemanden Wertvolles mitnehmen, wenn sie abreisen.“

Das Tintenfass brannte weiter in Professor Quirrells Händen, winzige Metalltropfen, die immer noch brannten, tropften nun auf den Schreibtisch, als würde das Tintenfass weinen.

„Du läufst nicht weg“, bemerkte Professor Quirrell.

Harry öffnete den Mund.

„Wenn du jetzt sagen willst, dass du keine Angst vor mir hast“, sagte Professor Quirrell, „dann lass es.“

„Sie sind der furchterregendste Mensch, den ich kenne“, sagte Harry, „und einer der Hauptgründe dafür ist Ihre Beherrschung. Ich kann mir einfach nicht vorstellen, dass Sie jemanden verletzen würden, den Sie nicht absichtlich verletzt wollen.“

Das Feuer in Professor Quirrells Händen erlosch, und er stellte das zerstörte Tintenfass vorsichtig auf seinen Schreibtisch.

„Sie sagen die nettesten Dinge, Mr~Potter. Haben Sie Unterricht in Schmeichelei genommen? Vielleicht von Mr~Malfoy?“

Harry hielt seine Miene ausdruckslos und merkte eine Sekunde zu spät, dass es genauso gut ein unterschriebenes Geständnis hätte sein können.

Professor Quirrell war es egal, wie der Gesichtsausdruck aussah, ihn interessierte, welche Geisteszustände ihn wahrscheinlich machten.

„Ich verstehe“, sagte Professor Quirrell. „Mr~Malfoy ist ein nützlicher Freund, Mr~Potter, und es gibt viel, was er Ihnen beibringen kann, aber ich hoffe, Sie haben nicht den Fehler gemacht, ihm zu viele Vertraulichkeiten anzuvertrauen.“

„Er weiß nichts, von dem ich befürchte, dass es bekannt wird“, sagte Harry.

„Gut gemacht“, sagte Professor Quirrell und lächelte leicht. „Also, was war Ihr ursprüngliches Anliegen hier?“

„Ich denke, ich bin mit den Vorübungen in Okklumentik fertig und bereit für den Tutor.“

Professor Quirrell nickte.

„Ich werde Sie am Sonntag zu Gringotts begleiten.“

Er hielt inne, sah Harry an und lächelte.

„Und wir könnten sogar einen kleinen Ausflug daraus machen, wenn Sie möchten. Mir ist gerade ein angenehmer Gedanke gekommen.“

Harry nickte und lächelte zurück. Als Harry das Büro verließ, hörte er Professor Quirrell eine \emph{kleine Melodie} summen.

Harry war froh, dass er ihn hatte aufheitern können.

An diesem Sonntag schienen ziemlich viele Leute in den Gängen zu tuscheln, zumindest, wenn Harry Potter an ihnen vorbeiging.

Und eine Menge spitzer Finger.

Und sehr viel weibliches Kichern.

Es hatte beim Frühstück angefangen, als jemand Harry gefragt hatte, ob er die Nachrichten gehört hätte, und Harry hatte schnell unterbrochen und gesagt, wenn die Nachrichten von Rita Kimmkorn stammten, dann wolle er sie nicht hören, er wolle sie selbst in der Zeitung lesen.

Es hatte sich dann herausgestellt, dass nicht viele Schüler in Hogwarts Exemplare des Tagespropheten bekamen und dass die Exemplare, die nicht schon von ihren Besitzern aufgekauft worden waren, in einer Art komplizierter Reihenfolge herumgereicht wurden und niemand wirklich wusste, wer im Moment eines hatte…

Also hatte Harry einen Beruhigungszauber benutzt und war zum Frühstück gegangen, im Vertrauen darauf, dass seine Sitznachbarn die vielen, vielen Fragesteller abwinken würden, und er tat sein Bestes, um die Ungläubigkeit, das Gelächter, die gratulierenden Lächeln, die mitleidigen Blicke, die ängstlichen Blicke und die heruntergefallenen Teller zu ignorieren, wenn neue Leute zum Frühstück kamen und zuhörten.

Harry war ziemlich neugierig, aber es hätte wirklich geschadet, die Kunstfertigkeit zu verderben, indem er sie aus zweiter Hand hörte.

Er machte seine Hausaufgaben für die nächsten paar Stunden in der Sicherheit seines Koffers, nachdem er seinen Mitschülern gesagt hatte, sie sollten ihn abholen, falls jemand eine Originalzeitung finden würde.

Harry war immer noch unwissend, als er um 10~Uhr morgens Hogwarts in einer Kutsche mit Professor Quirrell verlassen hatte, der vorne rechts saß und gerade im Zombie-Modus zusammengesackt war.

Harry saß schräg gegenüber, so weit weg, wie es der Wagen zuließ, hinten links. Trotzdem hatte Harry ein ständiges \emph{Gefühl des Untergangs}, als die Kutsche über einen kleinen Pfad durch einen Abschnitt des nicht verbotenen Waldes ratterte.

Das machte das Lesen etwas schwierig, zumal der Stoff schwierig war, und Harry wünschte sich plötzlich, er würde stattdessen eines der Science-Fiction-Bücher aus seiner Kindheit lesen—

„Wir sind außerhalb der Schutzräume, Mr~Potter“, sagte Professor Quirrells Stimme von vorne.

„Zeit zu gehen.“

Professor Quirrell stieg vorsichtig aus der Kutsche aus und stützte sich beim Absteigen ab.

Harry sprang seinerseits ab. Harry fragte sich gerade, wie sie dorthin kommen würden, als Professor Quirrell „Fang!“ sagte und einen bronzenen Knut nach ihm warf, den Harry ohne nachzudenken auffing.

Ein riesiger, nicht greifbarer Haken griff nach Harrys Unterleib und riss ihn zurück, hart, nur ohne jegliches Gefühl der Beschleunigung, und einen Augenblick später stand Harry mitten in der Winkelgasse.

(\emph{Wie bitte, was?} sagte sein Gehirn.)

(\emph{Wir haben uns gerade teleportiert,} erklärte Harry seinem Gehirn.)

(\emph{Das ist in der angestammten Umgebung früher nicht passiert,} beschwerte sich Harrys Gehirn und verwirrte ihn.)

Harry taumelte, als sich seine Füße an den Ziegelstein der Straße statt an den Schmutz des Waldkorridors, den sie durchquert hatten, gewöhnten.

Er richtete sich auf, immer noch benommen, während die geschäftigen Hexen und Zauberer leicht zu schwanken schienen und die Schreie der Ladenbesitzer sich in seinem Gehör zu bewegen schienen, während sein Gehirn versuchte, eine Welt zu finden, in der er sich befand.

Einige Augenblicke später ertönte eine Art saugendes Geräusch ein paar Schritte hinter Harry, und als Harry sich umdrehte, war Professor Quirrell da.

„Macht es Ihnen etwas aus—“

sagte Harry, zur gleichen Zeit wie Professor Quirrell,

„ich fürchte, ich muss—“

Harry blieb stehen, Professor Quirrell nicht.

„- etwas in die Wege leiten, Mr~Potter. Da mir gründlich erklärt wurde, dass ich für alles, was Ihnen zustößt, verantwortlich bin, werde ich Sie mit—“

„Kiosk“, sagte Harry.

„Verzeihung?“

„Oder irgendwo, wo ich eine Ausgabe des Tagespropheten kaufen kann. Bringen Sie mich dorthin und ich werde glücklich sein.“

Kurz darauf war Harry in einen Buchladen gebracht worden, begleitet von mehreren leise gesprochenen, zweideutigen Drohungen.

Und der Ladenbesitzer hatte \emph{weniger zweideutige} Drohungen erhalten, der Art nach zu urteilen, wie er zusammengezuckt war und wie seine Augen nun ständig zwischen Harry und dem Eingang hin und her huschten.

Wenn der Buchladen abbrannte, würde Harry in der Mitte des Feuers stehen bleiben, bis Professor Quirrell zurückkam.

Währenddessen - Harry warf einen kurzen Blick in die Runde. Der Buchladen schien ziemlich klein und schäbig zu sein, nur vier Reihen von Bücherregalen waren zu sehen, und das nächstgelegene Regal, zu dem Harrys Augen gesprungen waren, schien sich mit schmalen, billig gebundenen Büchern mit düsteren Titeln wie \emph{Das Massaker von Albanien im fünfzehnten Jahrhundert} zu befassen. Das Wichtigste zuerst. Harry trat zum Tresen des Verkäufers hinüber.

„Verzeihung“, sagte Harry, „ein Exemplar des Tagespropheten, bitte.“

„Fünf Sickel“, sagte der Ladenbesitzer.

„Tut mir leid, Junge, ich habe nur noch drei.“

Fünf Sickles fielen auf den Tresen. Harry hatte das Gefühl, dass er ihn um ein paar Knuts hätte herunterhandeln können, aber das war ihm im Moment egal.

Die Augen des Ladenbesitzers weiteten sich und er schien Harry zum ersten Mal wirklich zu bemerken.

„Sie!?“

„Ich!?“

„Ist es wahr? Bist du wirklich—“

„Halt die Klappe! Tut mir leid, ich habe den ganzen Tag darauf gewartet, das in der Originalzeitung zu lesen, anstatt es aus zweiter Hand zu erfahren, also geben Sie es mir bitte einfach, in Ordnung?“

Der Ladenbesitzer starrte Harry einen Moment lang an, dann griff er wortlos unter den Tresen und reichte ihm ein gefaltetes Exemplar des Tagespropheten. Die Schlagzeile lautete:

\textbf{HARRY POTTER HEIMLICH MIT GINEVRA WEASLEY VERHEIRATE}T

Harry starrte sie an. Er hob die Zeitung von der Theke, sanft, ehrfürchtig, als würde er ein Original-Escher-Kunstwerk anfassen, und knickte sie auf, um zu lesen…

… über den Beweis, der Rita Kimmkorn überzeugt hatte.

… und einige weitere interessante Details.

… und noch mehr Beweise.

\emph{Fred und George hatten das doch sicher vorher mit ihrer Schwester geklärt, oder? Ja, natürlich hatten sie das.}

Es gab ein Foto von Ginevra Weasley, die sehnsüchtig über einem Foto von Harry seufzte.

\emph{Das muss inszeniert gewesen sein. Aber wie um alles in der Welt…?}

Harry saß gerade in einem billigen Klappstuhl und las zum vierten Mal die Zeitung, als die Tür leise flüsterte und Professor Quirrell wieder in den Laden kam.

„Entschuldigen Sie bitte - was in Merlins Namen lesen Sie da?“

„Es scheint“, sagte Harry mit Ehrfurcht in der Stimme, „dass ein Mr~Arthur Weasley von einem Todesser, den mein Vater getötet hat, mit dem Imperius-Fluch belegt wurde, wodurch eine Schuld gegenüber dem Haus Potter entstand, von der mein Vater verlangte, dass sie durch die Heirat mit der kürzlich geborenen Ginevra Weasley zurückgezahlt wird. Macht man so etwas hier eigentlich?“

„Wie konnte Miss~Kimmkorn nur so dumm sein, zu glauben—“

Und Professor Quirrells Stimme brach ab.

Harry hatte die Zeitung senkrecht gehalten und aufgeklappt gelesen, was bedeutete, dass Professor Quirrell von dort aus, wo er stand, den Text unter der Überschrift sehen konnte.

Der Ausdruck des Schocks auf Professor Quirrells Gesicht war ein Kunstwerk, das der Zeitung selbst fast ebenbürtig war.

„Keine Sorge“, sagte Harry fröhlich, „es ist alles gefälscht.“

Von irgendwoher im Laden hörte er den Ladenbesitzer keuchen. Man hörte, wie ein Stapel Bücher umkippte.

„Mr~Potter…“ sagte Professor Quirrell langsam, „sind Sie sich da sicher?“

„Ziemlich sicher. Sollen wir gehen?“

Professor Quirrell nickte und sah ziemlich abwesend aus, und Harry faltete die Zeitung wieder zusammen und folgte ihm aus der Tür. \emph{Aus irgendeinem Grund schien Harry jetzt keine Straßengeräusche mehr zu hören}. Sie gingen dreißig Sekunden lang schweigend, bevor Professor Quirrell das Wort ergriff.

„Miss~Kimmkorn hat sich das Originalprotokoll der nichtöffentlichen Zaubergamot-Sitzung angesehen.“

„Ja.“

„Das Originalprotokoll des Zaubergamots.“

„Ja.“

„Das würde mir schwerfallen.“

„Wirklich?“, sagte Harry. „Denn wenn mein Verdacht richtig ist, wurde das von einem Hogwartsschüler gemacht.“

„Das ist mehr als unmöglich“, sagte Professor Quirrell barsch.

„Mr~Potter… ich bedaure, sagen zu müssen, dass diese junge Dame erwartet, Sie zu heiraten.“

„Aber das ist unwahrscheinlich“, sagte Harry.

„Um Douglas Adams zu zitieren: Das Unmögliche hat oft eine Art von Integrität, die dem bloß Unwahrscheinlichen fehlt.“

„Ich verstehe, was Sie meinen“, sagte Professor Quirrell langsam.

"Aber…nein, Mr~Potter. Es mag unmöglich sein, aber ich kann mir keine Manipulation des Zaubergamot-Verfahrens vorstellen.

Es ist unvorstellbar, dass der Große Direktor von Gringotts das Siegel seines Amtes als Zeuge für einen falschen Verlobungsvertrag anbringt, und Miss~Kimmkorn hat dieses Siegel persönlich überprüft."

„In der Tat“, sagte Harry, „man würde erwarten, dass der Direktor von Gringotts sich damit beschäftigt, dass so viel Geld den Besitzer wechselt. Es scheint, dass Mr~Weasley hoch verschuldet war und deshalb eine Nachzahlung von zehntausend Galleonen verlangte—“

„10.000 Galleonen für einen Weasley? Dafür könnte man die Tochter eines Adelshauses kaufen!“

„Entschuldigen Sie“, sagte Harry. „Ich muss an dieser Stelle wirklich fragen, ob die Leute hier so etwas tatsächlich tun—“

„Selten“, sagte Professor Quirrell mit einem Stirnrunzeln im Gesicht.

„Und überhaupt nicht mehr, vermute ich, seit der Dunkle Lord weg ist. Ich nehme an, dass Ihr Vater laut der Zeitung einfach bezahlt hat?“

„Er hatte keine andere Wahl“, sagte Harry. „Nicht, wenn er die Bedingungen der Prophezeiung erfüllen wollte.“

„Gib das her“, sagte Professor Quirrell, und die Zeitung sprang Harry so schnell aus der Hand, dass er einen Papierschnitt bekam.

Harry steckte sich automatisch den Finger in den Mund, um daran zu lutschen, und drehte sich um, um mit Professor Quirrell zu schimpfen - Professor Quirrell war mitten auf der Straße stehen geblieben, und seine Augen flackerten schnell hin und her, während eine unsichtbare Kraft die Zeitung vor ihm in der Schwebe hielt.

Harry sah mit offenem Mund zu, wie sich die Zeitung öffnete und die Seiten zwei und drei zum Vorschein kamen. Und nicht viel später auch vier und fünf. Es war, als hätte der Mann den Anschein der Sterblichkeit abgelegt.

Und nach einer beunruhigend kurzen Zeit faltete sich die Zeitung wieder ordentlich zusammen. Professor Quirrell pflückte es aus der Luft und warf es Harry zu, der es aus reinem Reflex auffing; und dann begann Professor Quirrell wieder zu gehen, und Harry stapfte automatisch hinterher.

„Nein“, sagte Professor Quirrell, „diese Prophezeiung klang für mich auch nicht ganz richtig.“

Harry nickte, immer noch fassungslos.

„Die Zentauren könnten unter einen Imperius gestellt worden sein“, sagte Professor Quirrell und runzelte die Stirn, "das scheint verständlich. Was Magie machen kann, kann Magie korrumpieren, und es ist nicht undenkbar, dass das Große Siegel von Gringotts in fremde Hände gelangt sein könnte. Der Unaussprechliche könnte mit Vielsaft verkörpert worden sein, ebenso wie der bayrische Seher. Und mit genug Aufwand wäre es möglich, die Sitzungen des Zaubergamots zu manipulieren.

Haben Sie eine Idee, wie das gemacht wurde?"

„Ich habe nicht eine einzige plausible Hypothese“, sagte Harry. „Ich weiß nur, dass es mit einem Gesamtbudget von vierzig Galleonen gemacht wurde.“

Professor Quirrell blieb kurz stehen und wirbelte zu Harry herum. Sein Gesichtsausdruck war nun völlig ungläubig.

"Vierzig Galleonen bezahlen einen kompetenten Einbrecher, um einen Weg in ein Haus

zu öffnen, in das Sie einbrechen wollen! Vierzigtausend Galleonen könnten ein Team der größten Berufsverbrecher der Welt bezahlen, um die Arbeit des Zaubergamots zu stören!"

Harry zuckte hilflos mit den Schultern.

„Daran werde ich denken, wenn ich das nächste Mal neununddreißigtausendneunhundertsechzig Galleonen sparen will, indem ich den richtigen Auftragnehmer finde.“

„Das sage ich nicht oft“, sagte Professor Quirrell. „Ich bin beeindruckt.“

„Gleichfalls“, sagte Harry.

„Und wer ist dieser unglaubliche Hogwarts-Schüler?“

„Das kann ich leider nicht sagen.“

Etwas zu Harrys Überraschung erhob Professor Quirrell keinen Einwand dagegen.

Sie gingen in Richtung des Gringotts-Gebäudes und dachten nach, denn sie waren beide nicht die Art von Menschen, die ein Problem aufgeben würden, ohne mindestens fünf Minuten darüber nachzudenken.

„Ich habe das Gefühl“, sagte Harry schließlich, "dass wir die Sache von der falschen Seite angehen.

Es gibt eine Geschichte, die ich einmal von einigen Studenten gehört habe, die in eine Physikstunde kamen, und die Lehrerin zeigte ihnen eine große Metallplatte in der Nähe eines Feuers.

Sie befahl ihnen, die Metallplatte zu ertasten, und sie spürten, dass das Metall, das näher am Feuer war, kühler war, und das Metall, das weiter weg war, wärmer. Und sie sagte, schreibt eure Vermutung auf, warum das so ist.

Also schrieben einige Studenten auf, '\emph{weil das Metall Wärme leitet}', und einige Studenten schrieben auf, \emph{'weil sich die Luft bewegt}', und niemand sagte, \emph{'das scheint einfach unmöglich zu sein',} und die wirkliche Antwort war, dass die Lehrerin die Platte umgedreht hat, bevor die Studenten in den Raum kamen."

„Interessant“, sagte Professor Quirrell. „Das klingt wirklich ähnlich. Gibt es eine Moral?“

„Dass deine Stärke als Rationalist darin besteht, dass du dich von der Fiktion mehr verwirren lässt als von der Realität“, sagte Harry.

"Wenn du jedes Ergebnis gleich gut erklären kannst, hast du null Wissen. Die Studenten dachten, sie könnten Worte wie '\emph{wegen der Wärmeleitung'} verwenden, um alles zu erklären, sogar eine Metallplatte, die auf der Seite kühler ist, die dem Feuer näher ist.

Sie merkten also nicht, wie verwirrt sie waren, und das bedeutete, dass sie durch die Unwahrheit nicht mehr verwirrt sein konnten als durch die Wahrheit.

Wenn Sie mir sagen, dass die Zentauren unter dem Imperius-Fluch standen, habe ich immer noch das Gefühl, dass etwas nicht ganz stimmt. Ich merke, dass ich immer noch verwirrt bin, selbst nachdem ich Ihre Erklärung gehört habe."

„Hm“, sagte Professor Quirrell.

Sie gingen weiter.

„Ich nehme nicht an“, sagte Harry, „dass es tatsächlich möglich ist, Menschen in alternative Universen zu tauschen? Zum Beispiel, dass das nicht unsere eigene Rita Kimmkorn ist, oder dass man sie vorübergehend in ein anderes Universum geschickt hat?“

„Wenn das möglich wäre“, sagte Professor Quirrell mit ziemlich trockener Stimme, „wäre ich dann noch hier?“

.

Und gerade, als sie fast an der riesigen weißen Fassade des Gringotts-Gebäudes waren, sagte Professor Quirrell: „Ah. Natürlich. Jetzt sehe ich es. Lass mich raten, die Weasley-Zwillinge?“

„Was?“, sagte Harry, wobei seine Stimme eine weitere Oktave in der Tonhöhe anstieg. „Wie?“

„Ich fürchte, das kann ich nicht sagen.“

„… Das ist nicht fair.“

„Ich denke, es ist äußerst fair“, sagte Professor Quirrell, und sie traten durch die Bronzetüren ein.

Es war kurz vor Mittag, und Harry und Professor Quirrell saßen am Fuß und am Kopf eines breiten, langen, flachen Tisches, in einem prächtig ausgestatteten Privatraum mit gut gepolsterten Sofas und Stühlen entlang der Wände und weichen Vorhängen, die überall hingen.

Sie waren im Begriff, im Mary's Place zu Mittag zu essen, von dem Professor Quirrell gesagt hatte, es sei ihm als eines der besten Restaurants in der Winkelgasse bekannt, besonders für—

seine Stimme war bedeutungsvoll gesunken—

\emph{bestimmte Zwecke.}

Es war das schönste Restaurant, in dem Harry je gewesen war, und es machte Harry wirklich zu schaffen, dass Professor Quirrell ihn zu diesem Essen einlud.

Der erste Teil der Mission, einen Okklumentik-Lehrer zu finden, war ein Erfolg gewesen. Professor Quirrell hatte mit einem bösen Lächeln zu Griphook gesagt, er solle den Besten empfehlen, den er kenne, und sich nicht um die Kosten kümmern, da Dumbledore sie bezahle; und der Kobold hatte im Gegenzug gelächelt. Ein gewisses Lächeln war vielleicht auch auf Harrys Seite zu sehen.

Der zweite Teil des Plans war ein kompletter Fehlschlag gewesen.

Harry durfte kein Geld aus dem Tresor nehmen, ohne dass Schulleiter Dumbledore oder ein anderer Schulbeamter anwesend war, und Professor Quirrell hatte den Tresorschlüssel nicht erhalten.

Harrys Muggel-Eltern konnten es nicht genehmigen, weil sie Muggel waren, und Muggel hatten in etwa die gleiche rechtliche Stellung wie Kinder oder Kätzchen: Sie waren niedlich, und wenn man sie in der Öffentlichkeit quälte, konnte man verhaftet werden, aber sie waren keine Menschen.

Es gab einige widerwillige Vorkehrungen, die Eltern von Muggelgeborenen in einem begrenzten Sinne als Menschen anzuerkennen, aber Harrys Adoptiveltern fielen nicht in diese rechtliche Kategorie.

Es schien, dass Harry in den Augen der zaubernden Welt effektiv ein Waisenkind war. Als solches waren der Schulleiter von Hogwarts oder seine Beauftragten innerhalb des Schulsystems Harrys Vormund, bis er seinen Abschluss gemacht hatte. Harry konnte ohne Dumbledores Erlaubnis atmen, aber nur so lange, wie der Schulleiter es nicht ausdrücklich verbot.

Harry hatte dann gefragt, ob er Griphook einfach sagen könne, wie er seine Investitionen über die Stapel von Goldmünzen, die in seinem Tresor lagen, hinaus diversifizieren könne. Griphook hatte ihn ausdruckslos angestarrt und gefragt, was „\emph{diversifizieren}“ bedeute.

Banken, so schien es, machten keine Investitionen. Banken lagerten die Goldmünzen in sicheren Tresoren gegen eine jährliche Gebühr. In der Zaubererwelt gab es kein Konzept für Investitionen.

Oder Aktien. Oder Aktiengesellschaften. Unternehmen wurden von Familien aus ihren persönlichen Tresoren geführt.

Kredite wurden von reichen Leuten vergeben, nicht von Banken. Obwohl Gringotts den Vertrag gegen eine Gebühr beglaubigte und für eine viel höhere Gebühr eintrieb.

Gute reiche Leute ließen ihre Freunde Geld leihen und zahlten es wann Sie wollten zurück. Schlechte reiche Leute verlangten Zinsen. Es gab keinen Sekundärmarkt für Kredite. Böse reiche Leute verlangten von dir jährliche Zinssätze von mindestens 20\%.

Harry war aufgestanden, hatte sich abgewandt und seinen Kopf an die Wand gelehnt. Harry hatte gefragt, ob er die Erlaubnis des Schulleiters bräuchte, bevor er eine Bank gründen könnte.

Professor Quirrell hatte ihn an dieser Stelle unterbrochen und gesagt, dass es Zeit für das Mittagessen sei.

Er führte den wütenden Harry schnell aus den Bronzetüren von Gringotts, durch die Gasse und zu einem feinen Restaurant namens Mary's Place, wo ein Raum für sie reserviert worden war.

Der Besitzer hatte schockiert geschaut, als er Professor Quirrell in Begleitung von Harry Potter sah, hatte sie aber ohne Beanstandung in das Zimmer geführt.

Und Professor Quirrell hatte ganz bewusst angekündigt, dass er die Rechnung bezahlen würde, wobei er Harrys Gesichtsausdruck ziemlich zu genießen schien.

„Nein“, sagte Professor Quirrell zur Kellnerin, „wir werden keine Speisekarten benötigen. Ich nehme das Tagesgericht mit einer Flasche Chianti, und Mr~Potter nimmt als Vorspeise die Diracawl-Suppe, gefolgt von einem Teller Roopo-Bällchen und einem Sirup-Pudding zum Nachtisch.“

Die Kellnerin, die in Roben gekleidet war, die immer noch streng und förmlich aussahen, obwohl sie etwas kleiner war als sonst, verbeugte sich respektvoll und ging, wobei sie die Tür hinter sich schloss.

Professor Quirrell winkte mit einer Hand in Richtung der Tür, und ein Riegel schob sich zu.

"Beachten Sie den Riegel auf der Innenseite. Dieser Raum, Mr~Potter, ist als Marys Zimmer bekannt. Es ist gegen jede Art von Hellseherei gefeit, und ich meine wirklich alle; selbst Dumbledore könnte nichts von dem feststellen, was hier passiert. Marys Zimmer wird von 2 Arten von Menschen benutzt. Die erste Sorte ist in unerlaubte Liebschaften verwickelt.

Und die zweite Sorte führt ein interessantes Leben."

„Wirklich“, sagte Harry.

Professor Quirrell nickte.

Harrys Lippen waren in Erwartung gespalten.

„Es wäre also eine Verschwendung, nur hier zu sitzen und zu Mittag zu essen, ohne etwas Besonderes zu tun.“

Professor Quirrell grinste, dann zückte er seinen Zauberstab und schnippte ihn in Richtung der Tür.

„Natürlich“, sagte er, "treffen Leute, die ein interessantes Leben führen, gründlichere Vorsichtsmaßnahmen als die Malliers.

Ich habe uns soeben eingeschlossen. Nichts wird mehr in diesen Raum hinein- oder hinausgelangen - durch den Spalt unter der Tür zum Beispiel. Und…"

Dann sprach Professor Quirrell nicht weniger als vier verschiedene Zaubersprüche, von denen Harry keinen erkannte.

„Selbst das reicht nicht wirklich aus“, sagte Professor Quirrell.

"Wenn wir etwas wirklich Wichtiges vorhätten, müssten wir noch dreiundzwanzig weitere Überprüfungen durchführen.

Wenn, sagen wir, Rita Kimmkorn wusste oder ahnte, dass wir hierher kommen würden, ist es möglich, dass sie in diesem Raum ist und den echten Unsichtbarkeitsumhang trägt.

Oder sie könnte ein Animagus mit einer winzigen Form sein, vielleicht. Es gibt Tests, um solche seltenen Möglichkeiten auszuschließen, aber sie alle durchzuführen, wäre mühsam.

Trotzdem frage ich mich, ob ich sie nicht trotzdem durchführen sollte, um dir keine schlechten Angewohnheiten beizubringen?"

Professor Quirrell tippte mit einem Finger auf seine Wange und sah abwesend aus.

„Schon gut“, sagte Harry, „ich verstehe, und ich werde es mir merken.“

Obwohl er ein wenig enttäuscht war, dass sie nichts wirklich Wichtiges vorhatten.

„Sehr gut“, sagte Professor Quirrell. Er lehnte sich in seinem Stuhl zurück und lächelte breit.

"Das haben Sie heute sehr gut gemacht, Mr~Potter. Der Grundgedanke war sicher Ihrer, auch wenn Sie die Ausführung delegiert haben. Ich glaube nicht, dass wir nach dieser Sache noch viel von Rita Kimmkorn hören werden.

Lucius Malfoy wird über ihr Versagen nicht erfreut sein. Wenn sie klug ist, flieht sie aus dem Land, sobald sie merkt, dass sie reingelegt wurde."

Ein flaues Gefühl machte sich in Harrys Magen breit.

„Lucius steckte hinter Rita Kimmkorn…?“

„Oh, das haben Sie nicht bemerkt?“, sagte Professor Quirrell.

Harry hatte nicht darüber nachgedacht, was mit Rita Kimmkorn danach passieren würde. Überhaupt nicht. Nicht im Geringsten. Aber sie würde von ihrem Job gefeuert werden, natürlich würde sie gefeuert werden, sie könnte Kinder haben, die durch Hogwarts gehen, soweit Harry wusste, und jetzt war es schlimmer, viel schlimmer—

„Wird Lucius sie töten lassen?“ sagte Harry mit kaum hörbarer Stimme.

Irgendwo in seinem Kopf schrie ihn der Sprechende Hut an.

Professor Quirrell lächelte trocken.

„Wenn Sie noch nie mit Journalisten zu tun hatten, lassen Sie sich gesagt sein, dass die Welt jedes Mal ein bisschen heller wird, wenn einer stirbt.“

Harry sprang mit einer krampfhaften Bewegung aus seinem Stuhl, er musste Rita Kimmkorn finden und sie warnen, bevor es zu spät war—

„Setzen Sie sich“, sagte Professor Quirrell scharf. "Nein, Lucius wird sie nicht umbringen. Aber Lucius macht denen, die ihm schlecht dienen, das Leben äußerst unangenehm.

Miss~Kimmkorn wird fliehen und ihr Leben unter einem neuen Namen neu beginnen. Setzen Sie sich, Mr~Potter; es gibt nichts, was Sie zu diesem Zeitpunkt tun können, und Sie haben eine Lektion zu lernen."

Harry setzte sich hin, langsam.

Auf Professor Quirrells Gesicht lag ein enttäuschter, verärgerter Blick, der ihn mehr aufhielt als die Worte.

„Es gibt Zeiten“, sagte Professor Quirrell mit schneidender Stimme, "in denen ich mir Sorgen mache, dass dein brillanter Slytherin-Verstand einfach an dich verschwendet ist.

Sprech mir nach. Rita Kimmkorn war eine abscheuliche, ekelhafte Frau."

„Rita Kimmkorn war eine abscheuliche, ekelhafte Frau“, sagte Harry.

Es war ihm nicht wohl dabei, es zu sagen, aber es schien keine anderen Handlungsmöglichkeiten zu geben, gar keine.

„Rita Kimmkorn versuchte, meinen Ruf zu zerstören, aber ich führte einen genialen Plan aus und zerstörte ihren Ruf zuerst.“

„Rita Kimmkorn hat mich herausgefordert. Sie verlor das Spiel, und ich gewann.“

„Rita Kimmkorn war ein Hindernis für meine Zukunftspläne. Ich hatte keine andere Wahl, als mich mit ihr auseinanderzusetzen, wenn ich wollte, dass diese Pläne gelingen.“

„Rita Kimmkorn war mein Feind.“

„Ich kann im Leben nichts erreichen, wenn ich nicht bereit bin, meine Feinde zu besiegen.“

„Ich habe heute einen meiner Feinde besiegt.“

„\emph{Ich bin ein guter Junge}.“

„\emph{Ich verdiene eine besondere Belohnung}.“

„Ah“, sagte Professor Quirrell, der während der letzten Zeilen ein wohlwollendes Lächeln aufgesetzt hatte, „ich sehe, es ist mir gelungen, Ihre Aufmerksamkeit zu erregen.“

Das stimmte.

Und obwohl Harry das Gefühl hatte, in etwas hineingezogen zu werden - nein, das war nicht nur ein Gefühl, er war hineingezogen worden -, konnte er nicht leugnen, dass er sich besser fühlte, als er diese Dinge sagte und Professor Quirrells Lächeln sah.

Professor Quirrell griff in seinen Umhang, die Geste langsam und absichtlich bedeutungsvoll, und zog……ein Buch heraus. Es war anders als alle Bücher, die Harry je gesehen hatte, die Ecken und Kanten waren sichtlich unförmig; grob behauen war der Ausdruck, der ihm in den Sinn kam, als wäre es aus einer Buchmine gehackt worden.

„Was ist es?“, hauchte Harry.

„Ein Tagebuch“, sagte Professor Quirrell.

„Wessen?“

„Das von einer berühmten Person.“ Professor Quirrell lächelte breit.

„Okay…“ Professor Quirrells Ausdruck wurde ernster.

"Mr~Potter, eine der Voraussetzungen, um ein mächtiger Zauberer zu werden, ist ein ausgezeichnetes Gedächtnis.

Der Schlüssel zu einem Rätsel ist oft etwas, das man vor zwanzig Jahren in einer alten Schriftrolle gelesen hat, oder ein merkwürdiger Ring, den man am Finger eines Mannes gesehen hat, den man nur einmal getroffen hat.

Ich erwähne das, um zu erklären, wie ich mich an diesen Gegenstand und das daran befestigte Schild erinnern konnte, nachdem ich Sie erst viel später getroffen habe.

Sehen Sie, Mr~Potter, im Laufe meines Lebens habe ich eine Reihe von Privatsammlungen gesehen, die sich im Besitz von Personen befanden, die vielleicht nicht ganz so viel verdient haben -"

„Sie haben sie gestohlen?“ Harry sagte ungläubig.

„Das ist richtig“, sagte Professor Quirrell. „Und zwar erst vor kurzem. Ich denke, Sie werden diesen besonderen Gegenstand viel mehr zu schätzen wissen als der gemeine kleine Mann, der ihn zu keinem anderen Zweck besaß, als seine ebenso gemeinen Freunde mit seiner Seltenheit zu beeindrucken.“

Harry klaffte jetzt einfach auf.

„Aber wenn Sie der Meinung sind, dass mein Handeln nicht korrekt war, Mr~Potter, dann brauchen Sie Ihr besonderes Geschenk wohl nicht anzunehmen. Obwohl ich mir natürlich nicht die Mühe machen werde, es zurück zu stehlen. Also, was soll es sein?“

Professor Quirrell warf das Buch von einer Hand in die andere, woraufhin Harry unwillkürlich einen erschrockenen Blick aufsetzte.

„Oh“, sagte Professor Quirrell, "machen Sie sich keine Sorgen über ein wenig grobe Behandlung. Sie könnten dieses Tagebuch in einen Kamin werfen und es würde unversehrt wieder auftauchen.

Auf jeden Fall warte ich auf Ihre Entscheidung."

Professor Quirrell warf das Buch lässig in die Luft, fing es wieder auf und grinste.

\emph{Nein}, sagten Gryffindor und Hufflepuff.

\emph{Ja}, sagte Ravenclaw. \emph{Welchen Teil des Wortes 'Buch' habt ihr zwei nicht verstanden?}

\emph{Den Teil mit dem Diebstahl}, sagte Hufflepuff.

\emph{Oh, komm schon,} sagte Ravenclaw, \emph{du kannst nicht ernsthaft von uns verlangen, nein zu sagen und den Rest unseres Lebens damit zu verbringen, uns zu fragen, was es war}.

\emph{Vom utilitaristischen Standpunkt aus betrachtet, klingt es nach einem Netto-Positiv,} sagte Slytherin.

\emph{Betrachte es als eine wirtschaftliche Transaktion, die Gewinne aus dem Handel generiert, nur ohne den Handelsteil.

Außerdem haben wir es nicht gestohlen und es wird niemandem helfen, wenn Professor Quirrell es behält.}

\emph{Er versucht, dich dunkel zu machen!} schrie Gryffindor, und Hufflepuff nickte fest.

\emph{Sei kein naiver kleiner Junge,} sagte Slytherin, \emph{er versucht, dich Slytherin zu lehren}.

\emph{Ja}, sagte Ravenclaw. \emph{Wer auch immer das Buch ursprünglich besessen hat, war wahrscheinlich ein Todesser oder so.} \emph{Es gehört zu uns.}

Harrys Mund öffnete sich, dann hielt er inne, ein gequälter Blick auf seinem Gesicht.

Professor Quirrell schien sich gut zu amüsieren. Er hatte das Buch in der Ecke auf einem Finger balanciert und hielt es aufrecht, während er eine kleine Melodie summte.

Ein Klopfen an der Tür ertönte. Das Buch verschwand wieder in Professor Quirrells Robe, und er erhob sich von seinem Stuhl. Professor Quirrell begann, zur Tür zu gehen - \emph{und taumelte, wobei er plötzlich gegen die Wand prallte.}

„Es ist alles in Ordnung“, sagte Professor Quirrells Stimme, die plötzlich viel schwächer klang als sonst. „Setzen Sie sich, Mr~Potter, es ist nur ein Schwindelanfall. Setzen Sie sich.“

Harrys Finger umklammerten die Kante seines Stuhls, unsicher, was er tun sollte, was er tun konnte. Harry konnte Professor Quirrell nicht einmal zu nahe kommen, es sei denn, er wollte diesem Gefühl des Untergangs trotzen - Professor Quirrell richtete sich auf, dann schien sein Atem etwas schwer zu sein, und öffnete die Tür.

Die Kellnerin kam herein und brachte einen Teller mit Essen; und während sie die Teller verteilte, ging Professor Quirrell langsam zurück zum Tisch.

Aber als die Kellnerin sich hinausbeugte, saß Professor Quirrell schon wieder aufrecht und lächelte.

Trotzdem hatte die kurze Episode von „\emph{Was-auch-immer}“ Harry entschieden. Er konnte nicht nein sagen, nicht nachdem Professor Quirrell sich so viel Mühe gemacht hatte.

„Ja“, sagte Harry.

Professor Quirrell hob mahnend den Finger, nahm dann wieder seinen Zauberstab, verschloss die Tür erneut und wiederholte drei der gleichen Zauber von vorhin.

Dann holte Professor Quirrell das Buch wieder aus seinem Umhang und warf es Harry zu, der es fast in seine Suppe fallen ließ.

Harry warf Professor Quirrell einen Blick der hilflosen Entrüstung zu. So etwas durfte man mit Büchern nicht machen, ob verzaubert oder nicht.

Harry öffnete das Buch mit tief verwurzelter, instinktiver Vorsicht.

Die Seiten schienen zu dick zu sein, mit einer Beschaffenheit, die weder Muggelpapier noch Zaubererpergament entsprach. Und der Inhalt war… leer?

„Soll ich etwa sehen…“

„Sehen Sie sich den Anfang an“, sagte Professor Quirrell, und Harry blätterte

(wieder mit dieser hilflosen, tief verwurzelten Vorsicht) einen Seitenblock um.

Die Schrift war offensichtlich handgeschrieben und sehr schwer zu lesen, aber Harry dachte, die Worte könnten Latein sein.

„Was ist das?“, fragte Harry.

„Das“, sagte Professor Quirrell, "ist eine Aufzeichnung über die magischen Forschungen eines Muggelgeborenen, der nie nach Hogwarts kam.

Er verweigerte den Brief und führte seine eigenen kleinen Untersuchungen durch, die ohne Zauberstab nie sehr weit kamen.

Nach der Beschreibung auf dem Plakat nehme ich an, dass sein Name für Sie eine größere Bedeutung hat als für mich.

Das, Harry Potter, ist das Tagebuch von \emph{Roger Bacon.}"

Harry wurde fast ohnmächtig.

\emph{An der Wand, an der Professor Quirrell gestolpert war, glitzerten die zerquetschten Überreste eines wunderschönen blauen Käfers.}

