

\hypertarget{die-wahrheit-teil-4}{% \section{106. Die Wahrheit, Teil 4}\label{die-wahrheit-teil-4}}

\textbf{\uline{Die Wahrheit, Teil 4}}

\hfill\break Die sich windenden Blätter der gigantischen Dieffenbachia fühlten sich unter Harrys Schuhen wie Waldlehm an, nicht so unnachgiebig wie Beton, aber sie trugen sein Gewicht. Harry behielt die Ranken wachsam im Auge, aber sie blieben passiv. Als Harry den Fuß der belaubten Wendeltreppe erreichte, peitschten die Ranken plötzlich hervor und ergriffen Harrys Arme und Beine. Nach einem kurzen Kampf ließ Harry sich erschlaffen.

„Interessant“, sagte Professor Quirrell, als er von oben herabschwebte, ohne eines der Blätter oder Ranken der Pflanze zu berühren. „Mir ist aufgefallen, dass du keine Probleme damit zu haben scheinst, gegen eine Pflanze zu verlieren."

Harry schaute sich den Verteidigungsprofessor genauer an und sah ihn jetzt ohne die Linse der Panik. Professor Quirrell war aufrecht und bewegte sich, er flog ohne offensichtliche Schwierigkeiten; das Gefühl des Untergangs um ihn war stark. Aber seine Augen waren immer noch in den Schädel eingesunken, seine Arme dünn und abgemagert. Die Krankheit war kein Bluff gewesen, und die naheliegende Hypothese war, dass der Verteidigungsprofessor vor kurzem ein weiteres Einhorn gegessen hatte, um vorübergehend wieder zu Kräften zu kommen. Und der Verteidigungsprofessor sprach auch wie die Maske von Professor Quirrell, nicht wie Lord Voldemort, was aus Harrys Sicht vielleicht gar nicht so schlecht war. Harry wusste nicht, warum - es sei denn, der Verteidigungsprofessor brauchte ihn noch für irgendetwas - aber es schien sicherlich in Harrys eigenem Interesse zu sein, mitzuspielen.

„Du hast mich absichtlich in diese Falle laufen lassen, Professor“, antwortete Harry, genau so, wie er mit Professor Quirrell gesprochen hätte.\\ \emph{Rollen, Masken, erinnern ihn daran, wie es zwischen uns war...}\\ "Auf eigene Faust hätte ich meinen Besen benutzt."

"Vielleicht. Wie würde ein normaler Erstklässler diese Aufgabe lösen? Wenn du deinen Zauberstab hättest, meine ich."

Die Pflanze streckte nun ihre Ranken in Richtung Professor Quirrell aus, aber Professor Quirrell schwebte gerade außerhalb ihrer Reichweite.

Harry erinnerte sich jetzt daran, dass Professor Sprout von einer Teufelsschlingenpflanze gesprochen hatte, von der es im Kräuterkunde-Lehrbuch geheißen hatte, dass sie kühle, dunkle Orte wie Höhlen mochte - wie das allerdings auf eine Blattpflanze zutreffen konnte, war eine Frage des Ermessens.

"Ich würde sagen, das ist eine Teufelsschlingenpflanze und sie könnte sich vor Licht oder Hitze zurückziehen. Also könnte ein Erstklässler vielleicht Lumos benutzen? Heute würde ich Inflammare verwenden, aber ich habe diesen Zauberspruch erst im Mai gelernt."

Eine Drehung des Zauberstabs des Verteidigungsprofessors, und ein Muster von Flüssigkeitsspritzern schoss aus ihm heraus und traf die Pflanze in der Nähe ihrer Ranken, traf sie mit einem leisen Platschen und dann mit einem leisen Zischen.

Alle Ranken, die Harry berührten, schossen verzweifelt zurück und begannen, auf die wachsenden Wunden zu schlagen, die auf der Haut der Pflanze erschienen, als ob sie versuchten, den Schmerzreiz zu beseitigen; irgendetwas an der Pflanze vermittelte den Eindruck, dass sie lautlos schrie.

Professor Quirrell ließ sich nach unten treiben.\\ "Jetzt hat sie Angst vor Licht, Hitze, Säure und vor mir."

Harry trat von den letzten Blättern auf den Boden, nachdem er einen vorsichtigen Blick auf seine Roben und dann auf den Boden geworfen hatte, um sich zu vergewissern, dass nirgendwo etwas von der Säure hingespritzt war. Harry hatte den Verdacht, dass Professor Quirrell auf irgendetwas hinauswollte, aber er wusste nicht, worauf er hinauswollte.

"Ich dachte, wir wären auf einer Mission, Professor. Ich kann dich nicht aufhalten, aber ist es klug, so viel Zeit damit zu verbringen, so vor mir anzugeben?"

„Oh, wir haben Zeit“, sagte Professor Quirrell und klang amüsiert. „Es würde einen großen Aufruhr geben, wenn wir hier entdeckt würden, bewacht von einem Inferius. Du hast nicht so getan, als hättest du bei deinem Quidditch-Match von einem solchen Aufruhr gehört, bevor du in dieser Zeit angekommen bist und mit Snape gesprochen hast, wie du es getan hast."

Ein leichter Schauer überkam Harry, als er dies begriff. Alles, was er tat, um Professor Quirrell zu besiegen, würde die Schule nicht stören dürfen, oder zumindest das Quidditchspiel nicht. Selbst wenn genügend Kräfte aufgeboten werden könnten, um Lord Voldemort zu bezwingen, würde es nicht einfach sein, es zu tun, ohne dass Professor McGonagall oder Professor Flitwick oder irgendjemand anderes beim Quidditchspiel es bemerkte.

\emph{... Einen klugen Feind zu bekämpfen, war schwer.}

Und trotzdem... trotzdem schien es Harry, dass, wenn er in Professor Quirrells Schuhen stünde, er keine gemütlichen Gespräche führen und Gedankenspiele spielen würde. Professor Quirrell hatte etwas davon, wenn er sich hier Zeit ließ. Aber was? Gab es einen anderen Prozess, der zu Ende geführt werden musste?

„Übrigens, hast du mich schon verraten?“, fragte Professor Quirrell.

„\emph{Ich habe Sie noch nicht verraten}“, zischte Harry.

Der Verteidigungsprofessor gestikulierte demonstrativ mit der Waffe, die er jetzt in der linken Hand hielt, und Harry ging voraus zu der großen Holztür am Ende des Raumes und öffnete sie.

Die nächste Kammer war kleiner im Durchmesser, mit einer höheren Decke. Das Licht, das aus den gewölbten Nischen schien, war weiß, statt blau. Um sie herum schwirrten Hunderte von geflügelten Schlüsseln, die hektisch durch die Luft schlugen. Nachdem sie ein paar Sekunden lang zugeschaut hatten, wurde klar, dass nur ein einziger Schlüssel die goldene Farbe eines Schnatzes hatte - allerdings bewegte er sich langsamer als ein Schnatz in einem echten Quidditch-Spiel.\\ Am anderen Ende des Raumes befand sich eine Tür mit einem großen, auffälligen Schlüsselloch. An der linken Wand lehnte ein Besen, das Arbeitspferd der Schule, Saubermacher Sieben.

„Professor“, sagte Harry und starrte hinauf zu den Wolken und den Schwärmen von surrenden Schlüsseln, „Du sagtest, du würdest meine Fragen beantworten. Worum genau geht es hier? Wenn man glaubt, eine Tür so gesichert zu haben, dass sie sich ohne Schlüssel nicht öffnen lässt, bewahrt man den Schlüssel an einem sicheren Ort auf und gibt nur autorisierten Personen eine Kopie. Man gibt dem Schlüssel keine Flügel und lässt dann einen Besen an die Wand gelehnt stehen. Also, was zum Teufel machen wir hier drin und was ist hier los? Es ist eine offensichtliche Vermutung, dass der Zauberspiegel der einzige wirkliche Faktor ist, der den Stein bewacht, aber warum der Rest von all dem - und warum die Erstklässler ermutigen, hierher zu kommen?"

„Ich bin mir wirklich nicht sicher“, sagte der Verteidigungsprofessor.\\ Er hatte den Raum betreten und sich weit rechts von Harry postiert, um den Abstand zwischen ihnen zu wahren.\\ "Aber ich werde antworten, wie ich es versprochen habe. Dumbledores Art ist es, ein Dutzend Dinge zu tun, die verrückt zu sein scheinen, und dann verbergen nur acht davon, oder vielleicht neun, eine innere Bedeutung. Meine Vermutung ist, dass Dumbledore den Anschein erwecken will, dass ich eingeladen bin, einen Schüler als meinen Vertreter zu schicken. Eben damit Lord Voldemort, wie Dumbledore ihn sich vorstellt, weniger in Versuchung gerät, sich dadurch für schlau zu halten.\\ Stell dir vor, Dumbledore überlegt zuerst, wie er den Stein schützen kann. Stell dir vor, Dumbledore überlegt, ob er den Spiegel mit wahren Gefahren bewachen soll.\\ Stell dir vor, wie er sich vorstellt, dass ein junger Schüler auf mein Geheiß durch diese Gefahren stolpert. Ich denke, das ist es, was Dumbledore zu vermeiden versucht, indem er den Anschein erweckt, diese Strategie sei einladend und daher nicht schlau. Es sei denn, ich habe missverstanden, was Dumbledore denkt, was Lord Voldemort denken wird."

Professor Quirrell grinste, und es sah bei ihm genauso natürlich aus wie jedes Grinsen, das er Harry zuvor gezeigt hatte.

"Verschwörungen zu schmieden, ist für Dumbledore nicht selbstverständlich, aber er versucht es, weil er es muss. Für diese Aufgabe bringt Dumbledore Intelligenz, Hingabe, die Fähigkeit, aus seinen Fehlern zu lernen, und einen völligen Mangel an angeborenem Talent mit. Er ist allein schon deshalb wunderbar schwer einzuschätzen."

Harry wandte sich ab und blickte auf die Tür auf der gegenüberliegenden Seite des Raumes.

\emph{Für ihn war das kein Spiel, Professor.}

"Meine Vermutung ist, dass die vorgesehene Lösung für Erstklässler ist, den Besen zu ignorieren und Wingardium Leviosa zu benutzen, um den Schlüssel zu schnappen, da dies kein Quidditch-Spiel ist und es keine Regeln gibt, die das verbieten. Welchen absurd übermächtigen Zauber wirst du also auf diese Rätsel loslassen?"

Es herrschte eine kurze Stille, bis auf das Zischen der Schlüssel.

Harry ging einige Schritte von Professor Quirrell weg.\\ "Das hätte ich wohl nicht sagen sollen, oder?"

„Oh, nein“, sagte Professor Quirrell. „Ich denke, das ist eine ganz vernünftige Sache, die man dem mächtigsten dunklen Zauberer der Welt sagen kann, wenn er keine zehn Schritte von einem entfernt steht."

Professor Quirrell steckte seinen Zauberstab zurück in den Ärmel seiner anderen Hand, der Hand, die manchmal die Waffe hielt. Dann griff der Verteidigungsprofessor in seinen Mund und holte etwas heraus, das wie ein Zahn aussah. Er warf den falschen Zahn hoch in die Luft, und als er herunterkam, hatte er sich in einen Zauberstab verwandelt, der ein seltsames Gefühl des Wiedererkennens in Harrys Geist auslöste, als ob ein Teil von ihm diesen Zauberstab als... einen Teil von ihm erkannte... \emph{Dreizehneinhalb Zoll, Eibe, mit einem Kern aus Phönixfedern.}

Harry hatte sich die Information gemerkt, als der Zauberstabmacher Olli-irgendwas sie ihm gegeben hatte, weil es ihm so vorkam, als könnte sie für die Handlung relevant sein. Das Ereignis und der Gedanke, der ihm zugrunde lag, fühlten sich beide ein Leben lang entfernt an.

Der Verteidigungsprofessor hob den Zauberstab und zeichnete eine flammende Rune in die Luft, die ganz zackig und bösartig aussah; Harry wich instinktiv einen weiteren Schritt zurück.

Dann sprach Professor Quirrell.\\ "\textbf{\emph{Az-reth. Az-reth. Az-reth.}}"

Die flammende Rune begann, Feuer auszustoßen, das... verdreht war, als ob die gezackten Ränder der Rune die Natur des Feuers selbst geworden wären. Das Feuer loderte karmesinrot, weiter rot schattiert als Blut, glühend so intensiv wie ein Lichtbogenschweißgerät. Diese Brillanz in diesem Farbton schien an sich falsch zu sein, als ob nichts, das so weit rot schattiert war, so viel Licht abgeben sollte; und das sengende Karminrot war von schwarzen Adern durchzogen, die das Licht aus dem Feuer zu saugen schienen. In dem geschwärzten Feuer, das sich im Wechselspiel von Karminrot und Dunkelheit abzeichnete, drehten sich Tiergestalten wild von einem Raubtier zum anderen, von der Kobra über die Hyäne bis zum Skorpion.

"\textbf{\emph{Az-reth. Az-reth. Az-reth.}}"

Als Professor Quirrell das Wort sechsmal wiederholt hatte, hatte sich so viel schwarzkarminrotes Feuer ergossen wie das Volumen eines kleinen Busches. Das verfluchte Feuer verlangsamte sich in seinen Veränderungen, als Professor Quirrell die Augen darauf richtete, und nahm eine einzige Form an, die Form eines geschwärzten, blutverbrennenden Phönix.

Und etwas sagte Harry mit einer schrecklichen Gewissheit, dass, wenn dieser schwarz brennende Phönix auf Fawkes treffen würde, der wahre Phönix sterben und niemals wiedergeboren werden würde.

Professor Quirrell machte eine einzige Geste mit seinem Zauberstab, und das geschwärzte Feuer flog quer durch den Raum. Es traf die Tür und ihr Schlüsselloch, und mit einem einzigen Schwung der karmesinroten Flügel wurde der größte Teil der Tür und ein Teil des Torbogens verbrannt. Dann fegte die verdorbene karminrote Flamme weiter. Harry hatte nur einen kurzen Blick durch das Loch, um riesige Statuen zu sehen, die gerade begannen, Schwerter und Keulen zu erheben, als das geschwärzte Feuer zwischen sie eindrang und sie zerbrachen und verbrannten.\\ Als es endete, fegte der Phönix aus geschwärztem Feuer durch das Loch zurück und schwebte über Professor Quirrells linker Schulter, die sonnenglühenden karmesinroten Krallen blieben einen Zentimeter von seiner Robe entfernt.

„Geh weiter“, sagte Professor Quirrell. „Jetzt ist es sicher."

Harry schritt vorwärts, wobei er die kognitiven Muster seiner dunklen Seite aufrufen musste, um ruhig genug zu bleiben, um dies zu tun. Harry trat über die glühenden Kanten des verbliebenen Teils der Tür und blickte auf ein Schachbrett aus zerstörten riesigen Schachfiguren. Die abwechselnden Fliesen aus schwarzem und weißem Marmor auf dem Boden begannen fünf Meter nach der zerstörten Türöffnung und erstreckten sich von Wand zu Wand, hörten aber fünf Meter vor der nächsten Tür auf der gegenüberliegenden Seite des Raumes auf. Die Decke war deutlich höher, als sie eine der Statuen hätte erreichen können.

„Ich würde vermuten“, sagte Harry, und die kognitiven Muster seiner dunklen Seite sorgten dafür, dass seine Stimme ruhig blieb, „dass die beabsichtigte Lösung darin besteht, mit dem Besen aus dem vorherigen Raum über die Statuen zu fliegen, da man ihn ja eigentlich nicht brauchte, um den Schlüssel zu bekommen."

Von hinten lachte Professor Quirrell, und es war das Lachen von Lord Voldemort.\\ „Fahre fort“, sagte eine Stimme, die kälter und höher wurde. „Geh in den nächsten Raum. Ich möchte sehen, was du aus dem, was dort ist, machen wirst."

Von Dumbledore für Erstklässler eingerichtet, erinnerte sich Harry, wird es sicher sein, und er ging über das zerstörte Schachbrett, legte seine Hand auf die Klinke der Tür und drückte sie nach innen.

Eine halbe Sekunde später schlug Harry die Tür zu und sprang zurück. Harry brauchte einige Sekunden, um seine Atmung und sich selbst in den Griff zu bekommen. Von hinter der Tür kam ein fortgesetztes lautes Bellen und ein dumpfes Knallen, als ob ein Baseballschläger auf den Boden hämmert.

„Ich nehme an“, sagte Harry mit einer ebenfalls kalt gewordenen Stimme, „dass, da Dumbledore wohl kaum einen echten Bergtroll dort hineinstellen würde, die nächste Herausforderung eine Illusion meiner schlimmsten Erinnerungen ist. Wie ein Dementor, wobei die Erinnerung in die Außenwelt projiziert wird. Sehr amüsant, Professor."

Professor Quirrell bewegte sich auf die Tür zu, und Harry trat einen großen Schritt zur Seite. Abgesehen von dem Gefühl des Unheils, das den Professor jetzt stark umgab, riet Harrys dunkle Seite oder einfach nur sein Instinkt ihm, nicht in die Nähe des schwarzkarmesinroten Feuers zu kommen, das über Professor Quirrells Schulter schwebte. Professor Quirrell schwang die Tür auf und schaute hinein.

„Hm“, sagte Professor Quirrell. „Nur der Troll, wie du gesagt hast. Ah, nun ja. Ich hatte gehofft, etwas Interessanteres über dich zu erfahren. Was da drin liegt, ist ein Kokorhekkus, auch bekannt als der gemeine Irrwicht."

"Ein Irrwicht? Was macht das - nein, ich nehme an, ich weiß, was es macht."

„Ein Irrwicht“, sagte Professor Quirrell, und jetzt war seine Stimme wieder die eines Hogwarts-Professors, der eine Vorlesung hält, „zieht es in dunkle Räume, die selten geöffnet werden, wie zum Beispiel einen vernachlässigten Schrank auf dem Dachboden. Es will in Ruhe gelassen werden und wird sich in jeder Form manifestieren, von der es glaubt, dass sie dich verscheucht."

„Mich verscheuchen?“ sagte Harry. „Ich habe den Troll getötet."

"Du bist rückwärts aus dem Zimmer gesprungen, ohne nachzudenken. Ein Irrwicht sucht das instinktive Zusammenzucken, nicht die durchdachte Bedrohung. Sonst hätte er etwas Glaubwürdigeres gewählt. Auf jeden Fall ist der Standard-Gegenzauber für einen Irrwicht natürlich...\emph{verfluchtes Feuer.}"

Professor Quirrell gestikulierte, und das geschwärzte Feuer sprang von seiner Schulter und ergoss sich durch die Türöffnung.

Aus dem Inneren des Raumes ertönte ein einzelnes Quietschen und dann nichts mehr.

Sie stürmten in den ehemaligen Raum des Irrwichts, wobei Professor Quirrell dieses Mal den Anfang machte. Ohne den scheinbaren Bergtroll war der Raum nur eine weitere riesige Kammer, die von kalten blauen Lichtern erhellt wurde.

Professor Quirrells Blick wirkte distanziert, nachdenklich. Er durchquerte den Raum, ohne auf Harry zu warten, und schwang die Tür an der gegenüberliegenden Wand aus eigenem Antrieb auf. Harry folgte ihm, aber nicht dicht.

Die nächste Kammer enthielt einen Kessel, ein Regal mit abgefüllten Zutaten, Schneidebretter, Rührstäbchen und die anderen Apparate für Zaubertränke. Das Licht, das aus den gewölbten Nischen kam, war weiß statt blau, vermutlich, weil Farbsehen beim Brauen von Zaubertränken wichtig war.

Professor Quirrell stand bereits neben dem Braugerät und begutachtete ein langes Pergament, das er aufgeschnappt hatte. Die Tür zur nächsten Kammer wurde von einem Vorhang aus violettem Feuer bewacht, der viel bedrohlicher gewirkt hätte, wenn er nicht im Vergleich zu der geschwärzten Flamme, die über Professor Quirrells Schulter schwebte, blass und schwach erschienen wäre.

Harrys Glaube diese Rätsel seien zur Sicherheit gedacht hatte sich zu diesem Zeitpunkt bereits in den Urlaub verabschiedet, also sagte er nichts darüber, dass Sicherheitssysteme in der realen Welt das Ziel hatten, autorisiertes von unautorisiertem Personal zu unterscheiden, was bedeutete, dass sie Herausforderungen ausstellten, die sich in der Nähe von Leuten, die dort sein sollten oder nicht, unterschiedlich verhielten. Eine gute Sicherheitsherausforderung würde zum Beispiel testen, ob der Teilnehmer eine Schlosskombination kennt, die nur autorisierten Personen mitgeteilt wurde, und eine schlechte Sicherheitsherausforderung würde testen, ob der Teilnehmer einen Zaubertrank gemäß schriftlichen Anweisungen brauen kann, die hilfreicherweise beigefügt worden waren.

Professor Quirrell warf das Pergament in Harrys Richtung, und es flatterte zwischen ihnen auf den Boden.\\ „Was hältst du davon?“, sagte Professor Quirrell, der dann zurücktrat, damit Harry nach vorne kommen und das Pergament aufheben konnte.

„Nein“, sagte Harry, nachdem er das Pergament überflogen hatte. „Zu testen, ob der Teilnehmer ein lächerlich einfaches Logikrätsel über die Reihenfolge der Zutaten lösen kann, ist immer noch keine Herausforderung, die sich für autorisiertes und nicht autorisiertes Personal unterschiedlich verhält. Es spielt keine Rolle, ob man ein interessanteres Logikrätsel über drei Idole oder eine Reihe von Leuten mit farbigen Hüten verwendet, man verfehlt immer noch völlig den Sinn."

„Sieh dir die andere Seite an“, sagte Professor Quirrell.

Harry drehte das zwei Fuß lange Pergament um. Auf der anderen Seite stand in winzigen Buchstaben die längste Liste von Brauanweisungen, die Harry je gesehen hatte.

"Was in aller Welt-"

„Ein Trank des Glanzes, um das Purpurfeuer zu löschen“, sagte Professor Quirrell.\\ "Er wird hergestellt, indem man die gleichen Zutaten immer wieder in leicht abgewandelter Form hinzufügt. Stell dir eine Gruppe eifriger Erstklässler vor, die an allen anderen Kammern vorbeigekommen ist und denken, dass sie jetzt gleich den Zauberspiegel erreichen, und dann auf diese Aufgabe stoßen. Dieser Raum ist in der Tat das Werk des Zaubertränkemeisters."

Harry warf einen spitzen Blick auf die schwarze Feuerform auf Professor Quirrells Schulter. „Feuer kann kein Feuer schlagen?"

„Es kann“, sagte Professor Quirrell. „Ich bin mir nicht sicher, ob es das sollte.\\ Nehmen wir an, dieser Raum ist gesichert?"

Harry wollte nicht festsitzen, wenn er diesen Trank zum Spaß braute, oder aus welchem anderen Grund Professor Quirrell sie so langsam durch diese Kammern führte. Das Rezept für den Zaubertrank sah fünfunddreißig verschiedene Gelegenheiten für die Zugabe von Glockenblumen vor, vierzehn Mal für die Zugabe von \emph{"eine Locke hellen Haares"}...

"Vielleicht gibt der Trank ein tödliches Gas ab, das für erwachsene Zauberer tödlich ist, aber nicht für Kinder. Oder einer von hundert anderen tödlichen Tricks, wenn wir plötzlich ernsthaft sein wollen. Sind wir jetzt ernst?"

„Dieser Raum ist das Werk von Severus Snape“, sagte Professor Quirrell und sah wieder nachdenklich aus. „Snape ist kein Zuschauer in diesem Spiel, nicht ganz. Ihm fehlt die Intelligenz von Dumbledore, aber er besitzt die tödliche Absicht, die Dumbledore nie hatte."

„Nun, was auch immer hier vor sich geht, es hält die Kinder nicht wirklich fern“, bemerkte Harry. „Viele Erstklässler haben es durchgeschafft. Und wenn man irgendwie alle außer Kindern aussperren kann, dann zwingt das, aus Dumbledores Sicht, Lord Voldemort, ein Kind zu besitzen, um einzutreten. Ich sehe keinen Sinn darin, angesichts deiner Ziele."

„In der Tat“, sagte Professor Quirrell und rieb sich den Nasenrücken.\\ "Aber sieh es mal so, Junge, diesem Raum fehlen die Auslöser und Sicherungen, die in den anderen sind. Es gibt keine subtilen Schutzzauber, die besiegt werden müssen.\\ Es ist, als ob ich eingeladen wäre, den Zaubertrank zu umgehen und einfach einzutreten - aber Snape weiß, dass Lord Voldemort dies bemerken wird. Wenn tatsächlich eine Falle für jeden aufgestellt wurde, der den Zaubertrank nicht gebraut hat, dann wäre es klüger, Schutzzauber anzulegen und kein Anzeichen zu geben, dass dieser Raum anders ist als die anderen."

Harry hörte zu und runzelte konzentriert die Stirn.\\ "Also... der einzige Sinn, die Schutzzauber wegzulassen, besteht darin, dass du diesen Raum nicht wie ein Bulldozer durchquerst."

„Ich nehme an, dass Snape erwartet, dass ich auch das ableiten kann“, sagte der Verteidigungsprofessor. „Und darüber hinaus kann ich nicht vorhersagen, auf welchem Niveau er glaubt, dass ich spielen werde. Ich bin geduldig, und ich habe mir \emph{viel} Zeit für dieses Unterfangen gegeben. Aber Snape kennt mich nicht, er kennt nur Lord Voldemort. Er hat manchmal gesehen, wie Lord Voldemort vor Frustration schreit und Impulsen folgt, die kontraproduktiv erscheinen. Betrachte die Angelegenheit aus Snapes Perspektive: Es ist der Meister der Zaubertränke von Hogwarts, der Lord Voldemort sagt, dass er geduldig sein und Anweisungen befolgen soll, wenn er eintreten will, als wäre Lord Voldemort ein einfacher Schuljunge. Es wäre ein Leichtes, dem nachzukommen, dabei zu lächeln und sich später zu rächen. Aber Snape weiß nicht, dass es Lord Voldemort leicht fällt, so zu denken."\\ Professor Quirrell sah Harry an.\\ "Junge, du hast mich bei der Teufelsschlinge in der Luft schweben sehen, nicht wahr?"

Harry nickte.\\ Dann bemerkte er seine Verwirrung.\\ "In meinem Lehrbuch für Zauberei steht, dass es für Zauberer unmöglich ist, selbst zu schweben."

„Ja“, sagte Professor Quirrell, „so steht es in deinem Zauberlehrbuch. Kein Zauberer kann sich selbst schweben lassen oder irgendeinen Gegenstand, der sein eigenes Gewicht trägt; es ist, als würde man versuchen, sich an seinen eigenen Stiefelschlaufen hochzuziehen. Doch Lord Voldemort allein kann fliegen - \emph{wie}? Antworte so schnell du kannst."

Wenn die Frage von einem Erstklässler beantwortet werden konnte -\\ "Du hast dir von jemand anderem Besenzauber auf die Unterwäsche zaubern lassen, und anschließend hast du sie vergiftet."

„Nicht ganz“, sagte Professor Quirrell. „Besenverzauberungen erfordern eine lange schmale Form, die fest sein muss. Stoff ist nicht geeignet."

Harrys Augenbrauen zogen sich zusammen.\\ "Wie lang muss die Form sein? Kann man ein paar kurze Besen an einer Art Hilfsgerüst am Körper befestigen und damit fliegen?"

"In der Tat, anfangs habe ich mir verzauberte Stangen an Arme und Beine geschnallt, aber das war nur, um mir eine neue Flugart beizubringen."\\ Professor Quirrell zog den Ärmel seines Umhangs zurück und enthüllte den nackten Arm.\\ "Wie du sehen kannst, habe ich im Moment nichts in petto."

Harry überdachte diese weitere Einschränkung.\\ "Du hast dir von jemandem Besenverzauberungen auf deine Knochen wirken lassen?"

Professor Quirrell seufzte.\\ "Das war eine von Voldemorts gefürchtetsten Taten, zumindest wurde mir das erzählt. Nach all den Jahren und einem gewissen Maß an widerwilliger Legilimation begreife ich immer noch nicht wirklich, was mit den normalen Menschen falsch ist... Aber du gehörst nicht zu ihnen. Es ist an der Zeit, dass du einen Beitrag zu dieser Expedition leistest. Du kennst Severus Snape schon länger als ich. Erzählen mir deine eigene Analyse dieses Raumes."

Harry zögerte und versuchte, nachdenklich auszusehen.

„Ich werde erwähnen“, sagte Professor Quirrell, während der geschwärzte Feuer-Phönix auf seiner Schulter seinen Kopf auszustrecken und Harry anzustarren schien,\\ "dass ich es als Verrat bezeichnen werde, wenn du mich wissentlich scheitern lässt.\\ Ich erinnere dich daran, dass der Stein der Schlüssel zu Miss Grangers Wiederauferstehung ist und dass ich das Leben von Hunderten von Schülern als Geisel halte."

„Ich erinnere mich“, sagte Harry, und in diesem Moment kam Harrys wunderbar erfinderisches Gehirn auf einen Gedanken. Harry war sich nicht sicher, ob er ihn aussprechen sollte.

Das Schweigen dehnte sich.

„Ist dir schon etwas eingefallen?“, fragte Professor Quirrell. „Antworte in Parsel."

\emph{Nein, das würde nicht einfach werden, nicht gegen einen schlauen Gegner, der einen jederzeit zwingen konnte, die wörtliche Wahrheit zu sagen.\\ }\strut \\ „Severus, zumindest der heutige Severus, hat großen Respekt vor deiner Intelligenz“, sagte Harry stattdessen. „Ich denke... Ich denke, er würde erwarten, dass Voldemort glaubt, Severus würde nicht glauben, dass Voldemort seine Geduldsprobe bestehen könnte, aber Severus würde erwarten, dass Voldemort sie besteht."

Professor Quirrell nickte.\\ "Das ist eine plausible Theorie. Glaubst du sie selbst? Antworte in Parsel."

„\emph{Jawohl}“, zischte Harry.

Es war vielleicht nicht sicher, Informationen zurückzuhalten, nicht einmal Gedanken und Ideen...

"Der Sinn dieses Raumes ist es also, Lord Voldemort eine Stunde lang aufzuhalten.\\ Und wenn ich dich töten soll, was Dumbledore glaubt das ich tun muss, wäre das Naheliegendste, es mit einem Dementor-Kuss zu versuchen. Ich meine, man hält dich immerhin für eine körperlose Seele -- bist du das eigentlich?"

Professor Quirrell war still.\\ „Dumbledore würde nicht an diese Methode denken“, sagte der Verteidigungsprofessor nach einer Weile. „Aber Severus vielleicht."\\ Professor Quirrell tippte sich mit einem Finger gegen die Wange, sein Blick war distanziert.\\ "Du hast Macht über Dementoren, Junge, kannst du mir sagen, ob welche in der Nähe sind?"

Harry schloss die Augen.\\ Wenn es in der Umgebung Wunden in der Welt gab, konnte er sie nicht spüren.

"Keine, die ich spüren kann."

"Antworte in Parsel."

"\emph{Ich spüre keine Lebensfresser.}"

"Aber du warst ehrlich zu mir, als du diese Möglichkeit vorgeschlagen hast? Du wolltest nicht tricksen?"

"\emph{War aufrichtig. Keine Tricks.}"

"Vielleicht gibt es eine Möglichkeit, die Dementoren zu verbergen, indem man ihnen sagt, sie sollen herausspringen und eine besessene Seele fressen, wenn sie eine sehen..."\\ Professor Quirrell tippte sich immer noch an die Wange.\\ "Es ist nicht unmöglich. Oder man kann ihm sagen, dass er jeden fressen soll, der zu schnell durch diesen Raum geht, oder jeden, der kein Kind ist. Wenn man bedenkt, dass ich Hermine und Hunderte von anderen Schülern als Geiseln halte, würdest du deine Macht über Dementoren einsetzen, um mich zu verteidigen, wenn sich ein Dementor enttarnt? Antworte in Parsel."

„\emph{Ich weiß es nicht}“, zischte Harry.

„\textbf{\emph{Lebensfresser können mich nicht vernichten, denke ich}}“, zischte Professor Quirrell. „\textbf{\emph{Und ich werde diesen Körper einfach verlassen, wenn sie mir zu nahe kommen. Ich werde diesmal schnell zurückkehren, und dann wird mich niemand mehr aufhalten können. Ich werde deine Eltern jahrelang foltern, um dich dafür zu bestrafen, dass du dich mir widersetzt hast. Hunderte von Schülern sterben, auch die, die du Freunde nennst. Jetzt frage ich wieder. Wirst du die Macht über die Lebensfresser nutzen, um mich zu beschützen, wenn die Lebensfresser kommen?}}"

„\emph{Jawohl}“, flüsterte Harry.

Die Traurigkeit und das Entsetzen, die Harry verdrängt hatte, flammten wieder auf, und seine dunkle Seite hatte keine gespeicherten Muster für den Umgang mit den Emotionen.

\emph{Warum, Professor Quirrell, warum bist du so...}

Professor Quirrell lächelte.\\ "Das erinnert mich an etwas. Hast du mich schon verraten?"

"\emph{Habe dich noch nicht verraten.}"

Professor Quirrell ging zu den Zaubertränkegeräten hinüber und begann einhändig eine Wurzel zu schneiden, das Messer bewegte sich fast unsichtbar schnell und ohne sichtbare Anstrengung. Der Phönix aus verfluchten Feuer ließ sich in die gegenüberliegende Ecke des Raumes treiben und wartete dort.

„In Anbetracht der Ungewissheit scheint es klüger zu sein, die Zeit damit zu verbringen, diesen Raum wie ein Erstklässler zu passieren“, sagte der Verteidigungsprofessor. „Wir können genauso gut reden, während wir warten.\\ Du hattest Fragen, Junge? Ich habe gesagt, dass ich sie beantworten werde, also frag.“

