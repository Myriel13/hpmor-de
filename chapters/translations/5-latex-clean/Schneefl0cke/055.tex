

\hypertarget{das-stanford-gefuxe4ngnis-experiment-teil-6}{% \section{56. Das Stanford-Gefängnis-Experiment, Teil 6}\label{das-stanford-gefuxe4ngnis-experiment-teil-6}}

\textbf{\uline{Das Stanford-Gefängnis-Experiment, Teil 6}}

Still, es war zum Glück still, hinter der Metalltür auf der nächsten Ebene nach unten.

Entweder war dort hinten niemand, vielleicht schrien sie, aber ihre Stimme hatte schon aufgegeben, oder sie murmelten nur leise vor sich hin in der Dunkelheit.

\emph{.. Ich bin mir nicht sicher, ob ich das schaffe,} dachte Harry, und er konnte den verzweifelten Gedanken auch nicht auf die Dementoren schieben. \emph{Es wäre besser, unten zu sein, sicherer, sein Plan würde Zeit brauchen, um ihn umzusetzen, und die Auroren waren wahrscheinlich schon dabei, sich nach unten vorzuarbeiten. Aber wenn Harry noch mehr dieser Metalltüren passieren musste, während er still blieb und seine Atmung vollkommen regelmäßig hielt, könnte er verrückt werden; wenn er an jeder ein Stück von sich zurücklassen musste, würde bald nichts mehr von ihm übrig sein—}

Eine leuchtende, mondbeschienene Katze sprang ins Dasein und landete vor Harrys Patronus. Harry hätte fast geschrien, was seinem Image bei Bellatrix nicht gerade zuträglich gewesen wäre.

„Harry!“, sagte die Stimme von Professor McGonagall und klang so alarmiert, wie Harry es noch nie von ihr gehört hatte. „Wo bist du? Ist alles in Ordnung mit dir? Das ist mein Patronus, antworte mir!“

Mit einer krampfhaften Anstrengung räusperte sich Harry, räusperte sich, erzwang Ruhe, schaltete eine andere Persönlichkeit ein wie eine Okklumentikbarriere. Es dauerte ein paar Sekunden, und er hoffte wie der Teufel, dass Professor McGonagall dank der Kommunikationsverzögerung nichts davon mitbekam, genauso wie er wie der Teufel hoffte, dass die Patronusse nicht über ihre Umgebung berichteten.

Die unschuldige Stimme eines Jungen sagte:

„Ich bin in Marys Restaurant, Professor, in der Winkelgasse. Ich gehe gerade auf die Toilette. Was ist los?“

Die Katze sprang weg, und Bellatrix begann leise zu kichern, ein staubiges, anerkennendes Lachen, aber sie unterbrach sich abrupt bei einem Zischen von Harry.

Einen Moment später kehrte die Katze zurück und sagte mit der Stimme von Professor McGonagall: „Ich komme dich jetzt gleich abholen. Geh nirgendwo hin, wenn du nicht in der Nähe des Verteidigungsprofessors bist, geh nicht zu ihm zurück, sag zu niemandem etwas, ich werde so schnell wie möglich da sein!“

Und die leuchtende Katze huschte nach vorne und verschwand.

Harry blickte auf seine Uhr und notierte sich die Zeit, damit er, nachdem er alle von hier weggebracht hatte und Professor Quirrell den Zeitdreher wieder verankert hatte, zurückgehen und zur passenden Zeit in der Toilette von Marys Restaurant sein konnte.

\emph{.. Weißt du, sagte der problemlösende Teil seines Gehirns, es gibt eine Grenze, wie viele Beschränkungen man einem Problem hinzufügen kann, bevor es wirklich unmöglich ist, weißt du das?}

Es hätte keine Rolle spielen sollen, und das tat es auch nicht, es war kein Vergleich zum Leiden eines einzelnen Gefangenen in Askaban, und dennoch war Harry sich sehr bewusst, dass Professor McGonagall ihn umbringen würde, wenn sein Plan nicht damit endete, dass er von Marys Platz abgeholt wurde, als wäre er nie weg gewesen, und der Verteidigungsprofessor völlig unschuldig an jeglichem Fehlverhalten aussah.

…

Während sich ihr Team darauf vorbereitete, einen weiteren Bissen Territorium aus der C-Spirale zu fressen, indem sie den vorherigen Schild abschirmten und abtasteten, bevor sie ihn wieder auflösten, tippte Amelia mit den Fingern auf ihre Tasche und überlegte, ob sie den offensichtlichen Experten zu Rate ziehen sollte. \emph{Wenn er nur nicht so wäre} - Amelia hörte das vertraute Knacken von Feuer und wusste, was sie sehen würde, als sie sich umdrehte. Ein Drittel ihrer Auroren wirbelte herum und richtete ihre Zauberstäbe auf den alten Zauberer mit der Halbmondbrille und dem langen Silberbart, der direkt in ihrer Mitte erschienen war, einen leuchtend rotgoldenen Phönix auf der Schulter.

„Nicht schießen!“

Vielsaft machte es leicht, das Gesicht zu fälschen, aber die Phönixreise vorzutäuschen, wäre etwas schwieriger gewesen - die Zauberer ließen sie als einen der schnellen Wege nach Askaban zu, obwohl es keine schnellen Wege nach draußen gab. Die alte Hexe und der alte Zauberer starrten sich einen langen Moment lang an.

(Amelia fragte sich im Hinterkopf, wer von ihren Auroren die Nachricht geschickt hatte, es waren mehrere ehemalige Mitglieder des Ordens des Phönix dabei; sie versuchte sich im Hinterkopf zu erinnern, ob sie Emmelines Spatz oder Andys Katze in der Schar der hellen Geschöpfe vermisst hatte; aber sie wusste, dass es zwecklos war. Vielleicht war es nicht einmal einer ihrer Leute, denn der alte Einmischer wusste oft Dinge, die er gar nicht wissen konnte.)

Albus Dumbledore neigte den Kopf in einer höflichen Geste zu Amelia.

„Ich hoffe, ich bin hier nicht unwillkommen“, sagte der Zauberer ruhig.

„Wir stehen doch alle auf derselben Seite, nicht wahr?“

„Das kommt darauf an“, sagte Amelia mit harter Stimme. „Bist du hier, um uns zu helfen, Verbrecher zu fangen, oder um sie vor den Konsequenzen ihres Handelns zu schützen?“

\emph{Wirst du versuchen, die Mörderin meines Bruders daran zu hindern, ihren wohlverdienten Kuss zu bekommen, alter Wichtigtuer?}

Nach dem, was Amelia gehört hatte, war Dumbledore gegen Ende des Krieges klüger geworden, hauptsächlich wegen Mad-Eyes ständiger Nörgelei; aber er war in seine törichte Barmherzigkeit zurückgefallen, sobald Voldemorts Leiche gefunden wurde.

Ein Dutzend kleiner weißer und silberner Punkte, Reflektionen der leuchtenden Tiere, schimmerten von der Halbmondbrille des alten Zauberers, als er sprach.

„Noch weniger als du möchte ich, dass Bellatrix Black befreit wird“, sagte der alte Zauberer. „Sie darf dieses Gefängnis nicht lebend verlassen, Amelia.“

Bevor Amelia noch etwas sagen konnte, um ihre überraschte Genugtuung auszudrücken, gestikulierte der alte Zauberer mit seinem langen Zauberstab, und ein glühender silberner Phönix entstand, heller vielleicht als all ihre anderen Patronusse zusammengenommen. Es war das erste Mal, dass sie diesen Zauberspruch wortlos gezaubert sah.

„Befehl allen deinen Auroren, ihre Patronus-Zauber für zehn Sekunden aufzuheben“, sagte der alte Zauberer. „Was die Dunkelheit nicht findet, kann das Licht finden.“

Amelia gab den Befehl an den Kommunikationsoffizier weiter, der alle Auroren über ihre Spiegel benachrichtigte und befahl, Dumbledores Willen zu erfüllen. Das dauerte ein paar Augenblicke, und es wurde eine Zeit schrecklicher Stille, in der keiner der Auroren zu sprechen wagte, während Amelia versuchte, ihre eigenen Gedanken abzuwägen.

\emph{Sie durfte dieses Gefängnis nicht lebend verlassen.}.. Albus Dumbledore würde sich nicht ohne einen triftigen Grund in Bartemius Crouch verwandeln. \emph{Wenn er vorgehabt hätte, ihr zu sagen, warum, hätte er es bereits getan; aber es war sicherlich kein positives Zeichen. Trotzdem war es gut zu wissen, dass sie in dieser Sache zusammenarbeiten würden.}

„Jetzt“, sagte ein Chor von Spiegeln, und alle Patronuszauber erloschen, bis auf den flammenden silbernen Phönix.

„Ist noch ein Patronus vorhanden?“, sagte der alte Zauberer deutlich zu dem hellen Wesen. Das helle Wesen neigte den Kopf und nickte. „Kannst du ihn finden?“

Der Silberkopf nickte erneut. „Wirst du dich an es erinnern, sollte es fortgehen und wiederkommen?“ Ein letztes Nicken des flammenden Phönix. „Es ist vollbracht“, sagte Dumbledore.

„Vorbei“, sagten alle Spiegel einen Moment später, und Amelia hob ihren Zauberstab und begann, ihren eigenen Patronus neu zu zaubern.

(Obwohl es zusätzliche Konzentration erforderte, mit diesem wölfischen Lächeln auf dem Gesicht, an das erste Mal zu denken, als Susan sie auf die Wange geküsst hatte, anstatt sich mit dem drohenden Schicksal von Bellatrix Black zu beschäftigen.

Dieser andere Kuss war zwar ein glücklicher Gedanke, aber nicht ganz die richtige Art für den Patronus-Zauber.)

…

Sie waren noch nicht einmal am Ende des Korridors angekommen, als Harrys Patronus die Hand hob, höflich, wie in einem Klassenzimmer. Harry dachte schnell nach.

\emph{Die Frage war nur, wie - nein, auch das war offensichtlich.}

„Es scheint“, sagte Harry mit kalt amüsierter Stimme, „dass jemand diesen Patronus angewiesen hat, seine Botschaft nur zu mir zu sprechen.“

Er gluckste. „Nun denn. Verzeih mir, liebe Bella. Quietus.“

Sofort sagte der silberne Humanoide in Harrys eigener Stimme:

„Es gibt einen anderen Patronus, der diesen Patronus sucht.“

„Was?!“, sagte Harry.

Und dann, ohne eine Pause zu machen, um darüber nachzudenken, was geschah:

„Kannst du es blockieren? Es davon abhalten, dich zu finden?“

Der silberne Humanoide schüttelte den Kopf.

…

Kaum hatten Amelia und die anderen Auroren ihre Patronus-Zauber zu Ende gewirkt, als - der flammende silberne Phönix davonflog, der echte rot-goldene Phönix folgte ihm, und der alte Zauberer schritt ruhig hinter den beiden her, den langen Zauberstab tief in der Hand. Die Schilde um ihr Territorium teilten sich um den alten Zauberer wie Wasser und schlossen sich hinter ihm mit kaum einer Kräuselung.

„Albus!?“, rief Amelia. „Was glaubst du, was du da tust?“ Aber sie wusste es bereits.

„Folge mir nicht“, sagte die Stimme des alten Zauberers streng.

„Ich kann mich selbst schützen, andere kann ich nicht schützen.“

Der Fluch, den Amelia ihm hinterher rief, ließ selbst ihre eigenen Auroren zusammenzucken.

….

\emph{Das ist nicht fair, das ist nicht fair, das ist nicht fair! Es gibt eine Grenze, wie viele Beschränkungen man einem Problem hinzufügen kann, bevor es wirklich unmöglich ist!}

Harry blockte die nutzlosen Gedanken ab, ignorierte die Müdigkeit, die er verspürte, und zwang seinen Verstand, sich mit den neuen Anforderungen auseinanderzusetzen, er musste schnell denken, das Adrenalin dazu verwenden, den Ketten der Logik schnell und ohne Zögern zu folgen, anstatt es an Verzweiflung zu verschwenden.

Damit die Mission gelingen konnte, musste

(1) Harry seinen Patronus auflösen.

(2) Bellatrix musste vor den Dementoren versteckt werden, nachdem der Patronus gebannt war.

(3) Harry musste dem Sog der Dementoren widerstehen, nachdem sein Patronus gebannt war. …

\emph{Wenn ich das hier löse,} sagte Harrys Gehirn, \emph{will ich danach einen Keks, und wenn du das Problem noch schwieriger machst als das hier, ich meine das kleinste bisschen schwieriger, klettere ich aus deinem Schädel und fahre nach Tahiti.}

Harry und sein Gehirn dachten über das Problem nach.

Askaban hatte jahrhundertelang unbesiegbar dagestanden und sich auf die Unmöglichkeit verlassen, dem Blick der Dementoren zu entgehen. Wenn Harry also einen anderen Weg fand, Bellatrix vor den Dementoren zu verstecken, würde er sich entweder auf sein wissenschaftliches Wissen stützen oder auf seine Erkenntnis, dass die Dementoren der Tod waren.

Harrys Gehirn schlug vor, dass ein offensichtlicher Weg, die Dementoren daran zu hindern, Bellatrix zu sehen, darin bestand, sie dazu zu bringen, nicht mehr zu existieren, d.h. sie zu töten. Harry beglückwünschte sein Gehirn, dass es außerhalb der Box dachte und sagte ihm, es solle weitersuchen.

\emph{Töte sie und bring sie dann zurück,} kam der nächste Vorschlag. \emph{Benutze Frigideiro, um Bellatrix bis zu dem Punkt abzukühlen, an dem ihre Hirnaktivität aufhört, und wärme sie danach mit der Thermos wieder auf, so wie Leute, die in sehr kaltes Wasser fallen, eine halbe Stunde später erfolgreich wiederbelebt werden können, ohne dass ihr Gehirn Schaden nimmt.}

Harry zog dies in Betracht. Bellatrix würde in ihrem geschwächten Zustand vielleicht nicht überleben. Und es könnte den Tod nicht davon abhalten, sie zu sehen. Und er hätte Schwierigkeiten, eine kalte, bewusstlose Bellatrix weit zu tragen. Und Harry konnte sich nicht an die Forschung erinnern, welche genaue Körpertemperatur nicht tödlich, aber vorübergehend hirntötend sein sollte. Es war eine weitere gute, unkonventionelle Idee, aber Harry befahl seinem Gehirn, weiter an…

… Möglichkeiten zu denken, sich vor dem Tod zu verstecken…

Ein Stirnrunzeln wanderte über Harrys Gesicht. Irgendwo hatte er das schon mal gehört.

\emph{Eine der Voraussetzungen, um ein mächtiger Zauberer zu werden, sei ein gutes Gedächtnis,} hatte Professor Quirrell gesagt.\emph{\hfill\break Der Schlüssel zu einem Rätsel ist oft etwas, das man vor zwanzig Jahren in einer alten Schriftrolle gelesen hat, oder ein merkwürdiger Ring, den man am Finger eines Mannes gesehen hat, den man nur einmal getroffen hat.}..

Harry konzentrierte sich so gut er konnte, aber er konnte sich nicht erinnern, es lag ihm auf der Zunge, aber er konnte sich nicht erinnern; also sagte er seinem Unterbewusstsein, es solle weiter versuchen, sich zu erinnern, und richtete seine Aufmerksamkeit wieder auf die andere Hälfte des Problems.

\emph{Wie kann ich mich ohne einen Patronus-Zauber vor den Dementoren schützen?}

Der Schulleiter war einen ganzen Tag lang immer wieder aus wenigen Schritten Entfernung einem Dementor ausgesetzt gewesen und hatte dabei nur müde ausgesehen. \emph{Wie hatte der Schulleiter das geschafft? Konnte Harry das auch?} Es könnte eine zufällige genetische Veranlagung sein, in diesem Fall war Harry aufgeschmissen. \emph{Aber angenommen, das Problem wäre lösbar.}..

Dann war die naheliegende Antwort, dass Dumbledore keine Angst vor dem Tod hatte. Dumbledore hatte wirklich keine Angst vor dem Tod. Dumbledore hat ehrlich und aufrichtig geglaubt, dass der Tod das nächste große Abenteuer ist. Er glaubte es in seinem Innersten, nicht nur als bequeme Worte, um kognitive Dissonanzen zu unterdrücken, nicht nur, um vorzugeben, weise zu sein. Dumbledore hatte entschieden, dass der Tod die natürliche und normative Ordnung war, und was auch immer für eine winzige Restangst noch in ihm steckte, es hatte lange Zeit und wiederholte Aussetzungen gebraucht, bis der Dementor ihn durch diese kleine Schwachstelle hindurch entwässert hatte.

Dieser Weg war für Harry verschlossen. Und dann dachte Harry an die Kehrseite, an die offensichtliche umgekehrte Frage: \emph{Warum bin ich so viel verletzlicher als der Durchschnitt? Andere Schüler kippten nicht um, wenn sie dem Dementor gegenüberstanden.} Harry wollte den Tod vernichten, ihn beenden, wenn er es könnte. Er wollte ewig leben, wenn er es könnte; er hatte Hoffnung darauf, der Gedanke an den Tod brachte ihm kein Gefühl der Verzweiflung oder Unausweichlichkeit.

Er hing nicht blind an seinem eigenen Leben; in der Tat hatte es ihn eine Anstrengung gekostet, nicht sein ganzes Leben an der Notwendigkeit zu verbrennen, andere vor dem Tod zu schützen.

\emph{Warum hatten die Schatten des Todes eine solche Macht über Harry? Er hätte nicht gedacht, dass er sich so fürchten würde. War es Harry, der die ganze Zeit über} \emph{rational gehandelt hatte? Wer hatte insgeheim solche Angst vor dem Tod, dass er seine eigenen Gedanken verdrehte, wie Harry es Dumbledore vorgeworfen hatte?}

Harry dachte darüber nach und verhinderte, dass er zurückschreckte. Es fühlte sich unangenehm an, aber… Aber… Aber unangenehme Gedanken waren nicht immer wahr, und dieser hier klang nicht ganz richtig. Als wäre da ein Körnchen Wahrheit, aber es war nicht dort versteckt, wo die Hypothese es behauptete - und das war, als Harry begriff.

\emph{Oh. Oh, jetzt verstehe ich. Derjenige, der Angst hat, ist.}.. Harry fragte seine dunkle Seite, was sie vom Tod hielt.

Und Harrys Patronus schwankte, verdunkelte sich, erlosch fast augenblicklich, denn dieser verzweifelte, schluchzende, schreiende Schrecken, eine unsagbare Angst, die alles tun würde, um nicht zu sterben, die alles beiseite werfen würde, um nicht zu sterben, die nicht klar denken oder fühlen konnte in der Gegenwart dieses absoluten Grauens, das genauso wenig in den Abgrund der Nichtexistenz blicken konnte, wie es direkt in die Sonne hätte starren können, ein blindes, verängstigtes Ding, das nur eine dunkle Ecke finden und sich verstecken wollte, um nicht mehr darüber nachdenken zu müssen—

Die silberne Gestalt hatte sich zum Mondlicht verdunkelt, flackerte wie eine ausfallende Kerze—

\emph{Es ist alles gut,} dachte Harry, \emph{es ist alles gut.} Er stellte sich vor, wie er seine dunkle Seite wie ein verängstigtes Kind in die Arme schloss.

\emph{Es ist richtig und angemessen, entsetzt zu sein, denn der Tod ist schrecklich. Man muss sein Entsetzen nicht verstecken, man muss sich nicht dafür schämen, man kann es wie ein Ehrenabzeichen tragen, offen in der Sonne.}

Es war seltsam, sich so gespalten zu fühlen, die Gedankenspur, die den Trost spendete, die Gedankenspur, die dem Unverständnis seiner dunklen Seite über die Fremdartigkeit der Gedanken des gewöhnlichen Harry folgte; von all den Dingen, die seine dunkle Seite mit ihrer eigenen Angst vor dem Tod verband, war das eine, was sie nie erwartet oder sich vorgestellt hatte, dass sie es finden würde, Akzeptanz und Lob und Hilfe.

\emph{.. Du musst nicht allein kämpfen,} sagte Harry leise zu seiner dunklen Seite. \emph{Der Rest von mir wird dich in dieser Sache unterstützen. Ich werde mich nicht sterben lassen, und meine Freunde auch nicht. Nicht du/Ich, nicht Hermine, nicht Mum oder Dad, nicht Neville oder Draco oder sonst wer, das ist der Wille zu beschützen.}

.. Er visualisierte Flügel aus Sonnenlicht, wie die Flügel des Patronus, die er ausgebreitet hatte, um dem verängstigten Kind Schutz zu geben.

Der Patronus leuchtete wieder auf, die Welt drehte sich um Harry oder war es sein eigener Geist, der sich drehte? \emph{Nimm meine Hand}, dachte Harry und stellte es sich vor, \emph{komm mit mir, und wir werden diese Sache zusammen machen.}

.. Es gab einen Ruck in Harrys Geist, als hätte sein Gehirn einen Schritt nach links gemacht, oder das Universum einen Schritt nach rechts.

Und in einem hell erleuchteten Korridor in Askaban, die schummrigen Gaslichter weit überstrahlt von dem gleichmäßigen und unerschütterlichen Licht eines menschenförmigen Patronus, stand ein unsichtbarer Junge mit einem seltsamen kleinen Lächeln auf dem Gesicht, der nur leicht zitterte.

Harry wusste irgendwie, dass er soeben etwas Bedeutendes getan hatte, etwas, das über die bloße Stärkung seiner Widerstandskraft gegen Dementoren hinausging.

Und mehr als das, er hatte sich erinnert. An den Tod als anthropomorphe Figur zu denken, hatte den Trick bewirkt, ironischerweise. Jetzt konnte Harry sich daran erinnern, was angeblich jemanden vor dem Blick des Todes selbst verbarg.

…

In einem Korridor von Askaban kamen die schreitenden Beine eines Zauberers abrupt zum Stillstand, denn das helle, silberne Ding, das ihn begleitete, war mitten in der Luft stehen geblieben und flatterte verzweifelt mit den Flügeln. Der strahlend weiße Phönix neigte den Kopf und schaute verwirrt hin und her, dann drehte er sich zu seinem Herrn um und schüttelte entschuldigend den Kopf. Ohne ein weiteres Wort drehte sich der alte Zauberer um und schritt den Weg zurück, den er gekommen war.

…

Harry stand gerade und aufrecht und fühlte, wie die Angst über ihn und um ihn herum hereinbrach. Ein winziger Teil von ihm mochte von den Wellen der Leere, die sich unablässig an dem unbeweglichen Stein brachen, ein wenig ausgehöhlt worden sein, aber seine Glieder waren nicht kalt, und seine Magie war bei ihm. Mit der Zeit könnten diese Wellen ihn zerfressen und verzehren, durch den winzigen Teil von ihm hindurchschleichen, der noch immer vor dem Tod kauerte, anstatt seine Angst zu nutzen, um sich für den Kampf zu stärken. Aber dieser Untergang würde Zeit brauchen, denn die Schatten des Todes waren weit weg und kümmerten sich nicht um ihn. Der Makel, der Riss, die Bruchlinie, die in ihm war, war repariert worden, und die Sterne leuchteten hell in seinem Geist, weit und furchtlos und strahlend inmitten von Kälte und Dunkelheit. Für die Augen eines anderen hätte es den Anschein gehabt, dass der Junge allein in dem schwach beleuchteten Metallkorridor stand und dieses

seltsame Lächeln trug. Denn Bellatrix Black und die Schlange, die um ihre Schultern drapiert war, wurden vom Umhang der Unsichtbarkeit verdeckt, einem der drei Heiligtümer des Todes, der seinen Träger angeblich vor dem Blick des Todes selbst verbergen sollte. Das Rätsel, dessen Antwort verloren gegangen war, und das Harry wiedergefunden hatte. Und Harry wusste jetzt, dass die Verborgenheit des Umhangs mehr war als die bloße Transparenz der Desillusionierung, dass der Umhang einen verborgen hielt und nicht nur unsichtbar, so unsichtbar wie Thestrale für den Unwissenden. Und Harry wusste auch, dass es Thestral-Blut war, das das Symbol der Heiligtümer des Todes auf die Innenseite des Umhangs malte und damit jenen Teil der Macht des Todes in den Umhang band, der es dem Umhang ermöglichte, den Dementoren auf ihrer eigenen Ebene zu begegnen und sie zu blockieren.

Es hatte sich wie eine Vermutung angefühlt, und doch eine sichere Vermutung, das Wissen kam ihm in dem Moment, als er das Rätsel gelöst hatte.

Bellatrix war immer noch transparent im Mantel, aber für Harry war sie nicht mehr verborgen, er wusste, dass sie da war, so offensichtlich für ihn wie ein Thestral.

Denn Harry hatte seinen Umhang nur geliehen, nicht geschenkt; und er hatte das Heiligtum des Todes, das durch die Potter-Linie weitergegeben worden war, verstanden und gemeistert. Harry blickte die unsichtbare Frau direkt an und sagte:

„Können die Dementoren dich erreichen, Bella?“

„Nein“, sagte die Frau mit leiser, verwunderter Stimme. Dann: „Aber mein Herr… Du…“

„Wenn du etwas Dummes sagst, wird es mich ärgern“, sagte Harry kalt.

„Oder hast du den Eindruck, dass ich mich für dich opfern würde?“

„Nein, mein Herr“, antwortete der Diener des Dunklen Lords und klang verwirrt, vielleicht auch ehrfürchtig.

„Folg mir“, flüsterte Harry kalt.

Und sie setzten ihre Reise nach unten fort, während der Dunkle Lord in seinen Beutel griff, einen Keks nahm und ihn aß. Wenn Bellatrix gefragt hätte, hätte Harry behauptet, es sei wegen der Schokolade, aber sie fragte nicht.

…

Der alte Zauberer schritt zurück in die Mitte der Auroren, der silberne und der rotgoldene Phönix folgten ihm nun.

„Du—“ begann Amelia zu brüllen.

„Sie haben ihren Patronus entlassen“, sagte Dumbledore. Der alte Zauberer schien seine Stimme nicht zu erheben, aber seine ruhigen Worte überlagerten irgendwie ihre eigenen. „Ich kann sie jetzt nicht finden.“

Amelia biss die Zähne zusammen, schob eine Reihe von bissigen Bemerkungen vor sich her und wandte sich an den Kommunikationsoffizier.

„Sagen Sie dem Auror im Kommunikationsbüro, er soll die Dementoren noch einmal fragen, ob sie Bellatrix Black spüren können.“

Die Kommunikationsspezialistin sprach einen Moment lang mit ihrem Spiegel und schaute ein paar Sekunden später überrascht auf. „Nein—“

Amelia fluchte bereits heftig in Gedanken.

„- aber sie können auf den unteren Ebenen jemand anderen sehen, der kein Gefangener ist.“

„Gut!“, schnauzte Amelia. „Lassen Sie ihn dem Dementor sagen, dass ein Dutzend ihrer Art befugt ist, Askaban zu betreten und denjenigen, der das ist, und jeden, der in seiner Begleitung ist, zu ergreifen! Und wenn sie Bellatrix Black sehen, sollen sie sie sofort küssen!“

Amelia drehte sich zu Dumbledore um und starrte ihn an, um ihn zum Widerspruch herauszufordern; aber der alte Zauberer sah sie nur ein wenig traurig an und schwieg.

…

Auror McCusker beendete das Gespräch mit dem Leichnam, der vor dem Fenster schwebte, und übermittelte die Anweisungen des Direktors. Der Leichnam schenkte ihm ein tödliches Lächeln, das ihm einen Schauer über den Rücken jagte, und schwebte dann nach unten.

Kurz darauf erhob sich ein Dutzend Dementoren von dort, wo sie in der zentralen Grube von Askaban getrieben hatten, und bewegte sich auf die Wände der riesigen Metallstruktur zu, die über ihnen aufragte.

\emph{Die dunkelsten aller Kreaturen traten durch Löcher im Boden von Askaban ein und begannen ihren Marsch des Grauens.}

