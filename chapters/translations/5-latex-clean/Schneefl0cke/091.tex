

\hypertarget{rollen-teil-3}{% \section{92. Rollen, Teil 3}\label{rollen-teil-3}}

\textbf{\uline{Rollen, Teil 3}}

Es gab nichts mehr zu tun. Es gab nichts mehr zu planen. Es gab nichts mehr, woran man denken konnte. In diese Leere hinein stieg die neue schlimmste Erinnerung - der Junge, der nicht wie sein bester Freund lebte, stapfte die langen, hallenden Korridore in Richtung der Großen Halle. Da all seine Gedankenkräfte erschöpft waren, fing sein Verstand an, Gedanken wie ein Bild von Hermine, die neben ihm lief, und wortlose Begriffe wie \emph{"Das wird nie wieder vorkommen"} auszuwerfen, bis ein anderer Teil \emph{"Nein"} schrie und mit der Entschlossenheit, sie zurückzubringen, nur die Stimme dieses Teils wurde müde und der andere Teil schien unermüdlich. Ein anderer Teil seines Verstandes bestand darauf, noch einmal zu überprüfen, was er zu Professor McGonagall und Dad und Mum gesagt hatte, obwohl er nur versucht hatte, sie so schnell wie möglich von dort wegzubringen und mit begrenzter geistiger Energie gearbeitet hatte. Als ob er es irgendwie hätte besser machen können, durch einen Akt seines fehlerhaften Willens. Was von seiner Beziehung zu seinen Eltern jetzt noch übrig sein würde, konnte Harry nicht erahnen.

Schließlich kam er an eine Kreuzung, an der ein älterer Junge in grün-gesäumten schwarzen Roben wartete, der schweigend ein Lehrbuch las, auf dem Weg, den jeder wählen würde, wenn er jemanden abfangen wollte, der von den Räumen der Heiler zur Großen Halle ging. Harry trug natürlich den Unsichtbarkeitsumhang, den er nach dem Verlassen des Büros angelegt hatte und der ihn gegen fast alle Formen der magischen Entdeckung immun machte. Es hatte keinen Sinn, es jemandem leicht zu machen, der versuchte, ihn zu finden und zu töten. Und Harry wollte schon weitergehen, ohne sich die Mühe zu machen, herauszufinden, was hier vor sich ging, als er das Gesicht des Slytherin-Jungen erkannte. Da dämmerte Harry die Erkenntnis. Einer der Schüler, die über die Osterferien in der Schule geblieben waren, wäre natürlich -

"Du hast auf mich gewartet", sagte Harry laut, ohne den Umhang abzunehmen. Der Slytherin-Junge zuckte zurück, schlug mit dem Kopf gegen die Wand, sein Zauberlehrbuch aus dem fünften Jahr fiel ihm aus den Händen, bevor er mit großen Augen aufblickte.

"Du bist -"

"Unsichtbar. Ja. Sag, was du zu sagen hast."

Lesath Lestrange rappelte sich auf, stellte sich aufmerksam hin und platzte dann heraus: "Mein Herr, habe ich das Richtige getan - ich dachte, Sie würden nicht wollen, dass ich vor all den anderen auftrete, dass sie unsere Verbindung vermuten könnten - ich dachte, wenn Sie meine Hilfe wünschen, würden Sie mich sicher aufrufen -"

Es war erstaunlich, wie viele verschiedene Möglichkeiten es gab, seinen besten Freund durch Dummheit umzubringen.

"Ich -" Lesath zögerte, dann sagte er mit leiser Stimme: "Ich habe mich geirrt, nicht wahr?"

"Du hast genau so gehandelt, wie du es unter den gegebenen Umständen hättest tun sollen. Ich bin es, der ein Narr war."

"Es tut mir leid, Mylord", flüsterte Lesath.

"Wenn du mit mir gekommen wärst, hättest du dann den Troll töten können?"

Das war nicht einmal die richtige Frage, die richtige Frage war, ob Harry selbst Lesath für ausreichend gehalten hätte und sechzig Sekunden früher losgeflogen wäre, aber trotzdem…

"Ich… ich bin mir nicht sicher, Mylord… Ich bin nicht sehr willkommen bei den Duellierübungen in Slytherin, ich habe die Gesten des Tötungsfluchs nicht gelernt - soll ich diese Künste studieren, um Ihnen besser zu dienen, mein Herr?"

"Ich bestehe weiterhin darauf, dass ich nicht Ihr Herr bin", sagte Harry.

"Ja, Mylord."

"Obwohl", sagte Harry, "und das ist kein Befehl, nur eine Bemerkung, jeder sollte wissen, wie man sich verteidigt, besonders du. Ich bin sicher, der Verteidigungsprofessor würde dir dabei ganz allgemein helfen, wenn du darum bittest."

Lesath Lestrange verbeugte sich und sagte: "Ja, Mylord, ich werde Ihre Befehle befolgen, wenn ich kann, Mylord."

Harry hätte sich darüber beschwert, missverstanden worden zu sein, wenn man ihn nicht perfekt verstanden hätte. Lesath ging.

Harry starrte die Wand an.

Ehrlich gesagt hatte er gedacht, dass er bereits alle Möglichkeiten herausgefunden hatte, wie dumm er gewesen war, nachdem er einen halben Tag lang darüber nachgedacht hatte. Offensichtlich war dies nur eine weitere Selbstüberschätzung seinerseits gewesen.

\emph{Verstehst du, was du falsch gemacht hast?} sagte seine Slytherin-Seite kalt.

\emph{Ja}, dachte Harry.

\emph{Deine ethischen Bedenken machen keinen Sinn. Du legst Lesath nicht rein. Du hast genau das getan, was Lesath denkt, dass du getan hast. Du müsstest keine Ausreden erfinden, warum Lesath dir geholfen hat, du könntest einfach sagen, dass du die Schulden einforderst, weil du ihn vor Schlägern gerettet hast, dafür gibt es sechs Zeugen. Hermine starb, weil du eine extrem wertvolle Ressource vergessen hast, und du hast Lesath vergessen, weil… warum genau?}

\emph{Weil es irgendwie Dunkler-Lord-mäßig erschien, Lesath Lestrange als Lakai zu haben?} sagte Hufflepuff mit einer kleinen mentalen Stimme. \emph{Ich meine… diese Entscheidung lag wahrscheinlich hauptsächlich an mir…}

Harrys Slytherin-Seite antwortete darauf nicht mit Worten, strahlte nur Verachtung aus und ließ ein Bild von Hermines Leiche aufblitzen.

\emph{Hör auf damit! s}chrie Harry innerlich.

\emph{Nächstes Mal,} sagte Slytherin eisig, \emph{schlage ich vor, dass wir mehr Zeit damit verbringen, uns darüber Gedanken zu machen, was effizient und effektiv ist, und weniger Zeit damit, uns darüber Gedanken zu machen, was irgendwie dunkel und böse erscheint.}

\emph{Punkt für dich,} dachte Harry, \emph{das werde ich.}

\emph{Nein, wirst du nicht}, sagte Slytherin. \emph{Du wirst dir noch mehr Rationalisierungen für deine kleinlichen Bedenken einfallen lassen. Du fängst an, auf mich zu hören, wenn dein nächster Freund stirbt.}

Harry machte sich langsam Sorgen, dass er wahnsinnig werden könnte. Die Gespräche, die er mit den Stimmen in seinem Kopf führte, waren normalerweise nicht so. Der Junge-der-lebte Harry Schmerz Verres stapfte alleine weiter, während Harry durch die stillen Korridore ging.

…\\ "Wie geht es Mr. Potter?", fragte Professor Quirrell.

Es lag eine Spannung über dem Mann, man konnte es nicht ganz Besorgnis nennen, eher wie jemand im Hinterhalt, der den Zeitpunkt zum Zuschlagen abwägt. Kaum waren die Grangers mit Madam Pomfrey gegangen, klopfte der Verteidigungsprofessor an die Tür zu ihrem Büro und trat dann ein, ohne ihre Antwort abzuwarten, und sprach, bevor sie ein Wort sagen konnte. Ein Teil von Minerva fragte sich aus der Ferne, ob Harry Potter diese Angewohnheit von seinem Verteidigungsprofessor aufgeschnappt hatte, sich des Schmerzes anderer nicht bewusst zu sein, wenn er etwas anderes im Kopf hatte, oder ob es nur eine kindliche Schwäche war, der dieser Mann irgendwie nicht entwachsen war.

"Mr. Potter hat aufgehört, Miss Grangers Leiche zu bewachen", sagte sie und legte etwas von dem Frösteln, das sie empfand, in ihre Stimme.\\ Sie war sich sicher, dass der Verteidigungsprofessor nicht so viel Kummer empfand wie sie, der Mann hatte kein einziges Wort über Hermine Granger verloren. Dass er Forderungen an sie stellte -\\ "Ich glaube, er ist zum Essen gegangen."

"Ich frage nicht nach dem körperlichen Zustand des Jungen! Haben Sie - hat er -" Professor Quirrell machte eine scharfe Geste, als wolle er einen Begriff andeuten, für den er keine Worte hatte.

"Nicht wirklich", sagte sie.\\ Sie war etwa dreißig Sekunden davon entfernt, den Verteidigungsprofessor aus ihrem Büro zu befehlen.

Professor Quirrell begann, in der kleinen Enge ihres Büros auf und ab zu gehen. "Miss Granger war die einzige, deren Sorgen er wirklich beachtet hat - jetzt, wo sie weg ist, sind alle Kontrollen für die Rücksichtslosigkeit des Jungen weg. Ich sehe es jetzt. Wer ist noch da? Mr. Longbottom? Mr. Potter tut nicht so, als wären sie Gleichaltrige. Flitwick? Sein Koboldblut würde nur nach Rache schreien. Mr. Malfoy, wenn er zurückkäme? Zu welchem Zweck? Snape? Eine wandelnde Katastrophe. Dumbledore? Pfah. Die Ereignisse sind bereits auf eine Katastrophe ausgerichtet. Sie müssen in eine Richtung gelenkt werden, die sie von Natur aus nicht einschlagen würden. Auf wen könnte Mr. Potter hören, der normalerweise nicht mit ihm sprechen würde? Cedric Diggory hat ihn unterrichtet, aber was würde Mr. Diggory als Ratschlag geben? Eine Unbekannte. Mr. Potter hat lange mit Remus Lupin gesprochen. Ihm habe ich wenig Beachtung geschenkt. Würde Lupin die Worte kennen, die zu sprechen sind, die Tat, die getan werden muss, das Opfer, das gebracht werden muss, um den Kurs des Jungen zu ändern?"\\ Professor Quirrell wirbelte auf sie zu.\\ "Hat Remus Lupin während seiner Zeit beim Orden des Phönix die Trauernden getröstet oder die zu unüberlegten Taten Bewegten aufgehalten?"

"Das ist kein schlechter Gedanke", sagte sie langsam. "Ich glaube, dass Mr. Lupin in seiner Zeit in Hogwarts oft eine Stimme der Zurückhaltung für James Potter war."

"James Potter", sagte Professor Quirrell, seine Augen verengten sich. "Der Junge hat nicht viel Ähnlichkeit mit James Potter. Sind Sie vom Erfolg dieses Plans überzeugt? Nein, das ist die falsche Frage, wir sind nicht auf einen einzigen Plan beschränkt. Sind Sie sicher, dass dieser Plan ausreicht, dass wir keinen anderen aufstellen müssen? Wenn man die Frage so stellt, beantwortet sie sich von selbst. Der Weg, der in die Katastrophe führt, muss an \emph{jedem möglichen Eingriffspunkt} abgewendet werden."

Der Verteidigungsprofessor hatte wieder begonnen, durch die Enge ihres Büros zu schreiten, erreichte die eine Wand, drehte sich auf dem Absatz um und schritt zur anderen.

"Verzeihen Sie, Professor", sie gab sich keine Mühe, die Schärfe aus ihrer Stimme zu halten, "aber ich habe für heute meine Grenzen erreicht. Sie können gehen."

"Sie." Professor Quirrell drehte sich um, und sie fand sich direkt in eisblaue Augen blickend wieder. "Sie wären nach Miss Granger die Erste, an die ich denken würde, um den Jungen von einer Torheit abzuhalten. Haben Sie schon Ihr Möglichstes getan? Natürlich haben Sie das nicht."

\emph{Wie kann er es wagen, das anzudeuten.}\\ "Wenn Sie nichts mehr zu sagen haben, Professor, dann werden Sie gehen."

"Hat Ihre Konföderation herausgefunden, wer ich wirklich bin?"

Die Worte waren mit trügerischer Milde gesprochen.

"Ja, in der Tat. Jetzt -"

\textbf{Reine Magie, reine Macht krachte in den Raum wie ein Blitz, wie ein Donnerschlag, der um ihre Ohren hallte und ihre anderen Sinne betäubte, die Papiere auf ihrem Schreibtisch wurden nicht von einem beschworenen Wind, sondern von der schieren, rohen Kraft der arkanen Macht beiseite geweht.}

Dann ließ die Kraft nach, und nur Hermine Grangers Totenscheine schwebten durch die Luft auf den Boden.

"Ich bin David Monroe, der gegen Voldemort gekämpft hat", sagte der Mann, immer noch in mildem Ton. "Hören Sie auf meine Worte. Der Junge darf nicht in diesem Geisteszustand bleiben. Er wird gefährlich werden. Es ist möglich, dass Sie bereits alles getan haben, was Sie können. Aber ich finde, das ist ein sehr seltenes Ereignis, und oft mehr gesagt als getan. Ich vermute eher, dass Sie nur das getan haben, was Sie üblicherweise tun. Ich kann nicht wirklich nachvollziehen, was andere dazu treibt, ihre Grenzen zu überschreiten, da ich sie nie hatte. Menschen bleiben erstaunlich passiv, wenn sie mit der Aussicht auf den Tod konfrontiert werden. Die Angst vor öffentlichem Spott oder dem Verlust des Lebensunterhalts treibt die Menschen eher zu Extremen und dem Bruch ihrer Gewohnheiten. Auf der anderen Seite hatte der Dunkle Lord ausgezeichnete Ergebnisse mit dem Cruciatus-Fluch, der mit Bedacht bei markierten Dienern eingesetzt wird, die der Strafe nur durch Erfolg entgehen können, wobei keine Ausreden akzeptiert wurden. Stellen Sie sich diesen Zustand in Ihrem Inneren vor und fragen Sie sich, ob Sie wirklich alles getan haben, um Harry Potter aus seiner Bahn zu reißen."

"Ich bin eine Gryffindor und lasse mich nicht von Angst bewegen", schnauzte sie zurück. "Sie werden sich in meinem Büro anständig benehmen!"

"Ich finde, Angst ist eine ausgezeichnete Motivation, und in der Tat ist es Angst, die \textbf{\emph{mich}} jetzt bewegt. Du-weißt-schon-wer hielt sich bei allem Schrecken immer noch an gewisse Grenzen. Als gelehrter Zauberer, der Dumbledore oder Du-weißt-schon-wem, fast ebenbürtig ist, bin ich der Meinung, dass der Junge sich in die Reihe derer einreihen könnte, deren ausgeführten Rituale auf den Grabsteinen von Ländern eingraviert sind. Das ist keine müßige Sorge, McGonagall, ich habe bereits Worte gehört, die die schlimmsten Befürchtungen hervorrufen."

"Sind Sie verrückt? Sie glauben, Mr. Potter könnte - das ist lächerlich. Mr. Potter kann unmöglich -"\\ Ein wortloses Bild kam ihr in den Sinn, ein Stück Glas auf einer Stahlkugel.\\ "- Mr. Potter würde so etwas nicht tun!"

"Er muss es nicht absichtlich tun. Zauberer machen sich selten auf den Weg, um ihr eigenes Verhängnis heraufzubeschwören. Mr. Potter mag auf Sie nicht bösartig wirken. Aber kommt er Ihnen nicht rücksichtslos vor, wenn er sich zu einem Ziel entschlossen hat? Ich sage noch einmal, dass ich konkreten Anlass zu allergrößten Bedenken habe!"

"Haben Sie mit dem Schulleiter darüber gesprochen?", fragte sie langsam.

"Das wäre mehr als sinnlos. Dumbledore kann den Jungen nicht erreichen. Bestenfalls ist er klug genug, dies zu wissen und die Dinge nicht noch schlimmer zu machen. Mir fehlt die nötige Geisteshaltung. Sie sind diejenige, die - aber ich sehe, dass Sie immer noch nach anderen suchen, um sich zu retten."

Der Verteidigungsprofessor wandte sich von ihr ab und schritt zur Tür.\\ "Ich denke, ich werde mich mit Severus Snape beraten. Der Mann mag eine wandelnde Katastrophe sein, aber er weiß das, und vielleicht hat er ein besseres Verständnis für die Stimmung dieses Jungen. Was Sie betrifft, Madam, stellen Sie sich vor, Sie stünden am Ende Ihres Lebens und wüssten, dass Britannien - aber nein, Britannien ist ja nicht Ihre wahre Heimat, nicht wahr? Stellen Sie sich am Ende Ihres Lebens vor, wie sich die Dunkelheit durch die verblassenden Mauern von Hogwarts frisst, wissend, dass Ihre Schüler mit Ihnen sterben werden, sich an diesen Tag erinnernd und sich bewusst werdend, dass es etwas anderes gab, was Sie hätten tun können."

