

\hypertarget{rollen-teil-6}{% \section{95. Rollen, Teil 6}\label{rollen-teil-6}}

\textbf{\uline{Rollen, Teil 6}}

Das dritte Treffen (10:31 Uhr, 17. April 1992)

Der Frühling hatte begonnen, die Luft am späten Vormittag war noch kühl von den Hinterlassenschaften des Winters. Narzissen blühten inmitten des sprießenden Grases des Waldes, die zarten gelben Blütenblätter mit ihren goldenen Herzen hingen schlaff an ihren toten, ergrauten Stängeln, verwundet oder getötet von einem der plötzlichen Fröste, die man im April oft sah. Im Verbotenen Wald würde es fremdartige Lebensformen geben, Zentauren und Einhörner zumindest, und Harry hatte von Werwölfen gehört. Doch nach dem, was Harry über Werwölfe im wirklichen Leben gelesen hatte, ergab das nicht den geringsten Sinn. Harry wagte sich nicht in die Nähe der Grenze des Verbotenen Waldes, denn es gab keinen Grund, das Risiko einzugehen. Er schritt unsichtbar zwischen den gewöhnlichen Lebensformen des erlaubten Waldes umher, den Zauberstab in der Hand, einen Besen auf den Rücken geschnallt, für den Fall der Fälle. Er hatte eigentlich keine Angst; Harry fand es seltsam, dass er keine Angst empfand.

\emph{Der Zustand der ständigen Wachsamkeit, der Bereitschaft zum Kampf oder zur Flucht, fühlte sich nicht belastend oder gar abnormal an.}

Am Rande des erlaubten Waldes ging Harry, seine Füße verirrten sich nie in die Nähe der ausgetretenen Pfade, wo er leichter gefunden werden konnte, und er verließ nie die Sicht auf die Fenster von Hogwarts. Harry hatte den Wecker seiner mechanischen Uhr so eingestellt, dass er wusste, wann es Mittag war, da er nicht auf sein Handgelenk schauen konnte, da er unsichtbar war und so weiter. Es stellte sich die Frage, wie seine Brille funktionierte, während er den Umhang trug. Das Gesetz der ausgeschlossenen Mitte schien zu bedeuten, dass entweder die Rhodopsin-Komplexe in seiner Netzhaut Photonen absorbierten und sie in neurale Spikes umwandelten, oder dass diese Photonen direkt durch seinen Körper gingen und auf der anderen Seite wieder herauskamen, aber nicht beides. Es schien wirklich immer wahrscheinlicher zu werden, dass Unsichtbarkeitsmäntel einen nach außen sehen ließen, während man selbst unsichtbar war, denn auf irgendeiner fundamentalen Ebene war das die Art und Weise, wie der Entwickler - nicht gewollt - aber implizit geglaubt hatte, dass Unsichtbarkeit funktionieren sollte. Daraufhin musste man sich fragen, ob irgendjemand versucht hatte, jemanden dazu zu überreden oder zu legitimieren, dass er implizit und sachlich glaubte, dass \emph{Fixus Allesus} ein einfacher Erstjahreszauber sein sollte, und dann versuchte, ihn zu erfinden. Oder vielleicht einen würdigen Muggelgeborenen in einem Land finden, das Muggelgeborene nicht identifizierte, und ihm ein paar ausgiebige Lügen erzählen, eine Umgebungsgeschichte und entsprechende Beweise erfinden, damit er von Anfang an eine andere Vorstellung davon hatte, was Magie tun kann. Obwohl sie anscheinend immer noch eine Reihe von vorherigen Zaubern lernen müssten, bevor sie in der Lage wären, ihre eigenen zu erfinden… Es könnte nicht funktionieren. Sicherlich hatte es einige organisch verrückte Zauberer gegeben, die wirklich an ihre eigene Möglichkeit der Gottheit geglaubt hatten und es dennoch nicht geschafft hatten, Gott zu werden. Aber selbst die Verrückten hatten wahrscheinlich geglaubt, dass der Aufstiegszauber ein grandioses, dramatisches Ritual sein sollte und nicht etwas, das man mit einem sorgfältig komponierten Zucken des Zauberstabs und der Beschwörungsformel \emph{Becomus Goddus} durchführte. Harry war sich schon ziemlich sicher, dass es nicht so einfach sein würde.

Aber dann war die Frage, warum nicht? Welches Muster hatte sein Gehirn gelernt? Konnte man den Grund im Voraus vorhersagen? Ein leichter Anflug von Besorgnis durchströmte Harry, ein Hauch von Sorge, als er über diese Frage nachdachte. Die namenlose Sorge verstärkte sich, wurde größer - \emph{Professor Quirrell?}

"Mr. Potter", rief eine leise Stimme hinter ihm. Harry wirbelte herum, seine Hand ging zum Zeitumkehrer unter seinem Umhang; wieder fühlte sich das Prinzip, sofort zur Flucht bereit zu sein, nur gewöhnlich an. Langsam, mit leeren Handflächen und nach außen gewandt, schritt Professor Quirrell am Rande des Waldes auf ihn zu, aus der allgemeinen Richtung des Hogwarts-Schlosses kommend.

"Mr. Potter", sagte Professor Quirrell wieder. "Ich weiß, dass du hier bist. Du weißt, dass ich weiß, dass du hier bist. Ich muss mit dir sprechen."

Noch immer sagte Harry nichts. Professor Quirrell hatte nicht wirklich gesagt, worum es ging, und Harrys sonniger Morgenspaziergang am Waldrand hatte eine Stimmung des Schweigens in ihm erzeugt. Professor Quirrell machte einen kleinen Schritt nach links, einen Schritt nach vorne, einen weiteren nach rechts. Er neigte den Kopf mit einem Blick der Berechnung, und dann ging er fast direkt auf die Stelle zu, an der Harry stand, und blieb ein paar Schritte entfernt mit dem Gefühl des Untergangs bis zur Grenze des Erträglichen stehen.

"Bist du immer noch fest entschlossen, deinen Weg zu gehen?" sagte Professor Quirrell. "Derselbe Kurs, von dem du gestern gesprochen hast?"

Wieder antwortete Harry nicht.

Professor Quirrell seufzte. "Es gibt vieles, was ich für dich getan habe", sagte der Mann. "Was du dich auch sonst über mich wundern magst, das kannst du nicht leugnen. Ich fordere einen Teil der Schulden ein. Rede mit mir, Mr. Potter."

\emph{Dazu habe ich jetzt keine Lust}, dachte Harry; dann: \emph{Oh, richtig.}

Zwei Stunden später, nachdem Harry den Zeitdreher einmal gedreht, die genaue Zeit notiert und sich seinen genauen Standort gemerkt hatte, eine weitere Stunde gelaufen war, hineingegangen war und Professor McGonagall mitgeteilt hatte, dass er gerade mit dem Verteidigungsprofessor im Wald vor Hogwarts sprach (nur für den Fall, dass ihm etwas zustoßen würde), eine weitere Stunde gelaufen war, dann genau eine Stunde nach seinem Aufbruch an seinen ursprünglichen Standort zurückgekehrt war und den Zeitdreher erneut gedreht hatte -

"Was war das?" sagte Professor Quirrell und blinzelte. "Hast du gerade -"

"Nichts Wichtiges", sagte Harry, ohne die Kapuze seines Unsichtbarkeitsmantels zurückzuziehen oder die Hand von seinem Zeitdreher zu nehmen. "Ja, ich bin immer noch entschlossen. Um ehrlich zu sein, ich denke, ich hätte nichts sagen sollen."

Professor Quirrell legte den Kopf schief. "Ein Gedanke, der dir im Leben gute Dienste leisten wird. Gibt es irgendetwas, das deine Meinung ändern könnte?"

"Professor, wenn ich bereits wüsste, dass es ein Argument gibt, das meine Entscheidung ändern würde -"

"Das stimmt, für Leute wie uns. Aber du wärst überrascht, wie oft jemand weiß, was er zu hören erwartet, und dennoch darauf warten muss, es gesagt zu bekommen." Professor Quirrell schüttelte den Kopf. "Um es in deinen Worten auszudrücken … es gibt eine wahre Tatsache, die mir bekannt ist, dir aber nicht, und von der ich dich gerne überzeugen würde, Mr. Potter."

Harrys Augenbrauen hoben sich, obwohl ihm im nächsten Moment klar wurde, dass Professor Quirrell das nicht sehen konnte. "Das könnte Interessant sein, in Ordnung. Fahr fort."

"Dein Vorhaben ist weitaus gefährlicher, als dir bewusst ist."

Um auf diese überraschende Aussage zu antworten, brauchte Harry nicht lange nachzudenken. "Definiere gefährlich, und sage mir, was du zu wissen glaubst und woher du es zu wissen glaubst."

"Manchmal", sagte Professor Quirrell, "wenn man jemandem von einer Gefahr erzählt, kann das dazu führen, dass er direkt in sie hineinläuft. Ich habe nicht die Absicht, dass das dieses Mal passiert. Erwartest du, dass ich dir genau sage, was du nicht tun darfst? Genau, warum ich Angst habe?" Der Mann schüttelte den Kopf. "Wenn du ein Zauberer wärst, Mr. Potter, wüsstest du, dass du es ernst nehmen musst, wenn ein mächtiger Magus dir sagt, dass du dich vorsehen musst."

Es wäre eine Lüge gewesen zu sagen, dass Harry nicht verärgert war, aber er war auch kein Idiot; also sagte Harry lediglich: "Gibt es irgendetwas, das du mir sagen kannst?"

Vorsichtig setzte sich Professor Quirrell auf das Gras und zückte seinen Zauberstab, wobei seine Hand eine Haltung einnahm, die Harry wiedererkannte. Harrys Atem stockte.

"Das ist das letzte Mal, dass ich das für dich tun kann", sagte Professor Quirrell leise. Dann begann der Mann Worte zu sprechen, die seltsam waren, in einer Sprache, die Harry nicht erkennen konnte, eine Intonation, die nicht ganz menschlich schien, Worte, die Harrys Gedächtnis zu entgleiten schienen, selbst als er versuchte, sie zu erfassen, und die so schnell aus seinem Geist verschwanden, wie sie hineingekommen waren.

Der Zauber wirkte dieses Mal langsamer. Die Bäume schienen sich zu verdunkeln, Äste und Blätter verfärbten sich, als sähe man sie durch eine perfekte Sonnenbrille, die das Licht verblasste und abschwächte, ohne es zu verzerren. Die blaue Schale des Himmels wich zurück, der Horizont, dem Harrys Gehirn fälschlicherweise eine endliche Entfernung zuordnete, zog sich zurück, während er grau und dunkler grau wurde. Die Wolken wurden durchscheinend, transparent, verzogen sich und ließen die Dunkelheit durchscheinen. Der Wald wurde schattig, verblasste, versank in Schwärze. Der große Himmelsfluss wurde wieder sichtbar, als Harrys Augen sich daran gewöhnten, das größte Objekt zu sehen, das menschliche Augen jemals als mehr als einen Punkt betrachten konnten, die umgebende Milchstraße. Und die Sterne, durchdringend hell und doch fern, aus einer großen Tiefe. Professor Quirrell atmete tief durch. Dann hob er wieder seinen Zauberstab (gerade noch sichtbar, im Sternenlicht ohne Sonne und Mond) und klopfte sich mit einem Geräusch wie ein knackendes Ei auf den Kopf. Der Verteidigungsprofessor verblasste ebenfalls, wurde ebenfalls unsichtbar. Eine winzige, von wenig Licht beleuchtete Grasscheibe trieb unbesetzt im leeren Raum. Eine Zeit lang sprach keiner der beiden. Harry begnügte sich damit, in die Sterne zu schauen, ungestört sogar von seinem eigenen Körper. Was auch immer Professor Quirrell ihn hierher gerufen hatte, um es zu sagen, es würde zur rechten Zeit gesagt werden.

Zu gegebener Zeit sprach eine Stimme. "Hier gibt es keinen Krieg", sagte eine sanfte Stimme, die aus der Leere drang. "Kein Konflikt und keine Schlacht, keine Politik und kein Verrat, kein Tod und kein Leben. Das ist alles für die Torheit der Menschen. Die Sterne sind über solcher Torheit, unberührt von ihr. Hier herrscht Frieden und ewige Stille. So dachte ich einst."

Harry drehte sich um, um zu sehen, woher die Stimme kam, und sah nur Sterne.

"So dachtest du einst?" sagte Harry, als keine weiteren Worte zu kommen schienen.

"Es gibt nichts, was über die Torheit der Menschen hinausgeht", flüsterte die Stimme aus der Leere. "Es gibt nichts, was über die zerstörerischen Kräfte ausreichend intelligenter Idiotie hinausgeht, nicht einmal die Sterne selbst. Ich habe mir große Mühe gegeben, eine bestimmte goldene Schallplatte für immer zu erhalten. Ich würde es nicht gerne sehen, wenn sie durch menschliche Dummheit zerstört würde."

Wieder huschten Harrys Augen reflexartig dorthin, wo die Stimme hätte sein sollen, wieder sah er nur Leere.

"Ich denke, ich kann dich in dieser Hinsicht beruhigen, Professor. Nuklearwaffen haben keinen Feuerball, der sich über … wie weit ist Pioneer 11 entfernt? Irgendwo bei einer Milliarde Kilometer vielleicht? Muggel reden davon, dass Atomwaffen die Welt zerstören, aber was sie eigentlich meinen, ist eine leichte Erwärmung eines Teils der Erdoberfläche. Die Sonne ist eine gigantische Fusionsreaktion und sie verdampft keine entfernten Raumsonden. Das Worst-Case-Szenario eines Atomkrieges würde nicht einmal annähernd das Sonnensystem zerstören, nicht dass das ein großer Trost wäre."

"Stimmt, solange wir von Muggeln sprechen", sagte die sanfte Stimme im Sternenlicht. "Aber was wissen Muggel schon von wahrer Macht? Es sind nicht sie, die mir jetzt Angst machen. Du bist es."

"Professor", sagte Harry vorsichtig, "ich muss zwar zugeben, dass ich in meinem Leben schon ein paar kritische Fehler begannen habe, aber das ist noch ein Stück weit von einem so gewaltigem Fehler entfernt, dass die Pioneer-11-Sonde in den Explosionsradius gerät. Es gibt keine realistische Möglichkeit, das zu tun, ohne die Sonne in die Luft zu jagen. Und bevor Sie fragen, unsere Sonne ist ein Hauptreihenstern vom Typ G, sie kann nicht explodieren. Jeder Energieeintrag würde nur das Volumen des Wasserstoffplasmas erhöhen, die Sonne hat keinen entarteten Kern, der explodieren könnte. Die Sonne hat nicht genug Masse, um als Supernova zu explodieren, nicht einmal am Ende ihrer Lebenszeit."

"Welch erstaunliche Dinge die Muggel gelernt haben", murmelte die andere Stimme. "Wie Sterne leben, wie sie vor dem Tod bewahrt werden, wie sie sterben. Und sie fragen sich nie, ob dieses Wissen gefährlich sein könnte."

"In aller Offenheit, Professor, dieser spezielle Gedanke ist mir auch noch nie gekommen."

"Du bist Muggelstämmig. Ich spreche nicht von Blut, ich spreche davon, wie du deine Kindheitsjahre verbracht hast. Darin liegt eine Freiheit des Denkens, wahr. Aber es liegt auch eine gewisse Weisheit in der Vorsicht der Zaubererwelt. Es ist dreihundertdreiundzwanzig Jahre her, dass die magischen Territorien Siziliens durch die Torheit eines einzigen Mannes ruiniert wurden. Solche Vorfälle waren in den Jahren, als Hogwarts entstand, häufiger. Häufiger noch in der Zeit nach Merlin. Von der Zeit vor Merlin ist nur noch wenig zu erforschen übrig."

"Es gibt ungefähr dreißig Größenordnungen Unterschied zwischen dem und der Sprengung der Sonne", bemerkte Harry, dann fing er sich. "Aber das ist eine sinnlose Spitzfindigkeit, ein Land in die Luft zu jagen wäre auch schlimm, da stimme ich zu. Auf jeden Fall, Professor, habe ich nicht vor, so etwas zu tun."

"Deine Entscheidung ist nicht erforderlich, Mr. Potter. Wenn du mehr Romane von Zauberern und weniger Muggelgeschichten gelesen hätten, wüsstest du das. In seriöser Literatur wird der Zauberer, dessen Dummheit droht, die watschelnden Knochenmänner zu entfesseln, nicht absichtlich auf ein solches Ziel hinarbeiten, das ist etwas für Kinderbücher. Dieser wirklich gefährliche Zauberer wird vielleicht ein Projekt verfolgen, von dem er sich großen Ruhm verspricht, und die sichere Aussicht, diesen Ruhm zu verlieren und sein Leben in der Dunkelheit zu verbringen, wird ihm schlimmer erscheinen als die unbekannte Aussicht, sein Land zu zerstören. Oder er hat jemandem Erfolg versprochen, den er nicht enttäuschen kann. Vielleicht hat er Kinder, die Schulden haben. Es steckt viel literarische Weisheit in diesen Geschichten. Sie ist aus harter Erfahrung und Städten aus Asche geboren. Der wahrscheinlichste Kandidat für eine Katastrophe ist ein mächtiger Zauberer, der sich, aus welchen Gründen auch immer, nicht dazu durchringen kann, innezuhalten, wenn Warnzeichen erscheinen. Auch wenn er viel und laut von Vorsicht spricht, wird er sich nicht dazu durchringen können, tatsächlich innezuhalten. Ich frage mich, Mr. Potter, hast du daran gedacht, etwas zu versuchen, wovon Hermine Granger selbst dir abgeraten hätte?"

"In Ordnung, Punkt verstanden", sagte Harry. "Professor, ich bin mir sehr wohl bewusst, dass ich, wenn ich Hermine um den Preis von zwei anderen Menschenleben rette, vom utilitaristischen Standpunkt aus gesehen insgesamt an Punkten verloren habe. Ich bin mir sehr wohl bewusst, dass Hermine nicht wollen würde, dass ich das Risiko eingehe, ein ganzes Land zu zerstören, nur um sie zu retten. Das ist einfach gesunder Menschenverstand."

"Kind, das Dementoren vernichtet", sagte die sanfte Stimme, "wenn es nur ein Land wäre, von dem ich befürchte, dass du es vernichten könntest, wäre ich weniger besorgt. Ich habe anfangs nicht geglaubt, dass dein Wissen über Muggelwissenschaft und Muggelpraktiken eine Quelle großer Macht sein könnte. Jetzt weiß ich das ich mich geirrt habe. Ich bin, in aller Aufrichtigkeit, um die Sicherheit dieser goldenen Schallplatte besorgt."

"Nun, wenn mich Science-Fiction etwas gelehrt hat", sagte Harry, "dann, dass die Zerstörung des Sonnensystems moralisch nicht akzeptabel ist, besonders wenn man es tut, bevor die Menschheit andere Sternensysteme kolonisiert hat."

"Dann wirst du dein Vorhaben aufgeben?"

"Nein", sagte Harry, ohne zu überlegen, bevor er den Mund öffnete. Nach einem Moment fügte er hinzu: "Aber ich verstehe, was du mir zu sagen versuchst."

Stille. Die Sterne hatten sich nicht verschoben, nicht einmal so, wie sie es an einem irdischen Nachthimmel mit der Zeit getan hätten. Ein ganz leichtes Rascheln, als ob jemand seinen Körper verlagern würde. Harry erkannte, dass er schon eine Weile in der gleichen Position stand, und ließ sich auf den fast unsichtbaren Kreis aus Gras fallen, der immer noch unter ihm lag, wobei er darauf achtete, die Ränder des Zaubers nicht zu berühren.

"Sag mir eins", sagte die sanfte Stimme. "Warum ist dir dieses Mädchen so wichtig?"

"Weil sie meine Freundin ist."

"In der englischen Sprache, wie sie üblicherweise verwendet wird, Mr. Potter, wird das Wort 'Freund' nicht mit dem verzweifelten Versuch in Verbindung gebracht, Tote zu erwecken. Hast du den Eindruck, dass sie deine wahre Liebe ist, oder etwas Ähnliches?"

"Oh, nicht auch noch du", sagte Harry müde. "Nicht ausgerechnet du, Professor. Gut, wir sind beste Freunde, aber das ist alles, okay? Das ist genug. Freunde lassen Freunde nicht tot bleiben."

"Gewöhnliche Leute tun nicht so viel, für die, die sie Freunde nennen." Die Stimme klang jetzt distanzierter, abstrakter. "Nicht einmal für die, von denen sie sagen, dass sie sie lieben. Deine Gefährten sterben, und sie gehen nicht auf die Suche nach der Macht, um dich wieder auferstehen zu lassen."

Harry konnte sich nicht helfen. Er schaute noch einmal hinüber, obwohl er wusste, dass es aussichtslos war, und sah nur noch mehr Sterne.

"Lass mich raten, daraus schließt du, dass … die Menschen sich nicht so sehr um ihre Freunde kümmern, wie sie vorgeben."

Ein kurzes Lachen.

"Sie würden kaum so tun, als wärst du ihnen gleichgültig."

"Sie sorgen sich, Professor, und nicht nur um ihre wahren Lieben. Soldaten werfen sich auf Granaten, um ihre Freunde zu retten, Mütter rennen in brennende Häuser, um ihre Kinder zu retten. Aber wenn man ein Muggel ist, glaubt man nicht, dass es so etwas wie Magie gibt, um jemanden ins Leben zurückzubringen. Und normale Zauberer denken nicht… so über den Tellerrand hinaus. Ich meine, die meisten Zauberer suchen nicht nach Kräften, um sich unsterblich zu machen. Beweist das, dass sie sich nicht um ihr eigenes Leben kümmern?"

"Wie du meinst, Mr. Potter. Ich selbst würde ihr Leben als sinnlos und ohne jeden Wert betrachten. Vielleicht glauben sie irgendwo in ihren verborgenen Herzen auch, dass meine Meinung über sie die richtige ist."

Harry schüttelte den Kopf, und dann warf er verärgert die Kapuze seines Umhangs zurück und schüttelte erneut den Kopf. "Das scheint ein ziemlich konstruiertes Weltbild zu sein, Professor", sagte der schwach beleuchtete Kopf eines Jungen, der freistehend auf einem Kreis aus dunklem Gras inmitten von Sternen schwebte. "Der Versuch, einen Auferstehungszauber zu erfinden, ist einfach nichts, woran normale Menschen denken würden, also kann man nichts daraus ableiten, dass sie die Möglichkeit nicht wahrnehmen."

Einen Moment später war auch der schwach beleuchtete Umriss eines Mannes zu sehen, der auf dem Graskreis saß. "Wenn sie sich wirklich um ihre vermeintlich geliebten Menschen sorgen würden", sagte der Verteidigungsprofessor leise, "würden sie doch daran denken, oder nicht?"

"Gehirne funktionieren nicht so. Sie laden sich nicht plötzlich auf, wenn etwas auf dem Spiel steht - oder wenn doch, dann nur innerhalb harter Grenzen. Ich könnte nicht einmal die tausendste Stelle von Pi berechnen, wenn das Leben von jemandem davon abhinge."

Der schummerige Kopf neigte sich. "Aber es gibt noch eine andere mögliche Erklärung, Mr. Potter. Nämlich, dass die Menschen die Rolle der Freundschaft spielen. Sie tun nur so viel, wie diese Rolle von ihnen verlangt, und nicht mehr. Mir kommt der Gedanke, dass der Unterschied zwischen dir und ihnen vielleicht nicht darin besteht, dass du dich mehr kümmerst als sie. Warum solltest du mit so ungewöhnlich starken Gefühlen der Freundschaft geboren worden sein, dass du allein unter den Zauberern dazu getrieben wirst, Hermine Granger nach ihrem Tod wiederzubeleben? Nein, der wahrscheinlichste Unterschied ist nicht, dass du dich mehr kümmerst. Da du ein logischeres Wesen bist als sie, hast nur du gedacht, dass die Rolle des Freundes dies von dir verlangen würde."

Harry starrte hinaus zu den Sternen. Er hätte gelogen, wenn er behauptet hätte, nicht erschüttert zu sein. "Das … kann nicht wahr sein, Professor. Ich könnte ein Dutzend Beispiele in Muggelromanen nennen, in denen Menschen dazu getrieben wurden, ihre toten Freunde wieder auferstehen zu lassen. Die Autoren dieser Geschichten haben offensichtlich genau verstanden, was ich für Hermine empfinde. Obwohl du sie wohl nicht gelesen hast… vielleicht Orpheus und Eurydike? Die habe ich zwar nicht gelesen, aber ich weiß, was da drin steht."

"Solche Geschichten werden auch unter Zauberern erzählt. Es gibt die Geschichte von den Brüdern Elric. Die Geschichte von Dora Kent, die von ihrem Sohn Saul beschützt wurde. Da ist Ronald Mallett und seine zum Scheitern verurteilte Herausforderung an die Zeit. In Sizilien vor dem Untergang, das Drama von Precia Testarossa. In Nippon erzählt man von Akemi Homura und ihrer verlorenen Liebe. Was diese Geschichten gemeinsam haben, Mr. Potter, ist, dass sie alle erfunden sind. Die Zauberer im wirklichen Leben versuchen nicht dasselbe, auch wenn die Vorstellung offensichtlich nicht jenseits ihrer Vorstellungskraft liegt."

"Weil sie es sich nicht zutrauen!" Harrys Stimme erhob sich.

"Sollen wir der guten Professor McGonagall von deiner Absicht erzählen, einen Weg zu finden, Miss Granger wiederzubeleben, und sehen, was sie davon hält? Vielleicht ist es ihr einfach noch nie in den Sinn gekommen, diese Möglichkeit in Betracht zu ziehen… Ah, aber du zögerst. Du kennst ihre Antwort bereits, Mr. Potter. Weißt du, warum du sie weißt?" Man konnte das kalte Lächeln in seiner Stimme hören. "Eine reizende Technik. Danke, dass du sie mir beigebracht hast."

Harry war sich der Anspannung bewusst, die sich in seinem Gesicht aufgebaut hatte, seine Worte kamen wie abgebissen heraus. "Professor McGonagall ist nicht mit dem Muggelkonzept der zunehmenden Macht der Wissenschaft aufgewachsen, und niemand hat ihr je gesagt, dass man sehr rational denken muss, wenn das Leben eines Freundes auf dem Spiel steht -"

Die Stimme des Verteidigungsprofessors erhob sich ebenfalls. "Die Verwandlungsprofessorin liest aus einem Drehbuch ab, Mr. Potter! Dieses Drehbuch verlangt, dass sie trauert und öffentlich trauern soll, damit alle wissen, wie sehr sie sich gesorgt hat. Gewöhnliche Menschen reagieren schlecht, wenn man vorschlägt, dass sie vom Drehbuch abweichen. Wie du bereits weißt!"

"Das ist lustig, ich hätte schwören können, dass ich Professor McGonagall gestern beim Essen gesehen habe, wie sie vom Protokoll abwich. Wenn ich sie noch zehn weitere Male abschweifen sehen würde, würde ich vielleicht tatsächlich versuchen, mit ihr über die Wiederbelebung von Hermine zu reden, aber im Moment ist das neu für sie und sie braucht Übung. Letzten Endes, Professor, ist das, was du wegzuerklären versuchst, indem du Liebe und Freundschaft und alles andere als Lüge bezeichnest, nur ein menschliches Wesen, das es nicht besser weiß."

Die Stimme des Verteidigungsprofessors erhob sich in der Tonlage. "Wenn du es gewesen wärst, der von diesem Troll getötet worden wäre, würde es Hermine Granger nicht einmal in den Sinn kommen, das zu tun, was du für sie tust! Es würde weder Draco Malfoy, noch Neville Longbottom, noch McGonagall oder irgendeinem deiner wertvollen Freunde einfallen! Es gibt nicht einen Menschen auf dieser Welt, der dir die Fürsorge, die du ihr entgegenbringst, erwidern würde! Warum also? Warum tust du es, Mr. Potter?"

Es lag eine seltsame, wilde Verzweiflung in dieser Stimme.

"Warum der Einzige auf der Welt sein, der sich solche Mühe gibt, den Schein aufrechtzuerhalten, wenn keiner von ihnen jemals das Gleiche für dich tun wird?"

"Ich glaube, du irrst dich in der Tat, Professor", gab Harry gleichmäßig zurück. "In einer Reihe von Dingen sogar. Zumindest ist dein Modell meiner Gefühle fehlerhaft. Denn du verstehst mich nicht im Geringsten, wenn du glaubst, dass es mich aufhalten würde, wenn alles, was du sagst, wahr wäre. Alles auf der Welt muss irgendwo beginnen, jedes Ereignis, das geschieht, muss zum ersten Mal geschehen. Das Leben auf der Erde musste mit einem kleinen selbstreplizierenden Molekül in einer Schlammpfütze beginnen. Und wenn ich der erste Mensch auf der Welt wäre, nein -" Harrys Hand streckte sich aus, um auf die furchtbar weit entfernten Lichtpunkte zu zeigen. "- wenn ich der erste Mensch im Universum wäre, der sich jemals wirklich um jemand anderen gekümmert hat, was ich übrigens nicht bin, dann würde ich mich geehrt fühlen, dieser Mensch zu sein, und ich würde versuchen, dem gerecht zu werden."

Es herrschte eine lange Stille.

"Du sorgst dich wirklich um das Mädchen", sagte der Mann mit den schemenhaften Umrissen leise. "Du sorgst dich um sie auf eine Art und Weise, wie keiner von ihnen in der Lage ist, sich um sein eigenes Leben zu kümmern, geschweige denn um das der anderen." Die Stimme des Verteidigungsprofessors war seltsam geworden, erfüllt von einer unentzifferbaren Emotion. "Ich verstehe es nicht, aber ich weiß, wie weit du deswegen gehen wirst. Du wirst den Tod selbst herausfordern, für sie. Nichts wird dich davon abbringen."

"Ich sorge mich genug, um mich tatsächlich zu bemühen", sagte Harry leise. "Ja, das ist richtig."

Das Sternenlicht begann sich langsam zu brechen, die Welt leuchtete durch die Risse; Schlitze durch die Nacht zeigten Baumstämme und Blätter, die im Sonnenlicht glühten. Harry hob eine Hand und blinzelte angestrengt, als die zurückkehrende Helligkeit in seine an die Dunkelheit angepassten Augen schlug; und seine Augen gingen automatisch zum Verteidigungsprofessor, nur für den Fall, dass ein Angriff erfolgte, während er geblendet war. Als alle Sterne verschwunden waren und nur noch Tageslicht übrig war, saß Professor Quirrell immer noch im Gras.

"Nun, Mr. Potter", sagte er mit seiner normalen Stimme, "wenn das so ist, dann werde ich dir helfen, so gut ich kann, solange ich kann."

"Du wirst was?!" sagte Harry unwillkürlich.

"Mein Angebot, das ich es gestern gemacht habe, steht immer noch. Frag und ich werde antworten. Zeig mir dieselben naturwissenschaftlichen Bücher, die du Mr. Malfoy gegeben hast, und ich werde sie durchsehen und dir sagen, was mir einfällt. Schau nicht so überrascht, Mr. Potter, ich würde dich kaum dir selbst überlassen."

Harry starrte vor sich hin, die Tränenkanäle tränten noch immer von dem plötzlichen Licht.

Professor Quirrell sah ihn wieder an. Etwas Seltsames glitzerte in den blassen Augen. "Ich habe getan, was ich konnte, und nun fürchte ich, muss ich mich von dir verabschieden. Gut -" und der Verteidigungsprofessor zögerte. "\emph{Guten Tag, Mr. Potter.}"

"Gut -" Harry begann.

Der Mann, der im Gras saß, fiel um, sein Kopf schlug mit einem leichten Aufprall auf dem Boden auf. Gleichzeitig nahm das Gefühl des Unheils so stark ab, dass Harry auf die Füße sprang und ihm das Herz plötzlich im Hals stecken blieb. Aber die Gestalt am Boden drückte sich langsam wieder in eine kriechende Position hoch. Sie drehte sich um und sah Harry an, die Augen leer, der Mund schlaff. Sie versuchte aufzustehen und fiel zurück auf den Boden. Harry machte einen Schritt nach vorne, schierer Instinkt sagte ihm, er solle eine Hand anbieten, obwohl das falsch war; die Befürchtung, die in ihm aufstieg, wie schwach auch immer, sprach von anhaltender Gefahr. Aber die gefallene Gestalt wich von Harry zurück und begann dann langsam von ihm wegzukriechen, in die allgemeine Richtung des entfernten Schlosses.

Der Junge, der inmitten des Waldes stand, blickte hinterher.

