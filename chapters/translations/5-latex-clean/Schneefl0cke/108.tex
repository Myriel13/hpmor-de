

\hypertarget{reflektionen-2}{% \section{109. Reflektionen 2}\label{reflektionen-2}}

\textbf{\uline{Reflektionen 2}}

Die Grimmigkeit auf Albus Dumbledores Gesicht währte nur einen Augenblick, bevor sie der Fassungslosigkeit wich.

"Quirinus? Was -"

Und dann gab es eine Pause.

"Nun", sagte Albus Dumbledore. "Ich komme mir wirklich dumm vor."

"Das will ich hoffen", sagte Professor Quirrell leichthin; wenn er selbst überhaupt schockiert darüber war, erwischt worden zu sein, zeigte es sich nicht. Mit einer beiläufigen Handbewegung verwandelte er seine Roben zurück in die Kleidung eines Professors.

Dumbledores Grimmigkeit war zurückgekehrt und hatte sich verdoppelt.

"Da bin ich so sehr auf der Suche nach Voldemorts Schatten, und bemerke nicht, dass der Verteidigungsprofessor von Hogwarts ein kränkliches, halb totes Opfer ist, das von einem Geist besessen ist, der weitaus mächtiger ist als er selbst. Ich würde es Senilität nennen, wenn es nicht auch so vielen anderen entgangen wäre."

"Durchaus", sagte Professor Quirrell. Er hob die Augenbrauen. "Wirklich, bin ich so schwer zu erkennen ohne die glühend roten Augen?"

"Oh ja, in der Tat", sagte Albus Dumbledore in ruhigem Ton. "Dein Schauspiel war perfekt; ich gestehe, ich habe mich völlig getäuscht. Quirinus Quirrell schien - wie lautet der Begriff, den ich suche? Ah ja, das ist das Wort. Er schien geistig gesund zu sein."

Professor Quirrell kicherte, es sah um alles in der Welt so aus, als würden die beiden nur ein lockeres Gespräch führen.

"Ich war nie geisteskrank, weißt du. Lord Voldemort war für mich nur ein weiteres Spiel, genau wie Professor Quirrell."

Albus Dumbledore sah nicht so aus, als würde er sich über ein zwangloses Gespräch freuen. "Ich dachte mir, dass du das sagen würdest. Ich bedaure, dir mitteilen zu müssen, Tom, dass jeder, der sich dazu durchringen kann, die Rolle von Voldemort zu spielen, Voldemort ist."

"Ah", sagte Professor Quirrell und hob einen mahnenden Finger. "In dieser Argumentation gibt es ein Schlupfloch, alter Mann. Jeder, der die Rolle von Voldemort spielt, muss das sein, was Moralisten 'böse' nennen, da sind wir uns einig. Aber vielleicht ist mein wahres Ich ganz, ganz, ganz unrettbar böse, und zwar auf eine interessante andere Art und Weise, als ich es mit Voldemort vorgetäuscht habe -"

"Ich denke", stieß Albus Dumbledore hervor, "dass es mich nicht interessiert."

"Dann denkst du wohl, dass du mich sehr bald los bist", sagte Professor Quirrell. "Wie interessant. Meine unsterbliche Existenz muss davon abhängen, dass ich herausfinde, welche Falle du gestellt hast, und dass ich einen Weg finde, ihr so schnell wie möglich zu entkommen." Professor Quirrell hielt inne. "Aber lass uns zuerst über andere Dinge sprechen. Wie kommt es, dass du im Inneren des Spiegels wartest? Ich dachte, du wärst woanders."

"Ich bin dort", sagte Albus Dumbledore, "und auch im Inneren des Spiegels, leider für dich. Ich war schon immer hier, die ganze Zeit."

"Ah", sagte Professor Quirrell und seufzte. "Dann war meine kleine Ablenkung wohl umsonst."

Und die Wut von Albus Dumbledore war nicht mehr zu bändigen.

„\textbf{Ablenkung}?“", brüllte Dumbledore, seine saphirblauen Augen starr vor Wut. "\textbf{Du hast Meister Flamel wegen einer Ablenkung getötet?!}"

Professor Quirrell sah bestürzt aus. "Ich bin verletzt über die Ungerechtigkeit deiner Anschuldigung. Ich habe den, den du als Flamel kennst, nicht getötet. Ich habe lediglich einem anderen befohlen, es zu tun."

"Wie konntest du nur? Selbst du, wie konntest du nur? Er war die Bibliothek all unserer Überlieferungen! Geheimnisse, die jetzt für immer an die Zauberei verloren sind!?"

Professor Quirrells Lächeln war jetzt etwas schärfer. "Weißt du, ich begreife immer noch nicht, wie dein verdrehter Geist es für akzeptabel hält, dass Flamel unsterblich ist, aber wenn ich dasselbe versuche, macht es mich zu einem Monster."

"Meister Flamel ist nie in die Unsterblichkeit hinabgestiegen! Er -" Dumbledore verschluckte sich. "Er ist nur über den Abend hinaus wach geblieben, uns zuliebe, durch den langen, langen Tag -"

"Ich weiß nicht, ob du dich daran erinnerst", sagte Professor Quirrell mit luftiger Stimme, "aber erinnerst du dich an den Tag in deinem Büro mit Tom Riddle? Den Tag, an dem ich dich auf Knien angefleht habe, mich Nicholas Flamel vorzustellen, damit ich ihn bitten kann, sein Lehrling zu werden, um mir eines Tages den Stein der Weisen zu erschaffen? Das war mein letzter Versuch, ein guter Mensch zu sein, falls du neugierig bist. Du hast mich abgewiesen und mir einen Vortrag darüber gehalten, wie untugendhaft es sei, Angst vor dem Tod zu haben. Ich verließ dein Büro in Bitterkeit und Wut. Ich kam zu dem Schluss, dass ich genauso gut böse sein könnte, wenn man mich sowieso böse nennen würde, nur weil ich nicht sterben will; und einen Monat später tötete ich Abigail Myrtle, um die Unsterblichkeit mit anderen Mitteln zu erreichen. Selbst als ich mehr von Flamel wusste, blieb ich von deiner Heuchelei ziemlich genervt; und aus diesem Grund habe ich dich und die deinen mehr gequält, als ich es sonst getan hätte. Ich habe oft gedacht, dass du das wissen solltest, aber wir hatten nie die Gelegenheit offen darüber zu sprechen."

"Ich lehne ab", sagte Albus Dumbledore, dessen Blick nicht wankte. "Ich übernehme nicht das kleinste Fünkchen Verantwortung für das, was aus dir geworden ist. Das lag einzig und allein an dir und deinen eigenen Entscheidungen."

"Es überrascht mich nicht, dich das sagen zu hören", sagte Professor Quirrell. "Nun, jetzt bin ich neugierig, \emph{welche} Verantwortung du übernimmst. Du hast Zugang zu einer ungewöhnlichen Kraft der Weissagung; so viel habe ich schon vor langer Zeit herausgefunden. Du hast zu viele unsinnige Schachzüge gemacht, und die Wege, auf denen sie sich zu deinen Gunsten entwickelten, waren zu lächerlich. Also sag mir. Hast du das Ergebnis vorausgeahnt, in jener Nacht von Allerheiligen, als ich für eine Weile besiegt wurde?"

"Ich wusste es", sagte Albus Dumbledore, seine Stimme war tief und kalt. "Dafür übernehme ich die Verantwortung, und das ist etwas, das du nie verstehen wirst."

"Du hast es arrangiert, dass Severus Snape die Prophezeiung hört, die er zu mir gebracht hat."

"Ich habe es zugelassen", sagte Albus Dumbledore.

"Und da war ich ganz aufgeregt, weil ich endlich mein eigenes Vorwissen erlangt hatte." Professor Quirrell schüttelte den Kopf, als ob er traurig wäre. "Also hat der große Held Dumbledore seine unwissenden Bauern, Lily und James Potter, geopfert, nur um mich für ein paar Jahre zu verbannen."

Die Augen von Albus Dumbledore waren wie Steine.

"James und Lily wären bereitwillig in den Tod gegangen, wenn sie es gewusst hätten."

"Und das kleine Baby?" sagte Professor Quirrell. "Irgendwie bezweifle ich, dass die Potters so erpicht darauf gewesen wären, ihn in den Weg von Du-weißt-schon-wem zu stellen."

Man konnte das Zusammenzucken kaum sehen.

"Der Junge-der-lebte kam gut genug aus der Sache heraus. Du hast versucht, ihn in dich selbst zu verwandeln, nicht wahr? Stattdessen hast du dich in eine Leiche verwandelt, und Harry Potter wurde der Zauberer, der du hättest sein sollen."

Jetzt war da so etwas wie der übliche Dumbledore hinter der Halbmondbrille, ein winziges Glitzern in den Augen.

"Tom Riddles ganze eisige Brillanz, gezähmt im Dienste von James und Lilys Wärme und Liebe. Ich frage mich, wie du dich gefühlt hast, als du gesehen hast was aus Tom Riddle hätte werden können, wenn er in einer liebevollen Familie aufgewachsen wäre."

Professor Quirrells Lippen schürzten sich.

"Ich war überrascht, ja schockiert über die abgrundtiefen Tiefen von Mr. Potters Naivität."

"Ich nehme an, der Humor der Situation ist dir entgangen."

In diesem Moment lächelte Albus Dumbledore endlich.

"Wie habe ich gelacht, als ich es merkte! Als ich sah, dass du einen guten Voldemort geschaffen hast, um dem bösen entgegenzutreten - ach, wie habe ich gelacht! Ich hatte nie den Stahl für meine Rolle, aber Harry Potter wird ihr mehr als gewachsen sein, wenn er in seine Macht kommt."

Das Lächeln von Albus Dumbledore verschwand.

"Obwohl ich vermute, dass Harry sich einen anderen Dunklen Lord suchen muss, den er dafür bezwingen kann, da du nicht dabei sein wirst."

"Ah, ja. Das."

Professor Quirrell machte Anstalten, vom Spiegel wegzugehen, und schien kurz vor dem Punkt stehen zu bleiben, an dem der Spiegel ihn nicht mehr reflektiert hätte, wenn er ihn reflektiert hätte.

"Interessant."

Dumbledores Lächeln war jetzt kälter.

"Nein, Tom. Du wirst nirgendwo hingehen."

Professor Quirrell nickte.

"Was genau hast du getan?"

"Du hast den Tod abgelehnt", sagte Dumbledore,

"und wenn ich deinen Körper zerstören würde, würde dein Geist nur zurückwandern, wie ein dummes Tier, das nicht verstehen kann, dass es weggeschickt wird. Also schicke ich dich außerhalb der Zeit, in einen gefrorenen Augenblick, aus dem weder ich noch ein anderer dich zurückbringen kann. Vielleicht kann Harry Potter dich eines Tages zurückholen, wenn die Prophezeiung wahr ist. Vielleicht möchte er mit dir besprechen, wer die Schuld am Tod seiner Eltern trägt. Für dich wird es nur ein Augenblick sein - wenn du überhaupt jemals zurückkehrst. So oder so, Tom, ich wünsche dir das Beste."

"Hm", sagte Professor Quirrell.

Der Verteidigungsprofessor war an der Stelle vorbeigeschritten, wo Harry stand und stumm und mit so etwas wie Entsetzen zusah, um dann am anderen Rand des Spiegels wieder stehen zu bleiben.

"Wie ich vermutet habe. Du verwendest Merlins alte Methode des Versiegelns, was die Sage von Topherius Chang als den Prozess des Zeitlosen bezeichnet. Wenn die Legende stimmt, kannst nicht einmal du den Prozess aufhalten, jetzt, wo er schon so lange in Gang ist."

"In der Tat", sagte Albus Dumbledore.

Aber seine Augen waren plötzlich wachsam.

Und Harry konnte es von dort, wo er kurz vor und rechts von der Tür stand und in Stille und kontrolliertem Schrecken wartete, in der Luft spüren; er konnte das Gefühl einer Präsenz spüren, die sich im Feld des Spiegels sammelte. Etwas, das fremder war als Magie, alles daran unverständlich, außer der Tatsache seiner Fremdartigkeit und der Tatsache seiner Macht. Sie war langsam gewesen, aber jetzt wuchs sie schneller, diese Präsenz.

"Aber man könnte den Effekt noch umkehren, wenn Changs Bericht stimmt", sagte Professor Quirrell. "Die meisten Kräfte des Spiegels sind der Legende nach doppelseitig. Du könntest also stattdessen das bannen, was sich auf der anderen Seite des Spiegels befindet. Dich selbst, anstelle von mir, in diesen gefrorenen Augenblick schicken. Das heißt, wenn du willst."

"Und warum sollte ich das tun?" Albus Dumbledores Stimme war fest.

"Ich nehme an, du willst mir sagen, dass du Geiseln genommen hast? Das war zwecklos, Tom, du Narr! Du völliger Narr! Du hättest wissen müssen, dass ich dir nichts für irgendwelche Geiseln geben würde, die du genommen hast."

"Du warst schon immer einen Schritt zu langsam", sagte Professor Quirrell.

"Erlaube mir, dich mit meiner Geisel bekannt zu machen."

Eine weitere Präsenz drang in die Luft um Harry ein, ein krabbelndes Gefühl auf seinem ganzen Fleisch, als ein weiterer von Tom Riddles Zauber ganz nah an seiner Haut vorbeizog. Der Unsichtbarkeitsumhang wurde ihm entrissen, und der schwarz schimmernde Umhang flog von ihm weg, durch die Luft. Professor Quirrell fing ihn auf und zog ihn schnell über sich; in weniger als einer Sekunde hatte er die Kapuze des Umhangs über seinen Kopf gezogen und war verschwunden.

Albus Dumbledore taumelte, als wäre ihm eine wichtige Stütze entzogen worden.

"Harry Potter", hauchte der Schulleiter. "Was machst du hier?"

Harry starrte auf das Bild von Albus Dumbledore, auf dessen Gesicht sich blanker Schock und blankes Entsetzen abspielten.

Die Schuld und die Scham waren zu viel, zu viel, die Harry auf einmal trafen, und er konnte spüren, wie die unverständliche Präsenz um ihn herum zu einem Höhepunkt aufstieg. Harry wusste ohne Worte, dass ihm keine Zeit mehr blieb, und dass er am Ende war.

"Es ist meine Schuld", sagte Harry mit winziger Stimme, von welchem Teil von ihm auch immer seine Kehle in letzter Konsequenz übernommen hatte. "Ich war dumm. Ich bin immer dumm gewesen. Du darfst mich nicht retten. Auf Wiedersehen."

"Na, sieh mal einer an", ertönte Professor Quirrells Stimme aus der leeren Luft, "ich scheine kein Spiegelbild mehr zu haben."

"Nein", sagte Albus Dumbledore. "Nein, nein, \textbf{NEIN}!"

In die Hand des Albus Dumbledore flog aus seinem Ärmel sein langer, dunkelgrauer Zauberstab, und in der anderen Hand erschien, wie aus dem Nichts, ein kurzer Stab aus dunklem Stein. Albus Dumbledore warf beides gewaltsam zur Seite, gerade als das sich aufbauende Gefühl der Macht zu einem unerträglichen Höhepunkt anstieg und dann verschwand.

Der Spiegel zeigte wieder das gewöhnliche Spiegelbild eines goldbeleuchteten Raumes aus weißem Stein, ohne eine Spur davon, wo Albus Dumbledore gewesen war.

