

\hypertarget{vorgeben-weise-zu-sein-teil-1}{% \section{39. Vorgeben weise zu sein Teil 1}\label{vorgeben-weise-zu-sein-teil-1}}

\textbf{\uline{Vorgeben, weise zu sein, Teil 1

}}

\emph{Pfeifen. Tick. Bzzzt. Ding. Glorp. Pop. Splat. Glockenspiel. Toot. Puff. Klimpern. Blasen. Piep. Knall. Knistern. Whoosh. Zisch.}

Professor Flitwick hatte Harry an jenem Montag während des Zauberunterrichts leise ein gefaltetes Pergament gereicht, auf dem stand, dass Harry den Schulleiter besuchen sollte, wann es ihm passte und auf eine Art und Weise, die niemand sonst bemerken würde, insbesondere nicht Draco Malfoy oder Professor Quirrell. Sein einmaliges Passwort für den Wasserspeier sollte lauten

„\emph{Zimperliche Zitrone}“. Dazu hatte der Professor eine bemerkenswert kunstvolle Tuschezeichnung von sich selbst angefertigt, auf der er ihn streng anstarrte und dessen Augen gelegentlich blinzelten; und am unteren Rand des Zettels stand dreimal unterstrichen der Satz

\textbf{\emph{\uline{MACH KEINEN ÄRGER!}}}

Und so hatte Harry den Verwandlungsunterricht beendet und mit Hermine gelernt und zu Abend gegessen und mit seinen Leutnants gesprochen und sich schließlich, als die Uhr neun schlug, unsichtbar gemacht und war auf sechs Uhr zurückgefallen und stapfte müde in Richtung des Wasserspeiers, der sich drehenden Wendeltreppe, der Holztür, des Raums voller kleiner fummeliger Dinge und der silberbärtigen Gestalt des Schulleiters.

Diesmal sah Dumbledore ziemlich ernst aus, das übliche Lächeln fehlte; und er trug einen Pyjama in einem dunkleren und nüchterneren Lila als sonst.

„Danke, dass du gekommen bist, Harry“, sagte der Schulleiter. Der alte Zauberer erhob sich von seinem Thron und begann, langsam durch den Raum und die seltsamen Geräte zu schreiten. „Hast du die Notizen von der gestrigen Begegnung mit Lucius Malfoy dabei?“

„Notizen?“, platzte Harry heraus.

„Du hast es doch sicher aufgeschrieben…“, sagte der alte Zauberer, und seine Stimme verstummte.

Harry fühlte sich ziemlich peinlich berührt. Ja, wenn man sich gerade durch ein mysteriöses Gespräch voller bedeutsamer Andeutungen gefummelt hatte, die man nicht verstand, wäre es das Naheliegendste, alles sofort aufzuschreiben, bevor die Erinnerung verblasste, damit man später versuchen konnte, es herauszufinden.

„Also gut“, sagte der Schulleiter, „dann aus dem Gedächtnis.“

Harry rezitierte verlegen, so gut er konnte, und war fast halb durch, bevor ihm klar wurde, dass es nicht klug war, dem möglicherweise verrückten Schulleiter einfach alles zu erzählen, zumindest nicht, ohne vorher darüber nachzudenken, aber dann war Lucius definitiv ein Bösewicht und Dumbledores Gegner, also war es wahrscheinlich eine gute Idee, es ihm zu sagen, und Harry hatte

schon angefangen zu reden, und es war zu spät, jetzt noch etwas zu berechnen…

Harry beendete seine Überlegungen ehrlich.

Dumbledores Gesicht wurde immer abweisender, je weiter Harry fortfuhr, und am

und am Ende lag ein altertümlicher Ausdruck in seiner Miene, eine Strenge in der Luft.

„Nun“, sagte Dumbledore. „Dann schlage ich vor, dass du so gut wie möglich dafür sorgst, dass dem Erben von Malfoy nichts zustößt. Und ich werde das Gleiche tun.“

Der Schulleiter runzelte die Stirn, seine Finger trommelten lautlos über die tiefschwarze Oberfläche einer Platte, auf der das Wort Leliel eingraviert war.

„Und ich halte es für äußerst klug, wenn du von nun an jeden Umgang mit Lord Malfoy vermeidest.“

„Haben Sie Eulen von ihm an mich abgefangen?“, fragte Harry.

Der Schulleiter starrte Harry einen langen Moment lang an, dann nickte er zögernd.

Aus irgendeinem Grund fühlte sich Harry nicht so empört, wie er es hätte sein sollen. Vielleicht lag es nur daran, dass es Harry im Moment sehr leicht fiel, den Standpunkt des Schulleiters nachzuvollziehen. Selbst Harry konnte verstehen, warum Dumbledore nicht wollte, dass er

mit Lucius Malfoy zu tun hatte; es schien keine böse Tat zu sein. Nicht so wie die Erpressung Zabinis durch den Schulleiter… für die er nur Zabinis Wort hatten, und Zabini war äußerst unzuverlässig, in der Tat war es schwer zu verstehen, warum Zabini nicht einfach die Geschichte erzählte, die ihm die meiste Sympathie von Professor Quirrell einbrachte…

.. „Wie wäre es, wenn ich, statt zu protestieren, sage, dass ich Ihren Standpunkt verstehe“, sagte Harry, „und Sie weiterhin meine Eulen abfangen, aber Sie sagen mir, von wem?“

„Ich habe sehr viele Eulen für dich abgefangen, fürchte ich“, sagte Dumbledore nüchtern. „Du bist eine Berühmtheit, Harry, und du würdest jeden Tag Dutzende von Briefen erhalten, manche von weit außerhalb dieses Landes, wenn ich sie nicht zurückweisen würde.“

„Das“, sagte Harry, der nun anfing, ein wenig Empörung zu empfinden, „scheint mir ein wenig zu weit zu gehen—“

„Viele dieser Briefe“, sagte der alte Zauberer leise, „werden dich um Dinge bitten, die du nicht geben kannst. Ich habe sie natürlich nicht gelesen, sondern sie nur unzustellbar an ihre Absender zurückgeschickt. Aber ich weiß es, denn ich erhalte sie auch. Und du bist zu jung, Harry, um dir jeden Morgen vor dem Frühstück sechsmal das Herz brechen zu lassen.“

Harry sah auf seine Schuhe hinunter. Er sollte darauf bestehen, die Briefe zu lesen und selbst zu urteilen, aber…es gab eine kleine Stimme des gesunden Menschenverstands in ihm, und die schrie gerade sehr laut.

„Danke“, murmelte Harry.

„Der andere Grund, warum ich dich hergebeten habe“, sagte der alte Zauberer, „war, dass ich dein einzigartiges Genie zu Rate ziehen wollte.“

„Verwandlung?“, sagte Harry, überrascht und geschmeichelt.

„Nein, nicht dieses einzigartige Genie“, sagte Dumbledore. „Sag mir, Harry, welches Übel könntest du anrichten, wenn ein Dementor auf das Gelände von Hogwarts gelassen würde?“

Es zeigte sich, dass Professor Quirrell darum gebeten, oder besser gesagt gefordert hatte, dass seine Schüler ihre Fähigkeiten gegen einen echten Dementor testen sollten, nachdem sie die Worte und Gesten für den Patronus-Zauber gelernt hatten.

„Professor Quirrell ist nicht in der Lage, den Patronus-Zauber selbst zu wirken“, sagte Dumbledore, während er langsam durch die Geräte schritt. „Was nie ein gutes Zeichen ist. Aber dann hat er mir diese Tatsache freiwillig mitgeteilt, als er verlangte, dass externe Lehrer geholt werden sollten, um jedem Schüler, der es lernen wollte, den Patronus-Zauber beizubringen; er bot an, die Kosten selbst zu tragen, wenn ich das nicht tun würde. Das hat mich sehr beeindruckt. Aber jetzt besteht er darauf, einen Dementor zu holen—“

„Herr Direktor“, sagte Harry leise, „Professor Quirrell glaubt sehr stark an Versuche unter realistischen Kampfbedingungen. Dass er einen echten Dementor mitbringen will, ist völlig normal für seine Rolle.“

Jetzt warf der Schulleiter Harry einen seltsamen Blick zu. „\emph{Seine Rolle?}“, sagte der alte Zauberer.

„Ich meine“, sagte Harry, „es ist völlig im Einklang mit der Art, wie Professor Quirrell sich normalerweise verhält…“ Harry brach ab.

\emph{Warum hatte er es so ausgedrückt?}

Der Schulleiter nickte. „Du hast also dasselbe Gefühl wie ich: dass es eine Ausrede ist. Eine sehr vernünftige Ausrede, um sicher zu sein; mehr, als dir vielleicht klar ist. Es kommt oft vor, dass Zauberer, die scheinbar nicht in der Lage sind, einen Patronus-Zauber zu wirken, in der Gegenwart eines echten Dementors erfolgreich sind und von keinem einzigen Flackern des Lichts zu einem vollständigen körperlichen Patronus werden. Warum das so ist, weiß niemand; aber es ist so.“

Harry runzelte die Stirn. „Dann verstehe ich wirklich nicht, warum Sie so misstrauisch sind—“

Der Schulleiter breitete seine Hände wie in Hilflosigkeit aus.

„Harry, der Verteidigungsprofessor hat mich gebeten, die dunkelste aller Kreaturen durch die Tore von Hogwarts zu lassen. Ich muss misstrauisch sein.“

Der Schulleiter seufzte.

„Und doch wird der Dementor in einem mächtigen Käfig bewacht werden, ich selbst werde da sein, um ihn jederzeit zu bewachen - ich kann mir nicht vorstellen, dass man ihm etwas Böses antun könnte. Aber vielleicht bin ich nur unfähig, es zu sehen. Und darum bitte ich dich.“

Harry starrte den Schulleiter mit offenem Mund an. Er war so schockiert, dass er sich nicht einmal geschmeichelt fühlte.

„Ich?!“, sagte Harry.

„Ja“, sagte Dumbledore und lächelte leicht. "Ich versuche mein Bestes, um meine Feinde vorauszuahnen, ihren bösen Geist zu erfassen und ihre bösen Gedanken vorherzusagen.

Aber ich hätte mir nie vorstellen können, die Knochen eines Hufflepuffs zu Waffen zu schärfen."

\emph{Ob dieses Detail jemails aus dem Wissen der Zauberschaft vergessen werden würde?}

„Schulleiter“, sagte Harry müde, „ich weiß, dass es sich nicht gut anhört, aber ganz im Ernst: Ich bin nicht böse, ich bin nur sehr kreativ—“

„Ich habe nicht gesagt, dass du böse bist“, sagte Dumbledore ernst. „Es gibt Leute, die sagen, das Böse zu begreifen, bedeutet, böse zu werden; aber sie tun nur so, als wären sie weise. Vielmehr ist es das Böse, das die Liebe nicht kennt und sich nicht vorzustellen wagt und das die Liebe niemals verstehen kann, ohne aufzuhören, böse zu sein. Aber ich vermute, dass du dich besser in die Gedankenwelt der dunklen Zauberer hineinversetzen kannst, als ich es je könnte, und dabei selbst noch die Liebe kennst. Also, Harry.“ Die Augen des Schulleiters waren aufmerksam. „Wenn du an Professor Quirrells Stelle wärst, welche Untaten könntest du begehen, nachdem du mich dazu gebracht hast, einen Dementor auf das Gelände von Hogwarts zu lassen?“

„Warten Sie“, sagte Harry und stapfte etwas benommen zu dem Stuhl vor dem Schreibtisch des Schulleiters hinüber und setzte sich. Diesmal war es ein großer und bequemer Stuhl, kein Holzschemel, und Harry spürte, wie er sich darin einhüllte, als er darin versank.

\emph{Dumbledore wollte, dass er Professor Quirrell überlistete.}

\emph{Punkt eins}: Harry mochte Professor Quirrell lieber als Dumbledore.

\emph{Punkt zwei:} Die Hypothese war, dass der Verteidigungsprofessor etwas Böses vorhatte, und in diesem Konjunktiv sollte Harry dem Schulleiter helfen, es zu verhindern.

\emph{Punkt 3…} „Schulleiter“, sagte Harry, „wenn Professor Quirrell etwas vorhat, bin ich mir nicht sicher, ob ich ihn überlisten kann. Er hat viel mehr Erfahrung als ich.“

Der alte Zauberer schüttelte den Kopf und schaffte es irgendwie, trotz seines Lächelns sehr feierlich zu wirken. „Du unterschätzt dich.“

\emph{Das war das erste Mal, dass jemand so etwas zu Harry gesagt hatte.}

„Ich erinnere mich“, fuhr der alte Zauberer fort, „an einen jungen Mann in genau diesem Büro, kalt und beherrscht, wie er dem Leiter des Hauses Slytherin gegenüberstand und seinen eigenen Schulleiter erpresste, um seine Mitschüler zu schützen. Und ich glaube, dass dieser junge Mann gerissener ist als Professor Quirrell, gerissener als Lucius Malfoy, dass er heranwachsen wird, um Voldemort selbst ebenbürtig zu sein. Er ist es, den ich zu Rate ziehen möchte.“

Harry unterdrückte den Schauer, der ihn bei dem Namen durchlief, runzelte nachdenklich die Stirn über den Schulleiter.

\emph{Wie viel weiß er …?}

Der Schulleiter hatte Harry im Griff seiner geheimnisvollen dunklen Seite gesehen, so tief, wie Harry jemals in sie versunken war. Harry erinnerte sich immer noch daran, wie es gewesen war, unsichtbar in der Zeit zu beobachten, wie sein vergangenes Ich sich den älteren Slytherins entgegenstellte; der Junge mit der Narbe auf der Stirn, der sich nicht wie die anderen verhielt. Natürlich hätte der Schulleiter in seinem Büro etwas Seltsames an dem Jungen bemerkt… Und Dumbledore war zu dem Schluss gekommen, dass sein Lieblingsheld so gerissen war wie sein auserkorener Feind, der Dunkle Lord.

\emph{Was nicht sehr viel verlangt war, wenn man bedenkt, dass der Dunkle Lord allen seinen Dienern ein deutlich sichtbares Dunkles Mal auf den linken Armen verpasst hatte und dass er das gesamte Kloster abgeschlachtet hatte, in dem die Kampfkunst gelehrt wurde, die er lernen wollte.}

\emph{Genug Gerissenheit, um es mit Professor Quirrell aufzunehmen, wäre ein ganz anderes Problem.}

Aber es war auch klar, dass der Schulleiter nicht zufrieden sein würde, bis Harry ganz kalt und düster wurde und mit irgendeiner Art von Antwort kam, die beeindruckend gerissen klang… \emph{die aber besser nicht Professor Quirrells in die Quere kommen sollte}… Und natürlich würde Harry zu seiner dunklen Seite übergehen und es aus dieser Richtung durchdenken, nur um ehrlich zu sein und für den Fall der Fälle.

„Erzählen Sie mir“, sagte Harry, „alles darüber, wie der Dementor hergebracht werden soll und wie er bewacht werden soll.“

Dumbledores Augenbrauen hoben sich für einen Moment, und dann begann der alte Zauberer zu sprechen.

Der Dementor würde von einem Auroren-Trio auf das Gelände von Hogwarts gebracht werden, alle drei persönlich mit dem Schulleiter bekannt und alle drei in der Lage, einen körperlichen Patronus-Zauber zu wirken. Sie würden am Rande des Geländes von Dumbledore empfangen werden, der den Dementor durch die Hogwarts-Schließfächer bringen würde - Harry fragte, ob der Passierschein dauerhaft oder vorübergehend sei - ob jemand denselben Dementor am nächsten Tag einfach wieder herbringen könne. Der Pass war temporär (antwortete der Schulleiter mit einem zustimmenden Nicken), und die Erklärung ging weiter: Der Dementor befände sich in einem Käfig aus massiven Titanstäben, nicht verwandelt, sondern echt geschmiedet; mit der Zeit würde die Anwesenheit eines Dementors dieses Metall zu Staub korrodieren, aber nicht an einem einzigen Tag. Schüler, die darauf warteten, an die Reihe zu kommen, hielten sich weit hinter dem Dementor auf, hinter zwei leibhaftigen Patronen, die jeweils von zwei der drei Auroren betreut wurden. Dumbledore würde mit seinem Patronus am Käfig des Dementors warten. Ein einzelner Schüler würde sich dem Dementor nähern, Dumbledore würde seinen Patronus auflösen und der Schüler würde versuchen, seinen eigenen Patronus-Zauber zu wirken; und wenn er scheitern würde, würde Dumbledore seinen Patronus wiederherstellen, bevor der Schüler einen dauerhaften Schaden erleiden könnte. Der frühere Duell-Champion Professor Flitwick würde ebenfalls anwesend sein, solange Schüler in der Nähe waren, nur um die Sicherheit zu erhöhen.

„Warum wartest nur du bei dem Dementor?“, fragte Harry. „Ich meine, sollten nicht Sie plus ein Auror—“

Der Schulleiter schüttelte den Kopf.

„Sie könnten dem wiederholten Kontakt mit dem Dementor nicht standhalten, jedes Mal, wenn ich meinen Patronus auflöse.“

Und wenn Dumbledores Patronus aus irgendeinem Grund versagen sollte, während einer der Schüler noch in der Nähe des Dementors war, würde der dritte Auror einen weiteren körperlichen Patronus zaubern und ihn zum Schutz des Schülers schicken… Harry stocherte und stocherte, aber er konnte keine Lücke in der Sicherheit erkennen. Also holte Harry tief Luft, sank weiter in den Stuhl, schloss die Augen und erinnerte sich:

\emph{„Und das wären dann…fünf Punkte? Nein, machen wir doch gleich zehn Punkte von Ravenclaw für Widerrede.}“

Die Kälte kam jetzt langsamer, widerwilliger, Harry hatte in letzter Zeit nicht viel von seiner dunklen Seite abgerufen… Harry musste die ganze Sitzung in Zaubertränke im Geiste durchgehen, bevor sein Blut in etwas annähernd tödlich kristalline Klarheit erstarrte. Und dann dachte er an den Dementor.

\emph{Und es war offensichtlich.}

„Der Dementor ist eine Ablenkung“, sagte Harry. Die Kälte war deutlich in seiner Stimme, denn das war es, was Dumbledore wollte und erwartete.

„Eine große, hervorstechende Bedrohung, aber im Endeffekt überschaubar und leicht abzuwehren. Während also all Ihre Aufmerksamkeit auf den Dementor gerichtet ist, findet die eigentliche Handlung woanders statt.“

Dumbledore starrte Harry einen Moment lang an, dann nickte er langsam.

„Ja…“, sagte der Schulleiter. „Und ich glaube, ich weiß, wovon es eine Ablenkung sein könnte, wenn Professor Quirrell es böse meint… Danke, Harry.“

Der Schulleiter starrte Harry immer noch an, ein seltsamer Blick in diesen uralten Augen.

„Was?“, sagte Harry mit einem Anflug von Verärgerung, da die Kälte noch immer in seinem Blut verweilte.

„Ich habe noch eine Frage an den jungen Mann“, sagte der Schulleiter. „Es ist etwas, das ich mich schon lange frage, aber nicht begreifen kann. Warum?“

Es lag ein Hauch von Schmerz in seiner Stimme. „Warum sollte sich jemand absichtlich zu einem Monster machen? Warum Böses tun, um des Bösen willen? Warum Voldemort?“

\emph{Zirpen, bzzzt, tick; ding, puff, platsch…}

Harry starrte den Schulleiter erstaunt an.

„Woher soll ich das wissen?“, fragte Harry. „Soll ich den Dunklen Lord auf magische Weise verstehen, weil ich der Held bin, oder was?“

„\textbf{Ja}!“, sagte Dumbledore. „Mein eigener großer Feind war Grindelwald, und ihn habe ich sehr gut verstanden. Grindelwald war mein dunkler Spiegel, der Mann, der ich so leicht hätte sein können, wenn ich der Versuchung nachgegeben hätte, zu glauben, ich sei ein guter Mensch und deshalb immer im Recht. \emph{Für das Allgemeinwohl}, das war sein Motto; und er glaubte es wirklich selbst, auch wenn er sich wie ein verwundetes Tier an ganz Europa vergriff. Und ihn habe ich am Ende besiegt. Aber dann kam nach ihm Voldemort, um alles zu zerstören, was ich in Britannien beschützt hatte.“ Der Schmerz lag nun deutlich in Dumbledores Stimme, entblößte sich in seinem Gesicht. "Er beging Taten, die bei weitem schlimmer waren als Grindelwalds schlimmste, Horror um des Horrors willen. Ich habe alles geopfert, nur um ihn zurückzuhalten, und ich verstehe immer noch nicht, \textbf{warum}! \textbf{Warum}, Harry? Warum hat er das getan? Er war nie mein auserkorener Feind, sondern deiner, also wenn du irgendeine Vermutung hast, Harry, dann sag es mir bitte!

\textbf{Warum}?"

Harry starrte auf seine Hände hinunter. Die Wahrheit war, dass Harry noch nichts über den Dunklen Lord gelesen hatte, und im Moment hatte er nicht den geringsten Anhaltspunkt. Und irgendwie schien das keine Antwort zu sein, die der Schulleiter hören wollte.

„Zu viele dunkle Rituale“, vielleicht? Am Anfang dachte er, er würde nur eines machen, aber das opferte einen Teil seiner guten Seite, und das machte ihn weniger abgeneigt, andere dunkle Rituale auszuführen, also machte er mehr und mehr Rituale in einem positiven Rückkopplungskreislauf, bis er schließlich zu einem ungeheuer mächtigen Monster wurde -"

„Nein!“ Jetzt war die Stimme des Schulleiters gequält. „Das kann ich nicht glauben, Harry! Da muss doch noch mehr dahinterstecken als nur das!“

\emph{Warum sollte es mehr geben}? dachte Harry, aber er sagte es nicht, denn es war klar, dass der Schulleiter dachte, das Universum sei eine Geschichte und habe eine Handlung, und dass große Tragödien nur aus ebenso großen, bedeutsamen Gründen geschehen durften.

„Es tut mir leid, Schulleiter. Der Dunkle Lord scheint mir nicht viel von einem dunklen Spiegel zu halten, überhaupt nicht. Es gibt nichts, was ich auch nur im Geringsten verlockend daran fände, die Häute von Yermy Wibbles Familie an eine Wand zu nageln.“

„Hast du keine Weisheit zu teilen?“, fragte Dumbledore. In der Stimme des alten Zauberers lag ein Flehen, fast ein Betteln.

\emph{Das Böse passiert,} dachte Harry, \emph{es bedeutet nichts und lehrt uns nichts, außer, nicht böse zu sein.} \emph{Der Dunkle Lord war wahrscheinlich nur ein selbstsüchtiger Bastard, dem es egal war, wen er verletzte, oder ein Idiot, der dumme, vermeidbare Fehler machte, die sich ausweiteten. Es gibt kein Schicksal hinter den Übeln dieser Welt; wenn man Hitler in die Architekturschule gelassen hätte, wie er es wollte, wäre die ganze Geschichte Europas anders verlaufen; wenn wir in der Art von Universum leben würden, in dem schreckliche Dinge nur aus guten Gründen passieren dürfen, würden sie gar nicht erst passieren.}

Und nichts davon war offensichtlich das, was der Schulleiter hören wollte. Der alte Zauberer schaute Harry immer noch über ein fummeliges Ding wie eine gefrorene Rauchwolke hinweg an, eine schmerzhafte Verzweiflung in diesen alten, wartenden Augen.

Nun, weise zu klingen war nicht schwer. Es war sogar viel einfacher als intelligent zu sein, da man nichts Überraschendes sagen oder mit neuen Erkenntnissen aufwarten musste. Man ließ einfach die Software seines Gehirns das Klischee vervollständigen, indem man das benutzte, was man zuvor an tiefer Weisheit gespeichert hatte.

„Schulleiter“, sagte Harry feierlich, „ich würde mich lieber nicht über meine Feinde definieren.“

Irgendwie herrschte selbst inmitten all des Surren und Tickens eine Art Stille.

\emph{Das war ein bisschen weiser rübergekommen, als Harry beabsichtigt hatte.}

„Du bist vielleicht sehr weise, Harry…“, sagte der Schulleiter langsam. „Ich wünschte…ich hätte mich von meinen Freunden definieren lassen können.“ Der Schmerz in seiner Stimme war noch tiefer geworden.

Harrys Verstand suchte hastig nach etwas anderem, was der Weise sagen könnte, um die unbeabsichtigte Wucht des Schlages abzumildern—

„Oder vielleicht“, sagte Harry leiser, „ist es der Feind, der den Gryffindor macht, so wie es der Freund ist, der den Hufflepuff macht, und der Ehrgeiz, der den Slytherin macht. Ich weiß, dass es immer, in jeder Generation, das Rätsel ist, das den Gelehrten macht.“

„Es ist ein furchtbares Schicksal, zu dem du mein Haus verdammst, Harry“, sagte der Schulleiter. Der Schmerz lag noch immer in seiner Stimme. „Denn jetzt, wo du es anmerkst, glaube ich, dass ich sehr wohl von meinen Feinden geschaffen wurde.“

Harry starrte auf seine eigenen Hände, wo sie in seinem Schoß lagen.

\emph{Vielleicht sollte er einfach die Klappe halten, solange er in Führung lag}.

„Aber du hast meine Frage beantwortet“, sagte Dumbledore leiser, als wäre er bei sich selbst. „Ich hätte erkennen müssen, dass das der Schlüssel eines Slytherins sein würde. Alles um seines Ehrgeizes willen; und das weiß ich, wenn auch nicht warum…“

Eine Zeit lang starrte Dumbledore ins Nichts; dann richtete er sich auf, und seine Augen schienen sich wieder auf Harry zu richten.

„Und du, Harry“, sagte der Schulleiter, „du nennst dich einen Wissenschaftler?“.

In seiner Stimme schwang Überraschung und leichte Missbilligung mit.

„Sie mögen die Wissenschaft nicht?“, sagte Harry ein wenig müde. Er hatte gehofft, dass Dumbledore Muggelsachen mehr zugetan wäre.

„Ich nehme an, sie ist nützlich für diejenigen, die keinen Zauberstab haben“, sagte Dumbledore und runzelte die Stirn. „Aber es scheint eine seltsame Sache zu sein, über die man sich definiert. Ist Wissenschaft so wichtig wie Liebe? Wie Freundlichkeit? Wie Freundschaft? Ist es die Wissenschaft, die dich für Minerva McGonagall schwärmen lässt? Ist es die Wissenschaft, die dich für Hermine Granger empfinden lässt? Wird es die Wissenschaft sein, an die du dich wendest, wenn du versuchst, Wärme in Draco Malfoys Herz zu entfachen?“

\emph{Das Traurige daran ist, dass du wahrscheinlich denkst, du hättest gerade eine Art unglaublich weises Totschlagargument vorgebracht. Nun, wie formuliert man die Erwiderung so, dass sie auch unglaublich weise klingt…}

„Du bist kein Ravenclaw“, sagte Harry mit ruhiger Würde, „und so ist es dir vielleicht nicht in den Sinn gekommen, dass die Wahrheit zu respektieren und sie alle Tage deines Lebens zu suchen, auch ein Akt der Gnade sein könnte.“

Die Augenbrauen des Schulleiters hoben sich. Und dann seufzte er.

„Wie bist du so weise geworden, so jung …?“

Der alte Zauberer klang traurig, als er das sagte.

„Vielleicht wird es sich als wertvoll für dich erweisen.“

\emph{Nur um alte Zauberer zu beeindrucken, die von sich selbst übermäßig beeindruckt sind,} dachte Harry. Er war tatsächlich ein wenig enttäuscht von Dumbledores Leichtgläubigkeit; es war nicht so, dass Harry gelogen hätte, aber Dumbledore schien viel zu beeindruckt von Harrys Fähigkeit, Dinge so zu formulieren, dass sie tiefgründig klangen, anstatt sie in einfaches Englisch zu fassen, wie Richard Feynman es mit seiner Weisheit getan hatte…

„Liebe ist wichtiger als Weisheit“, sagte Harry, nur um die Grenzen von Dumbledores Toleranz für blendend offensichtliche Klischees zu testen, die durch schiere Musteranpassung ohne irgendeine Art von detaillierter Analyse vervollständigt wurden.

Der Schulleiter nickte ernsthaft und sagte: „In der Tat.“

Harry erhob sich aus dem Stuhl und streckte die Arme aus.

\emph{Nun, dann gehe ich wohl besser los und liebe etwas, das mir helfen wird, den Dunklen Lord zu besiegen. Und wenn du mich das nächste Mal um Rat fragst, umarme ich dich einfach}—

„Heute hast du mir sehr geholfen, Harry“, sagte der Schulleiter.

„Und so gibt es eine letzte Sache, die ich den jungen Mann fragen würde.“

\emph{Na toll.}

„Sag mir, Harry“, sagte der Schulleiter (und jetzt klang seine Stimme einfach nur verwundert, obwohl immer noch ein Hauch von Schmerz in seinen Augen lag), „warum fürchten die dunklen Zauberer den Tod so sehr?“

„Äh“, sagte Harry, „tut mir leid, da muss ich den Dunklen Zauberern Recht geben.“

\emph{Zisch, zisch, läutet; glorp, pop, bubble—}

„Was?“, sagte Dumbledore.

„Der Tod ist schlecht“, sagte Harry und verwarf die Weisheit zugunsten einer klaren Kommunikation.

„Sehr schlecht. Äußerst schlecht. Angst vor dem Tod zu haben ist so, als hätte man Angst vor einem großen Monster mit giftigen Reißzähnen. Das macht sehr viel Sinn und deutet nicht darauf hin, dass Sie ein psychologisches Problem haben.“

Der Schulleiter starrte ihn an, als hätte er sich gerade in eine Katze verwandelt.

„Okay“, sagte Harry, „lassen Sie es mich so ausdrücken. Willst du sterben? Denn wenn ja, gibt es dieses Muggel-Ding, das sich Selbstmordpräventions-Hotline nennt—“

„Wenn es an der Zeit ist“, sagte der alte Zauberer leise. „Nicht vorher. Ich würde weder versuchen, den Tag zu beschleunigen, noch ihn zu verweigern, wenn er kommt.“

Harry runzelte ernsthaft die Stirn.

„Das hört sich nicht so an, als hätten Sie einen sehr starken Lebenswillen, Schulleiter!“

„Harry…“ Die Stimme des alten Zauberers klang allmählich etwas hilflos, und er war zu einer Stelle geschritten, an der sein silberner Bart unbemerkt in ein gläsernes Goldfischglas gewandert war und langsam eine grünliche Färbung annahm, die an den Haaren hochkroch.

„Ich glaube, ich habe mich vielleicht nicht klar ausgedrückt. Dunkle Zauberer sind nicht lebenshungrig. Sie fürchten den Tod. Sie strecken sich nicht nach dem Licht der Sonne aus, sondern fliehen vor der einbrechenden Nacht in unendlich dunkle Höhlen, die sie selbst geschaffen haben, ohne Mond und Sterne. Sie begehren nicht das Leben, sondern die Unsterblichkeit; und sie sind so getrieben, danach zu greifen, dass sie ihre eigenen Seelen opfern würden! Willst du ewig leben, Harry?“

„\textbf{Ja, und du auch}“, sagte Harry. „Ich will noch einen Tag leben. Auch morgen will ich noch einen Tag leben. Deshalb will ich ewig leben, Beweis durch Induktion über die positiven ganzen Zahlen. Wenn Sie nicht sterben wollen, bedeutet das, dass Sie ewig leben wollen. Wenn Sie nicht ewig leben wollen, bedeutet das, dass Sie sterben wollen. Man muss das eine oder das andere tun… Ich komme hier nicht durch, oder?“

Die beiden Kulturen starrten einander über eine riesige Kluft der Inkommensurabilität hinweg an.

„Ich habe hundertundzehn Jahre gelebt“, sagte der alte Zauberer leise (er nahm seinen Bart aus der Schale und schüttelte ihn, um die Farbe herauszuschütteln). „Ich habe viele Dinge gesehen und getan, zu viele, von denen ich wünschte, ich hätte sie nie gesehen oder getan. Und doch bereue ich es nicht, am Leben zu sein, denn meinen Schülern beim Wachsen zuzusehen, ist eine Freude, die mich nicht müde macht. Aber ich würde mir nicht wünschen, so lange zu leben, dass es das tut! Was würdest du mit der Ewigkeit anfangen, Harry?“

Harry nahm einen tiefen Atemzug. „Alle interessanten Menschen auf der Welt treffen, alle guten Bücher lesen und dann etwas noch Besseres schreiben, die zehnte Geburtstagsparty meines ersten Enkels auf dem Mond feiern, die hundertste Geburtstagsparty meines ersten Ur-Ur-Ur-Ur-Enkels um die Ringe des Saturns herum feiern, die tiefsten und letzten Regeln der Natur lernen, die Natur des Bewusstseins verstehen, herausfinden, warum überhaupt etwas existiert, andere Sterne besuchen, andere Sterne zu besuchen, Aliens zu entdecken, Aliens zu erschaffen, uns mit allen zu einer Party auf der anderen Seite der Milchstraße zu treffen, wenn wir alles erforscht haben, uns mit allen anderen zu treffen, die auf der alten Erde geboren wurden, um zu sehen, wie die Sonne endlich erlischt, und ich habe mir früher Sorgen gemacht, einen Weg zu finden, diesem Universum zu entkommen, bevor ihm die Entropie ausgeht, aber jetzt bin ich viel hoffnungsvoller, nachdem ich entdeckt habe, dass die sogenannten Gesetze der Physik nur optionale Richtlinien sind.“

„Ich habe nicht viel davon verstanden“, sagte Dumbledore. „Aber ich muss fragen, ob das Dinge sind, die du dir wirklich so verzweifelt wünschst, oder ob du sie dir nur einbildest, um dir vorzustellen, nicht müde zu sein, während du vor dem Tod davonläufst.“

„Das Leben ist keine endliche Liste von Dingen, die man abhakt, bevor man sterben darf“, sagte Harry fest. „Es ist das Leben, du lebst es einfach weiter. Wenn ich diese Dinge nicht tue, dann deshalb, weil ich etwas Besseres gefunden habe.“

Dumbledore seufzte. Seine Finger trommelten auf eine Uhr; als sie sie berührten, veränderten sich die Ziffern in eine unentzifferbare Schrift, und die Zeiger erschienen kurz in verschiedenen Positionen.

„Für den unwahrscheinlichen Fall, dass es mir erlaubt ist, bis hundertfünfzig zu bleiben“, sagte der alte Zauberer, „glaube ich nicht, dass es mir etwas ausmachen würde. Aber zweihundert Jahre wären völlig zu viel des Guten.“

„Ja, nun“, sagte Harry, und seine Stimme wurde ein wenig trocken, als er an seine Mum und Dad und ihre zugewiesene Lebensspanne dachte, wenn Harry nichts dagegen unternahm, „ich vermute, Schulleiter, dass, wenn Sie aus einer Kultur kämen, in der die Menschen daran gewöhnt waren, vierhundert Jahre zu leben, das Sterben mit zweihundert Jahren genauso tragisch verfrüht erscheinen würde wie das Sterben mit, sagen wir,achtzig.“

Harrys Stimme wurde bei diesem letzten Wort hart.

„Vielleicht“, sagte der alte Zauberer friedlich. „Ich würde mir nicht wünschen, vor meinen Freunden zu sterben, noch weiterzuleben, wenn sie alle fort sind. Am schwersten ist es, wenn die, die man am meisten geliebt hat, vor einem gegangen sind, und doch leben noch andere, um derentwillen man bleiben muss…“

Dumbledores Augen waren auf Harry gerichtet und wurden immer trauriger.

„Trauere nicht zu sehr um mich, Harry, wenn meine Zeit gekommen ist; ich werde bei denen sein, die ich lange vermisst habe, bei unserem nächsten großen Abenteuer.“

„Oh!“, sagte Harry in plötzlicher Erkenntnis. „Du glaubst an ein Leben nach dem Tod. Ich hatte den Eindruck, dass Zauberer keine Religion haben?“

\emph{Piep. Piep. Plopp.}

„Wie kannst du nicht daran glauben?!“, sagte der Schulleiter und sah völlig verblüfft aus. „Harry, du bist ein Zauberer! Du hast Geister gesehen!?“

„Geister“, sagte Harry, seine Stimme war flach. „Du meinst diese Dinge wie Porträts, gespeicherte Erinnerungen und Verhaltensweisen ohne Bewusstsein oder Leben, die durch den Ausbruch von Magie, der den gewaltsamen Tod eines Zauberers begleitet, versehentlich in die umgebende Materie eingeprägt werden—“

„Ich habe diese Theorie gehört“, sagte der Schulleiter, seine Stimme wurde scharf, „wiederholt von Zauberern, die Zynismus mit Weisheit verwechseln, die denken, dass sie sich selbst erhöhen, wenn sie auf andere herabschauen. Das ist eine der dümmsten Ideen, die ich in den letzten hundertzehn Jahren gehört habe! Ja, Geister lernen und wachsen nicht, denn hier gehören sie nicht hin! Seelen sind dazu bestimmt, weiterzuziehen, hier gibt es kein Leben mehr für sie! Und wenn nicht Geister, was ist dann mit dem Schleier? Was ist mit dem Stein der Auferstehung?“

„In Ordnung“, sagte Harry und versuchte, seine Stimme ruhig zu halten, „ich werde mir Ihre Beweise anhören, denn das ist es, was ein Wissenschaftler tut. Aber zuerst, Schulleiter, lassen Sie mich Ihnen eine kleine Geschichte erzählen.“ Harrys Stimme zitterte. „Wissen Sie, als ich hier ankam, als ich in King's Cross aus dem Zug stieg, ich meine nicht gestern, sondern damals im September, als ich damals aus dem Zug stieg, Schulleiter, hatte ich noch nie einen Geist gesehen. Ich hatte nicht mit Geistern gerechnet. Als ich sie also sah, Herr Direktor, tat ich etwas sehr Dummes. Ich zog voreilige Schlüsse. Ich, ich dachte, es gäbe ein Leben nach dem Tod, ich dachte, niemand wäre jemals wirklich gestorben, ich dachte, dass es allen, die die menschliche Spezies jemals verloren hatte, doch gut ging, ich dachte, dass Zauberer mit den Verstorbenen sprechen könnten, dass man nur den richtigen Zauberspruch bräuchte, um sie herbeizurufen, dass Zauberer das tun könnten, ich dachte, ich könnte meine Eltern treffen, die für mich gestorben sind, und ihnen sagen, dass ich von ihrem Opfer gehört habe und dass ich angefangen habe, sie meine Mutter und meinen Vater zu nennen—“

„Harry“, flüsterte Dumbledore. Wasser glitzerte in den Augen des alten Zauberers. Er trat einen Schritt näher durch das Büro—

„Und dann“, spuckte Harry, die Wut kam voll in seine Stimme, die kalte Wut auf das Universum, weil es so war, und auf sich selbst, weil er so dumm war,

„fragte ich Hermine und sie sagte, dass es nur Nachbilder waren, eingebrannt in den Stein des Schlosses durch den Tod eines Zauberers, wie die Silhouetten, die an den Wänden von Hiroshima zurückblieben. Und ich hätte es wissen müssen! Ich hätte es wissen müssen, ohne überhaupt zu fragen! Ich hätte es nicht einmal für 30 Sekunden glauben dürfen! Denn wenn die Menschen eine Seele hätten, gäbe es so etwas wie Hirnschäden nicht, wenn die Seele weiter sprechen könnte, nachdem das ganze Gehirn weg ist, wie könnte dann ein Schaden an der linken Gehirnhälfte einem die Fähigkeit zu sprechen nehmen? Und Professor McGonagall, als sie mir vom Tod meiner Eltern erzählte, tat sie nicht so, als wären sie einfach auf eine lange Reise in ein anderes Land gegangen, als wären sie nach Australien ausgewandert, als es noch Segelschiffe gab, so würden sich die Leute verhalten, wenn sie tatsächlich wüssten, dass der Tod einfach nur woanders hingeht, wenn sie harte Beweise für ein Leben nach dem Tod hätten, anstatt sich etwas auszudenken, um sich selbst zu trösten, es würde alles ändern, es würde keine Rolle spielen, dass jeder jemanden im Krieg verloren hat, es wäre ein bisschen traurig, aber nicht schrecklich! Und ich hatte schon gesehen, dass die Leute in der Zaubererwelt sich nicht so verhielten! Also hätte ich es besser wissen müssen! Und da wusste ich, dass meine Eltern wirklich tot und für immer und ewig weg waren, dass nichts mehr von ihnen übrig war, dass ich nie die Chance bekommen würde, sie kennenzulernen und, und, und die anderen Kinder dachten, ich würde weinen, weil ich Angst vor Geistern hatte—“

Das Gesicht des alten Zauberers war entsetzt, er öffnete den Mund, um zu sprechen—

„Also sagen Sie es mir, Herr Direktor! Erzählen Sie mir von den Beweisen! Aber wagen Sie es nicht, auch nur ein winziges bisschen davon zu übertreiben, denn wenn Sie mir noch einmal falsche Hoffnungen machen und ich später herausfinde, dass Sie gelogen oder die Dinge nur ein bisschen gedehnt haben, werde ich Ihnen das nie verzeihen! \textbf{Was ist der Schleier?!}“

Harry griff nach oben und wischte sich über die Wangen, während die Glasdinger des Büros von seinem letzten Schrei zu vibrieren aufhörten.

„Der Schleier“, sagte der alte Zauberer mit einem leichten Zittern in der Stimme, „ist ein großer steinerner Torbogen, der in der Abteilung für Mysterien aufbewahrt wird; ein Tor zum Land der Toten.“

„Und woher weiß man das?“, fragte Harry. „Sagen Sie mir nicht, was Sie glauben, sondern was Sie gesehen haben!“

Die physische Manifestation der Barriere zwischen den Welten war ein großer steinerner Torbogen, alt und hoch und spitz zulaufend, mit einem zerfledderten schwarzen Schleier wie die Oberfläche eines Wasserbeckens, der sich zwischen den Steinen spannte; kräuselnd, immer, von der ständigen und einseitigen Passage der Seelen. Wenn man am Schleier stand, konnte man die Stimmen der Toten rufen hören, immer im Flüsterton, kaum zu verstehen, immer lauter und zahlreicher werdend, wenn man blieb und zu hören versuchte, wie sie versuchten, zu kommunizieren; und wenn man zu lange zuhörte, ging man ihnen entgegen, und in dem Moment, in dem man den Schleier berührte, wurde man hindurchgesaugt und hörte nie wieder etwas von ihnen.

„Das hört sich nicht einmal wie ein interessanter Betrug an“, sagte Harry, seine Stimme wurde ruhiger, jetzt, da es nichts gab, was ihm Hoffnung machte oder ihn wütend machte, weil seine Hoffnungen enttäuscht worden waren. „Jemand hat einen steinernen Torbogen gebaut, eine kleine schwarze Wellenfläche dazwischen gelegt, die alles verschwinden ließ, was sie berührte, und sie verzaubert, damit sie den Leuten zuflüstert und sie hypnotisiert.“

„Harry…“, sagte der Schulleiter und begann, ziemlich besorgt auszusehen.

„Ich kann dir die Wahrheit sagen, aber wenn du dich weigerst, sie zu hören…“

\emph{Auch nicht interessant}.

„Was ist der Stein der Auferstehung?“

„Ich würde es dir nicht sagen“, sagte der Schulleiter langsam, „außer, dass ich fürchte, was dieser Unglaube dir antun könnte…also hör zu, Harry, bitte hör zu…“

Der Stein der Auferstehung war einer der drei legendären Heiligtümer des Todes, verwandt mit Harrys Umhang. Der Stein der Auferstehung konnte Seelen von den Toten zurückrufen - sie zurück in die Welt der Lebenden bringen, wenn auch nicht so, wie sie waren. Cadmus Peverell benutzte den Stein, um seine verlorene Geliebte von den Toten zurückzurufen, aber ihr Herz blieb bei den Toten, und nicht in der Welt der Lebenden. Und mit der Zeit trieb es ihn in den Wahnsinn, und er tötete sich selbst, um noch einmal wirklich mit ihr zusammen zu sein…

In aller Höflichkeit hob Harry seine Hand.

„Ja?“, sagte der Schulleiter zögernd.

„Der offensichtliche Test, um herauszufinden, ob der Stein der Auferstehung wirklich die Toten zurückruft oder nur ein Bild aus dem Geist des Anwenders projiziert, ist, eine Frage zu stellen, deren Antwort Sie nicht kennen, der Tote aber schon, und die in dieser Welt definitiv überprüft werden kann. Zum Beispiel: wenn man als Person…—“

Dann hielt Harry inne, denn diesmal hatte er es geschafft, seiner Zunge einen Schritt voraus zu denken, schnell genug, um nicht den ersten Namen und Test zu sagen, der ihm in den Sinn gekommen war.

„… Ihre tote Frau beschwört, und fragen Sie sie, wo sie ihren verlorenen Ohrring gelassen hat, oder so etwas in der Art“, beendete Harry. „Hat jemand so einen Test gemacht?“

„Der Stein der Auferstehung ist seit Jahrhunderten verschollen, Harry“, sagte der Schulleiter leise.

Harry zuckte mit den Schultern.

„Nun, ich bin Wissenschaftler und immer bereit, mich überzeugen zu lassen. Wenn Sie wirklich glauben, dass der Stein der Auferstehung die Toten zurückruft - dann müssen Sie doch auch glauben, dass so ein Test erfolgreich sein wird, oder? Weißt du etwas darüber, wo man den Stein der Auferstehung finden kann? Ich habe schon ein Artefakt unter höchst mysteriösen Umständen bekommen, und, na ja, wir wissen beide, wie der Rhythmus der Welt bei so etwas funktioniert.“

Dumbledore starrte Harry an. Harry starrte den Schulleiter gleichmütig zurück an. Der alte Zauberer fuhr sich mit der Hand über die Stirn und murmelte:

„Das ist Wahnsinn.“

(Irgendwie schaffte Harry es, sich das Lachen zu verkneifen.)

Und Dumbledore wies Harry an, den Unsichtbarkeitsumhang aus seiner Tasche zu ziehen; auf Anweisung des Schulleiters starrte Harry auf die Innenseite und die Rückseite der Kapuze, bis er es sah, schwach gegen das silbrige Gewebe gezeichnet, in verblasstem Scharlachrot wie getrocknetes Blut, das Symbol der Heiligtümer des Todes: ein Dreieck, mit einem Kreis im Inneren und einer Linie, die beides trennt.

„Danke“, sagte Harry höflich. „Ich werde auf jeden Fall nach einem so markierten Stein Ausschau halten. Haben Sie sonst noch irgendwelche Beweise?“

Dumbledore schien einen Kampf mit sich selbst zu führen.

„Harry“, sagte der alte Zauberer, und seine Stimme erhob sich, „das ist ein gefährlicher Weg, den du beschreitest, ich bin mir nicht sicher, ob ich das Richtige tue, wenn ich das sage, aber ich muss dich von diesem Weg wegreißen! Harry, wie hätte Voldemort den Tod seines Körpers überleben können, wenn er keine Seele gehabt hätte?“

Und das war der Moment, in dem Harry klar wurde, dass es genau eine Person gab, die Professor McGonagall ursprünglich gesagt hatte, dass der Dunkle Lord überhaupt noch lebte; und das war der verrückte Schulleiter ihres Irrenhauses von einer Schule, der glaubte, die Welt funktioniere wie Klischees.

„Gute Frage“, sagte Harry, nachdem er innerlich darüber nachgedacht hatte, wie er vorgehen sollte. „Vielleicht hat er einen Weg gefunden, die Kraft des Steins der Auferstehung zu duplizieren, nur hat er ihn vorher mit einer vollständigen Kopie seines Gehirnzustands geladen. Oder so etwas in der Art.“

Harry war sich plötzlich alles andere als sicher, dass er versuchte, eine Erklärung für etwas zu finden, das tatsächlich passiert war.

„Kannst du mir eigentlich alles erzählen, was du darüber weißt, wie der Dunkle Lord überlebt hat und was nötig wäre, um ihn zu töten?“

\emph{Wenn es ihn überhaupt noch gibt, als mehr als nur die Schlagzeilen des Klitterers.}

„Du machst mir nichts vor, Harry“, sagte der alte Zauberer; sein Gesicht sah jetzt uralt aus, und von mehr als nur Jahren gezeichnet. „Ich weiß, warum du diese Frage wirklich stellst. Nein, ich kann deine Gedanken nicht lesen, das muss ich auch nicht, dein Zögern verrät dich! Du suchst das Geheimnis der Unsterblichkeit des Dunklen Lords, um es für dich zu nutzen!“

„\textbf{Falsch! Ich will das Geheimnis der Unsterblichkeit des Dunklen Lords, um es für alle zu} \textbf{nutzen!}“

\emph{Tick, crackle, fzzzt…}

Albus Percival Wulfric Brian Dumbledore stand einfach nur da und starrte Harry mit stumm aufgerissenem Mund an.

(Harry vergab sich selbst einen Punkt für den Montag, da er es geschafft hatte, jemanden komplett umzuhauen, bevor der Tag zu Ende war.)

„Und falls es nicht klar war“, sagte Harry, „mit allen meine ich auch alle Muggel, nicht nur alle Zauberer.“

„Nein“, sagte der alte Zauberer und schüttelte den Kopf. Seine Stimme erhob sich. „Nein, nein, nein! Das ist Irrsinn!“

„\textbf{Bwa ha ha!}“, sagte Harry.

Das Gesicht des alten Zauberers war angespannt vor Wut und Sorge.

„Voldemort hat das Buch gestohlen, aus dem er sein Geheimnis entnommen hat; es war nicht mehr da, als ich danach suchte. Aber so viel weiß ich, und so viel will ich dir sagen: Seine Unsterblichkeit wurde durch ein Ritual geboren, das schrecklich und dunkel war, schwärzer als Pechschwarz! Und es war Myrtle, die arme, süße Myrtle, die dafür starb; seine Unsterblichkeit erforderte Opfer, sie erforderte Mord—“

„Nun, natürlich werde ich nicht eine Methode der Unsterblichkeit verwenden, die das Töten von Menschen erfordert! Das würde den ganzen Sinn zunichte machen!“

Es gab eine erschrockene Pause. Langsam entspannte sich das Gesicht des alten Zauberers aus seiner Wut, obwohl die Sorge immer noch da war.

„Du würdest kein Ritual anwenden, das Menschenopfer erfordert.“

„Ich weiß nicht, wofür Sie mich halten, Schulleiter“, sagte Harry kalt, sein eigener Zorn stieg auf, „aber vergessen wir nicht, dass ich derjenige bin, der will, dass die Menschen leben! Derjenige, der alle retten will! Sie sind derjenige, der denkt, dass der Tod großartig ist und jeder sterben sollte!“

„Ich bin ratlos, Harry“, sagte der alte Zauberer. Seine Füße stapften wieder durch sein seltsames Büro. „Ich weiß nicht, was ich sagen soll.“

Er hob eine Kristallkugel auf, die eine Hand in Flammen zu halten schien, und blickte mit trauriger Miene hinein.

„Nur, dass ich von dir sehr missverstanden werde… Ich will nicht, dass alle sterben, Harry!“

„Du willst nur nicht, dass jemand unsterblich ist“, sagte Harry mit beträchtlicher Ironie.

Es schien, dass elementare logische Tautologien wie

Alle x:Stirbt(x) = Nicht Existieren x:Nicht Sterben(x) = Existieren

jenseits der logischen Fähigkeiten des mächtigsten Zauberers der Welt war.

Der alte Zauberer nickte. „Ich habe weniger Angst als früher, aber ich mache mir immer noch große Sorgen um dich, Harry“, sagte er leise. Seine Hand, von der Zeit ein wenig schrumpelig geworden, aber immer noch stark, legte die Kristallkugel fest in ihren Ständer zurück.

„Denn die Angst vor dem Tod ist eine bittere Sache, eine Krankheit der Seele, durch die die Menschen verdreht und verzerrt werden. Voldemort ist nicht der einzige Dunkle Zauberer, der diesen düsteren Weg eingeschlagen hat, obwohl ich fürchte, dass er ihn weiter gegangen ist als alle anderen vor ihm.“

„Und du glaubst, du hast keine Angst vor dem Tod?“ sagte Harry und versuchte nicht einmal, die Ungläubigkeit in seiner Stimme zu verbergen.

Das Gesicht des alten Zauberers war friedlich.

„Ich bin nicht perfekt, Harry, aber ich glaube, ich habe meinen Tod als Teil von mir akzeptiert.“

„Aha“, sagte Harry. "Sehen Sie, es gibt da diese kleine Sache, die man kognitive Dissonanz nennt. Wenn man den Leuten einmal im Monat einen Knüppel auf den Kopf hauen würde und niemand etwas dagegen tun könnte, gäbe es bald alle möglichen Philosophen, die vorgeben, weise zu sein, wie Sie es ausdrücken, und die alle möglichen erstaunlichen Vorteile darin sehen, einmal im Monat einen Knüppel auf den Kopf zu bekommen.

Zum Beispiel macht es dich härter, oder es macht dich glücklicher an den Tagen, an denen du nicht mit einem Knüppel geschlagen wirst.

Aber wenn du zu jemandem gehst, der nicht geschlagen wird, und ihn fragst, ob er anfangen will, im Austausch für diese erstaunlichen Vorteile, würde er \emph{nein} sagen. Und wenn Sie nicht sterben müssten, wenn Sie von irgendwoher kämen, wo noch nie jemand etwas vom Tod gehört hat, und ich Ihnen vorschlagen würde, dass es eine erstaunliche, wunderbare, großartige Idee wäre, wenn die Menschen faltig und alt würden und schließlich aufhörten zu existieren, warum, würden Sie mich sofort in eine Irrenanstalt bringen lassen! Warum sollte jemand auf den dummen Gedanken kommen, dass der Tod eine gute Sache ist? Weil man Angst davor hat, weil man nicht wirklich sterben will, und dieser Gedanke schmerzt so sehr in einem, dass man ihn wegrationalisieren muss, etwas tun, um den Schmerz zu betäuben, damit man nicht daran denken muss -"

„Nein, Harry“, sagte der alte Zauberer. Sein Gesicht war sanft, seine Hand fuhr durch ein beleuchtetes Wasserbecken, das kleine musikalische Schläge machte, als seine Finger es umrührten. „Obwohl ich verstehen kann, dass du so denken musst.“

„Willst du den dunklen Zauberer verstehen?“ sagte Harry, seine Stimme jetzt hart und grimmig. „Dann schau in den Teil von dir, der nicht vor dem Tod flieht, sondern vor der Angst vor dem Tod, der diese Angst so unerträglich findet, dass er den Tod als Freund umarmt und sich bei ihm einschleimt, versucht, eins zu werden mit der Nacht, damit er sich für den Herrn des Abgrunds halten kann. Du hast das schrecklichste aller Übel genommen und es gut genannt! Mit nur einer kleinen Drehung würde derselbe Teil von dir Unschuldige ermorden und es Freundschaft nennen. Wenn du den Tod besser als das Leben nennen kannst, dann kannst du deinen moralischen Kompass so verdrehen, dass er überall hinzeigt—“

„Ich glaube“, sagte Dumbledore und schüttelte zum Klang kleiner bimmelnder Glöckchen Wassertropfen von seiner Hand, „dass du dunkle Zauberer sehr gut verstehst, ohne selbst einer zu sein.“

Es wurde in vollkommenem Ernst und ohne Vorwurf gesagt.

„Aber dein Verständnis für mich, fürchte ich, ist sehr mangelhaft.“

Der alte Zauberer lächelte jetzt, und es lag ein leises Lachen in seiner Stimme.

Harry bemühte sich, nicht noch kälter zu werden, als er es ohnehin schon war; von irgendwoher strömte eine lodernde Wut des Grolls auf Dumbledores Herablassung in seinen Geist, und all das Lachen, das weise alte Narren jemals anstelle von Argumenten benutzt hatten.

„Komisch, weißt du, ich dachte, Draco Malfoy würde so unmöglich sein, mit dem man reden kann, und stattdessen war er in seiner kindlichen Unschuld hundertmal stärker als du.“

Ein Blick der Verwunderung ging über das Gesicht des alten Zauberers.

„Was meinst du?“

„Ich meine“, sagte Harry mit beißender Stimme, „dass Draco seine eigenen Überzeugungen tatsächlich ernst genommen und meine Worte verarbeitet hat, anstatt sie mit einem Lächeln sanfter Überlegenheit aus dem Fenster zu werfen. Du bist so alt und weise, dass du nicht einmal etwas von dem, was ich sage, bemerken kannst! Nicht verstehen, bemerken!?“

„Ich habe dir zugehört, Harry“, sagte Dumbledore und sah nun etwas ernster aus, „aber zuhören heißt nicht immer zustimmen. Abgesehen von den Meinungsverschiedenheiten, was ist es, von dem du glaubst, dass ich es nicht verstehe?“

\emph{Dass du, wenn du wirklich an ein Leben nach dem Tod glauben würdest, zu St. Mungo's gehen und Nevilles Eltern, Alice und Frank Longbottom, töten würdest, damit sie zu ihrem nächsten großen Abenteuer aufbrechen können, anstatt sie hier in ihrem beschädigten Zustand verweilen zu lassen}—

Harry konnte sich gerade noch davon abhalten, es laut auszusprechen.

„In Ordnung“, sagte Harry kalt. „Dann werde ich deine ursprüngliche Frage beantworten. Sie haben gefragt, warum die dunklen Zauberer Angst vor dem Tod haben. Tun Sie so, Schulleiter, als würden Sie wirklich an Seelen glauben. Tun Sie so, als ob jeder jederzeit die Existenz von Seelen nachweisen könnte, tun Sie so, als ob niemand bei Beerdigungen weinen würde, weil sie wissen, dass ihre Lieben noch leben. Können Sie sich nun vorstellen, eine Seele zu zerstören? Sie in Stücke zu reißen, so dass nichts mehr übrig bleibt, um sich auf ihr nächstes großes Abenteuer zu begeben? Können Sie sich vorstellen, was für eine schreckliche Sache das wäre, das schlimmste Verbrechen, das jemals in der Geschichte des Universums begangen wurde, das Sie alles tun würden, um zu verhindern, dass es auch nur einmal passiert? Denn das ist es, was der Tod wirklich ist - die Auslöschung einer Seele!“

Der alte Zauberer starrte ihn an, ein trauriger Blick in den Augen.

„Ich glaube, ich verstehe jetzt“, sagte er leise.

„Oh?“, sagte Harry. „Was verstehen?“

„Voldemort“, sagte der alte Zauberer. „Ich verstehe ihn jetzt endlich. Denn um zu glauben, dass die Welt wirklich so ist, muss man glauben, dass es in ihr keine Gerechtigkeit gibt, dass sie in ihrem Kern aus Finsternis gewoben ist. Ich fragte dich, warum er ein Monster wurde, und du konntest keinen Grund nennen. Und wenn ich ihn fragen könnte, würde seine Antwort wohl lauten: \emph{Warum nicht?}“

Sie standen da und sahen sich in die Augen, der alte Zauberer in seinem Gewand und der Junge mit der Blitznarbe auf der Stirn.

„Sag, Harry“, sagte der alte Zauberer, „willst du ein Monster werden?“

„\textbf{Nein}“, sagte der Junge mit einer eisernen Gewissheit in seiner Stimme.

„Warum nicht?“, fragte der alte Zauberer.

Der Junge stand ganz gerade, das Kinn hoch und stolz erhoben, und sagte: "Es gibt keine Gerechtigkeit in den Gesetzen der Natur, Schulleiter, keinen Begriff für Fairness in den Gleichungen der Bewegung.

\textbf{Das Universum ist weder böse, noch gut, es ist ihm einfach egal. Den Sternen ist es egal, oder der Sonne, oder dem Himmel.}

\textbf{Aber das müssen sie auch nicht! Es ist uns nicht egal!}

\textbf{Es gibt Licht in der Welt, und das sind wir!}"

„Ich frage mich, was aus dir werden wird, Harry“, sagte der alte Zauberer. Seine Stimme war sanft, mit einem seltsamen Erstaunen und Bedauern darin. „Es ist genug, dass ich mir wünsche zu leben, nur um es zu sehen.“

Der Junge verbeugte sich mit schwerer Ironie vor ihm und ging, und die Eichentür schlug mit einem dumpfen Schlag hinter ihm zu.

