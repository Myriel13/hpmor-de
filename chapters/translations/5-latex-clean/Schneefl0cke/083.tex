

\hypertarget{tabubruxfcche-nachwirkungen-2}{% \section{84. Tabubrüche, Nachwirkungen 2}\label{tabubruxfcche-nachwirkungen-2}}

\textbf{\uline{Tabubrüche, Nachwirkungen 2}}

\hfill\break Als Hermine Granger erwachte, fand sie sich in einem weichen, bequemen Bett des Hogwarts-Krankenhauses liegend, mit einem Quadrat aus untergehendem Sonnenlicht, das auf ihre Hüfte fiel, warm durch die dünne Decke. Die Erinnerung sagte ihr, dass über ihr ein Vorhang war, der entweder um ihr Bett herumgezogen oder offen war, und dass der Rest von Madam Pomfreys Reich dahinter lag: die anderen Betten, besetzt oder unbesetzt, und helle Fenster, die in den geschwungenen Stein von Hogwarts eingelassen waren. Als Hermine ihre Augen öffnete, war das erste, was sie sah, das Gesicht von Professor McGonagall, die an der linken Seite ihres Bettes saß. Professor Flitwick war nicht da, aber das war verständlich, er war den ganzen Morgen in der Arrestzelle an ihrer Seite geblieben, sein silberner Rabe stand extra Wache vor dem Dementor und sein strenges kleines Gesicht war immer nach außen zu den Auroren gerichtet. Der Leiter von Ravenclaw hatte sicher viel zu viel Zeit mit ihr verbracht und musste wahrscheinlich wieder zurück in den Unterricht, anstatt eine verurteilte Mordversuchskandidatin zu bewachen.

Sie fühlte sich schrecklich krank und sie glaubte nicht, dass es an irgendwelchen Zaubertränken lag. Hermine hätte wieder angefangen zu weinen, nur ihre Kehle schmerzte, ihre Augen brannten immer noch, und ihr Geist fühlte sich einfach nur müde an. Sie hätte es nicht ertragen können, wieder zu weinen, fand nicht die Kraft für Tränen.

„Wo sind meine Eltern?“ flüsterte Hermine dem Hausoberhaupt der Gryffindor zu. Irgendwie schien es das Schlimmste auf der Welt zu sein, ihnen gegenüberzustehen, schlimmer noch als alles andere; und doch wollte sie sie sehen.

Der sanfte Blick auf Professor McGonagalls Gesicht veränderte sich zu etwas Traurigem. „Es tut mir leid, Miss Granger. Obwohl das nicht immer so war, haben wir in den letzten Jahren festgestellt, dass es klüger ist, den Eltern von Muggelgeborenen nichts über die Gefahren zu sagen, denen ihr Kind ausgesetzt ist. Ich würde Ihnen auch raten, zu schweigen, wenn Sie ohne Ärger mit ihnen in Hogwarts bleiben wollen.“

„Ich werde nicht von der Schule verwiesen?“, flüsterte das Mädchen. „Für das, was ich getan habe?“

„Nein“, sagte Professor McGonagall. „Miss Granger... Sie haben sicher gehört... Ich hoffe, Sie haben Mr. Potter gehört, als er sagte, dass Sie unschuldig sind?“

„Das hat er nur so gesagt“, sagte sie düster. „Um mich frei zu bekommen, meine ich.“

Die ältere Hexe schüttelte entschieden den Kopf. „Nein, Miss Granger. Mr. Potter glaubt, dass Sie einem Gedächtniszauber unterlagen und dass das ganze Duell nie stattgefunden hat. Der Schulleiter vermutet, dass noch dunklere Magie im Spiel gewesen sein könnte - dass Ihre eigene Hand den Zauber gesprochen haben könnte, aber nicht Ihr eigener Wille. Selbst Professor Snape findet die Sache völlig unglaubwürdig, auch wenn er das nicht öffentlich sagen kann. Er hat sich gefragt, ob vielleicht Muggeldrogen bei dir eingesetzt wurden.“

Hermines Augen starrten Sie weiterhin distanziert an; sie wusste, dass ihr gerade etwas Bedeutsames gesagt worden war, aber sie fand nicht die Energie, um irgendeine Veränderung in ihrem Kopf zu bewirken.

„Sie glauben es doch nicht etwa?“, sagte Professor McGonagall. „Miss Granger, Sie können doch nicht von sich aus glauben, dass Sie zu einem Mord imstande sind!“

„Aber ich -“ Ihr exzellentes Gedächtnis spielte es zum tausendsten Mal ab: Draco Malfoy, der ihr höhnisch sagte, dass sie ihn nie besiegen würde, wenn er nicht müde war, und der dann genau das bewies, indem er wie ein Duellant zwischen den umzäunten Trophäen tanzte, während sie sich verzweifelt aufrappelte, und ihr den finalen Schlag mit einem Fluch versetzte, der sie gegen die Wand schleuderte und das Blut das von ihrer Wange tropfte - und dann - dann hatte sie -

„Aber du erinnerst dich daran, es getan zu haben“, sagte die ältere Hexe, die mit freundlichem Verständnis über sie wachte. „Miss Granger, es gibt keinen Grund für ein zwölfjähriges Mädchen, solch furchtbare Erinnerungen zu ertragen. Sagen Sie nur ein Wort und ich schließe sie gerne für Sie weg.“

Es war wie ein Glas warmes Wasser, das ihr ins Gesicht geschüttet wurde.\\ „Was?“

Professor McGonagall zückte ihren Zauberstab, eine Geste, die so geübt und schnell war, dass sie wie ein Fingerzeig wirkte. „Ich kann Ihnen nicht anbieten, die Erinnerungen ganz loszuwerden, Miss Granger“, sagte die Professorin für Verwandlung mit ihrer gewohnten Präzision. „Es könnten wichtige Fakten darin versteckt sein. Aber es gibt eine Form des Erinnerungszaubers, die umkehrbar ist, und ich werde ihn gerne auf Sie anwenden.“

Hermine starrte auf den Zauberstab und spürte zum ersten Mal seit fast zwei Tagen das Aufkeimen von Hoffnung. \emph{Mach, dass es nicht passiert ist} ... das hatte sie sich immer wieder gewünscht, dass die Zeit sich zurückdreht und die schreckliche Entscheidung auslöscht, die nie und nimmer rückgängig gemacht werden konnte. Und wenn das Löschen der Erinnerung das nicht war, so war es doch eine Art Erlösung...

Sie blickte zurück in Professor McGonagalls freundliches Gesicht. „Sie glauben wirklich nicht, dass ich es getan habe?“ sagte Hermine, ihre Stimme zitterte.

„Ich bin mir ganz sicher, dass Sie so etwas niemals aus eigenem Antrieb tun würden.“

Unter ihren Decken klammerten sich Hermines Hände an die Laken.\\ „Harry glaubt nicht, dass ich es getan habe?“

„Mr. Potter ist der Meinung, dass deine Erinnerungen komplett erfunden sind. Ich kann seinen Standpunkt durchaus nachvollziehen.“

Dann ließen Hermines klammernde Finger das Laken los, und sie sackte zurück in das Bett, aus dem sie teilweise aufgestanden war.\\ \emph{Nein. Sie hatte nichts gesagt. Sie war aufgewacht und hatte sich daran erinnert, was letzte Nacht geschehen war, und es war wie - wie} - sie konnte nicht einmal in ihren eigenen Gedanken Worte dafür finden, wie es gewesen war. Aber sie hatte gewusst, dass Draco Malfoy bereits tot war, und sie hatte nichts gesagt, war nicht zu Professor Flitwick gegangen und hatte gestanden. Sie hatte sich einfach angezogen und war zum Frühstück gegangen und hatte versucht, sich normal zu verhalten, damit es niemand erfährt, und sie hatte gewusst, dass es falsch und falsch und furchtbar furchtbar FALSCH war, aber sie war so, so verängstigt gewesen - selbst wenn Harry Potter recht hatte, selbst wenn das Duell mit Draco Malfoy eine Lüge war, sie hatte diese Entscheidung ganz allein getroffen. Sie hatte es nicht verdient, das zu vergessen oder dafür Vergebung zu bekommen. Und wenn sie das Richtige getan hätte, direkt zu Professor Flitwick gegangen wäre, hätte das vielleicht - irgendwie geholfen, vielleicht hätte dann jeder gesehen, dass sie es bereute, und Harry hätte nicht sein ganzes Geld hergeben müssen, um sie zu retten - Hermine schloss die Augen, drückte sie ganz fest zu, sie konnte es nicht ertragen, wieder zu weinen.

„Ich bin ein furchtbarer Mensch“, sagte sie mit schwankender Stimme. „Ich bin schrecklich, ich bin überhaupt nicht heldenhaft -“

Professor McGonagalls Stimme war sehr scharf, als hätte Hermine gerade einen furchtbaren Fehler bei ihren Hausaufgaben gemacht.\\ „Seien Sie nicht dumm, Miss Granger! Schrecklich ist, wer immer Ihnen das angetan hat. Und was das Heldentum angeht - nun, Miss Granger, Sie haben bereits meine Meinung über junge Mädchen gehört, die versuchen, sich in solche Dinge zu verwickeln, bevor sie vierzehn sind, also werde ich Sie nicht noch einmal darüber belehren. Ich möchte nur sagen, dass Sie gerade eine absolut schreckliche Erfahrung gemacht haben, die Sie so gut überstanden haben, wie es jede Hexe Ihres Jahrgangs nur könnte. Heute darfst du so viel weinen, wie du willst. Morgen gehst du wieder in den Unterricht.“

Das war der Moment, in dem Hermine wusste, dass Professor McGonagall ihr nicht helfen konnte. Sie brauchte jemanden, der mit ihr schimpfte, sie konnte nicht freigesprochen werden, wenn sie nicht beschuldigt werden konnte, und Professor McGonagall würde das nie für sie tun, würde nie so viel von einem kleinen Ravenclaw-Mädchen verlangen. Es war etwas, bei dem ihr auch Harry Potter nicht helfen würde. Hermine drehte sich im Krankenbett um und kauerte sich in sich zusammen, weg von Professor McGonagall.

„Bitte“, flüsterte sie. „Ich möchte reden - mit dem Schulleiter -“

...\\ „Hermine.“

Als Hermine Granger ein zweites Mal die Augen öffnete, sah sie das sorgenvolle Gesicht von Albus Dumbledore, der sich über ihr Bett beugte und fast so aussah, als hätte er geweint, obwohl das unmöglich war; und Hermine fühlte einen weiteren stechenden Stich der Schuld, ihn so gestört zu haben.

„Minerva sagte, du wolltest mit mir sprechen“, sagte der alte Zauberer.

„Ich -“ Plötzlich wusste Hermine gar nicht mehr, was sie sagen sollte. Ihre Kehle schnürte sich zu und alles, was sie tun konnte, war zu stammeln: „Ich - ich bin -“ Irgendwie musste ihr Tonfall das andere Wort übermittelt haben, das, das sie gar nicht mehr aussprechen konnte. „Es tut mir Leid....“

„Was sollte Ihnen leid tun?“

Sie musste die Worte aus ihrer Kehle zwingen.\\ „Sie haben Harry gesagt - dass er nicht zahlen soll - also hätte ich nicht - tun sollen, was Professor McGonagall gesagt hat, ich hätte seinen Zauberstab nicht anfassen sollen -“

„Meine Liebe“, sagte Dumbledore, „hättest du dich nicht dem Haus Potter verschrieben, hätte Harry Askaban im Alleingang angegriffen - und womöglich gewonnen. Der Junge mag seine Worte mit Bedacht wählen, aber ich habe noch nie erlebt, dass er lügt; und in dem Jungen, der gelebt hat, steckt eine Macht, die der Dunkle Lord nie gekannt hat. Er hätte in der Tat versucht, Askaban zu durchbrechen, selbst wenn es ihn das Leben gekostet hätte.“\\ Die Stimme des alten Zauberers wurde sanfter und freundlicher.\\ „Nein, Hermine, du hast dir überhaupt nichts vorzuwerfen.“

„Ich hätte ihn dazu bringen können, es nicht zu tun.“

In Dumbledores Augen erschien ein kleines Glitzern, bevor es sich in Müdigkeit verlor. „Wirklich, Miss Granger? Vielleicht solltest du an meiner Stelle Schulleiterin werden, denn ich selbst habe keine solche Macht über starrköpfige Kinder.“

„Harry hat versprochen -“ Ihre Stimme stockte. Es fiel ihr schwer, die schreckliche Wahrheit auszusprechen. „Harry Potter hat mir versprochen - dass er mir niemals helfen würde - wenn ich es ihm verbieten würde.“

Es gab eine Pause.

Die entfernten Geräusche des Krankenflügels, die Professor McGonagall begleitet hatten, waren verstummt, wie Hermine feststellte, als Dumbledore sie geweckt hatte. Von dort, wo sie im Bett lag, konnte sie nur die Decke und den oberen Rand der Fenster einer Wand sehen, aber nichts in ihrem Sichtbereich bewegte sich, und wenn es Geräusche gab, konnte sie sie nicht hören.

„Ah“, sagte Dumbledore. Der alte Zauberer seufzte schwer. „Ich nehme an, es ist möglich, dass der Junge sein Versprechen gehalten hätte.“

„Ich hätte - ich hätte -“

„Aus freien Stücken nach Askaban gehen? Miss Granger, das ist mehr, als ich jemals von jemandem verlangen würde, auf sich zu nehmen.“

„Aber -“ Hermine schluckte. Sie konnte nicht umhin, das Schlupfloch zu bemerken, denn jeder, der durch die Porträttür in den Ravenclaw-Schlafsaal gelangen wollte, lernte schnell, auf genaue Formulierungen zu achten. „Aber es ist nicht mehr, als Sie selbst auf sich nehmen würden.“

„Hermine -“, begann der alte Zauberer.

„Warum?“, sagte Hermines Stimme, die jetzt ohne ihren Verstand weiterzulaufen schien. „Warum konnte ich nicht mutiger sein? Ich wollte vor den Dementor rennen - für Harry - vorher, ich meine, im Januar - also warum - warum - warum konnte ich nicht -“

Warum hatte der Gedanke, nach Askaban geschickt zu werden, sie einfach völlig aus der Bahn geworfen, warum hatte sie alles vergessen, was es heißt, gut zu sein -

„Mein liebes Mädchen“, sagte Dumbledore. Die blauen Augen hinter der Halbmondbrille zeigten völliges Verständnis für ihre Schuld. „Ich hätte es selbst nicht besser gemacht, in meinem ersten Jahr in Hogwarts. So wie du freundlich zu anderen sein würdest, sei auch freundlich zu dir selbst.“

„Also habe ich das Falsche getan.“\\ Irgendwie hatte sie das Bedürfnis, das zu sagen, das gesagt zu bekommen, obwohl sie es schon wusste.

Es gab eine Pause.

„Höre, junger Ravenclaw“, sagte der alte Zauberer, „höre mir gut zu, denn ich werde eine Wahrheit zu dir sprechen. Die meisten Übeltäter sehen sich selbst nicht als böse an, ja, die meisten halten sich für die Helden der Geschichten, die sie erzählen. Ich dachte einst, dass das größte Übel in dieser Welt im Namen des größeren Guten getan wurde. Ich habe mich geirrt. Schrecklich falsch. Es gibt ein Böses in dieser Welt, das sich selbst als böse erkennt und das Gute mit all seiner Kraft hasst. Alles Gute will es zerstören.“

Hermine zitterte in ihrem Bett, irgendwie kam es ihr sehr real vor, als Dumbledore das sagte.

Der alte Zauberer sprach weiter. „Du gehörst zu den schönen Dingen dieser Welt, Hermine Granger, und deshalb hasst dich auch das Böse. Wenn du auch in dieser Prüfung standhaft geblieben wärst, hätte es dich härter und noch härter geschlagen, bis du zerbrochen wärst. Glaub nicht, dass Helden nicht gebrochen werden können! Wir sind nur schwerer zu brechen, Hermine.“\\ Die Augen des alten Zauberers waren härter geworden, als sie sie je gesehen hatte. „Wenn man viele Stunden lang erschöpft ist, wenn Schmerz und Tod keine vorübergehende Angst, sondern eine Gewissheit ist, dann ist es schwieriger, ein Held zu sein. Wenn ich die Wahrheit sagen muss - dann würde ich heute, ja, im Angesicht von Askaban nicht wanken. Aber als ich ein Erstklässler in Hogwarts war, wäre ich vor den Dementoren, denen du gegenüberstandest, geflohen, denn mein Vater war in Askaban gestorben, und ich fürchtete sie. Wisse dies! Das Böse, das dich angegriffen hat, hätte jeden brechen können, sogar mich. Nur Harry Potter hat es in sich, sich diesem Grauen zu stellen, wenn er ganz in seine Kraft gekommen ist.“

Hermines Nacken konnte den alten Zauberer nicht länger anstarren; sie ließ den Kopf zurück ins Kissen fallen, wo sie zur Decke starrte und aufsaugte, was sie konnte.

„Warum?“ Ihre Stimme zitterte wieder. „Warum sollte jemand so böse sein? Ich verstehe es nicht.“

„Auch ich habe mich das gefragt“, sagte Dumbledores Stimme, in der eine tiefe Traurigkeit lag. „Dreimal zehn Jahre lang habe ich mich das gefragt, und ich verstehe es immer noch nicht. Du und ich werden es nie verstehen, Hermine Granger. Aber wenigstens weiß ich jetzt, was das wahre Böse von sich geben würde, wenn wir mit ihm sprechen und es fragen könnten, warum es böse ist. Es würde sagen: „Warum nicht?“

Ein kurzes Aufflackern von Empörung in ihr.\\ „Es muss doch eine Million Gründe geben, warum nicht!“

„In der Tat“, sagte die Stimme von Dumbledore. „Eine Million Gründe und mehr. Wir werden diese Gründe immer kennen, du und ich. Wenn du darauf bestehst, es so zu formulieren - dann ja, Hermine, die heutige Prüfung hat dich gebrochen. Aber was danach passiert, das gehört auch dazu, wenn man ein Held ist. Was du bist, Hermine Granger, und immer sein wirst.“

Sie hob wieder den Kopf und starrte ihn an. Der alte Zauberer stand von neben ihrem Bett auf. Sein silberner Bart senkte sich, als Dumbledore sich ernsthaft vor ihr verbeugte und ging. Sie schaute weiter dorthin, wohin der alte Zauberer gegangen war. Es hätte ihr etwas bedeuten müssen, hätte sie berühren müssen. Sie hätte sich innerlich besser fühlen sollen, dass Dumbledore, der zuvor so zurückhaltend schien, sie nun als Heldin anerkannt hatte.

Sie fühlte nichts.

Hermine ließ ihren Kopf zurück aufs Bett fallen, als Madam Pomfrey kam und ihr etwas zu trinken gab, das ihre Lippen wie der Nachgeschmack von scharfem Essen versengte und noch schärfer roch und nach gar nichts schmeckte.

Es bedeutete nichts für sie. Sie starrte weiter hinauf zu den fernen Steinfliesen der Decke.

Minerva wartete, ihr Bestes gebend, nicht zu schweben, neben den Doppeltüren zum Krankenflügel von Hogwarts, sie hatte diese Türen als Kind in Hogwarts immer als „die ominösen Tore“ bezeichnet und konnte nicht anders, als sich jetzt daran zu erinnern. Zu viele schlechte Nachrichten waren hier gesprochen worden -

Albus trat heraus. Der alte Zauberer machte auf dem Weg aus dem Krankenzimmer keine Pause, sondern ging einfach weiter in Richtung von Professor Flitwicks Büro; und Minerva folgte ihm. Professor McGonagall räusperte sich.

„Ist es vollbracht, Albus?“

Der alte Zauberer nickte bejahend. „Wenn ein feindlicher Zauber auf sie gewirkt wird oder ein Geist sie berührt, werde ich es wissen und kommen.“

„Ich habe mit Mr. Potter nach dem Unterricht gesprochen“, sagte Professor McGonagall. „Er war der Meinung, dass Miss Granger von nun an nach Beauxbatons und nicht mehr nach Hogwarts gehen sollte.“

Der alte Zauberer schüttelte den Kopf.\\ „Nein. Wenn Voldemort wirklich Miss Granger angreifen will - er ist über alle Maßen hartnäckig. Seine Diener kehren zu ihm zurück, er hätte Bellatrix nicht allein zurückholen können. Askaban selbst ist vor seiner Bosheit nicht sicher. Und was Beauxbatons angeht - nein, Minerva. Ich glaube nicht, dass Voldemort solche Macht wie die gezeigte oft oder gegen stärkere Ziele einsetzen kann, sonst wäre dieses Jahr ganz anders verlaufen. Und Harry Potter ist hier, den Voldemort fürchten muss, ob er es zugibt oder nicht. Jetzt, wo ich sie mit einem Schutzwall versehen habe, wird Miss Granger innerhalb von Hogwarts sicherer sein als außerhalb.“

„Mr. Potter schien das zu bezweifeln“, sagte Minerva. Sie konnte die Schärfe nicht ganz aus ihrer Stimme halten; es gab einen Teil von ihr, der ziemlich stark zustimmte. „Er schien der Meinung zu sein, dass der gesunde Menschenverstand sagte, dass Miss Granger ihre Ausbildung irgendwo anders als in Hogwarts fortsetzen sollte.“

Der alte Zauberer seufzte.\\ „Ich fürchte, der Junge hat zu viel Zeit unter den Muggeln verbracht. Immer streben sie nach Sicherheit; immer bilden sie sich ein, dass man Sicherheit erreichen kann. Wenn Miss Granger innerhalb des Zentrums unserer Festung nicht sicher ist, wird sie auch nicht sicherer sein, wenn sie es verlässt.“

„Nicht jeder scheint so zu denken“, sagte Professor McGonagall.\\ Es war fast der erste Brief gewesen, den sie gesehen hatte, als sie einen kurzen Blick auf ihren Schreibtisch geworfen hatte; ein Umschlag aus feinstem Schafsleder, versiegelt mit grünlich-silbernem Wachs, gepresst in das Abbild einer Schlange, die sich erhob und ihr entgegenzischte.\\ „Ich habe die Eule von Lord Malfoy erhalten, in der er seinen Sohn von Hogwarts abmeldet.“

Der alte Zauberer nickte, unterbrach aber nicht seinen Schritt.\\ „Weiß Harry davon?“

„Ja.“ Ihre Stimme stockte einen Moment lang, als sie sich an Harrys Gesichtsausdruck erinnerte. „Nach dem Unterricht hat Mr. Potter Lord Malfoys ausgezeichneten gesunden Menschenverstand gelobt und gesagt, dass er Madam Longbottom schreiben und ihr raten wird, dasselbe mit ihrem Enkel zu tun, falls er das nächste Ziel sein sollte. Für den Fall, dass Mr. Longbottoms Vormund so nachlässig war, ihn in Hogwarts zu behalten, wollte Mr. Potter, dass er einen Zeitumkehrer, einen Unsichtbarkeitsumhang, einen Besen und einen Beutel, in dem er diese Dinge mit sich führen konnte, bekam; außerdem einen Zehenring mit einem Notfall-Portschlüssel zu einem sicheren Ort, für den Fall, dass jemand Mr. Longbottom entführt und ihn außerhalb von Hogwarts in Sicherheit bringt. Ich habe Mr. Potter gesagt, dass ich nicht glaube, dass das Ministerium einer solchen Verwendung unserer Zeitdreher zustimmen würde, und er meinte, wir sollten nicht fragen. Ich nehme an, er wird wollen, dass Miss Granger dasselbe erhält, falls sie bleibt. Und für sich selbst möchte Mr. Potter einen dreisitzigen Besen, den er in seinem Beutel tragen kann.“\\ Sie war nicht ehrfürchtig von der Liste der Vorkehrungen zurückgeschreckt. Beeindruckt von der Cleverness, aber nicht ehrfürchtig; sie war schließlich eine Verwandlungslehrerin. Aber es jagte ihr immer noch einen Schauer des Unbehagens über den Rücken, dass Harry Potter Hogwarts nun für so gefährlich hielt wie den Verwandlungsunterricht.

„Der Mysteriumsabteilung widersetzt man sich nicht so leicht“, sagte Albus. „Aber im Übrigen -“ Der alte Zauberer schien leicht in sich zusammenzusacken. „Wir können dem Jungen ruhig geben, was er sich wünscht. Und ich werde auch Neville einweisen und Augusta schreiben, dass er über die Ferien hier bleiben soll.“

„Und schließlich“, sagte sie, „sagt Mr. Potter - das ist ein direktes Zitat, Albus - was auch immer für einen Köder für dunkle Zauberer der Schulleiter hier aufbewahrt, er muss ihn aus dieser Schule schaffen, und zwar sofort.“\\ Diesmal konnte sie die Schärfe in ihrer eigenen Stimme nicht unterdrücken.

„Ich habe Flamel darum gebeten“, sagte Albus, der Schmerz war deutlich in seiner Stimme. „Aber Meister Flamel hat gesagt - dass selbst er den Stein nicht mehr sicher verwahren kann - dass er glaubt, dass Voldemort Mittel hat, ihn zu finden, wo auch immer er versteckt ist - und dass er nicht damit einverstanden ist, dass er irgendwo anders als in Hogwarts bewacht wird. Minerva, es tut mir leid, aber es muss getan werden - muss!“

„Nun gut“, sagte Professor McGonagall. „Aber ich für meinen Teil denke, dass Mr. Potter in jedem einzelnen Punkt Recht hat.“

Der alte Zauberer blickte sie an, und seine Stimme überschlug sich, als er sagte: „Minerva, du kennst mich schon lange und so gut wie jede noch lebende Seele - sag mir, habe ich mich schon an die Dunkelheit verloren?“

„Was?“, sagte Professor McGonagall in echter Überraschung. Dann: „Oh, Albus, nein!“

Die Lippen des alten Zauberers pressten sich fest zusammen, bevor er sprach.\\ „Für das größere Wohl. Ich habe so viele geopfert, für das höhere Wohl. Heute hätte ich Hermine Granger fast nach Askaban verdammt, für das Wohl der Allgemeinheit. Und ich ertappe mich dabei - heute ertappe ich mich dabei -, wie ich anfange, mich über die Unschuld zu ärgern, die nicht mehr die meine ist -“\\ Die Stimme des alten Zauberers stockte.\\ „Das Böse wurde im Namen des Guten getan. Böses, das im Namen des Bösen getan wurde. Was ist schlimmer?“

„Du bist dumm, Albus.“

Der alte Zauberer blickte sie noch einmal an, bevor er seinen Blick wieder auf ihren Weg richtete. „Sag mir, Minerva - hast du innegehalten und die Konsequenzen abgewogen, bevor du Miss Granger gesagt hast, wie sie sich an die Familie Potter binden soll?“

Sie holte unwillkürlich Luft, als ihr klar wurde, was sie getan hatte -

„Also hast du es nicht getan.“ Albus' Augen waren traurig. „Nein, Minerva, du musst dich nicht entschuldigen. Es ist gut so. Für das, was du heute von mir gesehen hast - wenn deine erste Loyalität jetzt Harry Potter gilt und nicht mir, dann ist das gut und richtig.“

Sie öffnete ihre Lippen, um zu protestieren, aber Albus fuhr fort, bevor sie ein Wort sagen konnte. „In der Tat - das wird notwendig sein, und zwar mehr als notwendig, wenn der Dunkle Lord, den Harry besiegen muss, um in seine Macht zu kommen, nicht doch Voldemort ist -“

„Nicht das schon wieder!“ sagte Minerva. „Albus, es war Du-weißt-schon-wer, nicht du, der Harry als seinesgleichen bezeichnet hat. Es ist nicht möglich, dass die Prophezeiung von dir spricht!“

Der alte Zauberer nickte, aber seine Augen schienen immer noch weit weg, nur auf den Weg vor ihm gerichtet.

...\\ Die Arrestzelle, die wohl zum Zentrum der magischen Strafverfolgung gehörte, war luxuriös ausgestattet; mehr eine Bemerkung zu dem, was erwachsene Zauberer für selbstverständlich hielten, als irgendein besonderes Gefühl gegenüber Gefangenen. Es gab einen sich selbst zurücklehnenden, selbst schaukelnden Stuhl mit plüschigen, reich strukturierten, selbst wärmenden Kissen. Es gab ein Bücherregal mit zufälligen Büchern, die aus einer Schnäppchenkiste gerettet wurden, und ein ganzes Regal mit alten Zeitschriften, darunter eine von 1883. Was die Toilettenartikel anging, so war es nicht gerade luxuriös, aber es lag ein Zauber auf dem Zimmer, der all diese Dinge auf Eis legte; man durfte nirgendwo hingehen, wo der beobachtende Auror einen nicht sehen konnte. Aber abgesehen davon war es eine recht angenehme kleine Zelle.

Der Verteidigungsprofessor von Hogwarts wurde festgehalten, nicht verhaftet, nicht einmal eingeschüchtert. Es gab keine Beweise, um ihn anzuklagen... außer, dass ein schreckliches und ungewöhnliches Verbrechen an der Hogwarts-Schule für Hexerei und Zauberei begangen worden war, und wenn man von früheren Gelegenheiten ausgeht, standen die Chancen fünf zu eins, dass der jetzige Verteidigungsprofessor irgendwie darin verwickelt war. Hinzu kam, dass niemand in der A. M.S. überhaupt wusste, wer der Verteidigungsprofessor war, und dass der Mann alle Versuche, seine wahre Identität aufzudecken, buchstäblich abgenickt hatte.

Nein, Sie hatten '\emph{Quirinus Quirrell}' noch nicht zurück nach Hogwarts entlassen. Lassen Sie uns dies zur Betonung wiederholen: Der Verteidigungsprofessor. Wurde festgehalten. In einer Zelle.

Der Professor der Verteidigung starrte den beobachtenden Auror an und summte.\\ Der Professor der Verteidigung hatte kein einziges Wort gesprochen, seit er in dieser Zelle war. Er hatte nur gesummt. Das Summen begann mit einem einfachen Kinderschlaflied, das in Muggelbritannien mit „Schlaf, Kindlein schlaf“ beginnt...

Diese Melodie wurde ohne Variation gesummt, immer wieder, sieben Minuten lang, um das zugrunde liegende Muster zu etablieren. Dann begann die Ausarbeitung des Themas. Abschnitte, die zu langsam gesummt wurden, mit langen Pausen dazwischen, so dass der Geist des Zuhörers hilflos auf die nächste Note, die nächste Phrase wartet und wartet. Und dann, wenn diese nächste Phrase kommt, ist sie so abweichend von der Tonart, so unglaublich schrecklich abweichend von der Tonart, nicht nur abweichend von der Tonart der vorherigen Phrasen, sondern in einer Tonhöhe gesungen, die keiner Tonart entspricht, dass man glauben muss, diese Person hätte Stunden damit verbracht, ihr Summen absichtlich zu üben, nur um diese perfekte Anti-Tonhöhe zu erreichen. Es hatte dieselbe Ähnlichkeit mit Musik wie die schreckliche, tote Stimme eines Dementors mit der menschlichen Sprache. Und dieses furchtbare, schreckliche Brummen ist unmöglich zu ignorieren. Es ähnelt einem bekannten Wiegenlied, aber es weicht unvorhersehbar von diesem Muster ab. Es baut Erwartungen auf und verletzt sie dann, nie in einem konstanten Muster, das das Brummen in den Hintergrund treten lassen würde. Das Gehirn des Zuhörers kann weder verhindern, dass er den Abschluss der antimusikalischen Phasen erwartet, noch, dass er die Überraschungen bemerkt. Die einzig mögliche Erklärung dafür, wie diese Art des Summens zustande kam, ist, dass sie absichtlich von einem unsagbar grausamen Genie entwickelt wurde, das eines Tages aufwachte, weil es sich von der gewöhnlichen Folter gelangweilt fühlte, und das beschloss, sich selbst ein Handicap zu verpassen und herauszufinden, ob er den Verstand von jemandem brechen kann, indem er ihn einfach ansummt.

Der Auror hat diesem unvorstellbar schrecklichen Summen vier Stunden lang zugehört, während er von einer riesigen, kalten, tödlichen Präsenz angestarrt wurde, die sich genauso schrecklich anfühlte, egal ob er sie direkt ansieht oder sie am Rande seines Blickfeldes schweben lässt - Das Summen hörte auf. Es gab eine lange Wartezeit. Zeit genug, um falsche Hoffnungen aufkommen zu lassen, die dann von der Erinnerung an frühere Enttäuschungen zerdrückt wurden. Und dann, als das Intervall länger und länger wurde, stieg diese Hoffnung unaufhaltsam wieder auf - Das Brummen begann erneut. Der Auror brach.

Aus seinem Gürtel nahm der Auror einen Spiegel, tippte einmal darauf und sagte dann: „Hier ist Junior Auror Arjun Altunay, ich melde mich mit Code RJ-L20 auf Zelle 3.“

„Code RJ-L20?“, sagte der Spiegel in überraschtem Ton. Es gab ein Geräusch von umgeschlagenen Seiten, dann: „Sie wollen abgelöst werden, weil ein Gefangener versucht, psychologische Kriegsführung zu betreiben und damit Erfolg hat?“\\ (Amelia Bones ist wirklich ziemlich intelligent.)\\ „Was hat der Gefangene zu Ihnen gesagt?“, fragte der Spiegel.\\ (Diese Frage ist nicht Teil der Prozedur RJ-L20, aber leider hat Amelia Bones es versäumt, eine explizite Anweisung aufzunehmen, dass der befehlshabende Offizier sie nicht stellen darf.)

„Er ist -“, sagte der Auror und warf einen Blick zurück in die Zelle. Der Verteidigungsprofessor lehnte sich jetzt in seinem Stuhl zurück und sah ganz entspannt aus. „Er hat mich angestarrt! Und gesummt!“

Es gab eine Pause. Der Spiegel sprach wieder.\\ „Und deswegen rufen Sie einen RJ-L20 an? Sind Sie sicher, dass Sie nicht nur versuchen, sich davor zu drücken, ihn zu beobachten?“\\ (Amelia Bones ist von Idioten umgeben.)

„Sie verstehen das nicht!“, schrie Auror Altunay. „Es ist wirklich ein furchtbares Summen!“

Der Spiegel übertrug ein dumpfes Lachen im Hintergrund, das sich anhörte, als käme es von mehr als einer Person. Dann wieder die Rede.\\ „Herr Altunay, wenn Sie nicht zum Junior-Auror zweiter Klasse degradiert werden wollen, schlage ich vor, dass Sie sich zusammenreißen und wieder an die Arbeit gehen -“

„Streichen Sie das“, sagte eine klare Stimme, die aufgrund der Entfernung zum Spiegel etwas entfernt klang.\\ (Was der Grund dafür ist, dass Amelia Bones oft in einem Koordinationszentrum der A. M.S. sitzt, während sie ihren vom Ministerium geforderten Papierkram erledigt.)

„Auror Altunay“, sagte die klare Stimme und schien sich dem Spiegel zu nähern, „Sie werden in Kürze abgelöst werden. Auror Ben Gutierrez, die Prozedur für RJ-L20 besagt nicht, dass Sie fragen sollen, warum. Es besagt, dass Sie den Auror ablösen, der ihn anfordert. Wenn ich feststelle, dass die Auroren es missbrauchen, werde\\ ich das Verfahren ändern, um den Missbrauch zu verhindern...“

Der Spiegel wurde abrupt unterbrochen.

Der Auror drehte sich um und blickte triumphierend auf den aktuellen Verteidigungsprofessor von Hogwarts, der sich in seinem gepolsterten Stuhl zurücklehnte. Der Mann sprach dann die ersten Worte, die seine Lippen verlassen hatten, seit er die Zelle betreten hatte.

„Auf Wiedersehen, Mr. Altunay“, sagte der Verteidigungsprofessor.

Wenige Minuten später öffnete sich die Tür zur Arrestzelle, und eine grauhaarige Frau trat ein, gekleidet in die purpurrote Robe eines Aurors ohne Anzeichen von Rang oder anderen Verzierungen, die eine schwarze Ledermappe unter dem linken Arm trug.

„Sie sind abgelöst“, sagte die alte Frau abrupt.

Es gab eine kurze Verzögerung, während Auror Altunay versuchte zu erklären, was vorgefallen war. Dies wurde durch ein Nicken und einen strengen, einfachen Finger, der zur Tür hinaus zeigte, unterbrochen.

„Guten Abend, Frau Direktorin“, sagte der Verteidigungsprofessor.

Amelia Bones quittierte diese Aussage nicht, sondern setzte sich abrupt auf den frei gewordenen Stuhl. Die alte Hexe öffnete die schwarze Mappe und ihr Blick wanderte hinunter zu den Pergamenten, die darin lagen.

„Mögliche Hinweise auf die Identität des derzeitigen Hogwarts Verteidigungsprofessors, zusammengestellt von Auror Robards.“\\ Das Titel-Pergament wurde umgedreht und zur Seite geschoben. „Der Verteidigungsprofessor sagte, er sei in Slytherin einsortiert worden. Behauptete, dass seine Familie von Voldemort getötet wurde. Sagte, er habe in einem Kampfsportzentrum in Muggelasien studiert, das von Voldemort zerstört worden sei. Eine Anfrage bei der Abteilung für internationale magische Zusammenarbeit identifiziert diesen Vorfall als die Oni-Affäre von 1969.“\\ Ein weiteres Pergament wurde zur Seite geschoben. „Es scheint auch, dass dieser Verteidigungsprofessor kurz vor Weihnachten eine sehr aufrüttelnde Rede vor seinen Schülern gehalten hat, in der er die vorherige Generation für ihre Uneinigkeit gegen die Todesser geißelte.“\\ Die alte Hexe blickte von der Ledermappe auf.\\ „Madam Longbottom war ziemlich angetan davon und bestand darauf, dass ich das ganze Ding lese. Das Argument kam mir bekannt vor, obwohl ich es damals nicht einordnen konnte. Aber dann hatte ich Sie natürlich für tot gehalten.“

Der oberste Gesetzeshüter des magischen Britanniens blickte nun den amtierenden Verteidigungsprofessor von Hogwarts über die mit Zaubern verstärkte Glasscheibe hinweg scharf an, die sie trennte. Der Mann in der Zelle erwiderte den Blick gleichmäßig, ohne offensichtliche Beunruhigung.

„Ich werde keine Namen nennen“, sagte die alte Hexe. „Aber ich werde eine Geschichte erzählen und sehen, ob sie bekannt vorkommt.“\\ Amelia Bones blickte wieder nach unten und wandte sich dem nächsten Pergament zu. „Geboren 1927, 1938 in Hogwarts eingetreten, in Slytherin einsortiert, Abschluss 1945. Ging auf eine Abschlussfahrt ins Ausland und verschwand bei einem Besuch in Albanien. Wurde für tot gehalten, bis er 1970 ebenso plötzlich ins magische Britannien zurückkehrte, ohne jede Erklärung für die fehlenden fünfundzwanzig Jahre. Er blieb von seiner Familie und seinen Freunden entfremdet und lebte in Isolation. 1971 wehrte er bei einem Besuch in der Winkelgasse einen Versuch von Bellatrix Black ab, die Tochter des Zaubereiministers zu entführen, und tötete mit dem Tötungsfluch zwei der drei sie begleitenden Todesser. Darüber hinaus kennt ganz Britannien die Geschichte; muss ich sie fortsetzen?“\\ Die alte Hexe sah wieder von ihrer Mappe auf.\\ „Nun gut. Es gab einen Prozess im Zaubergamot, bei dem dieser junge Mann für die Anwendung des Tötungsfluchs entlastet wurde, nicht zuletzt dank der Bemühungen seiner Großmutter, der Herrin seines Hauses. Er wurde mit seiner Familie versöhnt, und sie veranstalteten eine Hausversammlung, um seine Rückkehr zu begrüßen. Der Ehrengast kam zu dieser Versammlung, um festzustellen, dass seine gesamte Familie von Todessern erschlagen worden war, sogar die Hauselfen; und dass er selbst, aus der Adelsfamilie, nun der letzte verbliebene Spross eines sehr alten Hauses war.“

Der Verteidigungsprofessor hatte auf nichts von alledem reagiert, außer dass er die Augen halb geschlossen hatte, wie vor Müdigkeit.

„Der junge Mann nahm den Sitz seiner Familie im Zaubergamot ein und wurde zu einer der standhaftesten Stimmen gegen Du-Weißt-Schon-Wer. Mehrere Male führte er Truppen gegen die Todesser an und kämpfte mit geschickter Taktik und außerordentlicher Kraft. Man begann, von ihm als dem nächsten Dumbledore zu sprechen, man dachte, dass er nach dem Sturz des Dunklen Lords Zaubereiminister werden könnte. Am 3. Juli 1973 erschien er nicht zu einer wichtigen Abstimmung des Zaubergamots und man hörte nie wieder von ihm. Wir nahmen an, Du-weißt-schon-wer hätte ihn getötet. Das war ein schwerer Schlag für uns alle, und von diesem Tag an ging es nur noch bergab.“\\ Der Blick der alten Hexe war fragend.\\ „Ich habe selbst um dich getrauert. Was ist passiert?“

Die Schultern des Verteidigungsprofessors bewegten sich leicht, ein kleines Achselzucken. „Sie stellen viele Vermutungen an“, sagte der Verteidigungsprofessor leise. „Ich für meinen Teil würde glauben, dass der Mann schon vor Jahren gestorben ist. Aber wenn dieser Mann dennoch lebt - dann ist es klar, dass er nicht möchte, dass diese Tatsache bekannt wird, und er hat Gründe genug zu schweigen. Dieser Mann war Ihnen einst eine Hilfe, wie es scheint.“\\ Die Lippen des Verteidigungsprofessors bogen sich zu einem zynischen Lächeln. „Aber ich bin nicht überrascht, wenn die Dankbarkeit flüchtig ist. Gibt es noch mehr, was Sie von ihm verlangen würden?„

Die alte Hexe lehnte sich in ihrem Auroren-Beobachtungsstuhl zurück und sah ziemlich erschrocken, vielleicht sogar verletzt aus. „Nein -“, sagte sie nach einem Moment. Ihre Finger klopften auf die Ledermappe; nervös, hätte man meinen können, wenn man geglaubt hätte, dass Amelia Bones jemals nervös sein könnte. „Aber Ihr Haus - es gibt nicht mehr viele Alte Häuser -“

„Es wird für dieses Land wenig ausmachen, ob acht Alte Häuser übrig bleiben oder sieben.“

Die alte Hexe seufzte. „Was hält Dumbledore davon?“

Der Mann in der Arrestzelle schüttelte den Kopf. „Er weiß nicht, wer ich bin, und hat versprochen, sich nicht zu erkundigen oder Forschungen anzustellen.“

Die Augenbrauen der alten Hexe hoben sich. „Wie hat er Sie dann bei den Hogwarts-Wachzaubern identifiziert?“

Ein leichtes Lächeln.\\ "Der Schulleiter hat einen Kreis gezeichnet und Hogwarts mitgeteilt, dass derjenige, der darin steht, der Verteidigungsprofessor ist. Wo wir gerade dabei sind -"\\ Der Ton wurde leiser, flacher.\\ „Ich verpasse meinen Unterricht, Direktor Bones.“

„Sie scheinen - ruhen, manchmal, auf eine merkwürdige Art und Weise. Das wurde auch schon berichtet. Und Sie scheinen sich immer häufiger auszuruhen, je weiter die Zeit fortschreitet.“\\ Die Finger der alten Hexe klopften wieder auf die Ledermappe.\\ „Ich kann mich nicht erinnern, von einem solchen Symptom gelesen zu haben, aber wenn man von so etwas hört, stellt man sich vor... Kämpfe gegen Dunkle Zauberer die schreckliche Flüche gegen Ihre Feinde verwenden...“

Der Verteidigungsprofessor blieb ausdruckslos.

„Brauchen Sie die Hilfe eines Heilers?“, fragte Amelia Bones.\\ Ihre eigene Maske war verrutscht und zeigte deutlich den Schmerz in ihren Augen. „Gibt es überhaupt etwas, das für dich noch getan werden kann?“

„Ich habe zugestimmt, in Hogwarts Verteidigung zu unterrichten“, sagte der Mann in der Zelle flach. „Zieh daraus deine eigenen Schlüsse, Madam. Und ich vermisse meine Stunden, von denen es nicht mehr viele gibt. Ich möchte nach Hogwarts zurückkehren, sofort.“

...\\ Als Hermine zum dritten Mal aufwachte (obwohl es sich anfühlte, als hätte sie nur kurz die Augen geschlossen), stand die Sonne noch tiefer am Himmel und war fast vollständig untergegangen. Sie fühlte sich ein wenig lebendiger und seltsamerweise noch erschöpfter. Diesmal war es Professor Flitwick, der neben ihrem Bett stand und sie an der Schulter rüttelte, ein Tablett mit dampfendem Essen schwebte neben ihm. Aus irgendeinem Grund hatte sie gedacht, dass Harry Potter über ihr Bett gebeugt sein müsste, aber er war nicht da.

\emph{Hatte sie das geträumt?} \emph{Sie konnte sich nicht daran erinnern, geträumt zu haben.}

Es stellte sich heraus (laut Professor Flitwick), dass Hermine das Abendessen in der Großen Halle verpasst hatte und zum Essen geweckt wurde. Und dann konnte sie zurück in den Ravenclaw-Schlafsaal und in ihr eigenes Bett gehen, um den Rest der Nacht zu schlafen.

Sie aß in Stille. Ein Teil von ihr wollte Professor Flitwick fragen, ob er glaubte, dass sie mit einem Gedächtniszauber belegt worden war oder dass sie versucht hatte, Draco Malfoy aus freien Stücken zu töten - wie sie sich daran erinnerte -, aber der größte Teil von ihr hatte Angst, es herauszufinden. Angst, es herauszufinden, war ein Warnzeichen, laut Harry Potter und seinen Büchern; aber ihr Verstand fühlte sich müde an, geprellt, und sie konnte nicht die Kraft aufbringen, ihn zu überwinden. Als sie und Professor Flitwick den Krankenflügel verließen, fanden sie Harry Potter, der im Schneidersitz vor der Tür saß und leise in einem Psychologie-Lehrbuch las.

„Ich übernehme sie von hier“, sagte der Junge-der-lebte. „Professor McGonagall sagte, es wäre in Ordnung.“

Professor Flitwick schien dies zu akzeptieren und verabschiedete sich nach einem strengen Blick auf die beiden. Sie konnte sich nicht vorstellen, was der strenge Blick sagen sollte, es sei denn, er lautete: \emph{Versuch nicht, noch mehr Schüler zu töten.}

Die Schritte von Professor Flitwick verklangen, und die beiden standen allein vor den Türen des Krankenflügels. Sie sah in die grünen Augen des Jungen, der lebte, in das Durcheinander von Haaren, das die Narbe auf seiner Stirn nicht ganz verdeckte; sie sah in das Gesicht des Jungen, der alles gegeben hatte, um sie zu retten, ohne einen zweiten Gedanken zu verschwenden. Da waren Gefühle in ihr - Schuld, Scham, Verlegenheit, auch andere Dinge - aber keine Worte. Es gab nichts, was sie zu sagen wusste.

„Also“, sagte Harry abrupt, „ich habe kurz in meinen Psychologiebüchern geblättert, um zu sehen, was sie über posttraumatische Belastungsstörung sagen. In den alten Büchern stand, dass man sofort danach mit einem Berater über das Erlebnis sprechen sollte. Die neuere Forschung sagt, dass sich bei Experimenten herausgestellt hat, dass es schlimmer wird, wenn man sofort danach darüber spricht. Anscheinend sollte man wirklich dem natürlichen Impuls des Verstandes folgen, die Erinnerungen zu verdrängen und einfach eine Weile nicht daran zu denken.“

Das war so normal für die Art und Weise, wie sie und Harry sich normalerweise unterhielten, dass sie ein plötzliches Brennen in ihrer Kehle spürte. \emph{Wir müssen nicht darüber reden.} Das war es, was Harry gerade gesagt hatte, mehr oder weniger. Es fühlte sich wie Betrug an, vielleicht sogar wie eine Lüge. Nichts war normal. Alles, was falsch war, war immer noch furchtbar falsch, alles, was ungesagt blieb, musste immer noch gesagt werden...

„Okay“, sagte Hermine, weil es nichts anderes zu sagen gab, überhaupt nichts anderes.

„Es tut mir leid, dass ich nicht gewartet habe, als du aufgewacht bist“, sagte Harry, als sie sich auf den Weg machten. „Madam Pomfrey wollte mich nicht reinlassen, also bin ich einfach hier draußen geblieben.“ Er zuckte mit den Schultern und sah etwas traurig aus. „Ich nehme an, ich sollte da draußen sein und versuchen, Schadensbegrenzung in der Öffentlichkeitsarbeit zu betreiben, aber... ehrlich gesagt war ich darin noch nie gut, ich ende nur damit, Leute scharf anzusprechen.“

„Wie schlimm ist es?“ Sie dachte, ihre Stimme hätte in einem Flüstern, einem Krächzen herauskommen müssen, aber das tat sie nicht.

„Nun -“ Harry sagte mit offensichtlichem Zögern. „Die Sache, die du verstehen musst, Hermine, ist, dass du heute beim Frühstück eine Menge Verteidiger hattest, aber alle auf deiner Seite haben... Sachen erfunden. Draco hat zuerst versucht, dich zu töten und solche Sachen. Es war Granger gegen Malfoy, so haben es die Leute gesehen, wie eine Wippe, wo seine Seite runtergedrückt wurde, um deine Seite hochzudrücken. Ich sagte ihnen, dass ihr wahrscheinlich beide unschuldig seid, dass ihr beide einen Gedächtniszauber hattet. Sie hörten nicht auf mich. Beide Seiten behandelten mich wie einen Verräter, der die Mitte spielen will. Und dann hörten die Leute, dass Draco unter Veritaserum ausgesagt hatte, dass er vor dem Kampf versucht hatte, dir zu helfen - hör auf, diesen Ausdruck zu machen, Hermine, du hast ihm eigentlich gar nichts getan. Wie auch immer, alles, was die Leute verstanden haben, war, dass die Pro-Malfoy-Fraktion Recht hatte und die Pro-Granger-Fraktion im Unrecht war.“ Harry stieß einen kleinen Seufzer aus. „Ich habe ihnen gesagt, dass es ihnen peinlich sein würde, wenn die Wahrheit später herauskäme...“

„Wie schlimm ist es?“, fragte sie erneut. Diesmal kam ihre Stimme schwächer heraus.

„Erinnerst du dich an Aschs Konformitätsexperiment?“ sagte Harry und drehte seinen Kopf, um ihr einen ernsten Blick zuzuwerfen.

Für ein paar Sekunden konnte sie sich nicht erinnern, was sie erschreckte, aber dann kam die Referenz zurück.

\emph{1951 hatte Solomon Asch einige Versuchspersonen genommen, und jeder von ihnen war in eine Reihe mit anderen Menschen gestellt worden, die wie sie aussahen, die wie andere Versuchspersonen aussahen, aber in Wirklichkeit Mitwisser des Experimentators waren.}\\ \emph{Sie hatten eine Referenzlinie auf einem Bildschirm gezeigt, die mit X beschriftet war, neben drei anderen Linien, die mit A, B und C beschriftet waren. Die richtige Antwort war offensichtlich C. Die anderen „Versuchspersonen“, die Konföderierten, hatten nacheinander gesagt, dass X die gleiche Länge wie B habe. Die echte Versuchsperson war in der Reihenfolge an die vorletzte Stelle gesetzt worden, um keinen Verdacht zu erregen, weil sie die letzte war. Der Test bestand darin, zu sehen, ob die echte Versuchsperson sich der standardmäßig falschen Antwort B „anpasst“ oder die offensichtlich richtige Antwort C ausspricht.}

\emph{75 \% der Versuchspersonen haben sich mindestens einmal „angepasst“.}\\ \emph{Ein Drittel der Probanden hatte sich mehr als die Hälfte der Zeit angepasst.}\\ \emph{Einige berichteten im Nachhinein, dass sie tatsächlich glaubten, X sei gleich lang wie B. Und das in einem Fall, in dem die Probanden keinen der Konföderierten kannten.}

\emph{Wenn man die Leute in die Nähe von anderen setzte, die zur selben Gruppe gehörten wie sie, wie jemand im Rollstuhl neben anderen Leuten im Rollstuhl, wurde der Konformitätseffekt noch stärker...}

Hermine hatte ein mulmiges Gefühl, wohin das führen würde.\\ „Ich erinnere mich“, flüsterte sie.

„Ich habe der Chaos-Legion ein Anti-Konformitäts-Training gegeben, weißt du. Ich ließ jeden Legionär in der Mitte stehen und sagen: 'Zweimal zwei ist vier!' oder 'Gras ist grün!', während alle anderen in der Chaos Legion sie als Idioten bezeichneten oder sie angrinsten - Flint konnte wirklich gut grinsen - oder ihnen sogar nur leere Blicke zuwarfen und dann weggingen. Die Sache, die du dir merken musst, ist, dass nur die Chaos Legion so etwas jemals praktiziert hat. Niemand sonst in Hogwarts weiß überhaupt, was Konformität ist.“

„Harry!“ Ihre Stimme schwankte. „Wie schlimm ist es?“

Harry zuckte wieder traurig mit den Schultern. „Jeder im zweiten Jahr und darüber, da sie dich nicht kennen. Jeder in der Drachenarmee. Alle aus Slytherin, natürlich. Und, na ja, auch der größte Teil des restlichen magischen Britanniens, denke ich. Denk dran, Lucius Malfoy kontrolliert den Tagespropheten.“

„Alle?“, flüsterte sie. Ihre Glieder hatten begonnen, sich kalt anzufühlen, als wäre sie gerade aus einem ungeheizten Schwimmbad gestiegen.

„Was die Leute wirklich glauben, fühlt sich nicht wie ein Glaube an, es fühlt sich an wie die Art, wie die Welt ist. Du und ich stehen in einer privaten kleinen Blase des Universums, in der Hermine Granger einen Erinnerungszauber bekommen hat. Alle anderen leben in der Welt, in der Hermine Granger versucht hat, Draco Malfoy zu ermorden. Wenn Ernie Macmillian -“

Ihr Atem blieb ihr im Hals stecken. Captain Macmillian -

“- denkt, dass es ihm ethisch verboten ist, jetzt dein Freund zu sein, nun, er versucht, das Richtige zu tun, so wie er es versteht, in der Welt, in der er zu leben glaubt.“

Harrys Augen waren sehr ernst. „Hermine, du hast mir schon oft gesagt, dass ich zu sehr auf andere Menschen herabschaue. Aber wenn ich zu viel von ihnen erwarten würde - wenn ich von den Leuten erwarten würde, dass sie alles richtig machen - dann würde ich sie wirklich hassen. Idealismus beiseite, Hogwarts-Schüler wissen eigentlich nicht genug über kognitive Wissenschaften, um Verantwortung dafür zu übernehmen, wie ihr eigener Verstand funktioniert. Es ist nicht ihre Schuld, dass sie verrückt und dumm sind.“\\ Harrys Stimme war seltsam sanft, fast wie die eines Erwachsenen.\\ „Ich weiß, dass es für dich schwieriger sein wird als für mich. Aber denk dran, irgendwann wird der wahre Schuldige festgenagelt. Die Wahrheit kommt heraus, und jeder, der sich sicher geirrt hat, wird in Verlegenheit gebracht.“

„Und wenn der wahre Schuldige nicht gefasst wird?“, fragte sie mit zitternder Stimme. \emph{...oder wenn sich herausstellt, dass ich es doch bin?}

„Dann kannst du Hogwarts verlassen und auf das Salem Hexeninstitut in Amerika gehen.“

„Hogwarts verlassen?“ An diese Möglichkeit hatte sie noch nie gedacht, außer als ultimative Strafe.

„Ich... Hermine, ich glaube, das solltest du auf jeden Fall tun. Hogwarts ist kein Schloss, es ist der Wahnsinn mit Mauern. Du hast andere Möglichkeiten.“

„Ich werde...“, stammelte sie. „Ich werde... darüber nachdenken müssen...“

Harry nickte. „Zumindest wird niemand versuchen, dich zu verhexen, nicht nach dem, was der Schulleiter heute Abend beim Essen gesagt hat. Oh, und Ron Weasley kam zu mir, sah sehr ernst aus und sagte mir, dass ich dir, wenn ich dich zuerst sehe, sagen soll, dass es ihm leid tut, dass er schlecht von dir gedacht hat, und dass er nie wieder schlecht von dir sprechen wird.“

„Ron glaubt, dass ich unschuldig bin?“, fragte Hermine.

„Nun... äh... er hält dich nicht unbedingt für unschuldig...“

….\\ Der ganze Ravenclaw-Schlafsaal wurde still, als die beiden hereinkamen. Sie starrten sie an. Und starrten sie an. (Sie hatte solche Albträume gehabt.) Und dann sah einer nach dem anderen von ihr weg. Penelope Clearwater, die Vertrauensschülerin im fünften Jahr, die für die Erstklässler zuständig war, schaute langsam und bedächtig weg und drehte ihren Kopf in eine andere Richtung. Su Li und Lisa Turpin und Michael Corner, die alle zusammen an einem Tisch saßen und denen sie schon einmal bei den Hausaufgaben geholfen hatte, sahen alle weg, ihre Gesichter plötzlich nervös, als sie versuchte, ihre Blicke zu erhaschen. Eine Hexe aus dem dritten Jahr namens Latisha Randle, die S. P.H. E.W. zweimal vor Slytherin-Tyrannen gerettet hatte, beugte sich schnell wieder über ihren Schreibtisch und begann wieder mit den Hausaufgaben. Mandy Brocklehurst wandte den Blick von ihr ab.

Wenn Hermine nicht in Tränen ausbrach, dann nur, weil sie es erwartet hatte, es in ihrem Kopf immer wieder durchgespielt hatte. Wenigstens schrien die Leute sie nicht an oder schubsten sie oder verfluchten sie. Sie schauten einfach weg -

Hermine ging schnurgerade auf die Treppe zu, die zu den Schlafsälen der Erstklässlerinnen führte. (Sie sah nicht, dass Padma Patil oder Anthony Goldstein ihr nachsahen, diese beiden einsamen Köpfe, die sich umdrehten, um sie zu verfolgen, als sie ging.) Von hinten hörte sie Harry Potter in einem sehr ruhigen Ton sagen: „Irgendwann wird die Wahrheit ans Licht kommen, ihr alle. Wenn ihr also alle so überzeugt davon seid, dass sie schuldig ist, kann ich euch alle bitten, dieses Papier hier zu unterschreiben, das besagt, dass, wenn sie sich später als unschuldig herausstellt, sie sagen darf: \emph{'Ich hab's euch ja gesagt'} und euch das dann für den Rest eures Lebens vorhalten darf? Kommt rauf, eine nach der anderen, seid nicht feige, wenn ihr wirklich glaubt, dass ihr keine Angst haben solltet, zu wetten -“

Sie war auf halbem Weg die Treppe hinauf, als ihr klar wurde, dass auch andere Mädchen in ihrem Schlafsaal sein würden.

...\\ Die Sterne waren noch nicht ganz herausgekommen, nur ein oder zwei der hellsten waren durch den rötlich-violetten Dunst des Horizonts zu sehen, obwohl die Sonne schon ganz untergegangen war. Hermines Hände gruben sich in den rauen Stein der Brüstung, die den kleinen Balkon bewachte, wo sie sich aus dem Treppenhaus geduckt hatte, nachdem ihr klar geworden war, dass - - sie nicht einfach wieder ins Bett gehen konnte - - die Worte in ihrem Kopf hallten, wie '\emph{Du kannst nicht wieder nach Hause gehen'} klingen sollte. Sie starrte hinaus auf das leere Gelände, den verblassenden Sonnenuntergang, das sprießende Gras so weit unten. Sie war müde, sie konnte jetzt nicht denken, sie musste schlafen. Professor Flitwick hatte ihr gesagt, dass sie schlafen müsse, und es hatte noch einen Zaubertrank zu ihrem Abendessen gegeben.

Vielleicht war das die Art und Weise, wie die zaubernde Gesellschaft mit schrecklichen Traumata an unschuldigen jungen Mädchen umging, sie danach einfach viel schlafen ließ. Sie sollte in ihr Zimmer gehen und schlafen, aber sie hatte Angst, irgendwohin zu gehen, wo andere Leute waren. Angst davor, wie sie sie ansehen oder wegschauen könnten. Fragmente von Gedanken jagten durch einen Verstand, der zu erschöpft war, um sie zu Ende zu bringen oder zu verbinden, während die Nacht voll einsetzte.

\emph{Warum - Warum ist das alles passiert - Vor einer Woche war noch alles in Ordnung -} \emph{Warum -}

Von hinter ihr kam das knarrende Geräusch einer sich öffnenden Tür. Sie drehte den Kopf und sah nach. Professor Quirrell lehnte an der Tür, durch die sie gegangen war, wie eine Pappfigur im Licht der Hogwarts-Fackeln, die hinter ihm in der offenen Tür brannten. Sie konnte seinen Gesichtsausdruck nicht sehen, obwohl die Türöffnung hinter ihm hell war; seine Augen, sein Gesicht, alles, was sie von hier aus sehen konnte, lag im Schatten der Nacht. Der Verteidigungsprofessor von Hogwarts, die Nummer eins auf der Liste der Leute, die das getan haben könnten. Bis zu diesem Moment hatte sie nicht einmal bemerkt, dass sie eine Verdächtigenliste hatte.

Der Mann stand in der Tür und sagte nichts; und sie konnte seine Augen nicht sehen. \emph{Was hatte er überhaupt hier zu suchen -}\\ „Sind Sie hier, um mich zu töten?“, fragte Hermine Granger.

Professor Quirrells Kopf neigte sich daraufhin. Dann ging der Verteidigungsprofessor auf sie zu, die dunkle Silhouette hob langsam und bedächtig eine Hand, als wolle er sie vom Ravenclaw-Turm stoßen -

„Stupor!“ Der Adrenalinstoß überlagerte alles, sie zog ihren Zauberstab, ohne nachzudenken, ihre Lippen formten das Wort wie von selbst, der Fluch sprang aus ihrem Zauberstab und - - kam langsam vor Professor Quirrells erhobener Hand zum Stehen, kräuselte sich in der Luft, als ob er noch zu fliegen versuchte, und gab ein leichtes Zischen von sich. Das rote Glühen beleuchtete zum ersten Mal Professor Quirrells Gesicht und zeigte ein seltsames, liebevolles Lächeln.

„Besser“, sagte Professor Quirrell. „Miss Granger, Sie sind immer noch eine Schülerin in meinem Verteidigungskurs. Wenn Sie mich also als Bedrohung ansehen, erwarte ich nicht, dass Sie mich nur traurig anschauen und fragen, ob ich da bin, um Sie zu töten. Ich ziehe Ihnen zwei Quirrell-Punkte ab.“

Sie war völlig unfähig, Worte zu bilden. Der Verteidigungsprofessor schnippte lässig mit dem Zeigefinger nach dem schwebendem Zauber und schickte den Fluch zurück über ihren Kopf, weit in die Nacht, so dass sie wieder in der Dunkelheit standen. Dann trat Professor Quirrell aus der Tür, die sich hinter ihm schloss, und ein sanftes weißes Licht umgab die beiden, so dass sie wieder sein Gesicht sehen konnte, immer noch mit diesem seltsamen, liebevollen Lächeln.

„Was sind Sie - was machen Sie hier?“

Ein paar weitere Schritte brachten Professor Quirrell zu einem höheren Teil der Balkonbrüstung, wo er seine Ellbogen auf den Stein stützte und sich schwer vorlehnte, um in die Nacht zu schauen.\\ „Ich bin direkt nach meiner Entlassung durch die Auroren hierher gekommen, sobald ich dem Schulleiter Bericht erstattet hatte“, sagte Professor Quirrell mit ruhiger Stimme, „denn ich bin dein Lehrer, und du bist mein Schüler, und ich bin für dich verantwortlich.“

Da verstand Hermine; sie erinnerte sich daran, was Professor Quirrell in der zweiten Verteidigungsstunde des Jahres zu Harry gesagt hatte, nämlich dass er seine Wut kontrollieren müsse. Sie spürte, wie die Schamesröte ihren ganzen Brustkorb hinunterlief. Es dauerte einen Moment, bis die Erkenntnis die Kasteiung überwand und sie die Worte herausbrachte -\\ „Ich -“, sagte Hermine. „Harry denkt - dass ich nicht - meine Beherrschung verloren habe, ich meine -“

„Das habe ich gehört“, sagte Professor Quirrell in eher trockenem Ton. Er schüttelte den Kopf, als ob er die Sterne selbst anschaute. „Der Junge hat Glück, dass ich die Grenze vom Ärger über seine Selbstzerstörungswut zur reinen Neugierde, was er als Nächstes tun wird, überschritten habe. Aber ich stimme mit Mr. Potters Einschätzung der Fakten überein. Dieser Mord war gut geplant, um der Entdeckung durch die Hogwarts-Schutzzauber und das wachsame Auge des Schulleiters zu entgehen. Natürlich würde bei einem so durchdachten Mord ein Unschuldiger die Schuld auf sich nehmen.“\\ Ein kurzes, schiefes Lächeln umspielte die Lippen des Verteidigungsprofessors, obwohl er sie nicht ansah.\\ „Was die Vorstellung angeht, dass Sie es selbst getan haben - ich halte mich für einen begabten Lehrer, aber selbst ich könnte einer so eigensinnigen und unbegabten Schülerin wie Hermine Granger keine mörderische Absicht beibringen.“

Der Teil ihres Gehirns, der entrüstet \emph{"Was?"} sagte, war nicht annähernd laut genug, um ihre Lippen zu erreichen.

„Nein...“, sagte Professor Quirrell. „Das ist nicht der Grund, warum ich hier bin. Sie haben sich keine Mühe gegeben, Ihre Abneigung gegen mich zu verbergen, Miss Granger. Ich danke Ihnen dafür, dass Sie sich nicht verstellt haben, denn ich ziehe wahren Hass einer falschen Liebe vor. Aber Sie sind immer noch meine Schülerin, und ich habe Ihnen etwas zu zu sagen, wenn Sie es hören wollen.“

Hermine sah ihn an, immer noch gegen die Nachwirkungen des Adrenalins von vorhin ankämpfend. Der Verteidigungsprofessor schien nur in den dunklen Himmel zu starren, in dem die Sterne sichtbar wurden.

„Ich wollte einmal ein Held sein“, sagte Professor Quirrell, immer noch nach oben blickend. „Können Sie das glauben, Miss Granger?“

„Nein.“

„Ich danke Ihnen nochmals, Miss Granger. Aber es ist dennoch wahr. Vor langer Zeit, lange vor Ihrer Zeit oder der von Harry Potter, gab es einen Mann, der als Retter gepriesen wurde. Ein Nachkomme, wie man ihn aus Erzählungen kennt, der Gerechtigkeit und Rache wie zwei Zauberstäbe gegen seine furchtbare Nemesis schwang.“\\ Professor Quirrell stieß ein leises, bitteres Lachen aus und blickte in den Nachthimmel.\\ „Wissen Sie, Miss Granger, ich hielt mich damals schon für zynisch, und doch... nun ja.“

Das Schweigen dehnte sich, in der Kälte und der Nacht.

„Um ehrlich zu sein“, sagte Professor Quirrell und blickte zu den Sternen hinauf, „ich verstehe es immer noch nicht. Sie hätten wissen müssen, dass ihr Leben vom Erfolg dieses Mannes abhängt. Und doch war es so, als ob sie alles tun wollten, um ihm das Leben unangenehm zu machen. Um ihm jedes mögliche Hindernis in den Weg zu legen. Ich war nicht naiv, Miss Granger, ich habe nicht erwartet, dass sich die Machthaber so schnell auf meine Seite schlagen würden - nicht ohne etwas für sich selbst davon zu haben. Aber auch ihre Macht war bedroht; und so war ich schockiert, wie sie sich damit zufrieden zu geben schienen, zurückzutreten und diesem Mann alle Last der Verantwortung zu überlassen. Sie spotteten über seine Leistung und bemerkten untereinander, dass sie es an seiner Stelle besser machen würden, obwohl sie sich nicht herabließen, vorzutreten.“\\ Professor Quirrell schüttelte den Kopf, als ob er sich amüsieren würde.\\ „Und es war das Seltsamste - der Dunkle Zauberer, die gefürchtete Nemesis dieses Mannes - nun, diejenigen, die ihm dienten, stürzten sich eifrig auf ihre Aufgaben. Der dunkle Zauberer wurde grausamer gegenüber seinen Anhängern, und sie folgten ihm umso mehr. Männer kämpften um die Chance, ihm zu dienen, und diejenigen, deren Leben von diesem Helden abhing, sich entschlossen daran machten ihm das Leben schwer zu machen... Ich konnte es nicht verstehen, Miss Granger.“\\ Professor Quirrells Gesicht lag im Schatten, als er nach oben blickte.\\ „Vielleicht hat dieser Mann, indem er den Fluch des Handelns auf sich genommen hat, ihn von allen anderen entfernt? War das der Grund, warum sie sich frei fühlten, seinen Kampf gegen den dunklen Zauberer zu verhindern, der sie alle versklavt hätte? Zu glauben, die Menschen würden in ihrem eigenen Interesse handeln, war kein Zynismus, wie sich herausstellte, sondern der reinste Optimismus; in Wirklichkeit erfüllen die Menschen nicht so hohe Ansprüche. Und so erkannte ich mit der Zeit, dass es besser wäre, den dunklen Zauberer allein zu bekämpfen, als mit solchen Anhängern im Rücken.“

„Also -“ Hermines Stimme klang fremd in der Nacht. „Du hast deine Freunde zurückgelassen, wo sie in Sicherheit waren, und hast versucht, den dunklen Zauberer ganz allein anzugreifen?“

„Aber nein“, lachte Professor Quirrell. „Ich habe aufgehört ein Held zu sein und bin losgezogen, um etwas anderes zu tun, das ich angenehmer fand.“

„Was?!“, sagte Hermine, ohne zu überlegen. „Das ist ja furchtbar!“

Der Verteidigungsprofessor drehte seinen Kopf vom Himmel herab, um sie zu betrachten; und sie sah im Licht der Türöffnung, dass er lächelte - oder zumindest lächelte die Hälfte seines Gesichts.

„Wollen Sie mir sagen, Miss Granger, dass ich ein furchtbarer Mensch bin? Nun, vielleicht bin ich das. Aber sind dann Menschen, die nie versuchen, ein Held zu sein, noch schlimmer? Hätten Sie etwas Besseres von mir gedacht, wenn ich überhaupt nichts getan hätte, so wie sie?“

Hermine öffnete den Mund und stellte dann fest, dass sie mal wieder nichts zu sagen hatte. Es war nicht richtig, ein Held zu sein, das konnte man nicht einfach so tun, aber sie wollte nicht sagen, dass jeder, der kein Held war, ein Nichts war, das war Quirrell-Denken...

Das Lächeln, oder Halblächeln, war verschwunden. „Du warst töricht“, sagte der Verteidigungsprofessor leise, „irgendeine dauerhafte Dankbarkeit von denen zu erwarten, die du zu beschützen versucht hast, sobald du dich selbst zur Heldin ernannt hast. Genauso wie du erwartest, dass dieser Mann weiterhin ein Held sein würde, und ihn schrecklich nanntest, weil er aufhörte, während tausend andere nie einen Finger rührten. Es wurde nur erwartet, dass du Tyrannen bekämpfst. Es war eine Steuer, die du geschuldet hast, und sie nahmen es an wie Prinzen, mit einem Spott für die Verspätung der Zahlung. Und du hast sicher schon erlebt, dass ihre Zuneigung wie Staub im Wind verschwand, sobald es nicht mehr in ihrem Interesse war, mit dir zu verkehren...“\\ Der Verteidigungsprofessor richtete sich langsam vom Balkon auf, stand fast gerade und drehte sich zu ihr um, um sie vollständig zu betrachten.\\ „Aber Sie müssen keine Heldin sein, Miss Granger“, sagte Professor Quirrell. „Sie können aufhören, wann immer Sie wollen.“

Dieser Gedanke......war ihr in den letzten zwei Tagen schon mehrmals gekommen. \emph{Der Mensch wird zu dem, was er sein soll, indem er das Richtige tut}, hatte Schulleiter Dumbledore ihr gesagt. Das Problem war nur, dass es anscheinend 2 verschiedene Arten gab, das Richtige zu tun. Da war der Teil von ihr, der sagte, das Richtige sei, weiterhin eine Heldin zu sein und in Hogwarts zu bleiben, sie wusste nicht, was los war, aber eine Heldin würde nicht einfach weglaufen. Und da war auch die Stimme des gesunden Menschenverstandes, die sagte, dass kleine Kinder sich niemals in der Nähe von Gefahren aufhalten sollten, dafür waren Erwachsene da; die Stimme von jedem Schulplakat, das sagte, dass man keine Süßigkeiten von Fremden annehmen sollte. Auch das war richtig.

Hermine Granger stand dort auf dem Balkon und schaute Professor Quirrell an, der von den aufgehenden Sternen umrissen wurde, und sie verstand nicht; sie verstand nicht, wie der Verteidigungsprofessor sie mit einem besorgten Gesichtsausdruck ansehen konnte; sie verstand nicht den Ton des Schmerzes in der Stimme des Verteidigungsprofessors, der sie ergriff; sie verstand nicht, warum ihr das alles gesagt wurde.

„Sie mögen mich nicht einmal, Professor“, sagte Hermine.

Ein kleines Lächeln flackerte auf Professor Quirrells Gesicht. „Ich könnte jetzt sagen, dass ich verärgert darüber bin, dass diese Angelegenheit meine wertvolle Zeit in Anspruch nimmt und meinen Verteidigungsunterricht stört. Aber vor allem, Miss Granger, sind Sie meine Schülerin, und was auch immer ich einst für andere Berufe ausgeübt haben mag, ich denke, ich war ein guter Lehrer in Hogwarts, nicht wahr?“ Plötzlich wirkten Professor Quirrells Augen sehr müde.\\ „Als Ihr Lehrer rate ich Ihnen also, dass Sie andere Berufsmöglichkeiten haben. Ich möchte nicht, dass noch jemand meinen Weg einschlägt.“

Hermine schluckte. Das war eine Seite von Professor Quirrell, die sie nie gesehen oder sich vorgestellt hatte, und sie fraß an ihren Vorurteilen.

Professor Quirrell sah sie einen Moment lang an und blickte dann wieder von ihr weg, zurück zu den Sternen. Als er dieses Mal sprach, war seine Stimme leiser. „Jemand hier hat es auf Sie abgesehen, Miss Granger, und ich kann Sie nicht beschützen, wie ich Mr. Malfoy beschützt habe. Der Schulleiter hat es aus, wie er behauptet, guten Gründen verhindert. Es ist leicht, Hogwarts lieb zu gewinnen, ich weiß es, denn ich mag es auch sehr. Aber in Frankreich hat man eine andere Einstellung zu den Alten Häusern als in England. Und Beauxbatons würde Sie nicht schlecht behandeln, denke ich. Was auch immer Sie sich sonst von mir vorstellen, ich schwöre, wenn Sie mich bitten würden, Sie sicher in Beauxbatons zu sehen, würde ich alles in meiner Macht stehende tun, um Sie dorthin zu bringen.“

„Ich kann nicht einfach -“ sagte Hermine.

„Doch, das können Sie, Miss Granger.“ Jetzt sahen die blassblauen Augen sie aufmerksam an. „Was auch immer Sie aus Ihrem Leben machen wollen, Sie können es in Hogwarts nicht erreichen, nicht mehr. Dieser Ort ist jetzt für Sie ruiniert, selbst wenn man alle anderen Bedrohungen außer Acht lässt. Bitte einfach Harry Potter, dir zu befehlen, nach Beauxbatons zu gehen und dein Leben in Frieden zu leben. Wenn Sie hier bleiben, ist er in den Augen von Britannien und seinen Gesetzen dein Herr!“

Daran hatte sie gar nicht gedacht, es verblasste so sehr im Vergleich dazu, von Dementoren gefressen zu werden; früher war es ihr wichtig gewesen, aber jetzt erschien ihr das alles kindisch, unwichtig, sinnlos, warum also brannten ihre Augen?

„Und wenn Sie das nicht rührt, Miss Granger, bedenken Sie auch, dass Mr. Potter gerade heute Mittag Lucius Malfoy, Albus Dumbledore und den gesamten Zaubergamot bedroht hat, weil er nicht vernünftig denken kann, wenn ihm etwas droht, das jemand ihm etwas wertvolles wegnimmt. Haben Sie keine Angst davor, was er als Nächstes tun wird?“

Das ergab einen Sinn. \emph{Schrecklich sinnvoll. Schrecklich schrecklich sinnvoll. Es machte zu viel Sinn} - sie hätte es nicht in Worte fassen können, was die Erkenntnis auslöste, es sei denn, es war der schiere Druck, den der Professor der Verteidigung auf sie ausübte. Dass, wenn der Verteidigungsprofessor hinter der ganzen Sache steckte - dann hatte Professor Quirrell das alles nur getan, um sie aus dem Weg für seine Pläne mit Harry zu bekommen. Ohne eine bewusste Entscheidung verlagerte sie ihr Gewicht auf den anderen Fuß, ihr Körper bewegte sich von dem Verteidigungsprofessor weg -

„Sie glauben also, dass ich derjenige bin, der dafür verantwortlich ist?“, sagte Professor Quirrell. Seine Stimme klang ein wenig traurig, als er es sagte, und ihr eigenes Herz blieb fast stehen, als sie es hörte. „Ich nehme an, ich kann es Ihnen nicht verübeln. Immerhin bin ich der Verteidigungsprofessor von Hogwarts. Aber Miss Granger, selbst wenn man annimmt, dass ich Ihr Feind bin, sollte Ihnen der gesunde Menschenverstand sagen, dass Sie sich sehr schnell von mir entfernen sollten. Sie können den Tötungsfluch nicht anwenden. Die richtige Taktik ist also, zu apparieren. Es macht mir nichts aus, der Bösewicht Ihrer Fantasie zu sein, wenn es die Sache klarer macht. Verlassen Sie Hogwarts. Und überlassen Sie mich denen, die mit mir umgehen können. Ich sorge dafür, dass der Transport durch eine angesehene Familie erfolgt. Mr. Potter wird wissen, dass er mir die Schuld geben kann, wenn Sie nicht sicher ankommen.

„Ich -“ Sie fühlte sich kalt, die Nachtluft kühlte ihre Haut, oder vielleicht wurde sie von ihr gekühlt. „Ich muss darüber nachdenken -“

Professor Quirrell schüttelte den Kopf. „Nein, Miss Granger. Ihre Abreise wird mich einige Zeit in Anspruch nehmen, und ich habe weniger Zeit übrig, als Sie vielleicht denken. Diese Entscheidung mag für Sie schmerzhaft sein, aber sie sollte nicht zweideutig sein; vieles wiegt in dieser Waage, aber nicht gleichmäßig. Ich muss heute Abend wissen, ob Sie zu gehen gedenken.“

\emph{Und wenn nicht - hatte der Verteidigungsprofessor sie absichtlich gewarnt? Dass er wieder zuschlagen würde, wenn sie nicht fliehen würde? Warum sollte es so wichtig sein, was wollte Professor Quirrell mit Harry machen?}\\ \emph{Hermine Granger, ich werde weniger subtil sein, als es für einen geheimnisvollen alten Zauberer üblich ist, und dir ganz offen sagen, dass du dir nicht vorstellen kannst, wie schlimm es werden könnte, wenn sich die Ereignisse um Harry Potter zum Schlechten wenden.}\\ Der mächtigste Zauberer der Welt hatte ihr das gesagt, als er darüber sprach, wie wichtig es war, dass sie nicht aufhörte, Harrys Freundin zu sein.

Hermine schluckte, sie schwankte ein wenig, wo sie stand, auf dem steinernen Balkon eines magischen Schlosses. Plötzlich schien sich die ganze tödliche Absurdität der Situation zu erheben und sie an der Kehle zu packen, dass zwölfjährige Mädchen nicht in Gefahr sein sollten, nicht über solche Dinge nachdenken sollten, dass Mum wollen würde, dass sie wegläuft, und ihr Vater einen Herzinfarkt bekäme, wenn er auch nur wüsste, dass sie mit dieser Frage konfrontiert wurde. Und da wusste sie, wie Harry und Dumbledore sie zu warnen versucht hatten, dass alles, was sie je über das Heldinnendasein gedacht hatte, ein Irrtum war. Dass es so etwas wie Helden außerhalb von Geschichten eigentlich gar nicht gab. Es gab nur schreckliche Gefahren, und von Auroren verhaftet und in Zellen neben Dementoren gesteckt zu werden, Schmerz und Angst und -

„Miss Granger?“, sagte der Verteidigungsprofessor.

Sie sagte nichts. Alle Worte blieben ihr in der Kehle stecken.

„Ich brauche eine Entscheidung, Miss Granger.“

Sie hielt ihren Kiefer verschlossen, ließ kein Wort herauskommen.

Schließlich seufzte der Verteidigungsprofessor. Langsam erlosch das weiße Licht, und langsam schwang die Tür hinter ihm auf, so dass er wieder als schwarze Silhouette vor der Öffnung stand.

„\emph{Gute Nacht,} Miss Granger“, sagte er, drehte ihr den Rücken zu und ging in Richtung Hogwarts davon.

Es dauerte eine Weile, bis sich ihr Atem wieder beruhigt hatte. Was auch immer heute Abend hier passiert war, es fühlte sich alles andere als ein Sieg an. Sie hatte so hart gekämpft, um sich selbst davon abzuhalten, im Angesicht des Drucks des Verteidigungsprofessors Ja zu sagen, und jetzt wusste sie nicht einmal, ob sie das Richtige getan hatte. Als sie selbst wieder ins Licht ging (nachdem die Erschöpfung alles überholt hatte und Schlaf wieder eine Möglichkeit war), glaubte sie, es zu hören, als sie in der Tür stand, von hinten und über ihr, einen entfernten krächzenden Schrei. Aber es war nicht für sie bestimmt, das wusste sie, und so begann sie, die Treppe zu ihrem Schlafsaal hinaufzusteigen. Die anderen Mädchen schliefen wahrscheinlich schon und würden sie nicht ansehen oder wegschauen - sie spürte, wie die Tränen begannen, und diesmal hielt sie sie nicht auf.

