

\hypertarget{humanismus-teil-3}{% \section{45. Humanismus Teil 3}\label{humanismus-teil-3}}

\textbf{\uline{Humanismus, Teil 3}}

Fawkes' Lied ging sanft ins Leere. Harry setzte sich auf, wo er auf dem winterlichen Gras gelegen hatte, Fawkes immer noch auf seiner Schulter sitzend. Alle um ihn herum atmeten tief ein.

"Harry", sagte Seamus mit schwankender Stimme, "geht es dir gut?"

\emph{Der Frieden des Phönix war immer noch in ihm, und Wärme, von dort, wo Fawkes hockte.

Wärme, die sich in ihm ausbreitete, und die Erinnerung an das Lied, das in der Gegenwart des Phönix noch lebendig war.}

Es gab schreckliche Dinge, die ihm widerfahren waren, schreckliche Gedanken, die ihn durchströmt hatten. Er hatte eine unmögliche Erinnerung wiedererlangt, obwohl der Dementor ihn dazu gebracht hatte, sie zu entweihen. Ein seltsames Wort hallte immer wieder in seinem Geist wider. Und all das konnte auf später verschoben werden, während der Phönix noch rot und golden unter der untergehenden Sonne leuchtete.

Fawkes krächzte ihn an. "Gibt es etwas, das ich tun muss?" sagte Harry zu Fawkes.

Fawkes wippte mit dem Kopf in Richtung des Dementors.

Harry schaute auf den unsichtbaren Schrecken, der immer noch in seinem Käfig saß, dann wieder auf den Phönix, verwirrt.

"Mr. Potter?", ertönte Minerva McGonagalls Stimme von hinter ihm. "Geht es Ihnen gut?"

Harry kletterte auf seine Füße und drehte sich um.

Minerva McGonagall sah ihn an und wirkte sehr besorgt; Albus Dumbledore neben ihr musterte ihn aufmerksam; Filius Flitwick wirkte ungeheuer erleichtert; und alle Schüler starrten ihn einfach nur an.

"Ich denke schon, Professor McGonagall", sagte Harry ruhig. Fast hätte er Minerva gesagt, bevor er es schaffte, sich zu stoppen. Solange Fawkes auf seiner Schulter lag, ging es Harry zumindest gut; es konnte sein, dass er gleich zusammenbrechen würde, wenn Fawkes weg war, aber irgendwie schienen solche Gedanken nicht wichtig zu sein.

"Ich glaube, es geht mir gut."

Es hätte Jubel geben müssen, oder Seufzer der Erleichterung, oder so etwas, aber niemand schien zu wissen, was er sagen sollte, niemand. Die Ruhe des Phönix verweilte. Harry drehte sich um.

"Hermine?", fragte er.

Jeder, der auch nur einen winzigen Funken Romantik im Herzen hatte, hielt den Atem an.

"Ich weiß nicht wirklich, wie ich mich anständig bedanken soll", sagte Harry leise, "genauso wenig wie ich weiß, wie ich mich entschuldigen soll. Ich kann nur sagen, wenn du dich fragst, ob es das Richtige war, dann war es das."

Der Junge und das Mädchen blickten sich in die Augen.

"Entschuldigung", sagte Harry. "Wegen dem, was als Nächstes passiert. Wenn ich irgendetwas tun kann -"

"Nein", sagte Hermine zurück. "Gibt es nicht. Aber es ist schon in Ordnung."

Dann wandte sie sich von Harry ab und ging weg, in Richtung des Weges, der zurück zu den Toren von Hogwarts führte. Einige Mädchen warfen Harry verwirrte Blicke zu und folgten ihr dann. Als sie gingen, konnte man hören, wie die aufgeregten Fragen begannen.

Harry sah ihnen nach, als sie gingen, drehte sich um und sah den anderen Schülern nach. Sie hatten ihn am Boden liegen sehen, schreiend, und… Fawkes kraulte seine Wange, kurz. … und das würde ihnen helfen, eines Tages zu verstehen, dass der Junge-der-lebte auch verletzt sein konnte, unglücklich sein konnte. Damit sie, wenn sie selbst verletzt und unglücklich waren, sich daran erinnerten, wie Harry sich am Boden krümmte, und wussten, dass ihr eigener Schmerz und ihre Probleme nicht bedeuteten, dass sie es nie zu etwas bringen würden.

\emph{\hfill\break Hatte der Schulleiter das bedacht, als er die anderen Schüler zum Zuschauen zugelassen hatte?} Harrys Blick ging wieder zu dem großen, zerfledderten Umhang, fast geistesabwesend, und ohne sich wirklich bewusst zu sein, was er sagte, sagte Harry:

"Es sollte nicht existieren."

"Ah", sagte eine trockene, präzise Stimme. "Ich dachte mir, dass du das sagen würdest. Es tut mir sehr leid, dir das zu müssen, Mr. Potter, aber Dementoren können nicht getötet werden. Viele haben es versucht."

"Wirklich?" sagte Harry, immer noch geistesabwesend. "Was haben sie versucht?"

"Es gibt einen bestimmten, äußerst gefährlichen und zerstörerischen Zauberspruch", sagte Professor Quirrell, "den ich hier nicht nennen will; einen Zauberspruch mit verfluchtem Feuer.

Das ist das, was man verwenden würde, um ein uraltes Gerät wie den Sprechenden Hut zu zerstören. Er hat keine Wirkung auf Dementoren. Sie sind unsterblich."

"Sie sind nicht unsterblich", sagte der Schulleiter. Die Worte mild, der Blick scharf.

"Sie besitzen kein ewiges Leben. Sie sind Wunden in der Welt, und wenn man eine Wunde angreift, wird sie nur größer."

"Hm", sagte Harry. "Und wenn man es in die Sonne wirft? Würde es zerstört werden?"

"In die Sonne werfen?!", quietschte Professor Flitwick und sah aus, als wollte er in Ohnmacht fallen.

"Das scheint unwahrscheinlich zu sein, Mr. Potter", sagte Professor Quirrell trocken.

"Aber die Sonne ist schließlich sehr groß; ich bezweifle, dass der Dementor eine große Wirkung auf sie haben würde. Trotzdem, einen solchen Test möchte ich nicht versuchen, Mr. Potter, nur für den Fall."

"Ich verstehe", sagte Harry.

Fawkes krächzte ein letztes Mal, legte seine Flügel um Harrys Kopf und stieß sich dann von Harry ab. Er stürzte sich direkt auf den Dementor und stieß einen durchdringenden Schrei aus, der auf dem ganzen Feld widerhallte. Und bevor jemand darauf reagieren konnte, gab es einen Feuerblitz, und Fawkes war weg. Der Frieden verblasste, ein wenig. Die Wärme verblasste, ein wenig. Harry atmete tief ein und ließ ihn wieder aus.

"Ja", sagte Harry. "Er lebt noch."

Wieder diese Stille, wieder die Abwesenheit von Jubel; niemand schien zu wissen, wie er darauf reagieren sollte -

"Es ist gut zu wissen, dass Sie vollständig genesen sind, Mr. Potter", sagte Professor Quirrell fest, als wollte er jede andere Möglichkeit ausschließen.

"Nun, ich glaube, Miss Ransom war als nächstes dran?"

Damit begann ein kleiner Streit, bei dem Professor Quirrell Recht hatte und alle anderen Unrecht.

Der Verteidigungsprofessor wies darauf hin, dass trotz der verständlichen Emotionen aller Beteiligten die Chance, dass einem anderen Schüler ein ähnliches Missgeschick passierte, gegen Null tendierte; zumal sie nun wussten, wie man Missgeschicke mit Zauberstäben vermeidet.

Und in der Zwischenzeit gab es andere Schüler, die ihre eigene beste Chance nutzen mussten, um einen körperlichen Patronus-Zauber zu wirken oder das Gefühl eines Dementors zu erlernen, damit sie fliehen konnten, und ihren eigenen Grad der Verwundbarkeit zu entdecken…

Am Ende stellte sich heraus, dass Dean Thomas und Ron Weasley aus Gryffindor die einzigen waren, die noch bereit waren, in die Nähe des Dementors zu gehen, was die Auseinandersetzung vereinfachte.

Harry warf einen Blick in die Richtung des Dementors. Das Wort hallte in seinen Gedanken wieder.

\emph{Also gut}, dachte Harry bei sich, \emph{wenn der Dementor ein Rätsel ist, was ist dann die Antwort?}

Und einfach so war es offensichtlich. Harry betrachtete den angeschlagenen, leicht korrodierten Käfig.

\emph{Er sah, was unter dem langen, zerfledderten Umhang lag.} \emph{Das war's also.}

Professor McGonagall kam und sprach mit Harry. Sie hatte das Schlimmste noch nicht gesehen, so dass ihr nur ein leichtes Glitzern in den Augen stand. Harry sagte ihr, dass er nachher mit ihr sprechen und ihr eine Frage stellen müsse, die er schon eine Weile aufgeschoben hatte, aber das musste nicht jetzt geschehen, wenn sie beschäftigt war. Sie hatte einen gewissen Blick aufgesetzt, der darauf hindeutete, dass sie von etwas Wichtigem weggezogen worden war; und Harry bemerkte dies ihr gegenüber und sagte, dass sie sich ehrlich gesagt nicht schuldig fühlen müsse, weil sie ging.

Das brachte ihm einen etwas scharfen Blick ein, aber dann ging sie, eilig, mit dem Versprechen, dass sie später reden würden.

Dean Thomas zauberte wieder seinen weißen Bären, sogar in Anwesenheit des Dementors; und Ron Weasley stellte einen angemessenen Schild aus funkelndem Nebel auf. Damit war der Tag beendet, soweit es alle anderen betraf, und Professor Flitwick begann, die Schüler zurück nach Hogwarts zu treiben.

Als klar war, dass Harry zurückbleiben wollte, schaute Professor Flitwick ihn fragend an; und Harry seinerseits warf einen bedeutungsvollen Blick auf Dumbledore. Harry wusste nicht, was Professor Flitwick davon hielt, aber nach einem scharfen, warnenden Blick entfernte sich sein Hausoberhaupt. Und so blieben nur Harry, Professor Quirrell, Schulleiter Dumbledore und ein Auroren-Trio übrig. Es wäre besser gewesen, das Trio zuerst loszuwerden, aber Harry fiel kein guter Weg ein, das zu tun.

"Also gut", sagte Auror Komodo, "bringen wir es zurück."

"Entschuldigen Sie mich", sagte Harry. "Ich würde gern noch mal auf den Dementor losgehen."

Harrys Bitte stieß auf einen gewissen Widerstand der Sorte "\emph{Du bist ja völlig verrückt}", obwohl nur Auror Butnaru das tatsächlich laut sagte.

"Fawkes hat mir gesagt, ich soll", sagte Harry. Das überwand nicht alle Widerstände, trotz des schockierten Blicks, den es auf Dumbledores Gesicht erzeugte. Der Streit ging weiter, und es begann, die Ränder der verbliebenen Ruhe des Phönix abzunutzen, was Harry ärgerte, wenn auch nur ein wenig.

"Seht ihr", sagte Harry, "ich bin mir ziemlich sicher, dass ich weiß, was ich vorher falsch gemacht habe. Es gibt eine Art von Mensch, der eine andere Art von warmen und glücklichen Gedanken benutzen muss. Lasst es mich einfach versuchen, okay?"

Auch dies erwies sich nicht als überzeugend.

"Ich glaube", sagte Professor Quirrell schließlich und starrte Harry mit zusammengekniffenen Augen an, "dass, wenn wir ihm nicht erlauben, dies unter Aufsicht zu tun, er sich irgendwann davonschleichen und auf eigene Faust nach einem Dementor suchen könnte. Beschuldige ich Sie etwa zu Unrecht, Mr. Potter?"

Daraufhin gab es eine entsetzte Pause. Es schien ein guter Zeitpunkt zu sein, seine

Trumpfkarte auszuspielen.

"Es macht mir nichts aus, wenn der Schulleiter seinen eigenen Patronus aufrechterhält", sagte Harry.

\emph{Denn ich werde genauso in der Gegenwart eines Dementors sein, Patronus hin oder her.}

Es herrschte Verwirrung darüber, sogar Professor Quirrell sah verwirrt aus; aber der Schulleiter stimmte schließlich zu, da es unwahrscheinlich schien, dass Harry durch vier Patronus verletzt werden könnte.

\emph{Wenn der Dementor nicht auf irgendeiner Ebene durch deinen Patronus gelangen könnte, Albus Dumbledore, würdest du nicht einen nackten Mann sehen, dessen Anblick schmerzhaft wäre…}

Harry hat es nicht laut ausgesprochen, aus naheliegenden Gründen. Und sie begannen, auf den Dementor zuzugehen.

"Schulleiter", sagte Harry, "nehmen wir an, die Ravenclaw-Tür stellt Ihnen dieses Rätsel: Was befindet sich in der Mitte eines Dementors? Was würden Sie sagen?"

"Furcht", sagte der Schulleiter.

Es war ein ganz einfacher Fehler. Der Dementor kam auf dich zu und die Angst überkam dich. Die Angst tat weh, man spürte, wie die Angst einen schwächte, man wollte, dass die Angst wegging.

Es war ganz natürlich, zu denken, die Angst sei das Problem. Also schloss man, dass der Dementor eine Kreatur der reinen Angst sei, dass man nichts fürchten müsse außer der Angst selbst.

Aber…

\emph{Was liegt im Zentrum eines Dementors?}

Furcht.

\emph{Was ist so furchtbar, dass der Verstand sich weigert, es zu sehen?}

Furcht.

\emph{Was ist unmöglich zu töten?}

Furcht.

Es passte nicht ganz, wenn man darüber nachdachte. Aber es war klar, warum die Leute nicht über die erste Antwort hinausgehen wollten. Die Menschen verstanden die Angst. Die Menschen wussten, was sie gegen die Angst tun sollten. Angesichts eines Dementors wäre es also nicht gerade beruhigend zu fragen: \emph{"Was ist, wenn die Angst nur eine Nebenwirkung und nicht das} \emph{Hauptproblem ist?}" Er war dem von vier Patroninnen bewachten Dementor-Käfig schon sehr nahe gekommen, als die drei Auroren und Professor Quirrell scharf einatmeten. Alle Gesichter wandten sich dem Dementor zu und schienen ihm zuzuhören; auf Auror Goryanofs Gesicht stand Entsetzen.

Dann hob Professor Quirrell den Kopf, sein Gesicht hart, und spuckte in Richtung des Dementors.

"Er mochte es wohl nicht, wenn ihm seine Beute weggenommen wurde", sagte Dumbledore leise.

"Nun. Wenn es nötig wird, Quirinus, wird es in Hogwarts immer eine Zuflucht für dich geben."

"Was sagte er?", fragte Harry.

Jeder Kopf drehte sich, um ihn anzustarren.

"Du hast es nicht gehört …?" sagte Dumbledore. Harry schüttelte den Kopf.

"Es hat zu mir gesagt", sagte Professor Quirrell, "dass es mich kennt und dass es mich eines Tages jagen wird, egal wo ich mich zu verstecken versuche."

Sein Gesicht war starr und zeigte keine Angst.

"Ah", sagte Harry. "Darüber würde ich mir keine Sorgen machen, Professor Quirrell."

\emph{Es ist ja nicht so, dass Dementoren tatsächlich reden oder denken können; die Struktur, die sie haben, ist von deinem eigenen Verstand und deinen Erwartungen geborgt.}..

Jetzt warfen ihm alle sehr seltsame Blicke zu. Die Auroren blickten nervös zueinander, zum Dementor, zu Harry. Und sie standen direkt vor dem Käfig des Dementors.

"Es sind Wunden in der Welt", sagte Harry. "Es ist nur eine wilde Vermutung, aber ich schätze, derjenige, der das gesagt hat, war Godric Gryffindor."

"Ja…", sagte Dumbledore. "Woher weißt du das?"

\emph{Es ist ein weit verbreiteter Irrglaube,} dachte Harry, \emph{dass alle besten Rationalisten nach Ravenclaw sortiert werden und keiner für andere Häuser übrig bleibt.}

Dem ist nicht so; in Ravenclaw sortiert zu sein, bedeutet, dass deine stärkste Tugend die Neugier ist, das Fragen und der Wunsch, die wahre Antwort zu erfahren. Und dies ist nicht die einzige Tugend, die ein Rationalist braucht. Manchmal muss man hart an einem Problem arbeiten und eine Weile dranbleiben. Manchmal braucht man einen cleveren Plan, um es herauszufinden. Und manchmal braucht man mehr als alles andere den Mut, sich einer Antwort zu stellen…

Harrys Blick ging zu dem, was unter dem Umhang lag, das Grauen weit schlimmer als jede verwesende Mumie. Rowena Ravenclaw hätte es auch wissen können, denn das Rätsel war offensichtlich genug, sobald man es als Rätsel sah. Und es war auch offensichtlich, warum die Patronusse Tiere waren. Die Tiere wussten es nicht, und so waren sie vor der Angst gefeit.

Aber Harry wusste es, und er würde es immer wissen und nie vergessen können. Er hatte versucht, sich beizubringen, der Realität ins Auge zu sehen, ohne mit der Wimper zu zucken, und auch wenn Harry diese Kunst noch nicht beherrschte, so hatten sich doch diese Furchen in seinen Geist eingegraben, der erlernte Reflex, dem schmerzhaften Gedanken entgegenzusehen statt wegzusehen.

Harry würde niemals in der Lage sein zu vergessen, indem er warme, glückliche Gedanken an etwas anderes dachte, und deshalb hatte der Zauber bei ihm nicht funktioniert. Also würde Harry einen warmen, glücklichen Gedanken denken, der nichts mit etwas anderem zu tun hatte.

Harry zückte seinen Zauberstab, den Professor Flitwick ihm zurückgegeben hatte, und stellte seine Füße in die Anfangsstellung für den Patronus-Zauber. In seinem Geist legte Harry die letzten Reste der Ruhe des Phönix ab, schob die Ruhe, den traumhaften Zustand beiseite, erinnerte sich stattdessen an Fawkes' durchdringenden Schrei und machte sich zum Kampf bereit. Rief alle Teile und Elemente seiner selbst zum Erwachen auf. Rief in sich all die Kraft auf, die der Patronus-Zauber jemals aufbringen konnte, um sich in den richtigen Geisteszustand für den letzten warmen und glücklichen Gedanken zu versetzen; erinnerte sich an alle hellen Dinge.

\emph{Die Bücher, die sein Vater ihm gekauft hatte. An Mums Lächeln, als Harry ihr eine Muttertagskarte gebastelt hatte, ein kunstvolles Ding, für das er ein halbes Pfund Elektronikteile aus der Garage verwendet hatte, um Lichter zum Blinken zu bringen und eine kleine Melodie zu piepsen, und für dessen Herstellung er drei Tage gebraucht hatte.}

\emph{Professor McGonagall sagte ihm, dass seine Eltern gut gestorben seien und ihn beschützt hätten. Wie sie es getan hatten. Die Erkenntnis, dass Hermine mit ihm mithalten und sogar schneller lernen konnte, dass sie wahre Rivalen und Freunde sein konnten.}

\emph{Wie er Draco aus der Dunkelheit lockte und zusah, wie er sich langsam dem Licht näherte. Neville und Seamus und Lavender und Dean und alle anderen, die zu ihm aufschauten, alle, für die er gekämpft hätte, um sie zu beschützen, wenn irgendetwas Hogwarts bedrohte. Alles, was das Leben lebenswert machte.}

Sein Zauberstab hob sich in die Ausgangsposition für den Patronus-Zauber.

\emph{Harry dachte an die Sterne, das Bild, das den Dementor auch ohne Patronus fast aufgehalten hatte.} Nur dieses Mal fügte Harry die fehlende Zutat hinzu, er hatte es nie wirklich gesehen, aber er hatte die Bilder und das Video gesehen.

Die Erde, strahlend blau und weiß vom reflektierten Sonnenlicht, wie sie im Weltraum hing, inmitten der schwarzen Leere und der leuchtenden Lichtpunkte.Sie gehörte dorthin, in dieses Bild, denn sie war es, die allem anderen seine Bedeutung gab. Die Erde war es, die den Sternen Bedeutung verlieh, die sie zu mehr als unkontrollierten Fusionsreaktionen machte, denn es war die Erde, die eines Tages die Galaxie kolonisieren und das Versprechen des Nachthimmels erfüllen würde.

\emph{Würden sie immer noch von Dementoren geplagt werden, die Kinder der Kinder der Kinder, die fernen Nachfahren der Menschheit, wenn sie von Stern zu Stern zogen?}

\emph{\textbf{Nein. Natürlich nicht.}}

\emph{Die Dementoren waren nur kleine Plagegeister, die im Licht dieser Verheißung verblassten.

\textbf{Nicht unbesiegbar, nicht mal annähernd.}}

\emph{Man musste sich mit kleinen Ärgernissen abfinden, wenn man zu den wenigen Glücklichen und Unglücklichen gehörte, die auf der Erde geboren wurden; auf der Alten Erde, wie sie eines Tages in Erinnerung bleiben würde. Auch das war Teil dessen, was es bedeutete, am Leben zu sein, wenn man zu der winzigen Handvoll empfindungsfähiger Wesen gehörte, die in den Anfang aller Dinge hineingeboren wurden, bevor intelligentes Leben voll in seine Kraft gekommen war. Dass die viel größere Zukunft davon abhing, was man hier, jetzt, in den frühesten Tagen der Dämmerung tat, als es noch so viel Dunkelheit zu bekämpfen gab und vorübergehende Ärgernisse wie Dementoren.}

\emph{\hfill\break Mum und Dad, Hermines Freundschaft und Dracos Reise, Neville und Seamus und Lavender und Dean, der blaue Himmel und die strahlende Sonne und alle hellen Dinge, die Erde, die Sterne, das Versprechen, alles, was die Menschheit war und alles, was sie werden würde.}

.. Auf dem Zauberstab bewegten sich Harrys Finger in ihre Ausgangsposition; er war jetzt bereit, die richtige Art von warmen und glücklichen Gedanken zu denken.

Und Harrys Augen starrten direkt auf das, was unter dem zerfledderten Umhang lag, sahen direkt auf das, was Dementor genannt worden war.

Das Nichts, die Leere, das Loch im Universum, die Abwesenheit von Farbe und Raum, der offene Abfluss, durch den die Wärme aus der Welt strömte. Die Angst, die er ausstrahlte, raubte alle glücklichen Gedanken, seine Nähe zehrte an deiner Kraft und Stärke, sein Kuss würde alles zerstören, was du warst.

\emph{Ich kenne dich jetzt,} dachte Harry, während sein Zauberstab einmal, zweimal, dreimal und viermal zuckte, während seine Finger genau die richtigen Abstände zurücklegten,

\emph{ich begreife dein Wesen,}

\textbf{\emph{du symbolisierst den Tod}}\emph{,}

\emph{durch irgendein Gesetz der Magie bist du ein Schatten, den der Tod in die Welt wirft.}

\textbf{\emph{Und der Tod ist nichts, was ich jemals umarmen werde.}}

\emph{Er ist nur eine kindische Sache, der die menschliche Spezies noch nicht entwachsen ist.

Und eines Tages…}

\textbf{\emph{werden wir darüber hinwegkommen…}}

\textbf{\emph{Und die Menschen werden sich nicht mehr verabschieden müssen.}}

.. Der Zauberstab hob sich und richtete sich direkt auf den Dementor.

\textbf{"EXPECTO PATRONUM!}"

Der Gedanke explodierte aus ihm wie ein brechender Damm, strömte seinen Arm hinunter in seinen Zauberstab, brach als gleißend weißes Licht aus ihm heraus. Licht, das körperlich wurde, Form und Substanz annahm.

\textbf{Eine Gestalt mit zwei Armen, zwei Beinen und einem Kopf, aufrecht stehend;}

\textbf{das Tier Homo Sapiens, die Gestalt eines Menschen}

\textbf{Sie leuchtete heller und heller, als Harry seine ganze Kraft in den Zauberspruch steckte, sie glühte heller als der Sonnenuntergang},

die Auroren und Professor Quirrell schirmten ihre Augen schockiert ab -

\emph{Und eines Tages, wenn die Nachkommen der Menschheit sich von Stern zu Stern ausgebreitet haben, werden sie den Kindern nichts über die Geschichte der Alten Erde erzählen, bis sie alt genug sind, es zu ertragen; und wenn sie es erfahren, werden sie weinen, wenn sie hören, dass so etwas wie der Tod jemals existiert hat!}

Die Gestalt eines Menschen leuchtete jetzt heller als die Mittagssonne, so strahlend, dass Harry ihre Wärme auf seiner Haut spüren konnte; und Harry sandte seinen ganzen Trotz gegen den Schatten des Todes aus und öffnete alle Schleusen in seinem Inneren, um diese helle Gestalt noch heller und noch heller leuchten zu lassen.

\emph{Du bist nicht unbesiegbar, und eines Tages wird die menschliche Spezies dich beenden.}

\emph{Ich werde dich beenden, wenn ich kann, durch die Macht des Geistes und der Magie und der Wissenschaft.

Ich werde mich nicht in Angst vor dem Tod ducken, nicht solange ich eine Chance habe zu gewinnen. Ich werde nicht zulassen, dass der Tod mich berührt, ich werde nicht zulassen, dass der Tod die berührt, die ich liebe.

\textbf{Und selbst wenn du mich beendest, bevor ich dich beende, wird ein anderer meinen Platz einnehmen, und noch einer, bis die Wunde in der Welt endlich geheilt ist…}}

Harry senkte seinen Zauberstab, und die helle Gestalt eines Menschen verblasste. Langsam atmete er aus. Wie das Erwachen aus einem Traum, wie das Öffnen der Augen nach dem Schlaf, bewegte sich Harrys Blick von dem Käfig weg, er sah sich um und sah, dass alle ihn anstarrten.

Albus Dumbledore starrte ihn an.

Professor Quirrell starrte ihn an.

Das Auroren-Trio starrte ihn an.

Sie alle starrten ihn an, als hätten sie gerade gesehen, wie er einen Dementor vernichtet hatte.

\emph{Der zerfledderte Umhang lag leer in dem Käfig.}

