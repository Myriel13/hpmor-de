

\hypertarget{das-stanford-gefuxe4ngnis-experiment-teil-10}{% \section{60. Das Stanford-Gefängnis-Experiment, Teil 10}\label{das-stanford-gefuxe4ngnis-experiment-teil-10}}

\textbf{\uline{Das Stanford-Gefängnis-Experiment, Teil 10}}

\hfill\break „Aufwachen.“

Harrys Augen flogen auf, als er mit einem erstickten Keuchen erwachte, ein ruckartiger Start seines liegenden Körpers. Er konnte sich an keine Träume erinnern, vielleicht war sein Gehirn zu erschöpft gewesen, um zu träumen, es schien, als hätte er nur die Augen geschlossen und dann einen Moment später dieses Wort gesprochen gehört.

„Du musst erwachen“, sagte die Stimme von Quirinus Quirrell. „Ich habe dir so viel Zeit gegeben, wie ich konnte, aber es wäre klug, wenigstens einen Einsatz deines Zeitumkehrers zu reservieren. Bald müssen wir vier Stunden rückwärts zu Marys Restaurant gehen und in jeder Hinsicht so aussehen, als hätten wir heute nichts Interessantes getan. Ich wollte vorher noch mit dir sprechen.“

Harry setzte sich langsam inmitten der Dunkelheit auf. Sein Körper schmerzte, und zwar nicht nur an den Stellen, an denen er auf dem harten Beton gelegen hatte. Bilder purzelten in seiner Erinnerung übereinander, alles, was sein bewusstloses Gehirn zu müde gewesen war, um es in einen richtigen Alptraum zu entladen. Zwölf schreckliche Wunden, die einen metallenen Korridor hinunterschwebten und das Metall um sie herum trübten, das Licht gedämpft und die Temperatur fallend, während die Leere versuchte, alles Leben aus der Welt zu saugen - ….

Kreideweiße Haut, Eine Metalltür -\\ Eine Frauenstimme -\\ \emph{Nein, ich wollte das nicht, bitte stirb nicht -\\ Ich kann mich nicht mehr an die Namen meiner Kinder erinnern -\\ Geh nicht, nimm es nicht weg, nicht, nicht, nicht} -

„Was war das für ein Ort? „ sagte Harry heiser, mit einer Stimme, die aus seiner Kehle gepresst wurde wie Wasser, das durch ein zu dünnes Rohr gepresst wurde, in der Dunkelheit klang sie fast so brüchig wie Bellatrix Blacks Stimme gewesen war. „Was war das für ein Ort? Das war kein Gefängnis, das war die HÖLLE!“

„Hölle?“, sagte die ruhige Stimme des Verteidigungsprofessors. „Du meinst die christliche Bestrafungsphantasie? Ich nehme an, es gibt eine Ähnlichkeit.“

„Wie -“ Harrys Stimme war blockiert, etwas Riesiges steckte in seiner Kehle. „Wie - wie konnten sie -“

\emph{Menschen hatten diesen Ort gebaut, jemand hatte Askaban erschaffen, sie hatten es absichtlich gemacht, sie hatten es absichtlich getan, diese Frau, sie hatte Kinder bekommen, Kinder, an die sie sich nicht erinnern würde, irgendein Richter hatte} \emph{entschieden, dass ihr das passieren sollte, jemand musste sie in diese Zelle zerren und die Tür verriegeln, während sie schrie, jemand fütterte sie} \emph{jeden Tag und ging weg, ohne sie rauszulassen -}

„\textbf{WIE KÖNNTEN MENSCHEN DAS TUN - ?}“

„Warum sollten sie das nicht tun?“, sagte der Verteidigungsprofessor.\\ Ein blassblaues Licht erhellte das Lagerhaus und zeigte eine hohe, höhlenartige Betondecke und einen staubigen Betonboden; und Professor Quirrell, der in einiger Entfernung von Harry saß und sich mit dem Rücken an eine gestrichene Wand lehnte; das blassblaue Licht verwandelte die Wände in Gletscherflächen, den Staub auf dem Boden in gesprenkelten Schnee, und der Mann selbst war zu einer Eisskulptur geworden, eingehüllt in Dunkelheit, wo seine schwarzen Roben über ihm lagen.\\ „Was nützen ihnen die Gefangenen von Askaban?“

Harrys Mund öffnete sich zu einem Krächzen. Es kamen keine Worte heraus.

Ein schwaches Lächeln zuckte auf den Lippen des Verteidigungsprofessors.\\ „Wissen Sie, Mr. Potter, wenn Er, der nicht genannt werden darf, gekommen wäre, um über das magische Britannien zu herrschen, und einen Ort wie Askaban gebaut hätte, dann hätte er ihn gebaut, weil es ihm Spaß macht, seine Feinde leiden zu sehen. Und wenn er stattdessen anfing, ihr Leiden geschmacklos zu finden, würde er am nächsten Tag den Abriss von Askaban anordnen. Was diejenigen angeht, die Askaban erbaut haben und diejenigen, die es nicht abreißen, während sie hochtrabende Predigten halten und sich einbilden, keine Schurken zu sein... nun, Mr. Potter, ich denke, wenn ich die Wahl hätte, mit ihnen Tee zu trinken oder mit Du-weißt-schon-wem Tee zu trinken, würde ich meine Empfindsamkeiten weniger durch den Dunklen Lord beleidigt sehen.“

„Ich verstehe nicht“, sagte Harry, seine Stimme zitterte, er hatte von dem klassischen Experiment über die Psychologie der Gefängnisse gelesen, von den gewöhnlichen College-Studenten, die sadistisch geworden waren, sobald man ihnen die Rolle von Gefängniswärtern zugewiesen hatte; Erst jetzt wurde ihm klar, dass das Experiment nicht die richtige Frage untersucht hatte, die wichtigste Frage, sie hatten nicht die Schlüsselpersonen betrachtet, nicht die Gefängniswärter, sondern alle anderen, „Ich verstehe es wirklich nicht, Professor Quirrell, wie können die Menschen einfach zusehen und so etwas geschehen lassen, warum tut das Land des magischen Britannien das -“ Harrys Stimme brach ab.

Die Augen des Verteidigungsprofessors schienen in dem blassblauen Licht die gleiche Farbe zu haben wie immer, denn dieses Licht hatte die gleiche Farbe wie die Iris von Quirinus Quirrell, diese nie tauenden Eissplitter.

„Willkommen, Mr. Potter, zu Ihrer ersten Begegnung mit den Realitäten der Politik. Was haben die erbärmlichen Kreaturen in Askaban irgendeiner Fraktion zu bieten? Wem würde es nützen, ihnen zu helfen? Ein Politiker, der sich offen auf ihre Seite stellt, würde sich mit Kriminellen assoziieren, mit Schwäche, mit widerwärtigen Dingen, an die die Menschen lieber nicht denken wollen. Alternativ könnte der Politiker seine Macht und Grausamkeit demonstrieren, indem er längere Strafen fordert; um Stärke zu demonstrieren, muss man schließlich ein Opfer unter sich zerquetschen. Und die Bevölkerung applaudiert, denn es ist ihr Instinkt, den Sieger zu unterstützen.“ Ein kaltes, amüsiertes Lachen. „Sehen Sie, Mr. Potter, niemand glaubt je so recht daran, dass er nach Askaban kommt, also sehen sie für sich selbst keinen Schaden darin. Was das angeht, was sie anderen antun... Ich nehme an, man hat Ihnen mal gesagt, dass die Menschen sich um so etwas kümmern? Das ist eine Lüge, Mr. Potter, die Leute interessiert das nicht im Geringsten. Hätten Sie nicht eine äußerst behütete Kindheit gehabt, hätten Sie das längst gemerkt. Trösten Sie sich mit Folgendem: Diejenigen, die jetzt in Askaban gefangen sind, haben dieselben Zaubereiminister gewählt, die versprochen haben, ihre Zellen näher an die Dementoren zu verlegen. Ich gebe zu, Mr. Potter, dass ich wenig Hoffnung für die Demokratie als effektive Regierungsform sehe, aber ich bewundere die Poesie, wie sie ihre Opfer zu Mitschuldigen an ihrer eigenen Zerstörung macht.“

Harrys vor kurzem noch kohärentes Selbst drohte wieder in Fragmente zu zerfallen, die Worte fielen wie Hammerschläge auf sein Bewusstsein und trieben ihn Schritt für Schritt zurück über die Stelle wo ein riesiger Abgrund lauerte; und er versuchte, etwas zu finden, um sich zu retten, irgendeine kluge Erwiderung, die die Worte widerlegen würde, aber sie kam nicht.

Der Verteidigungsprofessor beobachtete Harry, und in seinem Blick spiegelte sich mehr Neugierde als Befehl. „Es ist sehr einfach, Mr. Potter zu verstehen wie Askaban gebaut wurde und wie es immer noch ist. Die Menschen kümmern sich um das, was sie selbst zu erleiden oder zu gewinnen erwarten; und solange sie nicht erwarten, dass es auf sie selbst zurückfällt, ist ihre Grausamkeit und Rücksichtslosigkeit grenzenlos. Alle anderen Zauberer dieses Landes sind im Innern nicht anders als der, der über sie herrschen wollte, Du-weißt-schon-wer; es fehlt ihnen nur seine Macht und seine... Offenheit.“

Die Hände des Jungen waren zu Fäusten geballt, so fest, dass die Nägel in seine Handfläche schnitten, wenn seine Finger weiß oder sein Gesicht blass gewesen wären, hätte man das nicht sehen können, denn das schwache blaue Licht warf alles in Eis oder Schatten.

„Sie haben mir einmal angeboten, mich zu unterstützen, wenn mein Ehrgeiz darin bestünde, der nächste Dunkle Lord zu werden. Ist das der Grund, Professor?“

Der Verteidigungsprofessor neigte den Kopf, ein dünnes Lächeln auf seinen Lippen. „Lerne alles, was ich dir beibringen kann und du wirst mit der Zeit dieses Land regieren. Dann könnest du das Gefängnis einreißen, das die Demokratie geschaffen hat, wenn du findest, dass Askaban immer noch deine Empfindlichkeiten beleidigt. Ob es dir gefällt oder nicht, Mr. Potter, du hast heute gesehen, dass dein eigener Wille mit dem Willen der Bevölkerung dieses Landes kollidiert, und dass du nicht dein Haupt beugen und sich dieser Entscheidung unterwerfen wirst wenn das geschieht. Für sie, ob sie es wissen oder nicht, und ob du es anerkennst oder nicht, \emph{bist du also ihr nächster Dunkler Lord.}“

In dem monochromen Licht wirkten der Junge und der Verteidigungsprofessor beide wie unbewegliche Eisskulpturen, die Iris ihrer Augen auf ähnliche Farben reduziert, die in diesem Licht ganz gleich aussahen.

Harry starrte direkt in diese blassen Augen. All die lange unterdrückten Fragen, die, von denen er sich gesagt hatte, dass er sie bis Mitte Mai aufschieben würde.

\emph{Das war eine Lüge gewesen,} wusste Harry jetzt, \emph{eine Selbsttäuschung, er hatte geschwiegen aus Angst vor dem, was er hören könnte.} Und jetzt kam ihm alles auf einmal über die Lippen.

„An unserem ersten Schultag hast du versucht, meine Klassenkameraden davon zu überzeugen, dass ich ein Mörder bin.“

„Das bist du.“. In amüsiertem Tonfall ging es weiter „Aber wenn Ihre Frage lautet, warum ich ihnen das gesagt habe, Mr. Potter, dann lautet die Antwort, dass du in der Zweideutigkeit einen großartigen Verbündeten auf deinem Weg zur Macht finden wirst. Gebe an einem Tag ein Zeichen für Slytherin, und widerspreche dem am nächsten Tag mit einem Zeichen für Gryffindor; und die Slytherins werden in die Lage versetzt, zu glauben, was sie wollen, während die Gryffindors sich dazu durchringen, dich ebenfalls zu unterstützen. Solange Ungewissheit herrscht, können die Leute glauben, was immer zu ihrem eigenen Vorteil zu sein scheint. Und solange du stark erscheinst, solange du zu gewinnen scheinst, werden ihre Instinkte ihnen sagen, dass ihr Vorteil bei dir liegt. Geh immer im Schatten, und sowohl Licht als auch Dunkelheit werden dir folgen.“

„Und“, sagte der Junge mit ruhiger Stimme, „was genau wollen Sie von all dem?"

Professor Quirrell hatte sich weiter nach hinten an die Wand gelehnt und sein Gesicht in den Schatten geworfen, seine Augen verwandelten sich von blassem Eis in dunkle Gruben wie die seiner Schlangengestalt.

„Ich wünsche mir, dass Großbritannien unter einem starken Anführer groß wird; das ist mein Wunsch. Was meine Gründe dafür angeht“, Professor Quirrell lächelte ohne Heiterkeit, „ich denke, sie sollen meine eigenen bleiben.“

„Das Gefühl des Untergangs, das ich in Ihrer Nähe spüre.“\\ \emph{Die Worte fielen ihm immer schwerer, denn das Thema tanzte immer näher an etwas Schreckliches und Verbotenes heran.}\\ „Sie wussten immer, was es bedeutet.“

„Ich hatte mehrere Vermutungen“, sagte Professor Quirrell, seine Miene war unleserlich. „Und ich werde jetzt nicht sagen, was ich alles vermutet habe. Aber so viel will ich dir sagen: Es ist dein Verhängnis, das sich anbahnt, wenn wir uns nähern, nicht meines.“

Ausnahmsweise schaffte es Harrys Gehirn, dies als fragwürdige Behauptung und mögliche Lüge zu markieren, anstatt alles zu glauben, was es hörte.

„Warum verwandeln Sie sich manchmal in einen Zombie?“

„Persönliche Gründe“, sagte Professor Quirrell ohne jeglichen Humor in der Stimme.

„Was war Ihr Hintergedanke bei der Rettung von Bellatrix?“

Es herrschte eine kurze Stille, in der Harry sich bemühte, seine Atmung zu kontrollieren, sie ruhig zu halten. Schließlich zuckte der Verteidigungsprofessor mit den Schultern, als ob es keine Rolle spielen würde. „Ich habe es dir so gut wie gesagt, Mr. Potter. Ich habe dir alles gesagt, was du brauchst, um die Antwort abzuleiten, wenn du reif genug gewesen wärst, diese erste offensichtliche Frage zu bedenken. Bellatrix Black war die mächtigste Dienerin des Dunklen Lords, ihre Loyalität die sicherste; sie war die einzige Person, der am ehesten ein Teil des verlorenen Wissens von Slytherin anvertraut werden konnte, das dir hätte gehören sollen.“

Langsam kroch die Wut über Harry, langsam der Zorn, etwas Schreckliches begann sein Blut zu kochen, in wenigen Augenblicken würde er etwas sagen, was er wirklich nicht sagen sollte, während die beiden allein in einer verlassenen Lagerhalle waren -

„Aber sie war unschuldig“, sagte der Verteidigungsprofessor. „Und das Ausmaß, in dem ihr alle Wahlmöglichkeiten genommen wurden, so dass sie nie die Chance hatte, für ihre eigenen Fehler zu leiden... das kam mir übertrieben vor, Mr. Potter. Wenn sie Ihnen nichts Nützliches erzählt -“ Der Verteidigungsprofessor zuckte wieder leicht mit den Schultern. „Ich werde die Arbeit des Tages nicht als Verschwendung betrachten.“

„Wie altruistisch von Ihnen“, sagte Harry kalt. „Wenn also alle Zauberer im Inneren wie Du-weißt-schon-wer sind, sind Sie dann eine Ausnahme?“

Die Augen des Verteidigungsprofessors lagen immer noch im Schatten, dunkle Gruben, die man nicht treffen konnte.

„Nennen Sie es eine Laune, Mr. Potter. Es hat mich manchmal amüsiert, die Rolle eines Helden zu spielen. Wer weiß, ob Du-weißt-schon-wer nicht dasselbe sagen würde.“ Harry öffnete ein letztes Mal den Mund - und stellte fest, dass er es nicht sagen konnte, er konnte die letzte Frage nicht stellen, die letzte und wichtigste Frage, er konnte die Worte nicht herausbringen. Obwohl eine solche Verweigerung einem Rationalisten verboten war, obwohl er jemals die Litanei von Tarski oder die Litanei von Gendlin rezitiert oder geschworen hatte, dass alles, was von der Wahrheit zerstört werden könnte, es sein sollte, konnte er sich in diesem einen Moment nicht dazu durchringen, seine letzte Frage laut auszusprechen. Obwohl er wusste, dass er falsch dachte, obwohl er wusste, dass er besser sein sollte als das, konnte er es nicht sagen.

„Jetzt bin ich an der Reihe, dich zu befragen.“ Professor Quirrells Rücken richtete sich auf, wo er sich gegen die Gletscherwand aus gestrichenem Beton gelehnt hatte. „Ich habe mich gefragt, Mr. Potter, ob du etwas dazu zu sagen hast, dass du mich fast umgebracht und unser gemeinsames Unterfangen ruiniert hast. Man hat mir zu verstehen gegeben, dass eine Entschuldigung in solchen Fällen als Zeichen des Respekts gilt. Aber du hast mir keine angeboten. Ist es nur so, dass du noch nicht dazu gekommen bist, Mr. Potter?“

Der Ton war ruhig, die leise Kante so fein und scharf, dass sie einen ganz durchschneiden würde, bevor man merkte, dass man ermordet wurde. Und Harry sah den Verteidigungsprofessor einfach mit kühlen Augen an, die vor nichts zurückschrecken würden; nicht einmal vor dem Tod. Er war nicht mehr in Askaban, hatte keine Angst mehr vor dem Teil von sich, der furchtlos war; und der solide Edelstein, der Harry war, hatte sich gedreht, um der Belastung zu begegnen, drehte sich sanft von einer Facette zur anderen, von Licht zu Dunkelheit, warm zu kalt.

\emph{Ein kalkulierter Trick seinerseits, um mir ein schlechtes Gewissen zu machen, mich in eine Lage zu bringen, in der ich mich unterwerfen muss? Echte Emotionen von seiner Seite?}

„Ich verstehe“, sagte Professor Quirrell. „Ich nehme an, das beantwortet -“

„Nein“, sagte der Junge mit kühler, gefasster Stimme, „so einfach können Sie das Gespräch nicht gestalten, Professor. Ich habe beträchtliche Anstrengungen unternommen, um Sie zu beschützen und sicher aus Askaban herauszuholen, nachdem ich dachte, Sie hätten versucht, einen Polizisten zu töten. Dazu gehörte auch, 12 Dementoren ohne einen Patronus-Zauber zu bekämpfen. Ich frage mich, wenn ich mich entschuldigt hätte, als Sie es verlangt haben, hätten Sie sich dann im Gegenzug bedankt? Oder gehe ich recht in der Annahme, dass es meine Unterwerfung war, die Sie da gefordert haben, und nicht nur meinen Respekt?“

Es gab eine Pause, und dann kam Professor Quirrells Stimme als Antwort, offen eisig vor nicht mehr verhüllter Gefahr. „Es scheint, als könnten Sie es immer noch nicht über sich bringen, zu verlieren, Mr. Potter.“

Dunkelheit starrte aus Harrys Augen, ohne mit der Wimper zu zucken, der Verteidigungsprofessor selbst reduzierte sich darin zu einem sterblichen Ding.

„Oh, und überlegen Sie jetzt, ob Sie so tun sollen, als würden Sie gegen mich verlieren und sich vor meinem eigenen Zorn zu schützen, um Ihre eigenen Pläne zu wahren? Ist Ihnen der Gedanke an eine kalkulierte falsche Entschuldigung auch nur in den Sinn gekommen? Mir auch nicht, Professor Quirrell.“

Der Verteidigungsprofessor lachte, leise und humorlos, leerer als die Leere zwischen den Sternen, gefährlich wie jedes mit harter Strahlung gefüllte Vakuum. „Nein, Mr. Potter, du hast die Lektion nicht gelernt, ganz und gar nicht.“

„Ich habe schon oft daran gedacht, in Askaban zu verlieren“, sagte der Junge mit ruhiger Stimme. „Dass ich einfach aufgeben und mich den Auroren ausliefern sollte. Zu verlieren wäre das Vernünftigste, was ich tun könnte. Ich hörte ihre Stimme in meinem Kopf, die mir das sagte; und ich hätte es getan, wenn ich allein dort gewesen wäre. Aber ich konnte mich nicht dazu durchringen, Sie zu verlieren.“

Dann herrschte eine Zeit lang Schweigen, als ob selbst der Professor der Verteidigung nicht recht wüßte, was er darauf antworten sollte. „Ich bin neugierig“, sagte Professor Quirrell schließlich. „Wofür genau sollte ich mich deiner Meinung nach entschuldigen? Ich habe dir ausdrückliche Anweisungen für den Fall eines Kampfes gegeben. Du solltest am Boden bleiben, aus dem Weg gehen und keine Magie wirken. Du hast diese Anweisungen missachtet und die Mission zu Fall gebracht.“

„Ich habe keine Entscheidung getroffen“, sagte der Junge gleichmütig, „es gab keine Wahl, nur den Wunsch, dass der Auror nicht sterben sollte, und mein Patronus war zur Stelle. Damit dieser Wunsch nicht in Erfüllung geht, hätten Sie mich warnen müssen, dass Sie mit einem Tötungsfluch bluffen könnten. Standardmäßig gehe ich davon aus, dass, wenn man den Zauberstab auf jemanden richtet und Avada Kedavra sagt, es ist, weil man ihn tot sehen will. Sollte das nicht die erste Regel zum Schutz vor unverzeihlichen Flüchen sein?“

„Regeln sind für Duelle“, sagte der Verteidigungsprofessor. Etwas von der Kälte war in seine Stimme zurückgekehrt. „Und Duellieren ist ein Sport, kein Zweig der Kampfmagie. In einem echten Kampf ist ein Fluch, der nicht geblockt werden kann und dem man ausweichen muss, eine unverzichtbare Taktik. Ich hätte gedacht, dass dies für dich offensichtlich ist, aber es scheint, dass ich deinen Intellekt falsch eingeschätzt habe.“

„Es scheint mir auch unklug zu sein“, sagte der Junge und fuhr fort, als hätte der andere nicht gesprochen, „mir nicht zu sagen, dass ein Zauber, den ich auf Sie anwende, uns beide töten könnte. Was wäre, wenn Ihnen ein Missgeschick passiert wäre und ich einen Innervate- oder Schwebe-Zauber versucht hätte? Diese Unwissenheit, die Sie aus Gründen, die ich nicht erraten kann, zugelassen haben, hat auch eine gewisse Rolle in dieser Katastrophe gespielt.“

Wieder herrschte Schweigen. Die Augen des Verteidigungsprofessors hatten sich verengt, und es lag ein leicht verwirrter Ausdruck auf seinem Gesicht, als ob er sich in einer völlig ungewohnten Situation befände; und doch sprach der Mann kein Wort.

„Nun“, sagte der Junge. Seine Augen waren nicht von denen des Verteidigungsprofessors gewichen. „Ich bedaure es sehr, Ihnen wehgetan zu haben, Professor. Aber ich glaube nicht, dass die Situation verlangt, dass ich mich Ihnen unterwerfe. Ich habe das Konzept der Entschuldigung nie wirklich verstanden, noch weniger, wie es sich auf eine Situation wie diese bezieht; wenn Sie mein Bedauern haben, aber nicht meine Unterwerfung, zählt das als Entschuldigung?“

Wieder dieses kalte, kalte Lachen, dunkler als die Leere zwischen den Sternen. „Ich weiß es nicht“, sagte der Verteidigungsprofessor, „auch ich habe das Konzept der Entschuldigung nie verstanden. Dieser Trick wäre zwischen uns zwecklos, wie es scheint, da wir beide wissen, dass es eine Lüge ist. Lass uns also nicht mehr darüber sprechen. Die Schulden werden mit der Zeit zwischen uns beglichen werden.“

Eine Zeit lang herrschte Schweigen.

„Übrigens“, sagte der Junge. „Hermine Granger hätte Askaban nie gebaut, egal, wer dort hineingesteckt werden sollte. Und sie würde eher sterben, bevor sie einem Unschuldigen etwas antun würde. Ich habe das nur erwähnt, weil Sie vorhin gesagt haben, dass alle Zauberer innerlich wie Du-weißt-schon-wer sind, und das ist einfach nur falsch, weil es eine einfache Tatsache ist. Das wäre mir früher aufgefallen, wenn ich nicht“, der Junge lächelte kurz grimmig, „gestresst gewesen wäre.“

Die Augen des Verteidigungsprofessors waren halblidig, sein Ausdruck distanziert. „Das Innere der Menschen ist nicht immer wie ihr Äußeres, Mr. Potter. Vielleicht möchte sie einfach, dass andere sie für ein gutes Mädchen halten. Sie kann den Patronus-Zauber nicht benutzen -“

„Hah“, sagte der Junge; sein Lächeln schien jetzt echter, wärmer zu sein. „Sie hat aus genau demselben Grund Probleme wie ich. Es ist genug Licht in ihr, um Dementoren zu zerstören, da bin ich mir sicher. Sie würde sich selbst nicht davon abhalten können, Dementoren zu zerstören, selbst wenn es sie das eigene Leben kosten würde...“ Der Junge brach ab, und dann setzte seine Stimme wieder ein. „Ich bin vielleicht nicht so ein guter Mensch, aber es gibt sie, und sie ist eine von ihnen.„

\emph{Trocken}. „Sie ist jung, und eine Show der Freundlichkeit zu machen, kostet sie wenig.“

Daraufhin gab es eine Pause.

Dann sagte der Junge: „Professor, ich muss Sie fragen, wenn Sie etwas ganz dunkel und düster sehen, kommt es Ihnen dann nie in den Sinn, zu versuchen, es irgendwie zu verbessern? Zum Beispiel, \emph{ja,} irgendetwas läuft in den Köpfen der Leute furchtbar schief, das sie denken lässt, dass es großartig ist, Verbrecher zu foltern, aber das bedeutet nicht, dass sie im Inneren wirklich böse sind; und vielleicht, wenn man ihnen die richtigen Dinge beibringen würde, ihnen zeigen würde, was sie falsch machen, könnte man sie ändern -“

Dann lachte Professor Quirrell, und zwar nicht mit der Leere von vorher.\\ „Ah, Mr. Potter, manchmal vergesse ich wirklich, wie jung du bist. Eher könnte man die Farbe des Himmels ändern.“\\ Ein weiteres Kichern, dieses Mal kälter.\\ „Und der Grund, warum es dir so leicht fällt, solchen Narren zu verzeihen und gut von ihnen zu denken, Mr. Potter, ist, dass du selbst nicht schwer verletzt worden bist. Du wirst weniger zärtlich über gewöhnliche Idioten denken, wenn die Torheit der Narren dich das erste Mal etwas Teures kostet. Vielleicht nicht hundert Galleonen aus deiner eigenen Tasche aber den qualvollen Tod von einem Freund.“

Der Verteidigungsprofessor lächelte dünn. Er holte eine Taschenuhr aus seiner Robe und schaute sie an. „Lass uns jetzt gehen, wenn es zwischen uns nichts mehr zu sagen gibt.“

„Sie haben keine Fragen zu den unmöglichen Dingen, die ich getan habe, um uns aus Askaban herauszuholen?“

„Nein“, sagte der Verteidigungsprofessor. „Ich glaube, die meisten davon habe ich bereits gelöst. Was den Rest betrifft, so ist es zu selten, dass ich eine Person finde, die ich nicht sofort durchschaue, sei sie Freund oder Feind. Ich werde die Rätsel um die Ereignisse zu gegebener Zeit selbst enträtseln.“

Der Verteidigungsprofessor stemmte sich hoch, stützte sich mit beiden Händen an der Wand ab und erhob sich auf die Füße, geschmeidig, wenn auch zu langsam. Der Junge tat dasselbe, wenn auch weniger anmutig. Und der Junge platzte mit der letzten, furchtbarsten Frage heraus, die er vorher nicht hatte stellen können; als ob sie laut auszusprechen sie wahr machen würde, und als ob sie nicht schon ungemein offensichtlich wäre.

\emph{„Warum bin ich nicht wie die anderen Kinder in meinem Alter?“}

...\\ In einer verlassenen Seitenstraße der Winkelgasse, in der man an den Rändern der Ziegelsteinstraße und den sie umgebenden leeren Ziegelsteinseiten Fetzen von nicht beseitigtem Müll sehen konnte, zusammen mit verstreutem Schmutz und anderen Anzeichen von Vernachlässigung, erschienen ein alter Zauberer und sein Phönix in die Existenz. Der Zauberer griff bereits in seiner Robe nach seiner Sanduhr, als seine Augen aus Gewohnheit auf eine zufällige Stelle zwischen der Straße und der Mauer sprangen, um sie sich einzuprägen - und der alte Zauberer blinzelte überrascht; an dieser Stelle lag ein Fetzen Pergament. Ein Stirnrunzeln überzog Albus Dumbledores Gesicht, als er einen Schritt nach vorne machte, den zerknitterten Fetzen nahm und ihn entfaltete.

Darauf stand nur das Wort „NEIN“ und sonst nichts.

Langsam ließ der Zauberer es aus seinen Fingern flattern. Abwesend griff er nach dem nächstgelegenen Pergamentfetzen, der dem soeben genommenen auffallend ähnlich sah; er berührte ihn mit seinem Zauberstab, und einen Augenblick später stand darauf dasselbe Wort „NEIN“, in derselben Handschrift, die seine eigene war.

Der alte Zauberer hatte sich vorgenommen, drei Stunden zurückzugehen, bis zu der Zeit, als Harry Potter zum ersten Mal in der Winkelgasse ankam. Er hatte bereits mit seinen Instrumenten beobachtet, wie der Junge Hogwarts verließ, und das konnte nicht mehr rückgängig gemacht werden (sein einziger Versuch, seine eigenen Instrumente zu täuschen und so die Zeit zu kontrollieren, ohne ihr Erscheinungsbild für sich selbst zu verändern, hatte in einem ausreichend großem Desaster geendet, um ihn davon zu überzeugen, nie wieder solche Tricks zu versuchen). Er hatte gehofft, den Jungen so schnell wie möglich nach seiner Ankunft wiederzufinden und ihn an einen anderen sicheren Ort zu bringen, wenn nicht sogar nach Hogwarts (denn seine Instrumente hatten die Rückkehr des Jungen nicht angezeigt). Aber jetzt -

„Ein Paradoxon, wenn ich ihn sofort nach seiner Ankunft in der Winkelgasse zurückhole?“ murmelte der alte Zauberer zu sich selbst. „Vielleicht haben sie ihren Plan, Askaban auszurauben, erst in Gang gesetzt, nachdem sie seine Ankunft hier bestätigt hatten... oder aber... vielleicht...“

….\\ Gestrichener Beton, harter Boden und ferne Decken, zwei Gestalten, die sich gegenüberstanden. Ein Wesen, das die Gestalt eines Mannes in den späten Dreißigern trug und bereits eine Glatze hatte, und ein anderer Geist, der die Form eines elfjährigen Jungen mit einer Narbe auf der Stirn trug. Eis und Schatten, blassblaues Licht.

„Ich weiß es nicht“, sagte der Mann.

Der Junge schaute ihn nur an. Und sagte dann: „Ach, wirklich?“

„Wahrhaftig“, sagte der Mann. „Ich weiß nichts, und über meine Vermutungen werde ich nicht sprechen. Doch so viel will ich sagen -“

