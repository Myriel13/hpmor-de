

\hypertarget{das-stanford-gefuxe4ngnis-experiment-nachspiel}{% \section{63. Das-Stanford-Gefängnis-Experiment, Nachspiel}\label{das-stanford-gefuxe4ngnis-experiment-nachspiel}}

\textbf{\uline{Das-Stanford-Gefängnis-Experiment, Nachspiel}}

\hfill\break \textbf{Hermine Granger:}\\ Sie fing gerade an, ihre Bücher zu schließen und ihre Hausaufgaben wegzulegen, um sich auf den Schlaf vorzubereiten, während Padma und Mandy ihre eigenen Bücher gegenüber von ihr aufstapelten, als Harry Potter den Ravenclaw-Gemeinschaftsraum betrat; und erst da wurde ihr klar, dass sie ihn seit dem Frühstück überhaupt nicht mehr gesehen hatte. Diese Erkenntnis wurde schnell von einer viel erschreckenderen verdrängt. Da war ein gold-rot geflügeltes Wesen auf Harrys Schulter, ein leuchtender Feuervogel. Und Harry sah traurig und abgekämpft und wirklich müde aus, als wäre der Phönix das Einzige, was ihn auf den Beinen hielt, aber er hatte immer noch eine Wärme an sich, wenn man die Augen sah hätte man meinen können, den Schulleiter vor sich zu haben, das war der Eindruck, der Hermine durch den Kopf ging, auch wenn es keinen Sinn ergab.

Harry Potter stapfte durch den Ravenclaw-Gemeinschaftsraum, vorbei an Sofas voller starrender Mädchen, vorbei an Kartenspielkreisen starrender Jungen, auf sie zu. Theoretisch redete sie noch nicht mit Harry Potter, seine Woche war erst morgen zu Ende, aber was auch immer vor sich ging, war eindeutig viel wichtiger als das -

„Fawkes“, sagte Harry, gerade als sie den Mund öffnete, „das Mädchen da drüben ist Hermine Granger, sie redet gerade nicht mit mir, weil ich ein Idiot bin, aber wenn du dich an die Schulter einer guten Person hängen willst, ist sie besser als ich.“

So viel Erschöpfung und Schmerz in Harry Potters Stimme - aber bevor sie sich überlegen konnte, was sie dagegen tun sollte, war der Phönix von Harrys Schulter weggeglitten wie ein Feuer, das im Schnelldurchlauf an einem Streichholz hochkriecht, und flog auf sie zu; da war ein Phönix, der vor ihr flog und sie mit Augen aus Licht und Flamme anstarrte.

„\emph{Caw}?“, fragte der Phönix.

Hermine starrte ihn an, sie fühlte sich, als stünde sie vor einer Frage in einem Test, für den sie vergessen hatte zu lernen, die eine, wichtigste Frage, für die sie ihr ganzes Leben lang nicht gelernt hatte, sie konnte nichts sagen. „Ich bin -“, sagte sie. „Ich bin erst zwölf, ich habe noch nichts gemacht -“

Der Phönix glitt nur sanft herum, drehte sich um eine Flügelspitze wie das Wesen aus Licht und Luft, das er war, und schwebte zurück zu Harry Potters Schulter, wo er sich ganz fest niederließ.

„Du dummer Junge“, sagte Padma ihr gegenüber und sah aus, als wüsste sie nicht, ob sie lachen oder eine Grimasse schneiden sollte, „Phönixe sind nicht für kluge Mädchen, die ihre Hausaufgaben machen, sondern für Idioten, die sich direkt auf fünf ältere Slytherins stürzen. Es gibt einen Grund, warum die Farben von Gryffindor rot und gold sind, weißt du.“

Im Ravenclaw-Gemeinschaftsraum gab es viel freundliches Gelächter. Hermine gehörte nicht zu den Lachenden. Und Harry auch nicht. Harry hatte eine Hand über sein Gesicht gelegt.

„Sag Hermine, dass es mir leid tut“, sagte er zu Padma, wobei seine Stimme fast zu einem Flüstern sank. „Sag ihr, dass ich vergessen habe, dass Phönixe Tiere sind, sie verstehen nichts von Zeit und Planung, sie verstehen nicht, dass es Menschen gibt, die später gute Dinge tun werden - ich bin mir nicht sicher, ob sie wirklich die Vorstellung verstehen, dass es etwas gibt, das eine Person ist, sie sehen nur, was Menschen tun. Fawkes weiß nicht, was zwölf bedeutet. Sag Hermine, es tut mir leid - ich hätte es nicht tun sollen - es geht einfach alles schief, nicht wahr?“

Harry wandte sich zum Gehen, den Phönix immer noch auf der Schulter, und stapfte langsam auf die Treppe zu, die zu seinem Schlafsaal hinaufführte. Und Hermine konnte es nicht dabei belassen, sie konnte es einfach nicht dabei belassen. Sie wusste nicht, ob es ihr Konkurrenzkampf mit Harry war oder etwas anderes. Sie konnte es einfach nicht dabei belassen, dass der Phönix sich von ihr abwandte. Sie musste - ihr Verstand durchforstete verzweifelt die Gesamtheit ihres ausgezeichneten Gedächtnisses und fand nur eine Sache - „Ich wollte vor den Dementor rennen, um zu versuchen, Harry zu retten!“, rief sie ein wenig verzweifelt dem rot-goldenen Vogel zu. „Ich meine, ich bin tatsächlich losgerannt und alles! Das war doch dumm und mutig, oder?“

Mit einem trällernden Schrei stürzte sich der Phönix wieder von Harrys Schulter, zurück zu ihr, wie eine sich ausbreitende Feuersbrunst, er umkreiste sie dreimal, als wäre sie das Zentrum eines Infernos, und für einen kurzen Moment streifte sein Flügel ihre Wange, bevor der Phönix wieder zu Harry aufstieg.

Im Ravenclaw-Gemeinschaftsraum herrschte Schweigen. „Ich hab's ja gesagt“, sagte Harry laut, und dann begann er, die Treppe zu seinem Zimmer hinaufzusteigen; er schien sehr schnell zu steigen, als wäre er aus irgendeinem Grund sehr leichtfüßig, so dass er und Fawkes im Nu verschwunden waren.

Hermine hielt eine zitternde Hand an ihre Wange, wo Fawkes sie mit seinem Flügel gestreift hatte, ein Fleck Wärme verweilte dort, als ob dieser kleine Fleck Haut ganz sanft in Brand gesetzt worden wäre. Sie hatte die Frage nach dem Phönix beantwortet, nahm sie an, aber es fühlte sich für sie an, als wäre sie nur knapp an dem Test vorbeigeschrammt, als hätte sie eine 62 bekommen und sie hätte 104 bekommen können, wenn sie sich mehr angestrengt hätte. Wenn sie sich überhaupt bemüht hätte. Sie hatte sich nicht wirklich bemüht, wenn sie darüber nachdachte. Sie hatte nur ihre Hausaufgaben gemacht.

\emph{Wen hast du gerettet?}

\textbf{Fawkes:}\\ Albträume, hatte der Junge erwartet, Schreie und Betteln und heulende Orkane der Leere, die Entladung der Schrecken, die sich in der Erinnerung festsetzten und auf diese Weise vielleicht Teil der Vergangenheit wurden. Und der Junge wusste, dass die Albträume kommen würden. In der nächsten Nacht würden sie kommen. Der Junge träumte, und in seinen Träumen stand die Welt in Flammen, Hogwarts stand in Flammen, sein Zuhause stand in Flammen, die Straßen Oxfords standen in Flammen, alles brannte in goldenen Flammen, die leuchteten, aber nicht verzehrten, und alle Menschen, die durch die brennenden Straßen liefen, leuchteten in weißem Licht, heller als das Feuer, als wären sie selbst Flammen oder Sterne.

Die anderen Erstklässler kamen zu Bett und sahen es mit eigenen Augen, das Wunder, dessen Gerücht sie schon gehört hatten, dass Harry Potter still und regungslos in seinem Bett lag, ein sanftes Lächeln auf dem Gesicht, während auf seinem Kissen ein rotgoldener Vogel über ihn wachte, dessen helle Flügel über ihn hinwegflogen wie eine Decke, die über seinen Kopf gezogen war. Die Abrechnung war um eine Nacht verschoben worden.

\textbf{Draco Malfoy:}\\ Draco richtete seinen Umhang und achtete darauf, dass der grüne Saum gerade war. Er strich sich mit dem Zauberstab über den Kopf und sprach einen Zauberspruch, den Vater ihm beigebracht hatte, als andere Kinder noch im Schlamm spielten, einen Zauberspruch, der dafür sorgte, dass kein einziges Fussel- oder Staubkorn seine Zaubererroben beschmutzen würde. Draco hob den geheimnisvollen Umschlag auf, den Vater ihm mit der Eule geschickt hatte, und steckte ihn in seinen Umhang. Er hatte bereits Incendio und Everto für den geheimnisvollen Zettel benutzt. Und dann machte er sich auf den Weg zum Frühstück, um sich genau auf die Sekunde zu setzen, auf dem das Essen erschien wenn er es schaffen konnte, damit es so aussah, als hätten alle anderen auf sein Erscheinen gewartet, um zu essen. Denn wenn man der Spross der Malfoys war, war man in allem der Erste, auch beim Frühstück, das war der Grund.

Vincent und Gregory warteten vor der Tür seines privaten Zimmers auf ihn, noch vor ihm aufgestanden - wenn auch natürlich nicht ganz so schick gekleidet. Der Slytherin-Gemeinschaftsraum war menschenleer, jeder, der so früh aufstand, ging sowieso direkt zum Frühstück. Die Kerkerflure waren bis auf die eigenen Schritte still, leer und widerhallend.

In der Großen Halle herrschte trotz der relativ wenigen Ankömmlinge ein aufgeregtes Treiben, einige jüngere Kinder weinten, Schüler rannten zwischen den Tischen hin und her oder standen in Knoten und schrien sich gegenseitig an, ein rotgewandeter Vertrauensschüler stand vor zwei grüngekleideten Schülern und schrie sie an und Snape schritt auf die Menge zu - Der Lärm wurde etwas leiser, als die Leute Draco erblickten, als sich einige der Gesichter umdrehten, um ihn anzustarren, und still wurden.

Das Essen erschien auf den Tischen. Keiner sah es an. Und Snape machte auf dem Absatz kehrt, ließ sein Ziel aus den Augen und ging direkt auf Draco zu. Ein Knoten der Angst klammerte sich an Dracos Herz,\\ \emph{war Vater etwas zugestoßen - nein, Vater hätte es ihm sicher gesagt - was auch immer passiert war, warum hatte Vater es ihm nicht gesagt} -

Unter Snapes Augen waren Tränensäcke der Müdigkeit, sah Draco, als ihr Hausherr näher kam, der Meister der Zaubertränke war noch nie gut gekleidet gewesen (das war eine Untertreibung), aber seine Roben waren heute Morgen noch schmutziger und unordentlicher, mit zusätzlichen Fettflecken.

„Hast du es nicht gehört?“, zischte sein Hausoberhaupt, als er näher kam. „Um Himmels willen, Malfoy, hast du keine Zeitung geliefert bekommen?“

„Was gibt es, Profe-“

„Bellatrix Black wurde aus Askaban geholt!“

„Was?“, sagte Draco schockiert, als Gregory hinter ihm etwas sagte, was er wirklich nicht hätte sagen sollen, und Vincent keuchte nur.

Snape schaute ihn mit zusammengekniffenen Augen an, dann nickte er abrupt. „Lucius hat dir also nichts erzählt. Ich verstehe.“ Snape schnaubte, wandte sich ab -

„Professor!“, sagte Draco. Die Auswirkungen begannen ihm gerade zu dämmern, sein Verstand drehte sich wie wild. „Professor, was soll ich tun - Vater hat mich nicht instruiert -“

„Dann schlage ich vor“, sagte Snape spöttisch, während er wegging, „dass du ihnen sagst dass du keine Ahnung hattest, Malfoy, so wie es dein Vater beabsichtigt hat!“

Draco warf einen Blick zurück auf Vincent und Gregory, obwohl er nicht wusste, warum er sich die Mühe machte, denn natürlich sahen sie noch verwirrter aus als er selbst. Draco ging nach vorne zum Slytherin-Tisch und setzte sich an das hintere Ende, wo noch niemand saß. Draco legte ein Würstchen auf seinen Teller und begann es mit automatischen Bewegungen zu essen.

\emph{Bellatrix Black war aus Askaban geholt worden. Bellatrix Black war aus Askaban geholt worden...?}

Draco wusste nicht, was er davon halten sollte, es war so unerwartet wie das Erlöschen der Sonne - \emph{nun, die Sonne würde erwartungsgemäß in sechs Milliarden Jahren erlöschen, aber das war so unerwartet wie das Erlöschen der Sonne morgen.} \emph{Vater hätte es nicht getan, Dumbledore hätte es nicht getan, niemand sollte dazu in der Lage sein - was bedeutete es - was würde Bellatrix nach zehn Jahren in Askaban irgendjemandem nützen - selbst wenn sie wieder stark werden würde, was nützte eine mächtige Zauberin, die völlig böse und wahnsinnig war und einem Dunklen Lord fanatisch ergeben, der nicht mehr da war?}

„Hey“, sagte Vincent von dort, wo er neben Draco saß, „ich verstehe das nicht, Boss, warum haben wir das getan?“

„Wir haben es nicht getan, du Tölpel!“, schnauzte Draco. „Um Himmels willen, wenn selbst du denkst, dass wir - hat dir dein Vater nie etwas über Bellatrix Black erzählt? Sie hat Vater einmal gefoltert, sie hat deinen Vater gefoltert, sie hat jeden gefoltert, der Dunkle Lord hat ihr einmal befohlen, sich selbst zu foltern und sie hat es getan! Sie hat keine verrückten Dinge getan, um Angst und Gehorsam in der Bevölkerung zu wecken, sie hat verrückte Dinge getan, weil sie verrückt ist! Sie ist eine Schlampe, das ist sie!“

„Ach, wirklich?“, sagte eine empörte Stimme hinter Draco. Draco blickte nicht auf. Gregory und Vincent würden ihm den Rücken freihalten. „Ich hätte gedacht, du würdest dich freuen zu hören, dass ein Todesser befreit worden ist, Malfoy!“

Amycus Carrow war immer einer der anderen Problemfälle gewesen; Vater hatte Draco einmal gesagt, er solle dafür sorgen, dass er nie mit Amycus allein in einem Raum war... Draco drehte sich um und schenkte Flora und Hestia Carrow seinen Grinser Nummer drei, der besagte, dass er in einem edlen und sehr alten Haus war und sie nicht und \emph{ja, das war wichtig.}

Draco sagte in ihre allgemeine Richtung, ohne sich herabzulassen, sie besonders anzusprechen: „Es gibt Todesser und es gibt Todesser“, und wandte sich dann wieder seinem Essen zu.

Es gab zwei wütende Rufe im Einklang, und dann stürmten zwei Paar Schuhe in Richtung des anderen Endes des Slytherin-Tisches davon. Ein paar Minuten später rannte Millicent Bulstrode auf sie zu, sichtlich außer Atem, und sagte: „Mr. Malfoy, haben Sie gehört?“

„Von Bellatrix Black?“, fragte Draco. „Ja -“

„Nein, über Potter!“

„Was?“

„Potter ist gestern Abend mit einem Phönix auf der Schulter herumgelaufen, er sah aus, als wäre er durch zehn Meilen Schlamm geschleift worden, man sagt, der Phönix hätte ihn nach Askaban gebracht, um Bellatrix aufzuhalten, und er hat sich mit ihr duelliert, und sie haben die halbe Festung in die Luft gejagt!“

„Was?!“, sagte Draco. „Oh, es ist einfach unmöglich, dass -“ Draco hielt inne. Das hatte er schon oft über Harry Potter gesagt und er hatte begonnen, einen Trend zu bemerken.

Millicent rannte los, um es jemand anderem zu sagen.

„Du glaubst doch nicht wirklich -“, sagte Gregory.

„Ich weiß es ehrlich gesagt nicht mehr“, sagte Draco.

Ein paar Minuten später, nachdem Theodore Nott sich ihm gegenüber hingesetzt hatte und William Rosier gegangen war, um sich zu den Carrow-Zwillingen zu setzen, stupste Vincent ihn an und sagte: „Da.“

Harry Potter hatte die Große Halle betreten. Draco beobachtete ihn genau. Auf Harrys Gesicht war kein Alarm zu sehen, als er ihn sah, keine Überraschung oder Schock, er sah einfach nur... Es war der gleiche distanzierte, in sich gekehrte Blick, den Harry trug, wenn er versuchte, die Antwort auf eine Frage herauszufinden, die Draco noch nicht verstand. Draco schob sich hastig von der Bank des Slytherin-Tisches hoch, sagte: „Bleib zurück“, und ging mit aller anständigen Geschwindigkeit auf Harry zu. Harry schien seine Annäherung gerade zu bemerken, als der andere Junge sich zum Ravenclaw-Tisch umdrehte, und Draco warf Harry einen kurzen Blick zu und ging dann an ihm vorbei, direkt aus der Großen Halle hinaus. Eine Minute später spähte Harry um die Ecke der kleinen steinernen Ecke, in der Draco gewartet hatte, es würde vielleicht nicht jeden täuschen, aber es würde eine plausible Bestreitbarkeit schaffen.

„Quietus“, sagte Harry. „Draco, was -“

Draco holte den Umschlag aus seinem Umhang. „Ich habe eine Nachricht von Vater für dich.“

„Huh?“, sagte Harry und nahm Draco den Umschlag ab und riss ihn auf eine eher unschöne Weise auf und zog ein Blatt Pergament heraus und entfaltete es und - Harry atmete scharf ein. Dann sah Harry Draco an. Dann sah Harry wieder auf das Pergament hinunter. Es gab eine Pause.

Harry sagte: „Hat Lucius dir aufgetragen, über meine Reaktion darauf zu berichten?“ Draco hielt einen Moment inne, wog ab und öffnete dann den Mund.\\ „Wie ich sehe, hat er das“, sagte Harry und Draco verfluchte sich, er hätte es besser wissen müssen, nur war es schwer gewesen, sich zu entscheiden. „Was willst du ihm denn sagen?“

„Dass du überrascht warst“, sagte Draco.

„Überrascht“, sagte Harry flach. „Ja. Gut. Sag ihm das.“

„Was ist?“, fragte Draco.

Und dann, als er Harrys zwiespältigen Blick sah: „Wenn du hinter meinem Rücken mit Vater verhandelst -“

Und Harry gab Draco ohne ein Wort das Papier. Darauf stand:

\emph{Ich weiß, dass du es warst.}

„WAS ZUM -“

„Das wollte ich dich gerade fragen“, sagte Harry. „Hast du eine Ahnung, was mit deinem Dad los ist?“

Draco starrte Harry an. Dann sagte Draco: „Hast du es getan?“

„Was?“, sagte Harry. „Welchen möglichen Grund sollte ich - wie sollte ich -“

„Hast du es getan, Harry?“

„Nein!“, sagte Harry. „Natürlich nicht!“

Draco hatte genau zugehört, aber er hatte kein Zögern oder Zittern bemerkt. Also nickte Draco und sagte: „Ich habe keine Ahnung, was Vater denkt, aber es kann nicht, ich meine, es kann unmöglich gut sein. Und, ähm... die Leute sagen auch...“

„Was sagen sie denn, Draco?“ sagte Harry misstrauisch,

„Hat dich wirklich ein Phönix nach Askaban gebracht, um zu versuchen, Bellatrix Black an der Flucht zu hindern -“

\textbf{Neville Longbottom:}\\ Harry hatte sich gerade erst zum ersten Mal an den Ravenclaw-Tisch gesetzt, in der Hoffnung, einen schnellen Happen zu essen. Er wusste, dass er gehen und über die Dinge nachdenken musste, aber es gab noch einen winzigen Rest von Phönixfrieden (selbst nach der Begegnung mit Draco), an den er sich klammern wollte, ein schöner Traum, an den er sich nur erinnern konnte; und der Teil von ihm, der sich nicht friedlich fühlte, wartete darauf, dass all die Ambosse auf ihn herabfielen, damit er, wenn er ging, um nachzudenken und eine Weile allein zu sein, all die Katastrophen auf einmal verarbeiten konnte. Harrys Hand griff nach einer Gabel, hob einen Bissen Kartoffelpüree zum Mund - und es gab einen Schrei. Ab und zu schrie jemand, wenn er die Nachricht hörte, aber Harrys Ohren erkannten diesen - Harry stand sofort von der Bank auf und ging auf den Hufflepuff-Tisch zu, ein schreckliches ungutes Gefühl machte sich in seiner Magengrube breit. Es war eines dieser Dinge, an die er nicht gedacht hatte, als er sich entschlossen hatte, das Verbrechen zu begehen, denn Professor Quirrell hatte geplant, dass niemand davon erfährt; und jetzt, im Nachhinein, dachte Harry einfach - \emph{nicht daran} -

\emph{Das}, sagte Hufflepuff mit bitterer Intensität, \emph{ist auch deine Schuld.}

Aber als Harry ankam, saß Neville schon da und aß gebratene Wurstpastetchen mit Snippyfig-Soße. Die Hände des Hufflepuff-Jungen zitterten, aber er schnitt das Essen ab und aß es, ohne es fallen zu lassen.

„Hallo, General“, sagte Neville, wobei seine Stimme nur leicht schwankte. „Hast du dich gestern Abend mit Bellatrix Black duelliert?“

„Nein“, sagte Harry. Seine eigene Stimme war aus irgendeinem Grund ebenfalls schwankend.

„Hätte ich auch nicht geglaubt“, sagte Neville. Es gab ein schabendes Geräusch, als sein Messer wieder die Wurst schnitt. „Ich werde sie jagen und töten, kann ich auf deine Hilfe zählen?“

Die Masse der Hufflepuffs, die sich um Neville versammelt hatte, schnappte erschrocken nach Luft.

„Wenn sie hinter dir her ist“, sagte Harry heiser,\\ \emph{wenn das alles ein furchtbarer Irrtum war, wenn das alles eine Lüge war, werde ich dich verteidigen, sogar mit meinem Leben,}\\ „werde nicht zulassen, dass du verletzt wirst“.\\ \emph{Nicht für das, was ich getan habe, egal was,}\\ „aber ich werde dir nicht helfen, sie zu jagen, Neville, Freunde helfen Freunden nicht, Selbstmord zu begehen.“

Nevilles Gabel hielt auf dem Weg zu seinem Mund inne. Dann steckte Neville den Bissen in den Mund, kaute noch einmal. Und Neville schluckte ihn herunter. Und Neville sagte: „Ich meinte nicht jetzt, ich meine, nachdem ich Hogwarts abgeschlossen habe.“

„Neville“, sagte Harry, wobei er seine Stimme sehr sorgfältig unter Kontrolle hielt, „ich glaube, dass das auch nach deinem Abschluss noch eine ziemlich dumme Idee sein könnte. Es muss viele erfahrene Auroren geben, die sie verfolgen -“ \emph{oh oh,} \emph{warte, das ist nicht gut -}

„Hör auf ihn!“, sagte Ernie Macmillan, und dann sagte ein älter aussehendes Hufflepuff-Mädchen, das dicht neben Neville stand: „Nevvy, bitte, denk darüber nach, er hat recht!“

Neville stand auf.\\ Neville sagte: „Bitte folgt mir nicht.“ Neville ging von allen weg; Harry und Ernie streckten ihm unwillkürlich die Hand entgegen, und auch einige der anderen Hufflepuffs. Und Neville setzte sich an den Gryffindor-Tisch, und aus der Ferne (obwohl sie sich anstrengen mussten, um zu hören) hörten sie Neville sagen: „Ich werde sie jagen und nach meinem Abschluss töten, will jemand helfen?“ und mindestens fünf Stimmen sagten „Ja“ und dann sagte Ron Weasley laut: „Stellt\\ euch an, ihr alle, ich habe heute Morgen eine Eule von Mum bekommen, sie sagt, ich soll allen sagen, dass sie sich einen Platz reserviert hat“ und jemand sagte „Molly Weasley gegen Bellatrix Black?“

Jemand tippte Harry auf die Schulter, und als er sich umdrehte, sah er ein ihm unbekanntes älteres Mädchen mit grünem Haar, das ihm einen Pergamentumschlag reichte und dann schnell wegging. Harry starrte einen Moment lang auf den Umschlag, dann ging er auf die nächste Wand zu. Das war nicht sehr privat, aber es sollte privat genug sein, und Harry wollte nicht den Eindruck erwecken, viel zu verbergen. Das war eine Lieferung des Slytherin-Systems gewesen, was man benutzte, wenn man mit jemandem kommunizieren wollte, ohne dass jemand anderes wusste, dass die beiden miteinander gesprochen hatten. Der Absender gab jemandem, der den Ruf hatte, ein zuverlässiger Bote zu sein, einen Umschlag zusammen mit zehn Knuts; diese erste Person nahm fünf Knuts und gab den Umschlag zusammen mit den anderen fünf Knuts an einen anderen Boten weiter, und der zweite Bote öffnete diesen Umschlag und fand einen anderen Umschlag mit einem Namen darauf und übergab diesen Umschlag an diese Person. Auf diese Weise kannte keine der beiden Personen, die die Nachricht überbrachten, sowohl den Absender als auch den Empfänger, so dass niemand sonst wusste, dass diese beiden Parteien in Kontakt gestanden hatten...

Als Harry die Wand erreichte, steckte er den Umschlag in seinen Umhang, öffnete ihn unter den Stofffalten und warf vorsichtig einen Blick auf das Pergament, das er hervorzog. Darauf stand:

\emph{„Klassenzimmer links von der Verwandlung, 8 Uhr morgens. - LL.“}

Harry starrte es an und versuchte sich zu erinnern, ob er jemanden mit den Initialen LL kannte. Sein Verstand suchte... suchte... - „Das Quibbler-Mädchen?“ flüsterte Harry ungläubig und hielt sich dann den Mund zu. Sie war erst zehn Jahre alt, sie sollte überhaupt nicht in Hogwarts sein!

\textbf{Lesath Lestrange.}\\ Harry stand um 8 Uhr morgens in dem unbenutzten Klassenzimmer neben Verwandlung und wartete, er hatte es wenigstens geschafft, etwas Essen in sich hineinzustopfen, bevor er sich der nächsten Katastrophe stellte, \emph{Luna Lovegood...}

Die Tür zum Klassenzimmer öffnete sich, und Harry sah und verpasste sich einen wirklich harten mentalen Tritt. Noch eine Sache, an die er nicht gedacht hatte, eine Sache, die er wirklich hätte tun sollen. Der grüne Umhang des älteren Jungen war schief, es waren rote Flecken darauf, die aussahen wie kleine Punkte frischen Blutes, und ein Mundwinkel sah aus wie eine Schnittwunde, die durch einen Episkey oder einen anderen kleinen medizinischen Zauber, der den Schaden nicht ganz auslöschte, geheilt worden war.

Lesath Lestranges Gesicht war von Tränen übersät, frischen und halbgetrockneten Tränen, und in seinen Augen stand Wasser, ein Versprechen, dass noch mehr kommen würde. „Quietus“, sagte der ältere Junge, und dann „Homenum Revelio“ und einige andere Dinge, während Harry verzweifelt und ohne viel Glück nachdachte. Und dann senkte Lesath seinen Zauberstab und verstaute ihn in seinem Umhang, und langsam, diesmal förmlich, sank der ältere Junge auf dem staubigen Klassenraumboden auf die Knie. Beugte den Kopf ganz nach unten, bis auch seine Stirn den Staub berührte, und Harry hätte gesprochen, aber er war stimmlos.

Lesath Lestrange sagte mit brechender Stimme: „Mein Leben gehört Euch, mein Herr, und mein Tod ebenso.“

„Ich“, sagte Harry, ein riesiger Kloß saß ihm im Hals, und er hatte Mühe zu sprechen, „ich -“ \emph{hatte nichts damit zu tun,} er hätte sagen sollen, hätte jetzt sagen sollen, aber dann wiederum hätte der unschuldige Harry auch Mühe gehabt zu sprechen -

„Danke, „ flüsterte Lesath, „danke, mein Herr, oh, danke“, der Klang eines erstickten Schluchzens kam von dem knienden Jungen, alles, was Harry von ihm sehen konnte, waren die Haare am Hinterkopf, nichts von seinem Gesicht. „Ich bin ein Narr, mein Herr, ein undankbarer Bastard, unwürdig, Euch zu dienen, ich kann mich nicht genug erniedrigen, denn ich - ich habe Euch angeschrien, nachdem Ihr mir geholfen habt, weil ich dachte, Ihr würdet mich ablehnen, und ich habe erst heute Morgen gemerkt, dass ich so ein Narr war, Euch vor Longbottom zu bitten -“

„Ich hatte nichts damit zu tun“, sagte Harry. (Es war immer noch sehr schwer, so eine glatte Lüge zu erzählen.)

Langsam hob Lesath seinen Kopf vom Boden, sah zu Harry auf. „Ich verstehe, mein Herr“, sagte der ältere Junge, seine Stimme schwankte ein wenig, „Sie trauen meiner Gerissenheit nicht, und ich habe mich in der Tat als Narr erwiesen... Ich wollte Ihnen nur sagen, dass ich nicht undankbar bin, dass ich weiß, dass es schwer genug gewesen sein muss, nur einen Menschen zu retten, dass sie jetzt alarmiert sind, dass Sie es nicht können -- Vater zu holen - aber ich bin nicht undankbar, ich werde nie wieder undankbar zu Ihnen sein. Wenn Sie jemals eine Verwendung für diesen unwürdigen Diener haben, rufen Sie mich, wo immer ich bin, und ich werde antworten, mein Herr -“

„Ich war in keiner Weise beteiligt.“ (Aber es wurde von Mal zu Mal einfacher.)

Lesath blickte zu Harry auf und sagte unsicher: „Bin ich aus Eurer Gegenwart entlassen, mein Herr...?“

„Ich bin nicht dein Herr.“

Lesath sagte: „Ja, mein Herr, ich verstehe“, und erhob sich wieder vom Boden, stand aufrecht und verbeugte sich tief, dann wich er von Harry zurück, bis dieser sich umdrehte, um die Tür zum Klassenzimmer zu öffnen. Als Lesaths Hand den Türknauf berührte, hielt er inne. Harry konnte Lesaths Gesicht nicht sehen, als die Stimme des älteren Jungen sagte: „Hast du sie zu jemandem geschickt, der sich um sie kümmern würde? Hat sie überhaupt nach mir gefragt?“

Und Harry sagte, seine Stimme vollkommen gleichmäßig: „Bitte hör auf damit. Ich war in keiner Weise daran beteiligt.“

„Ja, mein Herr, es tut mir leid, mein Herr“, sagte Lesaths Stimme, und der Slytherin-Junge öffnete die Tür, ging hinaus und schloss die Tür hinter sich. Seine Füße beschleunigten sich, als er davonlief, aber nicht schnell genug, dass Harry nicht hören konnte, wie er anfing zu schluchzen.

\emph{Würde ich weinen?} fragte sich Harry. \emph{Wenn ich nichts wüsste, wenn ich unschuldig wäre, würde ich dann jetzt weinen?} Harry wusste es nicht, also schaute er einfach weiter auf die Tür. Und ein unglaublich taktloser Teil von ihm dachte:\\ \emph{„Hurra, wir haben eine Quest abgeschlossen und einen Diener bekommen} -\\ \emph{Halt die Klappe. Wenn du jemals wieder über etwas abstimmen willst... Halt die Klappe.“}

\hfill\break \textbf{Amelia Bones:}\\ „Dann ist sein Leben nicht in Gefahr, nehme ich an“, sagte Amelia.

Der Heiler, ein streng blickender alter Mann, der seine Roben weiß trug (er war ein Muggelgeborener und hielt sich an irgendeine seltsame Tradition der Muggel, nach der Amelia nie gefragt hatte, obwohl sie insgeheim dachte, dass es ihn zu sehr wie einen Geist aussehen ließ), schüttelte den Kopf und sagte: „Definitiv nicht.“

Amelia betrachtete die menschliche Gestalt, die bewusstlos auf dem Bett ruhte, das verbrannte und zerfetzte Fleisch, das dünne Laken, das ihn aus Gründen des Anstands bedeckte, wurde auf ihren Befehl hin zurückgeschlagen.

\emph{Vielleicht erholte er sich vollständig. Vielleicht aber auch nicht.}

Der Heiler hatte gesagt, es sei zu früh, um das zu sagen. Dann sah Amelia die andere Hexe im Raum an, die Auroren-Investigatorin „Und Sie sagen“, sagte Amelia, „dass die brennende Materie aus Wasser verwandelt wurde, vermutlich in Form von Eis.“

Sie nickte und sagte, verwirrt klingend: „Es hätte viel schlimmer sein können, wenn nicht -“

„Wie nett von ihnen“, spuckte sie aus und presste dann eine müde Hand an ihre Stirn.

\emph{Nein... nein, es war als Freundlichkeit gemeint gewesen. In der letzten Phase der Flucht würde es keinen Sinn mehr haben, jemanden zu täuschen. Wer auch immer das getan hatte, hatte also versucht, den Schaden zu begrenzen - und sie hatten dabei an die Auroren gedacht, die den Rauch einatmeten, nicht daran, dass irgendjemand mit dem Feuer angegriffen werden könnte. Wären sie es gewesen, die noch die Kontrolle hatten, hätten sie den Rocker zweifellos barmherziger gelenkt.}

Aber Bellatrix Black war allein auf dem Rocker aus Askaban geritten, da waren sich alle zuschauenden Auroren einig, sie hatten ihre Anti-Desillusionierungszauber aktiviert und es war nur eine Frau auf dem Rocker gewesen, obwohl er zwei Paar Steigbügel gehabt hatte. Irgendeine gute und unschuldige Person, die den Patronus-Zauber wirken konnte, war hereingelegt worden, um Bellatrix Black zu befreien. Ein Unschuldiger hatte gegen Bahry Einhand gekämpft und einen erfahrenen Auror vorsichtig überwältigt, ohne ihn nennenswert zu verletzen. Irgendein Unschuldiger hatte den Treibstoff für das Muggelartefakt, mit dem die beiden aus Askaban reiten sollten, aus gefrorenem Wasser zum Nutzen ihrer Auroren verwandelt. Und dann hatte ihre Nützlichkeit für Bellatrix Black geendet.\\ Man hätte erwartet, dass jeder, der fähig war, Bahry Einhand zu besiegen, diesen Teil vorausgesehen hätte. Aber dann hätte man auch nicht erwartet, dass jemand, der den Patronus-Zauber wirken konnte, überhaupt erst versuchte, Bellatrix Black zu retten.

Amelia fuhr sich mit der Hand über die Augen und schloss sie für einen Moment in stiller Trauer.

\emph{Ich frage mich, wer es war und wie Du-weißt-schon-wer sie manipuliert hat... welche Geschichte man ihnen wohl erzählt haben könnte...}

Sie merkte erst einen Moment später, dass der Gedanke bedeutete, dass sie anfing es zu glauben. Vielleicht, weil es, egal wie schwierig es war, Dumbledore zu glauben, immer schwieriger wurde, die Hand dieser kalten, dunklen Intelligenz nicht zu erkennen.

\textbf{Albus Dumbledore:}\\ Es waren vielleicht nur siebenundfünfzig Sekunden bis zum Ende des Frühstücks, und er hatte vier Umdrehungen seines Zeitdrehers gebraucht, aber am Ende hatte Albus Dumbledore es doch geschafft.

„Schulleiter?“, quietschte die höfliche Stimme von Professor Filius Flitwick, als der alte Zauberer auf dem Weg zu seinem Platz an ihm vorbeiging. „Mr. Potter hat eine Nachricht für Sie hinterlassen.“

Der alte Zauberer blieb stehen. Er schaute den Zaubereiprofessor fragend an.

„Mr. Potter sagte, dass ihm nach dem Aufwachen klar geworden ist, wie ungerecht die Dinge waren, die er zu Ihnen gesagt hat, nachdem Fawkes geschrien hat. Mr. Potter sagte, dass er nichts anderes zu sagen hat, sondern sich nur für diesen einen Teil entschuldigen will.“

Der alte Zauberer sah seinen Zauberkunst-Professor weiter an und sprach immer noch nicht.

„Schulleiter?“, quietschte Filius.

„Sag ihm, dass ich mich bedankt habe“, sagte Albus Dumbledore, „aber dass es klüger ist, auf Phönixe zu hören als auf weise alte Zauberer“, und setzte sich an seinen Platz, drei Sekunden bevor das ganze Essen verschwand.

\textbf{Professor Quirrell:}\\ „Nein“, schnauzte Madam Pomfrey das Kind an, „du darfst ihn nicht sehen! Du darfst ihn nicht belästigen! Du darfst ihm nicht die kleinste Frage stellen! Er muss im Bett liegen und darf mindestens drei Tage lang nichts tun!“

\textbf{Minerva McGonagall:}\\ Sie war auf dem Weg zum Krankenflügel, und Harry Potter wollte ihn gerade verlassen, als sie aneinander vorbeigingen. Der Blick, den er ihr zuwarf, war nicht wütend. Er war nicht traurig. Er sagte überhaupt nicht viel aus. Es war wie... als ob er sie gerade lange genug ansah, um deutlich zu machen, dass er es nicht absichtlich vermied, sie anzusehen. Und dann schaute er weg, bevor sie sich überlegen konnte, mit welchem Blick sie ihn erwidern sollte; als ob er ihr auch das ersparen wollte. Er sagte nichts, als er an ihr vorbeiging. Sie tat es auch nicht. Was sollte es da schon zu sagen geben?

\textbf{Fred und George Weasley:}\\ Sie schrien tatsächlich laut auf, als sie um die Ecke bogen und Dumbledore sahen. Es war nicht so, dass der Schulleiter aus dem Nichts aufgetaucht war und sie mit strengem Blick anstarrte. Das tat Dumbledore immer. Aber der Zauberer war in formelle schwarze Roben gekleidet und sah sehr alt und sehr mächtig aus und er warf den beiden einen \emph{SCHARFEN BLICK} zu.

„Fred und George Weasley!“, sprach Dumbledore mit einer Stimme der Macht.

„Ja, Schulleiter!“, sagten sie, richteten sich auf und gaben ihm einen knackigen Militärgruß, wie sie ihn auf alten Bildern gesehen hatten.

„Hört mir gut zu! Ihr seid Freunde von Harry Potter, ist das so?“

„Ja, Schulleiter!“

„Harry Potter ist in Gefahr. Er darf nicht über die Grenzen von Hogwarts hinausgehen. Hört mir zu, Söhne der Weasleys, ich bitte euch: Ihr wisst, dass ich genauso ein Gryffindor bin wie ihr selbst, dass auch ich weiß, dass es höhere Regeln gibt. Aber dies, Fred und George, diese eine Sache ist von furchtbarster Wichtigkeit, es darf diesmal keine Ausnahme geben, klein oder groß! Wenn ihr Harry helft, Hogwarts zu verlassen, kann er sterben! Schickt er euch auf eine Mission, dürft ihr gehen, bittet er euch, ihm Gegenstände zu bringen, dürft ihr helfen, aber wenn er euch bittet, seine eigene Person aus Hogwarts zu schmuggeln, müsst ihr ablehnen! Habt ihr das verstanden?“

„Ja, Schulleiter!“ Sie sagten es, ohne wirklich nachzudenken, und tauschten dann unsichere Blicke miteinander aus - die strahlend blauen Augen des Schulleiters waren auf sie gerichtet.

„Nein. Nicht ohne nachzudenken. Wenn Harry euch bittet, ihn herauszubringen, müsst ihr euch weigern, wenn er euch bittet, ihm den Weg zu sagen, müsst ihr euch weigern. Ich werde euch nicht bitten, ihn mir zu melden, denn das würdet ihr nie tun. Aber bittet ihn in meinem Namen, zu mir zu gehen, wenn es so wichtig ist, und ich werde ihn auf seinem Weg selbst bewachen. Fred, George, es tut mir leid, dass ich eure Freundschaft so strapaziere, aber es geht um sein Leben.“

Die beiden sahen sich lange an, ohne sich zu verständigen, sie dachten nur dasselbe zur gleichen Zeit. Dann sahen sie wieder zu Dumbledore. „Bellatrix Black“, sagten sie, wobei ihnen ein Schauer über den Rücken lief, als sie den Namen aussprachen.

„Ihr könnt mit Sicherheit davon ausgehen“, sagte der Schulleiter, „dass es mindestens so schlimm ist.“

„Okay -“\\ “- verstanden.“

\textbf{Alastor Moody und Severus Snape:}\\ Als Alastor Moody sein Auge verloren hatte, hatte er die Dienste eines äußerst gelehrten Ravenclaw, Samuel H. Lyall, in Anspruch genommen, dem Moody etwas weniger als dem Durchschnitt misstraute, weil er ihn nicht als unregistrierten Werwolf gemeldet hatte; und er hatte Lyall dafür bezahlt, eine Liste aller bekannten magischen Augen und jeden bekannten Hinweises auf ihren Standort zusammenzustellen. Als Moody die Liste zurückbekommen hatte, hatte er sich nicht die Mühe gemacht, das meiste davon zu lesen; denn an der Spitze der Liste stand das Auge von Vance, das aus einer Ära vor Hogwarts stammte und sich derzeit im Besitz eines mächtigen dunklen Zauberers befand, der über ein winziges, vergessenes Höllenloch herrschte, das weder in Großbritannien noch irgendwo anders lag, wo er sich um dumme Regeln kümmern musste. So hatte Alastor Moody seinen linken Fuß verloren und das Auge von Vance erworben, und so waren die unterdrückten Seelen von Urulat für einen Zeitraum von etwa zwei Wochen befreit worden, bevor ein anderer Dunkler Zauberer in das Machtvakuum einzog.

Er hatte erwogen, als Nächstes den Linken Fuß von Vance zu holen, hatte sich aber dagegen entschieden, nachdem ihm klar wurde, \emph{dass das genau das wäre, was sie erwarten würden.}

Jetzt drehte sich Mad-Eye Moody langsam, immer drehend, und überblickte den Friedhof von Little Hangleton. Es hätte viel düsterer sein sollen, dieser Ort, aber im hellen Tageslicht schien er nichts weiter zu sein als ein grasbewachsener Platz, gekennzeichnet durch gewöhnliche Grabsteine, abgegrenzt durch die angeketteten Windungen aus zerbrechlichem, leicht erklimmbaren Metall, die Muggel anstelle von Schutzzauber verwendeten. (Moody konnte nicht nachvollziehen, was die Muggel in dieser Hinsicht dachten, ob sie nur so taten, als hätten sie Schutzzauber oder was, und er hatte beschlossen, nicht zu fragen, ob Muggelverbrecher den Schein respektierten.)

Moody brauchte sich eigentlich nicht umzudrehen, um den Friedhof zu überblicken. Das Auge von Vance sah den gesamten Globus der Welt in jeder Richtung um ihn herum, egal, wohin es zeigte. Aber es gab keinen besonderen Grund, einen ehemaligen Todesser wie Severus Snape das wissen zu lassen. Manchmal nannten die Leute Moody „paranoid“. Moody sagte ihnen immer, sie sollten hundert Jahre Jagd auf Dunkle Zauberer überleben und sich dann bei ihm melden. Mad-Eye Moody hatte einmal ausgerechnet, wie lange er im Nachhinein gebraucht hatte, um das zu erreichen, was er jetzt als ein anständiges Maß an Vorsicht betrachtete - er wog ab, wie viel Erfahrung er gebraucht hatte, um gut zu werden, anstatt Glück zu haben - und hatte angefangen zu vermuten, dass die meisten Leute starben, bevor sie es erreichten. Moody hatte diesen Gedanken einmal Lyall gegenüber geäußert, der daraufhin ein wenig rechnete und ihm sagte, dass ein typischer Jäger der dunklen Magier im Durchschnitt achteinhalb Mal auf dem Weg zum „\emph{Paranoidwerden}“ sterben würde. Das erklärte eine ganze Menge, \emph{vorausgesetzt, Lyall log nicht.}

Gestern hatte Albus Dumbledore Mad-Eye Moody erzählt, dass der Dunkle Lord unaussprechliche dunkle Künste angewandt hatte, um den Tod seines Körpers zu überleben, und nun wach und unterwegs war, um seine Macht wiederzuerlangen und den Zaubererkrieg von neuem zu beginnen. Jemand anders hätte vielleicht mit Ungläubigkeit reagiert.

„Ich kann nicht glauben, dass ihr mir nie etwas von dieser Auferstehungssache erzählt habt“, sagte Mad-Eye Moody mit beträchtlicher Bitterkeit. „Ist euch klar, wie lange ich brauchen werde, um die Gräber aller Vorfahren aller dunklen Zauberer, die ich jemals getötet habe und die schlau genug gewesen sein könnten, einen Horkrux herzustellen, zu bearbeiten? Du machst das doch nicht erst jetzt, oder?“

„Ich sabotiere diesen hier jedes Jahr neu“, sagte Severus Snape ruhig, öffnete das dritte Fläschchen von angeblich siebzehn Flaschen und begann, mit seinem Zauberstab darüber zu schwenken. „Die anderen Ahnengräber, die wir ausfindig machen konnten, wurden nur mit den langanhaltenden Substanzen vergiftet, da einige von uns weniger Freizeit haben als du.“

Moody beobachtete, wie die Flüssigkeit spiralförmig aus dem Fläschchen floss und verschwand, um in den Knochen zu erscheinen, wo einst Knochenmark gewesen war.\\ „Aber du glaubst, dass es den Aufwand der Falle wert ist, anstatt die Knochen einfach verschwinden zu lassen.“

„Er hat noch andere Wege zum Leben, sollte er diesen als blockiert empfinden“, sagte Snape trocken und öffnete eine vierte Flasche. „Und bevor du fragst: Es muss das Originalgrab sein, der Ort der ersten Bestattung, der Knochen muss \emph{während} des Rituals entfernt werden und nicht vorher. Er kann es also nicht früher geholt haben; und es macht auch keinen Sinn, das Skelett eines Vorfahren zu ersetzen. Er würde merken, dass es jede Kraft verloren hat.“

„Wer weiß noch von dieser Falle?“ fragte Moody nach.

„Du. Ich. Der Schulleiter. Sonst niemand.“

Moody schnaubte. „Pfah. Hat Albus Amelia, Bartemius und dieser Frau McGonagall von dem Auferstehungsritual erzählt?“

„Ja -“

„Wenn Voldie herausfindet, dass Albus von dem Auferstehungsritual weiß und es ihnen erzählt hat, wird Voldie sich denken, dass Albus es mir erzählt hat, und Voldie weiß, dass ich an so etwas denken würde.“ Moody schüttelte angewidert den Kopf. „Was sind das für andere Möglichkeiten, wie Voldie wieder ins Leben zurückkehren könnte?“

Snapes Hand hielt an der fünften Flasche inne (es war natürlich alles desillusioniert, die ganze Operation war desillusioniert, aber das bedeutete Moody weniger als nichts, es markierte einen in seinem Auge nur als Versuch, sich zu verstecken), und der ehemalige Todesser sagte: „Das musst du nicht wissen.“

„Du lernst dazu, mein Sohn“, sagte Moody mit milder Zustimmung. „Was ist in den Flaschen?“

Snape öffnete die fünfte Flasche, gestikulierte mit seinem Zauberstab, um die Substanz in Richtung Grab fließen zu lassen, und sagte: „Diese hier? Ein Muggel-Narkotikum namens LSD. Ein Gespräch gestern hat mich an Muggelsachen denken lassen, und LSD schien die interessanteste Option zu sein, also habe ich mich beeilt, etwas davon zu besorgen. Wenn es dem Auferstehungstrank beigemischt wird, vermute ich, dass seine Wirkung dauerhaft sein wird.“

„Was bewirkt es?“, fragte Moody.

„Es heißt, dass die Wirkung für jemanden, der es nicht benutzt hat, unmöglich zu beschreiben ist“, lachte Snape, „und ich habe es nicht benutzt.“

Moody nickte zustimmend, als Snape das sechste Fläschchen öffnete. „Was ist mit dem hier?“

„Liebestrank.“

„Liebestrank?!“, sagte Moody.

„Nicht von der üblichen Sorte. Er soll eine wechselseitige Bindung mit einer unerträglich süßen Veela-Frau namens Verdandi auslösen, von der der Schulleiter hofft, dass sie sogar ihn erlösen könnte, wenn sie sich wirklich liebten.“

„Gah!“, sagte Moody. „Dieser verdammte sentimentale Narr -“

„Einverstanden“, sagte Severus Snape ruhig, seine Aufmerksamkeit auf seine Arbeit gerichtet.

„Sag mir, dass du wenigstens etwas Malaclaw-Gift da drin hast.“

„Zweites Fläschchen.“

„Iocane-Pulver.“

„Entweder die vierzehnte oder die fünfzehnte Flasche.“

„Bahls Betäubung“, sagte Moody und nannte ein extrem süchtig machendes Narkotikum mit interessanten Nebenwirkungen auf Menschen mit Slytherin-Tendenzen; Moody hatte einmal gesehen, wie ein süchtiger dunkler Zauberer lächerliche Anstrengungen unternahm, um ein Opfer dazu zu bringen, einen bestimmten exakten Portschlüssel in die Hand zu nehmen, anstatt die Zielperson bei ihrem nächsten Besuch in der Stadt einfach einen hingeworfenen Knut fangen zu lassen; und nach all dieser Arbeit hatte sich der Süchtige noch die Mühe gemacht, einen zweiten Portus auf denselben Portschlüssel zu legen, der das Opfer bei einer zweiten Berührung wieder in Sicherheit brachte. Bis heute konnte sich Moody, selbst unter Berücksichtigung der Droge, nicht vorstellen, was dem Mann durch den Kopf gegangen sein könnte, als er den zweiten Portus gelegt hatte.

„Zehnte Phiole“, sagte Snape.

„Basiliskengift“, bot Moody an.

„Was?“, spuckte Snape. „Schlangengift ist ein positiver Bestandteil des Auferstehungstranks! Ganz zu schweigen davon, dass es die Knochen und all die anderen Substanzen auflösen würde! Und woher sollten wir überhaupt -“

„Beruhige dich, mein Sohn, ich wollte nur sehen, ob man dir trauen kann.“ Mad-Eye Moody setzte seine (insgeheim unnötige) langsame Drehung fort und begutachtete den Friedhof, und der Meister der Zaubertränke goss weiter. „Moment mal“, sagte Moody plötzlich. „Woher weißt du, dass das wirklich der Ort ist, an dem -“

„Weil auf dem leicht zu bewegenden Grabstein \emph{'Tom Riddle'} steht“, sagte Snape trocken. „Und ich habe gerade zehn Sickel vom Schulleiter gewonnen, der gewettet hat, dass dir das vor der fünften Flasche einfallen würde. So viel zur \emph{ständigen Wachsamkeit.}“

Es entstand eine Pause. „Wie lange hat Albus gebraucht, um zu erkennen -“

„Drei Jahre, nachdem wir von dem Ritual erfahren haben“, sagte Snape in einem Ton, der nicht ganz seinem üblichen sardonischen Tonfall entsprach. „Im Nachhinein betrachtet, hätten wir dich früher konsultieren sollen.“ Snape öffnete die neunte Flasche. „Wir haben auch alle anderen Gräber vergiftet, mit lang anhaltenden Substanzen“, bemerkte der ehemalige Todesser. „Es ist möglich, dass wir auf dem richtigen Friedhof sind. Vielleicht hat er nicht so weit vorausgeplant, als er seine Familie abschlachtete, und er kann das Grab selbst nicht verlegen -“

„Der richtige Ort sieht nicht mehr wie ein Friedhof aus“, sagte Moody schlicht. „Er hat alle anderen Gräber hierher verlegt und die Muggel mit einem Gedächtniszauber belegt. Nicht einmal Bellatrix Black erfährt etwas davon, bis kurz vor Beginn des Rituals. Niemand außer ihm kennt jetzt den wahren Ort.“

\emph{Sie setzten ihre vergebliche Arbeit fort.}

\hfill\break \textbf{Blaise Zabini:}\\ Der Slytherin-Gemeinschaftsraum konnte genau und präzise als eine remilitarisierte Zone beschrieben werden; in dem Moment, in dem man durch das Porträtloch trat, sah man, dass die linke Hälfte des Raumes definitiv nicht mit der rechten Hälfte sprach und andersherum. Es war ganz klar, das brauchte man niemandem zu erklären, dass man nicht die Möglichkeit hatte, nicht Partei zu ergreifen. An einem Tisch genau in der Mitte des Raumes saß Blaise Zabini allein und machte grinsend seine Hausaufgaben. Er hatte jetzt einen Ruf, und er wollte ihn behalten.

\textbf{Daphne Greengrass und Tracey Davis:}\\ „Machst du heute etwas Interessantes?“, fragte Tracey. „Nö“, sagte Daphne.

\textbf{Harry Potter:}\\ Wenn man hoch genug in Hogwarts ging, sah man nicht viele andere Leute um sich herum, nur Korridore und Fenster und Treppen und das gelegentliche Porträt, und ab und zu einen interessanten Anblick, wie zum Beispiel eine Bronzestatue eines pelzigen Wesens, das wie ein kleines Kind aussah und einen seltsamen flachen Speer hielt...

Wenn man in Hogwarts hoch genug ging, sah man nicht viele andere Menschen, was Harry sehr gelegen kam. Es gab viel schlimmere Orte, um gefangen zu sein, vermutete Harry. Tatsächlich konnte man sich wahrscheinlich keinen besseren Ort vorstellen, um gefangen zu sein, als ein uraltes Schloss mit einer fraktalen, sich ständig verändernden Struktur, die bedeutete, dass einem nie die Orte ausgehen konnten, die man erkunden konnte, voller interessanter Leute und interessanter Bücher und unglaublich wichtigem Wissen, das der Muggelwissenschaft unbekannt war. Hätte man Harry nicht gesagt, dass er nicht gehen kann, hätte er wahrscheinlich die Chance ergriffen, mehr Zeit in Hogwarts zu verbringen, er hätte Ränke geschmiedet und Intrigen geschmiedet, um es zu bekommen. Hogwarts war buchstäblich optimal, vielleicht nicht in allen Bereichen des Möglichen, aber sicherlich auf dem realen Planeten Erde, war es der Ort für maximalen Spaß.

\emph{Wie konnte das Schloss und sein Gelände so viel kleiner, so viel beengender erscheinen, wie konnte der Rest der Welt so viel interessanter und wichtiger werden, in dem Moment, in dem Harry gesagt worden war, dass er nicht gehen dürfe?}\\ \emph{Er hatte Monate hier verbracht und hatte sich damals nicht klaustrophobisch gefühlt.}

\emph{Du kennst die Forschung dazu,} bemerkte ein Teil von ihm, \emph{es sind nur die üblichen Knappheitseffekte, wie damals, als, sobald ein Landkreis Phosphatwaschmittel} \emph{verbot, Leute, die sich vorher nie darum gekümmert hatten, in den nächsten Landkreis fuhren, um riesige Ladungen Phosphatwaschmittel zu kaufen, und Umfragen zeigten, dass sie Phosphatwaschmittel als sanfter und effektiver und sogar leichter ausgießbar einstuften.}.. \emph{und wenn man Zweijährige vor die Wahl stellt zwischen einem Spielzeug im Freien und einem, das durch eine Barriere geschützt ist, um die sie herumgehen können, werden sie das Spielzeug im Freien ignorieren und sich für das hinter der Barriere entscheiden... Verkäufer wissen, dass sie Dinge verkaufen können, indem sie dem Kunden einfach sagen, dass es vielleicht nicht verfügbar ist...}

das stand alles in Cialdinis Buch Influence, alles, was er gerade fühlte, das Gras ist immer grüner auf der Seite, die nicht erlaubt ist. Hätte man Harry nicht gesagt, dass er nicht gehen darf, hätte er wahrscheinlich die Chance ergriffen, den Sommer über in Hogwarts zu bleiben......aber nicht für den Rest seines Lebens. Das war sozusagen das Problem, wirklich. Wer wusste, ob es noch einen Dunklen Lord Voldemort gab, den er besiegen konnte? Wer wusste, ob Er, der nicht genannt werden darf, noch außerhalb der Fantasie eines möglicherweise nicht nur verrückten alten Zauberers existierte? Lord Voldemorts Leichnam wurde knusprig verbrannt aufgefunden, so etwas wie Seelen konnte es nicht wirklich geben. Wie konnte Lord Voldemort noch am Leben sein? Woher wusste Dumbledore, dass er noch am Leben war? Und wenn es keinen Dunklen Lord gab, konnte Harry ihn nicht besiegen und er wäre für immer in Hogwarts gefangen....vielleicht würde er nach seinem siebten Schuljahr, in sechs Jahren, vier Monaten und drei Wochen, legal entkommen dürfen. Es war nicht so lang, wie die Zeitspanne aussah, es schien nur lang genug, um Protonen zerfallen zu lassen.

Nur war es nicht nur das. Es war nicht nur Harrys Freiheit, die auf dem Spiel stand. Der Schulleiter von Hogwarts, der Oberste Hexenmeister des Wizengamots, der Oberste Mugwump der Internationalen Konföderation der Zauberer, schlug leise Alarm.

Er hätte sich an das Versprechen an Hermine erinnern sollen, bevor er nach Askaban ging. \emph{Warum hatte er sich dazu entschlossen, das wieder zu tun?}

\emph{Meine Arbeitshypothese ist, dass du dumm bist,} sagte Hufflepuff.

\emph{Das ist keine nützliche Fehleranalyse,} dachte Harry.

\emph{Wenn du es genauer wissen wills}t, sagte Hufflepuff, \emph{dann hat der Verteidigungsprofessor von Hogwarts gesagt: „Holen wir Bellatrix Black aus Askaban raus!“ und du hast gesagt: „Okay!“ Hey}, sagte Hufflepuff, \emph{ist dir aufgefallen, dass man, wenn man ganz oben ist und die einzelnen Bäume verschwimmen, die Form des Waldes erkennen kann?}

\emph{Warum hatte er das getan...?}

Nicht wegen einer Kosten-Nutzen-Rechnung, das war sicher. Es war ihm zu peinlich gewesen, ein Blatt Papier zu zücken und den zu erwartenden Nutzen zu berechnen, er hatte befürchtet, dass Professor Quirrell ihn nicht mehr respektieren würde, wenn er nein sagte oder auch nur zu sehr zögerte, einem Mädchen in Not zu helfen. Er hatte irgendwo tief in seinem Inneren gedacht, dass, wenn dein mysteriöser Lehrer dir die erste Mission, die erste Chance, den Ruf zum Abenteuer anbietet und du nein sagst, dann geht dein mysteriöser Lehrer angewidert von dir weg und du bekommst nie wieder eine Chance, ein Held zu sein...

...ja, das war es gewesen. Im Nachhinein betrachtet, war es das. Er hatte angefangen zu glauben, sein Leben hätte eine Handlung und hier war eine Wendung, hier war ein Vorschlag seines Mentors, Bellatrix Black aus Askaban zu befreien. Das war der wahre und ursprüngliche Grund für die Entscheidung in dem Sekundenbruchteil gewesen, in dem sie getroffen worden war, wobei sein Gehirn die Erzählung, in der er „Nein“ gesagt hatte, als dissonant wahrgenommen hatte. Und wenn man darüber nachdachte, war das keine rationale Art, Entscheidungen zu treffen.

Professor Quirrells Hintergedanke, die letzten Reste von Slytherins verlorenem Wissen zu erhalten, bevor Bellatrix starb und es unwiderruflich in Vergessenheit geriet, erschien im Vergleich dazu beeindruckend vernünftig; ein Nutzen, der dem, was damals als kleines Risiko erschienen war, angemessen war.\\ Es schien nicht fair, es schien nicht fair, dass dies geschah, wenn er nur für einen winzigen Bruchteil einer Sekunde den Halt an seiner Rationalität verlor, den winzigen Bruchteil einer Sekunde, den sein Gehirn brauchte, um zu entscheiden, dass ihm Ja-Argumente angenehmer waren als Nein-Argumente während der Diskussion, die gefolgt war.

Von hoch oben, weit genug oben, dass die einzelnen Bäume miteinander verschwammen, starrte Harry auf den Wald hinaus. Harry wollte nicht gestehen und seinen Ruf für immer ruinieren und alle wütend auf ihn machen und vielleicht später vom Dunklen Lord getötet werden. Lieber wäre er sechs Jahre lang in Hogwarts gefangen, als sich dem zu stellen. So fühlte er sich. Und so war es in der Tat hilfreich, eine Erleichterung, sich an einen einzigen entscheidenden Faktor klammern zu können, nämlich dass, wenn Harry gestand, Professor Quirrell nach Askaban gehen und dort sterben würde. (Ein Haken, eine Pause, ein Stottern in Harrys Atem.)

\emph{Wenn man es so formulierte... konnte man sogar vorgeben, ein Held zu sein, anstatt ein Feigling.}

Harry hob seinen Blick aus dem Verbotenen Wald, sah hinauf in den klaren, blauen, verbotenen Himmel. Starrte aus den Glasscheiben auf das große, helle, brennende Ding, die flauschigen Dinger, das geheimnisvolle, endlose Blau, in das sie eingebettet waren, diesen seltsamen, neuen, unbekannten Ort.

Es... half tatsächlich, es half ziemlich viel, zu denken, dass seine eigenen Probleme nichts waren im Vergleich zu dem, in Askaban zu sein. Dass es Menschen auf der Welt gab, die wirklich in Schwierigkeiten steckten, und dass Harry Potter nicht zu ihnen gehörte.

\emph{Was sollte er gegen Askaban unternehmen? Was sollte er gegen das magische Britannien unternehmen?...auf welcher Seite stand er jetzt?}

Im hellen Licht des Tages klang alles, was Albus Dumbledore gesagt hatte, sicherlich viel weiser als Professor Quirrell. Besser und heller, moralischer, bequemer, \emph{wäre es nicht schön, wenn es wahr wäre.} Und die Sache war, dass Dumbledore die Dinge glaubte, weil sie schön klangen, aber Professor Quirrell war derjenige, der zurechnungsfähig war. (Wieder das Hängenbleiben in seinem Atem, es passierte jedes Mal, wenn er an Professor Quirrell dachte.)

\emph{Aber nur weil sich etwas nett anhörte, war es auch nicht falsch.} Und wenn der Verteidigungsprofessor einen Makel in seiner Zurechnungsfähigkeit hatte, dann war es, dass er das Leben zu negativ sah.

\emph{Wirklich?} fragte sich der Teil von Harry, der achtzehn Millionen Versuchsergebnisse darüber gelesen hatte, dass Menschen zu optimistisch und zu zuversichtlich sind. \emph{Professor Quirrell ist zu pessimistisch? So pessimistisch, dass seine Erwartungen routinemäßig hinter der Realität zurückbleiben? Stopfe ihn aus und steck ihn in ein Museum, er ist einzigartig. Wer von euch beiden hat das perfekte Verbrechen geplant und dann all die Fehlermöglichkeiten und Rückschläge eingebaut, die euch am Ende den Arsch gerettet haben, nur für den Fall, dass das perfekte Verbrechen schief geht? Hinweis, Hinweis, sein Name war nicht Harry Potter.}

\emph{Aber „pessimistisch“ war nicht das richtige Wort, um Professor Quirrells Problem zu beschreiben - wenn es denn wirklich ein Problem war, und nicht die überlegene Weisheit der Erfahrung.} \emph{Aber für Harry sah es so aus, als würde Professor Quirrell ständig alles im schlechtesten möglichen Licht interpretieren.} \emph{Wenn man Professor Quirrell ein Glas reichte, das zu 90\% voll war, würde er einem sagen, dass die 10\% leere bewiesen, dass sich niemand wirklich für Wasser interessierte.}

Das war eine sehr gute Analogie, jetzt, wo Harry darüber nachdachte.

\emph{Nicht ganz magisches Britannien war wie Askaban, das Glas war weit mehr als halb voll...} Harry starrte hinauf in den strahlend blauen Himmel. \emph{...obwohl, der Analogie folgend, wenn Askaban existierte, dann bewies es vielleicht, dass der 90\% gute Teil aus anderen Gründen dort war, Menschen, die versuchten, eine Show der Freundlichkeit abzuziehen, wie Professor Quirrell es ausgedrückt hatte. Denn wenn sie wirklich gütig wären, hätten sie Askaban nicht erschaffen, sie würden die Festung stürmen, um sie niederzureißen... oder nicht?}

Harry starrte hinauf in den strahlend blauen Himmel.

\emph{Wenn man ein Rationalist sein wollte, musste man furchtbar viele Abhandlungen über Fehler in der menschlichen Natur lesen, und einige dieser Fehler waren unschuldige logische Ausfälle, und einige von ihnen sahen sehr viel dunkler aus.}

Harry starrte hinauf in den strahlend blauen Himmel und dachte an das Milgram-Experiment. Stanley Milgram hatte es gemacht, um die Ursachen des Zweiten Weltkriegs zu erforschen, um zu verstehen, warum die Bürger Deutschlands Hitler gehorcht hatten. Er hatte also ein Experiment entworfen, um den Gehorsam zu untersuchen, um zu sehen, ob die Deutschen aus irgendeinem Grund eher dazu neigten, schädliche Befehle von Autoritätspersonen zu befolgen. Zuerst hatte er eine Pilotversion seines Experiments an amerikanischen Probanden durchgeführt, als Kontrollversuch. Und danach hatte er sich nicht die Mühe gemacht, es in Deutschland zu versuchen.

\textbf{Versuchsapparat}:\\ Eine Reihe von 30 Schaltern, die in einer horizontalen Linie angeordnet waren, mit Beschriftungen, die bei „15 Volt“ begannen und bis zu „450 Volt“ reichten, mit Beschriftungen für jede Gruppe von vier Schaltern.\\ Die erste Gruppe von vier Schaltern ist mit „Leichter Schock“ beschriftet, die sechste Gruppe mit „Schock mit extremer Intensität“, die siebte Gruppe mit „Gefahr: Schwerer Schock', und die beiden letzten Schalter, die übrig blieben, waren einfach mit 'XXX' beschriftet.\\ Und ein Schauspieler, ein Vertrauter des Versuchsleiters, der den echten Versuchspersonen als jemand wie sie erschienen war: jemand, der auf die gleiche Anzeige für Teilnehmer an einem Lernexperiment geantwortet hatte, und der bei einer (manipulierten) Lotterie verloren hatte und zusammen mit den Elektroden auf einen Stuhl geschnallt worden war.

Die echten Versuchspersonen hatten einen leichten Schock von den Elektroden bekommen, nur damit sie sehen konnten, dass es funktioniert. Der echten Versuchsperson war gesagt worden, dass das Experiment die Auswirkungen von Bestrafung auf Lernen und Gedächtnis betraf und dass ein Teil des Tests darin bestand, herauszufinden, ob es einen Unterschied machte, welche Art von Person die Bestrafung verabreichte; und dass die an den Stuhl geschnallte Person versuchen würde, sich Sätze von Wortpaaren einzuprägen, und dass jedes Mal, wenn der „Lernende“ einen Fehler machte, der „Lehrer“ einen sukzessive stärkeren Schock verabreichen sollte. Beim 300-Volt-Pegel hörte der Akteur auf, Antworten zu rufen und begann, gegen die Wand zu treten, woraufhin der Versuchsleiter die Versuchspersonen anwies, Nicht-Antworten als falsche Antworten zu behandeln und fortzufahren. Beim 315-Volt-Pegel würde das Hämmern gegen die Wand wiederholt werden. Danach war nichts mehr zu hören.

Wenn die Versuchsperson widersprach oder sich weigerte, einen Schalter zu drücken, sagte der Versuchsleiter, der eine teilnahmslose Haltung bewahrte und einen grauen Laborkittel trug: „Bitte fahren Sie fort“, dann: „Das Experiment erfordert, dass Sie fortfahren“, dann: „Es ist absolut notwendig, dass Sie fortfahren“, dann: „Sie haben keine andere Wahl, Sie müssen fortfahren“. Wenn die vierte Aufforderung immer noch nicht funktionierte, wurde das Experiment dort abgebrochen.

Vor der Durchführung des Experiments hatte Milgram den Versuchsaufbau beschrieben und dann vierzehn Psychologie-Professoren gefragt, wie viel Prozent der Versuchspersonen ihrer Meinung nach den ganzen Weg bis zur 450-Volt-Stufe gehen würden, wie viel Prozent der Versuchspersonen den letzten der beiden mit XXX markierten Schalter drücken würden, nachdem das Opfer aufgehört hatte zu reagieren. Die pessimistischste Antwort war 3\%. \emph{Die tatsächliche Zahl lag bei 26 von 40.}

Die Versuchspersonen schwitzten, stöhnten, stotterten, lachten nervös, bissen sich auf die Lippen, gruben ihre Fingernägel in ihr Fleisch. Aber auf die Aufforderung des Versuchsleiters hin hatten sie, die meisten von ihnen, weitergemacht und das verabreicht, was sie für schmerzhafte, gefährliche, möglicherweise tödliche Elektroschocks hielten. \emph{Bis zum Ende.}

Harry konnte Professor Quirrell im Geiste lachen hören; die Stimme des Verteidigungsprofessors sagte etwas in der Art von: \emph{Nun, Mr. Potter, selbst ich war nicht so zynisch gewesen; ich wusste, dass Männer ihre wertvollsten Prinzipien für Geld und Macht verraten würden, aber mir war nicht klar, dass auch ein strenger Blick genügte.}

Es war gefährlich, zu versuchen, evolutionäre Psychologie zu erraten, wenn man kein professioneller Evolutionspsychologe war; aber als Harry über das Milgram-Experiment gelesen hatte, war ihm der Gedanke gekommen, dass Situationen wie diese in der Umgebung der Vorfahren wahrscheinlich oft vorgekommen waren, und dass die meisten potenziellen Vorfahren, die versucht hatten, der Autorität zu widersprechen, tot waren. Oder dass es ihnen zumindest weniger gut ergangen war als den Gehorsamen. Die Menschen hielten sich selbst für gut und moralisch, aber wenn es darauf ankam, legte sich irgendein Schalter in ihrem Gehirn um, und es war plötzlich viel schwieriger, sich der Autorität heldenhaft zu widersetzen, als sie dachten. Selbst wenn man es schaffen würde, wäre es nicht einfach, es wäre keine mühelose Zurschaustellung von Heldentum.

\emph{Du würdest zittern, deine Stimme würde brechen, du würdest Angst haben; würdest du selbst dann in der Lage sein, der Autorität zu trotzen?}

Harry blinzelte, denn sein Gehirn hatte gerade die Verbindung zwischen Milgrams Experiment und dem hergestellt, was Hermine an ihrem ersten Tag im Verteidigungskurs getan hatte: Sie hatte sich geweigert, einen Mitschüler zu erschießen, selbst als die Autorität ihr gesagt hatte, dass sie es tun müsse, sie hatte gezittert und Angst gehabt, aber sie hatte sich trotzdem geweigert. Harry hatte das direkt vor seinen eigenen Augen gesehen, und er hatte die Verbindung bis jetzt noch nicht hergestellt...

Harry starrte auf den röter werdenden Horizont, die Sonne sank tiefer, der Himmel verblasste, verdunkelte sich, auch wenn der größte Teil noch blau war, bald würde es Nacht werden. Die goldenen und roten Farben der Sonne und des Sonnenuntergangs erinnerten ihn an Fawkes; und Harry fragte sich einen Moment lang, ob es traurig sein musste, ein Phönix zu sein und zu rufen und zu schreien, ohne erhört zu werden. Aber Fawkes würde niemals aufgeben, so oft er auch starb, er würde immer wiedergeboren werden, denn Fawkes war ein Wesen des Lichts und des Feuers, und die Verzweiflung über Askaban gehörte genauso zur Dunkelheit wie Askaban selbst. Wenn man Ihnen ein halb leeres und ein halb volles Glas gab, dann war das die Art und Weise, wie die Realität war, das war die Wahrheit und es war so; aber Sie hatten immer noch die Wahl, wie Sie sich dabei fühlen wollten, ob Sie über die leere Hälfte verzweifeln oder sich über das Wasser freuen würden, das da war.

Milgram hatte einige andere Variationen seines Tests ausprobiert. Im achtzehnten Experiment brauchte die Versuchsperson dem auf dem Stuhl festgeschnallten Opfer nur die Testwörter zuzurufen und die Antworten zu notieren, während jemand anderes die Schalter drückte. Es war das gleiche offensichtliche Leiden, das gleiche hektische Hämmern, gefolgt von Stille; aber es waren nicht Sie, die den Schalter drückten. Sie schauten nur zu und lasen der gequälten Person die Fragen vor. 37 von 40 Versuchspersonen hatten ihre Teilnahme an diesem Experiment bis zum Ende fortgesetzt, dem 450-Volt-Ende, das mit „XXX“ markiert war.

Und wenn du Professor Quirrell wärst, hättest du vielleicht beschlossen, das zynisch zu finden. Aber 3 von 40 Probanden hatten sich geweigert, bis zum Ende mitzumachen. Wie Hermine. Es gab sie wirklich, die Leute, die keinen einfache Fluch auf einen Mitschüler abfeuern würden, selbst wenn der Verteidigungsprofessor es ihnen befehlen würde. Diejenigen, die während des Holocausts Zigeuner und Juden und Homosexuelle auf ihren Dachböden beherbergt hatten und dafür manchmal ihr Leben verloren. Gehörten diese Leute zu einer anderen Spezies als die Menschheit? Hatten sie ein zusätzliches Gerät in ihrem Kopf, ein zusätzliches Stück neuronaler Schaltkreise, das die weniger Sterblichen nicht besaßen? Aber das war unwahrscheinlich, angesichts der Logik der sexuellen Fortpflanzung, die besagte, dass die Gene für komplexe Maschinen irreparabel verschlüsselt würden, wenn sie nicht universell wären.

\emph{Aus welchen Teilen Hermine auch immer gemacht war, jeder hatte dieselben Teile irgendwo in sich...}

...naja, das war ein netter Gedanke, aber er stimmte nicht ganz, es gab so etwas wie buchstäbliche Hirnschäden, Menschen konnten Gene verlieren und die komplexe Maschinerie konnte aufhören zu funktionieren, es gab Soziopathen und Psychopathen, Menschen, denen das Zeug dazu fehlte. Vielleicht war Lord Voldemort so geboren worden, oder vielleicht hatte er das Gute gekannt und sich dennoch für das Böse entschieden; an diesem Punkt spielte es nicht die geringste Rolle. Aber eine übergroße Mehrheit der Bevölkerung sollte in der Lage sein, das zu lernen, was Hermine und die Holocaust-Verweigerer taten. Die Leute, die das Milgram-Experiment durchlaufen hatten, die gezittert und geschwitzt und nervös gelacht hatten, als sie bis zum Drücken der Schalter mit der Aufschrift „XXX“ vorgedrungen waren, viele von ihnen hatten Milgram hinterher geschrieben, um sich dafür zu bedanken, was sie über sich selbst gelernt hatten. Auch das war Teil der Geschichte, der Legende dieses legendären Experiments.

Die Sonne war jetzt fast unter den Horizont gesunken, ein letzter goldener Zipfel lugte über die fernen Baumwipfel. Harry schaute sie an, diese Sonnenspitze, seine Brille sollte gegen UV-Strahlung geschützt sein, so dass er direkt in sie schauen können sollte, ohne seine Augen zu schädigen. Harry starrte direkt darauf, auf den winzigen Bruchteil des Lichts, der nicht verdeckt und blockiert und versteckt war, \emph{auch wenn es nur 3 Teile von 40 waren, die anderen 37 Teile waren irgendwo da. Die 7,5 \% des Glases, die voll waren, was bewies, dass die Menschen sich wirklich um das Wasser kümmerten, auch wenn diese Kraft der Fürsorge in ihnen selbst zu oft besiegt wurde. Wenn die Menschen sich wirklich nicht gekümmert hätten, wäre das Glas wirklich leer gewesen. Wenn jeder innerlich wie Du-weißt-schon-wer gewesen wäre, insgeheim klug und egoistisch, dann hätte es überhaupt keinen Widerstand gegen den Holocaust gegeben.}

Harry schaute in den Sonnenuntergang, am zweiten Tag des Restes seines Lebens, und wusste, dass er die Seiten gewechselt hatte. Weil er nicht mehr daran glauben konnte, er konnte es wirklich nicht, nicht nachdem er in Askaban war. Er konnte nicht tun, wofür 37 von 40 Leuten ihn wählen würden. Jeder mochte in sich haben, was es brauchte, um Hermine zu sein, und eines Tages würden sie es vielleicht erfahren; aber eines Tages war nicht jetzt, nicht hier, nicht heute, nicht in der realen Welt. Wenn man auf der Seite von drei von vierzig Leuten stand, dann war man keine politische Mehrheit, und Professor Quirrell hatte Recht gehabt, Harry würde seinen Kopf nicht in Unterwerfung neigen, wenn das geschah.

Es hatte eine Art von schrecklicher Angemessenheit an sich. Man sollte nicht nach Askaban gehen und zurückkommen, ohne seine Meinung über etwas Wichtiges geändert zu haben.

\emph{Hat Professor Quirrell also recht?} fragte Slytherin. \emph{Mal abgesehen davon, ob er gut oder böse ist, hat er Recht? Bist du für sie, ob sie es wissen oder nicht, ihr nächster Lord? Wir lassen den dunklen Teil einfach weg, da ist er wieder zynisch. Aber ist es deine Absicht, jetzt zu herrschen?}

\emph{Ich muss sagen, das macht sogar mich nervös. Glaubst du, man kann dir die Macht anvertrauen?,} sagte Gryffindor.

\emph{Gibt es nicht irgendeine Regel, dass Leute, die Macht wollen, sie nicht haben sollten? Vielleicht sollten wir stattdessen Hermine zur Herrscherin machen. Glaubst du, du bist in der Lage, eine Gesellschaft zu leiten, ohne dass sie innerhalb von drei Wochen im totalen Chaos versinkt?} sagte Hufflepuff. \emph{Stell dir vor, wie laut Mum schreien würde, wenn sie hören würde, dass du zum Premierminister gewählt wurdest, und jetzt frag dich, ob sie damit wirklich falsch liegt?}

\emph{Eigentlich,} sagte Ravenclaw, \emph{muss ich darauf hinweisen, dass dieser ganze politische Kram überwältigend langweilig klingt. Wie wär's, wenn wir den ganzen Wahlkampf Draco überlassen und uns an die Wissenschaft halten? Darin sind wir eigentlich gut, und es ist bekannt, dass das auch den Zustand der Menschheit verbessert, wisst ihr.}

\emph{Langsam}, dachte Harry bei seinen Bauteilen, \emph{wir müssen nicht gleich alles entscheiden. Wir dürfen das Problem so gut wie möglich durchdenken, bevor wir zu einer Lösung kommen.}

Der letzte Teil der Sonne sank unter den Horizont. Es war seltsam, dieses Gefühl, nicht so recht zu wissen, wer man war, auf welcher Seite man stand, sich über etwas so Wesentliches noch nicht entschieden zu haben, es lag ein ungewohntes Gefühl von Freiheit darin... Und das erinnerte ihn an das, was Professor Quirrell auf seine letzte Frage gesagt hatte, das ihn an Professor Quirrell erinnerte, das ihm das Atmen wieder schwer machte, das dieses Brennen in Harrys Kehle auslöste, das seine Gedanken wieder um diese Schleife der Steigspirale schickte.

\emph{Warum war er jetzt so traurig, wann immer er an Professor Quirrell dachte?}

Harry war es gewohnt, sich selbst zu kennen, und er wusste nicht, warum er so traurig war... Es fühlte sich an, als hätte er Professor Quirrell für immer verloren, ihn in Askaban verloren, so fühlte es sich an. So sicher, als wäre der Verteidigungsprofessor von Dementoren gefressen worden, verzehrt in den leeren Räumen.

\emph{Ich habe ihn verloren! Warum habe ich ihn verloren? Weil er Avada Kedavra gesagt hat und es in Wirklichkeit einen ganz guten Grund gab, auch wenn ich ihn ein paar Stunden lang nicht gesehen habe? Warum können die Dinge nicht wieder so werden, wie sie waren?}

Aber es war nicht das Avada Kedavra gewesen. Das mochte eine Rolle dabei gespielt haben, eine Struktur von Rationalisierungen und Zurückweichen und vorsichtigem Nicht-Denken über bestimmte Dinge irreversibel zum Einsturz zu bringen. Aber es war nicht das Avada Kedavra gewesen, das war nicht das Beunruhigende gewesen, das Harry gesehen hatte.

\emph{Was habe ich gesehen...?}

Harry blickte in den verblassenden Himmel.

\emph{Er hatte gesehen, wie Professor Quirrell sich in einen hartgesottenen Verbrecher verwandelte, während er dem Auror gegenüberstand, und der scheinbare Wechsel der Persönlichkeiten war mühelos und vollständig gewesen. Eine andere Frau hatte den Verteidigungsprofessor als 'Jeremy Jaffe' gekannt.}

\emph{Wie viele verschiedene Menschen sind Sie eigentlich?}\\ \emph{\textbf{Ich kann nicht behaupten, dass ich mitgezählt hätte.}}\\ \emph{\textbf}\strut \\ Man konnte nicht anders, als sich zu fragen, ob „Professor Quirrell“ nur ein weiterer Name auf der Liste war, nur eine weitere Person, in die man sich verwandelt hatte, die man erfunden hatte, um irgendeinem unerklärlichen Ziel zu dienen.\\ Harry würde sich jetzt immer fragen, jedes Mal, wenn er mit Professor Quirrell sprach, ob es eine Maske war und welches Motiv hinter dieser Maske steckte. Mit jedem trockenen Lächeln würde Harry versuchen zu erkennen, was die Hebel auf den Lippen betätigte.

\emph{Werden andere Leute anfangen, so von mir zu denken, wenn ich zu sehr Slytherin werde? Wenn ich zu viele Intrigen spinne, werde ich dann nie wieder jemanden anlächeln können, ohne dass sie sich fragen, was ich wirklich damit meine?}

Vielleicht gab es eine Möglichkeit, das Vertrauen in den äußeren Schein wiederherzustellen und eine normale menschliche Beziehung wieder möglich zu machen, aber Harry fiel nicht ein, was das sein könnte.

So hatte Harry Professor Quirrell verloren, nicht die Person, aber die... \emph{Verbindung}...

\emph{Warum tat das so sehr weh? Warum fühlte er sich jetzt so einsam? Sicherlich gab es andere Menschen, vielleicht bessere Menschen, denen er vertrauen und mit denen er sich anfreunden konnte? Professor McGonagall, Professor Flitwick, Hermine, Draco, ganz zu schweigen von Mum und Dad, Harry war ja nicht allein... Nur...}

Ein würgendes Gefühl stieg in Harrys Kehle auf, als er verstand. Nur Professor McGonagall, Professor Flitwick, Hermine, Draco, sie alle wussten manchmal Dinge, die Harry nicht wusste, aber... Sie überragten Harry nicht in seinem eigenen Machtbereich; das Genie, das sie besaßen, war nicht wie sein Genie, und sein Genie war nicht wie ihres; er konnte sie als Gleichgestellte ansehen, aber nicht als seine Vorgesetzten. Keiner von ihnen war, keiner von ihnen konnte jemals...

\emph{Harrys Mentor sein...}

Das war es, was Professor Quirrell gewesen war. Er war es, den Harry verloren hatte. Und die Art und Weise, wie er seinen ersten Mentor verloren hatte, würde es Harry vielleicht erlauben, ihn zurückzubekommen, oder auch nicht. Vielleicht würde er eines Tages alle verborgenen Absichten von Professor Quirrell kennen und die Zweifel zwischen ihnen würden verschwinden; aber selbst wenn das möglich wäre, schien es nicht sehr wahrscheinlich zu sein.

Es gab einen Windstoß, außerhalb von Hogwarts, er beugte die leeren Bäume, kräuselte den See, dessen Herz noch nicht gefroren war, machte ein flüsterndes Geräusch, als er an dem Fenster vorbeiging, das auf die halbgefrorene Welt blickte, und Harrys Gedanken wanderten eine Zeit lang nach außen. Dann kehrten sie wieder nach innen zurück, zur nächsten Stufe der Spirale.

\emph{Warum bin ich anders als die anderen Kinder in meinem Alter?}

Wenn Professor Quirrells Antwort darauf eine Ausflucht gewesen war, dann war es eine sehr gut kalkulierte. Tiefgründig genug und komplex genug, voll von Andeutungen versteckter Bedeutungen, um als Falle für einen Ravenclaw zu dienen, der sich nicht durch weniger ablenken ließ. Oder vielleicht hatte Professor Quirrell seine Antwort ehrlich gemeint. Wer wusste schon, welches Motiv den Hebel auf diesen Lippen betätigt haben könnte?

So viel kann ich sagen, Mr. Potter: Du bist bereits ein Okklumens, und ich denke, du wirst bald ein perfekter Okklumens sein. Identität bedeutet für solche wie uns nicht das, was sie für andere Menschen bedeutet. Jeder, den wir uns vorstellen können, können wir sein; und der wahre Unterschied zu Ihnen, Mr. Potter, ist, dass du eine ungewöhnlich gute Vorstellungskraft hast. Ein Dramatiker muss seine Figuren in sich aufnehmen, er muss größer sein als sie, um sie in seinem Geist darstellen zu können. Für einen Schauspieler oder Spion oder Politiker ist die Grenze seines eigenen Durchmessers die Grenze dessen, wer er vorgeben kann zu sein, die Grenze dessen, welches Gesicht er als Maske tragen darf. Aber für solche wie dich und mich kann jeder, den wir uns vorstellen können, in Wirklichkeit und nicht zum Schein sein. Während du dich als ein Kind vorstelltest, Mr. Potter, warst du ein Kind. Und doch gibt es andere Existenzen, die du unterstützen könntest, größere Existenzen, wenn du es willst. Warum bist du so frei und so groß in deinem Umfang, während andere Kinder deines Alters klein und beschränkt sind? Warum kannst du dir ein erwachseneres Selbst vorstellen und werden, als ein bloßes Kind eines Dramatikers zu komponieren imstande sein sollte? Das weiß ich nicht, und ich darf nicht sagen, was ich vermute. Aber was du hast ist Freiheit.

\emph{Wenn das ein Schneegestöber war, dann war es ein verdammt ablenkendes.}

Und der noch beunruhigendere Gedanke war, dass Professor Quirrell nicht bemerkt hatte, wie verstört Harry sein würde, wie falsch diese Rede für ihn klingen würde, wie sehr sie sein Vertrauen in Professor Quirrell beschädigen würde.

\emph{Es sollte immer eine wirkliche Person geben, die man wirklich war, die im Zentrum} \emph{von allem stand...}

Harry starrte hinaus in die hereinbrechende Nacht, in die zunehmende Dunkelheit.

\emph{...oder?}

Es war schon fast Schlafenszeit, als Hermine die verstreuten Atemzüge hörte und von ihrem Exemplar \emph{'Die Geschichte von Beauxbatons'} aufblickte auf, um den vermissten Jungen zu sehen, den Jungen, der an jenem Sonntag beim Mittagessen vermisst worden war, dessen Nichterscheinen beim Abendessen von Gerüchten begleitet worden war - und sie hatte ihnen nicht geglaubt, weil sie völlig lächerlich waren, aber sie hatte ein kleines mulmiges Gefühl in sich gespürt -, dass er sich von Hogwarts zurückgezogen hatte, um Bellatrix Black zur Strecke zu bringen.

„Harry!?“, kreischte sie, ohne zu bemerken, dass sie zum ersten Mal seit einer Woche direkt mit ihm sprach, und ohne zu bemerken, wie einige andere Schüler beim Klang ihres Geschreis quer durch den Ravenclaw-Gemeinschaftsraum aufhorchten.

Harrys Blick hatte sich bereits zu ihr gehoben, er ging bereits auf sie zu, so dass sie auf halbem Weg von ihrem Stuhl stehen blieb - ein paar Augenblicke später saß Harry ihr gegenüber und steckte seinen Zauberstab weg, nachdem er einen Schweigezauber um sie herum gewirkt hatte. (Und eine ganze Menge Ravenclaws versuchten, nicht so auszusehen, als würden sie zusehen.)

„Hey“, sagte Harry. Seine Stimme schwankte. „Ich habe dich vermisst. Wirst du... jetzt wieder mit mir reden?“

Hermine nickte nur, ihr fiel nicht ein, was sie sagen sollte. Sie hatte Harry auch vermisst, aber sie stellte mit einer Art Schuldgefühl fest, dass es für ihn vielleicht noch viel schlimmer gewesen wäre. Sie hatte andere Freunde, Harry... es fühlte sich manchmal nicht fair an, dass Harry nur mit ihr so redete, so dass sie mit ihm reden musste; aber Harry hatte einen Blick auf sich, als wären auch ihm unfaire Dinge widerfahren.

„Was ist denn los?“, sagte sie. „Es gibt alle möglichen Gerüchte. Es gab Leute, die sagten, du wärst abgehauen, um gegen Bellatrix Black zu kämpfen, es gab Leute, die sagten, du wärst abgehauen, um dich Bellatrix Black anzuschließen -“ und diese Gerüchte hatten besagt, dass Hermine die Sache mit dem Phönix nur erfunden hatte, und sie hatte geschrien, dass der ganze Ravenclaw-Gemeinschaftsraum es gesehen hatte, und dann hatte das nächste Gerücht behauptet, sie hätte auch diesen Teil erfunden, was eine Dummheit von so unvorstellbarem Ausmaß war, dass es sie völlig verblüffte.

„Ich kann nicht darüber reden“, sagte Harry im leisen Flüsterton. „Ich kann über vieles nicht reden. Ich wünschte, ich könnte dir alles erzählen“, seine Stimme schwankte, „aber ich kann nicht... Ich schätze, wenn es hilft oder so, ich gehe nicht mehr zum Mittagessen mit Professor Quirrell...“ Harry legte seine Hände über sein Gesicht und bedeckte seine Augen.

Hermine spürte das mulmige Gefühl in ihrem Magen. „Weinst du etwa?“, fragte Hermine.

„Ja“, sagte Harry, seine Stimme klang ein wenig atemlos. „Ich will nicht, dass es jemand anderes sieht.„

Es herrschte ein kurzes Schweigen. Hermine wollte helfen, aber sie wusste nicht, was sie bei einem weinenden Jungen tun sollte, und sie wusste nicht, was vor sich ging; sie hatte das Gefühl, dass um sie herum - nein, um Harry herum - riesige Dinge geschahen, und wenn sie wüsste, was es war, wäre sie wahrscheinlich verängstigt oder alarmiert oder so, aber sie wusste nichts. „Hat Professor Quirrell etwas falsch gemacht?“, fragte sie schließlich.

„Das ist nicht der Grund, warum ich nicht mehr mit ihm zum Mittagessen gehen kann“, sagte Harry, immer noch in diesem knappen Flüsterton, die Hände über die Augen gepresst. „Das war die Entscheidung des Schulleiters. Aber ja, Professor Quirrell hat einige Dinge zu mir gesagt, die mich dazu gebracht haben, ihm weniger zu vertrauen, denke ich...“ Harrys Stimme klang sehr zittrig. „Ich fühle mich im Moment ziemlich allein.“

Hermine legte ihre Hand auf ihre Wange, wo Fawkes sie gestern berührt hatte. Sie musste immer wieder an diese Berührung denken, vielleicht weil sie wollte, dass sie wichtig war, dass sie ihr etwas bedeutete... „Kann ich irgendwie helfen?“, fragte sie.

„Ich möchte etwas Normales tun“, sagte Harry hinter seinen Händen. „Etwas ganz Normales für Hogwartsschüler im ersten Jahr. Etwas, das Elf- und Zwölfjährige wie wir tun sollten. Zum Beispiel eine Partie Snape explodiert spielen oder so... Ich nehme nicht an, dass du die Karten hast oder die Regeln kennst oder so was in der Art?“

„Ähm... Ich kenne die Regeln nicht, um ehrlich zu sein...“, sagte Hermine. „Ich weiß, dass sie explodieren.“

„Ich nehme an Koboldsteine auch nicht?“, sagte Harry.

„Du kennst die Regeln nicht, ich auch nicht und sie spucken dich an. Das sind Jungenspiele, Harry!“

Es gab eine Pause. Harry rieb sich mit den Händen über das Gesicht, um es abzuwischen, und dann nahm er die Hände weg; und dann sah er sie an und wirkte ein wenig hilflos. „Nun“, sagte Harry, „was machen Zauberer und Hexen in unserem Alter, wenn sie, du weißt schon, die Art von sinnlosen, albernen Spielen spielen, die wir in diesem Alter spielen sollen?“

„Himmel und Hölle?“, fragte Hermine. „Springseil? Einhornangriff? Ich weiß es nicht, ich lese Bücher!“

Harry fing an zu lachen, und Hermine fing an, mit ihm zu kichern, obwohl\\ sie nicht genau wusste, warum, aber es war lustig. „Ich schätze, das hat ein bisschen geholfen“, sagte Harry. „Eigentlich glaube ich, dass es mehr geholfen hat, als eine Stunde lang mit Koboldsteinen zu spielen, also danke, dass du es warst. Und egal was passiert, ich lasse mir nicht alles, was ich über Kalkül weiß, von irgendjemandem abnehmen. Eher sterbe ich.“

„Was?“, sagte Hermine. „Warum - warum solltest du das jemals tun wollen?“

Harry stand vom Tisch auf, und es gab einen Schwall wiederhergestellter Hintergrundgeräusche, als sein Aufstehen den Zauber durchbrach. „Ich bin ein bisschen müde, also gehe ich ins Bett“, sagte Harry, jetzt war seine Stimme normal und schief, „ich muss ein bisschen verlorene Zeit aufholen, aber wir sehen uns beim Frühstück und dann in Kräuterkunde, wenn das in Ordnung ist. Ganz zu schweigen davon, dass es nicht fair wäre, meine ganze Depression bei dir abzuladen. Gute Nacht, Hermine.“

„Gute Nacht, Harry“, sagte sie und fühlte sich sehr verwirrt und beunruhigt. „Angenehme Träume.“

Harry stolperte ein wenig, als sie das sagte, und dann ging er weiter in Richtung der Treppe, die zu den Schlafsälen der Erstklässler führte.

Harry drehte den Schweigezauber ganz nach oben, an das Kopfende seines Bettes, damit er niemanden mehr aufwecken würde, wenn er schrie. Stellte seinen Wecker so ein, dass er zum Frühstück geweckt wurde (falls er um diese Zeit nicht schon wach war, falls er überhaupt schlief). Ging ins Bett, legte sich hin -\\ \emph{\hfill\break - fühlte die Beule unter seinem Kissen.}

Harry starrte hinauf zum Baldachin über seinem Bett. Zischte unter seinem Atem: „Oh, das soll wohl ein Scherz sein...“

Es dauerte ein paar Sekunden, bis Harry den Mut aufbrachte, sich im Bett aufzusetzen, die Decke über sich und sein Kissen zu ziehen, um die Tat vor den anderen Jungen zu verbergen, ein schwach leuchtendes Lumos zu zaubern und zu sehen, was unter seinem Kissen lag. Es war ein Pergament und ein Spielkartenspiel. Auf dem Pergament stand:

\emph{Ein kleiner Vogel hat mir gesagt, dass Dumbledore die Tür deines Käfigs geschlossen hat. In diesem Fall muss ich zugeben, dass Dumbledore Recht haben könnte. Bellatrix Black ist wieder auf die Welt losgelassen worden, und das ist für keinen guten Menschen eine gute Nachricht. Wenn ich an Dumbledores Stelle wäre, würde ich wohl dasselbe tun. Aber nur für den Fall... Das Hexeninstitut von Salem in Amerika nimmt auch Jungs auf, trotz des Namens. Sie sind gute Menschen und würden dich sogar vor Dumbledore beschützen, wenn du es brauchst. In England gilt, dass man Dumbledores Erlaubnis braucht, um ins magische Amerika zu emigrieren, aber das magische Amerika ist anderer Meinung.}

\hfill\break

\emph{Willst du dich in letzter Konsequenz also außerhalb der Mauern von Hogwarts begeben musst du den Herzkönig aus diesem Kartenspiel in zwei Teile reißen. Dass du dazu nur im äußersten Notfall greifen solltest, versteht sich von selbst.}

\emph{Mach's gut, Harry Potter.\\ - Weihnachtsmann}

\hfill\break Harry starrte auf das Kartenspiel hinunter. Es konnte ihn nirgendwo anders hinbringen, nicht im Moment, Portschlüssel funktionierten hier nicht. Aber er fühlte sich immer noch entnervt von der Aussicht, es aufzuheben, selbst um es in seinem Koffer zu verstecken... Nun, er hatte bereits das Pergament aufgehoben, das genauso gut mit einer Falle hätte verzaubert sein können, falls es sich um eine Falle handelte. \emph{Aber trotzdem.}\\ „Wingardium Leviosa“, flüsterte Harry und ließ das Päckchen mit den Karten neben der Stelle schweben, wo sein Wecker in einer Tasche des Kopfteils ruhte. Er würde sich morgen darum kümmern. Und dann legte sich Harry zurück ins Bett und schloss die Augen, um ohne Phönix zu träumen, der ihn beschützen sollte, und um seine Rechnung zu begleichen.

Er erwachte mit einem Keuchen des Entsetzens, kein Schrei, er hatte in dieser Nacht noch nicht geschrien, aber seine Decke hatte sich um ihn herum verheddert, wo seine schlafende Gestalt zusammengezuckt war, als er davon träumte, zu rennen, zu versuchen, den Lücken im Raum zu entkommen, die ihn durch einen Korridor aus Metall verfolgten, der von schwachem Gaslicht erhellt wurde, ein endlos langer Korridor aus Metall, beleuchtet von schummrigem Gaslicht, und er hatte im Traum nicht gewusst, dass die Berührung dieser Lücken bedeutete, dass er schrecklich sterben und seinen noch atmenden Körper leer zurücklassen würde, Alles, was er wusste, war, dass er rennen musste und rennen und rennen vor den Wunden in der Welt, die ihm nachrutschten -

Harry begann wieder zu weinen, nicht wegen des Schreckens der Verfolgung, sondern weil er weggelaufen war, während hinter ihm jemand um Hilfe schrie, schrie, dass er zurückkommen und sie retten sollte, ihr helfen, sie wurde gefressen, sie würde sterben, und in dem Traum war Harry weggelaufen, anstatt ihr zu helfen.

\textbf{"GEH NICHT!"} Die Stimme kam schreiend von hinter der Metalltür.

\hfill\break

\textbf{"Nein, nein, nein, geh nicht, nimm es nicht weg, nimm es nicht weg -"}

\emph{Warum hatte sich Fawkes jemals auf seine Schulter gestützt? Er war weggelaufen. Fawkes sollte ihn hassen. Fawkes sollte Dumbledore hassen. Er war weggelaufen. Fawkes sollte jeden hassen -}

Der Junge war nicht wach, träumte nicht, seine Gedanken waren durcheinander und verwirrt in den Schattenländern, die an Schlaf und Wachsein grenzten, ungeschützt durch die Sicherheitsschienen, die sein bewusster Verstand sich selbst auferlegte, die sorgfältigen Regeln und Zensoren. In diesem Schattenland war sein Gehirn genug aufgewacht, um zu denken, aber etwas anderes war zu schläfrig, um zu handeln; seine Gedanken liefen frei und wild, unbehindert von seinem Selbstkonzept, den Idealen seines wachen Ichs, was er nicht denken sollte. Das war die Freiheit der Träume seines Gehirns, während sein Selbstkonzept schlief. Frei, immer wieder Harrys neuen schlimmsten Albtraum zu wiederholen:

\emph{„Nein, ich wollte das nicht, bitte stirb nicht!“}

\hfill\break

\emph{„Nein, ich wollte das nicht, bitte stirb nicht!“}

\hfill\break

\emph{"Nein, ich wollte das nicht, bitte stirb nicht!"}

Neben dem Selbsthass wuchs in ihm eine Wut, ein furchtbar heißer Zorn und eiskalter Hass, auf die Welt, die ihr das angetan hatte und auf sich selbst, und in seinem halbwachen Zustand phantasierte Harry Auswege, phantasierte Wege aus dem moralischen Dilemma, er stellte sich vor, wie er über dem riesigen dreieckigen Horror von Askaban schwebte, und eine Beschwörungsformel flüsterte, wie man sie auf der Erde noch nie gehört hatte, ein Flüstern, das über den ganzen Himmel hallte und auf der anderen Seite der Welt zu hören war, und es gab eine Explosion aus silbernem Patronusfeuer wie eine Atomexplosion, die alle Dementoren in einem Augenblick zerriss und die Metallwände von Askaban zerriss, die langen Korridore und all die schummrigen orangenen Lichter zerschmetterte, und dann, einen Moment später, erinnerte sich sein Gehirn daran, dass da drin Menschen waren, und schrieb die Halbtraumfantasie um, um all die Gefangenen lachend zu zeigen, wie sie in Scharen aus dem brennenden Wrack von Askaban davonflogen, wobei das silberne Licht das Fleisch an ihren Gliedern wiederherstellte, während sie flogen,

Er hatte auf sein Leben und seine Magie und seine Kunst als Rationalist geschworen, er hatte auf alles geschworen, was ihm heilig war und auf all seine glücklichen Erinnerungen, er hatte seinen Eid geleistet, und jetzt musste er etwas tun, musste etwas tun, musste etwas tun - vielleicht war es sinnlos. Vielleicht war der Versuch, Regeln zu befolgen, sinnlos. Vielleicht brannte man Askaban einfach nieder. Und tatsächlich hatte er sich geschworen, es zu tun, also war es das, was er jetzt tun musste. Er würde alles tun, was nötig war, um Askaban loszuwerden, das war alles. Wenn das bedeutete, Britannien zu regieren, \emph{schön}, wenn das bedeutete, einen Zauberspruch zu finden, den er flüstern konnte, der im ganzen Himmel widerhallte, \emph{was auch immer,} \emph{das Wichtigste war, Askaban zu zerstören.}

\emph{Das war die Seite, auf der er stand, das war, wer er war, also war es erledigt.}

Sein wacher Verstand hätte noch viel mehr Details verlangt, bevor er das als Antwort akzeptiert hätte, aber in seinem halb träumenden Zustand fühlte es sich wie eine ausreichende Lösung an, um seinen müden Geist wieder richtig einschlafen zu lassen und den nächsten Albtraum zu träumen.

\textbf{Letztes Nachspiel:}\\ Sie erwachte mit einem Keuchen des Entsetzens, einer Unterbrechung ihrer Atmung, die ihr das Gefühl gab, keine Luft mehr zu bekommen, und doch bewegten sich ihre Lungen nicht, sie erwachte mit einem stimmlosen Schrei auf den Lippen und ohne Worte, keine Worte kamen hervor, denn sie konnte nicht verstehen, was sie gesehen hatte, es war zu groß für sie, um es zu erfassen, und es nahm immer noch Gestalt an, sie konnte diese formlose Gestalt nicht in Worte fassen, und so konnte sie sie nicht entladen, konnte sie nicht entladen und wieder unschuldig und unwissend werden.

„Wie spät ist es?“, flüsterte sie.\\ Ihr goldener, juwelenbesetzter Wecker, der schöne und magische und \emph{teure} Wecker, den ihr der Schulleiter bei ihrer Einstellung in Hogwarts geschenkt hatte, flüsterte zurück: „Etwa zwei Uhr morgens. Leg dich wieder schlafen.“

Ihre Laken waren schweißgetränkt, ihr Nachthemd schweißgetränkt, sie nahm ihren Zauberstab neben dem Kopfkissen und säuberte sich, bevor sie versuchte, wieder einzuschlafen, was ihr schließlich auch gelang.

\emph{Sybill Trelawney schlief wieder ein.}

