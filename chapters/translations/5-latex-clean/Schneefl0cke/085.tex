

\hypertarget{multiple-hypothesentests}{% \section{86. Multiple Hypothesentests}\label{multiple-hypothesentests}}

\textbf{\uline{Multiple Hypothesentests}}

(Internationale Nachrichten-Schlagzeilen vom 7. April 1992:)

Toronto Magical Tribune:

\textbf{DAS GESAMTE BRITISCHE ZAUBERGAMOT BERICHTET, DEN "JUNGEN-DER-LEBTE" GESEHEN ZU HABEN, WIE ER EINEN DEMENTOR ERSCHRECKT HAT}

- EXPERTE FÜR MAGISCHE KREATUREN "DAS IST EINE GLATTE LÜGE„-

FRANKREICH UND DEUTSCHLAND BESCHULDIGEN BRITANNIEN,

ALLES ZU ERFINDEN

New Zealand Spellcrafter's Diurnal Notice:

\textbf{WAS HAT DIE BRITISCHE GESETZGEBUNG IN DEN WAHNSINN GETRIEBEN?}

KÖNNTE UNSERE REGIERUNG DIE NÄCHSTE SEIN?

EXPERTEN LISTEN TOP 28 GRÜNDE AUF, DASS ES SCHON PASSIERT IST

American Mage:

WEREWOLF CLAN WIRD IN WYOMING EINGEBÜRGERT

Klitterer:

\textbf{MALFOY FLIEHT AUS HOGWARTS ALS SEINE VEELA KRÄFTE ERWACHEN}

Tagesprophet

\textbf{JURISTISCHE TRICKS VERHINDERN GERECHTE STRAFE VON “VERRÜCKTER MUGGELGEBORENEN"}

-POTTER BEDROHT MINISTERIUM MIT ANGRIFF AUF ASKABAN-

\textbf{Hypothese: Voldemort} (8. April 1992, 19:22 Uhr)

Die vier versammelten sich wieder einmal um den uralten Schreibtisch des Schulleiters von Hogwarts, mit seinen Schubladen über Schubladen über Schubladen, in denen all der vergangene Papierkram der Hogwarts-Schule aufbewahrt wurde; die Legende besagte, dass sich die Schulleiterin Shehla einst in diesen Schreibtisch verirrt hatte und tatsächlich immer noch dort war und nicht wieder herausgelassen werden würde, bis sie ihre Akten geordnet hatte.

Minerva freute sich nicht sonderlich darauf, diese Schubladen zu erben, wenn sie eines Tages diesen Schreibtisch erbte - falls überhaupt etwas davon übrig blieb.

Albus Dumbledore saß hinter seinem Schreibtisch und sah ernst und gefasst aus. Severus Snape stand neben dem leeren Kamin und seiner Asche und schwebte bedrohlich wie der Vampir, von dem die Schüler ihm manchmal vorwarfen, dass er so tat, als wäre er einer.

Mad-Eye Moody sollte sich zu ihnen gesellen, aber er war noch nicht da. Und Harry…

Die kleine, dünne Gestalt eines Jungen, der auf der Armlehne seines Stuhls hockte, als wären die Energien, die ihn durchströmten, zu groß, um ein normales Sitzen zu erlauben. Ein eingefallenes Gesicht, schweißnasses Haar, wissende grüne Augen, und in all dem der gezackte Blitz seiner nie verheilenden Narbe. Er schien jetzt noch grimmiger zu sein; selbst im Vergleich zu einer einzigen Woche zuvor. Für einen Moment erinnerte sich Minerva an ihren Ausflug mit Harry in die Winkelgasse, der eine Ewigkeit her zu sein schien. Irgendwie war schon damals dieser düstere Junge in diesem Harry gewesen. Es war nicht allein ihre Schuld oder Albus' Schuld. Und doch war da etwas fast unerträglich Trauriges an dem Kontrast zwischen dem Jungen, den sie zum ersten Mal getroffen hatte, und dem, was das magische Britannien aus ihm gemacht hatte. Harry hatte nie eine gewöhnliche Kindheit gehabt, wie sie erfahren hatte; Harrys Adoptiveltern hatten ihr erzählt, dass er wenig mit Muggelkindern gesprochen und gespielt hatte. Es war schmerzhaft, daran zu denken, dass Harry vielleicht nur ein paar Monate lang mit den anderen Kindern in Hogwarts gespielt hatte, bevor die Anforderungen des Krieges ihm das alles genommen hatten. Vielleicht gab es ein anderes Gesicht, das Harry den Kindern in seinem Alter zeigte, wenn er nicht gerade das Zaubergamot nieder starrte. Aber sie konnte sich nicht davon abhalten, sich Harry Potters Kindheit als einen Haufen Feuerholz vorzustellen, und sich selbst und Albus, wie sie die hölzernen Äste Stück für Stück in die Flammen schoben.

"Prophezeiungen sind seltsame Dinge", sagte Albus Dumbledore. Die Augen des alten Zauberers waren halb geschlossen, wie vor Müdigkeit. "Vage, unklar, die Bedeutung entweicht wie Wasser, das zwischen losen Fingern gehalten wird. Prophezeiungen sind immer eine Last, denn es gibt dort keine Antworten, nur Fragen."

Harry Potter saß angespannt.

"Schulleiter Dumbledore", sagte der Junge mit leiser Präzision, "meine Freunde werden ins Visier genommen. Hermine Granger wäre fast in Askaban gelandet. Der Krieg hat begonnen, wie Sie es ausdrücken. Professor Trelawneys Prophezeiung ist eine Schlüsselinformation für die Abwägung meiner Hypothesen darüber, was vor sich geht. Ganz zu schweigen davon, wie dumm - und gefährlich - es ist, dass der Dunkle Lord die Prophezeiung kennt und ich nicht."

Albus sah sie grimmig fragend an, und sie schüttelte den Kopf als Antwort; auf welche unvorstellbare Weise auch immer Harry herausgefunden hatte, dass Trelawney die Prophezeiung gemacht hatte und dass der Dunkle Lord davon wusste, so viel hatte er nicht von ihr erfahren.

"Voldemort hat versucht, genau diese Prophezeiung abzuwenden, und ist durch deine Hände in die Niederlage gegangen", sagte der alte Zauberer dann. "Sein Wissen hat ihm nur Unheil gebracht. Überlege dir das gut, Harry Potter."

"Ja, Schulleiter, das verstehe ich. Auch in meiner Heimatkultur gibt es eine literarische Tradition von selbsterfüllenden und fehlinterpretierten Prophezeiungen. Ich werde sie mit Bedacht interpretieren, seien Sie versichert. Aber ich habe schon eine Menge erraten. Ist es sicherer für mich, von Teilvermutungen auszugehen?"

Die Zeit verging.

"Minerva", sagte Albus. "Wenn du so nett wärst."

"Der …", begann sie. Die Worte kamen ihr stockend über die Kehle; sie war keine Schauspielerin. Sie konnte den tiefen, abschreckenden Ton der ursprünglichen Prophezeiung nicht imitieren; und doch schien dieser Ton irgendwie die ganze Bedeutung zu tragen.

"Derjenige, der die Macht hat, den Dunklen Lord zu bezwingen, naht… geboren von denen, die sich ihm dreimal widersetzt haben, geboren, wenn der siebte Monat stirbt…"

"\emph{Und der Dunkle Lord wird ihn als seinesgleichen kennzeichnen}", kam Severus' Stimme und ließ sie auf ihrem Stuhl zusammenzucken.

Der Meister der Zaubertränke stand hoch aufgerichtet vor dem Kamin.

"\emph{Aber er wird eine Macht haben, die der Dunkle Lord nicht kennt … und einer von beiden muss alles bis auf einen Rest des anderen zerstören, denn diese beiden unterschiedlichen Geister können nicht in derselben Welt existieren.}"

Diese letzte Zeile sprach Severus mit so viel Vorahnung, dass es ihr kalt in den Knochen wurde; es war fast so, als würde man Sybill Trelawney zuhören.

Harry hörte mit einem Stirnrunzeln zu. "Könnt ihr das wiederholen?", fragte Harry.

"\emph{Derjenige, der die Macht hat, den Dunklen Lord zu besiegen, nähert sich, geboren von denen, die sich ihm dreimal widersetzt haben, geboren wenn der siebte Monat} -"

"Moment mal, könnt ihr das aufschreiben? Ich muss das genau analysieren -"

Dies wurde getan, wobei sowohl Albus als auch Severus das Pergament wie ein Falke beobachteten, als wollten sie sichergehen, dass keine unsichtbare Hand hineingriff und die kostbare Information wegschnappte.

"Schauen wir mal …" sagte Harry. "Ich bin männlich und am 31. Juli geboren, check. Ich habe tatsächlich den Dunklen Lord besiegt, abgehakt. Zweideutiges Pronomen in Zeile zwei… aber ich war noch nicht geboren, also ist es schwer zu erkennen, wie meine Eltern mich dreimal hätten besiegen können. Diese Narbe ist ein offensichtlicher Kandidat für das Mal…" Harry berührte seine Stirn. "Dann ist da noch die Macht, die der Dunkle Lord nicht kennt, was sich wahrscheinlich auf meinen wissenschaftlichen Hintergrund bezieht -"

"Nein", sagte Severus. Harry sah den Zaubertränkemeister erstaunt an. Severus' Augen waren geschlossen, sein Gesicht vor Konzentration angespannt. "Der Dunkle Lord könnte diese Macht erlangen, indem er die gleichen Bücher studiert wie du, Potter. Aber die Prophezeiung sagte nicht, Macht, die der Dunkle Lord nicht hat. Und auch nicht, Macht, die der Dunkle Lord nicht haben kann. Sie sprach von einer Macht, die der Dunkle Lord nicht kennt. Es wird etwas sein, das ihm fremder ist als Muggel-Artefakte. Etwas, das er vielleicht gar nicht begreifen kann, selbst wenn er es gesehen hat…"

"Die Wissenschaft ist keine Tüte mit technologischen Tricks", sagte Harry. "Es ist nicht nur die Muggelversion eines Zauberstabs. Es ist nicht einmal Wissen wie das Auswendiglernen des Periodensystems. Es ist eine andere Art des Denkens."

"Vielleicht …", murmelte der Zaubertränkemeister, aber seine Stimme war skeptisch.

"Es ist gefährlich", sagte Albus, "zu viel in eine Prophezeiung hineinzulesen, selbst wenn man sie selbst gehört hat. Es handelt sich um etwas, das äußerst frustrierend ist."

"Das sehe ich", sagte Harry. Seine Hand hob sich, rieb über die Narbe auf seiner Stirn. "Aber… okay, wenn das wirklich alles ist, was wir wissen… seht mal, ich werde es einfach unverblümt sagen. Woher wisst ihr, dass der dunkle Lord tatsächlich überlebt hat?"

"Was?!", rief alle.

Albus seufzte nur und lehnte sich in dem riesigen Schulleiterstuhl zurück.

"Nun", sagte Harry, "stellt euch vor, wie diese Prophezeiung damals geklungen hat, als sie gemacht wurde. Du-weißt-schon-wer lernt die Prophezeiung, und es klingt, als sei ich dazu bestimmt, erwachsen zu werden und ihn zu stürzen. Dass wir beide dazu bestimmt sind, einen letzten Kampf zu führen, in dem einer von uns beiden den anderen bis auf einen Rest vernichten muss. Du-weißt-schon-wer greift Godrics Hollow an, wird sofort besiegt und hinterlässt ein Überbleibsel, das seine körperlose Seele sein kann oder auch nicht. Vielleicht sind die Todesser sein Überbleibsel. Oder das Dunkle Mal. Diese Prophezeiung könnte sich bereits erfüllt haben, das will ich damit sagen. Versteht mich nicht falsch - mir ist klar, dass meine Interpretation weit hergeholt klingt. Trelawneys Formulierung scheint nicht natürlich zu sein, um nur die Ereignisse zu beschreiben, die historisch gesehen am 31. Oktober 1981 passiert sind. Ein Baby anzugreifen und den Zauber abprallen zu lassen, ist nicht etwas, das man normalerweise als \emph{"die Macht zu besiegen}" bezeichnen würde. Aber wenn man sich die Prophezeiung so vorstellt, dass es um mehrere mögliche Zukünfte geht, von denen nur eine an Halloween tatsächlich realisiert wurde, dann könnte die Prophezeiung schon vollständig sein."

"Aber -" Minerva platzte heraus. "Aber der Überfall auf Askaban -"

"Wenn der Dunkle Lord überlebt hat, dann ist er natürlich der wahrscheinlichste Verdächtige für den Ausbruch aus Askaban", sagte Harry vernünftig. "Man könnte sogar sagen, dass der Ausbruch aus Askaban ein Bayes'scher Beweis dafür ist, dass der Dunkle Lord überlebt hat, denn ein Ausbruch aus Askaban ist in Welten, in denen er lebt, wahrscheinlicher als in Welten, in denen er tot ist. Aber es ist kein starker Bayes'scher Beweis. Es ist nicht etwas, das unmöglich passieren kann, wenn der Dunkle Lord nicht am Leben ist. Professor Quirrell, der nicht von der Annahme ausging, dass Du-weißt-schon-wer noch am Leben ist, hatte keine Probleme, sich eine eigene Erklärung auszudenken. Für ihn war es naheliegend, dass irgendein mächtiger Zauberer Bellatrix Black haben wollte, weil sie ein Geheimnis des Dunklen Lords kannte, wie zum Beispiel etwas von seinem magischen Wissen, das er nur ihr verraten hatte. Die Wahrscheinlichkeit, dass jemand den Tod seines Körpers überlebt, ist sehr gering, selbst wenn es magisch möglich ist. Meistens passiert es nicht. Wenn es also nur um den Ausbruch aus Askaban geht… müsste ich formell sagen, dass es nicht genug Bayes'scher Beweis ist. Die Unwahrscheinlichkeit des Beweises unter der Annahme, dass die Hypothese falsch ist, ist nicht angemessen mit der vorherigen Unwahrscheinlichkeit der Hypothese."

"Nein", sagte Severus barsch. "Die Prophezeiung ist noch nicht erfüllt. Ich würde es wissen, wenn sie es wäre."

"Bist du dir da sicher?"

"Ja, Potter. Wenn sich die Prophezeiung bereits erfüllt hätte, würde ich es verstehen! Ich habe Trelawneys Worte gehört, ich erinnere mich an Trelawneys Stimme, und wenn ich die Ereignisse kennen würde, die der Prophezeiung entsprechen, würde ich sie erkennen. Was bereits geschehen ist … passt nicht." Der Meister der Zaubertränke sprach mit Bestimmtheit.

"Ich bin mir nicht ganz sicher, was ich mit dieser Aussage anfangen soll", sagte Harry. Seine Hand hob sich und rieb abwesend über seine Stirn. "Vielleicht ist es nur das, was du denkst, dass passiert ist, was nicht passt, und die wahre Geschichte ist anders …"

"Voldemort ist am Leben", sagte Albus. "Es gibt andere Hinweise."

"Zum Beispiel?" Harrys Antwort kam sofort.

Albus hielt inne. "Es gibt schreckliche Rituale, durch die Zauberer vom Tod zurückgekehrt sind", sagte Albus langsam. "So viel kann jeder aus der Geschichte und den Legenden herauslesen. Und doch sind diese Bücher verschwunden, ich konnte sie nicht finden; es war Voldemort, der sie entfernt hat, da bin ich mir sicher -"

"Du kannst also keine Bücher über Unsterblichkeit finden, und das beweist, dass Du-weißt-schon-wer sie hat?"

"In der Tat", sagte Albus. "Es gibt ein bestimmtes Buch - ich will es nicht laut nennen -, das in der verbotenen Abteilung der Hogwarts-Bibliothek fehlt. Eine uralte Schriftrolle, die eigentlich bei Borgin und Burkes hätte sein sollen, mit nur einem leeren Platz in einem Regal, um zu zeigen, wo sie war -" Der alte Zauberer hielt inne. "Aber ich nehme an", sagte der alte Zauberer wie zu sich selbst, "Du wirst sagen, dass, selbst wenn Voldemort versucht hat, sich unsterblich zu machen, das nicht beweist, dass es ihm gelungen ist…"

Harry seufzte. "Beweise, Schulleiter? Es gibt immer nur Wahrscheinlichkeiten. Wenn bestimmte Bücher über Unsterblichkeitsrituale bekannt sind und fehlen, erhöht das die Wahrscheinlichkeit, dass jemand einen Versuch unternommen hat. Was wiederum die Wahrscheinlichkeit erhöht, dass der Dunkle Lord seinen Tod überlebt hat. Das räume ich ein, und ich danke Ihnen für Ihren Beitrag. Die Frage ist, ob die vorherige Wahrscheinlichkeit hoch genug ist."

"Sicherlich", sagte Albus leise, "wenn du auch nur eine Chance einräumst, dass Voldemort überlebt hat, ist das eine Absicherung wert?"

Harry legte den Kopf schief. "Wie Sie meinen, Schulleiter. Allerdings ist es auch ein Fehler, sich weiter damit zu beschäftigen, sobald die Wahrscheinlichkeit niedrig genug ist… In Anbetracht der Tatsache, dass Bücher über Unsterblichkeit fehlen und dass diese Prophezeiung etwas natürlicher klingen würde, wenn sie sich auf einen zukünftigen Kampf zwischen dem Dunklen Lord und mir bezöge, stimme ich zu, dass das Leben des Dunklen Lords eine Wahrscheinlichkeit ist, nicht nur eine Möglichkeit. Aber es müssen auch andere Wahrscheinlichkeiten in Betracht gezogen werden - und in den wahrscheinlichen Welten, in denen Du-weißt-schon-wer nicht lebt, hat jemand anderes Hermine reingelegt."

"Dummheit", sagte Severus leise. "Absoluter Blödsinn. Das Dunkle Mal ist nicht verblasst, und sein Meister auch nicht."

"Seht ihr, das ist es, was ich mit formal unzureichenden Bayes'schen Beweisen meine. Sicher, das klingt alles düster und vorhersehbar und so, aber ist es so unwahrscheinlich, dass ein magisches Mal nach dem Tod des Erschaffers bestehen bleibt? Nehmen wir an, das Mal besteht mit Sicherheit weiter, solange der Dunkle Herrscher lebt, aber a priori hätten wir nur eine zwanzigprozentige Chance, dass das Dunkle Mal nach dem Tod des Dunklen Herrschers weiter existiert. Dann ist die Beobachtung "Das Dunkle Mal ist nicht verblasst" in Welten, in denen der Dunkle Lord lebt, fünfmal so wahrscheinlich wie in Welten, in denen der Dunkle Lord tot ist. Steht das wirklich im Verhältnis zu der vorherigen Unwahrscheinlichkeit der Unsterblichkeit? Nehmen wir an, die Wahrscheinlichkeit, dass der Dunkle Lord überlebt, stünde 100:1. Wenn eine Hypothese hundertmal wahrscheinlicher ist als falsch, und man dann Beweise sieht, die fünfmal wahrscheinlicher sind, wenn die Hypothese wahr ist als falsch, sollte man dazu übergehen, zu glauben, dass die Hypothese zwanzigmal wahrscheinlicher ist als falsch. Eine Wahrscheinlichkeit von hundert zu eins, mal ein Wahrscheinlichkeitsverhältnis von eins zu fünf, ergibt eine Wahrscheinlichkeit von zwanzig zu eins, dass der Dunkle Lord tot ist -"

"Woher nimmst du all diese Zahlen, Potter?"

"Das ist die zugegebene Schwäche der Methode", sagte Harry bereitwillig. "Aber worauf ich qualitativ hinaus will, ist, warum die Beobachtung 'Das Dunkle Mal ist nicht verblasst' keine ausreichende Unterstützung für die Hypothese 'Der Dunkle Lord ist unsterblich' ist. Der Beweis ist nicht so außergewöhnlich wie die Behauptung." Harry hielt inne. "Ganz zu schweigen davon, dass, selbst wenn der Dunkle Lord am Leben ist, er nicht derjenige sein muss, der Hermine reingelegt hat. Wie ein schlauer Mann einmal sagte, könnte es mehr als einen Verschwörer und mehr als einen Plan geben."

"So wie der Verteidigungsprofessor", sagte Severus mit einem dünnen Lächeln. "Ich muss wohl zustimmen, dass er ein Verdächtiger ist. Immerhin war es der Verteidigungsprofessor letztes Jahr; und das Jahr davor und das Jahr davor."

Harrys Augen fielen zurück auf das Pergament in seinem Schoß.

"Lasst uns weitermachen. Sind wir sicher, dass diese Prophezeiung korrekt ist? Niemand hat an Professor McGonagalls Gedächtnis herumgepfuscht, vielleicht eine Zeile editiert oder subtrahiert?"

Albus hielt inne, dann sprach er langsam. "Es liegt ein mächtiger Zauber über Britannien, der jede Prophezeiung aufzeichnet, die innerhalb unserer Grenzen gesprochen wird. Weit unter der Ältesten Halle des Zaubergamot, in der Abteilung für Mysterien, werden sie aufgezeichnet."

"Die Halle der Prophezeiungen", flüsterte Minerva. Sie hatte von diesem Ort gelesen, von dem es hieß, er sei ein großer Raum mit Regalen, gefüllt mit leuchtenden Kugeln, von denen eine nach der anderen im Laufe der Jahre auftauchte. Merlin selbst hatte ihn geschaffen, hieß es; die letzte Ohrfeige des größten Zauberers für das Schicksal. Nicht alle Prophezeiungen führten zum Guten, und Merlin hatte sich gewünscht, dass wenigstens die, von denen in der Prophezeiung die Rede war, wussten, was über sie gesagt worden war. Das war der Respekt, den Merlin dem freien Willen entgegenbrachte, damit das Schicksal sie nicht von außen, unwissend, kontrollieren konnte. Diejenigen, die in einer Prophezeiung erwähnt wurden, ließen eine leuchtende Kugel zu ihrer Hand schweben und hörten dann die wahre Stimme des Propheten sprechen. Andere, die versuchten, eine Kugel zu berühren, so hieß es, würden in den Wahnsinn getrieben werden - oder möglicherweise einfach nur ihren Kopf explodieren lassen, die Legenden waren in diesem Punkt unklar. Was auch immer Merlins ursprüngliche Absicht war, die Unaussprechlichen hatten seit Jahrhunderten niemanden mehr hineingelassen, so weit sie gehört hatte. In den Werken der alten Zauberer war zu lesen, dass spätere Unaussprechliche entdeckt hatten, dass ein Hinweis auf die Subjekte der Prophezeiungen die Seher daran hindern könnte, den zeitlichen Druck freizusetzen, der sie auslöste; und so hatten die Erben Merlins seine Halle versiegelt. Es kam Minerva in den Sinn, sich zu fragen (jetzt, wo sie ein paar Monate in der Nähe von Mr. Potter verbracht hatte), wie jemand das wissen konnte; aber sie wusste auch, dass sie Albus besser nicht fragen sollte, für den Fall, dass er es ihr sagen wollte. Minerva war der festen Überzeugung, dass man sich nur dann Gedanken über die Zeit machen sollte, wenn man eine Uhr war.

"Die Halle der Prophezeiung", bestätigte Albus leise. "Diejenigen, von denen in einer Prophezeiung die Rede ist, dürfen diese Prophezeiung dort anhören. Verstehst du, was das bedeutet, Harry?"

Harry runzelte die Stirn. "Nun, ich könnte sie anhören, oder der Dunkle Lord … oh, meine Eltern. Diejenigen, die sich ihm dreimal widersetzt hatten. Sie wurden auch in der Prophezeiung erwähnt, also könnten sie die Aufnahme hören?"

"Wenn James und Lily etwas anderes gehört haben als das, was Minerva berichtet hat", sagte Albus gleichmäßig, "dann haben sie es mir nicht gesagt."

"Du hast James und Lily dorthin gebracht?!" sagte Minerva.

"Fawkes kann an viele Orte gehen", sagte Albus. "Erwähne das niemals gegenüber anderen."

Harry starrte Albus direkt an. "Kann ich zu diesem Ort in der Mysteriumsabteilung gehen und die aufgezeichnete Prophezeiung hören? Der Originalton könnte hilfreich sein, nach dem, was ich gehört habe."

Licht glitzerte in der Reflexion von Albus' Halbmondbrille, als der alte Zauberer langsam den Kopf schüttelte. "Ich glaube, das wäre unklug", sagte Albus. "Aus Gründen, die über das Offensichtliche hinausgehen. Es ist gefährlich, dieser Ort, den Merlin geschaffen hat; für manche Menschen gefährlicher als für andere."

"Ich verstehe", sagte Harry tonlos und sah wieder auf das Pergament hinunter. "Ich nehme an, dass die Prophezeiung vorerst zutreffend ist. Der nächste Teil besagt, dass der Dunkle Lord mich als Seinesgleichen markiert hat. Irgendwelche Ideen, was das genau bedeutet?"

"Sicherlich nicht", sagte Albus, "dass du in irgendeiner Weise seine Wege imitieren musst."

"Ich bin nicht dumm, Schulleiter. Muggel haben ein oder zwei Dinge über temporale Paradoxa herausgefunden, auch wenn das für sie alles theoretisch ist. Ich werde meine Ethik nicht über Bord werfen, nur weil ein Signal aus der Zukunft behauptet, dass es passieren wird, denn dann wird das der einzige Grund, warum es überhaupt passiert. Trotzdem, was bedeutet das?"

"Ich weiß es nicht", sagte Severus.

"Ich auch nicht", sagte sie.

Harry nahm seinen Zauberstab heraus, drehte ihn in den Händen und betrachtete nachdenklich das Holz. "Elf Zoll, Stechpalme, mit einem Kern aus Phönixfedern", sagte Harry. "Und der Phönix, dessen Schwanzfeder in diesem Zauberstab steckt, hat nur einen anderen gegeben, den Mr… wie hieß er noch, Olive-irgendwas… zum Kern des Zauberstabs des Dunklen Lords gemacht hat. Und ich bin ein Parselmund. Das schien mir schon damals ein großer Zufall zu sein. Und jetzt finde ich heraus, dass es eine Prophezeiung gibt, die besagt, dass ich dem Dunklen Lord ebenbürtig sein werde."

Severus' Augen waren nachdenklich; der Blick des Schulleiters war unleserlich.

"Könnte es sein", sagte Minerva zögernd, "dass Du-weißt-schon-wer - Voldemort - etwas von seinen eigenen Kräften auf Mr. Potter übertragen hat, in der Nacht, als er ihm diese Narbe verpasst hat? Das war sicher nicht seine Absicht. Trotzdem… Ich wüsste nicht, wie Mr. Potter ihm ebenbürtig sein könnte, wenn er weniger Magie hätte als der Dunkle Lord selbst…"

"Meh", sagte Harry und blickte immer noch nachdenklich auf seinen Zauberstab. "Ich würde den Dunklen Lord auch ohne jegliche Magie bekämpfen, wenn es sein müsste. Der Homo sapiens ist nicht zur dominanten Spezies auf diesem Planeten geworden, weil er die schärfsten Krallen oder die härteste Rüstung hat - obwohl ich vermute, dass dieser Punkt bei Zauberern etwas verloren gegangen ist. Trotzdem ist es unter meiner Würde als Mensch, vor etwas Angst zu haben, das nicht schlauer ist als ich; und nach dem, was ich gehört habe, war der Dunkle Lord in dieser Dimension nicht sehr furchterregend."

Der Meister der Zaubertränke sprach, seine Stimme nahm etwas von seinem gewohnten verächtlichen Tonfall an. "Du hältst dich für intelligenter als der Dunkle Lord, Potter?"

"Ja, in der Tat", sagte Harry, zog den linken Ärmel seines Umhangs zurück und rollte den Hemdsärmel darunter auf, so dass der nackte Ellbogen zum Vorschein kam. "Oh, das erinnert mich an etwas! Stellen wir sicher, dass niemand hier eine deutlich sichtbare Tätowierung an einer leicht zu überprüfenden Standardstelle hat, die ihn als geheimen feindlichen Spion ausweisen würde."

Albus machte eine beschwichtigende Geste, die den Zaubertränkemeister aufhielt, bevor er etwas Beleidigendes sagen konnte.

"Sag mir, Harry", sagte Albus, "wie hättest du das Dunkle Mal gemacht und wo hättest du es versteckt?"

"An unüblichen Stellen", sagte Harry prompt, "nicht leicht zu finden ohne Peinlichkeit und Aufregung, obwohl natürlich jeder sicherheitsbewusste Mensch sowieso nachsehen würde. Mach es kleiner, wenn möglich. Überlagere es mit einer anderen nichtmagische Tätowierung, um die genaue Form zu verdecken - besser noch, bedecke es mit einer Schicht falscher Haut -"

"In der Tat gerissen", sagte Albus. "Aber sag mir, nehmen wir an, du könntest jeden beliebigen Zustand in das Mal einarbeiten, es verblassen lassen oder erscheinen, wie es dir beliebt. Was würdest du dann tun?"

"Es jederzeit komplett unsichtbar machen", sagte Harry in einem Ton, der das Offensichtliche ausspricht. "Man will doch nicht, dass es einen erkennbaren Unterschied zwischen einem Spion und einem Nicht-Spion gibt."

"Nehmen wir an, du bist noch gerissener", sagte Albus. "Du bist ein Meister der List, ein Meister der Täuschung, und du setzt deine Fähigkeiten in vollem Umfang ein."

"Nun -" Der Junge hielt inne und runzelte die Stirn. "Es wirkt unnötig kompliziert, eher wie eine Taktik, die ein Schurke in einem Rollenspiel anwenden würde, als etwas, das man in einem echten Krieg versuchen würde. Aber ich nehme an, man könnte Leuten, die nicht wirklich Todesser sind, falsche Dunkle Male aufdrücken und die Dunklen Male der echten Todesser unsichtbar machen. Aber dann wäre da noch die Frage, warum die Leute überhaupt glauben, dass das Dunkle Mal einen Todesser identifiziert… Darüber müsste ich mindestens fünf Minuten lang nachdenken, wenn ich das Problem ernst nehmen würde."

"Tu das bitte", sagte Albus, immer noch in diesem milden Ton, "weil ich in der Tat in den frühen Tagen des Krieges solche Tests durchgeführt habe, wie du sie vorgeschlagen hast. Der Orden überlebte meine Torheit nur, weil Alastor nicht auf die bloßen Waffen vertraute, die wir sahen. Ich hatte im Nachhinein gedacht, dass die Träger des Zeichens es verstecken oder zeigen könnten, wie sie wollten. Doch als wir Igor Karkaroff vor das Zaubergamot zerrten, war das Mal deutlich auf seinem Arm zu sehen, auch wenn Karkaroff seine Unschuld beteuern wollte. Welche wahre Regel das Dunkle Mal beherrschen mag, weiß ich nicht. Selbst Severus ist noch immer durch sein Mal verpflichtet, seine Geheimnisse nicht an Unwissende weiterzugeben."

"Oh, das macht es offensichtlich", sagte Harry prompt. "Warte, warte mal - du warst ein Todesser?" Harry übertrug seinen Blick auf Severus.

Severus erwiderte ein dünnes Lächeln. "Das bin ich immer noch, so weit sie wissen."

"Harry", sagte Albus und sah den Jungen nur an. "Was meinst du damit, dass das offensichtlich ist?"

"Informationstheorie 101", sagte der Junge in einem belehrenden Ton. "Die Beobachtung der Variablen X vermittelt dann und nur dann Informationen über die Variable Y, wenn die möglichen Werte von X bei unterschiedlichen Zuständen von Y unterschiedliche Wahrscheinlichkeiten haben. Sobald du von irgendetwas hörst, das sich zwischen einem Spion und einem Nicht-Spion unterscheidet, solltest du sofort daran denken, es auszunutzen, um Spione von Nicht-Spionen zu unterscheiden. In ähnlicher Weise braucht man zur Unterscheidung von Wahrheit und Lüge einen Prozess, der sich bei Vorhandensein von Wahrheit und Unwahrheit unterschiedlich verhält - deshalb funktioniert "\emph{Glaube}" nicht als Unterscheidungsmerkmal, während "\emph{experimentelle Vorhersagen machen und sie testen}" funktioniert. Sie sagen, dass jemand mit dem Dunklen Mal niemandem seine Geheimnisse verraten kann, der sie nicht bereits kennt. Um also herauszufinden, wie das Dunkle Mal funktioniert, schreiben Sie jede Möglichkeit auf, die Sie sich vorstellen können, wie das Dunkle Mal funktionieren könnte, und beobachten Sie dann, wie Professor Snape versucht, jedes dieser Dinge einem Verbündeten zu erzählen - vielleicht einem, der nicht weiß, worum es bei dem Experiment geht - ich werde die binäre Suche später erklären, so dass Sie \emph{Zwanzig Fragen} spielen können, um die Dinge einzugrenzen - und was immer er nicht laut sagen kann, ist wahr. Sein Schweigen wäre etwas, das sich in Gegenwart von wahren Aussagen über das Zeichen anders verhält als bei falschen Aussagen, verstehen Sie."

Minervas Mund stand offen, das merkte sie; und sie schloss ihn abrupt. Selbst Albus sah überrascht aus.

"Und danach kann man, wie ich schon sagte, jeden Verhaltensunterschied zwischen Spionen und Nicht-Spionen nutzen, um Spione zu identifizieren. Sobald man mindestens ein magisch zensiertes Geheimnis des Dunklen Zeichens identifiziert hat, kann man jemanden auf das Dunkle Zeichen testen, indem man sieht, ob er dieses Geheimnis jemandem verraten kann, der es noch nicht kennt -"

"Danke, Mr. Potter." Alle sahen Severus an. Der Meister der Zaubertränke richtete sich auf, die Zähne in einer Grimasse des wütenden Triumphs gefletscht. "Schulleiter, ich kann jetzt frei über das Mal sprechen. Wenn wir wissen, dass wir für als Todesser überführt sind, offenbart sich unser Mal vor anderen, die unsere nackten Arme noch nicht gesehen haben, ob wir es wollen oder nicht. Aber wenn sie unsere nackten Arme bereits gesehen haben, offenbart es sich nicht; auch nicht, wenn wir nur auf Verdacht hin geprüft werden. So scheint das Dunkle Mal die Todesser zu identifizieren - aber nur die, die bereits gefunden wurden, wie man sieht."

"Ah …" sagte Albus. "Ich danke dir, Severus." Er schloss kurz die Augen. "Das würde in der Tat erklären, warum Black nicht einmal Peters Aufmerksamkeit erregt hat … ah, nun ja. Und der von Harry vorgeschlagene Test?"

Der Meister der Zaubertränke schüttelte den Kopf. "Der Dunkle Lord war kein Narr, trotz Potters Wahnvorstellungen. In dem Moment, in dem ein solcher Test vermutet wird, hört das Mal auf, unsere Zungen zu binden. Dennoch konnte ich die Möglichkeit nicht andeuten, sondern nur darauf warten, dass ein anderer sie herleitet." Wieder ein dünnes Lächeln. "Ich würde Ihnen eine Menge Hauspunkte geben, Mr. Potter, wenn es nicht meine Tarnung gefährden würde. Aber wie du sehen kannst, war der Dunkle Lord ziemlich gerissen."

Sein Blick wurde immer distanzierter.

"Oh", hauchte Severus, "er war wirklich sehr gerissen …"

Harry Potter saß einen langen Moment lang still. Dann -

"Nein", sagte Harry. Der Junge schüttelte den Kopf. "Nein, das kann eigentlich nicht wahr sein. Erstens reden wir über die Art von Logikrätseln, die im ersten Kapitel eines Raymond-Smully-Buches auftauchen würden, nicht annähernd auf dem Niveau dessen sind, womit Muggelwissenschaftler ihren Lebensunterhalt verdienen. Und zweitens, soweit ich weiß, hat der Dunkle Lord fünf Monate gebraucht, um das Rätsel zu erfinden, das ich gerade in fünf Sekunden gelöst habe -"

"Ist es für dich so unvorstellbar, Potter, dass jemand so intelligent sein kann wie du?" In der Stimme des Zaubertränkemeisters lag mehr Neugierde als Hohn.

"Das nennt man eine Basisrate, Professor Snape. Die Beweise sind gleichermaßen kompatibel damit, dass der Dunkle Lord dieses Rätsel im Laufe von fünf Monaten oder im Laufe von fünf Sekunden erfunden hat, aber in jeder gegebenen Bevölkerung wird es viel mehr Leute geben, die es in fünf Monaten schaffen als in fünf Sekunden…" Harry presste eine Hand an seine Stirn. "Verflixt, wie soll ich das erklären? Ich nehme an, aus deiner Sicht hat sich der Dunkle Lord ein cleveres Rätsel ausgedacht und ich habe es clever gelöst, und das lässt uns gleichwertig erscheinen."

"Ich erinnere mich an deinen ersten Tag in der Zaubertrankklasse", sagte der Zaubertrankmeister trocken. "Ich glaube, du hast noch einen weiten Weg vor dir."

"Frieden, Severus", sagte Albus. "Harry hat schon mehr erreicht, als du weißt. Doch sag mir, Harry - warum glaubst du, dass der Dunkle Lord weniger schlau ist als du? Sicherlich ist er in vielerlei Hinsicht eine beschädigte Seele. Aber seine List gegen deine List - du bist noch nicht bereit, dich ihm zu stellen, würde ich urteilen; und ich kenne die volle Bilanz deiner Taten."

Das Frustrierende an diesem Gespräch war, dass Harry seine eigentlichen Gründe für seine Ablehnung nicht nennen konnte, was gegen mehrere Grundprinzipien des kooperativen Diskurses verstieß. Er konnte nicht erklären, wie Bellatrix wirklich aus Askaban geholt worden war - nicht durch Du-weißt-schon-wen in irgendeiner Gestalt, sondern durch den kombinierten Verstand von Harry und Professor Quirrell. Harry wollte nicht vor Professor McGonagall sagen, dass die Existenz von Hirnschäden implizierte, dass es so etwas wie Seelen nicht gab. Das machte ein erfolgreiches Unsterblichkeitsritual… nun, nicht unmöglich, Harry hatte sicherlich vor, eines Tages einen Weg zur magischen Unsterblichkeit zu schmieden, aber es würde viel schwieriger sein und viel mehr Einfallsreichtum erfordern, als einfach eine bereits existierende Seele an das Phylakterium eines Lichs zu binden. Was kein intelligenter Zauberer überhaupt tun würde, wenn er wüsste, dass seine Seele unsterblich ist. Und der wahre und ehrliche Grund, warum Harry wusste, dass der Dunkle Lord nicht so schlau gewesen sein konnte… nun… es gab keine taktvolle Art, es zu sagen, aber… Harry war bei einer Versammlung des Zaubergamot gewesen. Er hatte die lächerlichen "Sicherheitsvorkehrungen" gesehen, wenn man sie so nennen konnte, die die tiefsten Ebenen des Zaubereiministeriums bewachten. Sie hatten nicht einmal den "Untergang des Diebes", mit dem die Kobolde die Leute, die Gringotts betraten, vor Vielsaft und Imperius-Flüchen prüften. Die offensichtliche Übernahme-Route wäre, den Zaubereiminister und ein paar Abteilungsleiter mit Imperius zu belegen und jedem, der zu mächtig für Imperius ist, eine Handgranate zu eulen. Oder sie mit K.O.-Gas außer Gefecht setzen, wenn man sie lebendig und in einem Zustand des lebenden Todes braucht, um Haare für Vielsafttränke zu entnehmen. Legilimenz, falsche Erinnerungen, der Confundus-Zauber - es war lächerlich, die magische Welt war übersättigt mit Möglichkeiten zu betrügen. Harry konnte bei seiner eigenen Übernahme von Britannien keines dieser Dinge selbst tun, da er durch die Ethik eingeschränkt war…. naja, Harry würde vielleicht ein paar von den kleineren machen, denn Vielsaft oder ein vorübergehender Confundus oder Legilimenz nur zum Lesen klangen alle besser als ein zusätzlicher Tag Askaban… aber… Wäre Harry nicht durch die Ethik eingeschränkt gewesen, hätte er an diesem Tag möglicherweise die böseren Teile des Zaubergamot auslöschen können; ganz allein, nur mit den magischen Kräften eines Erstklässlers, weil er schlau genug war, die Dementoren zu erkennen. Obwohl Harry danach vielleicht nicht mehr in einer so großartigen politischen Position war, hätten die überlebenden Mitglieder des Zaubergamot es Recht und billig gefunden, seine Taten zu PR-Zwecken zu desavouieren und ihn zu verurteilen, selbst wenn die Klügeren erkannt hätten, dass es dem größeren Wohl diente… aber trotzdem. Wenn man völlig frei von Ethik war, mit den alten Geheimnissen von Salazar Slytherin bewaffnet, Dutzende von mächtigen Anhängern hatte, darunter Lucius Malfoy, und mehr als zehn Jahre brauchte, um die Regierung des magischen Britannien zu stürzen, bedeutete das, dass man dumm war.

"Wie soll ich das sagen…" sagte Harry. "Sehen Sie, Schulleiter, Sie haben eine Ethik, es gibt eine Menge Kampftaktiken, die Sie nicht anwenden, weil Sie nicht böse sind. Und Sie haben gegen den Dunklen Lord gekämpft, einen ungeheuer mächtigen Zauberer, der nicht so zurückhaltend war, und Sie haben ihn trotzdem aufgehalten. Wäre Du-weißt-schon-wer obendrein noch superschlau gewesen, wärt ihr tot. Ihr alle. Ihr wärt auf der Stelle gestorben -"

"Harry", sagte Professor McGonagall. Ihre Stimme schwankte. "Harry, wir wären fast alle gestorben. Mehr als die Hälfte des Ordens des Phönix ist gestorben. Wenn Albus - Albus Dumbledore, der größte Zauberer seit zwei Jahrhunderten, Harry - nicht gewesen wäre, wären wir sicher umgekommen."

Harry fuhr sich mit einer Hand über die Stirn. "Es tut mir leid", sagte Harry. "Ich will nicht verharmlosen, was ihr durchgemacht habt. Ich weiß, dass Du-weißt-schon-wer ein absolut böser, unglaublich mächtiger dunkler Zauberer mit Dutzenden von mächtigen Anhängern war, und das ist… schlimm, ja, definitiv schlimm. Es ist nur…"

\emph{All das ist nicht im Entferntesten auf der gleichen Bedrohungsskala wie ein Feind, der schlau ist, in dem Fall, dass er Botulinumtoxin verwandelt und ein Millionstel Gramm in deine Teetasse schmuggelt. Gab es einen sicheren Weg, dieses Konzept zu vermitteln, ohne Einzelheiten zu nennen? Harry fiel keins ein.}

"Bitte, Harry", sagte Professor McGonagall. "Bitte, Harry, ich flehe dich an - nimm den Dunklen Lord ernst! Er ist gefährlicher als -" Die Oberhexe schien Schwierigkeiten zu haben, Worte zu finden. "Er ist viel gefährlicher als Verwandlung."

Harrys Augenbrauen gingen nach oben, bevor er sich zurückhalten konnte. Ein dunkles Kichern kam aus der Richtung von Severus Snape.

\emph{Ähm}, sagte die Stimme des Ravenclaw in ihm. \emph{Ähm, ehrlich gesagt hat Professor McGonagall recht, wir nehmen das nicht so ernst, wie wir ein wissenschaftliches Problem angehen würden.}

\emph{Das Schwierige ist, überhaupt auf neue Informationen zu reagieren, anstatt sie einfach aus dem Fenster zu werfen. Im Moment sieht es so aus, als hätten wir unseren Glauben überhaupt nicht geändert, nachdem wir auf ein unerwartetes, wichtiges Argument gestoßen sind. Unsere Ablehnung von Lord Voldemort als ernsthafte Bedrohung basierte ursprünglich darauf, dass das Dunkle Mal eklatant dumm ist. Es würde eine konzentrierte Anstrengung erfordern, den ganzen Gartenpfad der Argumentation, den wir aufgrund dieser falschen Annahme eingeschlagen haben, zu deaktualisieren und zu verdächtigen, und diese Anstrengung werden wir im Moment nicht unternehmen.}

"In Ordnung", sagte Harry, gerade als Professor McGonagall wieder zu sprechen schien. "In Ordnung, um das ernst zu nehmen, muss ich fünf Minuten innehalten und nachdenken."

"\emph{Bitte} tu das", sagte Albus Dumbledore.

Harry schloss die Augen. Seine Ravenclaw-Seite teilte sich in drei.

\emph{Wahrscheinlichkeitsschätzung,} sagte Ravenclaw Eins, der als Moderator fungierte. \emph{Nehmen wir an, dass der Dunkle Lord lebt und genauso schlau ist wie wir und somit eine echte Bedrohung darstellt.}

\emph{Warum sind nicht alle seine Feinde tot?} sagte Ravenclaw Zwei, der die Anklage vertrat.

\emph{Wohlgemerkt}, sagte Ravenclaw Eins, \emph{dieses Argument hatten wir schon einmal, also können wir es nicht jedes Mal, wenn wir es besprechen, erneut verwenden, um die Überzeugung zu ändern.}

\emph{Aber was ist der eigentliche Fehler in der Logik?} sagte Ravenclaw Zwei. \emph{In Welten mit einem schlauen Lord Voldemort sind alle Mitglieder des Ordens des Phönix in den ersten fünf Minuten des Krieges gestorben. Die Welt sieht nicht so aus, also leben wir nicht in dieser Welt. QED.}

\emph{Ist das wirklich sicher?} fragte Ravenclaw 3, der zum Verteidiger ernannt worden war. \emph{Vielleicht gibt es einen Grund, warum Lord Voldemort damals nicht mit aller Macht gekämpft hat -}

\emph{Welchen zum Beispiel?} fragte Ravenclaw Zwei. \emph{Außerdem verlange ich, dass die Wahrscheinlichkeit deiner Hypothese entsprechend ihrer zusätzlichen Komplexität bestraft wird -}

\emph{Lass Drei reden,} sagte Ravenclaw Eins.

\emph{Okay… hör zu,} sagte Ravenclaw Drei. \emph{Erstens wissen wir nicht, dass jemand das Ministerium nur mit Gedankenkontrolle übernehmen kann. Vielleicht ist das magische Britannien wirklich eine Oligarchie und man braucht genug militärische Macht, um die Familienoberhäupter zur Unterwerfung einzuschüchtern -}

\emph{Imperius sie auch,} warf Ravenclaw Zwei ein. -

\emph{und die Oligarchen haben Sicherheitsvorkehrungen in den Eingängen ihrer Häuser} -

\emph{Komplexitätsstrafe!} rief Ravenclaw Zwei. \emph{Mehr Epizyklen!} -

\emph{Oh, sei doch vernünftig,} sagte Ravenclaw Drei. \emph{Wir haben noch niemanden gesehen, der das Ministerium mit ein paar gut platzierten Imperius-Flüchen übernommen hat.} \emph{Wir wissen nicht, ob das wirklich so einfach geht.}

\emph{Aber,} sagte Ravenclaw 2, \emph{selbst wenn man das in Betracht zieht… Es hätte einen anderen Weg geben müssen. Zehn Jahre des Scheiterns, wirklich? Nur mit konventionellen Terroristen-Taktiken? Das ist einfach… nicht mal ein Versuch, das ist lächerlich.}

\emph{Vielleicht hatte Lord Voldemort kreativere Ideen,} antwortete Ravenclaw Drei, \emph{aber er wollte den Regierungen anderer Länder nicht die Hand reichen, wollte nicht, dass}

\emph{sie wussten, wie verwundbar sie waren und Sicherungen in ihren Ministerien installieren}. \emph{Nicht bevor er Großbritannien als Basis hatte und genug Diener, um alle anderen großen Regierungen gleichzeitig zu unterwandern.}

\emph{Du gehst davon aus, dass er die ganze Welt erobern will,} merkte Ravenclaw 2 an.

\emph{Trelawney prophezeite, dass er uns ebenbürtig sein würde,} intonierte Ravenclaw Drei feierlich. \emph{Deshalb wolle er die Welt erobern. Und wenn er uns ebenbürtig ist und wir gegen ihn kämpfen müssen} - für einen Augenblick versuchte Harrys Geist, sich das Gespenst zweier kreativer Zauberer vorzustellen, die einen totalen Krieg gegeneinander führen…

… Harry hatte sich in seinen Erstklassbüchern alle Zaubersprüche und -tränke notiert, die man kreativ einsetzen konnte, um Menschen zu töten. Er hatte sich nicht anders zu helfen gewusst. Buchstäblich. Er hatte jedes Mal versucht, sein Gehirn davon abzuhalten, es zu tun, aber es war, als ob man einen Fisch ansieht und versucht, sein Gehirn davon abzuhalten, zu bemerken, dass es ein Fisch ist. Was jemand mit der Magie eines Siebtklässlers, eines Aurors oder einer uralten, verlorenen Magie, wie sie Lord Voldemort besessen hatte, kreativ anstellen konnte… war nicht auszudenken. Ein mit magischen Superkräften ausgestatteter, kreativ-genialer Psychopath war keine 'Bedrohung', er war eine Apokalypse.

… dann schüttelte Harry den Kopf und verwarf den düsteren Gedankengang, den er verfolgt hatte. Die Frage war, ob es überhaupt eine signifikante Wahrscheinlichkeit gab, etwas so Schrecklichem wie einem Dunklen Rationalisten gegenüberzustehen. Die Wahrscheinlichkeit, dass jemand, der ein Unsterblichkeitsritual versuchte, und es tatsächlich funktionierte, war… Nennen wir es eins zu tausend, bei großzügiger Überschätzung; es war nicht so, dass etwa einer von tausend Zauberern seinen Tod überlebte. Obwohl, zugegebenermaßen hatte Harry keine Daten darüber, wie viele zuerst Unsterblichkeitsrituale versucht hatten.

\emph{Was, wenn der Dunkle Lord genauso schlau ist wie wir?} sagte Ravenclaw 3. \emph{So wie Trelawney prophezeit hat, dass er uns ebenbürtig ist. Dann würde er sein Unsterblichkeitsritual durchziehen. P.S., vergiss nicht den Spruch "alles bis auf einen Rest zerstören".}

Diese Intelligenz zu verlangen, war ein zusätzliches lästiges Detail; die Wahrscheinlichkeit, dass ein zufälliges Mitglied der Bevölkerung so intelligent war, war gering… Aber Lord Voldemort war kein zufällig ausgewählter Zauberer, er war ein bestimmter Zauberer in der Bevölkerung, der allen aufgefallen war. Das Rätsel des Mal implizierte ein gewisses Mindestmaß an Intelligenz, auch wenn (hypothetisch) der Dunkle Lord länger gebraucht hatte, um es zu durchdenken. Andererseits waren in der Muggelwelt alle extrem intelligenten Menschen, die Harry aus der Geschichte kannte, keine bösen Diktatoren oder Terroristen geworden. Am nächsten dran waren in der Muggelwelt Hedgefonds-Manager, und keiner von ihnen hatte versucht, auch nur ein Land der Dritten Welt zu übernehmen, was sowohl dem möglichen Bösen als auch dem möglichen Guten eine Obergrenze setzte. Es gab Hypothesen, in denen der Dunkle Lord schlau war und der Orden des Phönix nicht einfach auf der Stelle starb, aber diese Hypothesen waren komplizierter und sollten Komplexitätsabzüge bekommen. Nachdem die Komplexitätsabzüge der weiteren Ausreden berücksichtigt wurden, gäbe es ein großes Wahrscheinlichkeitsverhältnis von den Hypothesen "Der Dunkle Lord ist schlau" versus "Der Dunkle Lord war dumm" zu der Beobachtung "Der Dunkle Lord hat den Krieg nicht sofort gewonnen". Das war wahrscheinlich ein Wahrscheinlichkeitsverhältnis von 10:1 zugunsten des Dunklen Lords als dumm… aber vielleicht nicht 100:1. Man konnte nicht wirklich sagen, dass "Der Dunkle Lord gewinnt sofort" eine Wahrscheinlichkeit von mehr als 99 Prozent hatte, vorausgesetzt, der Dunkle Lord hat klug angefangen; die Summe über alle möglichen Ausreden wäre mehr als .01. Und dann war da noch die Prophezeiung… die vielleicht, vielleicht auch nicht, ursprünglich eine Zeile darüber enthielt, dass Lord Voldemort sofort sterben würde, wenn er den Potters gegenüberstünde. Die Albus Dumbledore dann in Professor McGonagalls Gedächtnis editiert hatte, um Lord Voldemort in sein Verderben zu locken. Wenn es diese Zeile nicht gab, klang die Prophezeiung eher danach, dass Du-weißt-schon-wer und der Junge-der-lebte zu einer späteren Konfrontation bestimmt waren. Aber in diesem Fall war es unwahrscheinlicher, dass Dumbledore sich eine plausibel klingende Ausrede einfallen lassen würde, um Harry nicht in die Halle der Prophezeiung zu bringen…

Harry fragte sich, ob er überhaupt eine Bayes'sche Berechnung anstellen konnte. Natürlich ging es bei einer subjektiven Bayes'schen Berechnung nicht darum, dass man, nachdem man einen Haufen Zahlen erfunden hatte, durch Ausmultiplizieren eine exakt richtige Antwort erhalten würde. Der eigentliche Punkt war, dass der Prozess des Erfindens von Zahlen einen dazu zwingen würde, alle relevanten Fakten zusammenzuzählen und alle relativen Wahrscheinlichkeiten abzuwägen. Zum Beispiel, dass man, sobald man über die Wahrscheinlichkeit nachdachte, dass das Dunkle Mal nicht verblasst, wenn Du-weißt-schon-wer tot ist, feststellte, dass die Wahrscheinlichkeit nicht gering genug war, um die Beobachtung als starken Beweis zu werten. Eine Version des Prozesses bestand darin, Hypothesen zu zählen und Beweise aufzulisten, alle Zahlen aufzustellen, die Berechnung durchzuführen und dann die endgültige Antwort zu verwerfen und sich auf das Bauchgefühl des Gehirns zu verlassen, nachdem man es gezwungen hatte, alles wirklich abzuwägen. Das Problem war, dass die Beweise nicht bedingt unabhängig waren und es mehrere interagierende Hintergrundfakten gab, die von Interesse waren…

… naja, eine Sache war zumindest sicher. Wenn die Rechnung überhaupt aufgehen konnte, dann nur mit einem Blatt Papier und einem Bleistift.

Im Kamin an der einen Seite des Schulleiterbüros loderten plötzlich die Flammen auf und färbten sich von Orange zu leuchtendem Grün.

"Ah!", sagte Professor McGonagall in die unbehagliche Stille hinein. "Das ist wohl Mad-Eye Moody, nehme ich an."

"Lassen wir diese Angelegenheit erst einmal ruhen", sagte der Schulleiter etwas erleichtert, als auch er sich umdrehte, um den Floo zu betrachten. "Ich glaube, wir werden auch in dieser Sache bald Neuigkeiten erfahren."

\textbf{Hypothese: Hermine Granger} (8. April 1992, 18:53 Uhr)

Währenddessen in der Großen Halle von Hogwarts, als die Schüler, die keine geheimen Treffen mit dem Schulleiter hatten, sich um ihr Abendessen an vier riesigen Tischen tummelten -

"Es ist komisch", sagte Dekan Thomas nachdenklich. "Ich habe dem General nicht geglaubt, als er sagte, dass das, was wir gelernt haben, uns für immer verändern würde und wir danach nie wieder in ein normales Leben zurückkehren könnten. Sobald wir es wussten. Sobald wir sahen, was er sehen konnte."

"Ich weiß!", sagte Seamus Finnigan. "Ich dachte auch, es wäre nur ein Scherz! Wie alles andere, was General Chaos jemals gesagt hat."

"Aber jetzt -" sagte Dean traurig. "Wir können nicht mehr zurück, oder? Das wäre so, als würden wir wieder auf eine Muggelschule gehen, nachdem wir in Hogwarts waren. Wir müssen einfach… wir müssen einfach in der Nähe der anderen bleiben. Das ist alles, was wir tun können, sonst werden wir noch verrückt."

Seamus Finnigan, der neben ihm stand, nickte nur wortlos und aß einen weiteren Bissen Veldbeest. Um sie herum ging die Unterhaltung am Gryffindor-Tisch weiter. Es war nicht mehr so unerbittlich wie gestern, aber ab und zu wanderte das Thema zurück.

"Also, es muss eine Art Dreiecksbeziehung gegeben haben", sagte eine Hexe aus dem zweiten Jahr namens Samantha Crowley (sie antwortete nicht, als sie gefragt wurde, ob es eine Beziehung gab). "Die Frage ist, in welche Richtung es ging, bevor alles schief ging. Wer war in wen verliebt - und ob diese Person sie zurückgeliebt hat oder nicht - ich weiß nicht, wie viele Möglichkeiten es gibt -"

"Vierundsechzig", sagte Sarah Varyabil, eine aufblühende Schönheit, die wahrscheinlich stattdessen in Ravenclaw oder Hufflepuff hätte einsortiert werden sollen.

"Nein, warte, das ist falsch. Ich meine, wenn niemand Malfoy liebte und Malfoy niemanden liebte, dann wäre er nicht wirklich Teil des Liebesdreiecks … dafür braucht man Arithmetik, könnt ihr alle zwei Minuten warten?"

"Ich für meinen Teil halte es für völlig klar, dass Granger Potters Geliebte ist und dass Potter sich zwischen Malfoy und Granger gestellt hat." Die Hexe, die gesprochen hatte, nickte mit der Selbstzufriedenheit von jemandem, der gerade einen komplizierten Sachverhalt präzise auf den Punkt gebracht hat.

"Das sind nicht einmal Worte", wandte ein junger Zauberer ein. "Du denkst dir das einfach so aus."

"Manchmal kann man eine Sache nicht mit richtigen Worten beschreiben."

"Es ist so traurig", sagte Sherice Ngaserin, die tatsächlich Tränen in den Augen hatte. "Sie waren einfach - sie waren einfach so offensichtlich dafür bestimmt, zusammen zu sein!"

"Du meinst Potter und Malfoy?", sagte eine Schülerin im zweiten Jahr namens Colleen Johnson. "Ich weiß - ihre Familien haben sich so sehr gehasst, dass sie sich auf keinen Fall ineinander verlieben konnten -"

"Nein, ich meine alle drei", sagte Sherice.

Das erzeugte eine kurze Pause in der zusammengekauerten Unterhaltung. Dean Thomas verschluckte sich leise an seiner Limonade und versuchte, keine Geräusche zu machen, während sie aus seinem Mund tropfte und sein Hemd durchnässte.

"Wow", sagte eine dunkelhaarige Hexe mit dem Namen Nancy Hua. "Das ist wirklich … anspruchsvoll von dir, Sherice."

"Hört mal, ihr alle, wir müssen das realistisch halten", sagte Eloise Rosen, eine große Hexe, die General einer Armee gewesen war und daher mit einer gewissen Autorität sprach. "Wir wissen - weil sie ihn geküsst hat -, dass Granger in Potter

verliebt war. Der einzige Grund, warum sie versuchen würde, Malfoy zu töten, wäre also, wenn sie wüsste, dass sie Potter an ihn verlieren würde. Es gibt keinen Grund, das alles so kompliziert klingen zu lassen - ihr tut alle so, als wäre das ein Theaterstück und nicht das echte Leben!"

"Aber selbst wenn Granger verliebt wäre, ist es immer noch komisch, dass sie einfach so ausrastet", sagte Chloe, deren schwarzer Umhang in Kombination mit ihrer nachtschwarzen Haut sie wie eine verdunkelte Silhouette aussehen ließ. "Ich weiß es nicht… Ich glaube, da steckt vielleicht mehr dahinter als nur ein schiefgelaufener Liebesroman. Ich denke, vielleicht haben die meisten Leute überhaupt keine Ahnung, was hier los ist."

"Ja! Danke!", platzte Dean Thomas heraus. "Seht ihr -- begreift ihr nicht, wie Harry Potter es uns allen gesagt hat - wenn wir nicht vorausgesehen haben, dass etwas passieren würde, wenn es uns völlig überrascht hat, dann reicht das, was wir über die Welt geglaubt haben, als wir es nicht kommen sahen, nicht aus, um zu erklären …" Deans Stimme verstummte, als er sah, dass niemand zuhörte. "Es ist völlig hoffnungslos, nicht wahr?"

"Das hast du noch nicht herausgefunden?", fragte Lavender Brown, die ihren beiden ehemaligen Chaotenkollegen gegenüber saß. "Wie hast du es je zum Leutnant geschafft?"

"Oh, ihr zwei seid ruhig!" Sherice schnauzte die beiden an. "Es ist offensichtlich, dass ihr beide die drei für euch haben wollt!"

"Ich meine es ernst!" sagte Chloe. "Was ist, wenn das, was wirklich vor sich geht, anders ist als all die, du weißt schon, normalen Dinge, über die all die normalen Leute reden? Was, wenn jemand - Granger dazu gebracht hat, das zu tun, was sie getan hat, genau wie Potter es allen weismachen wollte?"

"Ich glaube, Chloe hat recht", sagte ein fremdländisch aussehender Zaubererjunge, der sich immer als 'Adrian Turnipseed' vorstellte, obwohl seine Eltern ihn eigentlich Mad Drongo genannt hatten. "Ich glaube, die ganze Zeit über gab es …" Adrian senkte bedrohlich seine Stimme, "… eine verborgene Hand …" Adrian erhob seine Stimme wieder, "die alles, was geschehen ist, gestaltet hat. Eine Person, die hinter allem steckt, von Anfang an. Und ich meine auch nicht Professor Snape."

"Du meinst doch nicht etwa -", keuchte Sarah.

"Doch", sagte Adrian. "Die wirkliche, die hinter allem steckt, ist - Tracey Davis!"

"Das denke ich auch", sagte Chloe. "Immerhin -" Sie blickte sich schnell um. "Seit der Sache mit den Tyrannen und der Decke - selbst die Bäume in den Wäldern um Hogwarts sehen aus, als würden sie zittern, als hätten sie Angst -"

Seamus Finnigan runzelte nachdenklich die Stirn. "Ich glaube, ich weiß, woher Harry seine … du weißt schon … hat", sagte Seamus und senkte seine Stimme so, dass nur Lavender und Dean sie hören konnten.

"Oh, ich weiß genau, was du meinst", sagte Lavender. Sie machte sich nicht die Mühe, ihre eigene Stimme zu senken. "Es ist ein Wunder, dass er nicht schon vor Ewigkeiten ausgeflippt ist und angefangen hat, alle umzubringen."

"Ich persönlich", sagte Dean, ebenfalls mit leiserer Stimme, "würde sagen, der wirklich beängstigende Teil ist - das hätten wir sein können."

"Ja", sagte Lavender. "Es ist gut, dass wir jetzt alle vollkommen zurechnungsfähig sind."

Dean und Seamus nickten feierlich.

\textbf{Hypothese: G. L.} (8. April 1992, 8: 08pm)

Das Floo-Feuer im Büro des Schulleiters loderte in einem hellen Blassgrün, das Feuer konzentrierte sich in sich selbst zu einem sich drehenden smaragdgrünen Wirbelsturm, flammte dann noch heller auf und spuckte eine menschliche Gestalt in die Luft - Es gab eine verschwommene Bewegung, als die sich materialisierende Gestalt einen Zauberstab hochschnappte und sich mit dem Schwung des Floo sanft drehte, wie ein Balletttanzschritt, so dass sein Schussbogen den gesamten 360-Grad-Bogen des Raumes abdeckte; und dann blieb die Gestalt ebenso abrupt stehen. Im ersten Augenblick, in dem Harry den Mann sah, noch bevor er das Auge aufnahm, bemerkte er die Narben an den Händen, die Narben im Gesicht, als wäre der Mann am ganzen Körper verbrannt und zerschnitten worden; obwohl nur die Hände und das Gesicht des Mannes sichtbar waren, von seinem ganzen Fleisch. Der Rest des Körpers des Mannes war verborgen, nicht in Roben gehüllt, sondern in Leder, das mehr wie eine Rüstung als wie Kleidung aussah; dunkelgraues Leder, das zu dem Wirrwarr an ergrautem Haar des Mannes passte. Das nächste, was Harrys Vision erfasste, war das strahlend blaue Auge, das die rechte Seite des Gesichts des Mannes einnahm. Ein Teil von Harrys Verstand erkannte, dass die Person, die Professor McGonagall "Mad-Eye Moody" genannt hatte, dieselbe war wie die, die Dumbledore "Alastor" genannt hatte, in der Erinnerung, die Dumbledore Harry gezeigt hatte; ein Bild von vor dem Ereignis, das jeden Zentimeter des Körpers des Mannes vernarbt und ein Stück aus seiner Nase herausgerissen hatte - und ein anderer Teil seines Verstandes bemerkte den Adrenalinstoß. Harry hatte aus reinem Reflex seinen Zauberstab gezogen, als der Mann so aus dem Floo herausgeschossen war, es hatte sich wie ein Hinterhalt angefühlt, Harrys Hand hatte bereits begonnen, seinen Zauberstab für ein Somnium zu richten, bevor er sich selbst hatte stoppen können.

Selbst jetzt hielt der gepanzerte Mann seinen Zauberstab waagerecht, nicht auf eine bestimmte Person gerichtet, sondern den ganzen Raum überblickend, und dieser Zauberstab war bereits in perfekter Linie mit seinen Augen, wie ein Soldat, der eine Waffe anvisiert. Es lag Gefahr in der Haltung des Mannes und dem Absatz seiner Stiefel, Gefahr in der Lederrüstung, die er trug, und Gefahr in diesem strahlend blauen Auge. Als der vernarbte Mann sprach und sich an den Schulleiter wandte, war seine Stimme kantig.

"Ich nehme an, du glaubst dieser Raum ist sicher?"

"Hier gibt es nur Freunde", sagte Dumbledore.

Der Kopf des Mannes ruckte in Richtung Harry.

"Dazu gehört er?!"

"Wenn Harry Potter nicht unser Freund ist", sagte Dumbledore ernst, "dann sind wir alle mit Sicherheit dem Untergang geweiht; also können wir genauso gut davon ausgehen, dass er es ist."

Der Zauberstab des Mannes blieb waagerecht, nicht ganz auf Harry gerichtet.

"Der Junge hat mich vorhin fast auf mich geschossen."

"Äh …" sagte Harry. Er bemerkte, dass seine Hand den Zauberstab immer noch fest umklammert hielt, und entspannte bewusst seine Hand und ließ sie zurück an seine Seite fallen. "Tut mir leid, du sahst ein bisschen … kampfbereit aus."

Der Zauberstab des vernarbten Mannes bewegte sich leicht von der Stelle weg, an der er fast auf Harry gezielt hatte, senkte sich aber nicht, und der Mann stieß ein kurzes, bellendes Lachen aus.

"Ständige Wachsamkeit, was, Junge?", sagte der Mann.

"Es ist keine Paranoia, wenn sie es wirklich auf dich abgesehen haben", rezitierte Harry das Sprichwort.

Der Mann wandte sich Harry ganz zu; und soweit Harry irgendeinen Ausdruck auf dem vernarbten Gesicht lesen konnte, sah der Mann jetzt interessiert aus.

Dumbledores Augen hatten etwas von dem leuchtenden Funkeln zurückgewonnen, das sie vor dem Ausbruch aus Askaban gehabt hatten, und unter seinem silbernen Schnurrbart lächelte er, als wäre dieses Lächeln nie weg gewesen.

"Harry, das ist Alastor Moody, auch Mad-Eye genannt, der nach mir den Orden des Phönix befehligen wird - falls mir etwas zustoßen sollte. Alastor, das ist Harry Potter. Ich habe die Hoffnung, dass ihr beide euch fantastisch verstehen werdet."

"Ich habe schon viel von dir gehört, Junge", sagte Mad-Eye Moody.

Sein eines dunkles, natürliches Auge blieb auf Harry fixiert, während der leuchtend blaue Punkt sich wild drehte und in seiner Fassung zu rotieren schien.

"Nicht alles davon ist gut. Ich habe gehört, dass man dich in der Abteilung den Dementor-Spuk nennt."

Nach einigem Überlegen entschied sich Harry, mit einem wissenden Lächeln zu antworten.

"Wie hast du das hingekriegt, Junge?", fragte der Mann leise.

Nun war auch sein blaues Auge auf Harry gerichtet.

"Ich hatte eine kleine Unterhaltung mit einem der Auroren, die den Dementor aus Askaban dorthin eskortiert haben. Beth Martin sagte, er käme direkt aus der Grube, und niemand habe ihm irgendwelche besonderen Anweisungen mit auf den Weg gegeben. Natürlich könnte sie lügen."

"Da war kein hinterhältiger Trick dabei", sagte Harry. "Ich habe es einfach auf die harte Tour gemacht. Natürlich könnte ich auch lügen."

Dumbledore lehnte sich in seinem Stuhl zurück und kicherte im Hintergrund, als wäre er nur ein weiteres Gerät im Büro des Schulleiters, und das war das Geräusch, das er machte.

Der vernarbte Mann wandte sich wieder dem Schulleiter zu, obwohl sein Zauberstab tief und in Harrys allgemeine Richtung gerichtet blieb. Als er sprach, war seine Stimme schroff und geschäftsmäßig.

"Ich habe eine Spur zu einem der letzten Gastgeber von Voldie. Bist du sicher, dass sein Schatten jetzt in Hogwarts ist?"

"Nicht sicher -" begann Dumbledore.

"Wie bitte was?!" unterbrach Harry ihn.

Nachdem er fast zu dem Schluss gekommen war, dass der Dunkle Lord nicht existierte, war es ein Schock zu hören, dass so sachlich darüber gesprochen wurde.

"Voldies Wirt", sagte Moody kurz. "Derjenige, den er besaß, bevor er Granger übernommen hat."

"Wenn die Erzählungen wahr sind", sagte Dumbledore, "gibt es irgendeine Vorrichtung, die Voldemorts Schatten an diese Welt bindet; und auf diese Weise kann er mit einem Wirt um den Besitz seines Körpers handeln und ihm einen Teil seiner Macht und seines Wissens übertragen -"

"Die offensichtliche Frage ist also, wer zu schnell zu viel Macht erlangt hat", sagte Moody abrupt. "Und es stellt sich heraus, dass es einen Kerl gibt, der die Bandon Banshee verbannt hat, einen ganzen abtrünnigen Vampirclan in Asien gepfählt hat, den Wagga-Wagga-Werwolf aufgespürt und ein Rudel Ghule mit einem Teesieb ausgelöscht hat. Und er nutzt das alles aus; es gibt Gerüchte über den Orden von Merlin. Scheint sich in einen Charmeur und Politiker verwandelt zu haben, nicht nur in einen mächtigen Zauberer."

"Du liebe Zeit", murmelte Dumbledore. "Bist du sicher, dass er sich nicht auf seine eigenen Fähigkeiten verlässt?"

"Ich habe seine Noten überprüft", sagte Moody. "Die Aufzeichnungen zeigen, dass Gilderoy Lockhart einen Troll in seinem Verteidigungs-Prüfungen erhalten hat und sich nicht um den U.T.Z. gekümmert hat. Genau die Art von Trottel, die den Deal annimmt, den Voldie angeboten hat." Das blaue Auge wirbelte wie verrückt in seiner Fassung. "Es sei denn, du erinnerst dich an Lockhart als Schüler und glaubst, dass er genug Potenzial hatte, um das alles allein zu schaffen?"

"Nein", sagte Professor McGonagall. Sie runzelte die Stirn. "Nicht die geringste Chance, würde ich sagen."

"Ich fürchte, ich muss zustimmen", sagte Dumbledore mit einem Unterton des Schmerzes. "Ah, Gilderoy, du armer Narr…"

Moodys Grinsen war eher ein Knurren. "Ist drei Uhr morgens für dich in Ordnung, Albus? Lockhart sollte heute Abend bei sich zu Hause sein."

Harry hörte sich das mit zunehmender Besorgnis an und fragte sich, ob es sogar im Ministerium Regeln gab, nach denen Richter Haftbefehle ausstellen mussten - ganz zu schweigen von der illegalen Bürgerwehr, der sich Harry nun anzuschließen schien.

"Entschuldigung", sagte Harry. "Was genau passiert um drei Uhr nachts?"

Irgendetwas in Harrys Stimme musste ihn verraten haben, denn der vernarbte Mann wirbelte auf ihn zu. "Hast du ein Problem damit, Junge?"

Harry hielt inne und versuchte herauszufinden, wie er es dem Fremden gegenüber formulieren sollte -

"Du willst ihn selbst zur Strecke bringen?", drängte der vernarbte Mann. "Rache für deine Eltern nehmen, was?"

"Nein", sagte Harry so höflich, wie er konnte. "Ehrlich - sieh mal, wenn wir sicher wüssten, dass er ein williger Wirt für Du-weißt-schon-wen ist, ist das eine Sache, aber wenn wir uns nicht sicher sind und du losziehst, um ihn zu töten -"

"Töten?" Mad-Eye Moody schnaubte. "Es ist das, was in seinem Kopf eingeschlossen ist", Moody tippte sich an die Stirn, "das wir von ihm brauchen, Junge. Wenn wir Glück haben, kann Voldie die Erinnerungen des Trottels nicht so leicht löschen wie zu Lebzeiten, und Lockhart wird sich daran erinnern, wie der Horkrux aussah."

Harry notierte sich gedanklich das Wort \emph{Horkrux} für zukünftige Nachforschungen und sagte: "Ich mache mir nur Sorgen, dass jemand Unschuldiges - was sich nach einem ziemlich anständigen Menschen anhört, wenn er das alles selbst gemacht hat - verletzt werden könnte."

"Auroren verletzen Menschen", sagte der vernarbte Mann kurz. "Böse Menschen, wenn man Glück hat. An manchen Tagen hat man kein Glück, und das ist alles, was es zu sagen gibt. Denk nur daran, dass dunkle Zauberer viel mehr Menschen verletzen als wir."

Harry nahm einen tiefen Atemzug. "Kannst du wenigstens versuchen, diese Person nicht zu verletzen, für den Fall, dass er nicht -"

"Was macht ein Erstklässler in diesem Raum, Albus?", fragte der vernarbte Mann und wirbelte nun zum Schulleiter herum. "Und erzähl mir nicht, dass es darum geht, was er als Baby getan hat."

"Harry Potter ist kein gewöhnlicher Erstklässler", sagte der Schulleiter leise. "Er hat bereits Leistungen vollbracht, die so unmöglich sind, dass selbst ich schockiert bin, Alastor. Er ist der einzige Intellekt im Orden, der es eines Tages mit dem von Voldemort selbst aufnehmen könnte, wie du oder ich es nie könnten."

Der vernarbte Mann lehnte sich über das Pult des Schulleiters. "Er ist eine Belastung. \emph{Naiv}. Er hat keine Ahnung, wie es im Krieg zugeht. Ich will, dass er von hier verschwindet und dass all seine Erinnerungen an den Orden ausgelöscht werden, bevor einer von Voldies Dienern sie ihm direkt aus dem Kopf reißt -"

"Ich bin ein Okklumens."

Mad-Eye Moody warf einen schmalen Blick auf den Schulleiter, der nickte. Und dann drehte sich der vernarbte Mann zu Harry um, \emph{ihre Blicke trafen sich.}

Die plötzliche Wut des Legilimationsangriffs ließ Harry fast von seinem Stuhl fallen, als eine Klinge aus weißglühendem Stahl in die imaginäre Person in seinem Kopf schnitt. Harry hatte seit Mr. Besters Training keine Gelegenheit zum Üben gehabt, und Harry verlor fast den Halt an der imaginären Person, die sein Hinterkopf zu sein vorgab, als sich die Welt dieser Person in glühende Lava und eine wütende Sonde von Fragen verwandelte. Harry verlor fast den Halt daran, nur so zu tun, als ob er halluzinieren würde, nur so zu tun, als ob er die imaginäre Person wäre, die vor Schock und Schmerz schrie, als die Legilimation seinen Verstand zerriss und ihn so umformte, dass er glaubte, er stünde in Flammen - Harry schaffte es, den Blickkontakt zu unterbrechen und ließ seinen Blick auf Moodys Kinn fallen.

"Du bist aus der Übung, Junge", sagte Moody.

Harry schaute nicht in das Gesicht des Mannes, aber seine Stimme war todernst. "Und ich warne dich nur ein einziges Mal davor. Voldie ist nicht wie jeder andere Legilimens in der aufgezeichneten Geschichte. Er braucht dir nicht in die Augen zu sehen, und wenn deine Schilde so rostig sind schleicht er sich so leise an, dass du gar nichts merkst."

"Ordnungsgemäß notiert", sagte Harry mit Blick auf das vernarbte Kinn. Harry war mehr erschüttert, als er zugeben wollte; Mr. Bester war nicht annähernd so mächtig gewesen und hatte Harry nie auf diese Weise getestet. So zu tun, als wäre er jemand, der so sehr leidet, hatte… Harry konnte keine Worte finden, um zu beschreiben, wie es sich anfühlte, eine imaginäre Person mit so viel Schmerz in sich zu tragen, aber es war nicht normal gewesen.

"Bekomme ich irgendeine Anerkennung dafür, dass ich überhaupt ein Okklumens bin?"

"Du hältst dich also schon für erwachsen, was? Sieh mir in die Augen!"

Harry verstärkte seine Schilde und blickte noch einmal in das dunkelgraue und das strahlend blaue Auge. "Hast du schon mal jemanden sterben sehen?", fragte Mad-Eye Moody.

"Meine Eltern", sagte Harry gleichmütig. "Ich habe die Erinnerung im Januar wiedergewonnen, als ich vor einen Dementor trat, um den Patronus-Zauber zu lernen. Ich erinnere mich an Du-weißt-schon-wen's Stimme -" Ein Schauer ging durch Harrys Körper, sein Zauberstab zuckte in seiner Hand. "Mein wichtigster taktischer Bericht ist, dass Du-weißt-schon-wer den Tötungsfluch in weniger als einer halben Sekunde sprechen konnte, aber das wusstet ihr wahrscheinlich schon."

Ein Keuchen ertönte aus Professor McGonagalls Richtung, und Severus' Gesicht hatte sich verkniffen.

"In Ordnung", sagte Mad-Eye Moody leise. Ein seltsames, dünnes Grinsen umspielte die Lippen in dem vernarbten Gesicht. "Ich mache dir dasselbe Angebot, das ich jedem Aurorenanwärter machen würde. Berühre mich nur einmal, Junge - ein Schlag, ein Zauber - und ich gebe dir das Recht, mir zu widersprechen."

"Alastor!", rief die Stimme von Professor McGonagall. "Das ist doch ein unangemessener Test! Mr. Potter, was auch immer seine sonstigen Verdienste sein mögen, verfügt nicht über hundert Jahre Kampferfahrung!"

Harrys Augen huschten blitzschnell durch den Raum, überflogen die merkwürdigen Geräte, blickten an Dumbledore und Severus und dem Sprechenden Hut vorbei und blieben hier und da kurz stehen. Harry konnte Professor McGonagall von dort, wo er stand, nicht sehen, aber das machte nichts. Es gab nur ein Gerät, das er sich wirklich ansehen wollte, und der Sinn all der anderen Blicke war nur gewesen, zu verschleiern, welches es war.

"In Ordnung", sagte Harry und hüpfte von seinem Stuhl, wobei er das Einatmen von Professor McGonagall und das ungläubige Schnauben des Zaubertränkemeisters ignorierte. Dumbledores Augenbrauen hatten sich gehoben, und Moody grinste wie ein Tiger. "Wecken Sie mich unbedingt in vierzig Minuten, wenn er mich erwischt hat."

Harry stellte sich in eine Duellanten-Startposition, den Zauberstab tief gehalten.

"Dann lass uns los -"

Harry öffnete die Augen, sein Kopf fühlte sich an, als hätte man ihn mit Watte ausgestopft. Alle anderen waren aus dem Büro des Schulleiters verschwunden, das Floo-Feuer war erloschen; nur Dumbledore wartete noch hinter dem Schreibtisch.

"Hallo, Harry", sagte der Schulleiter leise.

"Ich habe nicht einmal gesehen, dass er sich bewegt hat", wunderte sich Harry, dessen Muskeln schmerzten, als er sich aufsetzte.

"Du standest zwei Schritte von Alastor Moody entfernt", sagte Dumbledore, "und du hast seinen Zauberstab nicht gesehen."

Harry nickte, während er den Unsichtbarkeitsumhang aus seiner Tasche nahm. "Ich meine - ich habe die Duellstellung eingenommen, damit er mich für einen normalen Idioten hält und mich unterschätzt - aber ich muss zugeben, das war beeindruckend."

"Du hast es also die ganze Zeit geplant, Harry?" sagte Dumbledore.

"Natürlich", sagte Harry. "Man beachte, dass ich das gleich nach dem Aufwachen tue und nicht erst innehalte, um darüber nachzudenken."

Harry zog sich die Kapuze des Umhangs über den Kopf und blickte wieder auf die Wanduhr, auf die er vorhin heimlich einen Blick geworfen hatte. Sie hatte damals etwa dreiundzwanzig Minuten nach acht angezeigt, und jetzt war es fünf Minuten nach neun…

Minerva starrte darauf, wie der Junge sich in die Duellstellung begab, den Zauberstab tief gehalten. Eine Sekunde lang fragte sich Minerva, ob Harry vielleicht - nein, das war völlig lächerlich, es war Mad-Eye Moody und das war völlig unmöglich. Natürlich war es das, was sie auch über seine teilweise Verwandlung gedacht hatte…

"Dann lass uns los -", sagte Harry und fiel um. Severus gluckste nur.

"Mr. Potter hat seine Argumente, das muss ich zugeben", sagte der Zaubertränkemeister. "Obwohl ich es nie sagen würde, wenn er wach ist, und wenn du diese Worte wiederholst, werde ich sie leugnen, denn das Ego des Jungen ist schon groß genug. Mr. Potter hat durchaus seine Stärken, Mad-Eye, aber Duellieren gehört nicht dazu."

Mad-Eyes eigenes Glucksen war leiser und grimmiger. "Oh", sagte Mad-Eye. "Nur

Narren duellieren sich. Wie er so dasteht und darauf wartet, dass ich angreife, was hat sich der Junge dabei gedacht? Ich sollte ihm eine Narbe verpassen, damit er sich an diese Gelegenheit erinnert -"

"Alastor!", bellte Albus, gerade als sie "Stopp!" schrie, stürzte Severus nach vorne, und Mad-Eye Moody richtete seinen Zauberstab gezielt auf Harry Potters Körper.

"Stupor!"

Mad-Eyes Körper schien fast zu flackern, als er sich blitzschnell auf seinem hölzernen Fuß drehte, schneller als sie jemals jemanden ohne Magie hatte sich bewegen sehen, wobei der rote Fluch durch die plötzlich leere Luft flog und Severus nur knapp verfehlte, um in die gegenüberliegende Wand zu krachen, und als ihr Blick wieder zu Moody zurückkehrte, waren siebzehn leuchtende Kugeln im Muster einer Sagitta Magica zu sehen, die nur einen Augenblick lang zu sehen waren, bevor sie strahlend aufleuchteten und etwas trafen, das mit einem dumpfen Schlag zu Boden fiel -

…

"Hallo noch mal, Harry", sagte Dumbledore.

"Ich kann nicht glauben, wie schnell der Kerl reagiert", sagte Harry und streifte seinen Umhang ab, als er sich von der Stelle erhob, an der er unsichtbar auf dem Boden gelegen hatte, ungesehen von seinem früheren Ich. "Ich kann auch seine Bewegungsgeschwindigkeit nicht glauben. Ich muss einen Weg finden, ihn zu erwischen, ohne eine Beschwörung zu sprechen, die mich verrät…"

…

- und dann duckte sich Mad-Eye hart und schnell, seine Hände schlugen flach auf dem Boden auf. Fast hätte sie die beiden winzigen weißen Fäden nicht gesehen, die durch die Stelle in dem er sich nur Momente vorher noch befunden hatte, zogen, aber ihr Blick ging zu dem blauen Funken, als die Fäden auf einem der Geräte des Schulleiters aufschlugen, und als sie es schaffte, den Blick wieder abzuwenden, hatte sich Mad-Eye geschmeidig auf die Füße gedreht, sein Zauberstab tanzte unsichtbar schnell und es gab ein weiteres dumpfes Geräusch -

…

"Hallo noch mal, Harry."

"Verzeihen Sie, Schulleiter, aber könnten Sie mich Ihre Treppe hinuntergehen lassen und dann wieder hochkommen lassen, bevor ich den letzten Sprung rückwärts mache? Das wird länger dauern als eine Stunde Vorbereitung -"

…

Minerva starrte Mad-Eye Moody an, der seinen Zauberstab nicht im Geringsten gesenkt hatte; und Severus hatte einen Gesichtsausdruck, der fast wie ein Schock war.

"Nun, Junge?", sagte Mad-Eye Moody. "Was hast du noch?"

Harry Potters Kopf erschien und schwebte in der Luft, als eine unsichtbare Hand die Kapuze seines Unsichtbarkeitsumhangs zurückzog.

"Dieses Auge", sagte Harry Potter. In den Augen des Jungen lag ein seltsames, grimmiges Licht. "Das ist kein gewöhnliches Gerät. Es kann direkt durch meinen Unsichtbarkeitsumhang hindurchsehen. Du bist meinem verwandelten Taser ausgewichen, sobald ich ihn zu heben begann, obwohl ich keine Beschwörungsformeln gesprochen habe. Und jetzt, wo ich es mir noch einmal angeschaut habe - du hast alle meine vergangenen Ichs in dem Moment entdeckt, als du in diesen Raum gekommen bist, nicht wahr?"

Mad-Eye Moody lächelte, dasselbe zähnefletschende Grinsen, mit dem sie ihn gesehen hatte, als sie gegen Voldemort selbst angetreten waren.

"Verbringt man hundert Jahre damit, dunkle Zauberer zu jagen, sieht man alles", sagte Moody. "Ich habe einmal einen jungen Japaner verhaftet, der einen ähnlichen Trick versuchte. Er hat auf die harte Tour herausgefunden, dass seine Schattenreplikationstechnik meinem Auge nicht gewachsen war."

"Du siehst in alle Richtungen", sagte Harry Potter, immer noch dieses seltsame, grimmige Licht in seinem Blick. "Egal, wohin dieses Auge zeigt, es sieht alles um dich herum."

Moodys Tigergrinsen wurde breiter. "Du bist jetzt nicht mehr in diesem Raum", sagte Mad-Eye. "Glaubst du, das liegt daran, dass du nach dieser Zeit aufgibst, oder daran, dass du gewinnst? Irgendwelche Wetten, Junge?"

"Es ist mein letzter Versuch, denn ich habe beschlossen, meine letzten drei Stunden auf einen Versuch zu setzen", sagte Harry Potter. "Was die Frage angeht, ob ich gewinne -"

Ein verschwommenes Geräusch erfüllte die ganze Luft im Büro des Schulleiters. Mad-Eye Moody sprang mit blendender Geschwindigkeit zur Seite und einen Augenblick später schnellte Harrys Kopf nach hinten, während er schrie: "Stuporfy!"

Drei Schimmer in der Luft flogen an Harrys sich bewegendem Kopf vorbei, gerade als ein roter Blitz aus Harrys Zauberstab ausbrach und an Moody vorbeischoss, als dieser in eine andere Richtung auswich - hätte sie geblinzelt, hätte sie ihn verpasst, \emph{denn der rote Blitz machte in der Luft eine schräge Drehung} und schlug in Moodys Ohr ein.

Moody fiel.

Harry Potters schwebender Kopf sank auf die Höhe eines Erstklässlers auf Händen und Knien, dann fiel er weiter zu Boden, sein Gesicht zeigte plötzliche Erschöpfung.

Minerva McGonagall sagte: "Was in Merlins Namen ist gerade -"

…

"Du bist also zu Flitwick gegangen", sagte Moody.

Der pensionierte Auror saß jetzt auf einem Stuhl und trank lange Schlucke aus einem Stärkungsmittel in einer Flasche, die er von seinem Gürtel genommen hatte.

Harry Potter nickte, der jetzt in seinem eigenen Stuhl saß, anstatt auf einer Armlehne zu hocken. "Ich habe es zuerst beim Verteidigungsprofessor versucht, aber -" Der Junge schnitt eine Grimasse. "Er… war nicht verfügbar. Nun, ich hatte beschlossen, dass es das Risiko von fünf Hauspunkten wert ist, und wenn man sagt, dass ein Risiko es wert ist, kann man sich nicht beschweren, wenn man es bezahlen muss. Wie auch immer, ich dachte mir, wenn man ein Auge hat, das Dinge sieht, die andere Menschen nicht sehen können, dann ist, wie Isaac Asimov in "Second Foundation" dargelegt hat, die Waffe, die man benutzt, ein helles Licht. Wenn man genug Science-Fiction liest, hat man alles mindestens einmal gelesen. Jedenfalls sagte ich Professor Flitwick, dass ich einen Zauber brauche, der eine riesige Anzahl von Formen erzeugt, die hell und flackernd sind und das ganze Büro ausfüllen, aber unsichtbar, so dass nur dein Auge sie sehen kann. Ich hatte keine Ahnung, was es überhaupt bedeuten würde, eine Illusion zu zaubern und sie dann unsichtbar zu machen, aber ich dachte mir, wenn ich das nicht laut sage, würde Professor Flitwick es sowieso tun, und das tat er auch. Es stellte sich heraus, dass es keinen solchen Zauber gab, den ich selbst sprechen konnte, aber Flitwick zauberte mir ein einmaliges Gerät dafür - obwohl ich ihn davon überzeugen musste, dass es kein Betrug war, da nichts gegen einen Auror, der lange genug gelebt hatte, um in den Ruhestand zu gehen, Betrug sein konnte. Und dann wusste ich immer noch nicht, wie ich dich treffen sollte, wenn du dich so schnell bewegst. Also fragte ich nach gezielten Zaubern, und da zeigte mir Flitwick den Fluch, den ich am Ende gewirkt hatte, den "Ausweichenden Betäuber". Es ist eine von Professor Flitwicks eigenen Erfindungen - er ist sowohl ein Meister im Duellieren als auch ein Meister der Zauberei -"

"Das weiß ich, mein Sohn."

"Tut mir leid. Wie auch immer, der Professor sagt, er hat den Duellierzirkus verlassen, bevor er die Chance hatte, diesen Zauber zu benutzen, da er nur als Finishing Move bei einem ungeschützten Gegner funktioniert. Das Feld kommt auf seiner ursprünglichen Flugbahn so nah wie möglich an das Ziel heran, und sobald es merkt, dass das Ziel wieder weiter weg ist, dreht es sich in der Luft und fliegt direkt auf das Ziel zu. Der Fluch kann das nur einmal - aber die Beschwörungsformel klingt sehr ähnlich wie "Stupor" und hat die gleiche rote Farbe, wenn also der Gegner denkt, es sei ein normaler Betäubungsfluch und versucht, normal auszuweichen, wird er durch die Neuausrichtung in der Luft erledigt. Oh, und der Professor hat darum gebeten, dass keiner von uns über seinen Spezialzug spricht, nur für den Fall, dass er eines Tages die Chance bekommt, ihn im Wettkampf einzusetzen."

"Aber -", sagte Professor McGonagall. Sie warf einen Blick auf Mad-Eye Moody, der zustimmend nickte, und auf Severus, der sein Gesicht ausdruckslos hielt. "Mr. Potter, Sie haben gerade Mad-Eye Moody verblüfft! Den berühmtesten Jäger der dunklen Zauberer in der Geschichte des Aurorenbüros! Das hätte unmöglich sein sollen!"

Moody stieß ein dunkles Glucksen aus. "Wie lautet deine Antwort darauf, Kleiner? Ich bin neugierig."

"Nun…" sagte Harry. "Zunächst einmal, Professor McGonagall, hat keiner von uns beiden ernsthaft gekämpft."

"Keiner von Ihnen?"

"Natürlich", sagte Harry. "In einem ernsthaften Kampf hätte Mr. Moody alle meine Kopien sofort verflucht, ohne zu warten, bis sie angreifen. Und von meiner Seite aus, wenn es tatsächlich nötig wäre, den berühmtesten Auror in der Geschichte des Amtes auszuschalten, würde ich Schulleiter Dumbledore dazu bringen, es für mich zu tun. Und darüber hinaus … da das kein echter Kampf war …" Harry hielt inne. "Wie soll ich das ausdrücken? Zauberer sind an Duelle gewöhnt, bei denen man eine Weile mit Zaubersprüchen herumspielt. Aber wenn zwei Muggel mit Pistolen in einem kleinen Raum stehen und sich gegenseitig beschießen… dann gewinnt derjenige, der zuerst trifft. Und wenn einer von ihnen absichtlich seine Schüsse verfehlt und dem anderen eine Chance nach der anderen gibt - so wie Mr. Moody mir eine Chance nach der anderen gegeben hat - nun, dann muss man schon ziemlich erbärmlich sein, um zu verlieren."

"Oh, nicht so erbärmlich", sagte Moody mit einem leicht bedrohlichen Grinsen.

Harry schien das nicht zu bemerken. "Man könnte sagen, dass Mr. Moody mich getestet hat, um zu sehen, ob ich versuchen würde, ihn zu bekämpfen oder zu gewinnen. Das heißt, ob ich die Rolle von jemandem, der kämpft, ausführen würde - Standardzauber verwenden, die ich bereits kannte, auch wenn ich nicht erwartete, dass die Folgen dieser Handlung der Sieg sein würden - oder ob ich ungewöhnliche Pläne durchforsten würde, bis ich etwas finden würde, das gewinnen könnte. Wie der Unterschied zwischen einem Schüler, der in der Klasse sitzt, weil das Schüler

eben so machen, und einem Schüler, der sich genug Gedanken macht, um sich zu fragen, was es braucht, um einen Stoff wirklich zu lernen, und der übt, wie es nötig ist - sehen Sie, Professor McGonagall? Wenn man es so betrachtet - und erkennt, dass Mr. Moody mir eine Chance gegeben hat, und dass ich gar nicht erst angreifen sollte, wenn ich nicht glaube, dass ich gewinnen kann - dann sehe ich nicht so gut aus, denn ich habe drei Versuche gebraucht, um ihn zu erwischen. Außerdem, wie gesagt, in einem richtigen Kampf hätte sich Mr. Moody unsichtbar machen können oder Schilde aufstellen -"

"Verlassen dich nicht zu sehr auf Schilde, Junge", sagte Mad-Eye. Der ledergekleidete Auror nahm einen weiteren Schluck aus seinem Fläschchen. "Was du in deinem ersten Jahr an der Akademie lernst, gilt nicht ewig, nicht gegen die stärksten dunklen Zauberer. Auf jedem Schild, der jemals hergestellt wurde, gibt es einen Fluch, der direkt durch ihn hindurchgeht, wenn man nicht schnell genug ist, den Gegenzauber zu wirken. Und es gibt einen Fluch, der durch alles hindurchgeht, und das ist ein Fluch, den jeder Todesser benutzen wird."

Harry Potter nickte ernsthaft. "Richtig, manche Flüche sind unmöglich zu blocken. Ich werde mir das merken, falls jemand den Tötungsfluch gegen mich einsetzt. Schon wieder."

"Diese Art von Cleverness bringt Menschen um, Junge, und vergiss das bloß nicht."

Ein traurig klingender Seufzer von dem Jungen, der gelebt hat. "Ich weiß. Tut mir leid."

"Also, mein Sohn. Du hattest etwas dazu zu sagen, wenn Albus und ich Lockhart verfolgen?"

Harry öffnete den Mund, dann hielt er inne. "Ich werde dir nicht sagen, wie man einen Krieg führt", sagte der Junge-der-lebte schließlich. "Darin habe ich keine Erfahrung. Ich weiß nur, dass es Konsequenzen hat. Bitte beachte, dass Lockhart nach meiner Einschätzung wahrscheinlich unschuldig ist, wenn du es also vermeiden kannst, ihn ohne allzu großes Risiko zu verletzen -" Der Junge zuckte mit den Schultern. "Ich kenne die Kosten nicht. Aber bitte, wenn du kannst, achte darauf, ihn nicht zu verletzen, wenn er unschuldig ist."

"Wenn ich kann", sagte Moody.

"Und - du hast vor, in seinem Verstand nach Beweisen für den Dunklen Lord zu suchen, nicht wahr? Ich weiß nicht, wie die Regeln im magischen Britannien bezüglich zulässiger Beweise sind - aber jeder ist immer schuldig, das eine oder andere Gesetz gebrochen zu haben, es gibt einfach zu viele Gesetze. Wenn es also nicht um den Dunklen Lord geht, liefer ihn nicht an das Ministerium aus, sondern mach ihn einfach unschädlich und gehe, okay?"

Moody runzelte die Stirn. "Sohn, niemand erlangt so schnell Macht, ohne etwas im Schilde zu führen."

"Dann überlasse es den normalen Auroren, wenn sie auf normalem Wege Beweise finden. Bitte, Mr. Moody. Nenn es eine Marotte meiner Muggel-Erziehung, aber wenn es nicht um den Krieg geht, will ich nicht, dass wir die böse Polizei sind, die mitten in der Nacht in die Häuser der Leute einbricht, in ihren Köpfen wühlt und sie nach Askaban schickt."

"Ich sehe den Sinn darin nicht, mein Sohn, aber ich könnte dir wohl den Gefallen tun."

"Gibt es sonst noch etwas, Alastor?", erkundigte sich Albus. "Ja", sagte Moody. "Wegen deines Verteidigungsprofessors -" Hypothese: Gilderoy Lockhart: ENDE

\textbf{Hypothese: Dumbledore} (9. April 1992, 17:32 Uhr)

Als Professor Quirrell seinen Tee langsam anhob, ruckte die Teetasse in der Luft und ließ die dunkle, durchscheinende Flüssigkeit gerade noch über den Rand schwappen, so dass nur drei einzelne Tropfen an der Seite der Teetasse herunterkrochen.

Harry hätte es übersehen, wenn er nicht zufällig genau hingesehen hätte; denn Professor Quirrells Hand lag vorher und nachher vollkommen ruhig auf der Tasse. Wenn sich diese kleine ungewollte ruckartige Bewegung zu einem ständigen Zittern ausweitete, wäre das das Ende jeglicher zauberstabloser Magie für den Verteidigungsprofessor gewesen. Präzision hatte keinen Platz für zitternde Finger. Wie sehr das Professor Quirrell tatsächlich behindern würde, wenn überhaupt, konnte Harry nicht einschätzen. Der Verteidigungsprofessor war sicherlich zur zauberstablosen Magie fähig, neigte aber dennoch dazu, für größere Dinge einen Zauberstab zu benutzen - aber das war für ihn vielleicht nur eine Bequemlichkeit…

"Wahnsinn", sagte Professor Quirrell, während er vorsichtig an seinem Tee nippte - er schaute auf die Teetasse, nicht auf Harry, was für ihn ungewöhnlich war - "kann eine ganz eigene Signatur haben."

Das kleine Büro des Verteidigungsprofessors war still, der schallgedämpfte Raum auf eine Weise ruhig, wie es das Büro des Schulleiters nie sein konnte. Manchmal kam es vor, dass sie beide gleichzeitig aus- oder einatmeten, und dann herrschte eine akustische Leere, die fast ein Geräusch für sich war.

"Dem stimme ich in einem Punkt zu", sagte Harry. "Wenn mir jemand erzählt, dass alle ihn anstarren und dass seine Unterwäsche mit gedankenkontrollierendem Puder bestäubt wird, weiß ich, dass er psychotisch ist, denn das ist die Standardsignatur einer Psychose. Aber wenn Sie mir sagen, dass alles, was verwirrend ist, auf Albus Dumbledore als Verdächtigen hindeutet, dann scheint das… übertrieben zu sein. Nur weil ich keinen Zweck erkennen kann, heißt das nicht, dass es keinen Zweck gibt."

"Zweck?", sagte Professor Quirrell. "Oh, aber der Wahnsinn von Dumbledore ist nicht, dass er zwecklos ist, sondern dass er zu viele Zwecke hat. Der Schulleiter könnte dies geplant haben, um Lucius Malfoy dazu zu bringen, sein Spiel wegzuwerfen, um sich an dir zu rächen - oder es könnte ein Dutzend anderer Pläne sein. Wer weiß, was der Schulleiter zu tun gedenkt, wenn er schon zu so vielen seltsamen Dingen Anlass gefunden hat?"

Harry hatte den Tee höflich abgelehnt, obwohl er wusste, dass Professor Quirrell wissen würde, was das bedeutete. Er hatte in Erwägung gezogen, seine eigene Limonade mitzubringen - hatte sich aber auch dagegen entschieden, nachdem ihm klar geworden war, wie leicht es für den Verteidigungsprofessor sein würde, ein bisschen Zaubertrank hineinzuteleportieren, selbst wenn die beiden sich nicht mit direkter Magie berühren konnten.

"Ich habe jetzt ein wenig von Dumbledore gesehen", sagte Harry. "Wenn nicht alles, was ich gesehen habe, eine Lüge ist, fällt es mir schwer zu glauben, dass er planen würde, irgendeinen Hogwarts-Schüler nach Askaban zu schicken. Niemals."

"Ah", sagte der Verteidigungsprofessor leise, die winzige Reflexion der Teetasse schimmerte in seinen blassen Augen. "Aber vielleicht ist das eine andere Signatur, Mr. Potter. Du hast die Perspektive eines Mannes wie Dumbledore noch nicht begriffen. Wenn er für eine hinreichend edle Sache einen Schüler opfern muss - wen sollte er wählen, wenn nicht die, die sich selbst zur Heldin erklärt hat?"

Das ließ Harry etwas innehalten. Vielleicht war es nur eine Voreingenommenheit im Nachhinein, aber das schien einen Teil der Wahrscheinlichkeitsmasse dieser Hypothese darauf zu konzentrieren, vor allem Hermine reinzulegen. In ähnlicher Weise hatte Professor Quirrell vorhergesagt, dass Dumbledore es auf Draco abgesehen haben könnte…

\emph{Aber wenn Sie hinter all dem stecken, Professor, könnten Sie Ihre Pläne so gestaltet haben, dass Sie dem Schulleiter etwas anhängen wollten, und dafür gesorgt haben, dass er sich im Voraus verdächtig gemacht hat. Der Begriff 'Beweise' hatte eine etwas andere Bedeutung, wenn man es mit jemandem zu tun hatte, der über sich selbst erklärt hatte, das Spiel auf 'einer höheren Ebene als du' zu spielen.}

"Ich verstehe, worauf Sie hinauswollen, Professor", sagte Harry gleichmäßig, ohne einen Hinweis auf seine weiteren Gedanken zu geben. "Sie halten es also für sehr wahrscheinlich, dass es der Schulleiter war, der Hermine reingelegt hat?"

"Nicht unbedingt, Mr. Potter." Professor Quirrell leerte seine Teetasse in einem Schluck und stellte sie dann ab, wobei die Tasse mit einem scharfen Knall zu Boden fiel. "Es gibt auch Severus Snape - was er allerdings davon zu halten gedenkt, kann ich nicht erahnen. Er ist also auch nicht mein Hauptverdächtiger."

"Wer ist es dann?" fragte Harry etwas verwundert.

\emph{Professor Quirrell hatte sicher nicht vor, "Du-weißt-schon-wer" zu antworten -}

"Die Auroren haben eine Regel", sagte Professor Quirrell. "Untersuche das Opfer. Viele Möchtegern-Verbrecher bilden sich ein, dass sie nicht verdächtigt werden, wenn sie das scheinbare Opfer eines Verbrechens sind. So viele Verbrecher bilden sich das ein, dass jeder Auror das schon ein Dutzend Mal gesehen hat."

"Sie wollen mir doch nicht ernsthaft weismachen, dass Hermine -"

Der Verteidigungsprofessor warf Harry einen dieser schlitzäugigen Blicke zu, die bedeuteten, dass er dumm war.

\emph{Draco?} Draco war unter Veritaserum verhört worden - aber Lucius könnte genug Kontrolle gehabt haben, um Auroren zu unterwandern, um … oh.

"Sie denken, Lucius Malfoy hat seinen eigenen Sohn benutzt?" sagte Harry.

"Warum nicht?" sagte Professor Quirrell leise. "Aus Mr. Malfoys aufgezeichneter Aussage, Mr. Potter, entnehme ich, dass Sie einigen Erfolg dabei hatten, Mr. Malfoys politische Ansichten zu ändern. Wenn Lucius Malfoy das früher erfahren hätte … wäre er vielleicht zu dem Schluss gekommen, dass sein ehemaliger Erbe zu einer Belastung geworden ist."

"Das kaufe ich Ihnen nicht ab", sagte Harry barsch.

"Du bist sträflich naiv, Mr. Potter. Die Geschichtsbücher sind voll von Familienstreitigkeiten, die mörderisch endeten, wegen Unannehmlichkeiten und Drohungen, die weitaus geringer waren als die, die Mr. Malfoy seinem Vater gegenüber aussprach. Ich nehme an, als Nächstes wirst du mir erzählen, dass Lord Malfoy von den Todessern viel zu sanftmütig ist, um seinem Sohn solches Leid zu wünschen."

Ein Anflug von schwerem Sarkasmus. "Nun, ja, ehrlich gesagt", sagte Harry. "Liebe ist real, Professor, ein Phänomen mit beobachtbaren Auswirkungen. Gehirne sind

real, Emotionen sind real, und die Liebe ist genauso Teil der realen Welt wie Äpfel und Bäume. Wenn Sie experimentelle Vorhersagen machen würden, ohne die elterliche Liebe zu berücksichtigen, hätten Sie große Schwierigkeiten zu erklären, warum meine eigenen Eltern mich nach dem Vorfall mit dem Wissenschaftsprojekt nicht in einem Waisenhaus ausgesetzt haben."

Der Verteidigungsprofessor reagierte darauf überhaupt nicht.

Harry fuhr fort. "Nach dem, was Draco sagt, hat Lucius ihn bei wichtigen Abstimmungen im Zaubergamot bevorzugt. Das ist ein gewichtiges Indiz, denn es gibt billigere Wege, Liebe vorzutäuschen, wenn man sie nur vortäuschen will. Und es ist nicht so, dass die Wahrscheinlichkeit, dass ein Elternteil sein Kind liebt, gering ist. Ich nehme an, es ist möglich, dass Lucius nur die Rolle des liebenden Vaters übernommen hat, und er hat diese Rolle aufgegeben, als er erfuhr, dass Draco mit Muggelgeborenen verkehrt. Aber wie man so schön sagt, Professor, man muss die Möglichkeit von der Wahrscheinlichkeit unterscheiden."

"Umso besser das Verbrechen", sagte der Verteidigungsprofessor, immer noch in diesem sanften Tonfall, "wenn es ihm niemand abnimmt."

"Und wie sollte Lucius überhaupt einen Gedächtniszauber auf Hermine ausüben, ohne die Schutzzauber auszulösen? Er ist kein Professor - oh, richtig, Sie denken, es ist Professor Snape."

"Falsch", sagte der Verteidigungsprofessor. "Lucius Malfoy würde keinen Diener mit dieser Mission betrauen. Aber nehmen wir an, ein Hogwarts-Professor, der intelligent genug ist, um einen gut geformten Gedächtniszauber zu wirken, aber nicht über große Kampffähigkeiten verfügt, ist in Hogsmeade unterwegs. Aus einer dunklen Gasse tritt die schwarzgekleidete Gestalt von Malfoy hervor - er würde dafür persönlich kommen - und spricht ein einziges Wort zu ihr."

"Imperio."

"Legilimens, eher", sagte Professor Quirrell. "Ich weiß nicht, ob die Hogwarts-Schutzzauber für einen zurückkehrenden Professor unter dem Imperius-Fluch auslösen würden. Und wenn ich es nicht weiß, weiß es Lord Malfoy wahrscheinlich auch nicht. Aber Malfoy ist immerhin ein perfekter Okklumentiker; er könnte Legilimenz einsetzen. Und was das Ziel angeht … vielleicht Aurora Sinistra; niemand würde den Astronomieprofessor in Frage stellen, der nachts herumläuft."

"Oder noch besser, Professor Sprout", sagte Harry. "Da sie die letzte Person ist, die jemand verdächtigen würde."

Der Verteidigungsprofessor zögerte kurz.

"Vielleicht."

"Eigentlich", sagte Harry dann und legte ein nachdenkliches Stirnrunzeln auf sein Gesicht, "ich nehme nicht an, dass Sie aus dem Stegreif wissen, ob einer der jetzigen Professoren in Hogwarts damals dabei war, als Mr. Hagrid 1943 zu unrecht verurteilt wurde?"

"Dumbledore unterrichtete Verwandlung, Kettleburn unterrichtete Magische Geschöpfe und Vector unterrichtete Arithmetik", sagte Professor Quirrell sofort. "Und ich glaube, dass Bathsheda Babbling, jetzt von Alte Runen, damals Vertrauensschülerin in Ravenclaw war. Aber Mr. Potter, es gibt keinen Grund zu der Annahme, dass irgendjemand außer Du-weißt-schon-wem in diese Affäre verwickelt war."

Harry zuckte kunstvoll mit den Schultern. "Es schien mir wert, die Frage zu stellen, nur um sicherzugehen. Wie dem auch sei, Professor, ich stimme zu, dass es möglich ist, dass ein Außenstehender ein Mitglied des Hogwarts-Personals legilimiert hat - und sie danach vergiftet hat, diesen Teil darf man nicht vergessen. Aber ich denke nicht, dass Lucius Malfoy ein wahrscheinlicher Kandidat für das Superhirn ist. Es ist möglich, aber nicht wahrscheinlich, dass Lucius' scheinbare Liebe zu Draco nur ein Pflichtgefühl war und sich alles in Rauch aufgelöst hat. Es ist möglich, aber nicht wahrscheinlich, dass alles, was Lucius vor dem Zaubergamot tat, nur ein Schauspiel war. Das Äußere eines Menschen gleicht nicht immer seinem Inneren, wie Sie sagten. Aber es gibt ein Beweisstück, das überhaupt nicht passt."

"Und das wäre?", fragte der Verteidigungsprofessor mit halbgeschlossenen Augen.

"Lucius hat versucht, hunderttausend Galleonen für Hermines Leben abzulehnen. Ich habe gesehen, wie überrascht das Zaubergamot war, als Lucius sagte, dass er es trotz der Regeln der Ehre ablehnt. Das hat das Zaubergamot nicht von ihm erwartet. Warum sollte er das Geld nicht einfach nehmen und dabei so tun, als würde er die Zähne zusammenbeißen? Es würde ihm nicht wirklich etwas ausmachen, Hermine nach Askaban zu werfen."

Es gab eine Pause.

"Vielleicht ist die Rolle, die er gespielt hat, mit ihm durchgebrannt", sagte Professor Quirrell. "Das kommt vor, Mr. Potter, in der Hitze des Gefechts."

"Vielleicht", sagte Harry. "Aber es ist immer noch eine Unwahrscheinlichkeit mehr, die postuliert werden muss - und wenn man so viele Ausreden in einer Theorie zusammenzählen muss, kann sie nicht mehr ganz oben auf der Liste stehen. Gibt es noch irgendetwas Spezielles, von dem Sie denken, dass ich darüber nachdenken sollte, im Rahmen aller anderen Möglichkeiten?"

Es herrschte ein langes Schweigen.

Der Blick des Verteidigungsprofessors fiel auf die leere Teetasse vor ihnen und wirkte ungewöhnlich distanziert.

"Ich denke, mir fällt noch ein letzter Verdächtiger ein", sagte der Verteidigungsprofessor schließlich.

Harry nickte.

Der Verteidigungsprofessor schien es nicht zu bemerken, sondern sprach einfach weiter. "Hat Ihnen der Schulleiter irgendetwas - auch nur eine Andeutung - über Professor Trelawneys Prophezeiung erzählt?"

"Was?" sagte Harry automatisch und wandelte seinen eigenen plötzlichen Schock in das beste Versteckspiel um, das er zustande brachte.

\emph{Wahrscheinlich war es auf dem falschen Niveau, um Professor Quirrell zu täuschen, aber Harry konnte sich keine Zeit zum Nachdenken nehmen, bevor er antwortete} - \emph{Moment, aber wie um alles in der Welt sollte Professor Quirrell davon wissen -}

"Professor Trelawney hat eine Prophezeiung gemacht?"

"Du warst dabei, als sie begann", sagte Professor Quirrell und runzelte die Stirn. "Du hast der ganzen Schule zugerufen, dass die Prophezeiung nicht von dir handeln könne, da du nicht hierher kämst, sondern bereits hier bist."

\emph{ER KOMMT. DER EINE; DER ZERRESIER DER S…-}

Und das war alles, was Professor Trelawney gesagt hatte, bevor Dumbledore sie gepackt hatte und verschwunden war.

"Oh, diese Prophezeiung", sagte Harry. "Tut mir Leid! Sie ist mir völlig aus dem Kopf gegangen." Harry glaubte, dass er zu viel Kraft in die letzte Aussage gesteckt hatte, und erwartete zu 80 \%, dass Professor Quirrell sagen würde: \emph{Aha, also Mr. Potter, was ist das für eine mysteriöse andere Prophezeiung, die Sie so sehr zu leugnen versuchten -}

"Das ist dumm", sagte der Verteidigungsprofessor scharf, "wenn du mir tatsächlich die Wahrheit sagst. Prophezeiungen sind keine trivialen Dinge. Ich habe mir viel den Kopf zerbrochen über das Wenige, das ich gehört habe, aber ein so kleines Fragment ist einfach zu wenig."

"Sie glauben, derjenige, der kommt, ist derjenige, der Hermine hereingelegt haben könnte?", fragte Harry. Während sein Verstand eine weitere Hypothese zuordnete, unsicherer Prädikatsreferent, er-der-kommt.

"Ohne Miss Granger zu nahe treten zu wollen", sagte der Verteidigungsprofessor mit einem weiteren Stirnrunzeln, "ihr Leben oder Tod scheint nicht so wichtig zu sein. Aber es sollte jemand kommen - einer, der nach deiner Interpretation nicht schon da war - und jemand, der so bedeutend ist und als Akteur unbekannt … wer weiß, was er sonst noch getan haben könnte?"

Harry nickte und seufzte innerlich, weil er seine Lord-Voldemort-Wahrscheinlichkeitsberechnung mit einem weiteren Beweisstück in der Mischung neu machen musste.

Professor Quirrell sprach mit halbgeschlossenen Augen, die wie durch Schlitze herausschauten. "Mehr als die Frage, von wem die Prophezeiung sprach - wer sollte sie hören? Es heißt, dass Schicksale zu denen gesprochen werden, die die Macht haben, sie herbeizuführen oder abzuwenden. Dumbledore. Ich selbst. Du. Und als entfernter 4. Severus Snape. Aber von diesen 4 würden Dumbledore und Snape oft in Trelawneys Gegenwart sein. Du und ich sind diejenigen, die vor diesem Sonntag nicht viel Zeit in ihrer Nähe verbracht haben. Ich halte es für ziemlich wahrscheinlich, dass die Prophezeiung für einen von uns bestimmt war - bevor Dumbledore die Prophetin weggebracht hat. Hat der Schulleiter nichts mehr zu dir gesagt?" Professor Quirrells Stimme war jetzt fordernd. "Ich dachte, ich hätte zu viel Kraft in diesem Dementi gehört, Mr. Potter."

"Ehrlich gesagt, nein", sagte Harry. "Es war mir ehrlich gesagt völlig entfallen."

"Dann bin ich ziemlich verärgert über ihn", sagte Professor Quirrell leise. "Ich glaube sogar, dass ich wütend bin."

Harry sagte nichts. Er schwitzte nicht einmal. Es mochte ein schlechter Grund für Zuversicht sein, aber in diesem speziellen Punkt war Harry tatsächlich unschuldig. Professor Quirrell nickte einmal heftig, wie zur Bestätigung.

"Wenn es zwischen uns nichts mehr zu sagen gibt, Mr. Potter, darfst du gehen."

"Mir fällt noch ein anderer Verdächtiger ein", sagte Harry. "Jemand, den Sie gar nicht auf Ihre Liste gesetzt haben. Würden Sie ihn für mich analysieren, Professor?"

Es gab wieder einen dieser Momente der Stille, die fast schon ein Geräusch an sich war.

"Was diesen Verdächtigen betrifft", sagte der Verteidigungsprofessor leise, "so denke ich, dass du ihn allein analysieren solltest, Mr. Potter, ohne meine Hilfe. Ich habe solche Bitten schon einmal gehört, und die Erfahrung bringt mich dazu, sie abzulehnen. Entweder werde ich die Anklage selbst zu gut machen und dich davon überzeugen, dass ich schuldig bin - oder du wirst entscheiden, dass meine Anklage zu halbherzig war und dass ich schuldig bin. Zu meiner Verteidigung möchte ich nur anmerken, dass ich in der Tat einen sehr guten Grund gebraucht hätte, um Ihre fragile Allianz mit dem Erben des Hauses Malfoy zu gefährden."

\textbf{Hypothese: Der Verteidigungsprofessor} (8. April 1992, 20:37 Uhr)

"… so fürchte ich, dass ich mich verabschieden muss", sagte Dumbledore mit ernster Miene. "Ich habe es Quirinus versprochen… das heißt, ich habe es dem Verteidigungsprofessor versprochen…. dass ich keinen Versuch unternehmen würde, seine wahre Identität aufzudecken, weder in meiner eigenen Person noch in einer anderen."

"Und warum hast du einem Dummkopf dann so ein Versprechen gegeben?", schnauzte Mad-Eye Moody.

"Es war eine unabänderliche Bedingung für seine Anstellung, zumindest sagte er das." Dumbledore warf einen Blick auf Professor McGonagall, ein schiefes Lächeln huschte kurz über sein Gesicht. "Und Minerva hat mir klargemacht, dass Hogwarts in diesem Jahr einen kompetenten Verteidigungsprofessor braucht, selbst wenn ich Grindelwald aus Nurmengard herausholen und alte Zuneigungen geltend machen müsste, um ihn zu überreden, die Stelle anzunehmen."

"Ich habe es nicht ganz so ausgedrückt -"

"Deine Ausdrucksweise hat es für dich gesagt, meine Liebe."

Und so saßen die vier - Harry, Professor McGonagall, der Zaubertränkemeister und Alastor Moody alias "Mad-Eye" - bald ganz allein im Büro des Schulleiters. Es war seltsam, wie das Büro des Schulleiters… unausgeglichen wirkte… ohne den Schulleiter darin. Ohne den alten, verhutzelten Meister, der alles feierlich erscheinen ließ, waren sie nur vier Leute, die versuchten, eine ernsthafte Besprechung abzuhalten, während sie von bizarren, lärmenden Artefakten umgeben waren. Von der Stelle aus, an der Harry auf der Armlehne seines Stuhls hockte, war ein kegelstumpfförmiges Objekt deutlich zu erkennen, wie ein Kegel mit abgeschnittener Spitze, das sich langsam um ein pulsierendes zentrales Licht drehte, das es zwar beschattete, aber nicht verdeckte; und jedes Mal, wenn das innere Licht pulsierte, gab die Versammlung ein \emph{vroop-vroop-vroop-}Geräusch von sich, das sich seltsam weit entfernt anhörte, gedämpft, als käme es von hinter vier massiven Wänden, obwohl das sich drehende kegelförmige Dingsda nur ein oder zwei Meter entfernt war. \emph{Vroop… vroop… vroop…}

Und dann waren da noch die verschiedenen noch atmenden Körper von Harry Potter, die er in einer ruhigen Ecke versteckt hatte, um ein Chaos aufzuräumen, das in mehr als einer Hinsicht sein eigenes war.

(Nur ein Körper befand sich nicht in einer Kopie des Unsichtbarkeitsumhangs; aber dann bedurfte es nur einer kleinen Konzentrationsanstrengung, damit Harry seine anderen Ichs unter dem Umhang, dessen Herr er war, wahrnehmen konnte - eine Anstrengung, die Harry vorhin sorgfältig unterlassen hatte, um nicht im Voraus zeitliche Informationen zu erhalten, die er selbst bestimmen wollte.) Das Traurige war, dass es zu diesem Zeitpunkt gar nicht mehr so verrückt erschien, seinen eigenen Körper sichtbar in einer Ecke liegen zu haben. Es war einfach… Hogwarts.

"Also gut", sagte Moody und sah dabei ziemlich säuerlich aus. Aus seiner ledernen Rüstung holte der vernarbte Mann einen schwarzen Ordner hervor. "Das ist eine Kopie von dem, was Amelias Leute zusammengestellt haben. Sie weiß mit Sicherheit, dass wir es haben, aber es ist alles inoffiziell, ist das klar? Wie auch immer -"

Und Moody erzählte ihnen, für wen die Abteilung für magische Strafverfolgung 'Quirinus Quirrell' wirklich hielt. Ein scheinbar gewöhnlicher Hogwarts-Schüler (wenn auch talentiert genug, dass er bei der Wahl zum Schulsprecher nur knapp geschlagen worden war), der nach seinem Abschluss Urlaub in Albanien gemacht hatte, verschwand, nach 25 Jahren zurückkehrte und dann in den Zaubererkrieg verwickelt worden war -

"Es war der Mord am Haus Monroe, der ihn bekannt gemacht hat", sagte Moody. "Bis dahin war Voldie nur ein weiterer dunkler Zauberer mit Größenwahn und Bellatrix Black. Aber danach -" Moody schnaubte. "Jeder Narr im Lande strömte herbei, um ihm zu dienen. Man hätte gehofft, dass das Zaubergamot ernst machen würde, sobald sie merkten, dass Voldie bereit war, ihr eigenes heiliges Selbst zu töten. Und genau das haben die Bastarde getan - gehofft, dass ein anderer Bastard ernst machen würde. Keiner der Feiglinge wollte nach vorne treten. Es waren Monroe, Crouch, Bones, und Longbottom. Das war fast jeder im Ministerium, der es wagen würde, ein Wort zu sagen, das Voldie beleidigen könnte."

"So kam es dazu, dass Ihr Haus geadelt wurde, Mr. Potter", sprach die feierliche Stimme von Professor McGonagall. "Es gibt ein uraltes Gesetz, das besagt, dass, wenn jemand ein uraltes Haus vernichtet, derjenige, der dieses Blut rächt, zum Adeligen gemacht wird. Gewiss, das Haus Potter war schon älter als einige Linien, die man Alt nennt. Aber es wurde nach dem Ende des Krieges zum Adeligen Haus von Britannien ernannt, in Anerkennung dessen, dass das Älteste Haus von Monroe gerächt wurde."

"Rausch der Dankbarkeit und so weiter", sagte Mad-Eye Moody säuerlich. "Es hat nicht lange gehalten, aber wenigstens haben James und Lily einen schicken Titel und eine nutzlose Medaille bekommen, die sie mit ins Grab nehmen können. Aber das lässt acht Jahre völligen Horrors aus, nachdem Monroe verschwunden und Regulus Black - er war Monroes private Quelle bei den Todessern, da sind wir uns ziemlich sicher - von Voldie hingerichtet worden war. Es war, als würde ein Damm brechen und das Blut herausströmen und das ganze Land ertränken. Albus, der verdammte Dumbledore, musste selbst in Monroes Fußstapfen treten, und das hat gerade so gereicht, dass wir überlebt haben."

Harry hörte mit einem seltsamen Gefühl der Unwirklichkeit zu. Einiges davon fühlte sich richtig an, stimmte mit der Beobachtung überein - besonders mit der Rede, die Professor Quirrell vor Weihnachten gehalten hatte - und doch… Das war Professor Quirrell, über den sie sprachen.

"Das ist also der, den das Ministerium für Ihren Verteidigungsprofessor hält", beendete Mad-Eye Moody seinen Bericht. "Und was denkst du, mein Sohn?"

"Nun…" sagte Harry langsam.

\emph{Es ist auch möglich, dass sich hinter der Maske eine Maske befindet.}

"Der naheliegende nächste Gedanke ist, dass dieser '\emph{David Monroe}' doch im Krieg gestorben ist und dies nur jemand anderes ist, der sich als David Monroe ausgibt und vorgibt, Quirinus Quirrell zu sein."

"Das ist offensichtlich?!", sagte Professor McGonagall. "Lieber Merlin…"

"Wirklich, Junge?", sagte Mad-Eye Moody, sein blaues Auge drehte sich schnell. "Ich würde sagen, das ist ein bisschen… paranoid."

\emph{Du kennst Professor Quirrell nicht,} sagte Harry nicht.

"Es ist eine leicht zu testende Theorie", sagte Harry laut. "Überprüfe einfach, ob der Verteidigungsprofessor sich an etwas über den Krieg erinnert, das der echte David Monroe gewusst hätte. Obwohl ich vermute, dass er, wenn er die Rolle des David Monroe spielt und vorgibt, jemand anderes zu sein, eine gute Ausrede hat, um so zu tun, als wüsste er nicht, wovon man spricht-"

"Nur ein bisschen paranoid", sagte der vernarbte Mann, und seine Stimme erhob sich. "\textbf{Nicht paranoid genug! STÄNDIGE WACHSAMKEIT!} Denk doch mal nach, Junge - was wäre, wenn der echte David Monroe nie aus Albanien zurück gekommen wäre?"

Es gab eine Pause.

"Ich verstehe …" sagte Harry.

"Natürlich tun Sie das", sagte Professor McGonagall. "Lasst euch nicht stören, bitte. Ich sitze hier nur still und werde verrückt."

"Wenn man in diesem Beruf überlebt, lernt man, dass es drei Arten von dunklen Zauberern gibt", sagte Moody grimmig; sein Zauberstab war auf niemanden gerichtet, er war leicht nach unten geneigt, aber er war in seiner Hand. Er hatte seine Hand nicht mehr verlassen, seit er den Raum betreten hatte. "Es gibt Dunkle Zauberer, die einen Namen haben. Es gibt dunkle Zauberer, die zwei Namen haben. Und es gibt dunkle Zauberer, die ihren Namen wechseln wie wir unsere Kleidung. Ich habe gesehen, wie "Monroe" durch 3 Todesser hindurchging, wie ein heißes Messer durch Butter. Es gibt nicht viele Zauberer, die mit 45 Jahren so gut sind. Dumbledore vielleicht, aber nicht viele andere."

"Vielleicht ist das wahr", sagte der Meister der Zaubertränke von dort, wo er lauerte. "Aber was soll's, Mad-Eye? Was auch immer seine Identität ist, Monroe war sicherlich ein Feind des Dunklen Lords. Ich habe Todesser seinen Namen verfluchen hören, selbst nachdem sie ihn für tot hielten. Sie haben ihn gefürchtet."

"Soweit es Verteidigungsprofessoren betrifft", sagte Professor McGonagall hochnäsig, "nehme ich ihn dankend an."

Moody drehte sich um, um sie anzustarren.

"Wo zum Teufel war 'Monroe' all die Jahre, die er weg war? Vielleicht dachte er, er könnte sich in Britannien einen Namen machen, indem er sich Voldie entgegenstellt, und verschwand, als er herausfand, dass er falsch lag. Warum ist er dann jetzt zurückgekommen, hah? Was ist sein neuer Plan?"

"Er, ah …" Harry wagte es zaghaft. "Er sagt, er wollte schon immer ein großer Verteidigungsprofessor werden, weil die besten Kampfzauberer in Hogwarts unterrichtet haben. Und irgendwie ist er auch ein unglaublich guter Verteidigungsprofessor, eigentlich… Ich meine, wenn er nur eine Verkleidung aufrechterhalten wollte, könnte er mit viel schlampigerer Arbeit durchkommen…"

Professor McGonagall nickte entschlossen. "Naiv", sagte Moody mit fester Stimme. "Ich nehme an, Sie alle haben sich noch nicht gefragt, ob Ihr Verteidigungsprofessor das ganze Haus Monroe auslöschen lassen \emph{wollte}?"

"Was?", rief Professor McGonagall.

"Unser mysteriöser Zauberer erfährt von einem vermissten Kind aus einem der ältesten Häuser Britanniens", sagte Moody. "Er tritt in die Fußstapfen von 'David Monroe', hält sich aber von der echten Familie Monroe fern. Doch irgendwann bemerkt das Haus, dass etwas nicht stimmt. Also stiftet dieser Hochstapler Voldie irgendwie dazu an, ihnen allen eins auszuwischen - vielleicht hat er ein Passwort verraten, das sie ihm für ihre Mündel gegeben hatten - und dann war er ein Lord des Zaubergamot!"

In Harrys Kopf schien ein Kampf zwischen Hufflepuff Eins, der dem Verteidigungsprofessor von vornherein nicht vertraut hatte, und

Hufflepuff Zwei, der Harrys Freund Professor Quirrell gegenüber viel zu loyal war, um so etwas zu glauben, nur weil Moody es sagte.

\emph{Es ist aber irgendwie offensichtlich,} sagte Slytherin. \emph{Ich meine, glaubst du wirklich, dass unter natürlichen Umständen jemand als letzter Erbe eines sehr alten Adels-Hauses enden würde UND Lord Voldemort seine Familie getötet hat UND er seinen Kampfsport-Sensei rächen muss? Wenn überhaupt, dann würde ich sagen, er hat es zu sehr übertrieben, als er seine neue Identität als idealer literarischer Held aufbaute. So etwas gibt es im wirklichen Leben nicht.}

\emph{Das von einem Waisenkind, das in Unkenntnis seiner Herkunft aufgewachsen ist,} kommentierte Harrys innerer Kritiker. \emph{Mit einer Prophezeiung über ihn.}

\emph{Weißt du, ich glaube nicht, dass wir jemals eine Geschichte über zwei gleichberechtigte Helden gelesen haben, die darum wetteifern, wer klischeehaft genug ist, den Bösewicht zur Strecke zu bringen} -

\emph{Ja}, erwiderte der zentrale Harry über das entfernte \emph{Vroop}-Geräusch im Hintergrund, \emph{es ist ein sehr trauriges Leben, das wir führen und DU HILFST NICHT.}

\emph{An diesem Punkt gibt es nur eines zu tun,} sagte der Ravenclaw. \emph{Und wir alle wissen, was es ist, also warum streiten?}

\emph{Aber, erwiderte Harry, wie können wir experimentell testen, ob Professor Quirrell der ursprüngliche David Monroe ist oder nicht? Ich meine, welche Art von Beobachtungsgegenstand verhält sich anders, je nachdem, ob er der echte David Monroe oder ein Hochstapler ist?}

"Was soll ich dagegen tun, Mad-Eye?" Professor McGonagall war fordernd. "Ich kann nicht -"

"Doch", sagte der vernarbte Mann und starrte sie wütend an. "Feuer einfach den verdammten Verteidigungsprofessor."

"Das sagst du jedes Jahr", sagte Professor McGonagall.

"Ja, und ich habe immer recht!"

"Ständige Wachsamkeit hin oder her, Alastor, die Schüler müssen unterrichtet werden!"

Moody schnaubte. "Pfah! Ich schwöre, der Fluch wird von Jahr zu Jahr schlimmer, weil ihr euch immer mehr sträubt, sie gehen zu lassen. Euer kostbarer Professor Quirrell müsste schon Grindelwald in Verkleidung sein, um sich selbst zu entlassen!"

"Ist er das?" Harry konnte sich die Frage nicht verkneifen. "Ich meine, könnte er tatsächlich -"

"Ich kontrolliere Grindelwalds Zelle alle zwei Monate", sagte Moody. "Er war im März dort."

"Könnte die Person in der Zelle ein Doppelgänger sein?"

"Ich mache einen Bluttest, um seine Identität festzustellen, mein Sohn."

"Wo bewahrst du das Blut auf, das du als Referenz verwendest?"

"An einem sicheren Ort."

So etwas wie ein Lächeln zog sich über die vernarbten Lippen.

"Hast du daran gedacht, nach deinem Abschluss ins Aurorenbüro zu gehen?"

"Alastor", sagte Professor McGonagall zögernd. "Der Verteidigungsprofessor hat eine … gesundheitliche Beeinträchtigung. Ich nehme an, du wirst es an sich als verdächtig bezeichnen - aber es ist keineswegs sicher, dass es irgendeine Krankheit sein wird, die uns daran hindert, seine Anstellung zu verlängern."

"Ja, seine kleinen Nickerchen", sagte Moody düster. "Amelia glaubt, dass er einem hochgradigen Fluch in den Weg getreten ist. Für mich klingt das eher nach einem schiefgelaufenen dunklen Ritual!"

"Dafür hast du keine Beweise!" sagte Professor McGonagall.

"Der Mann könnte genauso gut ein Schild mit der Aufschrift '\emph{Dunkler Zauberer}' in leuchtend grünen Buchstaben über dem Kopf tragen."

"Ah …" sagte Harry. Es schien kein besonders guter Zeitpunkt zu sein, um zu fragen, was Mr. Moody von dem Standpunkt \emph{'nicht alle Opferrituale sind böse'} hielt. "Entschuldigung, aber du sagtest vorhin, dass Professor Quirrell - ich meine den alten David Monroe - ich meine den Monroe aus den Siebzigern - jedenfalls sagtest du, dass diese Person den Tötungsfluch benutzt hat. Was impliziert das? Muss jemand ein dunkler Zauberer sein, um ihn zu benutzen?"

Moody schüttelte den Kopf. "Ich habe ihn selbst schon benutzt. Alles, was man dazu braucht, ist Macht und eine \emph{bestimmte Stimmung}."

Die Grimassen auf den Lippen zeigten Zähne.

"Das erste Mal, dass ich ihn gewirkt habe, war gegen einen Zauberer namens Gerald Grice, und du kannst mich fragen, was er getan hat, nachdem du deinen Abschluss in Hogwarts gemacht hast."

"Aber warum ist es dann unverzeihlich?" sagte Harry. "Ich meine, einen Schneidezauber kann auch jemanden töten. Warum ist es also besser, einen Reducto zu benutzen statt Avada Kedav-"

"Halt die Klappe!" Moody sagte scharf. "Jemand könnte es falsch auffassen, wenn du diese Beschwörung sagst. Du siehst zu jung aus, um ihn zu sprechen, aber es gibt so etwas wie Vielsaft. Und um deine Frage zu beantworten, Junge, es gibt zwei Gründe, warum dieser Zauberspruch im schwärzesten Buch steht. Der erste ist, dass der Tötungsfluch direkt auf die Seele zielt, und zwar so lange, bis er einen trifft. Direkt durch Schilde. Direkt durch Wände. Es gibt einen Grund, warum selbst Auroren, die gegen Todesser kämpfen, ihn vor dem Monroe-Gesetz nicht benutzen durften."

"Ah", sagte Harry. "Das scheint ein hervorragender Grund für ein Verbot zu sein -"

"Ich bin noch nicht fertig, Sohn. Der zweite Grund ist, dass man für den Tötungsfluch nicht nur ein mächtiges Stück Magie braucht. Man muss es ernst meinen. Man muss den Tod von jemandem \emph{wollen}, und zwar nicht für das Allgemeinwohl. Das Töten von Grice hat weder Blair Roche noch Nathan Rehfuss oder David Capito zurückgebracht. Es war nicht für die Gerechtigkeit, oder um ihn davon abzuhalten, es wieder zu tun. \emph{Ich wollte ihn tot sehen}. Verstehst du jetzt, Junge? Du musst kein dunkler Zauberer sein, um den Spruch zu benutzen. Aber du kannst auch nicht Albus Dumbledore sein. Und wenn man dich wegen Mordes damit verhaftet, gibt es keine Möglichkeit der Verteidigung."

"Ich… verstehe", murmelte der Junge-der-lebte.

\emph{Man kann die Person nicht als instrumentellen Wert auf dem Weg zu irgendeiner positiven zukünftigen Konsequenz tot sehen wollen, man kann sie nicht töten, wenn man glaubt, dass es ein notwendiges Übel ist, man muss sie tatsächlich um des Totseins willen tot sehen wollen, als Endwert in seiner Nutzenfunktion.}

"Eine magisch verkörperte Vorliebe für den Tod gegenüber dem Leben, die in der Ebene der reinen Lebenskraft zuschlägt … das klingt nach einem schwer zu blockierenden Zauber."

"Nicht schwierig", schnauzte Moody. "Unmöglich."

Harry nickte ernsthaft.

"Aber David Monroe - oder wer auch immer - hat den Tötungsfluch gegen ein paar Todesser eingesetzt, noch bevor sie seine Familie auslöschten. Bedeutet das, dass er sie bereits hassen musste? So wie die Geschichte mit den Kampfsportarten wahrscheinlich wahr war?"

Moody schüttelte leicht den Kopf.

"Eine der dunklen Wahrheiten des Tötungsfluchs, mein Sohn, ist, dass man, wenn man ihn einmal ausgesprochen hat, nicht viel Hass braucht, um ihn wieder auszusprechen."

"Er schädigt den Geist?"

Wieder schüttelte Moody den Kopf. "Nein. Es ist das Töten, das das tut. Mord zerreißt die Seele - aber das ist bei einem Schneidenden Fluch nicht anders. Der Tötungsfluch zerbricht nicht die Seele. Es braucht nur eine zerbrochene Seele, um ihn zu wirken."

Wenn es einen traurigen Ausdruck auf dem vernarbten Gesicht gab, konnte man ihn nicht lesen.

"Aber das sagt uns nicht viel über Monroe. Diejenigen wie Dumbledore, die ihr ganzes Leben lang nicht in der Lage sind, den Fluch zu wirken, weil sie niemals brechen, egal was passiert - das sind die Seltenen, sehr selten. Es braucht nur einen kleinen Riss."

Da war ein seltsames, schweres Gefühl in Harrys Brust. Er hatte sich gefragt, was genau es bedeutet hatte, dass Lily Potter mit ihrem letzten Atemzug versucht hatte, den Tötungsfluch auf Lord Voldemort zu wirken. Aber sicher war es verzeihlich, es war recht und billig für eine Mutter, den dunklen Zauberer zu hassen, der im Begriff war, ihr Baby zu töten, und sie dafür zu verhöhnen, dass sie ihn nicht aufhalten konnte. Es stimmte etwas nicht mit einem als Elternteil, wenn man in dieser Situation kein Avada Kedavra sprechen konnte. Und kein anderer Zauber hätte die Schilde des Dunklen Lords durchdringen können; man musste zumindest versuchen, den Dunklen Lord so sehr zu hassen, dass man ihn um des Tötens willen tot sehen wollte, wenn das der einzige Weg war, sein Baby zu retten. \emph{Es braucht nur einen kleinen Riss.}..

"Genug", sagte Professor McGonagall. "Was sollen wir deiner Meinung nach tun?"

Moodys Lächeln verzog sich. "Werdet den Verteidigungsprofessor los und schaut mal, ob sich all eure Probleme auf mysteriöse Weise auflösen. Ich wette eine Galleone, dass sie das tun."

Professor McGonagall sah aus, als würde sie Schmerzen haben.

"Alastor - aber - wirst du die Klassen unterrichten, wenn -"

"Ha!", sagte Moody. "Wenn ich diese Frage jemals mit Ja beantworten sollte, überprüft mich auf Vielsaft, denn ich bin es nicht."

"Ich werde es experimentell testen", sagte Harry.

Und dann, als alle ihn ansahen,

"Ich werde Professor Quirrell eine Frage stellen, die der echte David Monroe wissen würde - zum Beispiel, wer noch in der Slytherin-Klasse von 1945 war, oder so etwas in der Art - hoffentlich ohne es offensichtlich zu machen. Es wird kein definitiver Beweis sein, er könnte die Rolle studiert haben, aber es wäre ein Hinweis. Dennoch, Mr. Moody, selbst wenn Professor Quirrell nicht der ursprüngliche Monroe ist, bin ich mir nicht sicher, ob es eine freie Aktion ist, ihn loszuwerden. Er hat mir zweimal das Leben gerettet -"

"Was?", verlangte Moody. "Wann? Wie?"

"Einmal, als er einen Haufen Hexen niederschlug, die mich zu Boden zogen, einmal, als er herausfand, dass der Dementor mich durch meinen Zauberstab aussaugte. Und wenn Professor Quirrell nicht derjenige war, der Draco Malfoy in die Falle gelockt hat, dann hat er Draco Malfoy das Leben gerettet, und es wäre viel schlimmer, wenn er es nicht getan hätte. Wenn der Verteidigungsprofessor nicht hinter all dem steckt - er ist niemand, den wir uns leisten können, einfach loszuwerden."

Professor McGonagall nickte fest.

\textbf{Hypothese: Severus Snape} (8. April 1992, 21:03 Uhr)

Harry und Professor McGonagall standen nun auf der sich langsam drehenden Treppe, ohne abzusteigen; oder zumindest stand ein Harry auf dieser Treppe - seine anderen drei Ichs waren im Büro des Schulleiters zurückgeblieben.

"Darf ich Ihnen eine private Frage stellen?" sagte Harry, als er dachte, dass sie weit genug weg waren, um nicht gehört zu werden. "Und zwar etwas Privates vom Schulleiter."

"Ja", sagte Professor McGonagall, nicht ganz seufzend. "Obwohl ich hoffe, Ihnen ist klar, dass ich nichts tun kann, was mit meinen Pflichten gegenüber -"

"Ja", sagte Harry, "genau das ist es, was ich Sie fragen muss. Als Lucius Malfoy vor dem Zaubergamot sagte, dass Hermine nicht zum Haus Potter gehöre und dass er das Geld nicht annehmen würde, haben Sie Hermine gesagt, wie sie den Eid schwören soll. Ich möchte wissen, wenn so etwas noch einmal vorkommt, ob Ihre erste Pflicht der Hogwarts-Schülerin Hermine Granger gilt oder dem Oberhaupt des Ordens des Phönix, Albus Dumbledore."

Professor McGonagall sah aus, als hätte ihr jemand vor ein paar Minuten mit einer gusseisernen Bratpfanne ins Gesicht geschlagen, und jetzt hatte man ihr gesagt, dass es gleich wieder jemand tun würde, und sie solle nicht zusammenzucken.

Harry zuckte selbst ein wenig zusammen.

\emph{Irgendwann musste er lernen, die Dinge nicht so zu formulieren, dass sie Leute nicht so hart wie möglich treffen.}

Die Wände drehten sich um sie herum, hinter ihnen, und irgendwie kamen sie herunter.

"Oh, Mr. Potter", sagte Professor McGonagall mit einem tiefen Ausatmen. "Ich… wünschte, Sie würden mir nicht solche Fragen stellen… oh, Harry, da habe ich nicht nachgedacht, ganz und gar nicht. Ich habe nur eine Chance gesehen, Miss Granger zu helfen und… ich wurde ja schließlich nach Gryffindor sortiert."

"Jetzt haben Sie die Chance zu denken", sagte Harry. \emph{Es kam alles falsch rüber, aber er musste es trotzdem sagen, denn -} "Ich verlange ja nicht, dass Sie mir gegenüber loyal sind. Aber wenn Sie wissen - wenn Sie sich sicher sind - was Sie tun werden, wenn es ein zweites Mal auf einen unschuldigen Hogwarts-Schüler gegen den Orden des Phönix hinausläuft…"

Doch Professor McGonagall schüttelte den Kopf. "Ich bin mir nicht sicher", flüsterte die Professorin für Verwandlung. "Ich weiß nicht, ob es damals die richtige Entscheidung war. Es tut mir leid. Ich kann solche schrecklichen Dinge nicht entscheiden!"

"Aber Sie werden etwas tun, wenn es wieder passiert", sagte Harry. "Unentschlossenheit ist auch eine Entscheidung. Sie können sich nicht vorstellen, eine sofortige Entscheidung treffen zu müssen?"

"Nein", sagte Professor McGonagall und klang ein wenig fester; und Harry wurde klar, dass er versehentlich einen Ausweg angeboten hatte. Die nächsten Worte des Professors bestätigten Harrys Befürchtungen. "Eine so furchtbare Entscheidung wie diese, Mr. Potter - ich denke, ich sollte sie erst treffen, wenn ich es muss."

Harry stieß einen inneren Seufzer aus. Er nahm an, dass er kein Recht hatte, zu erwarten, dass Professor McGonagall etwas anderes sagen würde. In einem moralischen Dilemma, in dem man so oder so etwas verlor, würde sich die Entscheidung so oder so schlecht anfühlen, also konnte man sich vorübergehend ein wenig seelischen Schmerz ersparen, indem man sich weigerte, sich zu entscheiden. Um den Preis, dass man nichts im Voraus planen konnte, und um den Preis, dass man einen großen Hang zur Untätigkeit oder zum Abwarten hatte, bis es zu spät war… aber man konnte nicht erwarten, dass eine Hexe das alles wusste.

"In Ordnung", sagte Harry. \emph{Obwohl es gar nicht richtig war, nicht wirklich.} Dumbledore würde vielleicht wollen, dass die Schulden erlassen werden, Professor Quirrell würde auch wollen, dass Harry aus den Schulden herauskommt. Und wenn der Verteidigungsprofessor David Monroe war oder überzeugend den Anschein erwecken konnte, David Monroe zu sein, dann hatte Lord Voldemort das Haus Monroe technisch gesehen nicht ausgelöscht. In diesem Fall könnte jemand eine Zaubergamot-Resolution verabschieden, die den Adelsstatus des Hauses Potter aufhebt, der für die Rache am ältesten Haus der Monroes verliehen wurde. In diesem Fall könnte Hermines Gelübde, einem Adelshaus zu dienen, null und nichtig sein. Oder vielleicht auch nicht. Harry wusste nichts über die Gesetzmäßigkeiten, vor allem nicht, ob Haus Potter das Geld zurückbekam, wenn es jemand schaffte, Hermine nach Askaban zu schicken. Nur weil man etwas verloren hatte, hieß das nicht unbedingt, dass man das Geld zurückbekam, rechtlich gesehen. Harry war sich nicht sicher und traute sich nicht, einen magischen Anwalt zu fragen…

\emph{… es wäre schön gewesen, wenigstens einem Erwachsenen vertrauen zu können, der sich auf Hermines Seite stellte und nicht auf die von Dumbledore, wenn so ein Problem aufzutauchen drohte.}

Die Treppe, auf der sie sich befanden, hörte auf, sich zu drehen, und sie standen vor den Rücken der großen steinernen Wasserspeier, die zur Seite polterten und den Gang freigaben. Harry trat hinaus. Eine Hand griff nach Harrys Schulter.

"Mr. Potter", sagte Professor McGonagall mit leiser Stimme, "warum haben Sie mir gesagt, ich solle auf Professor Snape aufpassen?"

Harry drehte sich wieder um.

"Sie haben mir gesagt, ich solle Wache halten und nachsehen, ob er sich verändert hat", fuhr Professor McGonagall fort, ihr Tonfall war eindringlich. "Warum haben Sie das gesagt, Mr. Potter?"

Es dauerte an dieser Stelle einen Moment, bis Harry sich daran erinnerte, warum er das gesagt hatte. \emph{Harry und Neville hatten Lesath Lestrange vor Schlägern gerettet, und dann hatte Harry Severus im Flur konfrontiert und war, zumindest nach den eigenen Worten des Zaubertränkemeisters, "fast gestorben"} -

"Ich habe etwas gelernt, das mich beunruhigt hat", sagte Harry nach einem Moment. "Von jemandem, der mir das Versprechen abverlangte, es niemandem sonst zu erzählen."

Severus hatte Harry schwören lassen, dass ihre Gespräche mit niemandem geteilt werden würden, und Harry war immer noch daran gebunden.

"Mr. Potter -", begann Professor McGonagall, und atmete dann aus, wobei der Anflug von Schärfe so schnell verschwand, wie er gekommen war. "Schon gut. Wenn Sie es nicht sagen können, können Sie es nicht sagen."

"Warum fragen Sie?" fragte Harry.

Professor McGonagall schien zu zögern -

"Na gut, lassen Sie mich genauer sein", sagte Harry. Nachdem Professor Quirrell es ihm einige Male vorgemacht hatte, hatte Harry langsam den Dreh raus.

"Welche Veränderung haben Sie bereits bei Professor Snape beobachtet, von der Sie mir erzählen wollen?"

"Harry -", sagte die Professorin für Verwandlung und schloss dann den Mund.

"Offensichtlich weiß ich etwas, was Sie nicht wissen", sagte Harry hilfsbereit. "Sehen Sie, das ist der Grund, warum wir die Entscheidung über unsere schrecklichen moralischen Dilemmas nicht immer aufschieben können."

Professor McGonagall schloss die Augen, holte tief Luft, kniff sich in den Nasenrücken und drückte ihn mehrmals zusammen.

"Also gut", sagte sie. "Es ist eine subtile Sache … aber beunruhigend. Wie soll ich das ausdrücken… Mr. Potter, haben Sie viele Bücher gelesen, die für kleine Kinder nicht geeignet sind?"

"Ich habe sie alle gelesen."

"Natürlich haben Sie das. Na ja… Ich verstehe es selbst nicht ganz, aber so lange Severus in dieser Schule beschäftigt ist und in diesem schrecklichen fleckigen Umhang herumstolziert, gibt es eine bestimmte Sorte Mädchen, die ihn mit sehnsüchtigen Augen anstarrt -"

"Sie sagen das, als wäre es etwas Schlechtes." sagte Harry. "Ich meine, wenn ich eines aus diesen Büchern verstanden habe, dann, dass man die Vorlieben anderer nicht in Frage stellen soll."

Professor McGonagall warf Harry einen sehr seltsamen Blick zu.

"Ich meine", sagte Harry wieder, "nach dem, was ich gelesen habe, besteht eine etwa zehnprozentige Chance, dass ich Professor Snape attraktiv finde, wenn ich etwas älter bin, und das Wichtigste ist, dass ich einfach akzeptiere, was immer ich -"

"\textbf{\emph{Auf jeden Fall, Mr. Potter}}\emph{,} war Severus die Blicke der jungen Mädchen immer völlig gleichgültig. Aber jetzt -" Professor McGonagall schien etwas zu bemerken und sagte hastig, die Hände abwehrend erhoben: "Verstehen Sie mich bitte nicht falsch, Professor Snape hat ganz sicher keine jungen Hexen ausgenutzt! Ganz und gar nicht! Er hat nie auch nur eine angelächelt, nicht, dass ich es je gehört hätte. Er hat den jungen Mädchen gesagt, sie sollen aufhören, ihn anzustarren. Und wenn sie ihn trotzdem anstarren, schaut er weg. Das habe ich mit meinen eigenen Augen gesehen."

"Äh …" sagte Harry. "Tut mir leid, aber nur weil ich diese Bücher gelesen habe, heißt das nicht, dass ich sie verstanden habe. Was soll das alles bedeuten?"

"Dass er es bemerkt", sagte Professor McGonagall mit leiser Stimme. "Es ist eine subtile Sache, aber jetzt, wo ich es gesehen habe, bin ich mir sicher. Und das bedeutet… Ich habe große Angst… dass das Band, das Severus an Albus' Sache gebunden hat… schwächer geworden oder sogar gebrochen sein könnte."

\emph{2 + 2 = …}

"Snape und Dumbledore?!"

Dann hörte Harry die Worte, die gerade aus seinem Mund gekommen waren, und fügte hastig hinzu: "Nicht, dass daran etwas falsch wäre -"

"Nein!", sagte Professor McGonagall. "Oh, um Himmels willen - ich kann es Ihnen nicht erklären, Mr. Potter!"

\emph{Der andere Schuh war endlich gefallen.}

\emph{Er war immer noch in meine Mutter verliebt? Das schien irgendwo zwischen schön traurig und erbärmlich zu sein, für ungefähr fünf Sekunden, bevor der dritte Schuh fiel.}

\emph{Das war natürlich, bevor ich ihm meinen hilfreichen Beziehungsratschläge gab.}

"Ich verstehe", sagte Harry nach ein paar Augenblicken vorsichtig. Es gab Zeiten, in denen ein "\emph{Ups}" nicht ganz ausreichte, um es zu sagen.

"Sie haben recht, das ist kein gutes Zeichen."

Professor McGonagall legte beide Hände über ihr Gesicht. "Was auch immer Sie gerade denken", sagte sie mit leicht gedämpfter Stimme, "was, das versichere ich Ihnen, auch falsch ist, ich will nichts davon hören, niemals."

"Also …" sagte Harry. "Wenn, wie Sie sagten, das Band, das Professor Snape mit dem Schulleiter verband, gebrochen ist … was würde er dann tun?"

Es herrschte eine lange Stille.

\emph{Was würde er dann tun?}

Minerva ließ ihre Hände sinken und blickte auf das aufgewühlte Gesicht des Jungen-der-lebte hinunter. Eine einfache Frage hätte sie nicht so sehr bestürzen dürfen. Sie kannte Severus seit Jahren; die beiden waren auf eine seltsame Weise durch die Prophezeiung verbunden, die sie beide gehört hatten. Obwohl Minerva nach dem, was sie über die Regeln der Prophezeiung wusste, vermutete, dass sie sie selbst nur mitgehört hatte. Es waren Severus' Taten, die die Erfüllung der Prophezeiung herbeigeführt hatten. Und die Schuldgefühle, der Herzschmerz, die aus dieser Entscheidung entstanden waren, hatten den Zaubertränkemeister jahrelang gequält. Sie konnte sich nicht vorstellen, wer Severus ohne sie sein würde. Ihr Verstand wurde leer bei dem Versuch, sich vorzustellen; ihre Gedanken ein leeres Pergament.

\emph{Sicherlich war Severus nicht mehr der Mann, der er einmal gewesen war, dieser wütende und schrecklich törichte junge Mann, der die Prophezeiung vor Voldemort gebracht hatte, im Austausch dafür, dass er bei den Todessern aufgenommen wurde.}

Sie kannte ihn seit Jahren, und sicherlich war Severus nicht mehr dieser Mann…

\emph{Kannte sie ihn überhaupt wirklich? Hatte irgendjemand jemals den echten Severus Snape gesehen?}

"Ich weiß es nicht", sagte Professor McGonagall schließlich. "Ich weiß es wirklich überhaupt nicht. Ich kann es mir nicht einmal vorstellen. Wissen Sie etwas darüber, Mr. Potter?"

"Äh …" sagte Harry. "Ich denke, ich kann sagen, dass meine eigenen Beweise in dieselbe Richtung weisen wie Ihre. Ich meine, es erhöht die Wahrscheinlichkeit, dass Professor Snape nicht mehr in meine Mutter verliebt ist."

Professor McGonagall schloss die Augen. "Ich gebe auf."

"Ich wüsste allerdings nicht, was er sonst noch verbrochen haben sollte", fügte Harry hinzu. "Ich nehme an, der Schulleiter hat Ihnen die Erlaubnis erteilt, mich danach zu fragen?"

Professor McGonagall wandte den Blick von ihm ab und starrte an die Wand.

"Bitte nicht, Harry."

"In Ordnung", sagte Harry und drehte sich um und eilte hinaus in die Gänge, wobei er hörte, wie Professor McGonagall langsamer hinterherging, und das polternde Geräusch der Wasserspeier, die sich an ihren Platz bewegten.

Es war am übernächsten Morgen, während des Unterrichts in Zaubertränke, als Harrys Zaubertrank der Kälteresistenz mit grünem Schaum und leicht ekelerregendem Geruch überkochte und Professor Snape, der mehr resigniert als angewidert aussah, Harry aufforderte, nach dem Unterricht zu bleiben. Harry hatte seinen eigenen Verdacht, was diese Angelegenheit betraf, und sobald der Unterricht zu Ende war - Hermine war, wie immer in den letzten Tagen, die erste, die aus der Tür flüchtete - schwang die Tür zu und schloss sich hinter den abreisenden Schülern.

"Ich entschuldige mich, dass ich deinen Trank ruiniert habe, Mr. Potter", sagte Severus Snape leise. Auf seinem Gesicht lag der seltsam traurige Ausdruck, den Harry nur einmal zuvor gesehen hatte, vor einiger Zeit auf einem Flur. "Es wird sich nicht in deinen Noten niederschlagen. Bitte, setz dich."

Harry setzte sich wieder an seinen Schreibtisch und vertrieb sich die Zeit, indem er noch ein bisschen an dem grünen Fleck auf der Holzoberfläche schrubbte, während der Meister der Zaubertränke ein paar Geheimhaltungszauber beschwor. Als der Tränkemeister fertig war, sprach er wieder.

"Ich… weiß nicht, wie ich dieses Thema ansprechen soll, Mr. Potter, also werde ich es einfach sagen… vor dem Dementor haben Sie Ihre Erinnerung an die Nacht, in der Ihre Eltern starben, wiedergefunden?"

Harry nickte stumm.

"Wenn… ich weiß, es ist sicher keine angenehme Erinnerung, aber… wenn Sie mir sagen könnten, was passiert ist…?"

"Warum?" sagte Harry.

Seine Stimme war feierlich, definitiv nicht mit dem flehenden Blick, den Harry nie von dieser Person erwartet hatte.

"Ich würde denken, dass es auch für Sie nicht angenehm wäre, das zu hören, Professor -"

Die Stimme des Zaubertränkemeisters war fast ein Flüstern.

"Ich habe es mir in den letzten zehn Jahren jede Nacht vorgestellt."

\emph{Weißt du,} sagte Harrys Slytherin-Seite, \emph{es ist vielleicht keine so gute Idee, ihm einen Schlussstrich ziehen zu lassen, wenn seine auf Schuldgefühlen beruhende Loyalität} \emph{bereits ins Wanken gerät -}

\textbf{\emph{Klappe. Überstimm}t.}

Es war nicht etwas, das Harry tatsächlich dazu bringen konnte, zu leugnen. Er nahm einen Vorschlag von seiner Slytherin-Seite an, und das war's.

"Wirst du mir genau erzählen, wie du von der Prophezeiung erfahren hast?" sagte Harry. "Es tut mir leid, dass ich das zu einem Handel mache, ich werde es dir nachher erzählen, nur, es könnte wirklich wichtig sein -"

"Da gibt es wenig zu sagen. Ich war gekommen, um von der stellvertretenden Schulleiterin für die Stelle des Zaubertränkemeisters interviewt zu werden, und so wartete ich vor dem Raum des Honigtopf, als die Bewerberin vor mir, Sybill Trelawney, kam, um sich um die Stelle des Professors für Wahrsagerei zu bewerben. Sobald Trelawney mit ihren Worten fertig war, floh ich und verließ meine Chance auf die Meisterschaft in Hogwarts und ging zum Dunklen Lord."

Das Gesicht des Zaubertränkemeisters war gezeichnet und angespannt.

"Ich habe nicht einmal innegehalten, um darüber nachzudenken, wie das Rätsel zu mir gekommen sein könnte, bevor ich es an einen anderen verkauft habe."

"Ein Vorstellungsgespräch?" sagte Harry. "Wo Sie und Professor Trelawney sich zufällig beide beworben haben und Professor McGonagall das Vorstellungsgespräch geführt hat? Das scheint … ein ziemlich großer Zufall zu sein …"

"Seher sind die Spielfiguren der Zeit, Mr. Potter. Der Zufall ist unter ihnen, und sie stehen darüber. Ich war derjenige, der die Prophezeiung hören und ihr Narr werden sollte. Minervas Anwesenheit hat keinen endgültigen Unterschied gemacht, wie es dazu kam. Es gab keinen Erinnerungszauber, wie du vermutetest, ich weiß nicht, warum du das dachtest, aber es gab keinen Erinnerungszauber, es konnte keinen Erinnerungszauber geben. Die Stimme eines Sehers hat eine Qualität, ein Rätsel, das selbst die Legilimenz nicht teilen kann, wie könnte das in eine falsche Erinnerung eingepflanzt werden? Glaubst du, der Dunkle Lord würde meinen bloßen Worten Glauben schenken? Der Dunkle Lord ergriff meinen Geist und sah die Mystifikation darin, auch wenn er das Geheimnis nicht erfassen konnte, und so wusste er, dass die Prophezeiung wahr war. Der Dunkle Lord hätte mich damals töten können, nachdem er sich genommen hatte, was er wollte - ich war in der Tat ein Narr, zu ihm zu gehen -, aber er sah etwas in mir, das ich nicht kenne, und nahm mich zu den Todessern, wenn auch eher zu seinen Bedingungen als zu meinen. So habe ich es herbeigeführt, habe alles herbeigeführt, von Anfang bis Ende, immer mein eigenes Tun."

Severus' Stimme war ziemlich heiser geworden, und sein Gesicht war von nacktem Schmerz erfüllt.

"Jetzt sag mir bitte, wie ist Lily gestorben?"

Harry schluckte zweimal und begann mit seiner Erzählung.

"James Potter rief Lily zu, sie solle mit mir weglaufen, er würde Du-Weißt-Schon-Wen aufhalten. Du-weißt-schon-wer sagte -"

Harry hielt inne, ein Schauer lief ihm über die Haut, seine Muskeln spannten sich an, als ob er sich auf einen Anfall vorbereiten wollte.Die Erinnerung kehrte stark zurück, jetzt, begleitet von Kälte und Dunkelheit in Verbindung.

"Er benutzte… den Tötungsfluch… und dann kam er irgendwie die Treppe hinauf, ich glaube, er muss geflogen sein, ich erinnere mich nicht an irgendwelche Schritte auf der Treppe oder so etwas… und dann sagte meine Mutter: '\emph{Nein, nicht Harry, bitte nicht Harry!}' oder so etwas in der Art. Und der Dunkle Lord - seine Stimme war so hoch, wie Wasser, das aus einem Teekessel pfeift, nur kalt - der Dunkle Lord sagte" -

\textbf{\emph{Geh zur Seite, Frau! Für dich bin ich nicht gekommen, nur für den Jungen.}}

Die Worte waren sehr deutlich in Harrys Erinnerung. "- sagte zu meiner Mutter, sie solle ihm aus dem Weg gehen, er sei nur wegen mir da, und meine Mutter flehte ihn an, Erbarmen zu haben, und der Dunkle Lord sagte -"

\textbf{\emph{Ich gebe dir diese seltene Chance zu fliehen.}}

"- dass er großzügig sei und ihr eine Chance gäbe, zu fliehen, aber er würde sich nicht die Mühe machen, sie zu bekämpfen, und selbst wenn sie sterben würde, könnte sie mich nicht retten -" Harrys Stimme war unsicher, "- und deshalb sollte sie ihm aus dem Weg gehen. Und das war, als meine Mutter den Dunklen Lord anflehte, ihr Leben anstelle meines zu nehmen - und der Dunkle Lord - der Dunkle Lord sagte zu ihr - und seine Stimme war diesmal tiefer, als ob er eine Pose einnehmen würde -"

\textbf{\emph{Nun gut, ich akzeptiere den Handel.}}

"- sagte er, dass er ihr Angebot akzeptiere und dass sie ihren Zauberstab fallen lassen solle, damit er sie töten könne. Und dann hat der Dunkle Lord gewartet, einfach gewartet. Ich, ich weiß nicht, was Lily Potter sich dabei gedacht hat, es hatte überhaupt keinen Sinn gemacht, was sie sagte, es war ja nicht so, als würde der Dunkle Lord sie töten und dann einfach gehen, wenn er wegen mir da war. Lily Potter hat nichts gesagt, und dann hat der Dunkle Lord angefangen, sie auszulachen und es war schrecklich und - und sie hat schließlich das Einzige versucht, was übrig blieb, als mich aufzugeben oder einfach aufzugeben und zu sterben. Ich weiß nicht, ob sie es überhaupt gekonnt hätte, ob der Zauber bei ihr gewirkt hätte, aber wenn man darüber nachdenkt, musste sie es versuchen. Das letzte, was meine Mutter sagte, war 'Avada Ke-', aber der Dunkle Lord begann seinen eigenen Fluch, sobald sie 'Av' sagte, und er sagte es in weniger als einer halben Sekunde und es gab einen Blitz aus grünem Licht und dann - und dann - und dann -"

"Das ist genug."

Langsam, wie ein Körper, der an die Wasseroberfläche schwimmt, kehrte Harry von dort zurück, wo er gewesen war.

"Das reicht", sagte der Zaubertränkemeister heiser. "Sie starb… Lily ist also ohne Schmerzen gestorben? Der Dunkle Lord… hat ihr nichts angetan, bevor sie starb?"

\emph{Sie starb mit dem Gedanken, dass sie versagt hatte und dass der Dunkle Lord als nächstes ihr Baby töten würde. Das ist Schmerz.}

"Er - der Dunkle Lord hat sie nicht gefoltert -" sagte Harry. "Wenn du das wissen willst."

Hinter Harry entriegelte sich die Tür und schwang auf.

Harry ging hinaus.

Es war Freitag, der 10. April des Jahres 1992.

