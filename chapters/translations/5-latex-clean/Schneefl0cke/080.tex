

\hypertarget{tabubruxfcche-teil-3}{% \section{81. Tabubrüche, Teil 3}\label{tabubruxfcche-teil-3}}

\textbf{\uline{Tabubrüche, Teil 3}}

In aufsteigenden Halbkreisen aus dunklem Stein, ein großes Meer von erhobenen Händen. Die Lords und Ladies des Zaubergamot, in pflaumenfarbenen Roben, die mit einem silbernen '\emph{Z}' gekennzeichnet waren, starrten in strengem Tadel auf ein junges Mädchen herab, das in Ketten zitterte. Wenn sie sich in einem bestimmten ethischen System selbst verdammt hatten, dann hielten sie offensichtlich ziemlich viel von sich selbst, weil sie das getan hatten.

Harrys Atem zitterte in seiner Brust. Seine dunkle Seite hatte sich einen Plan ausgedacht - und sich dann wieder herausgedreht, weil ein zu eisiges Sprechen nicht zu Hermines Vorteil sein würde; eine Tatsache, die der nur halbwegs kalte Harry irgendwie nicht erkannt hatte…

"Die Abstimmung ist angenommen", stimmte der Sekretär an, als alles ausgezählt war und die erhobenen Hände wieder nach unten fielen. "Das Zaubergamot erkennt die Blutschuld an, die Hermine Granger dem Haus Malfoy für den versuchten Mord an seinem Spross und die Beendigung seiner Linie schuldet."

Lucius Malfoy lächelte in grimmiger Zufriedenheit.

"Und nun", sagte der weißmähnige Zauberer, "sage ich, dass ihre Schuld beglichen werden soll -"

Harry ballte die Fäuste unter der Bank und rief: "Mit der Schuld, die das Haus Malfoy dem Haus Potter schuldet!"

"Ruhe!", schnauzte die Frau mit zu viel rosa Make-up, die neben Minister Fudge saß. "Du hast die Sitzung schon genug gestört! Auroren, geleiten Sie ihn hinaus!"

"Warten Sie", sagte Augusta Longbottom von der obersten Sitzreihe aus. "Was ist das für eine Schuld?"

Lucius' Hände wurden auf seinem Stock weiß. "Das Haus Malfoy hat keine Schulden bei dir!"

Es war nicht die solideste Hoffnung der Welt, sie basierte auf einem Zeitungsartikel einer Frau, die mit einem falschen Gedächtniszauber belegt worden war, aber Rita Skeeter schien es plausibel zu finden, dass Mr. Weasley angeblich James Potter etwas schuldete, weil…

"Es wundert mich, dass du das vergessen hast", sagte Harry gleichmäßig. "Sicherlich war es eine grausame und schmerzhafte Zeit deines Lebens, als du unter dem Imperius-Fluch von Er-der-nicht-genannt-werden-muss gelitten hast, bis du durch die Bemühungen des Hauses Potter davon befreit wurdest. Durch meine Mutter, Lily Potter, die dafür starb, und durch meinen Vater, James Potter, der dafür starb, und natürlich durch mich."

Es herrschte eine kurze Stille in der Ältestenhalle.

"Was für ein ausgezeichnetes Argument, Mr. Potter", sagte die alte Hexe, die als Madam Bones identifiziert worden war. "Auch ich bin ziemlich überrascht, dass Lord Malfoy ein so bedeutendes Ereignis vergessen würde. Es muss ein so glücklicher Tag für ihn gewesen sein."

"Ja", sagte Augusta Longbottom. "Er muss so dankbar gewesen sein."

Madam Bones nickte. "Das Haus Malfoy könnte diese Schuld unmöglich leugnen - es sei denn, Lord Malfoy will uns sagen, dass er sich an etwas falsch erinnert hat? Daran hätte ich ein recht professionelles Interesse. Wir sind immer bemüht, unser Bild von diesen dunklen Tagen zu verbessern."

Lucius Malfoys Hände umklammerten den silbernen Schlangengriff seines Stocks, als wolle er damit zuschlagen, die Macht entfesseln, die er in sich trug - dann schien sich Lord Malfoy zu entspannen, und ein kühles Lächeln kam über sein Gesicht. "Natürlich", sagte er leichthin. "Ich gebe zu, ich hatte es nicht verstanden, aber das Kind hat recht. Aber ich glaube nicht, dass sich die beiden Schulden aufheben - Haus Potter hat schließlich nur versucht, sich selbst zu retten -"

"Nicht ganz", sagte Dumbledore von oben.

"- und deshalb", stimmte Lucius Malfoy ein, "verlange ich auch eine finanzielle Entschädigung, um die Blutschuld an meinem Sohn zu tilgen. Auch das ist das Gesetz."

Harry spürte ein seltsames inneres Zusammenzucken. Das hatte auch in dem Zeitungsartikel gestanden, Mr. Weasley hatte zusätzlich zehntausend Galleonen gefordert -

"Wie viel?", fragte der Junge-der-lebte.

Lucius trug immer noch das kalte Lächeln. "Einhunderttausend Galleonen. Wenn du nicht so viel in deinem Tresor hast, muss ich wohl einen Schuldschein für den Rest akzeptieren."

Ein Aufschrei des Protestes ging von Dumbledores Seite des Raumes aus, sogar einige der pflaumenfarbenen Roben in der Mitte schauten schockiert.

"Sollen wir das Zaubergamot darüber abstimmen lassen?", sagte Lucius Malfoy. "Ich glaube, nur wenige von uns würden es gerne sehen, wenn die kleine Mörderin frei kommt. Durch Handzeichen wäre die zusätzliche Entschädigung von hunderttausend Galleonen erforderlich, um die Schuld zu streichen!"

Der Schreiber begann zu zählen, aber auch dieses Votum war eindeutig.

Harry stand da und atmete tief durch.

\emph{Darüber solltest du besser gar nicht erst nachdenken,} sagte Harrys innerer Gryffindor drohend.

\emph{Es ist eine hohe Summe,} bemerkte der Ravenclaw. \emph{Wir sollten viel Zeit damit verbringen, darüber nachzudenken.}

\emph{Es sollte nicht schwer sein. Das hätte es nicht sollen. Zwei Millionen Pfund waren nur Geld, und Geld war nur das wert, was es kaufen konnte}…

Es war seltsam, wie viel psychologische Anhaftung man an "\emph{nur Geld}" haben konnte, oder wie schmerzhaft es sein konnte, sich vorzustellen, einen Banktresor voller Gold zu verlieren, von dem man sich noch ein Jahr zuvor nicht einmal vorstellen konnte, dass er existierte.

\emph{Kimball Kinnison würde nicht zögern}, sagte der Gryffindor. \emph{Ernsthaft, es wäre eine Blitzentscheidung. Was für ein Held bist du? Ich hasse dich schon allein dafür, dass du länger als 50 Millisekunden darüber nachdenken musst.}

\emph{Das ist das wahre Leben,} sagte Ravenclaw. \emph{Sein ganzes Geld zu verlieren, ist für echte Menschen im wirklichen Leben viel schmerzhafter als in Heldenbüchern.}

\emph{Was?} fragte Gryffindor. \emph{Auf wessen Seite stehst du?}

\emph{Ich habe nicht für eine bestimmte Antwort plädiert,} sagte Ravenclaw, \emph{ich habe es nur gesagt, weil es wahr ist.}

\emph{Könnte man mit hunderttausend Galleonen mehr als ein Leben retten, wenn man sie auf andere Weise ausgeben würde?} sagte Slytherin. \emph{Wir haben zu forschen, Schlachten zu schlagen, der Unterschied zwischen 40.000 Galleonen reich und 60.000 Galleonen verschuldet zu sein ist nicht trivial -}

\emph{also werden wir einfach eine unserer Möglichkeiten nutzen, um schnell Geld zu verdienen und alles zurückzubekommen,} sagte Hufflepuff.

\emph{Es ist nicht sicher, dass diese funktionieren}, sagte Slytherin, \emph{und viele von ihnen erfordern Startkapital -}

\emph{Ich persönlich,} sagte Gryffindor, \emph{bin dafür, dass wir Hermine retten und uns dann zusammentun und unseren inneren Slytherin töten}.

Die Stimme des Schreibers sagte, dass die Stimmenauszählung aufgezeichnet worden war und die Abstimmung angenommen wurde…

Harrys Lippen öffneten sich. "Ich nehme das Angebot an", sagten Harrys Lippen, ohne jegliches Zögern, ohne dass eine Entscheidung getroffen worden wäre; gerade so, als wäre die innere Debatte nur Schein und Illusion gewesen, ohne dass der wahre Kontrolleur der Stimme daran beteiligt gewesen wäre.

Lucius Malfoys Maske der Ruhe zerbrach, seine Augen weiteten sich, er starrte Harry in blankem Erstaunen an. Sein Mund hatte sich leicht geöffnet, obwohl er nicht sprach, und wenn er irgendwelche merkwürdigen Geräusche machte, war das nicht zu hören über dem Gebrüll des gleichzeitigen Keuchens des Zaubergamot -

Ein Klopfen auf Stein brachte die Menge zum Schweigen.

"Nein", sagte die Stimme von Dumbledore.

Harrys Kopf ruckte herum und starrte den alten Zauberer an. Dumbledores liniertes Gesicht war blass, der silberne Bart zitterte sichtlich, er sah aus, als befände er sich in den letzten Zügen einer unheilbaren Krankheit.

"Es - tut mir leid, Harry - aber diese Entscheidung liegt nicht bei dir - denn ich bin immer noch der Bevollmächtigte deines Vermögens."

"Was?!", sagte Harry, zu schockiert, um seine Antwort zu formulieren.

"Ich kann nicht zulassen, dass du dich bei Lucius Malfoy verschuldest, Harry! Das kann ich nicht! Du weißt nicht - du begreifst nicht -"

\textbf{\emph{STIRB}}.

Harry wusste nicht einmal, welcher Teil von ihm gesprochen hatte, es hätte eine einstimmige Abstimmung sein können, die pure Wut und der Zorn strömten durch ihn. Einen Augenblick lang dachte er, dass die schiere Kraft der Wut magische Flügel annehmen und hinausfliegen könnte, um den Schulleiter zu treffen, ihn tot vom Podium zurückstürzen zu lassen - aber als diese mentale Stimme gesprochen hatte, stand der alte Zauberer immer noch da und starrte Harry an, den langen dunklen Zauberstab in der rechten Hand, den kurzen schwarzen Stab in der linken. Und Harrys Augen gingen auch zu dem rotgoldenen Vogel, der mit seinen Krallen auf der Schulter von Dumbledores schwarzer Robe ruhte und schwieg, als kein Phönix hätte schweigen dürfen.

"Fawkes", sagte Harry, wobei seine Stimme in seinen eigenen Ohren seltsam klang, "kannst du ihn für mich anschreien?"

Der feurige Vogel auf der Schulter des alten Zauberers schrie nicht. Vielleicht hatte das Zaubergamot verlangt, dass ein Schweigezauber auf die Kreatur gelegt wurde, sonst hätte sie wahrscheinlich die ganze Zeit geschrien. Aber Fawkes schlug auf seinen Herrn ein, wobei ein goldener Flügel den Kopf des alten Zauberers umwarf.

"Ich kann nicht, Harry!", sagte der alte Zauberer, die Qual deutlich in seiner Stimme. "Ich tue nur, was ich tun muss!"

Und Harry wusste, als er den rotgoldenen Vogel ansah, was auch er tun musste. Es hätte von Anfang an klar sein müssen, diese Lösung.

"Dann werde auch ich tun, was ich tun muss", sagte Harry zu Dumbledore hinauf, als stünden sie beide allein im Raum. "Das ist Ihnen doch klar, oder?"

Der alte Zauberer schüttelte den zitternden Kopf. "Du wirst deine Meinung ändern, wenn du älter bist -"

"Davon spreche ich nicht", sagte Harry, seine Stimme noch immer fremd in seinen eigenen Ohren. "Ich meine, dass ich unter keinen Umständen zulassen werde, dass Hermine Granger von Dementoren gefressen wird. \textbf{Punkt}. Ungeachtet dessen, was irgendein Gesetz sagt, und ungeachtet dessen, was ich tun muss, um es zu verhindern. Muss ich es noch buchstabieren?"

Eine fremde männliche Stimme sprach von irgendwo in der Ferne: "Sorgen Sie dafür, dass das Mädchen direkt nach Askaban gebracht und unter besondere Bewachung gestellt wird."

Harry wartete, starrte den alten Zauberer an und sprach dann wieder.

"Ich werde nach Askaban gehen", sagte Harry zu dem alten Zauberer, als stünden sie allein auf der Welt, "bevor Hermine dorthin gebracht werden kann, und ich werde mit den Fingern schnippen. Es mag mich mein Leben kosten, aber wenn sie dort ankommt, wird es kein Askaban mehr geben."

Einige Mitglieder des Zaubergamot schnappten überrascht nach Luft.

Dann begann eine größere Anzahl zu lachen.

"Wie willst du überhaupt dorthin kommen, Kleiner?", sagte jemand aus den Reihen der Lachenden.

"Ich habe meine Wege, um dorthin zu gelangen", sagte die ferne Stimme des Jungen. Harry hielt seine Augen auf Dumbledore gerichtet, auf den alten Zauberer, der ihn schockiert anstarrte. Harry sah Fawkes nicht direkt an, verriet seinen Plan nicht; aber in seinem Geist bereitete er sich darauf vor, den Phönix zu beschwören, um ihn zu transportieren, bereit, seinen Geist mit Licht und Wut zu füllen, den Feuervogel mit all seiner Kraft zu rufen, er würde es vielleicht auf der Stelle tun müssen, wenn Dumbledore auf seinen Zauberstab zeigte -

"Würdest du wirklich?", sagte der alte Zauberer zu Harry, auch so, als stünden die beiden allein im Raum.

Der Raum verstummte wieder, während alle schockiert auf den Obersten Hexenmeister des Zaubergamot starrten, der die wahnsinnige Drohung völlig ernst zu nehmen schien. Die Augen des alten Zauberers waren nur auf Harry fixiert.

"Würdest du alles - alles - nur für sie riskieren?"

"Ja", erwiderte Harry.

\emph{Das ist die falsche Antwort, weißt du}, sagte Slytherin. \emph{Ganz im Ernst}.

\textbf{\emph{Aber es ist die wahre Antwort.}}

"Du willst nicht zur Vernunft kommen?", fragte der alte Zauberer.

"Offenbar nicht", sagte Harry zurück.

Sie blickten sich weiter an.

"Das ist eine schreckliche Torheit", sagte der alte Zauberer.

"Dessen bin ich mir bewusst", antwortete der Held. "Und jetzt geh mir aus dem Weg."

Ein seltsames Licht funkelte in den alten blauen Augen.

"Wie du willst, Harry Potter, aber wisse, dass dies noch nicht vorbei ist."

Der Rest der Welt verschwand wieder in der Existenz.

"Ich ziehe meinen Einspruch zurück", sagte der alte Zauberer, "Harry Potter mag tun, was er will", und das Zaubergamot explodierte in einem Brüllen des Entsetzens, nur um durch ein letztes Klopfen des Steinstabs zum Schweigen gebracht zu werden.

Harry drehte seinen Kopf zurück und sah Lord Malfoy an, der aussah, als hätte er gesehen, wie sich eine Katze in einen Menschen verwandelte und anfing, andere Katzen zu fressen. Den Blick verwirrt zu nennen, beschrieb es nicht ansatzweise.

"Du würdest wirklich…" Lucius Malfoy sagte langsam. "Du würdest wirklich hunderttausend Galleonen bezahlen, um ein Schlammblutmädchen zu retten."

"Ich glaube, in meinem Gringotts-Tresor befinden sich etwa vierzigtausend", sagte Harry. Es war seltsam, dass das immer noch mehr innere Schmerzen verursachte als der Gedanke, ein über fünfzigprozentiges Risiko für sein Leben einzugehen, um Askaban zu zerstören. "Was die anderen sechzigtausend angeht - wie lauten die Regeln genau?"

"Die werden fällig, wenn du deinen Abschluss in Hogwarts machst", sagte der alte Zauberer von oben herab. "Aber ich fürchte, Lord Malfoy hat schon vorher gewisse Rechte über dich."

Lucius Malfoy stand regungslos da und blickte stirnrunzelnd auf Harry herab.

"Wer ist sie für dich? Was ist sie für dich, dass du so viel bezahlen würdest, um sie vor Schaden zu bewahren?"

"Meine Freundin", sagte der Junge leise.

Lucius Malfoys Augen verengten sich.

"Nach dem Bericht, den ich erhalten habe, kannst du den Patronus-Zauber nicht wirken, und Dumbledore weiß das. Die Kraft eines einzigen Dementors hätte dich fast getötet. Du würdest dich nicht einmal in die Nähe von Askaban wagen -"

"Das war im Januar", sagte Harry. "Jetzt haben wir April."

Die Augen von Lucius Malfoy blieben kühl und berechnend.

"Du tust so, als könntest du Askaban zerstören, und Dumbledore tut so, als würde er es glauben."

Harry antwortete nicht. Der weißhaarige Mann drehte sich leicht in Richtung der Mitte des Halbkreises, als wolle er sich an das gesamte Zaubergamot wenden.

"Ich ziehe mein Angebot zurück!", rief der Lord von Malfoy. "Ich werde die Schuld gegenüber dem Haus Potter nicht in Zahlung nehmen, nicht einmal für hunderttausend Galleonen! Die Blutschuld des Mädchens gegenüber dem Haus Malfoy bleibt bestehen!"

Wieder das Gebrüll vieler Stimmen.

"Unehrenhaft!", rief jemand.

"Du erkennst die Schuld gegenüber dem Haus Potter an und dennoch würdest du -" und dann brach die Stimme ab.

"Ich erkenne die Schuld an, aber das Gesetz verpflichtet mich nicht strikt dazu, sie zu erlassen", sagte Lord Malfoy mit einem grimmigen Lächeln. "Das Mädchen ist kein Teil des Hauses Potter; die Schuld, die ich dem Haus Potter schulde, ist keine Schuld ihr gegenüber. Was die Entehrung angeht -" Lucius Malfoy hielt inne. "Was die große Scham betrifft, die ich wegen meiner Undankbarkeit gegenüber den Potters empfinde, die so viel für mich getan haben -"

Lucius Malfoy senkte den Kopf.

"Mögen meine Vorfahren mir vergeben."

"Nun, Junge?", rief der vernarbte Mann, der zur rechten Hand von Lord Malfoy saß. "Dann geh und zerstöre Askaban!"

"Das würde ich gerne sehen", sagte eine andere Stimme. "Verkaufst du Eintrittskarten?"

Es verstand sich von selbst, dass Harry nicht gerade diesen Moment wählte, um aufzugeben. \emph{Das Mädchen gehört nicht zum Haus Potter} - eigentlich hatte er den offensichtlichen Ausweg aus dem Dilemma fast sofort gesehen. Er hätte vielleicht länger gebraucht, wenn er nicht in letzter Zeit eine Reihe von Gesprächen zwischen älteren Ravenclaw-Mädchen belauscht und eine gewisse Anzahl von Klitterer-Geschichten gelesen hätte. Nichtsdestotrotz hatte er Schwierigkeiten, es zu akzeptieren.

\emph{Das ist lächerlich,} sagte ein Teil von Harry, der sich gerade als interner Konsistenzprüfer betitelt hatte. \emph{Unsere Handlungen hier sind völlig inkohärent. Erst empfindest du weniger emotionale Abneigung, dein verdammtes LEBEN zu riskieren und wahrscheinlich für Hermine zu sterben, als dich von einem dummen Haufen Gold zu trennen. Und jetzt sträubst du dich, nur wegen einer Heirat?}

\textbf{\emph{SYSTEMFEHLER.}}

\emph{Weißt du was?} Sagte der interne Konsistenzprüfer. \emph{Du bist dumm.}

\emph{Ich habe nicht nein gesagt, dachte Harry. Ich habe nur} \textbf{\emph{SYSTEMFEHLER}} \emph{gesagt.}

\emph{Ich bin dafür, Askaban zu zerstören,} sagte Gryffindor. \emph{Es muss sowieso gemacht werden.}

\emph{Wirklich, wirklich dumm}, sagte der interne Konsistenzprüfer. \emph{Oh, scheiß drauf, ich übernehme die Kontrolle über unseren Körper.}

Der Junge holte tief Luft und öffnete den Mund -

Zu diesem Zeitpunkt hatte Harry Potter die Existenz von Professor McGonagall schon völlig vergessen, die die ganze Zeit dort gesessen und eine Reihe interessanter Veränderungen des Gesichtsausdrucks durchgemacht hatte, die Harry nicht beachtet hatte, weil er abgelenkt war. Es wäre zu hart gewesen, zu sagen, dass Harry sie vergessen hatte, weil er sie nicht als Wichtig betrachtete. Man könnte freundlicher sagen, dass Professor McGonagall keine sichtbare Lösung für eines seiner aktuellen Probleme war, und deshalb war sie nicht Teil des Universums. Harry, der zu diesem Zeitpunkt eine ordentliche Portion Adrenalin im Blut hatte, schreckte auf und zuckte sichtlich zusammen, als Professor McGonagall, deren Augen nun vor unmöglicher Hoffnung glühten und deren Tränen auf den Wangen schon halb getrocknet waren, aufsprang und rief: "Mit mir, Mr. Potter!" und, ohne eine Antwort abzuwarten, die Treppe hinunterrannte, die zur unteren Plattform führte, wo ein Stuhl aus dunklem Metall wartete.

Es dauerte einen Moment, aber Harry rannte hinterher; allerdings dauerte es länger, bis er unten ankam, nachdem Professor McGonagall mit einer seltsamen katzenartigen Bewegung die halbe Treppe übersprungen hatte und bei dem erstaunt dreinblickenden Aurorentrio landete, das bereits seine Zauberstäbe auf sie richtete.

"Miss Granger!", rief Professor McGonagall. "Können Sie schon sprechen?"

Ähnlich wie bei Professor McGonagall konnte man in gewissem Sinne sagen, dass Harry die Existenz von Hermine Granger vergessen hatte, weil er seinen Nacken nach hinten geneigt hatte, um nach oben statt nach unten zu schauen, und weil er sie nicht als Lösung für irgendeines seiner aktuellen Probleme betrachtet hatte. Obwohl es kaum sicher war, ja, es war sogar unwahrscheinlich, dass es Harry auch nur im Geringsten geholfen hätte, wenn er daran gedacht hätte, Hermine anzuschauen oder darüber nachzudenken, was sie fühlen musste. Harry erreichte den Fuß der Treppe und sah Hermine Granger in voller Pracht - ohne nachzudenken, ohne sich helfen zu können, schloss Harry die Augen, aber er hatte alles gesehen.

Ihren Schulmantel um den Hals, ganz durchnässt von Tränen. Die Art, wie sie von ihm weggesehen hatte. Und das Auge der Erinnerung und des Mitgefühls, das sich nicht schließen ließ, das nicht wegschauen konnte, wusste, dass Hermine vor dem Adel des magischen Britanniens und Professor McGonagall und Dumbledore und Harry die schlimmste Schande ihres Lebens erzählt hatte; und dann nach Askaban verurteilt worden war, wo sie der Dunkelheit und der Kälte und all ihren schlimmsten Erinnerungen ausgesetzt sein würde, bis sie verrückt wurde und starb; und dann hatte sie gehört, dass Harry sein ganzes Geld verschenken und sich verschulden würde, um sie zu retten, und vielleicht sogar sein Leben opfern würde, und da der Dementor nur ein paar Schritte hinter ihr stand, hatte sie nichts gesagt. ..

"J-ja", flüsterte die Stimme von Hermine Granger. "Ich k-kann reden."

Harry öffnete die Augen wieder und sah ihr Gesicht, das ihn nun anschaute. Es sagte nichts von dem, was er dachte, was Hermine fühlte, Gesichter konnten nichts so Kompliziertes sagen, alles, was die Gesichtsmuskeln tun konnten, war, sich zu Knoten zu verziehen.

"H-H-Harry, mir geht's gut, lass mich -"

"Halt die Klappe", schlug Harry vor.

"tut m-m-mir L-leid -"

"Wenn du mich nie im Zug getroffen hättest, wärst du jetzt nicht in Schwierigkeiten. Also halt die Klappe", sagte Harry Potter.

"Hört beide auf, so albern zu sein", sagte Professor McGonagall in ihrem festen schottischen Akzent (es war seltsam, wie sehr das half). "Mr. Potter, halten Sie Ihren Zauberstab so, dass die Finger von Miss Granger ihn berühren können. Miss Granger, sprechen Sie mir nach. \emph{Bei meinem Leben und meiner Magie} -"

Harry tat, wie ihm geheißen, und stieß seinen Zauberstab vor, um Hermines Finger zu berühren; und dann sagte Hermines stockende Stimme:

"Bei meinem Leben und meiner Magie -"

"\emph{Ich schwöre, dem Haus Potter zu dienen} -", sagte Professor McGonagall.

Und Hermine sagte, ohne auf weitere Anweisungen zu warten, und die Worte sprudelten nur so aus ihr heraus:

"Ich schwöre, dem Haus Potter zu dienen, seinem Meister oder seiner Meisterin zu gehorchen, zu ihrer Rechten zu stehen, auf ihren Befehl hin zu kämpfen und ihnen zu folgen, wohin sie gehen, bis zum Tag meines Todes."

All diese Worte waren in einem verzweifelten Keuchen herausgeplatzt, bevor Harry etwas hätte denken oder sagen können, wenn er verrückt genug gewesen wäre, sie zu unterbrechen.

"Mr. Potter, wiederholen Sie diese Worte", sagte Professor McGonagall. "\emph{Ich, Harry, Erbe und letzter Spross der Potters, nehme deinen Dienst an, bis zum Ende der Welt und ihrer Magie.}"

Harry holte tief Luft und sagte: "Ich, Harry, Erbe und letzter Spross der Potters, nehme Ihre Dienste an, bis zum Ende der Welt und ihrer Magie."

"Das war's", sagte Professor McGonagall. "Gut gemacht."

Harry blickte auf und sah, dass das gesamte Zaubergamot, dessen Existenz er vergessen hatte, auf sie starrte. Und dann schaute Minerva McGonagall, die Leiterin des Hauses Gryffindor, auch wenn sie sich nicht immer so verhielt, hoch oben auf die Stelle, an der Lucius Malfoy stand; und sie sagte vor dem gesamten Zaubergamot zu ihm: "Ich bereue jeden Punkt, den ich dir jemals in Verwandlung gegeben habe, du widerlicher kleiner Wurm."

Was auch immer Lucius darauf erwidern wollte, wurde durch einen Schlag mit dem kurzen Stab in Dumbledores Hand zum Schweigen gebracht. "Ähem!", sagte der alte Zauberer von seinem Podium aus dunklem Stein. "Diese Sitzung hat sich ziemlich in die Länge gezogen, und wenn sie nicht bald beendet wird, könnten einige von uns ihr gesamtes Mittagessen verpassen. Das Gesetz in dieser Angelegenheit ist klar. Es wurde bereits über die Bedingungen des Abkommens abgestimmt, und Lord Malfoy kann es rechtlich nicht ablehnen. Da wir die uns zugewiesene Zeit weit überschritten haben, vertage ich nun, in Übereinstimmung mit dem letzten Beschluss der Überlebenden des achtundachtzigsten Zaubergamots, diese Sitzung."

Der alte Zauberer klopfte dreimal auf den Stab aus dunklem Stein.

"Ihr Narren!", rief Lucius Malfoy.

Das weiße Haar schüttelte sich wie im Wind, das Gesicht darunter war blass vor Wut. "Glaubt ihr, dass ihr mit dem, was ihr heute getan habt, davonkommt? Glaubst du, dieses Mädchen kann versuchen, meinen Sohn zu ermorden und ungeschoren davonkommen?"

Die krötenartige, rosa geschminkte Frau, an deren Namen sich Harry nicht mehr erinnern konnte, erhob sich von ihrem Platz.

"Natürlich nicht", sagte sie mit einem hämischen Lächeln. "Immerhin ist das Mädchen immer noch eine Mörderin, und ich denke, das Ministerium wird ihre Angelegenheiten sehr genau beobachten - es erscheint kaum klug, sie auf den Straßen herumlaufen zu lassen…"

Harry war an diesem Punkt genervt. Ohne zu warten, um zuzuhören, drehte sich Harry auf dem Absatz um und schritt in langen Schritten vorwärts in Richtung -

Das Grauen, das nur er wirklich sehen konnte, die Abwesenheit von Farbe und Raum, die Wunde in der Welt, über der ein zerfledderter Umhang schwebte; höchst unvollkommen bewacht von einem rennenden, mondbeschienenen Eichhörnchen und einem flatternden Silberspatz.

Seine dunkle Seite hatte auch bemerkt, als sie den ganzen Raum nach allem durchsuchte, was möglicherweise als Waffe verwendet werden konnte, dass der Feind dumm genug gewesen war, einen Dementor in Harrys Gegenwart zu bringen. Das war in der Tat eine mächtige Waffe, und eine, die Harry vielleicht besser beherrschte als seine angeblichen Meister. Es gab eine Zeit in Askaban, da hatte Harry zwölf Dementoren befohlen, sich umzudrehen und zu gehen, und sie waren gegangen.

\emph{Die Dementoren sind der Tod, und der Patronus-Zauber funktioniert, indem man an glückliche Gedanken denkt, statt an den Tod.}

Wenn Harrys Theorie richtig war, würde dieser eine Satz genügen, um die Patronus-Zauber der Auroren wie eine Seifenblase zerplatzen zu lassen und dafür zu sorgen, dass niemand in Reichweite seiner Stimme einen weiteren wirken konnte.

\emph{Ich werde die Patronus-Zauber aufheben und verhindern, dass noch mehr Patronusse gewirkt werden. Und dann wird mein Dementor, schneller fliegend als jeder Besen,} \emph{jeden hier küssen, der dafür gestimmt hat, ein zwölfjähriges Mädchen nach Askaban zu schicken. Sag das vorher, um die Wenn-dann-Erwartung aufzustellen, und warte, dass die Leute es verstehen und lachen. Dann spreche die tödliche Wahrheit; und wenn die Patronusse der Auroren zum Beweis ausblinzeln würden, würde entweder die Vorahnung der Menschen auf die geistlose Leere oder Harrys Drohung mit ihrer Zerstörung den Dementor gehorchen lassen.}

\textbf{\emph{Diejenigen, die einen Kompromiss mit der Dunkelheit gesucht hatten, würden von ihr verschlungen werden.}}

Das war die andere Lösung, die sich seine dunkle Seite ausgedacht hatte.

Das hinter ihm aufsteigende Keuchen ignorierend, durchquerte Harry den Radius des Patronus, schritt bis auf einen Schritt an den Tod heran. Seine ungehinderte Angst umgab ihn wie ein Strudel, als ob er neben den saugenden Abfluss einer riesigen Badewanne treten würde, die ihr Wasser entleert; aber da die falschen Patronusse die Ebene, auf der sie interagierten, nicht mehr verdeckten, konnte Harry den Dementor erreichen, genauso wie dieser ihn erreichen konnte. Harry blickte geradewegs in das ziehende Vakuum und -

\emph{die Erde zwischen den Sternen all sein Triumph, Hermine gerettet zu haben, eines Tages wird die Realität, von der du ein Schatten bist, aufhören zu existieren!}

Harry nahm all die silberne Emotion, die seinen Patronus-Zauber angetrieben hatte, und schob sie gedanklich auf den Dementor; und erwartete, dass der Schatten des Todes vor ihm fliehen würde - - und als Harry das tat, warf er seine Hände hoch und schrie "Buh!"

Die Leere wich scharf von Harry zurück, bis sie auf den dunklen Stein dahinter stieß.

In der Halle herrschte eine tödliche Stille. Harry drehte der leeren Leere den Rücken zu und schaute hinauf zu der Stelle, wo die Krötenfrau stand. Sie war blass unter der rosa Schminke, ihr Mund öffnete und schloss sich wie der eines Fisches.

"Ich mache dir dieses Angebot nur ein einziges Mal!", sagte der Junge-der-lebte. "Ich erfahre nie, dass du dich bei mir oder einem meiner Freunde eingemischt hast. Und du erfährst nie, warum das unbesiegbare, seelenfressende Monster Angst vor mir hat. Jetzt setz dich hin und halt den Mund."

Die Krötenfrau ließ sich wortlos auf ihre Bank zurückfallen. Harry blickte weiter nach oben.

"Ein Rätsel, Lord Malfoy!", rief der Junge-der-lebte quer durch die Allerälteste Halle. "Ich weiß, dass Sie nicht in Ravenclaw waren, aber versuchen Sie trotzdem, dieses zu beantworten! Was vernichtet die Dunklen Lords, erschreckt die Dementoren und schuldet dir sechzigtausend Galleonen?"

Einen Augenblick lang stand Lord Malfoy mit leicht geweiteten Augen da, dann verwandelte sich sein Gesicht wieder in ruhigen Hohn, und seine Stimme antwortete kühl.

"Wollen Sie mir offen drohen, Mr. Potter?"

"Ich bedrohe dich nicht", sagte der Junge-der-lebte. "Ich jage dir Angst ein. Das ist ein Unterschied."

"Genug, Mr. Potter", sagte Professor McGonagall. "Wir werden ohnehin zu spät zur Nachmittagsverwandlungsstunde kommen. Und kommen Sie bitte zurück, Sie machen dem armen Dementor immer noch Angst."

Sie wandte sich an die Auroren.

"Mr. Kleiner, wenn Sie so freundlich wären!"

Harry schritt zu ihnen zurück, während der angesprochene Auror sich nach vorne bewegte und einen kurzen Stab aus dunklem Metall an den dunklen Metallstuhl drückte und ein unhörbares Wort der Entlassung murmelte. Die Ketten glitten so geschmeidig zurück, wie sie herausgekommen waren; und Hermine schob sich so schnell sie konnte aus dem Stuhl und rannte halb rennend und halb torkelnd ein paar Schritte vorwärts.

Harry streckte seine Arme aus - - und Hermine sprang halb und fiel halb in Professor McGonagalls Arme und begann hysterisch zu schluchzen.

\emph{Hmpfh,} sagte eine Stimme in Harry. \emph{Ich dachte irgendwie, wir hätten uns das selbst verdient.}

\emph{Ach, halt die Klappe.}

Professor McGonagall hielt Hermine so fest, dass man hätte meinen können, es sei eine Mutter, die ihre Tochter oder vielleicht Enkelin festhält. Nach ein paar Augenblicken wurde Hermines Schluchzen langsamer und hörte dann auf. Professor McGonagall änderte plötzlich ihre Haltung und packte sie fester; die Hände des Mädchens baumelten jetzt schlaff herunter und ihre Augen waren geschlossen -

"Sie wird wieder gesund, Mr. Potter", sagte Professor McGonagall leise in Harrys Richtung, ohne ihn anzusehen. "Sie braucht nur ein paar Stunden in einem der Betten von Madam Pomfrey."

"Also gut", sagte Harry. "Bringen wir sie zu Madam Pomfrey."

"Ja", sagte Dumbledore, als er die dunkle Steintreppe hinunterstieg. "Lasst uns alle nach Hause gehen, in der Tat."

Seine blauen Augen waren auf Harry gerichtet, so hart wie Saphire.

…

Die Lords und Ladies des Zaubergamot verlassen ihre hölzernen Bänke, gehen so, wie sie gekommen sind, und sehen ziemlich nervös aus.

Die allermeisten denken: "Der Dementor hatte Angst vor dem Jungen!"

Einige der Klügeren fragen sich bereits, wie sich dies auf das empfindliche Machtgleichgewicht des Zaubergamots auswirken wird - wenn eine neue Figur auf dem Spielbrett erschienen ist.

\emph{Fast keiner denkt etwas in der Art von "Ich frage mich, wie er das gemacht hat."}

Das ist die Wahrheit über das Zaubergamot: Viele sind Adelige, viele sind reiche Wirtschaftsmagnaten, einige wenige sind auf andere Weise zu ihrem Status gekommen. Einige von ihnen sind dumm. Die meisten sind gewitzt in den Bereichen Wirtschaft und Politik, aber ihre Gewitztheit ist begrenzt. Fast keiner hat den Weg eines mächtigen Zauberers beschritten. Sie haben sich nicht durch uralte Bücher gelesen, alte Schriftrollen unter die Lupe genommen, auf der Suche nach Wahrheiten, die zu mächtig sind, um offen und getarnt in Rätseln zu wandeln, auf der Jagd nach wahrer Magie unter hundert fantastischen Märchen. Wenn sie nicht gerade einen Schuldvertrag vor sich haben, geben sie auf, was an Scharfsinn in ihnen steckt, und entspannen sich mit irgendeinem bequemen Blödsinn. Sie glauben an die Heiligtümer des Todes, aber sie glauben auch, dass Merlin den schrecklichen Totoro bekämpft und den Ree gefangen hat. Sie wissen (denn auch das ist Teil der Standardlegende), dass ein mächtiger Zauberer lernen muss, die Wahrheit unter hundert plausiblen Lügen zu unterscheiden. Aber es ist ihnen nicht in den Sinn gekommen, dass sie das auch tun könnten.

(Warum nicht? Nun, warum sollten Zauberer, die über genügend Status und Reichtum verfügen, um sich fast allen Unternehmungen zuzuwenden, ihr Leben damit verbringen, um lukrative Monopole für den Tintenimport zu kämpfen?

Der Schulleiter von Hogwarts würde die Frage kaum sehen; natürlich sollten die meisten Menschen keine mächtigen Zauberer sein, so wie die meisten Menschen keine Helden sein sollten.

Der Verteidigungsprofessor könnte in großer und zynischer Länge erklären, warum ihre Ambitionen so trivial sind; auch für ihn gibt es kein Rätsel.

Nur Harry Potter ist trotz aller Bücher, die er gelesen hat, nicht in der Lage, es zu verstehen; dem Jungen, der gelebt hat, erscheinen die Lebensentscheidungen der Lords und Ladies unverständlich - nicht das, was ein guter Mensch tun würde, aber auch nicht ein böser Mensch. Wer von den dreien ist nun am weisesten?)

Aus welchem Grund auch immer, die meisten der Mitglieder des Zaubergamot haben den Weg, der zu mächtigen Zauberern führt, nie beschritten; sie suchen nicht nach dem Verborgenen. Für sie gibt es kein \emph{Warum}. Es gibt keine Erklärung. Es gibt keine Kausalität. Der Junge-der-lebte, der bereits auf halbem Weg in die Geschichten der Legende war, ist nun ganz dorthin befördert worden; und es ist eine brachiale Tatsache, einfach und unerklärlich, dass der Junge-der-lebte die Dementoren erschreckt. Zehn Jahre zuvor wurde ihnen erzählt, dass ein einjähriger Junge den schrecklichsten Dunklen Lord ihrer Generation besiegt hat, vielleicht den bösesten Dunklen Lord, der je gelebt hat; und auch das haben sie einfach akzeptiert. \emph{Man soll so etwas nicht in Frage stellen} (das wissen sie auf eine unausgesprochene Weise). Wenn der schrecklichste Dunkle Lord der Geschichte einem unschuldigen Baby gegenübersteht - wie sollte er nicht besiegt werden? Der Rhythmus des Stücks verlangt es. Man soll applaudieren, nicht von seinem Platz im Publikum aufstehen und sagen: "\emph{Warum}?" Es ist nur das Schicksal der Geschichte, dass der Dunkle Lord am Ende von einem kleinen Kind besiegt wird; und wenn man das in Frage stellt, kann man das Stück genauso gut gar nicht erst besuchen. Es kommt ihnen nicht in den Sinn, die Anwendung einer solchen Argumentation auf die Ereignisse, die sie mit eigenen Augen in der Allerältesten Halle gesehen haben, zu hinterfragen. In der Tat sind sie sich nicht bewusst, dass sie die Argumentation von Geschichten auf das reale Leben anwenden. Welches Gehirn würde den Jungen-der-lebte mit der gleichen sorgfältigen Logik untersuchen, die sie bei einem politischen Bündnis oder einer Geschäftsvereinbarung anwenden würden, wenn es so bequem durch Schicksal und Legende erklärt wird?

Aber es gibt einige wenige, die auf diesen Holzbänken sitzen, die nicht so denken. Es gibt ein paar wenige aus dem Zaubergamot, die sich durch halb zerfallene Schriftrollen gelesen und Geschichten über Dinge angehört haben, die dem Cousin des Bruders von jemandem passiert sind, nicht zur Unterhaltung, sondern als Teil einer Suche nach Macht und Wahrheit.

Sie haben die Nacht von Godric's Hollow, wie sie von Albus Dumbledore berichtet wurde, bereits als anormales und potenziell wichtiges Ereignis markiert.

Sie haben sich gefragt, \emph{warum} es passiert ist, \emph{wenn} es wirklich passiert ist; oder \emph{wenn nicht}, \emph{warum} Dumbledore \emph{lügt}.

Und wenn sich ein elfjähriger Junge erhebt und \emph{"Lucius Malfoy"} mit dieser kalten Erwachsenenstimme sagt und dann noch Worte spricht, die man von einem Erstklässler in Hogwarts einfach nicht erwarten würde, dann lassen sie diese Tatsache nicht in die gesetzlosen Unschärfen von Legenden und die Prämissen von Theaterstücken gleiten.

Sie markieren es als einen Hinweis.

Sie fügen es der Liste hinzu.

\emph{Diese Liste sieht langsam ziemlich beunruhigend aus.}

Es hilft nicht gerade, wenn der Junge einen Dementor mit "\textbf{Buh}!" anschreit und der verwesende Kadaver sich flach an die gegenüberliegende Wand presst und seine schreckliche, ohrenbetäubende Stimme rasselt:

\textbf{\emph{"Macht dass er weggeht!"}}

