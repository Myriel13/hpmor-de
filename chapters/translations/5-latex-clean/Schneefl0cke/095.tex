

\hypertarget{rollen-teil-7}{% \section{96. Rollen, Teil 7}\label{rollen-teil-7}}

\textbf{\uline{Rollen, Teil 7}}

\emph{Anmerkung des original Autors:}

\emph{Für diejenigen, die den Kanon nicht gelesen haben: Das Holzschild hat sich etwas verändert, aber die Inschrift hier ist die gleiche wie in J.K. Rowlings Original.}

Das vierte Treffen: (16:38~Uhr, 17. April 1992)

Der Mann mit dem abgetragenen, warmen Mantel, in dessen Wange drei schwache Narben für immer eingeätzt waren, beobachtete Harry Potter so genau wie möglich, während der Junge sich höflich in den Reihen der Häuser umsah. Für jemanden, dessen bester Freund gestern gestorben war, wirkte Harry Potter seltsam gefasst, obwohl er in keiner Weise an Gefühllosigkeit oder Normalität erinnerte.

\emph{Ich wünsche nicht, darüber zu sprechen,} hatte der Junge gesagt, \emph{weder mit Ihnen noch mit sonst jemandem.} Er sagte „wünschen“ und nicht „wollen“, als wollte er betonen, dass er in der Lage war, erwachsene Worte zu benutzen und erwachsene Entscheidungen zu treffen. Es gab nur eine Sache, die Remus Lupin eingefallen war, die helfen könnte, nachdem er die Eulen von Professor McGonagall und diesem seltsamen Mann Quirinus Quirrell erhalten hatte.

„Es gibt eine Menge leerer Häuser“, sagte der Junge und blickte sich wieder um.

Godric's Hollow hatte sich verändert, in dem Jahrzehnt, seit Remus Lupin ein häufiger Besucher gewesen war. Viele der alten, spitz zulaufenden Häuser sahen verlassen aus, mit grünen Blattranken, die über ihre Fenster und Türen wuchsen. Großbritannien war nach dem Zaubererkrieg merklich geschrumpft, hatte nicht nur die Toten, sondern auch die Geflüchteten verloren. Godrics Hollow war schwer getroffen worden. Und danach waren noch mehr Familien woanders hingezogen, nach Hogsmeade oder ins magische London, die verlassenen Häuser waren eine zu unangenehme Erinnerung. Andere waren geblieben. Godric's Hollow war älter als Hogwarts, älter als Godric Gryffindor, dessen Namen es angenommen hatte, und es gab Familien, die bis zum Ende der Welt und ihrer Magie hier wohnen würden. Die Potters waren eine solche Familie gewesen und würden es wieder sein, wenn der letzte Potter es so wollte. Remus Lupin versuchte, all das zu erklären und vereinfachte es so gut er konnte für den Jungen.

Der Ravenclaw nickte nachdenklich und sagte nichts, als ob er alles verstanden hätte, ohne Fragen stellen zu müssen. Vielleicht war das so; das Kind von James Potter und Lily Evans, dem Schulleiter und der Schulleiterin von Hogwarts, würde kaum dumm sein. Das Kind schien auf jeden Fall hochintelligent zu sein, soweit er für die kurze Zeit, die sie im Januar miteinander gesprochen hatten das einschätzen konnte, obwohl damals Remus das meiste Reden übernommen hatte. (Da war auch diese Sache mit dem Zaubergamot, über die Remus Gerüchte gehört hatte, aber Remus glaubte kein einziges Wort davon, genauso wenig wie er es geglaubt hatte, dass James seinen Sohn mit Mollys Jüngster verlobt hatte.)

„Da ist das Denkmal“, sagte Remus und deutete vor ihnen her.

Harry ging neben Mr~Lupin in Richtung des Obelisken aus schwarzem Marmor und dachte im Stillen nach. Es schien Harry, dass dieses Abenteuer im Wesentlichen fehlgeleitet war; er hatte keine Verwendung für Trauerbegleitung, das war nicht Harrys gewählter Weg. Soweit es Harry betraf, waren die fünf Stadien der Trauer Wut, Reue, Entschlossenheit, Forschung und Auferstehung. (Nicht, dass es für die üblichen 'fünf Phasen der Trauer' irgendeinen experimentellen Beweis gäbe, von dem Harry jemals gehört hätte.) Aber Mr~Lupin schien zu aufrichtig, um abzulehnen; und der Besuch bei James und Lily war etwas, das Harry nicht ablehnen sollte. So ging Harry, sich seltsam distanziert fühlend, schweigend durch ein Stück, dessen Skript er nicht lesen wollte. Harry war gesagt worden, dass er für diese Reise nicht den Tarnumhang tragen sollte, damit Mr~Lupin ihn im Auge behalten konnte. Harry war sich sicher, dass Dumbledore, oder sowohl Dumbledore als auch Mad-Eye Moody, ihnen unsichtbar folgten, um zu sehen, ob jemand den Köder schluckte. Auf keinen Fall hätte man Harry mit nur Remus Lupin als Wächter aus Hogwarts herausgelassen. Harry erwartete allerdings nicht, dass etwas passieren würde. Er hatte nichts gesehen, was der Hypothese widersprach, dass die ganze Gefahr von Hogwarts und nur von Hogwarts ausging.

Als die beiden sich dem Dorfzentrum näherten, verwandelte sich der Marmorobelisk in - Harry holte tief Luft. Er hatte eine heroische Pose von James Potter mit dem Zauberstab gegen Lord Voldemort und Lily Potter mit ausgestreckten Armen vor der Krippe erwartet. Stattdessen sah man einen Mann mit ungepflegtem Haar und Brille, und eine Frau mit offenem Haar und einem Baby im Arm, und das war alles.

„Es sieht sehr…normal aus“, sagte Harry und spürte ein seltsames Kratzen in seinem Hals.

„Madam Longbottom und Professor Dumbledore haben ein Machtwort gesprochen“, sagte Mr~Lupin, der mehr auf Harry als auf das Denkmal blickte. „Sie sagten, dass man sich an die Potters so erinnern sollte, wie sie gelebt haben, nicht wie sie gestorben sind.“

Harry betrachtete die Statue und dachte nach. Es war sehr seltsam, sich selbst als Baby aus Stein zu sehen, ohne Narbe auf der Stirn. Es war ein Blick auf ein alternatives Universum, eines, in dem Harry James Potter (ohne Evans-Verres) ein intelligenter, aber gewöhnlicher Zauberschüler wurde, vielleicht in Gryffindor sortiert wie seine Eltern. Ein Harry Potter, der als ordentlicher junger Zauberer aufwuchs, der nur ein wenig von der Wissenschaft wusste, weil seine Mutter eine Muggelgeborene war. Letztendlich ändert sich… ziemlich viel. James und Lily hätten ihren Sohn nicht mit dem großgezogen, was Professor Quirrell Ehrgeiz genannt hätte und was Professor Verres-Evans das gemeinsame Streben genannt hätte. Seine leiblichen Eltern hätten ihn sehr geliebt, und das wäre für niemanden auf der Welt außer Harry eine große Hilfe gewesen. Wenn jemand ihren Tod rückgängig gemacht hätte—

„Du warst ihr Freund“, sagte Harry und drehte sich zu Lupin um.

„Für eine lange Zeit, seit Sie Kinder waren.“

Mr~Lupin nickte stumm.

Professor Quirrells Stimme hallte in Harrys ungefährer Erinnerung wieder: \emph{Der wahrscheinliche Unterschied ist nicht, dass du dich mehr sorgst. Vielmehr ist es so, dass du, da du ein logischeres Wesen als sie bist, nur dir bewusst ist, dass die Rolle des Freundes dies von dir verlangen sollte…}

„Als Lily und James starben“, sagte Harry, „haben Sie da überhaupt daran gedacht, ob es vielleicht einen magischen Weg gibt, sie zurückzuholen? Wie bei Orpheus und Eurydike? Oder die, was war es, Elrin-Brüder?“

„Es gibt keine Magie, die den Tod ungeschehen machen kann“, sagte Mr~Lupin leise. „Es gibt einige Geheimnisse, die die Zauberei nicht berühren kann.“

„Haben Sie gedanklich überprüft, was Sie zu wissen glaubten, wie Sie es zu wissen glaubten und wie hoch die Wahrscheinlichkeit dieser Schlussfolgerung war?“

„Was?“, sagte Mr~Lupin. „Könntest du das wiederholen, Harry?“

„Ich will damit sagen, haben Sie überhaupt darüber nachgedacht?“

Mr~Lupin schüttelte den Kopf.

„Warum nicht?“

„Weil es schon geschehen ist, und es war vorbei“, sagte Remus Lupin sanft. „Denn wo auch immer James und Lily jetzt sind, sie würden sich wünschen, dass ich zum Wohle der Lebenden handle, nicht zum Wohle der Toten.“

Harry nickte stumm. Er war sich der Antwort auf diese Frage ziemlich sicher gewesen, bevor er sie gestellt hatte. Er hatte das Drehbuch bereits gelesen. Aber er hatte trotzdem gefragt, nur für den Fall, dass Mr~Lupin eine Woche lang darüber nachgedacht hatte, denn Harry hätte sich ja auch irren können.

Die sanfte Stimme des Verteidigungsprofessors schien in Harrys Gedanken zu sprechen.

\textbf{\emph{Sicherlich, wenn Lupin sich wirklich sorgte, würde er für etwas so Einfaches wenigstens fünf Minuten nachdenken, bevor er aufgab, und keine besondere Anweisung brauchen..}}.

\emph{Doch, das würde er,} antwortete Harry der mentalen Stimme. \emph{Menschliche Wesen würden nicht plötzlich eine solche Fähigkeit erlangen, nur weil sie sich sorgten. Ich habe davon erfahren, weil ich Bücher in der Bibliothek gelesen habe, die von einer riesigen wissenschaftlichen Gemeinschaft produziert wurden}—

Und dieser andere Teil von Harry sagte mit dieser sanften Stimme:

\textbf{\emph{Aber es gibt auch eine andere Hypothese, Mr~Potter, und sie passt auf eine viel weniger komplizierte Weise zu den Daten.}}

\emph{Nein, tut sie nicht! Woher sollen die Leute überhaupt wissen, was sie vorgeben sollen, wenn es niemanden interessieren würde? Sie wissen es nicht besser. Das ist es, was du beobachtest.}

Die beiden gingen weiter auf ein bestimmtes Haus zu, vorbei an einer langen Reihe von bewohnten Zaubererhäuschen und anderen mit Weinreben überwucherten Häusern. Schließlich kamen sie zu dem Haus, dessen Dach zur Hälfte weggesprengt war und dessen grüne Blätter ins Innere wuchsen; dahinter säumte eine schulterhohe, wild wachsende Hecke den Gehweg, und ein schmales Metalltor (Mr~Hagrid war wahrscheinlich direkt darüber getreten, da er nicht hindurchpasste).

Die Lücke im Dach sah aus, als hätte ein riesiges Maul einen kreisrunden Biss in das Haus gemacht, so dass Holzstacheln, die vielleicht Stützbalken waren, herausragten. Auf der rechten Seite stand ein einzelner Schornstein noch aufrecht, nicht vom Riesenbiss zerfressen, aber ohne seine frühere Stütze gefährlich schief. Die Fenster waren zersplittert. Wo eine Eingangstür hätte sein sollen, waren nur noch Holzsplitter. An diesen Ort war Lord Voldemort gekommen, lautlos, mit weniger Geräusch als das tote Laub, das über den Bürgersteig schlitterte…

Remus Lupin legte eine Hand auf Harrys Schulter.

„Berühre das Tor“, drängte Mr~Lupin.

Harry streckte eine Hand aus und tat es. Wie eine schnell wachsende Blume brach ein Schild aus dem Unkrautgewirr im Boden hinter dem Tor hervor, ein hölzernes Schild mit goldenen Buchstaben, und darauf stand:

\emph{An dieser Stelle, in der Nacht des 31. Oktober 1981, verloren Lily und James Potter ihr Leben. Sie wurden von ihrem Sohn überlebt, Harry Potter, dem einzigen Zauberer, der jemals dem Tötungsfluch widerstanden hat, dem Jungen, der überlebte, der die Macht von Du-weißt-schon-wem gebrochen hat.}

\emph{Dieses Haus wurde in seinem zerstörten Zustand belassen, als Denkmal für die Potters, als Erinnerung an ihr Opfer.}

In einem leeren Raum unter den goldenen Buchstaben waren andere Botschaften geschrieben, Dutzende von ihnen, magische Tinte, die an die Oberfläche stieg und hell genug schimmerte, um gelesen zu werden, bevor sie verblasste und anderen Botschaften Platz machte.

\emph{Mein Gideon ist also gerächt. Ich danke dir, Harry Potter.}

\emph{Lebe wohl, wo immer du bist.}

\emph{Wir werden immer in der Schuld der Potters stehen.}

\emph{Oh James, oh Lily, es tut mir leid.}

\emph{Ich hoffe, du bist am Leben, Harry Potter.}

\strut

\emph{Es gibt immer einen Preis. Ich wünschte, unsere letzten Worte wären netter gewesen, James.}

\emph{Es tut mir leid.}

\emph{Es gibt immer eine Morgendämmerung nach der Nacht.}

\emph{Ruh dich aus, Lily. Gott segne dich, Junge-der-lebte. Du warst unser Wunder.}

„Ich schätze—“ sagte Harry. „Ich schätze, das ist es, was die Leute tun - anstatt zu versuchen, es besser zu machen—“ Harry hielt inne. Der Gedanke schien diesem Ort nicht würdig. Er sah auf und erblickte Remus Lupin, der ihn mit einem Blick ansah, der so sanft war, dass Harry seine Augen von dem gesprengten und zerbrochenen Dach wegzog. \emph{Du warst unser Wunder.} Harry hatte das Wort „Wunder“ immer in dem Zusammenhang gehört, dass es so etwas im natürlichen Universum nicht gab. Und doch, als er das zerstörte Haus betrachtete, wusste er plötzlich genau, was das Wort bedeutete, der Hauch von Gnade, der unerklärlich war, der Segen, der unerklärlich war. Der Dunkle Lord hatte fast gesiegt, und dann war in einer Nacht all die Dunkelheit und der Schrecken zu Ende gewesen, eine Erlösung ohne Grund, ein plötzliches Auftauchen aus der Dunkelheit, und selbst jetzt wusste niemand, warum - wenn Lily Potter über ihre Konfrontation mit Lord Voldemort hinaus gelebt hätte, hätte sie sich so gefühlt, als sie ihr Baby lebendig danach sah.

„Lass uns gehen“, flüsterte der kleine Junge, zehn Jahre später.

Sie gingen.

Der Eingang des Friedhofs wurde von einem schlosslosen Tor bewacht, von der Art, die Tiere fernhielt, mit einem Platz zum Stehen, während man die Tür von einer Seite des Standplatzes zur anderen schob. Remus zückte seinen Zauberstab (Harry hatte seinen bereits in der Hand) und es gab ein kurzes Verschwimmen, als sie hindurch traten. Einige der Steine, die aus dem Boden ragten, sahen so alt aus wie die Mauer in Oxford, von der sein Vater gesagt hatte, sie sei etwa tausend Jahre alt. \emph{Hallie Fleming,} sagte der erste Stein, den Harry sah, in einer Schnitzerei, die fast unsichtbar mit der Erosion der Zeit verblasste. \emph{Vienna Wood}, sagte ein anderer. Es war schon lange her, dass Harry einen Friedhof besucht hatte. Sein Geist war noch kindlich gewesen, als er das letzte Mal einen betreten hatte, lange bevor er in den Schatten des Todes getreten war. Jetzt hierher zu kommen war… seltsam, und traurig, und rätselhaft, und das passiert schon so lange, \emph{warum haben die Zauberer nicht versucht, es zu stoppen, warum setzen sie nicht ihre ganze Kraft dafür ein, wie die Muggel für die medizinische Forschung, nur mehr, die Zauberer haben mehr Grund zur Hoffnung…}

„Die Dumbledores haben auch in Godrics Hollow gewohnt?“ sagte Harry, als sie an einem relativ neuen Steinpaar vorbeikamen, auf dem Kendra Dumbledore und Ariana Dumbledore standen.

„Für eine lange, lange Zeit“, sagte Mr~Lupin.

Sie gingen weiter in den Friedhof hinein, weit zum Ende hin, vorbei an vielen Toten, die betrauert worden waren. Dann zeigte Mr~Lupin auf einen verbundenen Doppelgrabstein, aus Marmor, noch weiß und ungealtert.

„Gibt es dort Botschaften?“ sagte Harry. Er wollte sich nicht weiter mit der Art und Weise beschäftigen, wie andere Menschen mit dem Tod umgingen.

Mr~Lupin schüttelte den Kopf. Sie gingen auf die miteinander verbundenen weißen Steine zu. Und standen vor—

„Was ist das?“ flüsterte Harry. „Wer… wer hat das geschrieben?!“

\textbf{JAMES POTTER}

\textbf{GEBOREN 27. MÄRZ 1960}

\textbf{GESTORBEN 31. OKTOBER 1981}

„Was geschrieben?“, fragte Mr~Lupin verwirrt.

\textbf{LILY POTTER}

\textbf{GEBOREN 30. JANUAR 1960}

\textbf{GESTORBEN 31. OKTOBER 1981}

„Das!“ rief Harry. „Die Inschrift!“

Tränen stiegen in Harrys Augen auf, über die unpassende und unerklärliche Helligkeit, den Hauch von Anmut, wo keine Anmut hätte sein sollen, den geheimnisvollen Segen, Tränen stiegen auf bei

\textbf{DER LETZTE FEIND DER ZERSTÖRT WERDEN WIRD IST DER TOT}

„Das?“ Mr~Lupin sagte. „Das ist das… Motto, so könnte man es wohl nennen, der Potters. Obwohl ich nicht glaube, dass es jemals etwas so Formelles war. Nur ein Spruch, der vor langer, langer Zeit überliefert wurde…“

„Das - das—“ Harry krabbelte hinunter, um neben dem Grab zu knien, berührte die Inschrift mit einer zitternden Hand. „Wie? So etwas kann nicht einfach genetisch sein—“ Dann sah Harry, was die Tränen verwischt hatten, die schwache Einritzung einer Linie, innerhalb eines Kreises, innerhalb eines Dreiecks. Das Symbol der Heiligtümer des Todes. \emph{Und Harry verstand}. „Sie haben es versucht“, flüsterte Harry.

Die 3 Peverell-Brüder. Hatten sie jemanden verloren, der ihnen wertvoll war? Hatte es damit begonnen? „Mit ihrem ganzen Leben haben sie es versucht. Und sie haben Fortschritte gemacht—“ \emph{Der Unsichtbarkeitsumhang, der die Sicht der Dementoren ausschalten konnte}. „- aber ihre Forschung war noch nicht beendet—“ \emph{Sich vor dem Schatten des Todes zu verstecken, heißt nicht, den Tod selbst zu besiegen. Der Stein der Auferstehung konnte niemanden wirklich zurückbringen. Der Elderstab konnte einen nicht vor dem Alter schützen}. „- also gaben sie die Mission an ihre Kinder weiter, und deren Kindeskinder.“ \emph{Generation um Generation. Bis es zu mir kam. Könnte die Zeit so widerhallen, sich reimen, zwischen so weit in der Zukunft und so weit in der Vergangenheit? Das konnte kein Zufall sein, oder? Nicht diese Nachricht, nicht an diesem Ort. Meine Familie. Sie waren es wirklich, meine Mutter und mein Vater.}

„Es bedeutet nicht, die Toten auferstehen zu lassen, Harry“, sagte Mr~Lupin. „Es bedeutet, den Tod zu akzeptieren und so über den Tod hinaus zu sein, ihn zu meistern.“

„Hat James dir das gesagt?“ sagte Harry, seine Stimme war seltsam.

„Nein“, sagte Mr~Lupin, „aber—“

„Gut.“ Harry erhob sich langsam von dort, wo er gekniet hatte, und fühlte sich, als würde er eine Sonne auf seine Schultern drücken, die die Morgendämmerung über den Horizont heben würde.

\emph{Natürlich haben es andere Zauberer versucht. Ich bin nicht einzigartig. Ich war nie allein. Diese Gefühle in meinem Herzen, sie sind nichts Besonderes, weder in der Welt der Zauberer noch in der der Muggel}.

„Harry, dein Zauberstab!“ Es lag eine plötzliche Aufregung in Mr~Lupins Stimme, und als Harry seinen Zauberstab hob, um ihn genau zu betrachten, sah er, dass er ganz schwach in einem silbernen Licht schimmerte, das aus dem Holz hervorquoll. „Wirke den Patronus-Zauber!“, drängte Mr~Lupin. „Versuch ihn noch mal, Harry!“

\emph{Oh, richtig. Soweit Mr~Lupin weiß, kann ich das nicht—}

Harry lächelte und lachte sogar ein wenig.

„Das sollte ich lieber nicht“, sagte Harry. „Wenn ich versuchen würde, den Zauber in diesem Geisteszustand auszuführen, würde er mich wahrscheinlich umbringen.“

„Was?1“, sagte Mr~Lupin. „Der Patronus-Zauber bewirkt so etwas nicht!“

Harry James Potter-Evans-Verres hob seine linke Hand, immer noch lachend, und wischte sich noch ein paar Tränen weg. „Wissen Sie, Mr~Lupin“, sagte Harry, „es bedarf wirklich einer barocken Interpretation, um zu glauben, dass jemand herumläuft, darüber nachdenkt, dass der Tod einfach etwas ist, das wir alle akzeptieren müssen, und seinen Gemütszustand mitteilt, indem er sagt: '\emph{Der letzte Feind, der zerstört werden soll, ist der Tod.}' Vielleicht dachte jemand anderes, dass es sich poetisch anhört und hat den Satz aufgegriffen und versucht, ihn anders zu interpretieren, aber wer auch immer ihn zuerst gesagt hat, mochte den Tod nicht besonders.“

Manchmal verwunderte es Harry, dass die meisten Leute nicht einmal zu bemerken schienen, wenn sie etwas in das 180-Grad-Gegenstück zur ersten offensichtlichen Lesart verdrehten. Es konnte nicht an der rohen Gehirnleistung liegen, die Leute konnten die offensichtliche Lesart der meisten anderen englischen Sätze erkennen. „Außerdem bezieht sich '\emph{zerstört werden soll}' auf eine Veränderung des zukünftigen Zustands, es kann also nicht um die Art und Weise gehen, wie die Dinge jetzt sind.“

Remus Lupin starrte ihn mit großen Augen an.

„Du bist eindeutig das Kind von James und Lily“, sagte der Mann und klang ziemlich schockiert.

„Ja, das bin ich“, sagte Harry.

\emph{Aber das war nicht genug, er musste noch etwas tun,} also hob Harry seinen Zauberstab in die Luft und sagte, seine Stimme so fest, wie er sie machen konnte: „Ich bin Harry James Potter-Evans-Verres, der Sohn von Lily und James, aus dem Hause Potter, und ich nehme die Bestimmung meiner Familie an. Der Tod ist mein Feind, und ich werde ihn besiegen.“

\emph{Thrayen beyn Peverlas soona ahnd thrih heera toal thissoom Dath bey yewoonen.}

„Was?“ Harry sagte laut. Die Worte waren in seinem Bewusstseinsstrom aufgetaucht, wie aus seinen eigenen Gedanken, unerklärt.

„Was war das?“, sagte Remus Lupin zur gleichen Zeit. Harry drehte sich um und scannte den Friedhof, aber er sah nichts. Neben ihm tat Mr~Lupin das Gleiche. Keiner von ihnen bemerkte den großen, wie von tausend Jahren abgenutzten Stein, auf dem eine Linie in einem Kreis in einem Dreieck leuchtete, die so schwach silbern war, wie das Licht, das von Harrys Zauberstab ausgegangen war, unsichtbar in dieser Entfernung unter der noch hellen Sonne.

Einige Zeit später: „Nochmals vielen Dank, Mr~Lupin“, sagte Harry, als der hochgewachsene, leicht vernarbte Mann im Begriff war, sich wieder zu verabschieden. „Obwohl ich wirklich wünschte, Sie hätten nicht—“

„Professor Dumbledore hat gesagt, dass ich uns mit einem Portschlüssel nach Hogwarts zurückbringen soll, wenn irgendetwas Ungewöhnliches passiert, egal ob es wie ein Angriff aussieht oder nicht“, sagte Mr~Lupin fest. „Was äußerst vernünftig ist.“

Harry nickte. Und dann, nachdem er sich diese Frage sorgfältig für den Schluss aufgespart hatte: „Haben Sie eine Ahnung, was die Worte bedeuten?“

„Wenn ich es wüsste, würde ich es dir nicht sagen“, sagte Mr~Lupin und sah ziemlich ernst aus. „Schon gar nicht ohne die Erlaubnis von Professor Dumbledore. Ich kann deinen Eifer verstehen, aber du solltest nicht versuchen, irgendwelche uralten Geheimnisse der Potters aufzudecken, bis du erwachsen bist. Das heißt, nachdem du deine UTZs bestanden hast, Harry, oder zumindest deine ZAGs. Und ich glaube immer noch, dass du eine völlig falsche Vorstellung davon hast, was dein Familienmotto aussagen soll!“

Harry nickte, seufzte innerlich, und verabschiedete sich von Mr~Lupin.

Harry ging zurück durch Hogwarts, zum Ravenclaw-Turm, und fühlte sich fremd und gestärkt. Er hätte das alles nicht erwartet, aber es war gut so. Er ging durch den Ravenclaw-Gemeinschaftsraum, auf dem Weg zu seinem Schlafsaal.

\emph{Dort kam die leuchtende Kreatur zu ihm, die im Kerzenlicht des Ravenclaw-Gemeinschaftsraumes sanft weiß schimmerte, als sie aus dem Nichts herausschlüpfte, die silberne Schlange.}

\_\_\_\_\_\_\_\_\_\_\_\_\_\_\_\_\_\_\_\_\_\_\_\_\_\_\_\_\_\_\_\_\_\_\_\_\_\_\_\_\_\_\_\_\_\_\_\_\_\_\_\_\_\_\_\_\_\_\_\_\_\_\_\_\_\_\_\_

\emph{\textbf{\emph{Þregen béon Pefearles suna and þrie hira tól þissum Déað béo gewunen.}}}

\emph{Drei sollen Peverells Söhne sein und drei ihre Artefakte, mit denen der Tod besiegt werden soll.—}

\emph{Gesprochen in Anwesenheit der drei Peverell-Brüder, in einer kleinen Taverne am Rande dessen, was später Godric's Hollow genannt werden sollte.}

