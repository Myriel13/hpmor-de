

\hypertarget{vorgeben-weise-zu-sein-teil-2}{% \section{40. Vorgeben weise zu sein Teil 2}\label{vorgeben-weise-zu-sein-teil-2}}

\textbf{\uline{Vorgeben weise zu sein, Teil 2}}

Harry hielt die Teetasse genau so, wie Professor Quirrell es dreimal hatte demonstrieren müssen, und nahm einen kleinen, vorsichtigen Schluck. Auf der anderen Seite des langen, breiten Tisches, der das Herzstück von Marys Zimmer war, nahm Professor Quirrell einen Schluck aus seiner eigenen Tasse, \emph{was viel natürlicher und eleganter aussah}.

Der Tee selbst war etwas, dessen Namen Harry nicht einmal aussprechen konnte, oder zumindest hatte Professor Quirrell ihn jedes Mal korrigiert, wenn Harry versucht hatte, die chinesischen Worte zu wiederholen, bis Harry schließlich aufgegeben hatte. Harry hatte sich das letzte Mal dazu durchgerungen, einen Blick auf die Rechnung zu werfen, und Professor Quirrell hatte ihn damit davonkommen lassen. Er hatte den Drang verspürt, zuerst einen Comed-Tea zu trinken. Selbst wenn man das in Betracht zieht, war Harry immer noch wie vom Donner gerührt.

Und es schmeckte für ihn immer noch wie, nun ja, Tee. Es gab einen leisen, nagenden Verdacht in Harrys Kopf, dass Professor Quirrell das wusste und absichtlich lächerlich teuren Tee kaufte, den Harry nicht zu schätzen wusste, nur um sich mit ihm anzulegen. Professor Quirrell selbst mochte ihn vielleicht gar nicht so sehr. Vielleicht mochte niemand diesen Tee wirklich, und der einzige Sinn des Tees war, lächerlich teuer zu sein und dem Opfer das Gefühl zu geben, es nicht zu schätzen.

Vielleicht war es sogar ein ganz normaler Tee, den man nur mit einem bestimmten Code bestellt hatte, und sie setzten einen falschen, gigantischen Preis auf die Rechnung…

Professor Quirrells Ausdruck war gezeichnet und nachdenklich.

"Nein", sagte Professor Quirrell, "du hättest dem Schulleiter nicht von deinem Gespräch mit Lord Malfoy erzählen sollen. Bitte versuchen Sie das nächste Mal, schneller zu denken, Mr~Potter."

"Es tut mir leid, Professor Quirrell", sagte Harry kleinlaut. "Ich kann es immer noch nicht verstehen."

Es gab Zeiten, in denen Harry sich wie ein Hochstapler fühlte, der in Professor Quirrells Gegenwart so tat, als sei er schlau.

"Lord Malfoy ist der Gegenspieler von Albus Dumbledore", sagte Professor Quirrell.

"Zumindest für die gegenwärtige Zeit. Ganz Britannien ist ihr Schachbrett, alle Zauberer ihre Figuren. Bedenken Sie: Lord Malfoy hat gedroht, alles hinzuschmeißen, sein Spiel aufzugeben und sich an Ihnen zu rächen, wenn Mr~Malfoy verletzt wird. In diesem Fall, Mr~Potter…?"

Es dauerte weitere lange Sekunden, bis Harry es begriffen hatte, aber es war klar, dass Professor Quirrell keine weiteren Andeutungen machen würde, nicht dass Harry sie wollte.

Dann stellte Harrys Verstand endlich die Verbindung her, und er runzelte die Stirn.

"Dumbledore tötet Draco, lässt es so aussehen, als hätte ich es getan, und Lucius opfert sein Spiel gegen Dumbledore, um an mich heranzukommen? Das… scheint nicht der Stil des Schulleiters zu sein, Professor Quirrell…"

Harrys Gedanken blitzten zurück zu einer ähnlichen Warnung von Draco, die Harry dazu veranlasst hatte, das Gleiche zu sagen.

Professor Quirrell zuckte mit den Schultern und nippte an seinem Tee. Harry nippte an seinem eigenen Tee und saß schweigend da. Die Tischdecke, die über dem Tisch ausgebreitet war, hatte ein sehr friedliches Muster, das zunächst wie ein einfacher Stoff aussah, aber wenn man sie lange genug anstarrte oder lange genug schwieg, fing man an, ein schwaches Blumenmuster darauf schimmern zu sehen; die Vorhänge des Raumes hatten ihr Muster passend dazu verändert und schienen wie in einer stillen Brise zu schimmern. Professor Quirrell war an diesem Samstag in einer nachdenklichen Stimmung, und so war es auch Harry, und Marys Zimmer, so schien es, war dies nicht entgangen.

"Professor Quirrell", sagte Harry plötzlich, "gibt es ein Leben nach dem Tod?"

Harry hatte die Frage mit Bedacht gewählt.

\emph{Nicht: "Glauben Sie an ein Leben nach dem Tod?", sondern einfach: "Gibt es ein Leben nach dem Tod?}

Was die Leute wirklich glaubten, kam ihnen gar nicht wie ein Glaube vor.

Die Leute sagten nicht: "\emph{Ich glaube fest daran, dass der Himmel blau ist!'} Sie sagten einfach: '\emph{Der Himmel ist blau'}. Ihr wahres inneres Weltbild fühlte sich für Sie einfach so an, wie die Welt war…

Der Verteidigungsprofessor hob seine Tasse wieder an die Lippen, bevor er antwortete.

Sein Gesicht war nachdenklich.

"Wenn das so ist, Mr~Potter", sagte Professor Quirrell, "dann haben ziemlich viele Zauberer bei ihrer Suche nach Unsterblichkeit viel Mühe verschwendet."

"Das ist nicht wirklich eine Antwort", bemerkte Harry.

Er hatte inzwischen gelernt, so etwas zu bemerken, wenn er mit Professor Quirrell sprach. Professor Quirrell setzte seine Teetasse mit einem kleinen, hochtönenden Tackern auf seiner Untertasse ab.

"Einige dieser Zauberer waren einigermaßen intelligent, Mr~Potter, Sie können also davon ausgehen, dass die Existenz eines Lebens nach dem Tod nicht offensichtlich ist.

Ich habe mich selbst mit der Angelegenheit befasst. Es gab viele Behauptungen von der Art, wie man sie von Hoffnung und Furcht erwarten würde.

Unter den Berichten, deren Wahrheitsgehalt nicht angezweifelt wird, gibt es nichts, was nicht das Ergebnis von reiner Zauberei sein könnte.

Es gibt bestimmte Geräte, die angeblich mit den Toten kommunizieren, aber diese, so vermute ich, projizieren nur ein Bild aus dem Geist; das Ergebnis scheint ununterscheidbar von der Erinnerung, weil es Erinnerung ist.

Die angeblichen Geister verraten keine Geheimnisse, die sie zu Lebzeiten kannten oder nach dem Tod hätten erfahren können und die dem Träger nicht bekannt sind -"

"Deshalb ist der Stein der Auferstehung auch nicht das wertvollste magische Artefakt der Welt", sagte Harry.

"Genau", sagte Professor Quirrell, "obwohl ich zu einer Chance, ihn auszuprobieren, nicht nein sagen würde."

Ein trockenes, dünnes Lächeln lag auf seinen Lippen; und etwas Kälteres, Entfernteres, in seinen Augen. "Sie haben auch mit Dumbledore darüber gesprochen, nehme ich an."

Harry nickte. Die Vorhänge nahmen ein schwach blaues Muster an, und auf dem Tischtuch schien nun ein undeutliches Gefüge von kunstvollen Schneeflocken sichtbar zu werden. Professor Quirrells Stimme klang sehr ruhig.

"Der Schulleiter kann sehr überzeugend sein, Mr~Potter. Ich hoffe, er hat Sie nicht überredet."

"Auf keinen Fall", sagte Harry. "Er hat mich nicht eine Sekunde lang überredet."

"Das will ich nicht hoffen", sagte Professor Quirrell, immer noch in diesem sehr ruhigen Ton.

"Ich wäre äußerst beunruhigt, wenn ich herausfände, dass der Schulleiter dich davon überzeugt hat, dein Leben für irgendeinen dummen Plan wegzuwerfen, indem er dir sagt, der Tod sei das nächste große Abenteuer."

"Ich glaube nicht, dass der Schulleiter das selbst geglaubt hat", sagte Harry.

Er nippte wieder an seinem eigenen Tee.

"Er fragte mich, was ich mit der Ewigkeit anfangen könnte, gab mir den üblichen Spruch, dass sie langweilig sei, und er schien keinen Konflikt zwischen dem und seinem eigenen Anspruch, eine unsterbliche Seele zu haben, zu sehen. Tatsächlich hielt er mir einen ganzen langen Vortrag darüber, wie schrecklich es sei, Unsterblichkeit zu wollen, bevor er behauptete, eine unsterbliche Seele zu haben. Ich kann mir nicht ganz vorstellen, was in seinem Kopf vorgegangen sein muss, aber ich glaube nicht, dass er tatsächlich ein mentales Modell davon hatte, dass er für immer im Jenseits weiterleben würde…"

Die Temperatur des Raumes schien zu sinken.

"Sie nehmen wahr", sagte eine Stimme wie Eis vom anderen Ende des Tisches, "dass Dumbledore nicht wirklich glaubt, was er sagt. Es ist nicht so, dass er seine Prinzipien kompromittiert hat. Es ist vielmehr so, dass er sie von Anfang an nicht hatte. Werden Sie jetzt etwa zynisch, Mr~Potter?"

Harry hatte seinen Blick auf seine Teetasse gesenkt.

"Ein wenig", sagte Harry zu seinem möglicherweise ultrahochwertigen, vielleicht lächerlich teuren chinesischen Tee.

"Ich bin auf jeden Fall ein bisschen frustriert von … was auch immer in den Köpfen der Leute schief läuft."

"Ja", sagte die eisige Stimme. "Ich finde es auch frustrierend."

"Gibt es eine Möglichkeit, die Leute dazu zu bringen, das nicht zu tun?", sagte Harry zu seiner Teetasse.

"Es gibt tatsächlich einen gewissen nützlichen Zauber, der das Problem löst."

Harry blickte daraufhin hoffnungsvoll auf und sah ein kaltes, kaltes Lächeln auf dem Gesicht des Verteidigungsprofessors. Dann verstand Harry es.

"Ich meine, außer Avada Kedavra."

Der Verteidigungsprofessor lachte. Harry tat es nicht.

"Jedenfalls", sagte Harry hastig, "habe ich schnell genug nachgedacht, um die offensichtliche Idee mit dem Stein der Auferstehung nicht vor Dumbledore vorzubringen. Haben Sie jemals einen Stein mit einer Linie gesehen, in einem Kreis, in einem Dreieck?"

Die tödliche Kälte schien sich zurückzuziehen, in sich selbst zu falten, als der gewöhnliche Professor Quirrell zurückkehrte.

"Nicht, dass ich mich erinnern könnte", sagte Professor Quirrell nach einer Weile, ein nachdenkliches Stirnrunzeln auf seinem Gesicht.

"Das ist der Stein der Auferstehung?"

Harry stellte seine Teetasse beiseite, dann zeichnete er das Symbol, das er auf der Innenseite seines Umhangs gesehen hatte, auf seine Untertasse. Und bevor Harry seinen eigenen Zauberstab zücken konnte, um den Schwebezauber zu wirken, schwebte die Untertasse pflichtbewusst über den Tisch zu Professor Quirrell. Harry wollte wirklich dieses zauberstablose Zeug lernen, aber das lag anscheinend weit über seinem derzeitigen Lehrplan. Professor Quirrell betrachtete Harrys Teeuntertasse einen Moment lang, dann schüttelte er den Kopf; und einen Moment später schwebte die Untertasse zurück zu Harry. Harry stellte seine Teetasse zurück auf die Untertasse und bemerkte dabei abwesend, dass das Symbol, das er gezeichnet hatte, verschwunden war.

"Wenn Sie zufällig einen Stein mit diesem Symbol sehen", sagte Harry, "und er spricht tatsächlich mit dem Jenseits, lassen Sie es mich wissen. Ich habe ein paar Fragen an Merlin oder jemanden, der in Atlantis dabei war."

"Ganz recht", sagte Professor Quirrell.

Dann hob der Verteidigungsprofessor seine Teetasse wieder an und kippte sie zurück, als wolle er das Letzte, was darin war, austrinken.

"Übrigens, Mr~Potter, ich fürchte, wir müssen unseren heutigen Besuch in der Winkelgasse abkürzen. Ich hatte gehofft, es würde - aber egal. Lassen Sie es dabei bewenden, dass ich heute Nachmittag noch etwas anderes erledigen muss."

Harry nickte und trank seinen eigenen Tee aus, dann erhob er sich gleichzeitig mit Professor Quirrell von seinem Platz.

"Eine letzte Frage", sagte Harry, als sich Professor Quirrells Mantel von der Garderobe abhob und auf den Verteidigungsprofessor zu schwebte.

"Die Magie ist frei in der Welt, und ich traue meinen Vermutungen nicht mehr so sehr wie früher. Glauben Sie also nach Ihrer eigenen Einschätzung und ohne Wunschdenken, dass es ein Leben nach dem Tod gibt?"

"Wenn ich das täte, Mr~Potter", sagte Professor Quirrell, während er seinen Mantel zuckte,

"wäre ich dann noch hier?"

