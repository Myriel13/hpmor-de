

\hypertarget{die-wahrheit-teil-2}{% \section{104. Die Wahrheit, Teil 2}\label{die-wahrheit-teil-2}}

\textbf{\uline{Die Wahrheit, Teil 2}}

\textbf{\uline{Tom Riddle.}}

Die Worte schienen in Harrys Kopf widerzuhallen, lösten Resonanzen aus, die ebenso schnell wieder verklangen, gebrochene Muster, die versuchten, sich zu vervollständigen und scheiterten.

\emph{Tom Riddle, Tom Riddle….}

Es gab andere Prioritäten, die Harrys Aufmerksamkeit in Anspruch nahmen. Professor Quirrell richtete eine Waffe auf ihn. Und aus irgendeinem Grund hatte Lord Voldemort sie noch nicht abgefeuert.

Harrys Stimme kam eher krächzend heraus.

"Was ist es, was Sie von mir wollen?"

"Dein Tod…", sagte Professor Quirrell, "ist eindeutig nicht das, was ich sagen will, denn ich hatte genug Zeit, dich zu töten, wenn ich es gewollt hätte. Der schicksalhafte Kampf zwischen Lord Voldemort und dem Jungen-der-lebte ist ein Hirngespinst von Dumbledore. Ich weiß, wo das Haus deiner Familie in Oxford zu finden ist. Und ich bin mit dem Konzept von Scharfschützengewehren vertraut. Du wärst gestorben, bevor du jemals einen Zauberstab angefasst hättest. Ich hoffe, das ist dir klar, Tom?"

"Kristallklar", flüsterte Harry.

Sein Körper zitterte immer noch, seine Instinkte waren eher dazu geeignet, vor einem Tiger zu fliehen, als heikle Zaubersprüche zu wirken oder zu denken. Aber Harry konnte an eine Sache denken, die die Person, die eine Waffe auf ihn richtete, offensichtlich von ihm wollte, eine Frage, die diese Person darauf wartete, dass er sie stellte, und Harry tat es.

"Warum nennst du mich Tom?"

Professor Quirrell betrachtete ihn unverwandt.

"Warum ich dich Tom nenne? Antworte selbst. Dein Intellekt ist nicht alles, was ich mir erhofft habe, aber dafür sollte er ausreichen."

Harrys Mund schien die Antwort zu kennen, bevor sein Gehirn sich auf die Frage konzentrieren konnte.

"Tom Riddle ist dein Name. Unser Name. Das ist der, der Lord Voldemort ist, oder war, oder -- irgendwas."

Professor Quirrell nickte.

"Besser. Du hast den Dunklen Lord bereits besiegt, das einzige Mal, dass du das jemals tun wirst. Ich habe bereits alles bis auf einen Rest von Harry Potter vernichtet, den Unterschied zwischen unseren Geistern beseitigt und uns ermöglicht, in derselben Welt zu wohnen. Jetzt, da dir klar ist, dass der Kampf zwischen uns eine Lüge ist, könntest du vernünftig handeln, um deine eigenen Interessen durchzusetzen.

Oder du tust es nicht, aber dann…."

Die Waffe bewegte sich leicht nach vorne und verursachte Schweißperlen auf Harrys Stirn.

"Lass deinen Zauberstab fallen. Sofort."

Harry ließ ihn fallen.

"Geh von dem Zauberstab weg", sagte Professor Quirrell.

Harry gehorchte.

"Greif nach deinem Hals", sagte Professor Quirrell, "und nehm den Zeitumkehrer ab, wobei du ihn nur an der Kette berühren darfst. Lege den ihn auf den Boden und trete dann ebenfalls von ihm weg."

Auch das tat auch Harry.

Selbst in seinem Schockzustand suchte sein Verstand noch nach einer Möglichkeit, den Zeitumkehrer dabei zu drehen, eine plötzliche Bewegung, die zum Sieg führen würde; aber Harry wusste, dass Professor Quirrell sich bereits in Harrys Lage hineinversetzen würde und nach denselben Möglichkeiten suchte.

"Nimm deinen Beutel ab und lege ihn ebenfalls auf den Boden, dann tritt weg."

Harry tat es.

"Sehr gut", sagte der Verteidigungsprofessor. "Jetzt ist es an der Zeit, dass ich mir den Stein der Weisen besorge. Ich beabsichtigte, diese vier Erstklässler hierher zu bringen, die in geeigneter Weise ihrer jüngsten Erinnerungen beraubt sind, so dass sie sich noch an ihre ursprüngliche Bestimmung erinnern. Snape werde ich kontrollieren und diese Tür bewachen lassen. Nach getaner Arbeit werde ich Snape für den Verrat, den er an meiner anderen Identität begangen hat, töten. Die 3 Erbenkinder werde ich danach mit mir nehmen, um ihre zukünftige Loyalität zu formen. Und ich erzähle dir noch etwas. Ich habe Geiseln genommen. Ich habe ein Ritual in Gang gesetzt das Hunderte von Hogwartsschülern töten wird, darunter viele, die du Freunde nennst. Ich kann diesen Zauber mit dem Stein stoppen, wenn ich ihn erfolgreich beschaffe.

Wenn ich vorher unterbrochen werde oder den Zauber nicht stoppe, sterben Hunderte von Schülern."

Professor Quirrells Stimme war immer noch mild.

"Erkennst du noch irgendwelche Interessen, die für dich auf dem Spiel stehen, Junge? Ich würde lächeln, wenn du 'nein' sagen würdest, aber das ist zu viel zu hoffen."

"Ich würde gerne", schaffte Harry es zu sagen, durch das Entsetzen und den Herzschmerz und die Messer, die an einer emotionalen Verbindung schnitten, die wie lebendiges Fleisch schmerzte, als sie geschnitten wurde,

"dass du diese Dinge nicht tust, Professor."

\emph{Warum, Professor Quirrell, warum, warum musste es so kommen, ich will nicht, ich will nicht, ich will nicht, dass das passiert…}

"Nun gut", sagte Professor Quirrell. "Ich erteile dir die Erlaubnis, mir etwas anzubieten, das ich haben möchte."

Die Waffe gestikulierte einladend.

"Das ist ein seltenes Privileg, Kind. Lord Voldemort verhandelt normalerweise nicht um das, was er will."

Ein Teil von Harrys Verstand suchte krampfhaft nach etwas, irgendetwas, das für Lord Voldemort oder Professor Quirrell von größerem Wert sein könnte als Kindergeiseln oder Severus' Tod. Ein anderer Teil von ihm, der Teil, der nie aufgehört hatte zu denken, kannte die Antwort bereits.

"Du hast bereits eine Idee, was du von mir willst", sagte Harry durch die Krankheit und die blutenden Wunden in seiner Seele hindurch.

"Was ist es?"

"Deine Hilfe, um den Stein der Weisen zu beschaffen."

Harry schluckte. Er konnte nicht verhindern, dass seine Augen zur Waffe wanderten, dann wieder zu Professor Quirrells Gesicht. Ihm war bewusst, dass der Held in einem Märchenbuch \emph{"Nein"} sagen sollte, aber jetzt, wo er sich tatsächlich in einer Situation wie dieser befand, schien es keinen Sinn zu machen, \emph{"Nein"} zu sagen.

"Ich bin enttäuscht, dass du darüber nachdenken musst", sagte Professor Quirrell.

"Es ist doch ganz klar, dass du mir im Moment gehorchen solltest, da ich dir gegenüber im Vorteil bin. Ich habe dir diese Dinge besser beigebracht; in dieser Situation solltest du auf jeden Fall so tun, als würdest du verlieren. Du kannst nicht erwarten das du etwas gewinnst wenn du dich wehrst, außer Schmerzen. Du hättest auch daran denken müssen, dass es besser ist, früher zu antworten und mein Misstrauen nicht heraufzubeschwören."

Professor Quirrells Augen studierten ihn neugierig.

"Vielleicht hat Dumbledore deine Ohren mit Unsinn über edlen Trotz gefüllt? Ich finde solche Moralvorstellungen amüsant, da sie so leicht zu manipulieren sind. Ich versichere dir, dass ich Trotz moralisch schlechter erscheinen lassen kann, und du wärst gut beraten, dich zu fügen, bevor ich dir demonstriere, wie."

Die Pistole blieb auf Harry gerichtet; aber mit einem Wink von Professor Quirrells anderer Hand erhob sich Tracey Davis in die Luft, drehte sich träge, ihre Gliedmaßen weit ausgestreckt - - dann, selbst als neues Adrenalin auf Harrys Herz hämmerte, schwebte Tracey wieder nach unten.

"Wähle", sagte Professor Quirrell. "Das hier fängt an, meine Geduld zu strapazieren."

\emph{Ich hätte rechtzeitig etwas sagen sollen, er hätte Tracey die Beine abreißen können, nein, ich hätte es nicht tun sollen, der Schulleiter hat gesagt, ich darf Lord Voldemort nicht zeigen, dass ich etwas tue, wenn er meine Freunde bedroht, denn das wird ihn nur dazu bringen, noch mehr von ihnen zu bedrohen - nur ist das, was er vorhin gesagt hat, keine Drohung, es ist genau die Art von Dingen, die Lord Voldemort tut -}

Harry holte tief Luft. Welcher Teil von ihm auch immer auf Vollautomatik lief, schrie den Rest seines Verstandes an, dass er es sich nicht leisten konnte, im Schock zu verharren. Schocks waren von endlicher Dauer, die Neuronen feuerten trotzdem weiter, der einzige Grund, warum Harrys Verstand sich abschalten würde, während sein Gehirn weiterlief, war, dass Harrys Selbstmodell glaubte, sein Verstand würde sich abschalten -

"Ich will deine Geduld nicht auf die Probe stellen", sagte Harry.

Seine Stimme war brüchig. Das war gut so. Wenn er so klang, als stünde er noch unter Schock, bedeutete das, dass Lord Voldemort ihm vielleicht mehr Zeit geben würde.

"Aber wenn Lord Voldemort den Ruf hatte, seine Abmachungen einzuhalten, dann weiß ich nichts davon."

"Eine offensichtliche Sorge", sagte Professor Quirrell. "Es gibt eine einfache Antwort, und ich hätte sie dir auf jeden Fall aufgezwungen. Schlangen können nicht lügen. Und da ich eine enorme Abneigung gegen Dummheit habe, schlage ich vor, dass du nicht so etwas sagst wie: 'Was meinst du?' Du bist klüger als das, und ich habe keine Zeit für Gespräche wie Sie sich sich gewöhnliche Menschen gegenseitig aufzwingen."

Harry schluckte. \emph{Schlangen können nicht lügen.}

\emph{"Zwei plus zwei ist gleich vier."}

Harry hatte versucht zu sagen, dass zwei plus zwei gleich drei war, und stattdessen war ihm das Wort vier herausgerutscht.

"Gut. Als Salazar Slytherin den Parselmund-Fluch auf sich und alle seine Kinder aussprach, wollte er in Wahrheit sicherstellen, dass seine Nachkommen den Worten der anderen vertrauen konnten, egal welche Ränke sie gegen Außenstehende schmiedeten."

Professor Quirrell hatte seine dozierende Pose aus der Kampfmagie eingenommen, wie jemand, der sich eine abgegriffene Maske aufsetzt, aber die Waffe blieb in seiner Hand gerichtet.

"Okklumentik kann den Parsel-Fluch nicht so täuschen wie Veritaserum, und auch das darfst du auf die Probe stellen. Nun hör gut zu.

\textbf{\emph{Komm mit mir, versprich mir deine beste Hilfe, um Sssstein zu holen, und ich werde diese Kinder unversehrt zurücklassen. Hunderte von Schülern sterben heute Nacht, wenn ich die Ereignisse nicht aufhalte. Ich werde Hunderte von Schülern retten, wenn ich erfolgreich den Stein bekomme. Und merke dir auch dies, merke es dir gut: Ich kann durch kein mir bekanntes Mittel wahrhaftig besiegt werden, und der Verlust des Steins wird mich nicht davon abhalten, zurückzukehren, noch dir oder den deinen meinen Zorn ersparen. Jede ungestüme Handlung, die du in Erwägung ziehst, kann das Spiel nicht für dich gewinnen, Junge. Ich traue dir zu mich zu ärgern, und schlage vor, dass du das vermeidest.}}"

"Du hast gesagt", Harrys Stimme klang fremd in seinen eigenen Ohren, "dass der Stein der Weisen andere Kräfte hat, als die Legende sagt. Das hast du mir in Parsel gesagt. Sag mir, was der Stein wirklich kann, bevor ich zustimme, dir zu helfen, ihn zu bekommen."

\emph{Wenn es etwas in der Art war, wie die totale Macht über das Universum zu erlangen, dann war nichts eine geringfügig größere Chance wert, dass Lord Voldemort den Stein bekommt.}

"Ah", sagte Professor Quirrell und lächelte. "Du denkst. Das ist besser, und als Belohnung biete ich dir einen weiteren Anreiz zur Zusammenarbeit. Ewiges Leben und Jugend, die Erschaffung von Gold und Silber. Angenommen, das sind die wahren Vorteile des Besitzes des Steins. Sag mir, Junge. Was ist die Macht des Steins?"

Vielleicht war es das Adrenalin, das er noch in sich hatte und das sein Gehirn ausnahmsweise mal brauchte. Vielleicht war es auch die Macht, zu wissen, dass es eine Antwort gab und dass der Beweis keine Lüge war.

"Es kann Verwandlungen dauerhaft machen."

Dann hielt Harry inne, als er hörte, was sein eigener Mund gerade gesagt hatte.

"Richtig", sagte Professor Quirrell. "Somit ist derjenige, der den Stein der Weisen besitzt, in der Lage, menschliche Verwandlungen an sich selbst und anderen nach freiem Willen durchzuführen."

Harrys zerrissener Verstand wurde erneut umgeworfen, als er erkannte, welcher weitere Anreiz ihm geboten werden würde.

"Du hast die Überreste von Miss Granger gestohlen und sie in ein harmlos aussehendes Ziel verwandelt", sagte Professor Quirrell.

"Ein verwandeltes Ziel, das du irgendwo an deiner eigenen Person aufbewahren musst, um die Verwandlung aufrechtzuerhalten. Ah, ich sehe, wie dein Blick zu dem Ring an deiner Hand wandert, aber natürlich wäre Miss Granger nicht das kleine Juwel, das in den Ring eingesetzt ist, oder? Das wäre zu offensichtlich. Nein, ich nehme an, du hast Grangers Überreste in den Ring selbst verwandelt und die Aura des verwandelten Juwels die Magie im Ring überdecken lassen."

"Ja", sagte Harry und zwang sich, das Wort auszusprechen.

Es war eine Lüge, ausnahmsweise, und Harrys Blick war absichtlich gewesen. Harry hatte erwartet, dass ihn jemand wegen des Stahlrings herausfordern würde, er hatte versucht, diese Herausforderung zu provozieren, damit er wieder einmal seine Unschuld beweisen konnte, aber niemand hatte ihn darauf angesprochen - vielleicht hatte Dumbledore nur gespürt, dass der Stahl an sich nicht magisch war.

"Schön und gut", sagte Professor Quirrell. "Jetzt komm mit mir, helf mir, den Stein zu beschaffen, und ich werde Hermine Granger in deinem Namen wiederbeleben. Ihr Tod hat unerwünschte Auswirkungen auf dich gehabt, und ich hätte nichts dagegen, sie rückgängig zu machen. Das, so wie ich dich verstehe, ist dein größter Wunsch. Ich habe dir schon viele Gefallen getan, und es würde mir nichts ausmachen, dir einen weiteren zu erweisen."

Eine bleichäugige Professor Sprout hatte sich nun vom Boden erhoben und richtete ihren eigenen Zauberstab auf Harry.

"\textbf{\emph{Hilf mir, den Stein der Verwandlung zu beschaffen, und ich werde mein Bestes tun, um deine Freundin, das Mädchen, wieder zu wahrem und dauerhaftem Leben zu erwecken. Das heißt, Junge, meine Geduld mit dir geht langsam zu Ende, und was jetzt kommt, wird dir nicht gefallen.}}"

Diese letzte Zeile zischte er mit einer Stimme, die den Eindruck einer Schlange vermittelte, die sich zum Schlag aufbäumt.

Selbst dann. Selbst dann, als die ganze Welt auf den Kopf gestellt wurde, mit einem Schock nach dem anderen, selbst dann hörte Harrys Gehirn nicht auf, ein Gehirn zu sein, oder die Muster zu vervollständigen, für die seine Schaltkreise verdrahtet worden waren. Harry wusste, dass dies ein zu gutes Angebot war, um es jemandem zu machen, auf den man eine Waffe gerichtet hatte. Es sei denn, man brauchte \emph{dringend} deine Hilfe, um den Stein der Weisen aus dem Zauberspiegel zu holen.

Und es blieb keine Zeit zum Planen, nur der Gedanke, dass, wenn Professor Quirrell wirklich so weit gehen würde, um seine Hilfe zu bekommen - was Harry wollte, war, von Professor Quirrell das Versprechen zu verlangen, in Zukunft niemanden mehr zu töten, im Austausch für seine Hilfe jetzt, aber Harry hatte das starke Gefühl, dass Professor Quirrell antworten würde: "Sei nicht lächerlich", und es war keine Zeit für ein normales Gespräch - Harry musste die höchst sichere Bitte im Voraus erraten - Professor Quirrells Augen verengten sich, seine Lippen schlossen sich -

"Wenn ich dir helfe", sagte Harrys Mund, "möchte ich dein Versprechen, dass du nicht vorhast, dich gegen mich zu wenden, wenn dies vorbei ist. Ich will, dass du für mindestens eine Woche weder Professor Snape noch sonst jemanden in Hogwarts tötest. Und ich will Antworten, die Wahrheit über alles, was die ganze Zeit vor sich gegangen ist, alles, was du über meine Natur weißt."

Die blassblauen Augen betrachteten ihn teilnahmslos.

\emph{Ich glaube wirklich, wir hätten uns etwas Besseres einfallen lassen können, als das zu verlangen,} sagte Harrys Slytherin-Seite. \emph{Aber ich nehme an, dass wir zu Recht keine Zeit mehr hatten, und was auch immer wir als Nächstes tun müssen, Antworten werden helfen.}

Harry hörte in diesem Moment nicht auf diese Stimme. Ein kalter Schauer lief ihm noch immer über den Rücken, als er die Worte hörte, die gerade über seine Lippen gekommen waren, gerichtet an den Mann mit der Waffe.

"Das ist deine Bedingung, um mir zu helfen, den Stein zu bekommen?", fragte Professor Quirrell.

Harry nickte, unfähig, Worte zu bilden.

"\textbf{\emph{Einverstanden}}", zischte Professor Quirrell."\textbf{\emph{Hilf mir, und du wirst Antworten auf deine Fragen bekommen, solange sie sich auf vergangene Ereignisse beziehen und nicht auf meine Pläne für die Zukunft. Ich habe nicht vor, in Zukunft meine Hand oder Magie gegen dich zu erheben, so lange du nicht deine Hand oder Magie gegen mich erhebst. Ich werde eine Woche lang niemanden auf dem Schulgelände töten, es sei denn, ich muss es tun. Versprich mir, dass du nicht versuchen wirst mich aufzuhalten, Hilfe zu rufen oder zu fliehen. Versprich, dass du dein Bestes gibst, um mir zu helfen, den Stein zu bekommen. Und deine Freundin, das Mädchen, soll von mir wiederbelebt werden, zu wahrem Leben und Gesundheit, und weder ich noch die Meinen sollen ihr je etwas antun.}}"

Ein schiefes Lächeln.

"\textbf{\emph{Versprich es, Junge, und der Handel wird geschlossen.}}"

"Ich verspreche es", flüsterte Harry.

\emph{WAS?} schrien andere Teile seines Verstandes.

\emph{Ähm, er zielt immer noch mit einer Waffe auf uns,} betonte Slytherin. \emph{Wir haben eigentlich keine Wahl, wir holen nur so viel wie möglich aus der Sache heraus.}

\emph{Du Mistkerl,} sagte Hufflepuff. \emph{Glaubst du, Hermine hätte das gewollt? Das ist Lord Voldemort, über den wir hier reden, wissen wir überhaupt, wie viele Menschen er getötet hat und töten wird?}

\emph{Ich bestreite, dass wir um Hermines willen einen Kompromiss mit Lord Voldemort eingehen,} sagte Slytherin. \emph{Er hat eine Waffe und wir können ihn gerade nicht} \emph{aufhalten. Außerdem würden Mum und Dad wollen, dass wir einfach mitgehen und in Sicherheit sind.}

Professor Quirrell betrachtete ihn unverwandt.

"Wiederhole das volle Versprechen in Parsel, Junge."

"\emph{Ich werde dir helfen, den Stein zu beschaffen… Ich kann nicht versprechen, dass ich mein Bestes geben werde, ich fürchte, mein Herz wird nicht dabei sein. Ich beabsichtige, es zu versuchen. Ich werde nichts tun, wovon ich denke, dass es dich zu keinem guten Ende bringt. Ich werde keine Hilfe rufen, wenn ich erwarte, dass sie von dir getötet werden oder dass Geiseln sterben. Es tut mir Leid, Lehrer, aber es issst das Beste, was ich tun kann.}"

Harrys Gedanken beruhigten sich, setzten sich zusammen, als die Entscheidung getroffen war. Er würde bei Professor Quirrell bleiben, mit ihm gehen, um den Stein zu holen, die Schülergeiseln retten und … und … Harry wusste es nicht, nur, dass er weiterdenken würde.

"Das tut dir tatsächlich leid?"

Professor Quirrell sah amüsiert aus.

"Ich nehme an, das wird genügen. Dann behalte zwei andere Dinge im Hinterkopf: \textbf{\emph{Ich habe einen Plan, um selbst den Schulleiter aufzuhalten, wenn er vor uns auftaucht.}} Und auch dies: Ich werde dich gelegentlich in Parsel fragen, ob du mich verraten hast. Die Abmachung ist geschlossen."

Danach hob Professor Sprout Harrys Zauberstab auf und wickelte ihn in ein schimmerndes Tuch, dann legte sie ihn auf den Boden und richtete ihren Zauberstab erneut auf Harry. Erst dann senkte Professor Quirrell seine Waffe, die in seiner Hand zu verschwinden schien, und nahm Harrys eingewickelten Zauberstab auf und steckte ihn in seinen Umhang. Der Wahre Umhang der Unsichtbarkeit wurde von der schlafenden Gestalt von Lesath Lestrange entfernt und Professor Quirrell nahm den Umhang sowie Harrys Beutel und den Zeitumkehrer an sich. Dann zauberte Professor Quirrell auf alle anwesenden Schüler eine Erinnerungslöschung, gefolgt von der Massenversion des Zaubers der falschen Erinnerung, bei der die Versuchsperson die Lücken mit Hilfe ihrer eigenen Fantasie ausfüllen musste. Danach ließ Professor Sprout die schlafenden Kinder wegschweben, die nun einen Ausdruck trugen, der ärgerlich und besorgt wirkte, als wären sie in irgendeinen Kräuterkunde-Unfall verwickelt gewesen. Professor Quirrell wandte sich dann wieder der Stelle zu, an der der Meister der Zaubertränke ausgestreckt lag, beugte sich vor und legte seinen Zauberstab auf die Stirn von Professor Snape.

"Alienis nervus mobile lignum."

Der Verteidigungsprofessor trat zurück und begann, seine linken Finger in der Luft zu bewegen, als würde er eine Marionette an Fäden manipulieren. Professor Snape erhob sich mit geschmeidigen Bewegungen vom Boden und stand wieder vor der Korridortür.

"Alohomora", sagte Professor Quirrell und richtete seinen Zauberstab auf die verbotene Tür.

Der Verteidigungsprofessor sah ziemlich amüsiert aus.

"Würdest du uns die Ehre erweisen, Junge?"

Harry schluckte. Er hatte wieder einmal Zweifel und dritte Gedanken. Es war seltsam, wie man etwas tun konnte, obwohl man wusste, dass es das Falsche war, nicht das Selbstsüchtige, sondern auf einer tieferen Ebene das Falsche zu tun. Aber der Mann hinter ihm hielt die Waffe; sie war auf Harrys Zögern hin wieder in seiner Hand erschienen.

Harry legte seine Hand auf den Türklopfer und atmete einige Male tief durch, um seinen Verstand so gut es ging wieder zu sammeln.

\emph{Zieh es durch, lass dich nicht erschießen, lass die Geiseln nicht sterben, sei da, um die Ereignisse zu optimieren, sei da, um nach Gelegenheiten Ausschau zu halten und bleib fähig, sie zu ergreifen.}

Es war keine gute Wahl, aber alle anderen schienen schlimmer zu sein.

Harry stieß die verbotene Tür auf und schritt hindurch.

