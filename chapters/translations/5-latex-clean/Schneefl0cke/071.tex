

\hypertarget{selbstverwirklichung-teil-7}{% \section{72. Selbstverwirklichung, Teil 7}\label{selbstverwirklichung-teil-7}}

\textbf{\uline{Selbstverwirklichung, Teil 7}}

Die Wintersonne war längst untergegangen, als das Abendessen endete, und so machte sich Hermine im friedlichen Licht der Sterne, die von der verzauberten Decke der Großen Halle herab funkelten, zusammen mit ihrem Lernpartner Harry Potter, der in letzter Zeit lächerlich viel Zeit zum Lernen zu haben schien, auf den Weg zum Ravenclaw-Turm. Sie hatte nicht die leiseste Ahnung, wann Harry seine eigentlichen Hausaufgaben machte, außer dass sie erledigt wurden, vielleicht von Hauselfen, während er schlief. Fast jedes einzelne Augenpaar in der ganzen Halle lag auf ihnen, als sie durch die mächtigen Türen des Speisesaals gingen, die eher den Belagerungstoren einer Burg glichen als etwas, durch das Schüler auf dem Rückweg vom Abendessen gehen sollten. Sie gingen hinaus, ohne zu sprechen, und gingen, bis das ferne Geplapper der Schülergespräche verstummt war; und dann gingen die beiden noch ein Stück weiter durch die steinernen Gänge, bevor Hermine endlich sprach. "Warum hast du das getan, Harry?"

"Was getan?", sagte der Junge-der-lebte in einem abstrakten Ton, als wäre er mit seinen Gedanken ganz woanders und würde über weitaus wichtigere Dinge nachdenken.

"Ich meine, warum hast du ihnen nicht einfach nein gesagt?"

"Nun", sagte Harry, während ihre Schuhe über die Fliesen klapperten, "ich kann nicht einfach jedes Mal '\emph{Nein}' sagen, wenn mich jemand nach etwas fragt, das ich nicht getan habe. Ich meine, nehmen wir an, jemand fragt mich: '\emph{Harry, hast du den Streich mit der unsichtbaren Farbe gemacht?}' und ich sage '\emph{Nein}' und dann sagen sie '\emph{Harry, weißt du, wer sich am Besen des Gryffindor-Suchers zu schaffen gemacht hat?}' und ich sage '\emph{Ich weigere mich, diese Frage zu beantworten.}' würde es das ganze wohl ziemlich offensichtlich verraten."

"Und deshalb", sagte Hermine vorsichtig, "hast du allen erzählt …" Sie konzentrierte sich und erinnerte sich an die genauen Worte. "Dass, wenn es hypothetisch eine Verschwörung gäbe, du weder bestätigen noch leugnen könntest, dass der wahre Meister der Verschwörung der Geist von Salazar Slytherin ist, und dass du sogar nicht einmal in der Lage wärst, die Existenz der Verschwörung zuzugeben, damit die Leute aufhören sollten, dir Fragen darüber zu stellen."

"Jep", sagte Harry Potter und lächelte leicht. "Das wird sie lehren, hypothetische Szenarien zu ernst zu nehmen."

"Und du hast mir gesagt, ich solle nichts antworten -"

"Sie werden dir vielleicht nicht glauben, wenn du es leugnest", sagte Harry. "Also ist es besser, nichts zu sagen, es sei denn, du willst, dass sie dich für einen Lügner halten."

"Aber -" Hermine sagte hilflos. "Aber - aber jetzt denken die Leute, ich würde etwas für Salazar Slytherin tun!"

Die Art, wie die Gryffindors sie angeschaut hatten - die Art, wie die Slytherins sie angeschaut hatten -

"Das gehört dazu, wenn man ein Held ist", sagte Harry. "Hast du gesehen, was der Klitterer über mich schreibt?"

Für eine kurze Sekunde stellte Hermine sich vor, wie ihre Eltern einen Zeitungsartikel über sie lasen, und anstatt dass die Geschichte davon handelte, dass sie einen landesweiten Buchstabierwettbewerb gewonnen hatte oder auf irgendeine andere Art und Weise, die sie sich ausgemalt hatte, in die Zeitung zu kommen, stand in der Schlagzeile:

\textbf{"HERMINE GRANGER SCHWÄNGERT DRACO MALFOY ".}

Das war genug, um sich die ganze Sache mit der Heldin zweimal zu überdenken.

Harrys Stimme wurde ein wenig förmlicher. "Wo wir gerade dabei sind, Miss Granger, wie läuft Ihre letzte Suche?"

"Nun", sagte Hermine, "wenn der Geist von Salazar Slytherin nicht wirklich auftaucht und uns sagt, wo wir Tyrannen finden, glaube ich nicht, dass wir viel Glück haben werden." \emph{Nicht, dass sie das bedauerte}. Sie blickte zu Harry hinüber und sah, wie der Junge sie mit einem seltsam intensiven Blick bedachte.

"Weißt du, Hermine", sagte der Junge leise, als wolle er sichergehen, dass es sonst niemand auf der Welt hörte, "ich glaube, du hast recht. Ich glaube, manche Leute bekommen viel mehr Hilfe als andere, um Helden zu werden. Und ich denke, das ist auch nicht fair."

Und Harry griff nach ihren Hexenroben, wo sie über ihrem Arm lagen, und drängte sie in eine Seitenhalle des Korridors, durch den sie gingen, und ihr Mund klaffte vor Überraschung auf, selbst als Harrys Zauberstab in seine Hand kam, sie eine Kurve der Seitenhalle umrundeten und es so eng war, dass es sie und Harry fast ineinander schob, selbst als Harry auf den Weg zeigte, den sie gekommen waren, und leise "Quietus" sagte, dann einen Moment später, in die andere Richtung, wieder "Quietus". Der Junge schaute suchend um sie herum, nicht nur zu jeder Seite, sondern sogar nach oben zur Decke und nach unten zum Boden. Dann steckte Harry eine Hand in seinen Beutel und sagte: "Unsichtbarkeitsumhang."

"Was?!", sagte Hermine.

Harry zog bereits Falten aus schimmerndem schwarzem Stoff aus dem Beutel.

"Keine Sorge", sagte der Junge mit einem kleinen Grinsen, "die sind so selten, dass sich niemand die Mühe gemacht hat, eine Schulregel gegen sie aufzustellen …" Und dann hielt Harry ihr das dunkle Samtgewebe hin und sagte mit seltsam förmlicher Stimme: "Ich gebe dir nicht, sondern leihe dir meinen Umhang, für Hermine Jean Granger. Beschütze sie gut."

Sie starrte auf den schimmernden Samt des Umhangs, Stoff, der alles Licht verschluckte, das auf ihn fiel, außer dem, was von kleinen, seltsamen Reflexionen glitzerte, Stoff, der so perfekt schwarz war, dass er eigentlich Staub oder Flusen oder so etwas hätte zeigen müssen, aber das tat er nicht, je länger man hinschaute, desto mehr hatte man das Gefühl, dass das, was man sah, gar nicht wirklich da war, aber dann blinzelte man wieder und es war nur ein schwarzer Umhang.

"Nimm ihn, Hermine." Kaum dass sie nachgedacht hatte, streckte Hermine ihre Hand aus, um nach dem Stoff zu greifen; und dann, gerade als ihr Gehirn aufwachte und sie ihre Hand zurückziehen wollte, ließ Harry den Mantel los, und er begann zu fallen, und sie griff danach, ohne nachzudenken. Und in dem Moment, in dem ihre Finger den Mantel berührten und festhielten, fühlte sie einen nicht greifbaren Ruck durch sie hindurchgehen, als würde sie ihren Zauberstab zum ersten Mal in die Hand nehmen; und es war, als hörte sie im Hinterkopf ein Lied, das gesungen wurde, ganz leise.

"Das ist einer meiner Questgegenstände, Hermine", sagte Harry leise. "Es gehörte meinem Vater, und ich kann es nicht ersetzen, wenn es verloren geht. Leih es niemandem, zeig es niemandem, sag niemandem, dass es existiert … aber wenn du ihn für eine Weile ausleihen willst, komm einfach zu mir und frag."

Hermine riss endlich ihre Augen aus den tiefschwarzen Falten und starrte wieder zu Harry hoch. "Ich kann nicht -"

"Das kannst du sehr wohl", sagte Harry. "Denn es ist nicht im Geringsten fair, dass ich dieses Geschenk eines Morgens eingepackt in einer Schachtel neben meinem Bett finde und du … nicht." Harry hielt nachdenklich inne. "Es sei denn, du hast deinen eigenen Unsichtbarkeitsumhang bekommen, in diesem Fall ist es egal."

Dann dämmerte ihr endlich die Tragweite des Unsichtbarkeitsumhangs, und sie zeigte mit einem schockierten Finger auf Harry, obwohl sie so nah beieinander standen, dass sie ihren Arm nicht richtig ausstrecken konnte, und ihre Stimme erhob sich mit beträchtlicher Empörung, als sie sagte: "So bist du also aus dem Zaubertränkeschrank verschwunden! Und das Mal, als…", und dann brach sie ab, denn selbst mit einem Unsichtbarkeitsumhang konnte sie immer noch nicht sehen, wie Harry…

Harry polierte seine Fingernägel mit gekünstelter Nonchalance an seinem Umhang und sagte: "Na ja, du wusstest doch, dass es einen Trick geben musste, oder? Und jetzt wird die Heldin auf geheimnisvolle Weise wissen, wo und wann sie die Tyrannen finden kann - so als ob sie Ihnen bei ihrer Planung zugehört hat, obwohl sich niemand in ihrem Alter unsichtbar machen könnte, um sie auszuspionieren."

Es gab eine Pause und ein Schweigen.

"Harry -", sagte sie. "Ich - ich bin mir nicht mehr sicher, ob es so eine gute Idee ist, gegen Tyrannen zu kämpfen."

Harrys Augen blieben auf den ihren haften. "Weil die anderen Mädchen verletzt werden könnten?"

Sie nickte nur.

"Das ist ihre Entscheidung, Hermine, genau wie es deine ist. Ich habe mich entschieden, nicht die offensichtliche Dummheit zu machen, die jeder in den Büchern macht, zu versuchen, dich zu beschützen und hilflos zu sein, und dich dazu zu bringen, wirklich wütend auf mich zu werden und mich wegzustoßen, während du auf eigene Faust losziehst und in noch mehr Schwierigkeiten gerätst, und es dann heldenhaft erfolgreich durchzustehen, wonach ich endlich meine Erleuchtung habe und erkenne, dass blah blah blah und so weiter. Ich weiß, wie dieser Teil meiner Lebensgeschichte abläuft, also überspringe ich ihn einfach. Wenn ich vorhersagen kann, was ich später denken werde, kann ich es genauso gut jetzt schon denken. Wie auch immer, mein Punkt ist, du solltest deine Freunde auch nicht erdrücken, um sie zu beschützen. Sag ihnen einfach im Voraus, dass es vorhersehbar schrecklich schief gehen wird, und wenn sie danach immer noch Heldinnen sein wollen, gut."

In Momenten wie diesen fragte sich Hermine, ob sie sich jemals an die Art gewöhnen würde, wie Harry dachte. "Harry, ich will wirklich", ihre Stimme stockte für eine Sekunde, "wirklich, wirklich nicht, dass sie verletzt werden! Besonders wegen etwas, das ich angefangen habe!"

"Hermine", sagte Harry ernst, "ich bin mir ziemlich sicher, dass du das Richtige getan hast. Ich wüsste nicht, was ihnen realistischerweise passieren könnte, was auf lange Sicht schlimmer für sie wäre, als es nicht zu versuchen."

"Was ist, wenn sie schwer verletzt werden?" sagte Hermine. Ihre Stimme fühlte sich in ihrer Kehle blockiert an; sie erinnerte sich daran, wie Kapitän Ernie erzählt hatte, wie Harry gerade einem Tyrannen in die Augen gestarrt hatte, als dieser ihm den Finger zurückbog, bevor Professor Sprout gekommen war, um ihn zu retten; und es gab noch einen anderen Gedanken, der danach kam, über Hannah und ihre zarten Hände mit den Fingernägeln, die sie jeden Morgen sorgfältig in Hufflepuff-Gelb lackierte, aber das durfte man sich nicht vorstellen. "Und dann - werden sie nie wieder etwas Mutiges tun -"

"Ich glaube nicht, dass es so funktioniert", sagte Harry mit fester Stimme. "Selbst wenn alles wahnsinnig schief geht, glaube ich nicht, dass es in einem menschlichen Verstand so funktioniert. Das Wichtigste ist, dass man von sich selbst überzeugt ist, dass man jemand ist, der seine Grenzen überschreiten kann. Es zu versuchen und verletzt zu werden, kann unmöglich schlimmer für dich sein, als … festzustecken."

"Und wenn du dich irrst, Harry?"

Harry hielt einen Moment inne, dann zuckte er ein wenig traurig mit den Schultern und sagte: "Was, wenn ich recht habe?"

Hermine blickte wieder auf das schwarze Netz, das über ihre Hand lief. Von innen fühlte sich der Umhang seltsam weich und doch fest gegen ihre Handfläche an, als wollte er ihre Hand beruhigend umarmen. Dann hob sie ihren Arm wieder hoch und hielt Harry den Mantel hin. Harry bewegte sich nicht, um ihn zu nehmen. "Ich -", sagte Hermine. "Ich meine, danke, vielen Dank, aber ich denke immer noch darüber nach, also kannst du ihn fürs Erste zurücknehmen. Und… Harry, ich glaube nicht, dass es richtig ist, Leute auszuspionieren -"

"Nicht einmal bei bekannten Tyrannen, um ihre Opfer zu retten?" Harry sagte. "Ich bin noch nie gemobbt worden, aber ich habe eine realistische Simulation durchgemacht, und die war nicht sehr angenehm. Wurdest du jemals verprügelt, Hermine?"

"Nein", sagte sie mit ruhiger Stimme und hielt Harry weiterhin seinen Unsichtbarkeitsumhang hin.

Schließlich nahm Harry seinen Umhang zurück - sie spürte ein kleines Zucken des Verlustes, als das unhörbare Lied aus ihrem Hinterkopf verschwand - und begann, den schwarzen Stoff zurück in seinen Beutel zu stopfen. Als der Beutel den letzten Rest des Stoffes verschlang, wandte sich Harry von ihr ab, um die Ruhebarriere zu durchbrechen -

"Und, ähm", sagte Hermine. "Das ist doch nicht der Umhang der Unsichtbarkeit, oder? Der, von dem wir in der Bibliothek auf Seite achtzehn von Paula Vieiras Übersetzung von Gottschalks Eine illustrierte Schriftrolle der verlorenen Artefakte gelesen haben?"

Harry drehte den Kopf zurück, grinste leicht und sagte in genau demselben Tonfall, den er zuvor beim Abendessen mit den anderen Schülern benutzt hatte: "Ich kann weder bestätigen noch leugnen, dass ich magische Artefakte von unglaublicher Macht besitze."

…

Als Hermine an diesem Abend ins Bett kletterte, versuchte sie immer noch, sich zu entscheiden. Ihr Leben war zur Zeit des Abendessens einfacher gewesen, damals, als es noch keine praktische Möglichkeit gegeben hatte, Tyrannen zu finden; und jetzt musste sie sich wieder entscheiden; diesmal nicht für sich selbst, sondern für ihre Freunde. Vor ihrem geistigen Auge sah sie immer wieder Dumbledores faltiges Gesicht und den Schmerz, den es nicht ganz verbergen konnte, und in ihren geistigen Ohren hörte sie immer wieder Harrys Stimme, die sagte: \emph{Das ist ihre Entscheidung, Hermine, genau wie es deine ist}. Und ihre Hand erinnerte sich immer wieder an das Gefühl des Umhangs an ihren Fingern, spielte es immer wieder in ihrem Kopf ab. Das Gefühl hatte eine Kraft, die ihre Gedanken dazu zwang, zu ihm zurückzukehren, und zu dem Lied, das sie in einem Teil ihres Geistes und ihrer Magie gehört oder nicht gehört hatte, der jetzt wieder still lag.

\emph{Harry hatte mit dem Mantel gesprochen, als wäre er eine Person, und ihm gesagt, er solle gut auf sie aufpassen. Harry hatte gesagt, dass der Mantel seinem Vater gehört hatte, dass er ihn nicht ersetzen konnte, wenn er verloren ging… Aber… Harry würde das nicht wirklich tun, oder? Ihr einfach eines der drei Heiligtümer des Todes geben, die Jahrhunderte vor Hogwarts erschaffen wurden?}

Sie könnte sagen, dass sie sich geschmeichelt fühlte, aber das ging weit über das Gefühl des Schmeichelns hinaus und brachte sie dazu, sich zu fragen, was genau sie für Harry war. Vielleicht war Harry die Art von Mensch, die herumlief und uralte, verlorene magische Artefakte an jeden verlieh, den er als Freund betrachtete, aber - Aber wenn sie darüber nachdachte, welchen Teil seines Lebens Harry gesagt hatte, dass er ihn übersprungen hatte, den Teil, in dem er versuchte, sie zu beschützen…

Hermine starrte hinauf an die Decke des Ravenclaw-Schlafsaals. Irgendwo hinter ihrem Bett unterhielten sich Mandy und Su. Sie hatte ihren Ruhezauber so weit aufgedreht, dass sie die Worte nicht genau hören konnte, aber sie konnte immer noch ihr leises Gemurmel hören; es hatte etwas Beruhigendes, mit den anderen Mädchen in einem Schlafsaal zu schlafen. Harry hatte seinen eigenen Ruhezauber ganz aufgedreht, das wusste sie. Sie begann sich zu fragen, ob Harry vielleicht tatsächlich, na ja… \emph{Du weißt schon… sie mochte.}

Hermine Granger brauchte in dieser Nacht sehr lange, um einzuschlafen. Und als sie am nächsten Morgen aufwachte, lugte ein kleiner Zettel unter ihrem Kopfkissen hervor, auf dem stand:

\emph{"Um halb elf findest du einen Tyrannen im vierten Gang auf der linken Seite der Halle, wenn du das Klassenzimmer für Zaubertränke verlässt -}

\emph{S."}

Als Hermine an diesem Morgen die Große Halle betrat, war ihr Magen mit fliegenden Schmetterlingen von der Größe eines Hippogreifs gefüllt; selbst als sie sich dem Ravenclaw-Frühstückstisch näherte, hatte sie sich noch nicht entschieden, was sie tun sollte. Sie sah, dass neben Padma ein leerer Platz war. Dort würde sie sich hinsetzen, wenn sie es Padma sagen wollte und dann Padma bitten würde, es Daphne und Tracey zu sagen. Hermine ging auf den leeren Platz neben Padma zu. Die Worte warteten in ihrer Kehle, \emph{Padma, ich habe eine geheimnisvolle Nachricht bekommen} - und sie konnte eine riesige Mauer in sich spüren, die die Worte daran hinderte, herauszukommen. \emph{Sie würde Hannah, Susan und Daphne in Gefahr bringen. Sie würde sie an der Hand nehmen und sie direkt in Schwierigkeiten führen. Das war Falsch. Oder sie könnte einfach versuchen, mit dem Tyrannen selbst fertig zu werden, ohne ihren Freundinnen etwas zu sagen, und das war ganz offensichtlich auch Falsch.} Hermine wusste, dass sie vor einem moralischen Dilemma stand, genau wie all diese Zauberer und Hexen, von denen sie in Geschichten gelesen hatte. Nur dass in den Geschichten die Leute immer eine richtige und eine falsche Wahl hatten, nicht zwei falsche, was ein bisschen unfair schien. Aber irgendwie hatte sie das Gefühl - vielleicht kam es von der Art, wie Harry immer davon sprach, wie die Geschichtsbücher sie sehen würden -, dass sie vor einer heroischen Entscheidung stand und dass ihr ganzes Leben in die eine oder andere Richtung verlaufen könnte, je nachdem, was sie jetzt, an diesem Morgen, wählte.

Hermine setzte sich an den Tisch, ohne einen Blick zur Seite zu werfen, starrte nur auf den Teller und das Besteck, als ob darin Antworten verborgen sein könnten, und dachte so angestrengt nach, wie sie es noch nie getan hatte, und ein paar Sekunden später hörte sie Padmas Stimme, die fast in ihr Ohr flüsterte:

"Daphne sagt, sie weiß, wo heute um zehn Uhr dreißig ein Tyrann sein wird."

\emph{Zum Scheitern verurteilt.}

Nach Susan Bones' Meinung waren sie alle dem Untergang geweiht. Tantchen erzählte manchmal Geschichten, die so anfingen, Leute, die etwas taten, von dem sie wussten, dass es dumm war, und die Geschichten endeten in der Regel damit, dass jemand auf dem ganzen Boden und an den Wänden verteilt war und auf Tantchens Schuhe spritzte.

"Hey, Padma", murmelte Parvati, ihre Stimme war kaum zu hören über dem leisen Aufprall von acht Mädchen, die auf Zehenspitzen durch den Korridor zum Zaubertränke-Klassenzimmer schlichen, "weißt du, warum Hermine schon den ganzen Morgen seufzt -"

"Nicht reden!", zischte Lavender, das harsche Flüstern klang viel lauter als Parvatis Gemurmel. "Man kann nie wissen, wann das Böse zuhört!"

"Pssst!", sagten die drei anderen Mädchen noch lauter.

\emph{Das wird ganz, ganz, ganz extrem schiefgehen.}

Als sie sich dem vierten Gang links vom Zaubertränke-Klassenzimmer näherten, wo laut Daphnes geheimnisvollem Informanten das Mobbing stattfinden sollte, bewegten sich die acht langsamer, das Geräusch ihrer Füße wurde leiser, und schließlich machte General Granger die Geste, die Halt bedeutete, ich schaue voraus. Lavender hob daraufhin eine Hand, und als Hermine sich umdrehte, um sie anzusehen, zeigte Lavender mit verwirrtem Blick geradeaus den Korridor hinunter, gestikulierte vor sich hin und versuchte dann, etwas anderes zu gebärden, das Susan nicht verstand - General Granger schüttelte den Kopf und machte erneut, diesmal mit langsameren, übertriebeneren Bewegungen, das Zeichen für \emph{Halt, ich schaue voraus}. Lavender, die noch verwirrter aussah, zeigte den Weg zurück, den sie gekommen waren, und machte mit der anderen Hand eine hüpfende Geste. Jetzt schauten alle anderen noch verwirrter als Lavender, und Susan dachte mit einer gewissen Bitterkeit, dass eine Stunde Training vor zwei Tagen offensichtlich nicht ausreichte, um sich einen neuen Satz Codesignale zu merken. Hermine deutete auf Lavender, dann auf den Boden unter Lavenders Füßen, wobei ihr Gesichtsausdruck deutlich machte, dass die beabsichtigte Bedeutung \emph{Du. Bleib. Hier.} war Lavender nickte.

\emph{Verdamnis, Verdamnis, Verdamnis, ('doom, doom, doom'. Anm. des Übersetzers)} die Worte des Marschliedes der Chaoslegion gingen Susan durch den Kopf, \emph{Verdamnis}, \emph{Verdamnis}, \emph{Verdamnis}, \emph{Verdamnis}…

Hermine griff in ihren Umhang und zog einen kleinen Stab mit einem Spiegel und einem Okular heraus. Ganz, ganz leise schlich sich das Ravenclaw-Mädchen an die Wand, direkt neben die Stelle, wo der Durchgang vom Korridor abging, und spähte nur mit der Spitze des Okulars um die Ecke. Dann ein bisschen mehr. Dann noch ein bisschen mehr. Dann steckte General Granger vorsichtig den Kopf um die Seite. General Granger drehte sich wieder zu ihnen um, nickte und machte die Handbewegung für \emph{"Folge mir"}.

Susan fühlte sich ein wenig besser, als sie vorwärts schlich. Der Teil des Plans, der vorsah, dass sie dreißig Minuten vor dem Tyrann eintreffen sollten, hatte anscheinend tatsächlich funktioniert. \emph{Vielleicht waren sie nur geringfügig dem Untergang geweiht…?}

Um zehn Uhr neunundzwanzig, fast pünktlich, tauchte der Tyrann auf. Wenn jemand anwesend gewesen wäre, um ihn zu hören - obwohl der Korridor anscheinend leer war -, hätte er gehört, wie seine Schuhe fest durch den Hauptkorridor klackten, den Gang betraten, in Richtung der ersten Ecke des Ganges liefen, um diese Ecke bogen und dann überrascht stehen blieben, als sie sahen, dass der Gang nun in einer massiven Ziegelwand endete, wo vorher keine Wand war. Dann zuckte der Rüpel mit den Schultern und wandte sich ab, während er sich zurücklehnte, um den Hauptgang von der nächsten Ecke aus zu beobachten. Immerhin war es das Schloss Hogwarts. Hinter den hastig verwandelten dünnen Platten, die sie zum äußeren Erscheinungsbild einer Backsteinmauer zusammengesetzt hatten, warteten die Mädchen; sie sprachen nicht, bewegten sich nicht, atmeten nicht einmal, sondern beobachteten durch die Augenlöcher, die sie selbst hinterlassen hatten. Als Susans Blick den Jungen erfasste, spürte sie, wie sich ihre Brust bis in die Zehenspitzen zusammenzog.

Der Junge sah aus, als wäre er im siebten Jahr, wenn nicht sogar älter, und seine Roben waren grün statt wie erhofft rot, und er hatte Muskeln, und nachdem sie ihn noch etwas länger angestarrt hatte, erkannte Susan, dass seine Haltung eine gewisse Balance hatte, die bedeutete, dass er oft kämpfte und duellierte. Dann hörten sie alle das Geräusch von weiteren Füßen, die sich aus dem Korridor näherten. Die Gryffindors und Slytherins aus dem vierten Jahr waren gerade aus dem Zaubertrankunterricht entlassen worden. Die Schritte prasselten vorbei, wurden leiser und verblassten, und der Tyrann tat nichts.

Für einen Moment fühlte Susan einen Anflug von Erleichterung - dann näherte sich eine weitere, kleinere Gruppe von Schritten. Der Tyrann tat immer noch nichts, als die Schritte vorbeigingen.

Das passierte noch ein paar Mal.

Und dann, als sich eine letzte Gruppe von Schritten leise hörbar näherte, hörten die sieben Mädchen die Stimme des Tyrannen, die klar und kalt und leise "Protego" sagte.

Da keuchte jemand, wenn auch zum Glück sehr, sehr leise. Wenn sie nicht einmal einen einzigen Schuss abgeben konnten - die Tyrannen lernten schon dazu, erkannte Susan, sie hatte nicht erwartet, dass S.P.H.E.W. das schaffen würde, und die Tyrannen es kapierten - aber - \emph{Hermine hatte schon drei Tyrannen besiegt - und die Schule hatte gestern mit Spekulationen über Salazar Slytherins Geist geprahlt - Er erwartet uns!} Susan hätte geflüstert, aufzugeben, den Plan abzubrechen, nur gab es keine Möglichkeit, eine Nachricht an -

"Silencio", sagte der Tyrann mit leiser, bedächtiger Stimme, den Zauberstab auf den Korridor gerichtet, den blauen Schleier seines abschirmenden Zaubers um sich schimmernd.

"Accio Opfer."

Als der Junge aus dem vierten Jahr in ihr Blickfeld kam, baumelte er kopfüber, als ob eine unsichtbare Hand ihn an einem Bein hochhielte, und sein roter Umhang begann an den Oberschenkeln herunterzurutschen, um die darunter liegende Hose zu enthüllen. Sein Mund öffnete und schloss sich hilflos, ohne dass ein Ton herauskam.

"Ich nehme an, du fragst dich, was hier los ist", sagte der Slytherin im siebten Jahr mit ruhiger, kalter Stimme. "Mach dir keine Sorgen. Es ist so einfach, dass es sogar ein Gryffindor verstehen könnte."

Damit formte die linke Hand des Slytherins eine Faust und schlug hart in den Bauch des Gryffindors. Der Körper des Viertklässlers zuckte verzweifelt herum, aber noch immer verließ kein Wort seinen Mund. "Du bist mein Opfer", sagte der ältere Slytherin. "Ich bin ein Tyrann. Ich werde dich verprügeln. Und wir werden sehen, ob mich jemand aufhält."

In diesem Moment erkannte Susan, dass es eine Falle war. Und fast im selben Moment ertönte die mächtige und hohe Stimme eines jungen Mädchens, das rief: "Halt, Übeltäter! Finite Incantatem!"

\emph{Lavender}, dachte Susan gequält. Das Gryffindor-Mädchen hatte sich freiwillig gemeldet, um als Ablenkung zu dienen, während der Rest von ihnen einen Flankenangriff von dort ausführte, wo der Rüpel es nicht erwarten würde, das war der Plan gewesen, nur jetzt -

"Im Namen von Hogwarts", rief Lavenders Stimme, obwohl sie sie nicht sehen konnten, "und im Namen aller Heldinnen überall, befehle ich dir, dieses Opfer loszulassen!"

"Expelliarmus", sagte der Rüpel. "Stupor. Accio dumme Heldin."

Als Lavender in ihr Blickfeld schwebte, an einem Fuß baumelnd und bewusstlos, blinzelte Susan; das Mädchen war in einen leuchtend karmesinroten und goldenen Rock und eine Bluse gekleidet, anstelle ihrer üblichen Hogwarts-Roben. Der Tyrann warf dem Mädchen ebenfalls einen seltsamen Blick zu, dann richtete er seinen Zauberstab auf sie und sagte "Finite Incantatem", aber die Kleidung blieb gleich. Dann zuckte der Rüpel mit den Schultern und zog, immer noch in Richtung Lavender statt in Richtung des baumelnden Viertklässlers blickend, seine Faust zurück -

"Lagann!" schrien fünf Stimmen, und fünf grüne Spiralen schossen aus fünf Zauberstäben, die durch fünf Löcher in der falschen Wand zielten, und einen Augenblick später rief Hermines Stimme "Stupor!"

Fünf grüne Spiralen zerschellten wirkungslos am blauen Dunst, und Hermines roter Blitz prallte am Dunst ab und traf den Viertklässler, der zusammenzuckte und dann still war. Und der Siebtklässler drehte sich um und lächelte grimmig, als die Erstklässlerinnen schrien und angriffen.

…

Susans Augen flogen auf und augenblicklich rollte sie sich von dem Ort weg, an dem sie auf dem Boden gelegen hatte, ihre Lungen brannten noch immer und ihr ganzer Körper schmerzte noch immer von dem Schlag, der sie getroffen hatte, der Kampf war nur ein paar Sekunden lang gegangen bis Sie getroffen wurde, Hannahs Körper fiel, ihr Arm war noch immer nach Susan ausgestreckt, "Glisseo! " schrie Hermine, aber der ältere Junge schlug einfach seinen Zauberstab nach unten und hinterließ eine Spur von grünem Glühen und Hermines Zauber zerfiel sichtbar in einen Schauer von blau-weißen Funken, dann sagte der Junge fast in der gleichen Bewegung "Stupor!" und Hermine wurde nach hinten geschleudert und Susan beschwor alle Magie, die sie noch hatte und schrie "Innvervate!"

Der Zauberstab des Jungen zeigte wieder in ihre Richtung und dann schrie Padma "Prismatis!", kurz bevor der Rüpel "Impedimenta!" schrie, die Regenbogenkugel bildete sich um den Tyrannen und der Slytherin im siebten Jahr taumelte, als sein eigener Fluch auf ihn zurückgeworfen wurde, aber einen Augenblick später fegte der Zauberstab zurück, auf ihn selbst um den Fluch aufzuhebenund dann zersprang Padmas prismatische Kugel wie eine geplatzte Seifenblase, als der Zauberstab des Tyrannen sie durchtrennte und Padma schrie "Innervate!" und zeigte dabei auf Hannah. Tracey und Lavender schrien gleichzeitig: "Wingardium Leviosa!" -

….

Hannah Abbott streckte ihren Zauberstab mit einer Hand aus, die vor Erschöpfung zitterte, sie hatte jetzt nicht mehr genug Magie für ein einziges Innervate.

Der Rest des Ganges war still, verstreute Körper lagen auf dem Boden, Padma und Tracey und Lavender, Hermine und Parvati in einem Haufen an einer Wand, Susan stand in versteinerter Starre da, während ihre Augen alles hilflos verfolgten, sogar der Gryffindor-Junge lag ausgestreckt und regungslos da (Hermine hatte ihn geweckt und er hatte gekämpft, aber es war nicht genug gewesen). Es war ein sehr kurzer Kampf gewesen. Der Tyrann lächelte immer noch, die einzigen Anzeichen seiner Anstrengung waren ein schwankendes Kräuseln in dem blauen Schein, der ihn umgab, und ein paar Schweißperlen auf seiner Stirn. Er hob den Arm, wischte sich den Schweiß von der Stirn und pirschte sich wie ein mannshoher lebender Lethifold an sie heran. Hannah drehte sich um und floh, drehte sich und rannte mit Schreien, die in ihrer würgenden Kehle gefangen blieben, sprintete an der heruntergefallenen Verkleidung der falschen Backsteinmauer vorbei, rannte mit aller Geschwindigkeit, die sie aufbringen konnte, den Gang hinunter und rannte, so gut sie konnte -

Kurz bevor Hannah die Kurve im Gang erreichte, sagte die Stimme des Tyrannen hinter ihr "Cluthe!" und sie bekam schreckliche Krämpfe in den Beinen, sie fiel hin und rutschte und schlug mit dem Kopf gegen die Wand, nur dass sie den Schmerz des Aufpralls gar nicht bemerkte, als sie vor lauter Muskelzerrung zu schreien begann - der Rüpel pirschte sich immer noch an sie heran, sah Hannah, als sie den Kopf drehte; er kam langsam auf sie zu, immer noch mit diesem furchtbaren Lächeln. Und sie rollte sich, trotz des Schmerzes, als sich ihre Beinmuskeln um sich selbst verkrampften, sie rollte um die Ecke des Ganges und schrie: "Geh weg!"

"Ich glaube nicht", sagte der Tyrann, seine Stimme war tief und furchteinflößend wie die eines erwachsenen Mannes und klang jetzt ganz nah. Der Tyrann ging um die Ecke, und Daphne Greengrass die hinter der Ecke gewartet hatte stach ihre Zauberstab-Klinge direkt in seine Leiste. Es gab einen Blitz, der den ganzen Korridor erhellte -

mit gedämpfter Miene verließen sieben Mädchen das Büro von Madam Pomfrey und ließen eine der ihren in einem Krankenhausbett zurück. Hannah würde in etwa fünfunddreißig Minuten wieder in Ordnung sein, hatte die Heilerin gesagt; gerissene Muskeln waren leicht zu flicken. Daphne hatte die ganze Zeit geredet, und ihr zufolge hatte Hannah ein Missgeschick mit einem Straßenlaufzauber erlitten, das die Beinkrämpfe verursacht hatte.

Madam Pomfrey hatte ihnen einen scharfen Blick zugeworfen, aber nicht widersprochen, obwohl dieser Zauber etwa sechs Jahre über ihrem Niveau lag. Madam Pomfrey hatte Daphne auch einen Zaubertrank gegeben, um ihren Zustand der totalen magischen Erschöpfung zu lindern, und sie gewarnt, in den nächsten drei Stunden keine Zauber zu sprechen. Das lag angeblich daran, dass Daphne zu viel Magie verbraucht hatte, als sie versuchte, Hannah zu heilen, und nicht daran, dass die Klinge all ihre Kraft aufbrachte, um den Protego zu brechen. Der Rest von ihnen hatte beschlossen, nichts über die blauen Flecken unter ihren Roben zu sagen, bis sie einige ältere Mädchen dazu bringen konnten, \emph{Episkey} zu zaubern. Es gab Grenzen für das, was Daphne erzählen konnte.

\emph{Die ganze Sache}, dachte Susan, \emph{war zu knapp gewesen, viel zu knapp. Wenn der Rüpel auch nur um die Ecke geschaut hätte - wenn er sich einen Moment Zeit genommen hätte, seinen Abschirmzauber neu zu wirken} -

"Wir sollten aufhören", sagte Susan, sobald die sieben aus der Hörweite des Heilerbüros heraus waren. "Wir sollten aufhören, das zu tun." Aus irgendeinem Grund drehten sich dann alle zu General Granger um, obwohl sie eigentlich über so etwas abstimmen sollten.

Die Sonnenschein-Generalin schien nicht zu bemerken, dass sie sie ansahen, sie schritt einfach weiter und starrte geradeaus. Nach einer Weile sagte Hermine Granger mit einer Stimme, die nachdenklich und ein wenig traurig klang: "Hannah sagte, sie wolle nicht, dass wir aufhören. Ich bin mir nicht sicher, ob es richtig ist, dass wir … weniger tapfer für sie sind, als sie es ist."

Alle anderen Mädchen, außer Susan, nickten daraufhin.

"Ich denke, schlimmer kann es nicht mehr werden", sagte Parvati. "Und wir können es schaffen. Das haben wir jetzt bewiesen."

Susan fiel nichts ein, was sie darauf erwidern konnte. Sie glaubte nicht, dass es überzeugend sein würde, aus voller Kehle über eklatante Dummheit und den Untergang zu schreien. Und sie konnte die anderen Mädchen auch nicht einfach verlassen. \emph{Reichte es nicht, mit dem Fluch des Fleißes belegt zu sein, warum mussten Hufflepuffs auch noch loyal sein?}

"Übrigens, Lavender", sagte Padma. "Was in Merlins Namen hattest du vorhin für Unterhosen an?"

"Mein Helden-Outfit", sagte das Gryffindor-Mädchen.

Daphne klang müde, als sie sprach, ohne ihren eigenen Kopf von dort zu wenden, wo sie durch die Halle stapfte. "Es ist das Kostüm des Soldaten von Gryffindor aus dem Stück Chroniken der Lunarier."

"Hast du es verwandelt?", fragte Parvati und schaute verwirrt. "Aber der Tyrann hat dich doch mit Finite belegt -"

"Nö!" sagte Lavender. "Es ist echt! Ich habe mein Helden-Outfit vorher in ein normales Hemd und einen Rock verwandelt, also musste ich nur noch Finite auf mich wirken, nachdem ich den Tyrannen gesehen hatte. Willst du dein eigenes, Parvati? Ich habe meins gestern von Katarina und Joshua im sechsten Schuljahr anfertigen lassen, für zwölf Sickles -"

"Ich glaube", sagte General Granger mit vorsichtiger Stimme, "das würde uns alle ein bisschen albern aussehen lassen."

"Nun", sagte Lavender, "wir sollten darüber abstimmen, ob wir -"

"Ich denke", sagte General Granger, "egal, wie jemand abstimmt, ich werde mich nicht dabei erwischen lassen, eines dieser Kostüme zu tragen -"

Susan ignorierte den Streit. Sie versuchte, sich eine schlaue Strategie auszudenken, um weniger \emph{verdammt} zu sein.

Die ganze Große Halle verstummte, wenn auch nur für einen Moment, als die sieben in die Mittagspause gingen. Dann setzte der Applaus ein. Er war verstreut, nicht der massive Applaus, bei dem alle auf einmal applaudierten. Viel davon kam vom Gryffindor-Tisch, weniger von Hufflepuff und Ravenclaw, und keiner von Slytherin.

Daphne spürte, wie sich ihr Gesicht straffte. Sie hatte gehofft - na ja, vielleicht würden ihre Mitschüler es merken, nachdem sie einen Gryffindor-Rüpel zum Aufhalten und einen Slytherin zum Retten gefunden hatten - Sie sah zum Hufflepuff-Tisch. Neville Longbottom applaudierte mit hochgehaltenen Händen über seinem Kopf, obwohl er nicht lächelte. Vielleicht hatte er von Hannah gehört, oder er fragte sich, warum Hannah nicht da war. Dann, nicht ganz in der Lage, sich selbst zu helfen, blickte sie in Richtung des Haupttisches. Professor Sprouts Gesicht war von Sorge gezeichnet. Sie und Professor McGonagall neigten ihre Köpfe zu Schulleiter Dumbledore, der einen feierlichen Blick hatte, und alle bewegten ihre Lippen schnell. Professor Flitwick sah eher resigniert aus, und Quirrell, mit schlaffem Gesicht, stocherte mit einem zur Faust geballten Löffel zitternd in seiner Suppe herum. Professor Snape blickte direkt auf - sie? Oder - auf Hermine Granger, die neben ihr stand? Ein kleines, dünnes Lächeln ging über das Gesicht des Meisters der Zaubertränke, und er hob seine Hände, führte sie einmal in einer Bewegung zusammen, die zu langsam war, um ein echtes Klatschen zu sein; und dann wandte sich der Meister der Zaubertränke wieder seinem Teller zu und ignorierte die Gespräche um ihn herum. Daphne spürte, wie ihr ein kleiner Schauer über den Rücken lief, und sie drehte sich hastig um, um zum Slytherin-Tisch zu gehen.

Susan, Lavender und Parvati lösten sich von ihrer Gruppe und gingen in Richtung der Hufflepuff- und Gryffindor-Tische auf der anderen Seite der Großen Halle. Es passierte, als sie an dem Teil des Slytherin-Tisches vorbeikamen, an dem das Slytherin-Quidditch-Team saß.

In diesem Moment stolperte Hermine plötzlich, stolperte heftig, als würde sie von den Füßen gerissen, und stürzte in die Lücke zwischen Marcus Flint und Lucian Bole, die dort saßen, und es gab ein trauriges, kleines Platschen, als Hermines Gesicht in Flints Teller mit Steak und Kartoffelpüree landete. Dann schien alles zu schnell zu gehen, oder vielleicht war es nur Daphne selbst, die zu langsam dachte, denn Flint stieß einen Schrei der Empörung aus und seine Hand riss Hermine zurück und warf sie gegen den Ravenclaw-Tisch, woraufhin sie vom Rücken eines Schülers abprallte und auf dem Boden zusammenbrach -

Die Stille breitete sich in Wellen aus. Hermine stieß sich auf den Händen ab, obwohl sie nicht ganz auf die Beine kam, konnte Daphne sehen, dass ihr ganzer Körper zitterte und dass ihr Gesicht noch immer mit Kartoffelbrei und verstreuten Steakstücken bedeckt war. Einen langen Moment lang sprach niemand, niemand bewegte sich. Als könnte sich niemand in der ganzen Großen Halle vorstellen, genauso wenig wie Daphne, was als nächstes geschah.

Dann sagte Flints kräftige Stimme, die Stimme des Slytherin-Kapitäns, der auf dem Quidditchfeld Befehle brüllte, mit einem gefährlichen Grollen: "Du hast mein Essen ruiniert, Mädchen."

Ein weiterer Moment der eisigen Stille.

Hermines Kopf - Daphne konnte sehen, wie er zitterte - drehte sich, um den Slytherin-Quidditch-Kapitän anzusehen.

"Entschuldige dich bei mir", sagte Flint.

Harry Potter begann, sich vom Ravenclaw-Tisch hochzudrücken, und blieb dann abrupt stehen, auf halbem Weg zu seinen Füßen, als ob ihm gerade etwas eingefallen wäre -

Dann standen fünf andere Leute vom Ravenclaw-Tisch auf. Die gesamte Slytherin-Quidditch-Mannschaft stand auf, ihre Zauberstäbe in der Hand, und dann standen Schüler am Gryffindor-Tisch und am Hufflepuff-Tisch auf, und ohne nachzudenken, drehte sich Daphne um, um zum Haupttisch zu schauen, und sie sah, dass der Schulleiter noch immer saß, \emph{Er sah zu, sah einfach nur zu, Dumbledore sah einfach nur zu und er hatte eine Hand ausgestreckt, als wolle er Professor McGonagall zurückhalten} - \emph{in nur einer Sekunde würde jemand einen Zauberspruch rufen und dann wäre es zu spät, warum tat der Schulleiter nichts} -

und eine Stimme sagte: "Entschuldigung."

Daphne drehte sich um und schaute mit offenem Mund und absolutem Schock.

"Verzeihen Sie", sagte die sanfte Stimme, und das Kartoffelpüree verschwand aus Hermines Gesicht und enthüllte den überraschten Gesichtsausdruck der Ravenclaw, als Draco Malfoy auf sie zukam, seinen Zauberstab wieder in die Tasche steckte und dann neben ihr auf ein Knie kniete und ihr die Hand reichte. "Das tut mir leid, Miss Granger", sagte Draco Malfoys höfliche Stimme. "Da hat wohl jemand gedacht, er wäre witzig."

Hermine nahm Dracos Hand, und Daphne wurde plötzlich klar, was gleich passieren würde - aber Draco Malfoy hob Hermine nicht erst halb hoch und ließ sie dann fallen. Er zog sie einfach auf ihre Füße.

"Danke", sagte Hermine.

"Gern geschehen", sagte Draco Malfoy mit lauter Stimme und schaute nicht zur Seite, um zu sehen, wo alle vier Häuser von Hogwarts ihn völlig schockiert anstarrten. "Denk nur daran, dass gerissen und ehrgeizig zu sein, nicht bedeutet, dass man \emph{so} sein muss."

Und dann ging Draco Malfoy zurück zu seinem Platz an der Slytherinbank und setzte sich, als hätte er nicht - \emph{er hatte nicht nur - er hatte nur} -

Hermine ging zum nächsten freien Platz an der Ravenclawbank und setzte sich. Einige andere Leute setzten sich, eher langsam, hin.

"Daphne?", sagte Tracey. "Ist alles in Ordnung mit dir?"

…

Dracos Herz hämmerte so heftig in seiner Brust, dass er befürchtete, es könnte in einem Blutregen aus seiner Brust explodieren, wie der Fluch, den Amycus Carrow einmal bei einem Hundewelpen angewendet hatte. Dracos Gesicht blieb völlig beherrscht, denn er wusste (es war ihm immer wieder eingebläut worden), dass seine Mitbewohner ihn wie einen Schwarm Acromantulas in Stücke reißen würden, wenn er auch nur das kleinste Anzeichen von Angst zeigte, die er fühlte. Er hatte keine Zeit gehabt, sich mit Harry Potter abzusprechen, keine Zeit, einen Plan zu schmieden, keine Zeit zum Nachdenken, nur den Augenblick, in dem ihm klar wurde, dass es genau jetzt an der Zeit war, den Ruf von Hause Slytherin zu retten.

Von allen Seiten des langen Slytherin-Tisches starrten wütende Gesichter auf Draco. Aber sie waren in der Unterzahl gegenüber den Gesichtern, die einfach nur verwirrt aussahen.

"Also gut, ich gebe auf", sagte ein Junge aus dem sechsten Jahr, den Draco nicht erkannte und der ihm gegenüber und zwei Plätze weiter rechts saß. "Warum hast du das getan, Malfoy?"

Obwohl sein Mund sehr trocken war, schluckte Draco nicht. Das wäre ein Zeichen von Angst gewesen. Stattdessen nahm er einen Bissen Karotte, die von allem auf seinem Teller die meiste Feuchtigkeit hatten, und kaute und schluckte, während er so schnell wie möglich nachdachte.

"Weißt du", sagte Draco und machte seine Stimme so schneidend wie möglich - während sein Herz noch heftiger in seiner Brust pochte und alle um ihn herum aufhörten zu reden, um zuzuhören - "es gibt wahrscheinlich eine Möglichkeit, Haus Slytherin noch schlechter aussehen zu lassen, als acht Erstklässlerinnen aus allen vier Häusern anzugreifen, die zusammenarbeiten, um Tyrannen zu stoppen, aber mir fällt nicht ein wie. Auf diese Weise haben wir den Vorteil von dem, was Greengrass und die anderen tun."

Die verwirrten Gesichter blieben verwirrt.

"Was?", sagte der Sechstklässler, und

"Moment, welcher Nutzen?", sagte ein Mädchen aus dem fünften Schuljahr, das rechts neben ihm saß.

"Es lässt das Haus Slytherin besser aussehen", sagte Draco.

Die Slytherins um ihn herum warfen ihm fragende Blicke zu, als hätte er gerade versucht, Algebra zu erklären.

"Besser aussehen für wen?", sagte der Sechstklässler.

"Aber du hast gerade einem Schlammblut geholfen", sagte das Mädchen im fünften Jahr. "Wie soll das denn gut aussehen?"

Dracos Kehle schnürte sich zu. Sein Gehirn erlebte eine grässliche Fehlfunktion, während der ihm nichts einfiel, was er sagen konnte, außer der Wahrheit - Dann:

"Das ist wahrscheinlich eine Art ungeheuer cleverer Plan, den Malfoy ausgeheckt hat", sagte ein Fünftklässler. "Du weißt schon, wie in Die Tragödie des Lichts, wo alles, was wie ein Rückschlag aussieht, Teil des Plans ist. Und es endet damit, dass Grangers Kopf auf einem Stock steckt und niemand ahnt, dass er es war."

"Das macht Sinn", sagte jemand von weiter unten am Tisch, und es wurde viel genickt.

"Weißt du, was der Boss vorhat?" murmelte Vincent mit einem Unterton.

Gregory Goyle antwortete nicht. In seinen Gedanken konnte er ganz deutlich die Stimme seines Meisters hören, der sagte: \emph{Ich kann nicht glauben, dass ich jedes Wort davon geglaubt habe}, an dem Tag, als das Gerücht aufkam, Salazar Slytherins Geist habe Potter und Granger gezeigt, wo man Tyrannen findet.

"Mr. Goyle?", flüsterte Vincent.

Gregory Goyles Lippen formten die Worte "\emph{Oh nein}", aber es kam kein Ton heraus.

…

Hermine hatte das Mittagessen an diesem Tag früh verlassen, aus irgendeinem Grund hatte sie keinen Hunger verspürt. Diese paar Sekunden schrecklicher Demütigung waren ihr immer wieder durch den Kopf gegangen, das Gefühl, wie ihr Gesicht in das Kartoffelpüree gequetscht und dann durch die Luft geschleudert wurde, und dann die Stimme des Slytherin-Jungen, der sagte: "Entschuldige dich bei mir"… es war vielleicht das erste Mal in ihrem ganzen Leben, dass sie Lust hatte, jemanden zu hassen. Der Junge, der sie geworfen hatte (Marcus Flint, so hieß er angeblich) und derjenige, der den Stolperzauber auf sie gezaubert hatte… sie hatte es gefühlt, einen schrecklichen Moment lang hatte sie Harry sagen wollen, dass sie sich nicht dagegen wehren würde, wenn er für sie kreativ werden würde.

Sie war noch keine Minute aus der Großen Halle heraus, als sie hinter sich das Geräusch laufender Füße hörte und sich umdrehte, um zu sehen, wie Daphne auf sie zurannte. Und hörte sich an, was ihr Sonnenschein-Soldat zu sagen hatte…

"Verstehst du nicht?" Daphnes Stimme war kaum leiser als ein Schrei.

"Nur weil jemand nett zu dir ist, heißt das nicht, dass er dein Freund ist! Er ist Draco Malfoy! Sein Vater ist ein Todesser, die Eltern all seiner Freunde sind Todesser - Nott, Goyle, Crabbe, alle um ihn herum, verstehst du das? Sie alle verachten Muggelgeborene, sie wollen, dass jeder wie du stirbt, sie denken, du bist zu nichts anderem gut, als ein Opfer in schrecklichen dunklen Ritualen zu sein! Draco ist der nächste Lord Malfoy, er wurde von Geburt an dazu erzogen, dich zu hassen und er wurde von Geburt an dazu erzogen, zu lügen!" Daphnes graugrüne Augen starrten sie wütend an, verlangten Zustimmung und Verständnis.

"Er -" sagte Hermine zögernd. Sie erinnerte sich an das Dach, an den schrecklichen Ruck, als sie zu fallen begann, an Draco Malfoys Hand, die ihre packte und so fest hielt, dass sie danach blaue Flecken hatte. Sie hatte es ihm zweimal sagen müssen, bevor er sie endlich fallen ließ.

"Vielleicht ist Draco Malfoy nicht wie sie -"

Daphnes Flüstern war fast ein Schrei. "Wenn er dir am Ende nicht zehnmal so viel antut, wie er dir gerade geholfen hat, ist sein Leben vorbei, verstehst du? Ich meine, Lucius Malfoy würde ihn buchstäblich enterben! Weißt du, wie groß die Chance ist, dass er nicht etwas im Schilde führt?"

"Winzig?", fragte Hermine mit leiser Stimme.

"Null!", zischte Daphne. "Ich meine keine! Ich meine weniger als null! Ich meine, die Chance ist so gering, dass man ihn mit drei Vergrößerungszaubern und einem Zeig-mir-Zauber und - und - und einer alten Karte und einem Zentaurenpropheten nicht finden könnte! Jeder in Slytherin weiß, dass er etwas mit dir vorhat und will nicht verdächtigt werden. Ich habe gehört, dass jemand gesehen hat, wie er seinen Zauberstab auf dich gerichtet hat, kurz bevor du gestolpert bist - verstehst du das nicht? Das ist alles Teil von Malfoys Plan!"

…

Draco saß da und aß sein Steak mit gebratenen Blumenkohlröschen und Aschenwindersoße (sie war nicht aus echten Aschenwindern gemacht, sondern schmeckte nur nach Feuer) und versuchte, nicht zu lachen und nicht zu weinen. Er hatte von plausibler Bestreitbarkeit gehört, aber ihm war nicht klar, wie wichtig sie war, bis er herausfand, dass die Malfoys keine hatten.

"Ihr wollt meinen Plan wissen?", sagte Draco. "Hier ist mein Plan. Ich werde nichts tun, und wenn die Leute das nächste Mal denken, dass ich etwas vorhabe, werden sie sich nicht sicher sein."

"Hm…", sagte der Fünftklässler. "Ich glaube, ich glaube dir nicht, das klingt nicht gerissen genug, um es wirklich zu sein -"

"Das ist es, was er dich glauben machen will", sagte die Fünftklässlerin.

…

"Albus", sagte Minerva gefährlich, "hast du das alles geplant?"

"Nun, wenn ich mit den Fingern unter dem Tisch geschnippt hätte, würde ich dir das nicht einfach sagen -"

Die zitternde Hand des Verteidigungsprofessors ließ den Löffel wieder in die Suppe fallen.

….

"Was meinst du damit?", fragte Millicent, "Sie haben sich gewehrt?".

Die beiden saßen im Schneidersitz auf Daphnes Bett, nachdem sie nach dem Mittagessen direkt aus der Großen Halle dorthin gekommen waren.

"Mit den Augen einer Seherin, die durch die Zeit selbst starrt, sah ich dich gewinnen."

Daphne starrte Millicent an, ihre eigenen, bloß sterblichen Augen waren in diesem Moment ziemlich verengt.

"Der Junge hat mit uns gerechnet."

"Nun, ja!", sagte Millicent. "Jeder weiß, dass ihr Tyrannen jagt!"

"Hannah wurde von einem wirklich schmerzhaften Fluch getroffen", sagte Daphne. "Sie musste einen Heiler aufsuchen, Millicent! Wenn wir Freunde sind, hättest du mich vorwarnen müssen!"

"Hör zu, Daphne, ich habe dir gesagt -" Das Slytherin-Mädchen hielt inne, als ob sie versuchte, sich an etwas zu erinnern, und sagte dann: "Ich meine, ich habe dir gesagt, dass das, was ich sehe, eintreten muss. Wenn ich versuche, es zu ändern, wenn irgendjemand versucht, es zu ändern, werden wirklich schreckliche, furchtbare, nicht gute, extrem schlechte Dinge passieren. Und dann wird es trotzdem eintreten. Wenn ich sehe, dass du verprügelt wirst, kann ich dir das nicht sagen, denn dann würdest du versuchen, nicht zu gehen, und dann -" Millicent hielt inne.

"Und dann?" sagte Daphne skeptisch. "Ich meine, was passiert, wenn wir einfach nicht hingehen?"

"Ich weiß es nicht!", sagte Millicent. "Aber wahrscheinlich sieht von Lethifolds aufgefressen zu werden dagegen aus, wie eine Teeparty!"

"Hör mal, selbst ich weiß, dass Prophezeiungen so nicht funktionieren", sagte Daphne und hielt dann inne. "Zumindest funktionieren Prophezeiungen in Theaterstücken nicht so …" Zugegeben, es gab alle möglichen Tragödien, in denen der Versuch, eine Prophezeiung zu umgehen, dazu führte, dass sie eintrat, oder in denen andererseits der Versuch, einer Prophezeiung zu folgen, der einzige Grund war, warum sie eintrat. Aber man konnte Prophezeiungen auf seine eigene Weise eintreten lassen, wenn man klug genug war; oder jemand, der einen genug liebte, konnte seinen Platz einnehmen; oder mit genug Mühe war es möglich, eine Prophezeiung ganz zu brechen… Andererseits erinnerten sich die Seher in Theaterstücken auch nie daran, was sie sahen… Millicent muss Daphnes Zögern gesehen haben, denn das andere Mädchen begann, ein wenig selbstbewusster zu wirken.

"Nun", sagte Millicent scharf, "das ist kein Spiel! Hör mal, ich sage dir, ob ich sehe, dass es ein harter oder ein leichter Kampf sein wird. Aber das ist alles, was ich tun kann, verstehst du? Und wenn ich 'schwer' sage, kannst du nicht \emph{nicht} auftauchen! Oder - oder -" Millicents Augen rollten in ihrem Kopf zurück, und sie stimmte hohl ein: "Diejenigen, die versuchen, ihrem Schicksal ein Schnippchen zu schlagen, werden ein trauriges und düsteres Ende nehmen -"

….

Professor Sprout schüttelte den Kopf, ihr Gesicht wirkte angespannt.

"Aber -", sagte Susan. "Aber Sie haben Harry Potter damals geholfen -"

"Und mir wurde klar gemacht", sagte Professor Sprout mit einer Stimme, die sich anhörte, als würde ihr jemand mit einem Schrumpfungszauber die Kehle zuschnüren, "dass es Professor Snapes Aufgabe ist und nicht meine, im Haus Slytherin für Ordnung zu sorgen - Miss Bones, bitte, Sie müssen das nicht tun, wenn -"

"Doch, das muss ich", sagte Susan unglücklich. "Ich bin eine Hufflepuff, wir müssen loyal sein."

…

"Ein geheimnisvolles Pergament unter deinem Kopfkissen?", fragte Harry Potter und blickte von seinem Platz in der stillen Ecke auf, in der sie gerade lernten. Dann verengten sich die grünen Augen des Jungen. "Es war nicht vom Weihnachtsmann, oder?"

Pause.

"Okay", sagte Hermine. "Ich werde nicht fragen, und du wirst es mir nicht sagen, und wir werden beide so tun, als hättest du das nie gesagt und ich wüsste nichts davon -"

…

Susan näherte sich dem Tisch, sobald das ältere Mädchen allein war, und schaute sich im Hufflepuff-Gemeinschaftsraum um, um sicherzugehen, dass niemand sie beobachtete (so wie es ihr ihre Tante beigebracht hatte, damit es nicht offensichtlich war, dass sie schaute).

"Hey, Susie", sagte die Hufflepuff-Schülerin im siebten Jahr. "Brauchst du schon mehr -"

"Kann ich dich bitte kurz unter vier Augen sprechen?" sagte Susan.

…

Jaime Astorga, Slytherin im siebten Jahr und bis vor kurzem noch ein vielversprechender Emporkömmling im Jugendduellkreis, stand kerzengerade in Professor Snapes Büro, die Zähne fest zusammengebissen und der Schweiß rann ihm den Rücken hinunter.

"Ich erinnere mich deutlich daran", sagte der Hausherr in einem sardonischen Tonfall, "dass ich Sie und einige andere heute Morgen gewarnt habe, dass es gewisse Erstklässlerinnen gibt, die sich als lästig erweisen können, wenn ein Kämpfer unvorsichtig ist und sich überrumpeln lässt."

Professor Snape pirschte sich in einem langsamen Kreis um ihn herum.

"Ich -", sagte Jaime, während ihm noch mehr Schweiß auf die Stirn perlte. Er wusste, wie lächerlich es klang, wie sehr es eine erbärmliche Ausrede war. "Sir, sie hätten nicht in der Lage sein sollen -"

\emph{Eine Erstklässlerin hätte nicht in der Lage sein dürfen, sein Protego zu brechen, egal, was für einen Zauber sie benutzte - Greengrass musste Hilfe gehabt haben - Aber es war ganz klar, dass sein Hausoberhaupt das nicht glauben würde.}

"Oh, da stimme ich zu", murmelte Snape in einem tiefen, drohenden Tonfall. "Sie hätten es nicht tun können. Ich beginne mich zu fragen, ob Mr. Malfoy, was auch immer er vorhat, nicht doch Recht hat, Astorga. Es kann nicht gut für den Ruf des Hauses Slytherin sein, wenn unsere Kämpfer, anstatt ihre Stärke zu demonstrieren, gegen kleine Mädchen verlieren!" Snapes Stimme hatte sich erhoben. "Es ist gut, dass du den guten Geschmack hattest, dich von einem kleinen Mädchen besiegen zu lassen, das ein Slytherin aus einem noblen Haus ist, Astorga, sonst würde ich dir Punkte abziehen!"

Jaime Astorgas Fäuste ballten sich an seiner Seite, aber ihm fiel nichts ein, was er hätte sagen können. Es dauerte einige Zeit, bis Jaime Astorga die Gegenwart seines Hausherrn verlassen durfte.

Und danach sahen nur noch die Wände, der Boden und die Decke das Lächeln von Severus Snape.

…

An diesem Abend bekam Draco Besuch von der Eule seines Vaters, Tanaxu, die nicht grün war, aber nur, weil es so etwas wie grüne Eulen nicht gab. Das Beste, was Vater hatte finden können, war eine Eule mit den reinsten silbernen Federn, mit großen leuchtenden grünen Augen und einem Schnabel, der so scharf und grausam war wie die Reißzähne einer Schlange. Das Pergament, das um Tanaxus Bein gewickelt war, war kurz und bündig:

\emph{Was tust du, mein Sohn?}

Das Pergament, das Draco zurückschickte, war ebenso kurz und lautete:

\emph{Ich versuche, dem Ruf von Haus Slytherin zu schützen, Vater.}

In der gleichen Zeit, die eine Eule brauchte, um von Hogwarts nach Malfoy Manor und wieder zurück zu fliegen, trug die Familieneule eine weitere Nachricht an Draco, und diese lautete nur:

\emph{Was machst du wirklich?}

Draco starrte auf das Pergament, das er aus dem Bein der Eule ausgewickelt hatte. Seine Hände zitterten, als er das Pergament in das Licht des Kamins hielt. 5 Worte, eingeritzt mit schwarzer Tinte, sollten nicht furchterregender sein als der Tod. Es blieb nicht viel Zeit zum Nachdenken. Vater wusste genau, wie lange es dauerte, bis eine Nachricht von Malfoy Manor nach Hogwarts und wieder zurück ging; er würde es wissen, wenn Draco es hinauszögerte, um eine sorgfältige Lüge zu verfassen. Aber Draco wartete noch, bis seine Hand aufhörte zu zittern, bevor er seine Antwort schrieb, die einzige Antwort, die ihm einfiel und die Vater akzeptieren würde.

\emph{Ich bereite mich auf den nächsten Krieg vor.}

Draco wickelte das Pergament um das Bein der Eule und band es fest, dann schickte er Tanaxu aus seinem Zimmer, durch die Hallen von Hogwarts, in die Nacht hinaus.

\emph{Er wartete, aber es kam keine Antwort.}

