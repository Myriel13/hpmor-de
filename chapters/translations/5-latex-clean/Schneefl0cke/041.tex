

\hypertarget{mut}{% \section{42. Mut}\label{mut}}

\textbf{\uline{Mut}}

"Romantisch?" sagte Hermine. "Sie sind beide Jungs!"

"Wow", sagte Daphne und klang ein wenig schockiert.

"Du meinst, Muggel hassen das wirklich? Ich dachte, das wäre nur etwas, das sich die Todesser ausgedacht haben."

"Nein", sagte ein älteres Slytherin-Mädchen, das Hermine nicht erkannte,

"es ist wahr, sie müssen im Geheimen heiraten, und wenn sie jemals entdeckt werden, werden sie zusammen auf dem Scheiterhaufen verbrannt. Und wenn du ein Mädchen bist, das es romantisch findet, verbrennen sie dich auch."

"Das kann doch nicht wahr sein!", wandte ein Gryffindor-Mädchen ein, während Hermine noch überlegte, was sie darauf antworten sollte. "Dann gäbe es keine Muggelmädchen mehr!"

Sie hatte leise weitergelesen, und Harry Potter hatte weiter versucht, sich zu entschuldigen, und bald war Hermine klar geworden, dass Harry möglicherweise zum ersten Mal in seinem Leben begriffen hatte, dass er etwas Unangenehmes getan hatte; und dass Harry, definitiv zum ersten Mal in seinem Leben, Angst hatte, sie als Freundin zu verlieren; und sie hatte begonnen, sich

(a) schuldig zu fühlen und (b) besorgt über die Richtung, in die Harrys zunehmend verzweifelte Angebote gingen.

Aber sie hatte immer noch keine Ahnung, welche Art von Entschuldigung angemessen war, also hatte sie gesagt, dass die Ravenclaw-Mädchen darüber abstimmen sollten - und dieses Mal würde sie das Ergebnis nicht bestimmen, \emph{obwohl sie diesen Teil nicht erwähnt hatte} -, womit Harry sofort einverstanden war.

Am nächsten Tag hatten praktisch alle Ravenclaw-Mädchen über dreizehn Jahren dafür gestimmt, dass Draco Harry fallen lässt. Hermine war ein wenig enttäuscht gewesen, dass es so einfach war, obwohl es offensichtlich fair war. Im Moment jedoch, als sie vor den großen Türen des Schlosses inmitten der halben weiblichen Bevölkerung von Hogwarts stand, begann Hermine zu ahnen, dass hier Dinge vor sich gingen, die sie nicht verstand und von denen sie verzweifelt hoffte, dass keiner ihrer Mitschüler jemals davon erfuhr. Von dort oben konnte man die Details nicht wirklich sehen, nur die allgemeine Tatsache eines Meeres von erwartungsvollen weiblichen Gesichtern.

"Du hast keine Ahnung, worum es hier geht, oder?", sagte Draco und klang amüsiert. Harry hatte eine ganze Reihe von Büchern gelesen, die er nicht lesen sollte, ganz zu schweigen von ein paar Klitterer-Schlagzeilen.

"Junge-der-überlebte schwängert Draco Malfoy?", fragte Harry.

"Okay, du weißt schon, worum es hier geht", sagte Draco. "Ich dachte, Muggel hassen das?"

"Nur die Dummen", sagte Harry. "Aber, ähm, sind wir nicht, äh, ein bisschen jung?"

"Nicht zu jung für sie", sagte Draco. Er schnaubte. "Mädchen!"

Sie gingen schweigend auf den Rand des Daches zu.

"Ich mache das also aus Rache an dir", sagte Draco, "aber warum tust du das?"

Harrys Verstand machte eine blitzschnelle Berechnung, wog die Faktoren ab, ob es zu früh war…

"Ehrlich gesagt?", fragte Harry. "Weil ich sie die eisigen Wände hochklettern lassen wollte, aber ich wollte nicht, dass sie vom Dach fällt. Und, ähm, irgendwie habe ich mich deswegen wirklich schrecklich gefühlt. Ich meine, ich schätze, ich habe nach einer Weile tatsächlich angefangen, sie als meine freundliche Rivalin zu sehen. Also ist das eine echte Entschuldigung bei ihr, keine Verschwörung oder so."

Es gab eine Pause. Dann -

"Ja", sagte Draco. "Ich verstehe."

Harry lächelte nicht. Es könnte das schwierigste Nichtlächeln seines Lebens gewesen sein.

Draco blickte auf die Dachkante und verzog das Gesicht.

"Das wird mit Absicht viel schwieriger sein als aus Versehen, nicht wahr?"

Harrys andere Hand hielt das Dach in einem reflexartig erschrockenen Griff, seine Finger waren weiß auf dem kalten, kalten Stein. Man konnte mit seinem bewussten Verstand wissen, dass man den Federfall-Trank getrunken hatte. Es mit dem Unterbewusstsein zu wissen, war eine ganz andere Sache. Es war genauso beängstigend, wie Harry gedacht hatte, dass es für Hermine sein könnte, und das war gerecht.

"Draco", sagte Harry, es war nicht leicht, seine Stimme zu kontrollieren, aber die Ravenclaw-Mädchen hatten ihnen ein Skript gegeben das Sie spielen mussten, "Du musst mich loslassen!"

"Okay!", sagte Draco und ließ Harrys Arm los.

Harrys andere Hand klammerte an der Kante, und dann, ohne dass eine Entscheidung getroffen wurde, versagten seine Finger, und Harry fiel. Es gab einen kurzen Moment, in dem Harrys Magen versuchte, in seine Kehle hochzuspringen, und sein Körper versuchte verzweifelt, sich zu orientieren, da es keine Möglichkeit gab, dies zu tun. Es gab einen kurzen Moment, in dem Harry spürte, wie der Federfalltrank zu wirken begann und ihn zu verlangsamen begann, eine Art taumelndes, abfederndes Gefühl.

\emph{Und dann zog etwas an Harry und er beschleunigte wieder nach unten, schneller als die Schwerkraft} -

Harrys Mund hatte sich bereits geöffnet und begann zu schreien, während ein Teil seines Gehirns versuchte, an etwas Kreatives zu denken, das er tun könnte, ein Teil seines Gehirns versuchte zu berechnen, wie viel Zeit er noch hatte, um kreativ zu sein, und ein winziger Teil seines Gehirns bemerkte, dass er nicht einmal die Restzeitberechnung zu Ende bringen würde, bevor er auf dem Boden aufschlug -

Harry versuchte verzweifelt, sein Hyperventilieren zu kontrollieren, und es half ihm nicht, das Kreischen all der Mädchen zu hören, die nun haufenweise auf dem Boden und aufeinander lagen.

"Gütiger Himmel", sagte der unbekannte Mann mit der alt aussehenden Kleidung und dem leicht vernarbten Gesicht, der Harry in seinen Armen hielt.

"Von allen Möglichkeiten, die ich mir vorgestellt habe, dass wir uns eines Tages wiedersehen könnten, hätte ich nicht erwartet, dass du vom Himmel fällst."

\emph{Harry erinnerte sich an das Letzte, was er gesehen hatte, den fallenden Körper,} und schaffte es, zu keuchen:

"Professor… Quirrell…"

"In ein paar Stunden wird es ihm wieder gut gehen", sagte der unbekannte Mann, der Harry festhielt. "Er ist nur erschöpft. Ich hätte es nicht für möglich gehalten… er muss zweihundert Schüler niedergeschlagen haben, nur um sicherzugehen, dass er denjenigen erwischt, der dich verhext hat…"

Behutsam setzte der Mann Harry aufrecht auf den Boden und stützte ihn dabei. Harry balancierte sich vorsichtig aus und nickte dem Mann zu.

Der ließ los, und Harry fiel prompt um. Der Mann half ihm, sich wieder aufzurichten. Er achtete darauf, immer zwischen Harry und den Mädchen zu stehen, die sich gerade vom Boden aufhoben, und sein Kopf schaute ständig in diese Richtung.

"Harry", sagte der Mann leise und sehr ernst, "hast du eine Ahnung, welches dieser Mädchen dich vielleicht umbringen wollte?"

"Nicht Mord", sagte eine angespannte Stimme. "Nur Dummheit."

Diesmal war es der unbekannte Mann, der fast umzufallen schien, völliger Schock auf seinem Gesicht.

Professor Quirrell hatte sich bereits von der Stelle aufgerichtet, wo er ins Gras gefallen war.

"Gütiger Himmel!", keuchte der Mann. "Sie sollten nicht -"

"Mr~Lupin, Ihre Bedenken sind unangebracht. Kein Zauberer, egal wie mächtig, wirkt einen solchen Zauber allein durch Kraft. Man muss es durch Effizienz tun."

Professor Quirrell stand jedoch nicht auf.

"Ich danke Ihnen", flüsterte Harry.

Und dann:

"Danke", auch an den Mann, der neben ihm stand.

"Was ist passiert?", fragte der Mann.

"Ich hätte es selbst vorhersehen müssen", sagte Professor Quirrell, seine Stimme knackig vor Missbilligung.

"Einige Mädchen haben versucht, Mr~Potter in ihre eigenen, besonderen Arme zu schließen. Individuell, nehme ich an, dachten sie alle, sie wären sanftmütig."

\emph{Oh}.

"Betrachten Sie es als eine Lektion in Bereitschaft, Mr~Potter", sagte Professor Quirrell.

"Hätte ich nicht darauf bestanden, dass mehr als ein Erwachsener Zeuge dieses kleinen Ereignisses ist und dass wir beide unsere Zauberstäbe gezückt haben, wäre Mr~Lupin nicht in der Lage gewesen, Ihren Sturz abzubremsen, und Sie wären schwer verletzt worden."

"Sir!", sagte der Mann - Mr~Lupin, offenbar. "Sie sollten so etwas nicht zu dem Jungen sagen!"

"Wer ist -" begann Harry zu sagen.

"Die einzige andere Person, die außer mir zum Zuschauen zur Verfügung stand", sagte Professor Quirrell. "Ich stelle Ihnen Remus Lupin vor, der vorübergehend hier ist, um die Schüler im Patronus-Zauber zu unterrichten. Allerdings habe ich gehört, dass Sie beide sich bereits kennen."

Harry musterte den Mann verwirrt. Er hätte sich an das leicht vernarbte Gesicht erinnern sollen, an dieses seltsame, sanfte Lächeln.

"Wo haben wir uns getroffen?", fragte Harry.

"In Godric's Hollow", sagte der Mann. "Ich habe ein paar von deinen Windeln gewechselt."

Mr~Lupins vorübergehendes Büro war ein kleiner steinerner Raum mit einem kleinen hölzernen Schreibtisch, und Harry konnte nichts von dem sehen, worauf Mr~Lupin saß, was vermuten ließ, dass es ein kleiner Hocker war, genau wie der vor seinem Schreibtisch.

Harry vermutete, dass Mr~Lupin nicht lange in Hogwarts bleiben oder dieses Büro oft benutzen würde, und so hatte er den Hauselfen gesagt, sie sollten sich nicht die Mühe machen.

Es sagte etwas über einen Menschen aus, dass er versuchte, Hauselfen nicht zu belästigen. Genauer gesagt, dass er in Hufflepuff einsortiert worden war, da Hermine, soweit Harry wusste, die einzige Nicht-Hufflepuff war, die sich Sorgen machte, Hauselfen zu belästigen.

(Harry selbst fand ihre Bedenken eher albern. Wer auch immer Hauselfen erschaffen hatte, war offensichtlich unsagbar böse gewesen; aber das bedeutete nicht, dass Hermine jetzt das Richtige tat, indem sie empfindungsfähigen Wesen die Plackerei verweigerte, zu der sie geformt worden waren und die Sie sich wünschten.)

"Bitte setz dich, Harry", sagte der Mann leise. Seine formellen Roben waren von minderer Qualität, nicht ganz zerfleddert, aber sichtlich vom Lauf der Zeit auf eine Weise abgenutzt, die einfache Reparaturzauber nicht beheben konnten; \emph{schäbig} war das Wort, das in den Sinn kam.

Und trotzdem war da irgendwie eine Würde an ihm, die nicht durch feine und teure Roben hätte erreicht werden können, die nicht zu feinen Roben gepasst hätte, die das exklusive Eigentum des Schäbigen war. Harry hatte von Bescheidenheit gehört, aber Echte hatte er noch nie gesehen - nur die zufriedene Bescheidenheit von Leuten, die meinten, sie gehöre zu ihrem Stil und wollten, dass man es bemerkte.

Harry nahm auf dem kleinen Holzschemel vor Mr~Lupins kurzem Schreibtisch Platz.

"Danke, dass du gekommen bist", sagte der Mann.

"Nein, danke, dass Sie mich gerettet haben", sagte Harry. "Lassen Sie mich wissen, wenn Sie jemals etwas Unmögliches brauchen."

Der Mann schien zu zögern.

"Harry, darf ich … eine persönliche Frage stellen?"

"Sie können fragen, natürlich", sagte Harry. "Ich habe auch eine Menge Fragen an Sie."

Mr~Lupin nickte. "Harry, behandeln dich deine Stiefeltern gut?"

"Meine Eltern", sagte Harry. "Ich habe vier. Michael, James, Petunia und Lily."

"Ah", sagte Mr~Lupin. Und dann wieder "Ah"

Er schien ziemlich heftig zu blinzeln.

"Ich… das ist gut zu hören, Harry, Dumbledore wollte keinem von uns sagen, wo du bist… Ich hatte Angst, er könnte denken, du müsstest böse Stiefeltern haben, oder so etwas… "

Harry war sich nicht sicher, ob Mr~Lupins Sorge unangebracht war, wenn man seine eigene erste Begegnung mit Dumbledore bedenkt; aber es war alles gut ausgegangen, also sagte er nichts.

"Was ist mit meinem…"

Harry suchte nach einem Wort, das sie nicht höher oder tiefer stellte…

"anderen Eltern? Ich möchte, nun ja, alles wissen."

"Ein hoher Anspruch", sagte Mr~Lupin. Er wischte sich mit einer Hand über die Stirn.

"Nun, lass uns am Anfang beginnen. Als du geboren wurdest, war James so glücklich, dass er seinen Zauberstab nicht berühren konnte, ohne dass er golden glühte, eine ganze Woche lang.

Und auch danach, wann immer er dich im Arm hielt, oder Lily sah, wie sie dich hielt, oder einfach nur an dich dachte, passierte es wieder -"

Ab und zu schaute Harry auf die Uhr und stellte fest, dass schon wieder dreißig Minuten vergangen waren. Er fühlte sich ein wenig schlecht, weil er Remus dazu gebracht hatte, das Abendessen zu verpassen, vor allem, weil Harry selbst einfach später um 19 Uhr zurückkommen würde, aber das war nicht genug, um einen der beiden aufzuhalten. Schließlich nahm Harry genug Mut zusammen, um die kritische Frage zu stellen, während Remus mitten in einem ausgedehnten Diskurs über die Wunder von James' Quidditch steckte, den Harry nicht über das Herz brachte, direkter zu unterbrechen.

"Und das war, als", sagte Remus mit leuchtenden Augen, "James einen dreifachen Rückwärts Mulhanney Taucher mit extra Drehung hinlegte! Die ganze Menge drehte durch, sogar einige der Hufflepuffs jubelten -"

\emph{Ich schätze, man musste dabei sein,} dachte Harry - \emph{nicht dass es irgendwie geholfen hätte, dabei zu sein} - und sagte: "Mr~Lupin?"

Irgendetwas an Harrys Stimme muss den Mann erreicht haben, denn er hielt mitten im Satz inne.

"War mein Vater ein Tyrann der andere gemobbt hat?", fragte Harry.

Remus sah Harry einen langen Moment lang an. "Eine Zeit lang", sagte Remus.

"Er wuchs bald genug darüber hinaus. Wo hast du das gehört?"

Harry antwortete nicht, er versuchte, an etwas Wahres zu denken, das den Verdacht ablenken würde, aber er dachte nicht schnell genug.

"Schon gut", sagte Remus und seufzte. "Ich kann mir denken, wer."

Das schwach vernarbte Gesicht war missbilligend verkniffen.

"Was für eine Sache, um es dir zu erzählen -"

"Hatte mein Vater irgendwelche mildernden Umstände?" sagte Harry.

"Schlechtes Familienleben oder so etwas in der Art? Oder war er einfach… von Natur aus böse?"

\emph{Kalt}? \emph{Ohne Mitleid}?

Remus' Hand strich sein Haar zurück, die erste nervöse Geste, die Harry von ihm gesehen hatte.

"Harry", sagte Remus, "du kannst deinen Vater nicht danach beurteilen, was er als kleiner Junge getan hat!"

"Ich bin ein kleiner Junge", sagte Harry, "und ich beurteile mich selbst."

Remus blinzelte daraufhin zweimal.

"Ich will wissen, warum", sagte Harry. "Ich will es verstehen, denn für mich scheint es keine mögliche Entschuldigung dafür zu geben!"

Die Stimme zitterte ein wenig.

"Bitte sagen Sie mir alles, was Sie darüber wissen, warum er es getan hat, auch wenn es nicht nett klingt."

\emph{Damit ich nicht selbst in die gleiche Falle tappe, was auch immer es ist.}

"Es war das Richtige, wenn man in Gryffindor war", sagte Remus, langsam, widerstrebend.

"Und… ich habe damals nicht so gedacht, ich dachte, es wäre umgekehrt, aber… es könnte Black gewesen sein, der James dazu gebracht hat, wirklich… Black wollte so gerne allen zeigen, dass er gegen Slytherin ist, weißt du, wir alle wollten glauben, dass Blut kein Schicksal ist -"

…

"Nein, Harry", sagte Remus. "Ich weiß nicht, warum Black hinter Peter her war, anstatt zu fliehen. Es war, als ob Black an diesem Tag eine Tragödie um der Tragödie willen machte."

Die Stimme des Mannes war unsicher.

"Es gab keine Andeutung, keine Warnung, wir alle dachten - zu denken, dass er -"

Remus' Stimme brach ab.

Harry weinte, er konnte nicht anders, es tat mehr weh, es von Remus zu hören, als alles, was er jemals selbst empfunden hatte. Harry hatte zwei Eltern verloren, an die er sich nicht erinnern konnte, die er nur aus Erzählungen kannte.

\emph{Remus Lupin hatte alle vier seiner besten Freunde in weniger als vierundzwanzig Stunden verloren; und für den Verlust seines letzten verbliebenen Freundes, Peter Pettigrew, hatte es einfach keinen Grund gegeben.}

"Manchmal tut es immer noch weh, an ihn in Askaban zu denken", beendete Remus, seine Stimme fast ein Flüstern. "Ich bin froh, Harry, dass Todesser keine Besucher empfangen dürfen. Das bedeutet, dass ich mich nicht schämen muss, nicht zu gehen."

Harry musste einige Male schwer schlucken, bevor er sprechen konnte.

"Kannst du mir von Peter Pettigrew erzählen? Er war der Freund meines Vaters, und es scheint - dass ich es wissen sollte, dass ich mich erinnern sollte -"

Remus nickte, das Wasser glitzerte jetzt in seinen eigenen Augen.

"Ich glaube, Harry, wenn Peter gewusst hätte, dass es so enden würde -", die Stimme des Mannes erstickte. "Peter hatte mehr Angst vor dem Dunklen Lord als jeder von uns und wenn er gewusst hätte, dass es so enden würde, glaube ich nicht, dass er es getan hätte. Aber Peter kannte das Risiko, Harry, er wusste, dass das Risiko real war, dass es passieren konnte, und trotzdem blieb er an James und Lilys Seite. Während der ganzen Zeit in Hogwarts habe ich mich immer gefragt, warum Peter nicht nach Slytherin sortiert wurde, oder vielleicht nach Ravenclaw, weil Peter Geheimnisse so sehr liebte, er konnte ihnen nicht widerstehen, er fand Dinge über Leute heraus, Dinge, die sie geheim halten wollten -"

Ein kurzer schiefer Blick ging über Remus' Gesicht.

"Aber er hat diese Geheimnisse nicht benutzt, Harry. Er wollte sie einfach nur wissen. Und dann fiel der Schatten des Dunklen Lords über alles, und Peter stand James und Lily zur Seite und setzte seine Talente ein, und ich verstand, warum der Hut ihn nach Gryffindor geschickt hatte."

Remus' Stimme war jetzt grimmig und stolz.

"Es ist leicht, zu seinen Freunden zu stehen, wenn man ein Held wie Godric ist, mutig und stark, wie die Leute denken, dass Gryffindors sein sollten. Aber wenn Peter mehr Angst hatte als jeder von uns, macht ihn das nicht auch zum Mutigsten?"

"Das tut es", sagte Harry.

Seine eigene Stimme war so erstickt, dass er fast nicht mehr sprechen konnte.

"Wenn Sie Zeit hätten, Mr~Lupin, es gibt noch jemanden, der die Geschichte von Peter Pettigrew hören sollte, ein Schüler im ersten Jahr in Hufflepuff namens Neville Longbottom."

"Der Junge von Alice und Frank", sagte Remus und seine Stimme wurde traurig.

"Ich verstehe. Es ist keine glückliche Geschichte, Harry, aber ich kann sie noch einmal erzählen, wenn du denkst, dass es ihm helfen wird."

Harry nickte. Eine kurze Stille trat ein.

"Hatte Black eine unerledigte Angelegenheit mit Peter Pettigrew?" fragte Harry.

"Irgendetwas, das ihn dazu veranlassen würde, Mr~Pettigrew aufzusuchen, auch wenn es sich nicht um eine tödliche Angelegenheit handelte? Zum Beispiel ein Geheimnis, das Mr~Pettigrew kannte, das Black selbst wissen wollte, oder ihn töten wollte, um es zu verbergen?"

Etwas flackerte in Remus' Augen auf, aber der ältere Mann schüttelte den Kopf und sagte:

"Nicht wirklich."

"Das heißt, da ist etwas", sagte Harry.

Das schiefe Lächeln erschien wieder unter dem Salz-und-Pfeffer-Schnurrbart.

"Du hast selbst ein bisschen von Peter in dir, wie ich sehe. Aber das ist nicht wichtig, Harry."

"Ich bin ein Ravenclaw, ich darf der Versuchung von Geheimnissen nicht widerstehen.

Und", sagte Harry ernster, "wenn es Black wert war, erwischt zu werden, kann ich nicht anders, als zu denken, dass es wichtig sein könnte."

Remus sah ziemlich unbehaglich aus.

"Ich nehme an, ich könnte es dir erzählen, wenn du älter bist, aber wirklich, Harry, es ist nicht wichtig! Nur etwas aus unserer Schulzeit."

Harry hätte nicht genau sagen können, was ihm den Anstoß gab; vielleicht war es etwas über den genauen Ton der Nervosität in Remus' Stimme, oder die Art, wie der Mann gesagt hatte, \emph{wenn du älter bist}, was den plötzlichen Sprung von Harrys Intuition auslöste…

"Eigentlich", sagte Harry, "glaube ich, dass ich es irgendwie schon erraten habe, tut mir leid."

Remus hob die Augenbrauen. "Hast du?" Er klang ein wenig skeptisch.

"Sie waren ein Liebespaar, nicht wahr?"

Es gab eine peinliche Pause.

Remus gab ein langsames, ernstes Nicken von sich.

"Einmal", sagte Remus. "Vor langer Zeit. Eine traurige Angelegenheit, die in einer großen Tragödie endete, so schien es uns allen, als wir jung waren."

Die unglückliche Verwirrung war deutlich auf seinem Gesicht zu sehen.

"Aber ich hatte gedacht, das sei längst vorbei und unter erwachsener Freundschaft begraben, bis zu dem Tag, an dem Black Peter tötete."

