

\hypertarget{was-sich-zu-beschuxfctzen-lohnt-draco-malfoy}{% \section{119. Was sich zu beschützen lohnt: Draco Malfoy}\label{was-sich-zu-beschuxfctzen-lohnt-draco-malfoy}}

\textbf{\uline{Was sich zu beschützen lohnt: Draco Malfoy}}

Anmerkung des Autors E.Y.:

\emph{Lebe wohl, Terry Pratchett, 1948-2015.}

\emph{Deine Charaktere waren eine Inspiration für mich, und jetzt kann ich sehen, wie viel sie mich über Charakterintelligenz der Stufen 1 und 3 gelehrt haben: dass sich Selbstbewusstsein oft als Humor oder als Genre-Sparsamkeit manifestiert; dass ein innerer Funke der Optimierung genauso hell durch Charaktere leuchten kann, denen gesagt (aber nicht gezeigt) wird, dass sie niedrig und dumm sind; dass intelligente Charaktere mit einem Funken Güte und Licht einhergehen können, der sich durch eine Geschichte zieht, statt mit Zynismus. Ich wünschte, ich hätte dich treffen und mit dir über deine Methoden sprechen können. Du wurdest von so vielen geliebt, und sicherlich von mindestens einer Person, die die Fundamente der Realität zerreißen würde, um dich zurückzubringen; aber dein Gehirn ist jetzt tot und warm, und so endet deine Geschichte. Selbst wenn die Sterne im Himmel sterben sollten, Unsere Sünden können niemals ungeschehen gemacht werden. Kein einziger Tod wird vergeben, wenn endlich die letzte erleuchtete Sonne verblasst. Dann, in der kalten und stillen Schwärze, wenn Licht und Materie enden, werden wir einen letzten Blick zurück werfen und auf einen abwesenden Freund anstoßen.}

Der Junge saß in einem Büro in der Nähe der Stelle, wo die einstige stellvertretende Schulleiterin Hof gehalten hatte. Seine Tränen waren schon vor Stunden versiegt. Jetzt blieb nur noch das Warten darauf, was aus ihm werden würde, dem Waisenmündel von Hogwarts, dessen Leben und Glück in den Händen der Feinde seiner Familie lag. Der Junge war in diesen Raum gerufen worden, und er war gekommen, weil es nichts anderes zu tun gab und er nirgendwo anders hingehen konnte. Vincent und Gregory waren von seiner Seite gewichen, zurückgerufen von ihren Müttern zu den eiligen Beerdigungen ihrer Väter. Vielleicht hätte der Junge mit ihnen gehen sollen, aber er konnte sich nicht dazu durchringen, das zu tun. Er wäre nicht in der Lage gewesen, die Rolle eines Malfoys zu spielen. Das Gefühl der Leere, das ihn erfüllte, war so tiefgreifend, dass es nicht einmal für gespielte Höflichkeit Platz ließ.

\emph{Alle waren tot. Sein Vater war tot, und sein Patenonkel Mr~MacNair und sein Ersatzpate Mr~Avery. Sogar Sirius Black, der Cousin seiner Mutter, hatte es irgendwie geschafft, zu sterben, und der letzte Überrest des Hauses Black war für keinen Malfoy ein Freund. Alle waren tot.}

Es klopfte an der Tür des Büros, und als der Junge nicht antwortete, öffnete sich die Tür und gab den Blick frei auf -

"Geh weg", sagte Draco Malfoy zu dem Jungen, der lebte. Er konnte keine Kraft in den Worten aufbringen.

"Das werde ich bald", sagte Harry Potter, als er den Raum betrat. "Aber es gibt eine Entscheidung zu treffen, und nur du kannst sie treffen."

Draco drehte seinen Kopf zur Wand, denn Harry Potter nur anzusehen, kostete ihn mehr Energie, als er noch in sich hatte.

"Du musst entscheiden", sagte Harry, "was danach mit Draco Malfoy geschieht. Ich meine das nicht auf eine unheilvolle Art und Weise. Egal was passiert, du wirst immer noch als reicher Erbe eines edlen und uralten Hauses aufwachsen. Die Sache ist die", Harrys Stimme schwankte jetzt, "die Sache ist die, dass es eine schreckliche Wahrheit gibt, die du nicht kennst, und ich denke immer wieder, wenn du sie wüsstest, würdest du mir sagen, dass ich nicht mehr dein Freund sein soll. Und ich will nicht aufhören, dein Freund zu sein. Aber einfach - es dir nie zu sagen - und immer diese Lüge aufrechtzuerhalten, damit ich weiter dein Freund sein kann - das kann ich nicht tun. Es ist auch falsch. Ich will… ich will das nicht mehr, ich will dich nicht manipulieren. Ich habe dich schon zu sehr verletzt."

\emph{Dann hör auf zu versuchen, mein Freund zu sein, du bist sowieso nicht gut darin.}

Die Worte stiegen in Dracos Bewusstsein auf und kamen nicht über seine Lippen. Er hatte das Gefühl, Harry bereits größtenteils verloren zu haben, durch die Spiele, die Harry mit ihrer Freundschaft gespielt hatte, die Lügen und Manipulationen; und dann noch der Gedanke, allein nach Slytherin zurückzugehen, vielleicht ohne Vincent und Gregory, wenn ihre Mütter das Arrangement beendeten… Draco wollte das nicht tun, er wollte nicht zurück nach Slytherin und sein Leben nur unter Leuten verbringen, die zugestimmt hatten, ins Haus Slytherin sortiert zu werden.

Draco war kaum vernünftig genug, um sich daran zu erinnern, dass viele seiner echten Freunde auch mit Harry befreundet waren, dass Padma eine Ravenclaw war und sogar Theodore ein Chaotischer Leutnant war. Alles, was vom Haus Malfoy übrig geblieben war, war jetzt eine Tradition; und diese Tradition besagte, dass es nicht klug war, dem Sieger des Krieges zu sagen, er solle verschwinden und aufhören, mit einem befreundet sein zu wollen.

"Na schön", sagte Draco mit leerer Stimme. "Sag es mir."

"Genau das werde ich tun", sagte Harry. "Und dann wird die Schulleiterin kommen, nachdem ich gegangen bin, und deine letzte halbe Stunde Erinnerung löschen. Aber vorher, wenn du die ganze Wahrheit kennst, kannst du dich entscheiden, ob du immer noch mit mir befreundet sein willst."

Harrys Stimme zitterte.

"Ähm. Laut den Aufzeichnungen, die ich durchgelesen habe, bevor ich hierher kam, begann die Geschichte in Wirklichkeit 1926 mit der Geburt eines Halbblut-Zauberers namens Tom Morfin Riddle. Seine Mutter starb bei der Geburt und er wuchs in einem Muggelwaisenhaus auf, bis ihm Professor Dumbledore den Hogwarts-Brief brachte…"

Der Junge-der-lebte sprach weiter, Worte, die auf das, was von Dracos Verstand noch übrig war, einschlugen wie einstürzende Häuser.

\emph{Der Dunkle Lord war ein Halbblut gewesen.}

\emph{Er hatte nie auch nur den Bruchteil einer Sekunde an die Reinheit des Blutes geglaubt.}

\emph{Tom Riddle hatte sich die Idee von Lord Voldemort als schlechten Scherz ausgedacht. Die Todesser sollten gegen David Monroe verlieren, damit Monroe die Macht übernehmen konnte.}

\emph{Nachdem er das aufgegeben hatte, hatte Tom Riddle weiter Voldemort gespielt, anstatt zu versuchen, zu gewinnen, weil es ihm Spaß gemacht hatte, die Todesser herumzukommandieren.}

\emph{Voldemort benutzte mich, um zu versuchen, Vater meinen Mordversuch anzuhängen, dann benutzte er mich wieder, um den Stein der Weisen zu suchen.}

Draco konnte sich daran nicht mehr erinnern, aber man hatte ihm gesagt, er sei nur ein Bauer an der Seite von Professor Sprout gewesen und man würde ihn nicht anklagen. Und dann der letzte Schrecken.

"Du…", flüsterte Draco Malfoy. "Du…"

"Ich bin derjenige, der deinen Vater und all die anderen Todesser letzte Nacht getötet hat. Man hatte ihnen befohlen, das Feuer auf mich zu eröffnen, sobald ich etwas tue, also musste ich sie töten, um eine Chance zu haben, mit Voldemort fertig zu werden, der eine Gefahr für die ganze Welt war." Harry Potters Stimme war angespannt. "Ich habe nicht an dich und Theodore und Vincent und Gregory gedacht, aber wenn ich es getan hätte, hätte ich es trotzdem getan. Mein Verstand hat es geschafft, erst im Nachhinein zu erkennen, dass Mr~White Lucius war, aber wenn ich es erkannt hätte, hätte ich es trotzdem nicht riskiert, ihn am Leben zu lassen, für den Fall, dass er zauberstabfreie Magie kennt. Der Gedanke kam mir schon lange vorher, dass es in Bezug auf die politische Landschaft ziemlich praktisch wäre, wenn alle Todesser plötzlich sterben würden. Ich habe die Todesser schon immer für schreckliche Menschen gehalten, viel mehr, als ich es dir gegenüber jemals zugegeben habe, seit dem ersten Tag, an dem wir uns kennengelernt haben. Aber wenn dein Vater nicht da gewesen wäre und ich einen Knopf gehabt hätte, der ihn aus der Ferne hätte töten können, hätte ich den Knopf nicht nur aus politischen Gründen gedrückt. Die Art, wie ich darüber denke, was ich getan habe, und ob ich Reue empfinde… nun, es gibt einen Teil von mir, der in allgemeinem Entsetzen darüber schreit, jemanden getötet zu haben. Und ein anderer Teil, der sagt, dass die Todesser aus moralischer Sicht ihr Leben weggegeben haben, als sie sich mit Voldemort verbündeten. Sie haben ihre Zauberstäbe zuerst auf mich gerichtet, blah blah und so weiter. Aber im Moment fühle ich mich einfach nur schlecht wegen dem, was ich dir angetan habe. Schon wieder. Ich habe das Gefühl", Harry Potters Stimme wackelte ein wenig, "dass alles, was ich tue, dir nur weh tut, dass du trotz all meiner guten Absichten immer nur etwas verloren hast, wenn du in meiner Nähe bist, also wenn du mir sagst, ich soll mich danach ganz von Draco Malfoy fernhalten, dann werde ich das tun. Und wenn du willst, dass ich diesmal wirklich versuche, dein Freund zu sein, ohne dich jemals wieder zu manipulieren, ohne dich jemals wieder zu benutzen oder zu riskieren, dich wieder zu verletzen, dann werde ich das tun, ich schwöre es."

Im nächsten Moment weinte Lord Malfoy, offen vor seinem Feind, Anstand und Gelassenheit aufgegeben, weil er niemanden mehr hatte, dem er es zuliebe behalten konnte.

\emph{Eine Lüge. Eine Lüge. Alles war eine Lüge gewesen, es war alles Lüge auf Lüge, Lüge auf Lüge, Lüge auf Lüge -}

"Du solltest sterben", presste Draco hervor. "Du solltest sterben, weil du Vater getötet hast." Die Worte erfüllten ihn nur mit noch mehr Leere, aber sie mussten gesagt werden.

Harry Potter schüttelte nur den Kopf. "Und wenn das keine Option ist?"

"Dann sollte man dir wehtun."

Harry schüttelte nur wieder den Kopf.

Der Junge-der-lebte drängte den Lord Malfoy zu seiner Entscheidung.

Der Lord Malfoy weigerte sich, sie zu geben. Er konnte es nicht sagen, brachte es nicht über sich, es zu sagen, so oder so. Er wollte nicht, dass der Sieger des Krieges und ihre gemeinsamen Freunde ihn im Stich ließen, und er wollte Harry auch nicht die Absolution erteilen, die er wollte. Also verweigerte Draco Malfoy die Antwort, und damit endete die Zeit der Erinnerung an dieses Ich.

Der Junge saß in einem Büro in der Nähe der Stelle, an der die einstige stellvertretende Schulleiterin Hof gehalten hatte. Seine Tränen waren schon vor Stunden versiegt. Jetzt gab es nur noch das Warten darauf, was aus ihm werden würde, dem Waisenmündel von Hogwarts, dessen Leben und Glück in den Händen der Feinde seiner Familie lag. Der Junge war in diesen Raum gerufen worden, und er war gekommen, weil es nichts anderes zu tun gab und er nirgendwo anders hingehen konnte. Vincent und Gregory waren von seiner Seite gewichen, zurückgerufen von ihren Müttern zu den eiligen Beerdigungen ihrer Väter. Vielleicht hätte der Junge mit ihnen gehen sollen, aber er konnte sich nicht dazu durchringen, das zu tun. Er wäre nicht in der Lage gewesen, die Rolle eines Malfoys zu spielen. Das Gefühl der Leere, das ihn erfüllte, war so tiefgreifend, dass es nicht einmal Platz für Lügen ließ.

\emph{Alle waren tot. Alle waren tot, und es war von Anfang an alles vergeblich gewesen.}

Es klopfte an der Bürotür, und nach einer höflichen Pause öffnete sie sich und gab den Blick auf Schulleiterin McGonagall frei, die genauso gekleidet war wie damals, als sie noch Professorin war.

"Mr~Malfoy?", sagte der siegreiche Feind seiner Familie. "Bitte kommen Sie mit mir."

Lustlos erhob sich Draco und folgte ihr aus dem Büro. Der Anblick von Harry Potter, der neben ihr wartete, ließ ihn kurz innehalten, doch dann schaltete sein Verstand einfach aus.

"Hier ist die letzte Sache", sagte Harry Potter. "Ich fand es in einem gefalteten Pergament, auf dessen Außenseite stand, dass es die letzte Waffe sei, die gegen das Haus Malfoy eingesetzt werden sollte, und dass ich es nicht weiter lesen sollte, bis der ganze Krieg in der Schwebe sei. Ich wollte es dir vorher nicht sagen, weil ich dachte, es könnte deine Entscheidung ungerechtfertigt beeinflussen. Wenn du ein guter Mensch wärst, der nie tötet oder lügt, aber du müsstest das eine oder das andere tun, was wäre schlimmer?"

Draco ignorierte ihn und ging in Begleitung von Schulleiterin McGonagall weiter, wobei er Harry zurückließ, der ihm traurig hinterherblickte.

Sie kamen zum alten Büro der Schulleiterin, wo sie mit einem Schwung ihres Zauberstabs das Floo-Feuer entzündete, zu der grünen Flamme "Gringotts-Reisebüro" sagte und nach einem festen Blick in seine Richtung hindurch trat. In Ermangelung einer anderen Möglichkeit folgte ihr Draco Malfoy.

…

Sie lag im Bett und fühlte sich lustloser als sonst an diesem Morgen. Sie war zu früh aufgewacht und die Sonne begann gerade aufzugehen - obwohl das direkte Sonnenlicht von den Wolkenkratzern, die ihr Haus überschatteten, blockiert wurde. Ein schwacher Hauch von Kater nagte an ihren Schläfen, trocknete ihren Mund aus; sie versuchte, sparsam mit dem Alkohol umzugehen (obwohl sie nicht wusste, warum sie sich die Mühe machte), aber gestern hatte sie sich gefühlt…. noch deprimierter als sonst, als ob sie etwas verloren hätte, irgendwie.

Nicht zum ersten Mal, nicht zum hundertsten Mal, dachte sie daran, umzuziehen - nach Adelaide, nach Perth, vielleicht nach Perth Amboy, wenn es sein musste. Sie hatte immer das Gefühl, dass es einen anderen Ort gab, an dem sie sein sollte; aber während sie mit den Zahlungen der Versicherung ein bequemes Leben führen konnte, konnte sie sich keinen Luxus leisten. Sie konnte es sich nicht leisten, auf der Suche nach einem Ort, der ihrem unbefriedigten Gefühl der Zugehörigkeit entsprach, durch die Welt zu ziehen. Sie hatte lange genug ferngesehen, sie hatte sich genug Reiseberichte ausgeliehen, um zu wissen, dass der Videorekorder ihr nirgendwo ein besseres Gefühl der Zugehörigkeit vermittelte als in Sydney. Sie fühlte sich wie eingefroren, in der Zeit stehen geblieben, seit dem Verkehrsunfall, der ihr die Erinnerungen geraubt hatte - nicht nur an eine tote Familie, die ihr jetzt nichts mehr bedeutete, sondern auch Erinnerungen daran, wie ein Herd funktionierte.

Sie ahnte, nein, sie wusste, dass, worauf auch immer ihr Herz wartete, welcher Schlüssel auch immer in ihr gedreht werden musste, damit ihr Leben wieder in Bewegung kam, es war eine weitere Sache, die sie an diesen betrunkenen Minivan Fahrer verloren hatte. Sie dachte fast jeden Morgen darüber nach und versuchte zu erraten, was ihr fehlte, was sie vermisste, was ihrem Leben und ihrem Geist fehlte.

Jemand läutete an ihrer Tür. Sie stöhnte und drehte ihren Kopf weit genug, um auf den LED-Wecker an der Seite ihres Bettes zu schauen.

\textbf{6:31} Uhr, stand da, und der Punkt \textbf{AM} leuchtete.

\emph{Ernsthaft? Nun, dieser Idiot konnte warten, während sie in ihrem eigenen Tempo aus dem Bett taumelte.}

Sie taumelte aus dem Bett, ignorierte die Türklingel, als sie wieder klingelte, ging ins Bad und zog sich an. Sie kletterte die Treppe hinunter und ignorierte das ständig nagende Gefühl, dass jemand anderes für sie an die Tür gehen sollte.

"Wer ist da?", rief sie zu der geschlossenen Tür; die Tür hatte ein Guckloch, aber es war beschlagen.

"Sind Sie Nancy Manson?", kam eine Frauenstimme, die mit einem präzisen schottischen Akzent sprach.

"Ja", sagte sie behutsam.

"Eunoe", sprach die schottische Stimme, und Nancy sprang erschrocken zurück, als ein Lichtblitz aus der Tür kam und sie traf und …

Nancy schwankte und legte eine Hand an ihre Stirn. Lichtblitze, die einfach durch Türen gehen und Leute treffen, das war… das war… das war nicht besonders überraschend…

"Würden Sie bitte die Tür öffnen?", sagte die Stimme der schottischen Frau. "Der Krieg ist vorbei und Ihre Erinnerungen sollten in Kürze zurückkehren. Es ist jemand hier, den Sie sehen sollten."

\emph{Meine Erinnerungen} - Nancys Kopf fühlte sich bereits verstopft an, als würde sie gleich anfangen, etwas aus ihrem Gehirn zu hacken, aber sie schaffte es, die Hand auszustrecken und die Tür aufzureißen.

Vor ihr stand eine Frau, die wie eine \emph{(ganz normale)} Hexe gekleidet war, von der schwarzen Robe bis zum hohen spitzen Hut - und neben ihr ein Junge mit kurzen weißblonden Haaren, der \emph{(ganz normale)} dunkle, grün gesäumte Roben trug und sie mit heruntergefallener Kinnlade und großen Augen, die sich mit Tränen zu füllen begannen, anstarrte.

\emph{Grün gesäumte Roben und weiß-blondes Haar…}

Etwas Warmes regte sich in ihrer Erinnerung. Sie spürte, wie ihr das Herz in die Kehle stieg, als ihr klar wurde, dass das, wonach sie in den letzten zehn Jahren gesucht hatte, genau in diesem Augenblick vor ihr liegen könnte. Irgendwo tief in ihrem Inneren knackte das Eis um ihr Herz, das Stück von ihr, das so lange angehalten worden war, bereitete sich darauf vor, sich wieder zu bewegen.

Der Junge starrte sie an, sein Mund arbeitete lautlos.

Ein geheimnisvoller Name kam ihr in den Sinn, stieg ihr auf die Lippen.

"Lucius?", flüsterte sie.

