

\hypertarget{selbstverwirklichung-teil-1}{% \section{66. Selbstverwirklichung, Teil 1}\label{selbstverwirklichung-teil-1}}

\textbf{\uline{Selbstverwirklichung, Teil 1}}

\hfill\break \emph{Zögern ist immer leicht, selten nützlich.} So hatte es ihm der Verteidigungsprofessor gesagt; und während man über die Details des Sprichworts streiten konnte, verstand Harry die Schwäche eines Ravenclaw gut genug, um zu wissen, dass man versuchen musste, seine eigenen Spitzfindigkeiten zu beantworten.

\emph{Verlangten manche Pläne nach Warten?}\\ Ja, viele Pläne erforderten ein verzögertes Handeln; aber das war nicht dasselbe wie ein Zögern bei der Wahl. Nicht das Zögern, weil man den richtigen Zeitpunkt kannte, um das Notwendige zu tun, sondern das Zögern, weil man sich nicht entscheiden konnte - es gab keinen schlauen Plan, der das erforderte.

\emph{Brauchte man manchmal mehr Informationen, um sich zu entscheiden?}\\ Ja, aber das konnte auch zu einer Ausrede für das Zögern werden; und es wäre verlockend, zu zögern, wenn man vor der Wahl zwischen zwei schmerzhaften Alternativen stünde und das Nichtwählen den mentalen Schmerz für eine Weile vermeiden würde. Man würde sich also eine Information heraussuchen, die man nicht so einfach bekommen könnte, und behaupten, dass man ohne diese Information unmöglich eine Entscheidung treffen könnte; das wäre deine Ausrede. Obwohl, wenn du wüsstest, welche Informationen du benötigst, wann und wie du diese Informationen erhalten würdest und was du in Abhängigkeit von jeder möglichen Beobachtung tun würdest, dann wäre das als Ausrede für das Zögern weniger verdächtig. Wenn du nicht nur zögerst, sollte man in der Lage sein, im Voraus zu entscheiden, was man tun würde, sobald man die zusätzlichen Informationen hatte, die man angeblich benötigt.

\emph{Wenn der Dunkle Lord wirklich da draußen ist, wäre es dann klug, auf Professor Quirrells Plan einzugehen, jemanden zu beauftragen, sich für den Dunklen Lord auszugeben?}\\ Nein. Eindeutig nein. Auf gar keinen Fall.

\emph{Und wenn Harry genau wüsste, dass der Dunkle Lord nicht wirklich da draußen ist...} In dem Fall...

Das Büro des Verteidigungsprofessors war ein kleiner Raum, zumindest heute; es hatte sich verändert, seit Harry es das letzte Mal gesehen hatte, der Stein des Raumes wurde dunkler, polierter. Hinter dem Schreibtisch des Verteidigungsprofessors stand das einzige leere Bücherregal, das den Raum immer zierte, ein hohes Bücherregal, das fast vom Boden bis zur Decke reichte, mit sieben leeren Holzregalen. Harry hatte Professor Quirrell nur ein einziges Mal gesehen, wie er ein Buch aus diesen leeren Regalen nahm, und nie, wie er ein Buch zurückstellte. Die grüne Schlange schwankte über der Sitzfläche des Stuhls hinter dem Schreibtisch des Verteidigungsprofessors, die lidlosen Augen starrten Harry blinzelnd aus der Nähe seiner eigenen Augenhöhe an. Sie waren jetzt mit zweiundzwanzig Zaubern geschützt, alle, die man in Hogwarts wirken konnte, ohne die Aufmerksamkeit des Schulleiters zu erregen.

„\emph{Nein}“, zischte Harry.

Die grüne Schlange legte den Kopf schief und neigte ihn leicht; die Geste vermittelte keine Emotion, nicht dass Harrys Parselmund-Talent ihm etwas vermittelte.

„\textbf{\emph{Wassss ist der Grund?}}“, sagte die grüne Schlange.

„\emph{Zu rissikant}“, sagte Harry schlicht.

Das stimmte, ob der Dunkle Lord nun da draußen war oder nicht. Der Zwang, sich im Voraus zu entscheiden, hatte ihn erkennen lassen, dass er die unbeantwortete Frage nur als Vorwand zum Zögern benutzt hatte; die vernünftige Entscheidung war so oder so dieselbe.

Einen Moment lang schienen die dunklen, entsteinten Augen schwarz zu glänzen, einen Moment lang klaffte das geschuppte Maul auf, um die Reißzähne zu entblößen. „\textbf{\emph{Ich denke, du hast aus früheren Misserfolgen falsch gelernt, mein Junge. Meine Pläne sind nicht daran gewöhnt, zu scheitern, und der letzte wäre einwandfrei verlaufen, wenn nicht deine eigene Dummheit gewesen wäre. Die richtige Lektion ist es, den Schritten zu folgen, die von älteren und weiseren Sslytherin für dich festgelegt wurden, um deine wilden Impulssse zu zähmen.}}“

„\emph{Lektion die ich lernte issst es, keine Intrigen zu verssssuchen die Mädchen-Freund denken lasssssen würde ich bin bösssse oder Jungen-Freund denken lasssssen würde ich sei dumm.}“, schnauzte Harry zurück.

Er hatte eine abwartende Antwort geplant, aber irgendwie waren die Worte einfach herausgerutscht.\\ Das \textbf{\emph{sssss}}-ing-Geräusch, das von der Schlange kam, hörte Harry nicht als Worte, sondern nur als pure Wut.

Einen Moment später: „\textbf{\emph{Du hast es gesagt -}}“

„\emph{Natürlich nicht! Aber ich weiß, was sie sagen würde.}“

Es gab eine lange Pause, während der Schlangenkopf schwankte und Harry anstarrte; wieder kam keine erkennbare Emotion durch, und Harry fragte sich, was Professor Quirrell denken könnte, dass er so lange zum Denken brauchen würde.

„\textbf{\emph{Du kümmersssst dich wirklich um dassss was beide denken?}}“, kam das letzte Zischen der Schlange. „\textbf{\emph{Wahre Jünglinge sind diese beiden, nicht wie du. Könnten nicht erwachsene Dinge abwägen.}}“

„\emph{Hätten es besser gemacht als ich}“, zischte Harry. „\emph{Junge-Kind-Freund hätte nach geheimen Motiven geforscht, bevor er sich bereit erklärt hätte, die Frau zu retten -}"

„\textbf{\emph{Gut, dass du das jetzt verstehst}}“, zischte die Schlange kalt. „\textbf{\emph{Frage immer nach Vorteil von anderen. Lerne als Nächstes, immer auf deinen eigenen Vorteil zu achten. Wenn mein Plan nicht zu deinem Vorteil ist, was ist dann dein Plan?}}“

„\emph{Wenns notwendig ist -- bleibe ich sechs Jahre in der sSchule und sstudiere. Hogwarts scheint ein guter Ort zum Verweilen zu sein. Bücher, Freunde, seltsames, aber schmackhaftes Essen.}“ Harry wollte kichern, aber es gab keine Geste in Parsel für die Art von Lachen, die er ausdrücken wollte.

Die Augenhöhlen der Schlange schienen fast schwarz zu sein. „\textbf{\emph{Dasss lässt sich leicht sagen. Ssolche wie du und ich, wir dulden keine Unverschämtheiten und Resssstriktionen. Du wirst die Geduld lange vor dem siebten Jahr verlieren, vielleicht sogar vor dem Ende dieses Jahres. Ich werde entsprechend planen.}}“

Und bevor Harry ein weiteres Wort in Parseltongue zischen konnte, saß die Menschengestalt von Professor Quirrell wieder auf seinem Stuhl. „Also, Mr. Potter“, sagte der Verteidigungsprofessor, seine Stimme so ruhig, als ob sie nichts Wichtiges besprochen hätten, als ob das ganze Gespräch gar nicht stattgefunden hätte, „ich habe gehört, dass Sie angefangen haben, sich im Duellieren zu üben. Nicht die wertlose Art mit Regeln, hoffe ich?“

….\\ Hannah Abbott sah so entnervt aus, wie Hermine sie noch nie gesehen hatte (außer am Tag des Phönix, dem Tag, an dem Bellatrix Black entkommen war, was aber für niemanden zählen sollte). Das Hufflepuff-Mädchen war während des Abendessens an den Ravenclaw-Tisch gekommen, hatte Hermine auf die Schulter getippt und sie fast weggezerrt -

„Neville und Harry Potter lernen Duellieren bei Mr. Diggory!“ platzte Hannah heraus, sobald sie ein paar Schritte vom Tisch entfernt waren.

„Wer?“, fragte Hermine.

„Cedric Diggory!“, sagte Hannah. „Er ist der Kapitän unserer Quidditch-Mannschaft und General einer Armee, und er belegt alle Wahlfächer und bekommt bessere Noten als alle anderen, und ich habe gehört, dass er im Sommer von professionellen Tutoren das Duellieren lernt und einmal zwei Siebtklässler besiegt hat, und sogar einige Lehrer nennen ihn den Super-Hufflepuff, und Professor Sprout sagt, wir sollten ihn alle emu, äh, emudieren oder so ähnlich, und -"

Nachdem Hannah endlich nach Luft geschnappt hatte (die Liste war schon eine Weile lang), gelang es Hermine, ein Wort einzufügen.

„Sonnenschein-Soldat Abbott!“, sagte Hermine. „Beruhigen Sie sich. Wir werden doch nicht gegen General Diggory kämpfen, oder? Sicher, Neville lernt, um uns zu schlagen, aber wir können auch lernen -“

„Siehst du es nicht?“\\ Hannah kreischte und erhob ihre Stimme viel lauter, als es nötig gewesen wäre, wenn sie das Gespräch vor all den Ravenclaws, die sie beobachteten, geheim halten wollten. „Neville lernt nicht, um uns zu schlagen! Er übt, damit er gegen Bellatrix Black kämpfen kann! Sie werden durch uns hindurchgehen wie ein Klatscher durch einen Stapel Pfannkuchen!“

Die Sonnenschein-Generalin warf ihrem Soldaten einen Blick zu. „Hör mal“, sagte Hermine, „ich glaube nicht, dass ein paar Wochen Training jemanden zu einem unbesiegbaren Kämpfer machen. Außerdem wissen wir bereits, wie man mit unbesiegbaren Kämpfern umgeht. Wir werden das Feuer auf sie konzentrieren und sie werden genau wie Draco zu Boden gehen.“

Das Hufflepuff-Mädchen sah sie mit einer Mischung aus Bewunderung und Skepsis an. „Bist du nicht einmal, du weißt schon, besorgt?“

„Oh, ehrlich!?“, sagte Hermine. Manchmal war es schwer, die einzige vernünftige Person im ganzen Schuljahr zu sein. „Hast du noch nie das Sprichwort gehört, dass das Einzige, was wir zu fürchten haben, die Angst selbst ist?“

„Was?“, sagte Hannah. „Das ist verrückt, was ist mit Lethifolds, die in der Dunkelheit lauern, und mit dem Imperius-Fluch, und schrecklichen Verwandlungsunfällen und -"

„Ich meine“, sagte Hermine, und Verzweiflung sickerte in ihre nun erhobene Stimme, sie hörte so etwas schon die ganze Woche, „wie wäre es, wenn wir warten, bis die Chaos-Legion uns tatsächlich zerquetscht, um so viel Angst vor ihnen zu haben, und hast du gerade '\emph{Gryffindors}' unter deiner Stimme gemurmelt?“

Ein paar Augenblicke später ging Hermine zurück zu ihrem Platz am Tisch, mit einem süßen Lächeln, das auf ihr junges Gesicht gepflastert war, es war nicht der schrecklich kalte Blick von Harrys dunkler Seite, aber es war das furchterregendste Gesicht, das sie machen konnte.

...\\ Harry Potter ging zu Boden. „Das ist verrückt“, keuchte Neville, mit dem winzigen Rest an Atem, den er noch erübrigen konnte, weil er völlig außer Atem war.

„Das ist brillant!“, sagte Cedric Diggory. Die Augen des Super-Hufflepuffs leuchteten vor manischem Enthusiasmus und glänzten wie der Schweiß auf seiner Stirn, als er mit den Füßen durch den Tanz einer seiner Duellstellungen stampfte. Seine normalerweise leichten Schritte hatten sich in schwerere Stampfbewegungen verwandelt, was vielleicht etwas mit den verwandelten Metallgewichten zu tun hatte, die sie alle an ihren Armen und Beinen befestigt und sich über die Brust geschnallt hatten. „Woher haben Sie diese Ideen, Mr. Potter?“

„Ein seltsamer alter Laden... in Oxford... und ich werde dort nie... wieder einkaufen...“

\emph{Er schlug hart auf dem Boden auf}.

