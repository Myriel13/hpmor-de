

\hypertarget{koordinationsprobleme-teil-3}{% \section{35. Koordinationsprobleme, Teil 3}\label{koordinationsprobleme-teil-3}}

\textbf{\uline{Koordinationsprobleme, Teil 3}}

Sie waren in das Büro des Verteidigungsprofessors gegangen, und Professor Quirrell hatte die Tür versiegelt, bevor er sich in seinem Stuhl zurücklehnte und sprach. Die Stimme des Verteidigungsprofessors war sehr ruhig, und das beunruhigte Harry ein gutes Stück mehr, als wenn Professor Quirrell geschrien hätte.

"Ich versuche", sagte Professor Quirrell leise, "der Tatsache Rechnung zu tragen, dass Sie jung sind. Dass ich selbst in demselben Alter ein ganz außergewöhnlicher Narr war. Sie sprechen mit erwachsenem Stil und mischen sich in erwachsene Spiele ein, und manchmal vergesse ich, dass Sie nur ein Einmischer sind. Ich hoffe, Mr. Potter, dass Ihre kindische Einmischung Sie nicht gerade umgebracht, Ihr Land ruiniert und den nächsten Krieg verloren hat."

Es fiel Harry sehr schwer, seine Atmung zu kontrollieren.

"Professor Quirrell, ich habe viel weniger gesagt, als ich eigentlich sagen wollte, aber ich musste etwas sagen. Ihre Vorschläge sind äußerst alarmierend für jeden, der auch nur die geringste Vertrautheit mit der Muggelgeschichte des letzten Jahrhunderts hat. Die italienischen Faschisten, ein paar sehr böse Leute, haben ihren Namen von der Fasces, einem Bündel zusammengebundener Ruten, das die Idee symbolisieren soll, dass Einigkeit Stärke bedeutet -"

"Die bösen italienischen Faschisten glaubten also, dass Einigkeit stärker ist als Spaltung", sagte Professor Quirrell. Schärfe begann sich in seine Stimme zu schleichen. "Vielleicht haben sie auch geglaubt, dass der Himmel blau ist, und haben eine Politik befürwortet, bei der man keine Steine auf Köpfe fallen lässt."

\emph{Umgekehrte Dummheit ist keine Intelligenz; der dümmste Mensch der Welt mag sagen, dass die Sonne scheint, aber das macht es nicht dunkel…}

"Gut, du hast recht, das war ein ad hominem Argument, es ist nicht falsch, weil die Faschisten es gesagt haben. Aber Professor Quirrell, man kann nicht jeden in einem Land das Zeichen eines Diktators nehmen lassen! Das ist ein einziger Punkt des Versagens! Ich will's mal so ausdrücken. Angenommen, der Feind imperiusiert einfach denjenigen, der das Mal kontrolliert -"

"Mächtige Zauberer sind nicht so leicht zu imperiusieren", sagte Professor Quirrell trocken.

"Und wenn man keinen würdigen Anführer findet, ist man ohnehin dem Untergang geweiht. Aber würdige Anführer gibt es; die Frage ist, ob das Volk ihnen folgen wird."

Harry fuhr sich frustriert mit den Händen durch die Haare. Er wollte eine Auszeit einlegen und Professor Quirrell dazu bringen, \emph{Der Aufstieg und Fall des dritten Reiches} zu lesen und dann das Gespräch von vorne zu beginnen.

"Ich nehme nicht an, dass, wenn ich vorschlagen würde, dass Demokratie eine bessere Regierungsform ist als Diktatur -"

"Ich verstehe", sagte Professor Quirrell.

Seine Augen schlossen sich kurz, dann öffneten sie sich.

"Mr. Potter, die Dummheit von Quidditch ist für Sie durchschaubar, weil Sie nicht mit der Verehrung des Spiels aufgewachsen sind. Wenn Sie noch nie etwas von Wahlen gehört hätten, Mr. Potter, und Sie einfach nur sehen würden, was da ist, würde Ihnen das, was Sie sehen, nicht gefallen. Sehen Sie sich unseren gewählten Zaubereiminister an. Ist er der weiseste, der stärkste, der größte unserer Nation? Nein, er ist ein Possenreißer, der sich im Besitz von Lucius Malfoy befindet.

Die Zauberer sind zur Wahl gegangen und haben sich zwischen Cornelius Fudge und Tania Leach entschieden, die in einem großartigen und unterhaltsamen Wettbewerb gegeneinander angetreten waren, nachdem der Tagesprophet, der ebenfalls von Lucius Malfoy kontrolliert wird, entschieden hatte, dass sie die einzigen ernsthaften Kandidaten waren.

Dass Cornelius Fudge wirklich als der beste Anführer gewählt wurde, den unser Land zu bieten hat, ist keine Behauptung, die irgendjemand mit einem geraden Gesicht machen könnte.

In der Muggelwelt ist es nicht anders, nach dem, was ich gehört und gesehen habe; in der letzten Muggelzeitung, die ich gelesen habe, stand, dass der vorherige Präsident der Vereinigten Staaten ein Filmschauspieler im Ruhestand gewesen sei. Wenn Sie nicht mit Wahlen aufgewachsen wären, Mr. Potter, wären sie für Sie so durchschaubar albern wie Quidditch."

Harry saß mit offenem Mund da und rang nach Worten.

"Der Sinn von Wahlen ist nicht, den einen besten Anführer hervorzubringen, sondern die Politiker so in Angst vor den Wählern zu halten, dass sie nicht völlig böse werden, wie es Diktatoren tun -"

"Der letzte Krieg, Mr. Potter, wurde zwischen dem Dunklen Lord und Dumbledore geführt. Und obwohl Dumbledore ein fehlerhafter Anführer war, der den Krieg verlor, ist es lächerlich zu behaupten, dass irgendeiner der in dieser Zeit gewählten Zaubereiminister Dumbledores Platz hätte einnehmen können! Stärke kommt von mächtigen Zauberern und ihren Anhängern, nicht von Wahlen und den Narren, die sie wählen. Das ist die Lektion der jüngsten Geschichte des magischen Britanniens; und ich bezweifle, dass der nächste Krieg Ihnen eine andere Lektion erteilen wird.

Wenn Sie ihn überleben, Mr. Potter, was Sie nicht tun werden, wenn Sie nicht die enthusiastischen Illusionen der Kindheit aufgeben!"

"Wenn Sie glauben, dass die Vorgehensweise, die Sie befürworten, keine Gefahren birgt", sagte Harry, und trotz allem wurde seine Stimme scharf, "dann ist auch das kindlicher Enthusiasmus."

Harry starrte grimmig in die Augen von Professor Quirrell, der ohne zu blinzeln zurückstarrte.

"Solche Gefahren", sagte Professor Quirrell kalt, "sind in Büros wie diesem zu besprechen, nicht in Reden. Die Narren, die Cornelius Fudge gewählt haben, sind nicht an Komplikationen und Vorsicht interessiert. Präsentieren Sie ihnen irgendetwas, das nuancierter ist als ein mitreißender Beifall, und Sie werden sich Ihrem Krieg allein stellen. Das, Mr. Potter, war Ihr kindischer Fehler, den Draco Malfoy selbst mit 8 Jahren nicht gemacht hätte. Es hätte selbst Ihnen klar sein müssen, dass Sie schweigen und sich erst mit mir beraten sollten, anstatt Ihre Sorgen vor der Menge auszusprechen!"

"Ich bin kein Freund von Albus Dumbledore", sagte Harry, eine Kälte in der Stimme, die zu der von Professor Quirrell passte. "Aber er ist kein Kind, und er schien meine Sorgen weder für kindisch zu halten, noch dafür, dass ich damit hätte warten sollen, sie auszusprechen."

"Oh", sagte Professor Quirrell, "Sie lassen sich also vom Schulleiter beraten?"

und stand hinter seinem Schreibtisch auf.

Als Blaise auf dem Weg zu seinem Büro um die Ecke bog, sah er, dass Professor Quirrell bereits an der Wand lehnte.

"Blaise Zabini", sagte der Verteidigungsprofessor und richtete sich auf; seine Augen standen wie dunkle Steine in seinem Gesicht, und seine Stimme ließ Blaise einen Schauer der Angst über den Rücken laufen.

\emph{Er kann nichts gegen mich ausrichten, ich muss nur daran denken, dass -}

"Ich glaube", sagte Professor Quirrell mit klarer, kalter Stimme, "dass ich den Namen Ihres Arbeitgebers bereits erraten habe. Aber ich möchte ihn von deinen eigenen Lippen hören, und mir auch den Preis nennen, mit dem du gekauft wurdest."

Blaise wusste, dass er unter seinem Gewand schwitzte und dass die Feuchtigkeit bereits auf seiner Stirn zu sehen war.

"Ich hatte die Chance zu zeigen, dass ich besser bin als alle drei Generäle, und ich habe sie genutzt.

Viele Leute hassen mich jetzt, aber es gibt auch viele Slytherins, die mich dafür lieben werden. Wie kommst du darauf, dass ich -"

"Sie haben sich den Plan für die heutige Schlacht nicht ausgedacht, Mr. Zabini. Sag mir, wer es war."

Blaise schluckte schwer. "Nun… Ich meine, wenn das so ist… dann wissen Sie doch schon, wer es war, oder? Der Einzige, der so verrückt ist, ist Dumbledore. Und er wird mich beschützen, wenn Sie versuchen, etwas zu tun."

"In der Tat. Nennen Sie mir den Preis."

Die Augen des Verteidigungsprofessors waren immer noch hart.

"Es geht um meine Cousine Kimberly", sagte Blaise, schluckte wieder und versuchte, seine Stimme zu kontrollieren.

"Sie ist echt, und sie wird wirklich gemobbt", Potter überprüfte das, er war nicht dumm. Nur Dumbledore hat gesagt, dass er die Tyrannen dazu angestiftet hat, nur wegen des Plans, und wenn ich für ihn arbeite, wird es ihr danach gut gehen, aber wenn ich mit Potter gehe, kann Kimberly in noch mehr Schwierigkeiten geraten!"

Professor Quirrell schwieg einen langen Moment lang.

"Ich verstehe", sagte Professor Quirrell, seine Stimme war nun viel milder.

"Mr. Zabini, sollte so ein Vorfall noch einmal vorkommen, können Sie sich direkt an mich wenden.

Ich habe meine eigenen Methoden, um meine Freunde zu schützen. Nun eine letzte Frage: Selbst mit all der Macht, die Sie in die Hand genommen haben, wäre es schwierig gewesen, ein Unentschieden zu erzwingen. Hat Dumbledore Sie instruiert, wer sonst gewinnen sollte?"

"Sonnenschein", sagte Blaise. Professor Quirrell nickte.

"Wie ich dachte."

Der Verteidigungsprofessor seufzte.

"In Ihrer zukünftigen Karriere, Mr. Zabini, rate ich Ihnen, keine so komplizierten Komplotte zu versuchen. Sie haben die Tendenz, zu scheitern."

"Ähm, das habe ich dem Schulleiter auch schon gesagt", sagte Blaise,

"und er meinte, dass es deshalb wichtig sei, mehr als einen Plot gleichzeitig am Laufen zu haben."

Professor Quirrell fuhr sich mit einer müden Hand über die Stirn.

"Es ist ein Wunder, dass der Dunkle Lord vom Kampf gegen ihn nicht verrückt geworden ist. Sie können jetzt zu Ihrer Besprechung mit dem Schulleiter gehen, Mr. Zabini. Ich werde nichts darüber sagen, aber sollte der Schulleiter irgendwie herausfinden, dass wir miteinander gesprochen haben, denken Sie an mein ständiges Angebot, Ihnen so viel Schutz zu geben, wie ich kann.

Sie können wegtreten."

Blaise wartete nicht auf ein weiteres Wort, drehte sich einfach um und floh.

Professor Quirrell wartete eine Zeit lang und sagte dann:

"Gehen Sie, Mr. Potter."

Harry riss sich den Unsichtbarkeitsumhang vom Kopf und stopfte ihn in seinen Beutel.

Er zitterte so sehr vor Wut, dass er kaum sprechen konnte.

"Er hat was? Er hat was getan?"

"Sie hätten es selbst herausfinden müssen, Mr. Potter", sagte Professor Quirrell milde.

"Sie müssen lernen, Ihren Blick zu trüben, bis Sie den Wald vor lauter Bäumen nicht mehr sehen können. Jeder, der die Geschichten über Sie gehört hat und nicht weiß, dass Sie der mysteriöse Junge-der-lebte sind, könnte leicht auf Ihren Besitz eines Unsichtbarkeitsumhangs schließen.

Treten Sie von diesen Ereignissen zurück, verwischen Sie ihre Details, und was sehen wir? Es gab eine große Rivalität zwischen den Schülern, und ihr Wettkampf endete mit einem perfekten Unentschieden. So etwas passiert nur in Geschichten, Mr. Potter, und es gibt eine Person in dieser Schule, die in Geschichten denkt. Es gab eine merkwürdige und komplizierte Handlung, die für den jungen Slytherin, dem Sie gegenüberstanden, untypisch sein sollte. Aber es gibt eine Person in dieser Schule, die in so ausgeklügelten Plots denkt, und sein Name ist nicht Zabini.

Und ich habe dich gewarnt, dass es einen Vierfachagenten gibt; du wusstest, dass Zabini mindestens ein Dreifachagent ist, und du hättest eine hohe Wahrscheinlichkeit vermuten müssen, dass er es ist. Nein, ich werde den Kampf nicht für ungültig erklären. Ihr habt alle drei den Test nicht bestanden und gegen euren gemeinsamen Feind verloren."

Harry interessierte sich zu diesem Zeitpunkt nicht für Tests.

"Dumbledore hat Zabini erpresst, indem er seinen Cousin bedroht hat? Nur damit unser Kampf mit einem Unentschieden endet? Warum?"

Professor Quirrell gab ein freudloses Lachen von sich.

"Vielleicht dachte der Schulleiter, die Rivalität sei gut für seinen Lieblingshelden und wollte, dass sie weitergeht. Zum Wohle der Allgemeinheit, Sie verstehen. Oder vielleicht war er einfach nur verrückt. Sehen Sie, Mr. Potter, jeder weiß, dass Dumbledores Wahnsinn eine Maske ist, dass er geistig gesund ist und vorgibt, wahnsinnig zu sein. Sie sind stolz auf diese kluge Erkenntnis. Und da sie die geheime Erklärung kennen, hören sie auf zu suchen. Es kommt ihnen nicht in den Sinn, dass es auch möglich ist, eine Maske hinter der Maske zu haben, also geisteskrank zu sein, der vorgibt, geistig gesund zu sein, der vorgibt, geisteskrank zu sein. Und ich fürchte, Mr. Potter, ich habe anderweitig dringende Geschäfte zu erledigen und muss abreisen; aber ich würde Ihnen dringend raten, sich nicht von Albus Dumbledore inspirieren zu lassen, wenn Sie einen Krieg führen.

Bis später, Mr. Potter."

Der Verteidigungsprofessor neigte ironisch den Kopf und schritt dann in dieselbe Richtung davon, in die Zabini geflohen war, während Harry immer noch mit offenem Mund vor Schreck dastand.

\textbf{Nachwirkungen: Harry Potter.}

Harry stapfte langsam auf den Ravenclaw-Schlafsaal zu, die Augen nicht auf Wände, Gemälde oder andere Schüler gerichtet; er ging Treppen hinauf und Rampen hinunter, ohne zu verlangsamen, zu beschleunigen oder darauf zu achten, wo er hintrat.

Er hatte mehr als eine Minute nach Professor Quirrells Abgang gebraucht, um zu begreifen, dass seine einzige Informationsquelle über Dumbledores Verwicklung

a) Blaise Zabini war, dem er ein absoluter Idiot sein müsste, um ihm wieder zu vertrauen, und

b) Professor Quirrell, der leicht ein Komplott in Dumbledores Stil hätte fabrizieren können und der vielleicht auch dachte, dass ein bisschen Schülerrivalität eine feine Sache sei; und der, wenn man zurücktrat und die Details verwischte, gerade vorgeschlagen hatte, das Land in eine magische Diktatur zu verwandeln.

Und es war auch möglich, dass Dumbledore derjenige war, der hinter Zabini stand, und dass Professor Quirrell aufrichtig versucht hatte, das Dunkle Mal in natura zu bekämpfen und die Wiederholung einer Vorstellung zu verhindern, die er als erbärmlich ansah.

Er wollte verhindern, dass Harry allein gegen den Dunklen Lord kämpfte, während alle anderen sich verängstigt versteckten und versuchten, sich aus der Schusslinie zu halten, und darauf warteten, dass Harry sie retten würde.

Aber die Wahrheit war… Na ja… Harry kam damit irgendwie klar. Er wusste, dass das die Art von Dingen war, die Helden nachtragend und verbittert machen sollte.

\emph{Zur Hölle damit.} Harry war sehr dafür, dass sich alle anderen aus der Gefahr heraushielten, während der Junge-der-lebte den Dunklen Lord allein zur Strecke brachte, plus oder minus einer kleinen Anzahl von Begleitern.

\emph{Wenn der nächste Konflikt mit dem Dunklen Lord zu einem Zweiten Zaubererkrieg führen würde, der viele Menschen tötete und ein ganzes Land verwickelte, würde das bedeuten, dass Harry bereits versagt hatte.}

\emph{\hfill\break Und wenn danach ein Krieg zwischen Zauberern und Muggeln ausbrach, war es egal, wer gewann, Harry hätte bereits versagt, indem er es so weit kommen ließ.}

Außerdem, wer sagte, dass sich die Gesellschaften nicht friedlich integrieren könnten, wenn die Geheimhaltung unweigerlich zusammenbricht?

(Obwohl Harry die trockene Stimme von Professor Quirrell in seinem Kopf hören konnte, die ihn fragte, ob er ein Narr sei, und all die offensichtlichen Dinge sagte…)

Und wenn Magier und Muggel nicht in Frieden leben konnten, dann würde Harry Magie und Wissenschaft kombinieren und herausfinden, wie man alle Zauberer auf den Mars oder sonst wohin evakuieren konnte, anstatt einen Krieg ausbrechen zu lassen.

Denn wenn es zu einem Vernichtungskrieg käme… Das war die Sache, die Professor Quirrell nicht erkannt hatte, die wichtigste Frage, die er vergessen hatte, seinem jungen General zu stellen.

Der wahre Grund, warum Harry nicht die Absicht hatte, sich zu einem Lichtmal überreden zu lassen, egal wie sehr es ihm im Kampf gegen den Dunklen Lord helfen würde.

Ein Dunkler Lord und fünfzig Gefolgsleute mit einem Mal waren eine Gefahr für das gesamte magische Britannien. Wenn ganz Britannien das Zeichen eines starken Anführers annehmen würde, wären sie eine Gefahr für die gesamte magische Welt. Und wenn die gesamte Zaubererwelt ein einziges Mal annahm, wäre es eine Gefahr für den Rest der Menschheit.

Keiner wusste genau, wie viele Zauberer es auf der Welt gab. Er hatte mit Hermine ein paar Schätzungen angestellt und kam auf Zahlen im Bereich von einer Million.

Aber es gab 6 Milliarden Muggel. Wenn es zu einem endgültigen Krieg käme…

\emph{Professor Quirrell hatte vergessen, Harry zu fragen, welche Seite er schützen würde.}

Eine wissenschaftliche Zivilisation, die nach oben blickt und weiß, dass es ihr Schicksal ist, nach den Sternen zu greifen. Oder eine magische Zivilisation, die langsam verblasste, weil das Wissen verloren ging, die immer noch von einem Adel regiert wurde, der Muggel als nicht ganz menschlich ansah.

Es war ein furchtbar trauriges Gefühl, aber keines, das auch nur einen Hauch von Zweifel enthielt.

\textbf{Nachwehen}: \textbf{Blaise Zabini.}

Blaise schlenderte mit vorsichtiger, selbst auferlegter Langsamkeit durch die Gänge, sein Herz schlug wild, während er versuchte, sich zu beruhigen -

"Ähem", sagte eine trockene, flüsternde Stimme aus einer schattigen Nische, als er vorbeiging.

Blaise zuckte zusammen, aber er schrie nicht. Langsam drehte er sich um. In der kleinen, schattigen Ecke lag ein schwarzer Umhang, der so breit und wogend war, dass es unmöglich war, festzustellen, ob die Gestalt darunter männlich oder weiblich war, und über dem Umhang ein breitkrempiger schwarzer Hut, und ein schwarzer Nebel schien sich darunter zu sammeln und das Gesicht dessen zu verdecken, wer oder was auch immer darunter liegen mochte.

"Berichten Sie", flüsterte Mr. Hut und Mantel.

"Ich habe genau das gesagt, was Sie mir aufgetragen haben", sagte Blaise.

Seine Stimme war ein wenig ruhiger, jetzt, da er niemanden mehr anlügen musste.

"Und Professor Quirrell hat genau so reagiert, wie Sie es erwartet haben."

Der breite schwarze Hut neigte sich und richtete sich auf, als ob der Kopf darunter nicken würde.

"Ausgezeichnet", sagte das nicht identifizierbare Flüstern. "Die Belohnung, die ich dir versprochen habe, ist bereits auf dem Weg zu deiner Mutter, per Eule."

Blaise zögerte, aber seine Neugierde fraß ihn auf.

"Darf ich jetzt fragen, warum du Ärger zwischen Professor Quirrell und Dumbledore verursachen willst?"

Der Schulleiter hatte nichts mit den Gryffindor-Tyrannen zu tun, von denen Blaise wusste, und er hatte Kimberly nicht nur geholfen, sondern auch angeboten, Professor Binns dazu zu bringen, ihm in Geschichte der Zauberei hervorragende Noten zu geben, selbst wenn er leere Pergamente für seine Hausaufgaben abgäbe, obwohl er trotzdem zum Unterricht erscheinen und so tun müsste, als würde er sie abgeben. Eigentlich hätte Blaise alle drei Generäle umsonst verraten, und auch seinen Cousin, aber er sah keine Notwendigkeit, das zu sagen.

Der breite schwarze Hut neigte sich zu einer Seite, als wolle er einen fragenden Blick vermitteln.

"Sagen Sie, Freund Blaise, ist es Ihnen schon einmal in den Sinn gekommen, dass Verräter, die so oft verraten, oft ein böses Ende finden?"

"Nö", sagte Blaise und blickte direkt in den schwarzen Nebel unter dem Hut.

"Jeder weiß, dass den Schülern in Hogwarts nie etwas wirklich Schlimmes passiert."

Mr. Hut und Mantel gab ein flüsterndes Kichern von sich.

"In der Tat", sagte das Flüstern. "Wobei der Mord an einem Schüler vor fünf Jahrzehnten die Ausnahme ist, die die Regel bestätigt, da Salazar Slytherin sein Monster auf einer höheren Ebene als der Schulleiter selbst in die alten Zaubersprüche eingegeben haben muss."

Blaise starrte auf den schwarzen Nebel und fühlte sich nun ein wenig unwohl.

Aber es müsste schon ein Hogwarts-Professor sein, der ihm etwas Bedeutsames antat, ohne Alarm auszulösen. Quirrell und Snape waren die einzigen Professoren, die so etwas tun würden, und Quirrell würde sich nicht darum scheren, sich etwas vorzumachen, und Snape würde keinen seiner eigenen Slytherins verletzen … oder?

"Nein, Freund Blaise", flüsterte der schwarze Nebel, "ich wollte dir nur raten, so etwas niemals in deinem Erwachsenenleben zu versuchen. So viel Verrat würde sicherlich zu mindestens einer Rache führen."

"Meine Mutter hat nie eine Rache bekommen", sagte Blaise stolz.

"Obwohl sie sieben Ehemänner geheiratet hat und jeder einzelne von ihnen auf mysteriöse Weise gestorben ist und ihr viel Geld hinterlassen hat."

"Wirklich?", flüsterte sie. "Wie hat sie aber den siebten überredet, sie zu heiraten, nachdem er gehört hat, was mit den ersten sechs passiert ist?"

"Das habe ich Mum gefragt", sagte Blaise, "und sie sagte, ich dürfe es nicht wissen, bis ich alt genug sei, und ich fragte sie, wie alt alt genug sei, und sie sagte, älter als sie."

Wieder das flüsternde Glucksen.

"Nun denn, Freund Blaise, meinen Glückwunsch, dass du in die Fußstapfen deiner Mutter getreten bist. Geh, und wenn du nichts davon sagst, werden wir uns nicht wiedersehen."

Blaise wich unbehaglich zurück und fühlte einen seltsamen Widerwillen, sich umzudrehen.

Der Hut neigte sich.

"Ach, komm schon, kleiner Slytherin. Wenn du wirklich Harry Potter oder Draco Malfoy ebenbürtig wärst, hättest du bereits erkannt, dass meine angedeuteten Drohungen nur dazu dienten, dein Schweigen vor Albus sicherzustellen. Hätte ich vorgehabt, dir zu schaden, hätte ich es nicht angedeutet; hätte ich nichts gesagt, hättest du dir Sorgen machen müssen."

Blaise richtete sich auf, fühlte sich ein wenig beleidigt und nickte Mr. Hut und Mantel zu; dann drehte er sich entschlossen um und schritt in Richtung seines Treffens mit dem Schulleiter davon.

Er hatte bis zuletzt gehofft, dass jemand anderes auftauchen und ihm die Chance geben würde, Mr. Hat und Cloak zu verraten.

Aber dann hatte Mum nicht sieben verschiedene Ehemänner auf einmal betrogen. Wenn man es so betrachtet, ging es ihm immer noch besser als ihr. Und Blaise Zabini ging weiter auf das Büro des Schulleiters zu, lächelnd, zufrieden damit, ein Fünffachagent zu sein - \emph{für einen Moment stolperte der Junge, richtete sich dann aber auf und schüttelte das seltsame Gefühl der Orientierungslosigkeit ab.}

Und Blaise Zabini ging weiter auf das Büro des Schulleiters zu, lächelte, zufrieden damit, ein vierfacher Agent zu sein.

\textbf{Nachwirkungen: Hermine Granger.}

Der Bote sprach sie erst an, als sie allein war. Hermine verließ gerade die Mädchentoilette, in der sie sich manchmal zum Nachdenken versteckte, als eine hell leuchtende Katze aus dem Nichts auftauchte und sagte: "\emph{Miss Granger?}" Sie stieß einen kleinen Schrei aus, bevor sie merkte, dass die Katze mit der Stimme von Professor McGonagall gesprochen hatte.

Trotzdem hatte sie sich nicht erschreckt, nur erschrocken; die Katze war hell und strahlend und schön, glühend mit einem weißen, silbernen Glanz wie mondfarbenes Sonnenlicht, und sie konnte sich nicht vorstellen, Angst zu haben.

"Was sind Sie?", fragte Hermine.

"Das ist eine Nachricht von Professor McGonagall", sagte die Katze, immer noch mit der Stimme des Professors.

"Kannst du in mein Büro kommen und mit niemandem darüber sprechen?"

"Ich bin sofort da", sagte Hermine, immer noch überrascht, und die Katze sprang und verschwand; nur verschwand sie nicht, sie reiste irgendwie weg; oder das war es, was ihr Verstand sagte, obwohl ihre Augen sie gerade verschwinden sahen.

Als Hermine im Büro ihres Lieblingsprofessors ankam, wirbelten ihre Gedanken nur so vor Spekulationen. War etwas mit ihren Noten in Verwandlung nicht in Ordnung? Aber warum würde Professor McGonagall dann sagen, dass sie es niemandem erzählen sollte? Wahrscheinlich ging es darum, dass Harry seine Teilverwandlung übte…

Professor McGonagalls Gesicht sah eher besorgt als streng aus, als Hermine sich vor den Schreibtisch setzte - sie versuchte, ihre Augen davon abzuhalten, zu dem Regal von Fächern zu gehen, in denen Professor McGonagalls Hausaufgaben lagen, denn sie hatte sich schon immer gefragt, welche Art von Arbeit die Erwachsenen zu erledigen hatten, um die Schule am Laufen zu halten, und ob sie Hilfe von ihr gebrauchen konnten…

"Miss Granger", sagte Professor McGonagall, "lassen Sie mich zunächst sagen, dass ich bereits weiß, dass der Schulleiter Sie gebeten hat, diesen Wunsch zu äußern -"

"Er hat es Ihnen gesagt?", platzte Hermine erschrocken heraus.

\emph{Der Schulleiter hatte gesagt, dass niemand sonst es wissen sollte!}

Professor McGonagall hielt inne, sah Hermine an und gab ein trauriges kleines Kichern von sich.

"Es ist gut zu sehen, dass Mr. Potter Sie nicht zu sehr verdorben hat. Miss Granger, Sie dürfen nichts zugeben, nur weil ich sage, dass ich es weiß. Zufälligerweise hat mir der Schulleiter nichts gesagt, ich kenne ihn einfach zu gut."

Hermine errötete jetzt heftig.

"Es ist in Ordnung, Miss Granger!", sagte Professor McGonagall hastig. "Sie sind eine Ravenclaw in Ihrem ersten Jahr, niemand erwartet, dass Sie eine Slytherin sind."

\emph{Das stach wirklich.}

"Na schön", sagte Hermine etwas sauer, "dann werde ich Harry Potter um Slytherin-Unterricht bitten."

"Das wollte ich nicht….", sagte Professor McGonagall, und ihre Stimme verstummte.

"Miss Granger, ich mache mir darüber Sorgen, denn junge Ravenclaw-Mädchen sollten keine Slytherins sein müssen! Wenn der Schulleiter Sie bittet, sich auf etwas einzulassen, bei dem Sie sich nicht wohl fühlen, Miss Granger, ist es wirklich in Ordnung, nein zu sagen. Und wenn Sie sich unter Druck gesetzt fühlen, sagen Sie bitte dem Schulleiter, dass Sie mich dabei haben möchten oder dass Sie mich vorher fragen möchten."

Hermines Augen wurden ganz groß.

"Macht der Schulleiter Dinge, die falsch sind?"

Professor McGonagall sah daraufhin ein wenig traurig aus.

"Nicht absichtlich, Miss Granger, aber ich denke… nun, es ist wahrscheinlich wahr, dass der Schulleiter manchmal Schwierigkeiten hat, sich daran zu erinnern, wie es ist, ein Kind zu sein.

Selbst als Kind muss er brillant gewesen sein und stark im Kopf und im Herzen, mit genug Mut für 3 Gryffindors. Manchmal verlangt der Schulleiter zu viel von seinen jungen Schülern, Miss Granger, oder ist nicht vorsichtig genug, um sie nicht zu verletzen. Er ist ein guter Mann, aber manchmal gehen seine Intrigen zu weit."

"Aber es ist gut für Schüler, stark zu sein und Mut zu haben", sagte Hermine.

"Deshalb haben Sie ja auch Gryffindor für mich vorgeschlagen, oder?"

Professor McGonagall lächelte verschmitzt.

"Vielleicht war ich nur egoistisch, weil ich dich für mein eigenes Haus haben wollte. Hat der Sprechende Hut dir angeboten - nein, ich hätte nicht fragen sollen."

"Er hat mir gesagt, ich könnte überall hingehen, nur nicht nach Slytherin", sagte Hermine.

\emph{Beinahe hätte sie gefragt, warum sie nicht gut genug für Slytherin war, bevor sie es geschafft hatte, sich zu stoppen …}

"Also ich habe Mut, Professor!"

Professor McGonagall beugte sich über ihren Schreibtisch nach vorne. Die Sorge zeichnete sich jetzt deutlicher auf ihrem Gesicht ab.

"Miss Granger, es geht nicht um Mut, es geht darum, was für junge Mädchen gesund ist!

Der Schulleiter zieht Sie in seine Intrigen hinein, Harry Potter gibt Ihnen seine Geheimnisse , und jetzt verbünden Sie sich mit Draco Malfoy! Und ich habe deiner Mutter versprochen, dass du in Hogwarts sicher sein würdest!"

Hermine wusste einfach nicht, was sie darauf antworten sollte.

Aber ihr kam der Gedanke, dass Professor McGonagall sie vielleicht nicht gewarnt hätte, wenn sie ein Junge in Gryffindor statt eines Mädchens in Ravenclaw gewesen wäre und das war, nun ja…

"Ich werde versuchen, brav zu sein", sagte sie, "und ich werde mir von niemandem etwas anderes sagen lassen."

Professor McGonagall presste ihre Hände über ihre Augen. Als sie sie wegnahm, sah ihr faltiges Gesicht sehr alt aus.

"Ja", sagte sie flüsternd, "Sie wären in meinem Haus gut aufgehoben gewesen. Bleiben Sie in Sicherheit, Miss Granger, und seien Sie vorsichtig. Und wenn Sie sich jemals wegen irgendetwas Sorgen machen oder sich unwohl fühlen, kommen Sie bitte sofort zu mir. Ich will Sie nicht länger aufhalten."

\textbf{Nachwirkungen, Draco Malfoy:}

Keiner von beiden hatte an diesem Samstag wirklich Lust, irgendetwas Kompliziertes zu tun, nicht nach dem Kampf von vorhin. Also saß Draco einfach in einem unbenutzten Klassenzimmer und versuchte, ein Buch zu lesen, das \emph{"Denksport Physik"} hieß.

Es war eines der faszinierendsten Dinge, die Draco je in seinem Leben gelesen hatte, zumindest die Teile, die er verstehen konnte, zumindest wenn der verfluchte Idiot, der sich weigerte, seine Bücher aus den Augen zu lassen, es schaffte, die Klappe zu halten und Draco sich konzentrieren zu lassen -

"Hermine Granger ist ein Schlammbluuuuuut", sang Harry Potter von dort, wo er an einem nahegelegenen Schreibtisch saß und ein weitaus fortgeschritteneres Buch von ihm las.

"Ich weiß, was du vorhast", sagte Draco ruhig, ohne von den Seiten aufzublicken.

"Es wird nicht funktionieren. Wir werden dich trotzdem zerquetschen."

"Ein Maaaalfoy arbeitet mit einem Schlammbluuuuuut zusammen, was werden die ganzen Freunde deines Vaters denken -"

"Die werden denken, dass Malfoys nicht so leicht zu manipulieren sind, wie du anscheinend glaubst, Potter!"

\emph{Der Verteidigungsprofessor war noch verrückter als Dumbledore, so kindisch und würdelos konnte kein zukünftiger Retter der Welt in irgendeinem Alter sein.}

"Hey, Draco, weißt du, was wirklich zum Kotzen ist? Du weißt, dass Hermine Granger zwei Kopien des magischen Gens hat, genau wie du und genau wie ich, aber alle deine Klassenkameraden in Slytherin wissen das nicht und du darfst es nicht erklären -"

Dracos Finger wurden weiß, wo sie das Buch umklammerten. Geschlagen und bespuckt zu werden, konnte unmöglich so viel Selbstbeherrschung erfordern, und wenn er sich nicht bald an Harry rächte, würde er etwas Schlimmes tun -

"Und was hast du dir beim ersten Mal gewünscht?", fragte Draco.

Harry sagte nichts, also schaute Draco von seinem Buch auf und fühlte einen Stich bösartiger Befriedigung bei dem traurigen Blick auf Harrys Gesicht.

"Ähm", sagte Harry. "Viele Leute haben mich das gefragt, aber ich glaube, Professor Quirrell hätte nicht gewollt, dass ich darüber rede."

Draco setzte selbst einen ernsten Gesichtsausdruck auf.

"Du kannst mit mir darüber reden. Es ist wahrscheinlich nicht wichtig im Vergleich zu den anderen Geheimnissen, die du mir erzählt hast, und wozu sind Freunde sonst da?"

\emph{Stimmt, ich bin dein Freund! Fühle dich schuldig!}

"Es war eigentlich gar nicht so interessant", sagte Harry mit offensichtlich künstlicher Leichtigkeit.

"Ich wünschte nur, Professor Quirrell würde nächstes Jahr wieder Kampfmagie unterrichten."

Harry seufzte und sah wieder auf sein Buch hinunter.

Nach ein paar weiteren Sekunden sagte er:

"Dein Vater wird dieses Weihnachten wahrscheinlich ziemlich sauer auf dich sein, aber wenn du ihm versprichst, dass du das Schlammblutmädchen verrätst und ihre Armee auslöschst, wird alles wieder in Ordnung kommen und du bekommst trotzdem deine Weihnachtsgeschenke."

\emph{Wenn er und Granger Professor Quirrell besonders höflich bitten und einige ihrer Quirrell-Punkte einsetzen würden, dürften die beiden vielleicht etwas Interessanteres mit General Chaos machen, als ihn einzuschläfern.}

