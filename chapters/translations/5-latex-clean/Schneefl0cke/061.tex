

\hypertarget{das-stanford-gefuxe4ngnis-experiment-finale}{% \section{62. Das Stanford-Gefängnis-Experiment, Finale}\label{das-stanford-gefuxe4ngnis-experiment-finale}}

\textbf{\uline{Das Stanford-Gefängnis-Experiment, Finale}}

Minerva starrte auf die Uhr, die goldenen Zeiger und silbernen Ziffern, die ruckartige Bewegung. Muggel hatten das erfunden, und bis sie es getan hatten, hatten sich Zauberer nicht die Mühe gemacht, die Zeit einzuhalten. Glocken, getaktet durch eine geschliffene Sanduhr, hatten bei der Erbauung von Hogwarts für den Unterricht gedient. Es war eines der Dinge, die Blutpuristen nicht wahr haben wollten, und deshalb wusste Minerva es. Sie hatte ein „Ohnegleichen“ auf ihren Muggelkunde-U.T.Z. erhalten, was ihr jetzt wie eine Schande vorkam, wenn man bedenkt, wie wenig sie wusste. Ihr jüngeres Ich hatte schon damals erkannt, dass der Kurs eine Mogelpackung war, unterrichtet von einem Reinblüter, angeblich, weil Muggelgeborene nicht verstehen konnten, was man Zauberern beibringen musste, und eigentlich, weil der Oberste Rat Muggel überhaupt nicht guthieß. Aber als sie siebzehn war, war die Ohnegleichen Note das Wichtigste gewesen, was ihr wichtig war, wie sie sich traurig erinnerte…

\emph{Wenn Harry Potter und Voldemort ihren Krieg mit Muggelwaffen führen, wird von der Welt nichts mehr übrig sein außer Feuer!}

Sie konnte es sich nicht vorstellen, und der Grund, warum sie es sich nicht vorstellen konnte, war, dass sie sich nicht vorstellen konnte, dass Harry gegen Du-weißt-schon-wen kämpfen würde. Sie war dem Dunklen Lord schon viermal begegnet und hatte jedes Mal überlebt, dreimal mit Albus als Schutzschild und einmal mit Moody an ihrer Seite. Sie erinnerte sich an das verunstaltete, schlangenartige Gesicht, die schwachen grünen Schuppen, die über die Haut verstreut waren, die glühend roten Augen, die Stimme, die in einem hohen Zischen lachte und nichts als Grausamkeit und Qual versprach: das Monster in Reinkultur.

Und Harry Potter war leicht vor ihrem geistigen Auge auszumalen, der strahlende Gesichtsausdruck eines kleinen Jungen, der schwankte zwischen dem Ernst-nehmen des Lächerlichen und dem Lachen über den Ernst.

Und der Gedanke, dass die beiden sich Zauberstab an Zauberstab gegenüberstanden, war zu schmerzhaft, um ihn sich vorzustellen. Sie hatten kein Recht, überhaupt kein Recht, das einem elfjährigen Jungen anzutun. Sie wusste, was der Schulleiter an diesem Tag für ihn beschlossen hatte, denn sie war beauftragt worden, die Vorkehrungen zu treffen; und wäre sie in demselben Alter gewesen, hätte sie gewütet und geschrien und geweint und wäre wochenlang untröstlich gewesen, und… \emph{Er ist kein gewöhnlicher Erstklässler,} hatte Albus gesagt. \emph{Er ist dem Dunklen Lord ebenbürtig, und er hat eine Macht, die der Dunkle Lord nicht kennt.}

Die schreckliche, hohle Stimme, die aus Sybill Trelawneys Kehle dröhnte, die wahre und ursprüngliche Prophezeiung, hallte noch einmal in ihrem Kopf wider. Sie hatte das Gefühl, dass sie nicht das bedeutete, was der Schulleiter dachte, aber es gab keine Möglichkeit, den Unterschied in Worte zu fassen. Und trotzdem schien es wahr zu sein, dass, wenn es irgendeinen Elfjährigen auf der ganzen Erde gab, der diese Last tragen konnte, dieser Junge jetzt in ihr Büro kam. Und wenn sie ihm so etwas wie '\emph{armer Harry'} ins Gesicht sagte…nun, das würde ihm nicht gefallen.

\emph{Jetzt muss ich also einen Weg finden, einen unsterblichen dunklen Zauberer zu töten,} hatte Harry an dem Tag gesagt, an dem er es zum ersten Mal erfahren hatte. \emph{Ich wünschte wirklich, Sie hätten mir das gesagt, bevor ich anfing einzukaufen…}

Sie war lange genug Leiterin des Hauses Gryffindor gewesen, sie hatte genug Freunde sterben sehen, um zu wissen, dass es einige Leute gab, die man nicht davor bewahren konnte, ein Held zu werden.

Es klopfte an der Tür, und Professor McGonagall sagte: „Herein.“

Als Harry eintrat, hatte sein Gesicht denselben kalten, wachen Blick, den sie in Marys Restaurant gesehen hatte, und sie fragte sich einen Moment lang, ob er den ganzen Tag über dieselbe Maske, dasselbe Selbst getragen hatte.

Der Junge setzte sich auf den Stuhl vor ihrem Schreibtisch und sagte: „Ist es also an der Zeit, mir zu sagen, was los ist?“ Neutral die Worte, nicht die Schärfe, die zu dem Ausdruck hätte gehören sollen.

Professor McGonagalls Augen hoben sich vor Überraschung, bevor sie sie aufhalten konnte, und sie sagte: „Der Schulleiter hat Ihnen nichts gesagt, Mr~Potter?“

Der Junge schüttelte den Kopf. „Nur, dass er eine Warnung erhalten hatte, dass ich in Gefahr sein könnte, aber dass ich jetzt in Sicherheit sei.“

Minerva hatte Mühe, seinen Blick zu erwidern. \emph{Wie konnten sie ihm das antun, wie konnten sie das einem elfjährigen Jungen aufbürden, diesen Krieg, dieses Schicksal, diese Prophezeiung…und sie trauten ihm nicht einmal…} Sie zwang sich, Harry direkt anzuschauen, und sah, dass seine grünen Augen ruhig auf ihr ruhten.

„Professor McGonagall?“, sagte der Junge leise.

„Mr~Potter“, sagte Professor McGonagall, „ich fürchte, es steht mir nicht zu, Ihnen das zu erklären, aber wenn der Schulleiter Ihnen danach immer noch nichts sagt, können Sie zu mir kommen, und ich werde ihn für Sie anschreien gehen.“

Die Augen des Jungen weiteten sich, etwas von dem echten Harry zeigte sich durch den Spalt, bevor die kühle Maske wieder aufgesetzt wurde.

„Auf jeden Fall“, sagte Professor McGonagall zügig. „Es tut mir leid für die Unannehmlichkeiten, Mr~Potter, aber ich muss Sie bitten, Ihren Zeitumkehrer zu benutzen, um sechs Stunden zurück auf drei Uhr zu gehen und Professor Flitwick folgende Nachricht zu übermitteln: \emph{Silber ist auf dem Baum}. Bitten Sie den Professor, die Zeit zu notieren, zu der Sie ihm diese Nachricht gegeben haben. Danach möchte sich der Schulleiter mit Ihnen treffen, wenn es Ihnen passt.“

Es gab eine Pause. Dann sagte der Junge: „Ich werde also verdächtigt, meinen Zeitumkehrer zu missbrauchen?“

„Nicht von mir!“ sagte Professor McGonagall hastig. „Es tut mir leid für die Unannehmlichkeiten, Mr~Potter.“

Es gab eine weitere Pause, und dann zuckte der Junge mit den Schultern. „Es wird meinen Schlafrhythmus durcheinanderbringen, aber ich nehme an, es lässt sich nicht ändern. Bitte lassen Sie die Hauselfen wissen, dass ich morgen früh um, sagen wir, drei Uhr ein frühes Frühstück bekomme, wenn ich darum bitte.“

„Natürlich, Mr~Potter“, sagte sie. „Ich danke Ihnen für Ihr Verständnis.“

Der Junge erhob sich von seinem Stuhl und nickte ihr förmlich zu, dann schlüpfte er zur Tür hinaus, wobei seine Hand bereits unter sein Hemd wanderte, wo sein Zeitumkehrer wartete; und fast hätte sie „Harry!“ gerufen, nur hätte sie nicht gewusst, was sie danach sagen sollte. Stattdessen wartete sie, die Augen auf die Uhr gerichtet.

Wie lange musste sie warten, bis Harry Potter in der Zeit zurückging? Eigentlich brauchte sie gar nicht zu warten; wenn er es getan hatte, dann war es bereits geschehen… Minerva wusste also, dass sie zögerte, weil sie nervös war, und die Erkenntnis machte sie traurig.

\emph{Unfug, ja, unaussprechlicher, undenkbarer Unfug mit all der Umsicht und Voraussicht eines fallenden Felsens} - sie wusste nicht, wie der Junge den Hut ausgetrickst hatte, ihn nicht nach Gryffindor zu sortieren, wo er offensichtlich hingehörte - \emph{aber nichts Dunkles oder Schädliches, niemals.} Unter diesem Unfug war seine Güte so tief und wahrhaftig wie die der Weasley-Zwillinge, obwohl nicht einmal der Cruciatus-Fluch sie dazu hätte bringen können, das laut auszusprechen.

„Expecto Patronum“, sagte sie und dann: „Geh zu Professor Flitwick und bring seine Antwort zurück, nachdem du ihm Folgendes gefragt hast: 'Hat Mr~Potter dir eine Nachricht von mir gegeben, wie lautete diese Nachricht und wann hast du sie erhalten?'“

Eine Stunde zuvor, nachdem er die letzte verbleibende Umdrehung seines Zeitumkehrers benutzt hatte, nachdem er den Unsichtbarkeitsumhang angelegt hatte, steckte Harry die Sanduhr zurück in sein Hemd. Und er machte sich auf den Weg zu den Slytherin-Kerkern, wobei er so schnell schritt, wie es seine unsichtbaren Beine schafften, wenn auch nicht rannte. Zum Glück befand sich das Büro der stellvertretenden Schulleiterin bereits in einem niedrigeren Stockwerk von Hogwarts… Ein paar Treppenhäuser später, zwei Stufen, aber nicht drei Stufen auf einmal genommen, blieb Harry an einem Korridor stehen, um dessen letzte Biegung der Eingang zu den Slytherin-Schlafsälen lag. Harry nahm ein Stück Pergament (kein Papier) aus seiner Pergamenttasche, holte einen Diktierfeder aus seinem Beutel und sagte der Feder: „Schreibe diese Buchstaben genau so, wie ich sie sage: Z-P-G-B-S-Y, Leerzeichen, F-V-Y-I-R-E-B-A-G-U-R-G-E-R-R.“

Es gab zwei Arten von Codes in der Kryptographie, Codes, die deinen kleinen Bruder davon abhalten, deine Nachricht zu lesen und Codes, die große Regierungen davon abhalten, deine Nachricht zu lesen, und dies war die erste Art von Code, aber es war besser als nichts. Theoretisch sollte es sowieso niemand lesen; aber selbst wenn jemand es täte, würde er sich an nichts Interessantes erinnern, wenn er nicht zuerst Kryptographie betrieben hätte.

Harry steckte das Pergament dann in einen Pergamentumschlag und schmolz mit seinem Zauberstab ein wenig grünes Wachs, um ihn zu versiegeln. Im Prinzip hätte Harry das natürlich auch schon Stunden früher tun können, aber irgendwie schien es ihm weniger wie ein Spiel mit der Zeit, zu warten, bis er die Nachricht von Professor McGonagalls eigenen Lippen gehört hatte. Dann steckte Harry den Umschlag in einen anderen Umschlag, der bereits ein weiteres Blatt Papier mit anderen Anweisungen und fünf silberne Sickel enthielt. Er verschloss diesen Umschlag (auf dessen Außenseite bereits ein Name stand), versiegelte ihn mit noch mehr grünem Wachs und drückte eine letzte Sickel in dieses Siegel. Dann steckte Harry diesen Umschlag in den allerletzten Umschlag, auf dem in großen Buchstaben der Name „Merry Tavington“ stand. Und Harry spähte um die Kurve, wo das finstere Porträt, das als Tür zu den Slytherin-Schlafsälen diente, wartete; und da er nicht wollte, dass das Porträt sich daran erinnerte, dass es jemanden unsichtbar gemacht hatte, benutzte Harry den Schwebezauber, um den Umschlag zu dem finsteren Mann schweben zu lassen und ihn ihm anzutippen.

Der finstere Mann schaute auf den Umschlag hinunter, betrachtete ihn durch ein Monokel, seufzte und drehte sich um, um in das Innere der Slytherin-Schlafsäle zu schauen, und rief: „Nachricht für Merry Tavington!“ Dann ließ er den Umschlag auf den Boden fallen. Ein paar Augenblicke später öffnete sich die Porträttür und Merry schnappte sich den Umschlag vom Boden. Sie öffnete ihn und fand einen Sickel und einen Umschlag, der an eine Viertklässlerin namens Margaret Bulstrode adressiert war. (Slytherins machten so etwas ständig, und ein Sickel war definitiv ein Eilauftrag.) Margaret würde ihren Umschlag öffnen und fünf Sickel finden, zusammen mit einem Umschlag, den sie in einem unbenutzten Klassenzimmer abgeben sollte, nachdem sie ihren Zeitumkehrer benutzt hatte, um fünf Stunden zurückzugehen, woraufhin sie weitere fünf Sickel finden würde, die auf sie warteten, wenn sie schnell dorthin kommen würde.

Und ein unsichtbarer Harry Potter würde von 15~Uhr bis 15:30~Uhr in diesem Klassenzimmer warten, nur für den Fall, dass jemand den \emph{offensichtlichen Test} versuchen würde.

Nun, für Professor Quirrell war es jedenfalls offensichtlich gewesen. Für Professor Quirrell war es auch offensichtlich gewesen, dass (a) Margaret Bulstrode einen Zeitumkehrer hatte und (b) sie nicht sehr streng damit war, wie sie ihn benutzte, z.B. indem sie ihrer jüngeren Schwester wirklich guten Klatsch erzählte, „bevor“ es jemand anderes gehört hatte.

Etwas von der Anspannung sickerte von Harry ab, als er von der Porträttür wegschritt, immer noch unsichtbar. Irgendwie hatte es sein Verstand immer noch geschafft, sich Gedanken über den Plan zu machen, obwohl er wusste, dass er bereits gelungen war. Jetzt blieb nur noch die Konfrontation mit Dumbledore, und dann war er für den Tag fertig… er würde um 9~Uhr zu den Wasserspeiern des Schulleiters gehen, da es verdächtiger wirken würde, es um 8~Uhr zu tun. Auf diese Weise könnte er behaupten, er hätte nur missverstanden, was Professor McGonagall mit „danach“ gemeint hatte…

Der obskure Schmerz klammerte sich wieder an Harrys Herz, als er an Professor McGonagall dachte. Also zog sich Harry ein wenig weiter in seine dunkle Seite zurück, die den ruhigen Ausdruck getragen und die Müdigkeit aus seinem Gesicht gehalten hatte, und ging weiter. Es würde eine Abrechnung kommen, aber manchmal musste man sich heute alles leihen, was man konnte, und die Zahlungen morgen fällig werden lassen. Sogar Harrys dunkle Seite spürte die Erschöpfung, als die Wendeltreppe ihn zu der großen Eichentür brachte, die das letzte Tor zu Dumbledores Büro war; aber da Harry jetzt rechtmäßig vier Stunden über seine natürliche Schlafenszeit hinaus war, war es sicher, etwas von der Müdigkeit zu zeigen, die physische, wenn nicht die emotionale.

Die Eichentür schwang auf - Harrys Augen waren bereits in Richtung des großen Schreibtisches und des dahinter stehenden Thrones gerichtet gewesen; es dauerte also einen Moment, bis er registrierte, dass der Thron leer war, der Schreibtisch kahl bis auf einen einzigen ledergebundenen Band; und dann verlagerte Harry seinen Blick, um den Zauberer zwischen seinen mechanischen Apparaten stehen zu sehen, die geheimnisvollen unbekannten Geräte in ihren Partituren. Fawkes und der Sprechende Hut besetzten ihre jeweiligen Sitzplätze, ein helles, fröhliches Feuer knisterte in einer Ecke, von der Harry vorher nicht wusste, dass es ein Kamin war, und da waren die zwei Regenschirme und drei rote Pantoffeln für die linken Füße. Alle Dinge an ihrem Platz und in ihrer gewohnten Erscheinung, bis auf den alten Zauberer selbst, der hoch aufgerichtet und in die förmlichsten schwarzen Gewänder gekleidet war.

Es war ein Schock für die Augen, diese Roben an dieser Person, es war, als hätte Harry seinen Vater in einem Geschäftsanzug gesehen. Sehr altertümlich war die Erscheinung von Albus Dumbledore, und traurig. „Hallo, Harry“, sagte der alte Zauberer.

Aus dem Inneren eines alternativen Selbst, das wie ein Okklumentik-Konstrukt aufrechterhalten wurde, neigte ein unschuldiger Harry, der absolut keine Ahnung hatte, was geschah, kalt den Kopf und sagte: „Schulleiter. Ich gehe davon aus, dass Sie inzwischen von der stellvertretenden Schulleiterin McGonagall gehört haben, und wenn es Ihnen recht ist, würde ich wirklich gerne wissen, was los ist.“

„Ja“, sagte der alte Zauberer, „es ist Zeit, Harry Potter.“ Der Rücken richtete sich auf, nur geringfügig, denn der Zauberer stand bereits aufrecht; aber irgendwie ließ selbst diese kleine Veränderung den Zauberer einen Fuß größer erscheinen, und stärker, wenn auch nicht jünger, furchterregend, wenn auch nicht gefährlich, seine Potenz sammelte sich um ihn wie eine Kutte. Dann sprach er mit klarer Stimme: „Mit dem heutigen Tag hat dein Krieg gegen Voldemort begonnen.“

„Was?“, sagte der äußere Harry, der nichts wusste, während etwas, das ihn von innen beobachtete, das Gleiche dachte, nur mit viel mehr Schimpfworten versehen.

„Bellatrix Black ist aus Askaban entführt worden, sie ist aus einem unentrinnbaren Gefängnis entkommen“, sagte der alte Zauberer. „Es ist ein Kunststück, das Voldemorts Handschrift trägt, wenn ich es je gesehen habe; und sie, seine treueste Dienerin, ist eine von drei Requisiten, die er erhalten muss, um in einem neuen Körper wieder aufzuerstehen. Nach zehn Jahren ist der Feind, den du einst besiegt hast, zurückgekehrt, wie es vorausgesagt wurde.“

Keinem Teil von Harry fiel etwas ein, was er darauf erwidern konnte, zumindest nicht für die wenigen Sekunden, bevor der alte Zauberer fortfuhr. „Es muss sich für dich vorerst wenig ändern“, sagte der alte Zauberer. „Ich habe damit begonnen, den Orden des Phönix neu zu gründen, der dir dienen wird, ich habe die wenigen Seelen alarmiert, die es verstehen können und sollen: Amelia Bones, Alastor Moody, Bartemius Crouch und einige andere. Von der Prophezeiung - ja, es gibt eine Prophezeiung - habe ich ihnen nichts erzählt, aber sie wissen, dass Voldemort zurückgekehrt ist, und sie wissen, dass du eine wichtige Rolle spielen wirst. Sie und ich werden deinen Krieg in seinen kleinen Anfängen kämpfen, während du hier in Hogwarts stärker und vielleicht weiser wirst.“ Die Hand des alten Zauberers hob sich, als würde er sie anflehen. „Für dich gibt es also vorerst nur eine Änderung, und ich bitte dich inständig, deren Notwendigkeit zu verstehen. Erkennst du das Buch auf meinem Schreibtisch, Harry?“

Der innere Teil von Harry schrie auf und schlug seinen Kopf gegen imaginäre Wände, während der äußere Harry sich umdrehte und auf das starrte, was sich als - Es gab eine ziemlich lange Pause. Dann sagte Harry: „Es ist eine Ausgabe von Der Herr der Ringe von J. R. R. Tolkien.“

„Du hast ein Zitat aus diesem Buch erkannt“, sagte Dumbledore mit einem wissenden Blick in den Augen, „ich nehme also an, dass du dich gut daran erinnerst. Sollte ich mich irren, korrigiere mich bitte.“

Harry starrte ihn nur an.

„Es ist wichtig zu verstehen“, sagte Dumbledore, „dass dieses Buch keine realistische Darstellung eines Zaubererkrieges ist. John Tolkien hat nie gegen Voldemort gekämpft. Dein Krieg wird nicht wie die Bücher sein, die du gelesen hast. Das wahre Leben ist nicht wie Geschichten. Verstehst du das, Harry?“

Harry nickte, eher langsam, mit Ja; und schüttelte dann den Kopf mit Nein.

„Insbesondere“, sagte Dumbledore, „gibt es eine gewisse sehr törichte Sache, die Gandalf im ersten Buch tut. Er macht viele Fehler, der Zauberer von Tolkien; aber dieser eine Fehler ist der unverzeihlichste. Dieser Fehler ist dieser: Als Gandalf zum ersten Mal, wenn auch nur für einen Moment, den Verdacht hatte, dass Frodo den Einen Ring besitzt, hätte er Frodo sofort nach Bruchtal bringen sollen. Es wäre ihm vielleicht peinlich gewesen, dem alten Zauberer, wenn sich sein Verdacht als falsch erwiesen hätte. Er hätte es vielleicht als unangenehm empfunden, Frodo so zu kommandieren, und Frodo wäre in große Unannehmlichkeiten geraten, weil er viele andere Pläne und Zeitvertreibe hätte aufgeben müssen. Aber eine kleine Verlegenheit und Unannehmlichkeit ist nichts im Vergleich zum Verlust des ganzen Krieges, wenn die neun Nazgul über das Auenland herfallen, während du in Minas Tirith alte Schriftrollen liest, und den Ring sofort an sich nehmen. Und nicht nur Frodo wäre verletzt worden, ganz Mittelerde wäre in die Sklaverei gefallen. Wäre es nicht nur eine Geschichte gewesen, Harry, hätten sie ihren Krieg verloren. Verstehst du, was ich damit sagen will?“

„Äh…“, sagte Harry, „nicht ganz…“ Es war etwas an Dumbledore, wenn er so war, das es schwer machte, richtig kalt zu bleiben; seine dunkle Seite hatte Schwierigkeiten mit dem Unheimlichen.

„Dann werde ich es erklären“, sagte der alte Zauberer. Seine Stimme war streng, seine Augen waren traurig. „Frodo hätte sofort von Gandalf selbst nach Bruchtal gebracht werden müssen - und Frodo hätte Bruchtal niemals ohne Wache verlassen dürfen. Es hätte keine Schreckensnacht in Bree geben dürfen, keine Grabhügel, keine Wetterspitze, auf der Frodo verwundet wurde - sie hätten den ganzen Krieg verlieren können, wegen Gandalfs Torheit! Verstehst du jetzt, was ich dir sage, Sohn von Michael und Petunia?“

Und der Harry, der nichts wusste, hat es verstanden. Und der Harry, der nichts wusste, sah, dass es das Kluge, das Weise, das Intelligente und Vernünftige war, das Richtige zu tun. Und der Harry, der nichts wusste, sagte genau das, was ein unschuldiger Harry gesagt hätte, während der stumme Wächter vor Verwirrung und Qual schrie. „Du sagst“, sagte Harry, seine Stimme zitterte, als die inneren Emotionen sich durch die äußere Ruhe brannten, „dass ich zu Ostern nicht nach Hause zu meinen Eltern fahre.“

„Du wirst sie wiedersehen“, sagte der alte Zauberer schnell. „Ich werde sie anflehen, hierher zu kommen, um bei dir zu sein, ich werde ihnen bei ihren Besuchen jede Höflichkeit erweisen. Aber du fährst nicht über Ostern nach Hause, Harry. Du fährst nicht über den Sommer nach Hause. Du wirst nicht mehr in der Winkelgasse zu Mittag essen, auch nicht mit Professor Quirrell als Aufpasser. Dein Blut ist die 2. Requisite, die Voldemort braucht, um wieder so stark zu werden wie früher. Du verlässt also nie wieder die Grenzen der Hogwarts-Schutzwälle ohne einen wichtigen Grund und einen Wächter, der stark genug ist, jeden Angriff lange genug abzuwehren, um dich in Sicherheit zu bringen. “

Wasser begann in Harrys Augenwinkeln zu stehen. „Ist das eine Bitte?“, fragte seine zitternde Stimme. „Oder ein Befehl?“

„Es tut mir leid, Harry“, sagte der alte Zauberer sanft. „Deine Eltern werden die Notwendigkeit einsehen, hoffe ich; aber wenn nicht… Ich fürchte, sie haben keine Handhabe; das Gesetz erkennt sie, wenn auch zu Unrecht, nicht als deine Vormünder an. Es tut mir leid, Harry, und ich werde verstehen, wenn du mich dafür verachtest, aber es muss getan werden.“

Harry wirbelte herum, sah zur Tür, er konnte Dumbledore nicht mehr ansehen, konnte seinem eigenen Gesicht nicht trauen.

\emph{Das ist der Preis, den du dir selbst auferlegt hast}, sagte der Hufflepuff in seinem Kopf, \emph{so wie du auch anderen Kosten auferlegt hast. Wird das deine ganze Sicht der Dinge ändern, so wie Professor Quirrell es sich vorstellt?}

Automatisch sagte die Maske des unschuldigen Harry genau das, was sie gesagt hätte: „Sind meine Eltern in Gefahr? Müssen sie hierher gebracht werden?“

„Nein“, sagte die Stimme des alten Zauberers. „Das glaube ich nicht. Die Todesser haben gegen Ende des Krieges gelernt, dass sie die Familien des Ordens nicht angreifen sollten. Und wenn Voldemort jetzt ohne seine früheren Gefährten handelt, so weiß er doch, dass ich es bin, der im Moment die Entscheidungen trifft, und er weiß, dass ich ihm nichts für eine Bedrohung deiner Familie geben würde. Ich habe ihm beigebracht, dass ich Erpressungen nicht nachgebe, und deshalb wird er es nicht versuchen.“

Harry drehte sich wieder um und sah eine Kälte auf dem Gesicht des alten Zauberers, die zu der Veränderung in seiner Stimme passte, Dumbledores blaue Augen waren hinter der Brille hart wie Stahl geworden, sie passten nicht zur Person, aber sie passten zu den formellen schwarzen Roben.

„Ist das dann alles?“, fragte Harrys zitternde Stimme. Später würde er darüber nachdenken, später würde er sich eine schlaue Gegenmaßnahme ausdenken, später würde er Professor Quirrell fragen, ob es eine Möglichkeit gab, den Schulleiter davon zu überzeugen, dass er sich geirrt hatte. Im Moment nahm das Aufrechterhalten der Maske Harrys ganze Aufmerksamkeit in Anspruch.

„Voldemort hat ein Muggelartefakt benutzt, um aus Askaban zu entkommen“, sagte der alte Zauberer. „Er beobachtet dich und lernt von dir, Harry Potter. Bald wird ein Mann namens Arthur Weasley im Ministerium einen Erlass herausgeben, dass alle Verwendung von Muggelartefakten in den Kämpfen des Verteidigungsprofessors aufhören muss. In Zukunft, wenn du eine gute Idee hast, behalte sie besser für dich.“

Im Vergleich dazu schien es nicht wichtig zu sein. Harry nickte nur und fragte erneut: „Ist das alles?“

Es gab eine Pause.

„Bitte“, sagte der alte Zauberer im Flüsterton. „Ich habe kein Recht, dich um Verzeihung zu bitten, Harry James Potter-Evans-Verres, aber bitte, sag wenigstens, dass du verstehst, warum.“ In den Augen des alten Zauberers stand Wasser.

„Ich verstehe“, sagte die Stimme des äußeren Harrys, der tatsächlich verstand, „ich meine… ich habe sowieso darüber nachgedacht…ich habe mich gefragt, ob ich Sie und meine Eltern dazu bringen kann, dass ich den Sommer über in Hogwarts bleiben darf, wie die Waisen, damit ich hier in der Bibliothek lesen kann, es ist in Hogwarts sowieso interessanter…“

Ein würgender Laut kam aus Albus Dumbledores Kehle. Harry wandte sich wieder der Tür zu. Es war kein ungeschorenes Entkommen, aber es war ein Entkommen. Er machte einen Schritt nach vorne. Seine Hand griff nach der Türklinke.

Ein durchdringender Schrei zerriss die Luft - wie in Zeitlupe sah Harry, als er sich drehte, den Phönix bereits durch die Luft geschossen und auf ihn zugeflogen kommen. Aus dem wahren Harry, der seine eigene Schuld kannte, kam ein Anflug von Panik, daran hatte er nicht gedacht, er hatte es nicht erwartet, er hatte sich darauf vorbereitet, Dumbledore gegenüberzutreten, aber er hatte Fawkes vergessen—

Flatter, flatter und flatter, dreimal flatterten die Flügel des Phönix wie das Aufflackern und Erlöschen eines Feuers, die Zeit schien zu langsam zu vergehen, als Fawkes über die geheimnisvollen Geräte hinweg auf Harry zusteuerte. Und der rotgoldene Vogel schwebte mit sanften Flügelschlägen vor ihm, wippte in der Luft wie eine Kerzenflamme.

„Was ist los, Fawkes?“, fragte der falsche Harry verwirrt und sah dem Phönix in die Augen, wie er es tun würde, wenn er unschuldig wäre. Der echte Harry, der innerlich die gleiche schreckliche Übelkeit verspürte wie damals, als Professor McGonagall ihm ihr Vertrauen ausgesprochen hatte, dachte: \emph{"Bin ich heute böse geworden,} \emph{Fawkes? Ich dachte nicht, dass ich böse bin… Hasst du mich jetzt? Wenn ich etwas geworden bin, was ein Phönix hasst, sollte ich es vielleicht einfach aufgeben, jetzt alles aufgeben und gestehen}—

Fawkes schrie, der schrecklichste Schrei, den Harry je gehört hatte, ein Schrei, der alle Geräte zum Vibrieren brachte und alle schlafenden Gestalten in ihren Porträts aufschrecken ließ. Er durchdrang alle Abwehrkräfte Harrys wie ein weißglühendes Schwert durch Butter, ließ alle seine Schichten zusammenfallen wie einen geplatzten Luftballon, änderte seine Prioritäten in einem Augenblick, als er sich an das eine Wichtigste erinnerte; die Tränen liefen Harry ungehindert aus den Augen, über die Wangen, seine Stimme erstickte, als die Worte aus seiner Kehle kamen, als würde er Lava aushusten—

„Fawkes sagt“, sagte Harrys Stimme, „er will, dass ich etwas tue, wegen der Gefangenen in Askaban—“

„Fawkes, nein!“, sagte der alte Zauberer. Dumbledore schritt vorwärts und streckte dem Phönix eine flehende Hand entgegen. Die Stimme des alten Zauberers war fast so verzweifelt wie der Schrei des Phönix gewesen war. „Das kannst du nicht von ihm verlangen, Fawkes, er ist doch noch ein Junge!“

„Du bist nach Askaban gegangen“, flüsterte Harry, „du hast Fawkes mitgenommen, er hat es gesehen - du hast es gesehen - du warst dort, du hast es gesehen - WARUM HAST DU NICHTS GEMACHT? WARUM HAST DU SIE NICHT RAUSGELASSEN!!!“

Als die Instrumente aufhörten zu vibrieren, erkannte Harry, dass Fawkes gleichzeitig mit seinem eigenen Schrei geschrien hatte, dass der Phönix nun neben Harry flog und Dumbledore an seiner Seite gegenüberstand, der rotgoldene Kopf auf gleicher Höhe mit seinem eigenen.

„Kannst du“, flüsterte der alte Zauberer, „kannst du wirklich die Stimme des Phönix so deutlich hören?“

Harry schluchzte fast zu heftig, um zu sprechen, denn all die Metalltüren, an denen er vorbeigegangen war, die Stimmen, die er gehört hatte, die schlimmsten Erinnerungen, das verzweifelte Betteln beim Weggehen, all das war bei dem Schrei des Phönix wie Feuer in seinen Geist geplatzt, alle inneren Bollwerke zerschlagen. Harry wusste nicht, ob er die Stimme des Phönix wirklich so deutlich hören konnte, ob er Fawkes verstanden hätte, ohne es bereits zu wissen. Alles, was Harry wusste, war, dass er eine plausible Ausrede hatte, um die Dinge zu sagen, von denen Professor Quirrell ihm gesagt hatte, dass er sie von diesem Tag an niemals in einem Gespräch erwähnen dürfe; denn das war genau das, was ein unschuldiger Harry gesagt hätte, was er getan hätte, wenn er so deutlich gehört hätte.

„Sie leiden - wir müssen ihnen helfen—“

„Ich kann nicht!“, schrie Albus Dumbledore. „Harry, Fawkes, ich kann nicht, es gibt nichts, was ich tun kann!“

Ein weiterer durchdringender Schrei. „WARUM NICHT? GEH EINFACH REIN UND HOLE SIE RAUS!“

Der alte Zauberer riss seinen Blick von dem Phönix los, seine Augen trafen stattdessen die von Harry. „Harry, sag es Fawkes von mir! Sag ihm, dass es nicht so einfach ist! Phönixe sind keine einfachen Tiere, aber sie sind Tiere, Harry, sie können nicht verstehen—“

„Ich verstehe es auch nicht“, sagte Harry, und seine Stimme zitterte. „Ich verstehe nicht, warum sie Menschen an Dementoren verfüttern! Askaban ist kein Gefängnis, es ist eine Folterkammer und Sie foltern diese Leute zu Tode!“

„Percival“, sagte der alte Zauberer heiser, „Percival Dumbledore, mein eigener Vater, Harry, mein eigener Vater ist in Askaban gestorben! Ich weiß, ich weiß, es ist ein Graus! Aber was wollt ihr von mir? Dass ich Askaban mit Gewalt breche? Wollt ihr, dass ich eine offene Rebellion gegen das Ministerium erkläre?“

\textbf{CAW}!

Es gab eine Pause, und Harrys zitternde Stimme sagte: „Fawkes weiß nichts von Regierungen, er will nur, dass Sie - die Gefangenen aus ihren Zellen holen - und er wird Ihnen helfen zu kämpfen, wenn sich Ihnen jemand in den Weg stellt - und - und das werde ich auch, Schulleiter! Ich gehe mit Ihnen und vernichte jeden Dementor, der sich nähert! Um die politischen Folgen werden wir uns danach kümmern, ich wette, wir beide zusammen könnten damit durchkommen—“

„Harry“, flüsterte der alte Zauberer, „Phönixe verstehen nicht, wie man eine Schlacht gewinnen und einen Krieg verlieren kann.“ Tränen liefen über die Wangen des alten Zauberers und tropften in seinen silbernen Bart. „Die Schlacht ist alles, was sie kennen. Sie sind gut, aber nicht weise. Deshalb wählen sie Zauberer als ihre Herren.“

„Kannst du die Dementoren dorthin bringen, wo ich an sie rankomme?“ Harrys Stimme klang jetzt flehend. „Bringen Sie sie in Gruppen von fünfzehn heraus - ich glaube, so viele könnte ich auf einmal vernichten, ohne mich zu verletzen—“

Der alte Zauberer schüttelte den Kopf. „Es war schon schwer genug, den Verlust von einem zu verschmerzen - sie könnten mir vielleicht noch einen geben, aber niemals zwei - sie gelten als Nationaleigentum, Harry, als Waffen im Kriegsfall—“

Da loderte Wut in Harry auf, loderte auf wie Feuer, sie mochte von dort kommen, wo jetzt ein Phönix auf seiner eigenen Schulter ruhte, und sie mochte von seiner eigenen dunklen Seite kommen, und die beiden Wutgefühle mischten sich in ihm, die kalte und die heiße, und es war eine fremde Stimme, die aus seiner Kehle sprach:

„Sag mir etwas Schulleiter. Was muss eine Regierung tun, was müssen die Wähler mit ihrer Demokratie tun, was müssen die Menschen eines Landes tun, bevor ich beschließen sollte, dass ich nicht mehr auf ihrer Seite bin?“

Die Augen des alten Zauberers weiteten sich, als er den Jungen mit einem Phönix auf der Schulter anstarrte. „Harry…sind das deine Worte, oder die des Verteidigungsprofessors—“

„Weil es irgendeinen Punkt geben muss, nicht wahr? Und wenn es nicht Askaban ist, wo ist es dann?“

„Harry, hör zu, bitte, hör mir zu! Zauberer können nicht zusammenleben, wenn jeder von ihnen bei jeder Meinungsverschiedenheit die Rebellion gegen das Ganze ausruft! Es wird immer etwas geben—“

„Askaban ist nicht nur etwas! Es ist böse!“

„Ja, sogar böse! Sogar etwas Böses, Harry, denn Zauberer sind nicht vollkommen gut! Und doch ist es besser, dass wir in Frieden leben, als im Chaos; und für dich und mich wäre ein gewaltsamer Ausbruch aus Askaban der Beginn des Chaos, siehst du das nicht?“ Die Stimme des alten Zauberers war flehend. „Und es ist möglich, sich dem Willen seiner Mitmenschen offen oder heimlich zu widersetzen, ohne sie zu hassen, ohne sie zum Bösen und Feind zu erklären! Ich glaube nicht, dass die Menschen in diesem Land das von dir verdient haben, Harry! Und selbst wenn es einige von ihnen täten - was ist mit den Kindern, was mit den Schülern in Hogwarts, was mit den vielen guten Menschen, die sich unter die schlechten mischen?“

Harry schaute über seine Schulter, wo Fawkes gehockt hatte, sah, wie die Augen des Phönix zu ihm zurückblickten, sie glühten nicht und doch loderten sie, rote Flammen in einem Meer aus goldenem Feuer.

\emph{Was denkst du, Fawkes?}

„CAW?“, sagte der Phönix.

Fawkes verstand das Gespräch nicht. Der Junge schaute den alten Zauberer an und sagte mit dicker Stimme: „Oder vielleicht sind die Phönixe weiser als wir, schlauer als wir, vielleicht folgen sie uns in der Hoffnung, dass wir eines Tages auf sie hören, dass wir es eines Tages kapieren, dass wir die Gefangenen einfach aus ihren Zellen holen—“

Harry drehte sich und riss die Eichentür auf, trat auf die Treppe und schlug die Tür hinter sich zu. Die Treppe begann sich zu drehen, Harry begann hinabzusteigen, und er legte sein Gesicht in seine Hände und begann zu weinen. Erst als er auf halbem Weg nach unten war, bemerkte er den Unterschied, bemerkte die Wärme, die sich immer noch in ihm ausbreitete, und erkannte, dass—

„Fawkes? flüsterte Harry. - der Phönix war immer noch auf seiner Schulter, thronte dort, wie er ihn ein paar Mal auf Dumbledores gesehen hatte. Harry schaute wieder in die Augen, rote Flammen in goldenem Feuer. “Du bist nicht mein Phönix…oder doch?"

\emph{CAW}.

„Oh“, sagte Harry, seine Stimme zitterte ein wenig, „ich bin froh, das zu hören, Fawkes, denn ich glaube nicht - der Schulleiter - ich glaube nicht, dass er es verdient—“ Harry hielt inne, holte Luft. „Ich glaube nicht, dass er das verdient hat, Fawkes, er hat versucht, das Richtige zu tun…“

\emph{CAW}!

„Aber du bist wütend auf ihn und versuchst, einen Standpunkt zu vertreten. Ich verstehe das.“

Der Phönix schmiegte seinen Kopf an Harrys Schulter, und der steinerne Wasserspeier ging sanft zur Seite, um Harry zurück in die Korridore von Hogwarts zu lassen.

