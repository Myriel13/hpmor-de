

\hypertarget{warte-bevor-du-luxf6sungen-vorschluxe4gst}{% \section{25. Warte bevor du Lösungen vorschlägst}\label{warte-bevor-du-luxf6sungen-vorschluxe4gst}}

\textbf{\uline{Warte bevor du Lösungen vorschlägst}}

\emph{Anm. des Übersetzers:Das die Einzelnen Akte hier nicht in der richtigen Reihenfolge auftauchen ist Absicht des Autors (und vielleicht ein Hinweis in welcher Reihenfolge Harry seinen Zeitumkehrer eingesetzt hat ;) )}

\textbf{Akt 2:}

(Die Sonne schien strahlend in die Große Halle von der verwunschenen Himmelsdecke darüber und beleuchtete die Schüler, als säßen sie unter dem nackten Himmel, von ihren Tellern und Schüsseln, während sie, erfrischt von einer Nacht Schlaf, das Frühstück inhalierten, um sich auf das vorzubereiten, was sie für ihren Sonntag geplant hatten.)

So. Es gab nur eine Sache, die dich zu einem Zauberer machte. Das war nicht überraschend, wenn man darüber nachdachte.

Was die DNA hauptsächlich tat, war den Ribosomen zu sagen, wie sie Aminosäuren zu Proteinen verketten sollten.

Die konventionelle Physik schien durchaus in der Lage zu sein, Aminosäuren zu beschreiben, und egal, wie viele Aminosäuren man aneinanderreihte, die konventionelle Physik sagte, dass man daraus niemals, niemals Magie gewinnen würde.

Und doch schien Magie vererbbar zu sein, der DNA folgend. Das lag dann wahrscheinlich nicht daran, dass die DNA nichtmagische Aminosäuren zu magischen Proteinen verkettete.

Vielmehr war es so, dass die Schlüsselsequenz der DNA an sich gar nicht für die Magie verantwortlich war. Die Magie kam von irgendwo anders her.

(Am Ravenclaw-Tisch gab es einen Jungen, der ins Leere starrte, während seine rechte Hand automatisch irgendeine unwichtige Speise in seinen Mund löffelte, von was auch immer vor ihm lag. Man hätte es wahrscheinlich durch einen Haufen Dreck ersetzen können und er hätte es nicht bemerkt.)

Und aus irgendeinem Grund achtete die Quelle der Magie auf einen bestimmten DNA-Marker bei Individuen, die in jeder anderen Hinsicht normale Menschen mit Affenabstammung waren.

(Tatsächlich gab es eine ganze Reihe von Jungen und Mädchen, die ins Leere starrten. Es war schließlich der Ravenclaw-Tisch.)

Es gab andere Linien der Logik, die zur gleichen Schlussfolgerung führten. Komplexe Maschinerie war immer universell innerhalb einer sich sexuell fortpflanzenden Spezies.

Wenn Gen B von Gen A abhing, dann musste A von sich aus nützlich sein und im Genpool fast universell werden, bevor B oft genug nützlich sein würde, um einen Fitnessvorteil zu bringen.

Sobald B universell war, gab es eine Variante A*, die auf B angewiesen war, und dann C, das auf A* und B angewiesen war, dann B*, das auf C angewiesen war, bis die ganze Maschine auseinanderfiel, wenn man ein einziges Teil entfernte.

Aber das musste alles inkrementell geschehen - die Evolution hat nie in die Zukunft geschaut, die Evolution würde nie damit beginnen, B zu fördern, in Vorbereitung darauf, dass A später universell wird.

\emph{Evolution war die einfache historische Tatsache, dass die Gene derjenigen Organismen, die tatsächlich die meisten Kinder hatten, in der nächsten Generation häufiger vorkommen würden.}

Jedes Teil einer komplexen Maschine musste also nahezu universell werden, bevor sich andere Teile der Maschine so entwickeln würden, dass sie von ihrem Vorhandensein abhängen.

Die komplexe, voneinander abhängige Maschinerie, die mächtigen, hochentwickelten Proteinmaschinen, die das Leben antrieben, waren also innerhalb einer sich sexuell fortpflanzenden Spezies immer universell - abgesehen von einer kleinen Handvoll nicht voneinander abhängiger Varianten, auf die zu jeder Zeit selektiert wurde, während weitere Komplexität langsam angelegt wurde.

\emph{Das war der Grund, warum alle Menschen das gleiche zugrundeliegende Gehirndesign hatten, die gleichen Emotionen, die gleichen Gesichtsausdrücke, die mit diesen Emotionen verdrahtet waren; diese Anpassungen waren komplex, also mussten sie universell sein.}

Wenn die Magie so gewesen wäre, eine große, komplexe Anpassung mit vielen notwendigen Genen, dann hätte die Paarung eines Zauberers mit einem Muggel ein Kind hervorgebracht, das nur die Hälfte dieser Teile besaß, und die Hälfte der Maschine hätte nicht viel ausrichten können.

Und so hätte es keine Muggelgeborenen gegeben, niemals. Selbst wenn alle Teile einzeln in den Muggel-Genpool gelangt wären, würden sie sich niemals alle an einem Ort wieder zusammensetzen, um einen Zauberer zu bilden.

Es gab nicht irgendein genetisch isoliertes Tal von Menschen, die auf einen evolutionären Pfad gestoßen sind, der zu hochentwickelten magischen Teilen des Gehirns führt.

Diese komplexe genetische Maschinerie hätte sich, wenn sich Zauberer mit Muggeln gekreuzt hätten, nie wieder zu Muggelgeborenen zusammengesetzt.

Wie auch immer deine Gene dich zu einem Zauberer gemacht haben, es war also nicht, weil sie die Baupläne für eine komplizierte Maschinerie enthielten.

Das war der andere Grund, warum Harry das Mendelsche Muster vermutet hatte. Wenn magische Gene nicht kompliziert waren, warum sollte es dann mehr als eines geben?

\emph{Und doch schien die Magie selbst ziemlich kompliziert zu sein.}

Ein Türverriegelungszauber würde verhindern, dass sich die Tür öffnet und dass man die Scharniere verwandelt und Finite Incantatem und Alohomora widersteht.

Viele Elemente, die alle in die gleiche Richtung weisen: Man könnte das Zielgerichtetheit nennen, oder einfacher ausgedrückt, Zielstrebigkeit.

Es gab nur zwei bekannte Ursachen für zielgerichtete Komplexität. Die natürliche Auslese, die Dinge wie \emph{Schmetterlinge} hervorbrachte. Und intelligente Technik, die Dinge wie \emph{Autos} hervorbrachte.

Magie schien nicht etwas zu sein, das sich selbst repliziert hatte.

Zaubersprüche waren absichtlich kompliziert, aber nicht, wie bei einem Schmetterling, kompliziert mit dem Ziel, Kopien von sich selbst zu machen.

Zaubersprüche waren kompliziert, um ihrem Benutzer zu dienen, \emph{wie ein Auto}. Ein \emph{intelligenter Ingenieur hatte also die Quelle der Magie erschaffen} und ihr gesagt, sie solle auf einen bestimmten DNA-Marker achten.

Der naheliegende nächste Gedanke war, dass dies etwas mit \emph{"Atlantis"} zu tun hatte. Harry hatte Hermine schon früher danach gefragt - im Zug nach Hogwarts, nachdem er Draco das sagen gehört hatte - und soweit sie wusste, war nicht mehr bekannt als das Wort selbst.

Es mochte eine reine Legende sein. Aber es war auch plausibel genug, dass eine Zivilisation von Magieanwendern, besonders eine aus der Zeit vor dem Merlin-Interdikt, es geschafft haben könnte, sich selbst in die Luft zu jagen.

Die Argumentation ging weiter: Atlantis war eine isolierte Zivilisation gewesen, die irgendwie die Quelle der Magie ins Leben gerufen hatte und ihr sagte, dass sie nur Menschen mit dem atlantischen genetischen Marker, dem Blut von Atlantis, dienen sollte.

Und nach ähnlicher Logik: Die Worte, die ein Zauberer sprach, die Bewegungen des Zauberstabs, die waren von sich aus nicht kompliziert genug, um die Zaubereffekte von Grund auf aufzubauen - nicht so, wie die drei Milliarden Basenpaare der menschlichen DNS tatsächlich kompliziert genug waren, um einen menschlichen Körper von Grund auf aufzubauen, nicht so, wie Computerprogramme Tausende von Bytes an Daten benötigten.

Die Worte und die Bewegungen des Zauberstabs waren also nur Auslöser, Hebel, die an einer verborgenen und noch komplexeren Maschine gezogen wurden.

Knöpfe, keine Blaupausen. Und so wie ein Computerprogramm nichts tun würde, wenn man einen einzigen Rechtschreibfehler machte, würde die Quelle der Magie nicht auf einen reagieren, wenn man seine Zaubersprüche nicht genau auf die richtige Weise sprach.

Die Kette der Logik war unerbittlich. Und sie führte unweigerlich zu einer einzigen, endgültigen Schlussfolgerung.

Die alten Vorfahren der Zauberer hatten der Quelle der Magie Tausende von Jahren zuvor gesagt, dass sie Dinge nur schweben lassen würde, wenn man sagte.

… \emph{'Wingardium Leviosa'}.

Harry sackte am Frühstückstisch in sich zusammen und stützte seine Stirn müde auf seine rechte Hand.

Es gab eine Geschichte aus den Anfangstagen der Künstlichen Intelligenz - damals, als sie gerade erst anfingen und noch niemand erkannt hatte, dass das Problem schwierig sein würde - über einen Professor, der einen seiner Doktoranden damit beauftragt hatte, das Problem des Computersehens zu lösen.

Harry begann zu verstehen, wie sich dieser Doktorand gefühlt haben musste.

\emph{Das konnte eine Weile dauern.}

Warum kostete es mehr Mühe, den Alohomora-Zauber zu sprechen, wenn es doch nur wie ein Knopfdruck war? \emph{Wer war so dumm gewesen, einen Zauber für Avada Kedavra einzubauen, der nur mit Hass gesprochen werden konnte?} Warum musste man bei der wortlosen Verwandlung eine vollständige mentale Trennung zwischen dem Begriff der Form und dem Begriff des Materials vornehmen? Harry war mit diesem Problem vielleicht noch nicht fertig, wenn er seinen Abschluss in Hogwarts machte.

Er könnte immer noch an diesem Problem arbeiten, wenn er \emph{dreißig Jahre alt war}.

Hermine hatte Recht gehabt, das war Harry aus dem Bauch heraus noch nicht bewusst gewesen.

Er hatte nur eine inspirierende Rede über Entschlossenheit gehalten. Harrys Verstand überlegte kurz, ob er auf einer Bauch-Ebene zu dem Schluss kommen sollte, dass er das Problem vielleicht nie lösen würde, entschied dann aber, dass das viel zu weit führen würde. Außerdem, solange er es in den ersten paar Jahrzehnten bis zur Unsterblichkeit schaffte, würde es ihm gut gehen.

Welche Methode hatte der Dunkle Lord benutzt? Wenn man bedenkt, dass die Tatsache, dass der Dunkle Lord es irgendwie geschafft hatte, den Tod seines ersten Körpers zu überleben, fast unendlich viel wichtiger war als die Tatsache, dass er versucht hatte, das magische Britannien zu übernehmen -

"Entschuldigen Sie", sagte eine erwartungsvolle Stimme hinter ihm in sehr unerwartetem Ton.

"Wenn es Ihnen recht ist, bittet Mr. Malfoy um den Gefallen eines Gesprächs."

Harry verschluckte sich nicht an seinem Frühstücksmüsli.

Stattdessen drehte er sich um und erblickte Mr. Crabbe.

"Entschuldigung", sagte Harry. "Meinst du nicht: 'Der Boss will mit dir reden?'"

Mr. Crabbe sah nicht glücklich aus.

"Mr. Malfoy hat mich angewiesen, richtig zu sprechen."

"Ich kann dich nicht hören", sagte Harry, "Du sprichst nicht richtig."

Er wandte sich wieder seiner Schüssel mit kleinen blauen Kristallschneeflocken zu und aß bedächtig einen weiteren Löffel.

"Der Boss will mit dir reden", kam eine drohende Stimme von hinten.

"Du solltest besser zu ihm kommen, wenn du weißt was gut für dich ist."

\emph{So. Jetzt lief alles wieder nach Plan.}

\textbf{Akt 1:}

"Ein Grund?", sagte der alte Zauberer. Er zügelte die Wut aus seinem Gesicht.

Der Junge vor ihm war das Opfer gewesen und brauchte sicher nicht weiter erschreckt zu werden. "Es gibt nichts, was entschuldigen könnte -"

"Was ich ihm angetan habe, war schlimmer."

Der alte Zauberer versteifte sich in plötzlichem Entsetzen.

"Harry, was hast du getan?"

"Ich habe Draco mit einem Trick dazu gebracht, an einem Ritual teilzunehmen, das seinen Glauben an den Blutpurismus opferte.

Und das bedeutete, dass er kein Todesser sein konnte, wenn er erwachsen war. Er hatte alles verloren, Schulleiter."

Es herrschte eine lange Stille im Büro, die nur durch das winzige Schnaufen und Pfeifen der fummeligen Apparate unterbrochen wurde, die nach einiger Zeit wie Stille wirkten.

"Du liebe Zeit", sagte der alte Zauberer, "ich komme mir wirklich albern vor. Und dabei hatte ich erwartet, dass du versuchen würdest, den Erbe von Malfoy zu erlösen, indem du ihm, sagen wir, wahre Freundschaft und Freundlichkeit zeigst."

"Ha! Ja, als ob das funktioniert hätte."

Der alte Zauberer seufzte. Das ging zu weit.

"Sag mal, Harry. Ist es dir überhaupt in den Sinn gekommen, dass es etwas Unpassend ist, jemanden durch Lügen und Betrug zu erlösen?"

"Ich habe es getan, ohne direkt zu lügen, und da wir hier von Draco Malfoy sprechen, denke ich, dass das Wort, das Sie suchen, deckungsgleich ist."

Der Junge sah ziemlich selbstgefällig aus. Der alte Zauberer schüttelte verzweifelt den Kopf.

"\emph{Und das ist der Held. Wir sind alle dem Untergang geweiht.}"

\textbf{Akt 5:} Der lange, schmale Tunnel aus rauem Stein, der nur von einem Kinderstab beleuchtet wurde, schien sich kilometerweit zu erstrecken.

Der Grund dafür war einfach: Er zog sich tatsächlich meilenweit hin. Es war drei Uhr morgens, und Fred und George machten sich auf den langen Weg durch den Geheimgang, der von der Statue einer einäugigen Hexe in Hogwarts bis in den Keller des Honigtopfs in Hogsmeade führte.

"Wie sieht es aus?", fragte Fred mit leiser Stimme.

(Nicht, dass jemand zugehört hätte, aber es hatte etwas Seltsames, mit normaler Stimme zu sprechen, wenn man durch einen Geheimgang ging.)

"Immer noch kaputt", sagte George.

"Beide, oder -"

"Der eine hat sich wieder selbst repariert. Die andere ist so wie immer."

Die Karte war ein außerordentlich mächtiges Artefakt, das in der Lage war, jedes empfindungsfähige Wesen auf dem Schulgelände in Echtzeit namentlich zu erfassen.

Mit ziemlicher Sicherheit war sie während der ursprünglichen Gründung von Hogwarts erstellt worden.

Es war nicht gut, dass Fehler aufgetaucht sind. Die Chancen standen gut, dass niemand außer Dumbledore es reparieren konnte, wenn es kaputt war.

Und die Weasley-Zwillinge hatten nicht vor, die Karte an Dumbledore zu übergeben. Es wäre eine unverzeihliche Beleidigung für die Rumtreiber gewesen - die vier Unbekannten, die es geschafft hatten, einen Teil des Sicherheitssystems von Hogwarts zu stehlen, etwas, das wahrscheinlich von Salazar Slytherin selbst erschaffen worden war, und es in ein Werkzeug für Schülerstreiche zu verwandeln.

Manche hätten es als respektlos empfunden. Manche hätten es für kriminell gehalten. Die Weasley-Zwillinge glaubten fest daran, dass Godric Gryffindor es gutgeheißen hätte, wäre er in der Nähe gewesen, um es zu sehen.

Die Brüder gingen weiter und weiter und weiter, meistens schweigend. Die Weasley-Zwillinge sprachen miteinander, wenn sie sich neue Streiche ausdachten, oder wenn einer von ihnen etwas wusste, was der andere nicht wusste.

Ansonsten gab es nicht viel Sinnvolles zu besprechen. Wenn sie bereits die gleichen Informationen kannten, neigten sie dazu, die gleichen Gedanken zu denken und die gleichen Entscheidungen zu treffen.

(Früher war es bei der Geburt von magischen eineiigen Zwillingen üblich gewesen, einen von ihnen nach der Geburt zu töten.)

Irgendwann kletterten Fred und George in einen staubigen Keller, der mit Fässern und Regalen voller seltsamer Zutaten übersät war.

Fred und George warteten. Es wäre nicht höflich gewesen, etwas anderes zu tun. Nach nicht allzu langer Zeit kletterte ein dünner alter Mann in einem schwarzen Pyjama die Stufen hinunter, die in den Keller führten, und gähnte.

"Hallo, Jungs", sagte Ambrosius Flume. "Ich habe euch heute Abend nicht erwartet. Schon ausverkauft?"

Fred und George beschlossen, dass Fred das Wort ergreifen würde.

"Nicht ganz, Mr. Flume", sagte Fred. "Wir hatten gehofft, Sie könnten uns bei etwas wesentlich… Interessanterem helfen."

"Also, Jungs", sagte Flume und klang streng,

"ich hoffe, ihr habt mich nicht geweckt, nur damit ich euch noch einmal sagen kann, dass ich euch keine Ware verkaufe, die euch in echte Schwierigkeiten bringen könnte.

Jedenfalls nicht, bis ihr sechzehn seid -"

George zog einen Gegenstand aus seinem Gewand hervor und reichte ihn wortlos an Flume weiter.

"Haben Sie das gesehen?", fragte Fred.

Flume blickte auf die gestrige Ausgabe des Tagespropheten und nickte finster.

Die Schlagzeile auf der Zeitung lautete

\textbf{DER NÄCHSTE DUNKLE LORD?}

und zeigte einen Jungen, den die Kamera irgendeines Schülers mit einem uncharakteristisch kalten und grimmigen Gesichtsausdruck eingefangen hatte.

"Ich kann das nicht glauben", schnauzte Flume.

"Auf den Jungen loszugehen, obwohl er erst elf Jahre alt ist! Malfoy sollte man zermahlen und zu Pralinen verarbeiten!"

Fred und George blinzelten unisono. Malfoy stand hinter Rita Kimmkorn? Harry Potter hatte sie nicht davor gewarnt….was sicherlich bedeutete, dass Harry es nicht wusste. Sonst hätte er sie nie hergebracht… Fred und George tauschten Blicke aus.

\emph{Nun, Harry brauchte es nicht zu wissen, bis der Job erledigt war.}

"Mr. Flume", sagte Fred leise, "der Junge-der-lebte, braucht Ihre Hilfe."

Flume sah sie beide an. Dann ließ er mit einem Seufzer den Atem aus.

"In Ordnung", sagte Flume, "was wollt ihr?"

\textbf{Akt 6:} Wenn Rita Kimmkorn auf eine leckere Beute aus war, neigte sie dazu, die herumhuschenden Ameisen, die den Rest des Universums ausmachten, nicht zu bemerken, und so stieß sie fast mit dem glatzköpfigen jungen Mann zusammen, der ihr in den Weg getreten war.

"Miss Kimmkorn", sagte der Mann und klang ziemlich streng und kalt für jemanden, dessen Gesicht so jung aussah.

"Komisch, dass ich Sie hier treffe."

"Aus dem Weg, Freundchen!", schnappte Rita und versuchte, um ihn herumzutreten.

Der Mann in ihrem Weg passte die Bewegung so perfekt an, dass es so war, als hätte sich keiner von ihnen bewegt, sie standen einfach still, während sich die Straße um sie herum verschob. Ritas Augen verengten sich.

"Was glauben Sie, wer Sie sind?"

"Wie töricht", sagte der Mann trocken.

"Es wäre klug gewesen, sich das Gesicht des verkleideten Todessers einzuprägen, der Harry Potter zum nächsten Dunklen Lord ausbildet. Schließlich",

ein dünnes Lächeln,

"klingt das sicherlich nach jemandem, dem man nicht auf der Straße begegnen möchte, vor allem nicht, nachdem man in der Zeitung über ihn hergezogen hat."

Rita brauchte einen Moment, um die Anspielung einzuordnen.

\emph{Das war Quirinus Quirrell?}

Er sah zu jung und zu alt zugleich aus; sein Gesicht, wenn es sich von seiner strengen und herablassenden Pose löste, würde zu jemandem in den späten Dreißigern gehören.

\emph{Und seine Haare fielen bereits aus? Konnte er sich keinen Heiler leisten?}

Nein, das war nicht wichtig, sie hatte eine Zeit und einen Ort um ein Käfer zu sein.

Sie hatte gerade einen anonymen Tipp erhalten, dass Madam Bones sich mit einem ihrer jüngeren Assistenten vergnügte.

Das wäre einen ziemlichen Bonus wert, wenn sie es schaffen würde, ihn zu verifizieren, Bones stand ganz oben auf der Abschussliste.

Der Tippgeber hatte gesagt, dass Bones und ihre junge Assistentin in einem speziellen Raum im Mary's Place zu Mittag essen sollten, einem sehr beliebten Raum für bestimmte Zwecke; ein Raum, der, wie sie herausgefunden hatte, sicher gegen alle Abhörgeräte war, aber nicht sicher gegen einen schönen blauen Käfer, der sich an eine Wand schmiegte…

"Aus dem Weg!"

sagte Rita und versuchte, Quirrell aus dem Weg zu schieben. Quirrells Arm streifte ihren eigenen, lenkte ab, und Rita taumelte, als der Stoß in die dünne Luft ging.

Quirrell zog den Ärmel seiner linken Robe hoch und zeigte seinen linken Arm.

"Beachten Sie", sagte Quirrell, "kein Dunkles Mal. Ich möchte, dass Ihre Zeitung eine Entschuldigung veröffentlicht."

Rita stieß ein ungläubiges Lachen aus. Natürlich war der Mann kein echter Todesser. Wenn er einer wäre, hätte die Zeitung es nicht veröffentlicht.

"Vergiss es, Freundchen. Und jetzt zieh Leine."

Quirrell starrte sie einen Moment lang an. Dann lächelte er.

"Miss Kimmkorn", sagte Quirrell,

"ich hatte gehofft, einen Hebel zu finden, der sich als überzeugend erweisen würde.

Doch ich finde, dass ich mir das Vergnügen nicht versagen kann, \emph{Sie einfach zu zerquetschen}."

"Es wurde schon oft \emph{versucht}. Und jetzt geh mir aus dem Weg, Freundchen, oder ich finde ein paar Auroren und lasse dich wegen Behinderung des Journalismus verhaften."

Quirrell machte eine kleine Verbeugung vor ihr und ging dann vorbei.

"\emph{Auf Wiedersehen, Rita Kimmkorn}", sagte seine Stimme hinter ihr.

Als Rita weiterlief, bemerkte sie im Hinterkopf, dass der Mann eine Melodie pfiff, während er davonlief. \emph{Als ob sie das erschrecken würde.}

\textbf{Akt 4:}

"Tut mir leid, ich bin nicht dabei", sagte Lee Jordan. "Ich bin eher der Riesenspinnen-Typ."

Der Junge-der-lebte hatte gesagt, er habe eine wichtige Aufgabe für den Orden des Chaos, etwas Ernstes und Geheimes, bedeutsamer und schwieriger als ihre übliche Reihe von Streichen.

Und dann hatte Harry Potter mit einer Rede begonnen, die inspirierend, aber auch vage war. Eine Rede, die darauf hinauslief, dass Fred und George und Lee ein enormes Potenzial hätten, wenn sie nur lernen könnten, noch verrückter zu sein.

Das Leben der Leute surreal zu gestalten, anstatt sie nur mit dem Äquivalent von Wassereimern über Türen zu überraschen.

(Fred und George hatten interessierte Blicke ausgetauscht, daran hatten sie nie gedacht.)

Harry Potter hatte ein Bild von dem Streich heraufbeschworen, den sie Neville gespielt hatten - für den, wie Harry mit einiger Reue erwähnt hatte, der Sprechende Hut ihn ausgepfiffen hatte -, der Neville aber wohl an seinem eigenen Verstand zweifeln ließ.

Für Neville muss es sich angefühlt haben, als wäre er plötzlich in ein anderes Universum versetzt worden.

So, wie sich alle anderen gefühlt hatten, als sie gesehen hatten, wie Snape sich entschuldigte. \emph{Das war die wahre Macht der Streiche.}

\emph{Seid ihr dabei?} hatte Harry Potter gerufen, und Lee Jordan hatte mit Nein geantwortet.

"Wir sind dabei", sagte Fred, oder möglicherweise George, denn es bestand kein Zweifel, dass Godric Gryffindor \emph{Ja} gesagt hätte.

Lee Jordan grinste bedauernd, stand auf und verließ den verlassenen und stillen Korridor, in dem sich die vier Mitglieder des Ordens des Chaos getroffen hatten und sich in einen konspirativen Kreis setzten.

Die drei Mitglieder des Ordens des Chaos kamen zur Sache.

(So traurig war es nicht. Fred und George würden immer noch mit Lee an den Riesenspinnenstreichen arbeiten, so wie immer. Sie hatten nur angefangen, sie den Orden des Chaos zu nennen, um Harry Potter zu rekrutieren, nachdem Ron ihnen erzählt hatte, dass Harry seltsam und böse sei, und \emph{Fred und George beschlossen hatten, Harry zu retten, indem sie ihm wahre Freundschaft und Liebe zeigten.}

Zum Glück schien das nicht mehr nötig zu sein -

obwohl sie sich da nicht ganz sicher waren…)

"Also", sagte einer der Zwillinge, "worum geht es hier?"

"Rita Kimmkorn", sagte Harry. "Wisst ihr, wer sie ist?"

Fred und George nickten und runzelten die Stirn.

"Sie hat Fragen über mich gestellt."

Das waren keine guten Nachrichten.

"Könnt ihr erraten, was ich von euch will?"

Fred und George sahen sich verwirrt an.

"Du willst, dass wir ihr ein paar unserer interessanteren Süßigkeiten zustecken?"

"Nein", sagte Harry. "Nein, nein, nein! Das ist ein Riesenspinnen-Denken! Kommt schon, was würdet ihr tun, wenn ihr hören würdet, dass Rita Kimmkorn auf der Suche nach Gerüchten über euch ist?"

Damit war es klar. Langsam begann das Grinsen auf den Gesichtern von Fred und George.

"Gerüchte über uns selbst in die Welt setzen", antworteten sie.

"Genau", sagte Harry und grinste breit. "Aber das können nicht nur irgendwelche Gerüchte sein. Ich möchte den Leuten beibringen, niemals zu glauben, was in der Zeitung über Harry Potter steht, genauso wenig wie Muggel glauben, was in der Zeitung über Elvis steht.

Zuerst dachte ich nur daran, Rita Kimmkorn mit so vielen Gerüchten zu überschwemmen, dass sie nicht weiß, was sie glauben soll, aber dann wird sie sich nur die herauspicken, die plausibel und schlecht klingen.

Ich möchte also, dass ihr eine falsche Geschichte über mich erfindtn und Rita Kimmkorn irgendwie dazu bringt, sie zu glauben.

Aber es muss etwas sein, von dem hinterher jeder weiß, dass es gefälscht war. Wir wollen Rita Kimmkorn und ihre Redakteure täuschen, und hinterher soll der Beweis herauskommen, dass es falsch war.

Und natürlich - das sind die Voraussetzungen - muss die Geschichte so lächerlich sein, wie sie nur sein kann, und trotzdem gedruckt werden.

Versteht ihr, was ich von Ihnen verlange?"

"Nicht ganz…" sagten Fred oder George langsam. "Du willst, dass wir die Geschichte erfinden?"

"Ich möchte, dass ihr das alles macht", sagte Harry Potter.

"Ich bin im Moment ziemlich beschäftigt, außerdem möchte ich wahrheitsgemäß sagen können, dass ich keine Ahnung hatte, was auf mich zukommt.

Überrascht mich."

Für einen Moment war ein sehr böses Grinsen auf den Gesichtern von Fred und George zu sehen. Dann wurden sie ernst.

"Aber Harry, wir wissen doch gar nicht, wie man so etwas macht -"

"Dann findet es heraus", sagte Harry.

"Ich habe Vertrauen in euch. Nicht das totale Vertrauen, aber wenn ihr es nicht könnt, sagt mir das, und ich versuche es mit jemand anderem oder mache es selbst.

Wenn ihr eine wirklich gute Idee habt sowohl für die lächerliche Geschichte, als auch dafür, wie ihr Rita Kimmkorn und ihre Redakteure davon überzeugen könnt, sie zu drucken - dann könnt ihr es ruhig tun.

Aber versucht es nicht mit etwas Mittelmäßigem. Wenn euch nichts Geniales einfällt, sag es einfach."

Fred und George tauschten besorgte Blicke aus.

"Mir fällt nichts ein", sagte George.

"Mir auch nicht", sagte Fred. "Tut mir leid."

Harry starrte sie an.

Und dann begann Harry zu erklären, wie man an Dinge denkt. Es sei allgemein bekannt, dass es länger als zwei Sekunden dauert, sagte Harry.

\emph{Man nannte nie eine Frage unmöglich}, sagte Harry, bis man eine echte Uhr genommen und fünf Minuten darüber nachgedacht hatte, anhand der Bewegung des Minutenzeigers.

\emph{Nicht fünf Minuten metaphorisch, sondern fünf Minuten mit einer physischen Uhr.}

Und außerdem, sagte Harry, seine Stimme war nachdrücklich und seine rechte Hand klopfte hart auf den Boden, fing man nicht sofort an, nach Lösungen zu suchen.

Harry begann dann mit der Erklärung eines Tests, der von jemandem namens Norman Maier durchgeführt wurde, der so etwas wie ein Organisationspsychologe war, und der zwei verschiedene Gruppen von Problemlösern gebeten hatte, ein Problem zu lösen.

Das Problem, sagte Harry, bestand darin, dass drei Angestellte drei Aufgaben zu erledigen hatten. Der jüngere Mitarbeiter wollte nur den einfachsten Job machen.

Der ältere Mitarbeiter wollte zwischen den Aufgaben rotieren, um Langeweile zu vermeiden. Ein Effizienzexperte hatte empfohlen, dem jüngeren Mitarbeiter die leichteste Aufgabe und dem älteren Mitarbeiter die schwerste Aufgabe zu geben, was 20 \% produktiver sein würde.

Eine Gruppe von Problemlösern hatte die Anweisung erhalten:

"Schlagen Sie keine Lösungen vor, bis das Problem so gründlich wie möglich besprochen wurde, ohne dass Sie Lösungen vorschlagen."

Die andere Gruppe von Problemlösern hatte keine Anweisungen erhalten.

Und diese Leute hatten das Natürliche getan und auf das Vorhandensein eines Problems mit Lösungsvorschlägen reagiert.

Und die Leute hingen an diesen Lösungen und fingen an, darüber zu streiten und über die relative Wichtigkeit von Freiheit gegenüber Effizienz zu argumentieren und so weiter.

Die erste Gruppe von Problemlösern, also diejenigen, die die Anweisung erhielten, das Problem erst zu diskutieren und dann zu lösen, waren viel eher auf die Lösung gekommen, den jüngeren Mitarbeiter den einfachsten Job behalten zu lassen und die beiden anderen zwischen den beiden anderen Jobs zu rotieren, was laut den Daten des Experten eine Verbesserung von 19 \% bedeuten würde.

Mit der Suche nach Lösungen anzufangen, war die völlig falsche Reihenfolge, wie ein Essen mit dem Nachtisch zu beginnen, nur schlecht.

(Harry zitierte auch jemanden namens Robyn Dawes, der sagte, dass je schwieriger ein Problem sei, desto eher würden die Leute versuchen, es sofort zu lösen).

Harry wollte dieses Problem also Fred und George überlassen, und sie würden alle Aspekte des Problems diskutieren und ein Brainstorming über alles machen, von dem sie dachten, dass es auch nur im Entferntesten relevant sein könnte.

Und sie sollten erst dann versuchen, eine tatsächliche Lösung zu finden, wenn sie damit fertig waren, es sei denn natürlich, dass ihnen zufällig etwas Tolles einfiel; in diesem Fall könnten sie es für später aufschreiben und dann wieder weiterdenken.

Und er wollte mindestens eine Woche lang keine Rückmeldung von ihnen hören, wenn ihnen angeblich nichts einfiel.

\emph{Manche Leute verbrachten Jahrzehnte damit, sich etwas einfallen zu lassen.}

"Irgendwelche Fragen?", fragte Harry.

Fred und George starrten sich gegenseitig an.

"Ich kann mir keine vorstellen."

"Ich mir auch nicht."

Harry hustete leise.

"Ihr habt nicht nach dem Budget gefragt."

\emph{Budget}? dachten sie.

"Ich könnte euch einfach den Betrag nennen", sagte Harry.

"Aber ich denke, das wird inspirierender sein."

Harrys Hände tauchten in seinen Mantel und holten etwas hervor -

Fred und George fielen fast um, obwohl sie saßen.

"Gebt es nicht um des Ausgebens willen aus", sagte Harry. Auf dem Steinboden vor ihnen glänzte eine absolut lächerliche Menge Geld.

"Gebt es nur aus, wenn es die Aufgabe erfordert; und wenn es die Aufgabe erfordert, zögert nicht, es auszugeben.

Wenn etwas übrig bleibt, gebt es hinterher einfach zurück, ich vertraue euch. Oh, und ihr bekommt zehn Prozent von dem, was da ist, egal, wie viel ihr am Ende ausgebt -"

"Das geht nicht!", platzte einer der Zwillinge heraus. "Wir nehmen kein Geld für so etwas an!"

(Die Zwillinge haben nie Geld dafür genommen, dass sie etwas Illegales getan haben. Ohne das er es wusste, verkauften sie Ambrosius Flume alle seine Waren mit null Prozent Aufschlag. Fred und George wollten bezeugen können - notfalls unter Veritaserum -, dass sie keine profitgierigen Kriminellen waren, sondern nur einen öffentlichen Dienst leisteten.)

Harry sah sie stirnrunzelnd an.

"Aber ich verlange von euch, dass ihr hier richtige Arbeit leistet.

Ein Erwachsener würde für so etwas bezahlt werden, und es würde immer noch als ein Gefallen für einen Freund zählen. Man kann nicht einfach Leute für so etwas anstellen."

Fred und George schüttelten den Kopf.

"Gut", sagte Harry.

"Ich besorge dann einfach teure Weihnachtsgeschenke, und wenn ihr versucht, sie mir zurückzugeben, werde ich sie verbrennen.

Jetzt wisst ihr nicht einmal, wie viel ich für euch ausgeben werde, außer dass es natürlich mehr sein wird, als wenn ihr das Geld einfach genommen hättet.

Und ich werde euch diese Geschenke sowieso kaufen, also denkt darüber nach, bevor ihr mir sagt, dass euch nichts Tolles einfällt."

Harry stand auf, lächelte und wandte sich zum Gehen, während Fred und George immer noch geschockt dreinschauten. Er schritt ein paar Schritte weg und drehte sich dann wieder um.

"Oh, eine letzte Sache", sagte Harry.

"Lasst Professor Quirrell aus allem heraus, was ihr tut. Er mag keine Publicity. Ich weiß, dass es leichter wäre, die Leute dazu zu bringen, seltsame Dinge über den Verteidigungsprofessor zu glauben als über irgendjemand anderen, und es tut mir leid, dass ich euch so in die Quere kommen muss, aber bitte lassen Sie Professor Quirrell da raus."

Und Harry drehte sich wieder um und ging ein paar Schritte weiter - schaute ein letztes Mal zurück und sagte leise: "Danke."

Und ging.

Es gab eine lange Pause, nachdem er gegangen war.

"Also", sagte der eine.

"\emph{Also}", sagte der andere.

"Der Verteidigungsprofessor mag keine Publicity, oder?"

"\emph{Harry kennt uns nicht sehr gut, oder}?"

"Nein, tut er nicht."

"\emph{Aber wir werden sein Geld natürlich nicht dafür verwenden}."

"Natürlich nicht, das wäre nicht richtig. Wir machen den Verteidigungsprofessor separat."

"\emph{Wir lassen ein paar Gryffindors an Kimmkorn schreiben und sagen…}"

"… sein Ärmel hat sich einmal in der Verteidigungsstunde gehoben und sie haben das Dunkle Mal gesehen…"

"\emph{… und dass er Harry Potter wahrscheinlich alle möglichen schrecklichen Dinge} \emph{beibringt..}."

".. und er ist der schlechteste Verteidigungsprofessor, an den man sich in Hogwarts erinnern kann, er unterrichtet uns nicht nur nicht, er macht alles falsch, das komplette Gegenteil von dem, was sein sollte…"

".\emph{..wie zum Beispiel, als er behauptete, man könne den Tötungsfluch nur mit Liebe sprechen, was ihn ziemlich nutzlos machte}."

"Das gefällt mir."

"\emph{Danke}."

"Ich wette, der Verteidigungsprofessor mag es auch."

"\emph{Er hat einen Sinn für Humor. Er hätte uns nicht so genannt, wenn er keinen Sinn für Humor hätte.}"

"Aber werden wir wirklich in der Lage sein, Harrys Job zu erledigen?"

„\emph{Harry sagte, wir sollten das Problem besprechen, bevor wir versuchen, es zu lösen, also sollten wir das tun}.“

Die Weasley-Zwillinge beschlossen, dass George der Enthusiast sein würde, während Fred zweifelte.

"Es scheint alles irgendwie widersprüchlich zu sein", sagte Fred.

"Er will, dass es so lächerlich ist, dass alle über Kimmkorn lachen und wissen, dass es falsch ist, und er will, dass Kimmkorn es glaubt. Wir können nicht beides gleichzeitig machen."

"\emph{Wir werden Beweise fälschen müssen, um Kimmkorn zu überzeugen}", sagte George.

"Ist das eine Lösung?", fragte Fred. Sie überlegten.

"\emph{Vielleicht}", sagte George, "a\emph{ber ich glaube nicht, dass wir das so streng handhaben sollten, oder}?"

Die Zwillinge zuckten hilflos mit den Schultern.

"Dann muss der gefälschte Beweis also gut genug sein, um Kimmkorn zu überzeugen", sagte Fred. "Können wir das wirklich allein schaffen?"

"\emph{Wir müssen es nicht allein machen}", sagte George und zeigte auf den Geldstapel.

"\emph{Wir können andere Leute anheuern, die uns helfen.}"

Die Zwillinge bekamen einen nachdenklichen Ausdruck im Gesicht.

"Das könnte Harrys Budget ziemlich schnell aufbrauchen", sagte Fred.

"Das ist eine Menge Geld für uns, aber es ist nicht viel Geld für jemanden wie Flume."

"\emph{Vielleicht geben die Leute Rabatte, wenn sie wissen, dass es für Harry ist}", sagte George.

"\emph{Aber das Wichtigste ist, was immer wir tun, es muss unmöglich sein.}"

Fred blinzelte. "Was meinst du damit, unmöglich?"

"\emph{So unmöglich, dass wir keinen Ärger bekommen, weil niemand glaubt, dass wir es geschafft haben könnten. So unmöglich, dass sogar Harry anfängt, sich zu wundern. Es muss surreal sein, es muss die Leute an ihrem eigenen Verstand zweifeln lassen, es muss … besser als Harry sein.}"

Freds Augen waren vor Erstaunen weit aufgerissen. Das kam manchmal zwischen ihnen vor,, aber nicht oft.

"Aber warum?"

"\emph{Es waren Streiche. Es waren alles Streiche. Der Kuchen war ein Streich. Das Erinnermich war ein Streich.

Kevin Entwhistles Katze war ein Streich. Snape war ein Streich. Wir sind die besten Scherzkekse in Hogwarts. Sollen wir etwa kampflos aufgeben?}"

"\emph{Er ist der Junge-der-lebte}", sagte George. "\emph{Und wir sind die Weasley-Zwillinge!

Er fordert uns heraus. Er sagt, wir könnten tun, was er tut. Aber ich wette, er glaubt nicht, dass wir jemals so gut sein werden wie er.}"

"Er hat recht", sagte Fred und fühlte sich ziemlich nervös.

Die Weasley-Zwillinge waren sich manchmal nicht einig, selbst wenn sie alle die gleichen Informationen hatten, aber jedes Mal, wenn sie es taten, schien es unnatürlich, als ob mindestens einer von ihnen etwas falsch machen müsste.

"Das ist Harry Potter, über den wir hier reden. Er kann das Unmögliche tun. Wir können es nicht."

"\emph{Doch, das können wir}", sagte George. "\emph{Und wir müssen noch unmöglicher sein als er.}"

"Aber -", sagte Fred.

"\emph{Das würde Godric Gryffindor auch tun}", sagte George.

Damit war die Sache erledigt und die Zwillinge kehrten zurück in… was auch immer für sie normal war.

"In Ordnung, dann -"

"\emph{- lass uns darüber nachdenken.}"

