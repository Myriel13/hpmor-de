

\hypertarget{tests}{% \section{102. Tests}\label{tests}}

\uline{Tests}

Juni 1992.

Daphne Greengrass befand sich im Slytherin-Gemeinschaftsraum und schrieb einen Brief an ihre Herrin Mutter (die erstaunlich unnachgiebig war, was die Teilung der Macht anging, obwohl sie nicht einmal in Hogwarts war, um die Kontrolle auszuüben), als sie Draco Malfoy durch die Porträttür taumeln sah, der wohl ein Dutzend Bücher trug, Vincent und Gregory hinter ihm, jeder mit einem weiteren Dutzend. Der Auror, der Malfoy begleitet hatte, steckte kurz den Kopf herein und zog sich dann nach wer-weiß-wohin zurück. Draco sah sich um, dann schien er eine glänzende Idee zu haben, denn er taumelte auf sie zu, Vincent und Gregory folgten ihm.

„Kannst du mir helfen, die zu lesen?“, sagte Draco und klang leicht außer Atem, als er sich näherte.

„Was?“ Der Unterricht war vorbei, jetzt standen nur noch die Prüfungen an, und seit wann baten Malfoys Greengrasses um Hilfe bei ihren Hausaufgaben?

„Das“, sagte Draco Malfoy wichtig, „sind alle Bibliotheksbücher, die Miss~Granger zwischen dem 1. und 16. April ausgeliehen hat. Ich dachte, ich schaue sie durch, falls es dort irgendwelche Hinweise gibt, nur dachte ich dann, vielleicht solltest du helfen, weil du Miss~Granger besser kennst.“

Daphne starrte auf die Bücher.

„Der General hat das alles in zwei Wochen gelesen?“ Ein schmerzhafter Stich durchfuhr ihr Herz, aber sie unterdrückte ihn.

„Nun, ich weiß nicht, ob Miss~Granger sie alle gelesen hat“, sagte Draco. Er hielt einen mahnenden Finger hoch. „Tatsächlich wissen wir nicht, ob sie eines davon gelesen hat, oder ob sie es wirklich ausgeliehen hat, ich meine, alles, was wir beobachtet haben, ist, dass im Bibliotheksbuch steht, dass sie es ausgeliehen hat—“

Daphne unterdrückte ein Stöhnen. Malfoy redete schon seit Wochen so. Es gab einige Leute, die eindeutig nicht dazu bestimmt waren, mit mysteriösen Morden zu tun zu haben, denn es machte seltsame Dinge mit ihrem Verstand.

„Mr~Malfoy, ich könnte das alles nicht lesen, wenn ich den ganzen Sommer nichts anderes machen würde.“

„Dann überfliege sie bitte einfach? Vor allem, wenn da, du weißt schon, geheimnisvolle Worte in ihrer Handschrift gekritzelt sind, oder ein Lesezeichen drin steckt, oder—“

„Ich habe diese Stücke auch schon gesehen, Mr~Malfoy.“ Daphne rollte mit den Augen. „Haben wir jetzt nicht Auroren für—“

„Wir sind dem Untergang geweiht!“, kreischte Millicent Bulstrode, als sie aus den unteren Räumen in den Slytherin-Gemeinschaftsraum stürmte.

Die Leute hielten inne und sahen sie an.

„Es ist Professor Quirrell!“

Eine plötzliche Aufmerksamkeit, als ob ein langjähriger Streit geschlichtet werden sollte.

„Na, endlich“, sagte jemand, während Millicent versuchte, zu Atem zu kommen. „Er hatte ja nur noch, was, zehn Tage, um gefeuert zu werden?“

„Elf Tage“, sagte der Siebtklässler, der das Wettbüro leitete.

„Es geht ihm plötzlich etwas besser, und er wird die Erstklässler für unsere Verteidigungsprüfungen holen! Überraschend! In fünfzig Minuten!“

„Eine Verteidigungsprüfung?“ sagte Pansy verständnislos. „Aber Professor Quirrell hält keine Prüfungen.“

„Die Verteidigungsprüfung des Ministeriums!“, kreischte Millicent.

„Aber Professor Quirrell unterrichtet doch gar nichts aus dem Lehrplan des Ministeriums“, wandte Pansy ein.

Daphne war bereits in ihr Zimmer geflüchtet, rannte zu dem Verteidigungslehrbuch für das erste Jahr, das sie seit September nicht mehr angerührt hatte, und schrie in Gedanken Flüche.

…

Ein Pult hinter ihr weinte jemand, dessen leises Schluchzen einen Hintergrundgesang der Verzweiflung für das Klassenzimmer darstellte. Daphne schaute zurück, in der Erwartung, einen Hufflepuff zu sehen und in der Hoffnung, dass es nicht Hannah war, und war im ersten Moment überrascht (wenn auch nicht bei näherem Nachdenken) zu sehen, dass es eine Ravenclaw war. Vor ihnen lagen die Prüfungspergamente, umgedreht, und sie warteten auf die Glocke. Fünfzig Minuten waren nicht annähernd genug Vorbereitungszeit gewesen, aber es war zumindest etwas, und Daphne schämte sich jetzt, dass sie nicht daran gedacht hatte, Boten zu schicken, um die Häuser Hufflepuff, Ravenclaw und Gryffindor zu warnen. Sie hatten erst vor drei Tagen, Anfang Juni, wieder angefangen, Hauspunkte zu vergeben, aber das Hilfsschutz-Sonderkomitee sollte immer noch die Einheit der Häuser fördern. Eine andere Ravenclaw, die vier Tische links von ihr saß, begann ebenfalls zu weinen. Das war Katherine Tung von der Drachenarmee, wenn sie sich richtig erinnerte, die sie einmal gesehen hatte, wie sie es mit drei Sonnenschein Soldaten gleichzeitig aufgenommen hatte, ohne mit der Wimper zu zucken. Daphne hatte sich nach den ersten paar Minuten des hektischen Lesens wieder beruhigt. Es war nur ein Test, kein Mord oder so etwas; und wenn fast alle Erstklässler überwiegend leere Pergamente abgaben, dann lag es auf der Hand, dass sich niemand schämen würde. Aber Daphne konnte verstehen, wenn auch nicht gerade mitfühlen, dass Ravenclaws und Hufflepuffs das vielleicht nicht so sahen.

„Er ist böse“, sagte eine andere Ravenclaw-Hexe mit zitternder Stimme. „Ein hundertprozentiger, reiner dunkler Zauberer bis auf die Knochen. Der Dunkle Lord Grindelwald würde so etwas nicht tun, nicht mit Kindern, er ist schlimmer als Du-weißt-schon-wer.“

Daphne schaute reflexartig zu der Stelle, wo Professor Quirrell saß, zur Seite gesackt, aber mit wachen Augen; und sie glaubte, den Verteidigungsprofessor für einen winzigen Augenblick lächeln zu sehen. Nein, das musste sie sich einbilden, das konnte der Verteidigungsprofessor auf keinen Fall gehört haben.

\emph{Die Glocke läutete.}

Daphne klappte das Pergament um. Oben waren die Siegel des Ministeriums, des Obersten Rates von Hogwarts und der Abteilung für magische Erziehung sowie Runen zur Erkennung von Betrug aufgedruckt. Darunter befand sich eine Zeile, in die sie ihren Namen schreiben konnte, und eine Liste der Prüfungsregeln mit einem Bild von Lindsay Gagnon, der Direktorin der Abteilung für magische Erziehung, die jedem einen mahnenden Finger entgegenstreckte. Auf der Hälfte der Seite stand die erste Prüfungsfrage. Sie lautete:

„\emph{Warum ist es für Kinder wichtig, sich von fremden Wesen fernzuhalten?}“

Es gab eine verblüffte Pause.

Eine Schülerin begann zu lachen, sie dachte, es sei aus dem Gryffindor-Teil der Klasse. Professor Quirrell machte keine Anstalten, es zu unterbinden, und das Lachen breitete sich aus. Niemand sprach laut, aber die Schüler sahen sich um, tauschten Blicke aus, als das Lachen abebbte, und dann sahen sie alle wie durch eine unausgesprochene Übereinkunft zu Professor Quirrell, der wohlwollend auf sie herablächelte.

Daphne beugte sich über ihre Prüfung, mit einem trotzigen, bösen Lächeln, das sowohl Godric Gryffindor als auch Grindelwald zur Ehre gereicht hätte; und sie schrieb auf:

\emph{Weil mein Betäubungsfluch, mein Klingenzauber und mein Patronus nicht gegen alles wirken.}

…

Harry Potter blätterte die letzte Seite seiner Verteidigungsprüfung um. Selbst Harry hatte ein kleines bisschen Nervosität unterdrücken müssen, ein winziges Überbleibsel seiner Kindheit, als er die erste richtige Frage las

(„\emph{Wie kann man einen Schreienden Aal zum Schweigen bringen?}“).

Professor Quirrells Unterricht hatte so gut wie keine Zeit auf die überraschenden, aber nutzlosen Belanglosigkeiten verwendet, die sich irgendein Idiot bei einem '\emph{Verteidigungskurs}' vorgestellt hatte. Im Prinzip hätte Harry seinen Zeitumkehrer benutzen können, um das Erstklässler-Verteidigungsbuch durchzulesen, nachdem er von der Überraschungsprüfung erfahren hatte; aber das hätte die Benotungskurve für die anderen unfairerweise verzerrt.

Nachdem er ein paar Sekunden lang auf die Frage gestarrt hatte, schrieb Harry „Schweigezauber“ auf und fügte die Zauberanweisungen hinzu, falls der Prüfer des Ministeriums nicht glaubte, dass Harry ihn kannte.

Nachdem Harry beschlossen hatte, einfach alle Fragen richtig zu beantworten, war die Prüfung sehr schnell vorbei gegangen. Die realistischste Antwort auf mehr als die Hälfte der Fragen war '\emph{Betäubungsfluch}', und viele der anderen Fragen hatten optimale Lösungen nach dem Muster '\emph{Dreh dich um und gehe in die entgegengesetzte Richtun}g' oder '\emph{Werfe den Käse weg und kaufe ein neues Paar Schuhe.'}

Die letzte Frage des Tests lautete:

„\emph{Was würdest du tun, wenn du vermutest, dass sich unter deinem Bett eine Kraytschlange befindet?}“

Die vom Ministerium zugelassene Antwort, an die sich Harry noch aus der Lektüre des Lehrbuchs zu Beginn des Jahres erinnern konnte, lautete: \emph{Sag es deinen Eltern.}

Das Problem dabei war Harry sofort aufgefallen, deshalb hatte er sich auch daran erinnert.

Nach einigem Grübeln schrieb Harry auf:

\emph{Sehr geehrter Prüfer des Ministeriums:}

\emph{Ich fürchte, die wirkliche Antwort darauf ist ein Geheimnis, aber seien Sie versichert, dass eine Kraytschlange mir nicht mehr Ärger bereiten würde als ein Bergtroll, ein Dementor oder Du-weißt-schon-wer.}

\emph{\hfill\break Bitte teilen Sie Ihren Vorgesetzten mit, dass ich Ihre Standardantwort gegenüber} \emph{Muggelgeborenen als unangemessen empfinde, und dass ich erwarte, dass dieser Fehler sofort korrigiert wird, ohne dass mein direktes Eingreifen erforderlich ist.}

\emph{\hfill\break Mit freundlichen Grüßen,}

\emph{der Junge-der-lebte.}

Harry unterschrieb das letzte Pergament mit einem breiten Schwung, legte es in seinen Stapel, legte seinen Federkiel nieder und setzte sich auf. Als er sich umschaute, sah Harry, dass Professor Quirrell in seine ungefähre Richtung zu schauen schien, obwohl der Kopf des Verteidigungsprofessors zur Seite genickt hatte.

Die anderen Schüler waren immer noch am Schreiben. Einige von ihnen weinten leise, aber sie schrieben immer noch. Weiter zu kämpfen war auch eine Lektion, die Professor Quirrell erteilt hatte.

Wenig später war die offizielle Prüfungszeit um. Ein Siebtklässler ging von Pult zu Pult und sammelte an Professor Quirrells Stelle die Klausuren ein. Das letzte Prüfungspergament wurde eingesammelt, und Professor Quirrell setzte sich aufrecht hin.

„Meine jungen Schüler“, sagte er leise.

Die Schülerin im siebten Jahr hatte ihren Zauberstab auf den Mund des Verteidigungsprofessors gerichtet, so dass sie alle seine Stimme hörten, die von direkt neben ihnen zu kommen schien.

"Ich weiß… das erschien einigen von euch wahrscheinlich sehr furchteinflößend… es ist eine andere Art von Angst, als sich dem Zauberstab des Feindes zu stellen… ihr müsst sie selbst besiegen. Deshalb… werde ich euch jetzt Folgendes sagen. Es ist Brauch in Hogwarts, dass die Noten in der zweiten Juniwoche vergeben werden.

Aber für meinen Fall… kann man eine Ausnahme machen, denke ich."

Der Verteidigungsprofessor lächelte sein vertrautes trockenes Lächeln, das jetzt wie von einer unterdrückten Grimasse aus Schmerzen entsellt war.

„Ich weiß, dass Sie sich Sorgen machen… dass Sie nicht auf diese Prüfung vorbereitet waren… dass mein Unterricht diesen Stoff nicht behandelt hat… und ich habe ganz vergessen zu erwähnen… dass sie bevorsteht… obwohl Sie hätten wissen müssen… dass sie rechtzeitig kommen würde. Aber ich habe soeben auf magische Weise… die Antworten überprüft, die ihr bei dieser… furchtbar, furchtbar wichtigen Abschlussprüfung gegeben habt… obwohl natürlich nur die Note des Ministeriums offiziell ist… und eure Jahresnoten unter Berücksichtigung der Ergebnisse zugewiesen… und auf magische Weise eure vollen Noten auf diese Pergamente geschrieben wie ich Sie vergeben würde“,

Professor Quirrell tippte auf einen Stapel Pergamente auf der Seite seines Schreibtisches,

„die jetzt ausgeteilt werden… ein unglaublicher Zauber… nicht wahr?“

Ein paar Schüler auf der Ravenclaw-Seite schauten entrüstet, aber zum größten Teil sahen die Schüler einfach nur erleichtert aus, und einige Slytherins kicherten. Harry hätte auch gelacht, wenn es nicht so weh getan hätte, Professor Quirrell dabei zuzusehen, wie er die Worte ausstieß.

Die Schülerin im siebten Jahr, die neben Professor Quirrell stand, zeigte ihren Zauberstab und sprach eine Beschwörungsformel in magischem Pseudolatein.

Die Pergamente erhoben sich und begannen durch die Luft zu schweben, trennten sich in der Mitte der Halle, um zu jedem Schüler zu schweben.

Harry wartete, bis sein Pergament auf seinem Schreibtisch angekommen war, und entfaltete es dann. Auf dem Pergament stand \textbf{E+}, was für \emph{Erwartungen übertroffen} stand. Es war der zweithöchste Notenbuchstabe, der höchste war \emph{Ohnegleichen}.

In einer anderen Welt, einer fernen, verschwundenen Welt, hätte ein kleiner Junge namens Harry vor Empörung darüber geschrien, dass er nur die zweithöchste Note bekommen hatte. Dieser Harry saß still da und dachte nach. Professor Quirrell wollte damit irgendetwas sagen, und es war nicht so, dass der genaue Notenbuchstabe in irgendeiner Weise von Bedeutung war.

Wollte Professor Quirrell damit sagen, dass Harry zwar relativ gut abgeschnitten, aber nicht sein volles Potenzial ausgeschöpft hatte? Oder war die Note wörtlich zu lesen, dass Harry tatsächlich die Erwartungen des Verteidigungsprofessors übertroffen hatte?

„Ihr habt alle… bestanden“, sagte Professor Quirrell, als alle Schüler auf ihre Abschlussnoten blickten, während Seufzer der Erleichterung von den Tischen aufstiegen und Lavender Brown ihr Pergament in einer geballten, triumphierend hochgehaltenen Faust erhob.

„Jeder Schüler im ersten Jahr Kampfmagie hat bestanden…bis auf einen.“

Eine Reihe von Schülern blickte plötzlich erschrocken auf. Harry saß schweigend da. Er hatte den Punkt sofort erkannt, und selbst wenn es ein falscher Punkt war, wusste er, dass Professor Quirrell sich niemals, niemals davon abbringen lassen würde, ihn zu machen.

„Jeder von euch in diesem Raum… hat mindestens die Note Annehmbar erhalten. Neville Longbottom… der diesen Test im Hause Longbottom abgelegt hat… hat ein Ohnegleichen erhalten. Aber die andere Schülerin, die nicht anwesend ist, hat die Note “\emph{Schrecklich}„ in ihre Akte eingetragen bekommen, weil sie bei der einzigen wichtigen Prüfung, die sie in diesem Jahr abgelegt hat, durchgefallen ist. Ich hätte sie noch schlechter benotet… aber das wäre geschmacklos gewesen.“

Im Raum war es sehr still, obwohl einige Schüler den Professor wütend anstarrten.

„Sie werden vielleicht denken, dass eine Note von Schrecklich… nicht fair ist. Dass Miss~Granger mit einem Test konfrontiert wurde… auf den ihr Unterricht… sie nicht vorbereitet hatte. Dass ihr nicht gesagt wurde… dass die Prüfung an diesem Tag stattfinden würde.“

Der Verteidigungsprofessor holte zitternd Luft.

„So ist der Realismus“, sagte Professor Quirrell. „Die einzige wichtige Prüfung… kann jederzeit kommen… seid besser darauf vorbereitet… als sie es war. Was den Rest von euch angeht… diejenigen, die \emph{Erwartungen übertroffen} oder besser abgeschnitten haben… habe ich Empfehlungsschreiben geschrieben… an bestimmte Organisationen jenseits der britischen Küste… wo eure Ausbildung vervollständigt werden könnte. Man wird sich mit euch in Verbindung setzen, wenn ihr alt genug seid, wenn ihr noch würdig erscheint und wenn ihr nicht bei einem wichtigen Test durchgefallen seid. Und denkt daran…von diesem Tag an…müsst Ihr Euch selbst ausbilden…Ihr könnt Euch nicht…auf zukünftige Verteidigungsprofessoren verlassen. Euer erstes Jahr in Kampfmagie ist vorbei… ihr seid entlassen.“

Professor Quirrell lehnte sich mit geschlossenen Augen zurück und schien das aufgeregte Geplapper, das um ihn herum ausbrach, zu ignorieren. Mit der Zeit hatten sich die meisten Schüler entfernt, einer blieb zurück und hielt einen vorgeschriebenen Abstand zum Verteidigungsprofessor.

Der Verteidigungsprofessor öffnete seine Augen.

Harry hob das Pergament mit seinem \textbf{E+}, immer noch schweigend.

Der Verteidigungsprofessor lächelte, und es ging bis in die müden Augen.

„Es ist die gleiche Note…die ich in meinem eigenen ersten Jahr erhalten habe.“

„Da… dan…“,

Harry brachte das Wort „\emph{Danke}“ nicht heraus, sie blieben in seiner plötzlich zugeschnürten Kehle stecken.

Der Verteidigungsprofessor neigte den Kopf und warf ihm einen fragenden Blick zu, also verbeugte sich Harry nur ruckartig und verließ dann den Raum.

