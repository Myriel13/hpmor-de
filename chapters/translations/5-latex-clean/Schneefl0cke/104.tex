

\hypertarget{die-wahrheit-teil-3}{% \section{105. Die Wahrheit, Teil 3}\label{die-wahrheit-teil-3}}

\textbf{\uline{Die Wahrheit, Teil 3}}

\hfill\break Nach einem einzigen Schritt in Dumbledores verbotene Kammer schrie Harry auf, sprang zurück und prallte mit Professor Snape zusammen, so dass die beiden auf einen Haufen fielen. Professor Snape rappelte sich auf und stellte sich wieder vor die Tür. Sein Kopf hob sich, um Harry anzusehen.

„Ich bewache diese Tür auf Anweisung des Schulleiters“, sagte Professor Snape in seinem üblichen sardonischen Ton. „Verschwinde sofort, sonst ziehe ich dir Hauspunkte ab."

Das war unheimlich, aber Harrys Aufmerksamkeit wurde von dem riesigen dreiköpfigen Hund in Anspruch genommen, der sich auf ihn gestürzt hatte, nur um von den Ketten an seinen drei Halsbändern zentimerweit vor Harry aufgehalten zu werden.

„Das - das - das -“, sagte Harry.

„Ja“, sagte Professor Quirrell ein Stück hinter ihm, „das ist in der Tat der übliche Bewohner dieser Kammer, die für alle Schüler, insbesondere für Erstklässler, tabu ist."

"Das ist selbst für Zauberer-Verhältnisse nicht sicher!"

In der Kammer gab die riesige schwarze Bestie ein mehrstimmiges Brüllen von sich, wobei Flecken von weißem Speichel aus drei Reißzahnmäulern flogen.

Professor Quirrell seufzte.\\ "Es ist verzaubert, dass es keine Schüler frisst, sondern sie durch die Tür wieder ausspuckt. Also, Junge, was würdest du empfehlen, wie wir mit dieser gefährlichen Kreatur umgehen sollen?"

„Äh“, stotterte Harry und versuchte, über das anhaltende Gebrüll des Wächters der Kammer hinwegzudenken. „Äh. Wenn es wie der Zerberus aus der Muggelsage von Orpheus und Eurydike ist, dann müssen wir es in den Schlaf singen, damit wir -"

"Avada Kedavra."

Das dreiköpfige Biest fiel um.

Harry schaute wieder zu Professor Quirrell, der ihm einen äußerst enttäuschten Blick zuwarf, als wollte er fragen, ob Harry jemals einen seiner Kurse besucht hatte.

„Ich habe irgendwie angenommen“, sagte Harry, der immer noch versuchte, nach Luft zu schnappen, „dass es vielleicht einen Alarm auslösen könnte, wenn man diese Herausforderung auf irgendeine andere Weise als die der Erstklässler durchführt."

"Das ist eine Lüge, Junge, du hast dich einfach nicht an deine Lektionen erinnert, als du im wahren Leben vor dieser Aufgabe standest. Was die Alarme angeht, so habe ich Monate damit verbracht, alle Wach-Zauber und Fallen in diesen Kammern zu verwirren."

"Warum hast du mich dann zuerst reingeschickt?"

Professor Quirrell lächelte nur. Es sah deutlich böser aus als sonst.

„Egal“, sagte Harry und ging langsam in die Kammer, seine Glieder zitterten immer noch.

Die Kammer war ganz aus Stein und wurde von einem blassblauen Licht erhellt, das aus gewölbten, in die Wand gehauenen Nischen schien; so als würde das Licht eines grauen Himmels durch Fenster eindringen, obwohl es keine Fenster gab. Am anderen Ende der Kammer war eine hölzerne Falltür auf dem Boden, an der ein einzelner Ring befestigt war. In der Mitte der Kammer lag ein riesiger toter Hund mit drei leblosen Köpfen. Harry wandte sich einer der gewölbten Nischen zu und schaute in sie hinein. Dort war nichts außer dem sauerstofflosen blauen Schein, also ging er hinüber und schaute in die nächste, wobei er auch die Wand im Vorbeigehen untersuchte.

"Was machst du da?"

„Ich durchsuche den Raum“, sagte Harry. „Da könnte ein Hinweis sein, oder eine Inschrift, oder ein Schlüssel, den wir später brauchen, oder irgendetwas -"

"Ist das dein Ernst, oder versuchst du absichtlich, uns aufzuhalten? Antworte in Parseltongue.„

Harry blickte zurück.

„\emph{Mein es ernssst}“, zischte Harry. „\emph{Hätte dassselbe getan, wenn ich allein gekommen wäre.}"

Professor Quirrell massierte sich kurz die Stirn.

„Ich gebe zu“, sagte er, „dass deine Herangehensweise dir bei der Erforschung des Grabes von Amon-Set gute Dienste leisten würde, also werde ich dich nicht ganz als Idioten bezeichnen, aber dennoch. Das falsche Rätsel, die äußere Form der Herausforderung, dies hier ist ein Spiel, das für Erstklässler gedacht ist. Wir gehen einfach durch die Falltür nach unten."

Unter der Falltür befand sich eine gigantische Pflanze, so etwas wie eine riesige Dieffenbachia mit breiten Blättern, die aus dem zentralen Stamm wie eine Wendeltreppe hervorgingen, aber dunkler gefärbt als eine normale Dieffenbachia, mit rankenartigen Stufen, die aus dem zentralen Stamm hervorgingen und herunterhingen. Die Basis breitete sich mit größeren Blättern und Ranken weit aus, als ob sie verspräche, den Sturz eines jeden abzufedern. Darunter befand sich eine weitere Steinkammer wie die erste, mit den gleichen Nischen wie falsche Bogenfenster, die das gleiche graublaue Licht ausstrahlten.

„Der naheliegende Gedanke ist, auf dem Besen in meinem Beutel hinunterzufliegen oder etwas Schweres zu werfen, um zu sehen, ob diese Ranken Fallen sind“, sagte Harry und blickte hinunter. „Aber ich nehme an, du wirst sagen, wir gehen einfach die Blätter hinunter."

Sie sahen auf jeden Fall so aus, als sollten sie eine Wendeltreppe sein.

„Nach dir“, sagte Professor Quirrell.

Harry setzte vorsichtig einen Fuß auf ein Blatt und stellte fest, dass es tatsächlich sein Gewicht trug. Dann warf Harry einen letzten Blick in den Raum, bevor er sich verabschiedete, um zu sehen, ob es irgendetwas gab, das es wert war, bemerkt zu werden. Der riesige tote Hund zog genug Aufmerksamkeit auf sich, dass es schwer war, sich auf etwas anderes zu konzentrieren.

„Professor Quirrell“, sagte Harry und ließ den Satz weg, dass Ihre Herangehensweise an Hindernisse gewisse Nachteile hat, „was ist, wenn jemand zur Tür hereinschaut und sieht, dass der Cerberus tot ist?"

„Dann haben sie wahrscheinlich schon bemerkt, dass mit Snape etwas nicht stimmt“, sagte Professor Quirrell. „Aber da du darauf bestehst …„

Der Verteidigungsprofessor ging zu der dreiköpfigen Leiche hinüber und setzte seinen Zauberstab dagegen. Er begann eine lateinisch klingende Beschwörungsformel, die von einem Gefühl aufsteigender Beklemmung begleitet wurde. Der Junge-der-lebte spürte die Macht des Dunklen Lords wie immer. Das letzte Wort, das er sprach, war „Inferius“, und es wurde von einem letzten Aufbäumen von \textbf{STOPP, NICHT} begleitet.

Und der dreiköpfige Hund erhob sich, seine sechs Augen stumpf und leer, und wandte sich wieder der Tür zu.

Harry starrte den riesigen Inferius mit einem schrecklichen Gefühl im Magen an, dem drittschlimmsten Gefühl, das er je in seinem Leben empfunden hatte. Jetzt wusste er, dass er diesen Vorgang schon einmal gesehen und gespürt hatte, nur ohne das gesprochene Latein. Der Zentaur, der ihn im Verbotenen Wald konfrontiert hatte, war tot. Der Verteidigungsprofessor hatte ihn mit einem echten Avada Kedavra getroffen, nicht mit einem gefälschten. Irgendwo in seinem Hinterkopf hatte Harry gedacht, wenn er nur Hermine zurückbekommen könnte, dann könnte er zum Kodex des Nichtsterbens zurückkehren, der Ethik von Batman. Die meisten Menschen gingen durch ihr ganzes Leben, ohne dass jemand bei ihren Abenteuern ums Leben kam.\\ Und das sollte nicht sein. Er hatte es nicht einmal bemerkt, an dem Tag, als er seine letzte Chance auf den Sieg verlor. Selbst wenn Hermine jetzt wieder auferstanden wäre, hätte Harry den ganzen Schlamassel nicht überstanden, ohne dass jemand getötet worden wäre. Er hatte nicht einmal den Namen des Zentauren erfahren.

Harry sagte nichts laut. Der Verteidigungsprofessor würde entweder die Anschuldigung in Parsel bestätigen oder in Klartext lügen, und so oder so würde der Verteidigungsprofessor mehr Grund haben, Harrys nächste Handlungen zu verdächtigen. Aber Harry wusste, dass - obwohl er nicht wusste, wie er Professor Quirrell aufhalten sollte, obwohl er keinen positiven Akt des Verrats wagte, vielleicht nicht einmal den Entschluss fasste, bis die Zeit zum Sieg gekommen war - es niemals eine gütliche Einigung zwischen ihm und Lord Voldemort geben würde, denn diese beiden unterschiedlichen Geister konnten nicht in derselben Welt existieren. Und es war, als ob dieser Entschluss, dieses Wissen um den Gegensatz, eine Kraft aus dem hervorrief, was Harry für seine dunkle Seite gehalten hatte. Harry hatte nach dem Tag, an dem er den Troll getötet hatte, aufgehört, seine dunkle Seite bewusst anzusprechen. Aber seine dunkle Seite war nie etwas, das von ihm getrennt war. Sie war etwas, das er von Tom Riddle übernommen hatte. Harry wusste nicht, wie es dazu gekommen war, aber wenn man die Vermutung aufstellte, sollte er die Anklänge an kognitive Fähigkeiten in seiner dunklen Seite nutzen können. Nicht als separater Modus, wie Harry es sich zuerst vorgestellt hatte, sondern einfach als neuronale Muster mit einer starken Tendenz zur Verkettung, da sie einmal Teil eines zusammenhängenden Ganzen gewesen waren.

Das änderte leider nichts daran, dass Professor Quirrell die gleichen Fähigkeiten mit weitaus mehr Lebenserfahrung im Rücken besaß und außerdem die Pistole hatte.

Harry drehte sich um, setzte einen Fuß auf die riesige Pflanze und begann, die Wendeltreppe hinunterzugehen, die von den Blättern gebildet wurde.\\ Diesmal hatte Harry nicht zu lange gebraucht, aber er hatte sich einigermaßen erholt, obwohl der Kummer immer noch wie dickes Wasser auf ihm lastete. Es war keine kalte Stahlstange in seiner Wirbelsäule, aber es war dennoch etwas Gerades und Festes. Er würde das durchziehen, zuerst Hermine ins Leben zurückholen und dann, irgendwie, Professor Quirrell aufhalten. Oder erst Professor Quirrell aufhalten und dann selbst den Stein holen. Es musste irgendetwas geben, irgendeine Möglichkeit, irgendeine Gelegenheit, die sich bieten würde, irgendeinen Weg, Voldemort aufzuhalten und Hermine ins Leben zurückzuholen...

Harry setzte seinen Abstieg fort. Hinter ihm wartete der dreiköpfige Hund, der das Tor bewachte.

