

\hypertarget{die-falschen-fragen}{% \section{13. Die falschen Fragen}\label{die-falschen-fragen}}

\textbf{Die falschen Fragen}

\emph{Anmerkung Original Autor E. Y.: Keine Panik. Ich schwöre feierlich, dass es für alles, was in diesem Kapitel passiert, eine logische, vorhersehbare, kanonkonforme Erklärung gibt. Es ist ein Rätsel, du sollst versuchen, es zu lösen, und wenn nicht, lies einfach das Kapitel.}

\emph{„Das ist eines der offensichtlichsten Rätsel, die ich je gehört habe.“}

Als Harry am Morgen seines ersten vollen Tages in Hogwarts die Augen im Ravenclaw-Schlafsaal der Erstklässler öffnete, wusste er, dass etwas nicht stimmte. Es war still. Zu still. Oh, richtig... Am Kopfende seines Bettes befand sich ein Quietus-Zauber, der durch einen kleinen Schieberegler gesteuert wurde, was der einzige Grund war, warum es überhaupt möglich war, in Ravenclaw einzuschlafen.

Harry setzte sich auf und schaute sich um, in der Erwartung, andere zu sehen, die für den Tag aufstanden - der Schlafsaal, leer. Die Betten, zerwühlt und ungemacht. Die Sonne, die in einem ziemlich hohen Winkel hereinkam. Sein Quieter war ganz auf Maximum gestellt. Und sein mechanischer Wecker lief noch, aber der Alarm war ausgeschaltet. Man hatte ihm offenbar erlaubt, bis 9:52 Uhr zu schlafen. Trotz seiner Bemühungen, seinen 26-Stunden-Schlafzyklus mit seiner Ankunft in Hogwarts zu synchronisieren, war er letzte Nacht erst gegen 1 Uhr nachts ins Bett gekommen. Er hatte vorgehabt, mit den anderen Schülern um 7 Uhr morgens aufzuwachen, er konnte es verkraften, an seinem ersten Tag ein wenig unter Schlafentzug zu leiden, solange er vor dem morgigen Tag eine Art magischen Fix bekam. Aber jetzt hatte er das Frühstück verpasst. Und seine allererste Unterrichtsstunde in Hogwarts, in Kräuterkunde, hatte vor einer Stunde und zweiundzwanzig Minuten begonnen.

Die Wut erwachte langsam, ganz langsam in ihm. Oh, was für ein netter kleiner Streich. Schaltet seinen Wecker aus. Dreh den Leiser auf. Und lassen Sie Mr. Großkotz Harry Potter seine erste Stunde verpassen und sich vorwerfen lassen, dass er ein Langschläfer ist. Als Harry herausfand, wer das getan hatte...\\ Nein, das war nur möglich, wenn alle 12 anderen Jungs im Ravenclaw-Schlafsaal mitgemacht hätten. Jeder von ihnen hätte seine schlafende Gestalt gesehen. Sie alle hatten ihn beim Frühstück schlafen lassen. Die Wut verflog und wurde durch Verwirrung und ein furchtbar verletztes Gefühl ersetzt. Sie hatten ihn gemocht. Das hatte er gedacht. Letzte Nacht hatte er gedacht, dass sie ihn mochten. Warum... Als Harry aus dem Bett trat, sah er ein Stück Papier, das aus dem Kopfteil herausschaute.

Auf dem Zettel stand: „\emph{Meine lieben Ravenclaws, es war ein besonders langer Tag. Bitte lasst mich ausschlafen und macht euch keine Sorgen wegen meines verpassten Frühstücks. Ich habe meine 1. Klasse nicht vergessen. Mit freundlichen Grüßen, Harry Potter“}

Und Harry stand da, gefroren, Eiswasser begann durch seine Adern zu rinnen. Der Zettel war in seiner eigenen Handschrift, mit seinem eigenen Druckbleistift. Und er konnte sich nicht erinnern, ihn geschrieben zu haben.\\ Und... Harry blinzelte auf das Stück Papier. Und wenn er es sich nicht einbildete, waren die Worte \emph{„Ich habe es nicht vergessen“} in einem anderen Stil geschrieben, als ob er versuchte, sich selbst etwas zu sagen...?

Hatte er gewusst, dass seine Erinnerung gelöscht werden würde? War er lange aufgeblieben, hatte irgendeine Art von Verbrechen oder verdeckter Aktivität begangen, und dann... aber er kannte den Obliviate-Zauber nicht... hatte jemand anderes... was... Ein Gedanke kam Harry in den Sinn. Wenn er gewusst hätte, dass „Obliviate“ auf ihn gezaubert werden würde...\\ Noch im Pyjama lief Harry um sein Bett herum zu seinem Koffer, drückte mit dem Daumen gegen das Schloss, zog seinen Beutel heraus, steckte ihn in die Hand und sagte:

„Notiz an mich selbst.“

Und ein weiteres Stück Papier fiel ihm in die Hand. Harry nahm es heraus und starrte es an. Auch es war in seiner eigenen Handschrift. Auf dem Zettel stand:

\emph{Liebes Ich, bitte spiele das Spiel.}\\ \emph{Man kann das Spiel nur einmal im Leben spielen. Das ist eine unersetzliche Gelegenheit.}\\ \emph{Erkennungscode 927, ich bin eine Kartoffel.}\\ \emph{Mit freundlichen Grüßen,}\\ \emph{Du.}

Harry nickte langsam. \emph{"Erkennungscode 927, ich bin eine Kartoffel„} war in der Tat die Botschaft, die er im Voraus ausgearbeitet hatte - einige Jahre zuvor, während er fernsah -, die nur er kennen würde. Wenn er ein Duplikat von sich als wirklich ihn identifizieren müsste, oder so. Nur für den Fall. Sei vorbereitet. Harry konnte der Nachricht nicht trauen, es könnten andere Zaubersprüche im Spiel sein. Aber das schloss einen einfachen Streich aus. Er hatte das definitiv geschrieben und er erinnerte sich definitiv nicht daran, es geschrieben zu haben. Harry starrte auf das Papier und wurde sich der Tinte bewusst, die auf der anderen Seite durchschien. Er drehte es um. Auf der Rückseite stand:

\emph{ANWEISUNGEN FÜR DAS SPIEL: Sie kennen die Spielregeln nicht Sie kennen den Einsatz des Spiels nicht Sie kennen das Ziel des Spiels nicht Sie wissen nicht, wer das Spiel kontrolliert Sie wissen nicht, wie Sie das Spiel beenden Sie beginnen mit 100 Punkten.}\\ \emph{Beginnen Sie.}

Harry starrte auf die „Anleitung“. Diese Seite war nicht handgeschrieben; die Schrift war vollkommen regelmäßig, also künstlich. Sie sah aus, als wäre sie mit einer Zitierfeder geschrieben worden, wie die, die er sich zum Diktieren gekauft hatte. Er hatte absolut keine Ahnung, was da vor sich ging.\\ Nun... Schritt eins war, sich anzuziehen und zu essen. Vielleicht sollte er die Reihenfolge umkehren. Sein Magen fühlte sich ziemlich leer an. Natürlich hatte er das Frühstück verpasst, aber er war auf diese Eventualität vorbereitet, da er es sich im Voraus ausgemalt hatte. Harry griff in seine Tasche und sagte\\ „Snack-Riegel“,\\ in der Erwartung, die Schachtel mit den Müsliriegeln zu bekommen, die er vor seiner Abreise nach Hogwarts gekauft hatte. Was auftauchte, fühlte sich nicht wie eine Schachtel mit Müsliriegeln an. Als Harry seine Hand in sein Blickfeld brachte, sah er zwei winzige Schokoriegel - nicht annähernd genug für eine Mahlzeit -, die an einem Zettel befestigt waren, und der Zettel war in der gleichen Schrift wie die Spielanleitung beschriftet. Auf dem Zettel stand:

\emph{VERSUCH FEHLGESCHLAGEN: -1 PUNKT}\\ \emph{AKTUELLER PUNKTESTAND: 99}\\ \emph{KÖRPERLICHER ZUSTAND: IMMER NOCH HUNGRIG}\\ \emph{MENTALER ZUSTAND: VERWIRRT}

„Gleehhhh“, sagte Harrys Mund ohne irgendeine Art von bewusstem Eingriff oder Entscheidung seinerseits. Er stand etwa eine Minute lang da. Eine Minute später machte es immer noch keinen Sinn und er hatte immer noch absolut keine Ahnung, was vor sich ging, und sein Gehirn hatte noch nicht einmal angefangen, nach irgendwelchen Hypothesen zu greifen, als wären seine geistigen Hände in Gummibällen eingeschlossen und könnten nichts aufheben. Sein Magen, der seine eigenen Prioritäten hatte, schlug eine mögliche experimentelle Untersuchung vor.

„Ah...“ sagte Harry in den leeren Raum hinein. „Ich nehme nicht an, dass ich einen Punkt ausgeben und meine Schachtel Müsliriegel zurückbekommen könnte?“

Es herrschte nur Schweigen. Harry steckte seine Hand in den Beutel und sagte: „Eine Schachtel Müsliriegel.“

Eine Schachtel, die sich wie die richtige Form anfühlte, tauchte in seiner Hand auf... aber sie war zu leicht, und sie war offen, und sie war leer, und auf dem Zettel, der daran befestigt war, stand:

\emph{AUSGEGEBENE PUNKTE: 1}\\ \emph{AKTUELLE PUNKTE: 98}\\ \emph{SIE HABEN GEWONNEN: EINE SCHACHTEL MÜSLIRIEGEL}

„Ich würde gerne einen Punkt ausgeben und die echten Müsliriegel zurückbekommen“, sagte Harry.

Wieder war es still. Harry steckte seine Hand in den Beutel und sagte „Müsliriegel“. Nichts kam heraus. Harry zuckte verzweifelt mit den Schultern und ging zu dem Schrank hinüber, den er neben seinem Bett bekommen hatte, um seine Zaubererroben für den Tag zu holen. Auf dem Boden des Schranks, unter seinen Roben, lagen die Müsliriegel und ein Zettel:

\emph{PUNKTE VERBRAUCHT: 1}\\ \emph{AKTUELLE PUNKTE: 97}\\ \emph{DU HAST GEWONNEN: 6 MÜSLIRIEGEL,}\\ \emph{DU TRÄGST NOCH: PYJAMAS}\\ \emph{ESSEN SIE NICHT, WÄHREND SIE IHRE PYJAMAS TRAGEN, SIE WERDEN EINE PYJAMA-STRAFE ERHALTEN}

Und jetzt weiß ich, dass derjenige, der das Spiel steuert, wahnsinnig ist.

„Ich vermute, dass das Spiel von Dumbledore kontrolliert wird“, sagte Harry laut.

Vielleicht konnte er dieses Mal einen neuen Geschwindigkeitsrekord aufstellen, weil er so schnell war. Schweigen. Aber Harry begann, das Muster zu erkennen; der Zettel würde an der nächsten Stelle sein, an der er nachsah. Also schaute Harry unter seinem Bett nach.

\emph{HA! HA HA HA HA HA! HA HA HA HA HA HA! HA! HA! HA! HA! HA! HA! DUMBLEDORE KONTROLLIERT DAS SPIEL NICHT FALSCH GERATEN SEHR FALSCH RATEN}\\ \emph{-20 PUNKTE UND DU TRAGST IMMER NOCH PYJAMAS ES IST DEIN VIERTER ZUG UND DU TRAGST IMMER NOCH PYJAMAS}\\ \emph{PYJAMA STRAFE: -2 PUNKTE}\\ \emph{AKTUELLE PUNKTE: 75}

Puh, das war ein Puzzlespiel, wirklich. Es war erst sein erster Tag an der Schule und wenn man einmal Dumbledore ausschloss, kannte er niemanden hier, der so verrückt war. Sein Körper lief mehr oder weniger auf Autopilot, Harry sammelte einen Satz Bademäntel und Unterwäsche ein, zog die Kellerebene seines Koffers heraus (er war ein sehr privater Mensch und es könnte jemand in den Schlafsaal kommen), zog sich an und ging dann wieder nach oben, um seinen Schlafanzug wegzulegen. Harry hielt inne, bevor er die Schrankschublade herauszog, in der sein Pyjama lag. Wenn das Muster hier stimmte...

„Wie kann ich mehr Punkte sammeln?“ sagte Harry laut. Dann zog er die Schublade heraus.

\emph{GELEGENHEITEN, GUTES ZU TUN, GIBT ES ÜBERALL, ABER DIE DUNKELHEIT IST DER ORT, AN DEM DAS LICHT SEIN MUSS}\\ \emph{KOSTEN DER FRAGE: 1 PUNKT}\\ \emph{AKTUELLE PUNKTE: 74}\\ \emph{SCHÖNE UNTERWÄSCHE HAT DEINE MUTTER SIE AUSGESUCHT?}

Harry zerdrückte den Zettel in seiner Hand, das Gesicht flammte scharlachrot. Dracos Fluch kam zu ihm zurück. \emph{Sohn eines Schlammbluts} - Zu diesem Zeitpunkt wusste er, dass er es besser nicht laut aussprechen sollte. Er würde wahrscheinlich eine Strafe wegen Obszönität bekommen.

Harry umgürtete sich mit seinem Beutel und seinem Zauberstab. Er riss die Verpackung eines seiner Müsliriegel ab und warf ihn in den Mülleimer des Zimmers, wo er auf einem größtenteils angefressenen Schokoladenfrosch, einem zerknitterten Umschlag und etwas grünem und rotem Geschenkpapier landete. Die anderen Müsliriegel steckte er in seinen Beutel. Er sah sich in einer letzten, verzweifelten und letztlich vergeblichen Suche nach Hinweisen um. Und dann verließ Harry den Schlafsaal, während er aß, auf der Suche nach den Slytherin-Kerkern. Zumindest dachte er, dass es in der Zeile darum ging.

Der Versuch, sich in den Gängen von Hogwarts zurechtzufinden, war wie... wahrscheinlich nicht ganz so schlimm, wie im Inneren eines Escher-Gemäldes herumzuwandern, das war die Art von Dingen, die man eher wegen des rhetorischen Effekts sagte, als weil sie wahr waren. Kurze Zeit später dachte Harry darüber nach, dass ein Escher-Gemälde im Vergleich zu Hogwarts in der Tat sowohl Vor- als auch Nachteile hätte.\\ Minuspunkte: Keine einheitliche Gravitationsausrichtung.\\ Pluspunkte: Wenigstens würde sich die Treppe nicht bewegen, WÄHREND man noch auf ihr war.

Harry war ursprünglich vier Stockwerke hochgestiegen, um zu seinem Schlafsaal zu gelangen. Nachdem er nicht weniger als zwölf Stockwerke hinuntergeklettert war, ohne auch nur in die Nähe der Kerker zu kommen, war Harry zu dem Schluss gekommen, dass\\ (1)ein Escher-Gemälde im Vergleich dazu ein Kinderspiel wäre,\\ (2) er sich irgendwie höher im Schloss befand als zu Beginn und\\ (3) er sich so gründlich verlaufen hatte, dass es ihn nicht überrascht hätte, aus dem nächsten Fenster zu schauen und zwei Monde am Himmel zu sehen.

Ersatzplan A war gewesen, anzuhalten und nach dem Weg zu fragen, aber es schien ein extremer Mangel an Leuten herumzulaufen, als ob die Schüler alle zum Unterricht gingen, wie es sich gehörte oder so.

Ersatzplan B... „Ich habe mich verlaufen“, sagte Harry laut. „Kann, ähm, der Geist des Hogwarts-Schlosses mir helfen oder so?“

„Ich glaube nicht, dass dieses Schloss einen Geist hat“, bemerkte eine verhutzelte alte Dame auf einem der Gemälde an den Wänden. „Leben vielleicht, aber keinen Geist.„

Es gab eine kurze Pause. „Sind Sie -“ sagte Harry und hielt sich dann den Mund zu. Bei näherem Nachdenken, nein, er hatte NICHT vor, das Gemälde zu fragen, ob es bei vollem Bewusstsein war, in dem Sinne, dass es sich seines eigenen Bewusstsein bewusst war.

„Ich bin Harry Potter“,\\ sagte sein Mund, mehr oder weniger auf Autopilot. Ebenfalls mehr oder weniger automatisch streckte Harry eine Hand in Richtung des Gemäldes aus. Die Frau auf dem Gemälde schaute auf Harrys Hand hinunter und hob die Augenbrauen. Langsam sank die Hand zurück an Harrys Seite.

„Entschuldigung“, sagte Harry, „ich bin irgendwie neu hier.“\\ „Das nehme ich wahr, junger Rabe. Wo willst du denn hin?“\\ Harry zögerte. „Ich bin mir nicht ganz sicher“, sagte er.\\ „Dann bist du vielleicht schon dort.“\\ „Nun, wo auch immer ich hin will, ich glaube nicht, dass es hier ist...“

Harry hielt sich den Mund zu, weil er merkte, wie sehr er sich wie ein Idiot anhörte.

„Lass mich von vorne anfangen. Ich spiele dieses Spiel, nur weiß ich nicht, wie die Regeln lauten -“

Das hat auch nicht wirklich funktioniert, oder?

„Okay, dritter Versuch. Ich suche nach Möglichkeiten, Gutes zu tun, damit ich Punkte bekomme, und alles, was ich habe, ist dieser kryptische Hinweis, dass die Dunkelheit dort ist, wo das Licht sein muss, also habe ich versucht, nach unten zu gehen, aber ich scheine stattdessen immer wieder nach oben zu gehen...“

Die alte Dame auf dem Bild schaute ihn ziemlich skeptisch an. Harry seufzte.

„Mein Leben neigt dazu, ein wenig seltsam zu werden.“

„Wäre es fair zu sagen, dass du nicht weißt, wohin du gehtst oder warum du versuchst, dorthin zu gelangen?“

„Völlig fair.“

Die alte Dame nickte. „Ich bin mir nicht sicher, ob das Verlorensein dein wichtigstes Problem ist, junger Mann.“

„Stimmt, aber im Gegensatz zu den wichtigeren Problemen ist es ein Problem, von dem ich verstehen kann, wie es zu lösen ist, und wow, diese Unterhaltung entwickelt sich zu einer Metapher für die menschliche Existenz, das habe ich bis gerade eben gar nicht bemerkt.“

Die Dame beäugte Harry abschätzend. „Du bist ein kluger junger Rabe, nicht wahr? Einen Moment lang habe ich mich schon gewundert. Nun denn, als allgemeine Regel gilt: Wenn du dich weiter nach links wendest, wirst du zwangsläufig weiter nach unten gehen.“

Das klang seltsam vertraut, aber Harry konnte sich nicht erinnern, wo er es schon einmal gehört hatte.

„Ähm... Sie scheinen ein sehr intelligenter Mensch zu sein. Oder ein Bild von einer sehr intelligenten Person... Jedenfalls, haben Sie schon mal von einem geheimnisvollen Spiel gehört, das man nur einmal spielen kann und dessen Regeln sie einem nicht verraten wollen?“

„Das Leben“, sagte die Dame sofort. „Das ist eines der offensichtlichsten Rätsel, die ich je gehört habe.“

Harry blinzelte. „Nein“, sagte er langsam. „Ich meine, ich habe einen Zettel bekommen und alles, auf dem steht, dass ich das Spiel spielen muss, aber die Regeln werden mir nicht gesagt, und jemand hinterlässt mir kleine Zettel, auf denen steht, wie viele Punkte ich für Regelverstöße verliere, etwa minus zwei Punkte für das Tragen eines Pyjamas. Kennst du jemanden hier in Hogwarts, der verrückt genug und mächtig genug ist, so etwas zu tun? Außer Dumbledore, meine ich?“

Das Bild einer Dame seufzte. „Ich bin nur ein Bild, junger Mann. Ich erinnere mich an Hogwarts, wie es war - nicht an Hogwarts, wie es ist. Alles, was ich dir sagen kann, ist, dass, wenn dies ein Rätsel wäre, die Antwort lauten würde, dass das Spiel das Leben ist, und dass wir zwar nicht alle Regeln selbst machen, aber derjenige, der Punkte vergibt oder nimmt, bist immer du. Wenn es kein Rätsel ist, sondern Realität - dann weiß ich es nicht.“

Harry verbeugte sich tief vor dem Bild. „Vielen Dank, Mylady.“

Die Dame verbeugte sich vor ihm. „Ich wünschte, ich könnte sagen, dass ich mich gern an dich erinnern werde“, sagte sie, „aber wahrscheinlich werde ich mich überhaupt nicht an dich erinnern. Lebe wohl, Harry Potter.“

Er verbeugte sich noch einmal als Antwort und begann, die nächste Treppe hinunterzusteigen. Vier Linkskurven später fand er sich in einem Korridor wieder, der abrupt in einem umgestürzten Haufen großer Steine endete - als hätte es einen Einsturz gegeben, nur die umliegenden Wände und die Decke waren intakt und bestanden aus ganz normalen Burgsteinen.

„In Ordnung“, sagte Harry in die leere Luft, „ich gebe auf. Ich bitte um einen weiteren Hinweis. Wie komme ich dahin, wo ich hinmuss?“

„Ein Tipp! Ein Hinweis, sagst du?“ Die aufgeregte Stimme kam von einem Gemälde an der nicht weit entfernten Wand, diesmal ein Porträt eines Mannes mittleren Alters in den knalligsten rosa Roben, die Harry je gesehen oder sich auch nur vorgestellt hatte. Auf dem Porträt trug er einen schlaffen alten Spitzhut mit einem Fisch darauf (keine Zeichnung eines Fisches, wohlgemerkt, aber ein Fisch).

„Ja!“ sagte Harry. „Eine Andeutung! Ein Hinweis, sage ich! Nur nicht irgendeinen Hinweis, ich suche einen ganz bestimmten Hinweis, für ein Spiel, das ich spiele -"

„Ja, ja! Einen Hinweis für das Spiel! Du bist Harry Potter, nicht wahr? Ich bin Cornelion Flubberwalt! Ich habe es von Erin, dem Gemahl, erfahren, der es von Lord Weaselnose erfahren hat, der es von, ich weiß nicht mehr genau. Aber es war eine Nachricht für mich, die ich dir geben sollte! Für mich! Niemand hat sich um mich gekümmert, seit, ich weiß nicht wie lange, vielleicht schon immer, ich hier unten in diesem verdammt nutzlosen alten Korridor festsitze - ein Hinweis! Ich habe deinen Hinweis. Er kostet dich nur drei Punkte. Willst du ihn?“

„Ja! Ich will ihn!“ Harry war sich bewusst, dass er seinen Sarkasmus wahrscheinlich unter Kontrolle halten sollte, aber er konnte sich einfach nicht zurückhalten.

„Die Dunkelheit befindet sich zwischen den grünen Lernräumen und McGonagalls Verwandlungsklasse! Das ist der Hinweis! Und beeil dich, du bist langsamer als ein Sack Schnecken! Minus 10 Punkte für deine Langsamkeit! Jetzt hast du 61 Punkte! Das war der Rest der Nachricht!“

„Danke“, sagte Harry. Er war hier wirklich im Rückstand. „Ähm... ich nehme nicht an, dass Sie wissen, woher die Nachricht ursprünglich kam, oder?“

„Sie wurde von einer hohlen Stimme gesprochen, die aus einem Spalt in der Luft selbst hervorbrach, einem Spalt, der sich zu einem feurigen Abgrund öffnete! Das hat man mir gesagt!“

Harry war sich an diesem Punkt nicht mehr sicher, ob er skeptisch sein sollte oder ob er die Sache einfach so hinnehmen sollte.

„Und wie finde ich den Weg zwischen den grünen Lernräumen und der Verwandlungsklasse?“

„Dreh dich einfach um und geh nach links, rechts, unten, unten, rechts, links, rechts, oben und wieder links, dann bist du beim grünen Studierzimmer, und wenn du hineingehst und auf der gegenüberliegenden Seite geradeaus gehst, befindest du dich auf einem großen, kurvigen Korridor, der zu einer Kreuzung führt, und auf der rechten Seite dieser Kreuzung befindet sich ein langer, gerader Gang, der zum Verwandlungsklassenzimmer führt!“

Die Gestalt des Mannes mittleren Alters hielt inne. „Zumindest war es so, als ich in Hogwarts war. Heute ist ein Montag in einem ungeraden Jahr, nicht wahr?“

„Bleistift und Millimeterpapier“, sagte Harry zu seinem Beutel. „Äh, streichen Sie das, Papier und Druckbleistift.“ Er blickte auf. „Könnten Sie das wiederholen?„

Nachdem er sich noch zwei weitere Male verirrt hatte, hatte Harry das Gefühl, dass er langsam die Grundregel für die Navigation durch das sich ständig verändernde Labyrinth, das Hogwarts war, verstand, nämlich ein Gemälde nach dem Weg zu fragen.\\ Wenn dies eine Art von unglaublich tiefgründiger Lebenslektion widerspiegelte, konnte er nicht herausfinden, was es war.

Der grüne Studienraum war ein überraschend angenehmer Raum mit Sonnenlicht, das durch Fenster aus grün gefärbtem Glas hereinströmte, die Drachen in ruhigen, pastoralen Szenen zeigten. Es gab Stühle, die äußerst bequem aussahen, und Tische, die sehr gut geeignet schienen, um in Gesellschaft von ein bis drei Freunden zu lernen. Harry konnte eigentlich nicht gerade hindurchgehen und auf der anderen Seite durch die Tür hinausgehen. Es gab Bücherregale, die in die Wand eingelassen waren, und er musste hinübergehen und einige der Titel lesen, um seinen Anspruch auf den Familiennamen Verres nicht zu verlieren. Aber er tat es schnell, eingedenk der Beschwerde über seine Langsamkeit, und ging dann auf der anderen Seite hinaus. Er ging gerade den „großen kurvigen Korridor“ hinunter, als er die Stimme eines kleinen Jungen rufen hörte.

In solchen Momenten hatte Harry eine Ausrede, um ohne Rücksicht auf Energie zu sparen oder richtige Aufwärmübungen zu machen oder sich Gedanken darüber zu machen, mit Dingen zusammenzustoßen, einen plötzlichen rasenden Flug, der fast ebenso plötzlich zum Stillstand kam, als er fast eine Gruppe von sechs Hufflepuffs aus dem ersten Jahr überfuhr...\\ ...die zusammengekauert waren und ziemlich verängstigt aussahen und als ob sie verzweifelt etwas tun wollten, aber nicht wussten, was, was wahrscheinlich etwas mit der Gruppe von fünf älteren Slytherins zu tun hatte, die einen anderen Jungen zu umringen schienen. Harry war plötzlich ziemlich wütend.

„Entschuldigung!“, schrie Harry aus vollem Halse.

Das wäre vielleicht nicht nötig gewesen. Die Leute sahen ihn bereits an. Aber es diente auf jeden Fall dazu, die ganze Aktion kalt zu stoppen. Harry ging an der Traube von Hufflepuffs vorbei auf die Slytherins zu. Sie sahen auf ihn herab mit Ausdrücken, die von Ärger über Belustigung bis hin zu Freude reichten. Ein Teil von Harrys Gehirn schrie in Panik, dass dies viel ältere und größere Jungen waren, die ihn platt treten konnten. Ein anderer Teil sagte trocken, dass jeder, der dabei erwischt wurde, wie er den Jungen-der-lebte ernsthaft niedertrampelte, in einer ganzen Welt von Schwierigkeiten steckte, besonders wenn es ein Rudel älterer Slytherins war und es sieben Hufflepuffs gab, die es sahen, und dass die Chance, dass sie ihm in Anwesenheit von Zeugen irgendeinen dauerhaften Schaden zufügten, fast null war.\\ Die einzige wirkliche Waffe, die die älteren Jungen gegen ihn hatten, war seine eigene Angst, wenn er das zuließ.

Dann sah Harry, dass der Junge, den sie gefangen hatten, Neville Longbottom war. Ja, natürlich. Damit war die Sache erledigt. Harry hatte beschlossen, sich unterwürfig bei Neville zu entschuldigen, und das bedeutete, \emph{dass Neville ihm gehörte, wie konnten sie es wagen?}\\ Harry streckte die Hand aus, packte Neville am Handgelenk und zog ihn zwischen den Slytherins hervor. Der Junge stolperte vor Schreck, als Harry ihn herauszog und sich fast mit der gleichen Bewegung selbst durch die gleiche Lücke schob.

Und Harry stand in der Mitte der Slytherins, wo Neville gestanden hatte, und blickte zu den viel älteren, größeren und stärkeren Jungen auf.\\ „Hallo“, sagte Harry. „Ich bin der Junge-der-lebte.“

Es gab eine ziemlich unangenehme Pause. Niemand schien zu wissen, wohin das Gespräch von da an gehen sollte. Harrys Blick fiel nach unten und sah einige Bücher und Papiere, die auf dem Boden verstreut lagen. Oh, das alte Spiel, bei dem man den Jungen versuchen lässt, seine Bücher aufzuheben und sie ihm dann wieder aus der Hand schlägt. Harry konnte sich nicht daran erinnern, jemals selbst das Objekt dieses Spiels gewesen zu sein, aber er hatte eine gute Vorstellungskraft und seine Fantasie machte ihn wütend. Nun, sobald die größere Situation geklärt war, würde es für Neville leicht genug sein, alles zurückzukommen und seine Materialien abzuholen, vorausgesetzt, die Slytherins blieben zu sehr auf ihn konzentriert, um daran zu denken, den Büchern etwas anzutun.

Leider war sein abschweifender Blick bemerkt worden.\\ „Ooh“, sagte der größte der Jungen, „wollte der kleine tietsi pietze junge.. -“\\ „Halt die Klappe“, sagte Harry kalt.

\emph{Bring sie aus dem Gleichgewicht. Tu nicht, was sie erwarten. Fall nicht in ein Muster, das sie auffordert, dich zu schikanieren.}

„Ist das Teil eines unglaublich cleveren Plans, der dir in Zukunft Vorteile verschaffen wird, oder ist es eine ebenso sinnlose Schande für den Namen Salazar Slytherin, wie es -“

Der größte Junge schubste Harry Potter hart, und er schleuderte aus dem Kreis der Slytherins auf den harten Steinboden von Hogwarts. Und die Slytherins lachten.

Harry erhob sich in etwas, das ihm wie eine schreckliche Zeitlupe vorkam. Er wusste noch nicht, wie er seinen Zauberstab benutzen sollte, aber es gab keinen Grund, sich unter diesen Umständen davon aufhalten zu lassen.

„Ich möchte so viele Punkte zahlen, wie es nötig ist, um diese Person loszuwerden“, sagte Harry und deutete mit dem Finger auf den größten Slytherin.

Dann hob Harry seine andere Hand, sagte \emph{„Abrakadabra“} und schnippte mit den Fingern.

Bei dem Wort Abrakadabra schrien zwei der Hufflepuffs, darunter Neville, drei andere Slytherins sprangen verzweifelt aus dem Weg von Harrys Finger und der größte Slytherin taumelte mit einem Ausdruck des Schocks zurück, ein plötzlicher Spritzer Rot zierte sein Gesicht und seinen Hals und seine Brust.\\ Damit hatte Harry nicht gerechnet.

Langsam griff der größte Slytherin nach seinem Kopf und schälte die Reste des Kirschkuchens ab, der gerade über ihm materialisiert war. Der größte Slytherin hielt die Kuchen-Pfanne einen Moment lang in der Hand und starrte sie an, dann ließ er sie auf den Boden fallen.

Es war wahrscheinlich nicht der beste Zeitpunkt auf der Welt, um zu Lachen, aber genau das tat einer der Hufflepuffs. Dann erblickte Harry den Zettel auf dem Boden der Pfanne.

„Warte mal“, sagte Harry und stürzte nach vorne, um den Zettel aufzuheben. „Dieser Zettel ist für mich, glaube ich-“

„Du“, knurrte der größte Slytherin, „du, wirst, zu -“

„Sieh dir das an!“, rief Harry und hielt dem älteren Slytherin den Zettel hin. „Ich meine, sieh dir das einfach an! Kannst du glauben, dass mir 30 Punkte für Versand und Bearbeitung für einen lausigen Kuchen berechnet werden? 30 Punkte! Ich mache einen Verlust bei dem Geschäft, obwohl ich einen unschuldigen Jungen in Not gerettet habe! Und \emph{Lagergebühren}? \emph{Überführungskosten}? \textbf{\emph{Transportkosten}}? Wie kommt man auf Transportkosten bei einem Kuchen?!“

Wieder gab es eine dieser peinlichen Pausen. Harry machte sich tödliche Gedanken, welcher Hufflepuff auch immer nicht aufhören konnte zu kichern, dieser Idiot würde ihm noch wehtun. Harry trat zurück und schoss den Slytherins seinen besten tödlichen Blick zu.

„Und jetzt verschwindet, oder ich werde eure Existenz immer unwirklicher machen, bis ihr es tut. Lass mich dich warnen... wenn du dich in mein Leben einmischst, wird dein Leben... ein bisschen haarig. Verstehst du das?“

In einer einzigen schrecklichen Bewegung holte der größte Slytherin mit seinem Zauberstab aus, um auf Harry zu zielen, und wurde im selben Moment von einem weiteren Kuchen an der anderen Seite seines Kopfes getroffen, diesmal von einem hellen Heidelbeerkuchen.

\emph{Der Zettel auf diesem Kuchen war ziemlich groß und deutlich zu lesen.}

„Du solltest vielleicht den Zettel auf diesem Kuchen lesen“, bemerkte Harry. „Ich glaube, er ist diesmal für dich.“

Der Slytherin griff langsam nach oben, nahm die Kuchenform, drehte sie mit einem feuchten Glibbergeräusch um, was noch mehr Blaubeeren auf den Boden fallen ließ, und las einen Zettel, auf dem stand:

\emph{WARNUNG KEINE MAGIE DARF AUF DEN WETTKÄMPFER ANGEWENDET WERDEN, WÄHREND DAS SPIEL LÄUFT}\\ \emph{WEITERES EINGREIFEN IN DAS SPIEL WIRD DEN SPIELBEHÖRDEN GEMELDET}

Der Ausdruck purer Verblüffung auf dem Gesicht des Slytherins war ein Kunstblick. Harry dachte, dass er vielleicht anfing, diesen Spielleiter zu mögen.

„Hört mal“, sagte Harry, „wollen wir Schluss machen? Ich glaube, die Dinge laufen hier aus dem Ruder. Wie wäre es, wenn du zurück nach Slytherin gehst und ich zurück nach Ravenclaw und wir uns alle ein bisschen abkühlen, okay?“

„Ich habe eine bessere Idee“, zischte der größte Slytherin. „Wie wäre es, wenn du dir aus Versehen alle Finger brichst?“

„Wie in Merlins Namen willst du einen glaubhaften Unfall inszenieren, nachdem du die Drohung vor einem Dutzend Zeugen ausgesprochen hast, du Idiot -“

Der größte Slytherin griff langsam und bedächtig nach Harrys Händen, und Harry erstarrte an Ort und Stelle, der Teil seines Gehirns, der das Alter und die Stärke des anderen Jungen bemerkte, schaffte es endlich, sich Gehör zu verschaffen und schrie:

\emph{WAS ZUM Henker tue ich da?}

„Warte!“, sagte einer der anderen Slytherins, seine Stimme plötzlich panisch. „Stopp, das solltest du eigentlich nicht tun!“

Der größte Slytherin ignorierte ihn und nahm Harrys rechte Hand fest in seine linke Hand und nahm Harrys Zeigefinger in seine rechte Hand. Harry starrte dem Slytherin direkt in die Augen. Ein Teil von Harry schrie, das sollte nicht passieren, das durfte nicht passieren, Erwachsene würden so etwas nie zulassen -

Langsam begann der Slytherin, seinen Zeigefinger nach hinten zu biegen.

\emph{Er hat mir nicht wirklich den Finger gebrochen und es ist unter meiner Würde, auch nur mit der Wimper zu zucken, bis er es tut. Bis dahin ist das nur ein weiterer Versuch, Angst zu machen.}

„Stopp!“, sagte der Slytherin, der sich zuvor beschwert hatte. „Stopp, das ist eine sehr schlechte Idee!“

„Da stimme ich eher zu“, sagte eine eisige Stimme. Die Stimme einer älteren Frau.

Der größte Slytherin ließ Harrys Hand los und sprang rückwärts, als ob er verbrannt wäre. „\textbf{Professor Sprout!"}, rief einer der Hufflepuffs und klang dabei so froh wie niemand sonst, den Harry je in seinem Leben gehört hatte.

In Harrys Blickfeld, als er sich umdrehte, stakste eine plumpe kleine Frau mit unordentlich gelockten grauen Haaren und schmutzverschmierter Kleidung. Sie zeigte mit einem anklagenden Finger auf die Slytherins.

„Erklärt euch“, sagte sie. „Was macht ihr mit meinen Hufflepuffs und...“,\\ sie sah ihn an.\\ „Meinem guten Schüler, Harry Potter.“

\emph{Uh oh. Stimmt, es war IHR Unterricht, den ich heute Morgen verpasst hatte.}

„Er hat gedroht, uns umzubringen!“, platzte einer der anderen Slytherins heraus, derselbe, der eine Pause gefordert hatte.

„Was?“ sagte Harry ausdruckslos. „Das habe ich nicht! Wenn ich vorhätte, euch zu töten, würde ich nicht erst öffentlich drohen!“

Ein dritter Slytherin lachte hilflos auf und blieb dann abrupt stehen, als die anderen Jungen ihm tödliche Blicke zuwarfen. Professor Sprout hatte eine eher skeptische Miene angenommen.

„Was für eine Todesdrohung soll das genau sein?“

„Der Tötungsfluch! Er hat so getan, als würde er den Tötungsfluch gegen uns anwenden!“

Professor Sprout drehte sich um und sah Harry an. „Ja, eine ziemlich schreckliche Drohung von einem elfjährigen Jungen. Trotzdem solltest du nicht im Traum daran denken, so zu tun, als ob, Harry Potter.“

„Ich kenne nicht einmal die Worte für den Tötungsfluch“, sagte Harry prompt. „Und ich hatte zu keiner Zeit meinen Zauberstab in der Hand.“

Nun warf Professor Sprout Harry einen skeptischen Blick zu.

„Dann hat sich der Junge wohl selbst mit zwei Torten beworfen?“

„Er hat seinen Zauberstab nicht benutzt!“, platzte einer der jungen Hufflepuffs heraus. „Ich weiß auch nicht, wie er es gemacht hat, er hat einfach mit den Fingern geschnippt und da war Kuchen!“

„Wirklich“, sagte Professor Sprout nach einer Pause. Sie zog ihren eigenen Zauberstab. „Ich werde ihn nicht verlangen, da du hier das Opfer zu sein scheinst, aber würde es dir etwas ausmachen, wenn ich deinen Zauberstab überprüfe, um das zu bestätigen?“

Harry zückte seinen Zauberstab. „Was muss ich -“

„Prior Incantato“, sagte Sprout. Sie runzelte die Stirn. „Das ist seltsam, dein Zauberstab scheint überhaupt nicht benutzt worden zu sein.“

Harry zuckte mit den Schultern. „Nein, eigentlich nicht, ich habe meinen Zauberstab und meine Schulbücher erst vor ein paar Tagen bekommen.“

Sprout nickte. „Dann haben wir einen klaren Fall von versehentlicher Magie durch einen Jungen, der sich bedroht fühlte. Und in den Regeln steht ganz klar, dass du nicht dafür verantwortlich gemacht werden kannst. Was euch betrifft...“, wandte sie sich an die Slytherins.\\ Ihr Blick fiel absichtlich auf die Bücher von Neville, die auf dem Boden lagen. Es herrschte eine lange Stille, während der sie die fünf Slytherins ansah.

„Drei Punkte von jedem Slytherin“, sagte sie schließlich. „Und sechs von ihm“, und deutete auf den Jungen, der mit Kuchen bedeckt war.\\ „Mischt euch nie wieder bei meinen Hufflepuffs ein, auch nicht bei meinem Schüler Harry Potter. Und jetzt geht.“

Sie brauchte sich nicht zu wiederholen; die Slytherins drehten sich um und gingen sehr schnell weg. Neville ging und fing an, seine Bücher aufzusammeln. Er schien zu weinen, aber nur ein wenig. Vielleicht war es der verspätete Schock, vielleicht aber auch, weil die anderen Jungen ihm halfen.

„Vielen Dank, Harry Potter“, sagte Professor Sprout zu ihm. „Sieben Punkte für Ravenclaw, einen für jeden Hufflepuff, den du beschützt hast. Und mehr werde ich nicht sagen.“

Harry blinzelte. Er hatte eher einen Vortrag darüber erwartet, wie man sich aus Schwierigkeiten heraushält, und eine ziemlich strenge Schelte dafür, dass er seine allererste Stunde verpasst hatte.

\emph{Vielleicht hätte er nach Hufflepuff gehen sollen. Sprout war cool.}

„Inervate“, sagte Sprout zu dem Durcheinander von Kuchen auf dem Boden, der prompt verschwand. Und sie ging weg, den Flur entlang, der zum grünen Lernraum führte.

„Wie hast du das gemacht?“, zischte einer der Hufflepuff-Jungen, sobald sie weg war. Harry lächelte süffisant. „Ich kann alles machen, was ich will, indem ich nur mit den Fingern schnippe.“

Die Augen des Jungen weiteten sich. „Wirklich?“

„Nein“, sagte Harry. „Aber wenn du allen diese Geschichte erzählst, dann erzähl sie auch Hermine Granger im ersten Jahr in Ravenclaw, sie hat eine Anekdote, die du vielleicht amüsant findest.“

Er hatte absolut keine Ahnung, was hier vor sich ging, aber er wollte sich die Gelegenheit nicht entgehen lassen, seine wachsende Legende zu vervollständigen.

„Oh, und was war das mit dem Tötungsfluch?“

Der Junge warf ihm einen seltsamen Blick zu. „Du weißt es wirklich nicht?“

„Wenn ich es wüsste, würde ich nicht fragen.“

„Die Worte des Tötungsfluchs lauten“, der Junge schluckte, und seine Stimme sank zu einem Flüstern, und er hielt die Hände von den Seiten weg, als wollte er deutlich machen, dass er keinen Zauberstab in der Hand hielt, \emph{"Avada Kedavra.„}

Nun, natürlich. Harry setzte dies auf seine wachsende Liste von Dingen, die er seinem Vater, Professor Michael Verres-Evans, niemals sagen durfte. Es war schon schlimm genug, darüber zu reden, dass man die einzige Person war, die den furchterregenden Tötungsfluch überlebt hatte, ohne zugeben zu müssen, dass der Tötungsfluch „Abrakadabra“ war.

„Ich verstehe“, sagte Harry nach einer Pause. „Nun, das ist das letzte Mal, dass ich das sage, bevor ich mit den Fingern schnippe.“

Obwohl es einen Effekt erzeugt hatte, der taktisch nützlich sein könnte.

„Warum hast du -“

„Von Muggeln aufgezogen, Muggel denken dass es ein Scherz ist und dass es lustig ist. Ernsthaft, das ist es, was passiert ist. Entschuldigung, aber kannst du mir deinen Namen sagen?“

„Ich bin Ernie Macmillan“, sagte der Hufflepuff. Er streckte seine Hand aus, und Harry schüttelte sie. „Es ist mir eine Ehre, dich kennenzulernen.“

Harry machte eine leichte Verbeugung. „Erfreut, dich kennenzulernen, überspring das das Geehrte.“

Dann drängten sich die anderen Jungen um ihn und es gab eine plötzliche Flut von Vorstellungsgesprächen. Als sie fertig waren, schluckte Harry. Das würde sehr schwierig werden.

„Ähm... wenn mich alle entschuldigen würden... Ich habe Neville etwas zu sagen -“ Alle Augen richteten sich auf Neville, der einen Schritt zurücktrat, sein Gesicht sah besorgt aus.

„Ich nehme an“, sagte Neville mit leiser Stimme, „du willst sagen, ich hätte mutiger sein sollen -“

„Oh, nein, nichts dergleichen!“ sagte Harry hastig. „Das hat nichts damit zu tun. Es ist nur, ähm, etwas, das mir der Sprechende Hut gesagt hat -“

Plötzlich sahen die anderen Jungen sehr interessiert aus, bis auf Neville, der noch besorgter dreinschaute. Irgendetwas schien Harrys Kehle zu blockieren. Er wusste, dass er es einfach herausplatzen lassen sollte, aber es war, als hätte er einen großen Ziegelstein verschluckt, der einfach im Weg war. Es war, als müsste Harry die Kontrolle über seine Lippen übernehmen und jede Silbe einzeln hervorbringen, aber er schaffte es, es zu tun.

„Ich bin, äh, … es tut mir Leid.“ Er atmete aus und nahm einen tiefen Atemzug. „Für das, was ich, ähm, neulich getan habe. Du... du musst deswegen nicht gnädig sein oder so, ich verstehe, wenn du mich einfach hasst. Es geht nicht darum, dass ich versuche, cool auszusehen, indem ich mich entschuldige, oder dass du es akzeptieren musst. Was ich getan habe, war falsch.“

Es gab eine Pause. Neville drückte seine Bücher fester an seine Brust. „Warum hast du es getan?“, fragte er mit dünner, schwankender Stimme. Er blinzelte, als wolle er die Tränen zurückhalten. „Warum tut mir das jeder an, sogar der Junge, der gelebt hat?“

Harry fühlte sich plötzlich kleiner als je zuvor in seinem Leben. „Es tut mir leid“, sagte Harry wieder, seine Stimme war nun heiser. „Es ist nur... du sahst so verängstigt aus, es war wie ein Schild über deinem Kopf, auf dem '\emph{Opfer}' stand, und ich wollte dir zeigen, dass die Dinge nicht immer schlecht ausgehen, dass die Monster dir manchmal Schokolade geben... Ich dachte, wenn ich dir das zeige, merkst du vielleicht, dass es nicht so viel gibt, wovor du Angst haben musst -“

„Aber es gibt sie“, flüsterte Neville. „Du hast es heute gesehen, es gibt sie!“

„Sie würden nichts wirklich Schlimmes vor Zeugen tun. Ihre Hauptwaffe ist die Angst. Deshalb haben sie es auf dich abgesehen, weil sie sehen können, dass du Angst hast. Ich wollte dir die Angst nehmen... dir zeigen, dass die Angst schlimmer ist als die Sache selbst... zumindest habe ich mir das eingeredet, aber der Sprechende Hut hat mir gesagt, dass ich mich selbst belogen habe und dass ich es wirklich getan habe, weil es Spaß gemacht hat. Deshalb entschuldige ich mich -“

„Du hast mich verletzt“, sagte Neville. „Gerade eben. Als du mich gepackt und von ihnen weggezogen hast.“ Neville streckte seinen Arm aus und zeigte auf die Stelle, an der Harry ihn gepackt hatte. „Vielleicht habe ich hier später einen blauen Fleck, weil du so stark gezogen hast. Du hast mich schlimmer verletzt als alles, was die Slytherins getan haben, indem du mich angerempelt hast, um ehrlich zu sein."

„Neville!“, zischte Ernie. „Er hat versucht, dich zu retten!“

„Es tut mir leid“, flüsterte Harry. „Als ich das gesehen habe, wurde ich einfach... richtig wütend...“

Neville sah ihn unverwandt an. „Also hast du mich einfach rausgerissen und dich an meine Stelle gesetzt und gesagt: '\emph{Hallo, ich bin der Junge, der gelebt hat'}.“\\ Harry nickte.\\ „Ich glaube, du wirst eines Tages richtig cool sein“, sagte Neville. „Aber im Moment bist du es noch nicht.“

Harry schluckte den plötzlichen Knoten in seiner Kehle hinunter und ging weiter. Er ging den Korridor weiter hinunter bis zur Kreuzung, bog dann links in einen Gang ein und ging blindlings weiter.

\emph{Was sollte er hier tun? Niemals wütend werden?}

Er war sich nicht sicher, ob er etwas hätte tun können, ohne wütend zu werden, und wer weiß, was dann mit Neville und seinen Büchern passiert wäre. Außerdem hatte Harry genug Fantasy-Bücher gelesen, um zu wissen, wie so etwas ablief. Er würde versuchen, die Wut zu unterdrücken, und es würde ihm nicht gelingen, und sie würde immer wieder zum Vorschein kommen. Und nach dieser ganzen langen Reise der Selbstfindung würde er am Ende lernen, dass seine Wut ein Teil von ihm selbst war und dass er nur, indem er sie akzeptierte, lernen konnte, sie weise einzusetzen.

Star Wars war das einzige Universum, in dem die Antwort tatsächlich lautete, dass man sich von negativen Emotionen komplett abkapseln sollte, und irgendetwas an Yoda hatte Harry schon immer dazu gebracht, den kleinen grünen Trottel zu hassen.

Der offensichtliche zeitsparende Plan war also, die Selbstfindungsreise zu überspringen und direkt zu dem Teil überzugehen, in dem er erkannte, dass er seine Wut nur dann unter Kontrolle halten konnte, wenn er sie als Teil seiner selbst akzeptierte.

Das Problem war, dass er sich nicht außer Kontrolle fühlte, wenn er wütend war. Die kalte Wut gab ihm das Gefühl, dass er sich unter Kontrolle hatte. Nur wenn er zurückblickte, schienen die Ereignisse als Ganzes... irgendwie außer Kontrolle geraten zu sein. Er fragte sich, wie sehr sich der Spielleiter für solche Dinge interessierte und ob er dafür Punkte gewonnen oder verloren hatte. Harry selbst hatte das Gefühl, dass er ziemlich viele Punkte verloren hatte, und er war sich sicher, dass die alte Dame auf dem Bild ihm gesagt hätte, dass seine Meinung die einzige war, die zählte. Und Harry fragte sich auch, ob der Spielleiter Professor Sprout geschickt hatte. Es war der logische Gedanke: Der Zettel hatte gedroht, die Autoritäten zu benachrichtigen, und dann war da Professor Sprout.

\emph{Vielleicht war Professor Sprout der Spielleiter} - die Leiterin des Hauses Hufflepuff wäre die letzte Person, die jemand verdächtigen würde, was sie an die Spitze von Harrys Liste setzen sollte. Er hatte auch schon den einen oder anderen Kriminalroman gelesen.

„Also, wie mache ich mich im Spiel?“ sagte Harry laut.

Ein Blatt Papier flog über seinen Kopf, als hätte es jemand von hinten geworfen - Harry drehte sich um, aber es war niemand da - und als Harry sich wieder nach vorne drehte, lag der Zettel auf dem Boden. Auf dem Zettel stand:

\emph{PUNKTE FÜR STIL: 10}\\ \emph{PUNKTE FÜR GUTES DENKEN: -3.000.000}\\ \emph{RAVENCLAW-HAUS-PUNKTE-ZUSATZ: 70}\\ \emph{AKTUELLE PUNKTE: -2.999.871}\\ \emph{RESTLICHE RUNDEN: 2}

„Minus drei Millionen Punkte?“ sagte Harry entrüstet in den leeren Flur. „Das scheint mir übertrieben! Ich will bei der Spielleitung Einspruch einlegen! Und wie soll ich in den nächsten zwei Runden drei Millionen Punkte aufholen?“

Ein weiterer Zettel flog über seinen Kopf.

\emph{ANFRAGE: FEHLGESCHLAGEN}\\ \emph{STRAFE FÜR DAS FRAGEN NACH FALSCHEN ANTWORTEN: -1.000.000.000.000}\\ \emph{AKTUELLE PUNKTE: -1.000.002.999.871}\\ \emph{RESTLICHE RUNDEN: 1}

Harry gab auf. Mit nur noch einem Zug konnte er nur noch sein Bestes geben, auch wenn es nicht sehr gut war.

„Ich vermute, dass das Spiel das Leben darstellt.“

Ein letztes Blatt Papier flog über seinen Kopf, auf dem stand:

\emph{VERSUCH GESCHEITERT GESCHEITERT GESCHEITERT GESCHEITERT AKTUELLE PUNKTE: MINUS UNENDLICH}\\ \emph{DU HAST DAS SPIEL VERLOREN, LETZTE ANWEISUNG:}\\ \emph{geh zu McGonagall Büro}

Die letzte Zeile war in seiner eigenen Handschrift. Harry starrte eine Weile auf die letzte Zeile und zuckte dann mit den Schultern. Also gut. Das Büro von Professor McGonagall würde es sein. Wenn sie der Spielleiter war...

Okay, ehrlich gesagt, hatte Harry absolut keine Ahnung, wie er sich fühlen würde, wenn Professor McGonagall der Spielleiter wäre. Sein Verstand war einfach komplett leer. Es war, buchstäblich, unvorstellbar.

Ein paar Porträts später - es war kein langer Weg, Professor McGonagalls Büro war nicht weit von ihrem Klassenzimmer für Verwandlung entfernt, zumindest nicht montags in ungeraden Jahren - stand Harry vor der Tür zu ihrem Büro. Er klopfte.

„Herein“, sagte die gedämpfte Stimme von Professor McGonagall.

Er trat ein.

\emph{Anm. des Übersetzers:}\\ \emph{Falls ihr es noch nicht kennt, versucht in den Reviews ruhig das Rätsel zu lösen, bzw. eure Lösung vorzustellen. Ich sage euch wie nah ihr dran wart:-)}

