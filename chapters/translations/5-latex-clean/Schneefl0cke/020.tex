

\hypertarget{rationalisierung}{% \section{21. Rationalisierung}\label{rationalisierung}}

\textbf{\uline{Rationalisierung}}

Hermine Granger hatte sich Sorgen gemacht, dass sie böse werden könnte.

Der Unterschied zwischen Gut und Böse war normalerweise leicht zu begreifen, sie hatte nie verstanden, warum andere Leute so viel Mühe hatten.

In Hogwarts waren "\emph{Gut}" Professor Flitwick und Professor McGonagall und Professor Sprout.

"\emph{Böse}" waren Professor Snape und Professor Quirrell und Draco Malfoy.

Harry Potter… war einer dieser ungewöhnlichen Fälle, bei denen man es nicht auf den ersten Blick erkennen konnte. Sie versuchte immer noch, herauszufinden, wo er hingehörte. Aber wenn es um sie selbst ging… hatte Hermine zu viel Spaß daran, Harry Potter zu zerquetschen.

Sie war in jedem einzelnen Kurs, den sie belegt hatten, besser als er gewesen.

(Außer beim Besenreiten, das war wie Sportunterricht, das zählte nicht.)

Sie hatte fast jeden Tag in der ersten Woche echte Hauspunkte bekommen, nicht für komische Heldentaten, sondern für kluge Dinge wie schnelles Lernen von Zaubersprüchen und Hilfe für andere Schüler.

Sie wusste, dass diese Art von Hauspunkten besser war, und das Beste daran war, dass Harry Potter es auch wusste. Sie konnte es jedes Mal in seinen Augen sehen, wenn sie einen weiteren echten Hauspunkt gewann.

\emph{Wenn man Gut war, durfte man sich nicht so sehr über einen Sieg freuen.}

Es hatte am Tag der Zugfahrt angefangen, auch wenn es eine Weile gedauert hatte, bis der Wirbelwind sich gesetzt hatte. Erst später in der Nacht war Hermine klar geworden, wie sehr sie sich von diesem Jungen hatte übergehen lassen. Bevor sie Harry Potter kennengelernt hatte, hatte sie niemanden gehabt, in den sie sich hätte verknallen können.

Wenn jemand im Unterricht nicht so gut war wie sie, war es ihre Aufgabe, ihm zu helfen, nicht, es ihm unter die Nase zu reiben. Das war es, was es bedeutete, gut zu sein. Und jetzt… ..\emph{.jetzt gewann sie,} Harry Potter zuckte jedes Mal zusammen, wenn sie einen weiteren Hauspunkt bekam, und es machte so viel Spaß, ihre Eltern hatten sie vor Drogen gewarnt, und sie vermutete, dass dies mehr Spaß machte als das.

Sie hatte immer das Lächeln gemocht, das die Lehrer ihr schenkten, wenn sie etwas richtig gemacht hatte. Sie hatte es immer gemocht, die lange Reihe von Häkchen bei einem perfekt beantworteten Test zu sehen. Aber jetzt, wenn sie im Unterricht gut war, schaute sie sich beiläufig um und erhaschte einen Blick auf Harry Potter, der mit den Zähnen knirschte, und das brachte sie dazu, in ein Lied auszubrechen, wie in einem Disney-Film.

Das war schlecht, nicht wahr? Hermine hatte sich Sorgen gemacht, dass sie böse werden könnte. Und dann war ihr ein Gedanke gekommen, der alle ihre Ängste wegwischte.

\textbf{Sie und Harry hatten sich auf eine Romanze eingelassen!}

Ja, natürlich! Jeder wusste, was es bedeutete, wenn ein Junge und ein Mädchen anfingen, sich ständig zu streiten. Sie machten sich gegenseitig den Hof! Daran war nichts Schlechtes. Es konnte nicht sein, dass es ihr einfach nur Spaß machte, den berühmtesten Schüler der Schule zu verprügeln, jemanden, der in Büchern stand und wie Bücher redete, den Jungen, der irgendwie den Dunklen Lord besiegt hatte und sogar Professor Snape wie einen traurigen kleinen Käfer zerquetscht hatte, den Jungen, der, wie Professor Quirrell es ausgedrückt hätte, dominant war, über alle anderen im ersten Jahr in Ravenclaw, außer Hermine Granger, die den Jungen, der lebte, in all seinen Klassen außer dem Besenreiten völlig zerquetschte. Denn das wäre schlecht gewesen. Nein. Es war Romantik. Das war es. Das war der Grund, warum sie sich stritten.

Hermine war froh, dass sie das rechtzeitig für den heutigen Tag herausgefunden hatte, an dem Harry ihren Buchlesewettbewerb verlieren würde, von dem die ganze Schule wusste, und sie wollte vor lauter überschwänglicher Freude anfangen zu tanzen.

Es war 14:45 Uhr am Samstag und Harry Potter hatte noch die Hälfte von Bathilda Bagshots \emph{Eine Geschichte der Magie} zu lesen, und sie starrte auf ihre Taschenuhr, während sie mit furchtbarer Langsamkeit auf 14:47 Uhr zu tickte.

Und der gesamte Ravenclaw-Gemeinschaftsraum schaute zu. Es waren nicht nur die Erstklässler, die Neuigkeit hatte sich herumgesprochen wie verschüttete Milch und die Hälfte von Ravenclaw war in den Raum gedrängt, in Sofas gequetscht, an Bücherregale gelehnt und auf den Armlehnen der Stühle sitzend.

Alle sechs Vertrauensschüler waren da, einschließlich der Schulsprecherin von Hogwarts. Jemand hatte einen Lufterfrischungszauber wirken müssen, nur damit genügend Sauerstoff vorhanden war.

Der Lärm der Konversation war in ein Flüstern übergegangen, das nun in völliger Stille verhallte.

14:46 Uhr.

Die Anspannung war unerträglich. Wenn es irgendjemand anderes gewesen wäre, irgendjemand anderes, wäre seine Niederlage eine ausgemachte Sache gewesen.

Aber dies war Harry Potter, und man konnte nicht ausschließen, dass er irgendwann in den nächsten Sekunden eine Hand heben und mit den Fingern schnippen würde.

Mit plötzlichem Schrecken erkannte sie, dass Harry Potter genau das tun könnte. Es käme ihm gerade recht, wenn er die zweite Hälfte des Buches schon vorher zu Ende gelesen hätte.

.. Hermines Sicht begann zu schwimmen. Sie versuchte, sich zum Atmen zu zwingen, und stellte fest, dass sie es einfach nicht konnte.

Noch zehn Sekunden, und er hatte immer noch nicht die Hand gehoben. Noch fünf Sekunden.

14:47 Uhr.

Harry Potter legte sorgfältig ein Lesezeichen in sein Buch, klappte es zu und legte es beiseite.

"Ich möchte für die Nachwelt anmerken", sagte der Junge-der-lebte mit klarer Stimme,

"dass ich nur noch ein halbes Buch hatte und dass ich auf einige unerwartete Verzögerungen gestoßen bin -"

"Du hast dich verschätzt!", kreischte Hermine. "Das hast du! Du hast unseren Wettbewerb verloren!"

Es gab ein kollektives Ausatmen, als alle wieder zu atmen begannen.

Harry Potter schoss ihr einen flammenden Feuerblick zu, aber sie schwebte in einem Heiligenschein aus reinem weißen Glück und nichts konnte ihr etwas anhaben.

"Ist euch klar, was für eine Woche ich hatte?", sagte Harry Potter. "Jedes geringere Wesen hätte sich schwer getan, acht Dr. Seuss-Bücher zu lesen!"

"Du hast das Zeitlimit gesetzt."

Harrys feuriger Blick wurde noch heißer.

"Ich konnte nicht ahnen, dass ich die ganze Schule vor Professor Snape retten muss oder im Verteidigungsunterricht verprügelt werde, und wenn ich dir erzählen würde, wie ich die Zeit zwischen 17 Uhr und dem Abendessen am Donnerstag verloren habe, würdest du mich für verrückt halten -"

"Awww, das klingt, als wäre da jemand dem Planungsirrtum zum Opfer gefallen."

Roher Schock zeigte sich auf Harry Potters Gesicht.

"Oh, da fällt mir ein, ich habe den ersten Stapel Bücher, den du mir geliehen hast, fertig gelesen", sagte Hermine mit ihrem besten unschuldigen Blick.

Ein paar von ihnen waren auch schwere Bücher gewesen. Sie fragte sich, wie lange er gebraucht hatte, um sie zu Ende zu lesen.

"Eines Tages", sagte der Junge-der-lebte, "wenn die entfernten Nachfahren des Homo sapiens auf die Geschichte der Galaxie zurückblicken und sich fragen, wie alles so schief gehen konnte, werden sie zu dem Schluss kommen, dass der ursprüngliche Fehler darin bestand, dass jemand Hermine Granger das Lesen beigebracht hat."

"Aber du verlierst trotzdem", sagte Hermine.

Sie hielt eine Hand an ihr Kinn und sah nachdenklich aus.

"Was genau solltest du denn verlieren, frage ich mich?"

"Was?"

"Du hast die Wette verloren", erklärte Hermine, "also musst du eine Aufwandsentschädigung zahlen."

"Ich kann mich nicht erinnern, dem zugestimmt zu haben!"

"Wirklich?", sagte Hermine Granger. Sie setzte einen nachdenklichen Gesichtsausdruck auf. Dann, als ob ihr die Idee erst jetzt gekommen wäre: "Dann stimmen wir ab.

Jeder in Ravenclaw, der der Meinung ist, dass Harry Potter zahlen muss, hebt die Hand!"

"Was?", kreischte Harry Potter erneut.

Er drehte sich um und sah, dass er von einem Meer von erhobenen Händen umgeben war. \emph{Und wenn Harry Potter genauer hingesehen hätte, wäre ihm aufgefallen, dass eine ganze Menge der Zuschauer Mädchen zu sein schienen und dass praktisch jedes weibliche Wesen im Raum die Hand hob.}

"Stopp!", jammerte Harry Potter.

"Ihr wisst nicht, was sie fragen wird! Begreifst ihr nicht, was sie vorhat? Sie bringt euch jetzt dazu, eine Vorabzusage zu machen, und dann wird der Druck der Gruppe euch dazu bringen, dem zuzustimmen, was sie danach sagt!"

"Keine Sorge", sagte Penelope Clearwater.

"Wenn sie etwas Unvernünftiges verlangt, können wir es uns einfach anders überlegen. Stimmt's, Leute?"

Und es gab eifriges Nicken von allen Mädchen, denen Penelope Clearwater von Hermines Plan erzählt hatte.

Eine schweigsame Gestalt schlich leise durch die kühlen Hallen der Hogwarts-Verliese. Er sollte sich um 18 Uhr in einem bestimmten Raum einfinden, um eine bestimmte Person zu treffen, und wenn es irgend möglich war, war es am besten, früh zu sein, um Respekt zu zeigen.

Doch als seine Hand den Türknauf umdrehte und die Tür in dieses dunkle, stille, unbenutzte Klassenzimmer öffnete, stand dort bereits eine Silhouette inmitten der Reihen staubiger, alter Schreibtische.

Eine Silhouette, die einen kleinen grün leuchtenden Stab in der Hand hielt, der ein fahles Licht ausstrahlte, das selbst den, der ihn hielt, kaum erhellte, geschweige denn den umliegenden Raum.

Das Licht des Flurs erlosch, als sich die Tür hinter ihm schloss, und Dracos Augen begannen sich an das schwache Licht zu gewöhnen.

Die Silhouette drehte sich langsam zu ihm um und offenbarte ein schattenhaftes Gesicht, das nur teilweise von dem unheimlichen grünen Licht erhellt wurde.

Draco mochte diese Begegnung bereits. Behalten wir das unheimliche grüne Licht, stellen wir uns größer vor, gebe uns Kapuzen und Masken, verlege es von einem Klassenzimmer auf einen

Friedhof, und es wäre genau wie der Anfang der Hälfte der Geschichten, die die Freunde seines Vaters über die Todesser erzählten.

"Ich möchte, dass du weißt, Draco Malfoy", sagte die Silhouette in einem Tonfall tödlicher Ruhe,

"dass ich dich nicht für meine jüngste Niederlage verantwortlich mache."

Draco öffnete den Mund in gedankenlosem Protest, es gab keinen Grund, warum er beschuldigt werden sollte -

"Es lag mehr als alles andere an meiner eigenen Dummheit", fuhr die schattenhafte Gestalt fort.

"Es gab viele andere Dinge, die ich hätte tun können, bei jedem Schritt auf dem Weg. Du hast mich nicht darum gebeten, genau das zu tun, was ich getan habe.

Du hast nur um Hilfe gebeten. Ich war derjenige, der unklugerweise diese spezielle Methode gewählt hat.

Aber die Tatsache bleibt, dass ich den Wettbewerb um ein halbes Buch verloren habe. Die Aktionen deines Lieblingsidioten und der Gefallen, um den du gebeten hast, und, ja, meine eigene Dummheit dabei, haben mich Zeit verlieren lassen.

Mehr Zeit, als du ahnst. Zeit, die sich im Endeffekt als entscheidend erwiesen hat. Tatsache ist, Draco Malfoy, hättest du mich nicht um diesen Gefallen gebeten, hätte ich gewonnen.

Und nicht… stattdessen… verloren."

Draco hatte bereits von Harrys Niederlage gehört und von der Einbuße, die Granger von ihm gefordert hatte. Die Nachricht hatte sich schneller verbreitet, als Eulen sie hätten tragen können.

"Ich verstehe", sagte Draco. "Es tut mir leid."

Es gab nichts anderes, was er sagen konnte, wenn er wollte, dass Harry Potter mit ihm befreundet war.

"Ich bitte nicht um Verständnis oder Mitleid", sagte die dunkle Silhouette, immer noch mit dieser tödlichen Ruhe.

"Aber ich habe gerade zwei volle Stunden in der Gegenwart von Hermine Granger verbracht, gekleidet in die Kleidung, die mir zur Verfügung gestellt wurde, und habe so faszinierende Orte in Hogwarts besucht wie einen \emph{winzigen plätschernden Wasserfall}, der für mich wie Rotz aussah, begleitet von einer Reihe anderer Mädchen, die auf so hilfreiche Aktivitäten bestanden, wie unseren Weg mit verwandelten Rosenblättern zu bestreuen.

Ich hatte ein \textbf{\emph{Date}}, Spross der Malfoys. Mein erstes \textbf{\emph{Date}}. Und wenn ich diesen Gefallen einfordere, wirst du ihn bezahlen."

Draco nickte feierlich.

Vor seiner Ankunft hatte er die weise Vorsichtsmaßnahme getroffen, jedes verfügbare Detail von Harrys Verabredung in Erfahrung zu bringen, damit er sein hysterisches Lachen vor der verabredeten Verabredung hinter sich bringen konnte und nicht einen Fauxpas beging, indem er ununterbrochen kicherte, bis er das Bewusstsein verlor.

"Meinst du", sagte Draco, "dass dem Granger-Mädchen etwas Trauriges zustoßen sollte -"

"Verbreite in Slytherin das Wort, dass das Granger-Mädchen mir gehört und dass jeder, der sich in meine Angelegenheiten einmischt, seine Überreste über ein Gebiet verstreut bekommt, das groß genug ist, um zwölf verschiedene Sprachen zu umfassen.

Und da ich nicht in Gryffindor bin und eher auf List als auf unmittelbare Frontalangriffe setze, sollten sie nicht in Panik geraten, wenn ich dabei gesehen werde, wie ich sie anlächle."

"Oder wenn du bei einem zweiten Date gesehen wirst?" sagte Draco und ließ nur einen winzigen Ton von Skepsis in seiner Stimme zu.

\textbf{"Es wird kein zweites Date geben"}, sagte die grün beleuchtete Silhouette mit einer Stimme, die so furchterregend klang, dass sie nicht nur wie ein Todesser klang, sondern wie Amycus Carrow das eine Mal, kurz bevor Vater ihm sagte, er solle damit aufhören, er sei nicht der Dunkle Lord.

Natürlich war es immer noch die hohe, ungebrochene Stimme eines kleinen Jungen, und wenn man das mit den eigentlichen Worten kombinierte, nun, es funktionierte einfach nicht.

Sollte Harry Potter tatsächlich eines Tages der nächste Dunkle Lord werden, würde Draco in einem Denkarium eine Kopie dieser Erinnerung irgendwo sicher aufbewahren, und Harry Potter würde es niemals wagen, ihn zu verraten.

"Aber lasst uns von glücklicheren Dingen reden", sagte die grünschattige Gestalt. "Lass uns von Wissen und von Macht sprechen. Draco Malfoy, lass uns über die Wissenschaft sprechen."

"Ja", sagte Draco. "Lasst uns sprechen."

Draco fragte sich, wie viel von seinem eigenen Gesicht in diesem unheimlichen grünen Licht zu sehen war und wie viel im Schatten lag. Und obwohl Draco sein Gesicht ernst hielt, war da ein Lächeln in seinem Herzen. Endlich führte er ein richtiges Gespräch mit einem Erwachsenen.

"Ich biete dir Macht", sagte die schattenhafte Gestalt, "und ich werde dir von dieser Macht und ihrem Preis erzählen. Die Macht kommt daher, dass man die Form der Realität kennt und so die Kontrolle über sie erlangt.

Was du verstehst, kannst du beherrschen, und das ist Macht genug, um auf dem Mond zu wandeln. Der Preis für diese Macht ist, dass du lernen musst, der Natur Fragen zu stellen und, was noch viel schwieriger ist, die Antworten der Natur zu akzeptieren.

Du wirst Experimente machen, Tests durchführen und sehen, was passiert. Und du musst die Bedeutung dieser Ergebnisse akzeptieren, wenn sie dir sagen, dass du dich geirrt hast.

Du wirst lernen müssen, zu verlieren, nicht gegen mich, sondern gegen die Natur.

Wenn du dich mit der Realität streitest, wirst du die Realität gewinnen lassen müssen.

Du wirst das schmerzhaft finden, Draco Malfoy, und ich weiß nicht, ob du in dieser Hinsicht stark bist.

In Kenntnis des Preises, ist es immer noch dein Wunsch, die menschliche Macht zu erlernen?"

Draco holte tief Luft.

Er hatte darüber nachgedacht. Und es war schwer zu erkennen, wie er anders antworten konnte. Er war angewiesen worden, jeden Weg der Freundschaft mit Harry Potter zu gehen.

Es war nur zum Lernen, er hatte nicht versprochen, etwas zu tun. Er konnte den Unterricht jederzeit abbrechen… Es gab sicherlich eine ganze Reihe von Dingen an der Situation, die sie wie eine Falle aussehen ließen, aber ganz ehrlich, Draco sah nicht, wie das schief gehen konnte.

Außerdem wollte Draco irgendwie die Welt beherrschen.

"Ja", sagte Draco.

"Ausgezeichnet", sagte die schattenhafte Gestalt. "Ich hatte eine etwas überfüllte Woche, und es wird Zeit brauchen, deinen Lehrplan zu planen -"

"Ich habe selbst eine Menge zu tun, um meine Macht in Slytherin zu festigen", sagte Draco,

"von den Hausaufgaben ganz zu schweigen. Vielleicht sollten wir einfach im Oktober anfangen?"

"Klingt vernünftig", sagte die schattenhafte Gestalt, "aber was ich sagen wollte, ist, dass ich für die Planung deines Lehrplans wissen muss, was ich dir beibringen werde.

Drei Gedanken kommen mir in den Sinn. Die erste ist, dass ich dich über den menschlichen Geist und das Gehirn unterrichte.

Die zweite Möglichkeit ist, dass ich dich über das physikalische Universum unterrichte, jene Künste, die auf dem Weg zum Besuch des Mondes liegen.

Dies beinhaltet eine große Menge an Zahlen, aber für eine bestimmte Art von Geist sind diese Zahlen schöner als alles andere, was die Wissenschaft zu lehren hat. Magst du Zahlen, Draco?"

Draco schüttelte den Kopf.

"So viel dazu. Irgendwann wirst du die Mathematik lernen, aber nicht sofort, denke ich.

Die dritte Möglichkeit ist, dass ich dir etwas über Genetik und Evolution und Vererbung beibringe, was du als Blut bezeichnen würdest -"

"Das", sagte Draco.

Die Gestalt nickte.

"Ich dachte mir schon, dass du das sagen würdest. Aber ich denke, es wird der schmerzhafteste Weg für dich sein, Draco.

Was ist, wenn deine Familie und Ihre Freunde, die Blutpuristen, das eine sagen, und du feststellen musst, dass der experimentelle Test etwas anderes sagt?"

"Dann werde ich herausfinden, wie ich den experimentellen Test dazu bringe, die richtige Antwort zu sagen!"

Es gab eine Pause, als die schattenhafte Gestalt eine kurze Zeit lang mit offenem Mund dastand.

"Ähm", sagte die schemenhafte Gestalt.

"So funktioniert es eigentlich nicht. Das ist es, wovor ich dich hier warnen wollte, Draco.

Du kannst die Antwort nicht so hinbekommen, wie du willst."

"Du kannst die Antwort immer so ausfallen lassen, wie du willst", sagte Draco.

Das war praktisch das Erste gewesen, was seine Tutoren ihm beigebracht hatten.

"Es geht nur darum, die richtigen Argumente zu finden."

"Nein", sagte die schattenhafte Gestalt, und ihre Stimme erhob sich vor Frustration,

"\textbf{nein, nein, nein!}

Dann hast du die falsche Antwort, und so kannst du nicht zum Mond gehen! Die Natur ist kein Mensch, man kann sie nicht austricksen, damit sie etwas anderes glaubt, wenn du versuchst, dem Mond zu sagen, dass er aus Käse ist, kannst du tagelang argumentieren und es wird den Mond nicht ändern!

Was du meinst, ist Rationalisierung, so als ob du mit einem Blatt Papier anfängst und direkt in der untersten Zeile, mit Tinte schreibst \emph{"und deshalb ist der Mond aus Käse"}, und dann wieder nach oben gehen, um alle möglichen cleveren Argumente darüber zu schreiben.

Aber entweder ist der Mond aus Käse oder er ist es nicht. In dem Moment, in dem du die untere Zeile geschrieben hast, war sie bereits wahr oder bereits falsch.

Ob das ganze Blatt Papier mit der richtigen oder der falschen Schlussfolgerung endet, steht in dem Moment fest, in dem du die unterste Zeile aufschreibst.

Wenn du versuchst, zwischen zwei teuren Koffern zu wählen, und du magst den glänzenden, ist es egal, welche cleveren Argumente du dir für den Kauf ausdenkst, die wirkliche Regel, die du benutzt hast, um zu entscheiden, für welchen Koffer du argumentierst, war \emph{"nimm den glänzenden"}, und wie effektiv diese Regel auch ist, um gute Koffer auszuwählen, das ist die Art von Koffer, die du bekommen wirst.

Rationalität kann nicht verwendet werden, um für eine bestimmte Seite zu argumentieren, dein einziger möglicher Nutzen ist die Entscheidung, für welche Seite man argumentiert.

Wissenschaft ist nicht dazu da, jemanden davon zu überzeugen, dass die Blutpuristen recht haben. \textbf{Das ist Politik!}

Die Macht der Wissenschaft kommt daher, dass du herausfindest, wie die Natur wirklich ist, die nicht durch Argumente verändert werden kann!

Was die Wissenschaft tun kann, ist uns zu sagen, wie Blut wirklich funktioniert, wie Zauberer wirklich ihre Kräfte von ihren Eltern erben und ob Muggelgeborene wirklich schwächer oder stärker sind -"

"Stärker!?", sagte Draco. Er hatte versucht, dem zu folgen, ein verwirrtes Stirnrunzeln auf dem Gesicht, er konnte sehen, dass es irgendwie Sinn machte, aber es war ganz sicher nicht wie irgendetwas, das er jemals zuvor gehört hatte.

Und dann hatte Harry Potter etwas gesagt, das Draco unmöglich durchgehen lassen konnte.

"Du denkst, Schlammblüter sind stärker?"

"Ich glaube gar nichts", sagte die schattenhafte Gestalt.

"Ich weiß nichts. Ich glaube nichts. Mein Fazit ist noch nicht geschrieben. Ich werde herausfinden, wie ich die magische Stärke von Muggelgeborenen und die magische Stärke von Reinblütern testen kann.

Wenn meine Tests mir sagen, dass Muggelgeborene schwächer sind, werde ich glauben, dass sie schwächer sind.

Wenn meine Tests mir sagen, dass Muggelgeborene stärker sind, werde ich glauben, dass sie stärker sind.

Wenn ich diese und andere Wahrheiten kenne, werde ich ein gewisses Maß an Macht erlangen -"

"Und du erwartest von mir, dass ich alles glaube, was du sagst?" forderte Draco scharf.

"Ich erwarte, dass du die Tests persönlich durchführst", sagte die schattenhafte Gestalt leise.

"Hast du Angst vor dem, was du finden wirst?"

Draco starrte die schattenhafte Gestalt eine Weile an, seine Augen verengten sich.

"Nette Falle, Harry", sagte er. "Die muss ich mir merken, sie ist neu."

Die schattenhafte Gestalt schüttelte den Kopf.

"Es ist keine Falle, Draco. Denk daran - ich weiß nicht, was wir finden werden. Aber du verstehst das Universum nicht, indem du mit ihm streitest oder ihm sagst, es solle beim nächsten Mal mit einer anderen Antwort zurückkommen.

Wenn du das Gewand eines Wissenschaftlers anziehst, musst du all deine Politik und Argumente und Fraktionen und Seiten vergessen, das verzweifelte Klammern deines Verstandes zum Schweigen bringen und nur die Antwort der Natur hören wollen."

Die schattenhafte Gestalt hielt inne.

"Die meisten Menschen können das nicht tun. Deshalb ist es so schwierig. Bist du sicher, dass du nicht lieber nur etwas über das Gehirn lernen willst?"

"Und wenn ich dir sage, dass ich lieber etwas über das Gehirn lernen möchte", sagte Draco, seine Stimme war jetzt hart, "wirst du herumgehen und den Leuten erzählen, dass ich Angst vor dem hatte, was ich finden würde."

"Nein", sagte die schattenhafte Gestalt. "So etwas werde ich nicht tun."

"Aber du könntest selbst solche Tests machen, und wenn du die falsche Antwort bekommst, wäre ich nicht da, um etwas zu sagen, bevor du es jemand anderem gezeigt hast." Dracos Stimme war immer noch hart.

"Ich würde dich trotzdem zuerst fragen, Draco", sagte die schattenhafte Gestalt leise.

Draco hielt inne. Damit hatte er nicht gerechnet, er hatte gedacht, er hätte die Falle gesehen, aber…

"Du würdest?"

"Natürlich. Woher sollte ich wissen, wen ich erpressen könnte oder was wir von ihnen verlangen könnten? Draco, ich sage noch mal, dass das keine Falle ist, die ich dir gestellt habe.

Zumindest nicht für dich persönlich. Wäre deine Politik eine andere, würde ich sagen, was, wenn der Test zeigt, dass Reinblüter stärker sind."

"Wirklich."

"Ja! Das ist der Preis, den man zahlen muss, um Wissenschaftler zu werden!"

Draco hob eine Hand. Er musste nachdenken.

Die schattenhafte, grün beleuchtete Gestalt wartete. Es dauerte aber nicht lange, bis er nachdachte.

Wenn man all die verwirrenden Teile wegließ… dann hatte Harry Potter vor, an etwas herumzupfuschen, das eine gigantische politische Explosion auslösen konnte, und es wäre verrückt, einfach wegzugehen und ihn das alleine machen zu lassen.

"Wir werden Blut untersuchen", sagte Draco.

"Ausgezeichnet", sagte die Gestalt und lächelte. "Glückwunsch, dass du bereit bist, die Frage zu stellen."

"Danke", sagte Draco und schaffte es nicht ganz, die Ironie aus seiner Stimme zu halten.

"Hey, hast du geglaubt, zum Mond zu gehen sei einfach? Sei froh, dass es nur darum geht, deine Meinung manchmal zu ändern, und nicht um ein Menschenopfer!"

"Ein Menschenopfer wäre viel einfacher!"

Es gab eine kleine Pause, und dann nickte die Gestalt. "Gutes Argument."

"Sieh mal, Harry", sagte Draco ohne große Hoffnung,

"ich dachte, die Idee wäre, all die Dinge zu nehmen, die Muggel wissen, sie mit Dingen zu kombinieren, die Zauberer wissen, und Meister beider Welten zu werden.

Wäre es nicht viel einfacher, einfach all die Dinge zu studieren, die Muggel bereits herausgefunden haben, wie das mit dem Mond, und diese Macht zu nutzen -"

"Nein", sagte die Gestalt mit einem scharfen Kopfschütteln, wobei sich grüne Schatten um seine Nase und Augen bewegten.

Seine Stimme war sehr grimmig geworden.

"Wenn du die Kunst des Wissenschaftlers, die Realität zu akzeptieren, nicht erlernen kannst, dann darf ich dir nicht sagen, was diese Akzeptanz entdeckt hat.

Das wäre, als würde dir ein mächtiger Zauberer von den Toren erzählen, die nicht geöffnet werden dürfen, und von den Siegeln, die nicht gebrochen werden dürfen, bevor du deine Intelligenz und Disziplin bewiesen hast, indem du die kleineren Gefahren überlebt hast."

Ein Schauer lief Draco über den Rücken und er erschauderte unwillkürlich. Er wusste, dass es selbst im schwachen Licht sichtbar gewesen war.

"In Ordnung", sagte Draco. "Ich verstehe."

Das hatte Vater ihm schon oft gesagt. Wenn ein mächtigerer Zauberer einem sagte, dass man noch nicht bereit war, es zu wissen, fragte man nicht weiter nach, wenn man leben wollte.

Die Gestalt legte den Kopf schief.

"In der Tat. Aber es gibt noch etwas, das du verstehen solltest.

Den ersten Wissenschaftlern, die Muggel waren, fehlten eure Zauberer-Traditionen. Am Anfang verstanden sie den Begriff des gefährlichen Wissens einfach nicht und dachten, dass man alles, was man weiß, frei aussprechen sollte.

Als ihre Forschungen gefährlich wurden, erzählten sie ihren Politikern von Dingen, die hätten geheim bleiben sollen -

schau nicht so, Draco, es war keine einfache Dummheit. Sie mussten schon klug genug sein, um das Geheimnis überhaupt zu lüften. Aber sie waren Muggel, es war das erste Mal, dass sie etwas wirklich Gefährliches gefunden hatten, und sie hatten nicht mit einer Tradition der Geheimhaltung begonnen. Es herrschte Krieg, und die Wissenschaftler auf der einen Seite hatten Angst, dass, wenn sie nicht reden würden, die Wissenschaftler des feindlichen Landes es ihren Politikern zuerst erzählen würden…"

Die Stimme verstummte deutlich.

"\emph{Sie haben die Welt nicht zerstört. Aber es war knapp}.

Und wir werden diesen Fehler nicht wiederholen."

"Richtig", sagte Draco, seine Stimme war jetzt sehr fest. "Das werden wir nicht. Wir sind Zauberer, und das Studium der Wissenschaft macht uns nicht zu Muggeln."

"Genau", sagte die grün beleuchtete Silhouette. "Wir werden unsere eigene Wissenschaft gründen, eine magische Wissenschaft, und diese Wissenschaft wird von Anfang an klügere Traditionen haben."

Die Stimme wurde hart.

"Das Wissen, das ich mit dir teile, wird zusammen mit den Disziplinen der Wahrheitsannahme gelehrt werden, der Grad dieses Wissens wird an deinen Fortschritt in diesen Disziplinen gekoppelt sein, und du wirst dieses Wissen mit niemandem teilen, der diese Disziplinen nicht gelernt hat.

Akzeptierst du das?"

"Ja", sagte Draco. Was hätte er tun sollen, nein sagen?

"Gut. Und was du für dich selbst entdeckst, wirst du für dich behalten, es sei denn, du denkst, dass andere Wissenschaftler bereit sind, es zu erfahren.

Was wir untereinander austauschen, werden wir der Welt nicht erzählen, es sei denn, wir sind uns einig, dass es für die Welt sicher ist, es zu wissen.

Und was auch immer unsere eigene Politik und Loyalität sein mag, wir alle werden jeden von uns bestrafen, der gefährliche Magie offenbart oder gefährliche Waffen weitergibt, ganz gleich, was für ein Krieg gerade stattfindet.

Von diesem Tag an wird das die Tradition und das Gesetz der Wissenschaft unter Zauberern sein. Sind wir uns da einig?"

"Ja", sagte Draco.

Eigentlich fing das an, ziemlich attraktiv zu klingen. Die Todesser hatten versucht, die Macht zu übernehmen, indem sie furchteinflößender waren als alle anderen, und sie hatten bisher nicht wirklich gewonnen.

Vielleicht war es an der Zeit, stattdessen zu versuchen, mit Geheimnissen zu regieren.

"Und unsere Gruppe bleibt so lange wie möglich im Verborgenen, und jeder in ihr muss unseren Regeln zustimmen."

"Natürlich. Auf jeden Fall."

Es gab eine sehr kurze Pause.

"Wir werden bessere Roben brauchen", sagte die schattenhafte Gestalt, "mit Kapuzen und so weiter -"

"Das habe ich gerade gedacht", sagte Draco. "Wir brauchen aber keine ganz neuen Roben, nur Umhänge mit Kapuzen zum Überziehen. Ich habe eine Freundin in Slytherin, sie wird für dich Maß nehmen -"

"Sag ihr aber nicht, wofür sie ist -"

"Ich bin doch nicht blöd!"

"Und keine Masken für jetzt, nicht wenn wir beide allein sind -" sagte die schattenhafte Gestalt.

"Richtig! Aber später sollten wir eine Art spezielles Zeichen haben, das alle unsere Diener haben, das Zeichen der Wissenschaft, wie eine Schlange, die den Mond frisst, auf ihrem rechten Arm -"

"Das nennt man einen Doktortitel, und würde es das nicht zu einfach machen, unsere Leute zu identifizieren?"

"Hm?"

"Ich meine, was ist, wenn jemand sagt: \emph{'Okay, jetzt ziehen alle ihre Roben über den rechten Arm hoch}', und unser Typ sagt: \emph{'Ups, sorry, sieht aus, als wäre ich ein Spion'} -"

"Vergiss, dass ich was gesagt habe", sagte Draco, dem plötzlich der Schweiß am ganzen Körper ausbrach. Er brauchte eine Ablenkung, schnell -

"Und wie nennen wir uns? Die Wissens-Esser?"

"Nein", sagte die schattenhafte Gestalt langsam. "Das klingt nicht richtig…"

Draco wischte sich mit dem Arm in seinem Gewand über die Stirn und wischte die feuchten Perlen weg. \emph{Was hatte sich der Dunkle Lord nur dabei gedacht? Vater hatte gesagt, der Dunkle Lord sei klug!}

"Ich hab's!", sagte die schattenhafte Gestalt plötzlich. "Du wirst es noch nicht verstehen, aber vertrau mir, es passt."

Im Moment hätte Draco \emph{'Malfoy Fresser'} akzeptiert, solange es das Thema wechselte.

"Was ist es?"

Und inmitten der staubigen Tische in einem unbenutzten Klassenzimmer in den Kerkern von Hogwarts stand die grün beleuchtete Silhouette von Harry Potter, breitete dramatisch die Arme aus und sagte:

"Dieser Tag soll den Beginn der … \emph{Bayes-Verschwörung} markieren."

Eine stumme Gestalt stapfte müde durch die Hallen von Hogwarts in Richtung Ravenclaw.

Harry war von der Besprechung mit Draco direkt zum Abendessen gegangen und blieb beim Abendessen gerade lange genug, um ein paar schnelle Schlucke hinunterzuwürgen, bevor er sich ins Bett begab.

Es war noch nicht einmal 19 Uhr, aber für Harry war es schon längst Schlafenszeit. Ihm war gestern Abend klar geworden, dass er den Zeitdreher am Samstag erst benutzen konnte, wenn der Buchlesewettbewerb bereits vorbei war.

Aber er konnte den Zeitdreher immer noch am Freitagabend benutzen und auf diese Weise Zeit gewinnen.

Also zwang sich Harry, am Freitag bis 21 Uhr wach zu bleiben, als sich die Schutzhülle öffnete, und nutzte dann die verbleibenden vier Stunden auf dem Time-Turner, um bis 17 Uhr zurückzuspulen und in den Schlaf zu fallen.

Er war am Samstagmorgen gegen 2 Uhr aufgewacht, genau wie geplant, und hatte die nächsten zwölf Stunden durchgehend gelesen … und es war immer noch nicht genug gewesen. Und nun würde Harry in den nächsten Tagen ziemlich früh schlafen gehen, bis sein Schlafzyklus wieder aufgeholt hatte.

Das Porträt an der Tür stellte Harry irgendein dummes Rätsel für Elfjährige, das er beantwortete, ohne dass ihm die Worte überhaupt durch den Kopf gingen, und dann taumelte Harry die Treppe zu seinem Schlafsaal hinauf, zog sich seinen Pyjama an und fiel ins Bett.

Und stellte fest, dass sein Kopfkissen ziemlich klumpig zu sein schien. Harry stöhnte auf. Er setzte sich widerwillig auf, drehte sich im Bett und hob sein Kissen an.

Zum Vorschein kamen ein Zettel, zwei goldene Galleonen und ein Buch mit dem Titel \emph{Okklumentik, die versteckte Kunst} . Harry hob den Zettel auf und las:

\emph{Meine Güte, du bringst dich aber auch schnell in Schwierigkeiten.

Dein Vater war kein Gegner für dich.}

\emph{Du hast dir einen mächtigen Feind gemacht. Snape genießt die Loyalität, die Bewunderung und die Furcht des gesamten Hauses Slytherin.

Du kannst keinem aus diesem Haus mehr trauen, egal, ob er in freundlicher oder in furchterregender Gestalt zu dir kommt.

Von nun an darfst du Snapes Augen nicht mehr begegnen.}

\emph{Er ist ein Legilimens und kann deine Gedanken lesen, wenn du es tust.

Ich habe ein Buch beigelegt, das dir helfen kann, dich selbst zu schützen, auch wenn du ohne einen Lehrer nicht sehr weit kommst.

Dennoch kannst du hoffen, dass du zumindest Eindringlinge erkennst. Damit du etwas mehr Zeit hast, um Okklumentik zu studieren, habe ich 2 Galleonen beigelegt, das ist der Preis für Antwortbögen und Hausaufgaben für den Kurs Geschichte der Magie im ersten Jahr (Professor Binns hat seit seinem Tod jedes Jahr die gleichen Tests und die gleichen Aufgaben gestellt).

Deine neu gewonnenen Freunde, die Weasley-Zwillinge, sollten in der Lage sein, dir eine Kopie zu verkaufen.

Es versteht sich von selbst, dass du dich nicht mit ihm in deinem Besitz erwischen lassen darfst. Über Professor Quirrell weiß ich wenig.

Er ist ein Slytherin und ein Verteidigungsprofessor, und das sind 2 Punkte gegen ihn. Überleg dir gut, was er dir rät.

Und sagt ihm nichts, was er nicht wissen soll.}

\emph{Dumbledore tut nur so, als wäre er wahnsinnig. Er ist äußerst intelligent, und wenn du weiterhin in Schränke trittst und verschwindest, wird er sicher auf deinen Besitz eines Unsichtbarkeitsumhangs schließen, wenn er es nicht schon getan hat.}

\emph{\hfill\break Meide ihn, wann immer es möglich ist, verstecke den Unsichtbarkeitsumhang an einem sicheren Ort (NICHT in der Tasche), wenn du ihm nicht ausweichen kannst, und gehe in seiner Gegenwart sehr vorsichtig vor.

Bitte sei in Zukunft etwas vorsichtiger, Harry Potter. -}

\emph{Weihnachtsmann}

Harry starrte auf den Zettel.

Es schien ein ziemlich guter Rat zu sein. Natürlich würde Harry nicht im Geschichtsunterricht schummeln, selbst wenn sie ihm einen toten Affen als Professor geben würden.

Aber Severus' Legilimation… wer auch immer diese Notiz geschickt hatte, wusste eine Menge wichtiger, geheimer Dinge und war bereit, Harry davon zu erzählen.

Die Notiz warnte ihn immer noch davor, dass Dumbledore den Umhang gestohlen hatte, aber zu diesem Zeitpunkt hatte Harry ehrlich gesagt keine Ahnung, ob das ein schlechtes Zeichen war, es könnte auch einfach ein verständlicher Fehler sein.

Es schien eine Art Intrige in Hogwarts im Gange zu sein. Vielleicht könnte Harry, wenn er die Geschichten zwischen Dumbledore und dem Absender des Zettels vergleichen würde, ein kombiniertes Bild ausarbeiten, das korrekt wäre? Wenn sie sich also beide einig waren, dann.

.. … wie auch immer… Harry stopfte alles in seinen Beutel, drehte den Schweigezauber auf und zog sich die Decke über den Kopf und schlief ein.

Es war Sonntagmorgen und Harry aß in der Großen Halle Pfannkuchen, scharfe, schnelle Bissen, wobei er alle paar Sekunden nervös auf seine Uhr schaute.

Es war 8:02 Uhr, und in genau zwei Stunden und einer Minute würde es genau eine Woche her sein, dass er die Weasleys gesehen und auf Bahnsteig neun und drei Viertel übergesetzt hatte.

Und der Gedanke war ihm gekommen… Harry wusste nicht, ob das eine gültige Art war, über das Universum zu denken, er wusste gar nichts mehr, aber es schien möglich.

.. \emph{Dass… In der letzten Woche waren nicht genug interessante Dinge mit ihm passiert.} Als er mit dem Frühstück fertig war, nahm sich Harry vor, direkt in sein Zimmer zu gehen und sich in der untersten Etage seines Koffers zu verstecken und bis 10:03 Uhr mit niemandem zu sprechen.

In diesem Moment sah Harry die Weasley-Zwillinge auf ihn zugehen. Einer von ihnen trug etwas versteckt hinter seinem Rücken.

Er sollte schreien und weglaufen. Er sollte schreien und wegrennen. Was auch immer das war… es könnte sehr wohl… … das große Finale… Er sollte wirklich einfach schreien und wegrennen.

Mit dem resignierten Gefühl, dass das Universum ihn sowieso holen würde, schnitt Harry weiter mit Messer und Gabel in den Pfannkuchen.

Er konnte die Energie nicht aufbringen. Das war die traurige Wahrheit. Harry wusste jetzt, wie sich die Menschen fühlten, wenn sie des Laufens müde waren, müde von dem Versuch, dem Schicksal zu entkommen, und sie einfach auf den Boden fielen und sich von den schrecklich verpesteten und tentakeligen Dämonen des dunkelsten Abgrunds in ihr unaussprechliches Schicksal ziehen ließen.

Die Weasley-Zwillinge kamen näher. Und noch näher. Harry aß einen weiteren Bissen Pfannkuchen. Die Weasley-Zwillinge kamen an und grinsten breit.

"Hallo, Fred", sagte Harry dumpf. Einer der Zwillinge nickte. "Hallo, George."

Der andere Zwilling nickte.

"Du klingst müde", sagte George.

"Du solltest dich aufmuntern", sagte Fred.

"Schau, was wir für dich haben!"

Und George nahm, hinter Freds Rücken, einen Kuchen mit zwölf brennenden Kerzen.

Es gab eine Pause, als der Ravenclaw-Tisch sie anstarrte.

"Das ist nicht richtig", sagte jemand. "Harry Potter wurde am 31. Juli geboren…"

\textbf{"ER KOMMT"}, sagte eine riesige, hohle Stimme, die wie ein Schwert aus Eis durch alle Gespräche schnitt.

\textbf{"DERJENIGE, DER DIE GANZE WELT ZERREISSEN, UND DIE S -"}

Dumbledore war aus seinem Thron gesprungen und direkt über den Kopftisch gerannt und hatte die Frau gepackt, die diese schrecklichen Worte sprach, Fawkes war blitzschnell aufgetaucht, und alle drei verschwanden mit einem Feuerstoß.

Es gab eine schockierte Pause… … gefolgt von Köpfen, die sich in Richtung Harry Potter drehten.

"Ich habe es nicht getan", sagte Harry mit müder Stimme.

"Das war eine Prophezeiung!", zischte jemand am Tisch. "Und ich wette, sie handelt von dir!"

Harry seufzte. Er stand von seinem Platz auf, erhob seine Stimme und sagte sehr laut über die aufkommenden Gespräche hinweg:

"Es geht nicht um mich! Offensichtlich! Ich komme nicht hierher, ich bin schon hier!"

Harry setzte sich wieder hin. Die Leute, die ihn angeschaut hatten, wandten sich wieder ab.

Jemand anderes am Tisch sagte:

"Um wen geht es dann?"

Und mit einem dumpfen, bleiernen Gefühl wurde Harry klar, wer nicht schon in Hogwarts war.

\emph{Nennen Sie es eine wilde Vermutung, aber Harry hatte das Gefühl, dass der untote Dunkle Lord eines Tages auftauchen würde.}

Das Gespräch ging um ihn herum weiter. "Ganz zu schweigen davon, was zu zerreißen?"

"Ich dachte, ich hätte Trelawney etwas mit einem 'S' sagen hören, kurz bevor der Schulleiter sie packte."

"Wie… Seele? Sonne?"

"Wenn jemand die Sonne zerreißt, sind wir wirklich in Schwierigkeiten!"

Das erschien Harry eher unwahrscheinlich, es sei denn, die Welt enthielt unheimliche Dinge, die von David Criswells Ideen über das Heben von Sternen gehört hatten.

"Also", sagte Harry in müdem Tonfall, "das passiert doch jeden Sonntag beim Frühstück, oder?"

"Nein", sagte ein Schüler, der vielleicht im siebten Jahr gewesen sein könnte, und runzelte grimmig die Stirn. "Das tut es nicht."

Harry zuckte mit den Schultern.

"Wie auch immer. Möchte jemand ein Stück Geburtstagskuchen?"

"Aber du hast doch gar nicht Geburtstag!", sagte derselbe Schüler, der beim letzten Mal Einspruch erhoben hatte.

Das war natürlich das Stichwort für Fred und George, die anfingen zu lachen.

Selbst Harry gelang ein müdes Lächeln.

Als ihm das erste Stück serviert wurde, sagte Harry:

"Ich habe eine wirklich lange Woche hinter mir."

Und Harry saß im Keller seines Koffers, der zugeschoben und verriegelt war, damit niemand hineinkam, eine Decke über den Kopf gezogen, und wartete darauf, dass die Woche vorbei war.

\emph{10:01. 10:02. 10:03,} aber nur um sicher zu gehen… \emph{10:04} Uhr und die erste Woche war geschafft. Harry atmete erleichtert auf und zog sich vorsichtig die Decke vom Kopf.

Wenige Augenblicke später war er in der hellen, sonnenbeschienenen Luft seines Schlafsaals aufgetaucht.

Kurze Zeit später war er im Ravenclaw-Gemeinschaftsraum. Ein paar Leute sahen ihn an, aber niemand sagte etwas oder versuchte, mit ihm zu sprechen.

Harry fand einen schönen breiten Schreibtisch, zog sich einen bequemen Stuhl zurück und setzte sich.

Aus seiner Tasche zog er ein Blatt Papier und einen Bleistift.

Mum und Dad hatten Harry unmissverständlich gesagt, dass sie zwar seine Begeisterung dafür verstanden, von zu Hause wegzugehen und von seinen Eltern wegzukommen, dass er ihnen aber unbedingt jede Woche schreiben sollte, nur damit sie wussten, dass er am Leben, unverletzt und nicht im Gefängnis war.

Harry starrte auf das leere Blatt Papier hinunter. Schauen wir mal… Nachdem er seine Eltern am Bahnhof verlassen hatte, hatte er…

… einen Jungen kennengelernt, der von Darth Vader aufgezogen worden war, sich mit den drei berüchtigtsten Strolchen von Hogwarts angefreundet, Hermine kennengelernt, dann war da noch der Zwischenfall mit dem Sortierhut.

.. Am Montag hatte er eine Zeitmaschine bekommen, um seine Schlafstörung zu behandeln, einen legendären Unsichtbarkeitsumhang von einem unbekannten Wohltäter erhalten, sieben Hufflepuffs gerettet, indem er fünf furchterregende ältere Jungen niedergestreckt hatte, von denen einer gedroht hatte, ihm den Finger zu brechen, erkannt, dass er eine mysteriöse dunkle Seite besaß, in der Zauberei gelernt, wie man Frigideiro zaubert, und seine Rivalität mit Hermine in Gang gebracht.

.. Der Dienstag hatte Astronomie eingeführt, unterrichtet von Professor Aurora Sinistra, die nett war, und Geschichte der Zauberei, unterrichtet von einem Geist, der exorziert und durch ein Tonbandgerät ersetzt werden sollte.

.. Mittwoch wurde er zum gefährlichsten Schüler im Klassenzimmer gekürt. Donnerstag, an den wollen wir gar nicht denken.

… Freitag, der Vorfall in der Zaubertränkeklasse, gefolgt von seiner Erpressung des Schulleiters, gefolgt davon, dass der Verteidigungsprofessor ihn in der Klasse verprügeln ließ, gefolgt davon, dass der Verteidigungsprofessor sich als der furchteinflößendste Mensch entpuppte, der noch auf Erden wandelte.

.. Am Samstag hatte er eine Wette verloren und war zu seinem ersten Date gegangen und hatte begonnen, Draco zu erlösen.

.. und dann heute Morgen Professor Trelawneys unerhörte Prophezeiung, dass ein unsterblicher Dunkler Lord im Begriff war, Hogwarts anzugreifen, oder auch nicht.

Harry ordnete im Geiste sein Material und begann zu schreiben.

\emph{Liebe Mum und Dad: Hogwarts macht sehr viel Spaß.

In Zauberei habe ich gelernt, wie man den zweiten Hauptsatz der Thermodynamik bricht, und ich habe ein Mädchen namens Hermine Granger kennengelernt, das schneller liest als ich. Ich sollte es dabei belassen. Euer euch liebender Sohn, Harry James Potter-Evans-Verres.}

