

\hypertarget{tabubruxfcche-finale}{% \section{82. Tabubrüche, Finale}\label{tabubruxfcche-finale}}

\textbf{\uline{Tabubrüche, Finale}}

Die Reise mit einem Phönix war ein Gefühl, das völlig anders war als Apparition oder ein Portschlüssel. Man fing Feuer - man fühlte definitiv, dass man Feuer fing, obwohl es keinen Schmerz gab - und anstatt zu Asche zu verbrennen, brannte das Feuer den ganzen Weg durch einen hindurch und man wurde zu Feuer, und dann erlosch man an einem Ort und loderte an einem anderen auf. Es verursachte keine Magenschmerzen wie Portschlüssel oder Apparition, aber es war dennoch eine ziemlich nervenaufreibende Erfahrung. Wenn die zugrundeliegende Wahrheit des Phönix-Reisens wirklich darin bestand, eine spezifische Instanziierung eines allgemeineren Feuers zu werden, dann schien das darauf hinzudeuten, dass man potenziell überall brennen konnte - sogar in der fernen Vergangenheit oder in einem anderen Universum oder an zwei Orten gleichzeitig. Man konnte an einem Ort ausgehen und an hundert anderen in Flammen aufgehen, und derjenige, der in Hogwarts ankam, würde den Unterschied nicht merken. Obwohl Harry alles über Phönixe gelesen hatte, was er konnte, um herauszufinden, wie er selbst einen bekommen könnte, gab es keinen Hinweis auf irgendetwas, das auch nur im Entferntesten mit dieser Fähigkeit vergleichbar wäre.

Harry fing Feuer und ging hinaus und loderte an einem anderen Ort auf; und einfach so waren er und der Schulleiter und die bewusstlose Gestalt von Hermine Granger, die in den Armen des Schulleiters gehalten wurde, an einem anderen Ort; mit Fawkes über ihnen allen. Ein ruhiger, warmer Raum mit hellen Steinsäulen, der an allen vier Seiten von Oberlichtern erhellt wurde, bevölkert von weißen Betten in langen Reihen, von denen vier mit schalldämpfenden Schleiern verhüllt waren und der Rest leer. In einer Ecke von Harrys Blickfeld drehte sich eine überrascht aussehende Madam Pomfrey zu ihnen um. Dumbledore schien die Oberheilerin nicht zu beachten, während er Hermine vorsichtig auf ein unbesetztes weißes Bett legte. Aus einer entfernten Ecke blitzte es grün auf, und aus dem Kamin schritt Professor McGonagall, die sich leicht von der Floo-Asche befreite. Der alte Zauberer drehte sich vom Bett weg und legte wieder einen seiner Arme um Harry; und dann verschwanden der Junge-der-lebte und sein Zauberer in einem weiteren Feuerstoß. Als Harry wieder vollständig bei Sinnen war, stand er im Büro des Schulleiters, inmitten der Geräusche von einem Dutzend unerklärlicher Artefakte. Der Junge ging einen Schritt von dem alten Zauberer weg und drehte sich dann zu ihm um, wobei sich smaragd- und saphirfarbene Augen trafen. Die beiden sprachen eine Zeit lang nicht, sondern sahen sich nur an, als ob alles, was sie zu sagen hatten, nur durch Blicke gesagt werden konnte, und nicht auf andere Weise.

Mit der Zeit sprach der Junge die Worte langsam und präzise aus.

"Ich kann nicht glauben, dass ein Phönix noch auf Ihrer Schulter sitzt."

"Der Phönix wählt nur einmal", sagte der alte Zauberer. "Sie mögen vielleicht einen Meister verlassen, der das Böse dem Guten vorzieht; aber sie werden keinen Meister verlassen, der gezwungen ist, zwischen einem Guten und einem anderen zu wählen. Phönixe sind nicht arrogant. Sie kennen die Grenzen ihrer eigenen Weisheit."

\emph{Streng in der Tat, dieser uralte Blick.}

"Im Gegensatz zu dir, Harry."

"Zwischen einem Gut und einem anderen wählen", wiederholte Harry barsch. "Wie das Leben von Hermine Granger gegen hunderttausend Galleonen."

Die Wut und Empörung, die Harry in seine Stimme legen wollte, war nicht ganz da, aus irgendeinem Grund, vielleicht weil -

"Du bist kaum in der Lage, mir das zu sagen, Harry Potter."

Die Stimme des Schulleiters war trügerisch sanft.

"Oder was war das für ein Ausdruck des Widerwillens, den ich auf deinem Gesicht gesehen habe, dort in der Allerältesten Halle?"

Das Gefühl der inneren Leere verstärkte sich.

"Ich habe nach anderen Alternativen gesucht", stieß Harry hervor. "Eine Möglichkeit, sie zu retten, ohne das Geld zu verlieren."

\emph{Wow}, sagte der Ravenclaw. \emph{Du hast gerade eine glatte Lüge erzählt. Nicht nur das, ich glaube, du hast es für die Sekunden, die du gebraucht hast, um es zu sagen, tatsächlich geglaubt. Das ist irgendwie beängstigend.}

"Ist es das, was du gedacht hast, Harry?"

Die blauen Augen waren scharf, und es gab einen erschreckenden Moment, in dem Harry sich fragte, ob der mächtigste Zauberer der Welt hinter seine Okklumentikbarriere sehen konnte.

"Also gut, ja!", sagte Harry, "ich bin vor dem Schmerz, das ganze Geld in meinem Tresor zu verlieren, zurückgeschreckt. Aber ich habe es getan! Das ist es, was zählt! Und Sie -"

Die Empörung, die aus Harrys Stimme gewichen war, kehrte zurück.

"Sie haben tatsächlich einen Preis auf Hermine Grangers Leben gesetzt, und zwar unter hunderttausend Galleonen!"

"Oh?", sagte der alte Zauberer leise. "Und welchen Preis setzt du dann auf ihr Leben? Eine Million Galleonen?"

"Sind Sie mit dem wirtschaftlichen Konzept des 'Wiederbeschaffungswertes' vertraut?" Die Worte sprudelten fast schneller aus Harrys Lippen, als er sie bedenken konnte. "Der Wiederbeschaffungswert von Hermine ist unendlich! Ich kann mir nirgendwo eine andere kaufen!"

\emph{Jetzt redest du nur noch mathematischen Blödsinn}, sagte Slytherin. \emph{Ravenclaw, hilfst du mir mal auf die Sprünge?}

"Ist Minervas Leben auch von unendlichem Wert?", fragte der alte Zauberer barsch. "Würdest du Minerva opfern, um Hermine zu retten?"

"Ja und ja", schnauzte Harry. "Das ist Teil von Professor McGonagalls Beruf und das weiß sie."

"Dann ist Minervas Wert nicht unendlich", sagte der alte Zauberer, "so sehr sie auch geliebt wird. Es kann nur einen König auf einem Schachbrett geben, Harry Potter, nur eine Figur, für die man jede andere Figur opfern würde, um sie zu retten. Und Hermine Granger ist nicht diese Figur. Täusche dich nicht, Harry Potter, an diesem Tag hast du vielleicht deinen Krieg verloren."

Und wenn die Worte des alten Zauberers nicht so hart getroffen hätten, und so nah an seinem Zuhause, hätte Harry vielleicht nicht gesagt, was er dann sagte.

"Lucius hatte recht", stieß Harry hervor. "Du hattest nie eine Frau, du hattest nie eine Tochter, du hattest nie etwas anderes als Krieg -"

Die linke Hand des alten Zauberers schloss sich hart um Harrys Handgelenk, knochige Finger gruben sich in den sich noch entwickelnden Muskel von Harrys Arm, und einen Moment lang war Harry wie gelähmt vor Schreck, er hatte vergessen, was es bedeutete, dass Erwachsene stärker waren. Albus Dumbledore schien das nicht zu bemerken. Er drehte sich nur um, zog Harry mit sich und bewegte sich mit schweren Schritten auf die Wand des Raumes zu.

"\textbf{\emph{Der Preis des Phönix.}}"

Harry wurde die schwarze Treppe hinaufgezogen.

"\textbf{\emph{Das Schicksal des Phönix'}}."

Der Raum mit den schwarzen Sockeln, das silberne Licht fiel auf die zerbrochenen Zauberstäbe.

"Du glaubst", schrie Harry, nachdem sich seine Lippen gelöst hatten, "dass du jeden Streit gewinnen kannst, nur weil du hier stehst?"

Der alte Zauberer ignorierte ihn und zerrte Harry durch den Raum. Seine rechte Hand, die nicht mehr seinen Zauberstab hielt, griff nach einer Phiole mit silberner Flüssigkeit - Harry blinzelte schockiert; die Phiole mit der silbernen Flüssigkeit hatte neben einem Bild von Dumbledore gestanden, so war es Harry in dem kurzen Moment erschienen, bevor er vorbeigeschleift wurde. Hinter dem Ende aller Podeste, im hintersten Teil des Raumes, erhob sich ein großes steinernes Becken mit eingemeißelten Runen, die Harry nicht erkannte. In der Mitte befand sich eine flache Vertiefung, die mit einer durchsichtigen Flüssigkeit gefüllt war, und in diese schüttete der alte Zauberer den Kanister mit der silbernen Flüssigkeit, die sich sofort auszubreiten begann, zu wirbeln und das gesamte Becken in ein unheimliches Weiß zu tauchen.

Die Hand des alten Zauberers ließ Harrys Arm los und wies mit einer Geste auf das glühende Becken und befahl barsch: "Schau hin!"

Wie gefordert starrte Harry auf das glühende Wasser.

"Steck deinen Kopf in das Becken, Harry Potter."

Die Stimme des alten Zauberers war streng.

Harry hatte dieses Wort schon einmal gehört, aber er konnte sich nicht erinnern, wo . "Was - macht das -"

"Erinnerungen", sagte der alte Zauberer. "Du wirst mein Gedächtnis sehen. Ich schwöre, dass es sicher ist. Nun schau in das Becken, Ravenclaw, wenn dir überhaupt noch etwas an deiner kostbaren Wahrheit liegt!"

Das war eine Aufforderung, die Harry nicht ablehnen konnte, und er trat vor und stieß seinen Kopf in das glühende Wasser.

…

\emph{Harry saß hinter dem Schreibtisch im Büro des Schulleiters von Hogwarts, und seine faltigen Hände, die sich an seinen Kopf klammerten, waren von Alter und weißen Haaren gezeichnet.}

\emph{"Er ist alles, was ich habe!", weinte eine Stimme, sehr seltsam war die Stimme, wie Dumbledore selbst sie in Erinnerung hatte, von innen heraus wirkte sie weit weniger streng und weise.}

\emph{"Der Letzte meiner Familie! Alles, was ich noch habe!"}

\emph{Keine Emotion dran durch die Erinnerung, nur die physische Empfindung, dass er die Worte zu sprechen schien. Harry hörte die völlige Verzweiflung in Dumbledores Worten, die Laute, die aus Harrys eigener Kehle zu kommen schienen, aber Harry fühlte sie nicht über das Hören hinaus.}

\emph{"Du hast keine Wahl", sagte eine raue Stimme. Die Augen bewegten sich, das Blickfeld sprang auf einen Mann, den Harry nicht erkannte, in einer von Aurorenkarminrot gefärbten Kleidung aus festem Leder mit vielen Taschen. Sein rechtes Auge war übergroß, mit einer elektrisch-blauen Pupille, die ständig zuckte und sich bewegte.}

\emph{"Das kannst du nicht von mir verlangen, Alastor!" Dumbledores Stimme war wild. "Nicht das! Alles, nur das nicht!"}

\emph{"Ich bitte dich nicht", knurrte der Mann. "Voldie ist derjenige, der fragt, und \textbf{du wirst ihm}} \textbf{Nein} sagen."

\emph{"Für Geld, Alastor?" Dumbledores Stimme war flehend. "Nur für Geld?"}

\emph{"Wenn du Aberforth freikaufst, verlierst du den Krieg", sagte der Mann schroff. "So einfach ist das. Hunderttausend Galleonen sind fast alles, was wir in der Kriegskasse haben, und wenn du sie so benutzt, wird sie nicht wieder aufgefüllt. Was willst du tun, versuchen, die Potters zu überreden, ihren Tresor zu leeren, wie es die Longbottoms bereits getan haben? Voldie wird einfach jemand anderen entführen und eine weitere Forderung stellen. Alice, Minerva, alle, die dir wichtig sind, werden zur Zielscheibe, wenn du die Todesser auszahlst. Das ist nicht die Lektion, die du ihnen beibringen solltest."}

\emph{"Wenn ich das tue, werde ich niemanden haben. Niemanden." Dumbledores Stimme brach, die Welt kippte, als der aussehende Kopf in die uralten Hände fiel, und schreckliche Laute kamen aus Nicht-Harrys Kehle, als er wie ein Kind zu schluchzen begann.}

\emph{"Soll ich Voldies Boten abweisen?", sagte Alastors Stimme, jetzt seltsam sanft. "Du brauchst es nicht selbst zu tun, alter Freund."}

\emph{"Nein - ich werde es selbst sagen - ich muss -"}

…

Die Erinnerung endete mit einem Schock und Harry riss seinen Kopf aus dem glühenden Wasser, keuchend, als hätte man ihm die Luft weggenommen. Der Übergang zwischen den Szenen, zwischen der jahrzehntealten Realität und dem gegenwärtigen Moment, war ein weiterer Ruck für Harrys Verstand; auf irgendeine Weise hatte ihn das Eintauchen in die Vergangenheit aus der Verankerung gerissen. Der gebrochene alte Mann, der in seinem Büro schluchzte, war eine andere Person in einer anderen Ära gewesen, so viel hatte Harry verstanden; jemand Weicheres - bevor sich alles wie auflösender Rauch verflüchtigt hatte und das Jetzt, die Gegenwart, zurückkehrte.

Schrecklich und streng stand der alte Zauberer da, als wäre er aus Stein gemeißelt; der Bart aus Fäden gewoben wie Eisen, die halbmondförmigen Gläser wie Spiegel, und die Pupillen dahinter so scharf und unnachgiebig wie schwarzer Diamant.

"Wünschst du auch, meinen Bruder zu sehen, wie er unter dem Cruciatus zu Tode gefoltert wurde?", fragte Albus Dumbledore. "Voldemort hat mir auch diese Erinnerung geschickt!"

"Und das -" Harry hatte Mühe, eine Stimme hervorzubringen, wegen der wachsenden Übelkeit in seiner Brust. "Das war, als -"

Die Worte schienen in seiner Kehle zu brennen, als ihm das schreckliche Wissen dämmerte, die schreckliche Erkenntnis.

"Das war, als du Narcissa Malfoy in ihrem eigenen Schlafzimmer lebendig verbrannt hast."

Albus Dumbledores Blick war kalt, als er antwortete.

"Auf diese Frage würde nur ein Narr mit Ja oder Nein antworten. Was zählt, ist, dass die Todesser glauben, ich hätte sie getötet, und dieser Glaube hat die Familien aller, die dem Orden des Phönix gedient haben, in Sicherheit gebracht - bis zum heutigen Tag. Ist dir jetzt klar, was du getan hast? Was du deinen Freunden angetan hast, Harry Potter, und allen, die zu dir halten?"

Der alte Zauberer schien noch größer und schrecklicher zu werden, als seine Stimme lauter wurde.

"Du hast sie alle zu Zielscheiben gemacht, und das werden sie auch bleiben! Bis du beweist, auf die einzige Art und Weise, wie es bewiesen werden kann, dass du nicht länger bereit bist, solche Preise zu zahlen!"

"Und ist es wahr?" sagte Harry.

Ein brummendes Gefühl erfüllte ihn, sein Körper wurde immer unruhiger.

"Was Draco gesagt hat, dass Narcissa Malfoy sich nie die Hände schmutzig gemacht hat, dass sie nur die Frau von Lucius war? Sie war eine Ermöglicherin, das verstehe ich, aber ich kann nicht unterstützen, dass sie es verdient, lebendig verbrannt zu werden."

"Nichts Geringeres hätte sie davon überzeugt, dass ich mit dem Zögern fertig bin." Die Stimme des alten Zauberers duldete keine Frage und keine Ablehnung.

"Immer war ich zu zögerlich, um zu tun, was ich tun musste, immer waren es andere, die den Preis für meine Barmherzigkeit zahlten. Das hat mir Alastor von Anfang an gesagt, aber ich habe nicht auf ihn gehört. Du, so erwarte ich, wirst dich bei solchen Entscheidungen als besser erweisen als ich."

"Ich bin überrascht", sagte Harry und war erstaunt, dass seine Stimme fast gleichmäßig war. "Ich hätte erwartet, dass die Todesser auf eine andere Familie des Lichts losgehen und einen Kreislauf eskalierender Vergeltung beginnen, wenn du sie nicht alle mit deinem ersten Schlag erwischt hättest."

"Wenn mein Gegner Lucius gewesen wäre, vielleicht."

Dumbledores Augen waren wie Steine.

"Voldemort soll darüber gelacht und seinen Todessern verkündet haben, ich sei endlich erwachsen geworden und ein würdiger Gegner. Vielleicht hatte er recht. Nach dem Tag, an dem ich meinen Bruder zum Tode verurteilte, begann ich, diejenigen, die mir folgten, abzuwägen, sie gegeneinander abzuwägen, zu fragen, wen ich riskieren und wen ich opfern würde, zu welchem Zweck. Es war seltsam, wie viel weniger Stücke ich verlor, sobald ich wusste, was sie wert waren."

Harrys Kiefer schien verschlossen, als ob es eine gewaltige Anstrengung kostete, seine Lippen zu bewegen.

"Aber dann ist es ja nicht so, dass Lucius Hermine absichtlich als Lösegeld entführt hat", sagte Harrys Stimme dünn. "Aus Lucius' Sicht hat jemand anderes den Waffenstillstand zuerst gebrochen. Wenn man das bedenkt, wie viele Galleonen war Hermine dann genau wert? Abgesehen von der Lösegeld-Sache, wenn es nur eine gewöhnliche Bedrohung für ihr Leben war, wie viel hätte ich zahlen müssen, um sie zu retten? Zehntausend Galleonen? Fünftausend?"

Der alte Zauberer antwortete nicht.

"Es ist eine komische Sache", sagte Harry, seine Stimme schwankte wie etwas, das man durch Wasser sieht. "Weißt du, was meine schlimmste Erinnerung an dem Tag war, an dem ich vor den Dementor trat? Es war, als meine Eltern starben. Ich habe ihre Stimmen gehört und alles."

Die Augen des alten Zauberers weiteten sich hinter der Halbmondbrille.

"Und hier ist die Sache", sagte Harry, "hier ist die Sache, über die ich immer und immer wieder nachgedacht habe. Der Dunkle Lord gab Lily Potter die Möglichkeit zu fliehen. Er sagte, sie könne fliehen. Er sagte ihr, dass das Sterben vor der Krippe ihr Baby nicht retten würde.

\textbf{\emph{Tritt beiseite, törichte Frau, wenn du auch nur einen Funken Verstand in dir hast}} -"

Ein schreckliches Frösteln überkam Harry, als er diese Worte aus seinem eigenen Mund sprach, aber er schüttelte es ab und fuhr fort.

"Und danach dachte ich immer wieder, ich konnte mich nicht davon abhalten, zu denken: Hatte der Dunkle Lord nicht recht? Wenn Mutter doch nur weggegangen wäre. Sie hat versucht, den Dunklen Lord zu verfluchen, aber es war Selbstmord, sie musste wissen, dass es Selbstmord war. Sie hatte nicht die Wahl zwischen ihrem Leben und meinem. Sie hatte die Wahl, entweder zu leben oder zu sterben, wir beide! Wenn sie nur das Logische getan hätte und weggegangen wäre, ich meine, ich liebe Mum auch, aber Lily Potter wäre jetzt am Leben und sie wäre meine Mutter!"

Tränen trübten Harrys Augen.

"Erst jetzt verstehe ich, ich weiß, was Mutter gefühlt haben muss. Sie konnte nicht von der Krippe weggehen. Sie konnte es nicht! \textbf{Liebe geht nicht weg!}"

Es war, als wäre der alte Zauberer getroffen worden, getroffen von einem Meißel, der ihn in der Mitte zerschmetterte.

"Was habe ich gesagt?", flüsterte der alte Zauberer. "Was habe ich zu dir gesagt?"

"Ich weiß es nicht!", rief Harry. "Ich habe auch nicht zugehört!"

"Ich - es tut mir leid, Harry - ich -"

Der alte Zauberer presste seine Hände auf sein Gesicht, und Harry sah, dass Albus Dumbledore weinte.

"Ich hätte so etwas nicht zu dir sagen dürfen - ich hätte dir deine Unschuld nicht übel nehmen dürfen -"

Harry starrte den Zauberer noch eine Sekunde lang an, dann drehte er sich um und marschierte aus dem schwarzen Raum, die Treppe hinunter, durch das Büro -

"Ich weiß wirklich nicht, warum du immer noch auf seiner Schulter sitzt", sagte Harry zu Fawkes. -

durch die Eichentür und in die sich endlos drehende Spirale.

Harry war vor allen anderen im Klassenzimmer für Verwandlung angekommen, sogar vor Professor McGonagall. Zuvor hatte er noch den Zauberkunstunterricht seines Jahrgangs besucht, aber er hatte sich nicht einmal die Mühe gemacht aufzupassen. Ob Professor McGonagall den heutigen Unterricht leiten würde, wusste er nicht.

All die leeren Tische neben ihm, die Abwesenheit an der Tafel, hatten etwas Unheilvolles an sich. Als stünde er allein in Hogwarts, da alle seine Freunde abgereist waren. Laut Stundenplan ging es in der heutigen Stunde um ständige Verwandlungen, deren Regeln Harry schon auswendig gelernt hatte, als er einen riesigen Stein in den kleinen Diamanten verwandelt hatte, der an seinem kleinen Finger glänzte. Für den Rest der Klasse würde es eher ein theoretisches als ein praktisches Thema sein; was schade war, denn er hätte eine Dosis Verwandlungsspaß gebrauchen können.

Harry bemerkte aus der Ferne, dass seine Hand zitterte, und zwar so sehr, dass er Mühe hatte, die Kordel des Beutels zu öffnen, als er das Schulbuch herauszog.

\emph{Du warst ungeheuerlich unfair zu Dumbledore}, sagte die Stimme, die Harry Slytherin genannt hatte, nur schien sie jetzt auch die Stimme der wirtschaftlichen Vernunft und vielleicht auch des Gewissens zu sein.

Harrys Augen fielen auf sein Lehrbuch, aber der Abschnitt war so vertraut, dass er genauso gut ein leeres Pergament hätte sein können.

\emph{Dumbledore kämpfte einen Krieg gegen einen Dunklen Lord, der bewusst darauf aus war, ihn auf die grausamste Art und Weise zu brechen. Er musste wählen, ob er den Krieg oder seinen Bruder verlieren wollte. Albus Dumbledore weiß, er hat auf die schlimmstmögliche Art und Weise gelernt, dass es Grenzen für den Wert eines Lebens gibt; und es hat fast seinen Verstand gebrochen, das zuzugeben. Aber du, Harry Potter - du wusstest es bereits besser.}

"Halt die Klappe", flüsterte der Junge in das leere Klassenzimmer, obwohl dort niemand war, der es hören konnte.

\emph{Du hattest bereits von Philip Tetlocks Experimenten mit Menschen gelesen, die aufgefordert wurden, einen emotionalen Wert gegen einen weltlichen einzutauschen, wie ein Krankenhausverwalter, der sich entscheiden muss, ob er eine Million Dollar}

\emph{für eine Leber ausgibt, um ein fünfjähriges Kind zu retten, oder ob er die Million ausgibt, um andere Krankenhausgeräte zu kaufen oder die Gehälter der Ärzte zu bezahlen. Und die Versuchspersonen wurden empört und wollten den Krankenhausverwalter dafür bestrafen, dass er überhaupt an die Wahl dachte.}

\emph{Erinnerst du dich daran, davon gelesen zu haben, Harry Potter?}

\emph{Erinnerst du dich, dass du dachtest, wie dumm das war, denn wenn Krankenhausausrüstung und Arztgehälter nicht auch Leben retten würden, gäbe es keinen Grund, Krankenhäuser oder Ärzte zu haben? Sollte der Krankenhausverwalter eine Milliarde Pfund für diese Leber bezahlen, auch wenn das Krankenhaus am nächsten Tag bankrott ist?}

"Halt die Klappe!", flüsterte der Junge.

\emph{Jedes Mal, wenn du Geld ausgibst, um ein Leben mit einer gewissen Wahrscheinlichkeit zu retten, legst du eine untere Grenze für den Geldwert eines Lebens fest. Jedes Mal, wenn du dich weigerst, Geld auszugeben, um mit einer gewissen Wahrscheinlichkeit ein Leben zu retten, legst du eine Obergrenze für den Geldwert eines Lebens fest. Wenn deine Ober- und Untergrenzen inkonsistent sind, bedeutet dies, dass du Geld von einem Ort zum anderen verschieben und mehr Leben zu den gleichen Kosten retten könntest. Wenn du also eine begrenzte Menge an Geld verwenden willst, um so viele Leben wie möglich zu retten, müssen deine Entscheidungen mit dem einem Menschenleben zugewiesenen Geldwert übereinstimmen; wenn nicht, kannst du das gleiche Geld nicht verwenden um es besser zu machen.}

\emph{Wie traurig, wie hohl die Empörung derjenigen, die sich weigern, zu sagen, dass Geld und Leben jemals verglichen werden können, wenn alles, was sie tun, ist, die Strategie zu verbieten, die die meisten Menschen rettet, um der anmaßenden moralischen Selbstdarstellung willen…}

\emph{Du wusstest das, und du hast trotzdem gesagt, was du zu Dumbledore gesagt hast. Du hast absichtlich versucht, Dumbledores Gefühle zu verletzen. Er hat nie versucht, dich zu verletzen, Harry Potter, nicht ein einziges Mal.}

Harrys Kopf fiel in seine Hände.

\emph{Warum hatte Harry gesagt, was er gesagt hatte, zu einem traurigen alten Zauberer,} \emph{der hart gekämpft und mehr ertragen hatte, als irgendjemand jemals hätte ertragen sollen?} \emph{Selbst wenn der alte Zauberer Unrecht hatte, hatte er es verdient, dafür verletzt zu werden, nach allem, was ihm widerfahren war? Warum gab es einen Teil von ihm, der über die Maßen wütend auf den alten Zauberer zu werden schien, ihn härter schlug, als Harry jemals jemanden geschlagen hatte, ohne an Mäßigung zu denken, sobald die Wut hochkochte, nur um sich zu beruhigen, sobald Harry seine Gegenwart verließ?}

\emph{Ist es, weil du weißt, dass Dumbledore sich nicht wehren wird? Dass er, egal was du zu ihm sagst, wie unfair es auch sein mag, niemals seine eigene Macht gegen dich einsetzen wird, er dich niemals so behandeln wird, wie du ihn behandelst? Behandelt man so Leute, wenn man weiß, dass sie nicht zurückschlagen werden? James Potters Mobbing-Gene, die sich endlich manifestieren?}

Harry schloss die Augen. Es war als würde der Sprechende Hut in seinem Kopf sprechen -

\emph{Was ist der wahre Grund für deine Wut? Wovor hast du Angst?}

Ein Wirbelwind von Bildern schien durch Harrys Geist zu blitzen:

der vergangene Dumbledore, der in seine Hände weinte;

die gegenwärtige Gestalt des alten Zauberers, groß und schrecklich stehend;

eine Vision von Hermine, die in ihren Ketten schrie, in dem Metallstuhl, als Harry sie den Dementoren überließ;

und eine Vorstellung von einer Frau mit langem weißen Haar

(hatte sie wie ihr Mann ausgesehen?), die inmitten der Flammen ihres Schlafzimmers fiel, während ein Zauberstab auf sie gerichtet war und orangefarbenes Licht von halbmondförmigen Gläsern reflektiert wurde.

Albus Dumbledore hatte anscheinend geglaubt, dass Harry in solchen Dingen besser sein würde als er. Und Harry wusste, dass er es wahrscheinlich sein würde. Immerhin kannte er die Mathematik. Aber es war klar, irgendwie war es klar, dass utilitaristische Ethiker eigentlich keine Banken ausraubten, um das Geld den Armen zu geben. Das Endergebnis des Wegwerfens aller ethischen Zwänge würde nicht wirklich Sonnenschein und Rosen und Glück für alle sein. Das Rezept des Konsequentialismus war, die Handlung zu ergreifen, die zu den besten Nettokonsequenzen führte, nicht Handlungen, die eine positive Konsequenz hatten und auf dem Weg dorthin alles andere zerstörten.

Erwartete Nutzenmaximierer durften den gesunden Menschenverstand berücksichtigen, wenn sie ihre Erwartungen berechneten.

Irgendwie hatte Harry das verstanden, noch bevor ihn jemand anderes erzählt hatte, dass er es verstanden hatte.Bevor er über Wladimir Lenin oder die Geschichte der Französischen Revolution gelesen hatte, hatte er es gewusst. Vielleicht waren es seine frühesten Science-Fiction-Bücher, die ihn vor Menschen mit guten Absichten warnten, oder vielleicht hatte Harry einfach die Logik für sich selbst gesehen. Irgendwie hatte er von Anfang an gewusst, dass das Endergebnis nicht gut sein würde, wenn er aus irgendeinem Grund aus seiner Ethik heraustrat.

Dann kam ihm ein letztes Bild in den Sinn: Lily Potter, die vor der Wiege ihres Babys steht und die Intervalle zwischen den Ergebnissen abwägt:

das Endergebnis, wenn sie bleibt und versucht, ihren Feind zu töten (tote Lily, toter Harry),

das Endergebnis, wenn sie geht (lebende Lily, toter Harry),

die zu erwartenden Vorteile abwägt und die einzig vernünftige Entscheidung trifft.

Sie wäre Harrys Mutter gewesen, wenn sie es getan hätte.

"Aber menschliche Wesen können so nicht leben", flüsterten die Lippen des Jungen in das leere Klassenzimmer. "\emph{Menschliche Wesen können so nicht leben}."

