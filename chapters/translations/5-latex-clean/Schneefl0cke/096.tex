

\hypertarget{rollen-teil-8}{% \section{97. Rollen, Teil 8}\label{rollen-teil-8}}

\textbf{\uline{Rollen, Teil 8}}

Zum zweiten Mal an diesem Tag füllten sich Harrys Augen mit Tränen. Ohne auf die verwunderten Blicke der Ravenclaws im Gemeinschaftsraum zu achten, griff er nach der silbernen Kreatur, die Draco Malfoy ihm geschickt hatte, und wiegte sie in seinen Armen wie ein lebendiges Wesen; dann stolperte er in Richtung seines Schlafsaals und steuerte halb blind auf den Boden seines Koffers zu, während die silberne Schlange stumm in seinen Armen wartete.

\textbf{Das fünfte Treffen:} 10:12~Uhr, Sonntag, 19. April.

Das Schuldnertreffen, das Lord Malfoy von Harry Potter, der Lucius Malfoy 58.203 Galleonen schuldete, verlangt hatte, fand in der Gringotts-Zentralbank statt, in Übereinstimmung mit den britischen Gesetzen. Es gab einige Widerstände seitens des Obersten Hexenmeisters Dumbledore, der Harry Potter daran hindern wollte, die Sicherheit von Hogwarts zu verlassen (ein Satz, der Harry Potter dazu veranlasste, seine Finger zu heben und stumm Anführungszeichen in die Luft zu machen). Der Junge-der-lebte hatte seinerseits scheinbar still gegrübelt und dann dem Treffen zugestimmt, seltsam nachgiebig angesichts der Forderung seines Feindes. Der Schulleiter von Hogwarts, der in den Augen des magischen Britanniens als Harry Potters gesetzlicher Vormund fungierte, hatte die Zustimmung seines Mündels überstimmt. Der Schuldenausschuss des Zaubergamot hatte den Schulleiter von Hogwarts überstimmt. Der Oberste Hexenmeister hatte den Schuldenausschuss überstimmt. Das Zaubergamot hatte den Obersten Hexenmeister überstimmt.

Und so war der Junge-der-lebte unter der strengen Bewachung von Mad-Eye Moody und einem Auroren-Trio zur Gringotts-Zentralbank aufgebrochen; mit Moodys hellblauem Auge, das wild in alle Richtungen rotierte, als wolle er jedem möglichen Angreifer signalisieren, dass er auf der Hut und ständig wachsam sei und jedem, der in die allgemeine Richtung des Jungen-der-lebte nieste, fröhlich die Nieren verbrennen würde.

Harry Potter beobachtete mit noch größerer Aufmerksamkeit als zuvor, wie sie durch die weit geöffneten Eingangstüren von Gringotts marschierten, unter dem Motto \emph{Fortius Quo Fidelius.} Bei Harrys letzten drei Besuchen in Gringotts hatte er lediglich die Marmorsäulen bewundert, die goldglühenden Fackellichter, die Architektur, die nicht ganz den menschlichen Teilen des magischen Britanniens entsprach. Seitdem gab es den Vorfall in Askaban und andere Dinge; und jetzt, bei seinem vierten Besuch, dachte Harry an die Koboldaufstände und den anhaltenden Unmut der Kobolde darüber, dass sie keine Zauberstäbe besitzen durften, und an bestimmte Fakten, die nicht im Geschichtslehrbuch der ersten Klasse gestanden hatten, die Harry durch Mustervergleiche erraten hatte und die Professor Flitwick mit sehr leiser Stimme bestätigt hatte.

Lord Voldemort hatte sowohl Kobolde als auch Zauberer getötet - ein unglaublich dummer Schachzug von Lord Voldemort, es sei denn, Harry hatte wirklich etwas übersehen - aber was Kobolde über den Jungen-der-überlebte dachten, wusste Harry nicht. Kobolde hatten den Ruf, zu zahlen, was sie schuldig waren, und zu nehmen, was sie glaubten, ihnen zu schulden, zusammen mit dem Ruf, diese Rechnungen auf eine etwas voreingenommene Weise zu interpretieren.

Heute starrten die Wachen, die in regelmäßigen Abständen um die Bank herum in ihren Rüstungen aufrecht standen, den Jungen-der-lebte mit leeren Gesichtern an und blickten Moody und die Auroren mit einem Anflug von bitterer Verachtung an. An den Ständen und Schaltern im Foyer der Bank starrten die koboldhaften Kassierer mit der gleichen Verachtung auf die Zauberer, deren Hände sie mit Galleonen füllten; ein Kassierer lächelte einer Hexe, die wütend und verzweifelt dreinschaute, ein spitzzahniges Grinsen entgegen.

\emph{Wenn ich die menschliche Natur richtig verstehe - und wenn ich Recht habe, dass alle humanoiden magischen Spezies genetisch menschlich plus einem vererbbaren magischen Effekt sind - dann werden Sie sich wahrscheinlich nicht mit einem Zauberer anfreunden, nur weil ich höflich zu Ihnen bin oder sage, dass ich sympathisch bin. Aber ich frage mich, ob Sie den Jungen-der-überlebte unterstützen würden, um das Ministerium zu stürzen, wenn ich versprechen würde, das Zauberstabgesetz danach aufzuheben… oder wenn ich Ihnen im Gegenzug für Ihre Unterstützung heimlich Zauberstäbe und Zauberbücher geben würde… ist das der Grund, warum das Geheimnis der Zauberstabherstellung Leuten wie Ollivander vorbehalten ist? Aber wenn Sie wirklich Menschen sind, einfach nur Menschen, dann hat das Koboldvolk wahrscheinlich seine eigenen inneren Schrecken, seine eigenes Askaban, denn auch das ist die menschliche Natur; in diesem Fall muss ich früher oder später auch Ihre eigene Regierung stürzen oder reformieren. Hm…}

Ein gealterter Kobold erschien vor ihnen, und Harry neigte seinen Kopf mit vorsichtiger Höflichkeit, eine Geste, die der alte Kobold mit einem abrupten halben Nicken erwiderte. Es gab keine wilde Zugfahrt; stattdessen führte der alte Kobold sie in einen kurzen Flur, der in einem kleinen Warteraum endete, mit drei koboldgroßen Bänken und einem zauberergroßen Stuhl, auf dem niemand saß.

„Unterschreibe nichts, was Lucius Malfoy dir gibt“, sagte Mad-Eye Moody. „Nichts, hast du mich verstanden, Junge?! Wenn Malfoy dir ein Exemplar von \emph{'Die wundervollen Abenteuer des Jungen der überlebte'} in die Hand drückt und dich um ein Autogramm bittet, sag ihm, du hättest dir einen Finger verstaucht. Nimm keine Feder in die Hand, solange du in Gringotts bist. Wenn dir jemand einen Federkiel reicht, zerbreche den Federkiel und dann deine eigenen Finger. Muss ich das weiter erklären, mein Sohn?“

„Nicht unbedingt“, sagte Harry. „Wir haben auch Anwälte in Muggelbritannien, und die würden eure Anwälte süß finden.“

Kurze Zeit später übergab Harry Potter seinen Zauberstab an einen gepanzerten Koboldwächter, der ihn mit allerlei interessant aussehenden Sonden abtastete, und gab Moody seinen Beutel zur Aufbewahrung. Und dann trat Harry durch eine weitere Tür, und ein kurzer Wasserfall von Diebesuntergang, der von seiner Haut verdampfte, sobald er hinaus trat. Auf der anderen Seite der Tür befand sich ein größerer Raum, reich getäfelt und ausgestattet, mit einem großen goldenen Tisch, der sich quer darüber erstreckte; zwei riesige Lederstühle auf der einen Seite des Tisches und ein kleiner Holzschemel auf der anderen, der Stuhl des Schuldners. Zwei Kobolde in voller Rüstung, die verzierte Ohrbügel und Brillen trugen, standen im Raum Wache. Keine der beiden Seiten hatte Zauberstäbe oder andere magische Geräte bei sich, und die Koboldwachen würden sofort angreifen, wenn es jemand wagen würde, in diesem friedlichen, von der Gringotts Bank überwachten Treffen zauberstabfreie Magie einzusetzen. Die verzierten Ohrstöpsel würden die Koboldwachen daran hindern, das Gespräch zu hören, es sei denn, sie würden direkt angesprochen, und die Okulare würden die Gesichter der Zauberer verschwommen erscheinen lassen. Es war, kurz gesagt, so etwas wie eine echte Sicherheit, zumindest wenn man ein Okklumens war.

Harry kletterte auf seinen unbequemen Holzschemel, und wartete auf seine Gläubiger. Es war nur eine kurze Zeitspanne später, viel kürzer als die Zeit, die man einen Schuldner legal warten lassen konnte, als Lucius Malfoy den Raum betrat und mit geübten, geschmeidigen Bewegungen auf seinem Lederstuhl Platz nahm. Sein schlangenköpfiger Gehstock fehlte in seinen Händen, seine lange weiße Mähne wehte wie immer hinter ihm, sein Gesicht war nicht zu erkennen. Leise folgte ihm ein Junge mit weiß-blondem Haar, der jetzt schwarze Roben trug, die weitaus feiner waren als jede Hogwarts-Uniform, und der mit beherrschter Miene in die Fußstapfen seines Vaters trat. Ein Junge, der auch Harrys Gläubiger in Höhe von vierzig Galleonen war und außerdem aus dem Hause Malfoy stammte und daher technisch gesehen unter den Beschluss des Zaubergamot fiel, der dieses Treffen ermöglichte.

\emph{Draco.}

Harry sprach es nicht laut aus, ließ nicht zu, dass sich sein Gesichtsausdruck veränderte. Ihm fiel nicht ein, was er sagen sollte. Nicht einmal „Es tut mir leid“ schien angemessen. Harry hatte sich auch nicht getraut, etwas davon zu Dracos Patronus zu sagen, als sie in einem kurzen Austausch dieses Treffen arrangiert hatten; und das nicht nur, weil Lucius zuhören könnte. Es hatte gereicht, zu wissen, dass Dracos glücklicher Gedanke immer noch glücklich war, und dass er immer noch wollte, dass Harry es wusste.

Lucius Malfoy sprach zuerst, seine Stimme war ruhig, sein Gesicht gefasst. „Ich verstehe nicht, was in Hogwarts vor sich geht, Harry Potter. Würdest du es mir bitte erklären?“

„Ich weiß es nicht“, sagte Harry. „Wenn ich diese Ereignisse verstehen würde, hätte ich sie nicht geschehen lassen, Lord Malfoy.“

„Dann beantworte mir diese Frage. Wer bist du?!“

Harry blickte ruhig in das Gesicht seines Gläubigers.

„Ich bin nicht Du-weißt-schon-wer, wie du dachtest“, sagte Harry.

Da er kein kompletter Idiot war, hatte er schließlich herausgefunden, mit wem Lucius Malfoy vor dem Zaubergamot zu sprechen geglaubt hatte.

„Offensichtlich bin ich kein normaler Junge. Genauso offensichtlich hat das wahrscheinlich etwas mit der Junge-der-überlebte-Sache zu tun. Aber ich weiß nicht, was, oder warum, genauso wenig wie du. Ich habe den Sprechenden Hut gefragt und er wusste es auch nicht.“

Lucius Malfoy nickte distanziert.

„Ich wüsste keinen Grund, warum man hunderttausend Galleonen zahlen sollte, um das Leben eines Schlammbluts zu retten. Kein Grund außer einem, der deine Macht und deinen Blutdurst gleichermaßen erklären würde; aber dann starb sie durch die Hand eines Trolls, und du hast trotzdem überlebt. Und auch mein Sohn hat mir vieles von dir erzählt, Harry Potter, was nicht den geringsten Sinn ergab, ich habe das Toben der Verrückten in St. Mungo's gehört und sie waren bei weitem vernünftiger als die Ereignisse, die mein Sohn mir unter Veritaserum erzählt hat, die du inszeniert hast, und den Teil dieses Tobsuchtsanfalls, den du persönlich ausgeführt hast, möchte ich von dir erklärt haben, und zwar jetzt.“

Harry drehte sich um und schaute Draco an, der ihn mit einem Gesicht ansah, das sich verrenkte, kontrolliert wurde und sich dann wieder anspannte.

„Ich würde auch“, sagte Draco Malfoy mit hoher und schwankender Stimme, „gerne wissen, \emph{warum}, Potter.“

Harry schloss die Augen und sprach, ohne aufzusehen.

„Ein Junge, der von Muggeln aufgezogen wurde und sich für klug hielt. Du hast mich gesehen, Draco, und du hast daran gedacht, wie nützlich es wäre, wenn man dem Jungen-der-lebte unter all den anderen Kindern deines Jahrgangs die Wahrheit der Dinge zeigen könnte, wenn wir Freunde sein könnten. Und ich dachte dasselbe über dich. Nur glaubten wir beide an unterschiedliche Wahrheiten. Nicht, dass ich damit sagen will, dass es verschiedene Wahrheiten gibt, ich meine, es gibt verschiedene Überzeugungen, aber es gibt nur eine Realität, nur ein Universum, das diese Überzeugungen wahr oder falsch machen kann—“

„Du hast mich angelogen.“

Harry öffnete seine Augen und sah Draco an.

„Ich würde eher sagen“, sagte Harry, nicht ganz mit fester Stimme, „dass die Dinge, die ich dir erzählt habe, von einem bestimmten Standpunkt aus wahr waren.“

„Ein gewisser Standpunkt?!“

Draco Malfoy sah genauso wütend aus, wie Luke Skywalker das Recht dazu gehabt hätte, und er war auch nicht in der Stimmung, Obi Wan-Kenobis Entschuldigungen zu akzeptieren.

„Es gibt ein Wort für Dinge, die aus einer bestimmten Sichtweise heraus wahr sind. Man nennt sie Lügen!“

„Oder Tricks“, sagte Harry gleichmütig. „Aussagen, die technisch gesehen wahr sind, die aber den Zuhörer dazu verleiten, weitere Überzeugungen zu bilden, die falsch sind. Ich denke, es lohnt sich, diese Unterscheidung zu treffen. Was ich dir gesagt habe, war eine selbsterfüllende Prophezeiung; Du hast geglaubt, dass du dich nicht selbst täuschen kannst, also hast du es nicht versucht. Die Fähigkeiten, die du erlernt hast, sind real, und es wäre sehr schlecht für dich gewesen, innerlich gegen sie anzukämpfen. Menschen können sich nicht durch einen Willensakt dazu bringen, zu glauben, dass blau grün ist, aber sie denken, dass sie es können, und das kann fast genauso schlimm sein.“

„Du hast mich benutzt“, sagte Draco Malfoy.

„Ich habe dich nur auf eine Weise benutzt, die dich stärker gemacht hat. Das ist es, was es bedeutet, von einem Freund benutzt zu werden.“

„Selbst ich weiß, dass das nicht das ist, was Freundschaft ausmacht!“

Jetzt ergriff Lucius Malfoy wieder das Wort. „Zu welchem Zweck? Zu welchem Zweck?!“ Selbst die Stimme des älteren Malfoy war nicht ganz fest. „\emph{Warum}?“

Harry betrachtete ihn einen Moment lang und wandte sich dann an Draco.

„Dein Vater wird das wahrscheinlich nicht glauben“, sagte Harry. „Aber du, Draco, solltest in der Lage sein zu erkennen, dass alles, was geschehen ist, mit dieser Hypothese vereinbar ist. Und dass jede zynischere Hypothese nicht erklären würde, warum ich dich nicht stärker bedrängt habe, als du dachtest, ich hätte ein Druckmittel, oder warum ich dir so viel beigebracht habe. Ich dachte, dass der Erbe des Hauses Malfoy, der öffentlich gesehen wurde, wie er ein Muggelgeborenes Mädchen gepackt hat, um zu verhindern, dass sie vom Dach von Hogwarts fällt, ein guter Kompromisskandidat wäre, um das magische Großbritannien nach der Reformation zu regieren.“

„Du willst mir also weismachen“, sagte Lucius Malfoy mit dünner Stimme, „dass du behauptest, verrückt zu sein. Nun, lassen wir das alles beiseite. Sag mir, wer den Troll auf Hogwarts angesetzt hat.“ „Ich weiß es nicht“, sagte Harry.

„Sag mir, wen du verdächtigst, Harry Potter.“

„Ich habe vier Verdächtige. Einer von ihnen ist Professor Snape—“

„Snape?“ platzte Draco heraus.

„Der zweite ist natürlich der Verteidigungsprofessor von Hogwarts, einfach weil er der Verteidigungsprofessor ist.“ Harry hätte ihn ausgelassen, weil er die Malfoys nicht auf Professor Quirrell aufmerksam machen wollte, wenn er unschuldig war, aber Draco hätte ihn vielleicht darauf angesprochen.

„Das dritte würdest du mir nicht glauben. Das Vierte ist eine Auffangkategorie namens Alles andere.“

\emph{Und den fünften, Lord Voldemort, sollte ich dir wohl nicht nennen.}

Lucius Malfoys Gesicht verzerrte sich zu einem Knurren.

„Glaubst du, ich erkenne den Köder an deinem Haken nicht? Erzähl mir von der dritten Möglichkeit, Potter, von der du willst, dass ich glaube, sie sei die wahre Antwort, und lass die Spielchen.“

Harry betrachtete Lord Malfoy mit festem Blick.

„Ich habe einmal ein Buch gelesen, das ich nicht lesen sollte, und es sagte mir Folgendes: Kommunikation ist ein Vorgang, der unter Gleichen stattfindet. Angestellte belügen ihre Chefs, die ihrerseits erwarten, belogen zu werden. Ich ziere mich nicht, ich stelle nur fest, dass es mir in unserer gegenwärtigen Situation einfach nicht möglich ist, dir von dem dritten Verdächtigen zu erzählen und dich glauben zu lassen, dass meine Geschichte etwas anderes als ein Köder war.“

Dann ergriff Draco das Wort.

„Es ist Vater, nicht wahr?“

Harry warf Draco einen erschrockenen Blick zu. Draco sprach gleichmütig.

„Du vermutest, dass Vater den Troll nach Hogwarts geschickt hat, um an Granger heranzukommen, nicht wahr? Das denkst du doch, oder!“

Harry öffnete den Mund, um zu sagen: „\emph{Eigentlich nicht}“, und schaffte es dann, sich ausnahmsweise einmal in seinem Leben zu beherrschen.

„Ich verstehe…“ sagte Harry langsam. „Darum geht es also. Lucius Malfoy sagt öffentlich, dass Hermine nicht damit durchkommt, was sie getan hat, und siehe da, ein Troll tötet sie.“ Harry lächelte dann, auf eine Art, die seine Zähne entblößte. „Und wenn ich das hier leugne, dann kann Draco, der kein Okklumentiker ist, unter Veritaserum bezeugen, dass der Junge-der-lebte nicht Lucius Malfoy verdächtigt, einen Troll nach Hogwarts geschickt zu haben, um Hermine Granger zu töten, die auf das edle Haus Potter vereidigt ist, deren Blutschuld kürzlich für hunderttausend Galleonen erkauft wurde und so weiter.“

Harry lehnte sich leicht zurück, obwohl sein Holzschemel keine Lehne hatte, mit der er das hätte tun können.

„Aber jetzt, wo es aufgezeigt wurde, sehe ich, dass es sehr vernünftig ist. Offensichtlich warst du es, der Hermine Granger getötet hat, so wie du es vor dem gesamten Zaubergamot angedroht hast.“

„Das habe ich nicht“, sagte Lucius Malfoy, wieder ausdruckslos.

Harry fletschte wieder seine Zähne in diesem Nicht-Lächeln.

„Nun denn, in diesem Fall muss es da draußen noch jemanden geben, der Hermine getötet und die Hogwarts-Zauber durcheinander gebracht hat, dieselbe Person, die zuvor versucht hat, Hermine den Mord an Draco Malfoy anzuhängen. Entweder hast du Hermine Granger getötet, nachdem du für ihr Leben bezahlt wurdest, oder du hast den Mordversuch deines Sohnes einem unschuldigen Mädchen in die Schuhe geschoben und das ganze Geld meiner Familie unter falschem Vorwand genommen, eines dieser beiden Dinge muss wahr sein.“

„Vielleicht hast \emph{du} sie in der Hoffnung getötet, dass du dein Geld zurückbekommst.“ Lucius Malfoy hatte sich nach vorne gelehnt und starrte Harry fest an.

„Dann hätte ich mein Geld gar nicht erst für sie hergegeben. Wie du bereits weißt. Beleidige nicht meine Intelligenz, Lord Malfoy - nein, warte, Entschuldigung, das musstest du gerade sagen, für den Fall, dass Draco es bezeugen muss, schon gut.“

Lucius Malfoy lehnte sich in seinem Stuhl zurück und starrte.

„Ich habe versucht, es dir zu sagen, Vater“, sagte Draco unter seinem Atem, „aber niemand kann sich ein Bild von Harry Potter machen, bevor er ihn nicht tatsächlich getroffen hat…“

Harry tippte mit einem Finger auf seine Wange. „Die Leute fangen also an, das Offensichtliche zu begreifen? Ich bin ehrlich gesagt überrascht. Ich hätte nicht gedacht, dass das passiert.“

Harry hatte inzwischen den allgemeinen Rhythmus von Professor Quirrells Zynismus und Sprechweise erfasst und war in der Lage, es recht gut nachzuahmen.

"Ich hätte nicht gedacht, dass eine Zeitung über ein Konzept wie '\emph{Entweder X oder Y muss wahr sein, aber wir wissen nicht, welches.}' berichten kann. Ich würde nur erwarten, dass Journalisten über Geschichten berichten, die aus einer Reihe von atomaren Propositionen bestehen, wie '\emph{X ist wahr}', '\emph{Y ist falsch}', oder '\emph{X ist wahr und Y ist falsch}'. Nicht aus komplexeren logischen Verbindungen wie '\emph{Wenn X wahr ist, dann ist Y wahr, aber wir wissen nicht, ob X wahr ist}'. Und alle deine Anhänger sollten schnell zwischen '\emph{Man kann nicht beweisen, dass Lord Malfoy Granger getötet hat, es könnte auch jemand anderes gewesen sein}' und '\emph{Man kann nicht beweisen, dass es jemand anderes war, der Granger die Schuld zugeschoben hat}' hin und her wechseln, solange es unsicher ist, sollten sie versuchen, beides auf einmal zu haben …

Moment, gehört Ihnen nicht der Tagesprophet?"

„Der Tagesprophet“, sagte Lucius Malfoy dünn, „der mir gewiss nicht gehört, ist viel zu seriös, um solch einen niederträchtigen Unsinn zu veröffentlichen. Leider sind nicht alle Zauberer mit Einfluss so vernünftig.“

„Aha. Verstehe.“ Harry nickte.

Lucius warf einen Blick auf Draco. „Der Rest von dem, was er gesagt hat - war irgendetwas davon wichtig?“

„Nein, Vater, war es nicht.“

„Danke, mein Sohn.“ Lucius richtete seinen Blick wieder auf Harry. Seine Stimme, als er sprach, war etwas näher an seinem üblichen Tonfall, kühl und selbstbewusst. „Es ist möglich, dass ich mich dazu überreden ließe, dir einen Gefallen zu tun, wenn du vor dem Zaubergamot zugibst, was du offensichtlich weißt, dass ich für diese Tat nicht verantwortlich war. Ich wäre bereit, deine Restschuld gegenüber dem Haus Malfoy ganz erheblich zu reduzieren oder sogar die Bedingungen für eine spätere Rückzahlung anzupassen.“

Harry betrachtete Lucius Malfoy mit festem Blick.

„Lucius Malfoy. Du weißt jetzt ganz genau, dass Hermine Granger in Wirklichkeit hereingelegt wurde, indem man deinen Sohn als Köder benutzte, dass sie mit einem falschen Gedächtniszauber oder Schlimmerem belegt wurde und dass das Haus Potter dir vorher nichts vorwerfen konnte. Mein Gegenvorschlag ist, dass du das ganze Geld meiner Familie zurückgibst, ich verkünde vor dem Zaubergamot, dass Haus Potter nichts gegen Haus Malfoy hat, und wir präsentieren eine vereinte Front gegen denjenigen, der das tut. Wir beschließen, unsere Rollen zu vergessen und uns zu verbünden, statt zu kämpfen. Es könnte die eine Sache sein, die der Feind nicht von uns erwartet.“

Es herrschte eine kurze Stille im Raum, bis auf die beiden Goblinwachen, die trotzdem weiteratmeten.

„Du bist verrückt“, sagte Lucius Malfoy kalt.

„Das nennt man Gerechtigkeit, Lord Malfoy. Du kannst unmöglich erwarten, dass ich mit dir kooperiere, während du den Reichtum des Hauses Potter unter einem, wie du jetzt weißt, falschen Vorwand an dich bringst. Ich verstehe, wie es für dich damals aussah, aber jetzt weißt du es besser.“

„Du hast mir nichts zu bieten, was 100.000 Galleonen wert ist.“

„Habe ich nicht?“ sagte Harry distanziert. „Ich frage mich ob das stimmt. Ich halte es für ziemlich wahrscheinlich, dass dir das langfristige Wohlergehen des Hauses Malfoy mehr am Herzen liegt als die Frage, welches politische Thema der gescheiterte Dunkle Lord der letzten Generation zu seinem persönlichen Hobby gemacht hat.“

Harry warf einen bedeutungsvollen Blick auf Draco.

„Die nächste Generation zieht ihre eigenen Kampflinien und bildet neue Allianzen. Dein Sohn kann davon ausgeschlossen werden, oder er kann sich direkt an die Spitze setzen. Ist dir das mehr wert als vierzigtausend Galleonen, die du nicht wirklich erwartet hast und nicht besonders brauchst?“ Harry lächelte dünn. „Vierzigtausend Galleonen. Zwei Millionen Muggelpfund Sterling. Dein Sohn weiß ein paar Dinge über die Größe der Muggelwirtschaft, die dich überraschen würden. Muggel würden es lächerlich finden, dass sich das Schicksal eines Landes um zwei Millionen Pfund Sterling entscheidet. Sie würden es niedlich finden. Und ich sehe das ähnlich, Lord Malfoy. Es geht nicht darum, dass ich verzweifelt bin. Es geht darum, dass du eine faire Chance bekommst, fair zu sein.“

„Oh?“, sagte Lord Malfoy. „Und wenn ich deine faire Chance ablehne, was dann?“

Harry zuckte mit den Schultern.

„Kommt darauf an, was für eine Koalitionsregierung ohne die Malfoys zustande kommt. Wenn die Regierung friedlich reformiert werden kann und es den Frieden stören würde, es anders zu machen, dann zahle ich dir das Geld aus der Portokasse. Oder vielleicht werden manche Todesser wegen vergangener Verbrechen erneut angeklagt und im Sinne der Gerechtigkeit hingerichtet, natürlich nach einem ordentlichen Gerichtsverfahren.“

„Du bist wirklich verrückt“, sagte Lucius Malfoy leise. „Du hast keine Macht, keinen Reichtum, und doch sagst du solche Dinge zu mir.“

„Ja, es ist dumm zu glauben, ich könnte dir Angst einjagen. Immerhin bist du kein Dementor.“ Und Harry lächelte weiter.

\emph{Er hatte nachgeschaut, und anscheinend würde ein Bezoar fast jedes Gift heilen, wenn man ihn jemandem schnell genug in den Mund steckte. Vielleicht würde das nicht den Strahlenschaden von verwandeltem Polonium reparieren, aber andererseits vielleicht doch. Harry hatte also die Gefrierpunkte verschiedener Säuren nachgeschlagen und es stellte sich heraus, dass Schwefelsäure bei nur zehn Grad Celsius gefriert, was bedeutete, dass Harry einen Liter Säure bei Muggeln kaufen, sie fest gefrieren lassen und zu einem winzig kleinem, unauffälligen Wassereis-Tropfen verwandeln konnte, den man jemandem in den Mund schweben oder, unsichtbar versteckt, trinken lassen konnte. Kein Bezoar würde das kompensieren, sobald die Verwandlung nachliese.}

Harry hatte natürlich nicht die Absicht, es laut auszusprechen, aber jetzt, wo er entscheidend versagt hatte, irgendwelche Todesfälle während seiner Suche zu verhindern, hatte er nicht mehr die Absicht, sich durch das Gesetz oder gar den Kodex von Batman zurückhalten zu lassen.

\emph{Dies ist deine letzte Chance zu leben, Lucius.}

\emph{Ethisch gesehen wurde dein Leben an dem Tag gekauft und bezahlt, an dem du deine erste Gräueltat für die Todesser begangen hast. Du bist immer noch ein Mensch und dein Leben hat immer noch einen Wert, aber du hast nicht mehr den deontologischen Schutz eines Unschuldigen. Jeder gute Mensch hat die Lizenz, dich jetzt zu töten, wenn er glaubt, dass es auf lange Sicht Leben rettet; und ich werde genauso von dir denken, wenn du anfängst, mir in die Quere zu kommen.}

\emph{Wer auch immer den Troll auf Granger angesetzt hat, muss es auch auf dich abgesehen haben und dich mit irgendeinem Fluch belegt haben, der ehemalige Todesser zu einem Haufen Glibber schmelzen lässt. Sehr traurig.}

„Vater“, sagte Draco mit leiser Stimme. „Ich denke, du solltest es in Betracht ziehen, Vater.“

Lucius Malfoy sah seinen Sohn an.

„Du scherzt.“

„Es ist wahr. Ich glaube nicht, dass Potter seine Bücher einfach nur erfunden hat, das kann niemand allein geschrieben haben, und es stehen Dinge darin, die ich selbst nachprüfen könnte. Und wenn auch nur die Hälfte von all dem wahr ist, hat er recht, dann bedeuten hunderttausend Galleonen nicht viel. Wenn wir sie ihm geben, wird er wirklich wieder mit dem Haus Malfoy befreundet sein - jedenfalls so, wie er es sich vorstellt, befreundet zu sein. Und wenn wir es nicht tun, wird er dein Feind sein, ob es nun in seinem eigenen Interesse ist oder nicht, er wird dich einfach verfolgen. Harry Potter denkt wirklich so. Es geht ihm nicht um Geld, es geht um das, was er für Ehre hält.“

Harry Potter legte den Kopf schief, immer noch lächelnd.

„Aber lass uns eines klarstellen“, sagte Draco, der ihn nun direkt anstarrte. In seinen Augen lag ein grimmiges Licht. „Du hast mir Unrecht getan. Und du bist mir etwas schuldig.“

„Ja“, sagte Harry leise. „Natürlich unter der Bedingung, dass wir zu einer Einigung kommen.“

Lucius Malfoy öffnete den Mund, um wer-weiß-was zu sagen, und schloss ihn dann wieder. „Verrückt“, sagte er wieder.

Es gab einen langen Vater-Sohn-Streit, während dessen Harry es schaffte, seinen Mund zu halten. Als es schien, dass selbst Draco nicht in der Lage sein würde, seinen Vater zu überreden, meldete sich Harry wieder zu Wort und schlug seine beabsichtigten nächsten Schritte vor, falls die Häuser Potter und Malfoy kooperieren könnten. Dann kam es zu einem weiteren Streit zwischen Lucius und Draco, bei dem Harry wieder schwieg.

Schließlich wandte sich Lucius Malfoys Blick zu Harry.

„Und du glaubst“, sagte Lucius Malfoy, „dass du Longbottom und Bones davon überzeugen kannst, sich dieser Idee anzuschließen, auch wenn Dumbledore dagegen ist.“

Harry nickte.

„Sie werden natürlich misstrauisch sein, was deine Beteiligung angeht. Aber ich werde ihnen sagen, dass es von Anfang an mein Plan war, und das sollte helfen.“

„Ich nehme an“, sagte Lucius Malfoy nach einer Pause, „dass ich einen Vertrag aufsetzen lassen könnte, der dich von fast allen restlichen Schulden befreit, falls ich zufällig auf diese verrückte Idee komme. Es bedarf natürlich weiterer Garantien—“

Harry griff prompt in seinen Umhang und holte ein Pergament hervor, das er entfaltete und auf dem goldenen Tisch ausbreitete.

„Eigentlich habe ich mir das selbst erlaubt“, sagte Harry.

Er hatte einige sorgfältige Stunden in der Hogwarts-Bibliothek mit den verfügbaren Gesetzesbüchern verbracht. Zum Glück waren die Gesetze des magischen Britanniens, soweit Harry das beurteilen konnte, für Muggelverhältnisse bestechend einfach. Zu schreiben, dass die ursprüngliche Blutschuld und die Zahlung annulliert, das Vermögen der Potters und alle anderen Tresorgegenstände zurückgegeben und die Restschuld annulliert würde, alles ohne Schuld der Malfoys, waren nur ein paar Zeilen mehr, als es brauchte, um sie laut auszusprechen.

„Ich musste meinen Begleitern versprechen, nichts zu unterschreiben, was du mir gibst. Also habe ich es selbst verfasst und unterschrieben, bevor ich gegangen bin.“

Draco stieß ein ersticktes Lachen aus.

Lucius las sich den Vertrag durch und lächelte humorlos.

„Wie charmant direkt.“

„Ich habe auch versprochen, keinen Federkiel anzufassen, solange ich in Gringotts bin“, sagte Harry. Er griff wieder in seinen Umhang und holte einen Muggelstift heraus, zusammen mit einem Blatt normalen Papiers. „Wird diese Formulierung in Ordnung sein?“

Harry kritzelte schnell eine juristisch klingende Erklärung hin, die besagte, dass das Haus Potter das Haus Malfoy in keiner Weise für den Mord an Hermine Granger verantwortlich machte und auch nicht glaubte, dass sie etwas damit zu tun hatten, und hielt dann das Papier in die Luft, damit Lord Malfoy es prüfen konnte.

Lord Malfoy sah sich das Papier an, rollte leicht mit den Augen und sagte: „Gut genug, nehme ich an. Allerdings solltest du, um die richtige Bedeutung zu haben, den juristischen Begriff “\emph{entschädigen}„ verwenden und nicht “\emph{entlasten}„—“

„Netter Versuch, aber nein. Ich weiß genau, was dieses Wort bedeutet, Lord Malfoy.“ Harry nahm sein Pergament und begann, seinen ursprünglichen Wortlaut sorgfältiger abzuschreiben.

Als Harry fertig war, griff Lord Malfoy über den goldenen Tisch, nahm den Stift und betrachtete ihn nachdenklich. „Eines deiner Muggel-Artefakte, nehme ich an? Wozu ist das gut, mein Sohn?“

„Er schreibt, ohne dass man ein Tintenfass braucht“, antwortete Draco.

„Das kann ich sehen. Ich nehme an, manche finden es ein amüsantes Schmuckstück.“ Lucius glättete den Pergamentvertrag auf dem Tisch, dann legte er seine Hand an die Unterschriftenzeile und tippte nachdenklich mit der Feder auf den Startpunkt. Harry riss seinen Blick weg, hinauf zu Lucius Malfoys Gesicht, und zwang sich, regelmäßig zu atmen, wobei er nicht ganz verhindern konnte, dass sich seine Muskeln anspannten.

„Unser guter Freund Severus Snape“, sagte Lucius Malfoy und tippte immer noch mit dem Stift auf die Zeile, die auf seine Unterschrift wartete. „Der Verteidigungsprofessor, der sich Quirrell nennt. Nun frage ich noch einmal, wer ist dein dritter Verdächtiger, Harry Potter?“

„Ich würde dir dringend raten, zuerst zu unterschreiben, Lord Malfoy, wenn du es ohnehin tun willst. Du wirst mehr von dieser Information profitieren, wenn du nicht denkst, dass ich versuche dich zu etwas zu überreden.“

Wieder ein humorloses Lächeln.

„Ich werde das Risiko eingehen. Sprich, wenn du willst, dass das hier weitergeht.“

Harry zögerte, dann sagte er gleichmäßig: „Mein dritter Verdächtiger ist Albus Dumbledore.“

Die klopfende Feder verstummte auf dem Pergament.

„Eine seltsame Anschuldigung“, murmelte Lucius. „Dumbledore hat viel Gesicht verloren, als ein Hogwarts-Schüler während seiner Amtszeit starb. Glaubst du etwa, dass ich dir alles glauben werde, nur weil er mein Feind ist?“

„Er ist ein Verdächtiger unter mehreren, Lord Malfoy, und nicht unbedingt der plausibelste. Aber der Grund, warum ich in der Lage war, einen ausgewachsenen Bergtroll zu töten, war, dass ich eine Waffe hatte, die Dumbledore mir zu Beginn des Schuljahres gegeben hat. Das ist kein starker Beweis, aber es ist verdächtig. Und wenn du denkst, dass es nicht Dumbledores Stil ist, einen seiner Schüler zu ermorden, nun, derselbe Gedanke kam mir auch schon.“

„Es ist nicht sein Stil?“ Sagte Draco Malfoy.

Lucius Malfoy schüttelte den Kopf in einer gemessenen, vorsichtigen Bewegung. „Nicht ganz, mein Sohn. Dumbledore ist wählerisch, was seine Bösartigkeiten angeht.“ Lord Malfoy lehnte sich in seinem Stuhl zurück und saß dann ganz still. „Erzähl mir von dieser Waffe.“

„Ich bin mir noch nicht sicher, ob ich in deiner Gegenwart darüber ins Detail gehen sollte, Lord Malfoy.“

Harry holte tief Luft.

„Lass mich das klarstellen. Ich versuche nicht, dir die Idee zu verkaufen, dass Dumbledore dahinter steckt, ich spreche nur die Möglichkeit an—“

Dann ergriff Draco Malfoy das Wort.

„Die Waffe oder das Gerät, das Dumbledore dir gegeben hat - war das etwas, um Trolle zu töten? Ich meine, nur Trolle? Kannst uns das sagen?“

Lucius drehte den Kopf und sah seinen Sohn etwas überrascht an.

„Nein…“ sagte Harry langsam. „Es war nicht speziell ein Schwert zur Trollbekämpfung oder etwas in der Art.“

Dracos Augen waren aufmerksam.

„Hätte das Gerät gegen einen Attentäter funktioniert?“

\emph{Nicht, wenn sie ihre Schilde erhoben hätten.}

„Nein.“

„Eine Schlägerei in der Schule?“

\emph{Ein expandierender Stein in der Kehle ist von Natur aus tödlich.}

„Nein. Ich glaube nicht, dass es für den Einsatz gegen Menschen gedacht war.“

Draco nickte.

„Also nur gegen magische Kreaturen. Wäre es eine gute Waffe gegen einen wütenden Hippogreif oder etwas in der Art gewesen?“

„Funktioniert der Betäubungsfluch bei Hippogreifs?“ sagte Harry langsam.

„Ich weiß es nicht“, sagte Draco.

„Ja“, sagte Lucius Malfoy.

\emph{Dann wäre ein Betäubungsfluch im Vergleich zu einem Wingardium Leviosa und Finite Incantatem die bessere Methode, um mit einem Hippogreif fertig zu werden.}

\emph{So gesehen schien es immer mehr so, als sei ein Verwandelter Stein nur gegen ein magisches Wesen aus Fleisch und Blut mit zauberresistenter Haut eine optimale Waffe.}

„Dann nein. Aber… Ich meine, es könnte gar nicht als Waffe gedacht gewesen sein, ich habe es auf eine seltsame Art und Weise benutzt, es könnte einfach eine verrückte Laune vom Schulleiter gewesen sein—“

„Nein“, sagte Lucius Malfoy leise. „Nicht eine Laune. Keine Laune. Nicht von Dumbledore.“

„Dann ist er es“, sagte Draco.

Langsam verengten sich Dracos Augen, und er nickte bösartig.

„Er war es von Anfang an. Die Hof-Legilimens sagten, dass jemand Legilimenz bei Granger eingesetzt hatte. Dumbledore hat zugegeben, dass er es war. Und ich wette, die Schutzzauber gingen los, als Granger fluchte und Dumbledore sie einfach ignorierte.“

„Aber—“ sagte Harry. Er sah Lucius an und fragte sich, ob es wirklich zu seinem Vorteil war, diese Idee in Frage zu stellen. „Was wäre sein Motiv? Sagen wir, er ist böse und belassen es einfach dabei?“

Draco Malfoy sprang von seinem Stuhl auf und begann, im Raum auf und ab zu gehen, schwarze Roben rauschten hinter dem Jungen her, die Koboldwächter starrten ihn durch ihre verzauberten Brillen etwas verwundert an.

„Um eine seltsame Verschwörung zu durchschauen, schau dir an, was passiert, und frag dann, wer davon profitiert. Nur hat Dumbledore nicht geplant, dass du versuchst, Granger bei ihrer Verhandlung zu retten, sondern er hat versucht, dich davon abzuhalten. Was wäre passiert, wenn Granger nach Askaban gegangen wäre? Das Haus Malfoy und das Haus Potter hätten sich für immer gehasst. Von allen Verdächtigen ist der Einzige, der das will, Dumbledore. Es passt also. Es passt alles. Derjenige, der den Mord wirklich begangen hat, ist - Albus Dumbledore!“

„Ähm“, sagte Harry. „Aber warum eine Anti-Troll-Waffe? Ich habe gesagt, dass es verdächtig ist, ich habe nicht gesagt, dass es einen Sinn ergibt.“

Draco nickte nachdenklich.

„Vielleicht dachte Dumbledore, du würdest den Troll aufhalten, bevor er Granger erwischt, und dann könnte er Vater die Schuld dafür geben, dass er ihn geschickt hat. Viele Leute wären sehr wütend, wenn sie dächten, Vater hätte so etwas auch nur versucht, in Hogwarts zu tun. Wie Vater sagte, hätte Dumbledore sicher sein Gesicht verloren, wenn die Leute herausgefunden hätten, dass tatsächlich ein Schüler in Hogwarts gestorben war, denn Hogwarts ist für seine Sicherheit bekannt. Also sollte dieser Teil wahrscheinlich nicht passieren.“

Harrys Gedanken blitzten unwillkürlich zurück zu dem Entsetzen in Dumbledores Augen, als er Hermine Grangers Leiche gesehen hatte.

\emph{Wäre ich rechtzeitig dort gewesen, wenn den Weasley-Zwillingen nicht die Zauberkarte gestohlen worden wäre? Könnte das der Plan gewesen sein?}

\emph{Und dann, ohne dass Dumbledore es wusste, hat jemand die Karte gestohlen und ich kam zu spät.}

\emph{Aber nein, das ergibt keinen Sinn. Ich hab's zu spät erfahren.}

\emph{Wie konnte Dumbledore ahnen, dass ich einen Besen benutze?}

\emph{Er wusste, dass ich einen habe.}

\emph{Es war unmöglich, dass so ein Plan funktioniert. Und das hat er auch nicht.}

\emph{Aber jemand, der ein bisschen senil wird, könnte erwarten, dass es funktioniert.}

\emph{Und ein Phönix könnte den Unterschied nicht erkennen.}

„Oder“, fuhr Draco Malfoy fort, immer noch energisch auf und ab gehend, „vielleicht hatte Dumbledore einen verzauberten Troll in der Nähe, von dem er erwartete, dass du ihn ein anderes Mal besiegst, für irgendeinen anderen Plan, und dann hat er den Troll stattdessen gegen Granger eingesetzt. Ich kann mir nicht vorstellen, dass Dumbledore das alles seit der ersten Unterrichtswoche geplant hat—“

„Ich kann es mir vorstellen“, sagte Lucius Malfoy in leisem Ton. „Das habe ich bei Dumbledore schon gesehen und selbst erlebt.“

Draco nickte entschlossen.

„Dann hätte ich bei der ersten Handlung nie sterben dürfen. Dumbledore wusste, dass Professor Quirrell nach mir suchte, oder Dumbledore plante, dass jemand anderes mich rechtzeitig finden sollte - ich hätte nicht gegen Granger aussagen können, wenn ich tot wäre, und er hätte sein Gesicht verloren, wenn ich gestorben wäre. Aber dass ich Hogwarts verlasse und nicht mehr da bin, um Slytherin anzuführen, würde ihm gerade recht kommen. Und beim nächsten Mal sollte Harry den Troll aufhalten, bevor er Granger erwischt und alle sollten dir die Schuld geben, Vater, nur lief es diesmal nicht so, wie Dumbledore es geplant hatte.“

Lucius Malfoy hob seine grauen Augen, von wo aus er seinen Sohn mit offenem Erstaunen anstarrte. „Wenn das wahr ist - aber ich frage mich, ob Harry Potter nur so tut, als würde er es nicht glauben wollen.“

„Vielleicht“, sagte Draco. „Aber ich bin mir ziemlich sicher, dass er es nicht tut.“

„Dann, wenn es wahr ist…“

Lucius Malfoys Stimme verstummte. Eine langsame Wut leuchtete in seinen Augen auf.

„Was würden wir dann genau tun?“ sagte Harry.

„Auch das ist mir klar“, sagte Draco. Er wirbelte auf sie zu und hob einen Finger hoch in die Luft. „Wir werden die Beweise finden, um Dumbledore dieses Verbrechens zu überführen und ihn vor Gericht zu bringen!“

Harry Potter und Lucius Malfoy sahen sich an. Keiner von beiden wusste so recht, was er sagen sollte.

„Mein Sohn“, sagte Lucius Malfoy nach einer Weile, „das hast du heute wirklich sehr gut gemacht.“

„Ich danke dir, Vater!“

„Aber das hier ist kein Spiel, wir sind keine Auroren, und wir verlassen uns nicht auf Gerichte.“

Etwas von dem Licht verschwand aus Dracos Augen. „Oh.“

„Ich, äh, habe eine sentimentale Abneigung gegen Gerichte“, warf Harry ein.

\emph{Ich kann nicht glauben, dass ich dieses Gespräch führe. Er sollte nach Hause gehen, sich ein Blatt Papier und einen Stift nehmen und versuchen, herauszufinden, ob Dracos Argumentation tatsächlich Sinn ergab.}

„Und keine Beweise.“

Da wandte Lucius Malfoy seinen Blick zu Harry Potter, und seine Augen glühten in purer grauer Wut. „Wenn du mich getäuscht hast“, sagte Lucius Malfoy in einem Tonfall leiser Wut, „wenn das alles eine Lüge ist, dann werde ich dir nicht verzeihen. Aber wenn dies keine Täuschung ist… Bring mir den Beweis, um Dumbledore vor dem Zaubergamot des Mordes zu überführen, oder Beweise, die ausreichen, um ihn zu stürzen, und es gibt nichts, was das Haus Malfoy nicht für dich tun würde, Harry Potter. \emph{Nichts}.“

Harry nahm einen tiefen Atemzug. Er musste das alles sortieren und die tatsächlichen Wahrscheinlichkeiten ausrechnen, aber dafür hatte er keine Zeit.

„Wenn es Dumbledore ist, dann hinterlässt seine Entfernung vom Spielbrett ein riesiges Loch in der Machtstruktur Großbritanniens.“

„So ist es“, sagte Lucius Malfoy mit einem grimmigen Lächeln. „Hattest du Ambitionen, es selbst zu füllen, Harry Potter?“

„Einigen deiner Gegner könnte das nicht gefallen. Sie könnten kämpfen.“

„Sie werden verlieren“, sagte Lucius Malfoy, jetzt mit einem Gesicht hart wie Eisen.

„Das ist es also, was das Haus Malfoy für mich tun soll, Lord Malfoy, wenn Dumbledore meinetwegen abgesetzt wird. Wenn die Opposition am ängstlichsten ist - dann wird man ihnen in letzter Minute ein Arrangement anbieten, um einen Bürgerkrieg zu vermeiden. Einigen deiner Verbündeten wird das nicht gefallen, aber viele Neutrale werden froh sein, Stabilität zu sehen. Die Abmachung wird sein, dass nicht du sofort die Macht übernimmst, sondern Draco Malfoy, wenn er volljährig ist.“

„Was?!“ sagte Draco.

„Draco hat unter Veritaserum ausgesagt, dass er versucht hat, Hermine Granger zu helfen. Ich wette, es gäbe eine Menge Leute in der Opposition, die es eher auf ihn ankommen lassen würden, als zu kämpfen. Ich bin mir nicht sicher, wie genau du das durchsetzen würdest - unbrechbare Schwüre oder Gringotts-Verträge oder was auch immer - aber es wird eine Art von durchsetzbarem Vertrag geben, dass die Macht an Draco geht, nachdem er Hogwarts abgeschlossen hat. Ich werde als Jungen-der-überlebte und so weiter, dich bei dieser Abmachung unterstützen. Versuchen, Longbottom und Bones zu überreden und so weiter. Unser erster Plan ebnet den Weg für später, wenn du diesmal darauf achtest, dich bei Longbottom und Bones ehrenhaft zu verhalten.“

„Vater, ich schwöre, ich habe nicht—“

Lucius' Gesicht verzog sich zu einem grimmigen Lächeln.

„Ich weiß, dass du das nicht beabsichtigt hast, mein Sohn. Nun.“

Der weißhaarige Mann starrte über den mächtigen goldenen Tisch hinweg auf Harry Potter. „Diese Bedingungen sind für mich akzeptabel. Aber wenn du irgendeinen Teil unserer Vereinbarung nicht einhältst, sei es die erste oder die zweite, dann wird das Konsequenzen für dich haben, Harry Potter. Kluge Worte werden das nicht aufhalten.“ Und Lucius Malfoy unterschrieb das Pergament.

…

Mad-Eye Moody starrte schon seit gefühlten Stunden auf die Bronzetür des Gringotts-Besprechungsraums, sofern ein Mann überhaupt auf eine Sache starren konnte, wenn sein Blick immer in alle Richtungen sah.

\emph{Das Problem bei dem Versuch, einem Mann wie Lucius Malfoy misstrauisch gegenüberzustehen,} dachte Moody, war, \emph{dass man einen ganzen Tag damit verbringen konnte, über alles nachzudenken, was er vorhaben könnte, und trotzdem nicht fertig wurde.}

Die Tür ging auf und Harry Potter stapfte heraus, kleine Schweißperlen noch auf der Stirn.

„Hast du etwas unterschrieben?“ verlangte Mad-Eye augenblicklich. Harry Potter sah ihn stumm an, dann griff er in seinen Umhang und zog ein gefaltetes Pergament heraus. „Die Kobolde sind schon dabei, es auszuführen“, sagte Harry Potter. „Sie haben drei Kopien gemacht, bevor ich gegangen bin.“

„MERLIN, VERDAMMT NOCH MAL, MEIN SOHN—“

Moody hielt inne, als sein Auge die zweite Hälfte des Dokuments erblickte, während Harry Potter langsam, wie widerwillig, begann, den oberen Teil nach oben zu entfalten. Ein Blick genügte, um die in sorgfältiger Handschrift gezeichneten Absätze zu erfassen, Lucius Malfoys elegante Unterschrift unter Harry Potters. Und dann explodierte Moody, gerade als auch die obere Hälfte des Dokuments in sein Blickfeld geriet.

„Du entlastest das Haus Malfoy von jeglicher Beteiligung am Tod von Hermine Granger? Hast du eine Ahnung, was du getan hast du verdammter Idiot? Warum in Merlins Namen solltest du so etwas tun, -WAS STEHT DA AUF DER ANDEREN SEITE?!“

