

\hypertarget{informationen-aus-der-vergangenheit}{% \section{49. Informationen aus der Vergangenheit}\label{informationen-aus-der-vergangenheit}}

\textbf{\uline{Informationen aus der Vergangenheit}}

Ein Junge wartet auf einer kleinen Lichtung am Rande des nicht verbotenen Waldes, neben einem Feldweg, der in einer Richtung zu den Toren von Hogwarts zurückführt und in der anderen in die Ferne. In der Nähe steht eine Kutsche, und der Junge steht weit davon entfernt und schaut sie an, wobei sein Blick selten von ihr abweicht.

In der Ferne nähert sich eine Gestalt auf dem Feldweg: Ein Mann in Professorenrobe, der mit hängenden Schultern langsam stapft und mit seinen förmlichen Schuhen beim Gehen kleine Staubwolken aufwirbelt. Eine halbe Minute später wirft der Junge noch einmal einen kurzen Blick auf den Mann, bevor er zu seiner Überwachung zurückkehrt; und dieser Blick zeigt, dass sich die Schultern des Mannes aufgerichtet haben, sein Gesicht entspannt ist und dass seine Schuhe jetzt leicht über den Schmutz laufen und keine Spur von Staub in der Luft zurücklassen.

„Hallo, Professor Quirrell“, sagte Harry, ohne seinen Blick wieder von der Richtung ihres Wagens abzuwenden.

„Ich grüße Sie“, sagte die ruhige Stimme von Professor Quirrell.

„Sie scheinen Abstand zu halten, Mr~Potter. Ich nehme nicht an, dass Sie etwas Seltsames an unserem Transportmittel bemerken?“

„Seltsam?“ echote Harry.

„Nein, ich kann nicht sagen, dass ich etwas Seltsames sehe. Es scheint von allem eine gerade Anzahl zu geben. Vier Sitze, vier Räder, zwei riesige, skelettartige, geflügelte Pferde…“

Ein häutiger Schädel drehte sich zu ihm um und blitzte Zähne auf, fest und weiß in dem schwarzen, höhlenartigen Mund, als wollte er damit andeuten, dass er ihn genauso gern hatte wie er ihn. Das andere schwarze, ledrige Pferdeskelett warf den Kopf hin und her, als würde es wimmern, aber es gab keinen Laut von sich.

„Das sind Thestrale, und sie haben die Kutsche schon immer gezogen“, sagte Professor Quirrell und klang ganz ungestört, als er in die vordere Bank der Kutsche kletterte und sich so weit rechts wie möglich hinsetzte.

„Sie sind nur für diejenigen sichtbar, die den Tod gesehen und begriffen haben, eine nützliche Verteidigung gegen die meisten tierischen Raubtiere. Hm. Ich nehme an, dass Ihre schlimmste Erinnerung an das erste Mal vor dem Dementor die Nacht Ihrer Begegnung mit Er-der-nicht-genannt-werden-muss war?“

Harry nickte grimmig. Es war die richtige Vermutung, wenn auch aus den falschen Gründen. Diejenigen, die den Tod gesehen haben…

„Haben Sie sich dabei an etwas Interessantes erinnert?“

„Ja“, sagte Harry, „das habe ich“, nur das und nichts weiter, denn er war noch nicht bereit, Anschuldigungen zu machen.

Der Verteidigungsprofessor lächelte eines seiner trockenen Lächeln und winkte mit einem ungeduldigen Finger. Harry schloss den Abstand und kletterte in die Kutsche, wobei er zusammenzuckte. Das Gefühl des Unheils war nach dem Tag des Dementors deutlich stärker geworden, auch wenn es schon vorher langsam stärker geworden war. Der größte Abstand, den die Kutsche ihm von Professor Quirrell ließ, schien nicht mehr annähernd weit genug zu sein.

Dann trabten die Skelettpferde vorwärts und die Kutsche setzte sich in Bewegung, um sie in Richtung der äußeren Grenzen von Hogwarts zu bringen. Als die Kutsche sich in Bewegung setzte, sackte Professor Quirrell wieder in den Zombie-Modus zurück und das Gefühl des Unheils wich zurück, obwohl es immer noch am Rande von Harrys Wahrnehmung schwebte, unauslöschlich…

Der Wald rollte an der Kutsche vorbei, die Bäume bewegten sich mit einer Geschwindigkeit, die im Vergleich zu Besenstielen oder sogar Autos geradezu eisig erschien. Es hatte etwas seltsam Entspannendes, dachte Harry, so langsam zu fahren. Auf jeden Fall hatte es den Verteidigungsprofessor entspannt, der zusammengesackt war und dem ein kleiner Strom von Sabber aus seinem schlaffen Mund kam und auf seine Robe tropfte.

Harry hatte immer noch nicht entschieden, was er zum Mittagessen essen durfte. Seine Nachforschungen in der Bibliothek hatten keinen Hinweis darauf ergeben, dass Zauberer mit nichtmagischen Pflanzen sprechen. Oder irgendwelche anderen nichtmagischen Tiere außer Schlangen, obwohl Zaubersprüche und Sprechen von Paul Breedlove die wahrscheinlich mythische Geschichte von einer Zauberin namens Lday der fliegenden Eichhörnchen erzählt hatte. Was Harry tun wollte, war, Professor Quirrell zu fragen. Das Problem war, dass Professor Quirrell zu schlau war.

Nach dem zu urteilen, was Draco gesagt hatte, war die Sache mit dem Erben von Slytherin eine große Bombe, und Harry war sich nicht sicher, ob er wollte, dass jemand anderes davon erfuhr. Und in dem Moment, in dem Harry nach der Parsel fragte, würde Professor Quirrell ihn mit diesen blassblauen Augen fixieren und sagen:

\emph{Verstehe, Mr~Potter, Sie haben also Mr~Malfoy den Patronus-Zauber beigebracht und versehentlich mit seiner Schlange gesprochen.}

Es würde keine Rolle spielen, dass es nicht genug Beweise sein sollten, um die wahre Erklärung als Hypothese zu lokalisieren, geschweige denn ihre Last der vorherigen Unwahrscheinlichkeit zu überwinden. Irgendwie würde der Verteidigungsprofessor sie sowieso herleiten. Es gab Zeiten, in denen Harry vermutete, dass Professor Quirrell viel mehr Hintergrundinformationen hatte, als er erzählte, seine Priors waren einfach zu gut. Manchmal lagen seine verblüffenden Schlussfolgerungen richtig, selbst wenn seine Gründe falsch waren.

Das Problem war, dass Harry nicht sehen konnte, wie Professor Quirrell bei der Hälfte der Dinge, die er vermutete, einen zusätzlichen Hinweis hätte einbauen können. Nur ein einziges Mal hätte Harry gerne eine unglaubliche Schlussfolgerung aus etwas gezogen, das Professor Quirrell sagte und ihn völlig unvorbereitet traf.

„Ich nehme eine Schüssel grüne Linsensuppe mit Sojasauce“, sagte Professor Quirrell zur Kellnerin. „Und für Mr~Potter einen Teller von Tenormans Familien-Chili.“

Harry zögerte in plötzlicher Bestürzung. Er hatte sich vorgenommen, vorerst bei vegetarischen Gerichten zu bleiben, aber er hatte in seinen Überlegungen vergessen, dass Professor Quirrell die eigentliche Bestellung aufgab - und es wäre unangenehm, wenn er jetzt protestieren würde - Die Kellnerin verbeugte sich vor ihnen und wandte sich zum Gehen—

„Ähm, entschuldigen Sie, ist da auch Fleisch von Schlangen oder fliegenden Eichhörnchen drin?“

Die Kellnerin blinzelte nicht einmal mit der Wimper, drehte sich nur zu Harry um, schüttelte den Kopf, verbeugte sich wieder höflich vor ihm und setzte ihren Weg zur Tür fort.

(Die anderen Teile von Harry grinsten ihn an. Gryffindor machte sardonische Kommentare darüber, wie ein wenig soziales Unbehagen ausreichte, um ihn dazu zu bringen, zum Kannibalismus zu greifen! (gerufen von Hufflepuff), und Slytherin bemerkte, wie schön es war, dass Harrys Moral flexibel war, wenn es um wichtige Ziele wie die Aufrechterhaltung seiner Beziehung zu Professor Quirrell ging.)

Nachdem die Kellnerin die Tür hinter sich geschlossen hatte, winkte Professor Quirrell mit einer Hand, um den Riegel zurückzuschieben, sprach die üblichen vier Zauber, um die Privatsphäre sicherzustellen, und sagte dann:

„Eine interessante Frage, Mr~Potter. Ich frage mich, warum Sie sie gestellt haben?“

Harry behielt sein Gesicht ruhig.

„Ich habe vorhin ein paar Fakten über den Patronus-Zauber nachgeschlagen“, sagte er. "Laut dem Buch \emph{Der Patronuszauber, Zauberer die ihn konnten und die die ihn nicht konnten}, stellt sich heraus, dass Godric es nicht konnte und Salazar schon.

Ich war überrascht, also habe ich die Referenz nachgeschlagen, in Vier Leben der Macht. Und dann habe ich entdeckt, dass Salazar Slytherin angeblich mit Schlangen sprechen konnte."

(Zeitliche Abfolge war nicht dasselbe wie Kausalität, es war nicht Harrys Schuld, wenn Professor Quirrell das übersehen würde.)

„Weitere Nachforschungen ergaben eine alte Geschichte über eine Art Muttergöttin, die mit fliegenden Eichhörnchen sprechen konnte. Ich war etwas beunruhigt über die Aussicht, etwas zu essen, das sprechen kann.“

Und Harry nahm einen beiläufigen Schluck von seinem Wasser—

—gerade als Professor Quirrell sagte:

„Mr~Potter, liege ich richtig in der Annahme, dass Sie auch ein Parselmund sind?“

Als Harry mit dem Husten fertig war, stellte er sein Glas Wasser wieder auf den Tisch, fixierte seinen Blick auf Professor Quirrells Kinn, anstatt ihm in die Augen zu sehen, und sagte:

„Sie sind also in der Lage, Legilimenz durch meine Okklumentik-Barrieren hindurch durchzuführen.“

Professor Quirrell grinste breit.

„Ich fasse das als Kompliment auf, Mr~Potter, aber nein.“

„Das kaufe ich Ihnen nicht mehr ab“, sagte Harry.

„Es ist unmöglich, dass Sie aufgrund dieser Beweise zu diesem Schluss gekommen sind.“

„Natürlich nicht“, sagte Professor Quirrell gleichmütig.

„Ich hatte ohnehin vor, Ihnen diese Frage heute zu stellen, und habe einfach einen günstigen Moment gewählt. Ich habe seit Dezember den Verdacht—“

„Dezember?“, sagte Harry. „Ich habe es gestern erfahren!“

„Ah, Sie haben also nicht bemerkt, dass die Nachricht des Sprechenden Hutes an Sie in Parsel war?“

Der Verteidigungsprofessor hatte es auch beim zweiten Mal genau richtig getimt, gerade als Harry einen Schluck Wasser nahm, um sich vom ersten Hustenanfall zu befreien. Harry hatte es nicht bemerkt, nicht bis gerade eben. Natürlich war es sofort klar, als Professor Quirrell es sagte.

\emph{Richtig, Professor McGonagall hatte ihm sogar gesagt, er solle nicht mit Schlangen reden, wenn jemand ihn sehen könnte, aber er hatte gedacht, sie hätte gemeint, man solle nicht mit irgendwelchen Statuen oder architektonischen Elementen in Hogwarts reden, die wie Schlangen aussahen.}

Doppelte Illusion von Transparenz, er hatte gedacht, er hätte sie verstanden, sie hatte gedacht, er hätte sie verstanden - aber wie zum Teufel—

„Also“, sagte Harry, „Sie haben während meiner ersten Verteidigungsstunde Legilimenz an mir durchgeführt, um herauszufinden, was mit dem Sprechenden Hut passiert ist—“

„Dann hätte ich es im Dezember nicht herausgefunden.“

Professor Quirell lehnte sich zurück und lächelte.

„Das ist kein Rätsel, das Sie allein lösen können, Mr~Potter, also werde ich Ihnen die Antwort verraten. Während der Winterferien wurde ich darauf aufmerksam gemacht, dass der Schulleiter einen Antrag auf ein geschlossenes Gericht gestellt hat, um den Fall eines gewissen Mr~Rubeus Hagrid zu überprüfen, den Sie als Hüter der Schlüssel und des Geländes in Hogwarts kennen und der des Mordes an Abigail Myrtle im Jahr 1943 beschuldigt wurde.“

„Oh, natürlich“, sagte Harry, „das macht es geradezu offensichtlich, dass ich ein Parselmund bin. Professor, was die süßen Schlangen—“

"Der andere Verdächtige für diesen Mord war das Monster von Slytherin, der legendäre Bewohner von Slytherins Kammer des Schreckens. Deshalb haben mich gewisse Quellen darauf aufmerksam gemacht, und deshalb hat es meine Aufmerksamkeit so sehr erregt, dass ich eine Menge Bestechungsgeld ausgegeben habe, um die Details des Falles zu erfahren. Tatsache ist, Mr~Potter, dass Mr~Hagrid unschuldig ist. Lächerlich offensichtlich unschuldig. Er ist der eklatanteste Unschuldige, den das magische britische Rechtssystem verurteilt hat, seit Grindelwalds Ermordung von Neville Chamberlain Amanda Knox angehängt wurde.

Schulleiter Dippet veranlasste eine Schüler-Marionette, Mr~Hagrid zu beschuldigen, weil Dippet einen Sündenbock brauchte, der die Schuld für den Tod von Miss~Myrtle auf sich nimmt, und unsere wunderbare Justiz war sich einig, dass dies plausibel genug war, um Mr~Hagrids Verweis und das Zerbrechen seines Zauberstabs zu rechtfertigen. Unser jetziger Schulleiter muss nur ein neues Beweisstück vorlegen, das bedeutend genug ist, um den Fall neu aufzurollen; und da Dumbledore statt Dippet Druck ausübt, ist das Ergebnis eine ausgemachte Sache. Lucius Malfoy hat keinen besonderen Grund, Mr~Hagrids Rechtfertigung zu fürchten; daher wird sich Lucius Malfoy nur in dem Maße widersetzen, wie er es ohne Kosten tun kann, um Dumbledore Kosten aufzuerlegen, und Dumbledore ist eindeutig gewillt, den Fall trotzdem zu verfolgen."

Professor Quirrell nahm einen Schluck von seinem Wasser.

"Aber ich schweife ab. Der neue Beweis, den der Schulleiter zu liefern verspricht, ist die Ausstellung eines bisher unentdeckten Zaubers auf dem Sprechenden Hut, von dem der Schulleiter behauptet, er habe persönlich festgestellt, dass er nur auf Slytherins reagiert, die auch Parselmünder sind. Der Schulleiter argumentiert weiter, dass dies die Interpretation begünstigt, dass die Kammer des Schreckens tatsächlich 1943 geöffnet wurde, was ungefähr der richtige Zeitraum ist, in dem Er-der-nicht-genannt-werden-muss, ein bekannter Parselmund, Hogwarts besucht haben könnte.

Es ist eine ziemlich fragwürdige Logik, aber ein Gericht könnte entscheiden, dass sie den Fall weit genug schwingt, um Mr~Hagrids Schuld in Zweifel zu ziehen, wenn sie es schaffen, ein gerades Gesicht zu behalten, während sie es sagen. Und jetzt kommen wir zur Schlüsselfrage: Wie hat der Schulleiter diesen versteckten Zauber auf dem Sprechenden Hut entdeckt?"

Professor Quirrell lächelte jetzt dünn.

„Nun, nehmen wir einmal an, es gäbe einen Parselmund unter den diesjährigen Schülern, einen potenziellen Erben Slytherins. Sie müssen zugeben, Mr~Potter, dass Sie als Möglichkeit in Frage kommen, wenn es um außergewöhnliche Menschen geht. Und wenn ich mich dann weiter frage, welcher neue Slytherin am ehesten vom Schulleiter in seine mentale Privatsphäre eingedrungen wäre, um speziell die Erinnerungen an seine Sortierung zu jagen, dann stechen Sie noch mehr heraus.“

Das Lächeln verschwand.

„Sie sehen also, Mr~Potter, nicht ich war es, der in Ihre Gedanken eingedrungen ist, obwohl ich nicht von Ihnen verlange, dass Sie sich entschuldigen. Es ist nicht Ihre Schuld, dass Sie Dumbledores Beteuerungen, Ihre geistige Privatsphäre zu respektieren, geglaubt haben.“

„Ich entschuldige mich aufrichtig“, sagte Harry und hielt sein Gesicht ausdruckslos.

Die starre Beherrschung war an sich schon ein Geständnis, ebenso wie der Schweiß, der ihm auf der Stirn stand; aber er glaubte nicht, dass der Verteidigungsprofessor daraus irgendwelche Schlüsse ziehen würde. Professor Quirrell würde nur denken, dass Harry nervös war, weil er als Erbe von Slytherin entdeckt worden war. Anstatt nervös zu sein, dass Professor Quirrell erkennen könnte, dass Harry absichtlich das Geheimnis von Slytherin verraten hatte .

\emph{.. was wiederum nicht mehr so klug zu sein schien.

}\strut

„Also, Mr~Potter. Gibt es Fortschritte bei der Suche nach der Kammer des Schreckens?“

\emph{Nein}, dachte Harry. Aber um glaubhaft leugnen zu können, musste man manchmal Fragen ausweichen, auch wenn man nichts zu verbergen hatte…

„Bei allem Respekt, Professor Quirrell, wenn ich solche Fortschritte gemacht hätte, ist es mir nicht ganz klar, dass ich Ihnen davon erzählen sollte.“

Professor Quirrell nahm wieder einen Schluck aus seinem Wasserglas.

"Nun denn, Mr~Potter, ich werde Ihnen frei heraus sagen, was ich weiß oder vermute.

Erstens glaube ich, dass die Kammer des Schreckens real ist, ebenso wie das Monster von Slytherin. Miss~Myrtles Tod wurde erst Stunden nach ihrem Ableben entdeckt, obwohl die Zauberstäbe den Schulleiter sofort hätten alarmieren müssen. Der Mord an Miss~Myrtle wurde also entweder von Schulleiter Dippet begangen, was unwahrscheinlich ist, oder von einem Wesen, das Salazar Slytherin auf einer höheren Ebene als der Schulleiter selbst in die Wachzauber der Schule verankerte.

Zweitens vermute ich, dass der Zweck von Slytherins Monster, entgegen der populären Legende, nicht darin bestand, Hogwarts von Muggelgeborenen zu befreien. Wenn Slytherins Monster nicht mächtig genug wäre, den Schulleiter von Hogwarts und alle Lehrer zu besiegen, könnte es nicht mit Gewalt triumphieren.

Mehrere Morde im Verborgenen würden zur Schließung der Schule führen, wie es 1943 fast geschehen wäre, oder zur Einsetzung neuer Mündel. Warum also das Monster von Slytherin, Mr~Potter? Welchem wahren Zweck dient es?"

„Ah…“

Harry senkte den Blick auf sein Wasserglas und versuchte, nachzudenken.

„Um jeden zu töten, der in die Kammer gelangt ist und dort nicht hingehört—“

„Ein Monster, das mächtig genug ist, um ein Team von Zauberern zu besiegen, das die besten Schutzvorrichtungen, die Salazar an seiner Kammer anbringen konnte, überwunden hat? Unwahrscheinlich.“

Harry fühlte sich jetzt ein wenig bedrängt.

„Nun, sie heißt Kammer des Schreckens, also hat das Monster vielleicht ein Geheimnis, oder ist ein Geheimnis?“

\emph{Übrigens, was für Geheimnisse waren überhaupt in der Kammer des Schreckens?} Harry hatte nicht viel zu diesem Thema recherchiert, auch weil er den Eindruck gewonnen hatte, dass niemand etwas wusste - Professor Quirrell lächelte.

„Warum schreiben Sie das Geheimnis nicht einfach auf?“

„Ahhh…“, sagte Harry. „Weil, wenn das Monster Parsel spricht, das sicherstellen würde, dass nur ein echter Nachkomme Slytherins das Geheimnis erfahren kann?“

"Es ist leicht genug, den Schlüssel für die Kammer auf einen in Parsel gesprochenen Satz einzustellen. Warum sollte man sich die Mühe machen, das Monster von Slytherin zu erschaffen? Es kann nicht einfach gewesen sein, eine Kreatur mit einer Lebensspanne von Jahrhunderten zu erschaffen.

Kommen Sie, Mr~Potter, es sollte offensichtlich sein; was sind die Geheimnisse, die von einem lebenden Geist zum anderen weitergegeben werden können, aber niemals aufgeschrieben werden?"

Da sah Harry es, mit einem Adrenalinstoß, der sein Herz zum Rasen brachte und seinen Atem schneller werden ließ.

„Oh.“

\emph{Salazar Slytherin war in der Tat sehr gerissen gewesen. Gerissen genug, um einen Weg zu finden, das Interdikt von Merlin zu umgehen.}

\emph{Mächtige Zaubersprüche konnten nicht durch Bücher oder Geister übertragen werden, aber wenn man ein Wesen erschaffen konnte, das lange genug lebte und ein gutes Gedächtnis hatte—}

„Es scheint mir sehr wahrscheinlich“, sagte Professor Quirrell,

„dass Er, der nicht genannt werden darf, seinen Aufstieg zur Macht mit Geheimnissen begann, die er von Slytherins Monster erhalten hatte. Dass Salazars verlorenes Wissen die Quelle von Du-weißt-schon-wem für seine außerordentlich mächtigen Zaubereien ist. Daher mein Interesse an der Kammer des Schreckens und dem Fall von Mr~Hagrid.“

„Ich verstehe“, sagte Harry.

\emph{Und wenn er, Harry, Salazars Kammer des Schreckens finden könnte… dann würde das ganze verlorene Wissen, das Lord Voldemort erlangt hatte, auch ihm gehören.

Ja. Genau so sollte die Geschichte ablaufen. Wenn man Harrys überlegene Intelligenz, einige originelle magische Forschungen und einige Muggel-Raketenwerfer hinzufügte, würde der daraus resultierende Kampf völlig einseitig sein, und das war genau so, wie Harry es wollte.}

Harry grinste jetzt, ein sehr böses Grinsen.

\emph{Neue Priorität: Finde alles in Hogwarts, was auch nur im Entferntesten wie eine Schlange aussieht, und versuche, es anzusprechen. Beginne mit allem, was du schon versucht hast, nur benutze diesmal unbedingt Parsel statt Englisch - bringe Draco dazu, dich in die Slytherin-Schlafsäle zu lassen—}

„Regen Sie sich nicht zu sehr auf, Mr~Potter“, sagte Professor Quirrell. Sein eigenes Gesicht war inzwischen ausdruckslos geworden.

„Sie müssen weiter nachdenken. Was waren die Abschiedsworte des Dunklen Lords an das Monster von Slytherin?“

„Was?“ sagte Harry. „Woher soll einer von uns das wissen?“

"Stellen Sie sich die Szene vor, Mr~Potter. Lassen Sie Ihre Fantasie die Details ausfüllen. Das Monster von Slytherin - wahrscheinlich eine große Schlange, so dass nur ein Parselmund mit ihm sprechen kann - hat sein gesamtes Wissen an Er, der nicht genannt werden darf, weitergegeben. Sie überbringt ihm Salazars letzten Segen und warnt ihn, dass die Kammer des Schreckens nun verschlossen bleiben muss, bis der nächste Nachkomme Salazars sich als schlau genug erweisen sollte, sie zu öffnen.

Und er, der der Dunkle Lord werden wird, nickt und sagt dazu -"

„Avada Kedavra“, sagte Harry, dem plötzlich ganz übel wurde.

„Regel Zwölf“, sagte Professor Quirrell leise. „Lass die Quelle deiner Macht niemals dort herumliegen, wo jemand anderes sie finden kann.“

Harrys Blick fiel auf das Tischtuch, das sich mit einem traurigen Muster aus schwarzen Blumen und Schatten geschmückt hatte.

Irgendwie schien das… zu traurig, um es sich vorstellen zu können, die große Schlange von Slytherin hatte nur Lord Voldemort helfen wollen, und Lord Voldemort hatte nur…. es hatte etwas unerträglich Trauriges an sich, was für ein Mensch würde das einem Wesen antun, das ihm nichts als Freundschaft angeboten hatte…

„Glauben Sie, der Dunkle Lord hätte—“

„Ja“, sagte Professor Quirrell barsch. „Er, der nicht genannt werden darf, hat eine ziemliche Spur von Leichen hinterlassen, Mr~Potter; ich bezweifle, dass er diese hier ausgelassen hätte. Wenn es dort irgendwelche Artefakte gab, die bewegt werden konnten, hätte der Dunkle Lord auch diese mitgenommen. Vielleicht gibt es in der Kammer des Schreckens noch etwas zu sehen, und es zu finden, würde Sie als den wahren Erben Slytherins ausweisen. Aber machen Sie sich nicht zu große Hoffnungen. Ich vermute, dass nur die Überreste des Ungeheuers von Slytherin zu finden sind, das ruhig in seinem Grab ruht.“

Sie saßen eine Weile schweigend da.

„Ich könnte mich irren“, sagte Professor Quirrell.

„Letztlich ist es nur eine Vermutung. Aber ich wollte Sie warnen, Mr~Potter, damit Sie nicht allzu sehr enttäuscht werden.“

Harry nickte kurz.

„Man könnte sogar den Sieg Ihres kindlichen Ichs bedauern“, sagte Professor Quirrell. Sein Lächeln verzog sich. „Wenn nur Du-Weißt-Schon-Wer gelebt hätte, hätten Sie ihn vielleicht überreden können, dir etwas von dem Wissen beizubringen, das dein Erbe gewesen wäre, \emph{von einem Erben Slytherins zum anderen.}“

Das Lächeln verdrehte sich weiter, als wolle es sich über die offensichtliche Unmöglichkeit lustig machen, selbst unter dieser Voraussetzung.

\emph{Notiz an mich selbst,} dachte Harry mit einem leichten Frösteln und einem Anflug von Wut, \emph{stelle sicher, dass ich mein Erbe aus dem Geist des Dunklen Lords heraushole, so oder so.}

Wieder herrschte Stille. Professor Quirrell sah Harry an, als ob er darauf wartete, dass er etwas fragen würde.

„Nun“, sagte Harry, „wenn wir schon beim Thema sind, darf ich fragen, wie Sie sich die ganze Sache mit dem Parselmund eigentlich vorstellen—“

Dann klopfte es an der Tür.

Professor Quirrell hob mahnend den Finger, dann öffnete er die Tür mit einem Wink. Die Kellnerin kam herein und balancierte einen riesigen Teller mit den Mahlzeiten, als ob die ganze Ansammlung nichts wöge

(was in der Tat wahrscheinlich der Fall war).

Sie gab Professor Quirrell seine Schüssel mit grüner Suppe und ein Glas seines üblichen Chianti und stellte Harry einen Teller mit kleinen Fleischstreifen in einer schwer aussehenden Soße und ein Glas seines gewohnten Sirupgetränks hin.

Dann verbeugte sie sich, wobei sie es schaffte, dass es eher wie aufrichtiger Respekt als wie eine oberflächliche Anerkennung wirkte, und ging weg. Als sie weg war, hob Professor Quirrell wieder einen Finger zum Zeichen der Stille und zog seinen Zauberstab.

Und dann begann Professor Quirrell eine bestimmte Reihe von Beschwörungsformeln auszuführen, die Harry wiedererkannte und die ihn scharf einatmen ließen. Es war die Reihe und Reihenfolge, die Mr~Bester verwendet hatte, der vollständige Satz von siebenundzwanzig Zaubern, die man vor der Besprechung von etwas wirklich Wichtigem durchführen würde. Wenn die Diskussion über die Kammer des Schreckens nicht als wichtig gezählt hätte - als Professor Quirrell fertig war - er hatte \emph{dreißig Zaubersprüche} vorgetragen, von denen Harry drei noch nie gehört hatte - sagte der Verteidigungsprofessor:

„Jetzt werden wir eine Zeit lang nicht unterbrochen. Können Sie ein Geheimnis bewahren, Mr~Potter?“

Harry nickte.

„Ein ernstes Geheimnis, Mr~Potter“, sagte Professor Quirrell.

Seine Augen waren aufmerksam, sein Gesicht ernst.

„Eines, das mich möglicherweise nach Askaban schicken könnte. Denken Sie darüber nach, bevor Sie antworten.“

Einen Moment lang sah Harry nicht einmal, warum die Frage schwer sein sollte, angesichts seiner wachsenden Sammlung von Geheimnissen. Dann - \emph{wenn dieses Geheimnis Professor Quirrell nach Askaban schicken könnte, bedeutet das, dass er etwas Illegales getan hat.}.. Harrys Gehirn führte ein paar Berechnungen durch.

\emph{Was auch immer das Geheimnis war, Professor Quirrell glaubte nicht, dass seine illegale Tat in Harrys Augen ein schlechtes Licht auf ihn werfen würde.

Es hatte keinen Vorteil, es nicht zu hören. Und wenn es etwas enthüllte, was mit Professor Quirrell nicht stimmte, dann war es sehr zu Harrys Vorteil, es zu wissen, auch wenn er versprochen hatte, es niemandem zu sagen.}

„Ich hatte nie viel Respekt vor Autoritäten“, sagte Harry. „Juristische und staatliche Autorität eingeschlossen. Ich werde Ihr Geheimnis bewahren.“

Harry machte sich nicht die Mühe zu fragen, ob die Enthüllung die Gefahr wert war, die sie für Professor Quirrell darstellen würde. Der Verteidigungsprofessor war nicht dumm.

„Dann muss ich prüfen, ob Sie wirklich ein Nachkomme von Salazar sind“, sagte Professor Quirrell und stand von seinem Stuhl auf. Harry, mehr aus Reflex und Instinkt als aus Berechnung, schob sich ebenfalls aus seinem Stuhl hoch.

\textbf{\emph{Da war eine Unschärfe, eine Verschiebung, eine plötzliche Bewegung.}}

Harry brach seinen panischen Sprung nach hinten auf halbem Weg ab, so dass er mit den Armen herumfuchtelte und versuchte, nicht umzufallen, wobei ihn ein hektischer Adrenalinschub durchfuhr.

\textbf{Am anderen Ende des Raumes schwankte eine meterhohe Schlange, leuchtend grün und kunstvoll in Weiß und Blau gebändert.}

Harry kannte nicht genug Schlangenkunde, um sie zu erkennen, aber er wusste, dass

\emph{„leuchtend bunt“ „giftig“} bedeutete. Das ständige Gefühl des Unheils hatte sich ironischerweise gelegt, nachdem sich der Verteidigungsprofessor von Hogwarts in eine Giftschlange verwandelt hatte.

Harry schluckte hart und sagte:

„Gruß - ah, hssss, nein, ah, \emph{Grußss}.“

„\textbf{\emph{Sso}}“, zischte die Schlange. \textbf{„\emph{Du sprichst, ich höre. Ich spreche,du hörsssst?}}“

„\emph{Ja, ich höre}“, zischte Harry. „\emph{Du bist ein Animaguss?}“

\textbf{„\emph{Offensichtlich}}“, zischte die Schlange.

\textbf{„\emph{Regel Nummer vierunddreißig: Werde Animagusssss. Alle vernünftigen Menschen tun es, wenn sie es können. Desssshalb, sehr selten.}}“

Die Augen der Schlange waren flache Flächen, eingebettet in dunkle Gruben, scharfe schwarze Pupillen in dunkelgrauen Feldern.

\textbf{„\emph{Das ist die sicherste Art zu sprechen. Siehst du? Keiner ssversteht uns.}}“

\emph{\hfill\break „Selbst wenn sie ssSchlangen Animagi sind?“

}\strut

\textbf{„\emph{Nicht, es sei denn, der Erbe von Sslytherin willss.}}“

Die Schlange gab eine Reihe von kurzen Zischlauten von sich, die Harrys Gehirn als sardonisches Lachen übersetzte.

\textbf{„\emph{Sslytherin nicht ssdumm. Sschlangen Animaguss nicht dasssselbe wie Parsselmund. Wäre großer Fehler im sssPlan.}}“

\emph{Nun, das sprach definitiv dafür, dass Parsel persönliche Magie war und nicht dafür, dass Schlangen fühlende Wesen mit einer erlernbaren Sprache waren—}

\textbf{„\emph{Ich bin nicht regisstriert}}“, zischte die Schlange.

Die dunklen Gruben ihrer Augen starrten Harry an.

\textbf{\emph{„Animaguss muss registriert werden. Die Strafe ist zwei Jahre Haft. Wirst du mein Geheimnis bewahren, Junge?}}“

„\emph{Ja}“, zischte Harry. „\emph{Ich würde nie mein Versprechen brechen.}“

Die Schlange schien wie unter Schock stillzuhalten und begann dann wieder zu schwanken.

\textbf{\emph{„Wir kommen das nächste Mal in sieben Tagen hierher. Bringt den Umhang, um ungesehen zu passieren, bringt die Sanduhr, um durch die Zeit zu reisen—} }“

\emph{„Du weißssst?}“, zischte Harry schockiert. „\emph{Wie}—“

Wieder die Reihe von kurzen, schnellen Zischlauten, die als sardonisches Lachen übersetzt wurden.

„\textbf{\emph{Du kommst in meine erste Klasse, während ich noch in der anderen Klasse bin, schlägst den Feind mit Kuchen nieder, zwei Gedächtniskugeln—} }“

„\emph{Vergiss es}“, zischte Harry. „\emph{Dumme Frage, ich vergaß, dass du sSchlau bist.}“

„\textbf{\emph{Dummes Ding, das man vergisst}}“, sagte die Schlange, aber das Zischen schien nicht beleidigt zu sein.

„\emph{Sssstundenglasss ist behindert}“, sagte Harry. „\emph{Kann bis zur neunten Stunde nicht benutzt werden.}“

Die Schlange zuckte mit dem Kopf, ein schlangenhaftes Nicken.

"\textbf{\emph{Viele Restriktionen. Nur für den eigenen Gebrauch gesperrt, kann nicht gestohlen werden. Kann keine anderen Menschen transssportieren. Aber ich nehme an, dass ich als Schlange in der Tasche mitkommen kann. Ich denke, es ist möglich, die Sanduhr in der Schale zu halten, ohne die Wachen zu stören, während Sie die Schale drehen.

Wir werden es in sieben Tagen testen. Wir sprechen nicht über Pläne, die darüber hinausgehen.Sie ssagen nichts, zu niemandem. Geben Sie kein ssZeichen der Erwartung, keins. Verstehst?}}"

Harry nickte.

„\textbf{\emph{Antworte in SchlangenssSprache.}}“

„\emph{Jawohl}.“

„\textbf{\emph{Wirssst tun wie ich ssssage?!}}“

„\emph{Jawohl}. \emph{Aber}“, Harry gab ein schwankendes Raspeln von sich, das sein Verstand als ein zögerndes '\emph{Ahhh}' ins Schlangenhafte übersetzt hatte,

„\emph{ich verspreche nicht, zu tun, was immer das ist, du hast nicht gesssagt}—“

Die Schlange vollführte ein Zittern, das Harrys Verstand als einen strengen Blick übersetzte.

„\textbf{\emph{Natürlich nicht, besssssprechen die Einzelheiten beim nächsten Treffen.}}“

Die Unschärfe und Bewegung kehrte sich um, und Professor Quirrell stand wieder da. Einen Moment lang schien der Verteidigungsprofessor selbst zu schwanken, wie die Schlange geschwankt hatte, und seine Augen wirkten kalt und flach; dann richteten sich seine Schultern auf und er war wieder ein Mensch.

Und die Aura des Unheils war zurückgekehrt. Professor Quirrells Stuhl rutschte für ihn zurück, und er setzte sich darauf.

„Es macht keinen Sinn, das hier zu vergeuden“, sagte Professor Quirrell, während er seinen Löffel aufhob, "obwohl ich im Moment eine lebende Maus vorziehen würde.

Man kann den Geist nie ganz von dem Körper trennen, den er trägt, wissen Sie…"

Harry nahm langsam Platz und begann zu essen.

„Die Linie von Salazar ist also doch nicht mit Du-Weißt-Schon-Wer gestorben“, sagte Professor Quirrell nach einer Weile.

„Es scheint, dass sich bereits Gerüchte unter unserer feinen Schülerschaft verbreitet haben, dass du ein Dunkler Lord bist; ich frage mich, was sie denken würden, wenn sie das wüssten.“

„Oder wenn sie wüssten, dass ich einen Dementor zerstört habe“, sagte Harry und zuckte mit den Schultern.

„Ich denke, die ganze Aufregung wird sich legen, wenn ich das nächste Mal etwas Interessantes mache. Hermine hat allerdings Probleme, und ich habe mich gefragt, ob Sie vielleicht irgendwelche Vorschläge für sie hast.“

Der Verteidigungsprofessor aß mehrere Löffel Suppe schweigend, und als er wieder sprach, war seine Stimme seltsam flach.

„Sie sorgen sich wirklich um das Mädchen.“

„Ja“, sagte Harry leise.

„Ich nehme an, dass sie deshalb in der Lage war, dich aus deiner Dementation herauszuholen?“

„Mehr oder weniger“, sagte Harry.

Die Aussage stimmte in gewisser Weise, nur nicht genau; es war nicht so, dass sein dementiertes Ich sich darum gekümmert hätte, sondern dass es verwirrt gewesen war.

„Ich hatte keine solchen Freunde, als ich jung war.“

Immer noch die gleiche emotionslose Stimme.

„Was wäre wohl aus dir geworden, wenn du allein gewesen wärst?“

Harry zitterte, bevor er sich zurückhalten konnte.

„Du musst dich ihr gegenüber dankbar fühlen.“

Harry nickte nur. Nicht ganz genau, aber wahr.

„Dann dies das, was ich in deinem Alter getan hätte, wenn es jemanden gegeben hätte, für den ich es hätte tun wollen—“

