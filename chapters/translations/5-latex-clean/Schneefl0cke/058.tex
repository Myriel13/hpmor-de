

\hypertarget{das-stanford-gefuxe4ngnis-experiment-teil-9}{% \section{59. Das Stanford-Gefängnis-Experiment, Teil 9}\label{das-stanford-gefuxe4ngnis-experiment-teil-9}}

\textbf{\uline{Das Stanford-Gefängnis-Experiment, Teil 9}}

Besen waren in einer Zeit erfunden worden, die ein Muggel als das finstere Mittelalter bezeichnen würde, angeblich von einer legendären Hexe namens Celestria Relevo, angeblich die Ur-Ur-Enkelin von Merlin. Celestria Relevo, oder welche Person oder Gruppe auch immer diese Zaubersprüche wirklich erfunden hatte, hatte nicht die geringste Ahnung von Newtonscher Mechanik. Besen funktionierten also nach der aristotelischen Physik. Sie gingen dorthin, wohin man sie richtete. Wenn man sich geradeaus bewegen wollte, richtete man sie geradeaus; man kümmerte sich nicht darum, dass ein Teil des Schubs nach unten ging, um die Wirkung der Schwerkraft aufzuheben. Wenn man einen Besen drehte, ging die gesamte neue Geschwindigkeit in die neue Richtung, in die man ihn richtete, und nicht seitwärts, basierend auf dem alten Schwung. Besen hatten maximale Geschwindigkeiten, nicht maximale Beschleunigungen. Nicht wegen irgendetwas, das mit dem Luftwiderstand zu tun hatte, sondern weil ein Besen einen maximalen aristotelischen Impuls hatte, den seine Verzauberungen ausüben konnten.

Harry hatte das noch nie explizit bemerkt, obwohl er geschickt genug war, um im Flugunterricht die besten Noten zu bekommen. Besen funktionierten so sehr, wie der menschliche Verstand sie instinktiv erwartete, dass sein Gehirn es geschafft hatte, ihre physikalische Absurdität völlig zu übersehen. Harry war an seinem ersten Donnerstag im Besen-Unterricht von interessanter erscheinenden Phänomenen abgelenkt worden, von auf Papier geschriebenen Worten und einem glühenden roten Ball. So hatte sein Gehirn einfach seinen Unglauben suspendiert, die Realität der Besen als akzeptiert markiert und sich weiter vergnügt, ohne auch nur ein einziges Mal an die Frage zu denken, deren Antwort offensichtlich gewesen wäre. Denn es ist eine traurige Tatsache, dass wir immer nur über einen winzigen Bruchteil aller Phänomene nachdenken, die uns begegnen…

Das ist die Geschichte, wie Harry James Potter-Evans-Verres fast an seiner eigenen mangelnden Neugierde zugrunde gegangen wäre. Denn Raketen funktionierten nicht nach der aristotelischen Physik. Raketen funktionierten nicht so, wie ein menschlicher Verstand instinktiv dachte, dass ein fliegendes Ding funktionieren sollte. Ein raketengestützter Besen bewegte sich also nicht wie die magischen Besen, auf denen Harry so gut fliegen konnte.

Nichts von alledem ging Harry zu diesem Zeitpunkt wirklich durch den Kopf. Zum einen hinderte ihn das lauteste Geräusch, das er je in seinem Leben gehört hatte, daran, sich selbst denken zu hören. Zum anderen bedeutete die Beschleunigung nach oben mit 4G, dass er insgesamt etwa zweieinhalb Sekunden Zeit hatte, um vom Boden bis zur Spitze von Askaban zu gelangen. Und selbst wenn es zweieinhalb der längsten Sekunden in der Geschichte der Zeit waren, war das nicht genug Raum, um viel zu denken. Es blieb nur Zeit, die Lichter der Flüche der Auroren auf sich niederprasseln zu sehen, den Besen leicht anzuwinkeln, um ihnen auszuweichen, zu erkennen, dass der Besen einfach mit demselben Schwung weiterflog, statt in die Richtung, in die er ihn richtete, und aktivierte die wortlosen Begriffe \emph{*Scheiße*} und \emph{*Newton*}, woraufhin Harry den Besen viel stärker anwinkelte und sie sich sehr schnell der Wand näherten, also winkelte er ihn wieder in die andere Richtung und es kamen noch mehr Lichter herunter und die Dementoren glitten sanft auf sie zu, zusammen mit einer Art riesiger geflügelter Kreatur aus weißer- goldenen Flamme, also riss Harry den Besen zurück in Richtung Himmel, aber jetzt rutschte er immer noch auf eine andere Wand zu, also kippte er den Besen leicht und stoppte die Annäherung, aber er war zu nah, also kippte er ihn wieder und dann waren die entfernten Auroren auf ihren Besen gar nicht mehr so weit entfernt und er war im Begriff, mit dieser Frau zusammenzustoßen, also drehte er seinen Besen direkt von ihr weg und dann, in einem anderen Augenblick, flog er mit dem Besen von ihr weg. Also drehte er seinen Besen direkt von ihr weg und in einem weiteren Augenblick wurde ihm klar, dass seine Rakete ein extrem starker Flammenwerfer war und im Bruchteil einer Sekunde direkt auf die Aurorin gerichtet sein würde, also drehte er den Besen seitwärts, während er weiter nach oben flog und er konnte sich nicht erinnern, ob er jetzt auf irgendwelche Auroren gerichtet war, aber zumindest war er nicht auf sie gerichtet.

Harry verpasste einen anderen Auror um etwa einen Meter, Er sauste auf einem seitwärts gerichteten Flammenwerfer an ihm vorbei, der sich mit, wie Harry später schätzte, etwa 300 Stundenkilometern nach oben bewegte. Wenn es irgendwelche Schreie von gebratenen Auroren gab, hörte er sie nicht, aber das war kein Beweis für die eine oder andere Seite, denn alles, was Harry im Moment hörte, war ein extrem lautes Geräusch.

Ein paar ruhigere, wenn auch nicht leisere Sekunden später, schienen keine Auroren in der Nähe zu sein, auch keine Dementoren oder riesige geflügelte Flammenwesen, und das riesige und schreckliche Gebäude von Askaban sah aus dieser Höhe überraschend winzig aus. Harry richtete den Besen auf die Sonne, die durch die Wolken schwach zu sehen war, sie stand zu dieser Tageszeit und im Wintermonat nicht hoch am Himmel, und der Besen beschleunigte noch zwei Sekunden lang in diese Richtung und nahm sehr schnell erstaunlich viel Geschwindigkeit auf, bevor die Feststoffrakete ausbrannte.

Danach, als Harry wieder denken konnte, als es nur noch den heulenden Wind gab, der von ihrer lächerlichen Geschwindigkeit herrührte, und Harrys verzauberte Finger, die den Besen umklammerten, nur noch dem abbremsenden Wind widerstanden, weil sie sich viel schneller als mit Endgeschwindigkeit bewegten, das war der Moment, in dem Harry all das Zeug über Newtonsche Mechanik und Aristotelische Physik und Besen und Raketen und die Wichtigkeit von Neugier dachte und darüber, dass er nie wieder etwas derartig Gryffindor-typisches machen würde oder zumindest nicht, bis er das Geheimnis der Unsterblichkeit des Dunklen Lords erfahren hatte und warum er auf\\ Professor Quirinus "\emph{Ich versichere dir, Junge, ich würde das nicht versuchen, wenn ich nicht mit meinem eigenen Überleben rechnen würde}" Quirrell

statt auf Professor Michael "\emph{Junge, wenn du auf eigene Faust irgendetwas versuchst, was mit Raketen zu tun hat, ich meine irgendetwas ohne die Aufsicht eines ausgebildeten Professors, wirst du sterben und das wird Mama traurig machen}" Verres-Evans gehört hatte.

…\\ "\textbf{WAS?!}", kreischte Amelia am Spiegel.

…\\ Der Wind hatte sich auf ein erträgliches Maß gelegt, da der Luftwiderstand sie verlangsamte, was Harry reichlich Gelegenheit gab, dem summenden, klingelnden Geräusch zu lauschen, das sein ganzes Gehirn auszufüllen schien. Professor Quirrell hatte den Auspuff der Rakete mit einem Ruhezauber belegen sollen…\\ \emph{anscheinend gab es Grenzen für die Wirkung von Ruhezaubern…}\\ im Nachhinein betrachtet hätte Harry ein Paar Ohrstöpsel verwandeln sollen, statt sich nur auf den Ruhezauber zu verlassen, obwohl das wahrscheinlich auch nicht gereicht hätte…

\emph{Nun, magische Heilung hatte wahrscheinlich etwas, um dauerhafte Hörschäden zu behandeln. Nein, wirklich, magische Heilung hatte wahrscheinlich etwas, um das zu behandeln. Er hat schon Schüler mit Verletzungen zu Madam Pomfrey gehen sehen, die viel schlimmer klangen…}

\emph{Gibt es eine Möglichkeit, eine imaginäre Persönlichkeit in den Kopf eines anderen zu transplantieren?} fragte Hufflepuff. \emph{Ich will nicht mehr in deinem leben.}

Harry schob das alles in den Hinterkopf, es gab wirklich nichts, was er im Moment dagegen tun konnte.

\emph{Gab es irgendetwas, worüber er sich Sorgen machen sollte} -

Dann blickte Harry hinter sich und dachte zum ersten Mal daran, nachzusehen, ob Bellatrix oder Professor Quirrell vom Besen geweht worden war. Aber die grüne Schlange steckte immer noch in ihrem Gurtzeug, und die ausgemergelte Frau klammerte sich immer noch an den Besen, ihr Gesicht immer noch mit ungesunder Farbe aufgeladen und ihre Augen immer noch hell und gefährlich. Ihre Schultern zitterten, als ob sie hysterisch lachen würde, und ihre Lippen bewegten sich, als ob sie schreien wollte, aber es kam kein Ton heraus -

\emph{Oh, richtig.} Harry nahm die Kapuze seines Umhangs ab, tippte sich an die Ohren, um sie wissen zu lassen, dass er nichts hören konnte. Daraufhin ergriff Bellatrix ihren Zauberstab, richtete ihn auf Harry, und plötzlich wurde das Klingeln in seinen Ohren leiser, er konnte sie hören. Einen Moment später bereute er es; die Verwünschungen, die sie über Askaban, Dementoren, Auroren, Dumbledore, Lucius, Bartemy Couch, etwas, das sich der Orden des Phönix nannte, und alle, die sich ihrem Dunklen Lord in den Weg stellten, und so weiter, schrie, waren für jüngere und empfindlichere Zuhörer nicht geeignet; und ihr Lachen schmerzte in seinen frisch verheilten Ohren.

"Genug, Bella", sagte Harry schließlich, und ihre Stimme verstummte auf der Stelle. Es gab eine Pause. Harry zog den Umhang wieder über seinen Kopf, und erkannte im selben Moment, dass sie dort unten vielleicht Teleskope oder so etwas hatten, im Nachhinein betrachtet war es unglaublich dumme seine Kapuze auch nur für einen Moment herunterzuziehen. Er hoffte, dass nicht die ganze Mission an diesem einen Fehler scheiterte….

\emph{Wir sind nicht wirklich für so etwas geschaffen, oder?} beobachtete Slytherin.

\emph{Hey}, widersprach Hufflepuff aus reinem Reflex. \emph{Wir können nicht erwarten, irgendetwas beim ersten Mal perfekt zu machen, wir brauchen wahrscheinlich nur mehr Übung, VERGISS, DASS ICH DAS GESAGT HABE.}

Harry blickte wieder zurück, sah Bellatrix, die sich mit einem mit einem verwirrten, fragenden Gesichtsausdruck. Ihr Kopf drehte sich immer wieder, drehte sich. Und schließlich sagte Bellatrix, ihre Stimme war nun tiefer: "Mein Herr, wo sind wir?"

\emph{Was meinst du?} wollte Harry sagen, aber der Dunkle Lord würde nie zugeben, dass er etwas nicht versteht, also antwortete Harry trocken: "Wir sind auf einem Besen."

\emph{Glaubt sie, dass sie tot ist, dass dies der Himmel ist?}

Bellatrix' Hände waren immer noch an den Besenstiel gefesselt, also war es nur ein Finger, der hochkam und zeigte, als sie sagte: "Was ist das?"

Harry folgte der Richtung ihres Fingers und sah… eigentlich nichts Bestimmtes…\\ Dann wurde es Harry klar. Nachdem sie hoch genug aufgestiegen waren, hatte es keine Wolken mehr gegeben, die es verdeckt hätten.

"Das ist die Sonne, liebe Bella." Es kam bemerkenswert kontrolliert heraus, der Dunkle Lord klang vollkommen ruhig und vielleicht ein wenig ungeduldig mit ihr, selbst als die Tränen über Harrys Wangen liefen.

\emph{In der endlosen Kälte, in der pechschwarzen Nacht, wäre die Sonne sicherlich eine… Eine glückliche Erinnerung}…

Bellatrix' Kopf drehte sich weiter. "Und die flauschigen Dinger?", fragte sie.

"Wolken."

Es gab eine Pause, und dann sagte Bellatrix: "Aber was sind sie?"

Harry antwortete ihr nicht, seine Stimme hätte auf keinen Fall ruhig sein können, er konnte nur seine Atmung vollkommen regelmäßig halten, während er weinte. Nach einer Weile hauchte Bellatrix, so leise, dass Harry es fast nicht hörte hörte: "Hübsch…" Ihr Gesicht entspannte sich langsam, die Farbe verließ ihre Blässe fast so schnell, wie sie gekommen war. Ihr skelettartiger Körper sackte gegen den Besenstiel. Der geliehene Zauberstab baumelte leblos an dem Riemen, der an ihrer ihrer unbeweglichen Hand gebunden war.

\emph{DAS KANN DOCH NICHT WAHR SEIN.}

Harrys Verstand erinnerte sich dann, dass der Pfeffer-Trank einen Preis hatte. Bellatrix würde für eine beträchtliche Zeit schlafen, hatte Professor Quirrell gesagt. Und im selben Moment wurde ein anderer Teil von Harry vollkommen überzeugt und blickte zurück auf die kreideweiße, abgemagerte Frau, die im hellen Sonnenlicht die im hellen Sonnenlicht toter wirkte als alles, was Harry je lebendig gesehen hatte.

\emph{Dass sie tot war, dass sie gerade ihr letztes Wort gesprochen hatte dass Professor Quirrell die Dosierung falsch eingeschätzt hatte -}\\ \emph{- oder Bellatrix absichtlich geopfert hatte, um ihre eigene Flucht zu sichern}\\ \emph{- Atmet sie noch?}

Harry konnte nicht sehen, ob sie atmete. Auf dem Besen gab es keine Möglichkeit, nach hinten zu greifen und ihren Puls zu fühlen. Harry schaute nach vorn, um sicherzugehen, dass sie nicht in fliegende Steine trafen, und steuerte den Besen weiter auf die Sonne zu, der unsichtbare Jungen und die möglicherweise tote Frau, die in den Nachmittag reiten. Während seine Finger das Holz so fest umklammerten, dass sie weiß wurden. Er konnte nicht nach hinten greifen und eine künstliche Beatmung durchführen. Er konnte nichts aus seinem Heilerkoffer benutzen.

\emph{Konnte er Professor Quirrell vertrauen, dass er sie nicht gefährdet hatte?}

Es war seltsam, dass selbst der aufrichtige Glaube, dass Professor Quirrell nicht vorhatte, den Auror zu töten (denn das wäre wirklich dumm gewesen), nicht verhinderte dass sich die Beteuerungen des Professors nicht mehr beruhigend anfühlten.

Dann fiel Harry ein, dass er noch nicht nachgesehen hatte - Harry blickte zurück und zischte: "\emph{Lehrer}?".

Die Schlange rührte sich nicht in ihrem Gurtzeug und sagte kein Wort. …

\emph{Vielleicht war die Schlange, da sie kein echter Reiter war, nicht nicht vor der Beschleunigung geschützt. Oder vielleicht kam sie zu nah an die Dementoren ohne Schild , selbst für einen Moment in Animagus-Form, und das hatte den Verteidigungsprofessor außer Gefecht gesetzt. Das war nicht gut}. \emph{Es sollte Professor Quirrell sein, der Harry sagte, wann es wann es sicher war, den Portschlüssel zu benutzen.}

Harry lenkte den Besen mit bleichen Fingern und dachte nach, er dachte sehr intensiv nach, für eine kleine, nicht gemessene Zeitspanne, während der Bellatrix geatmet haben könnte oder auch nicht, während der Professor Quirrell selbst schon eine Weile nicht mehr geatmet haben könnte. Und Harry entschied, dass es zwar möglich war, sich von dem Fehler zu erholen, den Portschlüssel in seinem Besitz zu verschwenden, dass es aber nicht möglich war, sich von dem Fehler zu erholen, ein Gehirn zu lange ohne Sauerstoff zu lassen.

Also nahm Harry den nächsten Portschlüssel in der Reihe aus seiner Tasche, während er seinen Besen in der hellblauen Luft zum Stehen brachte (Harry wusste nicht, wenn er darüber nachdachte, ob die Fähigkeit eines Portschlüssels, sich der Erdrotation anzupassen, auch die Fähigkeit einschloss, die Geschwindigkeit im Allgemeinen an seine neue Umgebung anzupassen), berührte den Portschlüssel am Besen und…

Harry hielt inne, immer noch den Zweig in der Hand, das Gegenstück zu dem Zweig, den er vor gefühlten zwei Wochen abgebrochen hatte. Er verspürte einen plötzlichen Widerwillen; sein Gehirn schien durch einen rein neuronalen Prozess der operanten Konditionierung die Regel gelernt zu haben, dass das Schnappen von Zweigen eine schlechte Idee war. Aber das war eigentlich nicht logisch, also knackte Harry den Zweig trotzdem.

…\\ Es gab einen donnernden Knall hinter der nahen Metalltür, der Amelia dazu veranlasste, den Spiegel, den sie in der Hand hielt, fallen zu lassen und sich mit ihrem Zauberstab in der Hand herumzudrehen, und dann platzte die Tür auf und enthüllte Albus Dumbledore, der vor einem großen rauchenden Loch in der Gefängniswand stand.

"Amelia", sagte der alte Zauberer. Von seiner gewohnten Leichtigkeit war keine Spur zu sehen, seine Augen waren hart wie Saphire unter seiner Halbmondbrille. "Ich muss Askaban verlassen, und zwar sofort. Gibt es einen schnelleren Weg als einen Besen, um hinter die Absperrungen zu gelangen?"

"Nein -"

"Dann brauche ich deinen schnellsten Besen, und zwar sofort!"

Der Ort, an dem Amelia sein wollte, war bei dem Auror, der von diesem verfluchtem Feuer oder was immer es gewesen war, verletzt worden war. Aber sie musste noch herausfinden, was Dumbledore wusste.\\ "Ihr!", bellte die alte Hexe das Team um sie herum an. "Räumt weiter die Korridore ab, bis ihr ganz unten seid, vielleicht sind sie noch nicht alle entkommen!" Und dann, an den alten Zauberer gewandt:\\ "Zwei Besen. Du kannst mich einweisen, wenn wir in der Luft sind."

Es gab einen Schlagabtausch der Blicke, aber nicht lange.

…\\ Ein unangenehm harter Ruck erfasste Harrys Unterleib, wesentlich härter als der Ruck, der ihn nach Askaban befördert hatte, und diesmal war die zurückgelegte Strecke groß genug, dass er einen Augenblick lang die Stille hören, den unsichtbaren Raum zwischen den Räumen beobachten konnte, in der Spalte zwischen einem Ort und einem anderen. Die Sonne, die nur kurz auf die beiden geschienen hatte, wurde schnell von einer Regenwolke verdeckt, als sie von Askaban wegschossen, in Richtung des Windes und schneller als der Wind.

…\\ "Wer steckt dahinter?", rief Amelia dem Besen zu, der einen Schritt von ihr wegflog.

"Einer von zwei Zauberern", gab Dumbledore zurück, "ich weiß in diesem Moment nicht, wer. Wenn es der erste ist, dann sind wir in Schwierigkeiten. Wenn es der zweite ist, dann sind wir alle in weitaus größeren Schwierigkeiten."

Amelia machte keine Anstalten zu seufzen. "Wann wirst du es wissen?"

Die Stimme des alten Zauberers war grimmig, leise und doch irgendwie über den Wind erhaben. "Drei Dinge braucht der eine für die Vollkommenheit, wenn es das ist: Das Fleisch des treuesten Dieners des Dunklen Lords, das Blut des größten Feindes des Dunklen Lords und den Zugang zu einem bestimmten Grab. Ich hatte Harry Potter für sicher gehalten, da ihr Versuch, Askaban zu stürmen, so gut wie gescheitert war - obwohl ich immer noch Wachen auf ihn angesetzt habe -, aber jetzt habe ich wirklich Angst. Sie haben Zugang zur Zeit, jemand mit einem Zeitdreher schickt Nachrichten für sie; und ich vermute, dass der Entführungsversuch auf Harry Potter schon vor einigen Stunden stattgefunden hat. Deshalb haben wir nichts davon gehört, weil wir in Askaban sind, wo die Zeit sich nicht verknoten kann. Diese Vergangenheit kam nach unserer eigenen Zukunft, verstehst du."

"Und wenn es der andere ist?", rief Amelia. \emph{Was sie schon gehört hatte, war beunruhigend genug; das klang nach dem dunkelsten aller dunklen Rituale, und im Mittelpunkt stand der tote Dunkle Lord selbst.}

Der alte Zauberer, dessen Gesicht jetzt noch grimmiger war, sagte nichts, schüttelte nur den Kopf.

…\\ Als das Ziehen des Portschlüssels nachgelassen hatte, lugte die Sonne gerade erst über den Horizont und sah eher wie eine Morgendämmerung als wie ein Sonnenuntergang aus, als ihr Besen tief über einer kurzen Fläche aus dunkel-orangem Fels und Sand schwebte, die zu klumpigen Hügeln angeordnet waren, als hätte jemand den Teig des Landes ein paar Mal geknetet und dann vergessen, ihn flach zu walzen. In der nahen Ferne rollten Wellen in einer endlosen Wasserfläche vorbei, obwohl der Boden, über dem der Besen schwebte, mindestens einige Meter über dem Meeresspiegel lag. Harry blinzelte über die Farben der Morgendämmerung und bemerkte dann, dass der Portschlüssel international gewesen war.

"Oy!", kam ein lebhafter, weiblicher Ruf von hinter ihm, und Harry drehte sich, um nachzusehen. Eine Dame mittleren Alters hielt sich eine Hand vor den Mund, um ihn zu rufen, und eilte nach vorne. Ihre freundlichen Gesichtszüge, die schmalen Augen und die bernsteinfarbene Haut kennzeichneten eine Kultur, die Harry nicht kannte; sie war in leuchtend violette Roben gekleidet, die Harry noch nie gesehen hatte; und als sich ihre Lippen wieder öffneten, sprach sie mit einem Akzent, den Harry nicht einordnen konnte, denn er war noch nicht weit gereist.

"Wo warst du? Du bist zwei Stunden zu spät! Ich hätte euch fast aufgegeben … Hallo?" Es gab eine kurze Pause.

Harrys Gedanken schienen sich seltsam zu bewegen, zu langsam, alles fühlte sich weit weg an, als wäre eine dicke Glasscheibe zwischen ihm und der Welt, und eine weitere dicke Glasscheibe zwischen ihm und seinen Gefühlen, so dass er zwar sehen, aber nicht berühren konnte. Es war über ihn gekommen, als er das Licht der Morgendämmerung und die freundliche Hexe sah und dachte, dass das alles wie ein angemessenes Ende des Abenteuers aussah.

Dann eilte die Hexe herbei und zog ihren Zauberstab; ein gemurmeltes Wort löste die Fesseln, die die ausgemergelte Frau an den Besenstiel banden, und Bellatrix wurde auf den sandigen Felsen hinuntergelassen, wobei ihre skelettartigen Arme und bleichen Beine wie leblose Dinge baumelten.

"Oh, Merlin", flüsterte die Hexe, "Merlin, Merlin, Merlin …"

\emph{Sie wirkt besorgt,} dachte ein abstraktes, fernes Ding zwischen zwei Glasscheiben. I\emph{st es das, was ein echter Heiler sagen würde, oder ist es das, was jemand sagen würde, dem man sagt, er solle eine Vorstellung geben?}

Als wäre es nicht Harry, der sprach, sondern ein anderer Teil von ihm hinter einer weiteren Glasscheibe, kam ein Flüstern von seinen Lippen.\\ "Die grüne Schlange auf ihrem Rücken ist ein Animagus." Nicht hoch das Flüstern, nicht kalt, nur leise. "Er ist bewusstlos."

Der Kopf der Hexe zuckte hoch, um dorthin zu blicken, wo die Stimme aus der leeren Luft zu sprechen schien, und blickte dann wieder zu Bellatrix hinunter.

"Sie sind nicht Mister Jaffe."

"Das wäre der Animagus", flüsterten Harrys Lippen.\\ \emph{Oh}, dachte der Harry hinter Glas, der dem Klang seiner eigenen Lippen lauschte, \emph{das macht Sinn; Professor Quirrell muss einen anderen Namen benutzt haben.}

"Seit wann ist er ein - bah, vergiss es." Die Hexe legte ihren Zauberstab einen Moment lang auf die Nase der Schlange, dann schüttelte sie heftig den Kopf. "Mit ihm ist nichts los, was ein Tag Ruhe nicht heilen könnte. Sie …"

"Kannst du ihn aufwecken?", flüsterte Harrys Lippen.\\ \emph{Ist das eine gute Idee?} dachte Harry, aber seine Lippen schienen definitiv anders zu denken.

Wieder das scharfe Kopfschütteln. "Wenn ein Innervate bei ihm nicht gewirkt hat -", begann die Hexe.

"Ich habe es nicht versucht", flüsterten Harrys Lippen.

"Was? Warum - oh, egal. Innervate."

Es gab eine Pause, und dann kroch langsam eine Schlange aus ihrem Gurt. Langsam kam der grüne Kopf hoch, schaute sich um. Einen Wimpernschlag später stand Professor Quirrell, und einen Moment später war er auf die Knie gesunken.

"Leg dich hin", sagte die Hexe, ohne von Bellatrix aufzublicken. "Bist du das da drinnen, Jeremy?"

"Ja", sagte der Verteidigungsprofessor ziemlich heiser, als er sich vorsichtig auf einen relativ flachen Fleck aus sandigem, orangefarbenem Gestein legte. Er war nicht so blass wie Bellatrix, aber sein Gesicht war im schummrigen Dämmerlicht blutleer. "Ich grüße Sie, Miss Camblebunker."

"Ich habe dir gesagt", sagte die Hexe mit Schärfe in der Stimme und einem leichten Lächeln auf dem Gesicht, "nenn mich Crystal, das hier ist nicht Britannien, und ich will hier nichts von eurer Förmlichkeit haben. Und es heißt jetzt Doktor, nicht Miss."

"Ich bitte um Entschuldigung, Doktor Camblebunker."

Es folgte ein trockenes Kichern. Das Lächeln der Hexe wurde noch ein wenig breiter, ihre Stimme noch schärfer. "Wer ist Ihr Freund?"

"Das brauchen Sie nicht zu wissen." Die Augen des Verteidigungsprofessors waren geschlossen, als er auf dem Boden lag.

"Wie schief ist es gelaufen?"

\emph{Sehr trocken:} "Das können Sie morgen in jeder Zeitung mit einem internationalen Teil nachlesen."

Der Zauberstab der Hexe tippte hierhin und dorthin, stocherte und stupste am ganzen Körper von Bellatrix herum. "Ich habe dich vermisst, Jeremy."

"Wirklich?", sagte der Verteidigungsprofessor und klang leicht überrascht.

"Nicht einmal ein klitzekleines bisschen. Wenn ich dir nicht etwas schulden würde -"

Der Verteidigungsprofessor begann zu lachen, und dann verwandelte es sich mehr in einen Hustenanfall.

\emph{Was denkst du?} sagte Slytherin zum inneren Kritiker, während Harry hinter den Glaswänden zuhörte. \emph{Vorstellung oder Realität?}

\emph{Kann ich nicht sagen,} sagte Harrys Innerer Kritiker. \emph{Ich bin im Moment nicht in bester kritischer Form.}

\emph{Fällt jemandem eine gute Sonde ein, um mehr Informationen zu sammeln?} sagte der Ravenclaw.

Wieder dieses Geflüster aus der leeren Luft über dem Besen: "Wie groß ist die Chance, all das ungeschehen zu machen, was man ihr angetan hat?"

"Oh, lass mal sehen. Legilimenz und unbekannte dunkle Rituale, zehn Jahre lang, bis sich das festgesetzt hat, gefolgt von zehn Jahren Dementor-Exposition? Das rückgängig machen? Sie haben den Verstand verloren, Mister "\emph{Wer-auch-immer-Sie-sind}". Die Frage ist, ob noch etwas übrig ist, und ich würde sagen, dass die Chance vielleicht eins zu drei -"\\ Die Hexe unterbrach sich plötzlich. Ihre Stimme, als sie wieder sprach, war leiser.\\ "Wenn du ihr Freund warst, vorher … dann nein, du wirst sie nie wieder zurückbekommen. Das solltest du jetzt verstehen."

\emph{Ich stimme dafür, dass dies eine Vorstellung ist,} sagte der innere Kritiker. \emph{Sie würde das alles nicht einfach auf eine Frage hin ausplaudern, es sei denn, sie suchte nach einer Gelegenheit.}

\emph{Zur Kenntnis genommen, aber ich schätze das als wenig vertrauenswürdig ein,} sagte der Ravenclaw. \emph{Es ist sehr schwer, sich nicht von seinen Verdächtigungen leiten zu lassen, wenn man versucht, so subtile Beweise abzuwägen.}

"Welchen Trank hast du ihr gegeben?", fragte die Hexe, nachdem sie Bellatrix' Mund geöffnet und hineingespäht hatte, wobei ihr Zauberstab in mehreren Farben aufblitzte.

Der Mann, der auf dem Boden lag, sagte ruhig: "Pfeffer-Trank -"

"Warst du nicht bei Sinnen?!"

Wieder das hustende Lachen.

"Sie wird eine Woche schlafen, wenn sie Glück hat", sagte die Hexe und schnalzte mit der Zunge. "Ich werde dir eine Eule schicken, wenn sie ihre Augen öffnet, damit du zurückkommst und ihr den unbrechbaren Schwur aufhalsen kannst. Hast du irgendetwas, das sie davon abhält, mich auf der Stelle zu töten, wenn sie es schafft, sich auch nur einen Moment lang zu bewegen?"

Der Verteidigungsprofessor, die Augen immer noch geschlossen, nahm ein Blatt Papier aus seiner Robe; einen Moment später begannen darauf Worte zu erscheinen, begleitet von winzigen Rauchschwaden. Als der Rauch aufgehört hatte, aufzusteigen, schwebte das Papier zu der Frau hinüber. Die Frau betrachtete das Papier mit hochgezogenen Augenbrauen und schnaubte hämisch. "Das sollte besser funktionieren, Jeremy, oder mein letzter Wille besagt, dass mein ganzes Vermögen dafür verwendet wird, ein Kopfgeld auf dich auszusetzen. Wo wir gerade dabei sind -"

Der Verteidigungsprofessor griff wieder in seine Robe und warf der Hexe einen Beutel zu, der ein klirrendes Geräusch machte. Die Hexe fing ihn auf, wog ihn und gab einen zufriedenen Laut von sich. Dann stand sie auf, und die blasse, skelettierte Frau schwebte neben ihr vom Boden auf. "Ich gehe zurück", sagte die Hexe. "Ich kann hier nicht mit meiner Arbeit beginnen."

"Warte", sagte der Verteidigungsprofessor und holte mit einer Geste seinen Zauberstab aus Bellatrix' Hand und Gurt. Dann richtete er den Zauberstab auf Bellatrix und bewegte sich in einer kleinen kreisförmigen Geste, begleitet von einem leisen "Obliviate."

"Das war's", schnappte die Hexe, "ich bringe sie hier raus, bevor ihr noch mehr Schaden zugefügt wird -" Ein Arm kam herum, um die knochige Gestalt von Bellatrix Black an ihre Seite zu drücken, und sie verschwanden beide mit dem lauten \textbf{\emph{POP}}!

Und es war still an diesem klumpigen Ort, bis auf das sanfte Rauschen der vorbeiziehenden Wellen und einen kleinen Windhauch.

\emph{Ich denke, die Vorstellung ist beendet,} sagte der innere Kritiker. \emph{Ich gebe ihr zweieinhalb von fünf Sternen. Sie ist wahrscheinlich keine sehr erfahrene Schauspielerin.}

\emph{Ich frage mich, ob ein echter Heiler falscher wirken würde als ein Schauspieler, dem man sagt, er solle} \emph{einen spielen}? überlegte der Ravenclaw.

\emph{Wie eine Fernsehsendung,} so fühlte es sich an, \emph{wie eine Fernsehsendung, mit deren Charakteren man nicht besonders mitfühlte, das war alles, was man hinter den Glaswänden sehen und fühlen konnte.}

Irgendwie schaffte Harry es, seine Lippen selbst zu bewegen, seine eigene Stimme in die stille Morgenluft hinauszuschicken, und war dann überrascht, seine eigene Frage zu hören.

"Wie viele verschiedene Menschen sind Sie eigentlich?"

Der bleiche Mann, der am Boden lag, lachte nicht, aber vom Besenstiel aus sahen Harrys Augen die Seiten von Professor Quirrells Lippen, die sich kräuselten, den Rand dieses bekannten sardonischen Lächelns.\\ "Ich kann nicht behaupten, dass ich mir die Mühe gemacht habe, mitzuzählen. Wie viele bist du?"

Es hätte den inneren Harry nicht so sehr erschüttern sollen, diese Antwort zu hören, und doch fühlte er sich - er fühlte sich - instabil, als ob sein eigenes Zentrum abgezogen worden wäre -- \emph{Oh}.

"Entschuldigen Sie", sagte Harrys Stimme. Sie klang jetzt so distanziert und losgelöst, wie der schwindende Harry sich fühlte. "Ich werde in ein paar Sekunden ohnmächtig, glaube ich."

"Benutze den vierten Portschlüssel, den ich dir gegeben habe, den, von dem ich gesagt habe, dass er unser Rückzugsort ist", sagte der Mann, der auf dem Boden lag, ruhig, aber schnell. "Dort wird es sicherer sein. Und trage weiterhin den Umhang."

Harrys freie Hand holte einen weiteren Zweig aus seiner Tasche und knickte ihn. Es gab ein weiteres, internationales Ruckeln, und dann war er irgendwo schwarz.

"Lumos", kam es über Harrys Lippen, denn ein Teil von ihm dachte an die Sicherheit des Ganzen. Er befand sich in etwas, das wie ein Muggel-Lagerhaus aussah, ein verlassenes.

Harrys Beine kletterten vom Besen und lagen auf dem Boden. Er schloss die Augen, und ein kleiner Teil seines Selbst wollte, dass das Licht ausging, bevor ihn die Dunkelheit einholte.

…\\ "Wo willst du hin?", rief Amelia. Sie waren fast am Rande der Schutzmauern.

"Zurück in der Zeit, um Harry Potter zu beschützen", sagte der alte Zauberer, und bevor Amelia auch nur die Lippen öffnen konnte, um zu fragen, ob er Hilfe brauchte, spürte sie die Grenze der Schutzzauber, als sie sie überquerten. Es gab einen Knall, und der Zauberer und der Phönix verschwanden und ließen den geliehenen Besen zurück.

