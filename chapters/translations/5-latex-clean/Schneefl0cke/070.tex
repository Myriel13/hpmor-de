

\hypertarget{selbstverwirklichung-teil-6}{% \section{71. Selbstverwirklichung, Teil 6}\label{selbstverwirklichung-teil-6}}

\textbf{\uline{Selbstverwirklichung, Teil 6}}

"Na ja", flüsterte Daphne, wobei sie ihre Stimme so leise wie möglich hielt, "wenigstens fühle ich mich jetzt nicht mehr als die einzige vernünftige Person in Hogwarts."

"Weil du jetzt den Rest von uns als Freunde hast?", flüsterte Lavender Brown, die auf Zehenspitzen an ihrer linken Seite entlangschlich.

"Ich glaube nicht, dass sie das meint", murmelte General Granger von Lavenders eigener linker Seite.

Sie schlichen langsam und vorsichtig durch die Korridore von Hogwarts, alle acht hielten die Ohren nach dem leisesten Geräusch von Ärger offen, als wäre es eine Schlacht und sie suchten nach feindlichen Soldaten, die sie in einen Hinterhalt locken konnten; nur in diesem Fall suchten sie nach Tyrannen, die sie besiegen konnten, und nach Opfern, die sie retten konnten, in der Zeitspanne zwischen dem Ende der Frühstückszeit und dem Zeitpunkt, an dem Lavender und Parvati zu ihrem Kräuterkurs gehen mussten. Lavender hatte argumentiert, dass, wenn eine Erstklässlerin drei ältere Tyrannen besiegen konnte, acht Erstklässlerinnen aufgrund der Multiplikation in der Lage sein müssten, vierundzwanzig ältere Tyrannen zu besiegen. Nach ihrem hektischen Geplapper und Fuchteln mit den Händen zu urteilen, hatte General Granger das nicht überzeugend gefunden. Padma hatte während des darauf folgenden Streits eine Weile geschwiegen und dann nachdenklich bemerkt, dass es selbst in Hogwarts wahrscheinlich nicht gut für den Ruf als Tyrann wäre, Erstklässlerinnen zu verprügeln. Parvati hatte sich daraufhin aufgerichtet und ausgerufen, dass dies bedeutete, dass sie die Einzigen waren, die etwas gegen das Tyrannenproblem in Hogwarts tun konnten, was es wirklich zu einem echten Heldeneinsatz machte. Außerdem war der ganze Grund, warum ihre Eltern nach Großbritannien gezogen waren, der, dass die beiden die einzige magische Schule der Welt mit einer Todesrate von 0 \% besuchen konnten, und was sollte das bringen, wenn sie nicht den Vorteil nutzen und ein paar Dinge ausprobieren würden? Worauf General Granger geantwortet hatte, dass Parvati den Sinn einer perfekten Sicherheitsbilanz überhaupt nicht verstand - Lavender hatte gesagt, wenn sie wirklich alle zusammen Freunde seien und nicht Hermines Gefolgsleute, wie Professor Quirrell dachte, dann sollten sie über solche Dinge abstimmen.

Daphne hatte erwartet, dass ihre Stimme den Ausschlag geben würde, nachdem Hermine und Susan und Hannah mit Nein gestimmt hatten. Und so hatte Daphne es sich gut überlegt, nachdem ihre erste Begeisterung abgeklungen war. Sie war schließlich eine Slytherin, und das bedeutete, dass es ihre Aufgabe war, ein wachsames Auge auf ihre eigenen Interessen zu haben, während sie alle herumliefen und versuchten, Menschen zu helfen - ihre Aufgabe, herauszufinden, wie riskant es wirklich war und ob es sich für sie lohnen würde, genau wie Mutter es an ihrer Stelle getan hätte. Immer auf sich selbst und seine Freunde aufzupassen, das war es, was einen echten Slytherin ausmachte…

Hannah Abbott, das nervöse kleine Hufflepuff-Mädchen, hatte mit kleiner, zitternder Stimme "Ja" gesagt. Und nun mussten Daphne und Susan und Hermine bei den anderen fünf bleiben, sie konnten die anderen unmöglich alleine losziehen lassen. Denn kein Gryffindor würde es je verkraften, das letzte überlebende Kind der Bones-Familie zu verletzen, und kein Slytherin würde es wagen, eine Tochter des edlen und uralten Hauses Greengrass anzugreifen. (\emph{Daphne hoffte es jedenfalls}.)\\ Und General Granger, die die ganze Sache angefangen hatte … man brauchte gar nicht erst zu fragen.

Die Gänge von Hogwarts zogen an ihnen vorbei, einer nach dem anderen, die angespannten Hände nie weit von ihren Zauberstäben entfernt, als Stein und Holz und Eewig brennende Fackeln in Sicht kamen und dann vorbeizogen. An einem Punkt hörten sie Schritte und holten tief Luft, die Hände schnellten zu ihren Zauberstäben, aber es war nur ein einsamer älterer Ravenclaw, der sie neugierig ansah, bevor er Ihnen zunickte und seinen Kopf wieder auf sein Buch sinken ließ, während er weiterging. Die Heldinnen schlichen an feierlichen, mit vergoldeten Fresken verzierten Eichenholzpaneelen vorbei und kamen in eine Sackgasse, die in eine Jungen-Toilette führte, und drehten um und wanderten zurück durch die feierlichen, mit vergoldeten Fresken verzierten Eichenholzpaneele und bogen dann durch staubige, alte, mit abgenutztem Zement verfugte Ziegelsteinkorridore ab, die sie eigentlich in einen Kreis führten, also konsultierten sie ein Porträt und gingen dann stattdessen einen anderen staubigen, alten Ziegelsteinkorridor hinunter, der sie zu einer kurzen Marmortreppe führte, die sie eigentlich in den dritten Stock hätte bringen müssen, wenn es irgendwo anders als in Hogwarts gewesen wäre, und dann ging es wieder zurück zu gefliestem Steinpflaster und Oberlichtern, durch die Sonnenstrahlen fielen, obwohl sie sich nicht in der Nähe des Daches befanden, und nachdem sie diesem Gang um ein paar Ecken gefolgt waren, führte er sie zu einer weiteren Jungen-Toilette, die deutlich mit einem Schild gekennzeichnet war, das die Silhouette einer gewandeten Gestalt zeigte, die in eine Toilette huschte. Zu acht standen sie vor der geschlossenen Tür und starrten mit einer gewissen Müdigkeit.

"Mir ist langweilig", sagte Lavender.

Padma machte eine Show daraus, eine Taschenuhr aus ihren Roben zu holen und darauf zu schauen. "Sechzehn Minuten und dreißig Sekunden", sagte sie. "Ein neuer Rekord für die längste Aufmerksamkeitsspanne in Gryffindor."

"Ich glaube auch nicht, dass das klappen wird", sagte Susan. "Und ich bin eine Hufflepuff."

"Weißt du", sagte Lavender nachdenklich, "ich frage mich, ob vielleicht das, was jemanden wirklich zu einem Helden macht, ist, dass, wenn er so etwas versucht, tatsächlich etwas Interessantes passiert."

"Ich wette, du hast recht", sagte Tracey. "Ich wette, wenn wir Harry Potter dabei hätten, würden wir in den ersten fünf Minuten auf drei Tyrannen und einen versteckten Raum voller Schätze stoßen. Ich wette, dass General Chaos nur auf die Toilette gehen muss, und schon findet er die Kammer des Schreckens von Slytherin oder so -"

Das konnte Daphne nicht so ganz durchgehen lassen. "Glaubst du, Lord Slytherin hätte den Eingang zur Kammer des Schreckens in ein Badezimmer gelegt -"

"Was ich sagen will", sagte Susan, als Tracey den Mund zu einer Antwort öffnete, "ist, dass wir keine Möglichkeit haben, irgendwelche Tyrannen zu finden. Ich meine, alles, was \emph{sie} tun müssen, ist, irgendwo einen Hufflepuff zu finden, aber wir müssen ihnen genau zur richtigen Zeit über den Weg laufen, versteht ihr? Was ein sehr großes Problem ist, denn wenn wir sie finden würden, würden wir alle wie Käfer zerquetscht werden. Können wir nicht einfach den verbotenen Korridor im dritten Stock durchwandern, so wie wir es sollten?"

Lavender schnaubte verächtlich. "Man wird keine echte Heldin, wenn man nur die verbotenen Dinge tut, die einem der Schulleiter verbietet!"

(Daphnes Verstand versuchte, diese Aussage zu verarbeiten, während sie im Stillen dem Sprechenden Hut dafür dankte, dass er sie nicht in die Nähe von Gryffindor gesteckt hatte.)

"Wenn ich so darüber nachdenke …" sagte Parvati langsam: "Ich meine, wie groß ist die Wahrscheinlichkeit, dass Harry Potter an seinem ersten Schulmorgen auf diese fünf Tyrannen trifft? Er muss einen Weg gefunden haben, sie zu finden."

Daphne stand zufällig an der Stelle, von der aus sie Parvati und Hermine sehen konnte, und so bemerkte sie, wie sich der Gesichtsausdruck des Ravenclaw-Mädchens veränderte - und dann wurde ihr klar, dass auch die Sonnenschein-Generalin erst vor kurzem ein paar Tyrannen gefunden hatte -

"Oh!", sagte Padma in einem Ton plötzlicher Erkenntnis. "Natürlich! Er wurde vom Geist von Salazar Slytherin geleitet!"

"Was?", sagte Daphne gleichzeitig mit mehreren anderen Leuten.

"Das war der Geist, der mich erschreckt hat, da bin ich mir ziemlich sicher", erklärte Padma. "Ich meine, ich habe es erst hinterher herausgefunden, aber … ja. Salazar Slytherins Geist mag es nicht, wenn Slytherins Leute schikanieren, er denkt, dass es seinem Namen Schande macht, und der Geist ist immer noch in die Hogwarts-Mauern eingebrannt, also weiß er alles, was passiert, wette ich."

Daphnes Mund stand offen; und sie sah, dass Hannah eine Hand an die Stirn gelegt hatte und sich gegen die Steinmauern lehnte, während Traceys Augen wie kleine braune Sterne loderten.

\emph{Der Geist von Salazar Slytherin? Hatte sich mit Harry Potter verbündet? Und hatte Hermine Granger geschickt, um Derricks Mannschaft aufzuhalten? Sie hätte 100 Galleonen bezahlt, um dabei zu sein, wenn Draco Malfoy davon erfährt. Obwohl, wenn man bedenkt, wie schnell sich Gerüchte in Hogwarts verbreiten, jetzt, wo Padma es ausgeplaudert hatte, hatte Millicent es ihm wahrscheinlich schon vor dreißig Minuten gesagt… In der Tat… jetzt, wo Daphne darüber nachdachte…}

"Also", sagte Parvati. "Wir müssen den Jungen-der-überlebte fragen, wo Salazar Slytherins Geist zu finden ist? Wow, ich schätze, wenn ich so etwas laut sage, werde ich vielleicht tatsächlich zu einer Heldin -"

"Ja!", sagte Lavender. "Wir müssen den Jungen-der-lebte fragen, wo Salazar Slytherins Geist zu finden ist!"

"Wir müssen… den Jungen-der-lebte… fragen, wo der Geist von Salazar Slytherin zu finden ist…", wiederholte Hannah mit nervöser Stimme, als würde sie sich zwingen, es zu sagen.

"Und wenn das nicht klappt", rief Tracey, "dann betäuben wir Harry Potter, fesseln ihn und nehmen ihn mit!"

\emph{Das sagte etwas aus,} dachte Hermine Granger, und es war etwas ziemlich Trauriges - als die acht zurück durch das Labyrinth der verwinkelten kleinen Gänge schlenderten, das Hogwarts war, nachdem ihre Zeit vor der nächsten Klasse abgelaufen war, ohne dass sie irgendwelche Tyrannen gefunden hatten -, \emph{dass sie wirklich nicht wusste, ob Harry Potter vom Geist von Salazar Slytherin oder einem Phönix oder was auch immer herumgeführt worden war. Und was auch immer Harry getan hatte, sie hoffte, dass es bei ihnen nicht funktionierte. Und vor allem hoffte sie, dass die anderen nicht für Traceys Idee stimmten, Harry Potter zu betäuben und seinen bewusstlosen Körper mit sich herumzuschleppen, um Abenteuer anzulocken. Das konnte im wirklichen Leben unmöglich funktionieren, und wenn doch, wollte sie aufgeben.}

Hermine schaute von Hexe zu Hexe, Tracey plauderte mit Lavender, die anderen machten gelegentliche Bemerkungen, und ihr Blick blieb an einem Mädchen hängen, das gedämpft und still war, die einzige Person, deren Gedanken sie im Moment überhaupt nicht erraten konnte.

"Hannah?", sagte sie zu dem Mädchen, das neben ihr lief. Hermine versuchte, ihre Stimme so sanft wie möglich zu machen. "Du musst nicht antworten, aber ist es in Ordnung, wenn ich frage, warum du für den Kampf gegen Tyrannen gestimmt hast?"\\ Hermine hatte gedacht, sie hätte ihre Stimme sanft gemacht, aber alle blieben stehen, und Lavender und Tracey unterbrachen ihr Gespräch und sahen sie an.

Hannahs Wangen röteten sich bereits, und gerade als Hannah den Mund öffnete -

"Es ist, weil sie offensichtlich mehr Mut hat, als du denkst", sagte Lavender.

Hannah hielt mit offenem Mund inne. Sie schloss ihren Mund. Sie schluckte, hart und sichtbar, während sich ihre Wangen noch mehr röteten. Dann holte Hannah tief Luft und sagte mit leiser Stimme: "Es gibt einen Jungen, den ich mag." Das Hufflepuff-Mädchen zuckte zusammen, als sie das sagte, und ihr Kopf huschte nervös herum, um alle anzuschauen, die sie ansahen, während sich die Pause und die Stille dehnten.

"Ähm, okay?" sagte Susan schließlich.

"Ich habe fünf Jungs, die ich mag", sagte Lavender.

"Padma und ich wussten, dass wir beide die gleichen Jungs mögen", sagte Parvati, "also haben wir eine Liste gemacht und einen Knut geworfen, um zu sehen, wer zuerst wählen darf."

"Ich weiß, wen ich heiraten werde", sagte Tracey. "Es ist mir egal, was die Welt sagt, er ist dazu bestimmt, mir zu gehören!"

Das ließ alle anderen Mädchen erwartungsvoll zu Hermine blicken, deren Gehirn Traceys letzte Aussage komplett verdrängt hatte, um sich nur auf das erste, was Hannah gesagt hatte, konzentrieren zu können. "Ähm", sagte Hermine. Sie fuhr vorsichtig fort und hielt ihre Stimme sanft. "Hannah, der Grund, warum du der Gesellschaft zur Förderung der heldenhaften Gleichberechtigung von Hexen beigetreten bist, war, dass es einen Jungen gibt, der dich vielleicht mehr mag, wenn du ein Held bist?"

Das Hufflepuff-Mädchen nickte erneut, wobei sich ihre Wangen noch mehr röteten, während sie auf ihr eigenes Spiegelbild in ihren schwarz polierten Schuhen starrte.

"Eigentlich mag sie Neville Longbottom", sagte Daphne. Die Slytherin stieß einen wehmütigen Seufzer aus. "Und unglücklicherweise für sie wird er eine andere heiraten. Es ist sehr tragisch."

Das brachte einen hohen, piepsenden Ton von Hannah hervor, die weiterhin auf ihre Füße starrte.

"Wie bitte?", fragte Lavender. "Neville wird eine andere heiraten? Woher weißt du davon? Wer?"

Daphne schüttelte nur traurig und mit niedergeschlagener Miene den Kopf.

"Entschuldigung", sagte Hermine, und dann, als die anderen sie wieder ansahen, "Ah…", während sie versuchte, ihre Gedanken zu ordnen. "Ich meine, ähm … Hannah … zu versuchen, ein Held zu werden, damit ein Junge dich mag, ist nicht sehr feministisch."

"Es wird eigentlich feminin ausgesprochen", sagte Padma.

"Und warum nennst du Hannah unweiblich?", fragte Susan. "Es ist nichts Unweibliches daran, einen Jungen beeindrucken zu wollen."

"Außerdem", sagte Parvati und klang verwirrt, "geht es nicht darum, dass wir versuchen, Helden zu sein, obwohl das nicht feminin ist?"

Die folgende Diskussion würde Hermine Granger nicht als einer ihrer erfolgreichsten Streifzüge in die Gefilde der politischen Bildung in Erinnerung bleiben. Sie versuchte es zu erklären, und nach dem darauf folgenden Streit versuchte sie es noch einmal, während die anderen sieben Mädchen sie immer skeptischer ansahen. Danach erklärte Daphne im herrischen Tonfall der zukünftigen Lady Greengrass, dass, wenn diese Sache mit dem Feminismus bedeute, dass es Mädchen nicht erlaubt sei, Jungen zu nehmen, wie es ihnen gefiele, dann könne der Feminismus in den Muggelländern bleiben, wo er hingehöre. Lavender schlug vor, dass der Hexismus vielleicht besagen könnte, dass Hexen alles tun dürfen, was sie wollen, was sich nach mehr Spaß anhört als Feminismus. Und schließlich beendete Padma die weitere Diskussion, indem sie müde feststellte, dass sie keinen Sinn darin sah, weiter zu streiten, da es bei S.P.H.E.W. gar nicht um irgendetwas ging, was mit Feminismus zu tun hatte, sondern nur darum, dass mehr Mädchen zu Heldinnen wurden. Hermine hatte an diesem Punkt aufgegeben.

Als die Zauberstunde an diesem Tag zu Ende war und die Ravenclaws des ersten Jahrgangs aus der Klasse schlurften, zuckte Hermine bereits mit den Schultern. Sie hatten es gerade noch vor dem Eröffnungsgong in die Klasse geschafft, sie mussten sofort zu ihren Tischen rennen und sich hinsetzen, so dass noch keine Zeit für die schreckliche Sache gewesen war; aber das bedeutete nur, dass Hermine sich auf die kommende Katastrophe für die ganze Klasse freuen konnte. Nachdem Professor Flitwick seine Entlassung gequietscht hatte und sich alle von ihren Stühlen erhoben hatten, begann Harry auf sie zuzugehen; und Hermine schob ihrerseits ihr Buch in ihren Beutel und ging sehr schnell zur Tür hinüber, warf sie auf und ging in die Korridore, und natürlich folgte Harry ihr mit einem überraschten Blick, denn sie hatten eine Bibliotheksstunde angesetzt -

"Hermine?" sagte Harry, als er die Tür hinter sich schloss. "Was ist los?"

Die Tür flog hinter Harry auf, nicht einen Moment nachdem er sie geschlossen hatte, und traf Harry fast, als er aus dem Weg trat, und Padma Patil trat aus dem Klassenzimmer mit einem furchtbar entschlossenen Blick auf ihrem Gesicht.

"Entschuldigen Sie, Mr. Potter", kamen die schrecklichen Worte, und die hohe Stimme des jungen Mädchens hallte durch den Korridor wie die düsteren Glocken des Unheils, "kann ich Sie um Hilfe bei etwas bitten?"

Harrys Augenbrauen zogen sich hoch, und er sagte: "Sie können natürlich fragen."

"Kannst du uns sagen, wie wir mit dem Geist von Salazar Slytherin sprechen können? Wir wollen, dass er uns sagt, wo wir Tyrannen finden können, so wie er es dir sagt."

Im Korridor vor dem Klassenzimmer herrschte ein wenig Stille. Die Tür öffnete sich wieder, und Susan spähte mit einem fragenden Blick hinaus -\\ "Nun, wir müssen zur Bibliothek", sagte Harry ganz beiläufig, sein Gesicht sah entspannt aus, "würde es dir etwas ausmachen, uns zu folgen?" und begann in die Richtung zu gehen, die an ungeraden Tagen des Monats zur Bibliothek führte, und Susan tat so, als wolle sie folgen, aber Harrys Gesicht wandte sich ihr für einen Moment zu. Erst als Harry um eine Ecke gebogen war, zog er seinen Zauberstab, sagte mit leiser, präziser Stimme "Quietus" und wandte sich dann an Padma und sagte: "Eine interessante Vermutung, Miss Patil."

Padma sah daraufhin ziemlich selbstgefällig aus; und sagte: "Ich hätte es eigentlich schon früher herausfinden müssen. Da war dieses Zischen in der Stimme des Geistes, ich hätte sofort an Parsel denken müssen, noch bevor er anfing, über Godric Gryffindor zu reden."

Harrys Gesicht veränderte sich nicht. "Darf ich fragen, Miss Patil, ob Sie diesen Gedanken mit -"

"Sie hat es vor allen in der S.P.H.E.W. gesagt", sagte Hermine.

Harrys Augen hatten diesen Blick, den sie hatten, wenn er sehr schnell etwas berechnete, und dann sagte er: "Hermine, wie groß ist die Chance, dass -"

"Sie hat es vor Lavender und Tracey gesagt."

"Ähm", sagte Padma. "Hätte ich das nicht tun sollen?"

…\\ "Wartet hier", knurrte Mr. Goyle und ging um die Ecke; da ertönte das Klopfen an Draco Malfoys Privatraum.

Tracey hatte ein mulmiges Gefühl im Magen, und sie erinnerte sich wieder daran, dass, seit Padma alles ausgeplaudert hatte, jemand es Draco Malfoy sagen musste, und das konnte genauso gut sie sein, und es war ja nicht so, dass sie Harry Potter etwas schuldete, und eine Slytherin musste tun, was nötig war, um ihre Ambitionen zu erfüllen. Seit Professor Quirrell ihr den Laufpass gegeben hatte, sammelte sie Ambitionen, und bis jetzt hatte sie beschlossen, dass sie ihren eigenen Nimbus 2000-Besen besitzen, superberühmt werden, Harry Potter heiraten, jeden Tag Schokofrösche zum Frühstück essen und mindestens drei Dunkle Lords besiegen wollte, nur um Professor Quirrell zu zeigen, wer hier Gewöhnliche ist.

"Mr. Malfoy wird dich sehen", sagte die tiefe, bedrohliche Stimme von Mr. Goyle, als er zurückkam. "Und du solltest hoffen, dass er nicht denkt, dass du seine Zeit verschwendest."

Der Junge warf ihr einen kurzen Blick zu und trat dann zur Seite. Tracey fügte ihrer Liste der Ambitionen hinzu, ihre eigenen Diener zu haben, und trat ein.

Das private Schlafzimmer der Malfoys sah genauso aus wie das von Daphne. Insgeheim hatte sie auf diamantene Kronleuchter oder goldene Fresken an den Wänden gehofft - sie hätte es nie vor Daphne gesagt, aber das Haus Malfoy war eine Stufe besser als Greengrass. Aber es war nur ein kleines Schlafzimmer wie das von Daphne, und der einzige Unterschied war, dass die Sachen von Malfoy mit silbernen Schlangen statt mit Smaragdpflanzen verziert waren. Als sie durch die Tür trat, erhob sich Draco Malfoy - der selbst in seinem eigenen Schlafzimmer perfekt gepflegt war - von seinem Schreibtischstuhl, um sie mit einer kleinen freundlichen Verbeugung zu begrüßen, wobei er ein charmantes Lächeln aufsetzte, als wäre sie jemand, der etwas bedeutete, was Tracey so aufgeregt machte, dass sie alles vergaß, was sie in ihrem Kopf einstudiert hatte, und einfach herausplatzte: "Ich muss dir etwas sagen!"

"Ja, Gregory hat es gesagt", sagte Draco Malfoy sanft. "Bitte, Miss Davis, setzen\\ Sie sich." Er gestikulierte zu seinem eigenen Schreibtischstuhl, während er sich selbst auf sein Bett setzte.

Sie fühlte sich etwas benommen, als sie sich vorsichtig auf Malfoys eigenen Stuhl setzte, ihre Finger fummelten gedankenlos daran herum, wie ihre Roben über ihre Knie fielen, und versuchten, sie so elegant und ungekräuselt aussehen zu lassen wie die von Draco Malfoy -

"Also, Miss Davis", sagte Draco Malfoy. "Was wollten Sie mir erzählen?"

Tracey zögerte, und als Malfoys Gesicht anfing, etwas ungeduldig auszusehen, stammelte sie einfach alles heraus, alles, was Padma darüber gesagt hatte, dass Salazar Slytherins Geist Harry Potter geschickt hatte, um Tyrannen zu stoppen, und auch, was Daphne ihr darüber erzählt hatte, dass Hermine Granger darin verwickelt war - Draco Malfoys Gesichtsausdruck änderte sich überhaupt nicht, als sie sprach, nicht einmal im Geringsten, und es dämmerte Tracey mit einem üblen Kribbeln im Magen. "Du glaubst mir nicht!", sagte sie.

Es gab eine kleine Pause.

"Nun", sagte Draco Malfoy mit einem Lächeln, das nicht ganz so charmant war wie sein letztes, "ich glaube, dass es das ist, was Padma gesagt hat und was Daphne gesagt hat, also danke trotzdem, Miss Davis." Der Junge erhob sich von dort, wo er auf dem Bett gesessen hatte, und Tracey erhob sich, ohne zu überlegen, vom Stuhl.

Als er sie zur Tür begleitete und gerade den Knauf drehen wollte, fiel Tracey ein - "Sie haben nicht gefragt, was ich für die Information wollte", sagte sie.

Draco Malfoy warf ihr eine Art Blick zu, sie wusste nicht genau, was er bedeuten sollte, und er sagte nichts.

"Na ja, jedenfalls", sagte Tracey und änderte kurzerhand ihre bisherigen Pläne, "ich will nichts für die Informationen, ich wollte nur freundlich sein."

Ein kurzer Blick der Überraschung ging für einen Augenblick über Draco Malfoys Gesicht, bevor sich seine Miene wieder verflachte und er sagte: "Es ist nicht so einfach, mit einem Malfoy befreundet zu sein, Miss Davis."

Tracey lächelte und meinte es auch so. "Nun, dann werde ich eben weiterhin freundlich sein", sagte sie und verließ den Raum mit einem Hüpfer im Schritt, wobei sie sich vielleicht zum ersten Mal in ihrem Leben wie eine echte Slytherin fühlte und gerade beschlossen hatte, dass Draco Malfoy auch einer ihrer Ehemänner sein würde.

Nachdem das Mädchen gegangen war, kam Gregory herein, schloss die Tür wieder und sagte: "Geht es dir gut?"

Draco sagte nichts zu seinem Diener und Freund. Seine Augen starrten ins Nichts, als wollte er durch die Wand seines Schlafzimmers starren, durch den Hogwarts-See, der die Slytherin-Kerker umgab, durch die Erdkruste und die Atmosphäre und den interstellaren Staub der Milchstraße, in die völlig leere und lichtlose Leere zwischen den Galaxien, die kein Zauberer und kein Wissenschaftler je gesehen hatte.

"Mr. Malfoy?" sagte Gregory und klang langsam ein wenig besorgt.

"Ich kann nicht glauben, dass ich jedes Wort davon geglaubt habe", sagte Draco.

…\\ Daphne beendete ihr letztes Stückchen Verwandlung und schaute quer durch den Slytherin-Gemeinschaftsraum, wo Millicent Bulstrode immer noch an ihren eigenen Hausaufgaben arbeitete. Es war an der Zeit, eine Entscheidung zu treffen. Wenn S.P.H.E.W. herumgehen und versuchen würde, Tyrannen zu betäuben, würde das den Tyrannen nicht gefallen, das war sicher. Und sie würden versuchen, etwas Unangenehmes dagegen zu tun, das war auch sicher. Andererseits, wenn die Tyrannen wirklich böse wurden, dann könnte Hermine Harry Potter um Hilfe bitten, oder sie könnten ihre kombinierten Quirrell-Punkte zusammenlegen und den Verteidigungsprofessor um einen Gefallen bitten… Nein, die Sache, über die sich Daphne wirklich Sorgen machte, war, ob diese Sache ihnen Ärger mit Professor Snape einbringen würde. Man wollte nie auf der falschen Seite von Professor Snape landen. Aber seit dem Tag, an dem sie Neville zu einem Duell herausgefordert hatte, hatte sie bemerkt, dass die Leute sie anders ansahen. Sogar die Slytherins, die sich über sie lustig gemacht hatten, sahen sie jetzt anders an. Es dämmerte Daphne, dass es viel mehr Respekt einbrachte, die Tochter des noblen und uralten Hauses Greengrass zu sein, wenn man eine schöne Heldin war, die aus einem uralten Haus stammte, und nicht nur ein hübsches adliges Mädchen. Es war der Unterschied zwischen einer Rolle, die von der Hauptdarstellerin gespielt wurde, und einer Rolle, die von einem Statisten mit zwei Galeonen und einer kreischenden Lache gespielt wurde. Tyrannen zu bekämpfen ist vielleicht nicht der beste Weg, eine Heldin zu werden. Aber Vater hatte ihr einmal gesagt, dass das Problem mit dem Verpassen von Gelegenheiten darin besteht, dass es zur Gewohnheit wird. Wenn man sich selbst sagte, dass man auf eine bessere Gelegenheit wartete, würde man sich beim nächsten Mal wahrscheinlich das Gleiche sagen. Vater hatte gesagt, dass die meisten Menschen ihr ganzes Leben damit verbrachten, auf eine Gelegenheit zu warten, die gut genug war, und dann starben sie. Vater hatte gesagt, dass das Ergreifen von Gelegenheiten zwar bedeuten würde, dass alle möglichen Dinge schief gehen würden, aber es war nicht annähernd so schlimm, wie ein hoffnungsloser Fall zu sein. Vater hatte gesagt, wenn sie sich angewöhnt habe, Gelegenheiten zu ergreifen, dann sei es an der Zeit, wählerisch zu werden. Andererseits hatte Mutter sie gewarnt, nicht alle Ratschläge Vaters zu befolgen, und gesagt, dass Daphne nicht nach Vaters sechstem Jahr in Hogwarts fragen dürfe, bis sie mindestens dreißig Jahre alt sei. Aber am Ende hatte Vater Mutter dazu gebracht, ihn zu heiraten, und sich erfolgreich einen Platz in einem der ältesten Häuser erschlichen, das war es also.

Millicent Bulstrode beendete ihre Hausaufgaben und begann, ihre Sachen wegzuräumen. Daphne stand von ihrem Schreibtisch auf und ging hinüber. Millicent schwang ihre Beine vom Tisch und stand auf, wobei sie ihre Büchertasche über eine Schulter warf, dann schaute sie hinüber, wo Daphne sich näherte, wobei der Ausdruck des Mädchens verwirrt war.

"Hey, Millicent", sagte Daphne, als sie näher kam, und ihre Stimme klang tief und aufgeregt, "rate mal, was ich heute herausgefunden habe?"

"Die Sache mit dem Geist von Salazar Slytherin, der Granger hilft?", fragte Millicent. "Davon habe ich schon gehört -"

"Nein", sagte Daphne im gedämpften Flüsterton, "das ist sogar noch besser."

"Wirklich?" sagte Millicent mit ebenso leiser, aufgeregter Stimme. "Was ist es?"

Daphne blickte sich verschwörerisch um. "Komm mit in mein Zimmer, dann erzähle ich es dir." Sie gingen in Richtung der Treppe, die nach unten führte; die Privaträume lagen noch tiefer im See als die Schlafsäle der Siebtklässler… Schon bald saß Daphne in ihrem bequemen Schreibtischstuhl und Millicent war an die Bettkante gehüpft. "Quietus", sagte Daphne, als sie beide saßen; und dann, anstatt ihren Zauberstab in ihrem Umhang zu verstauen, ließ Daphne ihre Hand ganz natürlich an ihre Seite fallen, den Zauberstab immer noch in der Hand haltend, nur für den Fall.

"Alles klar!", sagte Millicent. "Was ist los?"

"Weißt du, was ich herausgefunden habe?", sagte Daphne. "Ich habe herausgefunden, dass du den Klatsch und Tratsch so schnell mitbekommst, dass du von Dingen weißt, bevor sie tatsächlich passieren." Daphne hatte halb damit gerechnet, dass Millicent weiß werden und umfallen würde, und das tat sie auch nicht wirklich, aber das Mädchen zuckte ziemlich heftig zusammen, bevor sie anfing, Dementis zu stammeln. "Keine Sorge", sagte Daphne mit ihrem süßesten Lächeln, "ich werde niemandem sonst erzählen, dass du eine Seherin bist. Ich meine, wir sind doch Freunde, oder?"

…\\ Rianne Felthorne, siebte Klasse in Slytherin, arbeitete fleißig an einem weiteren Aufsatz (sie belegte alle Fächer außer Wahrsagen und Muggelkunde und ihr U.T.Z.-Jahr schien nur aus Hausaufgaben zu bestehen), als ihr Hausoberhaupt auf den Tisch zuging, an dem sie gerade arbeitete, und bellte: "Sie kommen mit mir, Miss Felthorne!" und ging davon, während sie noch verzweifelt begann, ihr Pergament, ihr Buch und ihren Federkiel wegzulegen.

Als sie Professor Snape einholte, wartete er vor dem Raum und starrte sie mit halb geschlossenen Augen an, die viel zu intensiv zu sein schienen; und bevor sie fragen konnte, worum es ging, drehte er sich wortlos um und schlenderte durch die Gänge, so dass sie sich anstrengen musste, um Schritt zu halten. Ihr Weg führte sie eine Treppe hinunter und dann eine weitere, unterhalb dessen, was sie für die unterste Ebene der Slytherin-Verliese gehalten hatte.Und die Korridore begannen, älter auszusehen, die Architektur war um Jahrhunderte zurückgeworfen und bestand nur noch aus aufgerautem Stein, der von grobem Mörtel zusammengehalten wurde. Sie begann sich zu fragen, ob Professor Snape sie in die echten Kerker führte, von denen sie Gerüchte gehört hatte, die wahren Kerker von Hogwarts, die für alle außer dem Lehrkörper verschlossen waren; und ob Professor Snape dort unten \emph{vielleicht schreckliche Dinge mit unschuldigen, hilflosen jungen Mädchen tat}, aber das war wahrscheinlich nur Wunschdenken ihrerseits.

Sie gingen eine weitere Treppe hinunter und kamen in einen Raum, der gar kein Raum war, sondern eine leere Felsenhöhle mit einer einzigen Tür, durchbohrt von vielen dunklen Öffnungen und erleuchtet von einer einzigen Fackel im antiken Stil, die brannte, als sie eintraten. Professor Snape zückte seinen Zauberstab und begann, einen Zauber nach dem anderen zu wirken, sie wusste nicht mehr, wie viele es waren. Als der Meister der Zaubertränke fertig war, drehte er sich wieder zu ihr um, sah sie mit seinen intensiven Augen an und sagte mit ruhiger Stimme, die so gar nicht zu seinem üblichen Tonfall passte: "Sie werden niemandem etwas von dieser Angelegenheit erzählen, Miss Felthorne, weder jetzt noch in Zukunft. Wenn das für Sie akzeptabel ist, nicken Sie. Wenn nicht, werden wir uns umdrehen und gehen."

Sie nickte, verängstigt und mit einer seltsamen Hoffnung, die in ihrem Herzen aufkeimte \emph{(na ja, nicht ganz ihr Herz)}.

"Die Aufgabe, die ich für Sie habe, ist sehr einfach, Miss Felthorne", sagte Professor Snapes tonlose Stimme, "und Ihr äußerst großzügiger Lohn von fünfzig Galleonen ist lediglich eine Entschädigung dafür, dass Sie danach einen Gedächtniszauber haben."

Sie holte unwillkürlich Luft. Ihre Familie mochte reich sein, aber sie hatten noch andere Töchter und hielten sie an der kurzen Leine, und für sie war das sicherlich eine Menge Geld. Dann fielen ihr die Worte Gedächtniszauber ein, und einen Moment lang fühlte sie sich empört, es hatte keinen Sinn, wenn sie die Erinnerungen nicht behalten konnte, für was für ein Mädchen hielt Professor Snape sie?

"Sie wissen sicher", sagte Severus Snape, "von Miss Hermine Granger, der Sonnenschein-Generalin?"

"Was?", sagte Rianne Felthorne in plötzlichem Entsetzen und Abscheu. "Sie ist erst zwölf! Bäh!"

