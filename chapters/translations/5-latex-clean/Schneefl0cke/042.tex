

\hypertarget{humanismus-teil-1}{% \section{43. Humanismus Teil 1}\label{humanismus-teil-1}}

\textbf{\uline{Humanismus, Teil 1}}

Die sanfte Sonne des Januars schien auf die kalten Felder außerhalb von Hogwarts.\\ Für einige der Schüler war es eine Lernstunde, andere waren aus dem Unterricht entlassen worden. Die Erstklässler, die sich dafür angemeldet hatten, übten einen bestimmten Zauber, einen Zauber, den man am besten im Freien, unter der hellen Sonne und dem klaren blauen Himmel, lernte und nicht in der Enge eines Klassenzimmers. Kekse und Limonade wurden ebenfalls als hilfreich angesehen. Die ersten Gesten des Zaubers waren komplex und präzise; man zupfte den Zauberstab einmal, zweimal, dreimal und viermal mit kleinen Neigungen in genau den richtigen relativen Winkeln, man verschob Zeigefinger und Daumen genau in den richtigen Abständen.

.. Das Ministerium hielt den Versuch, jemandem den Zauberspruch vor dem fünften Jahr beizubringen, für sinnlos. Es gab ein paar bekannte Fälle, in denen jüngere Kinder ihn lernten, und das wurde als "\emph{Genie}" abgetan. Es war vielleicht nicht sehr höflich ausgedrückt, aber Harry begann zu verstehen, warum Professor Quirrell behauptet hatte, dass der Lehrplanausschuss des Ministeriums einen größeren Nutzen für die Zaubererwelt gehabt hätte, wenn man sie als Mülldeponie benutzt hätte.

\emph{Die Gesten waren also kompliziert und heikel. Das hielt einen nicht davon ab, sie mit elf Jahren zu lernen. Es bedeutete nur, dass man besonders vorsichtig sein und jeden Teil viel länger als gewöhnlich üben musste.}

Die meisten Zauber, die nur von älteren Schülern erlernt werden konnten, waren so, weil sie mehr Kraft der Magie erforderten, als ein junger Schüler aufbringen konnte. Aber der Patronus-Zauber war nicht so, er war nicht schwierig, weil er zu viel Magie erforderte, er war schwierig, weil er mehr als nur Magie erforderte. Es brauchte die warmen, glücklichen Gefühle, die man im Herzen trug, die liebevollen Erinnerungen, eine andere Art von Kraft, die man für gewöhnliche Zaubersprüche nicht brauchte.

Harry zuckte einmal, zweimal, dreimal und viermal mit seinem Zauberstab, bewegte seine Finger genau in den richtigen Abständen…\\ "\emph{Viel Glück in der Schule, Harry. Glaubst du, ich hab dir genug Bücher gekauft?}"\\ "\emph{Man kann nie genug Bücher haben … aber du hast es auf jeden Fall versucht, es war ein wirklich, wirklich, wirklich guter Versuch …"}\\ Es hatte ihm die Tränen in die Augen getrieben, als Harry sich zum ersten Mal daran erinnerte und versuchte, den Zauberspruch zu sprechen. Harry hob den Zauberstab und schwang ihn, eine Geste, die nicht präzise sein musste, nur kühn und trotzig.

"Expecto Patronum!", rief Harry.

Doch nichts geschah. Nicht ein einziges Flackern von Licht. Als Harry aufblickte, studierte Remus Lupin immer noch den Zauberstab, mit einem ziemlich beunruhigten Blick auf seinem leicht vernarbten Gesicht. Schließlich schüttelte Remus den Kopf.\\ "Es tut mir leid, Harry", sagte der Mann leise. "Deine Zauberstabarbeit war genau richtig."

Und auch sonst flackerte kein Licht auf, denn alle anderen Erstklässler, die eigentlich ihre Patronus-Zauber üben sollten, blickten stattdessen aus den Augenwinkeln zu Harry. Die Tränen drohten wieder in Harrys Augen zu kommen, und es waren keine Freudentränen. Von allen Dingen, von allen Dingen, hatte Harry das nie erwartet. Es hatte etwas furchtbar Demütigendes, wenn einem mitgeteilt wurde, dass man nicht glücklich genug war. Was hatte Anthony Goldstein in sich, das Harry nicht hatte, das Anthonys Zauberstab in diesem hellen Licht erstrahlen ließ? Hat Anthony seinen eigenen Vater mehr geliebt?

"Mit welchem Gedanken hast du ihn gezaubert?", fragte Remus.

"Mein Vater", sagte Harry, wobei seine Stimme zitterte.\\ "Ich habe ihn gebeten, mir ein paar Bücher zu kaufen, bevor ich nach Hogwarts kam, und das hat er getan, und sie waren teuer, und dann hat er mich gefragt, ob sie genug sind -"\\ Harry versuchte nicht, etwas über das Familienmotto der Verres zu erklären.

"Ruhen dich erst einmal aus, bevor Sie einen anderen Gedanken fassen, Harry", sagte Remus. Er gestikulierte in Richtung der Stelle, wo einige andere Schüler auf dem Boden saßen und enttäuscht oder verlegen oder reumütig aussahen.

"Du wirst nicht in der Lage sein, einen Patronus-Zauber zu wirken, während du dich schämst, nicht dankbar genug zu sein." Es lag ein sanftes Mitgefühl in Mr. Lupins Stimme, und einen Moment lang hatte Harry das Gefühl, auf etwas einschlagen zu müssen. Stattdessen drehte Harry sich um und schlenderte dorthin, wo die anderen Versager saßen. Die anderen Schüler, deren Zauberstabarbeit ebenfalls für perfekt erklärt worden war, und die nun auf der Suche nach glücklicheren Gedanken sein sollten; so wie es aussah, machten sie keine großen Fortschritte.\\ Es gab dort viele dunkelblaue Roben und eine Handvoll roter, und ein einsames Hufflepuff-Mädchen, das immer noch weinte. Die Slytherins hatten sich nicht einmal die Mühe gemacht, aufzutauchen, mit Ausnahme von Daphne Greengrass und Tracey Davis, die immer noch versuchten, die Gesten richtig hin zu bekommen.

Harry ließ sich auf das kalte, tote Gras des Winters plumpsen, neben dem Schüler, dessen Versagen ihn am meisten überrascht hatte.

"Du hast es also auch nicht geschafft", sagte Hermine. Sie war zuerst vom Feld geflohen, aber dann war sie zurückgekommen, und man musste schon genau hinsehen, um in ihren geröteten Augen zu erkennen, dass sie geweint hatte.

"Ich", sagte Harry, "ich, ich würde mich wahrscheinlich viel schlechter fühlen, wenn du nicht versagt hättest, du bist die netteste, Person, die ich kenne, die ich je getroffen habe, Hermine, und wenn du es auch nicht schaffst, bedeutet das, dass ich vielleicht noch, gut bin…"

"Ich hätte nach Gryffindor gehen sollen", flüsterte Hermine. Sie blinzelte ein paar Mal heftig, aber sie wischte sich nicht die Augen.

Der Junge und das Mädchen gingen gemeinsam vorwärts, definitiv nicht händchenhaltend, aber jeder zog eine Art Kraft aus der Anwesenheit des anderen, etwas, das sie das Geflüster ihrer Mitschüler ignorieren ließ, während sie durch den Flur gingen und sich den großen Türen von Hogwarts näherten. Harry war nicht in der Lage gewesen, den Patronus-Zauber zu wirken, ganz gleich, was er an Glücksgefühlen versuchte. Die Leute schienen davon nicht überrascht zu sein, was es noch schlimmer machte.

Hermine war auch nicht in der Lage gewesen, ihn zu wirken. Die Leute waren darüber sehr überrascht gewesen und Harry hatte gesehen, wie sie die gleichen Seitenblicke wie er zu bekommen begann. Die anderen Ravenclaws, die versagt hatten, bekamen diese Blicke nicht. Aber Hermine war die Sonnenschein-Generalin, und ihre Fans behandelten sie, als hätte sie irgendwie versagt, als hätte sie ein Versprechen gebrochen, das sie nie gegeben hatte.

Die beiden waren in die Bibliothek gegangen, um über den Patronus-Zauber zu recherchieren, der Hermines Art war, mit Kummer umzugehen, wie es manchmal auch Harrys Art war. \emph{Studieren, lernen, versuchen zu verstehen, warum..}. Die Bücher hatten bestätigt, was der Schulleiter Harry gesagt hatte: Oft waren Zauberer, die den Patronus-Zauber in der Praxis nicht wirken konnten, in der Lage, ihn in Gegenwart eines echten Dementors zu wirken, und zwar von völligem Versagen bis hin zu einem vollständigen körperlichen Patronus. Es widersprach jeder Logik, die Aura der Angst des Dementors sollte es schwieriger machen, einen glücklichen Gedanken zu hegen; aber so war es nun einmal. Also wollten die beiden es ein letztes Mal versuchen, es gab keine Möglichkeit, dass einer von ihnen es nicht ein letztes Mal versuchen würde.

Es war der Tag, an dem der Dementor nach Hogwarts kam. Zuvor hatte Harry den Stein seines Vaters von der Stelle, an der er normalerweise in Form eines winzigen Diamanten auf seinem kleinen Ring ruhte, gelöst und den großen grauen Stein zurück in seinen Beutel gelegt. Nur für den Fall, dass Harrys Magie völlig versagte, wenn er der dunkelsten aller Kreaturen gegenüberstand.\\ Harry war bereits pessimistisch geworden, und er stand noch nicht einmal vor einem Dementor.

"Ich wette, du schaffst es und ich nicht", sagte Harry im Flüsterton. "Ich wette, das ist es, was passiert."

"Es fühlte sich für mich falsch an", sagte Hermine, ihre Stimme noch leiser als seine.\\ "Ich habe es heute Morgen ausprobiert und es wurde mir klar. Als ich den Wisch am Ende machte, sogar bevor ich die Worte sagte, fühlte es sich falsch an."

Harry sagte nichts. Er hatte das Gleiche gefühlt, von Anfang an, obwohl es weitere fünf Versuche mit fünf anderen glücklichen Gedanken gebraucht hatte, bevor er es sich eingestehen konnte.\\ Jedes Mal, wenn er versucht hatte, seinen Zauberstab zu schwingen, hatte es sich hohl angefühlt; der Zauberspruch, den er zu lernen versuchte, passte nicht zu ihm.

"Das heißt nicht, dass wir zu dunklen Zauberern werden", sagte Harry. "Viele Leute, die den Patronus-Zauber nicht wirken können, sind keine Dunklen Zauberer. Godric Gryffindor war kein Dunkler Zauberer…"\\ Godric hatte Dunkle Lords besiegt, gekämpft, um Bürgerliche vor Adelshäusern und Muggel vor Zauberern zu schützen. Er hatte viele gute und treue Freunde gehabt und nicht mehr als die Hälfte von ihnen für die eine oder andere gute Sache verloren. Er hatte die Schreie der Verwundeten in den Armeen gehört, die er zur Verteidigung der Unschuldigen aufgestellt hatte; junge, mutige Zauberer waren seinen Rufen gefolgt, und er hatte sie danach begraben. Bis er schließlich, als seine Zauberkünste ihn im Alter gerade zu verlassen begannen, die drei anderen mächtigsten Zauberer seiner Zeit zusammenbrachte, um Hogwarts aus dem nackten Boden zu stampfen; die einzige große Errungenschaft, die Godrics Namen trägt, hatte nichts mit Krieg zu tun, mit jeder Art von Krieg, egal wie gerecht. Es war Salazar und nicht Godric, der die erste Hogwarts-Klasse in Kampfmagie unterrichtet hatte. Godric hatte die erste Hogwarts-Stunde in Kräuterkunde unterrichtet, die Magie des grünen, wachsenden Lebens. Bis zu seinem letzten Tag war er nicht in der Lage gewesen, den Patronus-Zauber zu wirken.

\emph{Godric Gryffindor war ein guter Mann gewesen, kein glücklicher.}

Harry glaubte nicht an Angst, er konnte es nicht ertragen, von weinerlichen Helden zu lesen, er wusste, dass eine Milliarde anderer Menschen auf der Welt alles dafür gegeben hätten, mit ihm zu tauschen, und… Und auf dem Sterbebett hatte Godric zu Helga gesagt (denn Salazar hatte ihn im Stich gelassen, und Rowena war schon vorher gestorben), dass er nichts davon bereute, und er warnte seine Schüler nicht davor, in seine Fußstapfen zu treten, niemand sollte je sagen, er hätte jemandem geraten, nicht in seine Fußstapfen zu treten. Wenn es für ihn das Richtige gewesen war, dann würde er keinem anderen sagen, dass er sich falsch entschieden hatte, nicht einmal dem jüngsten Schüler in Hogwarts. Doch für diejenigen, die in seine Fußstapfen traten, hoffte er, dass sie sich daran erinnern würden, dass Gryffindor seinem Haus gesagt hatte, dass es für sie in Ordnung sei, glücklicher zu sein als er. Dass Rot und Gold leuchtende, warme Farben sein würden, von nun an. Und Helga hatte ihm weinend versprochen, dass sie, wenn sie Schulleiterin war, dafür sorgen würde. Daraufhin war Godric gestorben und hatte keinen Geist hinterlassen; und Harry hatte das Buch zurück zu Hermine geschoben und war ein Stück weggegangen, damit sie ihn nicht weinen sah.

\emph{Man würde nicht denken, dass ein Buch mit einem unschuldigen Titel wie "Der Patronus-Zauber: Zauberer, die es konnten und nicht konnten" das traurigste Buch sein würde, das Harry je gelesen hatte.}

Harry… Harry wollte das nicht. In diesem Buch zu sein. Harry wollte das nicht. Der Rest der Schule schien zu denken, dass "\emph{Kein Patronus}" schlicht und einfach "\emph{Schlechter Mensch}"\\ bedeutete. Irgendwie schien die Tatsache, dass Godric Gryffindor ebenfalls nicht in der Lage war, den Patronus-Zauber zu wirken, nicht wiederholt zu werden. Vielleicht sprach man nicht darüber, um seinen letzten Wunsch zu respektieren, Fred und George wussten es wahrscheinlich nicht und Harry hatte sicher nicht vor, es ihnen zu sagen. Oder vielleicht erwähnten die anderen Versager es nicht, weil es weniger beschämend war, der kleinere Verlust von Stolz und Status, eher dunkel als unglücklich gedacht zu werden. Harry sah, dass Hermine neben ihm heftig blinzelte; und er fragte sich, ob sie an Rowena Ravenclaw dachte, die auch Bücher geliebt hatte.

"Okay", flüsterte Harry. "Fröhlichere Gedanken. Wenn du einen vollkörperlichen Patronus bekommst, was denkst du, was dein Tier sein wird?"

"Ein Otter", sagte Hermine sofort.

"Ein Otter?" flüsterte Harry ungläubig.

"Ja, ein Otter", sagte Hermine. "Und was ist mit deinem?"

"Ein Wanderfalke", sagte Harry ohne zu zögern. "Er kann schneller als dreihundert Kilometer pro Stunde fliegen, er ist das schnellste Lebewesen, das es gibt."\\ \emph{Der Wanderfalke war schon immer Harrys Lieblingstier gewesen.}\\ Harry war fest entschlossen, eines Tages ein Animagus zu werden, nur um das als seine Form zu bekommen, und durch die Kraft seiner eigenen Flügel zu fliegen, und das Land unter ihm mit schärferen Augen zu sehen…\\ "Aber warum ein Otter?"

Hermine lächelte, sagte aber nichts. Und die riesigen Türen von Hogwarts schwangen auf.\\ Sie liefen eine Weile, die Kinder, über einen Weg, der in Richtung des Unerlaubten Waldes führte, und weiter durch den Wald selbst. Die Sonne senkte sich bis nahe an den Horizont, die Schatten waren lang, das Sonnenlicht wurde durch die kahlen Äste der Winterbäume gefiltert; denn es war Januar, und die Erstklässler lernten an diesem Tag als Letzte. Dann schlug der Weg eine neue Richtung ein, und sie alle sahen sie in der Ferne, die Lichtung im Wald und den kargen Winterboden, vergilbtes, vertrocknetes Gras, geweißt von ein paar kleinen Schneeresten.\\ Die menschlichen Gestalten noch klein in dieser Entfernung. Die zwei Flecken aus schwachem weißen Licht von den Patronen der Auroren und der hellere Fleck aus silbernem Licht von dem des Schulleiters, daneben etwas…\\ Harry blinzelte. Etwas… Es muss nur Harrys Einbildung gewesen sein, denn es hätte für einen Dementor keine Möglichkeit geben dürfen, an drei körperlichen Patronus vorbeizukommen, aber er glaubte zu spüren, wie ein Hauch von Leere seinen Geist streifte, direkt in sein weiches inneres Zentrum, ohne Rücksicht auf die Okklumentikbarrieren.

Seamus Finnigan war aschfahl und zitterte, als er sich wieder unter die Schüler mischte, die sich auf dem verdorrten und schneebedeckten Gras tummelten. Seamus' Patronus-Zauber war erfolgreich gewesen, aber es gab immer noch dieses Intervall zwischen dem Zeitpunkt, an dem der Schulleiter seinen eigenen Patronus auflöste, und dem Zeitpunkt, an dem man seinen eigenen zaubern sollte, an dem man der Angst des Dementors schutzlos ausgeliefert war.

Bis zu zwanzig Sekunden Exposition bei fünf Schritten war sicherlich sicher, selbst für einen elfjährigen Zauberer mit schwacher Abwehr und einem noch reifenden Gehirn. Es gab große Unterschiede darin, wie stark die Kraft des Dementors Menschen traf, was eine weitere Sache war, die nicht ganz verstanden wurde; aber zwanzig Sekunden waren definitiv sicher.

Vierzig Sekunden Dementoreinwirkung bei fünf Schritten hätten möglicherweise ausgereicht, um bleibende Schäden zu verursachen, allerdings nur bei den empfindlichsten Personen. Es war ein hartes Training, selbst nach den Maßstäben von Hogwarts, wo man das Fliegen auf einem Hippogreif lernte, indem man auf einen geworfen wurde und aufgefordert wurde, loszulegen.\\ Harry war kein Fan von Überfürsorglichkeit, und wenn man sich den Reifeunterschied zwischen einem Viertklässler in Hogwarts und einem vierzehnjährigen Muggel ansah, war es klar, dass die Muggel ihre Kinder verhätschelten… aber selbst Harry hatte angefangen, sich zu fragen, ob das nicht übertrieben war. Nicht jede Verletzung konnte im Nachhinein geheilt werden.

Aber wenn man den Zauber unter diesen Bedingungen nicht wirken konnte, bedeutete das, dass man sich nicht darauf verlassen konnte, sich mit dem Patronus-Zauber zu verteidigen; Überheblichkeit war für Zauberer noch gefährlicher als für Muggel. Dementoren konnten dir deine Magie und deine körperliche Vitalität entziehen, nicht nur deine fröhlichen Gedanken, was bedeutete, dass du vielleicht nicht in der Lage warst, wegzuapparieren, wenn du zu lange gewartet hast, oder wenn du die herannahende Angst nicht erkannt hast, bis der Dementor in Reichweite für seinen Angriff war.

(Während seiner Lektüre hatte Harry mit beträchtlichem Entsetzen festgestellt, dass in einigen Büchern behauptet wurde, der Kuss des Dementors würde die Seele auffressen und dies sei der Grund für das permanente geistlose Koma, in das er die Opfer versetzte. Und dass Zauberer, die dies glaubten, den Kuss des Dementors absichtlich eingesetzt hatten, um Verbrecher hinzurichten.\\ Es war eine Gewissheit, dass einige der so genannten Verbrecher unschuldig waren, und selbst wenn sie es nicht waren, \emph{ihre Seelen zu zerstören?} Wenn Harry an Seelen geglaubt hätte, hätte er … eine Leerstelle gezogen, ihm fiel einfach keine angemessene Antwort darauf ein).

Der Schulleiter nahm die Sicherheit ernst, ebenso wie die drei Auroren, die Wache standen.\\ Ihr Anführer war ein asiatisch aussehender Mann, feierlich, ohne grimmig zu sein, Auror Komodo, dessen Zauberstab nie seine Hand verließ. Sein Patronus, ein Orang-Utan aus massivem Mondlicht, schritt zwischen dem Dementor und den Erstklässlern, die darauf warteten, an die Reihe zu kommen, hin und her; neben dem Orang-Utan bewegte sich der strahlend weiße Panther von Auror Butnaru, einem Mann mit stechendem Blick, langem schwarzen Haar in einem Pferdeschwanz und einem langen geflochtenen Ziegenbart. Diese beiden Auroren und ihre beiden Patronusse beobachteten alle den Dementor. Auf der gegenüberliegenden Seite der Schüler befand sich der ruhende Auror Goryanof, groß und dünn und blass und unrasiert, der sich auf einem Stuhl zurücklehnte, den er wortlos und ohne Zauberstab herbeigezaubert hatte, und mit einem abwesenden Pokerface die ganze Szene abtastete.

Professor Quirrell war aufgetaucht, kurz nachdem die Erstklässler mit ihren Versuchen begonnen hatten, und seine Augen waren nie weit von Harry gewandert. Der winzige Professor Flitwick, der ein Meister im Duellieren gewesen war, fummelte abwesend an seinem Zauberstab herum; und seine Augen, die aus dem riesigen, geschwollenen Bart, der ihm als Gesicht diente, hervorlugten, blieben auf Professor Quirrell gerichtet. Und es muss Harrys Einbildung gewesen sein, aber Professor Quirrell schien jedes Mal leicht zusammenzuzucken, wenn der Patronus des Schulleiters verschwand, um den nächsten Schüler zu testen. Vielleicht bildete sich Professor Quirrell denselben Placebo-Effekt ein wie Harry, diesen Schwall von Leere, der an seinem Verstand nagte.

"Anthony Goldstein", rief die Stimme des Schulleiters.\\ Harry ging leise auf Seamus zu, auch als Anthony begann, sich dem silbern glänzenden Phönix zu nähern, und … was auch immer es unter dem zerfledderten Umhang war.

"Was hast du gesehen?" fragte Harry Seamus mit leiser Stimme.\\ Viele Schüler hatten Harry nicht geantwortet, als er versucht hatte, die Daten zu sammeln; aber Seamus war Finnigan von Chaos, einer von Harrys Leutnants. Vielleicht war das nicht fair, aber…\\ "Tot", sagte Seamus flüsternd, "gräulich und schleimig… tot und eine Weile im Wasser gelegen… "

Harry nickte.\\ "Das ist es, was viele Leute sehen", sagte Harry.\\ Er strahlte Zuversicht aus, auch wenn sie unecht war, denn Seamus brauchte sie.\\ "Geh und iss etwas Schokolade, dann geht es dir besser."\\ Seamus nickte und stolperte in Richtung des Tisches mit den heilenden Süßigkeiten.

"Expecto Patronum!", rief die Stimme eines kleinen Jungen. Dann gab es schockierte Blicke, sogar von den Auroren. Harry drehte sich um, um nachzusehen - da stand ein silberglänzender Vogel zwischen Anthony Goldstein und dem Käfig. Der Vogel bäumte sich auf und stieß einen Schrei aus, und der Schrei war ebenfalls silbern, so hell und hart und schön wie Metall.\\ Und etwas in Harrys Hinterkopf sagte:

\emph{Wenn das ein Wanderfalke ist, werde ich ihn im Schlaf erwürgen.}

\emph{Halt die Klappe,} sagte Harry zu dem Gedanken, \emph{willst du, dass wir ein dunkler Zauberer werden?}

\emph{Was soll das bringen? Du wirst irgendwann als einer enden.}

Das… war etwas, das Harry normalerweise nicht gedacht hätte…\\ \emph{Das ist ein Placebo-Effekt}, sagte sich Harry wieder. \emph{Der Dementor kann nicht wirklich durch 3 körperliche Patronusse an mich herankommen, ich bilde mir das nur ein, wie es ist. Wenn ich dem Dementor tatsächlich gegenüberstehe, wird es sich ganz anders anfühlen, und dann werde ich wissen, dass ich vorher nur dumm gewesen bin}.

Ein leichter Schauer lief Harry über den Rücken, denn er hatte das Gefühl, dass es sich in der Tat ganz anders anfühlen würde, und zwar nicht im positiven Sinne. Der flammende silberne Phönix entstand aus dem Zauberstab des Schulleiters, der kleine Vogel verschwand und Anthony Goldstein machte sich auf den Rückweg. Der Schulleiter kam mit Anthony, anstatt den nächsten Namen zu rufen, der Patronus wartete hinten, um den Dementor zu bewachen. Harry blickte hinüber zu der Stelle, an der Hermine stand, direkt hinter dem leuchtenden Panther. Hermine wäre als Nächste an der Reihe gewesen, hatte sich aber anscheinend nur verzögert. Sie sah gestresst aus. Vorhin hatte sie Harry höflich gebeten, er möge doch bitte aufhören, sie zu stressen. Dumbledore lächelte, als er Anthony zurück zu den anderen eskortierte; er lächelte nur leicht, denn der Schulleiter sah sehr, sehr müde.

"Unglaublich", sagte Dumbledore mit einer Stimme, die viel schwächer klang als sein gewohntes Dröhnen. "Ein leibhaftiger Patronus, im ersten Jahr. Und eine erstaunliche Anzahl von Erfolgen unter den anderen jungen Schülern. Quirinus, ich muss anerkennen, dass Sie Ihren Standpunkt bewiesen haben."

Professor Quirrell legte den Kopf schief.\\ "Eine einfache Vermutung, würde ich sagen. Ein Dementor greift durch Angst an, und Kinder haben weniger Angst."

"Weniger Angst?", fragte Auror Goryanof von seinem Platz aus.

"Das habe ich auch gesagt", sagte Dumbledore. "Und Professor Quirrell wies darauf hin, dass Erwachsene nicht weniger, sondern mehr Angst haben; ein Gedanke, der mir, wie ich gestehen muss, noch nie in den Sinn gekommen ist."

"Das war nicht meine präzise Formulierung", sagte Professor Quirrell trocken, "aber es wird reichen. Und der Rest unserer Vereinbarung, Schulleiter?"

"Wie Sie meinen", sagte Dumbledore zögernd. "Ich gebe zu, dass ich nicht damit gerechnet habe, diese Wette zu verlieren, Quirinus, aber Sie haben Ihre Klugheit bewiesen."

Alle Schüler sahen sie verwirrt an, außer Hermine, die in Richtung des Käfigs und der großen, verfallenden Roben starrte, und Harry, der alle beobachtete, da er sich einbildete, paranoid zu sein.

Professor Quirrell sagte in einem Tonfall, der nicht zu weiteren Kommentaren einlud:\\ "Es ist mir erlaubt, den Tötungsfluch Schülern beizubringen, die ihn lernen wollen. Das macht sie wesentlich sicherer vor dunklen Zauberern und anderen Plagegeistern, und es ist töricht zu glauben, dass sie sonst keine tödliche Magie kennen würden."\\ Professor Quirrell hielt inne, seine Augen verengten sich.\\ "Schulleiter, ich stelle respektvoll fest, dass Sie nicht gut aussehen. Ich schlage vor, den Rest der Tagesaufgabe Professor Flitwick zu überlassen."

Dumbledore schüttelte den Kopf.\\ "Wir sind fast fertig für den Tag, Quirinus. Ich werde der Letzte sein."

Hermine hatte sich Anthony genähert.\\ "Hauptmann Goldstein", sagte sie, und ihre Stimme zitterte nur ein wenig, "können Sie mir einen Rat geben?"

"Hab keine Angst", sagte Anthony fest.\\ "Denk nicht an das, woran es dich denken zu lassen versucht. Du hältst den Zauberstab nicht nur als Schutzschild gegen die Angst vor dir hoch, du schwingst deinen Zauberstab, um die Angst zu vertreiben, so machst du aus einem glücklichen Gedanken etwas Festes …"\\ Anthony zuckte hilflos mit den Schultern.\\ "Ich meine, ich habe das alles schon mal gehört, aber …"\\ Andere Schüler begannen, sich um Anthony zu versammeln, mit ihren eigenen Fragen.

"Miss Granger?", sagte der Schulleiter.\\ Seine Stimme war vielleicht sanft, oder einfach nur geschwächt. Hermine straffte die Schultern und folgte ihm.

"Was hast du unter dem Umhang gesehen?" sagte Harry zu Anthony. Anthony sah Harry überrascht an und antwortete dann:\\ "Einen sehr großen Mann, der tot war, ich meine, eine Art tote Form und eine tote Farbe…. es tat weh, ihn zu sehen, und ich wusste, dass das der Dementor war, der mich angreifen wollte."

Harry schaute wieder nach draußen, wo Hermine dem Käfig und dem Umhang gegenüberstand.\\ Hermine hob ihren Zauberstab in Position für die ersten Gesten. Der Phönix des Schulleiters blinzelte aus der Existenz.

\textbf{Und Hermine stieß einen winzigen, jämmerlichen Schrei aus, zuckte zurück} -\\ - trat einen Schritt zurück,\\ Harry konnte sehen, wie sich ihr Zauberstab bewegte, und dann schwang sie ihn und sagte\\ "Expecto Patronum!"

\emph{Nichts geschah.}

Hermine drehte sich um und rannte los.

"Expecto Patronum!", sagte die tiefe Stimme des Schulleiters, und der silberne Phönix flammte wieder auf.

Das junge Mädchen stolperte und rannte weiter, seltsame Laute kamen aus ihrer Kehle.

"Hermine!" Susan schrie es, und Hannah, und Daphne, und Ernie, und sie alle fingen an, auf sie zuzurennen; selbst als Harry, der immer einen Schritt voraus dachte, sich auf dem Absatz drehte und zu dem Tisch mit der Schokolade rannte. Selbst nachdem Harry die Schokolade in Hermines Mund geschoben hatte und sie gekaut und geschluckt hatte, atmete sie immer noch in großen Schlägen und weinte, ihre Augen schienen immer noch unkonzentriert zu sein.

\emph{Sie kann keine dauerhafte Schäden haben}, dachte Harry verzweifelt angesichts der Verwirrung in ihm, der schrecklichen Angst und der tödlichen Wut, die sich umeinander zu winden begannen, \emph{sie kann nicht verletzt sein, sie war nicht einmal zehn Sekunden lang ausgesetzt, geschweige denn vierzig} - Aber sie konnte vorübergehende Schäden haben, wie Harry in diesem Moment klar wurde, es gab keine Regel, dass man nicht in nur zehn Sekunden vorübergehend von einem Dementor verletzt werden konnte, wenn man empfindlich genug war.

Dann schienen sich Hermines Augen zu fokussieren, herumzuspringen und sich auf ihm niederzulassen.\\ "Harry", keuchte sie, und die anderen Schüler verstummten. "Harry, nicht. Nicht!"

Harry hatte plötzlich Angst zu fragen, was er nicht tun sollte, war er in ihren schlimmsten Erinnerungen oder in einem Alptraum aus dem Schlaf, den sie jetzt im Wachleben wiedererlebte?\\ "Geh nicht in die Nähe davon!", sagte Hermine.\\ Ihre Hand griff nach ihm, packte ihn am Revers seines Umhangs.\\ "Du darfst nicht in seine Nähe gehen, Harry! Es hat zu mir gesprochen, Harry, es kennt dich, es weiß, dass du hier bist!"

"Was -" sagte Harry und verfluchte sich für die Frage.

"Der Dementor!", sagte Hermine. Ihre Stimme stieg zu einem Schrei an. "Professor Quirrell will, dass er dich frisst!"

In der plötzlichen Stille trat Professor Quirrell ein paar Schritte vor; aber er kam nicht näher heran (schließlich war Harry da).\\ "Miss Granger", sagte er, und seine Stimme war ernst, "ich denke, Sie sollten noch etwas Schokolade bekommen."

"Professor Flitwick, lassen Sie Harry nicht versuchen, schicken Sie ihn zurück!"

Der Schulleiter war inzwischen eingetroffen, und er und Professor Flitwick tauschten besorgte Blicke aus.

"Ich habe den Dementor nicht sprechen hören", sagte der Schulleiter. "Trotzdem…"

"Fragen Sie einfach", sagte Professor Quirrell und klang ein wenig müde.

"Hat der Dementor gesagt, wie er an Harry herankommen würde?", fragte der Schulleiter.

"Alle seine leckersten Teile zuerst", sagte Hermine, "er würde - er würde essen -"\\ Hermine blinzelte.\\ Etwas Vernunft schien in ihre Augen zurückzukehren. Dann fing sie an zu weinen.

"Du warst zu tapfer, Hermine Granger", sagte der Schulleiter.\\ Seine Stimme war sanft, aber deutlich hörbar.\\ "Viel mutiger, als ich es verstanden habe. Du hättest dich umdrehen und weglaufen sollen, nicht ausharren und versuchen, deinen Zauber zu vollenden. Wenn Sie älter und stärker sind, Miss Granger, weiß ich, dass Sie es wieder versuchen werden, und ich weiß, dass Sie Erfolg haben werden."

"Es tut mir leid", sagte Hermine keuchend, "es tut mir leid, es tut mir leid, es tut mir leid… Es tut mir leid, Harry, ich kann dir nicht sagen, was ich gesehen habe, ich habe es nicht angeschaut, ich habe mich nicht getraut, es anzuschauen, ich wusste, es war zu schrecklich, um es jemals zu sehen…"

Es hätte Harry sein sollen, aber er hatte gezögert, weil seine Hände ganz schokoladig waren; und dann waren Ernie und Susan da, halfen Hermine von der Stelle, wo sie ins Gras gefallen war, und führten sie zum Snacktisch. Fünf Tafeln Schokolade später schien Hermine wieder in Ordnung zu sein, und sie ging hinüber und entschuldigte sich bei Professor Quirrell; aber sie beobachtete immer Harry, jedes Mal, wenn er in ihre Richtung blickte. Nur einmal war er auf sie zugegangen und hatte innegehalten, als sie weggetreten war. Ihre Augen hatten sich im Stillen entschuldigt und ihn im Stillen angefleht, sie in Ruhe zu lassen.

Neville Longbottom hatte etwas gesehen, das tot und halb aufgelöst war und mit einem Gesicht das wie ein zerquetschter Schwamm aussah aus dem Eiter lief.\\ Es war das Schlimmste, was bisher jemand beschrieben hatte, der es gesehen hatte.\\ Neville hatte zuvor ein kleines Flackern von Licht mit seinem Zauberstab erzeugen können, aber er hatte sich, klug und mit großer Geistesgegenwart, umgedreht und war weggelaufen, anstatt zu versuchen, seinen eigenen Patronus-Zauber zu wirken.

(Der Schulleiter hatte nichts zu den anderen Schülern gesagt, niemandem gesagt, er solle weniger mutig sein; aber Professor Quirrell hatte ruhig bemerkt, dass, wenn man den Fehler machte, nachdem man gewarnt worden war, dies der Zeitpunkt war, an dem Unwissenheit zu Dummheit wurde.)

"Professor Quirrell?" sagte Harry mit leiser Stimme, nachdem er so nahe an den Verteidigungsprofessor herangetreten war, wie er es wagte.\\ "Was sehen Sie, wenn Sie -"

"Frag nicht." Die Stimme war sehr flach. Harry nickte respektvoll.

"Wie haben Sie sich ursprünglich gegenüber dem Schulleiter ausgedrückt, wenn ich fragen darf?"

"Unsere schlimmsten Erinnerungen können nur schlimmer werden, je älter wir werden."

"Ah", sagte Harry. "Logisch."

Dann flackerte etwas Seltsames in Professor Quirrells Augen auf, als er Harry ansah.\\ "Hoffen wir", sagte Professor Quirrell, "dass Sie bei diesem Versuch erfolgreich sind, Mr. Potter. Denn dann könnte der Schulleiter Ihnen seinen Trick beibringen, mit einem Patronus Nachrichten zu verschicken, die nicht gefälscht oder abgefangen werden können, und die militärische Bedeutung dessen kann man gar nicht hoch genug einschätzen. Es wäre ein enormer Vorteil für die Chaos Legion und eines Tages, so vermute ich, auch für das ganze Land. Aber wenn Sie keinen Erfolg haben, Mr. Potter … nun, ich werde es verstehen."

Morag MacDougal hatte mit schwankender Stimme "Autsch" gesagt, und Dumbledore hatte sofort seinen Patronus neu gezaubert. Parvati Patil hatte einen leibhaftigen Patronus in Form eines Tigers hervorgebracht, größer als Dumbledores Phönix, wenn auch nicht annähernd so hell.\\ Es gab einen großen Beifallssturm von allen Zuschauern, wenn auch nicht so schockiert wie bei Anthony.

Und dann war Harry an der Reihe. Der Schulleiter rief den Namen von Harry Potter, und Harry hatte Angst. Harry wusste, er wusste, dass er versagen würde, und er wusste, dass es weh tun würde. Aber er musste es trotzdem versuchen; denn manchmal verwandelte sich ein Zauberer in der Gegenwart eines Dementors von einem nicht vorhandenen Lichtblitz in einen leibhaftigen Patronus, und niemand verstand warum. Und weil Harry, wenn er sich nicht gegen Dementoren verteidigen konnte, in der Lage sein musste, ihre Annäherung zu erkennen, ihr Gefühl in seinem Geist zu spüren und wegzulaufen, bevor es zu spät war.\\ \emph{\hfill\break Was ist meine schlimmste Erinnerung…?} Harry hatte erwartet, dass der Schulleiter ihm einen besorgten Blick zuwerfen würde, oder einen hoffnungsvollen Blick, oder einen zutiefst weisen Rat; aber stattdessen sah Albus Dumbledore ihn nur mit stiller Ruhe an.\\ \emph{Er glaubt, dass ich versagen werde, aber er wird mich nicht sabotieren, indem er es mir sagt,} dachte Harry, \emph{wenn er wahre Worte der Ermutigung zu sprechen hätte, würde er sie sprechen.}..

Der Käfig kam näher. Er war schon angelaufen, aber nicht zu nichts verrostet, noch nicht. Der Mantel kam näher. Er war aufgerissen und von ungeflickten Löchern durchzogen; er war an diesem Morgen neu gewesen, hatte Auror Goryanof gesagt.

"Schulleiter?" sagte Harry. "Was sehen Sie?"

Auch die Stimme des Schulleiters war ruhig.\\ "Die Dementoren sind Kreaturen der Angst, und in dem Maße, wie Ihre Angst vor dem Dementor abnimmt, nimmt auch die Furchterregendheit seiner Gestalt ab. Ich sehe einen großen, dünnen, nackten Mann. Er ist nicht verwest. Sein Anblick ist nur etwas schmerzhaft. Das ist alles. Was siehst du, Harry?"

… Harry konnte nicht unter den Mantel sehen. Oder es war so, dass sein Geist sich weigerte zu sehen, was unter dem Mantel war… Nein, sein Verstand versuchte, das Falsche unter dem Mantel zu sehen, Harry spürte es, seine Augen versuchten, einen Fehler zu erzwingen.\\ Aber Harry hatte sein Bestes getan, um sich zu trainieren, dieses winzige Gefühl der Verwirrung zu bemerken, um automatisch davor zurückzuschrecken, sich etwas auszudenken; und jedes Mal, wenn sein Verstand versuchte, eine Lüge darüber zu erfinden, was unter dem Mantel war, war dieser Reflex schnell genug, um ihn abzuschalten.\\ Harry schaute unter den Umhang und sah…

\emph{Eine offene Frage.}

Harry wollte nicht zulassen, dass sein Verstand etwas Falsches sah, und so sah er nichts, als ob der Teil seines visuellen Kortex, der das Signal bekam, einfach aufhörte zu existieren. Es gab einen blinden Fleck unter dem Umhang. Harry konnte nicht wissen, was da drunter war.

\emph{Nur, dass es viel schlimmer war als jede verwesende Mumie.}

Das unsichtbare Grauen unter dem Umhang war jetzt ganz nah, aber der flammende Vogel des Mondlichts, der weiße Phönix, lag noch zwischen ihnen.\\ Harry wollte weglaufen, wie einige der anderen Schüler. Die Hälfte derer, die mit ihren Patronus-Zaubern kein Glück gehabt hatten, war heute gar nicht erst aufgetaucht. Von den Übriggebliebenen war die Hälfte geflohen, noch bevor der Schulleiter seinen eigenen Patronus aufgelöst hatte, und niemand hatte ein Wort gesagt. Es hatte ein wenig Gelächter gegeben, als Terry sich umgedreht hatte und vor seinem eigenen Versuch zurückgelaufen war; und Susan und Hannah, die vor ihm gegangen waren, hatten alle angeschrien, sie sollten still sein. Aber Harry war der Junge, der lebte, und er würde viel Respekt verlieren, wenn man sah, dass er aufgab, ohne es überhaupt zu versuchen… Stolz und Rollen schienen zu schwinden und abzufallen, in der Gegenwart dessen, was unter dem Mantel lag.

\emph{Warum bin ich noch hier?} Es war nicht die Scham, dass andere ihn für feige hielten, die Harrys Füße an Ort und Stelle hielt. Es war nicht die Hoffnung, seinen Ruf wiederherzustellen, die seinen Zauberstab aufbrachte. Es war nicht der Wunsch, den Patronus-Zauber als Magie zu beherrschen, der seine Finger in die Ausgangsposition bewegte.\\ Es war etwas anderes, etwas, das sich dem entgegenstellen musste, was auch immer unter dem Umhang lag, dies war die wahre Dunkelheit und Harry musste herausfinden, ob sie in ihm lag, die Macht, sie zurückzutreiben. Harry hatte geplant, ein letztes Mal zu versuchen, an seinen Bücherbummel mit seinem Vater zu denken, aber stattdessen kam ihm im letzten Moment, im Angesicht des Dementors, eine andere Erinnerung in den Sinn, etwas, das er vorher nicht versucht hatte; ein Gedanke, der nicht auf die übliche Weise warm und glücklich war, sondern sich irgendwie richtiger anfühlte.

Und Harry erinnerte sich an die Sterne, erinnerte sich daran, wie sie furchtbar hell und unerschütterlich in der Stillen Nacht brannten; er ließ zu, dass dieses Bild ihn ausfüllte, ihn ganz ausfüllte, wie eine Okklumentik-Barriere, die sich über seinen gesamten Geist legte, wurde wieder zum körperlosen Bewusstsein der Leere.

\emph{Der helle, silbern leuchtende Phönix verschwand.}

\textbf{Und der Dementor schlug in seinen Geist ein wie die Faust Gottes.}

\textbf{ANGST / KÄLTE/ DUNKELHEIT}

\emph{Es gab einen Moment, in dem die beiden Kräfte frontal aufeinandertrafen, in dem die friedliche, sternenklare Erinnerung sich gegen die Angst behauptete}, selbst als Harrys Finger die Bewegungen des Zauberstabs begannen, geübt, bis sie automatisch geworden waren.

Sie waren nicht warm und glücklich, diese flammenden Lichtpunkte in vollkommener Schwärze; aber es war ein Bild, das der Dementor nicht leicht durchdringen konnte. Denn die still brennenden Sterne waren groß und furchtlos, und in der Kälte und Dunkelheit zu leuchten war ihr natürlicher Zustand.

Aber es gab einen Makel, einen Riss, eine Bruchlinie in dem unbeweglichen Objekt, das versuchte, dieser unwiderstehlichen Kraft zu widerstehen.

Harry spürte einen Stich der Wut auf den Dementor, weil er versuchte, sich von ihm zu ernähren, und es war wie ein Ausrutschen auf nassem Eis. Harrys Gedanken begannen seitwärts zu gleiten, in Bitterkeit, schwarze Wut, tödlichen Hass - Harrys Zauberstab kam im letzten Schwung hoch.\\

\textbf{\emph{Es fühlte sich falsch an.}}

"Expecto Patronum", sprach seine Stimme, die Worte hohl und sinnlos.

Und Harry fiel in seine dunkle Seite, fiel hinunter in seine dunkle Seite, weiter und schneller und tiefer als je zuvor, hinunter, während sich die Rutsche beschleunigte, während der Dementor sich an den entblößten und verletzlichen Stellen festhielt und sich von ihnen ernährte, das Licht auffraß.\\ Ein schwindender Reflex krabbelte nach Wärme, aber selbst als ein Bild von Hermine zu ihm kam, oder ein Bild von Mum und Dad, verdrehte der Dementor es, zeigte ihm Hermine, die tot auf dem Boden lag, die Leichen seiner Mutter und seines Vaters, und dann wurde auch das weggesaugt.\\ In das Vakuum stieg die Erinnerung, die schlimmste Erinnerung, etwas, das so lange vergessen war, dass die neuralen Muster nicht mehr hätten existieren dürfen.

…\\ "\emph{Lily, nimm Harry und geh! Er ist es}!", rief eine Männerstimme. "\emph{Geh! Lauf! Ich werde ihn aufhalten!}"

Und Harry konnte nicht anders, als in den leeren Tiefen seiner dunklen Seite zu denken, wie lächerlich übermütig James Potter gewesen war. \emph{Lord Voldemort aufhalten? Womit?}\\ Dann sprach die andere Stimme, hochtönend wie das Zischen eines Teekessels, und es war, als ob Trockeneis auf Harrys jeden Nerv gelegt würde, wie eine Marke aus Metall, die auf die Temperatur von flüssigem Helium gekühlt und auf jeden Teil von ihm gelegt wurde.\\ Und die Stimme sagte: "\textbf{\emph{Avadakedavra}}."\\ …

(Der Zauberstab flog aus den nervenlosen Fingern des Jungen, als sein Körper zu krampfen und zu sinken begann, die Augen des Schulleiters weiteten sich alarmiert, als er seinen eigenen Patronus-Zauber einsetzte.)

…\\ "\emph{Nicht Harry, nicht Harry, bitte nicht Harry!}", schrie die Frauenstimme.

Was auch immer von Harry übrig war, hörte mit allem Licht, das aus ihm herausgesaugt wurde, in die tote Leere seines Herzens und fragte sich, \emph{ob sie dachte, dass Lord Voldemort aufhören würde, weil sie höflich fragte.}

"\textbf{\emph{Tritt beiseite, Frau!}}", sagte die schrille Stimme von brennender Kälte.\\ "\textbf{\emph{Für dich bin ich nicht gekommen, nur für den Jungen.}}"

"\emph{Nicht Harry! Bitte … habt Erbarmen … habt Erbarmen …}"

\emph{Lily Potter,} dachte Harry, \emph{schien nicht zu verstehen, welche Art von Menschen überhaupt zu Dunklen Lords wurden; und wenn dies die beste Strategie war, die sie sich ausdenken konnte, um das Leben ihres Kindes zu retten, dann war das ihr endgültiges Versagen als Mutter}.

"\textbf{\emph{Ich gebe dir diese seltene Chance zu fliehen}}", sagte die schrille Stimme.\\ "\textbf{\emph{Aber ich werde mir nicht die Mühe machen, dich zu überwältigen, und dein Tod hier wird dein Kind nicht retten. Tritt beiseite, törichtes Weib, wenn du überhaupt noch einen Funken Verstand in dir hast!}}"

"\emph{Nicht Harry, bitte nicht, nimm mich, töte mich lieber!}"

Das leere Ding, das Harry war, fragte sich, \emph{ob Lily Potter sich ernsthaft vorstellte, dass Lord Voldemort ja sagen, sie töten und dann abreisen und ihren Sohn unversehrt lassen würde.}

"\textbf{\emph{Nun gut}}", sagte die Stimme des Todes, die jetzt kalt und amüsiert klang,\\ "\textbf{\emph{ich akzeptiere den Handel. Du wirst sterben und das Kind wird leben. Jetzt lass deinen Zauberstab fallen, damit ich dich ermorden kann.}}"

\emph{Es herrschte eine grässliche Stille.}

Lord Voldemort begann zu lachen, ein schreckliches, verächtliches Lachen. Und dann, endlich, kreischte Lily Potters Stimme in verzweifeltem Hass:\\ "\emph{Avada ke-}"\\ Die tödliche Stimme endete zuerst, der Fluch schnell und präzise.\\ "\textbf{\emph{Avadakedavra.}}"

Ein grelles, grünes Aufblitzen markierte das Ende von Lily Potter. Und der Junge in der Krippe sah es, die Augen, diese beiden purpurroten Augen, schienen hellrot zu glühen, zu lodern wie Miniatursonnen, erfüllten Harrys ganzes Blickfeld, als sie sich mit seinem eigenen verbanden -\\ …

Die anderen Kinder sahen Harry Potter fallen, sie hörten Harry Potter schreien, einen dünnen, hohen Schrei, der ihre Ohren wie Messer zu durchbohren schien.\\ Es gab einen leuchtenden Silberblitz, als der Schulleiter\\ "\textbf{Expecto Patronum!}" brüllte und der flammende Phönix wieder zum Leben erwachte.

Aber Harry Potters schrecklicher Schrei ging weiter und weiter und weiter, selbst als der Schulleiter den Jungen in seine Arme nahm und ihn von dem Dementor wegtrug, selbst als Neville Longbottom und Professor Flitwick gleichzeitig nach der Schokolade griffen und - Hermine wusste es, sie wusste es, als sie es sah, sie wusste, dass ihr Albtraum wahr gewesen war, er wurde wahr, irgendwie wurde er wahr.

"Holt ihm Schokolade!", forderte die Stimme von Professor Quirrell, sinnlos, denn Professor Flitwicks winzige Gestalt raste bereits auf die Stelle zu, an der der Schulleiter auf die Schüler zuraste. Hermine bewegte sich selbst vorwärts, obwohl sie nicht wusste, was sie sonst tun sollte -

"\textbf{Zaubert Patroni!}", rief der Schulleiter, als er Harry hinter die Auroren brachte.\\ "\textbf{Jeder, der kann! Bringt sie zwischen Harry und den Dementor! Er nährt sich immer noch an ihm!}"\\ Es gab einen Moment des erstarrten Entsetzens.

"Expecto Patronum! ", riefen Professor Flitwick und Auror Goryanof, und dann Anthony Goldstein, aber er scheiterte beim ersten Mal, und dann Parvati Patil, die es schaffte, und dann versuchte Anthony es noch einmal, und sein silberner Vogel breitete seine Flügel aus und schrie den Dementor an, und Dean Thomas brüllte die Worte, als wären sie in Feuerbuchstaben geschrieben worden, und sein Zauberstab gebar einen gewaltigen weißen Bären; acht flammende Patronusse standen in einer Reihe zwischen Harry und dem Dementor, \textbf{\emph{und Harry schrie und schrie weiter, als der Schulleiter ihn auf das trockene Gras legte.}}

Hermine konnte keinen Patronus-Zauber wirken, also rannte sie auf die Stelle zu, wo Harry lag. In ihrem Kopf versuchte etwas zu erraten, wie lange es schon her war.\\ \emph{Waren es zwanzig Sekunden? Mehr?}

Auf dem Gesicht von Albus Dumbledore lag eine furchtbare Agonie und Fassungslosigkeit.\\ Er hielt seinen langen schwarzen Zauberstab in der Hand, aber er sprach keinen Zauberspruch, sondern blickte nur entsetzt auf Harrys zuckenden Körper hinunter - Hermine wusste nicht, was sie tun sollte, sie verstand nicht, was geschah, und der mächtigste Zauberer der Welt schien ebenso ratlos zu sein.

"\textbf{Benutz deinen Phönix}!", brüllte Professor Quirrell.\\ "\textbf{Bringt ihn weit weg von diesem Dementor!}"

Ohne ein einziges Wort hob der Schulleiter Harry in seine Arme und verschwand in einem Feuerknall zusammen mit dem plötzlich auftauchenden Fawkes; und der Patronus des Schulleiters blinzelte heraus, wo er den Dementor bewacht hatte.

Entsetzen und Verwirrung und plötzliches Geplapper.

"Mr. Potter sollte sich erholen", sagte Professor Quirrell mit erhobener Stimme, doch sein Tonfall war nun wieder ruhig, "ich glaube, es waren etwas mehr als zwanzig Sekunden."

Dann erschien der strahlend weiße Phönix wieder, als würde er von irgendwoher vor ihnen herfliegen, zu Hermine Granger kam das Wesen aus Mondlicht, und es rief ihr mit Albus Dumbledores Stimme zu:\\

\textbf{\emph{"Es ernährt sich immer noch von ihm, sogar hier! Aber wie? Wenn du es weißt, Hermine Granger, musst du es mir sagen! Sag es mir!!!"}}

Der Senior Auror drehte sich um und starrte sie an, ebenso wie viele Schüler.\\ Professor Flitwick drehte sich nicht um, er richtete jetzt seinen Zauberstab auf Professor Quirrell, der eindeutig leere Hände vorhielt.

Die Sekunden tickten vorbei, ungezählt.\\ Sie konnte sich nicht erinnern, sie konnte sich nicht klar an den Albtraum erinnern, sie konnte sich nicht erinnern, warum sie es für möglich gehalten hatte, warum sie Angst gehabt hatte - da wurde Hermine klar, was sie tun sollte, und es war die schwerste Entscheidung ihres Lebens.

\emph{Was, wenn das, was Harry passiert war, auch ihr passierte?}\\ Alle ihre Glieder waren kalt wie der Tod, ihre Sicht wurde dunkel, die Angst überwältigte alles; sie hatte gesehen, wie Harry starb, wie Mum und Dad starben, wie all ihre Freunde starben, wie alle starben, so dass sie am Ende, wenn sie starb, allein sein würde.\\ \emph{Das war ihr geheimer Albtraum, über den sie nie mit jemandem gesprochen hatte, der dem Dementor seine Macht über sie gegeben hatte, das Einsamste war es, allein zu sterben.}\\ Sie wollte nicht noch einmal an diesen Ort gehen, sie, sie wollte nicht, sie wollte nicht für immer dort bleiben -

\emph{Du hast Mut genug für Gryffindor}, sagte die ruhige Stimme des Sprechenden Hutes in ihrer Erinnerung, \emph{aber du wirst tun, was richtig ist, in jedem Haus, das ich dir gebe. Du wirst lernen, du wirst deinen Freunden beistehen, in jedem Haus, das du wählst. Also hab keine Angst, Hermine Granger, entscheide einfach, wo du hingehörst.}..

Es war keine Zeit zum Entscheiden, Harry lag im Sterben.\\ "Ich kann mich jetzt nicht erinnern", sagte Hermine, ihre Stimme brach, "aber ich gehe wieder vor den Dementor…"

\textbf{Sie begann, auf den Dementor zuzulaufen.}

"Miss Granger!", quietschte Professor Flitwick, aber er machte keine Anstalten, sie aufzuhalten, sondern hielt nur seinen Zauberstab auf Professor Quirrell gerichtet.

"Alle!", rief Auror Komodo mit militärischer Befehlsstimme.\\ "Weißt ihr mit euren Schutzpatronen den Weg!"

"\textbf{FLITWICK}!", brüllte Professor Quirrell. "\textbf{POTTERS ZAUBERSTAB!}"

Noch während Hermine verstand, rief Professor Flitwick bereits "Accio!", und sie sah, wie der Holzstab von dort, wo er gelegen hatte, hochschnellte und fast den Käfig des Dementors berührte.

…\\ An einem anderem Ort öffneten sich die Augen, tot und ausdruckslos.

"Harry!?", keuchte eine Stimme in der farblosen Welt. "Harry! Sprich mit mir!"\\ Das Gesicht von Albus Dumbledore beugte sich in das Blickfeld, das von einer fernen Marmordecke eingenommen wurde.\\

…

…\\ "\textbf{\emph{Du bist lästig}}", sagte die leere Stimme. "\textbf{\emph{Du solltest sterben.}}"\\

