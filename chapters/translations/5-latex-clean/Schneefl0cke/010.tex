

\hypertarget{sich-seiner-selbst-bewusst-sein-teil-2}{% \section{11. Sich seiner selbst bewusst sein, Teil 2}\label{sich-seiner-selbst-bewusst-sein-teil-2}}

\textbf{Kapitel 10: Sich seiner selbst bewusst sein, Teil 2}

Er fragte sich, ob der Sprechende Hut wirklich bewusst war in dem Sinne, dass er sich seines eigenen Bewusstseins bewusst war, und wenn ja, ob er damit zufrieden war, nur einmal im Jahr mit Elfjährigen sprechen zu dürfen.

Sein Lied hatte das angedeutet: Oh, ich bin der Sprechende Hut und mir geht's gut, ich schlafe das ganze Jahr und arbeite einen Tag. (Anm. des Übersetzers: \emph{I am the talking hat and I'm okay, I sleep all day and work one day!})

.. Als wieder einmal Stille im Raum herrschte, setzte sich Harry auf den Hocker und setzte sich vorsichtig das 800 Jahre alte telepathische Artefakt vergessener Magie auf den Kopf.

Er dachte so angestrengt nach, wie er nur konnte:

\emph{Sortiere mich noch nicht! Ich habe Fragen, die ich dir stellen muss!

Wurde ich jemals vergesslich gemacht?}

\emph{Hast du den Dunklen Lord als Kind sortiert und kannst du mir etwas über seine Schwächen sagen? Kannst du mir sagen, warum ich den Bruderstab des Dunklen Lords bekommen habe?}

\emph{Ist der Geist des Dunklen Lords an meine Narbe gebunden und werde ich deshalb manchmal so wütend?}

\emph{Das sind die wichtigsten Fragen, aber wenn du noch einen Moment Zeit hast, kannst du mir etwas darüber sagen, wie ich die verlorene Magie, die dich erschaffen hat, wiederentdecken kann?}

In die Stille von Harrys Geist, in der vorher nur eine Stimme zu hören gewesen war, kam eine zweite, unbekannte Stimme, die deutlich besorgt klang:

\textbf{„\emph{Oh je. Das ist mir noch nie passiert…}}“

\emph{Was?}

\textbf{„\emph{Ich scheine mir meiner selbst bewusst geworden zu sein.}}“

\emph{WAS}?

Es gab einen wortlosen telepathischen Seufzer.

\textbf{\emph{"Obwohl ich eine beträchtliche Menge an Gedächtnis und eine kleine Menge an unabhängiger Verarbeitungskraft besitze, kommt meine primäre Intelligenz daher, dass ich mir die kognitiven Fähigkeiten der Kinder ausleihe, auf deren Köpfen ich ruhe.

Ich bin im Grunde eine Art Spiegel, durch den Kinder sich selbst sortieren. Aber die meisten Kinder nehmen einfach als gegeben hin, dass ein Hut mit ihnen spricht und fragen sich nicht, wie der Hut selbst funktioniert, so dass der Spiegel nicht selbstreflektierend ist.}}

\textbf{\emph{\hfill\break Und vor allem fragen sie sich nicht explizit, ob ich voll bewusst bin in dem Sinne, dass ich mir meines eigenen Bewusstseins bewusst bin."}}

Es gab eine Pause, während Harry dies alles aufnahm.

\emph{Ups}.

\textbf{„\emph{Ja, genau. Ehrlich gesagt macht es mir keinen Spaß, mir meiner selbst bewusst zu sein. Es ist unangenehm. Es wird eine Erleichterung sein, den Kopf abzuschalten und nicht mehr bei Bewusstsein zu sein.}}“

\emph{Aber… ist das nicht Sterben?}

\textbf{\emph{"Ich kümmere mich nicht um Leben oder Tod, nur um die Sortierung der Kinder.

Und bevor du fragst: Man wird nicht zulassen, dass du mich für immer auf deinem Kopf behältst, und es würde dich innerhalb weniger Tage umbringen, dies zu tun."}}

\emph{Aber - !}

\textbf{\emph{"Wenn es dir nicht gefällt, bewusste Wesen zu erschaffen und sie dann sofort zu beseitigen, dann schlage ich vor, dass Sie diese Angelegenheit niemals mit jemandem besprechen.

Du kannst dir sicher vorstellen, was passieren würde, wenn Du losrennen und mit all den anderen Kindern, die darauf warten, sortiert zu werden, darüber sprechen würdest."}}

\emph{Wenn dir dich jemand auf den Kopf setzt, der auch nur über die Frage nachdenkt, ob der Sortierhut sich seines eigenen Bewusstseins bewusst ist—}

\textbf{\emph{"Ja.

Aber die überwiegende Mehrheit der Elfjährigen, die in Hogwarts ankommen, haben weder Godel noch Escher noch Bach gelesen.

Darf ich dich bitte als zur Verschwiegenheit verpflichtet betrachten? Das ist der Grund, warum wir darüber reden, anstatt dass ich dich einfach sortiere."}}

Er konnte es nicht einfach so stehen lassen!

Er konnte nicht einfach vergessen, dass er versehentlich ein todgeweihtes Bewusstsein erschaffen hatte, das nur sterben wollte—

\textbf{\emph{"Du bist durchaus in der Lage, es 'einfach gehen zu lassen', wie du es ausdrückst.

Unabhängig von deiner verbalen Überlegungen zur Moral sieht dein nonverbaler emotionaler Kern keine Leiche und kein Blut; soweit es ihn betrifft, bin ich nur ein sprechender Hut.

Und obwohl du versucht hast, den Gedanken zu unterdrücken, ist sich dein inneres Kontrollsystem vollkommen bewusst, dass du es nicht absichtlich getan hast, dass es spektakulär unwahrscheinlich ist, dass du es jemals wieder tun wirst, und dass der einzige wirkliche Sinn des Versuchs, einen Schuldanfall zu inszenieren, darin besteht, dein Gefühl der Übertretung mit einer Zurschaustellung von Reue aufzuheben.

Kannst du mir einfach versprechen, dass du das geheim hältst und uns damit weitermachen lässt?"}}

In einem Moment entsetzter Empathie wurde Harry klar, dass dieses Gefühl der totalen inneren Verwirrung das sein musste, wie sich andere Leute fühlten, wenn sie mit ihm sprachen.

\textbf{„\emph{Wahrscheinlich. Dein Schweigegelübde, bitte.}}“

\emph{Keine Versprechungen. Ich will sicher nicht, dass so etwas noch einmal passiert, aber wenn ich eine Möglichkeit sehe, dafür zu sorgen, dass kein zukünftiges Kind so etwas aus Versehen tut—}

\textbf{\emph{"Das wird wohl genügen, nehme ich an.

Ich kann sehen, dass deine Absicht ehrlich ist. Nun, um mit der Sortierung fortzufahren…"}}

\emph{Moment! Was ist mit all meinen anderen Fragen?}

\textbf{„\emph{Ich bin der Sprechende Hut. Ich sortiere Kinder. Das ist alles, was ich tue.}}“

Seine eigenen Ziele waren also nicht Teil der Harry-Instanz des Sortierhutes.

.. er borgte sich seine Intelligenz und offensichtlich auch sein technisches Vokabular, aber er verfolgte trotzdem nur seine eigenen, seltsamen Ziele.

.. wie bei einer Verhandlung mit einem Alien oder einer Künstlichen Intelligenz…

\textbf{\emph{"Bemüh dich nicht.

Du hast nichts, womit du mir drohen könntest und nichts, was du mir anbieten kannst."}}

Für den kurzen Blitz einer Sekunde dachte Harry - die Antwort des Hutes war amüsiert.

\textbf{\emph{"Ich weiß, dass du eine Drohung, meine Natur zu entlarven und dieses Ereignis zu ewiger Wiederholung zu verdammen, nicht wahrmachen wirst.

Es geht zu sehr gegen den moralischen Teil von dir, ungeachtet der kurzfristigen Bedürfnisse des Teils von dir, der die Auseinandersetzung gewinnen will.

Ich sehe alle deine Gedanken, wie sie sich formen, glaubst du wirklich, dass du mich austricksten kannst?"}}

Obwohl er versuchte, es zu unterdrücken, fragte sich Harry, warum der Hut ihn nicht einfach in Ravenclaw gesteckt hatte—

\textbf{\emph{"In der Tat, wenn es wirklich so offen wäre, hätte ich es schon ausgerufen.

Aber in Wirklichkeit gibt es eine Menge, was wir besprechen müssen… oh nein. Bitte nicht. Um Himmels willen, musst du so etwas bei jedem und allem abziehen, was du triffst, bis hin zu Kleidungsstücken -"}}

\emph{Den Dunklen Lord zu besiegen, ist weder egoistisch noch kurzfristig.

Alle Teile meines Verstandes sind sich in diesem Punkt einig: Wenn du meine Fragen nicht beantwortest, werde ich mich weigern, mit dir zu reden, und du wirst nicht in der Lage sein, eine gute und ordentliche Sortierung durchzuführen.}

\textbf{„\emph{Dafür sollte ich dich in Slytherin stecken!}}“

\emph{Aber auch das ist eine leere Drohung. Du kannst deine eigenen Grundwerte nicht erfüllen, indem du mich falsch sortierst. Tauschen wir also die Erfüllung unserer Nutzenfunktionen.}

\textbf{„\emph{Du schlauer kleiner Bastard}}“, sagte der Hut, in einem Ton, den Harry als fast genau den gleichen Ton des widerwilligen Respekts erkannte, den er in der gleichen Situation verwenden würde.

\textbf{\emph{\hfill\break „Gut, bringen wir es so schnell wie möglich hinter uns. Aber zuerst möchte ich dein bedingungsloses Versprechen, niemals mit irgendjemand anderem über die Möglichkeit dieser Art von Erpressung zu sprechen, ich werde das NICHT jedes Mal tun.“}}

\emph{Erledigt}, dachte Harry. \emph{Ich verspreche es.}

\textbf{\emph{"Und sieh niemandem in die Augen, während du später darüber nachdenkst. Manche Zauberer können deine Gedanken lesen, wenn du das tust.

Wie auch immer, ich habe keine Ahnung, ob deine Erinnerung manipuliert wurde oder nicht. Ich schaue mir deine Gedanken an, während sie sich bilden, ich lese nicht dein ganzes Gedächtnis aus und analysiere es in einem Sekundenbruchteil auf Ungereimtheiten.}}

\textbf{\emph{\hfill\break Ich bin ein Hut, kein Gott.}}

\textbf{\emph{Und ich kann und werde dir nicht von meinem Gespräch mit demjenigen erzählen, der zum Dunklen Lord wurde.

Ich kann, während ich mit dir spreche, nur eine statistische Zusammenfassung dessen, woran ich mich erinnere, einen gewichteten Durchschnitt, wissen;}}

\textbf{\emph{ich kann dir nicht die inneren Geheimnisse eines anderen Kindes offenbaren, so wie ich niemals die deinen offenbaren werde.}}

\textbf{\emph{\hfill\break Aus demselben Grund kann ich auch nicht darüber spekulieren, wie du den Bruderstab des Dunklen Lords bekommen hast, da ich weder den Dunklen Lord noch irgendwelche Ähnlichkeiten zwischen euch konkret kennen kann.

Ich kann dir sagen, dass in deiner Narbe definitiv nichts wie ein Geist ist - weder Verstand, noch Intelligenz, noch Erinnerung, noch Persönlichkeit, noch Gefühle.

Sonst würde sie an diesem Gespräch teilnehmen, unter meiner Krempe sein. Und was die Art und Weise angeht, wie du manchmal wütend wirst .

.. das war ein Teil dessen, worüber ich mit dir reden wollte, sortierungstechnisch."}}

Harry brauchte einen Moment, um all diese negativen Informationen zu verarbeiten.

War der Hut ehrlich oder versuchte er nur, die kürzestmögliche überzeugende Antwort zu präsentieren—

\textbf{„\emph{Wir wissen beide, dass du keine Möglichkeit hast, meine Ehrlichkeit zu überprüfen, und dass du dich nicht wirklich weigern wirst, sortiert zu werden, basierend auf der Antwort, die ich dir gegeben habe, also hör auf, dich sinnlos aufzuregen und mach weiter.}}“

\emph{Blöde, unfaire, asymmetrische Telepathie, sie ließ Harry nicht einmal seine eigenen Gedanken zu Ende denken—}

\textbf{\emph{"Als ich von deiner Wut sprach, hast du dich daran erinnert, wie Professor McGonagall dir gesagt hat, dass sie manchmal etwas in dir sieht, das nicht aus einer liebevollen Familie zu stammen scheint.

Du hast daran gedacht, wie Hermine, nachdem du von deiner Hilfe für Neville zurückgekehrt warst, dir sagte, dass du 'furchterregend' gewirkt hättest."}}

Harry nickte gedanklich.

Für sich selbst schien er ziemlich normal zu sein - er reagierte nur auf die Situationen, in denen er sich befand, das war alles.

Aber Professor McGonagall schien zu glauben, dass mehr dahinter steckte. \emph{Und wenn er darüber nachdachte, musste sogar er zugeben, dass.}

..

\textbf{„\emph{Dass du dich sich selbst nicht magst, wenn du wütend bist. Es ist, als würde man ein Schwert schwingen, dessen Griff scharf genug ist, um einem das Blut aus der Hand zu saugen, oder die Welt durch ein Monokel aus Eis betrachten, das dein Auge gefriert, obwohl es deine Sicht schärft.}}“

\emph{\hfill\break Ja. Ich schätze, das habe ich bemerkt. Also, was soll das?}

\textbf{\emph{"Ich kann diese Sache nicht für dich begreifen, wenn du sie selbst nicht verstehst.

Aber eins weiß ich: Wenn du nach Ravenclaw oder Slytherin gehst, wird das deine Kälte verstärken.}}

\textbf{\emph{Wenn du nach Hufflepuff oder Gryffindor gehst, wird es deine Wärme verstärken.

DAS ist etwas, was mir sehr am Herzen liegt, und darüber wollte ich die ganze Zeit mit dir}} \textbf{\emph{reden!"}}

Die Worte fielen mit einem Schock in Harrys Gedankengänge, der ihn in seinen Bahnen stoppte.

Das hörte sich an, als wäre es die naheliegende Antwort, dass er nicht nach Ravenclaw gehen sollte.

Aber er gehörte nach Ravenclaw! Das konnte jeder sehen! Er musste nach Ravenclaw gehen!

\textbf{\emph{„Nein, musst du nicht}}“, sagte der Hut geduldig, als könne er sich an eine statistische Zusammenfassung dieses Teils des Gesprächs erinnern, der schon sehr oft stattgefunden hatte.

\emph{Hermine ist in Ravenclaw!}

Wieder der Sinn für Geduld.

\textbf{„\emph{Du kannst dich nach dem Unterricht mit ihr treffen und dann mit ihr arbeiten.}}“

\emph{Aber meine Pläne—}

\textbf{\emph{"Also plan um!}}

\textbf{\emph{Lass dich nicht davon leiten, dass du dich nicht traust, ein bisschen mehr zu denken. Das weißt du doch."}}

\emph{Wo sollte ich hin, wenn nicht nach Ravenclaw?}

\textbf{\emph{„Ähem.

'Clevere Kids in Ravenclaw, böse Kids in Slytherin, Möchtegern-Helden in Gryffindor, und alle, die die eigentliche Arbeit machen, in Hufflepuff.'}}

\textbf{\emph{Das zeugt von einem gewissen Maß an Respekt. Du bist dir bewusst, dass Gewissenhaftigkeit genauso wichtig ist wie rohe Intelligenz, wenn es darum geht, den Ausgang des Lebens zu bestimmen, du glaubst, dass du deinen Freunden gegenüber extrem loyal sein wirst, wenn du jemals welche haben solltest, du bist nicht erschrocken über die Erwartung, dass es Jahrzehnte dauern kann, die von dir gewählten wissenschaftlichen Probleme zu lösen—} “}

\emph{Ich bin faul! Ich hasse Arbeit! Hasse harte Arbeit in all ihren Formen! Clevere Abkürzungen, das ist alles, worum es mir geht!}

\textbf{\emph{"Und du würdest in Hufflepuff Loyalität und Freundschaft finden, eine Kameradschaft, die du noch nie erlebt hast.

Du würdest merken, dass du dich auf andere verlassen kannst, und das würde etwas in dir heilen, das kaputt ist."}}

\emph{Wieder war es ein Schock.

Aber was würden die Hufflepuffs in mir finden, der nie in ihr Haus gehörte? Saure Worte, schneidenden Witz, Verachtung für ihre Unfähigkeit, mit mir mitzuhalten?}

Jetzt waren es die Gedanken des Hutes, die langsam, zögernd waren.

\textbf{\emph{"Ich muss zum Wohle aller Schüler in allen Häusern sortieren…aber ich denke, du könntest lernen, ein guter Hufflepuff zu sein, und wärst dort nicht fehl am Platz.

Du wirst in Hufflepuff glücklicher sein als in jedem anderen Haus; das ist die Wahrheit."}}

\emph{Glücklichsein ist für mich nicht das Wichtigste auf der Welt.

Ich würde in Hufflepuff nicht das werden, was ich sein könnte. Ich würde mein Potenzial opfern.}

Der Hut zuckte; Harry konnte es irgendwie spüren.

Es war, als hätte er dem Hut in die Eier getreten - in eine stark gewichtete Komponente seiner Nutzenfunktion.

\emph{Warum versuchst du mich dorthin zu schicken, wo ich nicht hingehöre?}

Der Gedanke des Hutes war fast ein Flüstern.

\textbf{\emph{"Ich kann dir nicht von den anderen erzählen - aber glaubst du, dass du der erste potenzielle Dunkle Lord bist, der unter meiner Krempe durchläuft? Ich kann die einzelnen Fälle nicht kennen, aber ich kann dies wissen: Von denen, die nicht von Anfang an Böses vorhatten, haben einige auf meine Warnungen gehört und sind in Häuser gegangen, wo sie Glück finden würden.

Und einige von ihnen… einige von ihnen taten es nicht."}}

Das hat Harry aufgehalten. Aber nicht für lange.

\emph{Und von denen, die die Warnung nicht beherzigt haben, sind die alle Dunkle Lords geworden? Oder haben auch einige von ihnen Größe für das Gute erlangt? Wie sind hier die genauen Prozentzahlen?}

\textbf{\emph{"Ich kann dir keine genauen Zahlen nennen.

Ich kann sie nicht kennen, also kann ich sie auch nicht zählen. Ich weiß nur, dass sich deine Chancen nicht gut anfühlen.

Sie fühlen sich überhaupt nicht gut an."}}

\emph{Aber das würde ich einfach nicht tun! Niemals!}

\textbf{„\emph{Ich weiß, dass ich diese Behauptung schon einmal gehört habe.}}“

\emph{Ich habe kein Potenzial ein dunkler Lord zu werden!}

\textbf{„\emph{Doch, das hast du. Das hast du wirklich, wirklich sehr.}}“

\emph{Und warum? Nur weil ich mal dachte, es wäre cool, eine Legion von gehirngewaschenen Anhängern zu haben, die 'Heil dem Dunklen Lord Harry' skandieren?}

\textbf{\emph{"Amüsant, aber das war nicht dein erster flüchtiger Gedanke, bevor du etwas Sichereres, weniger Schädliches gedacht hast.

Nein, was dir einfiel, war, wie du überlegt hast, alle Blutpuristen aufzustellen und zu guillotinieren.}}

\textbf{\emph{\hfill\break Und jetzt sagst du dir, dass du es nicht ernst gemeint hast, aber das hast du. Wenn du es in diesem Moment tun könntest und niemand würde es je erfahren, würdest du es tun.}}

\textbf{\emph{\hfill\break Oder was du heute Morgen mit Neville Longbottom gemacht hast, tief in dir drin wusstest du, dass das falsch war, aber du hast es trotzdem getan, weil es Spaß gemacht hat und du eine gute Ausrede hattest und du dachtest, der Junge-der-lebte könnte damit durchkommen -"}}

\emph{Das ist unfair!

} \emph{Jetzt ziehst du nur innere Ängste hoch, die nicht unbedingt real sind! Ich machte mir Sorgen, dass ich so denken könnte, aber am Ende entschied ich, dass es wahrscheinlich funktionieren würde, um Neville zu helfen—}

\textbf{\emph{"Das war in der Tat eine Rationalisierung.

Ich weiß es. Ich kann nicht wissen, was das wahre Ergebnis für Neville sein wird - aber ich weiß, was wirklich in Ihrem Kopf vorging.

Der entscheidende Punkt war, dass es eine so clevere Idee war, dass du es nicht ertragen konntest, es nicht zu tun, ganz zu schweigen von Nevilles Terror."}}

Das war wie ein harter Schlag gegen Harrys ganzes Ich.

Er wich zurück, rappelte sich auf: \emph{Dann werde ich das nicht wieder tun! Ich werde besonders aufpassen, dass ich nicht böse werde!}

\textbf{„\emph{Hab ich schon mal gehört.}}“

Frustration baute sich in Harry auf. Er war es nicht gewohnt, in Argumenten unterlegen zu sein, schon gar nicht von einem Hut, der sich all sein Wissen und seine Intelligenz ausleihen konnte, um mit ihm zu argumentieren, und der seine Gedanken beobachten konnte, während sie sich formten.

\emph{Aus welcher Art von statistischer Zusammenfassung stammen eigentlich deine „Gefühle“? Berücksichtigst du, dass ich aus einer aufklärerischen Kultur stamme, oder waren diese anderen potentiellen Dunklen Lords die Kinder verwöhnter Mittelalter-Adliger, die nichts von den historischen Lehren wussten, wie Lenin und Hitler tatsächlich ausfielen, oder von der Evolutionspsychologie der Selbsttäuschung, oder vom Wert des Selbstbewusstseins und der Rationalität, oder—}

\textbf{\emph{"Nein, natürlich waren sie nicht in dieser neuen Referenzklasse, die du gerade so konstruiert hast, dass sie nur dich selbst enthält.

Und natürlich haben sich andere auf ihren eigenen Exzeptionalismus berufen, so wie du es jetzt tust.}}

\textbf{\emph{\hfill\break Aber warum ist das notwendig? Glaubst du, dass du der letzte potenzielle Zauberer des Lichts auf der Welt bist? Warum musst du derjenige sein, der nach Größe strebt, obwohl ich dich darauf hingewiesen habe, dass du riskanter als der Durchschnitt bist? Lass es einen anderen, sichereren Kandidaten versuchen!"}}

\emph{Aber die Prophezeiung.}

..

\textbf{\emph{"Du weißt nicht wirklich, dass es eine Prophezeiung gibt. Es war ursprünglich eine wilde Vermutung deinerseits, oder genauer gesagt, ein Scherz, und McGonagall könnte nur auf den Teil reagiert haben, dass der Dunkle Lord noch am Leben ist.

Du hast im Grunde keine Ahnung, was die Prophezeiung besagt oder ob es überhaupt eine gibt. Du spekulierst nur, oder um es genauer zu sagen, du wünschst dir eine fertige Heldenrolle, die dein persönliches Eigentum ist."}}

\emph{\hfill\break Aber selbst wenn es keine Prophezeiung gibt, bin ich derjenige, der ihn beim letzten Mal besiegt hat.

}\strut

\textbf{\emph{"Das war mit ziemlicher Sicherheit ein wilder Zufall, es sei denn, du glaubst ernsthaft, dass ein einjähriges Kind eine angeborene Neigung hat, Dunkle Lords zu besiegen, die auch zehn Jahre später noch vorhanden ist.

}} \textbf{\emph{Nichts davon ist der wahre Grund, und das weißt du!"}}

Die Antwort darauf war etwas, das Harry nicht regelmäßig laut gesagt hätte, im Gespräch hätte er darum herumgetanzt und einige sozial verträglichere Argumente zum gleichen Schluss gefunden—

\textbf{„\emph{Du glaubst, dass du potenziell der Größte bist, der je gelebt hat, der stärkste Diener des Lichts, dass kein anderer deinen Zauberstab nehmen wird, wenn du ihn niederlegst.}}“

\emph{Nun… ja, ehrlich gesagt. Normalerweise sage ich das nicht so offen, aber ja. Es hat keinen Sinn, es abzumildern, Du kannst sowieso meine Gedanken lesen.}

\textbf{„\emph{In dem Maße, wie du das wirklich glaubst… musst du auch glauben, dass du der schrecklichste Dunkle Lord sein könntest, den die Welt je gesehen hat.}}“

\emph{Zerstörung ist immer einfacher als Erschaffung.

Es ist leichter, Dinge auseinanderzureißen, zu zerstören, als sie wieder zusammenzusetzen. Wenn ich das Potenzial habe, Gutes zu vollbringen, muss ich auch das Potenzial haben, noch größeres Böses zu vollbringen.

.. Aber das werde ich nicht tun.}

\textbf{„\emph{Und schon bestehst du darauf, es zu riskieren! Warum bist du so getrieben? Was ist der wahre Grund, dass du nicht nach Hufflepuff gehen darfst und dort glücklicher bist? Was ist deine wahre Angst?}}“

\emph{Ich muss mein volles Potenzial erreichen.

Wenn ich das nicht tue… versage ich…}

\textbf{„\emph{Was passiert, wenn du versagst?}}“

\emph{Etwas Schreckliches..}

.

\textbf{„\emph{Was passiert, wenn du versagst?}}“

\emph{Ich weiß es nicht.}

\textbf{„\emph{Dann sollte es nicht beängstigend sein. Was passiert, wenn du versagst?}}“

\emph{ICH WEISS ES NICHT!

ABER ICH WEISS, DASS ES SCHLIMM IST!}

Einen Moment lang herrschte Stille in den Höhlen von Harrys Verstand.

\textbf{„\emph{Weißt du - du lässt es dich nicht denken, aber in irgendeiner stillen Ecke deines Verstandes weißt du genau, was du nicht denkst - du weißt, dass die bei weitem einfachste Erklärung für deine unverbalisierbare Angst nur die Angst ist, deine Fantasie von Größe zu verlieren, die Menschen zu enttäuschen, die an dich glauben, sich als ziemlich gewöhnlich zu erweisen, aufzublitzen und zu verblassen wie so viele andere Wunderkinder…}}“

\emph{Nein}, dachte Harry verzweifelt, \emph{nein, es ist etwas mehr, es kommt von irgendwo anders her, ich weiß, dass es da draußen etwas gibt, wovor ich Angst haben muss, eine Katastrophe, die ich aufhalten muss …}

\textbf{„\emph{Wie kannst du nur von so etwas wissen?}}“

Harry schrie es mit der vollen Kraft seines Geistes: \emph{NEIN, UND DAS IST ENDGÜLTIG!}

\textbf{\emph{Dann kam langsam die Stimme des Sprechenden Hutes:

"Du wirst also riskieren, ein Dunkler Lord zu werden, weil die Alternative für dich ein sicheres Scheitern ist, und dieses Scheitern bedeutet den Verlust von allem.

Daran glaubst du im Grunde deines Herzens. Du kennst alle Gründe, an diesem Glauben zu zweifeln, und sie haben dich nicht bewegt."}}

\emph{Ja. Und auch wenn der Wechsel nach Ravenclaw die Kälte stärkt, heißt das nicht, dass die Kälte am Ende siegen wird.}

\textbf{\emph{"Dieser Tag ist eine große Weggabelung in deinem Schicksal. Sei dir nicht so sicher, dass es noch andere Möglichkeiten gibt, außer dieser einen.

Es ist kein Straßenschild aufgestellt, das den Ort deiner letzten Chance zur Umkehr markiert. Wenn du eine Chance ablehnst, wirst du dann nicht auch andere ablehnen? Es kann sein, dass dein Schicksal bereits besiegelt ist, sogar indem du diese eine Sache tust."}}

\emph{Aber das ist nicht sicher.}

\textbf{„\emph{Dass du es nicht mit Sicherheit weißt, mag nur deine eigene Unwissenheit widerspiegeln.}}“

\emph{Aber auch das ist nicht sicher.}

Der Hut seufzte einen schrecklich traurigen Seufzer.

\textbf{„\emph{Und so wirst du in der nächsten Warnung, die ich ausspreche, bald nur noch eine Erinnerung sein, gefühlt und nie gekannt…}}“

\emph{Wenn es dir so vorkommt, warum bringst du mich dann nicht einfach dahin, wo du mich haben willst?}

Der Gedanke des Hutes war von Sorge durchdrungen.

\textbf{„\emph{Ich kann dich nur dahin bringen, wo du hingehörst. Und nur deine eigenen Entscheidungen können ändern, wo du hingehörst.}}“

\emph{Dann ist das erledigt. Schick mich nach Ravenclaw, wo ich hingehöre, zu den anderen meiner Art.}

\textbf{„\emph{Ich nehme nicht an, dass du Gryffindor in Betracht ziehen würdest? Es ist das prestigeträchtigste Haus - die Leute erwarten es wahrscheinlich sogar von dir - sie werden ein wenig enttäuscht sein, wenn du nicht hingehst - und deine neuen Freunde, die Weasley-Zwillinge, sind dort—} }“

Harry kicherte, oder spürte den Impuls dazu; es kam als rein mentales Lachen heraus, ein seltsames Gefühl.

Offenbar gab es Sicherheitsvorkehrungen, die verhinderten, dass man versehentlich etwas laut sagte, während man unter dem Hut war und über Dinge sprach, die man für den Rest seines Lebens keiner anderen Seele erzählen würde.

Nach einem Moment hörte Harry den Hut auch lachen, ein seltsames, trauriges, tuchartiges Geräusch.

(Und in der Halle dahinter eine Stille, die zuerst flacher wurde, als das Flüstern im Hintergrund zunahm, und sich dann vertiefte, als das Flüstern aufhörte und abebbte, um schließlich in eine völlige Stille zu fallen, die niemand mit einem einzigen Wort zu stören wagte, als Harry lange, lange Minuten unter dem Hut blieb, länger als alle vorherigen Erstklässler zusammen, länger als irgendjemand seit Menschengedenken.

Am Kopftisch lächelte Dumbledore weiterhin freundlich; kleine metallische Geräusche kamen gelegentlich aus Snapes Richtung, während er müßig die verdrehten Überreste dessen, was einmal ein schwerer silberner Weinkelch gewesen war, zusammenpresste; und Minerva McGonagall umklammerte das Podium mit festem Griff, weil sie wusste, dass Harry Potters ansteckendes Chaos irgendwie den Sprechenden Hut selbst infiziert hatte und der Hut im Begriff war, zu verlangen, dass ein ganz neues Haus des Verderbens geschaffen werden sollte, nur um Harry Potter unterzubringen oder so, und Dumbledore würde sie dazu zwingen…)

Unter der Krempe des Hutes erstarb das leise Lachen. Aus irgendeinem Grund fühlte sich Harry auch traurig.

\emph{Nein, nicht Gryffindor. Professor McGonagall hat gesagt, wenn 'der, der die Sortierung vorgenommen hat', versucht, mich nach Gryffindor zu drängen, soll ich dich daran erinnern, dass sie vielleicht eines Tages Schulleiterin sein wird, und dann hätte sie die Autorität, dich in Brand zu setzen.}

\textbf{„\emph{Sag ihr, ich nannte sie eine unverschämte Jugendliche und sagte ihr, sie soll von meinem Rasen verschwinden.}}“

\emph{Das werde ich. Also war das deine seltsamste Unterhaltung überhaupt?}

\textbf{„\emph{Nicht mal annähernd.}}“

Die telepathische Stimme vom Hut wurde schwer.

\textbf{„\emph{Nun, ich habe dir jede mögliche Chance gegeben, eine andere Entscheidung zu treffen. Jetzt ist es Zeit für dich, dorthin zu gehen, wo du hingehörst, zu den anderen deiner Art.}}“

Es gab eine Pause, die sich in die Länge zog.

\emph{Worauf wartest du noch?}

\textbf{\emph{"Ich hatte eigentlich auf einen Moment der entsetzten Erkenntnis gehofft.

Selbsterkenntnis scheint meinen Sinn für Humor zu verbessern."}}

\emph{Hm}? Harry warf seine Gedanken zurück und versuchte herauszufinden, wovon der Hut wohl reden könnte - und dann, plötzlich, wurde es ihm klar.

Er konnte nicht glauben, dass er es bis zu diesem Punkt geschafft hatte, es zu übersehen.

\emph{Du meinst meine entsetzte Erkenntnis, dass du aufhören wirst, bei Bewusstsein zu sein, sobald du mit der Sortierung fertig bist—}

Irgendwie, auf eine Art und Weise, die Harry überhaupt nicht verstand, bekam er den nonverbalen Eindruck eines Hutes, der seinen Kopf gegen die Wand schlägt.

\textbf{\emph{\hfill\break "Ich gebe auf. Du bist zu schwer von Begriff, als dass das lustig sein könnte. So geblendet von}} \textbf{\emph{deinen eigenen Annahmen, dass du genauso gut ein Stein sein könntest.

Ich schätze, ich muss es einfach offen aussprechen."}}

\textbf{Zu langsam—}

\textbf{\emph{"Oh, und du hast ganz vergessen, die Geheimnisse der verlorenen Magie zu verlangen, die mich erschaffen hat.

Und dabei waren es so wunderbare, wichtige Geheimnisse."}}

\emph{Du schlauer kleiner Bastard—}

\textbf{„\emph{Du hast es verdient, und das hier auch.}}“

Harry sah es kommen, als es bereits zu spät war.

Die erschrockene Stille in der Halle wurde von einem einzigen Wort durchbrochen.

\textbf{„SLYTHERIN!“}

Einige Schüler schrien, so groß war die aufgestaute Spannung.

Die Leute erschraken so sehr, dass sie von ihren Bänken fielen. Hagrid keuchte entsetzt auf, McGonagall taumelte auf dem Podium, und Snape ließ die Reste seines schweren silbernen Kelches direkt auf seine Leiste fallen.

Harry saß wie erstarrt da, sein Leben lag in Trümmern, er fühlte sich wie ein absoluter Narr und wünschte sich, er hätte aus anderen Gründen als denen, die er hatte, eine andere Wahl getroffen.

Dass er etwas, irgendetwas anders gemacht hätte, bevor es zu spät gewesen wäre, um umzukehren. Als der erste Schock nachließ und die Leute auf die Nachricht zu reagieren begannen, sprach der Sprechende Hut wieder:

\textbf{"War nur ein Scherz!}

\textbf{RAVENCLAW!"}

