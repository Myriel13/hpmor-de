

\hypertarget{koordinationsprobleme-teil-2}{% \section{34. Koordinationsprobleme, Teil 2}\label{koordinationsprobleme-teil-2}}

\textbf{\uline{Koordinationsprobleme, Teil 2}}

Minerva und Dumbledore hatten ihr gemeinsames Talent eingesetzt, um die große Bühne zu zaubern, auf die Quirrell nun langsam zustapfte; sie bestand im Kern aus stabilem Holz, aber die Außenflächen glänzten mit glitzerndem Marmor, der mit Platin eingelegt und mit Edelsteinen in allen Hausfarben besetzt war. Weder sie noch der Schulleiter war ein Gründer von Hogwarts, aber die Beschwörung musste nur ein paar Stunden halten.

Normalerweise genoss Minerva die wenigen Gelegenheiten, bei denen sie sich an großen Verwandlungen abmühen konnte; sie hätte die vielen kleinen Gelegenheiten zur Kunstfertigkeit und die Illusion von Opulenz genießen sollen; aber diesmal hatte sie die Arbeit mit dem schrecklichen Gefühl erledigt, ihr eigenes Grab zu schaufeln.

Aber Minerva fühlte sich jetzt ein wenig besser. Es hatte einen kurzen Moment gegeben, in dem die Explosion hätte kommen können; aber Dumbledore war bereits aufgestanden und hatte herzlich applaudiert, und niemand hatte sich als dumm genug erwiesen, vor dem Schulleiter zu randalieren.

Und die explosive Stimmung war schnell einer kollektiven Stimmung gewichen, die man vielleicht mit dem Satz beschreiben könnte: \emph{Lasst uns in Ruhe!}

\emph{Blaise Zabini hatte sich im Namen von Sunshine erschossen, und der Endstand hatte 254 zu 254 zu 254 gelautet.}

Hinter der Bühne, die darauf wartete, nach oben zu kommen, starrten sich drei Kinder in einer Mischung aus Wut und Frustration an.

Es half nicht, dass sie noch feucht waren, weil sie aus dem See gefischt worden waren, und dass die wärmenden Zauber nicht ganz ausreichten, um die knackige Dezemberluft auszugleichen, oder vielleicht war es nur ihre Stimmung.

"Das war's", sagte Granger. "Mir reicht's! Keine Verräter mehr!"

"Ich stimme Ihnen vollkommen zu, Miss Granger", sagte Draco eisig. "Genug ist genug."

"Und was habt ihr zwei vor, dagegen zu tun?", schnauzte Harry Potter. "Professor Quirrell hat bereits gesagt, dass er Spione nicht verbieten wird!"

"Wir werden sie für ihn verbieten", sagte Draco grimmig.

Er hatte noch nicht einmal verstanden, was er mit diesen Worten meinte, als er sie aussprach, aber der bloße Akt des Sprechens schien einen Plan herauszukristallisieren -

Die Bühne war wirklich gut gemacht, zumindest für ein provisorisches Bauwerk; die Macher waren nicht in die übliche Falle getappt, von ihrer eigenen Illusion des Reichtums beeindruckt zu sein, und wussten etwas über Architektur und visuellen Stil.

Von dort aus, wo Draco stand, an dem für ihn offensichtlichen Platz, sahen die zuschauenden Schüler, wie er von dem schwachen Glitzern der Smaragde umhüllt wurde; und Granger, die dort stand, wohin Draco sie dezent wies, wurde von dem Saphir von Ravenclaw umhüllt. Was Harry Potter betraf, so sah Draco ihn im Moment nicht an.

Professor Quirrell war… erwacht, oder was auch immer er tat, und lehnte an einem Platinpodest, das von allen Edelsteinen befreit war.

Mit offensichtlicher Kunstfertigkeit stapelte der Verteidigungsprofessor sorgfältig die drei Umschläge, die die drei Pergamente enthielten, auf die die drei Generäle ihre Wünsche geschrieben hatten, während alle Schüler von Hogwarts zusahen und warteten. Schließlich blickte Professor Quirrell von den Umschlägen auf.

"Nun", sagte der Verteidigungsprofessor. "Das ist unangenehm."

Ein leichtes Lachen ging durch die Menge, mit einem scharfen Unterton.

"Ich nehme an, Sie fragen sich alle, was ich tun werde?", sagte Professor Quirrell.

"Da gibt es nichts zu überlegen; ich werde tun müssen, was fair ist. Allerdings wollte ich vorher noch eine kleine Rede halten, und es scheint mir, dass Mr. Malfoy und Miss Granger etwas zu sagen haben."

Draco blinzelte, und dann tauschten er und Granger schnelle Blicke aus - darf ich? - ja, nur zu - und Draco erhob seine Stimme.

"General Granger und ich möchten beide sagen", sagte Draco mit seiner förmlichsten Stimme, wohl wissend, dass sie verstärkt und gehört wurde, "dass wir nicht länger die Hilfe von Verrätern akzeptieren werden. Und wenn wir in einer Schlacht feststellen, dass Potter Verräter aus einer unserer Armeen aufgenommen hat, werden wir uns zusammentun, um ihn zu vernichten."

Und Draco warf dem Jungen-der-lebte einen bösartigen Blick zu.

\emph{Nimm das, General Chaos!}

"Ich stimme General Malfoy vollkommen zu", sagte Granger, die neben ihm stand, mit ihrer hohen, klaren und starken Stimme. "Keiner von uns wird Verräter einsetzen, und wenn General Potter es doch tut, werden wir ihn vom Schlachtfeld fegen."

Es gab einen Aufschrei der Überraschung von den zuschauenden Schülern.

"Sehr gut", sagte ihr Verteidigungsprofessor und lächelte. "Ihr habt lange genug gebraucht, aber man muss euch trotzdem dazu gratulieren, dass ihr vor allen anderen Generälen darauf gekommen seid."

Es dauerte einen Moment, bis Sie das verinnerlicht hatten.

"In Zukunft, Mr. Malfoy, Miss Granger, bevor Sie mit einer Bitte in mein Büro kommen, überlegen Sie, ob es einen Weg gibt, sie ohne meine Hilfe zu erfüllen.

Bei dieser Gelegenheit werde ich Quirrell keine Punkte abziehen, aber beim nächsten Mal können Sie damit rechnen, die vollen fünfzig zu verlieren."

Professor Quirrell setzte ein amüsiertes Grinsen auf.

"Und was haben Sie dazu zu sagen, Mr. Potter?"

Harry Potters Blick ging zu Granger, dann zu Draco. Sein Gesicht wirkte ruhig; obwohl Draco sicher war, dass kontrolliert der bessere Ausdruck gewesen wäre.

Schließlich sprach Harry Potter, seine Stimme war ruhig.

"Die Chaos Legion nimmt immer noch gerne Verräter auf. Wir sehen uns auf dem Schlachtfeld."

Draco wusste, dass sich der Schock auf seinem eigenen Gesicht abzeichnete; es gab erstauntes Gemurmel von den zuschauenden Schülern, und als Draco einen Blick in die erste Reihe warf, sah er, dass sogar Harrys Chaoten verblüfft aussahen.

Grangers Gesicht war wütend, und es wurde noch wütender.

"Mr. Potter", sagte sie in einem scharfen Ton, als würde sie sich für eine Lehrerin halten, "versuchen Sie, unausstehlich zu sein?"

"Ganz bestimmt nicht", sagte Harry Potter ruhig. "Ich werde dich nicht jedes Mal dazu zwingen, es zu tun. Schlagt mich einmal, und ich bleibe geschlagen.

Aber Drohungen sind nicht immer genug, General Sonnenschein. Sie haben mich nicht gebeten, sich mit Ihnen zu verbünden, sondern haben einfach versucht, Ihren Willen durchzusetzen; und manchmal muss man den Feind tatsächlich besiegen, um ihm seinen Willen aufzuzwingen. Sehen Sie, ich bin skeptisch, dass Hermine Granger, der hellste akademische Stern von Hogwarts, und Draco, Sohn von Lucius, Spross des edlen und uralten Hauses Malfoy, zusammenarbeiten können, um ihren gemeinsamen Feind, Harry Potter, zu besiegen."

Ein amüsiertes Lächeln ging über Harry Potters Gesicht.

"Vielleicht mache ich einfach das, was Draco mit Zabini versucht hat, und schreibe einen Brief an Lucius Malfoy und sehe, was er davon hält."

"Harry!?", keuchte Granger und sah absolut entsetzt aus, und auch aus dem Publikum kam ein Keuchen.

Draco kontrollierte die Wut, die ihn durchströmte. Das war ein dummer Schachzug von Harry gewesen, das in der Öffentlichkeit zu sagen. Wenn Harry es einfach getan hätte, hätte es vielleicht funktioniert, daran hatte Draco gar nicht gedacht, aber wenn Vater das jetzt tat, würde es so aussehen, als würde er Harry in die Hände spielen -

"Wenn du glaubst, dass mein Vater, Lord Malfoy, so einfach von dir manipuliert werden kann", sagte Draco kalt, "dann steht dir eine Überraschung bevor, Harry Potter."

Und Draco erkannte, als die Worte seinen Mund verließen, dass er seinen eigenen Vater gerade in die Ecke gedrängt hatte, mehr oder weniger ohne es zu wollen.

Vater würde das wahrscheinlich nicht gefallen, nicht das kleinste bisschen, aber jetzt würde er es unmöglich sagen können… Draco würde sich dafür entschuldigen müssen, es war ein ehrliches Versehen gewesen, aber es war seltsam, dass er es überhaupt getan hatte.

"Dann geh doch und besiege den bösen General Chaos", sagte Harry und sah immer noch amüsiert aus. "Ich kann nicht gegen eure beiden Armeen gewinnen - nicht, wenn ihr wirklich zusammenarbeitet. Aber ich frage mich, ob ich euch vielleicht vorher auseinanderbringen kann."

"Das wirst du nicht, und wir werden dich zerquetschen!", sagte Draco Malfoy.

Und neben ihm nickte Hermine Granger entschlossen.

"Nun", sagte Professor Quirrell, nachdem sich das erstaunte Schweigen eine Weile hingezogen hatte. "So hatte ich mir den Verlauf dieser speziellen Unterhaltung nicht vorgestellt."

Der Verteidigungsprofessor hatte einen ziemlich verblüfften Gesichtsausdruck.

"Um ehrlich zu sein, Mr. Potter, habe ich erwartet, dass Sie sofort und mit einem Lächeln einräumen und dann verkünden, dass Sie meine beabsichtigte Lektion längst durchschaut haben, aber beschlossen haben, sie nicht für andere zu verderben.

In der Tat, ich habe meine Rede entsprechend geplant, Mr. Potter."

Harry Potter zuckte nur mit den Schultern.

"Das tut mir leid", sagte er und sagte nichts weiter.

"Oh, keine Sorge", sagte Professor Quirrell. "Ich kann improvisieren."

Und Professor Quirrell wandte sich von den drei Kindern ab und richtete sich am Podium auf, um sich an die ganze zuschauende Menge zu wenden; seine gewohnte Miene der distanzierten Belustigung fiel ab wie eine fallende Maske, und als er wieder sprach, war seine Stimme noch lauter als zuvor.

"\textbf{Wenn Harry Potter nicht gewesen wäre}", sagte Professor Quirrell, seine Stimme so klar und kalt wie im Dezember, "\textbf{hätte Du-Weißt-Schon-Wer gewonnen.}"

\emph{Die Stille war augenblicklich und vollkommen.}

"Macht keinen Fehler", sagte Professor Quirrell. "Der Dunkle Lord war am Gewinnen. Es gab immer weniger Auroren, die es wagten, sich ihm entgegenzustellen, die Wächter, die sich ihm entgegenstellten, wurden zur Strecke gebracht. Ein Dunkler Lord und vielleicht fünfzig Todesser gewannen gegen ein Land

von Tausenden. \textbf{\emph{Das ist mehr als lächerlich!}} \textbf{\emph{Es gibt für mich keine Noten, die niedrig genug sind, um diese Inkompetenz zu benoten!}}"

Auf dem Gesicht von Schulleiter Dumbledore lag ein Stirnrunzeln, auf den Gesichtern der Zuhörer Verwunderung, und es herrschte völliges Schweigen.

"Versteht ihr jetzt, wie es dazu kam? Ihr habt es heute gesehen. Ich habe Verräter zugelassen und den Generälen keine Möglichkeit gegeben, sie zurückzuhalten.

Ihr habt das Ergebnis gesehen. Geschickte Intrigen und geschickter Verrat, bis sich der letzte Soldat auf dem Schlachtfeld erschoss! Sie können unmöglich bezweifeln, dass alle drei dieser Armeen von einem äußeren Feind hätten besiegt werden können, wenn sie in sich selbst geeint gewesen wären."

Professor Quirrell lehnte sich am Podium nach vorne, seine Stimme war nun von einer grimmigen Intensität erfüllt. Seine rechte Hand war ausgestreckt, die Finger geöffnet und gespreizt.

"Spaltung ist Schwäche", sagte der Verteidigungsprofessor.

Seine Hand schloss sich zu einer festen Faust.

"Einigkeit ist Stärke.

Der Dunkle Lord verstand das sehr gut, ungeachtet seiner anderen Torheiten; und er nutzte dieses Verständnis, um die eine einfache Erfindung zu machen, die ihn zum schrecklichsten Dunklen Lord der Geschichte machte. Eure Eltern hatten es mit \emph{einem} Dunklen Lord zu tun. Und 50 Todessern, die sich vollkommen einig waren, weil sie wussten, dass jeder Bruch ihrer Loyalität mit dem Tod bestraft werden würde, dass jede Nachlässigkeit oder Inkompetenz mit Schmerzen bestraft werden würde. Keiner konnte sich dem Griff des Dunklen Lords entziehen, sobald sie sein Zeichen trugen. Und die Todesser stimmten zu, dieses furchtbare Zeichen zu nehmen, weil sie wussten, dass sie, sobald sie es genommen hatten, vereint einem geteilten Land gegenüberstehen würden. \emph{Ein Dunkler Lord und fünfzig Todesser hätten ein ganzes Land besiegt, durch die Macht des Dunklen Mal}."

Professor Quirrells Stimme war düster und hart.

"Eure Eltern hätten mit gleicher Münze zurückschlagen können. Haben sie aber nicht. Es gab einen Mann namens Yermy Wibble, der die Nation dazu aufrief, eine Wehrpflicht einzuführen, obwohl er nicht genug Weitblick hatte, um ein Mal von Britannien vorzuschlagen. Yermy Wibble wusste, was mit ihm geschehen würde; er hoffte, sein Tod würde andere inspirieren. Also nahm der Dunkle Lord seine Familie als Vorsichtsmaßnahme mit. Ihre leeren Häute lösten nichts als Furcht aus, und niemand wagte es, noch einmal zu sprechen. Und eure Eltern hätten die Konsequenzen ihrer verachtenswerten Feigheit zu tragen gehabt, wenn sie nicht von einem einjährigen Jungen gerettet worden wären."

Professor Quirrells Gesicht zeigte volle Verachtung.

"Ein Dramatiker hätte das eine deus ex machina genannt, denn sie haben nichts getan, um ihre Rettung zu verdienen. Er, der nicht genannt werden darf, hat es vielleicht nicht verdient zu gewinnen, aber keine Frage, \textbf{\emph{eure Eltern haben es verdient zu verlieren.}}"

Die Stimme des Verteidigungsprofessors ertönte wie Eisen.

"Und wisst dies: \textbf{\emph{Eure Eltern haben nichts gelernt}}! Die Nation ist immer noch zersplittert und schwach! Wie wenige Jahrzehnte sind zwischen Grindelwald und Du-weißt-schon-wem vergangen? Glaubt ihr, dass ihr die nächste Bedrohung nicht noch zu euren Lebzeiten erleben werdet? Wollt ihr dann die Dummheiten eurer Eltern wiederholen, wenn ihr die Ergebnisse heute so deutlich vor euch seht? Denn ich kann euch sagen, was eure Eltern tun werden, wenn der Tag der Finsternis kommt!

Ich kann Euch sagen, welche Lektion sie gelernt haben! S\textbf{\emph{ie haben gelernt, sich wie Feiglinge zu verstecken und nichts zu tun, während sie auf Harry Potter warten, um sie zu retten!}}"

In den Augen von Schulleiter Dumbledore lag ein verwunderter Blick; und auch die anderen Schüler sahen ihren Verteidigungsprofessor mit Fassungslosigkeit, Zorn und Ehrfurcht an. Professor Quirrells Augen waren jetzt genauso kalt wie seine Stimme.

"Merkt euch das, und merkt es euch gut. Er, der nicht genannt werden darf, wollte über dieses Land herrschen, um es für immer in seinem grausamen Griff zu halten.

Aber wenigstens wollte er über ein lebendiges Land herrschen und nicht über einen Haufen Asche! Es gab dunkle Lords, die verrückt waren und die Welt nur in einen riesigen Scheiterhaufen verwandeln wollten! Es hat Kriege gegeben, in denen ein ganzes Land gegen ein anderes marschierte! Eure Eltern hätten fast gegen ein halbes Hundert verloren, die dieses Land lebendig einnehmen wollten!

Wie schnell wären sie von einem Feind zermalmt worden, der zahlreicher war als sie, einem Feind, der sich um nichts anderes kümmerte als um ihre Vernichtung!

Das prophezeie ich: Wenn sich die nächste Bedrohung erhebt, wird Lucius Malfoy behaupten, dass ihr ihm folgen müsst oder zugrunde geht, dass eure einzige Hoffnung darin besteht, auf seine Grausamkeit und Stärke zu vertrauen.

Und obwohl Lucius Malfoy selbst daran glauben wird, wird das eine Lüge sein. Denn als der Dunkle Lord unterging, hat Lucius Malfoy die Todesser nicht geeint, sie wurden in einem Augenblick zerrissen, sie flohen wie geprügelte Hunde und verrieten sich gegenseitig! Lucius Malfoy ist nicht stark genug, um ein wahrer Lord zu sein, ob dunkel oder nicht."

Draco Malfoys Fäuste waren weiß geballt, in seinen Augen standen Tränen und Wut und eine unerträgliche Scham.

"Nein", sagte Professor Quirrell, "ich glaube nicht, dass es Lucius Malfoy sein wird, der euch rettet. Und damit ihr nicht denkt, dass ich in meinem eigenen Namen spreche, wird die Zeit noch früh genug deutlich machen, dass dem nicht so ist.

Ich gebe euch keine Empfehlung, meine Schüler. Aber ich sage, wenn ein ganzes Land einen Anführer finden würde, der so stark ist wie der Dunkle Lord, aber ehrenhaft und rein, und sein Zeichen annehmen würde, dann könnten sie jeden Dunklen Lord wie ein Insekt zerquetschen, und der ganze Rest unserer geteilten magischen Welt könnte sie nicht bedrohen. Und wenn sich ein noch größerer Feind in einem Vernichtungskrieg gegen uns erheben würde, dann könnte nur eine geeinte magische Welt überleben."

Es gab ein Aufatmen, vor allem von Muggelgeborenen; die Schüler in den grüngeschmückten Roben schauten nur verwirrt. Jetzt war es Harry Potter, dessen Fäuste fest geballt und zitternd waren; und Hermine Granger neben ihm war wütend und bestürzt. Der Schulleiter erhob sich von seinem Platz, sein Gesicht war nun streng, er sagte noch kein Wort; aber der Befehl war klar.

"Ich weiß nicht, was für eine Drohung kommen wird", sagte Professor Quirrell.

"Aber ihr werdet nicht euer ganzes Leben lang in Frieden leben, nicht, wenn die vergangene Geschichte der Welt überhaupt ein Hinweis auf ihre Zukunft ist.

Und wenn ihr in Zukunft so handelt, wie ihr es heute bei drei Armeen gesehen habt, wenn ihr nicht in der Lage seid, euer kleinliches Gezänk beiseite zu werfen und das Zeichen eines einzigen Anführers anzunehmen, \textbf{\emph{dann werdet ihr euch in der Tat wünschen, dass der Dunkle Lord gelebt hätte, um über euch zu herrschen, und ihr werdet den Tag bereuen, an dem Harry Potter jemals geboren wurde}} -"

"\textbf{Genug}!", brüllte Albus Dumbledore.

Es herrschte Stille. Professor Quirrell drehte langsam den Kopf und blickte auf die Stelle, an der Albus Dumbledore in der Wut seiner Zauberei stand; ihre Blicke trafen sich, und eine lautlose Anspannung drückte wie ein Gewicht auf alle Schüler, die zuhörten und sich nicht zu bewegen wagten.

"Auch du hast dieses Land im Stich gelassen", sagte Professor Quirrell.

"Und du kennst die Gefahr genauso gut wie ich."

"Solche Reden sind nicht für die Ohren von Schülern", sagte Albus Dumbledore mit gefährlich ansteigender Stimme. "Und auch nicht für die Münder von Professoren!"

Trocken ergriff dann Professor Quirrell das Wort: "Es wurden viele Reden für die Ohren der Erwachsenen gehalten, als der Dunkle Lord sich erhob. Und die Erwachsenen klatschten und jubelten und gingen nach Hause, nachdem sie ihre Tagesunterhaltung genossen hatten. Aber ich werde Ihnen gehorchen, Schulleiter, und keine weiteren Reden halten, wenn sie Ihnen nicht gefallen. Meine Lektion ist einfach. Ich werde weiterhin nichts gegen Verräter unternehmen, und wir werden sehen, was die Schüler selbst dagegen tun können, wenn sie nicht auf Professoren warten, die sie retten."

Und dann wandte sich Professor Quirrell wieder seinen Schülern zu, und sein Mund verzog sich zu einem schiefen Grinsen, das den furchtbaren Druck zu zerstreuen schien wie ein Gott, der die Wolken vertreibt.

"Aber bitte seid nett zu den Verrätern", sagte Professor Quirrell.

"bis jetzt haben Sie sich nur amüsiert."

Es gab Gelächter, wenn auch zunächst nervös, und dann schien es sich zu steigern, als Professor Quirrell mit einem schiefen Lächeln dastand und sich etwas von der Anspannung löste.

Dracos Verstand wirbelte noch immer durch tausend Fragen und einen Anflug von Entsetzen, als Professor Quirrell sich anschickte, die Umschläge zu öffnen, in die die drei ihre Wünsche geschrieben hatten. Es war Draco nie zuvor in den Sinn gekommen, dass mondreisende Muggel eine größere Bedrohung darstellten als der langsame Niedergang der Zauberei oder dass Vater sich als zu schwach erwiesen hatte, sie aufzuhalten. Und noch seltsamer war die offensichtliche Andeutung: Professor Quirrell glaubte, dass Harry es könnte. Der Verteidigungsprofessor behauptete, keine Empfehlung ausgesprochen zu haben, aber er hatte Harry Potter in seiner Rede immer wieder erwähnt; andere würden bereits das Gleiche denken wie Draco. Es war lächerlich.

\emph{Der Junge, der einen ausgestopften Stuhl mit Glitzer überzogen und ihn einen Thron genannt hatte} - \emph{der Junge, der sich Snape entgegenstellte und gewann}, flüsterte eine verräterische Stimme, dieser Junge könnte zu einem Lord heranwachsen, der stark genug war, um zu herrschen, stark genug, um uns alle zu retten -

Harry war von Muggeln aufgezogen worden! Er war praktisch selbst ein Schlammblut, er würde nicht gegen seine Adoptivfamilie kämpfen - Er kennt ihre Künste, ihre Geheimnisse und ihre Methoden; er kann die ganze Wissenschaft der Muggel nehmen und sie gegen sie einsetzen, neben unserer eigenen Macht als Zauberer.

\emph{Aber was, wenn er sich weigert? Was, wenn er zu schwach ist?}

\emph{Dann}, sagte diese innere Stimme, \emph{wirst du es sein müssen, nicht wahr, Draco Malfoy?}

Und dann wurde es wieder still in der Menge, als Professor Quirrell den ersten Umschlag öffnete.

"Mr. Malfoy", sagte Professor Quirrell, "Ihr Wunsch ist, dass… dass Slytherin den Hauspokal gewinnt."

Die Zuschauer machten eine verwunderte Pause.

"Ja, Professor", sagte Draco mit klarer Stimme, wohl wissend, dass sie noch einmal verstärkt wurde. "Wenn Sie das nicht tun können, dann etwas anderes für Slytherin -"

"Ich werde keine Hauspunkte ungerecht vergeben", sagte Professor Quirrell.

Er tippte sich an die Wange und sah nachdenklich aus.

"Was Ihren Wunsch schwierig genug macht, um interessant zu sein. Möchten Sie etwas darüber sagen, warum, Mr. Malfoy?"

Draco wandte sich von dem Verteidigungsprofessor ab und blickte auf die Menge vor dem Hintergrund von Platin und Smaragden hinaus. Nicht ganz Slytherin hatte der Drachenarmee zugejubelt, es gab Anti-Malfoy-Fraktionen, die ihre Unzufriedenheit zum Ausdruck gebracht hatten, indem sie den Jungen-der-lebte oder sogar Granger unterstützt hatten; und diese Fraktionen würden durch das, was Zabini getan hatte, sehr ermutigt werden. Er musste sie daran erinnern, dass Slytherin für Malfoy stand und Malfoy für Slytherin -

"Nein", sagte Draco. "Sie sind Slytherins, sie werden es verstehen."

Es gab einige Lacher aus dem Publikum, besonders in Slytherin, sogar von einigen Schülern, die sich noch vor einem Moment als Anti-Malfoy bezeichnet hätten.

\emph{Schmeicheleien waren eine schöne Sache.} Draco drehte sich wieder zu Professor Quirrell um und war überrascht, einen verlegenen Blick auf Grangers Gesicht zu sehen.

"Und für Miss Granger…", sagte Professor Quirrell. Es gab das Geräusch eines zerrissenen Umschlags. "Ihr Wunsch ist, dass… dass Ravenclaw den Hauspokal gewinnt?"

Es gab beträchtliches Gelächter aus dem Publikum, darunter auch ein Glucksen von Draco. Er hatte nicht gedacht, dass Granger dieses Spiel spielte.

"Nun, ähm", sagte Granger und klang, als würde sie plötzlich über eine auswendig gelernte Rede stolpern, "ich wollte sagen, dass…"

Sie holte tief Luft.

"Es waren Soldaten aus jedem Haus in meiner Armee, und ich möchte keinen von ihnen geringschätzen. Aber Häuser sollten doch auch etwas zählen. Es war traurig, als Schüler desselben Hauses sich gegenseitig verhext haben, nur weil sie in verschiedenen Armeen waren. Die Leute sollten sich auf den verlassen können, der in ihrem Haus ist. Deshalb haben Godric Gryffindor und Salazar Slytherin und Rowena Ravenclaw und Helga Hufflepuff die vier Häuser von Hogwarts überhaupt erst geschaffen. Ich bin die Generalin von Sonnenschein, aber noch davor bin ich Hermine Granger aus Ravenclaw, und ich bin stolz darauf, einem Haus anzugehören, das achthundert Jahre alt ist."

"Gut gesagt, Miss Granger!", sagte Dumbledores dröhnende Stimme.

Harry Potter runzelte die Stirn, und etwas kitzelte am Rande von Dracos Erkenntnis.

"Ein interessanter Gedanke, Miss Granger", sagte Professor Quirrell.

"Aber es gibt Zeiten, in denen es für einen Slytherin gut ist, Freunde in Ravenclaw zu haben, oder für einen Gryffindor, Freunde in Hufflepuff zu haben. Sicherlich wäre es am besten, wenn man sich sowohl auf seine Freunde im Haus als auch auf seine Freunde in der Armee verlassen könnte?"

Grangers Augen huschten kurz zu den zuschauenden Schülern und Lehrern, und sie sagte nichts.

Professor Quirrell nickte wie zu sich selbst, dann wandte er sich wieder dem Podium zu, nahm den letzten Umschlag und riss ihn auf. Neben Draco spannte sich Harry Potter sichtlich an, als der Verteidigungsprofessor das Pergament hervorzog.

"Und Mr. Potter wünscht sich -"

Es gab eine Pause, als Professor Quirrell das Pergament betrachtete.

Dann, ohne dass sich der Gesichtsausdruck von Professor Quirrell änderte, \textbf{\emph{ging das Pergamentblatt in Flammen auf und verbrannte mit einem kurzen, intensiven Feuer,}} das nur noch schwarzen Staub aufwirbelte, der von seiner Hand herabrieselte.

"Bitte beschränken Sie sich auf das Mögliche, Mr. Potter", sagte Professor Quirrell und klang dabei sehr trocken.

Es gab eine lange Pause; Harry, der neben Draco stand, sah ziemlich erschüttert aus. \emph{Worum in Merlins Namen hatte er gebeten?}

"Ich hoffe", sagte Professor Quirrell, "dass Sie einen weiteren Wunsch vorbereitet haben, falls ich diesen nicht erfüllen kann."

Wieder gab es eine Pause.

Harry holte tief Luft. "Das habe ich nicht", sagte er, "aber ich habe mir schon einen anderen ausgedacht." Harry Potter drehte sich um, um das Publikum anzuschauen, und seine Stimme wurde fester, als er sprach.

"Die Menschen fürchten Verräter wegen des Schadens, den sie direkt anrichten, wegen der Soldaten, die sie erschießen oder wegen der Geheimnisse, die sie verraten.

Aber das ist nur ein Teil der Gefahr. Was die Leute tun, weil sie Angst vor Verrätern haben, kostet sie auch. Ich habe diese Strategie heute gegen Sonnenschein und Drache angewandt. Ich habe meinen Verrätern nicht gesagt, dass sie so viel direkten Schaden wie möglich anrichten sollen. Ich habe ihnen gesagt, dass sie so handeln sollen, dass sie das meiste Misstrauen und die meiste Verwirrung stiften und die Generäle dazu bringen, die teuersten Dinge zu tun, um zu versuchen, sie davon abzuhalten, es noch einmal zu tun. Wenn es nur ein paar Verräter gibt und ein ganzes Land, das sich ihnen entgegenstellt, liegt es nahe, dass das, was ein paar Verräter tun, weniger Schaden anrichtet als das, was ein ganzes Land tut, um sie zu stoppen, dass das Heilmittel schlimmer sein könnte als die Krankheit -"

"Mr. Potter", sagte der Verteidigungsprofessor, seine Stimme plötzlich schneidend,

"die Lektion der Geschichte ist, dass Sie einfach falsch liegen. Die Generation Ihrer Eltern hat zu wenig getan, um sich zu vereinigen, nicht zu viel! Dieses ganze Land wäre fast gefallen, Mr. Potter, obwohl Sie nicht dabei waren, um es zu sehen. Ich schlage vor, Sie fragen Ihre Mitschüler in Ravenclaw, wie viele von ihnen Angehörige durch den Dunklen Lord verloren haben. Oder, wenn Sie klüger sind, fragen Sie nicht! Haben Sie einen Wunsch, Mr. Potter?"

"Wenn es Ihnen nichts ausmacht", sagte die milde Stimme von Albus Dumbledore,

"würde ich gerne hören, was der Junge-der-lebte zu sagen hat. Er hat mehr Erfahrung darin, Kriege zu stoppen, als wir beide."

Ein paar Leute lachten, aber nicht viele.

Harry Potters Blick wanderte zu Dumbledore, und er sah einen Moment lang nachdenklich aus.

"Ich sage nicht, dass Sie Unrecht haben, Professor Quirrell. Im letzten Krieg haben die Menschen nicht gemeinsam gehandelt, und ein ganzes Land ist fast an ein paar Dutzend Angreifer gefallen, und ja, das war erbärmlich. Und wenn wir beim nächsten Mal denselben Fehler machen, ja, dann ist das noch erbärmlicher. Aber man kämpft nie denselben Krieg zweimal. Und das Problem ist, dass der Feind auch schlau sein darf. Wenn man geteilt ist, ist man in einer Hinsicht verwundbar; aber wenn man versucht, sich zu vereinen, dann hat man andere Risiken und andere Kosten, und der Feind wird versuchen, auch diese auszunutzen. Man kann nicht aufhören, nur auf einer Ebene des Spiels zu denken."

"Auch die Einfachheit hat viel für sich, Mr. Potter", sagte die trockene Stimme des Verteidigungsprofessors. "Ich hoffe, Sie haben an diesem Tag etwas über die Gefahren von Strategien gelernt, die komplizierter sind als die, sein Volk zu vereinen und den Feind anzugreifen. Und wenn das alles nicht irgendwie mit Ihrem Wunsch zusammenhängt, werde ich ziemlich verärgert sein."

"Ja", sagte Harry Potter, "es war ziemlich schwierig, einen Wunsch zu finden, der die Kosten der Einigkeit symbolisiert. Aber das Problem des gemeinsamen Handelns gibt es nicht nur in Kriegen, es ist etwas, das wir unser ganzes Leben lang lösen müssen, jeden Tag. Wenn alle nach den gleichen Regeln koordinieren und die Regeln dumm sind, dann bricht eine Person, die sich entscheidet, Dinge anders zu machen, die Regeln. Aber wenn alle beschließen, die Dinge anders zu machen, können sie es. Es ist genau das gleiche Problem, dass alle gemeinsam handeln müssen. Aber für die erste Person, die sich zu Wort meldet, sieht es so aus, als würde sie sich gegen die Masse stellen. Und wenn man der Meinung wäre, dass das einzig Wichtige ist, dass die Leute sich immer einig sind, dann könnte man das Spiel nie ändern, egal wie dumm die Regeln sind.

\textbf{\emph{Deshalb ist mein eigener Wunsch, um zu symbolisieren, was passiert, wenn sich Menschen in die falsche Richtung vereinigen, dass wir in Hogwarts Quidditch ohne den Schnatz spielen}}."

"\textbf{WAS}?", schrien hundert Stimmen in der Menge, während Draco die Kinnlade herunterfiel.

"\textbf{Der Schnatz ruiniert das ganze Spiel}", sagte Harry Potter. "Alles, was die anderen Spieler tun, ist am Ende irrelevant. Es wäre überwältigend sinnvoller, einfach eine Uhr zu kaufen. Es ist eines dieser unglaublich dummen Dinge, die man nicht bemerkt, nur weil man damit aufgewachsen ist, die die Leute nur tun, weil alle anderen es auch tun -"

Aber an diesem Punkt war Harry Potters Stimme nicht mehr zu hören, denn der Aufstand hatte begonnen.

Der Aufruhr endete etwa fünfzehn Sekunden später, nachdem ein gigantischer Feuerspeier unter dem Klang von hundert Donnern aus dem höchsten Turm von Hogwarts geschossen war. Draco hatte nicht gewusst, dass Dumbledore so etwas kann. Die Schüler setzten sich ganz vorsichtig und leise wieder hin.

Professor Quirrell lachte, ohne Pause.

"So sei es, Mr. Potter. Ihr Wille geschehe."

Der Verteidigungsprofessor hielt bedächtig inne.

"Natürlich habe ich nur \emph{einen} gerissenen Plan versprochen. Und das ist alles, was ihr drei bekommen werdet."

Draco hatte die Worte schon halb erwartet, aber jetzt kamen sie doch wie ein Schock; Draco tauschte schnelle Blicke mit Granger, sie wären die offensichtlichen Verbündeten gewesen, aber ihre Wünsche waren direkt entgegengesetzt -

"Du meinst", sagte Harry, "wir müssen uns alle auf einen Wunsch einigen?"

"Oh, das wäre viel zu viel verlangt", sagte Professor Quirrell. "Ihr drei habt doch keinen gemeinsamen Feind, oder?"

\emph{Und für einen kurzen Moment, so schnell, dass Draco dachte, er könnte es sich eingebildet haben, flackerten die Augen des Verteidigungsprofessors in Richtung Dumbledore.}

"Nein", sagte Professor Quirrell, "ich meine, dass ich mit einer einzigen Handlung drei Wünsche erfüllen werde."

Es herrschte eine verwirrte Stille.

"Das können Sie nicht tun", sagte Harry flach neben Draco.

"Nicht einmal ich kann das tun. Zwei dieser Wünsche sind miteinander unvereinbar.

Es ist logisch unmöglich -" und dann unterbrach sich Harry.

"Sie sind ein paar Jahre zu jung, um mir zu sagen, was ich nicht tun kann, Mr.

Potter", sagte Professor Quirrell mit einem kurzen trockenen Lächeln. Dann wandte sich der Verteidigungsprofessor wieder an die zuschauenden Schüler.

"Ehrlich gesagt, habe ich kein Vertrauen in eure Fähigkeit, die heutige Lektion zu lernen. Geht nach Hause und genießt die Zeit mit euren Familien, oder was von ihnen übrig ist, solange sie noch leben. Meine eigene Familie ist schon lange durch die Hand des Dunklen Lords tot. Ich sehe euch alle wieder, wenn der Unterricht weitergeht."

In der darauf folgenden sprachlosen Stille, Professor Quirrell wandte sich bereits ab, um die Bühne zu verlassen, hörte Draco die Stimme des Verteidigungsprofessors leise und nicht mehr verstärkt sagen:

"\emph{Aber Sie, Mr. Potter, möchte ich jetzt sprechen.}"

