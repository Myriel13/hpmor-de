

\hypertarget{reziprozituxe4t}{% \section{8. Reziprozität}\label{reziprozituxe4t}}

Kapitel 7: Reziprozität.

Anmerkung des Original Autors E. Yudkowsky:\\ „... Es ist mir ein Bedürfnis, darauf hinzuweisen, dass bestimmte Teile dieses Kapitels nicht als „Bashing“ gedacht sind. Es ist nicht so, dass ich einen Groll hege, die Geschichte schreibt sich einfach von selbst, und wenn man einmal anfängt, Ambosse auf eine Figur zu werfen, ist es schwer, damit aufzuhören...“, Anmerkung Ende.

\emph{„Dein Dad ist fast so fantastisch wie mein Dad.“}

Petunia Evans-Verres Lippen zitterten und in ihren Augen glitzerten Tränen, als Harry auf Gleis 9 der King's Cross Station ihre Hüfte umarmte.

„Bist du sicher, dass du nicht willst, dass ich mitkomme, Harry?“

Harry blickte zu seinem Vater Michael Verres-Evans hinüber, der stereotypisch streng, aber stolz aussah, und dann wieder zu seiner Mutter, die wirklich ziemlich... unbeherrscht aussah.

„Mum, ich weiß, du magst die Zaubererwelt nicht besonders. Du musst nicht mitkommen. Ich meine es ernst.“

Petunia zuckte zusammen. „Harry, du solltest dir keine Sorgen um mich machen, ich bin deine Mutter und wenn du jemanden brauchst, der dich begleitet -“

„Mum, ich werde in Hogwarts monatelang auf mich allein gestellt sein. Wenn ich einen Bahnsteig nicht allein bewältigen kann, ist es besser, das früher als später herauszufinden, damit wir abbrechen können.“

Er senkte seine Stimme zu einem Flüstern. „Außerdem, Mum, lieben mich da drüben alle. Wenn ich Probleme habe, brauche ich nur mein Schweißband abzunehmen“,\\ Harry tippte auf das Übungsband, das seine Narbe bedeckte,\\ „und ich werde viel mehr Hilfe haben, als ich bewältigen kann.“

„Oh, Harry“, flüsterte Petunia.

Sie kniete sich hin und umarmte ihn fest, Gesicht an Gesicht, ihre Wangen ruhten aneinander. Harry konnte ihren raschen Atem spüren, und dann hörte er ein dumpfes Schluchzen entweichen.

„Oh, Harry, ich liebe dich, vergiss das nie.“

Es ist, als hätte sie Angst, mich nie wieder zu sehen, schoss Harry der Gedanke durch den Kopf. Er wusste, dass der Gedanke wahr war, aber er wusste nicht, warum Mum solche Angst hatte. Also stellte er eine Vermutung an.

„Mum, du weißt, dass ich mich nicht in deine Schwester verwandeln werde, nur weil ich Magie lerne, oder? Ich mache jeden Zauber, um den du mich bittest - wenn ich es kann, meine ich - oder wenn du willst, dass ich keine Magie im Haus benutze, dann mache ich das auch, ich verspreche, dass ich nie zulasse, dass Magie zwischen uns kommt -“

Eine feste Umarmung schnitt ihm das Wort ab. „Du hast ein gutes Herz“, flüsterte ihm seine Mutter ins Ohr. „Ein sehr gutes Herz, mein Sohn.“

Harry verschluckte sich dann selbst ein wenig. Seine Mutter ließ ihn los und stand auf. Sie nahm ein Taschentuch aus ihrer Handtasche und tupfte mit zitternder Hand das verlaufende Make-up um ihre Augen ab. Es gab keine Frage darüber, dass sein Vater ihn auf die magische Seite der King's Cross Station begleitete. Dad hatte sogar Schwierigkeiten, Harrys Koffer direkt anzuschauen. Magie lag in der Familie, und Michael Verres-Evans konnte nichts davon. Also räusperte sich sein Vater stattdessen einfach.

„Viel Glück in der Schule, Harry“, sagte er. „Meinst du, ich habe dir genug Bücher gekauft?“

Harry hatte seinem Vater erklärt, wie er dachte, dass dies seine große Chance sein könnte, etwas wirklich Revolutionäres und Wichtiges zu tun, und Professor Verres-Evans hatte genickt und seinen extrem vollen Terminkalender für zwei volle Tage über den Haufen geworfen, um sich auf den größten Secondhand-Buchladen-Raubzug aller Zeiten zu begeben, der vier Städte abgedeckt und dreißig Kisten mit wissenschaftlichen Büchern hervorgebracht hatte, die jetzt im Keller von Harrys Koffer lagen.

Die meisten Bücher waren für ein oder zwei Pfund weggegangen, aber einige definitiv nicht, wie das allerneueste „Handbuch der Physik und Chemie“ oder der komplette 1972er Satz der „Encyclopaedia Britannica“. Sein Vater hatte versucht, Harry den Blick auf den Kassenbon zu verwehren, aber Harry schätzte, dass sein Vater mindestens tausend Pfund ausgegeben haben musste.

Harry hatte zu seinem Vater gesagt, dass er es ihm zurückzahlen würde, sobald er herausgefunden hätte, wie man Zauberergold in Muggelgeld umwandeln könnte, und sein Vater hatte ihm gesagt, er solle in einen See springen.

Und dann hatte sein Vater ihn gefragt: Meinst du, ich habe dir genug Bücher gekauft? Es war ganz klar, welche Antwort Dad hören wollte. Harrys Kehle war aus irgendeinem Grund heiser.

„Man kann nie genug Bücher haben“, rezitierte er das Familienmotto der Verres, und sein Vater kniete nieder und umarmte ihn kurz und fest. „Aber du hast es auf jeden Fall versucht“, sagte Harry und spürte, wie er sich wieder verschluckte. „Es war ein wirklich, wirklich, wirklich guter Versuch.“

Sein Vater richtete sich auf. „Also...“, sagte er. „Siehst du einen Bahnsteig Neun und Dreiviertel?“

Die King's Cross Station war riesig und geschäftig, mit Wänden und Böden, die mit gewöhnlichen, schmutzigen Fliesen gepflastert waren. Er war voll von gewöhnlichen Menschen, die ihren gewöhnlichen Geschäften nachgingen und gewöhnliche Gespräche führten, die jede Menge gewöhnlichen Lärm erzeugten. King's Cross Station hatte einen Bahnsteig Neun (auf dem sie standen) und einen Bahnsteig Zehn (ganz in der Nähe), aber zwischen Bahnsteig Neun und Bahnsteig Zehn gab es nichts außer einem dünnen, wenig verheißungsvollen Absperrwand.

Ein großes Oberlicht über ihnen ließ viel Licht herein, um das völlige Fehlen eines Bahnsteigs 9-dreiviertel zu beleuchten. Harry starrte umher, bis seine Augen tränten, und dachte: Komm schon, Magieblick, komm schon, Magieblick, aber es erschien ihm absolut nichts.

Er überlegte, ob er seinen Zauberstab herausnehmen und damit wedeln sollte, aber Professor McGonagall hatte ihn davor gewarnt, seinen Zauberstab zu benutzen. Außerdem könnte ein weiterer bunter Funkenregen dazu führen, dass er wegen des Zündens von Feuerwerkskörpern in einem Bahnhof verhaftet würde. Und das setzte voraus, dass sein Zauberstab nicht beschloss, etwas anderes zu tun, etwa ganz King's Cross in die Luft zu jagen.

Harry hatte seine Schulbücher nur flüchtig überflogen (obwohl diese Überfliegung schon bizarr genug war), um schnell herauszufinden, welche Art von Wissenschaftsbüchern er sich für die nächsten 48 Stunden kaufen sollte.

Nun, er hatte - Harry schaute auf seine Uhr - eine ganze Stunde Zeit, um das Rätsel zu lösen, da er um elf Uhr im Zug sein sollte. Vielleicht war das das Äquivalent zu einem IQ-Test und die dummen Kinder konnten keine Zauberer werden.

(Und die Menge an Extrazeit, die man sich selbst gab, würde die eigene Gewissenhaftigkeit bestimmen, die der zweitwichtigste Faktor für den schulischen Erfolg war.)

„Ich werde es herausfinden“, sagte Harry zu seinen wartenden Eltern. „Es ist wahrscheinlich eine Art Test oder so.“

Sein Vater runzelte die Stirn. „Hm... vielleicht suchst du nach einer Spur von gemischten Fußabdrücken auf dem Boden, die irgendwohin führt, wo es keinen Sinn zu machen scheint -“

„Dad!“ sagte Harry. „Hör auf damit! Ich habe noch nicht einmal versucht, es selbst herauszufinden!“

Es war auch ein sehr guter Vorschlag, was noch schlimmer war.

„Tut mir leid“, entschuldigte sich sein Vater.

„Ah...“ sagte Harrys Mutter. „Ich glaube nicht, dass sie das mit einem Schüler machen würden, oder? Bist du sicher, dass Professor McGonagall dir nichts gesagt hat?“

„Vielleicht war sie abgelenkt“, sagte Harry ohne nachzudenken.

„Harry!“, zischten sein Vater und seine Mutter zusammen. „Was hast du getan?“

„Ich, ähm -“ Harry schluckte. „Hör zu, wir haben jetzt keine Zeit für so etwas -"

„Harry!“

„Ich meine es ernst! Wir haben jetzt keine Zeit für so was! Weil es eine wirklich lange Geschichte ist und ich herausfinden muss, wie ich zur Schule komme!“

Seine Mutter hatte eine Hand vor dem Gesicht.

„Wie schlimm war es?“

„Ich, ah“, \emph{Aus Gründen der nationalen Sicherheit darf ich darübe nicht sprechen,} „etwa halb so schlimm wie der Vorfall mit dem Schul-Projekt?“

„Harry!?“

„Ich, ähm, oh, sieh mal, da sind ein paar Leute mit einer Eule, ich werde sie fragen, wie man reinkommt!“\\ und Harry rannte von seinen Eltern weg in Richtung der Familie der feurigen Rothaarigen, wobei sein Koffer automatisch hinter ihm herschlitterte. Die mollige Frau schaute zu ihm, als er ankam.

„Hallo, mein Lieber. Das erste Mal in Hogwarts? Ron ist auch neu -“ und dann schaute sie ihn genau an. „Harry Potter?“

Vier Jungen, ein rothaariges Mädchen und eine Eule wirbelten herum und erstarrten dann auf der Stelle.

„Ach, komm schon!“ protestierte Harry. Er hatte vorgehabt, als Harry Verres zu gehen, zumindest bis er in Hogwarts ankam. „Ich habe ein Schweißband gekauft und alles! Woher wissen Sie, wer ich bin?“

„Ja“, sagte Harrys Vater und kam mit langen, leichten Schritten hinter ihm her, „woher wissen Sie, wer er ist?“

Seine Stimme verriet eine gewisse Furcht.

„Dein Bild war in der Zeitung“, sagte einer der beiden identisch aussehenden Zwillinge.

„HARRY!“

„Dad! So ist es nicht! Es ist, weil ich den dunklen Lord Du-weißt-schon-wen besiegt habe, als ich ein Jahr alt war!“

"WAS?!"

„Mum kann das erklären.“

„WAS???“

„Ah... Michael, mein Lieber, es gibt gewisse Dinge, von denen ich dachte, es wäre das Beste, dich erst jetzt damit zu behelligen -“

„Verzeihung“, sagte Harry zu der rothaarigen Familie, die ihn alle anstarrte, „aber es wäre ganz außerordentlich hilfreich, wenn Sie mir jetzt sagen könnten, wie ich zu Bahnsteig neun und drei Viertel komme.“

„Ah...“, sagte die Frau. Sie hob eine Hand und zeigte auf die Wand zwischen den Plattformen. „Geh einfach geradeaus auf die Barriere zwischen Bahnsteig neun und zehn zu. Bleib nicht stehen und hab keine Angst, dass du dagegen stoßen könntest, das ist sehr wichtig. Wenn du nervös bist, mach es am besten im Laufschritt.“

„Und was immer du tust, denk nicht an einen Elefanten.“

„George! Ignorier ihn, Harry, es gibt keinen Grund, nicht an einen Elefanten zu denken.“

„Ich bin Fred, Mum, nicht George...“

„Danke!“ sagte Harry und rannte in Richtung der Barriere -

\emph{Moment mal, es würde nicht funktionieren, wenn er nicht daran glaubte?} In solchen Momenten hasste Harry seinen Verstand dafür, dass er schnell genug arbeitete, um zu erkennen, dass dies ein Fall von „resonantem Zweifel“ war.

D. h. wenn er zu Beginn geglaubt hätte, dass er durch die Barriere gehen würde, wäre alles in Ordnung gewesen, nur jetzt machte er sich Sorgen darüber, ob er ausreichend daran glaubte, dass er durch die Barriere gehen würde, was bedeutete, dass er sich tatsächlich Sorgen darüber machte, in die Barriere zu krachen -

„Harry! Komm zurück, du hast etwas zu erklären!“

Das war sein Dad. Harry schloss die Augen und ignorierte alles, was er über\\ rechtfertigende Glaubwürdigkeit wusste und versuchte einfach, ganz fest daran zu glauben, dass er durch die Barriere gehen würde und -

- die Geräusche um ihn herum veränderten sich. Harry öffnete die Augen und blieb stolpernd stehen, mit dem ekligen Gefühl, sich absichtlich bemüht zu haben, etwas zu glauben.

Er stand auf einem hellen, offenen Bahnsteig neben einem einzigen riesigen Zug, vierzehn lange Waggons, angeführt von einer massiven, scharlachroten Metalldampflok mit einem hohen Schornstein, der den Tod für jede Luftqualität versprach.

Der Bahnsteig war bereits leicht bevölkert (obwohl Harry eine ganze Stunde zu früh dran war); Dutzende von Kindern und ihren Eltern drängten sich um Bänke, Tische und verschiedene Händler und Stände. Es versteht sich von selbst, dass es in der King's Cross Station keinen solchen Platz gab und auch keinen Platz, um ihn zu verstecken.

\emph{Okay, also entweder (a) habe ich mich gerade ganz woanders hin teleportiert, (b) sie können den Raum falten wie kein anderer oder (c) sie ignorieren einfach alle Regeln.}

Hinter ihm war ein schlurfendes Geräusch zu hören, und Harry drehte sich um, um festzustellen, dass sein Koffer ihm tatsächlich auf seinen kleinen krallenartigen Tentakeln gefolgt war. Offenbar hatte es sein Gepäck auch aus magischen Gründen geschafft, mit genügend Kraft zu glauben die Barriere zu passieren.\\ Das war tatsächlich ein wenig beunruhigend, wenn Harry darüber nachdachte.

Einen Moment später kam der jüngste rothaarige Junge im Laufschritt durch den eisernen Torbogen, zog seinen Koffer an einer Leine hinter sich her und krachte fast in Harry hinein. Harry, der sich dumm vorkam, weil er in der Nähe geblieben war, entfernte sich schnell vom Landeplatz, und der rothaarige Junge folgte ihm, wobei er kräftig an der Leine seines Koffers zerrte, um Schritt zu halten.

Einen Moment später flatterte eine weiße Eule durch den Torbogen und ließ sich auf der Schulter des Jungen nieder.

„Hey“, sagte der rothaarige Junge, „bist du wirklich Harry Potter?“

\emph{Nicht das schon wieder.}

„Ich habe keine logische Möglichkeit, das mit Sicherheit zu wissen. Meine Eltern haben mich in dem Glauben erzogen, dass ich Harry James Potter-Evans-Verres heiße, und viele Leute hier haben mir gesagt, dass ich wie meine Eltern aussehe, ich meine, meine anderen Eltern, aber“,\\ Harry runzelte die Stirn, als ihm etwas klar wurde,\\ „soweit ich weiß, könnte es leicht Zauber geben, die ein Kind\\ in ein bestimmtes Aussehen polymorphisieren -“

„Äh, was, Kumpel?“

\emph{Nicht auf dem Weg nach Ravenclaw, wie ich annehme.}

„Ja, ich bin Harry Potter.“

„Ich bin Ron Weasley“, sagte der große, dünne, sommersprossige, langnasige Junge und streckte eine Hand aus, die Harry höflich schüttelte, als sie weitergingen. Die Eule gab Harry einen seltsam gemessenen und höflichen Schrei (eigentlich eher ein eehhhh-Laut, was Harry überraschte).

An diesem Punkt erkannte Harry das Potenzial für eine bevorstehende Katastrophe.

„Nur eine Sekunde“, sagte er zu Ron und öffnete eine der Schubladen seines Koffers, diejenige, die, wenn er sich richtig erinnerte, für Winterkleidung war\\ - das war sie -\\ und dann fand er den leichtesten Schal, den er besaß, unter seinem Wintermantel.

Harry nahm sein Schweißband ab, und ebenso schnell entfaltete er den Schal und band ihn um sein Gesicht. Es war ein bisschen heiß, besonders im Sommer, aber Harry konnte damit leben. Dann klappte er die Schublade zu und zog eine weitere Schublade heraus und holte schwarze Zaubererroben hervor, die er sich über den Kopf zog, jetzt, da er sich nicht mehr auf Muggelgebiet befand.

„Da“, sagte Harry.

Der Ton kam leicht gedämpft durch den Schal über seinem Gesicht heraus. Er wandte sich an Ron.

„Wie sehe ich aus? Dumm, ich weiß, aber bin ich als Harry Potter zu erkennen?“

„Äh“, sagte Ron. Er schloss seinen Mund, der offen gestanden hatte. „Nicht wirklich, Harry.“

„Sehr gut“, sagte Harry. „Aber um den Sinn der ganzen Übung nicht zu vereiteln, wirst du mich von nun an mit“,\\ \emph{Verres geht vielleicht nicht mehr},\\ „Mr. Spoo ansprechen.“

„Okay, Harry“, sagte Ron unsicher.

\emph{Die Macht ist in seinem Fall nicht besonders stark.}

„Nenn... mich... Mister... Spoo.“

„Okay, Mister Spoo -“ Ron hielt inne.\\ „Ich kann das nicht tun, dabei komme ich mir dumm vor.“

\emph{Das ist nicht nur ein Gefühl.}

„Okay. Du suchst einen Namen aus.“

„Mr. Cannon“, sagte Ron sofort. „Für die Chudley Cannons.“

„Ah...“ Harry wusste, dass er es furchtbar bereuen würde, das zu fragen. „Wer oder was sind die Chudley Cannons?“

„Wer sind die Chudley Cannons? Nur die brillanteste Mannschaft in der ganzen Geschichte des Quidditch! Klar, letztes Jahr waren sie ganz unten in der Liga, aber -“

„Was ist Quidditch?“

Auch das zu fragen, war ein Fehler.

„Also, damit ich das richtig verstehe“, sagte Harry, als es schien, dass Rons Erklärung (mit den dazugehörigen Handgesten) zu Ende ging.\\ „Den Schnatz zu fangen ist einhundertfünfzig Punkte wert?“

„Ja -“

„Wie viele Zehn-Punkte-Tore erzielt eine Seite normalerweise, wenn man den Schnatz nicht mitzählt?“

„Ähm, vielleicht fünfzehn oder zwanzig in professionellen Spielen -“

„Das ist einfach falsch. Das verstößt gegen jede mögliche Regel des Spieldesigns. Sieh mal, der Rest des Spiels klingt so, als ob es Sinn machen würde, irgendwie, für einen Sport meine ich, aber du sagst im Grunde, dass das Fangen des Schnatzes fast jede gewöhnliche Toranzahl übersteigt.\\ Die beiden Sucher fliegen da oben herum und suchen nach dem Schnatz, und normalerweise interagieren sie mit niemandem sonst, und den Schnatz zuerst zu sehen, wird also hauptsächlich Glück sein -“

„Das ist kein Glück!“, protestierte Ron. „Man muss seine Augen im richtigen Muster bewegen -“

„Das ist nicht interaktiv, es gibt kein Hin und Her mit dem anderen Spieler und wie viel Spaß macht es, jemandem zuzusehen, der unglaublich gut darin ist, seine Augen zu bewegen? Und dann schnappt sich der Sucher, der Glück hat, den Schnatz und macht die Arbeit aller anderen zunichte.\\ Es ist, als hätte jemand ein echtes Spiel genommen und diese sinnlose Extraposition aufgepfropft, damit man der wichtigste Spieler sein kann, ohne sich wirklich engagieren oder den Rest lernen zu müssen.\\ Wer war der erste Sucher, der idiotische Sohn des Königs, der Quidditch spielen wollte, aber die Regeln nicht verstehen konnte?“

Eigentlich, jetzt wo Harry darüber nachdachte, schien das eine überraschend gute Hypothese zu sein. \emph{Setzt ihn auf einen Besen und sagt ihm, er soll das glänzende Ding fangen...}

Rons Gesicht verzog sich zu einem finsteren Ausdruck.

„Wenn du Quidditch nicht magst, musst du dich nicht darüber lustig machen!“

"Wenn du nicht kritisieren kannst, kannst du nicht optimieren. Ich schlage vor, wie man das Spiel verbessern kann. Und das ist ganz einfach. Schafft den Schnatz ab."

„Man wird das Spiel nicht ändern, nur weil du es sagst!“

„Ich bin der Junge-der-lebte, weißt du. Die Leute werden auf mich hören. Und wenn ich sie dazu überreden kann, das Spiel in Hogwarts zu ändern, wird sich die Neuerung vielleicht verbreiten.“

Ein Ausdruck des absoluten Entsetzens machte sich auf Rons Gesicht breit.

„Aber wenn du den Schnatz abschaffst, wie soll dann jemand wissen, wann das Spiel zu Ende ist?“

„Kauf... eine... Uhr. Das wäre viel fairer, als wenn das Spiel manchmal nach zehn Minuten endet und manchmal erst nach Stunden, und der Zeitplan wäre auch für die Zuschauer viel berechenbarer.“

Harry seufzte. „Oh, hör auf, mir diesen Blick des absoluten Entsetzens zuzuwerfen, ich werde mir wahrscheinlich nicht wirklich die Zeit nehmen, diese erbärmliche Entschuldigung für einen Nationalsport zu zerstören und ihn nach meiner eigenen Idee besser und klüger neu zu gestalten. Ich habe viel, viel, viel wichtigere Dinge, um die ich mich kümmern muss.“

Harry sah nachdenklich aus. „Andererseits würde es nicht viel Zeit kosten, die fünfundneunzig Thesen der schnatzfreien Reformation zu verfassen und sie an eine Kirchentür zu nageln -“

„Potter“, ertönte die Stimme eines jungen Mannes, „was ist das in deinem Gesicht und was steht da neben dir?“

Rons Blick des Entsetzens wurde von blankem Hass abgelöst. „Du!“

Harry drehte den Kopf; und in der Tat, Draco Malfoy, der zwar gezwungen war, die üblichen Schulroben zu tragen, dies aber mit einem Koffer wettmachte, der mindestens genauso magisch und weitaus eleganter aussah als Harrys eigener, verziert mit Silber und Smaragden und mit dem, was Harry für das Malfoy-Familienwappen hielt, einer wunderschönen Schlange mit Reißzähnen über gekreuzten Elfenbeinstäben.

„Draco!“ sagte Harry. „Äh, oder Malfoy, wenn dir das lieber ist, obwohl das für mich irgendwie nach Lucius klingt. Freut mich, dass es dir so gut geht nach, ähm, unserem letzten Treffen. Das ist Ron Weasley. Und ich versuche, inkognito zu bleiben, also nenn mich, äh“, Harry schaute auf seinen Umhang hinunter, „Mister Black."

„Harry!“, zischte Ron. „Du kannst diesen Namen nicht benutzen!“

Harry blinzelte. „Warum nicht?“ Er klang schön dunkel, wie ein internationaler Geheimagent -

„Ich würde sagen, es ist ein schöner Name“, sagte Draco, „aber er gehört zum edlen und uralten Haus Black. Ich werde dich Mr. Silber nennen.“

„Lass die Finger von... von Mr. Gold“, sagte Ron kalt und trat einen Schritt vor. „Er hat es nicht nötig, mit Leuten wie dir zu reden!“

Harry hob beschwichtigend eine Hand. „Ich werde mich mit Mr. Bronze anreden lassen, danke für das Namensschema. Und, Ron, ähm“,

Harry hatte Mühe, einen Weg zu finden, das zu sagen,

„ich bin froh, dass du so... begeistert bist, mich zu beschützen, aber es macht mir nicht besonders viel aus, mit Draco zu reden -“

Das war offenbar der letzte Strohhalm für Ron, der sich mit vor Empörung glühenden Augen auf Harry stürzte.

„Was? Weißt du, wer das ist?“

„Ja, Ron“, sagte Harry, „du erinnerst dich vielleicht, dass ich ihn Draco genannt habe, ohne dass er sich vorzustellen brauchte.“

Draco kicherte. Dann leuchteten seine Augen auf die weiße Eule auf Rons Schulter.

„Oh, was ist das?“ sagte Draco in einem Tonfall, der vor Bosheit strotzte. „Wo ist die berühmte Ratte der Familie Weasley?“

„Begraben im Hinterhof“, sagte Ron kalt.

„Oh, wie traurig. Pot... ah, Mr. Bronze, ich sollte erwähnen, dass die Familie Weasley nach allgemeiner Meinung die beste Haustiergeschichte aller Zeiten hat. Willst du sie erzählen, Weasley?“

Rons Gesicht verzerrte sich.

„Du würdest es nicht lustig finden, wenn es deiner Familie passieren würde!“

„Oh“, schnurrte Draco, „aber es würde den Malfoys niemals passieren.“

Rons Hände ballten sich zu Fäusten -

„Das reicht“, sagte Harry und legte so viel ruhige Autorität in die Stimme, wie er nur konnte. Es war klar, dass, worum auch immer es hier ging, es eine schmerzhafte Erinnerung für den rothaarigen Jungen war.

„Wenn Ron nicht darüber reden will, muss er nicht darüber reden, und ich würde dich bitten, auch nicht darüber zu reden.“

Draco warf Harry einen überraschten Blick zu, und Ron nickte.

„Das ist richtig, Harry! Ich meine Mr. Bronze! Siehst du, was für ein Mensch er ist? Und jetzt sag ihm, dass er verschwinden soll!“

Harry zählte in seinem Kopf bis zehn, was für ihn ein sehr schnelles \emph{12345678910} war -\\ eine seltsame Angewohnheit, die von dem Alter von fünf Jahren übrig geblieben war, als seine Mutter ihn zum ersten Mal angewiesen hatte, es zu tun, und Harry hatte sich überlegt, dass seine Art schneller war und genauso effektiv sein sollte.

„Ich sage ihm nicht, dass er weggehen soll“, sagte Harry ruhig. „Er kann gerne mit mir reden, wenn er will.“

„Nun, ich habe nicht vor, mit jemandem herumzuhängen, der mit Draco Malfoy herumhängt“, verkündete Ron kalt.

Harry zuckte mit den Schultern. „Das liegt an dir. Ich habe nicht vor, mir von irgendjemandem vorschreiben zu lassen, mit wem ich herumhängen darf und mit wem nicht.“

Leise dachte er: \emph{Bitte geh weg, bitte geh weg}...

Rons Gesicht wurde leer vor Überraschung, als hätte er tatsächlich erwartet, dass dieser Spruch funktioniert. Dann wirbelte Ron herum, zerrte an der Leine seines Gepäcks und stürmte den Bahnsteig hinunter.

„Wenn du ihn nicht mochtest“, sagte Draco neugierig, „warum bist du dann nicht einfach weggegangen?“

„Ähm... seine Mutter hat mir geholfen, herauszufinden, wie ich von der King's Cross Station zu diesem Bahnsteig komme, also war es irgendwie schwer, ihm zu sagen, dass er verschwinden soll. Und es ist nicht so, dass ich diesen\\ Ron hasse“, sagte Harry, „ich habe nur...“ Harry suchte nach Worten.

„Siehst du keinen Grund für seine Existenz?“, bot Draco an.

„So ziemlich.“

„Jedenfalls, Potter... wenn du wirklich von Muggeln aufgezogen wurdest -“\\ Draco hielt hier inne, als warte er auf ein Dementi, aber Harry sagte nichts\\ “- dann weißt du vielleicht nicht, wie es ist, berühmt zu sein. Die Leute wollen alle unsere Zeit in Anspruch nehmen. Man muss lernen, nein zu sagen.“

Harry nickte und setzte einen nachdenklichen Gesichtsausdruck auf.

„Das klingt nach einem guten Rat.“

"Wenn du versuchst, nett zu sein, endest du nur damit, dass du die meiste Zeit mit den aufdringlichsten Leuten verbringst. Entscheide, mit wem du Zeit verbringen willst, und schicke alle anderen weg. Du bist gerade erst hier angekommen, Potter, also wird dich jeder danach beurteilen, mit wem er dich sieht, und du willst nicht mit Leuten wie Ron Weasley gesehen werden."

Harry nickte wieder. „Wenn du mir die Frage gestattest, wie hast dumich erkannt?"

„Mister Bronze“, murmelte Draco, „ich bin dir schon begegnet, erinnerst du dich? Ich sah jemanden mit einem um den Kopf gewickelten Schal herumlaufen, der absolut lächerlich aussah. Also habe ich eine Vermutung angestellt.“

Harry senkte den Kopf und nahm das Kompliment an.

„Das tut mir furchtbar leid“, sagte Harry. „Unser erstes Treffen, meine ich. Ich wollte dich nicht vor Lucius in Verlegenheit bringen.“

Draco winkte ab, während er Harry einen seltsamen Blick zuwarf.

„Ich wünschte nur, Vater wäre hereingekommen, während du mir geschmeichelt hast -“ Draco lachte. „Aber danke für das, was du zu Vater gesagt hast. Wenn das nicht gewesen wäre, hätte ich es vielleicht schwieriger gehabt, das zu erklären."

Harry machte eine tiefere Verbeugung.

„Und danke, dass du dich mit dem, was du zu Professor McGonagall gesagt hast, revanchiert hast.“

„Gern geschehen. Allerdings muss eine der Assistentinnen ihre engste Freundin zur absoluten Verschwiegenheit verpflichtet haben, denn Vater sagt, es gehen seltsame Gerüchte um, als hätten wir beide uns gestritten oder so.“

„Autsch“, sagte Harry und zuckte zusammen. „Es tut mir wirklich leid -“

„Nein, wir sind daran gewöhnt, Merlin weiß, dass es schon eine Menge Gerüchte über die Familie Malfoy gibt.“

Harry nickte. „Ich bin froh zu hören, dass du nicht in Schwierigkeiten steckst.“

Draco grinste. „Vater hat, ähm, einen feinen Sinn für Humor, aber er versteht es, Freunde zu finden. Er versteht es sehr gut. Er ließ mich im letzten Monat jeden Abend vor dem Schlafengehen wiederholen: '\emph{Ich werde in Hogwarts Freunde finden}.' Als ich ihm alles erklärt habe und er gesehen hat, dass ich das tue, hat er mir ein Eis gekauft.“

Harrys Kinnlade fiel herunter.

„Du hast es geschafft, das in ein Eis zu verwandeln?“

Draco nickte und sah dabei so selbstgefällig aus, wie es diese Leistung verdient hatte.

„Nun, Vater wusste natürlich, was ich tat, aber er ist derjenige, der mir beigebracht hat, wie man es macht, und wenn ich dabei richtig grinse, macht es das zu einer Vater-Sohn-Sache, und dann muss er mir ein Eis kaufen, oder ich werfe einen traurigen Blick zu, als müsste ich ihn enttäuscht haben.“

Harry beäugte Draco berechnend, er spürte die Anwesenheit eines anderen Meisters.

„Du hast Unterricht darin gehabt, wie man Menschen manipuliert?“

„Natürlich“, sagte Draco voller Stolz. „Ich bin ein Malfoy. Vater hat mir Nachhilfeunterricht gegeben.“

„Wow“, sagte Harry. Die Lektüre von Robert Cialdinis 'Influenz: Wissenschaft und Praxis' war im Vergleich dazu wahrscheinlich nicht besonders gut (obwohl es immer noch ein verdammt gutes Buch war).

„Dein Vater ist fast so fantastisch wie mein Vater.“

Dracos Augenbrauen hoben sich in die Höhe.

„Oh? Und was macht dein Vater?“

„Er kauft mir Bücher.“

Draco überlegte. „Das klingt nicht sehr beeindruckend.“

„Du hättest dabei sein sollen. Wie auch immer, ich bin froh, das alles zu hören. So wie Lucius dich angeschaut hat, dachte ich, er würde dich foltern lassen.“

„Mein Vater liebt mich wirklich“, sagte Draco fest. „Das würde er nie tun.“

„Ähm...“ sagte Harry. Er erinnerte sich an die schwarz gewandete, weißhaarige, elegante Gestalt, die durch Madam Malkins Laden gestürmt war und diesen wunderschönen, tödlichen, silbernen Gehstock geschwungen hatte. Es war nicht leicht, ihn sich als vernarrten Vater vorzustellen.

„Versteh mich nicht falsch, aber woher weißt du das?“

„Hm?“ Es war klar, dass dies eine Frage war, die Draco sich nicht\\ üblicherweise nicht stellte.

„Ich stelle die grundlegende Frage der Rationalität: Warum glaubst du, was du glaubst? Was glaubst du zu wissen und woher glaubst du, es zu wissen? Was lässt dich glauben, dass Lucius dich nicht genauso opfern würde, wie er alles andere für seine Macht opfern würde?“

Draco warf Harry einen weiteren seltsamen Blick zu.

„Was genau weißt du über Vater?“

"Ähm... Sitz im Zaubergamot, Sitz im Vorstand von Hogwarts, Sitz im Gouverneursrat von Hogwarts, unglaublich reich, hat das Ohr von Minister Fudge, hat das Vertrauen von Minister Fudge, hat wahrscheinlich ein paar höchst peinliche Fotos von Minister Fudge, prominentester Vertreter der Blutpuristen, jetzt, wo der Dunkle Lord weg ist, ehemaliger Todesser, bei dem das Dunkle Mal gefunden wurde, der aber davonkam, indem er behauptete, unter dem Imperius-Fluch zu stehen, was lächerlich unglaubwürdig war und so ziemlich jeder wusste... böse mit einem großen 'B' und ein geborener Mörder... Ich glaube, das war's."

Dracos Augen hatten sich zu Schlitzen verengt.

„Das hat dir McGonagall erzählt, oder?“

„Nein, sie hat mir danach nichts mehr über Lucius gesagt, außer dass ich mich von ihm fernhalten soll. Also habe ich mir während des Vorfalls im Zaubertränke-Laden, während Professor McGonagall damit beschäftigt war, den Ladenbesitzer anzuschreien und zu versuchen, alles unter Kontrolle zu bringen, einen der Kunden geschnappt und ihn nach Lucius gefragt.“

Dracos Augen wurden wieder groß.

„Hast du das wirklich?“

Harry warf Draco einen verwirrten Blick zu. „Wenn ich beim ersten Mal gelogen habe, werde ich dir nicht die Wahrheit sagen, nur weil du zweimal fragst.“

Es gab eine gewisse Pause, während Draco dies verinnerlichte.

„Du wirst so was von in Slytherin sein.“

„Ich werde auf jeden Fall in Ravenclaw sein, vielen Dank. Ich will nur Macht, damit ich Bücher bekommen kann.“

Draco gekicherte. „Ja, klar. Jedenfalls... um deine Frage zu beantworten...“ Draco holte tief Luft, und sein Gesicht wurde ernst.\\ „Vater hat einmal eine Zaubergamot-Abstimmung für mich verpasst. Ich saß auf einem Besen und bin runtergefallen und habe mir eine Menge Rippen gebrochen. Es tat wirklich weh. Ich hatte mich noch nie so verletzt und ich dachte, ich würde sterben. Also verpasste Vater diese wirklich wichtige Abstimmung, weil er an meinem Bett im St. Mungo's war, meine Hände hielt und mir versprach, dass ich wieder gesund werden würde.“

Harry blickte unbehaglich weg, dann zwang er sich mit Mühe, wieder zu Draco zu schauen.

„Warum erzählst du mir das? Es scheint irgendwie... sehr privat zu sein...“

Draco warf Harry einen ernsten Blick zu. „Einer meiner Lehrer hat einmal gesagt, dass Menschen enge Freundschaften schließen, indem sie private Dinge übereinander wissen, und der Grund, warum die meisten Menschen keine engen Freundschaften schließen, ist, dass es ihnen zu peinlich ist, etwas wirklich Wichtiges über sich zu erzählen.“\\ Draco streckte seine Handflächen einladend aus. „Du bist dran?“

Das Wissen, dass Dracos hoffnungsvolles Gesicht ihm wahrscheinlich durch monatelange Übung eingebläut worden war, machte es nicht weniger effektiv, stellte Harry fest. Es machte es zwar weniger effektiv, aber leider nicht ineffektiv. Das Gleiche konnte man über Dracos cleveren Einsatz von Gegenseitigkeitsdruck für ein unaufgefordertes Geschenk sagen, eine Technik, über die Harry in seinen Büchern über Sozialpsychologie gelesen hatte

(\emph{ein Experiment hatte gezeigt, dass ein bedingungsloses Geschenk von 5 Dollar doppelt so effektiv war wie ein bedingtes Angebot von 50 Dollar, um Leute zum Ausfüllen von Umfragen zu bewegen)}.

Draco hatte unaufgefordert sein Vertrauen geschenkt und forderte Harry nun auf, im Gegenzug sein Vertrauen anzubieten... und die Sache war die, dass Harry sich unter Druck gesetzt fühlte. Eine Weigerung, da war sich Harry sicher, würde mit einem Blick der traurigen Enttäuschung beantwortet werden, und vielleicht mit einer kleinen Portion Verachtung, die anzeigte, dass Harry Punkte verloren hatte.

„Draco“, sagte Harry, „nur damit du es weißt, ich erkenne genau, was du gerade tust. In meinen Büchern wird das als Reziprozität bezeichnet, und man hat festgestellt, dass es doppelt so effektiv ist, jemandem zwei Sickles zu schenken, als ihm zwanzig Sickles anzubieten, um ihn dazu zu bringen, das zu tun, was man will...“

Harry brach ab. Draco sah traurig und enttäuscht aus.

"Es ist nicht als Trick gedacht, Harry. Es ist ein echter Weg, um Freunde zu werden.„

Harry hob eine Hand. „Ich habe nicht gesagt, dass ich nicht antworten werde. Ich brauche nur Zeit, um mir etwas auszusuchen, das privat ist, aber genauso wenig verletzend. Sagen wir... Ich will dich wissen lassen, dass ich mich nicht zu etwas drängen lasse.“

Eine Denkpause konnte viel dazu beitragen, die Macht vieler Manipulationstechniken zu entschärfen, wenn man erst einmal gelernt hatte, sie als das zu erkennen, was sie waren.

„In Ordnung“, sagte Draco. „Ich werde warten, während du dir etwas einfallen lässt. Oh, und bitte nimm den Schal ab, während du es sagst.“

Einfach, aber effektiv.\\ Und Harry konnte nicht umhin, zu bemerken, wie ungeschickt, unbeholfen, anmutlos sein Versuch, sich der Manipulation zu widersetzen / das Gesicht zu wahren / sich zu zeigen, im Vergleich zu Draco erschienen war.

\emph{Ich brauche diese Nachhilfelehrer.}

„In Ordnung“, sagte Harry nach einer Weile. „Hier ist meine.“\\ Er warf einen Blick in die Runde und rollte dann den Schal wieder über sein Gesicht, so dass alles außer der Narbe zu sehen war.

„Ähm... es klingt, als könntest du dich wirklich auf deinen Vater verlassen. Ich meine... wenn du ernsthaft mit ihm redest, wird er dir immer zuhören und dich ernst nehmen.“

Draco nickte.

„Manchmal“, sagte Harry und schluckte. Das fiel ihm überraschend schwer, aber so sollte es auch sein. „Manchmal wünschte ich, mein eigener Dad wäre\\ wie deiner.“

Harrys Augen zuckten von Dracos Gesicht weg, mehr oder weniger automatisch, und dann zwang sich Harry, Draco wieder anzusehen. Dann wurde Harry klar, was um alles in der Welt er gerade gesagt hatte, und Harry fügte hastig hinzu:

„Nicht, dass ich mir wünschte, mein Dad wäre ein makelloses Instrument des Todes wie Lucius, ich meine nur, mich ernst zu nehmen -“

„Ich verstehe“, sagte Draco mit einem Lächeln. „Da... fühlt es sich jetzt nicht so an, als wären wir der Freundschaft ein Stück näher gekommen?“

Harry nickte. „Ja. Das tut es tatsächlich. Ähm... nichts für ungut, aber ich werde mich wieder verkleiden, ich habe wirklich keine Lust, mich mit -“

„Ich verstehe.“

Harry rollte den Schal wieder über sein Gesicht.

„Mein Vater nimmt alle seine Freunde ernst“, sagte Draco. „Deshalb hat er auch so viele Freunde. Du solltest ihn kennenlernen.“

„Ich werde darüber nachdenken“, sagte Harry mit neutraler Stimme. Er schüttelte verwundert den Kopf. „Du bist also wirklich sein einziger Schwachpunkt. Hm.“

Jetzt warf Draco Harry einen wirklich seltsamen Blick zu.

„Willst du etwas zu trinken holen und uns einen Platz zum Sitzen suchen?“

Harry merkte, dass er zu lange an einer Stelle gestanden hatte, und streckte sich, wobei er versuchte, seinen Rücken zu verrenken.

„Sicher.“

Der Bahnsteig begann sich jetzt zu füllen, aber es gab immer noch einen ruhigeren Bereich auf der anderen Seite, weg von der roten Dampflok. Auf dem Weg dorthin kamen sie an einem Stand vorbei, an dem ein glatzköpfiger, bärtiger Mann Zeitungen und Comics anbot und neongrüne Dosen stapelte. Der Standbesitzer lehnte sich gerade zurück und trank aus einer der neongrünen Dosen, als er den raffinierten und eleganten Draco Malfoy zusammen mit einem mysteriösen Jungen, der unglaublich dumm aussah und einen Schal über das Gesicht gebunden hatte, auf sich zukommen sah, was dazu führte, dass der Standbesitzer mitten im Trinken einen plötzlichen Hustenanfall bekam und eine große Menge neongrüner Flüssigkeit auf seinen Bart tropfte.

„Verzeihung“, sagte Harry, „aber was ist das eigentlich für ein Zeug?“

„Comed-Tea“, sagte der Standbesitzer. „Wenn man es trinkt, wird etwas Überraschendes passieren, so dass man es auf sich selbst oder jemand anderen verschüttet. Aber es ist so verzaubert, dass es nur ein paar Sekunden später verschwindet. -“

In der Tat war der Fleck auf seinem Bart bereits verschwunden.

„Wie albern“, sagte Draco. „So unglaublich albern. Komm, Mr. Bronze, lassen uns einen anderen -“

„Moment mal“, sagte Harry.

„Ach, komm schon! Das ist einfach, einfach nur kindisch!“

„Nein, tut mir leid, Draco, ich muss das untersuchen. Was passiert, wenn ich Comed-Tea trinke, während ich mein Bestes gebe, das Gespräch völlig ernst zu halten?“

Der Standbesitzer lächelte geheimnisvoll. „Wer weiß? Kommt ein Freund im Froschkostüm vorbei? Etwas Unerwartetes wird garantiert passieren -“

„Nein. Es tut mir leid. Ich kann es einfach nicht glauben. Das verstößt gegen meine viel strapazierte Unglaubwürdigkeit auf so vielen Ebenen, dass mir die Sprache fehlt, um es zu beschreiben. Es gibt, es gibt einfach keinen Weg, wie ein verdammter Drink die Realität manipulieren kann, oder ich gebe auf und ziehe mich auf die Bahamas zurück -“

Draco stöhnte. „Wollen wir das wirklich durchziehen?“

"Du musst es nicht trinken, aber ich muss es untersuchen. Ich muss. Wie viel?"

„Fünf Knuts die Dose“, sagte der Standbesitzer.

„Fünf Knuts? Sie können realitätsmanipulierende Sprudelgetränke für fünf Knuts die Dose verkaufen?“

Harry griff in seine Tasche, sagte: „Vier Sicheln, vier Knuts“, und klatschte sie auf den Tresen. „Zwei Dutzend Dosen bitte.“

„Ich nehme auch eine“, seufzte Draco und begann, nach seinen Taschen zu greifen. Harry schüttelte schnell den Kopf.

„Nein, ich kaufe das hier, das zählt auch nicht als Gefallen, ich will sehen, ob es auch bei dir funktioniert.“

Er nahm eine Dose von dem Stapel, der jetzt auf dem Tresen stand, und warf sie Draco zu, dann begann er, seinen Beutel zu füttern. Die sich verbreiternde Lippe des Beutels fraß die Dosen, begleitet von kleinen Rülpsgeräuschen, was nicht gerade dazu beitrug, Harrys Glauben wiederherzustellen, dass er eines Tages eine vernünftige Erklärung für all das finden würde.

Zweiundzwanzig Rülpser später hatte Harry die letzte gekaufte Dose in der Hand, Draco sah ihn erwartungsvoll an, und beide zogen gleichzeitig am Ring. Harry krempelte seinen Schal hoch, um seinen Mund freizulegen, und sie neigten ihre Köpfe zurück und tranken den Comed-Tea. Er schmeckte irgendwie knallgrün - extra sprudelnd und limittiger als Limette. Abgesehen davon passierte nichts weiter. Harry sah den Standbesitzer an, der sie wohlwollend beobachtete.

\emph{Also gut, wenn dieser Kerl einfach einen natürlichen Zufall ausnutzt, um mir vierundzwanzig Dosen von nichts zu verkaufen, dann werde ich seinem kreativen Unternehmergeist applaudieren und ihn dann umbringen.}

„Es passiert nicht immer sofort“, sagte der Standbesitzer. „Aber es passiert garantiert einmal pro Dose, oder Sie bekommen Ihr Geld zurück.“

Harry nahm einen weiteren langen Schluck. Und wieder passierte nichts. V\emph{ielleicht sollte ich einfach alles so schnell wie möglich austrinken... und hoffen, dass mein Magen nicht vor lauter Kohlendioxid explodiert, oder dass ich nicht rülpse, während ich es trinke} ...

Nein, er konnte es sich leisten, ein wenig geduldig zu sein. Aber ganz ehrlich, Harry wusste nicht, wie das funktionieren sollte. Man konnte nicht zu jemandem gehen und sagen:\\ \emph{„Jetzt werde ich dich überraschen“} oder \emph{"Und jetzt erzähle ich dir die Pointe des Witzes, und es wird wirklich lustig sein."}\\ Das ruinierte den Schockwert.

In Harrys geistigem Zustand hätte Lucius Malfoy in einem Ballerina-Kostüm vorbeigehen können und er hätte sich nicht mal übergeben müssen. Was für einen verrückten Blödsinn sollte das Universum denn jetzt ausspucken?

„Wie auch immer, setzen wir uns“, sagte Harry.

Er machte sich bereit, einen weiteren Schluck zu nehmen und ging in Richtung der entfernten Sitzecke, was ihn in den richtigen Winkel brachte, um einen Blick zurück zu werfen und den Teil des Zeitungsständers der Bude zu sehen, der einer Zeitung namens \emph{'Der Klitterer'} gewidmet war, die folgende Schlagzeile zeigte

\textbf{„Junge-der-lebte schwängert Draco Malfoy!“}

„Gah!“, schrie Draco, als eine leuchtend grüne Flüssigkeit aus Harrys Richtung über ihn hinweg spritzte. Draco drehte sich mit Feuer in den Augen zu Harry um und griff nach seiner eigenen Dose.

„Du Sohn eines Schlammbluts! Mal sehen, wie es dir gefällt, bespuckt zu werden!“

Draco nahm einen bedächtigen Schluck aus der Dose, gerade als seine eigenen Augen die Schlagzeile sahen. Aus reinem Reflex versuchte Harry, sein Gesicht zu schützen, als die Flüssigkeitsspritzer in seine Richtung flogen. Leider wehrte er sich mit der Hand, in der er den Comed-Tea hielt, so dass der Rest der grünen Flüssigkeit über seine Schulter spritzte.

Harry starrte auf die Dose in seiner Hand, auch als er weiter würgte und stotterte und die grüne Farbe begann, aus Dracos Robe zu verschwinden. Dann sah er auf und starrte auf die Schlagzeile in der Zeitung.\\ \textbf{„Junge-der-lebte schwängert Draco Malfoy!“}

Harrys Lippen öffneten sich und sagten: „buh-bluh-buh-buh...“

Zu viele konkurrierende Einwände, das war das Problem. Jedes Mal, wenn Harry zu sagen versuchte:

\emph{„Aber wir sind doch erst elf!“,} verlangte der Einwand \emph{"Aber Männer können doch nicht schwanger werden!"} erste Priorität und wurde dann überrollt von \emph{"Aber zwischen uns ist doch nichts!"}

Dann schaute Harry wieder auf die Dose in seiner Hand hinunter. Er verspürte den tief sitzenden Wunsch, schreiend wegzulaufen, bis er aus Sauerstoffmangel umkippte, und das Einzige, was ihn davon abhielt, war, dass er einmal gelesen hatte, dass völlige Panik das Zeichen für ein wirklich wichtiges wissenschaftliches Problem war.

Harry knurrte, warf die Dose gewaltsam in einen nahegelegenen Mülleimer und pirschte sich wieder an den Stand heran.

„Ein Exemplar des Klitterer, bitte.“

Harry zahlte vier weitere Knuts, holte eine weitere Dose Comed-Tea aus seinem Beutel und schlenderte dann mit dem blonden Jungen, der seine eigene Dose mit einem Ausdruck ehrlicher Bewunderung anstarrte, zum Tisch hinüber.

„Ich nehme es zurück“, sagte Draco, „das war ziemlich gut.“

„Hey, Draco, weißt du, was ich wette, was noch besser ist, um Freunde zu werden, als Geheimnisse auszutauschen? Einen Mord zu begehen.“

„Ich habe einen Tutor, der das sagt“, erläuterte Draco. Er griff in seinen Umhang und kratzte sich mit einer leichten, natürlichen Bewegung. „An wen denkst du da?“

Harry knallte den Klitterer hart auf den Tisch.

„Der Typ, der sich diese Schlagzeile ausgedacht hat.“

Draco stöhnte auf. „Kein Typ. Ein Mädchen. Ein zehnjähriges Mädchen, kannst du das glauben? Sie ist nach dem Tod ihrer Mutter durchgedreht und ihr Vater, dem diese Zeitung gehört, ist davon überzeugt, dass sie eine Seherin ist, also fragt er, wenn er etwas nicht weiß Luna Lovegood und glaubt alles, was sie sagt.“

Ohne wirklich darüber nachzudenken, zog Harry den Ring an seiner nächsten Dose Comed-Tea und machte sich bereit, zu trinken.

„Willst du mich verarschen? Das ist ja noch schlimmer als Muggeljournalismus, was ich für physikalisch unmöglich gehalten hätte.“

Draco knurrte. „Sie hat auch so eine Art perverse Besessenheit von den Malfoys, und ihr Vater ist politisch gegen uns, also druckt er jedes Wort. Sobald ich alt genug bin, werde ich sie vergewaltigen.“

Grüne Flüssigkeit spritzte aus Harrys Nasenlöchern und sickerte in den Schal, der diesen Bereich noch bedeckte.

Flüssigkeit und Lunge passen nicht zusammen, und Harry verbrachte die nächsten Sekunden damit, verzweifelt zu husten. Draco schaute ihn scharf an.

„Stimmt etwas nicht?“

An diesem Punkt wurde Harry plötzlich klar, dass\\ (a) \emph{die Geräusche, die vom Rest des Bahnsteigs kamen, sich etwa zur gleichen Zeit, als Draco in seinen Umhang gegriffen hatte, in ein verschwommenes weißes Rauschen verwandelt hatten}, und dass\\ (b) als er über Mord als Bindungsmethode gesprochen hatte, es genau eine Person in dem Gespräch gegeben hatte, die gedacht hatte, dass sie scherzen würden.

\emph{Das stimmt. Weil er so ein normales Kind zu sein schien.}

Und er ist ein normales Kind, er ist genau das, was man von einem männlichen Standard-kind erwarten würde, wenn Darth Vader sein vernarrter Vater wäre.

„Ja, nun“, hustete Harry,\\ oh Gott, wie sollte er aus diesem Gespräch heil herauskommen,\\ „ich war nur überrascht, dass du bereit warst, so offen darüber zu reden, du schienst nicht besorgt zu sein, erwischt zu werden oder so.“

Draco schnaubte. „Machst du Witze? Luna Lovegoods Wort gegen meins?“

\emph{Heilige Scheiße auf einem heiligen Stock.}

„So etwas wie magische Wahrheitsfindung gibt es nicht, nehme ich an?“\\ \emph{Oder DNA-Tests... noch nicht.}

Draco sah sich um. Seine Augen verengten sich.

"Stimmt, du weißt gar nichts. Hör zu, ich werde dir die Dinge erklären, ich meine, wie es wirklich funktioniert, so wie wenn du schon in Slytherin wärst und mir die gleiche Frage gestellt hättest. Aber du musst schwören, dass du nichts darüber sagst."

„Ich schwöre“, sagte Harry.

„Die Gerichte benutzen Veritaserum, aber das ist eigentlich ein Witz, man lässt sich einfach die Erinnerung löschen, bevor man aussagt, und behauptet dann, die andere Person sei mit einem falschen Erinnerungszauber belegt worden. Natürlich, wenn du nur eine normale Person bist, gehen die Gerichte davon aus, dass deine Erinnerung manipuliert ist und keinen falscher Erinnerungszauber im Spiel ist. Aber das Gericht hat einen Ermessensspielraum, und wenn ich involviert bin, dann beeinträchtigt es die Ehre eines Adelshauses, also geht es an das Zaubergamot, wo Vater die Stimmen hat.

Wenn ich für nicht schuldig befunden werde, muss die Familie Lovegood Wiedergutmachung für die Befleckung meiner Ehre leisten. Und sie wissen von Anfang an, dass es so laufen wird, also werden sie einfach den Mund halten.“

Ein kalter Schauer überkam Harry, ein Schauer, der mit der Anweisung kam, seine Stimme und sein Gesicht normal zu halten.

\emph{Notiz an mich selbst: Stürze die Regierung des magischen Britanniens bei nächster Gelegenheit.}

Harry hustete erneut, um sich zu räuspern. „Draco, bitte versteh mich nicht falsch, mein Wort gilt, aber wie du gesagt hast, ich könnte in Slytherin sein und ich möchte wirklich nur zu Informationszwecken fragen, was würde theoretisch passieren, wenn ich aussagen würde, dass ich gehört habe, dass du es geplant hast?“

„Dann wäre ich in Schwierigkeiten, wenn ich jemand anderes als ein Malfoy wäre“, antwortete Draco süffisant. „Da ich ein Malfoy bin... hat Vater die Stimmen. Und danach würde er dich zerquetschen... na ja, ich schätze, nicht so leicht, da du der Junge-der-lebte bist, aber Vater ist ziemlich gut in solchen Dingen.“

Draco runzelte die Stirn. „Außerdem hast du davon geredet, sie zu ermorden, warum hast du dir keine Gedanken darüber gemacht, dass ich aussage, wenn sie tot auftaucht?“

\emph{Wie, oh wie konnte mein Tag nur so schief gehen?}

Harrys Mund bewegte sich schon schneller, als er denken konnte.

„Da dachte ich, sie ist älter! Ich weiß nicht, wie es hier funktioniert, aber in Muggelbritannien würden sich die Gerichte viel mehr aufregen, wenn jemand ein Kind tötet -“

„Das macht Sinn“, sagte Draco, der immer noch ein wenig misstrauisch dreinschaute. „Aber wie auch immer, es ist immer klüger, wenn es gar nicht erst zu den Auroren geht. Wenn wir darauf achten, nur Dinge zu tun, die mit Heilzaubern behoben werden können, können wir ihre Erinnerung hinterher einfach löschen und das Ganze nächste Woche wiederholen.“

Dann kicherte der blondhaarige Junge, ein jugendlich hoher Ton.

„Obwohl, stell dir mal vor, sie würde sagen, Draco Malfoy und dem Jungen-der-lebte hätten es ihr besorgt, nicht mal Dumbledore würde ihr glauben.“

\emph{Ich werde dein armseliges kleines magisches Überbleibsel aus dem finsteren Mittelalter in Stücke zerreißen, die kleiner sind als die Atome, aus denen sie bestehen.}

„Eigentlich, können wir damit noch warten? Nachdem ich herausgefunden habe, dass die Schlagzeile von einem Mädchen stammt, das ein Jahr jünger ist als ich, hatte ich einen anderen Gedanken für meine Rache.“

„Hm? Erzähl mal“, sagte Draco und nahm einen weiteren Schluck von seinem Comed-Tea.

Harry wusste nicht, ob der Zauber mehr als einmal pro Dose funktionierte, aber er wusste, dass er den Vorwurf vermeiden konnte, also achtete er auf den richtigen Zeitpunkt:

„Ich habe mir gedacht, dass ich diese Frau eines Tages heiraten werde.“

Draco gab ein grauenhaftes, krächzendes Geräusch von sich und ließ grüne Flüssigkeit aus seinen Mundwinkeln tropfen wie ein kaputter Autokühler.

„Bist du verrückt?“

„Ganz im Gegenteil, ich bin so vernünftig, dass es wie Eis brennt.“

„Du hast einen seltsameren Geschmack als eine Lestrange“, sagte Draco und klang dabei halb bewundernd. „Und ich nehme an, du willst sie ganz für dich allein, hm?“

„Jep. Ich kann dir dafür einen Gefallen schulden -“ Draco winkte ab. „Nee, das hier ist umsonst.“

Harry starrte auf die Dose in seiner Hand hinunter, die Kälte setzte sich in seinem Blut fest.

\emph{Charmant, fröhlich, großzügig mit seinen Gefälligkeiten für seine Freunde,} Draco war kein Psychopath. Das war das Traurige und Schreckliche daran, die menschliche Psychologie gut genug zu kennen, um zu wissen, dass Draco kein Monster war.

Es hatte in der Geschichte der Welt zehntausend Gesellschaften gegeben, in denen dieses Gespräch hätte stattfinden können. Nein, die Welt wäre in der Tat ein ganz anderer Ort gewesen, wenn es einen bösen Mutanten gebraucht hätte, um zu sagen, was Draco gesagt hatte.

Es war sehr einfach, sehr menschlich, es war der Standard, wenn nichts anderes dazwischenkam.

\emph{Für Draco waren seine Feinde keine Menschen}.

Und in der verlangsamten Zeit dieses verlangsamten Landes, hier und jetzt wie in der Dunkelheit vor dem Zeitalter der Vernunft, würde der Sohn eines hinreichend mächtigen Adligen einfach davon ausgehen, dass er über dem Gesetz stand, zumindest wenn es um irgendein Bauernmädchen ging.

Es gab Orte im Muggelland, an denen es immer noch so war, Länder, in denen diese Art von Adel noch existierte und immer noch so dachte, oder noch grimmigere Länder, in denen es nicht nur der Adel war.

So war es an jedem Ort und zu jeder Zeit, die nicht direkt von der Aufklärung abstammte. Eine Abstammungslinie, so schien es, die das magische Britannien nicht ganz mit einschloss, so sehr es auch eine kulturübergreifende Kontaminierung durch Dinge wie ringgezogene Getränkedosen gegeben hatte.

\emph{Und wenn Draco seine Meinung über Rache nicht ändert und ich nicht meine eigene Chance auf ein glückliches Leben wegwerfe, um ein armes verrücktes Mädchen zu heiraten, dann habe ich nur Zeit gekauft, und nicht zu viel davon...}

Für ein Mädchen. Nicht für andere.

\emph{Ich frage mich, wie schwierig es wäre, einfach eine Liste mit allen Top-Blutpuristen zu erstellen und sie zu töten.}

Sie hatten genau das während der Französischen Revolution versucht, mehr oder weniger - eine Liste aller Feinde des Fortschritts zu erstellen und alles oberhalb des Halses zu entfernen - und es hatte nicht gut funktioniert, soweit Harry sich erinnerte. Vielleicht sollte er ein paar der Geschichtsbücher abstauben, die sein Vater ihm gekauft hatte, und nachsehen, ob das, was bei der Französischen Revolution schief gelaufen war, nicht leicht zu beheben war.

Harry blickte in den Himmel und auf die blasse Form des Mondes, die an diesem Morgen durch die wolkenlose Luft sichtbar war.

\emph{Die Welt ist also kaputt und fehlerhaft und verrückt, und grausam und blutig und dunkel. Das sind Neuigkeiten? Das wusstest du sowieso schon immer...}

„Du guckst so ernst“, sagte Draco. „Lass mich raten, deine Muggel-Eltern haben dir gesagt, dass so etwas schlecht ist.“

Harry nickte, seiner Stimme nicht ganz trauend.

„Nun, wie Vater sagt, es mag vier Häuser geben, aber am Ende gehört jeder entweder zu Slytherin oder zu Hufflepuff. Und ehrlich gesagt bist du nicht auf der Seite von Hufflepuff. Wenn du dich unter dem Tisch für die Malfoys entscheidest, unsere Macht und deinen Ruf...\\ Du könntest mit Dingen davonkommen, die nicht mal ich tun kann. Willst du es mal ausprobieren? Um zu sehen, wie es ist?“

\emph{Sind wir nicht eine schlaue kleine Schlange. Elf Jahre alt und schon lockt man die Beute aus dem Versteck...}

Harry dachte nach, überlegte, wählte seine Waffe.

„Draco, willst du mir die Sache mit der Blutreinheit erklären? Ich bin ja noch ziemlich neu.“

Ein breites Lächeln ging über Dracos Gesicht. „Du solltest wirklich Vater treffen und ihn fragen, du weißt schon, er ist unser Anführer.“

„Gib mir die Zweiunddreißig-Sekunden-Version.“

„Okay“, sagte Draco. Er holte tief Luft, und seine Stimme wurde etwas leiser und nahm einen ernsten Tonfall an.

„Unsere Kräfte sind schwächer geworden, von Generation zu Generation, während die Schlammblutverschmutzung zunahm. Wo Salazar und Godric und Rowena und Helga einst Hogwarts durch ihre Macht erhoben haben, indem sie das Medaillon und das Schwert und das Diadem und den Kelch schufen, hat sich kein Zauberer dieser verblichenen Tage erhoben, um es mit ihnen aufzunehmen. Wir verblassen, wir werden alle zu Muggeln, wenn wir uns mit ihrer Brut vermischen und unsere Squibs leben lassen. Wenn dem Makel nicht Einhalt geboten wird, werden unsere Zauberstäbe bald zerbrechen und all unsere Künste erlöschen, die Linie Merlins wird enden und das Blut von Atlantis versagen. Unsere Kinder werden im Dreck kratzen müssen, um zu überleben wie die einfachen Muggel, und Dunkelheit wird die ganze Welt für immer bedecken.“

Draco nahm einen weiteren Schluck aus seiner Getränkedose und sah zufrieden aus; das schien das ganze Argument zu sein, soweit es ihn betraf.

„Überzeugend“, sagte Harry, wobei er es eher deskriptiv als normativ meinte.

Es war ein Standardmuster: Der Sündenfall, die Notwendigkeit, das, was an Reinheit übrig geblieben war, vor Verunreinigung zu bewahren, die Vergangenheit, die sich nach oben neigte, und die Zukunft, die sich nur nach unten neigte. Und dieses Muster hatte auch sein Gegenstück...

„Ich muss dich allerdings in einem Punkt korrigieren. Deine Informationen über die Muggel sind ein wenig veraltet. Wir kratzen nicht mehr in der Erde.“

Dracos Kopf ruckte herum.

„Was? Was soll das heißen, wir?“

„Wir. Die Wissenschaftler. Die Linie von Francis Bacon und das Blut der Aufklärung. Die Muggel saßen nicht nur herum und heulten, weil sie keine Zauberstäbe hatten, wir haben jetzt unsere eigenen Kräfte, mit oder ohne Magie. Wenn alle eure Kräfte versagen, dann haben wir alle etwas sehr Wertvolles verloren, denn eure Magie ist der einzige Hinweis, den wir haben, wie das Universum wirklich funktionieren muss -

aber ihr werdet nicht am Boden kratzen müssen. Eure Häuser werden immer noch im Sommer kühl und im Winter warm sein, es wird immer noch Ärzte und Medizin geben. Die Wissenschaft kann dich am Leben erhalten, wenn die Magie versagt. Es wäre eine Tragödie, aber nicht buchstäblich das Ende allen Lichts auf der Welt. Ich meine ja nur.“

Draco war einige Meter zurückgewichen und sein Gesicht war eine Mischung aus Angst und Unglauben.

„Wovon, um Himmels willen, redest du, Potter?“

„Hey, ich habe mir deine Geschichte angehört, willst du dir nicht auch meine anhören?“ \emph{Zu simpel}, schimpfte Harry mit sich selbst, aber Draco hörte tatsächlich auf, sich zurückzuziehen und schien zuzuhören.

„Wie dem auch sei“, sagte Harry, „ich will damit sagen, dass du anscheinend nicht viel darauf geachtet hast, was in der Muggelwelt vor sich geht.“

\emph{Wahrscheinlich, weil die gesamte Zaubererwelt den Rest der Erde als Slum zu betrachten schien, der ungefähr so viel Berichterstattung verdiente, wie die Financial Times den alltäglichen Qualen in Burundi zubilligte.}

„Alles klar. Kurze Überprüfung. Waren Zauberer jemals auf dem Mond? Du weißt schon, das Ding?“\\ Harry zeigte auf den riesigen, weit entfernten Globus.

„Was?!“ sagte Draco.

Es war ziemlich klar, dass der Gedanke dem Jungen nie in den Sinn gekommen war.

„Flieg zum Mond - es ist nur ein -“ Sein Finger zeigte auf das kleine, blasse Ding am Himmel. „Man kann nicht an einen Ort apparieren, an dem man noch nie gewesen ist, und wie sollte man überhaupt auf den Mond kommen?“

„Warte mal“, sagte Harry zu Draco, „ich möchte dir ein Buch zeigen, das ich mitgebracht habe, ich glaube, ich weiß noch, in welcher Kiste es ist.“

Und Harry stand auf und kniete sich hin und riss die Treppe zum Keller seines Koffers heraus, dann riss er die Treppe hinunter und hob eine Kiste von einer anderen Kiste, \emph{wobei er gefährlich nahe daran kam, seine Bücher respektlos zu behandeln,} und riss den Deckel der Kiste ab und holte schnell, aber vorsichtig einen Stapel Bücher heraus -\\ (Harry hatte die nahezu magische Fähigkeit von Verres geerbt, sich zu merken, wo alle seine Bücher waren, selbst nachdem er sie nur einmal gesehen hatte, was ziemlich mysteriös war, wenn man bedenkt, dass es keinen genetischen Zusammenhang gab).\\ Und Harry rannte die Treppe wieder hinauf, schob sie mit dem Absatz zurück in den Koffer und blätterte keuchend in dem Buch, bis er das Bild fand, das er Draco zeigen wollte.

Das mit dem weißen, trockenen, verkraterten Land und den angezogenen Menschen und dem blau-weißen Globus, der über allem hing.

\emph{Dieses Bild. Das eine Bild, wenn nur ein einziges Bild auf der ganzen Welt überleben sollte.}

„Das“, sagte Harry, und seine Stimme zitterte, weil er den Stolz nicht ganz unterdrücken konnte, „ist das, wie die Erde vom Mond aus aussieht.“

Draco beugte sich langsam vor. Es lag ein seltsamer Ausdruck auf seinem jungen Gesicht.

„Wenn das ein echtes Bild ist, warum bewegt es sich nicht?“

\emph{sich bewegen? Achja.}

„Muggel können bewegte Bilder machen, aber sie brauchen einen größeren Kasten, um sie zu zeigen, sie können sie noch nicht auf einzelne Buchseiten packen.“

Dracos Finger wanderte zu einem der Anzüge. „Was ist das?“ Seine Stimme begann zu schwanken.

„Das sind menschliche Wesen. Sie tragen Anzüge, die ihren ganzen Körper bedecken, damit sie Luft bekommen, denn auf dem Mond gibt es keine Luft.“

„Das ist unmöglich“, flüsterte Draco. In seinen Augen stand Entsetzen und völlige Verwirrung. „Kein Muggel könnte das jemals tun. Wie...“

Harry nahm das Buch zurück, blätterte die Seiten um, bis er fand, was er sah.

„Das ist eine Rakete, die nach oben steigt. Das Feuer treibt sie höher und höher, bis sie den Mond erreicht.“\\ Er blätterte wieder um.\\ "Das ist eine Rakete auf dem Boden. Der kleine Fleck daneben ist ein Mensch."

Draco schnappte nach Luft.

„Die Reise zum Mond hat umgerechnet... wahrscheinlich etwa eine Milliarde Galleonen gekostet.“

Draco verschluckte sich.

„Und es brauchte die Anstrengungen von... wahrscheinlich mehr Menschen, als es Magiekundige in Britannien gibt.“

\emph{Und als sie ankamen, hinterließen sie eine Tafel, auf der stand:}\\ \emph{Wir kamen in Frieden, für die ganze Menschheit.}\\ \emph{Obwohl du noch nicht bereit bist, diese Worte zu hören, Draco Malfoy..}.

„Du sagst die Wahrheit“, sagte Draco langsam. „Du würdest nicht ein ganzes Buch fälschen, nur um dies zu tun - und ich kann es in deiner Stimme hören. Aber... aber..."

„Wie, ohne Zauberstäbe oder Magie? Das ist eine lange Geschichte, Draco. Wissenschaft funktioniert nicht, indem man mit Zauberstäben wedelt und Zaubersprüche aufsagt, sondern indem man weiß, wie das Universum auf einer so tiefen Ebene funktioniert, dass man genau weiß, was man tun muss, damit das Universum das tut, was man will. Wenn Magie so ist, als würde man jemanden mit einem Imperius belegen, damit er tut, was man will, dann ist Wissenschaft so, als würde man ihn so gut kennen, dass man ihn davon überzeugen kann, dass es die ganze Zeit seine eigene Idee war.

Es ist viel schwieriger, als mit einem Zauberstab zu wedeln, aber es funktioniert, wenn der Zauberstab versagt, genauso wie man, wenn der Imperius versagt, immer noch versuchen kann, eine Person zu überreden. Und Wissenschaft baut sich von Generation zu Generation auf. Man muss wirklich wissen, was man tut, um Wissenschaft zu betreiben - und wenn man etwas wirklich versteht, kann man es auch jemand anderem erklären.

Die größten Wissenschaftler von vor einem Jahrhundert, die hellsten Namen, die immer noch mit Ehrfurcht gesprochen werden, ihre Kräfte sind wie nichts zu den größten Wissenschaftlern von heute. Es gibt in der Wissenschaft kein Äquivalent zu euren verlorenen Künsten, die Hogwarts groß gemacht haben. In der Wissenschaft wachsen unsere Kräfte von Jahr zu Jahr. Und wir beginnen, die Geheimnisse des Lebens und der Vererbung zu verstehen und zu enträtseln.

Wir können das Blut, von dem du sprichst, untersuchen und sehen, was dich zu einem Zauberer macht. Und in ein oder 2 Generationen können wir dieses Blut dazu bringen, auch deine Kinder zu mächtigen Zauberern zu machen. Du siehst also, dein Problem ist nicht annähernd so schlimm, wie es aussieht, denn in ein paar weiteren Jahrzehnten wird die Wissenschaft in der Lage sein, es für dich zu lösen.“

„Aber...“ sagte Draco. Seine Stimme zitterte. „Wenn Muggel diese Art von Macht haben... dann... was sind wir?“

„Nein, Draco, das ist es nicht, verstehst du nicht? Die Wissenschaft zapft die Macht des menschlichen Verstandes an, um die Welt zu betrachten und herauszufinden, wie sie funktioniert. Sie kann nicht versagen, ohne dass die Menschheit selbst versagt. Deine Magie könnte sich abschalten, und du würdest das hassen, aber du wärst immer noch du. Du wärst noch am Leben, um es zu bedauern. Aber weil die Wissenschaft auf meiner menschlichen Intelligenz beruht, ist sie die Macht, die nicht von mir entfernt werden kann, ohne mich zu entfernen.

Selbst wenn sich die Gesetze des Universums vor meinen Augen ändern, so dass all mein Wissen ungültig ist, werde ich einfach die neuen Gesetze herausfinden, so wie es schon einmal gemacht wurde. Es ist keine Muggel-Sache, es ist eine menschliche Sache, es verfeinert und trainiert nur die Kraft, die du jedes Mal einsetzt, wenn du etwas ansiehst, das du nicht verstehst, und dich fragst: 'Warum?' Du bist ein Slytherin, Draco, siehst du nicht die Konsequenz?“

Draco blickte von dem Buch zu Harry auf. Sein Gesicht zeigte dämmerndes Verständnis.

„Zauberer können lernen, diese Kraft zu nutzen.“

\emph{Ganz vorsichtig, jetzt... Der Köder ist ausgelegt, jetzt der Haken...}

„Wenn du lernst, dich als Mensch und nicht als Zauberer zu sehen, kannst du deine Kräfte als Mensch trainieren und verfeinern.“

\emph{Und wenn diese Anweisung nicht in einem naturwissenschaftlichen Lehrplan stand, brauchte Draco das ja nicht zu wissen, oder?}

Dracos Augen waren jetzt nachdenklich.

„Du hast... das schon gemacht?“

„Bis zu einem gewissen Grad“, erklärte Harry. „Meine Ausbildung ist noch nicht abgeschlossen. Nicht mit elf. Aber - mein Vater hat mir auch Nachhilfelehrer gekauft, weißt du.“

Sicher, sie waren hungernde Studenten gewesen, und das auch nur, weil Harry im 26-Stunden-Takt schlief, aber lassen wir das alles mal beiseite...

Langsam nickte Draco. „Du glaubst, du kannst beide Künste beherrschen, die Kräfte zusammenführen und...“ Draco starrte Harry an. „Dich zum Herrscher über die beiden Welten machen?“

Harry stieß ein böses Lachen aus, es schien in diesem Moment einfach natürlich zu kommen.

„Du musst begreifen, Draco, dass die ganze Welt, die du kennst, das ganze magische Britannien, nur ein Quadrat auf einem viel größeren Spielbrett ist. Das Spielbrett, das Orte wie den Mond und die Sterne am Nachthimmel einschließt, die Lichter wie die Sonne sind, nur unvorstellbar weit weg, und Dinge wie Galaxien, die viel größer sind als die Erde und die Sonne, Dinge, die so groß sind, dass nur Wissenschaftler sie sehen können und du nicht einmal weißt, dass sie existieren. Aber ich bin wirklich Ravenclaw, weißt du, nicht Slytherin. Ich will das Universum nicht beherrschen. Ich finde nur, dass es besser organisiert sein könnte.“

Auf Dracos Gesicht stand Ehrfurcht.

„Warum erzählst du mir das?“

„Oh... es gibt nicht viele Leute, die wissen, wie man echte Wissenschaft betreibt - etwas zum ersten Mal zu verstehen, auch wenn es einen total verwirrt. Hilfe wäre nützlich.“

Draco starrte Harry mit offenem Mund an.

„Aber mach keinen Fehler, Draco, wahre Wissenschaft ist nicht wie Magie, man kann sie nicht einfach machen und unverändert weggehen, wie wenn man lernt, wie man die Worte eines neuen Zaubers sagt. Die Macht kommt mit einem Preis, einem Preis, der so hoch ist, dass die meisten Leute sich weigern, ihn zu zahlen.“

Draco nickte dazu, als hätte er endlich etwas gehört, das er verstehen konnte.

„Und dieser Preis ist was?“

„Zu lernen, zuzugeben, dass man im Unrecht ist.“

„Ähm“, sagte Draco, nachdem sich die dramatische Pause noch eine Weile hingezogen hatte. „Kannst du das erklären?“

„Wenn man versucht, herauszufinden, wie etwas auf dieser tiefen Ebene funktioniert, sind die ersten neunundneunzig Erklärungen, die einem einfallen, falsch. Die hundertste ist richtig. Also musst du lernen, zuzugeben, dass du falsch liegst, immer und immer und immer wieder. Das hört sich nicht nach viel an, aber es ist so schwer, dass die meisten Menschen keine Wissenschaft betreiben können. Man muss sich selbst immer wieder in Frage stellen, Dinge, die man immer für selbstverständlich gehalten hat, immer wieder neu betrachten“,

\emph{wie zum Beispiel einen Schnatz im Quidditch zu haben},

„und jedes Mal, wenn man seine Meinung ändert, ändert man sich selbst. Aber ich greife hier viel zu weit vor. Viel zu weit. Ich will nur, dass du weißt... dass ich dir anbiete, etwas von meinem Wissen zu teilen. Wenn du willst. Es gibt nur eine Bedingung.“

„Äh, ja“, sagte Draco. „Weißt du, Vater sagt, wenn jemand so etwas zu dir sagt, ist das nie ein gutes Zeichen.“

Harry nickte. „Verstehen mich jetzt nicht falsch und denke, dass ich versuche, einen Keil zwischen dich und deinen Vater zu treiben. Darum geht es nicht. Es geht nur darum, dass ich lieber mit jemandem in meinem Alter zu tun haben möchte, als dass dies eine Sache zwischen mir und Lucius sein soll. Ich denke, dein Vater wäre auch damit einverstanden, er weiß, dass du irgendwann erwachsen werden musst. Aber deine Züge in unserem Spiel müssen deine eigenen sein. Das ist meine Bedingung - dass ich es mit dir zu tun habe, Draco, nicht mit deinem Vater.“

„Ich muss gehen“, sagte Draco. Er stand auf. „Ich muss los und über die Sache nachdenken.“

„Lass dir Zeit“, sagte Harry.

Die Geräusche auf dem Bahnsteig verwandelten sich von einem Verschwimmen in ein Murmeln, als Draco loslief. Harry atmete langsam die Luft aus, die er angehalten hatte, ohne sich dessen bewusst zu sein, und sah dann auf die Uhr an seinem Handgelenk, ein einfaches mechanisches Modell, das sein Vater ihm in der Hoffnung gekauft hatte, dass es in Gegenwart von Magie funktionieren würde. Der Sekundenzeiger tickte noch, und wenn der Minutenzeiger stimmte, dann war es noch nicht ganz elf. Wahrscheinlich sollte er bald in den Zug steigen und sich auf die Suche nach Wie-war-ihr-Name machen, aber es schien sich zu lohnen, sich vorher ein paar Minuten Zeit zu nehmen, um ein paar Atemübungen zu machen und zu sehen, ob sich sein Blut wieder erwärmte.

Doch als Harry von seiner Uhr aufblickte, sah er zwei Gestalten auf sich zukommen, die mit ihren von Winterschals verhüllten Gesichtern völlig lächerlich aussahen.

„Hallo, Mr. Bronze“, sagte eine der maskierten Gestalten. „Können wir Sie dafür interessieren, dem Orden des Chaos beizutreten?“

\textbf{Nachwirkungen}:\\ Nicht allzu lange danach, als sich die Aufregung des Tages endlich gelegt hatte, saß Draco über einen Schreibtisch gebeugt mit einer Schreibfeder in der Hand. Er hatte ein Privatzimmer in den Slytherin-Kerkern, mit eigenem Schreibtisch und eigenem Feuer - leider hatte nicht einmal er einen Anschluss an das Floh-Netzwerk, aber wenigstens glaubte Slytherin nicht an diesen völligen Unsinn, alle in Schlafsälen schlafen zu lassen.

Es gab nicht viele Privatzimmer, man musste schon der Allerbeste im Haus der besseren Sorte sein, aber das war im Hause Malfoy selbstverständlich.

Lieber Vater,\\ schrieb Draco. Und dann hielt er inne. Tinte tropfte langsam aus seiner Feder und befleckte das Pergament neben den Worten.

Draco war nicht dumm. Er war zwar jung, aber seine Hauslehrer hatten ihn gut ausgebildet. Draco wusste, dass Potter wahrscheinlich viel mehr Sympathie für Dumbledores Fraktion empfand, als er sich anmerken ließ... obwohl Draco durchaus glaubte, dass Potter in Versuchung geraten könnte.

Aber es war glasklar, dass Potter versuchte, Draco zu verführen, genauso wie Draco versuchte, ihn zu verführen. Und es war auch klar, dass Potter brillant und mehr als nur ein bisschen verrückt war und ein riesiges Spiel spielte, das Potter selbst größtenteils nicht verstand, improvisiert in Höchstgeschwindigkeit mit der Subtilität eines rasenden Nundu.

Aber Potter hatte es geschafft, eine Taktik zu wählen, vor der Draco nicht einfach weglaufen konnte. Er hatte Draco einen Teil seiner eigenen Macht angeboten und darauf gewettet, dass Draco sie nicht nutzen konnte, ohne ihm ähnlicher zu werden.

Sein Vater hatte dies eine fortgeschrittene Technik genannt und Draco gewarnt, dass sie oft nicht funktionierte. Draco wusste, dass er nicht alles verstanden hatte, was passiert war... aber Potter hatte ihm die Chance geboten, zu spielen, und jetzt war es seine Change. Und wenn er mit der ganzen Sache herausplatzte, würde sie Vater gehören.

Am Ende war es so einfach. Die geringeren Techniken erfordern die Unkenntnis des Ziels, oder zumindest seine Unsicherheit. Schmeicheleien müssen plausibel als Bewunderung getarnt sein.

(\emph{„Du hättest in Slytherin sein sollen“} ist ein alter Klassiker, sehr effektiv bei einer bestimmten Art von Person, die es nicht erwartet, und wenn es funktioniert, kann man es wiederholen.)

Aber wenn man den ultimativen Hebel von jemandem findet, ist es egal, ob er weiß, dass man es weiß. Potter hatte in seiner verrückten Eile einen Schlüssel zu Dracos Seele erraten. Und wenn Draco wusste, dass Potter es wusste - selbst wenn es eine offensichtliche Vermutung gewesen war -, änderte das nichts.

Jetzt hatte er also zum ersten Mal in seinem Leben echte Geheimnisse zu bewahren. Er spielte sein eigenes Spiel. Es war ein undurchsichtiger Schmerz, aber er wusste, dass Vater stolz sein würde, und das machte es in Ordnung. Er ließ die Tintentropfen an Ort und Stelle - es gab eine Botschaft, und zwar eine, die sein Vater verstehen würde, denn sie hatten das Spiel der Feinheiten mehr als einmal gespielt -

Draco schrieb die eine Frage auf, die ihn wirklich an der ganzen Angelegenheit genagt hatte, den Teil, von dem es schien, dass er ihn verstehen sollte, aber das tat er nicht, ganz und gar nicht.

Lieber Vater:\\ Angenommen, ich erzähle dir, dass ich in Hogwarts einen Schüler kennengelernt habe, der noch nicht zu unserem Bekanntenkreis gehört, der dich als „lupenreines Instrument des Todes“ bezeichnete und sagte, ich sei deine „einzige Schwachstelle“. Was würdest du über ihn sagen?

Es hatte nicht lange gedauert, bis die Familieneule die Antwort brachte.

Mein geliebter Sohn:\\ Ich würde sagen, dass du das Glück hattest, jemanden zu treffen, der das intime Vertrauen unseres Freundes und wertvollen Verbündeten Severus Snape genießt.

Draco starrte den Brief eine Weile an und warf ihn schließlich ins Feuer.

