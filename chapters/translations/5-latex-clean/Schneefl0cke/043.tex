

\hypertarget{humanismus-teil-2}{% \section{44. Humanismus Teil 2}\label{humanismus-teil-2}}

\textbf{\uline{Humanismus, Teil 2}}

„Fawkes“, sagte Albus Dumbledore mit brüchiger Stimme, „hilf ihm, bitte—“

Ein leuchtendes rot-goldenes Wesen schlurfte ins Blickfeld, schaute fragend zu Boden und begann zu zwitschern. Die bedeutungslosen Zirpen glitten an der Leere ab, es gab nichts, woran sie sich festhalten konnten.

„\textbf{\emph{Du bist laut}}“, sagte die Stimme, „\textbf{\emph{du solltest sterben.}}“

„Schokolade“, sagte Albus Dumbledore, „du brauchst Schokolade, und deine Freunde - aber ich traue mich nicht, dich zurückzunehmen—“

Da kam ein leuchtender Rabe und sprach mit Professor Flitwicks Stimme, woraufhin Albus Dumbledore in plötzlichem Verständnis keuchte und laut über seine eigene Dummheit fluchte.

Das leere Ding lachte darüber, denn es hatte sich die Fähigkeit bewahrt, sich zu amüsieren. Und einen Moment später waren sie alle in einem weiteren Feuerblitz verschwunden.

Es war nur ein Augenblick, so schien es, zwischen dem Zeitpunkt, an dem Flitwicks Rabe zu einem anderen Ort geflogen war, und dem Zeitpunkt, an dem Albus Dumbledore in einem weiteren rot-goldenen Feuerblitz mit Harry in den Armen wieder auftauchte; aber irgendwie hatte Hermine es in dieser Zeit bereits geschafft, ihre Hände mit Schokolade zu füllen. Noch bevor Hermine dazu kam, war die Schokolade vom Tisch und direkt in Harrys Mund gesaust, was ein winziger Teil ihres Verstandes sagte, dass es unfair war, dass er die Chance bekommen hatte, es für sie zu tun - Harry spuckte die Schokolade wieder aus.

„Geh weg“, sagte eine Stimme, die so leer war, dass sie nicht einmal kalt war. …Alles schien zu erstarren, jeder, der sich auf Harry zubewegte, hielt inne, alle Bewegungen wurden durch den Schock dieser zwei toten Worte unterbrochen.

Dann: „Nein“, sagte Albus Dumbledore, „das werde ich nicht“, und die Zeit ging wieder weiter, selbst als ein weiteres Stück Schokolade vom Tisch in Harrys Mund sauste.

Hermine war jetzt nah genug dran, um zu sehen, wie Harrys Gesichtsausdruck immer hasserfüllter wurde, während sein Mund mit einem mechanischen, unnatürlichen Rhythmus kaute.

Die Stimme des Schulleiters war grimmig wie Eisen.

„Filius, rufen Sie Minerva an und sagen Sie ihr, dass sie so schnell wie möglich kommen soll.“

Professor Flitwick flüsterte seinem silbernen Raben zu, und der flog in die Luft und verschwand.

Ein weiteres Stück Schokolade schwebte in Harrys Mund, und das mechanische Kauen ging weiter. Weitere Schüler versammelten sich um die Stelle, an der der Schulleiter mit grimmigem Blick über Harry wachte: Neville, Seamus, Dean, Lavender, Ernie, Terry, Anthony, keiner von ihnen wagte sich näher heran, als Hermine es getan hatte.

„Was können wir tun?“, sagte Dean mit zittriger Stimme.

„Zurücktreten und ihm mehr Raum geben—“ , sagte die trockene Stimme von Professor Quirrell.

„Nein!“, unterbrach der Schulleiter. „Lasst ihn von seinen Freunden umgeben sein.“

Harry schluckte seine Schokolade und sagte mit dieser leeren Stimme:

„\textbf{\emph{Sie sind dumm. Sie sollten stermmppphhh}}“, während ein weiteres Stück Schokolade in seinen Mund wanderte.

Hermine sah die schockierten Blicke, die über ihre Gesichter gingen.

„Er meint es doch nicht ernst, oder?“ Seamus sagte es, als ob er betteln würde.

„Du verstehst nicht“, sagte Hermine, ihre Stimme brach, „das ist nicht Harry—“

und sie hielt den Mund, bevor sie noch mehr sagen konnte, aber so viel musste sie sagen.

Sie sah an seinem Gesichtsausdruck, dass Neville verstand, und sie sah auch, dass die anderen es nicht taten.

\emph{Wenn Harry wirklich nie so etwas gedacht hätte, dann hätte ihn die Tatsache, dass er weniger als eine Minute lang einem Dementor ausgesetzt war, nicht dazu gebracht, es zu sagen.}

Das ist es, was sie wahrscheinlich dachten. Weniger als eine Minute, in der man einem Dementor ausgesetzt war, konnte nicht aus dem Nichts eine neue böse Person in einem erschaffen.

\emph{Aber wenn diese Person schon da war - weiß das der Schulleiter?}

Hermine blickte zum Schulleiter auf und stellte fest, dass Albus Dumbledore sie anschaute und dass seine blauen Augen plötzlich durchdringend geworden waren - Worte kamen ihr in den Sinn.

\textbf{\emph{Sprich nicht davon,}} sagte der Wille von Dumbledore zu ihr.

\emph{Du weißt es,} dachte Hermine. \emph{Über seine dunkle Seite.}

\textbf{\emph{Ja, ich weiß. Aber das hier übersteigt selbst das. Fawkes' Lied kann ihn nicht erreichen, er ist verloren.}}

\emph{Was können wir…}

\textbf{\emph{Ich habe einen Plan, schickte der Schulleiter. Geduld.}}

Irgendetwas am Tenor dieses Gedankens machte Hermine nervös. \emph{Was für einen Plan?}

\textbf{\emph{Es ist besser, wenn du das nicht weißt,}} sagte der Schulleiter.

Jetzt wurde Hermine wirklich nervös. Sie wusste nicht, wie viel der Schulleiter über Harrys dunkle Seite wusste—

\textbf{\emph{Stimmt}}, sagte der Schulleiter. \textbf{\emph{Ich werde es dir jetzt sagen; stähle dich, damit du nicht reagierst. Bist du bereit? Gut.}} \textbf{\emph{Ich werde jetzt so tun, als würde ich Professor McGonagall mit dem Tötungsfluch belegen. Nicht reagieren, Hermine!}}

\emph{Der Schulleiter war wirklich verrückt! Das würde Harry nicht aus seiner dunklen Seite holen, Harry würde völlig durchdrehen, er würde den Direktor umbringen}—

\textbf{\emph{Aber das ist keine wahre Dunkelheit,}} schickte Albus Dumbledore. \textbf{\emph{Das ist Beschützerinstinkt, das ist Liebe. Dann wird Fawkes ihn erreichen können. Und wenn Harry sieht, dass Minerva doch noch am Leben ist, wird sie ihn ganz zurückbringen.}}

Hermine kam der Gedanke—

\textbf{\emph{ich bezweifle, dass das klappt, , und es könnte dir nicht gefallen, wie er reagiert, wenn Sie es versuchen.}} \textbf{\emph{Aber du darfst es versuchen, wenn du willst.}}

\emph{Das hatte sie doch nicht wirklich ernst gemeint! Es war zu—}

Dann bewegten sich ihre Augen, brachen den Blick vom Schulleiter ab und gingen zu dem Jungen, der sich mit leeren, verachtenden Augen umsah, während sein Mund eine Tafel Schokolade nach der anderen ohne Wirkung kaute und schluckte. Ihr Herz klopfte, und plötzlich schien vieles egal zu sein, nur dass es eine Chance gab.

\textbf{\emph{…}}

\textbf{\emph{Es gab den Zwang, Schokolade zu kauen und zu schlucken. Die Reaktion auf Zwang war Tod.}}

Die Leute hatten sich um sie versammelt und starrten sie beide an.

\textbf{\emph{Das war ärgerlich. Die Reaktion auf Verärgerung war das Töten.}}

Andere Leute plapperten im Hintergrund.

\textbf{\emph{Das war unverschämt. Die Antwort auf Anmaßung war, Schmerz zuzufügen, aber da keiner von ihnen nützlich war, wäre es einfacher, sie zu töten. All diese Leute zu töten, wäre schwierig. Aber viele von ihnen vertrauten Quirrell nicht, der stark war. Genau den richtigen Auslöser zu finden, könnte dazu führen, dass sie sich alle gegenseitig umbringen.}}

Dann beugte sich eine Person ins Blickfeld und tat etwas völlig Fremdes, etwas, das zu einer fremden Denkweise gehörte, für die es nur eine einzige Antwort gab, die irgendwo gespeichert war—

Sie hörte das Keuchen um sich herum, aber das war egal, sie hielt den Kuss auf diesen schokoladenverschmierten Lippen aufrecht, während ihr die Tränen in die Augen stiegen.

Harrys Arme kamen hoch und schoben sie weg, und seine Lippen schrien:

\textbf{„Ich habe dir gesagt, nicht Küssen!!!“}

„Ich glaube, jetzt wird er wieder gesund“, sagte der Schulleiter und schaute auf die Stelle, wo Harry in großen, jämmerlichen Schluchzern weinte, während Fawkes über ihn krächzte.

„Hervorragend gemacht, Miss~Granger. Wissen Sie, nicht einmal ich hätte erwartet, dass das tatsächlich funktioniert.“

\emph{Das Lied des Phönix war nicht für sie bestimmt, das wusste Hermine, aber es konnte sie trotzdem besänftigen, was sie auch brauchte, denn ihr Leben war offiziell vorbei.}

