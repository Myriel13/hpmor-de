

\hypertarget{das-stanford-gefuxe4ngnis-experiment-teil-3}{% \section{53. Das Stanford-Gefängnis-Experiment, Teil 3}\label{das-stanford-gefuxe4ngnis-experiment-teil-3}}

\textbf{\uline{Das Stanford-Gefängnis-Experiment Teil 3}}

Die Frauenleiche öffnete ihre Augen, und die trüben, eingesunkenen Augenhöhlen starrten ins Leere.

"Verrückt", murmelte Bellatrix mit brüchiger Stimme, "es scheint, dass die kleine Bella verrückt wird…"

Professor Quirrell hatte Harry ruhig und genau instruiert, wie er sich in Bellatrix' Gegenwart verhalten sollte; wie er den Schein, den er aufrechterhalten würde, in seinem Kopf bilden sollte.

\emph{Du fandest es zweckmäßig, oder vielleicht auch nur amüsant, Bellatrix dazu zu bringen, sich in dich zu verlieben, sie an deine Dienste zu binden. Diese Liebe hätte durch Askaban hindurch Bestand gehabt,} hatte Professor Quirrell gesagt, \emph{denn für Bellatrix wäre es kein glücklicher Gedanke gewesen. Sie liebt dich, ganz und gar, mit ihrem ganzen Wesen. Du erwiderst ihre Liebe nicht, sondern betrachtest sie als nützlich. Das weiß sie. Sie war die tödlichste Waffe, die du hattest, und du nanntest sie deine liebe Bella.}

Harry erinnerte sich an die Nacht, in der der Dunkle Lord seine Eltern tötete: die kalte Belustigung, das verächtliche Lachen, diese hohe Stimme voller tödlichem Hass. Es war gar nicht so schwer zu erraten, was der Dunkle Lord sagen würde.

"Ich hoffe, du bist nicht verrückt, liebe Bella", sagte das kühle Flüstern. "Wahnsinn ist nicht hilfreich."

Bellatrix' Augen flackerten, versuchten, sich auf leere Luft zu konzentrieren.

"Mein… Herr… Ich habe auf Euch gewartet, aber Ihr seid nicht gekommen… Ich habe Euch gesucht, aber ich konnte Euch nicht finden … Ihr seid am Leben …"

Alle ihre Worte kamen in einem leisen Gemurmel heraus, ob Emotionen darin steckten, konnte Harry nicht sagen.

"\textbf{\emph{Zeig ihr dein Gesicht}}", zischte die Schlange zu Harrys Füßen. Harry warf die Kapuze des Unsichtbarkeitsmantels zurück. Der Teil von ihm, dem Harry die Kontrolle über seine Mimik übertragen hatte, sah Bella ohne die geringste Spur von Mitleid an, nur mit kühlem, ruhigem Interesse.

(Während in seinem Inneren Harry dachte: \emph{Ich werde dich retten, ich werde dich retten, egal was passiert…})

"Die Narbe…", murmelte Bellatrix. "Dieses Kind…"

"Das denken alle noch", sagte Harrys Stimme und gab ein dünnes, kleines Glucksen von sich. "Du hast an der falschen Stelle nach mir gesucht, liebe Bella."

(Harry hatte gefragt, warum Professor Quirrell nicht derjenige sein konnte, der die Rolle des Dunklen Lords spielen konnte, und Professor Quirrell hatte darauf hingewiesen, dass es keinen plausiblen Grund dafür gab, dass er von dem Schatten von Er-der-nicht-genannt-werden-muss besessen war.)

Bellatrix' Augen blieben auf Harry fixiert, sie sagte kein Wort.

"\textbf{\emph{Ssag etwassss in Parssel}}", zischte die Schlange.

Harrys Gesicht wandte sich der Schlange zu, um deutlich zu machen, dass er sie ansprach, und zischte: "\emph{Eins zwei drei vier fünf sechs sieben acht neun zehn}."

Es gab eine Pause.

"Diejenigen, die die Dunkelheit nicht fürchten …", murmelte Bellatrix. Die Schlange zischte:

"\textbf{\emph{Werden von ihr verschlungen.}}"

"Werden von ihr verschlungen", flüsterte die kalte Stimme.

Harry wollte nicht unbedingt darüber nachdenken, wie Professor Quirrell an das Passwort gekommen war. Sein Gehirn, das ohnehin darüber nachdachte, schlug vor, dass es wahrscheinlich einen Todesser, einen ruhigen, abgelegenen Ort und etwas Bleirohr-Legilimenz erfordert hatte.

"Dein Zauberstab", murmelte Bellatrix, "ich habe ihn aus dem Haus der Potters mitgenommen und versteckt, mein Herr… unter dem Grabstein rechts neben dem Grab deines Vaters… willst du mich jetzt töten, wenn das alles war, was du dir von mir gewünscht hast… Ich glaube, ich habe mir immer gewünscht, dass du derjenige bist, der mich tötet… aber ich kann mich jetzt nicht mehr erinnern, es muss ein glücklicher Gedanke gewesen sein…"

Harrys Herz zerriss in ihm, es war unerträglich, und - und er konnte nicht weinen, konnte seinen Patronus nicht verblassen lassen - Harrys Gesicht zeigte ein Aufflackern von Verärgerung, und seine Stimme war scharf, als er sagte:

"Genug der Dummheiten. Du wirst mit mir kommen, Bella Liebes, es sei denn, du ziehst die Gesellschaft der Dementoren vor."

Bellatrix' Gesicht zuckte kurz verwirrt, die geschrumpften Gliedmaßen rührten sich nicht.

"\emph{Du musst sie antreiben}", zischte Harry der Schlange zu. "\emph{Sie kann nicht mehr an ein Entkommen denken}."

"\textbf{\emph{Ja}}", zischte die Schlange, "\textbf{\emph{aber unterschätzt sie nicht, ssie war die tödlichste aller}} \textbf{\emph{Kriegerinnen.}}" Der grüne Kopf neigte sich warnend. "\textbf{\emph{Man wäre gut beraten, mich zu fürchten, Junge, selbst wenn ich verhungert und zu neun Zehnteln tot wäre; hüte dich vor ihr, erlaube dir keinen einzigen Fehler in deiner Verstellung.}}"

Die grüne Schlange glitt geschmeidig aus der Tür.

Und kurz darauf stürmte ein Mann mit fahler Haut und einem ängstlichen Ausdruck auf seinem bärtigen Gesicht mit dem Zauberstab in der Hand in den Raum.

"Mein Herr?", sagte der Diener zögernd.

"Tu, was dir aufgetragen wurde", flüsterte der Dunkle Lord mit dieser kalten Stimme, die aus dem Körper eines Kindes noch schrecklicher klang.

"Und lass deinen Patronus nicht schwanken. Denk daran, wenn ich nicht zurückkehre, wird es keine Belohnung für dich geben, und es wird lange dauern, bis deine Familie sterben darf."

Nachdem er diese schrecklichen Worte gesprochen hatte, zog der Dunkle Lord seinen Unsichtbarkeitsumhang über seinen Kopf und verschwand.

Der kriechende Diener öffnete die Tür zu Bellatrix' Käfig und zog eine winzige Nadel aus seinem Gewand, mit der er das menschliche Skelett anstach. Der einzige Tropfen roten Blutes, der dabei entstand, wurde bald in eine kleine Puppe aufgesogen, die auf den Boden gelegt wurde, und der Diener begann flüsternd zu singen. Bald lag ein weiteres lebendes Skelett auf dem Boden, regungslos. Danach schien der Diener einen Moment lang zu zögern, bis aus der leeren Luft ein ungeduldiger Befehl zischte. Dann richtete der Diener seinen Zauberstab auf Bellatrix und sprach ein Wort, und das lebende Skelett, das auf dem Bett lag, war nackt, und das Skelett, das auf dem Boden lag, war mit ihrem verblichenen Kleid bekleidet. Der Diener riss einen kleinen Stoffstreifen von dem Kleid, das auf der scheinbaren Leiche lag, und aus seinen eigenen Gewändern holte der furchtsame Mann dann einen leeren Glaskolben hervor, an dessen Innenseite kleine Spuren einer goldenen Flüssigkeit klebten. Dieses Fläschchen wurde in einer Ecke versteckt, der Rockstreifen darüber gelegt, wobei der ausgelaugte Stoff fast mit der grauen Metallwand verschmolz.

Ein weiterer Wink des Dieners mit dem Zauberstab ließ das menschliche Skelett, das auf dem Bett lag, in die Luft schweben und kleidete es fast in derselben Bewegung in neue schwarze Gewänder. Eine normal aussehende Flasche Schokoladenmilch wurde ihr in die Hand gedrückt, und ein kühles Flüstern befahl Bellatrix, die Flasche zu ergreifen und zu trinken, was sie auch tat, wobei ihr Gesicht immer noch nur verwirrt aussah. Dann machte der Diener Bellatrix unsichtbar, und machte sich selbst unsichtbar, und sie gingen. Die Tür schloss sich hinter ihnen und verriegelte sich mit einem Klicken, wodurch der Korridor wieder in Dunkelheit getaucht wurde, unverändert bis auf einen kleinen Flachmann, der in der Ecke einer Zelle versteckt war, und eine frische Leiche, die auf dem Boden lag.

Vorhin, in dem verlassenen Laden, hatte Professor Quirrell zu Harry gesagt, dass sie das perfekte Verbrechen begehen würden. Harry hatte gedankenlos begonnen, das Standardsprichwort zu wiederholen, dass es so etwas wie ein perfektes Verbrechen nicht gibt, bevor er tatsächlich zwei Drittel einer Sekunde darüber nachdachte, sich an ein weiseres Sprichwort erinnerte und mitten im Satz den Mund hielt.

\emph{Was glaube ich zu wissen, und woher weiß ich es? Wenn ich das perfekte Verbrechen begehen würde, würde es niemand jemals herausfinden - wie könnte also jemand wissen, dass es keine perfekten Verbrechen gibt?}

\emph{Und sobald man es so betrachtete, wurde einem klar, dass perfekte Verbrechen wahrscheinlich ständig begangen werden, und der Gerichtsmediziner vermerkte es als natürlichen Tod, oder die Zeitung berichtete, dass der Laden nie sehr profitabel gewesen war und schließlich das Geschäft aufgegeben hatte.}

Als Bellatrix Blacks Leiche am nächsten Morgen tot in ihrer Zelle gefunden wurde, dort im Gefängnis von Askaban, aus dem (wie jeder wusste) noch nie jemand entkommen war, machte sich niemand die Mühe, eine Autopsie durchzuführen.

Niemand dachte zweimal darüber nach. Sie sperrten einfach den Korridor ab und gingen, und der Tagesprophet berichtete am nächsten Tag in der Nachrufspalte darüber…

\emph{… das war das perfekte Verbrechen, das Professor Quirrell geplant hatte. Und es war nicht Professor Quirrell, der es \textbf{vermasselte}}.

