

\hypertarget{das-unbekannte-und-das-unwissende}{% \section{14. Das Unbekannte und das Unwissende}\label{das-unbekannte-und-das-unwissende}}

\textbf{Das Unbekannte und das Unwissende}

\emph{Es gibt geheimnisvolle Fragen, aber eine geheimnisvolle Antwort ist ein Widerspruch in sich.}

"Herein", sagte Professor McGonagalls gedämpfte Stimme. Harry tat es. Das Büro der stellvertretenden Schulleiterin war sauber und gut organisiert; an der Wand direkt neben dem Schreibtisch befand sich ein Labyrinth aus hölzernen Fächern in allen Formen und Größen, in die meisten waren mehrere Pergamentrollen hineingeschoben, und es war irgendwie sehr klar, dass Professor McGonagall genau wusste, was jedes Fach bedeutete, auch wenn das sonst niemand tat.

Ein einziges Pergament lag auf dem eigentlichen Schreibtisch, der davon abgesehen sauber war. Hinter dem Schreibtisch befand sich eine geschlossene Tür, die mit mehreren Schlössern verriegelt war.

Professor McGonagall saß auf einem lehnenlosen Hocker hinter dem Schreibtisch und sah verwirrt aus - ihre Augen hatten sich geweitet, vielleicht mit einem leichten Anflug von Besorgnis, als sie Harry sah.

"Mr~Potter?", fragte Professor McGonagall. "Worum geht es hier?"

Harrys Verstand wurde leer. Er war von dem Spiel angewiesen worden, hierher zu kommen, er hatte erwartet, dass sie etwas im Sinn hatte…

"Mr~Potter?", sagte Professor McGonagall und begann, leicht genervt auszusehen.

Glücklicherweise erinnerte sich Harrys panisches Gehirn in diesem Moment daran, dass er etwas vorhatte, das er mit Professor McGonagall besprechen wollte.

Etwas Wichtiges, das ihre Zeit wert war.

"Ähm …" sagte Harry. "Wenn es irgendwelche Zauber gibt, die Sie sprechen können, um sicherzustellen, dass uns niemand zuhört…"

Professor McGonagall stand von ihrem Stuhl auf, schloss die Außentür fest und begann, ihren Zauberstab herauszuholen und Zaubersprüche zu sagen.

An diesem Punkt wurde Harry klar, dass er vor einer unbezahlbaren und möglicherweise unersetzlichen Gelegenheit stand, Professor McGonagall einen Comed-Tea anzubieten, und er konnte nicht glauben, dass er das ernsthaft dachte, und es wäre in Ordnung, dass die Limonade nach ein paar Sekunden verschwinden würde, und er sagte diesem Teil von ihm, er solle den Mund halten.

Das tat es und Harry begann, sich mental zu organisieren, was er sagen wollte. Er hatte nicht geplant, diese Diskussion so bald zu führen, aber solange er hier war.

.. Professor McGonagall beendete einen Zauberspruch, der viel älter als Latein klang, und dann setzte sie sich wieder.

"In Ordnung", sagte sie mit ruhiger Stimme. "Keiner hört zu."

Ihr Gesicht war ziemlich angespannt.

\emph{Oh, richtig, sie erwartet, dass ich sie um Informationen über die Prophezeiung erpresse.} Eh, dazu würde Harry an einem anderen Tag kommen.

"Es geht um den Vorfall mit dem Sprechenden Hut", sagte Harry. (Professor McGonagall blinzelte.)

"Ähm… Ich glaube, es gibt einen zusätzlichen Zauber auf dem Sprechenden Hut, etwas, von dem der Sprechende Hut selbst nichts weiß, etwas, das ausgelöst wird, wenn der Sprechende Hut Slytherin sagt.

Ich habe eine Nachricht gehört, die Ravenclaws sicher nicht hören sollen. Sie kam in dem Moment, als der Sprechende Hut von meinem Kopf weg war und ich spürte, wie die Verbindung abbrach.

Es klang wie ein Zischen und gleichzeitig wie Englisch", McGonagall atmete scharf ein,

"und es hieß: Grüße von Slytherin zu Slytherin, wenn du meine Geheimnisse suchst, sprech mit meiner Schlange."

Professor McGonagall saß mit offenem Mund da und starrte Harry an, als wären ihm zwei weitere Köpfe gewachsen.

"Also…"

Professor McGonagall sagte langsam, als könnte sie die Worte, die aus ihren eigenen Lippen kamen, nicht glauben,

"Sie haben sich entschieden, sofort zu mir zu kommen und mir davon zu erzählen."

"Nun, ja, natürlich", sagte Harry.

Er brauchte nicht zuzugeben, wie lange er gebraucht hatte, um tatsächlich auf diesen Gedanken zu kommen.

"Im Gegensatz zu, sagen wir, dem Versuch, es selbst zu erforschen, oder es einem der anderen Kinder zu erzählen."

"Ich … verstehe", sagte Professor McGonagall.

"Und wenn Sie vielleicht den Eingang zu Salazar Slytherins legendärer Kammer des Schreckens entdecken würden, einen Eingang, den Sie und nur Sie allein öffnen könnten…"

"Ich würde den Eingang verschließen und Ihnen sofort Bericht erstatten, damit ein Team von erfahrenen magischen Archäologen zusammengestellt werden kann", sagte Harry prompt.

"Dann würde ich den Eingang wieder öffnen und sie würden sehr vorsichtig hineingehen, um sicherzugehen, dass es nichts Gefährliches gibt.

Ich würde vielleicht später hineingehen, um mich umzusehen, oder wenn sie mich brauchten, um etwas anderes zu öffnen, aber erst, nachdem das Gebiet für frei erklärt worden war und sie Fotos davon hatten, wie alles aussah, bevor die Leute anfingen, auf ihrer unbezahlbaren historischen Stätte herumzutrampeln."

Professor McGonagall saß mit offenem Mund da und starrte ihn an, als hätte er sich gerade in eine Katze verwandelt.

"Es ist offensichtlich, wenn man kein Gryffindor ist", sagte Harry freundlich.

"Ich glaube", sagte Professor McGonagall mit ziemlich erstickter Stimme, "dass Sie die Seltenheit des gesunden Menschenverstandes weit unterschätzen, Mr~Potter."

Das hörte sich ungefähr richtig an. Obwohl…

"Ein Hufflepuff hätte das Gleiche gesagt."

McGonagall hielt verblüfft inne.

"Das ist wahr."

"Der Sortierhut hat mir Hufflepuff angeboten."

Sie blinzelte ihn an, als könne sie ihren eigenen Ohren nicht trauen.

"Hat er wirklich?"

"Ja."

"Mr~Potter", sagte McGonagall, und jetzt war ihre Stimme leise,

"vor fünf Jahrzehnten ist das letzte Mal ein Schüler innerhalb der Mauern von Hogwarts gestorben, und ich bin mir jetzt sicher, dass vor fünf Jahrzehnten auch das letzte Mal jemand diese Nachricht gehört hat."

Ein Schauer durchlief Harry.

"Dann werde ich ganz sicher nichts in dieser Angelegenheit unternehmen, ohne Sie zu konsultieren, Professor McGonagall."

Er hielt inne.

"Und darf ich vorschlagen, dass Sie die besten Leute zusammentrommeln, die Sie finden können, um zu sehen, ob es möglich ist, diesen zusätzlichen Zauber vom Sprechenden Hut zu entfernen .

.. und wenn Sie das nicht schaffen, vielleicht einen anderen Zauber anwenden, einen Quietus, der kurz aktiviert wird, wenn der Hut vom Kopf eines Schülers entfernt wird, das könnte als Flicken funktionieren. So, keine toten Schüler mehr."

Harry nickte zufrieden. Professor McGonagall schaute noch fassungsloser, falls so etwas überhaupt vorstellbar war.

"Ich kann Ihnen unmöglich genug Punkte dafür geben, ohne den Hauspokal gleich an Ravenclaw zu vergeben."

"Ähm", sagte Harry. "Ähm. Ich würde lieber nicht so viele Hauspunkte vergeben."

Jetzt warf Professor McGonagall ihm einen seltsamen Blick zu.

"Warum nicht?"

Harry hatte ein wenig Mühe, es in Worte zu fassen.

"Weil es einfach zu traurig wäre. Wie … wie damals, als ich noch in der Muggelwelt zur Schule gehen wollte, und immer, wenn es ein Gruppenprojekt gab, habe ich das Ganze selbst gemacht, weil die anderen mich nur ausbremsen würden.

Ich habe kein Problem damit, viele Punkte zu sammeln, sogar mehr als alle anderen, aber wenn ich genug sammle, um den Hauspokal allein zu gewinnen, dann ist es, als würde ich das Haus Ravenclaw auf dem Rücken tragen, und das ist zu traurig."

"Ich verstehe…"

sagte McGonagall zögernd. Es war offensichtlich, dass ihr diese Denkweise nie in den Sinn gekommen war.

"Angenommen, ich würde Ihnen nur fünfzig Punkte geben?"

Harry schüttelte wieder den Kopf.

"Es ist den anderen Kindern gegenüber nicht fair, wenn ich viele Punkte für erwachsene Dinge bekomme, bei denen ich mitmachen kann, und sie nicht.

Wie soll Terry Boot fünfzig Punkte dafür bekommen, dass er von einem Geflüster berichtet, das er vom Sprechenden Hut gehört hat? Das wäre überhaupt nicht fair."

"Ich verstehe, warum der Sprechende Hut Ihnen Hufflepuff angeboten hat", sagte Professor McGonagall.

Sie beäugte ihn mit einem seltsamen Respekt. Das ließ Harry ein wenig zusammenzucken. Er hatte wirklich gedacht, er sei Hufflepuff nicht würdig.

Dass der Sprechende Hut gerade versucht hatte, ihn irgendwo anders hin zu schieben als nach Ravenclaw, in ein Haus, dessen Tugenden er nicht besaß.

.. Professor McGonagall lächelte jetzt.

"Und wenn ich versuchen würde, Ihnen zehn Punkte zu geben …?"

"Werden Sie erklären, woher diese zehn Punkte kommen, falls jemand fragt? Es könnte eine Menge Slytherins geben, und damit meine ich nicht die Kinder in Hogwarts, die wirklich sehr wütend wären, wenn sie von der Entzauberung des Sprechenden Hutes wüssten und herausfänden, dass ich daran beteiligt war.

Also denke ich, dass absolute Verschwiegenheit der bessere Teil der Tapferkeit ist. Sie brauchen mir nicht zu danken, Ma'am, Tugend ist ihre eigene Belohnung."

"So sei es", sagte Professor McGonagall,

"aber ich habe Ihnen noch etwas ganz Besonderes zu geben.

Ich sehe, dass ich Ihnen in meinen Gedanken sehr Unrecht getan habe, Mr~Potter. Bitte warten Sie hier."

Sie stand auf, ging hinüber zur verschlossenen Hintertür, schwenkte ihren Zauberstab, und eine Art verschwommener Vorhang entstand um sie herum.

Harry konnte weder sehen noch hören, was vor sich ging. Ein paar Minuten später verschwand der Vorhang und Professor McGonagall stand vor ihm, die Tür hinter ihr sah aus, als wäre sie nie geöffnet worden.

Und Professor McGonagall hielt in der einen Hand eine Halskette, eine dünne goldene Kette, die in ihrer Mitte einen silbernen Kreis trug, in dem sich die Vorrichtung einer Sanduhr befand.

In der anderen Hand hielt sie ein gefaltetes Pamphlet.

"Das ist für Sie", sagte sie.

\emph{Wow! Er würde irgendeinen netten magischen Gegenstand als Belohnung für eine Quest bekommen!

Offenbar funktionierte die Sache mit dem Ablehnen von Geldangeboten, bis man einen magischen Gegenstand bekam, auch im wirklichen Leben, nicht nur in Computerspielen.}

Harry nahm seine neue Halskette lächelnd entgegen. "Was ist es?"

Professor McGonagall holte tief Luft.

"Mr~Potter, dies ist ein Gegenstand, der normalerweise nur an Kinder verliehen wird, die sich bereits als sehr verantwortungsbewusst erwiesen haben, um ihnen bei schwierigen Klassenarbeiten zu helfen."

McGonagall zögerte, als wolle sie noch etwas hinzufügen.

"Ich muss betonen, Mr~Potter, dass die wahre Natur dieses Gegenstandes geheim ist und dass Sie keinem der anderen Schüler davon erzählen oder sie sehen lassen dürfen, wie Sie ihn benutzen.

Wenn das für Sie nicht akzeptabel ist, dann können Sie ihn jetzt zurückgeben."

"Ich kann Geheimnisse bewahren", sagte Harry.

"Und was bewirkt es?"

"Soweit es die anderen Schüler betrifft, ist dies ein Spimster-Wicket und wird zur Behandlung einer seltenen, nicht ansteckenden magischen Krankheit namens Spontane Duplikation verwendet.

Sie tragen es unter Ihrer Kleidung, und obwohl es keinen Grund gibt, es jemandem zu zeigen, gibt es auch keinen Grund, es als schreckliches Geheimnis zu behandeln.

Spimster-Wickets sind nicht interessant. Verstehen Sie das, Mr~Potter?"

Harry nickte, sein Lächeln wurde breiter.

Er spürte die Arbeit eines kompetenten Slytherins.

"Und was macht es wirklich?"

"Es ist ein Zeitumkehrer.

Jede Umdrehung der Sanduhr schickt dich eine Stunde in der Zeit zurück. Wenn man sie also benutzt, um jeden Tag zwei Stunden zurück zu gehen, sollte man immer zur gleichen Zeit einschlafen können."

Harrys Kopf explodierte.

S\emph{ie geben mir eine Zeitmaschine, um meine Schlafstörung zu behandeln.}

\emph{Sie geben mir eine} \textbf{\emph{ZEITMASCHINE}}\emph{, um meine} \textbf{\emph{SCHLAFSTÖRUNG}} \emph{zu behandeln.}

\emph{\hfill\break }\textbf{\emph{SIE GEBEN MIR EINE ZEITMASCHINE, UM MEINE SCHLAFSTÖRUNG ZU BEHANDELN}}\emph{.}

"Ehehehehheheh…"kam es aus Harrys Mund

Er hielt nun die Halskette von sich weg, als wäre sie eine scharfe Bombe. Nun, nein, nicht als wäre es eine scharfe Bombe, das beschrieb nicht ansatzweise den Ernst der Lage.

Harry hielt die Halskette von sich weg, als wäre sie eine Zeitmaschine.

\emph{Professor McGonagall, wussten Sie, dass in der Zeit umgewandelte gewöhnliche Materie wie Antimaterie aussieht? Aber ja, das tut sie!}

\emph{Wussten Sie, dass ein Kilogramm Antimaterie, das auf ein Kilogramm Materie trifft, in einer Explosion vernichtet wird, die 43 Millionen Tonnen TNT entspricht? Ist Ihnen klar, dass ich selbst 41 Kilogramm wiege und dass die resultierende Explosion einen RIESIGEN RAUCHKRATER dort hinterlassen würde, wo früher einmal SCHOTTLAND war?}

"Entschuldigung", schaffte Harry zu sagen, "aber das klingt wirklich, wirklich, wirklich \textbf{GEFÄHRLICH}!"

Harrys Stimme erhob sich nicht ganz zu einem Schrei, er konnte unmöglich laut genug schreien, um dieser Situation gerecht zu werden, also hatte es keinen Sinn, es zu versuchen.

Professor McGonagall sah ihn mit toleranter Zuneigung an.

"Ich bin froh, dass Sie das ernst nehmen, Mr~Potter, aber Zeitdreher sind nicht so gefährlich. Wir würden sie Kindern nicht geben, wenn sie es wären."

"Wirklich", sagte Harry.

"Ahahahaha. Natürlich würdet ihr Kindern keine Zeitmaschinen geben, wenn sie gefährlich wären, was habe ich mir nur dabei gedacht? Also nur um das klarzustellen, das Niesen auf diesem Gerät wird mich nicht ins Mittelalter schicken, wo ich Gutenberg mit einem Pferdewagen überfahre und die Aufklärung verhindere? Ich hasse es nämlich, wenn mir das passiert."

McGonagalls Lippen zuckten auf diese Weise, die sie hatte, wenn sie versuchte, nicht zu lächeln.

Sie bot Harry das Pamphlet an, das sie in der Hand hielt, aber Harry hielt die Halskette vorsichtig mit beiden Händen und starrte auf die Sanduhr, um sicherzugehen, dass sie sich nicht drehen würde.

"Keine Sorge", sagte McGonagall nach einer kurzen Pause, als klar wurde, dass Harry sich nicht bewegen würde, "das kann unmöglich passieren, Mr~Potter. Mit dem Zeitdreher kann man sich nicht mehr als sechs Stunden rückwärts bewegen. Er kann nicht mehr als sechsmal an einem Tag benutzt werden."

"Oh, gut, sehr gut, das ist sehr sehr gut. Und wenn mich jemand anrempelt, wird der Zeitumkehrer nicht zerbrechen und nicht ganz Hogwarts in einer sich endlos wiederholenden Schleife von Donnerstagen gefangen halten."

"Nun, sie können zerbrechlich sein…", sagte McGonagall.

"Und ich glaube, ich habe schon von seltsamen Dingen gehört, die passieren, wenn sie zerbrochen sind. Aber nichts dergleichen!"

"Vielleicht", sagte Harry, als er wieder sprechen konnte,

"sollten Sie Ihre Zeitmaschinen mit einer Art Schutzhülle versehen, anstatt das Glas offen zu lassen, damit so etwas nicht passiert."

McGonagall sah ganz erstaunt aus.

"Das ist eine ausgezeichnete Idee, Mr~Potter. Ich werde das Ministerium darüber informieren."

\emph{Das war's, jetzt ist es offiziell, sie haben es im Parlament ratifiziert, alle in der zaubernden Welt sind völlig verblödet.}

"Und obwohl ich es hasse, philosophisch zu werden", versuchte Harry verzweifelt, seine Stimme auf etwas weniger als einen Schrei zu senken,

"hat irgendjemand über die Konsequenzen nachgedacht, wenn man sechs Stunden zurückgeht und etwas tut, das die Zeit verändert, was so ziemlich ALLE BETROFFENEN PERSONEN LÖSCHEN und sie durch andere Versionen ersetzen würde -"

"Oh, Sie können die Zeit nicht ändern!" unterbrach Professor McGonagall.

"Gütiger Himmel, Mr~Potter, glauben Sie, dass dieses Gerät zugelassen würden, wenn das möglich wäre? Was wäre, wenn jemand versuchen würde, Testergebnisse zu ändern?"

Harry brauchte einen Moment, um dies zu verarbeiten.

Seine Hände entspannten sich, nur ein wenig, von ihrem weißen Griff um die Sanduhrkette. Als ob er keine Zeitmaschine in der Hand hätte, sondern nur einen scharfen Atomsprengkopf.

"Also …" sagte Harry langsam.

"Die Leute finden einfach, dass das Universum… irgendwie selbstkonsistent ist, auch wenn es Zeitreisen gibt.

Wenn ich und mein zukünftiges Ich interagieren, dann sehe ich dasselbe wie beide, obwohl mein zukünftiges Ich bei meinem ersten Durchlauf bereits in voller Kenntnis von Dingen handelt, die aus meiner eigenen Perspektive noch nicht passiert sind…"

Harrys Stimme versank in der Unzulänglichkeit des Englischen.

"Korrekt, denke ich", sagte Professor McGonagall.

"Allerdings wird Zauberern geraten, es zu vermeiden, von ihren früheren Ichs gesehen zu werden. Wenn du zum Beispiel zwei Klassen gleichzeitig besuchst und dir selbst über den Weg laufen musst, sollte die erste Version von dir zur Seite treten und zu einer bekannten Zeit die Augen schließen - du hast ja schon eine Uhr, gut -, damit das zukünftige Du vorbeigehen kann.

Steht alles in dem Pamphlet."

"Ahahahaa. Und was passiert, wenn jemand diesen Ratschlag ignoriert?"

Professor McGonagall schürzte die Lippen.

"Ich habe gehört, dass das ziemlich beunruhigend sein kann."

"Und es erzeugt nicht, sagen wir, ein Paradoxon, das das Universum vernichtet."

Sie lächelte tolerant.

"Mr~Potter, ich glaube, ich würde mich daran erinnern, wenn das jemals passiert wäre."

"DAS IST NICHT BERUHIGEND!

HABT IHR LEUTE DENN NOCH NIE ETWAS VOM ANTHROPISCHEN PRINZIP GEHÖRT? \textbf{UND WELCHER IDIOT HAT JEMALS SO EIN DING ZUM ERSTEN MAL GEBAUT}?!!!"

Professor McGonagall lachte tatsächlich.

Es war ein angenehmes, fröhliches Geräusch, das in diesem strengen Gesicht erstaunlich fehl am Platz wirkte.

"Sie haben einen weiteren '\emph{Sie haben sich in eine Katze verwandelt'}-Moment, nicht wahr, Mr~Potter. Sie wollen das wahrscheinlich nicht hören, aber es ist ziemlich liebenswert und niedlich."

"Sich in eine Katze zu verwandeln, ist nicht annähernd damit zu vergleichen.

Wissen Sie, bis zu diesem Moment hatte ich irgendwo im Hinterkopf diesen schrecklichen, unterdrückten Gedanken, dass die einzige verbleibende Antwort wäre, dass mein ganzes Universum eine Computersimulation ist, wie in dem Buch Simulacron 3, aber jetzt ist selbst das ausgeschlossen, weil dieses kleine Spielzeug NICHT TURING KOMPATIBEL ist!

Eine Turing-Maschine könnte simulieren, in einen definierten Moment der Vergangenheit zurückzugehen und von dort aus eine andere Zukunft zu berechnen, eine Orakelmaschine könnte sich auf das Verhalten von Maschinen niedrigerer Ordnung verlassen, aber was Sie sagen, ist, dass die Realität irgendwie selbstkonsistent in einem Schwung rechnet, indem sie Informationen verwendet, die … noch nicht … geschehen sind…"

Die Erkenntnis traf Harry wie ein Hammerschlag. Es machte jetzt alles Sinn.

Endlich ergab alles einen Sinn.

"SO FUNKTIONIERT ALSO DER COMMED-TEE!

Ja, natürlich! Der Zauber erzwingt keine komischen Ereignisse, er bewirkt nur, dass man einen Drang verspürt, zu trinken, kurz bevor sowieso komische Dinge passieren werden!

Ich bin so ein Narr, ich hätte es merken müssen, als ich den Impuls verspürte, den Tee vor Dumbledores zweiter Rede zu trinken, ihn nicht trank und stattdessen an meinem eigenen Speichel erstickte - das Trinken des Comed-Tee verursacht nicht die Komik, die Komik verursacht, dass man den Kometentee trinkt!

Ich sah, dass die beiden Ereignisse korreliert waren und nahm an, dass der Comed-Tee die Ursache und die Komödie die Wirkung sein musste, weil ich dachte, dass die zeitliche Ordnung die Kausalität einschränkt und dass Kausaldiagramme azyklisch sein müssen,

aber es macht alles Sinn wenn die Kausalpfeile in der Zeit rückwärts zeigen!"

Die Erkenntnis traf Harry der zweite Rammbock. Diesmal schaffte er es, ruhig zu bleiben und gab nur ein kleines würgendes Geräusch wie ein sterbendes Kätzchen von sich, als ihm klar wurde, wer den Zettel heute Morgen auf sein Bett gelegt hatte.

Die Augen von Professor McGonagall leuchteten.

"Nach Ihrem Abschluss, oder vielleicht sogar schon vorher, müssen Sie wirklich einige dieser Muggel-Theorien in Hogwarts unterrichten, Mr~Potter. Sie klingen ziemlich faszinierend, auch wenn sie alle falsch sind."

"\emph{Glehhahhh}…"

Professor McGonagall bot ihm noch ein paar Höflichkeiten

an, verlangte noch ein paar Versprechen, zu denen Harry nickte, sagte etwas darüber, \emph{dass er nicht mit Schlangen sprechen sollte, wo ihn jemand hören konnte}, erinnerte ihn daran, das Pamphlet zu lesen, und dann fand sich Harry irgendwie vor ihrem Büro stehend wieder, die Tür fest hinter sich geschlossen.

"\emph{Gaahhhrrrraa}…" sagte Harry.

Aber ja, er war völlig verblüfft. Nicht zuletzt durch die Tatsache, dass er ohne den Streich vielleicht gar nicht erst einen Zeitdreher bekommen hätte.

Oder hätte Professor McGonagall ihn ihm trotzdem gegeben, nur später am Tag, wenn er dazu gekommen wäre, nach seiner Schlafstörung zu fragen oder ihr von der Nachricht des Sprechenden Hutes zu erzählen? Und hätte er sich zu diesem Zeitpunkt einen Streich spielen wollen, der dazu geführt hätte, dass er den Zeitdreher früher bekommen hätte? So dass die einzige in sich schlüssige Möglichkeit die war, in der der Streich begann, bevor er am Morgen überhaupt aufwachte? Harry erwog zum ersten Mal in seinem Leben, dass die Antwort auf seine Frage buchstäblich unvorstellbar sein könnte.

Da sein eigenes Gehirn Neuronen enthielt, die nur in der Zeit vorwärts liefen, gab es nichts, was sein Gehirn tun konnte, keine Operation, die es durchführen konnte, die mit der Operation eines Zeitdrehers konjugiert war.

Bis zu diesem Punkt hatte Harry nach der Ermahnung von E. T. Jaynes gelebt, dass, wenn man über ein Phänomen unwissend war, dies eine Tatsache über den eigenen Geisteszustand war, nicht eine Tatsache über das Phänomen selbst; dass die eigene Ungewissheit eine Tatsache über einen selbst war, nicht eine Tatsache über das, worüber man unsicher war; dass Unwissenheit im Geist existierte, nicht in der Realität; dass eine leere Landkarte nicht einem leeren Gebiet entsprach.

Es gab mysteriöse Fragen, aber eine mysteriöse Antwort war ein Widerspruch in sich.

Ein Phänomen konnte für eine bestimmte Person mysteriös sein, aber es konnte keine Phänomene geben, die an sich mysteriös waren.

\emph{Ein heiliges Mysterium anzubeten, bedeutete nur, die eigene Unwissenheit anzubeten.}

So hatte Harry die Magie betrachtet und sich nicht einschüchtern lassen.

Die Menschen hatten keinen Sinn für Geschichte, sie lernten über Chemie und Biologie und Astronomie und dachten, dass diese Dinge schon immer das eigentliche Fleisch der Wissenschaft gewesen waren, dass sie niemals mysteriös gewesen waren.

Die Sterne waren einst Mysterien gewesen. Lord Kelvin hatte einmal die Natur des Lebens und der Biologie - die Reaktion der Muskeln auf den menschlichen Willen und die Erzeugung von Bäumen aus Samen - als ein Mysterium bezeichnet, das \emph{"unendlich jenseits"} der Reichweite derWissenschaft lag.

(Nicht nur ein wenig darüber hinaus, wohlgemerkt, sondern unendlich weit darüber hinaus. Lord Kelvin fühlte sicherlich eine enorme emotionale Belastung davon etwas nicht zu wissen.)

Jedes Mysterium, das jemals gelöst wurde, war ein Rätsel gewesen, von den Anfängen der menschlichen Spezies an bis zu dem Zeitpunkt, an dem es jemand gelöst hatte.

Jetzt stand er zum ersten Mal vor einem Rätsel, das dauerhaft zu sein drohte. Wenn die Zeit nicht durch azyklische kausale Netzwerke funktionierte, dann verstand Harry nicht, was mit Ursache und Wirkung gemeint war; und wenn Harry Ursachen und Wirkungen nicht verstand, dann verstand er nicht, aus was für einem Zeug die Realität stattdessen bestehen mochte; und es war durchaus möglich, dass sein menschlicher Verstand das nie verstehen konnte, weil sein Gehirn aus altmodischen Linear-Zeit-Neuronen bestand, und das hatte sich als eine verarmte Teilmenge der Realität herausgestellt.

Auf der positiven Seite hatte sich herausgestellt, dass der Comed-Tea, der einst allmächtig und unglaubwürdig erschien, eine viel einfachere Erklärung hatte. Die er nur deshalb übersehen hatte, weil die Wahrheit völlig außerhalb seines Hypothesenraums lag oder außerhalb dessen, was sein Gehirn entwickelt hatte, um es zu begreifen.

Aber jetzt hatte er es tatsächlich herausgefunden, wahrscheinlich. Was irgendwie ermutigend war. Irgendwie.

Harry warf einen Blick auf seine Uhr. Es war fast 11 Uhr, er war letzte Nacht um 1 Uhr schlafen gegangen, also würde er heute Nacht um 3 Uhr schlafen gehen.

Um um 22 Uhr schlafen zu gehen und um 7 Uhr aufzuwachen, müsste er also insgesamt fünf Stunden zurückgehen.

Was bedeutete, dass er sich besser beeilen sollte, wenn er gegen 6 Uhr morgens zurück in seinem Schlafsaal sein wollte, bevor irgendjemand wach war, und…

Selbst im Nachhinein verstand Harry nicht, wie er die Hälfte der Dinge, die mit dem Streich zu tun hatten, geschafft hatte.

Wo war der Kuchen hergekommen? Harry begann, ernsthaft Angst vor Zeitreisen zu haben. Andererseits musste er zugeben, dass es eine unersetzliche Gelegenheit gewesen war.

Ein Streich, den man sich nur einmal im Leben erlauben konnte, und zwar innerhalb von sechs Stunden, nachdem man das erste Mal von Zeitdrehern erfahren hatte.

Tatsächlich war das sogar noch rätselhafter, wenn Harry darüber nachdachte. Die Zeit hatte ihn mit dem fertigen Streich vor vollendete Tatsachen gestellt, und doch war es ganz klar sein eigenes Werk.

Konzept und Ausführung und Schreibstil. Jeder einzelne Teil, auch die, die er noch nicht verstand.

Nun, die Zeit drängte, und ein Tag hatte höchstens dreißig Stunden. Harry wusste einiges von dem, was er zu tun hatte, und den Rest, wie den Kuchen, würde er vielleicht herausfinden, während er arbeitete.

Es hatte keinen Sinn, es aufzuschieben. Er konnte nicht wirklich etwas erreichen, wenn er hier in der Zukunft festsaß.

Fünf Stunden schlich Harry in seinen Schlafsaal, die Roben über den Kopf gezogen, als eine Art einfache Verkleidung, nur für den Fall, dass jemand schon auf den Beinen war und ihn zur gleichen Zeit sah, wie Harry in seinem Bett lag.

Er wollte niemandem sein kleines medizinisches Problem mit der Spontanverdopplung erklären müssen.

Glücklicherweise schienen alle noch zu schlafen. Und es schien auch eine Schachtel, eingewickelt in rotes und grünes Papier mit einer leuchtend goldenen Schleife, neben seinem Bett zu liegen.

Das perfekte, klischeehafte Bild eines Weihnachtsgeschenks, obwohl es nicht Weihnachten war. Harry schlich sich so leise an, wie es ihm möglich war, nur für den Fall, dass jemand seinen Schweigezauber leise gedreht hatte.

An der Schachtel war ein Umschlag befestigt, der mit einfachem, klarem Wachs ohne eingeprägtes Siegel verschlossen war.

Harry riss den Umschlag vorsichtig auf und nahm den Brief heraus. Auf dem Brief stand:

\emph{Dies ist der Unsichtbarkeitsumhang von Ignotus Peverell, vererbt durch seine Nachkommen, den Potters. Im Gegensatz zu anderen Mänteln und Zaubern hat er die Macht, dich verborgen zu halten, nicht nur unsichtbar.

Dein Vater hat ihn mir kurz vor seinem Tod zum Studium geliehen, und ich muss gestehen, dass ich im Laufe der Jahre viel Nutzen daraus gezogen habe.

In Zukunft werde ich mit der Desillusionierung auskommen müssen, fürchte ich. Es ist an der Zeit, dass der Umhang an dich, seinen Erben, zurückgegeben wird.}

\emph{\hfill\break Ich hatte gedacht, dies als Weihnachtsgeschenk zu machen, aber er wollte vorher in deine Hand zurückkehren. Er scheint zu erwarten, dass du ihn brauchst. Verwende ihn gut. Zweifellos denkst du bereits an allerlei wunderbare Streiche, wie sie dein Vater zu seiner Zeit begangen hat.}

\emph{\hfill\break Wären seine Missetaten bekannt, würde sich jede Frau in Gryffindor versammeln, um sein Grab zu schänden.}

\emph{\hfill\break Ich werde nicht versuchen, die Wiederholung der Geschichte zu verhindern, aber pass auf, dass du dich nicht verrätst.

Wenn Dumbledore eine Chance sähe, eines der Heiligtümer des Todes zu besitzen, würde er es bis zu seinem Tod nicht aus den Händen geben.}

\emph{\hfill\break Ich wünsche dir frohe Weihnachten.}

Der Zettel war nicht unterschrieben.

Später.

"Moment", sagte Harry und hielt kurz inne, als die anderen Jungen den Ravenclaw-Schlafsaal verlassen wollten. "Tut mir leid, ich muss noch etwas mit meinem Koffer erledigen. Ich komme in ein paar Minuten zum Frühstück."

Terry Boot warf Harry einen finsteren Blick zu.

"Du hast hoffentlich nicht vor, unsere Sachen zu durchwühlen."

Harry hielt eine Hand hoch.

"Ich schwöre, dass ich nichts dergleichen mit euren Sachen vorhabe, dass ich nur auf Gegenstände zugreifen will, die mir selbst gehören, dass ich keine Streiche oder anderweitig fragwürdige Absichten gegenüber einem von euch hege und dass ich nicht erwarte, dass sich diese Absichten ändern, bevor ich zum Frühstück in die Große Halle komme."

Terry runzelte die Stirn. "Warte, ist das -"

"Keine Sorge", sagte Penelope Clearwater, die dabei war, um sie zu führen. "Es gab keine Schlupflöcher. Gut formuliert, Potter, du solltest Anwalt werden."

Harry Potter blinzelte daraufhin. Ah, ja, Ravenclaw-Vertrauensschüler.

"Danke", sagte er. "Aber ich denke nicht."

"Wenn du versuchst, die Große Halle zu finden, wirst du dich verirren."

Penelope sagte das im Tonfall einer nüchternen, unumstößlichen Tatsache.

"Sobald du das tust, frag ein Porträt, wie du in den ersten Stock kommst. Frag ein anderes Porträt, sobald du den Verdacht hast, dass du dich wieder verirren könntest.

Vor allem, wenn es scheint, als würdest du immer höher hinaufsteigen. Wenn du höher bist, als das ganze Schloss sein sollte, halte an und warten auf Suchtrupps.

Sonst sehen wir dich vier Monate später wieder, und du wirst fünf Monate älter sein und einen Lendenschurz tragen und mit Schnee bedeckt sein, und das ist, wenn du im Schloss bleibst."

"Verstanden", sagte Harry und schluckte schwer. "Ähm, solltest du den Schülern so etwas nicht gleich sagen?"

Penelope seufzte. "Was, alles? Das würde Wochen dauern. Sie werden es nach und nach lernen."

Sie wandte sich zum Gehen, gefolgt von den anderen Schülern.

"Wenn ich dich nicht in dreißig Minuten beim Frühstück sehe, Potter, werde ich mit der Suche beginnen."

Als alle weg waren, befestigte Harry den Zettel an seinem Bett - er hatte ihn und alle anderen Zettel bereits geschrieben und in seinem Koffer gearbeitet, bevor alle anderen aufgewacht waren.

Dann griff er vorsichtig in das Quietus-Feld und zog den Unsichtbarkeitsumhang von Harry 1 immer noch schlafendem Körper ab.

Und nur um Unfug zu treiben, steckte Harry den Umhang in Harry-1s Tasche, wohl wissend, dass er dadurch bereits in seiner eigenen sein würde.

"Ich kann dafür sorgen, dass die Nachricht an Cornelion Flubberwalt weitergegeben wird", sagte das Bild eines Mannes mit aristokratischen Allüren und eigentlich einer ganz normalen Nase.

"Aber darf ich fragen, woher sie ursprünglich stammt?"

Harry zuckte mit kunstvoller Hilflosigkeit mit den Schultern.

"Mir wurde gesagt, dass es von einer hohlen Stimme gesprochen wurde, die aus einer Lücke in der Luft selbst hervorbrach, einer Lücke, die sich zu einem feurigen Abgrund öffnete."

"Hey!", sagte Hermine in einem Ton der Entrüstung von ihrem Platz auf der anderen Seite des Frühstückstisches.

"Das ist der Nachtisch von allen! Du kannst nicht einfach einen ganzen Kuchen nehmen und ihn in deinen Beutel stecken!"

"Ich nehme nicht nur einen Kuchen, \emph{ich nehme zwei.} Tut mir leid, Leute, ich muss jetzt los!"

Harry ignorierte die Aufschreie der Empörung und verließ die Große Halle.

Er musste ein wenig früher zum Kräuterkunde Unterricht kommen. Professor Sprout warf ihm einen scharfen Blick zu.

"Und woher wissen Sie, was die Slytherins planen?"

"Ich kann meine Quelle nicht nennen", sagte Harry.

"Ich muss Sie sogar bitten, so zu tun, als hätte dieses Gespräch nie stattgefunden. Tu einfach so, als wären Sie zufällig auf sie gestoßen, während Sie eine Besorgung gemacht haben, oder so.

Ich laufe voraus, sobald Kräuterkunde fertig ist. Ich denke, ich kann die Slytherins ablenken, bis Sie da sind.

Ich bin nicht leicht einzuschüchtern. Sie werden es nicht wagen, den Jungen, der lebte, ernsthaft zu verletzen.

Obwohl… Ich verlange ja nicht, dass Sie durch die Gänge rennen, aber ich würde es begrüßen, wenn Sie auf dem Weg dorthin nicht trödeln würden."

Professor Sprout sah ihn einen langen Moment lang an, dann wurde ihre Miene weicher.

"Bitte seien Sie vorsichtig mit sich selbst, Harry Potter. Und … danke."

"Pass nur auf, dass Sie nicht zu spät kommen", sagte Harry.

"Und denken Sie daran, wenn Sie dort ankommen, haben Sie nicht erwartet, mich zu sehen, und dieses Gespräch hat nie stattgefunden."

Es war schrecklich, sich selbst dabei zuzusehen, wie er Neville aus dem Kreis der Slytherins herauszerrte.

Neville hatte recht gehabt, er hatte zu viel Gewalt angewendet, viel zu viel Gewalt.

"Hallo", sagte Harry Potter kalt. „Ich bin der Junge-der-lebte.“

Acht Jungen im ersten Jahr, meist gleich groß. Einer von ihnen hatte eine Narbe auf der Stirn und verhielt sich nicht wie die anderen.

\emph{Wäre es doch eine Gabe, uns so zu sehen, wie andere uns sehen! Es wäre wohl Möglich uns zu befreien, und törichte Vorstellung -}

Professor McGonagall hatte recht. Der Sprechende Hut hatte recht. Es war klar, sobald man es von außen sah.

\emph{Mit Harry Potter stimmte etwas nicht.}

\emph{Anm. des Übersetzers: Nun, dies war also die Lösung des Rätsel. Hinweise darauf sind im ganzen vorherigen Text verstreut, unter anderem schon im ersten Kapitel als McGonagall sagt „Ich werde mit der Zeit eine Lösung finden“. Ich hoffe ihr habt ein bisschen gegrübelt :-)}

