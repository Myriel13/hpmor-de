

\hypertarget{ein-uxe4uuxdferst-unwahrscheinlicher-tag}{% \section{2. Ein äußerst unwahrscheinlicher Tag}\label{ein-uxe4uuxdferst-unwahrscheinlicher-tag}}

\uline{Hinweiß des Autors E.Y.}: Der Text enthält viele Hinweise: offensichtliche Hinweise, nicht so offensichtliche Hinweise, wirklich obskure Hinweise, von denen ich schockiert war, dass sie von einigen Lesern erfolgreich entschlüsselt wurden, und massive Hinweise, die einfach liegen gelassen wurden. Dies ist eine rationalistische Geschichte; ihre Rätsel sind lösbar und dazu gedacht, gelöst zu werden.

Das Tempo der Geschichte ist das einer seriellen Fiktion, d. h. das einer Fernsehserie, die über eine bestimmte Anzahl von Staffeln läuft, deren Episoden einzeln gezeichnet sind, aber mit einem Gesamtbogen, der auf ein endgültiges Ende hinführt.

Alle erwähnte Wissenschaft ist echte Wissenschaft. Bitte bedenken Sie aber, dass die Ansichten der Charaktere außerhalb des Bereichs der Wissenschaft nicht unbedingt die des Autors sind.

Nicht alles, was der Protagonist tut, ist eine Lektion in Weisheit, und Ratschläge, die von dunkleren Charakteren angeboten werden, können unzuverlässig oder gefährlich zweischneidig sein.

\emph{Und damit wollen wir beginnen….}

\emph{Unter dem Mondlicht schimmert ein winziges Silberfragment, ein Bruchteil einer Linie… (fallende schwarze Roben) … Blut schwappt literweise heraus, und jemand schreit ein Wort.}

Jeder Zentimeter Wandfläche ist von einem Bücherregal bedeckt.

Jedes Bücherregal hat sechs Fächer, die fast bis zur Decke reichen. Einige Bücherregale sind bis zum Rand mit gebundenen Büchern bestückt: Naturwissenschaften, Mathematik, Geschichte und alles andere.

Andere Regale sind zweilagig mit Science-Fiction-Taschenbüchern bestückt, wobei die hintere Lage der Bücher auf alten Papiertaschentüchern oder Holzstücken ruht, so dass man die hintere Lage der Bücher über den vorderen Büchern sehen kann.

Und es ist immer noch nicht genug. Die Bücher quellen über auf die Tische und die Sofas und bilden kleine Haufen unter den Fenstern.

Dies ist das Wohnzimmer des Hauses, in dem der angesehene Professor Michael Verres-Evans und seine Frau, Mrs. Petunia Evans-Verres, und ihr Adoptivsohn Harry James Potter-Evans-Verres leben. Auf dem Wohnzimmertisch liegen ein Brief und ein unfrankierter Umschlag aus gelblichem Pergament, adressiert an Mr. H. Potter in smaragdgrüner Tinte. Der Professor und seine Frau sprechen sich scharf an, aber sie schreien nicht.

Der Professor hält Schreien für unzivilisiert.

"Du machst Witze", sagte Michael zu Petunia. Sein Ton verriet, dass er sehr befürchtete, dass sie es ernst meinte.

"Meine Schwester war eine Hexe", wiederholte Petunia. Sie sah erschrocken aus, blieb aber standhaft. "Ihr Mann war ein Zauberer."

"Das ist absurd!" sagte Michael scharf. "Sie waren auf unserer Hochzeit - sie haben uns zu Weihnachten besucht -"

"Ich habe ihnen gesagt, dass du es nicht wissen darfst", flüsterte Petunia.

"Aber es ist wahr. Ich habe Dinge gesehen -"

Der Professor rollte mit den Augen.

"Liebes, ich verstehe, dass du mit der skeptischen Literatur nicht vertraut bist.

Vielleicht ist dir nicht klar, wie leicht es für einen ausgebildeten Zauberer ist, das scheinbar Unmögliche vorzutäuschen. Weißt du noch, wie ich Harry beibrachte, Löffel zu verbiegen? Wenn es so aussah, als könnten sie immer erraten, was du denkst, nennt man das Cold Reading -"

"Es war nicht das Verbiegen von Löffeln -"

"Was war es dann?"

Petunia biss sich auf die Lippe. "Ich kann es dir nicht einfach sagen. Du wirst denken, dass ich -"

Sie schluckte. "Hör zu. Michael. Ich war nicht - immer so -"

Sie deutete auf sich selbst, als wolle sie ihre schöne Gestalt andeuten.

"Lily hat das getan. Weil ich - weil ich sie angefleht habe. Jahrelang habe ich sie angefleht. Lily war immer hübscher als ich, und ich… war deswegen gemein zu ihr, und dann bekam sie Magie, kannst du dir vorstellen, wie ich mich fühlte? Und ich habe sie angefleht, etwas von dieser Magie auf mich anzuwenden, damit ich auch hübsch sein kann, auch wenn ich ihre Magie nicht haben kann, so kann ich wenigstens hübsch sein." Tränen sammelten sich in Petunias Augen.

"Und Lily sagte immer nein und erfand die lächerlichsten Ausreden, wie dass die Welt untergehen würde, wenn sie nett zu ihrer Schwester wäre, oder dass ein Zentaur ihr sagte, sie solle es nicht tun - die lächerlichsten Dinge, und ich hasste sie dafür.

Und als ich gerade meinen Abschluss gemacht hatte, ging ich mit diesem Jungen aus, Vernon Dursley, er war fett und er war der einzige Junge, der mit mir reden wollte.

Und er sagte, er wolle Kinder und dass sein erster Sohn Dudley heißen würde. Und ich dachte mir, was für Eltern nennen ihr Kind Dudley Dursley? Es war, als sähe ich mein ganzes zukünftiges Leben vor mir ausgebreitet, und ich konnte es nicht ertragen.

Und ich schrieb meiner Schwester und sagte ihr, wenn sie mir nicht helfen würde, würde ich lieber -"

Petunia hielt inne.

"Jedenfalls", sagte Petunia mit leiser Stimme,

"hat sie nachgegeben. Sie sagte mir, es sei gefährlich, und ich sagte, es sei mir egal, und ich trank diesen Trank und war wochenlang krank, aber als es mir besser ging, wurde meine Haut klarer und ich wurde endlich satt und… ich war schön, die Leute waren nett zu mir", ihre Stimme brach,

"und danach konnte ich meine Schwester nicht mehr hassen, besonders als ich erfuhr, was ihre Magie ihr am Ende gebracht hat -"

"Liebling", sagte Michael sanft,

"du wurdest krank, du hast etwas zugenommen, während du im Bett lagst, und deine Haut hat sich von selbst aufgehellt. Oder die Krankheit hat dich dazu gebracht, deine Ernährung umzustellen -"

"Sie war eine Hexe", wiederholte Petunia. "Ich habe es gesehen."

"Petunia", sagte Michael. Die Verärgerung schlich sich in seine Stimme.

"Du weißt, dass das nicht wahr sein kann. Muss ich wirklich erklären, warum?"

Petunia rang die Hände. Sie schien den Tränen nahe zu sein.

"Meine Liebster, ich weiß, dass ich keinen Streit mit dir gewinnen kann, aber bitte, du musst mir in dieser Sache vertrauen -"

"Dad!

Mum! "

Die beiden blieben stehen und sahen Harry an, als hätten sie vergessen, dass noch eine dritte Person im Raum war.

Harry holte tief Luft.

"Mum, deine Eltern hatten keine Magie, oder?"

"Nein", sagte Petunia und sah verwirrt aus.

"Dann wusste niemand in deiner Familie von Magie, als Lily ihren Brief bekam. Wie konnten sie überzeugt werden?"

"Ah…" sagte Petunia.

"Sie haben nicht nur einen Brief geschickt. Sie schickten einen Professor aus Hogwarts.

Er -"

Petunias Augen flackerten zu Michael.

"Er hat uns etwas Magie gezeigt."

"Dann müsst ihr euch nicht darum streiten", sagte Harry fest.

Er hoffte inständig, dass sie diesmal, nur dieses eine Mal, auf ihn hören würden.

"Wenn es wahr ist, können wir einfach einen Hogwarts-Professor herholen und die Magie mit eigenen Augen sehen, und Dad wird zugeben, dass es wahr ist.

Und wenn nicht, dann wird Mum zugeben, dass es falsch ist. Dafür ist die experimentelle Methode ja da, damit wir die Dinge nicht nur durch Streitereien klären müssen."

Der Professor drehte sich um und schaute auf ihn herab, abweisend wie immer.

"Ach, komm schon, Harry. Wirklich, Magie? Ich dachte, du würdest es besser wissen, als das ernst zu nehmen, mein Sohn, auch wenn du erst zehn bist.

Magie ist so ziemlich die unwissenschaftlichste Sache, die es gibt!"

Harrys Mund verzog sich bitterlich.

Er wurde gut behandelt, wahrscheinlich besser, als die meisten genetischen Väter ihre eigenen Kinder behandelten.

Harry war auf die besten Grundschulen geschickt worden - und wenn das nicht klappte, hatte man ihm Nachhilfelehrer aus dem endlosen Pool der hungernden Studenten zur Verfügung gestellt.

Harry war immer ermutigt worden, alles zu studieren, was seine Aufmerksamkeit erregte, er kaufte alle Bücher, die ihn interessierten, er wurde gesponsert, egal an welchem Mathematik- oder Wissenschaftswettbewerb er teilnahm.

Man gab ihm alles Vernünftige, was er wollte, außer vielleicht das kleinste Fünkchen Respekt. Von

einem Doktor, der in Oxford Biochemie lehrte, konnte man kaum erwarten, dass er auf den Rat eines kleinen Jungen hörte.

Man würde zuhören, um Interesse zu zeigen, natürlich; das ist es, was ein guter Elternteil tun würde, und so, wenn man sich selbst als guter Elternteil begriff, würde man es tun.

Aber einen Zehnjährigen ernst nehmen? Wohl kaum. Manchmal wollte Harry seinen Vater anschreien.

"Mum", sagte Harry.

"Wenn du diesen Streit mit Dad gewinnen willst, dann schau im zweiten Kapitel des ersten Buches der Feynman-Vorlesungen über Physik nach.

Da gibt es ein Zitat darüber, wie Philosophen viel darüber sagen, was Wissenschaft unbedingt erfordert, und das ist alles falsch, denn die einzige Regel in der Wissenschaft ist, dass der letzte Schiedsrichter die Beobachtung ist - dass man sich einfach die Welt ansehen und berichten muss, was man sieht.

Ähm … mir fällt spontan nichts ein, wo ich etwas darüber finden könnte, dass es ein Ideal der Wissenschaft ist, Dinge durch Experimente statt durch Argumente zu klären -"

Seine Mutter schaute auf ihn herab und lächelte.

"Danke, Harry. Aber -"

ihr Kopf hob sich wieder, um ihren Mann anzustarren.

"Ich will keinen Streit mit deinem Vater gewinnen.

Ich möchte, dass mein Mann, dass er auf seine Frau hört, die ihn liebt, und ihr vertraut, nur dieses eine Mal -"

Harry schloss kurz die Augen.

Hoffnungslos. Seine beiden Eltern waren einfach hoffnungslos. Jetzt gerieten seine Eltern wieder in einen dieser Streitereien, in denen seine Mutter versuchte, seinem Vater ein schlechtes Gewissen einzureden, und sein Vater versuchte, seiner Mutter ein schlechtes Gewissen einzureden.

"Ich werde auf mein Zimmer gehen", verkündete Harry. Seine Stimme zitterte ein wenig.

"Bitte versucht, nicht zu sehr darüber zu streiten, Mum, Dad, wir werden früh genug wissen, wie es ausgeht, richtig?"

"Natürlich, Harry", sagte sein Vater, und seine Mutter gab ihm einen beruhigenden Kuss, und dann stritten sie weiter, während Harry die Treppe zu seinem Schlafzimmer hinaufstieg.

Er schloss die Tür hinter sich und versuchte, nachzudenken.

Das Komische war, dass er Dad eigentlich hätte zustimmen müssen.

Niemand hatte je einen Beweis für Magie gesehen, und laut Mum gab es da draußen eine ganze magische Welt.

Wie konnte jemand so etwas geheim halten? Mehr Magie? Das schien eine ziemlich verdächtige Art von Ausrede zu sein.

Es hätte ein klarer Fall für Mum sein müssen: ein Scherz, eine Lüge oder Geisteskrankheit, in aufsteigender Reihenfolge der Schrecklichkeit.

Wenn Mum den Brief selbst abgeschickt hatte, würde das erklären, wie er ohne Briefmarke im Briefkasten ankam.

Ein wenig Verrücktheit war viel, viel unwahrscheinlicher als dass das Universum wirklich so funktionierte.

Abgesehen davon, dass ein Teil von Harry felsenfest davon überzeugt war, dass Magie real war, und zwar seit dem Moment, in dem er den vermeintlichen Brief von der Hogwarts-Schule für Hexerei und Zauberei gesehen hatte.

Harry rieb sich die Stirn und zog eine Grimasse. Glaube nicht alles, was du denkst, hatte eines seiner Bücher gesagt.

Aber diese bizarre Gewissheit… Harry ertappte sich dabei, dass er einfach erwartete, dass, ja, ein Hogwarts-Professor auftauchen und mit einem Zauberstab herumfuchteln würde und Magie herauskommen würde.

Die seltsame Gewissheit machte keine Anstalten, sich gegen Verfälschungen zu schützen - machte keine Ausreden im Voraus, warum es keinen Professor geben würde, oder der Professor nur Löffel verbiegen könnte.

\emph{Woher kommst du, seltsame kleine Vorhersage?}

Harry richtete den Gedanken an sein Gehirn. \emph{Warum glaube ich, was ich glaube?} Normalerweise war Harry ziemlich gut darin, diese Frage zu beantworten, aber in diesem speziellen Fall hatte er keinen Schimmer, was sein Gehirn dachte.

Harry zuckte innerlich mit den Schultern. Eine flache Metallplatte an einer Tür ermöglicht es, zu drücken, und ein Griff an einer Tür ermöglicht es, zu ziehen, und das, was man mit einer überprüfbaren Hypothese tun sollte, ist, hinzugehen und sie zu testen.

Er nahm ein Stück liniertes Papier von seinem Schreibtisch und begann zu schreiben.

\emph{Sehr geehrte stellvertretende Schulleiterin,}

Harry hielt inne, überlegte; dann legte er das Papier für ein anderes weg und klopfte einen weiteren Millimeter Graphit aus seinem Druckbleistift.

Dies erforderte eine sorgfältige Kalligraphie.

Sehr geehrte stellvertretende Schulleiterin Minerva McGonagall, oder wen auch immer es betrifft:

Ich habe vor kurzem Ihren Brief über die Aufnahme in Hogwarts erhalten, adressiert an Mr.

H. Potter. Sie wissen vielleicht nicht, dass meine genetischen Eltern, James Potter und Lily Potter (ehemals Lily Evans), tot sind.

\emph{Ich wurde von Lilys Schwester, Petunia Evans-Verres, und ihrem Mann, Michael Verres-Evans, adoptiert.

Ich bin sehr daran interessiert, Hogwarts zu besuchen, unter der Voraussetzung, dass ein solcher Ort tatsächlich existiert.

Nur meine Mutter Petunia sagt, dass sie sich mit Magie auskennt, und sie kann sie selbst nicht anwenden.

Mein Vater ist höchst skeptisch. Ich selbst bin unsicher. Ich weiß auch nicht, woher ich eines der Bücher oder die Ausrüstung bekommen soll, die in deinem Zulassungsbrief aufgeführt sind.

Mutter erwähnte, dass Sie einen Hogwarts-Vertreter zu Lily Potter (damals Lily Evans) geschickt haben, um ihrer Familie zu zeigen, dass Magie real ist, und, wie ich annehme, um Lily zu helfen, ihre Schulmaterialien zu bekommen.

Wenn Sie dies für meine eigene Familie tun könnten, wäre das äußerst hilfreich.}

\emph{Mit freundlichen Grüßen, Harry James Potter-Evans-Verres.}

Harry fügte ihre aktuelle Adresse hinzu, dann faltete er den Brief zusammen und steckte ihn in einen Umschlag, den er an Hogwarts adressierte.

Nach weiteren Überlegungen besorgte er sich eine Kerze und tropfte Wachs auf die Klappe des Umschlags, in die er mit der Spitze eines Taschenmessers die Initialen H.

J.P.E.V. eindrückte. Wenn er sich schon in diesen Wahnsinn stürzte, dann mit Stil. Dann öffnete er seine Tür und ging wieder nach unten.

Sein Vater saß im Wohnzimmer und las ein Buch mit höherer Mathematik, um zu zeigen, wie klug er war; und seine Mutter war in der Küche und bereitete eines der Lieblingsgerichte seines Vaters zu, um zu zeigen, wie liebevoll sie war.

Es sah nicht so aus, als würden sie überhaupt miteinander reden. So beängstigend Streit auch sein konnte, nicht zu streiten war irgendwie viel schlimmer.

"Mum", sagte Harry in die beunruhigende Stille hinein,

"ich werde die Hypothese testen. Wie kann ich nach deiner Theorie eine Eule nach Hogwarts schicken?"

Seine Mutter drehte sich von der Küchenspüle um und starrte ihn mit schockiertem Blick an.

"Ich - ich weiß es nicht, ich glaube, man muss einfach eine magische Eule besitzen."

Das hätte eigentlich höchst verdächtig klingen müssen, oh, dann gibt es also keine Möglichkeit, ihre Theorie zu testen, aber die merkwürdige Gewissheit in Harry schien bereit zu sein, ihren Hals noch weiter herauszustrecken.

"Nun, der Brief ist irgendwie hierher gekommen", sagte Harry,

"also werde ich ihn einfach draußen herumfuchteln und 'Brief für Hogwarts!' rufen und sehen, ob eine Eule ihn aufhebt. Dad, willst du mitkommen und zusehen?"

Sein Vater schüttelte bedächtig den Kopf und las weiter.

Natürlich, dachte Harry bei sich. Magie war eine schändliche Sache, an die nur dumme Menschen glaubten; wenn sein Vater so weit ginge, die Hypothese zu testen oder gar zuzusehen, wie sie getestet wurde, käme ihm das so vor, als würde er sich damit assoziieren…

Erst als Harry durch die Hintertür in den Garten stapfte, kam ihm der Gedanke, dass es ihm schwer fallen würde, Dad davon zu erzählen, falls tatsächlich eine Eule herunterkäme und den Brief mitnehmen würde.

Aber - nun - das kann doch nicht wirklich passieren, oder? Egal, was mein Gehirn zu glauben scheint.

Wenn wirklich eine Eule herunterkommt und sich diesen Umschlag schnappt, werde ich mir viel wichtigere Sorgen machen als das, was Dad denkt.

Harry holte tief Luft und hob den Umschlag in die Luft.

Er schluckte.

\emph{"Brief für Hogwarts!"}

zu rufen, während man mitten im eigenen Garten einen Umschlag hoch in die Luft hielt, war… eigentlich ziemlich peinlich, jetzt, wo er darüber nachdachte.

\emph{Nein. Ich bin besser als Dad. Ich werde die wissenschaftliche Methode anwenden, auch wenn ich mir dabei dumm vorkomme.}

"Brief -" sagte Harry, aber es kam eher wie ein geflüstertes Krächzen heraus. Harry nahm seinen Willen zusammen und rief in den leeren Himmel:

\textbf{"Brief für Hogwarts! Kann ich eine Eule bekommen? "}

"Harry?", fragte eine verwirrte Frauenstimme, eine der Nachbarinnen.

Harry riss seine Hand herunter, als würde sie brennen, und versteckte den Umschlag hinter seinem Rücken, als wäre es Drogengeld.

Sein ganzes Gesicht war heiß vor Scham. Das Gesicht einer alten Frau lugte über den benachbarten Zaun hervor, das graue Haar entwich aus ihrem Haarnetz.

Mrs. Figg, die gelegentliche Babysitterin.

"Was machst du da, Harry?"

"Nichts", sagte Harry mit erstickter Stimme.

"Ich - teste nur eine wirklich dumme Theorie -"

"Hast du deinen Zulassungsbescheid von Hogwarts bekommen?"

Harry erstarrte auf der Stelle.

"Ja", kam es nach einer Weile über Harrys Lippen. "Ich habe einen Brief von Hogwarts bekommen.

Sie sagen, sie wollen meine Eule bis zum 31. Juli haben, aber -"

"Aber du hast doch gar keine Eule.

Armes Kind! Ich kann mir nicht vorstellen, was sich jemand dabei gedacht haben muss, dir nur den Standardbrief zu schicken."

Ein faltiger Arm streckte sich über den Zaun und öffnete eine erwartungsvolle Hand.

Kaum dass er nachgedacht hatte, überreichte Harry den Umschlag.

"Überlassen das einfach mir, mein Lieber", sagte Mrs. Figg,

"und in ein oder zwei Minuten werde ich jemanden vorbeischicken."

Und ihr Gesicht verschwand hinter dem Zaun.

Es herrschte eine lange Stille im Garten.

Dann sagte eine Jungenstimme, ruhig und leise:

"\emph{Was}."

