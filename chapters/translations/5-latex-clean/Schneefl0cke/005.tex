

\hypertarget{der-fundamentale-attributionsfehler}{% \section{6. Der fundamentale Attributionsfehler}\label{der-fundamentale-attributionsfehler}}

\emph{*****}\\ \emph{"Es hätte eines übernatürlichen Eingriffs bedurft, damit er in seiner Umgebung deine Moral besitzt."}\\ *****

Der Laden für nützliche Gegenstände war ein uriger kleiner Laden (manche würden sogar sagen: niedlich), der sich hinter einem Gemüsestand befand, der hinter einem Laden für magische Handschuhe lag, der sich in einer Gasse in einer Seitenstraße der Winkelgasse befand.

Enttäuschenderweise war die Ladenbesitzerin keine verhutzelte alte Kuh, sondern eine nervös wirkende junge Frau in verblichenen gelben Gewändern. Im Moment hielt sie einen Super Maulwurfsfellbeutel QX31 in der Hand, dessen Verkaufsargument darin bestand, dass er sowohl über eine größere Öffnung als auch über einen unauffindbaren Erweiterungszauber verfügte: Man konnte tatsächlich große Dinge darin verstauen, obwohl das Gesamtvolumen immer noch begrenzt war.

Harry hatte darauf bestanden, gleich als Erstes hierher zu kommen - so sehr, wie er glaubte, darauf bestehen zu können, ohne Professor McGonagall misstrauisch zu machen.

Harry hatte etwas, das er so schnell wie möglich in den Beutel stecken musste. Es war nicht der Beutel mit den Galleonen, die Professor McGonagall ihm erlaubt hatte, aus Gringotts zu entnehmen. Es waren all die anderen Galleonen, die Harry heimlich in seine Tasche gesteckt hatte, nachdem er in einen Haufen Goldmünzen gefallen war.

Das war ein echter Unfall gewesen, aber Harry war nie jemand, der eine Gelegenheit ausschlug… obwohl es eigentlich mehr eine spontane Idee gewesen war. Seitdem trug Harry den erlaubten Beutel mit Galleonen unbeholfen neben seiner Hosentasche, so dass jedes Klimpern von der richtigen Stelle zu kommen schien.

Damit blieb immer noch die Frage offen, wie er die anderen Münzen eigentlich in den Beutel bekommen wollte, ohne erwischt zu werden. Die goldenen Münzen mochten ihm gehören, aber sie waren trotzdem gestohlen - selbst gestohlen? Harry schaute von dem Beutel auf, der vor ihm auf dem Tresen lag.

"Kann ich das mal kurz ausprobieren? Um sicherzugehen, dass es, ähm, zuverlässig funktioniert?"

Er weitete seine Augen mit einem Ausdruck von jungenhafter, spielerischer Unschuld. Und tatsächlich, nach zehn Wiederholungen, in denen er den Beutel mit den Münzen in den Beutel steckte, hinein griff,\\ "Beutel mit Gold" flüsterte\\ und ihn wieder herausholte, trat Professor McGonagall einen Schritt zur Seite und begann, einige der anderen Gegenstände im Laden zu untersuchen, und die Ladenbesitzerin drehte ihren Kopf, um zuzusehen.

Harry ließ den Beutel mit dem Gold mit der linken Hand in den Beutel fallen; seine rechte Hand kam aus der Tasche, in der er einige der Goldmünzen festhielt, griff in den Beutel, ließ die losen Galleonen fallen und holte (mit einem geflüsterten "Beutel mit Gold") den ursprünglichen Beutel zurück.

Dann ging der Beutel zurück in seine linke Hand, um wieder hineingeworfen zu werden, und Harrys rechte Hand ging zurück in seine Tasche…

Professor McGonagall schaute einmal zu ihm zurück, aber Harry schaffte es, nicht zu erstarren oder zu zucken, und sie schien nichts zu bemerken. Obwohl man bei den Erwachsenen, die einen Sinn für Humor hatten, nie ganz sicher sein konnte.

Es brauchte drei Wiederholungen, um den Job zu erledigen, und Harry schätzte, dass er es geschafft hatte, sich selbst vielleicht dreißig Galleonen zu stehlen. Harry griff nach oben, wischte sich ein wenig Schweiß von der Stirn und atmete aus.

"Ich hätte gern den hier, bitte."

Fünfzehn Galleonen leichter (anscheinend doppelt so teuer wie ein Zauberstab) und einen Maulwurfsfell-Beutel QX31 schwerer, schoben sich Harry und Professor McGonagall aus der Tür. Die Tür formte eine Hand und winkte ihnen zum Abschied zu, wobei sie ihren Arm auf eine Weise ausstreckte, die Harry ein wenig mulmig werden ließ.

Und dann, leider…

"Bist du wirklich Harry Potter?", flüsterte der alte Mann, wobei ihm eine riesige Träne über die Wange glitt.\\ "Du würdest doch nicht lügen, oder? Ich habe nur Gerüchte gehört, dass du den Tötungsfluch nicht wirklich überlebt hast und man deshalb nie wieder etwas von dir gehört hat."

… es schien, dass Professor McGonagalls Verkleidungszauber gegen erfahrenere Magieanwender nicht ganz so effektiv war.

Professor McGonagall hatte Harry eine Hand auf die Schulter gelegt und ihn in die nächstgelegene Gasse gezerrt, als sie "Harry Potter?" hörte.

Der alte Mann war ihr gefolgt, aber zumindest sah es so aus, als hätte es sonst niemand gehört.

Harry dachte über die Frage nach. War er wirklich Harry Potter?

"Ich weiß nur, was andere Leute mir erzählt haben", sagte Harry.\\ "Es ist nicht so, dass ich mich an meine Geburt erinnere."\\ Seine Hand strich über seine Stirn.\\ "Ich habe diese Narbe, solange ich mich erinnern kann, und mir wurde gesagt, mein Name sei Harry Potter, solange ich mich erinnern kann. Aber",\\ sagte Harry nachdenklich,\\ "wenn es schon genügend Gründe gibt, eine Verschwörung zu postulieren, gibt es keinen Grund, warum sie nicht einfach ein anderes Waisenkind finden und es in dem Glauben aufziehen sollten, dass es Harry Potter sei -"

Professor McGonagall fuhr sich verärgert mit der Hand über das Gesicht.

"Du siehst fast genauso aus wie dein Vater, James, in dem Jahr, als er zum ersten Mal Hogwarts besuchte. Und ich kann Ihnen allein aufgrund Ihrer Persönlichkeit bescheinigen, dass Sie mit der Geißel von Gryffindor verwandt sind."

"Sie könnte auch Teil der Verschwörung sein", bemerkte Harry.

"Nein", zitterte der alte Mann. "Sie hat recht. Du hast die Augen deiner Mutter."

"Hmm", Harry runzelte die Stirn. "Ich nehme an, du könntest auch dabei sein -"

"Genug, Mr. Potter."

Der alte Mann hob eine Hand, als wolle er Harry berühren, ließ sie dann aber fallen. "Ich bin nur froh, dass du noch lebst", murmelte er.\\ "Ich danke dir, Harry Potter. Ich danke dir für das, was du getan hast… Ich werde dich jetzt in Ruhe lassen."

Und sein Stock klopfte langsam weg, aus der Gasse heraus und die Hauptstraße der Winkelgasse hinunter. Die Professorin sah sich um, ihre Miene war angespannt und grimmig. Harry sah sich automatisch selbst um. Doch die Gasse schien bis auf altes Laub leer zu sein, und von der Mündung, die in die Diagon Alley hinausführte, waren nur schnell schreitende Passanten zu sehen.\\ Endlich schien sich Professor McGonagall zu entspannen.

"Das war nicht gut gemacht", sagte sie mit leiser Stimme. "Ich weiß, dass Sie das nicht gewöhnt sind, Mr. Potter, aber die Leute sorgen sich um Sie. Bitte seien Sie nett zu ihnen."

Harry blickte auf seine Schuhe hinunter.

"Das sollten sie nicht", sagte er mit einem Anflug von Bitterkeit. "Sich Sorgen um mich machen, mein ich."

"Du hast sie vor Du-weißt-schon-wem gerettet", sagte Professor McGonagall. "Wie sollten sie sich nicht kümmern?"

Harry sah zu dem strengen Blick der Hexenmeisterin unter ihrem spitzen Hut auf und seufzte.

"Ich nehme an, Sie haben keine Ahnung, was das bedeutet, wenn ich von einem fundamentalen Zuordnungsfehler spreche."

"Nein", sagte die Professorin in ihrem präzisen schottischen Akzent, "aber bitte erklären Sie es, Mr. Potter, wenn Sie so freundlich wären."

"Nun …" sagte Harry und versuchte herauszufinden, wie er diesen besonderen Teil der Muggelwissenschaft beschreiben sollte.

"Angenommen, Sie kommen zur Arbeit und sehen, wie Ihr Kollege gegen seinen Schreibtisch tritt. Sie denken: '\emph{Was für ein wütender Mensch er sein muss}'.

Ihr Kollege denkt darüber nach, dass ihn jemand auf dem Weg zur Arbeit gegen eine Wand gestoßen hat und ihn dann angeschrien hat.

Jeder würde darüber wütend sein, denkt er.

Wenn wir andere betrachten, sehen wir Persönlichkeitsmerkmale, die ihr Verhalten erklären, aber wenn wir uns selbst betrachten, sehen wir Umstände, die unser Verhalten erklären.

Die Geschichten der Menschen machen für sie einen inneren Sinn, von innen heraus, aber wir sehen die Geschichten der Menschen nicht in der Luft hinter ihnen herziehen.

Wir sehen sie nur in einer Situation, und wir sehen nicht, wie sie in einer anderen Situation sein würden.

Der fundamentale Attributionsfehler besteht also darin, dass wir mit dauerhaften, beständigen Merkmalen erklären, was besser durch Umstände und Kontext zu erklären wäre."

Es gab einige elegante Experimente, die das bestätigten, aber Harry hatte nicht vor, auf sie einzugehen. Die Augenbrauen der Hexe zogen sich unter der Krempe ihres Hutes hoch.

"Ich glaube, ich verstehe …" sagte Professor McGonagall langsam.\\ "Aber was hat das mit Ihnen zu tun?"

Harry trat so fest gegen die Ziegelwand der Gasse, dass sein Fuß schmerzte.

"Die Leute denken, dass ich sie vor Du-weißt-schon-wem gerettet habe, weil ich eine Art großer Krieger des Lichts bin."

"Der mit der Macht, den Dunklen Lord zu besiegen …",\\ murmelte die Hexe, wobei eine seltsame Ironie in ihrer Stimme mitschwang.

"Ja", sagte Harry,\\ Ärger und Frustration kämpften in ihm,\\ "als hätte ich den Dunklen Lord vernichtet, weil ich eine Art permanente, dauerhafte Zerstör-den-Dunklen-Lord-Eigenschaft habe. Ich war zu der Zeit fünfzehn Monate alt! Ich weiß nicht, was passiert ist, aber ich würde annehmen, dass es etwas mit, wie man so schön sagt, bedingten Umweltumständen zu tun hatte.

Und es hatte ganz sicher nichts mit meiner Persönlichkeit zu tun. Die Leute interessieren sich nicht für mich, sie schenken mir nicht einmal Aufmerksamkeit, sie wollen sich nur mit einer schlechten Erklärung abgeben."

Harry hielt inne und sah McGonagall an.

"Wissen Sie, was wirklich passiert ist?"

"Ich habe mir eine Vorstellung gemacht …", sagte Professor McGonagall. "Nachdem ich Sie getroffen habe, meine ich."

"Ja?"

"Du hast über den Dunklen Lord triumphiert, indem du schrecklicher warst als er, und hast den Tötungsfluch überlebt, indem du schrecklicher warst als der Tod."

"Ha. Ha. Ha."

Harry trat wieder gegen die Wand. Professor McGonagall gluckste.

"Als Nächstes bringen wir Sie zu Madam Malkin. Ich fürchte, Ihre Muggelkleidung könnte Aufmerksamkeit erregen."

Unterwegs trafen sie auf zwei weitere Gratulanten.

Madam Malkins Roben hatte eine wirklich langweilige Ladenfront, rote gewöhnliche Ziegelsteine und Glasfenster, die schlichte schwarze Roben im Inneren zeigten. Keine Roben, die leuchteten oder sich veränderten oder drehten oder seltsame Strahlen aussandten, die durch das Hemd zu gehen und einen zu kitzeln schienen. Nur einfache schwarze Roben, das war alles, was man durch das Fenster sehen konnte.

Die Tür war weit aufgestoßen, als wollte sie verkünden, dass es hier keine Geheimnisse gab und nichts zu verbergen.

"Ich werde für ein paar Minuten weggehen, während Sie sich Ihre Roben anprobieren", sagte Professor McGonagall.\\ "Ist das in Ordnung für Sie, Mr. Potter?"

Harry nickte. Er hasste Kleiderkauf mit feuriger Leidenschaft und konnte es der älteren Hexe nicht verübeln, dass es ihr genauso ging.

Professor McGonagalls Zauberstab kam aus ihrem Ärmel und tippte leicht an Harrys Kopf.

"Und da Sie für Madam Malkins Sinne klar sein müssen, werde ich die Verschleierung entfernen."

"Äh …" sagte Harry.\\ Das beunruhigte ihn ein wenig; er war immer noch nicht an die Sache mit 'Harry Potter' gewöhnt.

"Ich war in Hogwarts mit Madam Malkin", sagte McGonagall.\\ "Schon damals war sie eine der gelassensten Menschen, die ich kannte. Sie würde mit keinem Auge zwinkern, wenn Du-weißt-schon-wer persönlich in ihren Laden käme."

McGonagalls Stimme war erinnerungsfern und sehr anerkennend.

"Madam Malkin wird dich nicht belästigen, und sie wird auch nicht zulassen, dass dich jemand anderes belästigt."

"Wohin gehen Sie?" erkundigte sich Harry. "Nur für den Fall, dass doch etwas passiert."

McGonagall warf Harry einen strengen Blick zu.

"Ich gehe dorthin",\\ sagte sie und deutete auf ein Gebäude auf der anderen Straßenseite, auf dem das Zeichen eines Holzfasses zu sehen war,\\ "und kaufe mir einen Drink, den ich dringend brauche. Du sollst dir deine Roben anpassen lassen, sonst nichts. Ich werde in Kürze wiederkommen, um nach Ihnen zu sehen, und ich erwarte, dass der Laden von Madam Malkin noch steht und in keiner Weise in Flammen steht."

Madam Malkin war eine geschäftige alte Frau, die kein Wort über Harry verlor, als sie die Narbe auf seiner Stirn sah, und sie schoss einen scharfen Blick auf eine Assistentin, als diese etwas sagen wollte.

Madam Malkin holte eine Reihe von animierten, sich windenden Stofffetzen heraus, die als Maßband zu dienen schienen, und machte sich an die Arbeit, das Medium ihrer Kunst zu untersuchen.

Neben Harry schien ein blasser Junge mit spitzem Gesicht und erstaunlich coolem blond-weißem Haar die letzten Stadien eines ähnlichen Prozesses zu durchlaufen.

Eine von Malkins zwei Assistentinnen untersuchte den weißhaarigen Jungen und die schachbrettartig gerasterte Robe, die er trug; gelegentlich tippte sie mit ihrem Zauberstab auf eine Ecke der Robe, und die Robe lockerte oder straffte sich.

"Hallo", sagte der Junge. "Auch in Hogwarts?"

Harry konnte vorhersehen, wohin dieses Gespräch führen würde, und er beschloss in einem Sekundenbruchteil der Frustration, dass es genug war.

"Gütiger Himmel", flüsterte Harry, "das kann doch nicht sein."

Er ließ seine Augen weit aufreißen. "Ihr… Name, Sir?"

"Draco Malfoy", sagte Draco Malfoy mit einem leicht verwirrten Blick.

"Sie sind es! Draco Malfoy. Ich - ich hätte nie gedacht, dass ich mich so geehrt fühlen würde, Sir."

Harry wünschte sich, dass ihm die Tränen aus den Augen kämen. Normalerweise fingen die anderen ungefähr an dieser Stelle an zu weinen.

"Oh", sagte Draco und klang ein wenig verwirrt. Dann spannten sich seine Lippen zu einem süffisanten Lächeln.\\ "Es ist gut, jemanden zu treffen, der seinen Platz kennt."

Einer der Assistenten, derjenige, der Harry zu erkennen schien, gab einen dumpfen, würgenden Laut von sich. Harry plusterte sich auf.

"Ich bin entzückt, Sie kennenzulernen, Mr. Malfoy. Einfach unsagbar erfreut. Und dass ich in Ihrem Jahrgang Hogwarts besuche! Das lässt mein Herz in Ohnmacht fallen."

Huch. Der letzte Teil klang vielleicht ein bisschen seltsam, als würde er mit Draco flirten oder so.

"Und ich freue mich zu erfahren, dass ich mit dem Respekt behandelt werde, der der Familie Malfoy gebührt",\\ lobte der andere Junge zurück, begleitet von einem Lächeln, wie es der höchste König dem geringsten seiner Untertanen schenken würde, wenn dieser ehrlich wäre, wenn auch arm.

Eh… Verdammt, Harry hatte Schwierigkeiten, sich seinen nächsten Satz auszudenken. Nun, jeder wollte Harry Potter die Hand schütteln, also…

"Wenn ich angezogen bin, Sir, würden Sie mir dann die Hand schütteln? Ich würde mir nichts sehnlicher wünschen, als diesen Tag, nein, diesen Monat, ja mein ganzes Leben so zu krönen."

Der weiß-blondhaarige Junge erwiderte den Blick.

"Und was hast du für die Malfoys getan, das dich zu so einem Gefallen berechtigt?"

Oh, ich werde diesen Spruch auf jeden Fall bei der nächsten Person ausprobieren, die mir die Hand schütteln will. Harry senkte den Kopf.

"Nein, nein, Sir, ich verstehe. Es tut mir leid, dass ich gefragt habe. Es wäre mir vielmehr eine Ehre, Ihre Stiefel zu putzen."

"In der Tat",\\ schnauzte der andere Junge. Sein strenges Gesicht hellte sich etwas auf.\\ "Sag mal, in welches Haus wirst du wohl einsortiert? Ich bin natürlich für das Haus Slytherin bestimmt, wie mein Vater Lucius vor mir. Und bei dir würde ich auf das Haus Hufflepuff tippen, oder vielleicht Hauself. \emph{(„House-elf", anm. des Übersetzers)}

Harry grinste verlegen.

"Professor McGonagall sagt, dass ich der größte Ravenclaw bin, den sie je gesehen oder von dem sie je gehört hat, und zwar so sehr, dass Rowena selbst mir raten würde, mehr rauszugehen, was auch immer das heißen mag, und dass ich zweifellos im Haus Ravenclaw landen werde, wenn der Hut nicht zu laut schreit, als dass der Rest von uns irgendwelche Worte verstehen könnte, Zitat Ende."

"Wow",\\ sagte Draco Malfoy und klang leicht beeindruckt. Der Junge stieß eine Art wehmütigen Seufzer aus.

"Deine Schmeicheleien waren großartig, zumindest dachte ich das - du würdest dich auch im Haus Slytherin gut machen. Normalerweise ist es nur mein Vater, der diese Art von Kriecherei bekommt. Ich hoffe, die anderen Slytherins schleimen sich bei mir ein, jetzt wo ich in Hogwarts bin… Dann ist das wohl ein gutes Zeichen."

Harry hustete.

"Eigentlich, tut mir Leid, ich habe keine Ahnung, wer du wirklich bist."

"Ach komm schon!?", sagte der Junge mit grimmiger Enttäuschung. "Warum hast du das dann gemacht?"

Dracos Augen weiteten sich vor plötzlichem Misstrauen.

"Und wieso weißt du nichts von den Malfoys? Und was sind das für Klamotten, die du da trägst? Sind deine Eltern Muggel?"

"Zwei meiner Eltern sind tot",\\ sagte Harry. Sein Herz pochte.\\ "Meine anderen beiden Eltern sind Muggel, und sie sind diejenigen, die mich aufgezogen haben", sagte er.

"Was?", sagte Draco. "Wer bist du?"

"Harry Potter, freut mich, dich kennenzulernen."

"Harry Potter?!", keuchte Draco. "Der Harry -"

und der Junge brach abrupt ab. Es herrschte eine kurze Stille. Dann, mit heller Begeisterung,

"Harry Potter? Der Harry Potter? Mensch, dich wollte ich schon immer mal kennenlernen!"

die Ladenangestellte stieß einen Laut aus, als würde sie würgen, fuhr aber mit ihrer Arbeit fort und hob Dracos Arme an, um ihm vorsichtig den karierten Umhang abzunehmen.

"Halt die Klappe", schlug Harry vor.

"Kann ich ein Autogramm von dir haben? Nein, warte, ich will erst ein Foto mit dir!"

"HaltsMaul, haltsmaul, haltsmaul!."

"Ich bin so erfreut, dich kennenzulernen!"

"Geh in Flammen auf und stirb!"

"Aber du bist Harry Potter, der glorreiche Retter der Zaubererwelt! Jedermanns Held, Harry Potter! Ich wollte immer so werden wie du, wenn ich groß bin, damit ich -"

Draco unterbrach die Worte mitten im Satz, sein Gesicht erstarrte in absolutem Entsetzen.

Groß, weißhaarig, kalt und elegant, in schwarzer Robe von feinster Qualität. In der einen Hand hielt er einen silbernen Stock, der allein dadurch, dass er in dieser Hand lag, den Charakter einer tödlichen Waffe annahm. Seine Augen betrachteten den Raum mit der leidenschaftslosen Qualität eines Henkers, eines Mannes, für den das Töten nicht schmerzhaft oder gar köstlich verboten war, sondern einfach eine Routinetätigkeit wie das Atmen.

Das war der Mann, der in diesem Moment durch die offene Tür hereinspaziert war.

"Draco", sagte der Mann, leise und sehr wütend, "was sagst du?".

In einem Bruchteil einer Sekunde mitfühlender Panik formulierte Harry einen Rettungsplan.

"Lucius Malfoy!", keuchte Harry Potter. "Der Lucius Malfoy?!"

Einer von Malkins Assistenten musste sich abwenden und an die Wand stellen.

Kühle, mörderische Augen betrachteten ihn.

"Harry Potter."

"Ich fühle mich so sehr geehrt, Sie kennenzulernen!"

Die dunklen Augen weiteten sich, schockierte Überraschung ersetzte tödliche Drohung.

"Ihr Sohn hat mir alles über Sie erzählt", sprudelte Harry weiter, kaum wissend, was aus seinem Mund kam, sondern einfach so schnell wie möglich redend.

"Aber natürlich wusste ich schon vorher alles über Sie, jeder weiß über Sie Bescheid, der große Lucius Malfoy! Der ehrenvollste Preisträger des ganzen Hauses Slytherin, ich habe schon darüber nachgedacht, selbst ins Haus Slytherin zu kommen, nur weil ich gehört habe, dass Sie als Kind dort waren -"

"Was sagen Sie da, Mr. Potter?!",\\ kam es fast schreiend von draußen, und Professor McGonagall platzte eine Sekunde später herein.

Auf ihrem Gesicht stand so pures Entsetzen, dass Harrys Mund sich automatisch öffnete, dann aber bei einem Nichts-zu-sagen stehen blieb.

"Professor McGonagall!", rief Draco. "Sind Sie es wirklich? Ich habe von meinem Vater so viel über Sie gehört, dass ich schon überlegt habe, mich nach Gryffindor sortieren zu lassen, damit ich -"

„Was?!,“

brüllten Lucius Malfoy und Professor McGonagall in perfektem Einklang und standen nebeneinander. Ihre Köpfe drehten sich, um einander in einer\\ doppelten Bewegung anzuschauen, und dann wichen die beiden voreinander zurück, als würden sie einen Synchrontanz aufführen.

Es gab eine plötzliche Hektik, als Lucius Draco packte und ihn aus dem Laden zerrte. Und dann herrschte Stille.

In Professor McGonagalls linker Hand lag ein kleines Trinkglas, das in der vergessenen Eile zur Seite gekippt war und nun langsam Tropfen in die winzige Rotweinpfütze tropfte, die sich auf dem Boden gebildet hatte.

Professor McGonagall schritt vorwärts in den Laden, bis sie Madam Malkin gegenüberstand.

"Madam Malkin", sagte Professor McGonagall, ihre Stimme war ruhig. "Was hat sich hier abgespielt?"

Madam Malkin blickte vier Sekunden lang schweigend zurück, dann brach sie zusammen. Sie fiel gegen die Wand und stieß ein Lachen aus, was ihre beiden Assistentinnen aufschreckte, von denen eine hysterisch kichernd auf den Boden fiel.

Professor McGonagall drehte sich langsam um und sah Harry an, ihre Miene war kühl.

"Ich lasse Sie für sechs Minuten allein. Sechs Minuten, Mr. Potter, genau nach der Uhr."

"Ich habe nur einen Scherz gemacht", protestierte Harry, während in der Nähe hysterisches Gelächter zu hören war.

"Draco Malfoy hat vor seinem Vater gesagt, dass er nach Gryffindor einsortiert werden will! Mit Scherzen alleine ist das nicht getan!"

Professor McGonagall hielt inne und holte sichtlich Luft.

"Welcher Teil von 'Roben anprobieren' klang für dich wie 'Bitte wirf einen Verwirrungs-Zauber auf das gesamte Universum!'?"

"Er befand sich in einem situativen Kontext, in dem diese Handlungen einen inneren Sinn ergaben -"

"Nein. Erklären Sie es nicht. Ich will nicht wissen, was hier drin passiert ist, niemals. Welche dunkle Macht auch immer in Ihnen wohnt, sie ist ansteckend, und ich will nicht so enden wie der arme Draco Malfoy, die arme Madam Malkin und ihre beiden armen Assistenten."

Harry seufzte. Es war klar, dass Professor McGonagall nicht in der Stimmung war, sich vernünftige Erklärungen anzuhören.

Er blickte auf Madam Malkin, die immer noch keuchend an der Wand lehnte, und auf Malkins zwei Assistenten, die nun beide auf die Knie gefallen waren, und schließlich auf seinen eigenen, mit Klebeband umwickelten Körper.

"Ich bin noch nicht ganz fertig mit dem Anpassen der Kleidung", sagte Harry freundlich. "Warum gehen Sie nicht zurück und trinken noch etwas?"

