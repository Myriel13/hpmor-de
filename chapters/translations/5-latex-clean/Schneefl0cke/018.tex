

\hypertarget{aufgeschobene-genugtuung}{% \section{19. Aufgeschobene Genugtuung}\label{aufgeschobene-genugtuung}}

\textbf{Aufgeschobene Genugtuung}

Draco hatte einen strengen Gesichtsausdruck, und seine grün gesäumten Roben sahen irgendwie viel förmlicher, ernster und besser gekleidet aus als genau die gleichen Roben, die die beiden Jungen hinter ihm trugen.

„Sprich“, sagte Draco.

„Ja! Rede!“

„Du hast den Boss gehört! Rede!“

„Ihr zwei hingegen, haltet die Klappe.“

Die letzte Unterrichtsstunde am Freitag sollte beginnen, in der riesigen Aula, in der alle vier Häuser Verteidigung, äh, Kampfmagie lernten. Die letzte Unterrichtsstunde am Freitag. Harry hoffte, dass diese Stunde nicht zu stressig werden würde und dass der brillante Professor Quirrell erkennen würde, dass dies vielleicht nicht der beste Zeitpunkt war, um Harry für irgendetwas herauszufordern.

Harry hatte sich ein wenig erholt, aber……aber vorsichtshalber war es wohl am besten, erst einmal ein wenig Stressabbau zu betreiben. Harry lehnte sich in seinem Stuhl zurück und schenkte Draco und seinen Schergen einen sehr ernsten Blick.

„Sie fragen, was unser Ziel ist ?“ deklamierte Harry.

„ ich kann in einem Worte erwidern: es ist der Sieg - Sieg um jeden Preis - Sieg trotz aller Schrecken, Sieg, wie lang und hart auch immer der Weg sein mag, denn ohne Sieg gibt es kein Überleben - seien Sie sich darüber klar - kein Überleben für ….“

„Du sollst über Snape reden“, zischte Draco. „Was hast du getan?“

Harry wischte die vorgetäuschte Feierlichkeit weg und schaute Draco ernster an.

„Du hast es gesehen“, sagte Harry. „Jeder hat es gesehen. Ich habe mit den Fingern geschnippt.“

„Harry! Hör auf, mich zu ärgern!?“

\emph{Er war also jetzt zu Harry befördert worden. Interessant.}

Und eigentlich war sich Harry ziemlich sicher, dass er das bemerken und sich schlecht fühlen sollte, wenn er nicht irgendwie reagierte… Harry spitzte die Ohren und warf einen bedeutungsvollen Blick auf die Minions.

„Sie werden nicht reden“, sagte Draco.

„Draco“, sagte Harry, „ich werde hier hundertprozentig ehrlich sein und sagen, dass ich gestern von Mr~Goyles Gerissenheit nicht besonders beeindruckt war.“

Mr~Goyle zuckte zusammen.

„Ich auch nicht“, sagte Draco.

„Ich habe ihm erklärt, dass ich dir deswegen am Ende einen Gefallen schulde.“

(Mr~Goyle zuckte wieder zusammen.)

„Aber es gibt einen großen Unterschied zwischen dieser Art von Fehler und Indiskretion. Das ist etwas, was man ihnen von Kindheit an beigebracht hat.“

„Also gut“, sagte Harry.

Er senkte seine Stimme, auch wenn die Hintergrundgeräusche in Dracos Gegenwart zu verschwimmen begannen.

„Ich bin einem von Severus' Geheimnissen auf die Schliche gekommen und habe ein bisschen Erpressung betrieben.“

Dracos Gesichtsausdruck verhärtete sich.

„Gut, und jetzt erzähl mir etwas, das du nicht streng vertraulich den Idioten in Gryffindor erzählt hast, das war die Geschichte, die du in der ganzen Schule bekannt machen wolltest.“

Harry grinste unwillkürlich und er wusste, dass Draco es mitbekommen hatte.

„Was hat Severus gesagt?“ Sagte Harry.

„Dass er nicht gemerkt hat, wie empfindlich die Gefühle von kleinen Kindern sind“, sagte Draco.

„Sogar in Slytherin! Sogar mir gegenüber!“

„Bist du sicher“, sagte Harry, „dass du etwas wissen willst, von dem dein Hausoberhaupt lieber nicht möchte, dass du es weißt?“

„Ja“, sagte Draco ohne zu zögern.

\emph{Interessant}.

„Dann schickst du deine Lakaien wirklich erst einmal weg, denn ich bin mir nicht sicher, ob ich alles glauben kann, was du über sie glaubst.“

Draco nickte.

„Okay.“

Mr~Crabbe und Mr~Goyle sahen sehr unglücklich aus.

„Boss—“ , sagte Mr~Crabbe.

„Ihr habt Mr~Potter keinen Grund gegeben, euch zu vertrauen“, sagte Draco. „Geht!“

Sie gingen.

„Vor allem“, sagte Harry und senkte seine Stimme noch weiter,

„bin ich mir nicht ganz sicher, ob sie nicht einfach Lucius berichten würden, was ich gesagt habe.“

„Vater würde das nicht tun!“ sagte Draco und sah wirklich entsetzt aus. „Sie gehören mir!“

„Es tut mir leid, Draco“, sagte Harry.

"Ich bin mir nur nicht sicher, ob ich alles glauben kann, was du über deinen Vater glaubst.

Stell dir vor, es wäre dein Geheimnis und ich würde dir sagen, dass mein Vater das nicht tun würde."

Draco nickte langsam.

„Du hast recht. Es tut mir leid, Harry. Es war falsch von mir, es von dir zu verlangen.“

\emph{Wie bin ich nur so weit gekommen? Sollte er mich jetzt nicht hassen?}

Harry hatte das Gefühl, etwas Verwertbares vor sich zu haben…er wünschte nur, sein Gehirn wäre nicht so erschöpft. Normalerweise hätte er sich gerne an einem komplizierten Komplott versucht.

„Wie auch immer“, sagte Harry.

„Ein Handel. Ich erzähle dir eine Tatsache, die sich nicht herumspricht und insbesondere nicht zu deinem Vater durchdringt, und im Gegenzug sagst du mir, was du und Slytherin von der ganzen Sache haltet.“

„Abgemacht!“

\emph{Um es so vage wie möglich zu formulieren… etwas, das nicht viel schaden würde, selbst wenn es rauskäme…}

„Was ich gesagt habe, ist wahr. Ich habe eines von Severus' Geheimnissen entdeckt und ich habe ihn erpresst. Aber Severus war nicht die einzige Person, die daran beteiligt war.“

„Ich wusste es!“ Sagte Draco überschwänglich.

Harrys Magen sank. Er hatte offenbar etwas sehr Bedeutsames gesagt und er wusste nicht, warum. Das war kein gutes Zeichen.

„In Ordnung“, sagte Draco. Er grinste jetzt breit. "Also, so sah die Reaktion in Slytherin aus.

Zuerst waren alle Idioten so: '\emph{Wir hassen Harry Potter! Lasst uns ihn verprügeln!}'" Harry hat sich verschluckt.

„Was ist mit dem Sprechenden Hut los? Das ist nicht Slytherin, das ist Gryffindor—“

„Nicht alle Kinder sind Wunderkinder“, sagte Draco, wobei er auf eine Art böse-verschwörerische Weise lächelte, als wolle er andeuten, dass er insgeheim mit Harrys Meinung übereinstimmte.

"Und es hat ungefähr fünfzehn Sekunden gedauert, bis ihnen jemand erklärt hat, warum das vielleicht nicht so ein Gefallen für Snape ist, also ist alles in Ordnung.

Wie auch immer, danach kam die zweite Welle von Idioten, die sagten: \emph{'Sieht so aus, als wäre Harry Potter doch nur ein weiterer Weltverbesserer.}'"

„Und dann?“ sagte Harry und lächelte, obwohl er keine Ahnung hatte, warum das dumm war.

„Und dann fingen die wirklich klugen Leute an zu reden. Es ist offensichtlich, dass du einen Weg gefunden hast, Snape unter Druck zu setzen. Und obwohl das mehr als eine Sache sein könnte… der naheliegende nächste Gedanke ist, dass es etwas mit Snapes unbekannter Macht über Dumbledore zu tun hat. Liege ich da richtig?“

„Kein Kommentar“, sagte Harry.

Wenigstens verarbeitete sein Gehirn diesen Teil richtig. Das Haus Slytherin hatte sich gefragt, warum Severus nicht gefeuert wurde. Und sie waren zu dem Schluss gekommen, dass Severus Dumbledore erpresst hatte. Könnte das tatsächlich wahr sein…? Aber Dumbledore hatte nicht so gewirkt…

Draco redete weiter.

"Und das nächste, worauf die schlauen Leute hinwiesen, war, dass, wenn man genug Druck auf Snape ausüben konnte, um ihn dazu zu bringen, halb Hogwarts in Ruhe zu lassen, das bedeutete, dass man wahrscheinlich genug Macht hatte, um ihn ganz loszuwerden, wenn man wollte.

Was du ihm angetan hast, war eine Demütigung, genauso wie er versucht hat, dich zu demütigen - aber du hast uns unseren Hausoberhaupt gelassen."

Harry verzog das Gesicht zu einem breiten Lächeln.

„Und dann sind die wirklich schlauen Leute“, sagte Draco, sein Gesicht nun ernst,

"losgezogen und haben eine kleine Diskussion unter sich geführt, und jemand hat darauf hingewiesen, dass es eine sehr dumme Sache wäre, einen Feind so in der Nähe zu lassen.

Wenn man seine Macht über Dumbledore brechen könnte, wäre es das Naheliegendste, es einfach zu tun.

Dumbledore würde Snape aus Hogwarts hinauswerfen und ihn vielleicht sogar töten lassen, er wäre dir sehr dankbar und du müsstest dir keine Sorgen machen, dass Snape sich nachts mit interessanten Zaubertränken in deinen Schlafsaal schleicht."

Harrys Gesicht war nun neutral.

Daran hatte er nicht gedacht und das hätte er wirklich, wirklich tun sollen.

„Und daraus schließt ihr…?“

„Snapes Macht war ein Geheimnis von Dumbledore und du hast das Geheimnis!“

Draco schaute jubelnd.

"Es kann nicht mächtig genug sein, um Dumbledore vollständig zu vernichten, sonst hätte Snape es schon längst eingesetzt.

Snape weigert sich, seinen Einfluss für irgendetwas anderes zu nutzen, als König des Hauses Slytherin in Hogwarts zu bleiben, und selbst dann bekommt er nicht immer, was er will, also muss es Grenzen haben.

Aber es muss wirklich gut sein! Vater versucht schon seit Jahren, Snape dazu zu bringen, es ihm zu sagen!?"

„Und“, sagte Harry, „jetzt denkt Lucius, dass ich es ihm vielleicht sagen kann. Hast du schon eine Eule bekommen—“

„Das werde ich heute Abend“, sagte Draco und lachte.

„Darin wird stehen“, seine Stimme nahm einen anderen, förmlicheren Tonfall an,

"Mein geliebter Sohn: Ich habe dir bereits von Harry Potters möglicher Bedeutung erzählt.

Wie du bereits erkannt hast, ist seine Bedeutung jetzt noch größer und dringlicher geworden. Wenn du irgendeinen möglichen Weg der Freundschaft oder einen Druckpunkt bei ihm siehst, musst du ihn verfolgen, und die gesamten Ressourcen von Malfoy stehen dir bei Bedarf zur Verfügung."

\emph{Donnerwetter}.

„Nun“, sagte Harry, „ohne zu kommentieren, ob dein ganzes kompliziertes Theoriegebäude wahr ist oder nicht, möchte ich nur sagen, dass wir noch nicht ganz so gute Freunde sind.“

„Ich weiß“, sagte Draco. Dann wurde sein Gesicht sehr ernst, und seine Stimme wurde sogar in der Unschärfe leise. „Harry, ist dir schon mal in den Sinn gekommen, dass, wenn du etwas weißt, von dem Dumbledore nicht will, dass es bekannt wird, Dumbledore dich einfach töten lassen könnte? Und das würde den Jungen-der-lebte von einem potenziellen konkurrierenden Anführer zu einem wertvollen Märtyrer machen.“

„Kein Kommentar“, sagte Harry noch einmal.

An den letzten Teil hatte er auch nicht gedacht. Schien nicht Dumbledores Stil zu sein…aber …

„Harry“, sagte Draco, „du hast offensichtlich unglaubliches Talent, aber du hast keine Ausbildung und keine Mentoren und du machst manchmal dumme Sachen und du brauchst wirklich einen Berater, der weiß, wie man das macht, oder du wirst verletzt werden!“

Dracos Gesicht war grimmig.

„Ah“, sagte Harry. „Einen Berater wie Lucius?“

„Wie ich!“, sagte Draco.

„Ich verspreche dir, deine Geheimnisse vor Vater und vor allen anderen zu bewahren, ich helfe dir einfach, herauszufinden, was du tun willst!“

\emph{Wow}!

Harry sah, dass Zombie-Quirrell durch die Türen herein taumelte.

„Der Unterricht fängt gleich an“, sagte Harry.

„Ich werde darüber nachdenken, was du gesagt hast, es gibt viele Momente, in denen ich mir wünschte, ich hätte deine ganze Ausbildung, es ist nur so, dass ich nicht weiß, wie ich dir so schnell vertrauen kann—“

„Das solltest du nicht“, sagte Draco, „es ist noch zu früh. Siehst du? Ich gebe dir einen guten Rat, auch wenn es mir wehtut. Aber wir sollten uns vielleicht beeilen und engere Freunde werden.“

„Dafür bin ich offen“, sagte Harry, der schon überlegte, wie er das ausnutzen könnte.

„Noch ein Ratschlag“, sagte Draco hastig, während Quirrell sich zu seinem Schreibtisch beugte,

„im Moment macht sich jeder in Slytherin Gedanken über dich, also wenn du uns den Hof machst, was ich glaube, solltest du etwas tun, das Slytherin Freundschaft signalisiert. Bald, so wie heute oder morgen.“

„Hat es nicht gereicht, dass Severus Slytherin bei den Hauspunkten noch bevorteilen darf?“

Es gab keinen Grund, warum Harry nicht die Lorbeeren dafür einheimsen konnte.

Dracos Augen flackerten vor Erkenntnis, dann sagte er schnell:

„Das ist nicht dasselbe, glaub mir, es muss etwas Offensichtliches sein. Schubs deine Schlammblutrivalin Granger gegen eine Wand oder so, jeder in Slytherin wird wissen, was das bedeutet—“

„So läuft das nicht in Ravenclaw, Draco! Wenn du jemanden gegen eine Wand schubsen musst, bedeutet das, dass dein Gehirn zu schwach ist, um ihn auf die richtige Art und Weise zu schlagen und jeder in Ravenclaw weiß das—“

Der Bildschirm auf Harrys Schreibtisch flackerte auf und löste einen plötzlichen Anflug von Nostalgie für Fernsehen und Computer aus.

„Ähem“, sagte Professor Quirrells Stimme und schien Harry persönlich aus dem Bildschirm anzusprechen.

„Bitte nehmt eure Plätze ein.“

Und die Kinder saßen alle und starrten auf die Bildschirme auf ihren Pulten oder blickten direkt auf die große weiße Marmorbühne, auf der Professor Quirrell stand, an sein Pult gelehnt auf dem kleinen Podest aus dunklerem Marmor.

„Heute“, sagte Professor Quirrell, "hatte ich geplant, euch euren ersten Verteidigungszauber beizubringen, einen kleinen Schild, der der Vorläufer des heutigen Protego war.

Aber nach reiflicher Überlegung habe ich den heutigen Unterrichtsplan im Lichte der jüngsten Ereignisse geändert."

Professor Quirrells Blick suchte die Sitzreihen ab.

Harry zuckte zusammen, als er in der letzten Reihe saß. Er hatte das Gefühl, dass er wusste, wer gleich aufgerufen werden würde.

„Draco, aus dem edlen und sehr alten Haus Malfoy“, sagte Professor Quirrell.

\emph{Uff}.

„Ja, Professor?“, sagte Draco.

Seine Stimme wurde verstärkt und schien von dem Verstärkerbildschirm auf Harrys Schreibtisch zu kommen, der Dracos Gesicht zeigte, während er sprach. Dann wechselte der Bildschirm zurück zu Professor Quirrell, der sagte:

„Ist es Ihr Ehrgeiz, der nächste Dunkle Lord zu werden?“

„Das ist eine merkwürdige Frage, Professor“, sagte Draco.

„Ich meine, wer wäre schon so dumm, das zuzugeben?“

Ein paar Schüler lachten, aber nicht viele.

„In der Tat“, sagte Professor Quirrell. "Es hat zwar keinen Sinn, einen von euch zu fragen, aber es würde mich nicht im Geringsten überraschen, wenn es in meiner Klasse den einen oder anderen Schüler gäbe, der Ambitionen hegt, der nächste Dunkle Lord zu werden.

Immerhin wollte ich der nächste Dunkle Lord sein, als ich ein junger Slytherin war."

Diesmal war das Gelächter viel lauter.

„Nun, es ist immerhin das Haus der Ehrgeizigen“, sagte Professor Quirrell und lächelte.

"Ich habe erst später erkannt, dass das, was mir wirklich Spaß machte, die Kampfmagie war, und dass mein wahrer Ehrgeiz darin bestand, ein großer Kampfzauberer zu werden und eines Tages in Hogwarts zu unterrichten.

Jedenfalls habe ich mit dreizehn Jahren die historischen Abteilungen der Hogwarts-Bibliothek durchgelesen, das Leben und die Schicksale vergangener Dunkler Lords unter die Lupe genommen und eine Liste mit all den Fehlern gemacht, die ich nie machen würde, wenn ich ein Dunkler Lord wäre -"

Harry kicherte, bevor er sich zurückhalten konnte.

„Ja, Mr~Potter, sehr amüsant. Also, Mr~Potter, können Sie erraten, was der allererste Punkt auf dieser Liste war?“

\emph{Na toll.}

„Ähm… nie eine komplizierte Methode anwenden, um mit einem Feind umzugehen, wenn man ihn einfach mit Abrakadabra erledigen kann?“

„Der Begriff, Mr~Potter, ist \emph{Avada Kedavra}“, Professor Quirrells Stimme klang aus irgendeinem Grund ein wenig scharf, „und nein, das stand nicht auf der Liste, die ich mit dreizehn Jahren gemacht habe. Wollen Sie noch einmal raten?“

„Ah…du gibst nie vor anderen mit deinem bösen Masterplan an?“

Professor Quirrell lachte.

„Ah, das war jetzt Nummer 2. Meine Güte, Mr~Potter, haben wir die gleichen Bücher gelesen?“

Es gab noch mehr Gelächter, mit einem Unterton von Nervosität.

Harry klappte seinen Kiefer fest zu und sagte nichts. Ein Leugnen würde nichts bewirken.

"Aber nein. Der erste Punkt war: \emph{'Ich werde nicht herumgehen und starke, bösartige Feinde provozieren.'} Die Weltgeschichte würde ganz anders aussehen, wenn Mornelithe Falconsbane oder Hitler diesen elementaren Punkt begriffen hätten.

Wenn Sie nun, Mr~Potter, zufällig einen ähnlichen Ehrgeiz hegen, wie ich ihn als junger Slytherin hegte, dann hoffe ich, dass es nicht Ihr Ehrgeiz ist, ein dummer Dunkler Lord zu werden."

„Professor Quirrell“, sagte Harry und knirschte mit den Zähnen, "ich bin ein Ravenclaw und es ist nicht mein Ehrgeiz, dumm zu sein, egal was ich tue.

Ich weiß, dass das, was ich heute getan habe, dumm war. Aber es war nicht dunkel! Ich war nicht derjenige, der in diesem Kampf den ersten Schlag getan hat!"

"Sie, Mr~Potter, sind ein Idiot. Aber das war ich in deinem Alter auch. Daher habe ich Ihre Antwort vorausgesehen und den heutigen Unterrichtsplan entsprechend geändert.

Mr~Gregory Goyle, wenn Sie bitte nach vorne kommen würden?"

Es gab eine überraschte Pause im Klassenzimmer.

Damit hatte Harry nicht gerechnet. Genauso wenig wie Mr~Goyle, der ziemlich unsicher und besorgt aussah, als er die Marmorbühne bestieg und sich dem Podium näherte.

Professor Quirrell richtete sich auf, wo er auf dem Pult lehnte. Er sah plötzlich kräftiger aus, seine Hände formten sich zu Fäusten und er richtete sich in einer deutlich erkennbaren Kampfsporthaltung auf. Harrys Augen weiteten sich bei diesem Anblick und ihm wurde klar, warum Mr~Goyle herbeigerufen worden war.

„Die meisten Zauberer“, sagte Professor Quirrell, "geben sich nicht viel mit dem ab, was ein Muggel als Kampfsport bezeichnen würde. Ist ein Zauberstab nicht stärker als eine Faust? Diese Einstellung ist dumm. Zauberstäbe werden in Fäusten gehalten.

Wenn Sie ein großer Kampfzauberer werden wollen, müssen Sie Kampfkünste auf einem Niveau erlernen, das selbst einen Muggel beeindrucken würde. Ich werde jetzt eine bestimmte, lebenswichtige Technik demonstrieren, die ich in einem Dojo gelernt habe, einer Muggel-Kampfkunstschule, über die ich gleich mehr sagen werde. Für den Moment…"

Professor Quirrell machte einige Schritte nach vorne, immer noch in Haltung, und ging auf die Stelle zu, an der Mr~Goyle stand.

„Mr~Goyle, ich werde Sie bitten, mich anzugreifen.“

„Professor Quirrell“, sagte Mr~Goyle, seine Stimme war nun ebenso verstärkt wie die des Professors, „darf ich fragen, welche Stufe—“

„Sechster Dan. Sie werden nicht verletzt und ich auch nicht. Und wenn Sie eine Öffnung sehen, nehmen Sie sie bitte.“

Mr~Goyle nickte und sah sehr erleichtert aus.

„Beachten Sie“, sagte Professor Quirrell, „dass Mr~Goyle Angst hatte, jemanden anzugreifen, der die Kampfkünste nicht auf einem akzeptablen Niveau beherrscht, aus Angst, dass ich oder er verletzt werden könnte. Mr~Goyles Einstellung ist genau richtig und er hat sich dafür drei Quirrell-Punkte verdient. Also, kämpf!“

Der Junge stürmte vorwärts, die Fäuste flogen, und der Professor blockte jeden Schlag ab, er tanzte rückwärts, Quirrell trat und Goyle blockte und drehte sich und versuchte, Quirrell mit einem geschwungenen Bein stolpern zu lassen, aber Quirrell sprang darüber hinweg, \emph{und es ging alles zu schnell, als dass Harry hätte verstehen können, was vor sich ging,} und dann lag Goyle auf dem Rücken und stieß mit den Beinen und Quirrell flog tatsächlich durch die Luft, und dann schlug er mit der Schulter zuerst auf dem Boden auf und rollte.

„Halt!“, rief Professor Quirrell vom Boden aus und klang ein wenig panisch. „Du hast gewonnen!“

Mr~Goyle holte so stark aus, dass er taumelte und durch den abgebrochenen Schwung seines Kopfstoßes auf Professor Quirrell fast stolperte und fiel.

Sein Gesicht zeigte völligen Schock. Professor Quirrell wölbte seinen Rücken und hüpfte mit einer seltsamen Federbewegung, bei der er seine Hände nicht benutzte, auf seine Füße.

Es herrschte eine Stille im Klassenzimmer, eine Stille, die aus völliger Verwirrung geboren war.

„Mr~Goyle“, sagte Professor Quirrell, „welche lebenswichtige Technik habe ich demonstriert?“

„Wie man richtig fällt, wenn man geworfen wird“, sagte Mr~Goyle. „Das ist eine der allerersten Lektionen, die man lernt—“

„Das auch“, sagte Professor Quirrell.

Es gab eine Pause.

„Die lebenswichtige Technik, die ich demonstriert habe“, sagte Professor Quirrell, „war, wie man verliert. Sie können gehen, Mr~Goyle, danke.“

Mr~Goyle ging vom Podium und sah ziemlich verwirrt aus. Harry ging es genauso.

Professor Quirrell ging zurück zu seinem Schreibtisch und lehnte sich wieder darauf.

"Manchmal vergessen wir die grundlegendsten Dinge, weil es zu lange her ist, seit wir sie gelernt haben. Mir wurde klar, dass ich dasselbe mit meinem eigenen Unterrichtsplan getan hatte. Man bringt Schülern nicht das Werfen bei, bevor man ihnen das Fallen beigebracht hat.

Und ich darf euch nicht lehren zu kämpfen, wenn ihr nicht wisst, wie man verliert."

Professor Quirrells Gesicht verhärtete sich, und Harry glaubte, einen Hauch von Schmerz, einen Hauch von Trauer in diesen Augen zu sehen.

"Ich habe das Verlieren in einem Dojo in Asien gelernt, wo, wie jeder Muggel weiß, alle guten Kampfsportler leben. Dieses Dojo lehrte einen Stil, der unter kämpfenden Zauberern den Ruf hatte, sich gut an magische Duelle anzupassen.

Der Meister dieses Dojos - ein alter Mann für Muggelverhältnisse - war der größte lebende Lehrer dieses Stils.

Er hatte natürlich keine Ahnung, dass Magie existierte. Ich bewarb mich, um dort zu studieren, und war einer der wenigen Schüler, die in jenem Jahr angenommen wurden, unter vielen Bewerbern.

Vielleicht war da ein kleines bisschen besonderer Einfluss im Spiel."

Im Klassenzimmer war Gelächter zu hören.

Harry teilte es nicht. Das war ganz und gar nicht richtig gewesen.

„Auf jeden Fall. Bei einem meiner ersten Kämpfe, nachdem ich auf besonders demütigende Weise geschlagen worden war, verlor ich die Kontrolle und griff meinen Sparringspartner an—“

\emph{Oh je.}

„- zum Glück mit meinen Fäusten und nicht mit meiner Magie. Der Meister verwies mich überraschenderweise nicht auf der Stelle. Aber er sagte mir, dass es einen Fehler in meinem Temperament gäbe. Er erklärte es mir, und ich wusste, dass er Recht hatte. Und dann sagte er, dass ich lernen würde, wie man verliert.“

Professor Quirrells Gesicht war ausdruckslos.

"Auf seinen strikten Befehl hin stellten sich alle Schüler des Dojos in einer Reihe auf. Einer nach dem anderen kamen sie auf mich zu.

Ich durfte mich nicht verteidigen. Ich sollte nur um Gnade betteln. Einer nach dem anderen ohrfeigte mich oder schlug mich und stieß mich zu Boden.

Einige von ihnen spuckten mich an. Sie beschimpften mich in ihrer Sprache auf schreckliche Weise. Und zu jedem musste ich sagen: \emph{'Ich verliere!}' und ähnliche Dinge, wie \emph{'Ich bitte euch, aufzuhören!}' und '\emph{Ich gebe zu, dass ihr besser seid als ich!}'"

Harry versuchte, sich das vorzustellen und scheiterte einfach. Es war unmöglich, dass dem ehrwürdigen Professor Quirrell so etwas passieren konnte.

"Ich war schon damals ein Wunderkind der Kampfmagie. Allein mit zauberstabloser Magie hätte ich jeden in diesem Dojo töten können.

Ich habe es nicht getan. Ich habe gelernt, zu verlieren. Bis heute erinnere ich mich daran als eine der unangenehmsten Stunden meines Lebens.

Und als ich dieses Dojo acht Monate später verließ - was nicht annähernd genug Zeit war, aber alles war, was ich mir leisten konnte - sagte mir der Meister, dass er hoffe, ich würde verstehen, warum das notwendig gewesen war.

Und ich sagte ihm, dass es eine der wertvollsten Lektionen war, die ich je gelernt hatte. Und das war und ist wahr."

Professor Quirrells Gesicht wurde bitter.

"Ihr fragt euch, wo dieses wunderbare Dojo ist, und ob ihr dort lernen könnt. Das könnt ihr nicht.

Denn nicht lange danach kam ein anderer Möchtegern-Schüler an diesen versteckten Ort, an diesen abgelegenen Berg. Er-der-nicht-genannt-werden-darf."

Es klang, als ob viele Atemzüge gleichzeitig eingezogen würden. Harry wurde ganz mulmig zumute. Er wusste, was kommen würde.

"Der Dunkle Lord kam offen in diese Schule, ohne Verkleidung, mit glühend roten Augen und allem. Die Schüler versuchten, ihm den Weg zu versperren und er apparierte einfach durch. Es herrschte Terror, aber auch Disziplin, und der Meister trat hervor.

Und der Dunkle Lord verlangte - nicht bat, sondern verlangte -, unterrichtet zu werden."

Professor Quirrells Gesicht war sehr hart.

„Vielleicht hatte der Meister zu viele Bücher gelesen, die die Lüge verbreiteten, ein wahrer Kampfkünstler könne sogar Dämonen besiegen. Aus welchem Grund auch immer, der Meister weigerte sich. Der Dunkle Lord fragte, warum er kein Schüler sein könne. Der Meister sagte ihm, er habe keine Geduld, und da riss ihm der Dunkle Lord die Zunge heraus.“

Es gab ein kollektives Aufatmen.

„Ihr könnt euch denken, was dann geschah. Die Schüler versuchten, sich auf den Dunklen Lord zu stürzen und fielen um, wie betäubt, wo sie standen. Und dann…“

Professor Quirrells Stimme stockte für einen Moment, dann setzte er wieder ein.

"Es gibt einen unverzeihlichen Fluch, den Cruciatus-Fluch, der unerträgliche Schmerzen erzeugt.

Wenn der Cruciatus länger als ein paar Minuten ausgedehnt wird, führt er zu dauerhaftem Wahnsinn. Einer nach dem anderen folterte der Dunkle Lord die Schüler des Meisters bis zum Wahnsinn und tötete sie dann mit dem Tötungsfluch, während der Meister gezwungen war, zuzusehen.

Als alle seine Schüler auf diese Weise gestorben waren, folgte der Meister. Ich erfuhr dies von dem einzigen überlebenden Schüler, den der Dunkle Lord am Leben gelassen hatte, um die Geschichte zu erzählen, und der ein Freund von mir gewesen war…"

Professor Quirrell wandte sich ab, und als er sich einen Moment später wieder umdrehte, wirkte er wieder ruhig und gefasst.

„Dunkle Zauberer können ihr Temperament nicht zügeln“, sagte Professor Quirrell leise.

"Es ist eine fast universelle Schwäche dieser Spezies, und jeder, der es sich zur Gewohnheit macht, gegen sie zu kämpfen, lernt bald, sich darauf zu verlassen.

Verstehen Sie, dass der Dunkle Lord an diesem Tag nicht gewonnen hat. Sein Ziel war es, die Kampfkunst zu erlernen, und doch ging er ohne eine einzige Lektion.

Der Dunkle Lord wäre töricht, diese Geschichte nacherzählt haben zu wollen. Sie zeigte nicht seine Stärke, sondern eher eine ausnutzbare Schwäche."

Professor Quirrells Blick konzentrierte sich auf ein einzelnes Kind im Klassenzimmer.

„Harry Potter“, sagte Professor Quirrell.

„Ja“, sagte Harry, seine Stimme war heiser.

„Was genau haben Sie heute falsch gemacht, Mr~Potter?“

Harry fühlte sich, als müsste er sich übergeben.

„Ich habe meine Beherrschung verloren.“

„Das ist nicht präzise“, sagte Professor Quirrell.

"Ich werde es genauer beschreiben. Es gibt viele Tiere, die so genannte Dominanzkämpfe austragen. Sie stürzen sich mit ihren Hörnern aufeinander - sie versuchen, sich gegenseitig niederzuschlagen, nicht sich zu zerfleischen.

Sie kämpfen mit ihren Pfoten - mit eingefahrenen Krallen. Aber warum mit eingefahrenen Krallen? Wenn sie ihre Krallen benutzen würden, hätten sie doch sicher eine bessere Chance zu gewinnen? Aber dann könnte ihr Gegner auch seine Krallen ausfahren, und statt den Dominanzkampf mit einem Sieger und einem Verlierer zu beenden, könnten beide schwer verletzt werden."

Professor Quirrells Blick schien Harry direkt aus dem Wiederholungsbildschirm heraus anzublicken.

"Was Sie heute demonstriert haben, Mr~Potter, ist, dass Sie - im Gegensatz zu jenen Tieren, die ihre Krallen einziehen und das Ergebnis akzeptieren - nicht wissen, wie man einen Dominanzwettbewerb verliert.

Als ein Hogwarts-Professor Sie herausforderte, haben Sie nicht klein beigegeben. Als es so aussah, als würden Sie verlieren, haben Sie die Krallen ausgefahren, ohne Rücksicht auf die Gefahr.

Sie haben eskaliert, und dann haben Sie wieder eskaliert. Es begann mit einer Ohrfeige von Professor Snape, der offensichtlich über Sie herrscht.

Anstatt zu verlieren, hast haben Sie zurückgeohrfeigt und zehn Punkte von Ravenclaw verloren. Schon bald haben Sie davon gesprochen, Hogwarts zu verlassen.

Die Tatsache, dass Sie noch weiter in eine unbekannte Richtung eskaliert haben und am Ende irgendwie gewonnen haben, ändert nichts an der Tatsache, dass Sie ein Idiot sind."

„Ich verstehe“, sagte Harry.

Seine Kehle war trocken. Das war präzise gewesen. Beängstigend präzise. Jetzt, wo Professor Quirrell es gesagt hatte, konnte Harry im Nachhinein sehen, dass es eine genau zutreffende Beschreibung dessen war, was passiert war. Wenn jemand ein so gutes Modell von einem hatte, musste man sich fragen, ob er auch mit anderen Dingen richtig lag, \emph{zum Beispiel mit seiner Absicht zu töten.}

"Das nächste Mal, Mr~Potter, wenn Sie sich dafür entscheiden, einen Streit lieber eskalieren zu lassen, als zu verlieren, verlieren Sie vielleicht alle Einsätze, die Sie auf den Tisch legen.

Ich kann nicht erraten, wie hoch sie heute waren. Ich kann nur raten, dass sie viel, viel zu hoch waren für den Verlust von zehn Hauspunkten."

\emph{Das Schicksal des magischen Britanniens.}

Das war es, was er getan hatte.

„Sie werden protestieren, dass Sie ganz Hogwarts helfen wollten, ein viel wichtigeres Ziel, das große Risiken wert ist. Das ist eine Lüge. Was hätten Sie tun sollen?“

„Ich hätte die Ohrfeige genommen, gewartet und den bestmöglichen Zeitpunkt gewählt, um meinen Zug zu machen“, sagte Harry mit heiserer Stimme.

„Aber das hätte bedeutet, zu verlieren. Ihn über mich herrschen zu lassen. Das konnte der Dunkle Lord nicht mit dem Meister machen, von dem er lernen wollte.“

Professor Quirrell nickte.

„Wie ich sehe, haben Sie das vollkommen verstanden. Und deshalb, Mr~Potter, werden Sie heute lernen, wie man verliert.“

„Ich—“

"Ich will keine Einwände hören, Mr~Potter. Es ist offensichtlich, dass Sie das brauchen und dass Sie stark genug sind, es zu ertragen.

Ich versichere Ihnen, dass Ihre Erfahrung nicht so hart sein wird wie das, was ich durchgemacht habe, auch wenn Sie sich vielleicht an die schlimmsten fünfzehn Minuten Ihres jungen Lebens erinnern werden."

Harry schluckte.

„Professor Quirrell“, sagte er mit leiser Stimme, „können wir das ein anderes Mal machen?“

„Nein“, sagte Professor Quirrell schlicht.

"Sie sind erst fünf Tage in der Hogwarts-Ausbildung und schon ist das passiert. Heute ist Freitag.

Unsere nächste Verteidigungsstunde ist am Mittwoch. Samstag, Sonntag, Montag, Dienstag, Mittwoch… Nein, wir haben keine Zeit zum Warten."

Darüber gab es ein paar Lacher, aber nur wenige.

"Bitte betrachten Sie es als eine Anweisung Ihres Professors, Mr~Potter. Ich möchte Ihnen sagen, dass ich Ihnen sonst keine Angriffszauber beibringen werde, weil ich dann hören würde, dass Sie jemanden schwer verletzt oder gar getötet haben.

Leider hat man mir gesagt, dass Ihre Finger bereits mächtige Waffen sind. Sie dürfen während dieser Lektion auf keinen Fall mit ihnen schnippen."

Weiteres verstreutes Gelächter, das ziemlich nervös klang. Harry hatte das Gefühl, er könnte weinen.

„Professor Quirrell, wenn Sie irgendetwas von dem tun, worüber Sie gesprochen haben, wird mich das wütend machen, und ich möchte heute wirklich lieber nicht mehr wütend werden—“

„Es geht nicht darum, nicht wütend zu werden“, sagte Professor Quirrell, sein Gesicht sah ernst aus. "Wut ist ganz natürlich. Sie müssen lernen, zu verlieren, auch wenn Sie wütend sind. \emph{Oder zumindest so tun, als ob man verliert, damit man die Rache planen kann.}

\textbf{So wie ich es heute mit Mr~Goyle gemacht habe, es sei denn natürlich, einer von euch glaubt, dass er wirklich besser ist -"}

„Das glaube ich nicht!“, rief Mr~Goyle von seinem Schreibtisch aus und klang ein wenig verzweifelt. „Ich weiß, dass Sie nicht wirklich verloren haben! Bitte planen Sie keine Racheakte!“

Harry fühlte sich mulmig zumute. Professor Quirrell wusste nichts von seiner geheimnisvollen dunklen Seite.

„Professor, wir müssen das wirklich nach dem Unterricht besprechen—“

„Das werden wir“, sagte Professor Quirrell mit dem Tonfall eines Versprechens.

„Nachdem Sie gelernt haben, wie man verliert.“

Sein Gesicht war ernst.

„Es sollte selbstverständlich sein, dass ich alles ausschließe, was Sie verletzen oder auch nur erhebliche Schmerzen bereiten könnte. Der Schmerz wird von der Schwierigkeit kommen, zu verlieren, anstatt sich zu wehren und den Kampf zu eskalieren, bis man gewinnt.“

Harrys Atem kam in kurzen, panischen Zügen. Er hatte mehr Angst, als er es nach dem Verlassen des Zaubertrank-Klassenzimmers gehabt hatte.

„Professor Quirrell“, schaffte er es zu sagen, „ich will nicht, dass Sie wegen dieser Sache gefeuert werden—“

„Das werde ich nicht“, sagte Professor Quirrell, „wenn Sie ihnen hinterher sagen, dass es notwendig war. Und das traue ich Ihnen zu.“

Einen Moment lang wurde Professor Quirrells Stimme sehr trocken.

"Glauben Sie mir, man hat schon Schlimmeres in diesen Fluren geduldet.

Dieser Fall ist nur insofern außergewöhnlich, als das er in einem Klassenzimmer passiert."

„Professor Quirrell“, flüsterte Harry, aber er glaubte, dass seine Stimme immer noch überall zu hören war, „glauben Sie wirklich, dass ich jemanden verletzen könnte, wenn ich das nicht tue?“

„Ja“, sagte Professor Quirrell schlicht.

„Dann“, Harry fühlte sich übel, „werde ich es tun.“

Professor Quirrell drehte sich um und betrachtete die Slytherins.

„Also… mit der vollen Zustimmung eures Lehrers und auf eine Art und Weise, dass Snape nicht für eure Aktionen verantwortlich gemacht werden kann… möchte einer von euch seine Dominanz über den Jungen-der-lebte zeigen? Ihn herumschubsen, zu Boden drücken, ihn um Gnade betteln hören?“

Fünf Hände gingen hoch.

„Jeder mit erhobener Hand, istt ein absoluter Schwachkopf. Welchen Teil von \emph{“so tun, als würde man verlieren„} habt ihr nicht verstanden? Wenn Harry Potter der nächste Dunkle Lord wird, wird er euch jagen und töten, nachdem er seinen Abschluss gemacht hat.“

Die fünf Hände sanken abrupt zurück auf ihre Tische.

„Das werde ich nicht“, sagte Harry, wobei seine Stimme eher schwach klang.

"Ich schwöre, mich niemals an denen zu rächen, die mir helfen, das Verlieren zu lernen.

Professor Quirrell…würden Sie bitte…damit aufhören?"

Professor Quirrell seufzte.

„Es tut mir leid, Mr~Potter. Mir ist klar, dass Sie das genauso ärgerlich finden müssen, ob Sie nun vorhaben, ein Dunkler Lord zu werden oder nicht. Aber diese Kinder hatten auch eine wichtige Lebenslektion zu lernen. Wäre es akzeptabel, wenn ich Ihnen als Entschuldigung einen Quirrell-Punkt zugestehe?“

„Machen Sie zwei daraus“, sagte Harry.

Es gab einen Strom von überraschtem Gelächter, was die Spannung etwas entschärfte.

„Abgemacht“, sagte Professor Quirrell.

„Und nachdem ich meinen Abschluss gemacht habe, werde ich Sie jagen und kitzeln.“

Es gab noch mehr Gelächter, obwohl Professor Quirrell nicht lächelte.

Harry fühlte sich, als würde er mit einer Anakonda ringen und versuchen, das Gespräch in die engen Bahnen zu lenken, die den Leuten klar machen würden, dass er doch kein Dunkler Lord war… warum war Professor Quirrell ihm gegenüber so misstrauisch?

„Professor“, sagte Dracos unverstärkte Stimme. „Es ist auch nicht mein eigener Ehrgeiz, ein dummer Dunkler Lord zu werden.“

Es herrschte eine schockierte Stille im Klassenzimmer. Das musst du nicht tun! Harry hätte das fast laut herausgeplatzt, aber er beherrschte sich noch rechtzeitig; Draco wollte vielleicht nicht, dass bekannt wurde, dass er das aus Freundschaft zu Harry tat… oder aus dem Wunsch heraus, freundlich zu erscheinen… Das als Wunsch, freundlich zu erscheinen, zu bezeichnen, ließ Harry sich klein und gemein fühlen. Wenn Draco vorhatte, ihn zu beeindrucken, funktionierte es perfekt.

Professor Quirrell betrachtete Draco mit ernstem Blick.

„Sie befürchten, dass Sie sich nicht verstellen können, Mr~Malfoy? Dass dieser Makel, der Mr~Potter beschreibt, auch Sie beschreibt? Sicherlich hat Ihr Vater Sie eines Besseren belehrt.“

„Wenn es um das Reden geht, vielleicht“, sagte Draco, jetzt auf dem Verstärkerbildschirm.

„Aber nicht, wenn es darum geht, herumgeschubst und zu Boden gedrückt zu werden. Ich will genau so stark sein wie Sie, Professor Quirrell.“

Professor Quirrells Augenbrauen gingen hoch und blieben oben.

„Ich fürchte, Mr~Malfoy“, sagte er nach einer Weile,

"dass die Vorkehrungen, die ich für Mr~Potter getroffen habe und die einige ältere Slytherins betreffen, denen man hinterher sagen wird, wie dumm sie waren, sich nicht auf Sie übertragen lassen. Aber es ist meine professionelle Meinung, dass Sie bereits sehr stark sind. Sollte ich hören, dass Sie versagt haben, so wie Mr~Potter heute versagt hat, werde ich die entsprechenden Vorkehrungen treffen und mich bei Ihnen und denen, die Sie verletzt haben, entschuldigen.

Ich denke jedoch nicht, dass dies notwendig sein wird."

„Ich verstehe, Professor“, sagte Draco. Professor Quirrell ließ seinen Blick über die Klasse schweifen.

„Möchte noch jemand stark werden?“

Einige Schüler blickten sich nervös um. Einige, so dachte Harry von seiner hinteren Reihe aus, sahen aus, als würden sie den Mund öffnen, aber nichts sagen. Schließlich sprach niemand mehr.

„Draco Malfoy wird einer der Generäle der Armeen eures Jahrgangs sein“, sagte Professor Quirrell,

„sollte er sich dazu herablassen, an dieser außerschulischen Aktivität teilzunehmen. Und nun, Mr~Potter, kommen Sie bitte nach vorne.“

\emph{Ja, hatte Professor Quirrell gesagt, es muss vor allen Leuten sein, vor Ihren Freunden, denn das ist der Ort, an dem Snape Sie konfrontiert hat und das ist der Ort, an dem Sie lernen müssen, zu verlieren.}

Also sahen die Erstklässler nun zu. In magisch erzwungenem Schweigen und mit der Bitte sowohl von Harry als auch vom Professor, nicht einzugreifen.

Hermine hatte ihr Gesicht abgewandt, aber sie hatte nichts gesagt oder ihm auch nur einen bedeutungsvollen Blick zugeworfen, vielleicht weil sie auch in Zaubertränke dabei gewesen war.

Harry stand auf einer weichen blauen Matte, wie man sie in einem Muggel-Dojo finden konnte, die Professor Quirrell auf dem Boden ausgelegt hatte, wenn Harry niedergestoßen wurde.

Harry hatte Angst davor, was er tun könnte. Wenn Professor Quirrell Recht hatte mit seiner Absicht zu töten… Harrys Zauberstab lag auf Professor Quirrells Schreibtisch, nicht weil Harry irgendwelche Zauber kannte, die ihn verteidigen konnten, sondern weil er sonst (so dachte Harry) vielleicht versucht hätte, ihn jemandem in die Augenhöhle zu rammen.

Dort lag sein Beutel, der nun seinen geschützten, aber immer noch potenziell zerbrechlichen Zeitumkehrer enthielt.

Harry hatte Professor Quirrell angefleht, ihm ein paar Boxhandschuhe zu verwandeln und sie an seinen Händen zu befestigen.

Professor Quirrell hatte ihm einen stummen, verständnisvollen Blick zugeworfen und abgelehnt.

\emph{Ich werde nicht auf ihre Augen losgehen, ich werde nicht auf ihre Augen losgehen, ich werde nicht auf ihre Augen losgehen, es wäre das Ende meines Lebens in Hogwarts, ich werde verhaftet werden,} skandierte Harry zu sich selbst und versuchte, den Gedanken in sein Gehirn zu hämmern, in der Hoffnung, dass er dort bleiben würde, wenn seine Absicht zu töten die Oberhand gewinnen würde.

Professor Quirrell kehrte zurück und eskortierte dreizehn ältere Slytherins aus verschiedenen Jahrgängen. Harry erkannte einen von ihnen als denjenigen, den er mit einem Kuchen geschlagen hatte. Zwei andere von dieser Konfrontation waren ebenfalls anwesend.

Derjenige, der gesagt hatte, dass sie aufhören sollten, dass sie das wirklich nicht tun sollten, fehlte.

„Ich wiederhole“, sagte Professor Quirrell und klang dabei sehr streng, „Potter darf nicht wirklich verletzt werden. Alle Unfälle werden als vorsätzlich behandelt. Habt ihr verstanden?“

Die älteren Slytherins nickten und grinsten.

„Dann darfst du den Jungen, der gelebt hat, gerne ein paar Stufen tiefer prügeln“, sagte Professor Quirrell mit einem schiefen Lächeln, das nur die Erstklässler verstanden.

Durch eine Art gegenseitiges Einverständnis war das Torten-Ziel an der Spitze der Gruppe.

„Potter“, sagte Professor Quirrell,

„das ist Mr~Peregrine Derrick. Er ist besser als Sie und er wird Ihnen das gleich zeigen.“

Derrick schritt vorwärts und Harrys Gehirn schrie unharmonisch, \emph{ich darf nicht weglaufen, ich darf mich nicht wehren} - Derrick blieb eine Armlänge von Harry entfernt stehen.

Harry war noch nicht wütend, nur verängstigt. Und das bedeutete, dass er einen Teenager erblickte, der gut einen halben Meter größer war als er selbst, mit klar definierten Muskeln, Gesichtsbehaarung und einem Grinsen von schrecklicher Vorfreude.

„Bitte ihn, dir nichts zu tun“, sagte Professor Quirrell. „Wenn er sieht, dass du erbärmlich genug bist, wird er vielleicht beschließen, dass du langweilig bist, und verschwinden.“

Es gab Gelächter von den zuschauenden älteren Slytherins.

„Bitte“, sagte Harry, und seine Stimme stockte, „tu mir nicht weh…“

„Das klang nicht sehr aufrichtig“, sagte Professor Quirrell.

Derricks Lächeln wurde breiter. \emph{Der tollpatschige Schwachkopf wirkte sehr überlegen und}…… Harrys Bluttemperatur sank…

„Bitte tu mir nicht weh!“, versuchte Harry erneut.

Professor Quirrell schüttelte den Kopf. „Wie in Merlins Namen haben Sie es geschafft, dass das wie eine Beleidigung klingt, Potter? Es gibt nur eine Antwort, die Sie von Mr~Derrick erwarten können.“

Derrick trat absichtlich einen Schritt vor und stieß mit Harry zusammen. Harry taumelte ein paar Schritte zurück und richtete sich, bevor er sich aufhalten konnte auf.

„Falsch“, sagte Professor Quirrell, „falsch, falsch, falsch.“

„Du hast mich angerempelt, Potter“, sagte Derrick. „Entschuldige dich.“

„Es tut mir leid!“

„Du klingst nicht so, als würde es dir leid tun“, sagte Derrick.

Harrys Augen weiteten sich vor Empörung, er hatte es geschafft, das flehend klingen zu lassen - Derrick stieß ihn, hart, und Harry fiel auf Händen und Knien auf die Matte.

Der blaue Stoff schien in Harrys Sicht zu schwanken, nicht weit entfernt. Er begann, an Professor Quirrells wahren Motiven für diese sogenannte Lektion zu zweifeln.

Ein Fuß ruhte auf Harrys Gesäß und einen Moment später wurde Harry hart zur Seite geschubst, so dass er auf den Rücken kippte. Derrick lachte.

„Das macht Spaß“, sagte er.

\emph{Er brauchte nur zu sagen, dass es vorbei war. Und die ganze Sache im Büro des Schulleiters melden. Das wäre das Ende dieses Verteidigungsprofessors und seines unglückseligen Aufenthalts in Hogwarts und… Professor McGonagall würde darüber wütend sein, aber…}

(Ein Bild von Professor McGonagalls Gesicht blitzte vor seinen Augen auf, sie sah nicht wütend aus, nur traurig -)

„Jetzt sag ihm, dass er besser ist als du, Potter“, sagte Professor Quirrells Stimme.

„Du bist, besser, als, ich.“ Harry wollte sich aufrichten, doch Derrick stellte einen Fuß auf seine Brust und schubste ihn zurück auf die Matte.

Die Welt wurde durchsichtig wie Kristall. Handlungsstränge und ihre Konsequenzen dehnten sich in völliger Klarheit vor ihm aus.

\emph{Der Narr würde nicht damit rechnen, dass er zurückschlagen würde, ein schneller Schlag in die Leiste würde ihn lange genug betäuben, um—}

„Versuchen Sie es noch einmal“, sagte Professor Quirrell und mit einer plötzlichen scharfen Bewegung rollte sich Harry ab, sprang auf die Füße und wirbelte herum, wo sein wahrer Feind stand, der Verteidigungsprofessor—

Professor Quirrell sagte: „Sie haben keine Geduld.“

Harry schwankte.

Sein in Pessimismus geübter Verstand zeichnete das Bild eines verhutzelten alten Mannes, dem Blut aus dem Mund floss, nachdem Harry ihm die Zunge herausgerissen hatte - Einen Moment später stieß Derrick Harry wieder auf die Matte und setzte sich dann auf ihn, was Harrys Atem zischend ausstieß.

„Stopp!“ Harry schrie. „Bitte hör auf!“

„Besser“, sagte Professor Quirrell. „Das klang sogar aufrichtig.“

\emph{Das war es auch.}

Das war das Schreckliche, das Abscheuliche, es war aufrichtig gewesen. Harry keuchte schnell, Angst und kalte Wut durchströmten ihn.

„Verliere“, sagte Professor Quirrell.

„Ich, verlieren“, presste Harry hervor.

„Das gefällt mir“, sagte Derrick von oben auf ihm. „Verliere noch mehr.“

Hände schubsten Harry und schickten ihn quer durch den Kreis der älteren Slytherins zu einem anderen Paar Hände, die ihn wieder schubsten. Harry hatte schon lange den Punkt überschritten, an dem er versuchte, nicht zu weinen, und versuchte jetzt nur noch, nicht umzufallen.

„Was bist du, Potter?“, fragte Derrick.

„Ein, l-Verlierer, ich verliere, ich gebe auf, du gewinnst, du bist b-besser, als ich, bitte hör auf—“

Harry stolperte über einen Fuß und stürzte zu Boden, die Hände konnten sich nicht ganz fangen.

Er war einen Moment lang benommen, dann kämpfte er sich wieder auf die Beine—

„Genug!“, sagte Professor Quirrells Stimme, die scharf genug klang, um Eisen zu schneiden.

„Gehen Sie weg von Mr~Potter!“

Harry sah die überraschten Blicke auf ihren Gesichtern. Der Schauer in seinem Blut, der bis dahin geflossen war, lächelte in kalter Zufriedenheit. Dann sackte Harry auf die Matte. Professor Quirrell sprach. Die älteren Slytherins schnappten nach Luft.

„Und ich glaube, der Spross von Malfoy hat euch auch etwas zu erklären“, beendete Professor Quirrell.

Dracos Stimme begann zu sprechen. Seine Stimme klang fast so scharf wie die von Professor Quirrell, sie hatte dieselbe Kadenz angenommen, mit der Draco seinen Vater imitiert hatte, und sie sagte Dinge wie:

"Das Haus Slytherin könnte in Gefahr sein und wer weiß, wie viele Verbündete es allein in dieser Schule gibt, und dass ihr völlig unaufmerksam seid, geschweige denn, dass ihr gerissen und statt dessen stumpfsinnig sind und zu nichts anderem taugen als zu Lakaien und Ritualopfern, \emph{und irgendetwas in Harrys Hinterhirn bezeichnete Draco trotz allem, was er wusste, als Verbündeten}.

Harry schmerzte am ganzen Körper, war wahrscheinlich geprellt, sein Körper fühlte sich kalt an, sein Geist war völlig erschöpft. Er versuchte, an Fawkes' Lied zu denken, aber ohne die Anwesenheit des Phönix konnte er sich nicht an die Melodie erinnern und als er versuchte, es sich vorzustellen, konnte er an nichts anderes denken als an einen zwitschernden Vogel.

Dann hörte Draco auf zu sprechen und Professor Quirrell teilte den älteren Slytherins mit, dass sie entlassen seien, und Harry öffnete seine Augen und setzte sich mühsam auf.

„Wartet“, sagte Harry und zwang die Worte heraus, „ich möchte ihnen etwas sagen—“

„Warten Sie, Mr~Potter“, sagte Professor Quirrell kalt zu den abreisenden Slytherins. Harry schwankte auf seine Füße. Er achtete darauf, nicht in die Richtung seiner Klassenkameraden zu schauen. Er wollte nicht sehen, wie sie ihn jetzt ansahen. Er wollte ihr Mitleid nicht sehen. Also schaute Harry stattdessen zu den älteren Slytherins, die immer noch in einem Schockzustand zu sein schienen. Sie starrten zurück zu ihm. Entsetzen stand auf ihren Gesichtern. Seine dunkle Seite, wenn sie die Kontrolle hatte, hatte sich an die Vorstellung dieses Moments gehalten und \emph{tat weiterhin so, als würde er verlieren.}

Harry sagte: „Keiner wird—“

„Stopp“, sagte Professor Quirrell. „Wenn es das ist, wofür ich es halte, dann warten Sie bitte, bis sie weg sind. Sie werden es später erfahren. Wir haben alle unsere Lektionen zu lernen, Mr~Potter.“

„In Ordnung“, sagte Harry.

„Ihr. Geht.“

Die älteren Slytherins flohen und die Tür schloss sich hinter ihnen.

„Keiner soll sich an ihnen rächen“, sagte Harry heiser.

„Das ist eine Bitte an jeden, der sich für meinen Freund hält. Ich hatte meine Lektion zu lernen, sie haben mir geholfen, sie hatten auch ihre Lektion zu lernen, es ist vorbei. Wenn ihr diese Geschichte erzählt, sorgt dafür, dass ihr auch diesen Teil erzählt.“

Harry drehte sich um und sah Professor Quirrell an.

„Du hast verloren“, sagte Professor Quirrell, seine Stimme war zum ersten Mal sanft. Es klang seltsam aus dem Munde des Professors, als sollte seine Stimme das gar nicht können.

Harry hatte verloren. Es hatte Momente gegeben, in denen die kalte Wut völlig verblasst und durch Angst ersetzt worden war, und in diesen Momenten hatte er die älteren Slytherins angefleht, und er hatte es ernst gemeint…

„Und Sie sind noch am Leben?“, fragte Professor Quirrell, immer noch mit dieser seltsamen Sanftmut.

Harry schaffte es zu nicken.

„Nicht alle Verluste sind so“, sagte Professor Quirrell. „Es gibt Kompromisse und ausgehandelte Kapitulationen. Es gibt andere Wege, Tyrannen zu besänftigen. Es gibt eine ganze Kunstform, andere zu manipulieren, indem man ihnen erlaubt, über einen zu herrschen. Aber zuerst muss das Verlieren denkbar sein. Werden Sie sich daran erinnern, wie Sie verloren haben?“

„Ja.“

„Wirst du in der Lage sein, zu verlieren?“

„Ich…denke schon…“

„Das glaube ich auch.“

Professor Quirrell verbeugte sich so tief, dass sein dünnes Haar fast den Boden berührte.

„Glückwunsch, Harry Potter, du hast gewonnen.“

Es gab keine einzelne Quelle, keinen Erststarter, der Applaus setzte auf einmal ein wie ein gewaltiger Donnerschlag.

Harry konnte den Schock nicht aus seinem Gesicht halten. Er riskierte einen Blick auf seine Klassenkameraden und sah, dass ihre Gesichter nicht Mitleid, sondern Ehrfurcht zeigten.

Der Applaus kam aus Ravenclaw und Gryffindor und Hufflepuff und sogar aus Slytherin, wahrscheinlich weil auch Draco Malfoy applaudierte.

Einige Schüler standen von ihren Stühlen auf und halb Gryffindor stand auf ihren Pulten. So stand Harry da, schwankend, ließ ihren Respekt über sich ergehen und fühlte sich stärker und vielleicht sogar ein wenig geheilt.

Professor Quirrell wartete darauf, dass der Applaus abebbte. Es dauerte eine ganze Weile.

„Überrascht, Mr~Potter?“ sagte Professor Quirrell. Seine Stimme klang amüsiert.

"Sie haben gerade herausgefunden, dass die reale Welt nicht immer so funktioniert wie Ihre schlimmsten Albträume. Ja, wenn Sie irgendein armer anonymer Junge gewesen wären, der missbraucht wurde, dann hätten sie Sie danach wahrscheinlich weniger respektiert, Sie bemitleidet, auch wenn sie Sie von ihren hohen Stühlen aus getröstet hätten.

Das ist die menschliche Natur, fürchte ich. Aber Sie kennen sie bereits als eine Figur der Macht. Und sie haben gesehen, wie Sie sich der Angst gestellt haben und sich ihr immer wieder gestellt haben, obwohl Sie jederzeit hätten weggehen können.

Haben Sie weniger von mir gehalten, als ich Ihnen sagte, dass ich es bewusst ertragen habe, bespuckt zu werden?"

Harry spürte ein brennendes Gefühl in seiner Kehle und presste sich krampfhaft zusammen.

Er traute diesem wundersamen Respekt nicht genug, um vor ihm wieder zu weinen.

„Diese außergewöhnliche Leistung in meiner Klasse verdient eine außergewöhnliche Belohnung, Harry Potter. Bitte nimm sie im Namen meines Hauses entgegen und denke von heute an daran, dass nicht alle Slytherins gleich sind. Es gibt Slytherins und dann gibt es Slytherins.“

Professor Quirrell lächelte ziemlich breit, als er dies sagte.

„Einundfünfzig Punkte für Ravenclaw.“

Es gab eine schockierte Pause und dann brach unter den Ravenclaw-Schülern ein Pandämonium aus, sie johlten und pfiffen und jubelten.

(Und im selben Moment fühlte Harry, dass daran etwas nicht stimmte, Professor McGonagall hatte Recht gehabt, es hätte Konsequenzen geben müssen, es hätte einen Preis geben müssen, den man hätte zahlen müssen, man konnte nicht einfach alles wieder so hinstellen, wie es war -)

Aber Harry sah die begeisterten Gesichter in Ravenclaw und wusste, dass er unmöglich Nein sagen konnte. Sein Gehirn machte einen Vorschlag. Es war ein guter Vorschlag. Harry konnte nicht einmal glauben, dass sein Gehirn ihn noch aufrecht hielt, geschweige denn gute Vorschläge produzierte.

„Professor Quirrell“, sagte Harry so deutlich wie möglich durch seine brennende Kehle.

„Sie sind alles, was ein Mitglied Ihres Hauses sein sollte, und ich glaube, Sie sind genau das, was Salazar Slytherin im Sinn hatte, als er half, Hogwarts zu gründen. Ich danke Ihnen und Ihrem Haus“, Draco nickte ganz leicht und drehte dezent den Daumen nach oben,

„und ich denke, das schreit nach einem dreifachen Hoch auf Slytherin. Alle mit mir?“

Harry hielt inne.

„Huzzah!“

Nur ein paar Leute schafften es, beim ersten Versuch mit einzustimmen.

„Huzzah!“

Diesmal waren fast alle aus Ravenclaw dabei.

„Huzzah!“

Das war fast ganz Ravenclaw, ein paar versprengte Hufflepuffs und etwa ein Viertel von Gryffindor. Dracos Hand bewegte sich zu einer kleinen, schnellen, Daumen-hoch-Geste. Die meisten Slytherins hatten Ausdrücke des puren Schocks.

Ein paar starrten Professor Quirrell erstaunt an. Blaise Zabini sah Harry mit einem berechnenden, faszinierten Ausdruck an.

Professor Quirrell verbeugte sich.

„Danke, Harry Potter“, sagte er, immer noch mit diesem breiten Lächeln.

Er wandte sich an die Klasse.

„Nun, ob Sie es glauben oder nicht, wir haben noch eine halbe Stunde in dieser Sitzung übrig, und das reicht aus, um den Einfachen Schild vorzustellen. Mr~Potter geht natürlich weg und gönnt sich eine wohlverdiente Pause.“

„Ich kann—“

„Idiot“, sagte Professor Quirrell liebevoll.

Die Klasse war bereits am Lachen.

"Deine Klassenkameraden können Sie nachher unterrichten, oder ich gebe Ihnen Nachhilfe, wenn es sein muss auch privat.

Aber jetzt gehen Sie erst einmal durch die dritte Tür von links in den hinteren Teil der Bühne, wo Sie ein Bett, eine Auswahl an außergewöhnlich leckeren Snacks und etwas extrem leichte Lektüre aus der Hogwarts-Bibliothek findet. Sie dürfen nichts anderes mitnehmen, schon gar nicht Ihre Lehrbücher. Jetzt geh."

Harry ging.

