

\hypertarget{omake-alternative-parallelen}{% \section{64. Omake: Alternative Parallelen}\label{omake-alternative-parallelen}}

\textbf{\uline{Alternative Parallelen}}

Anmerkung des Übersetzers:

dies hier sind einige Anspielungen und Gedankengänge zu alternativen Geschichten wenn sich E.Y. nicht für Harry Potter entschieden hätte.

Zitat des original Autors:

Wenn du fünf Stunden nach deiner Schlafenszeit bist und das hier immer noch liest, würde ich vorschlagen, etwas zu schlafen. Die Fic wird morgen immer noch hier sein…

es sei denn, es passiert etwas Schlimmes damit und am nächsten Morgen gibt es nur noch eine 404 unter dieser Adresse und du hast nichts als eine verblassende Erinnerung und ein ewiges Bedauern, daß du nicht länger wach geblieben bist und weitergelesen hast, als du noch die Chance dazu hattest…

aber hey, wie wahrscheinlich ist das?

Diese Geschichte verbreitet sich durch Bloggen, Tweeten, Mundpropaganda, Favorisieren, Einstellen in Foren und Hinzufügen zu Listen; und denk daran, wenn die Leser vor dir sich nicht einen Moment Zeit genommen hätten, das zu tun, hättest du dies wahrscheinlich nicht gefunden. Wenn das nicht genug ist, um dich zu motivieren, dann lass mich hinzufügen, dass Hermine traurig sein wird, wenn du nicht hilfst, Rationalität zu verbreiten.

Du willst doch nicht, dass sie traurig ist, oder?

Vergiss nicht, LessWrong.com zu besuchen und die Sequenzen zu lesen, von deren wahrer Existenz diese Fic nur ein Schatten ist. Ich empfehle, mit der Sequenz \emph{How to Actually Change Your Mind} zu beginnen. Und nun, da alle Universen ihren jeweiligen Schöpfern gehören, präsentiere ich:

\textbf{\uline{OMAKE FILES: DIE ANDEREN FANFICTIONEN, DIE HÄTTEN SEIN KÖNNEN}}

\textbf{LORD OF THE RATIONALITY}

Frodo blickte in alle Gesichter, aber sie waren nicht zu ihm gewandt. Der ganze Rat saß mit niedergeschlagenen Augen da, als ob er tief in Gedanken versunken wäre. Ein großes Grauen überkam ihn, als erwarte er die Verkündung eines Unheils, das er schon lange vorausgesehen und vergeblich gehofft hatte, dass es vielleicht doch nie ausgesprochen würde. Eine überwältigende Sehnsucht, an Bilbos Seite in Bruchtal zu ruhen und in Frieden zu bleiben, erfüllte sein ganzes Herz. Endlich sprach er mit einer Anstrengung und wunderte sich, seine eigenen Worte zu hören, als ob ein anderer Wille seine kleine Stimme benutzte.

„Wir können nicht“, sagte Frodo. „Wir dürfen nicht. Versteht ihr nicht? Es ist genau das, was der Feind will. All das hat er vorausgesehen.“

Die Gesichter wandten sich ihm zu, verwundert die Zwerge und ernst die Elben; Strenge in den Augen der Menschen; und so scharf die Blicke von Elrond und Gandalf, dass Frodo ihnen fast nicht standhalten konnte. Es war sehr schwer, den Ring nicht in die Hand zu nehmen, und noch schwerer, ihn nicht aufzusetzen, um ihnen als intelligenterer Frodo gegenüberzutreten.

„Stellt ihr es nicht in Frage?“ sagte Frodo, dünn wie der Wind seine Stimme, und schwankend wie ein Windhauch. „Ihr habt euch ausgerechnet dafür entschieden, den Ring nach Mordor zu schicken; solltet ihr euch nicht wundern? Wie konnte es so weit kommen? Dass wir von all unseren Möglichkeiten ausgerechnet das tun, was unser Feind am meisten begehrt? Vielleicht sind die Risse des Verderbens bereits bewacht, stark genug, um Gandalf und Elrond und Glorfindel alle zusammen abzuhalten; oder vielleicht hat der Herr dieses Ortes die Lava dort gekühlt, sie so eingestellt, dass sie den Ring einschließt, damit er ihn einfach herausholen kann, nachdem er hineingeworfen wurde…“

Da überkam Frodo eine Erinnerung von schrecklicher Klarheit und ein Aufblitzen von schwarzem Lachen, und der Gedanke kam ihm, dass es genau das war, was der Feind tun würde. Nur der Gedanke kam ihm so: S\emph{o würde es mich amüsieren, wenn ich zu herrschen beabsichtigte…}

Im Rat wurden zweifelnde Blicke ausgetauscht; Gimli und Boromir sahen den Elben nun skeptischer an als zuvor, als wären sie aus einem Traum von Worten erwacht.

„Der Feind ist sehr weise“, sagte Gandalf, „und wägt alle Dinge in der Waage seiner Bosheit genau ab. Aber das einzige Maß, das er kennt, ist das Verlangen, das Verlangen nach Macht; und so richtet er alle Herzen. In sein Herz wird der Gedanke nicht eindringen, daß irgend jemand es ablehnen wird, daß wir, die wir den Ring haben, danach trachten, ihn zu zerstören—“

„Er wird daran denken!“, rief Frodo. Er rang nach Worten und versuchte, Dinge auszudrücken, die ihm einst vollkommen erschienen und dann wie schmelzender Schnee verblassten. „Wenn der Feind glaubte, dass alle seine Feinde allein von Machtgelüsten bewegt würden - er würde sich irren, immer wieder, und der Schöpfer dieses Ringes würde das sehen, er würde wissen, dass er sich irgendwo geirrt hat!“

Frodo streckte flehend die Hände aus. Boromir regte sich, und seine Stimme war zweifelnd. „Du sprichst gut von dem Feind“, sagte Boromir, „für einen seiner Feinde.“

Frodos Mund öffnete und schloss sich in verzweifelter Verwirrung; denn Frodo wusste, er wusste, dass der Mann verrückt war, aber ihm fiel nichts ein, was er sagen konnte.

Dann sprach Bilbo, und seine welke Stimme ließ den ganzen Raum verstummen, sogar Elrond, der gerade etwas sagen wollte. „Frodo hat recht, fürchte ich“, flüsterte der alte Hobbit. „Ich erinnere mich, ich erinnere mich daran, wie es war. Mit dem Schwarzen Auge zu sehen. Ich erinnere mich. Der Feind wird denken, dass wir einander nicht trauen könnten, dass der Schwächere unter uns vorschlagen wird, den Ring zu zerstören, damit der Stärkere ihn nicht haben kann. Er weiß, dass selbst jemand, der nicht wahrhaftig gut ist, trotzdem danach rufen könnte, den Ring zu zerstören, um den anderen seine Güte vorzuspielen. Und der Feind wird es nicht für unmöglich halten, dass eine solche Entscheidung von diesem Rat getroffen wird, denn ihr seht, er traut uns nicht zu, weise zu sein.“ Ein flüsterndes Glucksen stieg aus der Kehle des alten Hobbits auf. „Und wenn er es täte - er würde dann immer noch die Risse des Verderbens bewachen. Es würde ihn wenig kosten.“

Jetzt stand die Vorahnung auch auf den Gesichtern der Elben und der Weisen; Elrond hatte die Stirn gerunzelt, und die scharfen Augenbrauen von Gandalf runzelten sich.

Frodo starrte sie alle an und fühlte, wie eine Wildheit über ihn kam, eine Verzweiflung; und als sein Herz schwächer wurde, kam ein Schatten über seine Vision, eine Dunkelheit und ein Schwanken. Aus dem Schatten heraus sah Frodo Gandalf, und die Stärke des Zauberers entpuppte sich als Schwäche, und seine Weisheit als Torheit. Denn Frodo wusste, während der Ring an seiner Brust zu zerren und zu wiegen schien, dass Gandalf überhaupt nicht an Geschichte und Überlieferung gedacht hatte, als der Zauberer davon sprach, dass der Feind kein anderes Verlangen als das nach Macht verstehen würde; dass Gandalf sich nicht daran erinnert hatte, wie Sauron die Menschen von Numenor in den Tagen ihrer Herrlichkeit niedergeworfen und verdorben hatte. Genauso wenig wie es Gandalf in den Sinn gekommen war, dass der Feind lernen könnte, Feinde guten Willens zu verstehen, indem er in Sie hineinblickt…

Frodos Blick schwenkte zu Elrond, aber dort gab es keine Hoffnung, keine Antwort und keine Rettung in der schattenhaften Vision; denn Elrond hatte Isildur gehen lassen, als er den Ring aus den Rissen des Schicksals trug, wo er hätte zerstört werden sollen, zum Preis dieses ganzen Krieges. Nicht um Isildurs willen, nicht aus Freundschaft war es getan worden, denn der Ring hatte Isildur am Ende getötet, und weit schlimmere Schicksale hätten ihm folgen können. Aber das Verhängnis, das von Isildurs Tat ausging, wäre Elrond damals ungewiss erschienen, ungewiss und zeitlich weit entfernt; und doch wäre der Preis, den Elrond selbst zu zahlen hatte, wenn er den Knauf seines Schwertes an Isildurs Hinterkopf führte, sicherer und näher gewesen…

Wie in Verzweiflung drehte sich Frodo um und sah Aragorn an, den wettergegerbten Mann, der für diesen Rat seine abgetragene Kleidung angezogen hatte, den Erben der Könige, der sanft zu den Hobbits sprach. Aber Frodos Sicht schien sich zu verdoppeln, und in dem schattenhaften zweiten Bild sah Frodo einen Mann, der zu viel von seiner Jugend unter Elben verbracht hatte, der gelernt hatte, bescheidene und befleckte Kleidung inmitten von Gold und Juwelen zu tragen, wissend, dass er es nicht mit ihnen aufnehmen konnte, Weisheit um Weisheit, und in der Hoffnung, sie auf eine Weise zu übertrumpfen, die sie nicht nachahmen würden…

Im Anblick des Ringes, der der Anblick seines eigenen Schöpfers war, verblassten alle edlen Dinge zu List und Lüge, eine Welt aus Grau und Dunkelheit ohne jedes Licht. Sie hatten ihre Entscheidungen nicht wissentlich getroffen, Gandalf oder Elrond oder Aragorn; die Impulse waren aus den dunklen, verborgenen Teilen ihrer selbst gekommen, den schwarzen, geheimen Tiefen, die der Ring in Frodos Vision deutlich gemacht hatte. Würden sie den Schatten überlisten, wenn sie nicht einmal ihr eigenes Selbst oder die Kräfte, die sie bewegten, begreifen konnten?

„Frodo!“ kam das scharfe Flüstern von Bilbos Stimme, und Frodo kam zu sich und hielt inne, als er seine Hand nach oben streckte, dorthin, wo der Ring auf seiner Brust lag, an seiner Kette, die sich wie ein riesiger Stein um seinen Hals zog. Er griff nach oben, um den Ring zu ergreifen, in dem alle Antworten lagen.

„Wie hast du dieses Ding ertragen?“ flüsterte Frodo Bilbo zu, als ob die beiden die einzigen Seelen im Raum wären, obwohl der ganze Rat sie beobachtete. „Jahrelang? Ich kann es mir nicht vorstellen.“

„Ich habe es in einem Raum verschlossen gehalten, zu dem nur Gandalf den Schlüssel hatte“, sagte sein Onkel, „und als ich anfing, mir Wege auszudenken, es zu öffnen, erinnerte ich mich an Gollum.“

Ein Schauder durchlief Frodo, als er sich an die Geschichten erinnerte. Der Schrecken des Nebelgebirges, das Denken, immer im Dunkeln; das Beherrschen der Orks aus den Schatten heraus und das Füllen der Tunnel mit Fallen; wenn Bilbo den Ring nicht zum ersten Mal getragen hätte, hätte kein einziger Zwerg überlebt. Und nun, so hatte ihnen Legolas, der Elf, erzählt, hatte Gollum es aufgegeben, seine Agenten gegen das Auenland zu schicken, hatte endlich den Mut gefunden, seine Berge zu verlassen und den Ring selbst zu suchen. Das war Gollum, das Schicksal, das Frodo selbst teilen würde, wenn der Ring nicht zerstört würde. Nur hatten sie keine Möglichkeit, den Ring zu zerstören. Der Schatten hatte jeden ihrer Schritte vorausgesehen.

Beinahe - Frodo konnte sich immer noch nicht vorstellen, wie es geschehen war, wie der Schatten so etwas arrangiert hatte - beinahe hätte er den Rat dazu gebracht, den Ring direkt nach Mordor zu schicken, mit nur einer winzigen Wache, wie sie es getan hätten, wenn Frodo und Bilbo nicht da gewesen wären. Und nachdem er auf diese schnellste aller möglichen Niederlagen verzichtet hatte, blieb nur noch die Frage, wie lange es dauern würde, den Krieg und den Ring zu verlieren. Gandalf hatte zu lange gezögert, viel zu lange gezögert, um diesen Marsch in Gang zu setzen. Es hätte so einfach sein können, wenn Bilbo nur achtzig Jahre früher aufgebrochen wäre, wenn Bilbo nur gesagt worden wäre, was Gandalf bereits vermutet hatte, wenn Gandalfs Herz nicht stillschweigend vor der Aussicht zurückgeschreckt wäre, peinlich falsch zu liegen…

Frodos Hand krampfte auf seiner Brust; ohne nachzudenken, begannen sich seine Finger wieder zu dem gewaltigen Gewicht der Kette zu erheben, an der der Ring hing. Alles, was er tun musste, war, den Ring aufzusetzen. Nur das, und alles würde ihm klar werden, noch einmal würde die Langsamkeit und der Schlamm aus seinen Gedanken weichen, alle Möglichkeiten und Zukünfte würden ihm durchsichtig werden, er würde die Pläne des Schattens durchschauen und einen unwiderstehlichen Gegenschlag ersinnen—

—und er würde den Ring nie mehr abnehmen können, nicht mehr, nicht durch irgendeinen Willen, der ihm bleiben würde. Alles, was Frodo von diesen Momenten hatte, waren verblassende Erinnerungen, aber er wusste, dass es sich wie Sterben angefühlt hatte, all seine Gedankentürme einstürzen zu lassen und nur noch Frodo zu sein. Es hatte sich wie Sterben angefühlt, so sehr er sich an die Wetterspitze erinnerte, auch wenn er sich an wenig anderes erinnerte. Und wenn er den Ring noch einmal trug, wäre es besser, mit ihm am Finger zu sterben, sein Leben zu beenden, solange er noch er selbst war; denn Frodo wusste, dass er die Auswirkungen des Tragens des Rings ein zweites Mal nicht aushalten konnte, nicht, wenn ihm danach die grenzenlose Klarheit abhanden gekommen war…

Frodo sah sich im Rat um, bei den armen, verlorenen, führerlosen Weisen, und er wusste, dass sie den Schatten nicht aus eigener Kraft besiegen konnten. „Ich werde ihn ein letztes Mal tragen“, sagte Frodo, seine Stimme gebrochen und versagend, wie er von Anfang an gewusst hatte, dass er es am Ende sagen würde, „ein letztes Mal, um die Antwort für diesen Rat zu finden, und dann wird es andere Hobbits geben.“

„Nein!“, schrie die Stimme von Sam, als der andere Hobbit von dort, wo er sich versteckt hatte, nach vorne stürmte; gerade als Frodo mit einer Bewegung so schnell und präzise wie ein Nazgul den Ring unter seinem Hemd hervorzog; und irgendwie stand Bilbo schon da und hatte seinen Finger schon hindurchgestochen. Das alles geschah, bevor auch nur Gandalf seinen Stab nehmen konnte, bevor Aragorn die Spitze seines Schwertes ausrichten konnte; die Zwerge schrien entsetzt auf, und die Elben waren bestürzt.

„Natürlich“, sagte Bilbos Stimme, als Frodo zu weinen begann, „jetzt sehe ich es, jetzt verstehe ich endlich alles. Hört zu, hört zu und schnell, hier ist, was ihr tun müsst—“

\textbf{DIE HEXE UND DER WÄRTER}

Mit kritischem Blick betrachtete Peter die Zentauren mit ihren Bögen, die Biber mit ihren langen Dolchen und die sprechenden Bären mit ihren übergestülpten Kettenhemden. Er hatte das Sagen, denn er war einer der mythischen Söhne Adams und hatte sich zum Hochkönig von Narnia erklärt; aber die Wahrheit war, dass er nicht wirklich viel über Lager, Waffen und Wachpatrouillen wusste. Am Ende konnte er nur sehen, dass sie alle stolz und selbstbewusst aussahen, und Peter musste hoffen, dass sie damit Recht hatten; denn wenn man nicht an seine eigenen Leute glauben konnte, konnte man an niemanden glauben.

„Sie würden mir Angst machen, wenn ich gegen sie kämpfen müsste“, sagte Peter schließlich, „aber ich weiß nicht, ob das reicht, um…sie zu besiegen.“

„Du glaubst doch nicht, dass dieser geheimnisvolle Löwe tatsächlich auftaucht und uns hilft, oder?“, sagte Lucy. Ihre Stimme war sehr leise, damit keines der Wesen um sie herum sie hören würde.

„Es wäre nur schön, ihn wirklich zu haben, meinst du nicht, anstatt die Leute nur glauben zu lassen, dass er uns die Verantwortung übertragen hat?“

Susan schüttelte den Kopf und schüttelte die magischen Pfeile in dem Köcher auf ihrem Rücken. „Wenn es wirklich so jemanden gäbe“, sagte Susan, „dann hätte er nicht zugelassen, dass die Weiße Hexe das Land hundert Jahre lang mit Winter bedeckt, oder?“

„Ich hatte den seltsamsten Traum“, sagte Lucy, ihre Stimme wurde noch leiser, „in dem wir keine Wesen organisieren oder sie zum Kämpfen überreden mussten, wir kamen einfach an diesen Ort, und der Löwe war schon da, mit allen Armeen, die bereits versammelt waren, und er ging hin und rettete Edmund, und dann ritten wir an seiner Seite in diese gewaltige Schlacht, in der er die Weiße Hexe tötete…“

„Hatte der Traum eine Moral?“, fragte Peter.

„Ich weiß es nicht“, sagte Lucy, blinzelte und sah ein wenig verwirrt aus. „In dem Traum schien alles irgendwie sinnlos zu sein.“

„Ich glaube, vielleicht wollte dir das Land Narnia etwas sagen“, sagte Susan, „oder vielleicht waren es nur deine eigenen Träume, die dir sagen wollten, dass, wenn es wirklich so etwas wie diesen Löwen gäbe, wir keinen Nutzen davon hätten.“

\textbf{MY LITTLE PONY: FREUNDSCHAFT IST WISSENSCHAFT}

„Applejack, die mir geradeheraus gesagt hat, dass ich mich geirrt habe, repräsentiert den Geist der… Ehrlichkeit! “

Twilight Sparkle hob ihren Kopf noch höher, die Mähne wehte wie ein Windhauch um den düsteren Himmel ihres Halses.

„Fluttershy, die sich dem Mantikor näherte, um etwas über den Dorn in seiner Pfote herauszufinden, repräsentiert den Geist der… Untersuchung! Pinkie Pie, die erkannte, dass die schrecklichen Gesichter nur Bäume waren, repräsentiert den Geist von… dem Formulieren alternativer Hypothesen! Rarity, die das Problem der Schlange gelöst hat, repräsentiert den Geist der… Kreativität! Rainbow Dash, die das falsche Angebot ihres Herzenswunsches durchschaute, repräsentiert den Geist der… Analyse! Marie-Susan, die uns dazu brachte, sie zu überzeugen, dass wir Recht hatten, bevor sie zustimmte, an unserer Expedition teilzunehmen, repräsentiert den Geist von… Peer Review! Und wenn diese Elemente durch den Funken der Neugierde, der in unser aller Herzen wohnt, entzündet werden, entsteht das siebte Element - das Element der Wissenschaft…“

Der Energiestoß, der aufkam, war wie ein Wind in mondloser Nacht, er erwischte Marie-Susan, bevor das Pony auch nur zucken konnte, und sie war spurlos verschwunden, bevor einer von ihnen die Chance hatte, sich erschrocken zurückzuziehen. Aus dem dunklen Ding, das in der Mitte des Podiums stand, wo die Elemente zerbrochen waren, aus dem kaum erkennbaren, leer-schwarzen Umriss eines Pferdes, kam eine Stimme, die an allen Ohren vorbeizugehen schien und wie kaltes Feuer brannte, das direkt in das Gehirn jedes Ponys klang, das sie hörte:

\emph{Hast du erwartet, dass ich einfach dastehe und dich ausreden lasse?}

Twilight Sparkle starrte auf die Stelle, an der Marie-Susan gestanden hatte, wo keine Spur des Einhorns mehr zu sehen war. \emph{Sie - sie hat nur - sie} - Im Hinterkopf, ungehört, war ihr bewusst, dass Rarity schrie.

\emph{Das war keine Desintegration, sagte die Stimme von Nightmare. Ich habe sie woanders hingeschickt.}

Raritys Schrei hörte abrupt auf. Twilight Sparkle hatte das Gefühl, dass ihr eigener Schrei gerade erst begann. \emph{Sieben. Es brauchte sieben Ponys, um die Elemente der Befragung zu benutzen. Jeder wusste, dass, egal wie ehrlich, forschend, skeptisch, kreativ, analytisch oder neugierig man war, das wirklich Wissenschaftliche an seiner Arbeit war, wenn man seine Ergebnisse in einer angesehenen Zeitschrift veröffentlichte. Jeder wusste das. Konnte es mehr als ein Element der Peer Review auf einmal geben - wie lange würde es dauern, ein anderes zu finden - und der Albtraum würde nicht einfach dastehen und sie machen lassen—}

„Wo?“, schrie Rainbow Dash. „Wo hast du sie hingeschickt?“

\emph{Ich legte das kleine Pony an denselben Ort, an dem ich meine erbärmliche Schwester} \emph{gefesselt hatte, in das Herz eurer erbärmlichen Sonne.}

„Sie wird sterben!“, schrie Fluttershy und starrte den Albtraum entsetzt an. „Es ist zu heiß, sie wird verbrennen!“

\emph{Oh, keine Sorge. Die Macht des Alptraums umgibt deine kleine Freundin, hält sie sicher und kühl und lässt sie ohne Essen und Trinken überleben. Sie wird nichts weiter als Langeweile erleiden…} Der leere schwarze Umriss trat vom Podium und ging langsam und bedächtig an den restlichen sechs Ponys vorbei. \emph{… solange die Macht des Alptraums nicht gebrochen ist. Zum Beispiel durch irgendwelche Backup-Pläne, die meine Schwester in Gang gesetzt hat und die euch bekannt sein könnten. In diesem Fall wird sie auf der Stelle verdampfen. So etwas Schönes, Freundschaft. Sie ist ein wunderbares Instrument der Erpressung. Bewahrt die Elemente der Untersuchung gut auf. Ihr wollt doch nicht, dass jemand anders sie gegen mich benutzt, oder?}

„Nein“, flüsterte Twilight Sparkle, als ihr das Grauen zu dämmern begann. Dann ein krabbelndes Gefühl auf ihrer ganzen Haut, als der Albtraum an ihr vorbeiging und die tödliche Macht sie mit ihrer kalten Liebkosung streifte.

\emph{Wenn ihr mich jetzt entschuldigen würdet, meine kleinen Ponys, ich habe eine ewige Nacht, über die ich herrschen muss.}

\textbf{DAS DORF IN DER KLARHEIT}

„Bedenken Sie die Rechenleistung, die nötig ist, um über hundert Schattenklone zu manifestieren“, sagte das Uchiha-Genie in seinem leidenschaftslosen Ton.

„Es ist ein Irrtum der Vernunft, Sakura, 'Fluke' zu sagen und zu glauben, du hättest irgendetwas erklärt. 'Fluke' ist einfach der Name, den man Daten gibt, die man ignoriert.“

„Aber es muss doch ein Zufall sein!“ schrie Sakura auf. Mühsam beruhigte sie ihre Stimme zu der vorsichtigen Präzision, die man von einem Vernunft-Ninja erwartete; es würde nicht gut gehen, wenn ihr Schwarm sie für dumm hielt.

„Wie du schon sagtest, die Rechenleistung, die nötig ist, um über hundert Kage Bunshin einzusetzen, ist einfach absurd. Wir reden hier von der Stufe einer großen Superintelligenz. Naruto ist der Allerletzte in unserer Klasse. Er ist nicht einmal auf Jounin-Niveau intelligent, geschweige denn eine Superintelligenz!“

Die Augen des Uchiha leuchteten, fast so, als hätte er sein Smartingan aktiviert. „Naruto kann hundert unabhängig voneinander agierende Klone manifestieren. Er muss die rohe Intelligenz besitzen. Aber unter normalen Umständen hindert ihn etwas daran, diese Rechenleistung effizient zu nutzen…wie ein Geist, der mit sich selbst im Krieg liegt, vielleicht? Wir haben jetzt Grund zu der Annahme, dass Naruto in irgendeiner Weise mit einer Superintelligenz verbunden ist, und als kürzlich graduierter Genin ist er, wie wir, fünfzehn Jahre alt. Was ist vor fünfzehn Jahren passiert, Sakura?“

Es dauerte einen Moment, bis Sakura begriff, sich erinnerte, und dann verstand sie. Der Angriff des Neun-Gehirne-Dämonenfuchses. Es war nur ein kleines, knochenweißes Wesen mit großen Ohren, einem noch größeren Schwanz und glänzenden roten Augen. Er war nicht stärker als ein gewöhnlicher Fuchs, er spuckte kein Feuer und blitzte nicht mit Laseraugen, er besaß kein Chakra und keine Magie, aber seine Intelligenz war mehr als neuntausendmal so groß wie die eines Menschen. Hunderte waren getötet worden, die Hälfte der Gebäude zerstört, fast das ganze Dorf Beisugakure war vernichtet worden.

„Du glaubst, der Kyubey versteckt sich in Naruto?“ sagte Sakura. Einen Moment später fuhr ihr Gehirn automatisch damit fort, die offensichtlichen Implikationen dieser Theorie auszufüllen. „Und der Softwarekonflikt zwischen ihren Existenzen ist der Grund, warum er sich die Hälfte der Zeit wie ein schnatternder Idiot verhält, aber hundert Kage Bunshin kontrollieren kann. Huh. Das macht…eine Menge Sinn…eigentlich…“

Sasuke schenkte ihr das kurze, verächtliche Nicken von jemandem, der das alles selbst herausgefunden hatte, ohne dass jemand anderes ihn dazu auffordern musste. „Ano…“, sagte Sakura. Nur jahrelange Vernunftübungen kanalisierten ihre komplette schreiende Panik in pragmatisch sinnvolle Handlungsoptionen. „Sollten wir nicht… jemandem davon erzählen? Irgendwann in den nächsten fünf Sekunden?“

„Die Erwachsenen wissen es schon“, sagte Sasuke emotionslos. „Es ist die offensichtliche Erklärung für ihre Behandlung von Naruto. Nein, die eigentliche Frage ist, wie das in die Überlistung der Uchiha passt…“

„Ich sehe nicht, wie es überhaupt passt—“ , begann Sakura.

„Es muss passen!“ Ein Hauch von verzweifelter Emotion flackerte in Sasukes Stimme auf. „Ich habe diesen Mann gefragt, warum er das getan hat, und er hat mir gesagt, wenn ich die Antwort darauf wüsste, würde das alles erklären! Sicherlich muss dies auch ein Teil dessen sein, was zu erklären ist!“

Sakura seufzte vor sich hin. Ihre persönliche Hypothese war, dass Itachi nur versucht hatte, seinen Bruder in klinische Paranoia zu treiben.

„Yo, Kinder“, sagte die Stimme ihres Rationalitäts-Sensei aus ihren Funkgeräten. „Da ist ein Dorf in Wave, das versucht, eine Brücke zu bauen, und sie stürzt immer wieder ein, ohne dass irgendjemand einen Grund dafür herausfinden kann. Wir treffen uns um 12~Uhr an den Toren. Es ist Zeit für Ihre erste C-Rang-Analyse-Mission.“

(Dies hat nun eine erweiterte Fanfiction inspiriert, \_Lighting Up the Dark\_ von Velorien.)

\textbf{NERDS IN CHAINS}

„Wie konntest du das tun, Anita?“, sagte Richard mit sehr fester Stimme. „Wie konntest du zusammen mit Jean-Claude eine Arbeit verfassen? Du studierst die Untoten, du arbeitest nicht mit ihnen an Papieren zusammen!“

„Und was ist mit dir?“ spuckte Ich. „Du bist Co-Autor einer Arbeit mit Sylvie! Es ist in Ordnung für dich produktiv zu sein, aber nicht für mich ?“

„Ich bin der Leiter dieses Instituts“, knurrte Richard.

Ich konnte die Wellen der Wissenschaft spüren, die von ihm ausgingen; er war wütend. „Ich muss mit Sylvie zusammenarbeiten, das hat nichts zu bedeuten! Ich dachte, unsere eigene Forschung sei etwas Besonderes, Anita!“

„Ist es auch“, sagte ich und fühlte mich hilflos über meine Unfähigkeit, Richard die Dinge zu erklären. Er verstand nicht den Nervenkitzel, ein Universalgelehrter zu sein, die neuen Welten, die sich mir eröffneten. „Ich habe unsere Forschung mit niemandem geteilt—“

„Aber du wolltest es“, sagte Richard.

Ich sagte nichts, aber ich wusste, dass der Ausdruck auf meinem Gesicht alles sagte. „Gott, Anita, du hast dich verändert“, sagte Richard. Er schien in sich zusammenzusacken. „Ist dir klar, dass die Monster jetzt Witze über Blake-Nummern machen? Früher war ich dein Partner in allem, und jetzt - bin ich nur ein weiterer Werwolf mit einer Blake-Nummer von 1.“

\textbf{THUNDERSMARTS}

„Ich habe es satt!“, rief Liono. „Ich habe es satt, das jede einzelne Woche zu machen! Unsere Spezies war zu interstellaren Reisen fähig, Panthro, ich kenne die Energiemengen, die damit verbunden sind! Es muss eine Möglichkeit geben, eine Atombombe zu bauen oder einen Asteroiden zu steuern oder irgendwie die Pyramide dieses ewigen Idioten zu sprengen!“

\textbf{HE-MAN UND DIE MEISTER DER RATIONALITÄT}

„Fabelhaftes Geheimwissen wurde mir an dem Tag offenbart, als ich mein magisches Buch in die Höhe hielt und sagte: Bei der Macht des Bayes'schen Theorems!“

\textbf{FATE/SANE NIGHT}

Ich bin der Kern meiner Gedanken Der Glaube ist mein Körper Und die Wahl ist mein Blut Ich habe über tausend Urteile revidiert Keine Angst vor Verlust noch vor Gewinn Ich habe dem Schmerz widerstanden, um viele Male zu aktualisieren Ich warte auf die Ankunft der Wahrheit. Dies ist der einzige unsichere Weg. Mein ganzes Leben war… Unbegrenzte Bayes-Werke!

\textbf{DER NAME DER RATIONALITÄT}

Der elfjährige Junge, der eines Tages zur Legende werden sollte - Drachentöter, Königsmörder - hatte nur einen Gedanken im Kopf, als er sich dem Sprechenden Hut näherte, um in das Studium der Mysterien einzutreten. Irgendwo, nur nicht in Ravenclaw, irgendwo, nur nicht in Ravenclaw, oh bitte, irgendwo, nur nicht in Ravenclaw…

Doch kaum war die Krempe des uralten Filzgeräts über seine Stirn gerutscht - „RAVENCLAW!“

Als der blau geschmückte Tisch anfing, ihm zu applaudieren, als er sich dem gefürchteten Tisch näherte, an dem er die nächsten sieben Jahre verbringen würde, zuckte Kvothe innerlich bereits zusammen, weil er auf das Unvermeidliche wartete; und das Unvermeidliche trat fast sofort ein, genau wie er es befürchtet hatte, noch bevor er überhaupt eine Chance gehabt hatte, sich richtig zu setzen.

„So!“, sagte ein älterer Junge mit dem glücklichen Gesichtsausdruck von jemandem, dem etwas furchtbar Schlaues eingefallen ist. „Kvothe der Rabe, hm?“

\textbf{TENGEN TOPPA GURREN RATIONALITÄT 40K}

Ich habe eine wirklich wunderbare Geschichte für dieses Crossover, für die dieser Rahmen zu schmal ist.

\textbf{UTILITARIAN TWILIGHT}

(Anmerkung: Geschrieben, nachdem ich gehört hatte, dass Alicorn eine Twilight-Fanfic schreibt, aber bevor ich \_Luminosity\_ gelesen habe. Es ist offensichtlich, wenn man einer von uns ist.)

„Edward“, sagte Isabella zärtlich. Sie streckte eine Hand aus und streichelte seine kalte, glänzende Wange. „Du musst mich vor nichts beschützen. Ich habe alle Vor- und Nachteile aufgelistet, sie konsequent relativ gewichtet, und es ist einfach ganz offensichtlich, dass die Vorteile, ein Vampir zu werden, die Nachteile überwiegen.“

„Bella“, sagte Edward und schluckte verzweifelt. „Bella—“

„Unsterblichkeit. Perfekte Gesundheit. Erwachende übersinnliche Kräfte. Einfach genug, um mit Tierblut zu überleben, wenn man es einmal getan hat. Selbst die Schönheit, Edward, es gibt Menschen, die ihr Leben dafür geben würden, hübsch zu sein, und wage es nicht, sie oberflächlich zu nennen, bevor du nicht versucht hast, hässlich zu sein. Denkst du, ich habe Angst vor dem Wort “Vampir„? Ich habe genug von deinen willkürlichen deontologischen Zwängen, Edward. Die ganze menschliche Spezies sollte an deinem Spaß teilhaben, und die Menschen sterben zu Tausenden, während du zögerst.“

Die Waffe in ihrer Hand war kalt an der Stirn Ihres Geliebten. Sie würde ihn nicht töten, aber lange genug außer Gefecht setzen—

\textbf{JASMINE UND DIE LAMPE}

Aladdins Gesicht war wehmütig, aber entschlossen, als der frischgebackene Straßenjunge sich ein letztes Mal an das blaue Wesen von kosmischer Macht wandte, bereit, den Reichtum und die Hoffnung, die er so kurz geschmeckt hatte, um seines Freundes willen zurückzulassen.

„Dschinni, ich äußere meinen dritten Wunsch. Ich wünsche mir, dass du—“

Prinzessin Jasmin, die mit offenem Mund vor sich hin starrte und nicht so recht glauben konnte, was sie da sah, schaffte es gerade noch, ihre Lähmung zu überwinden und dem Jungen die Lampe aus der Hand zu reißen, bevor er den fatalen Satz beenden konnte.

„Entschuldige mal“, sagte Jasmin. „Aladin, mein Schatz, du bist süß, aber du bist ein Idiot, weißt du das? Hast du nicht bemerkt, wie Jafar, sobald er diese Lampe in die Hände bekam, seine eigenen drei Wünsche bekam - ach, egal. Dschinni, ich wünsche mir, dass alle Menschen immer jung und gesund sind, ich wünsche mir, dass niemand jemals sterben muss, wenn er es nicht will, und ich wünsche mir, dass die Intelligenz aller Menschen allmählich um einen IQ-Punkt pro Jahr zunimmt.“

Sie warf die Lampe zurück zu Aladdin. „Geh zurück zu dem, was du getan hast.“

\textbf{RATIONALIST HAMLET}

(beigesteuert von Histocrat auf LiveJournal, Beitrag 13389, alias HonoreDB auf LessWrong) (mit Erlaubnis wiederveröffentlicht)

HAMLET Eindringling, gib diesen seltsamen Streich auf, der die Blindheit meines Kummers und das gute Herz meines guten Freundes Horatio grausam ausnutzt. Oder sonst, wenn du ein Recht auf diese geliebte Gestalt hast, sag mir: Welche Zeichnung gab ich dem König Hamlet, als ich 6 Jahre alt und kaum aus der Schlinge war?

GEIST: Ein Einhorn war, ganz in Panzer gekleidet.

HAMLET Was?

GEIST, merk dir das.

HAMLET Vater, ich will.

GEIST Meine Stunde ist fast gekommen, Wenn ich mich den schwefligen und quälenden Flammen hingeben muss.

HAMLET Du bist in Qualen?

GEIST Ja, wie alle, die ohne Beichte sterben.

HAMLET Wie jeder Däne ist dies, was man mich gelehrt hat. Doch hielt ich solche Willkür für unpassend dem allmächt'gen Gott. Denn alle, die unversehens den Tod erleiden, von Gottes auserwählten Priestern unbeaufsichtigt, um dann für die schlechte Ordnung der Welt bestraft zu werden…

GEIST, es war nicht die Welt, die mich tötete, noch irgendein Zufall.

HAMLET Was?

GEIST Wenn du je deinen lieben Vater liebtest, räch seinen schnöden, unnatürlichen Mord.

HAMLET O Gott!

GEIST Meine Zeit wird immer kürzer. Willst du die Erzählung hören?

HAMLET Nein.

GEIST Was?

HAMLET Meine Liebe zu dir ruft mich, deinen Tod zu rächen, doch größere Verbrechen hörte ich diese Nacht erzählen. Wenn alle Ermordeten zur Hölle fahren und auch andere, die gestanden hätten, wenn sie Zeit gehabt hätten, wenn Menschen, die im Großen und Ganzen gut sind, durch Gottes Hand grausam leiden, dann trotze ich Gottes Plan. Guter Geist, als einer, der hinter dem Schleier wohnt, weißt du Dinge, die wir Sterblichen kaum begreifen. Sag mir: Gibt es eine List, die der Natur fremd ist, durch die man dem Tod entrinnen kann?

GEIST Du willst der Hölle entfliehn?

HAMLET Ich will jedem die Hölle verwehren! und auch den Himmel, denn ich ahne, dass der Himmel unseres verrückten Gottes ein armseliges Ding sein wird, neben dem Himmel, den ich aus der Erde machen werde, wenn ich ihr unsterblicher König bin.

GEISTIch kümmere mich nicht um diese Dinge. Tod und Hölle haben mir alles Begehren genommen, außer der Rache an meinem Mörder.

HAMLET Du sollst nicht gerächt werden, es sei denn, du schwörst: Töte ich deinen Mörder, so bürgst du mir für die Mittel, mit denen ich den Tod töten kann. Der, der dich tötete, wird sich zu dir in die Grube gesellen, und das war's dann. Ich werde keine weitere Vermehrung der Hölle zulassen.

GEISTAbgemacht. Wenn mein Bruder erschlagen ist, der mir das Gift ins Ohr goss, dann werde ich dir die kostbare Wahrheit in deines gießen: die Herstellung des Steins der Weisen. Mit diesem Stein kannst du einen Philter beschaffen, der jeden Menschen gegen den Tod unempfindlich macht, und noch mehr unedles Metall in Gold umwandeln, um die Versorgung der ganzen Menschheit mit diesem Philter zu finanzieren.

HAMLET Wahrlich, es gibt nichts, was über die Träume der Philosophie hinausgeht. Wartet. Der Mann, den ich töten muss - mein Oheim, der König? In der Tat, er hat solche Gaben, dass ich fast verzweifle, ihn zu töten und doch den Thron zu besteigen. Es wird ein furchtbarer Kampf um furchtbare Einsätze sein. Hast du einen Rat?

\emph{Ein Hahn kräht. Abgang Geist.}

(HonoreDB hat dies nun zu einem kompletten ebook erweitert mit dem Titel

A Will Most Incorrect to Heaven: The Tragedy of Prince Hamlet and the Philosopher's Stone erhältlich für \$3 bei makefoil dot com. Ja, wirklich)

\textbf{MOBY DICK UND DIE METHODEN DER RATIONALITÄT}

(wie von Eneasz auf LessWrong erzählt)

„Rache?“, sagte der holzbeinige Mann. "An einem Wal? Nein, ich beschloss, einfach mit meinem Leben weiterzumachen.

\textbf{ALICE IM LAND WO ALLES NOCH VERRÜCKTER IST ALS HIER}

(wie zuerst von braindoll in einer Rezension dieses Kapitels geschrieben, mit einigen weiteren Änderungen)

Alice saß neben ihrer Schwester auf der Bank und las ein Buch. Sie hatte einige Freunde, die älter waren, und wenn sie nur nett fragte, waren sie oft bereit, ihr Bücher zu leihen, die nicht so viele Bilder und Gespräche enthielten, wie man es für ein Mädchen ihres Alters für angemessen hielt. An heißen Tagen fühlte sie sich oft schläfrig und dumm, deshalb hatte Alice in Gedanken ein Taschentuch nass gemacht und es sich in den Nacken gelegt. Trotzdem waren ihre Gedanken auf Wanderschaft gegangen (als wäre sie ein kleines Kätzchen, dessen Besitzerin ihr für einen Moment die Augen abgenommen hatte), und sie hatte gerade beschlossen, dass das Vergnügen, eine Gänseblümchenkette zu machen, etwa 4/3 der Mühe wert sein würde, aufzustehen und die Gänseblümchen zu pflücken, was jedoch nicht den Opportunitätskosten entsprach, ihr Buch wegzulegen, als plötzlich ein weißes Kaninchen mit rosa Augen dicht an ihr vorbei lief.

Daran war nichts besonders Bemerkenswertes; und tatsächlich fand Alice es auch nicht besonders abwegig, das Kaninchen zu sich sagen zu hören: „Oh je! Oh je! Ich werde zu spät kommen!“

Aber als das Kaninchen tatsächlich eine Uhr aus seiner Westentasche nahm, sie ansah und dann weiterlief, erstarrte Alice in plötzlicher Klarheit und Angst, denn sie hatte noch nie ein Kaninchen mit einer Westentasche oder einer Uhr gesehen, die es aus der Tasche nahm.

„Oh Schreck“, sagte sie zu sich selbst (wenn auch nicht laut; diese Angewohnheit hatte sie sich längst abgewöhnt, da die Leute sie dadurch noch weniger ernst nahmen, als sie es ohnehin schon taten). „Wenn ich nicht sofort erkannt habe, wie viel merkwürdiger das war als das durchschnittliche Kaninchen, dann stört etwas meine Neugier, und das ist das Merkwürdigste von allem.“

Also rannte sie, brennend vor Fragen, über das Feld hinterher und kam gerade noch rechtzeitig, um zu sehen, wie es in einem großen Kaninchenbau unter der Hecke verschwand.

\textbf{WILLKOMMEN IN DER REALEN WELT}

MORPHEUS: Die längste Zeit wollte ich es nicht glauben. Aber dann sah ich die Felder mit meinen eigenen Augen, sah, wie sie die Toten verflüssigten, um sie intravenös an die Lebenden zu verfüttern—

NEO (höflich): Verzeihung, bitte.

Ja, Neo?

Ich habe so lange geschwiegen, wie ich konnte, aber ich fühle ein gewisses Bedürfnis, an diesem Punkt etwas zu sagen. Der menschliche Körper ist die ineffizienteste Energiequelle, die man sich vorstellen kann. Der Wirkungsgrad eines Kraftwerks bei der Umwandlung von Wärmeenergie in Elektrizität nimmt ab, wenn man die Turbinen bei niedrigeren Temperaturen laufen lässt. Wenn Sie irgendeine Art von Nahrung hätten, die Menschen essen könnten, wäre es effizienter, sie in einem Ofen zu verbrennen, als sie an Menschen zu verfüttern. Und jetzt willst du mir erzählen, dass ihre Nahrung die Körper der Toten sind, die an die Lebenden verfüttert werden? Hast du noch nie etwas von den Gesetzen der Thermodynamik gehört?

Wo hast du von den Gesetzen der Thermodynamik gehört, Neo?

NEO: Jeder, der auch nur eine Schulstunde in Naturwissenschaften überstanden hat, sollte die Gesetze der Thermodynamik kennen!

MORPHEUS: Wo bist du zur High School gegangen, Neo?

(Pause.)

NEO:…in der Matrix.

MORPHEUS: Die Maschinen erzählen elegante Lügen.

(Pause.)

NEO (mit leiser Stimme): Könnte ich bitte ein richtiges Physiklehrbuch haben?

MORPHEUS: So etwas gibt es nicht, Neo. Das Universum läuft nicht mit Mathe.

