

\hypertarget{die-wissenschaftliche-methode}{% \section{22. Die Wissenschaftliche Methode}\label{die-wissenschaftliche-methode}}

\textbf{\uline{Die wissenschaftliche Methode}}

\emph{Einige Anmerkungen des original Autors: Die Meinungen der Charaktere in dieser Geschichte sind nicht unbedingt die des Autors. Was die gute Seite von Harry denkt, ist oft als ein gutes Muster gedacht, dem man folgen kann, besonders wenn Harry darüber nachdenkt, wie er wissenschaftliche Studien zitieren kann, um ein bestimmtes Prinzip zu untermauern. Aber nicht alles, was Harry tut oder denkt, ist eine gute Idee. Das würde als Geschichte nicht funktionieren. Und die weniger warmherzigen Charaktere mögen manchmal wertvolle Lektionen zu bieten haben, aber diese Lektionen können auch gefährlich zweischneidig sein.}

Ein kleiner Studienraum, in der Nähe, aber nicht im Ravenclaw-Schlafsaal, einer der vielen, vielen ungenutzten Räume von Hogwarts. Grauer Stein die Böden, roter Backstein die Wände, dunkel gebeiztes Holz die Decke, vier leuchtende Glaskugeln in die vier Wände des Raumes eingelassen. Ein runder Tisch, der wie eine breite Platte aus schwarzem Marmor aussah, die auf dicken schwarzen Marmorbeinen als Säulen stand, der sich aber als sehr leicht erwiesen hatte

(sowohl in Bezug auf das Gewicht als auch auf die Masse)

und nicht schwer zu heben und zu verschieben war, falls nötig. Zwei bequem gepolsterte Stühle, die zunächst an unbequemen Stellen am Boden befestigt zu sein schienen, die sich aber, wie die beiden schließlich herausfanden, dorthin bewegten, wo man stand, sobald man sich in einer Haltung vorbeugte, die aussah, als wolle man sich setzen. Außerdem schienen eine Reihe von Fledermäusen durch den Raum zu fliegen. Das war der Ort, an dem, wie zukünftige Historiker eines Tages festhalten würden -

falls das ganze Projekt überhaupt jemals zu etwas führen sollte -,

die wissenschaftliche Erforschung der Magie begonnen hatte, mit zwei jungen Hogwarts-Schülern im ersten Jahr.

\emph{Harry James Potter-Evans-Verres, Theoretiker}.

Und \emph{Hermine Jean Granger, Experimentatorin und Versuchsperson}.

Harry war jetzt besser im Unterricht, zumindest in den Klassen, die er für interessant hielt. Er las mehr Bücher, und auch keine Bücher für Elfjährige. Er übte jeden Tag in einer seiner Freistunden immer wieder Verwandlung und nutzte die andere Stunde, um mit Okklumentik zu beginnen. Er nahm den lohnenswerten Unterricht ernst, gab nicht nur jeden Tag seine Hausaufgaben ab, sondern nutzte seine Freizeit, um mehr zu lernen, als verlangt wurde, um über die vorgegebenen Lehrbücher hinaus weitere Bücher zu lesen, um das Fach zu beherrschen und nicht nur ein paar Testantworten auswendig zu lernen, um zu brillieren. So etwas sah man außerhalb von Ravenclaw nicht oft. Und selbst innerhalb von Ravenclaw waren seine einzigen verbliebenen Konkurrenten Padma Patil (deren Eltern aus einer nicht englischsprachigen Kultur stammten und sie daher mit einer echten Arbeitsmoral großgezogen hatten), Anthony Goldstein (aus einer bestimmten winzigen ethnischen Gruppe, die 25 \% der Nobelpreise gewann) und natürlich Hermine Granger, die weit über allen stand wie ein Titan, der durch ein Rudel Welpen schreitet.

Um dieses spezielle Experiment durchzuführen, musste die Testperson sechzehn neue Zaubersprüche lernen, und zwar allein, ohne Hilfe oder Korrektur. Das bedeutete, die Testperson war Hermine. Punkt.

Es sollte an dieser Stelle erwähnt werden, dass die Fledermäuse, die im Raum herumflogen, nicht leuchteten.

Harry fiel es schwer, die Konsequenzen daraus zu akzeptieren.

"Oogely boogely!" Sagte Hermine wieder. Wieder, an der Spitze von Hermines Zauberstab, gab es das plötzliche, übergangslose Erscheinen einer Fledermaus. Im einen Moment leere Luft.

Im nächsten Moment: Fledermaus. Ihre Flügel schienen sich in dem Moment, in dem sie erschien, bereits zu bewegen. Und sie leuchtete immer noch nicht.

"Kann ich jetzt aufhören?", fragte Hermine.

"Bist du sicher", sagte Harry durch etwas, das wie eine Blockade in seiner Kehle aussah, "dass du es mit ein bisschen mehr Übung nicht zum Leuchten bringen könntest?"

Er verstieß gegen die Versuchsanleitung, die er vorher aufgeschrieben hatte, was eine Sünde war, und er verstieß dagegen, weil ihm die Ergebnisse nicht gefielen, die er erhielt, was eine Todsünde war, \emph{für die man in die Wissenschaftshölle kommen konnte}, aber das schien sowieso keine Rolle zu spielen.

"Was hast du dieses Mal verändert?" sagte Hermine und klang ein wenig müde.

"Die Dauer der oo, eh, und ee Laute. Es sollte 3 zu 2 zu 2 sein, nicht 3 zu 1 zu 1."

"Oogely boogely!", sagte Hermine. Die Fledermaus materialisierte sich mit nur einem Flügel, drehte sich jämmerlich zu Boden und hüpfte im Kreis auf dem grauen Stein herum. "Wie geht es nun wirklich?", fragte Hermine.

"3 zu 2 zu 1."

"Oogely boogely!" Diesmal hatte die Fledermaus gar keine Flügel und fiel mit einem Plopp wie eine tote Maus um.

"3 zu 1 zu 2."

Und siehe da, die Fledermaus materialisierte sich und flog sofort zur Decke, gesund und in einem hellen Grün leuchtend. Hermine nickte zufrieden. "Okay, was jetzt?"

Es gab eine lange Pause. "Ernsthaft? Du musst ernsthaft Oogely boogely sagen, wobei die Dauer der oo, eh und ee-Laute ein Verhältnis von 3 zu 1 zu 2 haben muss, sonst leuchtet die Fledermaus nicht? Warum eigentlich? Warum? Bei der Liebe zu allem, was heilig ist, warum?!"

"Warum nicht?"

"\textbf{AAAAAAAAAAARRRRRRGHHHH}!" \emph{Pamm. Pamm. Pamm.}

Harry hatte eine Weile über die Natur der Magie nachgedacht und dann eine Reihe von Experimenten entworfen, die auf der Prämisse beruhten, dass praktisch alles, was Zauberer über Magie glaubten, falsch war. Man konnte nicht wirklich \emph{"Wingardium Leviosa"} auf genau die richtige Weise sagen müssen, um etwas schweben zu lassen, denn, komm schon, \emph{"Wingardium Leviosa"}? Das Universum würde überprüfen, ob man \emph{"Wingardium Leviosa"} genau richtig gesagt hat, und sonst würde es die Feder nicht schweben lassen? Nein. Offensichtlich nicht, wenn man ernsthaft darüber nachdenkt. Irgendjemand, möglicherweise ein echtes Vorschulkind, auf jeden Fall aber ein englischsprachiger Magieanwender, der fand, dass \emph{"Wingardium Leviosa"} fliegend und schwebend klang, hatte diese Worte ursprünglich gesprochen, als er den Zauber zum ersten Mal sprach. Und dann allen anderen erzählt, es sei notwendig.

Aber (so hatte Harry argumentiert) es musste nicht so sein, es war nicht in das Universum eingebaut, es war in dich eingebaut. Es gab eine alte Geschichte, die unter den Wissenschaftlern weitergegeben wurde, ein abschreckendes Märchen, die Geschichte von Blondlot und den N-Strahlen. Kurz nach der Entdeckung der Röntgenstrahlen hatte ein bedeutender französischer Physiker namens Prosper-Rene Blondlot - der als erster die Geschwindigkeit von Radiowellen gemessen und gezeigt hatte, dass sie sich mit Lichtgeschwindigkeit ausbreiten - die Entdeckung eines erstaunlichen neuen Phänomens bekannt gegeben, der N-Strahlen, die eine schwache Aufhellung des Bildschirms bewirken würden. Man musste schon genau hinschauen, um es zu sehen, aber es war da. N-Rays hatten alle möglichen interessanten Eigenschaften. Sie wurden durch Aluminium gebeugt und konnten durch ein Aluminiumprisma so fokussiert werden, dass sie auf einen behandelten Faden aus Cadmiumsulfid trafen, der dann im Dunkeln schwach leuchtete… Bald hatten Dutzende von anderen Wissenschaftlern Blondlots Ergebnisse bestätigt, vor allem in Frankreich. Aber es gab noch andere Wissenschaftler, in England und Deutschland, die sagten, sie seien sich nicht ganz sicher, ob sie dieses schwache Leuchten sehen könnten. Blondlot hatte gesagt, dass sie wahrscheinlich die Maschinen falsch eingestellt hatten. Eines Tages hatte Blondlot eine Demonstration von N-Strahlen gegeben. Die Lichter waren aus, und sein Assistent hatte das Aufhellen und Abdunkeln auf Zuruf getan, während Blondlot seine Manipulationen durchführte. Es war eine normale Demonstration gewesen, alle Ergebnisse verliefen wie erwartet. Auch wenn ein amerikanischer Wissenschaftler namens Robert Wood in aller Stille das Aluminiumprisma aus dem Zentrum von Blondlots Mechanismus gestohlen hatte. Und das war das Ende der N-Strahlen gewesen.

Die Realität, hatte Philip K. Dick einmal gesagt, ist das, was, wenn man aufhört, daran zu glauben, nicht verschwindet. Blondlots Sünde war im Nachhinein offensichtlich gewesen. Er hätte seinem Assistenten nicht sagen sollen, was er tat. Blondlot hätte sich vergewissern sollen, dass der Assistent nicht wusste, was versucht wurde und wann es versucht wurde, bevor er ihn bat, die Helligkeit des Bildschirms zu beschreiben. So einfach hätte es sein können. Heutzutage nannte man das "\emph{blenden}" und es war eines der Dinge, die moderne Wissenschaftler als selbstverständlich ansahen. Wenn man ein psychologisches Experiment durchführte, um zu sehen, ob Menschen wütender wurden, wenn man ihnen rote Knüppel auf den Kopf schlug, als wenn man ihnen grüne Knüppel auf den Kopf schlug, konnte man sich die Versuchspersonen nicht selbst ansehen und entscheiden, wie "\emph{wütend}" sie waren. Man machte Fotos von ihnen, nachdem sie mit dem Knüppel geschlagen worden waren, und schickte die Fotos an ein Gremium von Bewertern, die auf einer Skala von 1 bis 10 bewerteten, wie wütend jede Person aussah, natürlich ohne zu wissen, welche Farbe der Knüppel hatte, mit dem sie geschlagen worden war. In der Tat gab es keinen guten Grund, den Bewertern überhaupt zu sagen, worum es bei dem Experiment ging. Sie würden den Versuchspersonen sicher nicht sagen, dass sie wütender sein sollten, wenn sie von roten Knüppeln getroffen werden. Man würde ihnen einfach 20 Pfund anbieten, sie in einen Versuchsraum locken, sie mit einem Knüppel schlagen, natürlich in einer zufällig zugewiesenen Farbe, und das Foto machen. Das Schlagen mit dem Knüppel und das Fotografieren wurden von einem Assistenten durchgeführt, der nicht über die Hypothese informiert war, so dass er nicht erwartungsvoll schauen, härter schlagen oder das Foto im richtigen Moment schießen konnte. Blondlot hatte seinen Ruf mit der Art von Fehlern zerstört, die ihm eine durchfallende Note und wahrscheinlich höhnisches Gelächter von der Wissenschaft einbringen würde und zwar in einem Erstsemesterkurs über Versuchsplanung… im Jahr 1991.

Aber 1904 war schon etwas länger her und so hatte es Monate gedauert, bis Robert Wood die offensichtliche Alternativhypothese formuliert und herausgefunden hatte, wie man sie testen konnte, und Dutzende anderer Wissenschaftler waren mitgerissen worden. Mehr als zwei Jahrhunderte nachdem die Wissenschaft begonnen hatte. So spät in der Wissenschaftsgeschichte war es immer noch nicht offensichtlich. Was es völlig plausibel machte, dass in der winzigen Zaubererwelt, in der die Wissenschaft nicht viel bekannt zu sein schien, niemand jemals das Erste, das Einfachste, das Offensichtlichste ausprobiert hatte, das jeder moderne Wissenschaftler zu überprüfen gedachte. Die Bücher waren voll von komplizierten Anleitungen für all die Dinge, die man genau richtig machen musste, um einen Zauber zu wirken. Und, so hatte Harry vermutet, der Prozess des Befolgens dieser Anweisungen, das Überprüfen, ob man sie richtig befolgte, bewirkte wahrscheinlich etwas. Es zwang einen dazu, sich auf den Zauberspruch zu konzentrieren. Wenn einem gesagt wird, man solle einfach mit dem Zauberstab wedeln und sich etwas wünschen, würde das wahrscheinlich nicht so gut funktionieren. Und wenn man erst einmal geglaubt hat, dass der Zauber auf eine bestimmte Art und Weise funktioniert, wenn man ihn auf diese Art und Weise geübt hat, kann man sich vielleicht nicht mehr davon überzeugen, dass er auf eine andere Art und Weise funktionieren könnte …

… wenn man das Einfache, aber Falsche tut und versucht, alternative Formen selbst zu testen. Aber was wäre, wenn du nicht wüsstest, wie der ursprüngliche Spruch gewirkt hat? Was, wenn man Hermine eine Liste von Sprüchen gab, die sie noch nicht gelernt hatte, aus einem Buch mit albernen Streichen aus der Hogwarts-Bibliothek, und einige dieser Sprüche hatten die korrekten und ursprünglichen Anweisungen, während andere eine veränderte Geste, ein verändertes Wort hatten? Was wäre, wenn man die Anweisungen beibehielte, ihr aber sagte, dass ein Zauberspruch, der einen roten Wurm erzeugen sollte, stattdessen einen blauen Wurm erzeugen sollte? Nun, in diesem Fall hatte sich herausgestellt…

… dass Harry Schwierigkeiten hatte, seine Ergebnisse hier zu glauben…

wenn man Hermine sagte, sie solle "Oogely boogely" sagen, mit einer Vokaldauer von 3 zu 1 zu 1, statt des korrekten Verhältnisses von 3 zu 1 zu 2, bekam man immer noch die Fledermaus, aber sie leuchtete nicht mehr. Nicht, dass der Glaube hier irrelevant wäre. Nicht, dass es nur auf die Worte und die Bewegungen des Zauberstabs ankam. Wenn man Hermine völlig falsche Informationen darüber gab, was ein Zauberspruch bewirken sollte, würde er nicht mehr funktionieren. Wenn man ihr gar nicht sagte, was der Zauber bewirken sollte, würde er nicht mehr funktionieren. Wenn sie nur sehr vage wusste, was der Zauber tun sollte, oder wenn sie sich nur teilweise irrte, dann würde der Zauber so funktionieren, wie er ursprünglich im Buch beschrieben war, und nicht so, wie es ihr gesagt worden war.

Harry schlug in diesem Moment buchstäblich seinen Kopf gegen die Wand. Aber nicht hart. Er wollte sein kostbares Hirn nicht beschädigen. Aber wenn er nicht ein Ventil für seine Frustration hätte, würde er sich spontan entzünden.

\emph{Pamm. Pamm. Pamm.}

Es schien, als wolle das Universum tatsächlich, dass man "\emph{Wingardium Leviosa"} sagt, und zwar auf eine ganz bestimmte Art und Weise, und es war ihm genauso egal, wie man die Aussprache fand, wie man die Schwerkraft empfand.

\textbf{WARUM}?

Das Schlimmste daran war der selbstgefällige, amüsierte Blick auf Hermines Gesicht. Hermine war nicht damit einverstanden gewesen, gehorsam herumzusitzen und Harrys Anweisungen zu befolgen, ohne dass man ihr sagte, warum. Also hatte Harry ihr erklärt, was sie testeten. Harry hatte erklärt, warum sie es testeten. Harry hatte erklärt, warum es wahrscheinlich noch kein Zauberer vor ihnen ausprobiert hatte. Harry hatte erklärt, dass er eigentlich ziemlich zuversichtlich war, was seine Vorhersage anging. Denn, so hatte Harry gesagt, das Universum wollte auf keinen Fall, dass man \emph{"Wingardium Leviosa"} sagt.

Hermine hatte darauf hingewiesen, dass dies nicht das war, was in ihren Büchern stand. Hermine hatte gefragt, ob Harry wirklich glaubte, dass er mit elf Jahren und etwas mehr als einem Monat in seiner Hogwarts-Ausbildung schlauer war als alle anderen Zauberer auf der Welt, die anderer Meinung waren als er. Harry hatte genau die folgenden Worte gesagt:

"Natürlich."

Jetzt starrte Harry auf den roten Ziegelstein direkt vor ihm und überlegte, wie fest er sich den Kopf stoßen müsste, um sich eine Gehirnerschütterung zuzuziehen, die die Bildung des Langzeitgedächtnisses beeinträchtigen und ihn daran hindern würde, sich später daran zu erinnern.

Hermine lachte nicht, aber er konnte ihre Absicht zu lachen spüren, die von hinten auf ihn ausstrahlte wie ein furchtbarer Druck auf seiner Haut, ungefähr so, als wüsste man, dass man von einem Serienmörder verfolgt wurde, nur schlimmer.

"Sag es", sagte Harry.

"Das hatte ich nicht vor", sagte die freundliche Stimme von Hermine Granger. "Es schien nicht nett zu sein."

"Bring es einfach hinter dich", sagte Harry.

"Okay! Du hast mir also diesen ganzen langen Vortrag darüber gehalten, wie schwer es sei, Grundlagenforschung zu betreiben und dass wir vielleicht fünfunddreißig Jahre an dem Problem dranbleiben müssten, und dann bist du hingegangen und hast erwartet, dass wir in der ersten Stunde, in der wir zusammenarbeiten, die größte Entdeckung in der Geschichte der Magie machen. Du hast es nicht nur gehofft, du hast es wirklich erwartet. Du bist albern."

"Ich danke dir. Jetzt -"

"Ich habe alle Bücher gelesen, die du mir gegeben hast, und ich weiß immer noch nicht, wie ich das nennen soll. Selbstüberschätzung? Planungsirrtum? Super duper Lake Wobegon Effekt? Wir werden es nach dir benennen müssen. \emph{Harry Bias.}"

"Alles klar!"

"Aber es ist süß. Das ist so ein Jungen-Ding, das zu tun."

"Fall tot um."

"Ach, du sagst die romantischsten Sachen."

\emph{Pamm. Pamm Pamm.}

"Und was kommt jetzt?", fragte Hermine.

Harry lehnte seinen Kopf gegen die Ziegelsteine. Seine Stirn fing an zu schmerzen, wo er sie aufgeschlagen hatte.

"Nichts. Ich muss zurückgehen und andere Experimente entwerfen."

Während des letzten Monats hatte Harry im Voraus sorgfältig eine Reihe von Experimenten für sie ausgearbeitet, die bis Dezember gedauert hätten. Es wäre eine großartige Versuchsreihe gewesen, wenn nicht schon der erste Versuch die Grundannahme verfälscht hätte. Harry konnte nicht glauben, dass er so dumm gewesen war.

"Lass mich mich korrigieren", sagte Harry. "Ich muss \textbf{ein} neues Experiment entwerfen. Ich sage dir Bescheid, wenn wir es haben, und wir werden es durchführen, und dann entwerfe ich das nächste. Wie hört sich das an?"

"Das hört sich an, als hätte jemand eine Menge Mühe verschwendet."

\emph{Pamm. Aua.}

Das hatte er etwas härter gemacht, als er geplant hatte.

"Also", sagte Hermine. Sie lehnte sich in ihrem Stuhl zurück und der selbstgefällige Blick war wieder auf ihrem Gesicht. "Was haben wir heute entdeckt?"

"Ich habe entdeckt", sagte Harry mit zusammengebissenen Zähnen, "dass meine Bücher über wissenschaftliche Methodik einen Dreck wert sind, wenn es darum geht, ein wirklich grundlegendes Problem zu erforschen, bei dem man keine Ahnung hat, was vor sich geht -"

"Achten Sie auf Ihre Worte, Mr~Potter! Einige von uns sind unschuldige junge Mädchen!"

"Schön. Aber wenn meine Bücher einen Karpfen wert wären, das ist eine Fischart und nichts Schlechtes, dann hätten sie mir den folgenden wichtigen Rat gegeben: Wenn es ein verwirrendes Problem gibt und du gerade erst anfängst und eine falsifizierbare Hypothese hast, geh und teste sie. Finden Sie eine einfache, leichte Möglichkeit, eine grundlegende Überprüfung durchzuführen und tun Sie es sofort. Machen Sie sich keine Gedanken darüber, eine aufwendige Versuchsreihe zu entwerfen, die einen Förderantrag bei einer Förderorganisation beeindruckend aussehen lassen würde. Überprüfen Sie einfach so schnell wie möglich, ob Ihre Ideen falsch sind, bevor Sie anfangen, große Mengen an Aufwand in sie zu investieren. Wie klingt das für eine Moral?"

"Mmm … okay", sagte Hermine. "Aber ich hatte auch auf etwas gehofft wie '\emph{Hermines Bücher sind nicht wertlos. Sie sind von weisen alten Zauberern geschrieben, die viel mehr über Magie wissen als ich. Ich sollte darauf achten, was in Hermines Büchern steht.'} Können wir diese Moral auch haben?"

Harrys Kiefer schien zu fest zusammengebissen zu sein, um irgendwelche Worte herauszulassen, also nickte er nur.

"Großartig!" sagte Hermine. "Ich mochte dieses Experiment. Wir haben eine Menge daraus gelernt und es hat nur eine Stunde oder so gedauert."

\textbf{"AAAAAAAAAAAAAAAAHHHHHHHHHHHHHH!"}

In den Kerkern von Slytherin. Ein unbenutztes Klassenzimmer wurde von einem unheimlichen grünen Licht erhellt, das diesmal viel heller war und von einer kleinen Kristallkugel mit einer vorübergehenden Verzauberung kam, aber dennoch ein unheimliches grünes Licht warf, das seltsame Schatten auf die staubigen Tische warf. Zwei knabengroße Gestalten in verhüllten grauen Mänteln \emph{(keine Masken}) waren schweigend eingetreten und hatten sich auf zwei Stühlen gegenüber demselben Schreibtisch niedergelassen.

Es war das zweite Treffen der Bayes'schen Verschwörung. Draco Malfoy war sich nicht sicher gewesen, ob er sich darauf freuen sollte oder nicht. Harry Potter schien, seinem Gesichtsausdruck nach zu urteilen, keine Zweifel an der richtigen Stimmung zu haben. \emph{Harry Potter sah aus, als wäre er bereit, jemanden zu töten.}

"Hermine Granger", sagte Harry Potter, gerade als Draco den Mund öffnen wollte. "Frag nicht."

\emph{Er war doch nicht auf ein anderes Date gegangen, oder?} dachte Draco, aber das machte keinen Sinn.

"Harry", sagte Draco, "es tut mir leid, aber ich muss das trotzdem fragen, hast du dem Schlammblutmädchen wirklich einen teuren Beutel zum Geburtstag geschenkt?"

"Ja, das habe ich. Du hast natürlich schon herausgefunden, warum."

Draco griff nach oben und fuhr sich frustriert mit den Fingern durch die Haare, wobei seine Kutte über den Handrücken strich. Er war sich nicht ganz sicher gewesen, warum, aber jetzt konnte er es nicht mehr sagen. Und Slytherin wusste, dass er Harry Potter den Hof machte, er hatte es im Verteidigungsunterricht deutlich genug gemacht.

"Harry", sagte Draco, "die Leute wissen, dass ich mit dir befreundet bin, sie wissen natürlich nichts von der Verschwörung, aber sie wissen, dass wir Freunde sind, und es lässt mich schlecht aussehen, wenn du so etwas machst."

Harry Potters Gesicht straffte sich. "Jeder in Slytherin, der nicht begreift, dass man zu Leuten, die man eigentlich nicht mag, so tun kann als wäre man nett, sollte zermahlen und an Haustierschlangen verfüttert werden."

"Es gibt eine Menge Leute in Slytherin, die das nicht verstehen", sagte Draco, seine Stimme ernst. "Die meisten Leute sind dumm, und man muss vor ihnen trotzdem gut aussehen." Harry Potter musste das verstehen, wenn er jemals im Leben etwas erreichen wollte.

"Was kümmert es dich, was andere Leute denken? Willst du wirklich dein Leben damit verbringen, alles, was du tust, den dümmsten Idioten in Slytherin erklären zu müssen und dich von ihnen beurteilen zu lassen? Tut mir leid, Draco, aber ich senke meine schlauen Pläne nicht auf das Niveau dessen, was die dümmsten Slytherins verstehen können, nur weil es dich sonst vielleicht schlecht aussehen lässt. Nicht mal deine Freundschaft ist das wert. \emph{Es würde mir den ganzen Spaß am Leben nehmen.} Sag mir, dass du noch nie dasselbe gedacht hast, wenn jemand in Slytherin zu dumm zum Atmen ist, dass es unter der Würde eines Malfoys ist, sich ihnen anbiedern zu müssen."

Das hatte Draco wirklich nicht. Niemals. Sich Idioten anbiedern war wie atmen, man tat es, ohne darüber nachzudenken.

"Harry", sagte Draco schließlich. "Einfach zu tun, was man will, ohne sich Gedanken darüber zu machen, wie es aussieht, ist nicht klug. Der Dunkle Lord sorgte sich darum, wie er aussah! Er wurde gefürchtet und gehasst, und er wusste genau, welche Art von Angst und Hass er erzeugen wollte. Jeder muss sich Gedanken darüber machen, was andere Leute denken."

Die vermummte Gestalt zuckte mit den Schultern. "Vielleicht. Erinner mich daran, dir irgendwann einmal von etwas zu erzählen, das sich Aschs Konformitätsexperiment nennt, Du könntest es recht amüsant finden. Fürs Erste möchte ich nur anmerken, dass es gefährlich ist, sich aus Instinkt darüber Gedanken zu machen, was andere Leute denken, weil es einen tatsächlich interessiert und nicht aus kaltblütiger Berechnung. Erinner dich, ich wurde fünfzehn Minuten lang von älteren Slytherins geschlagen und schikaniert, und danach bin ich aufgestanden und habe ihnen gnädig vergeben. Wie es sich für einen guten und tugendhaften Jungen, der lebt, gehört.

Aber mein kaltblütiges Kalkül, Draco, sagt mir, dass ich keine Verwendung für die dümmsten Idioten in Slytherin habe, \emph{da ich keine Hausschlange besitze}. Also habe ich keinen Grund, mich darum zu kümmern, was sie darüber denken, wie ich mein Duell mit Hermine Granger führe."

Draco ballte seine Fäuste nicht vor Frustration. "Sie ist nur irgendein Schlammblut", sagte Draco, wobei er seine Stimme ruhig hielt, anstatt zu schreien. "Wenn du sie nicht magst, schubs sie die Treppe runter."

"Ravenclaw würde wissen -"

"Lass Pansy Parkinson sie die Treppe hinunterstoßen! Du müsstest sie nicht einmal manipulieren, biete ihr einen Sickel an und sie würde es tun!"

"Ich würde es wissen! Hermine hat mich in einem Lesewettbewerb geschlagen, sie hat bessere Noten als ich, ich muss sie mit meinem Verstand besiegen, sonst zählt es nicht!"

"Sie ist nur ein Schlammblut! Warum respektierst du sie so sehr?"

"Sie ist eine Macht unter den Ravenclaws! Was kümmert es dich, was irgendein machtloser Idiot in Slytherin denkt?"

"Das nennt man Politik! Und wenn man es nicht spielen kann, kann man keine Macht haben!"

"Auf dem Mond zu laufen ist Macht! Ein großer Zauberer zu sein ist Macht! Es gibt Arten von Macht, die nicht erfordern, dass ich den Rest meines Lebens damit verbringe, mich bei Idioten anzubiedern!"

Beide hielten inne und begannen, fast unisono, tief einzuatmen, um sich zu beruhigen.

"Tut mir leid", sagte Harry Potter nach ein paar Augenblicken und wischte sich den Schweiß von der Stirn.

"Tut mir leid, Draco. Du hast eine Menge politischer Macht und es ist sinnvoll, dass du sie behältst. Du solltest ausrechnen, was Slytherin denkt. Es ist ein wichtiges Spiel und ich hätte es nicht beleidigen dürfen. Aber du kannst nicht von mir verlangen, mein Spiel in Ravenclaw herunterzuziehen, nur damit du nicht schlecht dastehst, wenn du dich mit mir verbündest. Sag Slytherin, dass du mit den Zähnen knirschst, während du so tust, als wärst du mein Freund."

Das war genau das, was Draco Slytherin gesagt hatte, und er war sich immer noch nicht sicher, ob es wahr war.

"Wie auch immer", sagte Draco. "Wo wir gerade bei deinem Image sind. Ich fürchte, ich habe ein paar schlechte Nachrichten. Rita Kimmkorn hat einige der Geschichten über dich gehört und sie hat Fragen gestellt."

Harry Potter hob die Augenbrauen.

"Wer?"

"Sie schreibt für den Tagespropheten", sagte Draco. Er versuchte, die Sorge aus seiner Stimme zu halten. Der Tagesprophet war eines von Vaters wichtigsten Werkzeugen, er benutzte ihn wie einen Zauberstab.

"Das ist die Zeitung, der die Leute tatsächlich Aufmerksamkeit schenken. Rita Kimmkorn schreibt über Berühmtheiten und benutzt ihre Feder, um deren aufgeblasenen Ruf zu zerstören. Wenn sie keine Gerüchte über dich finden kann, erfindet sie einfach ihre eigenen."

"Ich verstehe", sagte Harry Potter. Sein grün beleuchtetes Gesicht sah unter der Kutte sehr nachdenklich aus.

Draco zögerte, bevor er sagte, was er als Nächstes zu sagen hatte. Inzwischen hatte sicher jemand Vater berichtet, dass er Harry Potter den Hof machte, und Vater würde auch wissen, dass Draco nicht nach Hause geschrieben hatte, und Vater würde verstehen, dass Draco nicht glaubte, es wirklich geheim halten zu können, was eine klare Botschaft aussandte, dass Draco jetzt sein eigenes Spiel trieb, aber immer noch auf Vaters Seite war, denn wenn Draco in Versuchung geraten wäre, hätte er falsche Berichte geschickt. Daraus folgte, dass Vater wahrscheinlich vorausgesehen hatte, was Draco als nächstes sagen würde.

Das Spiel mit Vater in echt zu spielen, war ein ziemlich zermürbendes Gefühl. Selbst wenn sie auf der gleichen Seite standen. Auf der einen Seite war es aufregend, aber Draco wusste auch, dass sich am Ende herausstellen würde, dass Vater das Spiel besser gespielt hatte. Anders konnte es gar nicht laufen.

"Harry", sagte Draco schließlich. "Das ist kein Vorschlag. Das ist kein Ratschlag von mir. Es ist einfach so, wie es ist. Mein Vater könnte diesen Artikel mit ziemlicher Sicherheit unterdrücken. Aber es würde dich etwas kosten."

Dass Vater erwartet hatte, dass Draco Harry Potter genau das sagen würde, sprach Draco nicht laut aus. Harry Potter würde es selbst herausfinden, oder auch nicht.

Doch stattdessen schüttelte Harry Potter den Kopf und lächelte unter der Kutte.

„Ich habe nicht die Absicht, zu versuchen, Rita Kimmkorn zu unterdrücken."

Draco versuchte gar nicht erst, die Ungläubigkeit aus seiner Stimme herauszuhalten. "Du kannst mir nicht erzählen, dass es dir egal ist, was die Zeitung über dich schreibt!"

"Es kümmert mich weniger, als du vielleicht denkst", sagte Harry Potter. "Aber ich habe meine eigenen Methoden, mit Leuten wie Kimmkorn umzugehen. Ich brauche die Hilfe von Lucius nicht."

Ein besorgter Blick kam über Dracos Gesicht, bevor er ihn aufhalten konnte. Was auch immer Harry Potter als Nächstes vorhatte, es würde etwas sein, womit Vater nicht gerechnet hatte, und Draco wurde sehr nervös, wohin das führen könnte. Draco bemerkte auch, dass sein Haar unter der Kutte verschwitzt war. Er hatte so etwas noch nie getragen und hatte nicht bemerkt, dass die Umhänge der Todesser wahrscheinlich so etwas wie Kühlzauber hatten.

Harry Potter wischte sich wieder den Schweiß von der Stirn, zog eine Grimasse, zückte seinen Zauberstab, richtete ihn nach oben, holte tief Luft und sagte: "Frigideiro!"

Augenblicke später spürte Draco den kalten Luftzug.

"Frigideiro! Frigideiro! Frigideiro! Frigideiro! Frigideiro!"

Dann ließ Harry Potter den Zauberstab sinken, obwohl seine Hand ein wenig zittrig wirkte, und steckte ihn zurück in seinen Umhang. Der ganze Raum schien spürbar kühler zu werden. Das hätte Draco auch tun können, \emph{aber trotzdem, nicht schlecht.}

"Also", sagte Draco. "Wissenschaft. Du wirst mir etwas über Blut erzählen."

"Wir werden etwas über Blut herausfinden", sagte Harry Potter. "Indem wir Experimente machen."

"Na schön", sagte Draco. "Was für Experimente?"

Harry Potter lächelte böse unter seiner Kutte und sagte: "Sag du es mir."

Draco hatte von etwas gehört, das man die sokratische Methode nannte, also das Lehren durch Fragen

(benannt nach einem antiken Philosophen, der zu klug gewesen war, um ein echter Muggel zu sein, und deshalb ein getarnter reinblütiger Zauberer gewesen war).

Einer seiner Tutoren hatte den sokratischen Unterricht sehr oft angewendet. Es war lästig gewesen, aber effektiv. Dann gab es noch die Potter-Methode, die wahnsinnig war.

Um fair zu sein, musste Draco zugeben, dass Harry Potter die sokratische Methode zuerst ausprobiert hatte und sie nicht besonders gut funktioniert hatte.

Harry Potter hatte gefragt, wie Draco die Hypothese der Blutpuristen widerlegen wolle, dass Zauberer heute nicht mehr die tollen Sachen machen könnten, die sie vor acht Jahrhunderten gemacht hatten, weil sie sich mit Muggelgeborenen und Squibs gekreuzt hätten.

Draco hatte gesagt, dass er nicht verstehe, wie Harry Potter mit geradem Gesicht dasitzen und behaupten könne, dies sei keine Falle.

Harry Potter hatte geantwortet, immer noch mit ernstem Gesicht, dass, wenn es eine Falle gewesen wäre, es so erbärmlich offensichtlich gewesen wäre, dass man ihn zermahlen und an Haustierschlangen verfüttern sollte, aber es war keine Falle, es war einfach eine Regel, wie Wissenschaftler arbeiteten, dass man versuchen musste, seine eigenen Theorien zu widerlegen, und wenn man einen ehrlichen Versuch unternahm und scheiterte, war das ein Sieg.

Draco hatte versucht, auf die umwerfende Dummheit dieses Vorgehens hinzuweisen, indem er vorschlug, dass der Schlüssel zum Überleben eines Duells darin bestand, Avada Kedavra auf den eigenen Fuß zu zaubern und zu verfehlen.

Harry Potter hatte genickt. Draco hatte den Kopf geschüttelt. Harry Potter hatte dann die Idee präsentiert, dass Wissenschaftler Ideen beim Kämpfen beobachten, um zu sehen, welche gewinnt, und man kann nicht ohne einen Gegner kämpfen, also musste Draco Gegner für die blutpuristische Hypothese finden, gegen die er kämpfen konnte, damit der Blutpurismus gewinnen konnte, was Draco ein wenig besser verstand, obwohl Harry Potter es mit einem eher geschmacklosen Blick gesagt hatte.

Zum Beispiel war es klar, dass, wenn der Blutpurismus die Art und Weise war, wie die Welt wirklich war, dann musste der Himmel einfach blau sein, und wenn irgendeine andere Theorie wahr war, musste der Himmel einfach grün sein; und niemand hatte den Himmel bisher gesehen; und dann ging man nach draußen und schaute nach und die Blutpuristen gewannen; und nachdem das sechsmal hintereinander passiert war, würden die Leute anfangen, den Trend zu bemerken. Harry Potter hatte dann weiter behauptet, dass alle Gegner, die Draco erfand, zu schwach waren, so dass der Blutpurismus keine Anerkennung für den Sieg über sie bekommen würde, weil der Kampf nicht beeindruckend genug wäre.

Auch das hatte Draco verstanden.

\emph{Dass Zauberer schwächer geworden sind, weil Hauselfen unsere Magie klauen, hatte sich für ihn auch nicht beeindruckend angehört.}

(Obwohl Harry Potter gesagt hatte, dass man das zumindest testen könne, indem man versuchen könne, zu überprüfen, ob die Hauselfen im Laufe der Zeit stärker geworden seien, und sogar ein Bild zeichnen könne, das die zunehmende Stärke der Hauselfen darstelle, und ein anderes Bild, das die abnehmende Stärke der Zauberer darstelle, und wenn die beiden Bilder übereinstimmten, würde das auf die Hauselfen hindeuten, und das alles in einem so völlig ernsten Tonfall, dass Draco den Impuls verspürte, Dobby unter Veritaserum ein paar spitze Fragen zu stellen, bevor er sich davon losriss.)

Und Harry Potter hatte schließlich gesagt, dass Draco den Kampf nicht manipulieren könne, Wissenschaftler seien nicht dumm, es wäre offensichtlich, wenn man den Kampf manipulieren würde, es müsse ein echter Kampf sein, zwischen zwei verschiedenen Theorien, die beide wirklich wahr sein könnten, mit einem Test, den nur die wahre Hypothese gewinnen würde, etwas, das tatsächlich unterschiedlich ausfallen würde, je nachdem, welche Hypothese tatsächlich richtig sei, und es gäbe erfahrene Wissenschaftler, die zusehen würden, um sicherzustellen, dass genau das passierte. Harry Potter hatte behauptet, dass er selbst nur wissen wollte, wie Blut wirklich funktionierte, und dafür musste er sehen, wie der Blutpurismus wirklich gewann, und Draco sollte ihn nicht mit Theorien täuschen, die nur dazu da waren, um niedergemacht zu werden.

Selbst nachdem er den Punkt gesehen hatte, war Draco nicht in der Lage gewesen, irgendwelche "plausiblen Alternativen", wie Harry Potter es ausdrückte, zu der Idee zu erfinden, dass Zauberer weniger mächtig wurden, weil sie ihr Blut mit Schlamm vermischten. Es war zu offensichtlich wahr. Da hatte Harry Potter ziemlich frustriert gesagt, dass er sich nicht vorstellen könne, dass Draco wirklich so schlecht darin sei, andere Standpunkte in Betracht zu ziehen, es habe doch sicher Todesser gegeben, die sich als Feinde des Blutpurismus ausgegeben und viel plausiblere Argumente gegen ihre eigene Seite vorgebracht hätten als Draco. Wenn Draco versucht hätte, sich als Mitglied von Dumbledores Fraktion auszugeben und mit der Hauselfen-Hypothese aufzuwarten, hätte er niemanden auch nur eine Sekunde lang getäuscht. Draco war gezwungen gewesen, zuzugeben, dass das ein Argument war.

\emph{Daher auch die Potter-Methode.}

"Bitte, Dr~Malfoy", jammerte Harry Potter, "warum wollen Sie meine Arbeit nicht annehmen?"

Harry Potter hatte den Satz \emph{"Tu einfach so, als wärst du ein Todesser der so tut als wäre er ein Wissenschaftler"} dreimal wiederholen müssen, bevor Draco es verstanden hatte.

In diesem Moment hatte Draco erkannt, dass mit Harry Potters Gehirn etwas zutiefst verkehrt war, und jeder, der Legilimenz daran ausprobierte, würde wahrscheinlich nie wieder herauskommen. Harry Potter war dann in weitere und erhebliche Details gegangen:

Draco sollte sich als Todesser ausgeben, der sich als Herausgeber einer wissenschaftlichen Zeitschrift, Dr~Malfoy, ausgab, der die Abhandlung seines Feindes Dr~Potter \emph{"Über die Vererbbarkeit magischer Fähigkeiten"} ablehnen wollte, und wenn der Todesser sich nicht so verhielte, wie es ein echter Wissenschaftler tun würde, würde er als Todesser enttarnt und hingerichtet werden, während Dr~Malfoy auch von seinen eigenen Rivalen beobachtet wurde und er musste den Anschein erwecken, dass er Dr~Potters Aufsatz aus neutralen wissenschaftlichen Gründen ablehnte, sonst würde er seine Position als Herausgeber der Zeitschrift verlieren.

Es war ein Wunder, dass der Sortierhut in St. Mungo's nicht wie verrückt klapperte. Es war auch die komplizierteste Sache, um die Draco jemals gebeten worden war, und es gab keine Möglichkeit, dass er die Herausforderung hätte ablehnen können. Gerade jetzt waren sie, wie Harry Potter es ausgedrückt hatte, dabei in Stimmung zu kommen.

"Ich fürchte, Dr~Potter, dass Sie das mit der falschen Tintenfarbe geschrieben haben", sagte Draco. "Der Nächste!"

Dr~Potters Gesicht zerknitterte vor Verzweiflung, und Draco konnte nicht umhin, einen Anflug von Dr~Malfoys Schadenfreude zu spüren, obwohl der Todesser nur so tat, als wäre er Dr~Malfoy. Dieser Teil hat Spaß gemacht. Er hätte das den ganzen Tag lang tun können.

Dr~Potter stand vom Stuhl auf, sackte erschrocken zusammen und stapfte davon. Er verwandelte sich in Harry Potter, der Draco einen Daumen hoch gab, und verwandelte sich dann wieder in Dr~Potter, der nun mit einem eifrigen Lächeln auf ihn zukam. Dr~Potter setzte sich hin und überreichte Dr~Malfoy ein Stück Pergament, auf dem geschrieben stand:

\emph{Über die Vererbbarkeit von magischen Fähigkeiten}

\emph{Dr~H. J. Potter-Evans-Verres,}

\emph{Institut für hinreichend fortgeschrittene Wissenschaft}

\emph{Meine Beobachtung: Heutige Zauberer können nicht so beeindruckende Dinge tun, wie es Zauberer vor 800 Jahren taten.}

\emph{Meine Schlussfolgerung: Die Zauberer sind schwächer geworden durch die Vermischung ihres Blutes mit Muggelgeborenen und Squibs.}

"Dr~Malfoy", sagte Dr~Potter mit einem hoffnungsvollen Blick, "ich habe mich gefragt, ob das Journal \emph{'nicht reproduzierbare Resultate'} meine Abhandlung mit dem Titel '\emph{Über die Vererbbarkeit von magischen Fähigkeiten}' zur Veröffentlichung in Betracht ziehen könnte."

Draco betrachtete das Pergament und lächelte, während er mögliche Ablehnungen in Betracht zog. Wenn er ein Professor wäre, hätte er den Aufsatz als zu kurz abgelehnt, also -

"Er ist zu lang, Dr~Potter", sagte Dr~Malfoy.

Einen Moment lang lag echte Ungläubigkeit auf Dr~Potters Gesicht. "Ah…", sagte Dr~Potter. "Wie wäre es, wenn ich die separaten Zeilen für Beobachtungen und Schlussfolgerungen weglasse und einfach eine dafür einfüge -"

"Dann wird es zu kurz. Der Nächste!"

Dr~Potter stapfte davon. "In Ordnung", sagte Harry Potter, "du wirst zu gut darin. Noch zwei Mal zum Üben, und dann ist das dritte Mal echt, keine Unterbrechungen dazwischen, ich komme einfach direkt auf dich zu, und dieses Mal wirst du das Papier aufgrund des tatsächlichen Inhalts ablehnen, denk daran, deine wissenschaftlichen Konkurrenten schauen zu."

Dr~Potters nächstes Paper war in jeder Hinsicht perfekt, ein Wunderwerk seiner Art, musste aber leider abgelehnt werden, weil Dr~Malfoys Journal Probleme mit dem Buchstaben E hatte. Dr~Potter bot an, es ohne diese Worte neu zu schreiben, und Dr~Malfoy erklärte, dass es in Wirklichkeit eher ein Problem mit den Vokalen war.

Der Aufsatz danach wurde abgelehnt, weil es Dienstag war. In Wirklichkeit war es aber ein Samstag. Dr~Potter versuchte, darauf hinzuweisen und bekam zu hören: \emph{"Der Nächste!"}

(Draco begann zu verstehen, warum Snape seine Macht über Dumbledore ausgenutzt hatte, um eine Position zu bekommen, die es ihm erlaubte, furchtbar zu Schülern zu sein.)

Und dann - näherte sich Dr~Potter mit einem überlegenen Grinsen im Gesicht.

"Dies ist meine neueste Arbeit, '\emph{Über die Vererbbarkeit von magischen Fähigkeiten}', erklärte Dr~Potter selbstbewusst und schob das Pergament vor.

"Ich habe mich entschlossen, Ihrer Zeitschrift die Veröffentlichung zu gestatten, und ich habe sie in perfekter Übereinstimmung mit Ihren Richtlinien vorbereitet, damit Sie sie schnell veröffentlichen können."

Der Todesser beschloss, Dr~Potter aufzuspüren und zu töten, nachdem seine Mission erfüllt war. Dr~Malfoy behielt ein höfliches Lächeln auf seinem Gesicht, da seine Rivalen zusahen, und sagte…

(Die Pause dehnte sich, wobei Dr~Potter ihn ungeduldig ansah.) …

"Lassen Sie mich bitte einen Blick darauf werfen."

Dr~Malfoy nahm das Pergament und betrachtete es sorgfältig. Der Todesser wurde langsam nervös, weil er kein richtiger Wissenschaftler war, und Draco versuchte, sich daran zu erinnern, wie man wie Harry Potter redet.

"Sie, ähm, müssen andere mögliche Erklärungen für Ihre, ähm, Beobachtung in Betracht ziehen, außer dieser einen -"

"Wirklich?", unterbrach ihn Dr~Potter. "Zum Beispiel was genau? Dass Hauselfen unsere Magie stehlen? Meine Daten lassen nur eine mögliche Schlussfolgerung zu, Dr~Malfoy. Es gibt keine anderen plausiblen Hypothesen."

Draco versuchte krampfhaft, seinem Gehirn zu befehlen, nachzudenken, was würde er sagen, wenn er sich als Mitglied von Dumbledores Fraktion ausgab, was war ihrer Meinung nach die Erklärung für den Niedergang der Zauberer, Draco hatte sich nie die Mühe gemacht, das wirklich zu fragen …

"Wenn Ihnen keine andere Erklärung für meine Daten einfällt, müssen Sie meine Arbeit veröffentlichen, Dr~Malfoy."

Es war das höhnische Grinsen auf Dr~Potters Gesicht, das den Ausschlag gab.

"Ach ja?", schnauzte Dr~Malfoy. "Woher wollen Sie wissen, dass die Magie selbst nicht verblasst?"

Die Zeit blieb stehen. Draco und Harry Potter tauschten entsetzte Blicke aus. Dann spuckte Harry Potter etwas aus, das wahrscheinlich ein extrem schlimmes Wort war, wenn man von Muggeln erzogen worden war.

"Daran habe ich nicht gedacht!", sagte Harry Potter. "Und ich hätte es tun sollen. Die Magie verschwindet. Verdammt, verdammt, verdammt!"

Der Alarm in Harry Potters Stimme war ansteckend. Ohne darüber nachzudenken, fuhr Dracos Hand in seinen Umhang und griff nach seinem Zauberstab. Er hatte gedacht, das Haus Malfoy sei sicher, solange man nur in Familien einheiratete, die ihre Blutlinien vier Generationen zurückverfolgen konnten, sollte man sicher sein, es war ihm noch nie in den Sinn gekommen, dass es vielleicht nichts gab, was man tun konnte, um das Ende der Magie aufzuhalten.

"Harry, was sollen wir tun?" Dracos Stimme erhob sich in Panik. "\textbf{Was sollen wir tun?!}"

"Lass mich nachdenken!"

Nach ein paar Augenblicken schnappte sich Harry von einem nahegelegenen Schreibtisch denselben Federkiel und dieselbe Pergamentrolle, die er für seinen Scheinaufsatz benutzt hatte, und begann etwas zu kritzeln.

"Wir werden es herausfinden", sagte Harry mit fester Stimme, "wenn die Magie aus der Welt schwindet, werden wir herausfinden, wie schnell sie schwindet und wie viel Zeit wir noch haben, um etwas zu tun, und dann werden wir herausfinden, warum sie schwindet, und dann werden wir etwas dagegen tun. Draco, haben die Zauberkräfte gleichmäßig abgenommen oder gab es plötzliche Einbrüche?"

"Ich… Ich weiß es nicht…"

"Du hast mir gesagt, dass niemand den vier Gründern von Hogwarts gewachsen ist. Dann geht das schon seit mindestens acht Jahrhunderten so? Du kannst dich nicht daran erinnern, etwas davon gehört zu haben, dass die Probleme plötzlich vor fünf Jahrhunderten aufgetaucht sind oder so etwas in der Art?"

Draco versuchte krampfhaft, nachzudenken. "Ich habe immer gehört, dass niemand so gut war wie Merlin und danach war niemand so gut wie die Gründer von Hogwarts."

"In Ordnung", sagte Harry. Er kritzelte immer noch. "Weil vor drei Jahrhunderten die Muggel anfingen, nicht mehr an Magie zu glauben, und ich dachte, das könnte etwas damit zu tun haben. Und vor etwa anderthalb Jahrhunderten begannen die Muggel, eine Art von Technologie zu benutzen, die nicht mehr mit Magie funktioniert, und ich habe mich gefragt, ob es vielleicht auch andersherum geht."

Draco explodierte aus seinem Stuhl, so wütend, dass er kaum noch sprechen konnte. "Es sind die Muggel -"

\textbf{"Verdammt!"}, brüllte Harry. "Hast du dir nicht einmal selbst zugehört? Das geht schon seit mindestens acht Jahrhunderten so und die Muggel haben damals nichts Interessantes getan! Wir müssen die wahre Wahrheit herausfinden! Die Muggel haben vielleicht etwas damit zu tun, aber wenn sie es nicht tun und du alles auf sie schiebst und das uns davon abhält, herauszufinden, was wirklich los ist, dann wirst du eines Tages morgens aufwachen und herausfinden, dass dein Zauberstab nur ein Stecken ist!"

Dracos Atem blieb ihm im Hals stecken. Sein Vater hatte in seinen Reden oft gesagt, dass unsere Zauberstäbe in unseren Händen zerbrechen würden, aber Draco hatte noch nie wirklich darüber nachgedacht, was das bedeutete, es würde ihm ja schließlich nicht passieren. Und jetzt schien es plötzlich sehr real zu sein. Nur ein Holzstäbchen. Draco konnte sich vorstellen, wie es wäre, seinen Zauberstab zu zücken und zu versuchen, einen Zauber zu sprechen und festzustellen, dass nichts passierte… Das könnte jedem passieren. Es würde keine Zauberer mehr geben, keine Magie mehr, niemals. Nur Muggel, die ein paar Legenden darüber hatten, wozu ihre Vorfahren fähig gewesen waren. Einige der Muggel würden Malfoy heißen, und das wäre alles, was von dem Namen übrig wäre.

\emph{Zum ersten Mal in seinem Leben wurde Draco klar, warum es die Todesser gab.}

Er war immer davon ausgegangen, dass man ein Todesser wird, wenn man erwachsen ist. Jetzt verstand Draco, er wusste, warum Vater und Vaters Freunde geschworen hatten, ihr Leben zu geben, um den Albtraum zu verhindern, es gab Dinge, bei denen man nicht einfach zusehen konnte.

\emph{Aber was, wenn es trotzdem passieren würde, was, wenn all die Opfer, all die Freunde, die sie an Dumbledore verloren hatten, die Familie, die sie verloren hatten, was, wenn das alles umsonst gewesen wäre?}

"Die Magie darf nicht schwinden", sagte Draco. Seine Stimme war brüchig. "Das wäre nicht fair."

Harry hörte auf zu kritzeln und sah auf. Sein Gesicht hatte einen wütenden Ausdruck. "Hat dir dein Vater nie gesagt, dass das Leben nicht fair ist?"

Vater hatte das jedes Mal gesagt, wenn Draco das Wort benutzte.

"Aber, aber, es ist zu furchtbar, um zu glauben, dass -"

"Draco, lass mich dir etwas vorstellen, das ich die Litanei von Tarski nenne. Sie ändert sich jedes Mal, wenn du sie benutzt. Bei dieser Gelegenheit läuft sie so ab:

\emph{Wenn die Magie aus der Welt verschwindet, möchte ich glauben, dass die Magie aus der Welt verschwindet. Wenn die Magie nicht aus der Welt verschwindet, möchte ich nicht glauben, dass die Magie aus der Welt verschwindet. Ich möchte mich nicht an Überzeugungen klammern, die ich vielleicht nicht will.}

Wenn wir in einer Welt leben, in der die Magie schwindet, dann müssen wir das glauben, wir müssen wissen, was auf uns zukommt, damit wir es aufhalten können, oder im allerschlimmsten Fall darauf vorbereitet sein, in der uns verbleibenden Zeit zu tun, was wir können.

\emph{Nicht daran zu glauben, wird es nicht verhindern, dass es passiert.}

Die einzige Frage, die wir uns stellen müssen, ist also, ob die Magie tatsächlich schwindet, und wenn das die Welt ist, in der wir leben, dann ist es das, was wir glauben wollen.

\emph{Litanei von Gendlin: Was wahr ist, ist schon so. Es zuzugeben, macht es nicht schlimmer.}

Hast du das verstanden, Draco? Du musst sie nachher auswendig lernen. Das wiederholst du jedes Mal, wenn du dich fragst, ob es eine gute Idee ist, etwas zu glauben, das eigentlich nicht wahr ist. Ich möchte sogar, dass du es jetzt sofort sagst. \emph{Was wahr ist, ist schon so, es sich einzugestehen, macht es nicht schlimmer.} Sag es."

"Was wahr ist, ist schon so", wiederholte Draco mit zitternder Stimme, "sich das einzugestehen, macht es nicht schlimmer."

"Wenn die Magie schwindet, will ich glauben, dass die Magie schwindet. Wenn die Magie nicht schwindet, will ich nicht glauben, dass sie schwindet. Sag es."

Draco wiederholte die Worte, während die Übelkeit in seinem Magen aufgewühlt war.

"Gut", sagte Harry, "denk daran, dass es vielleicht gar nicht passiert, und dann musst du es auch nicht glauben. Wir wollen erst einmal nur wissen, was wirklich vor sich geht, in welcher Welt wir eigentlich leben."

Harry wandte sich wieder seiner Arbeit zu, kritzelte noch etwas und drehte dann das Pergament so, dass Draco es sehen konnte. Draco lehnte sich über den Schreibtisch und Harry brachte das grüne Licht näher.

\emph{\textbf{\uline{Beobachtung}}}: Die Zauberei ist heute nicht mehr so mächtig wie zur Zeit der Gründung von Hogwarts.

\strut 

\emph{\textbf{\uline{Hypothesen}}}:

\emph{1. Die Magie selbst schwindet.

2. Die Zauberer kreuzen sich mit Muggeln und Squibs.

3. Das Wissen, mächtige Zaubersprüche zu wirken, geht verloren.

4. Zauberer essen als Kinder das Falsche, oder etwas anderes als Blut lässt sie schwächer aufwachsen.

5. Die Muggeltechnologie stört die Magie. (Seit 800 Jahren?)

6. Stärkere Zauberer haben weniger Kinder. (Draco = Einzelkind? Prüfen Sie, ob 3 mächtige Zauberer, Quirrell / Dumbledore / Dunkler Lord, Kinder hatten.)}

\textbf{\emph{\uline{Tests}}}\emph{:}

"In Ordnung", sagte Harry. Seine Atmung klang ein wenig ruhiger. "Wenn man es mit einem verwirrenden Problem zu tun hat und keine Ahnung hat, was los ist, ist es klug, sich ein paar wirklich einfache Tests auszudenken, Dinge, die man sich sofort ansehen kann. Wir brauchen schnelle Tests, die zwischen diesen Hypothesen unterscheiden. Beobachtungen, die bei mindestens einer von ihnen anders ausfallen würden als bei allen anderen."

Draco starrte geschockt auf die Liste. Ihm wurde plötzlich klar, dass er furchtbar viele Reinblüter kannte, die Einzelkinder waren. Er selbst, Vincent, Gregory, praktisch alle. Die beiden mächtigsten Zauberer, über die jeder sprach, waren Dumbledore und der Dunkle Lord, und beide hatten keine Kinder, genau wie Harry vermutet hatte…

"Es wird wirklich schwer sein, zwischen zwei und sechs zu unterscheiden", sagte Harry, "es liegt so oder so im Blut, man müsste versuchen, den Niedergang der Zauberei zu verfolgen und das damit vergleichen, wie viele Kinder verschiedene Zauberer hatten, und die Fähigkeiten von Muggelgeborenen im Vergleich zu Reinblütern messen…"

Harrys Finger klopften nervös auf den Schreibtisch. "Lass uns einfach 6 mit 2 in einen Topf werfen und sie erst einmal \emph{die Bluthypothese} nennen.

\textbf{4} ist unwahrscheinlich, weil dann jeder einen plötzlichen Rückgang bemerken würde, wenn die Zauberer auf neue Nahrungsmittel umsteigen, es ist schwer zu erkennen, was sich über 800 Jahre hinweg stetig verändert hätte.

\textbf{5} ist aus dem gleichen Grund unwahrscheinlich, kein plötzlicher Abfall, Muggel haben vor 800 Jahren nichts gemacht.

\textbf{4} sieht aus wie \textbf{2} und \textbf{5} sieht sowieso aus wie \textbf{1}. Wir sollten also hauptsächlich versuchen, zwischen \textbf{1, 2} und \textbf{3} zu unterscheiden."

Harry drehte das Pergament zu sich, zeichnete eine Ellipse um die drei Zahlen und drehte es zurück.

"\emph{Die Magie schwindet},

\emph{das Blut wird schwächer,}

\emph{das Wissen schwindet.}

Welcher Test fällt anders aus, je nachdem, was davon zutrifft? Was könnten wir sehen, das bedeuten würde, dass eine dieser Zahlen falsch ist?"

"Ich weiß es nicht!", platzte Draco heraus. "Warum fragst du mich? Du bist doch der Wissenschaftler!"

"Draco", sagte Harry mit einem Ton flehender Verzweiflung in der Stimme, "ich weiß nur, was Muggelwissenschaftler wissen! Du bist in der Welt der Zauberer aufgewachsen, ich nicht! Du kennst mehr Magie als ich und du weißt mehr über Magie als ich und du hast dir diese ganze Idee überhaupt erst ausgedacht, also fang an wie ein Wissenschaftler zu denken und löse das Problem!"

Draco schluckte schwer und starrte auf das Papier. Die Magie schwindet… Zauberer kreuzen sich mit Muggeln… das Wissen geht verloren…

"Wie sieht die Welt aus, wenn die Magie schwindet?", fragte Harry Potter. "Du weißt mehr über Magie, du solltest derjenige sein, der rät, nicht ich! Stell dir vor, du erzählst eine Geschichte darüber, was passiert in der Geschichte?"

Draco stellte es sich vor.

"Zaubersprüche, die früher funktioniert haben, funktionieren nicht mehr."

\emph{Zauberer wachen auf und stellen fest, dass ihre Zauberstäbe Holzstöcke sind…}

"Wie sieht die Welt aus, wenn das Blut der Zauberer schwächer wird?"

"Menschen können Dinge nicht mehr tun, die ihre Vorfahren tun konnten."

"Wie sieht die Welt aus, wenn das Wissen verloren geht?"

"Die Leute wissen gar nicht mehr, wie man die Zaubersprüche wirkt…", sagte Draco. Er hielt inne, überrascht über sich selbst. "Das ist ein Test, nicht wahr?"

Harry nickte entschlossen. "Das ist einer." Er schrieb es auf dem Pergament unter Tests auf:

\emph{A. Gibt es Zaubersprüche, die wir kennen, aber nicht wirken können (1 oder 2), oder sind die verlorenen Zaubersprüche nicht mehr bekannt (3)?}

"Das ist also die Unterscheidung zwischen 1 und 2 auf der einen Seite und 3 auf der anderen Seite", sagte Harry. "Nun brauchen wir eine Möglichkeit, zwischen 1 und 2 zu unterscheiden. Die Magie schwindet, das Blut wird schwächer, wie können wir das unterscheiden?"

"Welche Art von Zaubern haben die Schüler früher in ihrem ersten Jahr in Hogwarts gewirkt?", fragte Draco. "Wenn sie früher viel mächtigere wirken konnten, war das Blut stärker -"

Harry Potter schüttelte den Kopf. "Oder die Magie selbst war stärker. Wir müssen einen Weg finden, um den Unterschied festzustellen." Harry stand von seinem Stuhl auf und begann nervös durch das Klassenzimmer zu laufen.

"Nein, warte, das könnte immer noch funktionieren. Nehmen wir an, verschiedene Zaubersprüche verbrauchen unterschiedliche Mengen an magischer Energie. Wenn dann die Umgebungsmagie schwächer wird, würden die mächtigen Zaubersprüche zuerst sterben, aber die Zaubersprüche, die jeder im ersten Jahr lernt, würden gleich bleiben…" Harrys nervöser Schritt beschleunigte sich.

"Das ist kein sehr guter Test, es geht eher darum, dass mächtige Zauberei verloren geht, als dass alle Zauberei verloren geht, jemandes Blut könnte zu schwach für mächtige Zauberei sein, aber stark genug für einfache Zauber… Draco, weißt du, ob mächtigere Zauberer innerhalb einer Ära, wie etwa mächtige Zauberer aus diesem Jahrhundert, als Kinder mächtiger sind? Wenn der Dunkle Lord den Abkühlungszauber mit elf Jahren gewirkt hätte, hätte er dann den ganzen Raum einfrieren können?"

Dracos Gesicht verfinsterte sich, als er versuchte, sich zu erinnern.

"Ich kann mich nicht erinnern, etwas über den Dunklen Lord gehört zu haben, aber ich glaube, Dumbledore soll bei seinen Verwandlungs-Übungen im fünften Jahr etwas Erstaunliches gemacht haben… Ich glaube, andere mächtige Zauberer waren in Hogwarts auch gut…"

Harry blickte finster drein und ging immer noch auf und ab.

"Sie könnten einfach nur fleißig lernen. Trotzdem, wenn die Erstklässler die gleichen Zaubersprüche lernten und damals genauso mächtig schienen wie heute, könnten wir das als schwachen Beweis für die Bevorzugung von 1 gegenüber 2 bezeichnen… Moment, warte mal."

Harry blieb stehen, wo er stand.

"Ich habe einen weiteren Test, der zwischen 1 und 2 unterscheiden könnte. Es würde eine Weile dauern, ihn zu erklären, er nutzt einige Dinge, die Wissenschaftler über Blut und Vererbung wissen, aber die Frage ist einfach zu stellen. Und wenn wir meinen Test und deinen Test kombinieren und beide gleich herauskommen, ist das ein starker Hinweis auf die Antwort."

Harry rannte fast zurück zum Schreibtisch, nahm das Pergament und schrieb:

\emph{B. Haben die alten Erstklässler die gleiche Art von Zaubersprüchen mit der gleichen Kraft gesprochen wie heute? (Schwacher Beweis für 1 gegenüber 2, aber Blut könnte auch nur ein Verlust an mächtigen Zaubern sein.)}

\emph{C. Zusätzlicher Test, der 1 und 2 anhand der wissenschaftlichen Erkenntnisse über Blut unterscheidet, wird später erklärt.}

"Okay", sagte Harry, "wir können zumindest versuchen, den Unterschied zwischen 1 und 2 und 3 zu erkennen, also fangen wir gleich damit an, wir können uns weitere Tests überlegen, nachdem wir die, die wir schon haben, gemacht haben. Jetzt wird es ein bisschen seltsam aussehen, wenn Draco Malfoy und Harry Potter zusammen Fragen stellen, also hier ist meine Idee. Du gehst durch Hogwarts und findest alte Porträts und frag sie nach den Zaubersprüchen, die sie in ihren ersten Jahren gelernt haben. Es sind Porträts, also werden sie nicht wissen, dass es etwas Seltsames ist, wenn Draco Malfoy das tut.

Ich werde aktuelle Porträts und lebende Menschen nach Zaubersprüchen fragen, die wir kennen, aber nicht wirken können. Niemand wird etwas Ungewöhnliches bemerken, wenn Harry Potter seltsame Fragen stellt. Und ich werde komplizierte Nachforschungen über vergessene Zaubersprüche anstellen müssen, also möchte ich, dass du derjenige bist, der die Daten sammelt, die ich für meine eigene wissenschaftliche Frage brauche. Es ist eine einfache Frage und du solltest in der Lage sein, die Antwort zu finden, indem du Portraits fragst. Du solltest dir das vielleicht aufschreiben, fertig?"

Draco setzte sich wieder hin und kramte in seiner Büchertasche nach Pergament und Federkiel. Als es auf dem Schreibtisch lag, blickte Draco mit entschlossenem Gesicht auf.

"Leg los."

"Finde Porträts, die ein verheiratetes Squib-Paar kannten - mach nicht so ein Gesicht, Draco, das ist eine wichtige Information. Frag einfach aktuelle Porträts, die Gryffindors sind oder so. Finde Porträts, die ein verheiratetes Squib-Paar gut genug kannten, um die Namen all ihrer Kinder zu kennen. Schreibe den Namen jedes Kindes auf und ob das Kind ein Zauberer, ein Squib oder ein Muggel war. Wenn sie nicht wissen, ob das Kind ein Squib oder ein Muggel war, schreib \emph{"Nicht-Zauberer"} auf. Schreib das für jedes Kind auf, das das Paar hatte, lass keins aus. Wenn das Porträt nur den Namen der zaubernden Kinder kennt, nicht aber die Namen aller Kinder, dann schreib keine Daten von diesem Paar auf. Es ist sehr wichtig, dass du mir nur Daten von jemandem bringst, der alle Kinder eines Squib-Paares kennt, gut genug, um sie alle mit Namen zu kennen.

Versuch, mindestens vierzig Namen insgesamt zu bekommen, wenn du kannst, und wenn dz Zeit für mehr hast, umso besser. Hast du das alles?"

"Wiederhole es", sagte Draco, als er mit dem Schreiben fertig war, und Harry wiederholte es. "Ich habe es", sagte Draco, "aber warum -"

"Es hat mit einem der Geheimnisse des Blutes zu tun, das Wissenschaftler bereits entdeckt haben. Ich erkläre es dir, wenn du zurückkommst. Wir teilen uns auf und treffen uns in einer Stunde wieder hier, 18:22 Uhr müsste das sein. Sind wir bereit zu gehen?"

Draco nickte entschlossen. Es war alles sehr überstürzt, aber er hatte schon lange gelernt, wie man sich beeilt.

"Dann los!", sagte Harry Potter und riss sich seinen Umhang vom Leib, schob ihn in seinen Beutel, der anfing, ihn zu fressen, und ohne auch nur abzuwarten, bis sein Beutel leer war, drehte er sich um und begann, mit schnellen Schritten auf die Tür des Klassenzimmers zuzusteuern, wobei er gegen ein Pult stieß und in seiner Eile fast umfiel. Bis Draco es geschafft hatte, seinen eigenen Umhang abzulegen und ihn in seiner Büchertasche zu verstauen, war Harry Potter schon weg.

Draco rannte fast aus der Tür.

