

\hypertarget{egoismus}{% \section{50. Egoismus}\label{egoismus}}

\textbf{\uline{Egoismus}}

Padma Patil hatte ihr Abendessen etwas verspätet beendet, es ging auf halb acht zu, und schritt nun zügig aus der Großen Halle auf dem Weg zum Ravenclaw-Schlafsaal und den Lernräumen. Tratschen machte Spaß und Grangers Ruf zu zerstören machte noch mehr Spaß, aber es konnte von den Schularbeiten ablenken. Sie hatte einen Aufsatz über Lomillialorholz aufgeschoben, der am nächsten Morgen in der Kräuterkunde-Stunde fällig war, und sie musste ihn heute Abend fertigstellen.

Während sie durch einen langen, gewundenen, engen Steinkorridor ging, kam das Flüstern, das sich anhörte, als käme es direkt hinter ihr.

„\emph{Padma Patil …}“

Sie wirbelte blitzschnell herum, ihren Zauberstab schon aus einer Tasche ihres Umhangs hervorgeholt und in die Hände gesprungen, \emph{wenn Harry Potter glaubte, er könne sich so einfach anschleichen und sie erschrecken}—

da war niemand.

Sofort drehte sich Padma um und schaute in die andere Richtung, ob es ein Bauchrednerzauber gewesen war—

auch dort war niemand.

Das flüsternde Seufzen kam wieder, leise und gefährlich mit einem leicht zischenden Unterton.

„\emph{Padma Patil, Slytherin-Mädchen…}“

„Harry Potter, Slytherin-Junge“, sagte sie laut. Sie hatte schon ein Dutzend Mal gegen Potter und seine Chaos-Legion gekämpft, und sie wusste, dass das hier irgendwie Harry Potter war…

…\emph{obwohl der Bauchredner-Zauber nur auf Sichtweite funktionierte, und in dem verwinkelten Korridor konnte sie problemlos bis zur nächsten Kurve sehen, sowohl vorwärts als auch rückwärts, und dort war niemand.}..

… das war egal. Sie kannte ihren Feind. Da war ein flüsterndes Kichern, das jetzt von neben ihr kam, und sie wirbelte herum, richtete ihren Zauberstab auf das Flüstern und rief: „Luminos!“

Der rote Lichtblitz schoss heraus und traf die Wand, die in einem karmesinroten Schein aufleuchtete, der bald verblasste. Sie hatte nicht wirklich erwartet, dass es funktionieren würde. Harry Potter konnte unmöglich unsichtbar sein, nicht wirklich unsichtbar, das war Magie, die die meisten Erwachsenen nicht beherrschten, und sie hatte neun Zehntel der Geschichten über ihn nie geglaubt.

Die flüsternde Stimme lachte wieder, jetzt auf ihrer anderen Seite.

„\emph{Harry Potter steht am Abgrund}“, flüsterte die Stimme, die jetzt ganz nah an ihrem Ohr klang, „\emph{er schwankt, aber du, du fällst schon, Slytherin-Mädchen ..}.“

„Der Hut hat nie Slytherin nach meinem Namen gerufen, Potter!“

Sie lehnte sich mit dem Rücken an die Wand, um nicht hinter sich schauen zu müssen, und hob ihren Zauberstab in Angriffshaltung.

Wieder das leise Lachen.

„\emph{Harry Potter ist seit einer halben Stunde im Ravenclaw-Gemeinschaftsraum und hilft Kevin Entwhistle und Michael Corner beim Einstudieren von Zaubertränke-Rezepten. Aber das ist nicht wichtig. Ich bin hier, um dir eine Warnung zu überbringen, Padma Patil, und wenn du dich entscheidest, sie zu ignorieren, ist das deine eigene Sache.}“

„Schön“, sagte sie kalt. „Nur zu, warne mich, Potter, ich habe keine Angst vor dir.“

„\emph{Slytherin war einst ein großes Haus}“, flüsterte sie; es klang jetzt trauriger.

„\emph{Slytherin war einst ein Haus, auf dessen Wahl du stolz gewesen wärst, Padma Patil. Aber etwas ist schief gelaufen, etwas ist sauer geworden; weißt du, was im Haus Slytherin schief gelaufen ist, Padma Patil?}“

„Nein, und es ist mir egal!“

„\emph{Aber es sollte dir nicht egal sein}“, sagte das Flüstern, das jetzt so klang, als käme es direkt hinter ihrem Kopf, wo sie fast an die Wand gepresst stand.

„\emph{Denn du bist immer noch das Mädchen, dem der Sprechende Hut die Wahl gelassen hat. Glaubst du, nur weil du dich für Ravenclaw entschieden hast, bist du nicht Pansy Parkinson und wirst es auch nie werden, ganz gleich, wie du dich sonst benimmst?}“

Trotz allem breiteten sich jetzt kleine Schauer der Angst auf ihrer Wirbelsäule aus und liefen über ihre Haut. Sie hatte diese Geschichten auch über Harry Potter gehört, dass er ein heimlicher Legilimens war. Aber sie stand immer noch aufrecht, und sie legte all den Biss in ihre Stimme, den sie konnte, als sie sagte:

„Die Slytherins sind in die Dunkelheit gegangen, um Macht zu bekommen, genau wie du, Potter. Und das werde ich nicht, niemals.“

„\emph{Aber du verbreitest bösartige Gerüchte über ein unschuldiges Mädchen}“, flüsterte die Stimme, „\emph{auch wenn es dir nicht hilft, deine eigenen Ambitionen zu erreichen, und ohne zu bedenken, dass sie mächtige Verbündete hat, die daran Anstoß nehmen könnten. Das ist nicht das stolze Slytherin der alten Tage, Padma Patil, das ist nicht der Stolz von Salazar, das ist das verkommene Slytherin, Padma Parkinson, nicht Padma Malfoy.}..“

Ihr wurde so unheimlich wie noch nie in ihrem Leben, und die Möglichkeit kam ihr langsam in den Sinn, dass es sich wirklich um einen Geist handeln könnte.

Sie hatte noch nie gehört, dass Geister sich so verstecken konnten, aber vielleicht taten sie es einfach nicht - ganz zu schweigen davon, dass die meisten Geister nicht so unheimlich waren, sie waren schließlich nur tote Menschen—

„Wer sind Sie? Der Blutige Baron?“

„\emph{Als Harry Potter tyrannisiert und geschlagen wurde}“, flüsterte die Stimme,

„\emph{befahl er allen seinen Verbündeten, von Rache abzusehen; erinnerst du dich daran, Padma Patil? Denn Harry Potter ist schwankend, aber noch nicht verloren; er kämpft, er weiß, dass er in Gefahr ist. Aber Hermine Granger hat ihre eigenen Verbündeten nicht um so etwas gebeten. Harry Potter ist jetzt wütend auf Sie, Padma Patil, wütender, als er es in seinem eigenen Namen jemals sein würde…und er hat seine eigenen Verbündeten.}“

Ein Schauder durchlief sie, sie wusste, dass es sichtbar war, und sie hasste sich dafür.

„\emph{Oh, hab keine Angst}“, hauchte die Stimme. „I\emph{ch werde dir nicht wehtun. Denn weißt du, Padma Patil, Hermine Granger ist wirklich unschuldig. Sie steht nicht am Abgrund, sie wird nicht fallen. Sie hat ihre Verbündeten nicht gebeten, es zu unterlassen, Ihnen wehzutun, denn der Gedanke ist ihr nicht einmal als Möglichkeit in den Sinn gekommen. Und Harry Potter weiß sehr gut, dass, wenn er dir wehtun würde oder dafür sorgen würde, dass dir wehgetan wird, Hermine Granger zuliebe, dann würde sie nie wieder mit ihm sprechen, bis die Sonne tief brennt und der letzte Stern am Nachthimmel verschwunden ist.}“

Die Stimme war jetzt sehr traurig.

„\emph{Sie ist wirklich ein gütiges Mädchen, ein Mensch, wie ich ihn mir nur wünschen kann..}.“

„Granger kann den Patronus-Zauber nicht wirken!“, sagte Padma.

„Wenn sie wirklich so nett wäre, wie sie vorgibt zu sein—“

„\emph{Kannst du den Patronus-Zauber wirken, Padma Patil? Du hast dich nicht einmal getraut, es zu versuchen, weil du Angst hattest, was das Ergebnis sein würde}.“

„Das ist nicht wahr! Ich hatte keine Zeit, das war alles!“

Das Geflüster ging weiter.

„\emph{Aber Hermine Granger hat es versucht, ganz offen vor ihren Freunden, und als ihr Zauber versagte, war sie überrascht und bestürzt. Denn es gibt Geheimnisse des Patronus-Zaubers, die nur wenige je kannten, und vielleicht kennt sie jetzt keiner außer mir.}“

Ein leises, geflüstertes Kichern.

„\emph{Lasst mich dir sagen, dass es kein Makel ihres Geistes ist, der ihr Licht daran hindert, hervorzukommen. Hermine Granger kann den Patronus-Zauber nicht wirken, aus demselben Grund, aus dem Godric Gryffindor, der diese Hallen erschaffen hat, es nie konnte.}“

Der Korridor wurde kälter, dessen war sie sich sicher, als würde jemand den Kältezauber anwenden.

„\emph{Und Harry Potter ist nicht der einzige Verbündete von Hermine Granger.}“

Jetzt lag ein Unterton von trockener Belustigung in diesem Flüstern, es erinnerte sie plötzlich und erschreckend an Professor Quirrell.

„\emph{Filius Flitwick und Minerva McGonagall sind ziemlich angetan von ihr, glaube ich. Ist dir schon einmal in den Sinn gekommen, dass, wenn die beiden erfahren, was du Hermine Granger antust, sie dich vielleicht weniger gern haben? Sie würden sich vielleicht nicht offen einmischen, aber sie würden vielleicht etwas langsamer sein, dir Hauspunkte zu geben, etwas langsamer, Chancen in deine Richtung zu lenken—} “

„Potter hat mich verpetzt?“

Ein geisterhaftes Glucksen, ein trockenes

„\emph{Hahahaha glaubst du, die beiden sind dumm, taub und blind?}“

In einem traurigeren Flüsterton:

„\emph{Glaubst du, Hermine Granger ist ihnen nicht wertvoll, dass sie nicht sehen, wie sehr sie leidet? So wie sie dich vielleicht einmal gern hatten, ihre kluge junge Padma Patil, aber du wirfst sie weg …}“

Padmas Kehle war trocken. Daran hatte sie nicht gedacht, ganz und gar nicht.

„\emph{Ich frage mich, wie viele Menschen sich am Ende um dich kümmern werden, Padma Patil, auf diesem Weg, den du jetzt beschreitest. Ist es das wert, nur um dich weiter von deiner Schwester zu entfernen? Um der Schatten zu Parvatis Licht zu sein? Deine tiefste Angst war immer, in Harmonie mit ihr zu fallen, zurück in Harmonie mit ihr, sollte ich sagen; aber ist es das wert, ein unschuldiges Mädchen zu verletzen, nur um dich selbst so viel mehr zu unterscheiden? Musst du der böse Zwilling sein, Padma Patil, kannst du nicht etwas anderes Gutes finden, das du verfolgen kannst?}“

Ihr Herz hämmerte in ihrer Brust. Sie hatte, sie hatte noch nie mit jemandem darüber gesprochen—

„\emph{Ich habe mich immer darüber gewundert, warum sich Schüler gegenseitig schikanieren}“, seufzte die Stimme. "\emph{Wie Kinder sich selbst das Leben schwer machen, wie sie ihre Schulen in Gefängnisse verwandeln, sogar mit ihren eigenen Händen. Warum machen sich die Menschen ihr eigenes Leben so unangenehm? Ich kann dir einen Teil der Antwort geben, Padma Patil.

} \emph{Es liegt daran, dass die Menschen nicht innehalten und nachdenken, bevor sie Schmerz verursachen, wenn sie sich nicht vorstellen, dass sie selbst auch verletzt werden könnten, dass sie auch unter ihren eigenen Missetaten leiden könnten.

Aber leiden wirst du, oh ja, Padma Patil, leiden wirst du, wenn du auf diesem Weg bleibst. Du wirst denselben Schmerz der Einsamkeit erleiden, denselben Schmerz der Angst und des Misstrauens anderer, den du jetzt Hermine Granger zufügst. Nur für dich wird es verdient sein.}"

Ihr Zauberstab zitterte in ihrer Hand.

„\emph{Du hast dich nicht für eine Seite entschieden, als du nach Ravenclaw gegangen bist, Mädchen. Du wählst deine Seite durch die Art, wie du dein Leben lebst, was du anderen Menschen antust und was du dir selbst antust. Wirst du das Leben anderer erhellen oder es verdunkeln? Das ist die Wahl zwischen Licht und Dunkelheit, nicht irgendein Wort, das der Sprechende Hut ausruft. Und das Schwierige, Padma Patil, ist nicht, 'Licht' zu sagen, das Schwierige ist, zu entscheiden, was was ist, und es sich selbst einzugestehen, wenn man den falschen Weg einschlägt.}“

Es herrschte Schweigen. Es dauerte eine Weile, und Padma wurde klar, dass sie entlassen worden war. Padma ließ fast ihren Zauberstab fallen, als sie versuchte, ihn wieder in ihre Tasche zu stecken. Sie fiel fast hin, als sie einen Schritt von der Wand weg machte und sich zum Gehen wandte—

„\emph{Ich habe nicht immer richtig zwischen Licht und Dunkelheit gewählt}“, sagte das Flüstern, jetzt laut und hart direkt in ihr Ohr.

"\emph{Nimm meine Weisheit nicht als letztes Wort, Mädchen, fürchte dich nicht, sie in Frage zu stellen, denn obwohl ich es versuchte, habe ich manchmal versagt, oh ja, ich habe versagt. Aber du tust einem wahren Unschuldigen weh, und du wirst nichts von deinen Ambitionen dadurch erreichen, es ist nicht für irgendeinen schlauen Plan.

Du fügst einem Unschuldigen Schmerzen zu, nur um des Vergnügens willen, das es dir bereitet. Ich habe nicht immer richtig zwischen Licht und Dunkelheit gewählt, aber das erkenne ich als Dunkelheit, ganz sicher. Du tust einem unschuldigen Mädchen weh und entgehst der Vergeltung nur, weil sie zu freundlich ist, um zu dulden, dass ihre Verbündeten gegen dich vorgehen. Dafür kann ich dir nicht wehtun, du sollst nur wissen, dass ich es nicht respektieren kann.}

\textbf{\emph{Du bist unwürdig für Slytherin; geh und mach deine Hausaufgaben in Kräuterkunde, Ravenclaw-Mädchen!}}"

Das letzte Flüstern kam in einem lauten Zischen heraus, das fast wie eine Schlange klang, und Padma floh, sie floh die Korridore hinunter, als ob Lethifolds sie jagen würden, sie rannte ohne Rücksicht auf die Regeln über das Laufen in den Korridoren, selbst als sie an anderen Schülern vorbeikam, die sie überrascht ansahen, blieb sie nicht stehen, sie rannte den ganzen Weg zu den Ravenclaw-Schlafsälen mit pochendem Puls im Nacken, die Tür fragte sie „Warum scheint die Sonne am Tag und nicht in der Nacht?“, und sie brauchte drei Versuche, bevor sie ihre Antwort zusammenhängend formulieren konnte, und dann ging die Tür auf, und sie sah - - ein paar Mädchen und Jungen, einige jung und einige alt, die sie alle anstarrten, und in einer Ecke am fünfeckigen Tisch, Harry Potter und Michael Corner und Kevin Entwhistle, die von ihren Lehrbüchern aufschauten.

„Heiliger Merlin!“, rief Penelope Clearwater und erhob sich von einer Couch.

„Was ist mit dir passiert, Padma?“

„Ich“, stotterte sie, „ich, ich hörte - einen Geist—“

„Es war doch nicht der Blutige Baron, oder?“, sagte Clearwater. Sie zog ihren Zauberstab, und einen Moment später hielt sie einen Becher in der Hand, und ein Aguamenti später war der Becher mit Wasser gefüllt.

„Hier, trink das, setz dich hin—“

Padma schritt bereits auf den fünfeckigen Tisch zu. Sie sah Harry Potter an, der sie mit seinem eigenen Blick ansah, ruhig und ernst und ein wenig traurig.

„Du hast das getan!“ sagte Padma. „Wie - du - wie kannst du es wagen!“

Im Ravenclaw-Schlafsaal herrschte plötzlich Stille.

Harry sah sie nur an. Und sagte: „Gibt es irgendetwas, wobei ich dir helfen kann?“

„Leugne es nicht“, sagte Padma, ihre Stimme zitterte, „du hast diesen Geist auf mich gehetzt, er sagte—“

„Ich meine es ernst“, sagte Harry. „Kann ich dir mit irgendetwas helfen? Dir etwas zu essen holen, oder eine Limonade für dich holen, oder dir bei den Hausaufgaben helfen, oder so etwas?“

Alle starrten die beiden an.

„Warum?“ sagte Padma. Ihr fiel nichts mehr ein, was sie sagen konnte, sie verstand es nicht.

„Weil einige von uns am Abgrund stehen“, sagte Harry. „Und der Unterschied ist, was man für andere Menschen tut. Lass mich dir bitte bei etwas helfen, Padma.“

Sie starrte ihn an und wusste in diesem Moment, dass er seine eigene Warnung bekommen hatte, genau wie sie.

„Ich…“, sagte sie. „Ich muss sechs Zentimeter über Lomillialor schreiben—“

„Lass mich in meinen Schlafsaal laufen und meine Kräuterkundesachen holen“, sagte Harry. Er erhob sich von dem fünfeckigen Tisch, sah Entwhistle und Corner an.

„Tut mir leid, Leute, wir sehen uns später.“

Sie sagten nichts, starrten nur, zusammen mit allen anderen im Schlafsaal, als Harry Potter zur Treppe hinüberging. Und gerade als er hochging, sagte er:

„Und niemand soll sie mit Fragen löchern, es sei denn, sie will darüber reden, ich hoffe, jeder hat das verstanden?!“

„Verstanden“, sagten die meisten Erstklässler und einige der älteren Schüler, wobei ein paar von ihnen ziemlich ängstlich klangen.

Und sie sprach mit Harry Potter über viele Dinge, außer über Lomillialorholz - sogar über ihre Angst, wieder mit Parvati in Einklang zu kommen, worüber sie noch nie mit jemandem gesprochen hatte, aber darüber wusste Harrys geisterhafter Verbündeter schon Bescheid.

Und Harry hatte in seine Tasche gegriffen und ein paar seltsame Bücher herausgeholt, die er ihr unter der Bedingung absoluter Geheimhaltung geliehen hatte, weil er meinte, wenn sie diese Bücher verstehen könnte, würde sich ihr Denkmuster so verändern, dass sie nie wieder in Harmonie mit Parvati geraten würde…

Um neun Uhr, als Harry sagte, er müsse gehen, war der Aufsatz erst halb fertig. Und als Harry auf dem Weg nach draußen innehielt und sie ansah und sagte, dass er sie für würdig hielt, Slytherin zu sein, fühlte sie sich eine ganze Minute lang gut, bevor sie realisierte, was gerade zu ihr gesagt worden war und wer es gesagt hatte.

…

Als Padma an diesem Morgen zum Frühstück kam, sah sie, wie Mandy sie sah und dem Mädchen, das neben ihr am Ravenclaw-Tisch saß, etwas zuflüsterte. Sie sah, wie das Mädchen von der Bank aufstand und auf sie zuging. Gestern Abend war Padma froh gewesen, dass das Mädchen im anderen Schlafsaal wohnte; aber jetzt, wo sie darüber nachdachte, war das noch schlimmer, jetzt musste sie es vor allen tun.

Aber obwohl Padma schwitzte, wusste sie, was sie zu tun hatte.

Das Mädchen kam näher - „Es tut mir leid.“

„Was?“, sagte Padma. \emph{Das war ihr Satz.}

„Es tut mir leid“, wiederholte Hermine Granger. Ihre Stimme war so laut, dass es jeder hören konnte. „Ich… ich habe Harry nicht darum gebeten, das zu tun, und ich war wütend auf ihn, als ich es herausfand, und ich ließ ihn versprechen, es nie wieder jemandem anzutun, und ich werde eine Woche lang nicht mit ihm reden… Es tut mir wirklich, wirklich leid, Miss~Patil.“

Hermine Grangers Rücken war steif, ihr Gesicht war steif, man konnte den Schweiß auf ihrem Gesicht sehen.

„Ähm“, sagte Padma. Ihre eigenen Gedanken waren jetzt ziemlich durcheinander… Padmas Blick huschte zum Ravenclaw-Tisch, wo ein Junge sie mit zusammengekniffenen Augen und in seinem Schoß geballten Händen beobachtete.

…

\textbf{Vorher}:

„Ich hab dir gesagt, du sollst netter sein!?“, kreischte Hermine.

Harry begann zu schwitzen. Er hatte Hermine noch nie zuvor wirklich schreien gehört, und es war ziemlich laut in dem leeren Klassenzimmer.

„Ich - aber - aber ich war doch nett!“ protestierte Harry.

„Ich habe sie praktisch erlöst, Padma war auf dem falschen Weg und ich habe sie davon abgebracht! Ich habe wahrscheinlich ihr ganzes Leben verändert, damit sie glücklicher wird! Außerdem hättest du mal die Originalversion dessen hören sollen, was Professor Quirrell mir vorgeschlagen hat—“ , woraufhin Harry merkte, was er sagte und seinen Mund eine Sekunde zu spät schloss.

Hermine klammerte sich an ihre kastanienbraunen Locken, eine Geste, die Harry noch nie von ihr gesehen hatte.

„Was hat er gesagt, was ich tun soll? Sie umbringen?“

Der Verteidigungsprofessor hatte Harry vorgeschlagen, alle gerüchteverbreitenden Schüler innerhalb und außerhalb seines Jahrgangs ausfindig zu machen und zu versuchen, die Kontrolle über die gesamte Gerüchteküche von Hogwarts zu erlangen, mit der Bemerkung, dass dies eine allgemein nützliche und amüsante Herausforderung für jeden echten Slytherin in Hogwarts sei.

„Nichts dergleichen“, sagte Harry schnell, „er sagte nur ganz allgemein, dass ich Einfluss auf die Leute bekommen sollte, die Gerüchte verbreiten, und ich beschloss, dass die nette Version davon wäre, Padma einfach direkt über die Bedeutung dessen, was sie tut, und die möglichen Konsequenzen ihres Handelns zu informieren, anstatt zu versuchen, sie zu bedrohen oder so etwas—“

„Das nennst du, jemanden nicht zu bedrohen?“ Hermines Hände zerrten jetzt an ihren Haaren.

„Ähm…“ sagte Harry. „Ich schätze, sie hat sich vielleicht ein bisschen bedroht gefühlt, aber Hermine, die Leute werden alles tun, von dem sie glauben, dass sie damit durchkommen, es ist ihnen egal, wie sehr es andere Leute verletzt, wenn es sie selbst nicht verletzt, wenn Padma denkt, dass es keine Konsequenzen hat, wenn sie Lügen über dich verbreitet, dann wird sie es natürlich einfach weiter tun—“

„\textbf{Und du denkst, es wird keine Konsequenzen für das haben, was du getan hast?!}“

Harry bekam plötzlich ein flaues Gefühl im Magen.

Hermine hatte den wütendsten Blick aufgesetzt, den er je gesehen hatte.

„Was denkst du, was die anderen Schüler jetzt von dir denken, Harry? Von mir? Wenn es Harry nicht gefällt, wie du über Hermine sprichst, werden Geister auf dich angesetzt, ist es das, was du willst, was sie denken?“

Harry öffnete den Mund, aber es kamen keine Worte heraus, er hatte nur… so hatte er nicht darüber nachgedacht, eigentlich… Hermine griff nach unten, um ihre Bücher vom Tisch zu holen, wo sie sie hingeschmissen hatte.

„Ich werde eine Woche lang nicht mit dir reden, und ich werde jedem sagen, dass ich eine Woche lang nicht mit dir rede, und ich werde ihnen sagen, warum, und vielleicht macht das etwas von dem rückgängig, was du gerade getan hast. Und nach dieser Woche, werde ich - werde ich dann entscheiden, was zu tun ist, schätze ich—“

„Hermine!“ Harrys eigene Stimme erhob sich zu einem Schrei der Verzweiflung.

„Ich habe versucht zu helfen!“

Das Mädchen drehte sich um und sah ihn an, als sie die Klassenzimmertür öffnete.

„Harry“, sagte sie, und ihre Stimme zitterte ein wenig unter der Wut, „Professor Quirrell saugt dich in die Dunkelheit, das tut er wirklich, ich meine es ernst, Harry.“

\emph{Das…war nicht er, das war nicht das, was er gesagt hat, das zu tun, das war nur ich}

Hermines Stimme war jetzt fast ein Flüstern.

„Eines Tages wirst du mit ihm essen gehen, und es wird deine dunkle Seite sein, die zurückkommt, oder vielleicht kommst du auch gar nicht zurück.“

„Ich verspreche dir“, sagte Harry, „dass ich vom Mittagessen zurückkommen werde.“

Er hat nicht einmal nachgedacht, als er das sagte.

Und Hermine drehte sich einfach um, stürmte hinaus und schlug die Tür hinter sich zu.

\emph{So beschwört man die Gesetze der dramatischen Ironie, Schwachkopf}, bemerkte Harrys innerer Kritiker.

\emph{Jetzt wirst du diesen Samstag sterben, deine letzten Worte werden sein 'Es tut mir leid, Hermine', und sie wird immer bedauern, dass das Letzte, was sie getan hat, das} \emph{Zuschlagen der Tür war}—

\textbf{\emph{Oh, halt die Klappe.}}

…

Als Padma sich mit Hermine zum Frühstück hinsetzte und mit einer Stimme, die laut genug war, dass andere sie hören konnten, sagte, dass der Geist ihr Dinge gesagt hatte, die für sie wichtig zu hören waren, und dass Harry Potter Recht gehabt hatte, es zu tun, gab es einige Leute, die danach weniger Angst hatten, und einige, die mehr Angst hatten.

Und danach sagten die Leute weniger böse Dinge über Hermine, zumindest im ersten Jahr, zumindest in der Öffentlichkeit, wo Harry Potter es hören könnte.

Als Professor Flitwick Harry fragte, ob er für das, was mit Padma passiert war, verantwortlich sei, und Harry dies bejahte, sagte Professor Flitwick ihm, dass er zwei Tage nachsitzen müsse. Auch wenn es nur ein Geist gewesen war und Padma nicht verletzt worden war, so war das doch kein akzeptables Verhalten für einen Ravenclaw-Schüler.

Harry nickte und sagte, er verstehe, warum der Professor das tun musste, und würde nicht protestieren; aber wenn man bedenkt, dass es Padma anscheinend umgestimmt hat, dachte Professor Flitwick wirklich, inoffiziell, dass er das Falsche getan hatte?

Und Professor Flitwick hielt inne, schien tatsächlich darüber nachzudenken, und sagte dann zu Harry mit feierlich-quietschiger Stimme, dass er lernen müsse, sich mit anderen Schülern auf normale Weise zu verhalten.

Und Harry konnte sich des Eindrucks nicht erwehren, dass dies ein Rat war, den Professor Quirrell ihm niemals geben würde. Harry konnte sich des Gedankens nicht erwehren, dass, wenn er es auf Professor Quirrells Art gemacht hätte, auf die normale Slytherin-Art, eine Mischung aus positiven und negativen Anreizen, um Padma und die anderen Gerüchteköche unter seine ausdrückliche Kontrolle zu bringen, dann hätte Padma nicht darüber geredet und Hermine hätte es nie herausgefunden…

… in diesem Fall wäre Padma nicht erlöst worden, sie wäre auf dem falschen Weg geblieben und sie selbst hätte irgendwann darunter gelitten. Es war ja nicht so, dass Harry Padma in irgendeiner Weise belogen hätte, als er zeitumgewandelt und unsichtbar war und den Bauchrednerzauber benutzte.

Harry war sich immer noch nicht sicher, ob er das Richtige getan hatte, und Hermine hatte nicht nachgegeben, nicht mit ihm zu reden - obwohl sie viel mit Padma redete.

Es tat mehr weh, als Harry erwartet hatte, wieder allein zu lernen; als hätte sein Gehirn bereits begonnen, seine lang erlernte Fähigkeit, allein zu sein, zu vergessen.

\emph{\hfill\break Die Tage bis zum Mittagessen mit Professor Quirrell am Samstag schienen sehr, sehr langsam zu vergehen.}

