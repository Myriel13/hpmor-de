

\hypertarget{egozentrische-vorurteile}{% \section{29. Egozentrische Vorurteile}\label{egozentrische-vorurteile}}

\textbf{\uline{Egozentrische Vorurteile}}

In letzter Zeit hatte Hermine jedes Mal ein flaues Gefühl im Magen, wenn sie hörte, wie die anderen Schüler über sie und Harry sprachen. Sie war heute Morgen in einer Duschkabine gewesen, als sie ein Gespräch zwischen Morag und Padma belauscht hatte, das der letzte Strohhalm gewesen war, der sich zu einer ganzen Reihe anderer Strohhalme gesellt hatte.

Sie begann zu glauben, dass es ein schrecklicher Fehler gewesen war, sich auf eine Rivalität mit Harry Potter einzulassen.

Wenn sie sich einfach von Harry Potter ferngehalten hätte, hätte sie Hermine Granger sein können, der hellste akademische Stern von Hogwarts, der mehr Punkte für Ravenclaw sammelte als jeder andere. Sie wäre nicht so berühmt gewesen wie der Junge-der-lebte, aber sie wäre \emph{für sich selbst} berühmt gewesen.

Stattdessen hatte der Junge-der-lebte eine akademische Rivalin, und die hieß zufällig Hermine Granger. Und was noch schlimmer war, sie hatte sich mit ihm verabredet. Die Idee, sich auf eine Romanze mit Harry einzulassen, war anfangs verlockend gewesen. Sie hatte solche Bücher gelesen, und wenn es in Hogwarts jemanden gab, der ein Kandidat für das Liebesinteresse der Heldin war, dann war es offensichtlich Harry Potter.

Klug, witzig, berühmt, manchmal unheimlich…

Also hatte sie Harry zu einem Date mit ihr gezwungen. Und jetzt war sie sein Liebesinteresse. Oder schlimmer noch, eine der \emph{Optionen auf seiner Abendkarte.}

Sie war an diesem Morgen in einer Duschkabine und wollte gerade das Wasser aufdrehen, als sie Kichern von draußen hörte.

Und sie hörte, wie Morag darüber sprach, dass dieses Muggelgeborene Mädchen wahrscheinlich nicht hart genug kämpfen würde, um gegen Ginny Weasley zu gewinnen, und Padma darüber spekulierte, dass Harry Potter vielleicht beschließen würde, dass er beide wollte.

Es war, als würden sie nicht verstehen, dass \emph{MÄDCHEN Optionen auf ihrer Speisekarte hatten und JUNGE um sie kämpften.}

Aber das war nicht einmal der Teil, der wirklich weh tat. Es war, dass, als sie bei einem von Professor McGonagalls Tests 98 Punkte erreichte, die Nachricht nicht lautete, dass Hermine Granger die höchste Punktzahl in der Klasse erreicht hatte, sondern dass Harry Potters Rivale sieben Punkte mehr erreicht hatte als er.

Wenn man dem Jungen-der-lebte zu nahe kam, wurde man Teil seiner Geschichte.

\emph{Man bekam nicht seine eigene.}

Und Hermine war der Gedanke gekommen, dass sie einfach weggehen sollte, aber das wäre zu traurig gewesen. Aber sie wollte zurückbekommen, was sie versehentlich weggegeben hatte, indem sie sich als Harrys Rivalin hatte bekannt werden lassen.

Sie wollte wieder eine eigenständige Person sein und nicht Harry Potters drittes Bein, war das zu viel verlangt? Es war schwer, aus der Falle zu klettern, wenn man einmal hineingefallen war. Egal, wie gut man in der Klasse abschnitt, selbst wenn man etwas tat, das eine besondere Ankündigung beim Abendessen verdiente, bedeutete es nur, dass man wieder mit Harry Potter rivalisierte.

Aber sie dachte, sie würde einen Weg finden. Etwas zu tun, das nicht so gesehen würde, als würde man sich am anderen Ende von Harry Potters Wippe hochdrängen.

Es würde schwer sein. Es würde gegen ihre Natur gehen. Sie müsste jemanden bekämpfen, der sehr böse ist. Und sie würde jemanden, der noch böser war, um Hilfe bitten müssen.

Hermine hob ihre Hand, um an diese schreckliche Tür zu klopfen.

Sie zögerte. Hermine merkte, dass sie albern war, und hob die Hand ein bisschen höher. Sie versuchte erneut zu klopfen. Ihre Hand schaffte es nicht, die Tür zu berühren.

Und dann schwang die Tür trotzdem auf.

"Du liebe Zeit", sagte die Spinne, die in ihrem Netz saß. "War es wirklich so schwer, einen einzigen Quirrell-Punkt zu verlieren, Miss Granger?"

Hermine stand mit erhobener Hand da, ihre Wangen wurden rosa.

\emph{Das war es gewesen.}

"Nun, Miss Granger, ich werde gnädig sein", sagte der böse Professor Quirrell.

"Betrachten Sie sie als bereits verloren. So, ich habe Ihnen eine schwere Entscheidung abgenommen. Sind Sie mir nicht dankbar?"

"Professor Quirrell", schaffte es Hermine mit einer Stimme zu sagen, die ein wenig quietschte. "Ich habe eine Menge Quirrell-Punkte, nicht wahr?"

"Das haben Sie in der Tat", sagte Professor Quirrell.

"Allerdings einen weniger als vorher. Schrecklich, nicht wahr? Denken Sie nur, wenn mir Ihr Grund, hierher zu kommen, nicht gefällt, könnten Sie weitere fünfzig verlieren. Vielleicht würde ich sie dir wegnehmen, einen nach dem anderen … einen nach dem anderen …"

Hermine's Wangen wurden noch röter.

"Sie sind wirklich böse, hat Ihnen das schon mal jemand gesagt?"

"Miss Granger", sagte Professor Quirrell ernst, "es kann gefährlich sein, Leuten solche Komplimente zu machen, wenn sie nicht wirklich verdient sind.

Der Empfänger könnte sich schamhaft und unverdient fühlen und etwas tun wollen, das Ihres Lobes würdig ist. Also, worüber wollten Sie mit mir sprechen, Miss Granger?"

Es war nach dem Mittagessen am Donnerstagnachmittag, und Hermine und Harry hatten es sich in einer kleinen Bibliotheksecke gemütlich gemacht, wo ein Quietus-Feld aufgestellt war, damit sie sich unterhalten konnten.

Harry lag mit dem Bauch nach unten auf dem Boden, die Ellbogen auf den Boden gestützt, den Kopf in den Händen und die Füße lässig hinter sich aufgestützt.

Hermine saß auf einem ausgestopften Stuhl, der viel zu groß für sie war, so als wäre sie die Hermine in der Mitte einer Bonbonschale.

Harry hatte vorgeschlagen, dass sie in einem ersten Durchgang nur die Titel aller Bücher in der Bibliothek lesen könnten, und dann könnte Hermine alle Inhaltsverzeichnisse lesen.

Hermine hatte das für eine brillante Idee gehalten. So etwas hatte sie noch nie mit einer Bibliothek gemacht.

Leider gab es einen kleinen Schönheitsfehler in diesem Plan. Nämlich, dass sie beide Ravenclaws waren. Hermine las gerade ein Buch mit dem Titel M\emph{agische Mnemotechnik.} Harry las ein Buch namens \emph{"Der skeptische Zauberer"}

.

Jeder von ihnen hatte gedacht, dass sie nur dieses eine Mal eine Ausnahme machen würden, und keiner von ihnen hatte bisher begriffen, dass es unmöglich war, dass einer von ihnen jemals alle Buchtitel zu Ende lesen würde, egal wie sehr sie sich bemühten.

Die Stille in ihrer kleinen Ecke wurde von zwei Worten durchbrochen.

"Oh nein", sagte Harry plötzlich laut und es klang, als wären die Worte aus ihm herausgerissen worden.

Es wurde noch ein bisschen stiller.

"Hat er nicht", sagte Harry mit derselben Stimme. Dann hörte sie, wie Harry hilflos zu kichern begann.

Hermine blickte von ihrem Buch auf.

"Also gut", sagte sie, "was gibt es?"

"Ich habe gerade herausgefunden, warum man die Weasleys nie nach der Familienratte fragt", sagte Harry. "Es ist wirklich schrecklich und ich sollte nicht lachen und ich bin ein schrecklicher Mensch."

"Ja", sagte Hermine hochnäsig, "das bist du. Erzähl es mir auch."

"Okay, zuerst der Hintergrund. Es gibt ein ganzes Kapitel in diesem Buch über die Verschwörungstheorien von Sirius Black. Du weißt doch noch, wer das ist, oder?"

"Natürlich", sagte Hermine.

Sirius Black war ein Verräter, ein Freund von James Potter, der Voldemort in das geschützte Haus der Potters gelassen hatte.

"Es stellte sich heraus, dass es eine Reihe von, sagen wir mal, Unregelmäßigkeiten gab, die damit zusammenhingen, dass Black nach Askaban kam.

Er bekam keinen Prozess, und der Junior-Minister, der das Sagen hatte, als die Auroren Black verhafteten, war kein anderer als Cornelius Fudge, der unser jetziger Zaubereiminister wurde."

Das klang auch für Hermine ein wenig verdächtig, und das sagte sie auch.

Harry machte eine achselzuckende Bewegung mit den Schultern, während er auf dem Boden lag und auf sein Buch schaute.

"Verdächtige Dinge passieren die ganze Zeit, und wenn man ein Verschwörungstheoretiker ist, findet man immer etwas."

"Aber kein Gerichtsverfahren?", sagte Hermine.

"Es war gleich nach der Niederlage des Dunklen Lords", sagte Harry, seine Stimme war ernst, als er es sagte.

"Die Dinge waren unglaublich chaotisch, und als die Auroren Black aufspürten, stand er lachend auf der Straße, knöcheltief im Blut, mit zwanzig Augenzeugen, die berichteten, wie er einen Freund meines Vaters namens Peter Pettigrew und zwölf Umstehende getötet hatte.

Ich will nicht sagen, dass ich es gutheiße, dass Black keinen Prozess bekommt. Aber wir reden hier über Zauberer, also ist es nicht wirklich verdächtiger als, ich weiß nicht, die Art von Dingen, auf die Leute zeigen, wenn sie darüber streiten wollen, wer John F. Kennedy erschossen hat. Wie auch immer, Sirius Black ist der zaubernde Lee Harvey Oswald. Es gibt alle möglichen Verschwörungstheorien darüber, wer statt ihm meine Eltern wirklich verraten hat, und einer der Favoriten ist Peter Pettigrew, und hier fängt es an, kompliziert zu werden."

Hermine hörte fasziniert zu.

"Aber wie kommst du von dort auf die Hausratte der Weasleys -"

"Warte mal", sagte Harry, "ich komme da schon hin.

Nun, nach Pettigrews Tod kam heraus, dass er ein Spion für das Licht gewesen war - kein Doppelagent, nur jemand, der herumschlich und Dinge herausfand.

Darin war er gut, seit er ein Teenager war, sogar in Hogwarts hatte er den

Ruf, alle möglichen Geheimnisse herauszufinden.

Die Verschwörungstheorie ist also, dass Pettigrew ein unregistrierter Animagus wurde, während er noch in Hogwarts war, ein Animagus von etwas Kleinem, das herumwuseln und Gespräche belauschen konnte.

Das Hauptproblem ist, dass erfolgreiche Animagi selten sind und es als Teenager wirklich unwahrscheinlich wäre, also besagt die Verschwörungstheorie natürlich, dass mein Vater und Black auch unregistrierte Animagi waren.

Und in dieser Verschwörungstheorie tötete Pettigrew selbst die zwölf Umstehenden, verwandelte sich in seine kleine Animagus-Form und lief davon.

Michael Shermer sagt also, dass es vier weitere Probleme damit gibt.

Erstens: Black war neben meinen Eltern der Einzige, der wusste, wie man durch die Schutzzauber um ihr Haus kommt."

(Harrys Stimme war ein wenig hart, als er das sagte.)

"Zweitens, Black war von vornherein ein wahrscheinlicherer Verdächtiger als Pettigrew, es gibt das Gerücht, dass Black während seiner Zeit in Hogwarts absichtlich versucht hat, einen Schüler umbringen zu lassen, und er stammte aus dieser wirklich fiesen Reinblüterfamilie, Bellatrix Black war buchstäblich seine Cousine.

Drittens war Black zwanzigmal so kampfstark wie Pettigrew, auch wenn er nicht so schlau war. Das Duell zwischen ihnen wäre wie Professor Quirrell gegen Professor Sprout gewesen. Pettigrew hätte wahrscheinlich nicht einmal die Chance gehabt, seinen Zauberstab zu ziehen, geschweige denn alle Beweise vorzutäuschen, die die Verschwörungstheorie erfordert.

Und viertens: Black stand auf der Straße und lachte."

"Aber die Ratte -", sagte Hermine.

"Richtig", sagte Harry.

"Nun, um es kurz zu machen, Bill Weasley entschied, dass die Hausratte seines kleinen Bruders Percy Pettigrews Animagusform war -"

Hermine fiel die Kinnlade herunter.

"Ja", sagte Harry, "man würde nicht gerade erwarten, dass der böse Pettigrew ein trauriges und heimliches Leben als Hausratte einer feindlichen Zaubererfamilie führt, er wäre entweder bei den Malfoys oder, was wahrscheinlicher ist, nach einer kleinen Schönheitsoperation in der Karibik. Jedenfalls schlägt Bill seinen kleinen Bruder Percy k.o., betäubt und schnappt sich die Ratte, sendet all diese Eulennotrufe aus -"

"Oh, nein!" sagte Hermine, die Worte wurden aus ihr herausgerissen.

"- und irgendwie schafft er es, Dumbledore, den Zaubereiminister und den Chef-Auror zu versammeln -"

"Hat er nicht!", sagte Hermine.

"Und als sie dort ankommen, denken sie natürlich, dass er verrückt ist, aber sie wenden Veritas Oculum trotzdem auf die Ratte an, nur um sicherzugehen, und was entdecken sie dann?"

Sie wäre gestorben.

"Eine Ratte."

"Du gewinnst einen Keks! Also schleppten sie den armen Bill Weasley ins St. Mungo's und es stellte sich heraus, dass es ein ganz normaler schizophrener Ausbruch war, wie er bei manchen Leuten vorkommt, besonders bei jungen Männern im College-Alter. Der Typ war davon überzeugt, dass er siebenundneunzig Jahre alt war, gestorben war und über den Bahnhof in sein jüngeres Ich zurückgereist war.

Und er hat perfekt auf die Antipsychotika angesprochen und ist wieder normal und alles ist jetzt in Ordnung, außer dass die Leute nicht mehr so viel über Sirius-Black-Verschwörungstheorien reden und man die Weasleys nie nach der Familienratte fragt."

Hermine kicherte hilflos. Es war wirklich furchtbar und sie sollte nicht lachen und sie war eine schreckliche Person.

"Was ich nicht verstehe", sagte Harry, nachdem sich ihr Kichern gelegt hatte, "ist, warum Black Pettigrew jagen würde, anstatt so schnell wie möglich wegzulaufen.

Er musste doch wissen, dass die Auroren hinter ihm her sein würden. Ich frage mich, ob sie den Grund aus Black herausbekommen haben, bevor sie ihn nach Askaban gebracht haben?

Siehst du, das ist der Grund, warum Leute, die absolut eindeutig schuldig sind, immer noch durch das Rechtssystem gehen und einen Prozess bekommen."

Hermine musste dem zustimmen.

Bald war Harry mit seinem Buch fertig, während Hermine erst halb durch ihres durch war - \emph{ihres war ein viel schwierigeres Buch als Harrys}, aber das war ihr trotzdem peinlich.

Und dann musste sie Ihr Buch wieder ins Regal stellen und sich wegschleppen, denn es war Zeit für sie, sich der gefürchtetsten Klasse in Hogwarts zu stellen, dem BESEN REITEN.

Harry begleitete sie auf dem Weg dorthin, obwohl sein eigener Unterricht erst anderthalb Stunden später stattfand, wie ein Kampfjet, der ein trauriges kleines Propellerflugzeug auf dem Weg zu seiner eigenen Beerdigung eskortiert.

Der Junge wünschte ihr mit leiser, mitfühlender Stimme Lebewohl, und sie ging auf die grasbewachsenen Felder des Schicksals.

Und es gab viel Geschrei und Beinahe-Fälle und schreckliche Berührungen mit dem Tod und den Boden an der völlig falschen Stelle und die Sonne, die ihr in die Augen fiel und Morag, die sie anbrummte und Mandy, die dachte, sie sei subtil, weil sie immer nahe genug war, um sie aufzufangen, wenn sie fiel, und sie wusste, dass die anderen Schüler über sie beide lachten, aber sie sagte nie etwas zu Mandy, weil sie wirklich nicht sterben wollte. Nach zehn Millionen Jahren war die Klasse zu Ende, und sie war wieder auf dem Boden, wo sie bis zum nächsten Donnerstag hingehörte.

Manchmal hatte sie Alpträume, weil es immer Donnerstag war. Warum alle das lernen mussten, wenn sie, sobald sie erwachsen waren, einfach überall hin apparieren oder mit dem Floopulver oder dem Portschlüssel reisen würden, war Hermine ein völliges Rätsel. Niemand hatte es nötig, als Erwachsener auf Besen zu reiten, es war, als würde man gezwungen, im Sportunterricht Völkerball zu spielen.

Ein paar Stunden später saß sie mit Hannah, Susan, Leanne und Megan in einem Hufflepuff-Lernsaal. Professor Flitwick, der für einen Lehrer erstaunlich zurückhaltend war, hatte gefragt, ob sie den vier vielleicht eine Weile bei ihren Zauberkunst-Hausaufgaben helfen könnte, obwohl sie keine Ravenclaws waren, und Hermine hatte sich so stolz gefühlt, dass sie fast geplatzt wäre.

Hermine nahm ein Stück Pergament, verschüttete ein wenig Tinte darauf, riss es in vier Stücke, zerknüllte sie und warf die Stücke auf den Tisch.

Sie hätte es auch einfach zerknüllen können, aber dadurch sah es eher wie Müll aus, und das half, wenn jemand zum ersten Mal den Entsorgungszauber übte.

Hermine spitzte ihre Ohren und Augen und sagte:

"Ok, versucht es."

"Everto."

"Everto."

"Everto."

"Everto."

Hermine glaubte nicht, dass sie alle Probleme verstanden hatte.

"Könnt ihr es alle noch mal versuchen?"

Eine Stunde später war Hermine zu dem Schluss gekommen, dass

(1) Leanne und Megan irgendwie schlampig waren, aber wenn man sie bat, etwas weiter zu üben, würden sie es tun,

(2) Hannah und Susan waren so konzentriert und angetrieben, dass man ihnen immer wieder sagen musste, sie sollten langsamer machen und sich entspannen und über Dinge nachdenken, anstatt sich so sehr anzustrengen - und

(3) sie mochte es, Hufflepuffs zu helfen, der ganze Lernsaal hatte eine sehr fröhliche Atmosphäre.

Als sie zum Abendessen ging, fand sie den Jungen-der-lebte beim Lesen eines Buches, während er darauf wartete, sie zu begleiten.

Sie fühlte sich geschmeichelt, aber auch ein wenig beunruhigt, weil Harry nicht wirklich mit jemandem außer ihr zu reden schien.

"Wusstest du, dass es in Hufflepuff ein Mädchen gibt, das ein Metamorphmagus ist?", sagte Hermine, als sie in Richtung Große Halle gingen.

"Sie färbt sich die Haare richtig rot, so wie Stoppschild-Rot, nicht Weasley-Rot, und als sie sich mit heißem Tee bekleckert hat, hat sie sich in einen schwarzhaarigen Jungen verwandelt, bis sie es wieder unter Kontrolle hatte."

"Echt? Cool", sagte Harry und klang ein wenig abgelenkt. "Ähm, Hermine, nur zur Kontrolle, du weißt, dass morgen der letzte Tag ist, an dem man sich für Professor Quirrells Armeen anmelden kann, oder?"

"Ja", sagte Hermine. "Die Armeen des bösen Professor Quirrell."

Ihre Stimme war ein wenig wütend, obwohl Harry natürlich nicht wusste, warum.

"Hermine", sagte Harry mit verzweifelter Stimme, "er ist nicht böse. Er ist ein bisschen dunkel und eine ganze Menge Slytherin. Das ist nicht dasselbe wie böse sein."

Harry Potter hatte zu viele Worte für Dinge, das war sein Problem.

Er wäre besser dran gewesen, wenn er das Universum einfach in Gut und Böse unterteilt hätte.

"Professor Quirrell hat mich vor der ganzen Klasse aufgerufen und mir gesagt, ich solle jemanden erschießen!"

"Er hatte recht", sagte Harry mit nüchternem Gesicht. "Es tut mir leid, Hermine, aber das hatte er. Du hättest mich erschießen sollen, es hätte mir nichts ausgemacht. Du kannst keine Kampfmagie lernen, wenn du nicht gegen echte Gegner mit echten Zaubern üben kannst. Und jetzt machst du dich gut im Sparring, nicht wahr?"

Hermine war erst zwölf und wusste es, aber sie konnte es nicht in Worte fassen, sie fand nicht das, was sie sagen konnte, das Harry überzeugen würde.

Professor Quirrell hatte sich ein junges Mädchen geschnappt und dieses Mädchen vor allen Leuten aufgerufen und ihr befohlen, ohne Provokation das Feuer auf einen Mitschüler zu eröffnen.

Es spielte keine Rolle, ob Professor Quirrell Recht hatte, dass sie es lernen musste. Professor McGonagall hätte das nie getan.

Professor Flitwick hätte das niemals getan. Vielleicht hätte nicht mal Professor Snape das getan. Professor Quirrell war BÖSE.

Aber sie konnte die Worte nicht finden und sie wusste, dass Harry ihr niemals glauben würde.

"Hermine, ich habe mit älteren Schülern gesprochen", sagte Harry.

"Professor Quirrell könnte der einzige kompetente Verteidigungsprofessor sein, den wir in den ganzen sieben Jahren in Hogwarts bekommen. Alles andere können wir später lernen. Wenn wir Verteidigung studieren wollen, müssen wir es dieses Jahr tun.

Die Schüler, die sich für die außerschulischen Kurse anmelden, werden sehr viel lernen, weit über das hinaus, was das Ministerium den Erstklässlern vorschreibt - wusstest du, dass wir den Patronus-Zauber lernen werden? Im Januar?"

"Der Patronus-Zauber?" sagte Hermine und ihre Stimme erhob sich vor Überraschung. In ihren Büchern stand, dass dies einer der hellsten bekannten Zauber sei, eine Waffe gegen die dunkelsten Kreaturen, die mit rein positiven Emotionen gewirkt wurde. Es war nicht etwas, von dem sie erwarten würde, dass der böse Professor Quirrell es lehrte - oder dafür sorgte, dass es ihm beigebracht wurde, da Hermine sich nicht vorstellen konnte, dass er den Zauber selbst ausführen konnte.

"Ja", sagte Harry. "Schüler lernen den Patronus-Zauber normalerweise erst im fünften Jahr oder sogar noch später! Aber Professor Quirrell sagt, die Stundenpläne des Ministeriums seien von sprechenden Flobberwürmern erfunden worden, und die Fähigkeit, den Patronus-Zauber zu wirken, hänge mehr von Gefühlen als von magischer Kraft ab. Professor Quirrell sagt, dass er glaubt, dass die meisten Schüler viel weniger tun, als sie können, und dieses Jahr wird er es beweisen."

Da war der übliche Ton ehrfürchtiger Verehrung, den Harrys Stimme hatte, wenn er über Professor Quirrell sprach, und Hermine biss die Zähne zusammen und ging weiter.

"Ich habe mich eigentlich schon angemeldet", sagte Hermine, ihre Stimme ein wenig leise. "Ich habe es heute Morgen getan. Für alles, genau wie du gesagt hast."

\emph{In for a penny, in for a pound} war der übliche Ausdruck. Außerdem wollte sie nicht verlieren, und wenn sie gewinnen wollte, musste sie lernen.

"Du wirst also in der Armee sein?" Harrys Stimme war plötzlich enthusiastisch.

"Das ist großartig, Hermine! Ich habe meine Liste mit Soldaten schon bekommen, aber ich bin sicher, Professor Quirrell wird mir erlauben, noch einen hinzuzufügen, oder zu tauschen -"

"Ich trete deiner Armee nicht bei." Hermine's Stimme war scharf.

Sie wusste, dass es eine vernünftige Vermutung war, aber es ärgerte sie trotzdem.

Harry blinzelte. "Aber doch nicht in die von Draco Malfoy. Du willst also in der dritten Armee sein? Auch wenn wir noch nicht wissen, wer der General ist?"

Harry klang überrascht und ein wenig verletzt, und sie konnte es ihm nicht verübeln, obwohl sie ihm natürlich die Schuld gab, denn im Grunde war es alles seine Schuld.

"Aber warum nicht meine?"

"Denk darüber nach", schnauzte Hermine, "und vielleicht kommst du dann darauf!"

Und sie beschleunigte ihren Schritt und ließ Harry gaffend hinter sich.

"Professor Quirrell", sagte Draco in seiner förmlichsten Stimme,

"ich muss gegen Ihre Ernennung von Hermine Granger zum dritten General protestieren."

"Oh?", sagte Professor Quirrell und lehnte sich lässig und entspannt in seinem Stuhl zurück. "Protestieren Sie ruhig, Mr. Malfoy."

"Granger ist für diese Position ungeeignet", sagte Draco.

Professor Quirrell tippte nachdenklich mit einem Finger auf seine Wange.

"Aber ja, das ist sie. Haben Sie weitere Einwände?"

"Professor Quirrell", sagte Harry Potter neben ihm, "bei allem Respekt vor Miss Grangers vielen herausragenden akademischen Talenten und den Quirrell-Punkten, die sie sich in Ihren Kursen zu Recht verdient hat, ist ihre Persönlichkeit für ein militärisches Kommando nicht geeignet."

Draco war erleichtert gewesen, als Harry zugestimmt hatte, ihn zu Professor Quirrells Büro zu begleiten. Es war nicht nur so, dass Harry ein gigantischer, unverhohlener Lehrerschmeichler war, wenn es um Professor Quirrell ging.

Draco hatte auch angefangen, sich Sorgen zu machen, dass Harry tatsächlich mit Granger befreundet war, es war nun schon eine Weile her und er hatte immer noch nicht den ersten Schritt gemacht… aber das hier war schon eher der Fall.

"Ich stimme Mr. Potter zu", sagte Draco. "Sie zum General zu ernennen, macht das Ganze zu einer Farce."

"Hart ausgedrückt", sagte Harry, "aber ich kann mich nicht dazu durchringen, Mr.

Malfoy zu widersprechen. Um es ganz offen zu sagen, Professor Quirrell, Hermine Granger hat ungefähr so viel Tötungsabsicht wie eine Schüssel mit nassen Weintrauben."

"Das", sagte Professor Quirrell milde, "ist nichts, was mir selbst entgehen würde.

Du erzählst mir nichts, was ich nicht schon weiß."

Jetzt war Draco an der Reihe, etwas zu sagen, aber das Gespräch war plötzlich ins Stocken geraten. Diese Antwort hatte nicht zu den Möglichkeiten gehört, die er und Harry durchdacht hatten, bevor sie hierher kamen.

Was sollten Sie sagen, nachdem der Lehrer sagte, dass er alles wusste, was Sie wussten, und er trotzdem einen offensichtlichen Fehler begehen würde?

Das Schweigen dehnte sich aus.

"Ist das eine Art Verschwörung?" sagte Harry langsam.

"Muss alles, was ich tue, eine Art Verschwörung sein?", sagte Professor Quirrell.

"Kann ich nicht einmal Chaos nur um des Chaos willen verursachen?"

Draco verschluckte sich fast.

"Nicht in Ihrem Kampfmagiekurs", sagte Harry mit ruhiger Bestimmtheit.

"An anderen Orten vielleicht, aber nicht dort."

Professor Quirrell hob langsam die Augenbrauen.

Harry starrte ihn unverwandt an. Draco erschauderte.

"Nun denn", sagte Professor Quirrell. "Keiner von Ihnen scheint eine sehr einfache Frage in Betracht gezogen zu haben. Wen könnte ich anstelle von Miss Granger einsetzen?"

"Blaise Zabini", sagte Draco ohne zu zögern.

"Irgendwelche anderen Vorschläge?", fragte Professor Quirrell und klang dabei ziemlich amüsiert.

\emph{Anthony Goldstein und Ernie Macmillan,} kam der Gedanke, bevor sich Dracos gesunder Menschenverstand meldete und Schlammblüter und Hufflepuffs ausschloss, egal wie aggressiv sie sich duellierten.

Also sagte Draco stattdessen nur:

"Was ist falsch mit Zabini?"

"Ich verstehe…" sagte Harry langsam.

"Ich nicht", sagte Draco. "Warum nicht Zabini?"

Professor Quirrell sah Draco an. "Weil, Mr. Malfoy, egal wie sehr er sich anstrengt, er wird nie mit Ihnen oder Mr. Potter mithalten können."

Der Schock darüber versetzte Draco einen Stich. "Sie können doch nicht glauben, dass Granger -"

"Er wettet auf sie", sagte Harry leise. "Es ist nicht garantiert. Die Chancen stehen nicht einmal gut. Sie wird uns wahrscheinlich nie einen guten Kampf liefern, und selbst wenn sie es tut, könnte sie Monate brauchen, um zu lernen.

Aber sie ist die Einzige in unserem Jahrgang, die überhaupt eine Chance hat, zu wachsen und uns zu schlagen."

Dracos Hände zuckten, ballten sich aber nicht zu Fäusten.

Sich als Unterstützer zu zeigen und dann einen Rückzieher zu machen, war eine klassische Unterminierungstaktik, also war Harry Potter mit Granger im Bunde und das bedeutete -

"Aber Professor", fuhr Harry sanft fort, "ich mache mir Sorgen, dass Hermine als General einer Armee unglücklich sein wird. Ich spreche jetzt als ihr Freund, Professor Quirrell. Der Wettbewerb mag für Draco und mich gut sein, aber das, was Sie von ihr verlangen, ist nicht gut für sie!"

\emph{Also doch nicht.}

"Ihre Freundschaft zu Hermine Granger macht Ihnen alle Ehre", sagte Professor Quirrell trocken. "Vor allem, weil du gleichzeitig mit Draco Malfoy befreundet sein kannst. Eine ziemliche gute Leistung."

Harry sah plötzlich ein wenig nervös aus, was bedeutete, dass er sich wahrscheinlich noch viel nervöser fühlte, und Draco fluchte leise vor sich hin.

Natürlich würde Harry Professor Quirrell nicht täuschen.

"Und ich bezweifle, dass Miss Granger Ihre freundliche Besorgnis zu schätzen weiß", sagte Professor Quirrell. "Sie hat mich um die Stelle gebeten, Mr. Potter, ich habe sie nicht gebeten."

Harry schwieg daraufhin einen Moment lang.

Dann warf er Draco einen kurzen Blick zu, der eine Mischung aus Entschuldigung und Warnung darstellte und gleichzeitig sagte:

\emph{"Tut mir leid, ich habe mein Bestes getan und wir sollten die Sache besser nicht weiter forcieren. "}

Was ihr Unglück betrifft", fuhr Professor Quirrell fort, wobei ein leichtes Lächeln seine Lippen umspielte, "so vermute ich, dass sie mit den Härten ihrer Position viel leichter zurechtkommen wird, als Sie beide vermuten, und dass sie sich viel schneller wehren wird, als Sie denken."

Harry und Draco zuckten beide entsetzt zusammen.

"Sie wollen ihr doch nicht etwa einen Rat geben?", fragte Draco völlig entgeistert.

"Ich habe mich nie angemeldet, um gegen Sie zu kämpfen!", sagte Harry.

Das Lächeln, das um Professor Quirrells Lippen spielte, wurde breiter.

"In der Tat habe ich angeboten, ein paar Vorschläge für Miss Grangers erste Kämpfe zu machen."

"Professor Quirrell!", sagte Harry.

"Oh, keine Sorge", sagte Professor Quirrell. "Sie hat mich abgewiesen. Genau wie ich erwartet habe."

Dracos Augen verengten sich.

"Meine Güte, Mr. Potter", sagte Professor Quirrell, "hat Ihnen denn nie jemand gesagt, dass es unhöflich ist, zu starren?"

"Sie werden ihr doch nicht heimlich auf andere Weise helfen, oder?", fragte Harry.

"Würde ich das tun?", fragte Professor Quirrell.

"\textbf{Ja}", sagten Draco und Harry zur gleichen Zeit.

"Euer mangelndes Vertrauen hat mich tief verletzt. Nun denn, ich verspreche, dass ich General Granger in keiner Weise helfen werde, von der ihr beide nichts wisst.

Und nun schlage ich vor, dass Sie beide sich um Ihre militärischen Angelegenheiten kümmern. Der November naht, und zwar schnell."

Draco erkannte die Implikationen, bevor sich die Tür auf dem Weg aus Professor Quirrells Büro ganz hinter ihnen geschlossen hatte. Harry hatte einmal abschätzig von "\emph{Menschenkram}" gesprochen. Und jetzt war das Dracos einzige Hoffnung.

\emph{Bitte lass es ihn nicht merken,}

\emph{Bitte lass es ihn nicht merken…}

"Wir sollten einfach zuerst die Granger angreifen und sie aus dem Weg schaffen", sagte Draco. "Nachdem wir sie zerquetscht haben, können wir unseren eigenen Wettkampf ohne Ablenkung austragen."

"Das scheint ihr gegenüber nicht wirklich fair zu sein, oder?", sagte Harry mit milder Stimme.

"Was kümmert dich das?", sagte Draco. "Sie ist deine Rivalin, richtig?"

Dann, mit genau der richtigen Note von Misstrauen in seiner Stimme:

"Sag mir nicht, dass du angefangen hast, sie wirklich zu mögen, nachdem du die ganze Zeit ihr Rivale warst…"

"Oh Merlin nein!", sagte Harry.

"Was soll ich sagen, Draco? Ich habe nur einen natürlichen Sinn für Gerechtigkeit.

Den hat Granger auch, weißt du. Sie hat ein sehr sicheres Gespür für Gut und Böse, und sie wird wahrscheinlich das Böse zuerst angreifen.

Einen Namen wie '\emph{Malfoy}' zu haben, schreit geradezu danach, weißt du."

\textbf{\emph{VERDAMMT}}!

"Harry", sagte Draco und klang dabei verletzt und vielleicht ein wenig überlegen,

"willst du nicht fair gegen mich kämpfen?"

"Du meinst, anstatt dich anzugreifen, nachdem du bereits einen Teil deiner Kräfte gegen Granger verloren hast?", fragte Harry. "Oh, ich weiß nicht. Vielleicht probiere ich das 'fair' aus, wenn es mir zu langweilig wird, nur zu gewinnen."

"Vielleicht wird sie dich angreifen", sagte Draco. "Du bist ihr Rivale."

"Aber ich bin ihr freundlicher Rivale", sagte Harry mit einem bösen Grinsen.

"Ich habe ihr ein schönes Geburtstagsgeschenk gekauft und alles. Du würdest deinen freundlichen Rivalen nicht so sabotieren."

"Was ist damit, die Chance deines Freundes auf einen fairen Kampf zu sabotieren?", sagte Draco wütend. "Ich dachte, wir wären Freunde!"

"Lass mich das anders ausdrücken", sagte Harry. "Granger würde einen befreundeten Rivalen nicht sabotieren. Aber das liegt daran, dass sie die Tötungsabsicht einer Schale nasser Weintrauben hat. Du würdest es tun.

Du würdest es auf jeden Fall tun. Und ratet mal, ich würde es auch."

\textbf{\emph{VERDAMMT}}!

Wenn es ein Theaterstück gewesen wäre, hätte es dramatische Musik gegeben.

Der Held, tadellos herausgeputzt in grün gesäumten Gewändern und perfekt gekämmten weißblonden Haaren, stand dem Schurken gegenüber.

Die Schurkin lehnte sich in einem einfachen Holzstuhl zurück, ihre Backenzähne waren deutlich zu sehen, und ihre kastanienbraunen Locken wehten ihr über die Wangen, während sie dem Helden gegenüberstand.

Es war Mittwoch, der 30. Oktober, und die erste Schlacht stand am Sonntag an. Draco stand in General Grangers Büro, einem Raum von der Größe eines kleinen Klassenzimmers.

(Warum das Büro eines jeden Generals so groß war, war Draco nicht ganz klar. Ein Stuhl und ein Schreibtisch hätten für ihn gereicht.

Ihm war nicht einmal klar, warum die Generäle überhaupt Büros brauchten, seine Soldaten wussten, wo sie ihn finden konnten.

Es sei denn, Professor Quirrell hatte die riesigen Büros absichtlich für sie eingerichtet, als Zeichen des Status, in dem Fall war Draco ganz dafür.)

Granger saß auf dem einzigen Stuhl des Raumes wie auf einem Thron, ganz am anderen Ende des Büros, wo sich die Tür öffnete.

Es gab einen langen, länglichen Tisch, der sich in der Mitte des Raumes zwischen ihnen erstreckte, und vier kleine, runde Tische, die in den Ecken verstreut standen, aber nur diesen einen Stuhl, ganz am anderen Ende.

Der Raum hatte Fenster entlang einer Wand, und ein Sonnenstrahl berührte den Scheitel von Grangers Haar wie eine leuchtende Krone.

Es wäre schön gewesen, wenn Draco langsam vorwärts hätte gehen können. Aber da war ein Tisch im Weg, und Draco musste diagonal um ihn herumgehen, und es gab keine gute Möglichkeit, das auf dramatische und würdevolle Weise zu tun.

War das beabsichtigt gewesen? Wenn es sein Vater gewesen wäre, wäre es das sicher gewesen; aber dies war Granger, also sicher nicht.

Er konnte sich nirgendwo hinsetzen, und Granger war auch nicht aufgestanden. Draco hielt sich die Empörung komplett aus dem Gesicht.

"Nun, Mr. Draco Malfoy", sagte Granger, als er vor ihr stand,

"Sie haben um eine Audienz bei mir gebeten und ich war so gnädig, sie zu gewähren.

Wie lautete Ihre Bitte?"

\emph{Komm mit mir nach Malfoy Manor, mein Vater und ich würden dir gerne einige interessante Zaubersprüche zeigen.}

"Dein Rivale Potter ist mit einem Angebot zu mir gekommen", sagte Draco und setzte einen ernsten Gesichtsausdruck auf.

"Es macht ihm nichts aus, gegen mich zu verlieren, aber er wäre gedemütigt, wenn du gewinnen würdest. Also will er sich mit mir zusammentun und dich sofort auslöschen, nicht nur in unserer ersten Schlacht, sondern in allen. Wenn ich das nicht tue, will Potter, dass ich mich zurückhalte oder dich bedränge, während er als ersten Schritt einen Großangriff auf dich startet."

"Ich verstehe", sagte Granger und sah überrascht aus. "Und Sie bieten mir an, mir gegen ihn zu helfen?"

"Natürlich", sagte Draco sanft. "Ich fand das, was er dir antun wollte, nicht fair."

"Das ist aber sehr nett von Ihnen, Mr. Malfoy", sagte Granger.

"Es tut mir leid, wie ich vorhin mit Ihnen gesprochen habe. Wir sollten Freunde sein.

Darf ich dich \emph{Drakey} nennen?"

In Dracos Kopf begannen die Alarmglocken zu läuten, aber es gab eine Chance, dass sie es ernst meinte…

"Natürlich", sagte Draco, "wenn ich dich Hermy nennen darf."

Draco war sich ziemlich sicher, dass er ihren Gesichtsausdruck flackern sah.

"Wie auch immer", sagte Draco, "ich habe mir gedacht, es würde Potter recht geschehen, wenn wir ihn beide angreifen und auslöschen würden."

"Aber das wäre Mr. Potter gegenüber nicht fair, oder?", sagte Granger.

"Ich denke, es wäre sehr fair", sagte Draco. "Er hatte vor, es zuerst mit dir zu tun."

Granger warf ihm einen strengen Blick zu, der ihn möglicherweise eingeschüchtert hätte, wenn er ein Hufflepuff statt eines Malfoys gewesen wäre.

"Sie halten mich für ziemlich dumm, nicht wahr, Mr. Malfoy?"

Draco lächelte charmant.

"Nein, Miss Granger, aber ich dachte, ich würde es zumindest überprüfen.

Also, was wollen Sie?"

"Wollen Sie mich etwa bestechen?", fragte Granger.

"Sicher", sagte Draco. "Kann ich dir einfach eine Galleone zustecken und dich für den Rest des Jahres anstelle von mir gegen Potter antreten lassen?"

"Nö", sagte Granger, "aber du kannst mir zehn Galleonen anbieten und mich euch beide gleichermaßen angreifen lassen, statt nur dich."

"Zehn Galleonen sind eine Menge Geld", sagte Draco vorsichtig.

"Ich wusste nicht, dass die Malfoys arm sind", sagte Granger.

Draco starrte Granger an. Er fing an, ein seltsames Gefühl bei der Sache zu bekommen. Diese spezielle Antwort schien nicht von diesem speziellen Mädchen kommen zu können.

"Nun", sagte Draco, "man wird nicht reich, indem man Geld verschwendet, weißt du."

"Ich weiß nicht, ob Sie wissen, was ein Zahnarzt ist, Mr. Malfoy, aber meine Eltern sind Zahnärzte und alles unter zehn Galleonen ist meine Zeit überhaupt nicht wert."

"Drei Galleonen", sagte Draco, mehr als Sondierung denn als etwas anderes.

"Nö", sagte Granger. "Wenn Sie überhaupt einen gleichwertigen Kampf wollen, glaube ich nicht, dass ein Malfoy einen gleichwertigen Kampf weniger als zehn Galleonen haben will."

Draco bekam langsam ein sehr seltsames Gefühl bei dieser Sache. "Nein", sagte Draco.

"Nein?", sagte Granger. "Das ist ein zeitlich begrenztes Angebot, Mr. Malfoy.

Sind Sie sicher, dass Sie riskieren wollen, ein ganzes Jahr lang von dem Jungen, der gelebt hat, elendig zerquetscht zu werden? Das wäre ziemlich peinlich für das Haus Malfoy, oder?"

Es war ein sehr überzeugendes Argument, eines, das schwer abzulehnen war, aber man wurde nicht reich, indem man Geld ausgab, wenn einem das Herz sagte, dass es eine Falle war.

"Nein", sagte Draco.

"Wir sehen uns am Sonntag", sagte Granger.

Draco drehte sich um und ging ohne ein weiteres Wort aus dem Büro.

\emph{Das war nicht richtig gewesen…}

"Hermine", sagte Harry geduldig, "wir sollten ein Komplott gegeneinander schmieden. Du könntest mich sogar verraten und es würde außerhalb des Schlachtfeldes nichts bedeuten."

Hermine schüttelte den Kopf.

"Das wäre nicht nett, Harry."

Harry seufzte.

"Ich glaube, du bist überhaupt nicht in der Stimmung, die hier herrscht."

\emph{Es wäre nicht nett.} Das hatte sie tatsächlich gesagt.

Hermine wusste nicht, ob sie beleidigt sein sollte über das, was Harry von ihr dachte, oder ob sie sich Sorgen darüber machen sollte, ob sie sich wirklich so sehr nach einem Tugendbold anhörte wie sonst. Es war wohl an der Zeit, das Thema zu wechseln.

"Wie auch immer, hast du für morgen etwas Besonderes vor?", fragte Hermine.

"Es ist -"

Ihre Stimme brach abrupt ab, als sie es bemerkte.

"Ja, Hermine", sagte Harry ein wenig angespannt, "welcher Tag ist es?"

\textbf{Zwischenspiel}:

Es gab eine Zeit, da hatte man den 31. Oktober im magischen Großbritannien \emph{Halloween} genannt. Jetzt war es der \emph{Harry-Potter-Tag}. Harry hatte alle Angebote abgelehnt, sogar das von Minister Fudge, das vielleicht für zukünftige politische Gefälligkeiten gut gewesen wäre und das er eigentlich zähneknirschend hätte annehmen sollen.

Aber für Harry würde der 31. Oktober immer der Tag sein, an dem der Dunkle Lord seine Eltern tötete.

Es hätte irgendwo eine ruhige, würdevolle Gedenkfeier geben müssen, und wenn es eine gab, war er nicht eingeladen worden.

Hogwarts hatte den Tag frei, um zu feiern. Selbst die Slytherins trauten sich nicht, außerhalb ihres Schlafsaals Schwarz zu tragen.

Es gab besondere Veranstaltungen und besonderes Essen und die Lehrer schauten weg, wenn jemand durch die Gänge rannte. Immerhin war es der zehnte Jahrestag. Harry verbrachte den Tag in seinem Koffer, um ihn niemandem zu verderben, aß Snack-Riegel anstelle von Mahlzeiten, las einige seiner traurigeren Science-Fiction-Bücher (keine Fantasy) und schrieb einen Brief an Mum und Dad, der viel länger war als die, die er normalerweise verschickte.

