

\hypertarget{einfuxfchlungsvermuxf6gen}{% \section{27. Einfühlungsvermögen}\label{einfuxfchlungsvermuxf6gen}}

\textbf{\uline{Einfühlungsvermögen}}

\emph{Roger Bacon lebte im 13. Jahrhundert und wird als einer der frühesten Verfechter der wissenschaftlichen Methode angesehen.

Einem Wissenschaftler sein experimentelles Tagebuch zu geben, ist in etwa so, als würde man einem Schriftsteller die Feder geben - nicht die von Shakespeare, sondern die von jemandem, der die Schrift miterfunden hat.}

Es kam nicht jeden Tag vor, dass man Harry Potter betteln sah.

"Biiiiiitte", wimmerte Harry Potter.

Fred und George schüttelten wieder lächelnd den Kopf. Auf Harry Potters Gesicht lag ein gequälter Ausdruck.

"Aber ich habe euch doch erzählt, wie ich das mit Kevin Entwhistles Katze gemacht habe, und Hermine und die verschwindende Limonade, und ich kann euch nichts über den Sprechenden Hut oder den Erinnermich oder Professor Snape erzählen…"

Fred und George zuckten mit den Schultern und wollten gehen.

"Wenn du es jemals herausfindest", sagten die Weasley-Zwillinge, "lass es uns auf jeden Fall wissen."

"Ihr seid böse! Ihr seid beide böse!"

Fred und George schlossen fest die Tür zum leeren Klassenzimmer hinter sich und achteten darauf, das Grinsen noch eine Weile auf ihren Gesichtern zu behalten, nur für den Fall, dass Harry Potter durch Türen sehen konnte.

Dann bogen sie um eine Ecke und ihre Gesichter wurden schlaff.

"Ich nehme nicht an, dass Harrys Vermutungen -"

"- dich auf irgendeine Idee gebracht haben?", sagten sie gleichzeitig zueinander, und dann sackten ihre Schultern weiter zusammen.

Ihre letzte relevante Erinnerung war, dass Flume sich geweigert hatte, ihnen zu helfen, obwohl sie sich nicht erinnern konnten, worum sie ihn gebeten hatten…

… aber sie mussten sich anderweitig umgesehen haben und jemanden gefunden haben, der ihnen half, etwas Illegales zu tun, sonst hätten sie nicht zugestimmt das Ihre Erinnerung danach gelöscht werden würde.

Wie hatten sie das alles mit nur 40 Galleonen schaffen können? Zuerst hatten sie sich Sorgen gemacht, dass sie die Beweise so gut gefälscht hatten, dass Harry am Ende tatsächlich mit Ginny verheiratet sein würde.

.. aber daran hatten sie wohl auch gedacht. Das Zaubergamot-Verfahren war wieder so manipuliert worden, wie es ursprünglich gewesen war, der gefälschte Verlobungsvertrag war aus dem von Drachen bewachten Tresor in Gringotts verschwunden, und so weiter.

Es war ziemlich beängstigend, ehrlich gesagt. Die meisten Leute dachten nun, der Tagesprophet hätte sich die ganze Sache aus unerfindlichen Gründen nur ausgedacht, und der Klitterer hatte das Messer mit der Schlagzeile des nächsten Tages,

\textbf{HARRY POTTER GEHEIM MIT LUNA LOVEGOOD VERHEIRATET}

, noch tiefer ins Fleisch getrieben.

Wer auch immer sie angeheuert hatte, würde es ihnen nach Ablauf der Verjährungsfrist sagen, hofften sie verzweifelt.

Aber in der Zwischenzeit war es furchtbar, sie hatten den größten Streich aller Zeiten abgezogen, vielleicht den größten Streich in der Geschichte der Streiche, und sie wussten nicht mehr, wie.

Es war verrückt, beim ersten Mal war ihnen ein Weg eingefallen, warum konnten sie ihn jetzt nicht sehen, wo sie doch alles wussten, was sie getan hatten? Ihr einziger Trost war, dass Harry nicht wusste, dass sie es nicht wussten.

Nicht einmal Mum hatte sie dazu befragt, trotz der offensichtlichen Weasley-Verbindung. Was auch immer getan worden war, es war weit außerhalb der Reichweite eines jeden Hogwarts-Schülers… außer vielleicht einem, der, wenn gewisse Gerüchte stimmten, es mit einem Fingerschnippen getan haben könnte.

Harry war unter Veritaserum verhört worden, hatte er erzählt… in Anwesenheit von Dumbledore, der den Auroren böse Blicke zuwarf.

Die Auroren hatten gerade genug gefragt, um festzustellen, dass Harry den Streich nicht selbst gespielt oder jemanden verschwinden lassen hatte, und waren dann aus Hogwarts geflüchtet.

Fred und George hatten sich gefragt, ob sie sich beleidigt fühlen sollten, dass Harry Potter von den Auroren wegen ihres Streiches befragt wurde, aber der Ausdruck auf Harrys Gesicht, wahrscheinlich aus genau demselben Grund, war es alles wert.

Es war nicht überraschend, dass Rita Kimmkorn und der Herausgeber des Tagespropheten beide verschwunden waren und sich wahrscheinlich schon in einem anderen Land befanden.

Sie hätten ihrer Familie gerne von diesem Teil erzählt. \emph{Dad hätte ihnen gratuliert,} dachten sie, nachdem Mum mit dem Töten fertig war und Ginny die Überreste verbrannt hatte.

Aber noch war alles in Ordnung, sie würden es Dad eines Tages erzählen, und in der Zwischenzeit…

… in der Zwischenzeit hatte Dumbledore zufällig geniest, als er im Flur an ihnen vorbeiging, und ein kleines Päckchen war versehentlich aus seinen Taschen gefallen, und darin waren zwei passende Fluchbrecher-Monokel von unglaublicher Qualität gewesen.

Die Weasley-Zwillinge hatten ihre neuen Monokel auf dem "\emph{verbotenen}" Korridor im dritten Stock getestet, indem sie einen kurzen Abstecher zum Zauberspiegel und wieder zurück gemacht hatten, und sie hatten nicht alle Erkennungszauber deutlich sehen können, aber die Monokel hatten viel mehr gezeigt, als sie beim ersten Mal gesehen hatten.

Natürlich mussten sie sehr aufpassen, dass sie nicht mit den Monokeln in ihrem Besitz erwischt wurden, sonst würden sie im Büro des Schulleiters landen und eine strenge Lektion und vielleicht sogar die Androhung eines Schulverweises bekommen.

Es war gut zu wissen, dass nicht jeder, der nach Gryffindor sortiert wurde, zu Professor McGonagall heranwuchs.

Harry befand sich in einem weißen Raum, fensterlos, funktionslos, vor einem Schreibtisch sitzend, einem ausdruckslosen Mann in formellen Roben aus solidem Schwarz gegenüber. Der Raum war gegen Entdeckung abgeschirmt, und der Mann hatte genau siebenundzwanzig Zaubersprüche ausgeführt, bevor er auch nur

\emph{"Hallo, Mr. Potter"} sagte. Es war seltsam passend, dass der Mann in Schwarz gleich versuchen würde, Harrys Gedanken zu lesen.

"Mach dich bereit", sagte der Mann tonlos.

Ein menschlicher Geist, so hatte Harrys Okklumentik-Buch gesagt, war einem Legilimens nur entlang bestimmter Oberflächen ausgesetzt.

Wenn man seine Oberfläche nicht verteidigte, konnte der Legilimens durch sie hindurchgehen und auf jeden Teil von einem zugreifen, den sein eigener Verstand erfassen konnte…

… was meist nicht viel war. Der menschliche Verstand war für die Menschen nur sehr schwer zu verstehen.

Harry hatte sich gefragt, ob die Kenntnis vieler kognitiver Wissenschaften ihn zu einem unglaublich mächtigen Legilimens machen könnte, aber wiederholte Erfahrungen hatten ihm schließlich die Lektion eingetrichtert, dass er sich in seinen Vorahnungen über diese Art von Dingen ein wenig weniger aufregen sollte.

Es war ja nicht so, dass irgendein Kognitionswissenschaftler die Menschen gut genug verstand, um einen zu machen.

Um den Gegenspieler, Okklumentik, zu erlernen, war der erste Schritt, sich vorzustellen, eine andere Person zu sein, es so gründlich wie möglich vorzutäuschen, sich ganz in diese alternative Persona zu versenken.

Das musste man nicht immer tun, aber am Anfang lernte man auf diese Weise, wo die eigenen Oberflächen waren.

Der Legilimens würden versuchen, dich zu lesen, und du würdest es spüren, wenn man gut genug aufpasste, würde man spüren, dass jemand versucht, einzudringen.

Und deine Aufgabe war es, dafür zu sorgen, dass der Eindringling immer deine imaginäre Persona berührt und nicht die reale.

Wenn du darin gut genug warst, konnte man sich vorstellen, eine sehr einfache Art von Person zu sein, vorgeben, ein Stein zu sein, und es sich zur Gewohnheit machen, die Vorspiegelung dort zu lassen, wo alle deine Oberflächen waren.

Das war eine Standard-Okklumentik-Barriere. So zu tun, als wäre man ein Stein, war schwer zu erlernen, aber danach leicht zu bewerkstelligen, und die freiliegende Oberfläche eines Geistes war viel flacher als sein Inneres, so dass man es mit genügend Übung als Hintergrundgewohnheit beibehalten konnte.

Oder wenn man ein perfekter Okklument war, konnte man allen Sonden vorauslaufen und Fragen so schnell beantworten, wie sie gestellt wurden, so dass die Legilimens durch deine Oberfläche eintraten einen Geist sahen, der nicht von dem zu unterscheiden war, den man vorgab zu sein.

Selbst die besten Legilimens konnten auf diese Weise getäuscht werden. Wenn ein perfekter Okklument behauptete, er würde seine Okklumentik-Barrieren fallen lassen, gab es keine Möglichkeit zu wissen, ob er lügt.

Schlimmer noch, man konnte nicht wissen, dass man es mit einem perfekten Okklumenten zu tun hatte. Sie waren selten, aber die Tatsache, dass es sie gab, bedeutete, dass man der Legilimenz bei niemandem trauen konnte.

Es war ein trauriger Kommentar dazu, wie wenig die Menschen einander verstanden, wie wenig ein Zauberer die Tiefen verstand, die unter der Oberfläche des Geistes lagen, daß man die besten menschlichen Telepathen täuschen konnte, indem man vorgab, jemand anderes zu sein.

Aber dann verstanden sich die Menschen überhaupt nur, indem sie so taten, als ob.

Man macht keine Vorhersagen über Menschen, indem man die hundert Billionen Synapsen im Gehirn als separate Objekte modelliert.

Bitte den besten sozialen Manipulator der Welt, dir eine Künstliche Intelligenz von Grund auf zu bauen, und er würde dich nur dumm anschauen.

Man sagt Menschen voraus, indem man dem Gehirn sagt, dass es sich wie deren Gehirn verhalten soll.

Du hast dich in ihre Lage versetzt. Wenn man wissen wollte, was eine wütende Person tun würde, aktivierte man die Wutschaltkreise des eigenen Gehirns, und was immer diese Schaltkreise ausgaben, das war die Vorhersage.

Wie sah der neuronale Schaltkreis für Wut im Inneren tatsächlich aus? Wer wusste das schon? Der beste soziale Manipulator der Welt wusste vielleicht nicht, was Neuronen sind, und der beste Legilimens auch nicht.

Alles, was ein Legilimens verstehen konnte, konnte auch ein Okklument vorgeben zu sein. Es war in beiden Fällen derselbe Trick - wahrscheinlich wurde er von denselben neuronalen Schaltkreisen ausgeführt, ein einziger Satz von Kontrollschaltkreisen, um das eigene Gehirn so zu rekonfigurieren, dass es als Modell eines anderen fungierte.

Und so war das Rennen zwischen telepathischer Offensive und telepathischer Verteidigung ein entscheidender Sieg für die Verteidigung.

Sonst wäre die gesamte magische Welt, vielleicht sogar die ganze Erde, ein ganz anderer Ort gewesen.

.. Harry nahm einen tiefen Atemzug und konzentrierte sich. Ein leichtes Lächeln zeichnete sich auf seinem Gesicht ab.

Ausnahmsweise war Harry in der Abteilung für geheimnisvolle Kräfte mal nicht zu kurz gekommen. Nach fast einem Monat Arbeit und mehr aus einer Laune heraus als aus einer wirklichen Ahnung heraus hatte Harry beschlossen, sich \emph{eiskalt} zu ärgern und dann noch einmal die Okklumentik-Übungen aus dem Buch zu versuchen.

Zu diesem Zeitpunkt hatte er die Hoffnung auf so etwas eigentlich schon aufgegeben, aber es schien immer noch einen Versuch wert zu sein -

er hatte die schwierigsten Übungen des Buches in zwei Stunden durchgespielt und am nächsten Tag war er zu Professor Quirrell gegangen, um ihm zu sagen, dass er bereit war.

Seine dunkle Seite, so hatte sich herausgestellt, war sehr, sehr gut darin, so zu tun, als wäre er ein anderer Mensch.

Harry dachte an seinen Standardauslöser, als er das erste Mal ganz zu seiner dunklen Seite übergegangen war.

\emph{.. Severus hielt inne und sah recht zufrieden mit sich selbst aus.

"Und das wären dann… fünf Punkte? Nein, machen wir doch gleich zehn Punkte aus Ravenclaw für Widerrede."}

Harrys Lächeln wurde kühler, und er betrachtete den schwarzgewandeten Mann, der glaubte, Harrys Gedanken lesen zu können.

Und dann verwandelte sich Harry in jemand ganz anderen, jemand, der dem Anlass angemessen schien. .

\emph{..in einem weißen Raum, fensterlos, gesichtslos, vor einem Schreibtisch sitzend, einem ausdruckslosen Mann in förmlichen Roben aus solidem Schwarz gegenüber.

Kimball Kinnison betrachtete den schwarz gekleideten Mann, der glaubte, die Gedanken eines Leutnant der zweiten Stufe der Galaktischen Patrouille zu lesen.

Zu sagen, dass Kimball Kinnison sich des Ergebnisses sicher war, wäre eine Untertreibung. Er war von Mentor von Arisia ausgebildet worden, dem mächtigsten Geist, der in diesem oder jedem anderen Universum bekannt war, und der einfache Zauberer, der ihm gegenüber saß, würde genau das sehen, was der Graue Leutnant ihn sehen lassen wollte…}

\emph{… den Geist des Jungen, als den er gerade verkleidet war, ein unschuldiges Kind namens Harry Potter.}

"Ich bin bereit", sagte Kimball Kinnison in einem nervösen Ton, der für einen elfjährigen Jungen genau richtig war.

"Legilimens", sagte der schwarz gewandete Zauberer.

Es gab eine Pause.

Der schwarzgewandete Zauberer blinzelte, als hätte er etwas so Schockierendes gesehen, dass es ausreichte, um sogar seine Augenlider zu bewegen.

Seine Stimme war nicht ganz tonlos, als er sagte:

"Der Junge-der-lebte hat eine mysteriöse dunkle Seite?!"

Die Hitze kroch langsam in Harrys Wangen hoch.

"Nun", sagte der Mann. Sein Gesicht hatte sich nun wieder in vollkommener Ruhe eingefunden. "Entschuldigen Sie mich. Mr. Potter, es ist gut, seine Vorteile zu kennen, aber das ist nicht dasselbe, wie wildes Übervertrauen in sie zu haben.

Sie könnten tatsächlich in der Lage sein, mit elf Jahren Okklumentik zu lernen. Das verblüfft mich. Ich hatte schon gedacht, Mr. Dumbledore würde wieder so tun, als wäre er geisteskrank. Deine dissoziative Begabung ist so stark, dass ich überrascht bin, keine anderen Anzeichen von Kindheitsmissbrauch zu finden, und du könntest mit der Zeit ein perfekter Okklumentiker werden.

Aber es besteht ein erheblicher Unterschied zwischen dem und der Erwartung, beim ersten Versuch eine erfolgreiche Okklumentikbarriere zu errichten.

Das ist einfach lächerlich. Hast du etwas gespürt, als ich deine Gedanken gelesen habe?"

Harry schüttelte den Kopf und errötete nun heftig.

"Dann pass beim nächsten Mal besser auf. Das Ziel ist nicht, an deinem ersten Unterrichtstag ein perfektes Bild zu schaffen.

Das Ziel ist es, zu lernen, wo deine Oberflächen sind. Bereiten Sie sich vor."

Harry versuchte, wieder so zu tun, als wäre er Kimball Kinnison, versuchte, aufmerksamer zu sein, aber seine Gedanken waren ein wenig zerstreut, und er war sich plötzlich all der Dinge bewusst, an die er nicht denken sollte…

Oh, das würde ätzend werden. Harry biss die Zähne zusammen. Wenigstens würde dem Lehrer danach die Erinnerung gelöscht.

"Legilimens."

Es gab eine Pause -

… in einem weißen Raum, fensterlos, funktionslos, vor einem Schreibtisch sitzend, einem ausdruckslosen Mann in förmlicher, schwarzer Robe gegenüber.

Es war ihr vierter Tag, an einem Sonntagabend. Wenn man so viel bezahlte, bekam man seine Sitzungen, wann immer man wollte, ganz zu schweigen von den Wochenenden.

"Hallo, Mr. Potter", sagte der Telepath tonlos, nachdem er die gesamte Palette an Geheimhaltungszaubern angewendet hatte.

"Hallo, Mr. Bester", sagte Harry müde.

"Lassen wir den ersten Schock erst einmal hinter uns, ja?"

"Sie haben es geschafft, mich zu überraschen?", sagte der Mann und klang nun leicht interessiert. "Nun denn."

Er richtete seinen Zauberstab und starrte in Harrys Augen.

"Legilimens."

Es gab eine Pause, und dann zuckte der schwarzgewandete Zauberer zusammen, als hätte ihn jemand mit einem Viehtreiber berührt.

"Der Dunkle Lord ist noch am leben?!", würgte er. Seine Augen waren plötzlich wild. "Dumbledore macht sich unsichtbar und schleicht sich in die Schlafsäle der Mädchen?!"

Harry seufzte und schaute auf seine Uhr. In etwa drei Sekunden…

"Also", sagte der Mann. Er hatte seine Tonlosigkeit noch nicht ganz wiedererlangt.

"Du glaubst wirklich, dass du die geheimen Regeln der Magie entdecken und allmächtig werden wirst."

"Das stimmt", sagte Harry gleichmäßig und schaute immer noch auf seine Uhr.

"Ich bin so von mir selbst überzeugt."

"Das wundert mich. Es scheint, dass der Sprechende Hut denkt, dass du der nächste Dunkle Lord sein wirst."

"Und du weißt, dass ich mich ziemlich anstrenge, es nicht zu sein, und du hast gesehen, dass wir schon eine lange Diskussion darüber geführt haben, ob du bereit bist, mir Okklumentik beizubringen, und am Ende hast du dich \emph{dafür} entschieden, also können wir das einfach hinter uns bringen?"

"In Ordnung", sagte der Mann genau sechs Sekunden später, genau wie beim letzten Mal. "Bereiten Sie sich vor."

Er hielt inne und sagte dann mit etwas wehmütiger Stimme:

"Obwohl ich wünschte, ich könnte mich an den Trick mit dem Gold und dem Silber erinnern."

Harry fand es sehr beunruhigend, wie reproduzierbar menschliche Gedanken waren, wenn man Menschen auf die gleichen Ausgangsbedingungen zurücksetzte und sie den gleichen Reizen aussetzte.

Es war das Zerstreuen von Illusionen, die ein guter Reduktionist gar nicht erst haben sollte.

Harry war ziemlich schlecht gelaunt, als er am nächsten Montagmorgen aus seiner Kräuterkunde stapfte.

Hermine schäumte neben ihm. Die anderen Kinder waren noch drinnen und brauchten etwas länger, um ihre Sachen zusammenzusuchen, weil sie aufgeregt miteinander darüber plapperten, dass Ravenclaw das zweite Quidditch-Match des Jahres gewonnen hatte.

Es schien, dass gestern Abend nach dem Abendessen ein Mädchen dreißig Minuten lang auf einem Besen herumgeflogen war und dann eine Art Riesenmücke gefangen hatte.

Es gab noch andere Fakten darüber, was während dieses Spiels passiert war, aber sie waren irrelevant.

Harry hatte dieses aufregende Sportereignis wegen seiner Okklumentikstunde verpasst, und außerdem hatte er ein Leben.

Er hatte dann alle Unterhaltungen im Ravenclaw-Schlafsaal vermieden,

\emph{waren Ruhezauber und magische Koffer nicht wunderbar.}

Das Frühstück hatte er am Gryffindor-Tisch eingenommen. Aber Harry konnte Kräuterkunde nicht vermeiden, und die Ravenclaws hatten vor dem Unterricht darüber geredet, und nach dem Unterricht, und während des Unterrichts, bis Harry von dem Babypelz aufgeschaut hatte, dessen Windel er gerade wechselte, und laut verkündet hatte, dass einige von ihnen versuchten, etwas über Pflanzen zu lernen, und dass Schnatze nirgendwo wuchsen, also könnten sie bitte alle die Klappe über Quidditch halten.

Alle anderen Anwesenden hatten ihm schockierte Blicke zugeworfen, außer Hermine, die aussah, als wollte sie applaudieren, und Professor Sprout, die ihm einen Punkt für Ravenclaw gegeben hatte.

\emph{Einen Punkt für Ravenclaw. Einen Punkt.}

Die sieben Idioten auf ihren idiotischen Besen, die ihr idiotisches Spiel spielten, hatten \emph{einhundertneunzig Punkte} für Ravenclaw erzielt.

Es schien, als würden die Quidditch-Punkte direkt zu den Hauspunkten hinzugezählt. Mit anderen Worten: \emph{Einen goldenen Moskito zu fangen war 150 Hauspunkte wert.}

Harry konnte sich nicht einmal vorstellen, was er tun müsste, um hundertfünfzig Hauspunkte zu bekommen.

Außer, du weißt schon, hundertfünfzig Hufflepuffs zu retten, oder auf fünfzehn Ideen zu kommen, die so gut sind wie Schutzhüllen für Zeitmaschinen, oder eintausendfünfhundert kreative Wege zu erfinden, Leute zu töten, oder das ganze Jahr lang Hermine Granger zu sein.

"Wir sollten sie umbringen", sagte Harry zu Hermine, die mit ebenso beleidigter Miene neben ihm herging.

"Wen?", fragte Hermine. "Das Quidditch-Team?"

"Ich dachte an alle, die in irgendeiner Weise mit Quidditch zu tun haben, egal wo, aber das Ravenclaw-Team wäre ein Anfang, ja."

Hermines Lippen waren missbilligend geschürzt. "Du weißt schon, dass es falsch ist, Menschen zu töten, Harry?"

"Ja", sagte Harry.

"Okay, ich wollte nur sichergehen", sagte Hermine.

"Lass uns zuerst den Sucher holen. Ich habe ein paar Krimis von Agatha Christie gelesen, weißt du, wie wir sie in einen Zug bringen können?"

"Zwei Schüler schmieden ein Mordkomplott", sagte eine trockene Stimme.

"Wie schockierend."

Um eine nahe gelegene Ecke schlenderte ein Mann in leicht gefleckten Gewändern, sein fettiges Haar fiel lang und ungepflegt über seine Schultern.

Von ihm schien eine tödliche Gefahr auszugehen, die den Flur mit unsachgemäß gemischten Tränken und versehentlichen Stürzen füllte und mit Menschen, die im Bett an etwas starben, was die Auroren für einen natürlichen Tod halten würden.

Ohne auch nur einen Gedanken daran zu verschwenden, trat Harry vor Hermine. Hinter ihm war ein Einatmen zu hören, und einen Moment später schob sich Hermine an ihm vorbei und trat vor ihn.

"Lauf, Harry!", sagte sie. "Jungs sollten sich nicht in Gefahr begeben müssen."

Severus Snape lächelte freudlos.

"Amüsant. Ich bitte um einen Moment deiner Zeit, Potter, wenn du dich von deinem Flirt mit Miss Granger losreißen kannst."

Plötzlich war da ein sehr besorgter Ausdruck auf Hermines Gesicht.

Sie drehte sich zu Harry um und öffnete den Mund, dann hielt sie inne und sah verzweifelt aus.

"Oh, keine Sorge, Miss Granger", sagte Severus' seidige Stimme.

"Ich verspreche, Ihren Verehrer unbeschadet zurückzubringen."

Sein Lächeln verschwand.

"Jetzt werden Potter und ich uns auf den Weg machen und ein privates Gespräch führen, nur unter uns. Ich hoffe, es ist klar, dass du nicht eingeladen bist, aber nur für den Fall, betrachte das als Befehl eines Hogwarts-Professors. Ich bin sicher, ein braves kleines Mädchen wie du wird nicht ungehorsam sein."

Severus drehte sich um und ging zurück um die Ecke.

"Kommst du, Potter?", fragte seine Stimme.

"Ähm", sagte Harry zu Hermine. "Kann ich ihm einfach so folgen und dich überlegen lassen, was ich sagen soll, damit du nicht total beunruhigt und beleidigt bist?"

"Nein", sagte Hermine, ihre Stimme zitterte.

Severus' Lachen hallte von der Ecke herüber.

Harry senkte den Kopf. "Tut mir leid", sagte er leise, "wirklich", und er ging dem Zaubertränkemeister hinterher.

"Also", sagte Harry. Jetzt gab es keine weiteren Geräusche mehr, nur zwei Beinpaare, ein langes und ein kurzes, die über einen zufälligen Steinkorridor stapften.

Der Meister der Zaubertränke schritt schnell, aber nicht zu schnell, als dass Harry nicht hätte mithalten können, und soweit Harry das Konzept der Richtungsabhängigkeit auf Hogwarts anwenden konnte, bewegten sie sich von den frequentierten Bereichen weg.

"Worum geht es hier?"

"Ich nehme nicht an, dass du erklären kannst", sagte Severus trocken, "warum ihr beide den Mord an Cho Chang geplant habt?"

"Ich nehme nicht an, dass du erklären kannst", sagte Harry trocken, "in deiner Eigenschaft als Beamter des Hogwarts-Schulsystems, warum das Fangen einer goldenen Mücke als eine akademische Leistung gilt, die hundertfünfzig Hauspunkte wert ist?"

Ein Lächeln umspielte Severus' Lippen.

"Du liebe Zeit, und ich dachte, du wärst scharfsinnig. Bist du wirklich so unfähig, deine Klassenkameraden zu verstehen, Potter, oder magst du sie zu wenig, um es zu versuchen? Wenn Quidditch-Punkte nicht für den Hauspokal zählen würden, würde sich keiner von ihnen um Hauspunkte scheren.

Es wäre lediglich ein obskurer Wettbewerb für Schüler wie dich und Miss Granger."

Das war eine schockierend gute Antwort. Und dieser Schock holte Harrys Verstand vollends wach. Im Nachhinein hätte es nicht überraschen dürfen, dass Severus seine Schüler verstand, sogar sehr gut verstand.

Er hatte ihre Gedanken gelesen. Und… … im Buch stand, dass ein erfolgreicher Legilimens extrem selten war, seltener als ein perfekter Okklument, weil fast niemand genug mentale Disziplin hatte.

\emph{Mentale Disziplin?} Harry hatte Geschichten über einen Mann gesammelt, der im Unterricht regelmäßig die Beherrschung verlor und kleine Kinder anpöbelte.

… aber eben dieser Mann hatte, als Harry davon gesprochen hatte, dass der Dunkle Lord noch am Leben war, sofort und perfekt reagiert - genau so, wie jemand, der völlig unwissend war, reagieren würde.

Der Mann pirschte durch Hogwarts mit der Ausstrahlung eines Attentäters, der Gefahr ausstrahlte…

\emph{… was ein echter Attentäter eigentlich nicht tun sollte. Echte Attentäter sollten wie sanfte kleine Buchhalter aussehen, bis sie einen töten.}

Er war das Hausoberhaupt des stolzen und aristokratischen Slytherin und er trug eine Robe mit Flecken von Resten von Tränken und Zutaten, die zwei Minuten Magie hätten entfernen können.

Harry bemerkte, dass er verwirrt war. Und seine Einschätzung der Bedrohung durch den Leiter des Hauses Slytherin schoss astronomisch in die Höhe.

Dumbledore schien Severus für Loyal zu halten, und es hatte nichts gegeben, was dem widersprochen hätte; der Meister der Zaubertränke war "\emph{furchterregend, aber nicht beleidigend}" gewesen, wie versprochen.

Also, so hatte Harry vorhin argumentiert, war dies eine Angelegenheit der Gemeinschaft. Wenn Severus etwas Böses vorgehabt hätte, wäre er sicher nicht gekommen, um Harry vor Hermine, einer Zeugin, zu holen, wenn er einfach hätte warten können, bis Harry allein war… \emph{Harry biss sich leise auf die Lippe.}

"Ich kannte einmal einen Jungen, der Quidditch wirklich verehrte", sagte Severus Snape. "Er war ein absoluter Schwachkopf. Genau wie Sie und ich es erwarten würden."

"Was soll das?" Harry sagte langsam.

"Geduld, Potter."

Severus drehte den Kopf und glitt dann mit der Haltung eines Attentäters in eine nahe gelegene Öffnung in den Korridorwänden, von der ein kleinerer und engerer Gang abging.

Harry folgte ihm und fragte sich, ob es klüger wäre, einfach wegzulaufen. Sie bogen ab und machten eine weitere Kurve und kamen an eine Sackgasse, eine einfache leere Wand.

\emph{Wäre Hogwarts tatsächlich gebaut worden und nicht beschworen oder heraufbeschworen oder geboren oder was auch immer, hätte Harry dem Architekten ein paar scharfe Worte darüber gesagt, dass er Leute dafür bezahlt, Flure zu bauen, die nirgendwo hinführen.

}\strut 

"Quietus", sagte Severus, und auch noch ein paar andere Dinge. Harry lehnte sich zurück, verschränkte die Arme vor der Brust und beobachtete Severus' Gesicht.

"Siehst du mir in die Augen, Potter?", sagte Severus Snape.

"Deine Okklumentik-Lektionen können nicht weit genug fortgeschritten sein, damit du Legilimenz blockieren kannst.

Aber vielleicht sind sie auch weit genug fortgeschritten, dass du sie erkennen kannst. Da ich es nicht anders wissen kann, werde ich es nicht riskieren, es zu versuchen."

Der Mann lächelte dünn.

"Und dasselbe wird auch für Dumbledore gelten, denke ich. Deshalb führen wir jetzt dieses kleine Gespräch."

Harrys Augen weiteten sich unwillkürlich.

"Für den Anfang", sagte Severus mit funkelnden Augen,

"möchte ich, dass du versprichst, niemandem von unseren Gesprächen zu erzählen.

Soweit es die Schule betrifft, besprechen wir deine Zaubertrank-Hausaufgaben. Ob alle das glauben oder nicht, ist unwichtig.

Was Dumbledore und McGonagall betrifft, so verletze ich Draco Malfoys Vertrauen in mich, und keiner von uns beiden hält es für angebracht, weiter über die Einzelheiten zu sprechen."

Harrys Gehirn versuchte, die Verästelungen und Implikationen dieser Aussage zu berechnen und hatte keinen Platz mehr zum Denken.

"Nun?", fragte der Zaubertrankmeister.

"In Ordnung", sagte Harry langsam. Es war schwer zu begreifen, wie ein Gespräch zu führen und es niemandem erzählen zu können, einschränkender sein konnte als es nicht zu führen, weil man in diesem Fall auch niemandem den Inhalt erzählen konnte.

"Ich verspreche es."

Severus beobachtete Harry aufmerksam.

"Du hast einmal im Büro des Schulleiters gesagt, dass du weder Mobbing noch Missbrauch dulden würdest. Und so frage ich mich, Harry Potter. Wie sehr ähnelst du deinem Vater?"

"Wenn wir nicht gerade von Michael Verres-Evans sprechen", sagte Harry,

"lautet die Antwort, dass ich James Potter nicht kenne."

Severus nickte, wie zu sich selbst.

"Es gibt einen Slytherin im fünften Jahr. Ein Junge namens Lesath Lestrange. Er wird von Gryffindors gemobbt.

Ich bin… in meiner Fähigkeit eingeschränkt, mit solchen Situationen umzugehen. Du könntest ihm vielleicht helfen.

Wenn du das willst. Ich bitte dich nicht um einen Gefallen und bin dir auch keinen schuldig. Es ist einfach eine Gelegenheit, das zu tun, was du willst."

Harry starrte Severus an und dachte nach.

"Fragst du dich, ob es eine Falle ist?", sagte Severus, ein schwaches Lächeln umspielte seine Lippen.

"Ist es nicht. Es ist ein Test. Nenne es Neugierde von meiner Seite. Aber Lesaths Probleme sind real, genauso wie meine eigenen Schwierigkeiten, einzugreifen."

Das war das Problem, wenn andere Leute wussten, dass man ein guter Kerl war.

Selbst wenn man wusste, dass sie es wussten, konnte man den Köder nicht ignorieren. Und wenn sein Vater auch Schüler vor Tyrannen beschützt hatte… es spielte keine Rolle, ob Harry wusste, warum Severus es ihm gesagt hatte. Es gab ihm immer noch ein warmes Gefühl und Stolz, und es machte es unmöglich, wegzugehen.

"Gut", sagte Harry. "Erzähl mir von Lesath. Warum wird er gemobbt?"

Severus' Gesicht verlor das schwache Lächeln. "Denkst du, es gibt Gründe, Potter?"

"Vielleicht nicht", sagte Harry leise. "Aber mir ist der Gedanke gekommen, dass er irgendein unwichtiges Schlammblutmädchen die Treppe hinuntergestoßen haben könnte."

"Lesath Lestrange", sagte Severus, seine Stimme jetzt kalt,

"ist der Sohn von Bellatrix Black, der fanatischsten und bösesten Dienerin des Dunklen Lords. Lesath ist der anerkannte Bastard von Rabastan Lestrange. Kurz nach dem Tod des Dunklen Lords wurden Bellatrix und Rabastan sowie Rabastans Bruder Rodolphus gefangen genommen, als sie Alice und Frank Longbottom folterten.

Alle drei sitzen lebenslang in Askaban. Die Longbottoms wurden durch wiederholten Cruciatus in den Wahnsinn getrieben und bleiben in St. Mungos unheilbarer Abteilung. Ist irgendetwas davon ein guter Grund, ihn zu schikanieren, Potter?"

"Es ist überhaupt kein Grund", sagte Harry, immer noch leise. "Und Lesath selbst hat kein Unrecht getan, von dem du weißt?"

Das schwache Lächeln kam wieder über Severus' Lippen.

"Er ist genauso wenig ein Heiliger wie jeder andere. Aber er hat kein Schlammblutmädchen die Treppe hinuntergestoßen, nicht dass ich das je gehört hätte."

"Oder in seinen Gedanken gesehen habe", sagte Harry.

Severus' Ausdruck war kühl.

"Ich bin nicht in seine Privatsphäre eingedrungen, Potter. Ich habe vielmehr in die Gryffindors hineingeschaut. Er ist einfach ein bequemes Ziel für ihre kleinen Gelüste."

Ein kalter Anflug von Wut lief Harrys Rücken hinunter, und er musste sich daran erinnern, dass Severus vielleicht keine vertrauenswürdige Informationsquelle war.

"Und du denkst", sagte Harry, "dass ein Eingreifen von Harry Potter, dem Jungen, der gelebt hat, sich als wirksam erweisen könnte."

"In der Tat", sagte Severus Snape und \emph{verriet Harry, wann und wo die Gryffindors ihr nächstes kleines Spielchen planten.}

Es gibt einen Hauptgang, der in der Mitte des zweiten Stocks von Hogwarts auf der Nord-Süd-Achse verläuft, und in der Nähe der Mitte dieses Ganges gibt es eine Öffnung in einen kurzen Korridor, der ein Dutzend Schritte zurückgeht, bevor er im rechten Winkel abbiegt und eine L-Form bildet, und dann noch ein Dutzend Schritte weitergeht, bevor er an einem hellen, breiten Fenster endet, das von drei Stockwerken über dem Boden auf den leichten Nieselregen hinausschaut, der über das Ostgelände von Hogwarts fällt.

Wenn man am Fenster steht, kann man nichts vom Hauptgang hören, und niemand im Gang würde hören, was am Fenster vor sich geht.

Wenn du denkst, dass hier irgendetwas seltsam ist, bist du noch nicht lange in Hogwarts gewesen.

Vier Jungen in roter Robe lachen, und ein Junge in grüner Robe schreit und klammert sich verzweifelt mit den Händen an den Rand des geöffneten Fensters, während die vier Jungen so tun, als wollten sie ihn hinausstoßen.

Es ist natürlich nur ein Scherz, und außerdem würde ein Sturz aus dieser Höhe einen Zauberer nicht umbringen.

Alles ein guter Spaß. Wenn du denkst, dass daran etwas merkwürdig ist -

"Was macht ihr?!", sagt die Stimme eines sechsten Jungen.

Die vier Jungen in den roten Roben wirbeln mit einem plötzlichen Ruck herum, und der Junge in den grünen Roben stößt sich verzweifelt vom Fenster ab und fällt mit tränenüberströmtem Gesicht zu Boden.

"Oh", sagt der hübscheste der Jungen in den roten Roben und klingt erleichtert,

"du bist es. Hey, Lessy, weißt du, wer das ist?"

Der Junge auf dem Boden, der versucht, sein Schniefen unter Kontrolle zu bringen, antwortet nicht, und der Junge in den roten Roben zieht sein Bein für einen Tritt zurück -

"Hör auf!", schreit der sechste Junge. Der Junge in der roten Robe wackelt, als er den Tritt abbricht.

"Ähm", sagt er, "weißt du, wer das ist?" Das Atmen des sechsten Jungen klingt seltsam.

"Lesath Lestrange", sagt er, sein Atem kommt in kurzen Zügen,

"und er hat meinen Eltern nichts getan, er war fünf Jahre alt."

Neville Longbottom starrte auf die vier riesigen Fünftklässler vor ihm und versuchte angestrengt, sein Zittern zu kontrollieren.

\emph{Er hätte Harry Potter einfach nein sagen sollen.}

"Warum verteidigst du ihn?", fragte der Hübsche langsam und klang bei den ersten Andeutungen von Beleidigung verwirrt. "Er ist ein Slytherin. Und ein Lestrange."

"Er ist ein Junge, der seine Eltern verloren hat", sagte Neville Longbottom.

"Ich weiß, wie das ist."

Er wusste nicht, woher die Worte stammten. Es klang zu cool, wie etwas, das Harry Potter sagen würde. Das Zittern ging aber weiter.

"Was glaubst du, wer du bist?", sagte der Hübsche und begann, wütend zu klingen.

\emph{Ich bin Neville, der letzte Spross des edlen und uralten Hauses Longbottom} - Neville konnte es nicht aussprechen.

"Ich glaube, er ist ein Verräter", sagte einer der anderen Gryffindors, und in Nevilles Magen machte sich plötzlich ein flaues Gefühl breit.

\emph{Er hatte es gewusst, er hatte es einfach gewusst.}

Harry Potter hatte also doch Unrecht gehabt. Tyrannen würden nicht aufhören, nur weil Neville Longbottom ihnen sagte, sie sollten aufhören.

Der Hübsche machte einen Schritt nach vorne, und die drei anderen folgten.

"So ist das also bei euch", sagte Neville und war erstaunt, wie fest seine Stimme war.

"Für euch ist es egal, ob es Lesath Lestrange oder Neville Longbottom ist."

Lesath Lestrange stieß ein plötzliches Keuchen aus, von dort, wo er auf dem Boden lag.

"Böse ist böse", knurrte derselbe Junge, der zuvor gesprochen hatte,

"und wenn du mit dem Bösen befreundet bist, bist du auch böse."

Die vier machten einen weiteren Schritt nach vorne.

Lesath erhob sich, schwankend, auf seine Füße. Sein Gesicht war grau, und er ging ein paar Schritte vorwärts, lehnte sich gegen die Wand und sagte nichts.

Seine Augen waren auf die Biegung des Ganges gerichtet, den Weg nach draußen.

"Freunde", sagte Neville.

Jetzt stieg seine Stimme ein wenig in der Tonlage.

"Ja, ich habe Freunde. Einer von ihnen ist der Junge-der-lebte."

Ein paar der Gryffindors sahen plötzlich besorgt aus. Der Hübsche zuckte nicht zurück.

"Harry Potter ist nicht hier", sagte er mit harter Stimme, "und wenn er es wäre, glaube ich nicht, dass es ihm gefallen würde, wenn ein Longbottom eine Lestrange verteidigt."

Und die Gryffindors machten einen weiteren langen Schritt nach vorne, und hinter ihnen schlich Lesath an der Wand entlang und wartete auf seine Chance.

Neville schluckte und hob seine rechte Hand, wobei er Daumen und Zeigefinger zusammenpresste. Er schloss die Augen, denn Harry Potter hatte ihn versprechen lassen, nicht zu spähen.

\emph{Wenn das nicht klappte, würde er nie wieder jemandem trauen.}

Seine Stimme kam erstaunlich klar heraus, wenn man bedenkt was er tat.

"Harry James Potter-Evans-Verres. Harry James Potter-Evans-Verres. Harry James Potter-Evans-Verres.

Bei der Schuld, die du mir schuldest, und der Macht deines wahren Namens rufe ich dich herbei, ich öffne dir den Weg, ich rufe dich auf, dich vor mir zu offenbaren."

\emph{Neville schnippte mit den Fingern.}

Und dann öffnete Neville seine Augen. Lesath Lestrange starrte ihn an. Die vier Gryffindors starrten ihn an. Der gutaussehende fing an zu kichern, und das brachte die anderen drei dazu einzustimmen.

"Sollte Harry Potter um die Ecke kommen oder so?", sagte der Hübsche.

"Ach. Sieht aus, als wärst du versetzt worden."

Der Gutaussehende machte einen bedrohlichen Schritt auf Neville zu.

Die anderen drei folgten im Gleichschritt.

"\emph{Ähem}", sagte Harry Potter von hinten und lehnte sich gegen die Wand am Fenster, in der Sackgasse des Flurs, in die niemand hätte gelangen können, ohne gesehen zu werden.

Wenn es sich immer so gut anfühlte, Leute schreien zu sehen, konnte Neville irgendwie verstehen, warum Leute zu Rüpeln wurden.

Harry Potter pirschte sich vor und stellte sich zwischen Lesath Lestrange und die anderen. Er ließ seinen eisigen Blick über die Jungen in den roten Roben schweifen, und dann ruhten seine Augen auf dem gut aussehenden Anführer.

"Mr. Carl Sloper", sagte Harry Potter. "Ich glaube, ich habe die Situation voll und ganz begriffen. Wenn Lesath Lestrange jemals selbst ein einziges Übel begangen hat, anstatt von den falschen Eltern geboren worden zu sein, dann ist Ihnen das nicht bekannt. Sollte ich mich in diesem Punkt irren, Mr. Sloper, schlage ich vor, dass Sie mich sofort informieren."

Neville sah die Angst und Ehrfurcht in den Gesichtern der anderen Jungen.

Er fühlte es selbst. Harry hatte behauptet, es wäre alles nur ein Trick, aber wie konnte das sein?

"Aber er ist ein Lestrange", sagte der Rädelsführer.

"Er ist ein Junge, der seine Eltern verloren hat", sagte Harry Potter, und seine Stimme wurde noch kälter.

Dieses Mal zuckten alle drei anderen Gryffindors zusammen.

"Also", sagte Harry Potter. "Du hast gesehen, dass Neville nicht wollte, dass du im Namen der Longbottoms einen unschuldigen Jungen quälst.

Das hat dich nicht gerührt. Wenn ich dir sage, dass der Junge-der-lebte auch denkt, dass du im Unrecht bist, dass das, was du heute getan hast, ein schrecklicher Fehler war, macht das einen Unterschied?"

Der Rädelsführer machte einen Schritt auf Harry zu. Die anderen folgten ihm nicht.

"Carl", sagte einer von ihnen und schluckte. "Vielleicht sollten wir gehen."

"Man sagt, du wirst der nächste Dunkle Lord", sagte der Rädelsführer und starrte Harry an.

Ein Grinsen ging über Harry Potters Gesicht.

"Sie sagen auch, dass ich heimlich mit Ginevra Weasley verlobt bin und dass es eine Prophezeiung über die Eroberung Frankreichs gibt."

Das Lächeln verblasste.

"Da Sie die Sache unbedingt erzwingen wollen, Mr. Carl Sloper, lassen Sie mich etwas klarstellen. Lassen Sie Lesath in Ruhe. Ich werde es erfahren, wenn Sie es nicht tun."

"Lessy hat also bei dir gepetzt", sagte der Rädelsführer kalt.

"Sicher", sagte Harry Potter trocken, "und er hat mir auch erzählt, was du heute, nachdem du den Zauberkunstunterricht verlassen hast, an einem abgeschiedenen Ort, wo dich niemand sehen konnte, mit einem gewissen Hufflepuff-Mädchen gemacht hast, das eine weiße Schleife im Haar trug -"

Dem Rädelsführer fiel vor Schreck die Kinnlade herunter.

"Eep", sagte einer der anderen Gryffindors mit hoher Stimme, drehte sich auf den Fersen und rannte um die Ecke. Seine Schritte klapperten schnell weg und verblassten. Und dann waren es sechs.

"Ah", sagte Harry Potter, "da geht ein leicht intelligenter junger Mann.

Der Rest von euch könnte sich ein Beispiel an Bertram Kirke nehmen, bevor ihr in, sagen wir mal, Schwierigkeiten geratet."

"Drohst du damit, uns zu verhexen?", sagte der gutaussehende Gryffindor, wobei seine Stimme versuchte, wütend zu klingen, und ziemlich schwankte.

"Leuten wie dir passieren schlimme Dinge."

Die beiden anderen Gryffindors wichen langsam zurück. Harry Potter begann zu lachen.

"Oh, das hast du nicht gerade gesagt. Versuchen Sie wirklich, mich einzuschüchtern? Mich? Jetzt mal ehrlich, glaubst du, du bist furchteinflößender als Peregrine Derrick, Severus Snape oder gar \emph{Du-weißt-schon-wer?}"

Sogar der Rädelsführer zuckte bei dieser Aussage zusammen.

Harry Potter hob seine Hand, die Finger spitz, und alle drei Gryffindors sprangen zurück, und einer von ihnen platzte heraus: "Nicht - !"

"Seht ihr", sagte Harry Potter, "jetzt schnippe ich mit den Fingern und ihr werdet Teil einer wahnsinnig amüsanten Geschichte, die heute Abend beim Abendessen mit viel nervösem Gelächter erzählt werden wird.

Aber die Sache ist die, dass Leute, denen ich vertraue, mir immer wieder sagen, dass ich das nicht tun soll.

Professor McGonagall hat mir gesagt, dass ich alles auf die leichte Schulter nehme und Professor Quirrell sagt, ich muss lernen, wie man verliert.

Erinnerst ihr euch an die Geschichte, wo ich mich von ein paar älteren Slytherins verprügeln ließ? Das könnten wir machen.

Ihr könntet mich eine Weile schikanieren und ich könnte euch lassen. Nur erinnerst ihr euch an den Teil am Ende, wo ich meinen \emph{vielen, vielen Freunden} in dieser Schule sage, dass sie nichts dagegen tun sollen? Dieses Mal überspringen wir diesen Teil.

Also nur zu. Schikaniert mich."

Harry Potter trat vor, die Arme einladend weit geöffnet. Die drei Gryffindors brachen ab und rannten los, und Neville musste schnell ausweichen, um nicht überfahren zu werden. Es herrschte Stille, als ihre Schritte verklangen, und danach noch mehr Stille. Und dann waren es drei.

Harry Potter holte tief Luft und atmete dann aus.

"Uff", sagte er. "Wie geht's dir, Neville?"

Nevilles Stimme kam in einem hohen Quietschton heraus. "Okay, das war wirklich cool."

Ein Grinsen huschte über Harry Potters Gesicht.

"Du warst auch ziemlich cool, weißt du."

Neville wusste, dass Harry Potter das nur sagte, um ihm ein gutes Gefühl zu geben, und trotzdem löste es ein warmes Glühen in seiner Brust aus.

Harry drehte sich zu Lesath Lestrange um.

"Geht es dir gut, Lestrange?", sagte Neville, bevor Harry den Mund öffnen konnte.

\emph{Das war etwas, von dem man nicht erwartet hatte, dass man es jemals sagen würde.}

Lesath Lestrange drehte sich langsam um und starrte Neville an, sein Gesicht war angespannt, er weinte nicht mehr, die Tränen glitzerten, als sie trockneten.

"Du glaubst, du weißt, wie es ist?", sagte Lesath, seine Stimme hoch und zitternd.

"Du glaubst, du weißt es? Meine Eltern sind in Askaban, ich versuche, nicht daran zu denken, und sie erinnern mich immer daran, sie finden es toll, dass Mutter dort in der Kälte und der Dunkelheit ist, mit den Dementoren, die ihr das Leben aussaugen, ich wünschte, ich wäre wie Harry Potter, wenigstens tun seine Eltern nicht weh, meine Eltern tun immer weh, jede Sekunde an jedem Tag, \emph{ich wünschte, ich wäre wie du,} wenigstens kannst du deine Eltern manchmal sehen, wenigstens weißt du, dass sie dich geliebt haben, \emph{wenn Mutter mich jemals geliebt hat, dann haben die Dementoren diesen Gedanken schon aufgefressen -"}

Nevilles Augen waren groß vor Schreck. Damit hatte er nicht gerechnet. Lesath drehte sich zu Harry Potter um, dessen Augen voller Entsetzen waren.

Lesath warf sich vor Harry Potter auf den Boden, berührte mit der Stirn den Boden und flüsterte: \emph{"Hilf mir, Herr."}

Es herrschte eine furchtbare Stille.

Neville fiel nichts ein, was er sagen konnte, und dem nackten Schock auf Harrys Gesicht nach zu urteilen, fiel ihm auch nichts ein.

"Man sagt, du kannst alles tun, bitte, bitte mein Herr, holen Sie meine Eltern aus Askaban heraus, ich werde für immer Ihr treuer Diener sein, mein Leben wird Ihnen gehören und mein Tod auch, nur bitte -"

"Lesath", sagte Harry, seine Stimme brach, "Lesath, ich kann nicht, ich kann so etwas wirklich nicht tun, das sind alles nur dumme Tricks."

"Ist es nicht!", sagte Lesath, seine Stimme hoch und verzweifelt.

"Ich habe es gesehen, die Geschichten sind wahr, du kannst es!"

Harry schluckte.

"Lesath, ich habe die ganze Sache mit Neville eingefädelt, wir haben alles im Voraus geplant, frag ihn!"

Das hatten sie, obwohl Harry nicht gesagt hatte, wie er irgendetwas davon machen wollte.

.. Als Lesath vom Boden aufblickte, war sein Gesicht grässlich, und seine Stimme kam in einem Schrei heraus, der in Nevilles Ohren schmerzte.

"Du Sohn eines Schlammbluts! Du könntest sie rausholen, du tust es nur nicht! Ich bin auf die Knie gegangen und habe dich angefleht, und du willst immer noch nicht helfen!

Ich hätte es wissen müssen, du bist der Junge-der-lebte, du denkst sie gehört dort hin!"

"Ich kann nicht!" sagte Harry, seine Stimme war genauso verzweifelt wie die von Lesath. "Es geht nicht darum, was ich will, ich habe nicht die Macht dazu!"

Lesath stand auf und spuckte vor Harry auf den Boden, dann drehte er sich um und ging weg. Als er um die Ecke war, beschleunigte sich das Geräusch seiner Füße, und als sie verklangen, glaubte Neville, ein einzelnes Schluchzen zu hören.

Und dann waren es zwei. Neville sah Harry an. Harry sah Neville an.

"Wow", sagte Neville leise. "Er schien nicht sehr dankbar dafür zu sein, gerettet worden zu sein."

"Er dachte, ich könnte ihm helfen", sagte Harry, seine Stimme war heiser.

"Er hatte zum ersten Mal seit Jahren wieder Hoffnung."

Neville schluckte und sagte es dann. "Es tut mir leid."

"Was?", sagte Harry und klang völlig verwirrt.

"Ich war dir nicht dankbar, als du mir geholfen hast -"

"Alles, was du vorher gesagt hast, war völlig richtig", sagte der Junge-der-lebte.

"Nein", sagte Neville, "das war es nicht."

Sie schenkten sich gleichzeitig ein kurzes, trauriges Lächeln, jeder dem anderen gegenüber herablassend.

"Ich weiß, dass das nicht echt war", sagte Neville, "ich weiß, dass ich nichts hätte tun können, wenn du nicht hier gewesen wärst, aber danke, dass ich so tun durfte."

"Mach mal halblang", sagte Harry.

Harry hatte sich von Neville abgewandt und starrte aus dem Fenster auf die düsteren Wolken.

Ein völlig lächerlicher Gedanke kam Neville in den Sinn.

"Fühlst du dich schuldig, weil du Lesaths Eltern nicht aus Askaban herausholen kannst?"

"Nein", sagte Harry.

Ein paar Sekunden vergingen.

"Ja", sagte Harry.

"Du bist dumm", sagte Neville.

"Ich bin mir dessen bewusst", sagte Harry.

"Musst du buchstäblich alles tun, was man dich fragt?"

Der Junge-der-lebte drehte sich um und sah Neville wieder an.

"Tun? Nein. Sich schuldig fühlen, wenn man es nicht tut? Ja."

Neville hatte Mühe, Worte zu finden.

"Nach dem Tod des Dunklen Lords war Bellatrix Black buchstäblich der böseste Mensch auf der ganzen Welt, und das war, bevor sie nach Askaban kam.

Sie hat meine Mutter und meinen Vater in den Wahnsinn gefoltert, weil sie herausfinden wollte, was mit dem Dunklen Lord passiert ist -"

"Ich weiß", sagte Harry leise. "Das verstehe ich, aber -"

"Nein! Das tust du nicht! Sie hatte einen Grund, das zu tun, und meine Eltern waren beide Auroren! Das ist nicht einmal annähernd das Schlimmste, was sie je getan hat!"

Nevilles Stimme zitterte.

"Trotzdem", sagte der Junge-der-lebte, seine Augen starrten in die Ferne, an einen anderen Ort, den sich Neville nicht vorstellen konnte.

"Es könnte eine unglaublich clevere Lösung geben, die es möglich macht, alle zu retten und sie alle glücklich bis ans Ende ihrer Tage leben zu lassen, und wenn ich nur schlau genug wäre, hätte ich schon längst daran gedacht -"

"Du hast Probleme", sagte Neville. "Du denkst, du müsstest das sein, wofür Lesath Lestrange dich hält."

"Ja", sagte der Junge-der-lebte, "das trifft es so ziemlich auf den Punkt.

Jedes Mal, wenn jemand im Gebet schreit und ich nicht antworten kann, fühle ich mich schuldig, weil ich nicht Gott bin."

Neville verstand das zwar nicht ganz, aber… "Das hört sich nicht gut an."

Harry seufzte. "Ich verstehe, dass ich ein Problem habe, und ich weiß, was ich tun muss, um es zu lösen, in Ordnung? Ich arbeite daran."

Harry sah Neville beim Gehen zu. Natürlich hatte Harry nicht gesagt, was die Lösung war.

\emph{Die Lösung war offensichtlich, sich zu beeilen und Gott zu werden}.

Nevilles Schritte entfernten sich, und bald waren sie nicht mehr zu hören. Und dann war da noch einer.

"Ähem", sagte die Stimme von Severus Snape direkt hinter ihm. Harry stieß einen kleinen Schrei aus und hasste sich augenblicklich.

Langsam drehte sich Harry um. Der große, schmierige Mann in den gefleckten Roben lehnte an der Wand in der gleichen Position, die Harry eingenommen hatte.

"Ein schöner Unsichtbarkeitsumhang, Potter", murmelte der Zaubertränkemeister.

"Damit ist vieles geklärt."

\emph{Oh, verdammter Mist.}

"Und vielleicht bin ich schon zu lange in Dumbledores Gesellschaft", sagte Severus,

"aber ich komme nicht umhin, mich zu fragen, ob das der \emph{Umhang der Unsichtbarkeit} ist."

Harry verwandelte sich sofort in jemanden, der noch nie vom Tarnumhang gehört hatte und der genau so schlau war, wie Harry dachte, dass Severus Harry für schlau hielt.

"Oh, möglicherweise", sagte Harry. "Ich vertraue darauf, dass du dir über die Auswirkungen im Klaren bist, wenn es so ist?"

Severus' Stimme war herablassend.

"Du hast keine Ahnung, wovon ich rede, nicht wahr, Potter? Ein ziemlich ungeschickter Versuch, mehr herauszufinden."

(Professor Quirrell hatte während ihres Mittagessens bemerkt, dass Harry seinen Gemütszustand besser verbergen sollte, als ein ausdrucksloses Gesicht aufzusetzen, wenn jemand ein gefährliches Thema besprach, und er hatte etwas über einstufige Täuschungen, zweistufige Täuschungen und so weiter erklärt.

Entweder stellte Severus Harry also tatsächlich als einstufigen Spieler dar, was Severus selbst zu einem zweistufigen machte, und Harrys dreistufiger Zug war erfolgreich gewesen; oder Severus war ein vierstufiger Spieler und wollte, dass

Harry dachte, die Täuschung sei erfolgreich gewesen.

Harry hatte Professor Quirrell lächelnd gefragt, auf welcher Stufe er spielte, und Professor Quirrell hatte, ebenfalls lächelnd, geantwortet:

"Eine Stufe höher als du.")

"Du hast also die ganze Zeit zugesehen", sagte Harry. "Desillusionierung, glaube ich, nennt man das."

Ein dünnes Lächeln.

"Es wäre dumm von mir gewesen, auch nur das geringste Risiko einzugehen, dass du zu Schaden kommst."

"Und du wolltest die Ergebnisse deines Tests aus erster Hand sehen", sagte Harry.

"Also. Bin ich wie mein Vater?"

Ein seltsam trauriger Ausdruck überkam den Mann, einer, der seinem Gesicht fremd war.

"Ich sollte eher sagen, Harry Potter, dass du jemand anderem ähnlich bist -"

Severus hielt kurz inne. Er starrte Harry an.

"Lestrange hat dich einen Sohn eines Schlammbluts genannt", sagte Severus langsam.

"Es schien dich nicht sonderlich zu stören."

Harry runzelte die Augenbrauen. "Nicht unter diesen Umständen, nein."

"Du hattest ihm gerade geholfen", sagte Severus. Seine Augen waren auf Harry gerichtet. "Und er hat es dir ins Gesicht zurückgeworfen. Das ist doch sicher nichts, was du einfach so verzeihen würdest?"

"Er hatte gerade eine ziemlich erschütternde Erfahrung hinter sich", sagte Harry.

"Und dass er von Erstklässlern gerettet wurde, hat seinem Stolz wohl auch nicht gerade geholfen."

"Ich nehme an, es war leicht genug zu verzeihen", sagte Severus, und seine Stimme war seltsam, "denn Lestrange bedeutet dir nichts. Nur irgendein fremder Slytherin. Wenn es ein Freund gewesen wäre, hättest du dich vielleicht viel mehr durch das, was er gesagt hat, verletzt gefühlt."

"Wenn er ein Freund wäre", sagte Harry, "umso mehr Grund, ihm zu verzeihen."

Es herrschte ein langes Schweigen.

Harry fühlte, und er hätte nicht sagen können, warum oder woher, dass sich die Luft mit einer \emph{furchtbaren Spannung füllte, wie Wasser, das steigt und steigt und steigt.}

Dann lächelte Severus, sah plötzlich wieder entspannt aus, und die ganze Anspannung verschwand.

"Du bist ein sehr nachsichtiger Mensch", sagte Severus, immer noch lächelnd.

"Ich nehme an, dein Stiefvater, Michael Verres-Evans, war derjenige, der dir das beigebracht hat."

"Eher Dads Science-Fiction- und Fantasy-Sammlung", sagte Harry.

"Sozusagen mein fünftes Elternteil, wirklich. Ich habe das Leben all der Figuren in all meinen Büchern gelebt, und all ihre mächtigen Weisheiten donnern in meinem Kopf. Irgendwo da drin war jemand wie Lesath, nehme ich an, obwohl ich nicht sagen könnte, wer. Es war nicht schwer, mich in seine Lage zu versetzen.

Und es waren auch meine Bücher, die mir sagten, was ich tun sollte. Die guten Jungs verzeihen."

Severus stieß ein leichtes, amüsiertes Lachen aus.

"Ich fürchte, ich wüsste nicht viel darüber, was gute Menschen tun."

Harry sah ihn an. Das war irgendwie traurig, ehrlich gesagt.

"Ich kann dir ein paar Romane leihen, in denen gute Menschen vorkommen, wenn du willst."

"Ich würde dich gern in einer Sache um Rat fragen", sagte Severus mit lässiger Stimme.

"Ich weiß von einem anderen Slytherin aus dem fünften Jahr, der von Gryffindors schikaniert wurde. Er umwarb ein wunderschönes Muggelgeborenes Mädchen, das ihm beim Mobben über den Weg lief und versuchte, ihn zu retten.

Er nannte sie ein Schlammblut und das war das Ende für sie. Er hat sich entschuldigt, viele Male, aber sie hat ihm nie verziehen. Haben Sie eine Idee, was er hätte sagen oder tun können, um von ihr die Vergebung zu erlangen, die Sie Lestrange gegeben haben?"

"Ähm", sagte Harry, "nur aufgrund dieser Information bin ich mir nicht sicher, ob er das Hauptproblem war. Ich hätte ihm geraten, sich nicht mit jemandem zu verabreden, der so unfähig zur Vergebung ist. Angenommen, sie hätten geheiratet, können Sie sich das Leben in diesem Haushalt vorstellen?"

Es gab eine Pause.

"Oh, aber sie konnte verzeihen", sagte Severus mit Belustigung in der Stimme.

"Aber danach ging sie weg und wurde die Freundin des Tyrannen.

Sag mir, warum sollte sie dem Tyrannen verzeihen und nicht dem Opfer?"

Harry zuckte mit den Schultern.

"Ich vermute, weil der Tyrann jemand anderem sehr wehgetan hat und das Opfer ihr nur ein wenig wehgetan hat, und das erschien ihr irgendwie viel unverzeihlicher.

Oder, um es nicht zu sehr zuzuspitzen, war der Tyrann gut aussehend? Oder, was das betrifft, reich?"

Wieder gab es eine Pause.

"Ja zu beidem", sagte Severus.

"Und da haben Sie es", sagte Harry.

"Nicht, dass ich jemals selbst durch die Highschool gegangen wäre, aber aus meinen Büchern weiß ich, dass es eine bestimmte Art von Teenagermädchen gibt, die über eine einzige Beleidigung empört ist, wenn der Junge einfach oder arm ist, die aber irgendwie Platz in ihrem Herzen finden kann, um einem reichen und gut aussehenden Jungen seine Schikanen zu verzeihen.

Sie war \emph{oberflächlich}, mit anderen Worten. Sagen Sie demjenigen, der es war, dass sie seiner nicht würdig war und er darüber hinwegkommen und weiterziehen muss und sich das nächste Mal mit Mädchen verabreden sollte, die tiefgründig statt hübsch sind."

Severus starrte Harry schweigend an, seine Augen glitzerten.

Das Lächeln war verblasst, und obwohl Severus' Gesicht zuckte, kehrte es nicht zurück. Harry fing an, sich ein wenig nervös zu fühlen.

"Ähm, nicht, dass ich selbst Erfahrung auf diesem Gebiet hätte, natürlich, aber ich denke, ein weiser Ratgeber aus meinen Büchern würde das sagen."

Es herrschte noch mehr Schweigen und noch mehr Glitzern. Es war wahrscheinlich ein guter Zeitpunkt, das Thema zu wechseln.

"Also", sagte Harry. "Habe ich Ihren Test bestanden, was auch immer es war?"

"Ich denke", sagte Severus, "dass es keine weiteren Unterhaltungen zwischen uns geben sollte, Potter, und du wärst äußerst klug, niemals über diese zu sprechen."

Harry blinzelte.

"Würden Sie mir bitte sagen, was ich falsch gemacht habe?"

"Du hast mich beleidigt", sagte Severus.

"Und ich traue deiner Gerissenheit nicht mehr."

Harry starrte Severus verblüfft an.

"Aber Sie haben mir einen gut gemeinten Rat gegeben", sagte Severus Snape,

"und so werde ich Ihnen im Gegenzug einen wahren Rat geben."

Seine Stimme war fast vollkommen gleichmäßig.

Wie eine Schnur, die trotz des massiven Gewichts, das in ihrer Mitte hing, durch eine Million Tonnen Spannung, die an beiden Enden zogen, fast perfekt horizontal gespannt war.

"\emph{Du wärst heute fast gestorben}, Potter. In Zukunft solltest du niemals deine Weisheit mit jemandem teilen, wenn du nicht genau weißt, wovon du redest."

Harrys Verstand stellte endlich die Verbindung her. "Du warst das -"

Harrys Mund schnappte zu, als der "\emph{fast gestorben}"-Teil einsickerte, zwei Sekunden zu spät.

"Ja", sagte Severus, "das war ich."

Und die schreckliche Spannung flutete zurück in den Raum wie Wasser, das auf dem Grund des Ozeans unter Druck steht. Harry konnte nicht atmen.

\emph{Verlieren. Jetzt}.

"Ich wusste es nicht", flüsterte Harry. "Ich bin…"

"Nein", sagte Severus. Nur dieses eine Wort.

Harry stand schweigend da, sein Verstand suchte krampfhaft nach Möglichkeiten.

Severus stand zwischen ihm und dem Fenster, was wirklich schade war, \emph{denn ein Sturz aus dieser Höhe würde einen Zauberer nicht umbringen.}

"Deine Bücher haben dich verraten, Potter", sagte Severus, immer noch mit dieser Stimme, die von einer Million Tonnen Zugkraft angespannt wurde.

"Sie haben dir die eine Sache nicht gesagt, die du wissen musstest. Du kannst nicht aus Geschichten lernen, wie es ist, den zu verlieren, den du liebst. Das ist etwas, das man nie verstehen kann, ohne es selbst zu fühlen."

"Mein Vater", flüsterte Harry.

Es war seine beste Vermutung, das Einzige, was ihn retten konnte.

"Mein Vater hat versucht, dich vor den Tyrannen zu beschützen."

Ein grässliches Lächeln zog sich über Severus' Gesicht, und der Mann bewegte sich auf Harry zu.

Und an ihm vorbei.

"Auf Wiedersehen, Potter", sagte Severus und blickte auf dem Weg nach draußen nicht zurück.

"Von heute an werden wir uns nur noch wenig zu sagen haben."

An der Ecke blieb der Mann stehen, und ohne sich umzudrehen, sprach er ein letztes Mal.

"\textbf{Dein Vater war der Tyrann}", sagte Severus Snape, "und was deine Mutter in ihm sah, habe ich bis heute nicht verstanden."

\emph{Er ging.}

Harry drehte sich um und ging auf das Fenster zu. Seine zitternden Hände legten sich auf den Sims.

\emph{Gib niemals jemandem einen klugen Rat, wenn du nicht genau weißt, wovon ihr} \emph{beide redet.Ich hab's.}

Harry starrte eine Weile auf die Wolken und den leichten Nieselregen hinaus. Das Fenster blickte auf das Ostgelände hinaus, und es war Nachmittag, wenn also die Sonne überhaupt durch die Wolken sichtbar war, konnte Harry sie nicht sehen.

Seine Hände hatten aufgehört zu zittern, aber es gab ein enges Gefühl in Harrys Brust, als würde sie von Metallbändern zusammengedrückt werden.

Sein Vater war also ein Tyrann gewesen. Und seine Mutter war oberflächlich gewesen. Vielleicht waren sie später erwachsen geworden.

Gute Menschen wie Professor McGonagall schienen wirklich viel von ihnen zu halten, und das lag vielleicht nicht nur daran, dass sie heldenhafte Märtyrer waren.

Natürlich war das ein schwacher Trost, wenn man elf war und dabei war, ein Teenager zu werden, und sich fragte, was für ein Teenager man wohl werden würde.

\emph{So furchtbar. So furchtbar traurig. Was für ein schreckliches Leben Harry führte. Wenn er erfährt, dass seine genetischen Eltern nicht perfekt waren, dann sollte er eine Weile Trübsal blasen und sich selbst bemitleiden.

Vielleicht könnte er sich bei Lesath Lestrange beschweren.}

Harry hatte über Dementoren gelesen. Kälte und Dunkelheit umgaben sie, und Angst, sie saugten all deine glücklichen Gedanken weg, und in dieser Abwesenheit stiegen all deine schlimmsten Erinnerungen an die Oberfläche.

Er konnte sich vorstellen, in Lesaths Schuhen zu stecken, zu wissen, dass seine Eltern lebenslang in Askaban saßen, diesem Ort, aus dem noch nie jemand entkommen war.

Und Lesath würde sich an der Stelle seiner Mutter vorstellen, in der Kälte und der Dunkelheit und der Angst, allein mit all ihren schlimmsten Erinnerungen, sogar in ihren Träumen, jede Sekunde eines jeden Tages.

Für einen Augenblick stellte sich Harry seine eigene Mutter und seinen eigenen Vater in Askaban vor, mit den Dementoren, die ihnen das Leben aussaugten und die glücklichen Erinnerungen an ihre Liebe zu ihm aussaugten.

\emph{Nur für einen Augenblick, bevor seine Vorstellungskraft eine Sicherung durchbrannte und eine Notabschaltung auslöste und ihm sagte}\textbf{\emph{, er solle sich das nie wieder vorstellen.}}

War es richtig, das jemandem anzutun, selbst dem zweitbösesten Menschen auf der Welt?

\emph{Nein}, sagte die Weisheit von Harrys Büchern, n\emph{icht wenn es einen anderen Weg gibt, überhaupt einen anderen Weg.}

Und wenn das Justizsystem der Zauberer nicht so perfekt war wie ihre Gefängnisse - und das klang alles in allem eher unwahrscheinlich -, dann saß irgendwo in Askaban eine Person, die völlig unschuldig war, und wahrscheinlich mehr als eine.

In Harrys Kehle brannte es, und in seinen Augen sammelte sich Feuchtigkeit, und er wollte alle Gefangenen von Askaban in Sicherheit teleportieren und Feuer vom Himmel rufen, um diesen schrecklichen Ort in Grund und Boden zu sprengen.

Aber er konnte es nicht, weil er nicht Gott war.

Und Harry erinnerte sich daran, was Professor Quirrell unter dem Sternenlicht gesagt hatte: \emph{Manchmal, wenn diese fehlerhafte Welt ungewöhnlich hassenswert erscheint, frage ich mich, ob es vielleicht einen anderen Ort gibt, weit weg, wo ich hätte sein sollen… Aber die Sterne sind so sehr, sehr weit weg… Und ich frage mich, wovon ich träumen würde, wenn ich lange, lange schliefe.}

Im Moment erschien diese fehlerhafte Welt ungewöhnlich hasserfüllt. Und Harry konnte Professor Quirrells Worte nicht verstehen, es könnte ein Außerirdischer gewesen sein, der gesprochen hatte, oder eine künstliche Intelligenz, etwas, das so anders gebaut war als Harry, dass sein Gehirn nicht gezwungen werden konnte, in diesem Modus zu arbeiten.

\emph{Man konnte seinen Heimatplaneten nicht verlassen, solange es dort noch einen Ort wie Askaban gab. Man musste bleiben und kämpfen.}

