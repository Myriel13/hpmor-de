

\hypertarget{die-wahrheit-teil-1}{% \section{103. Die Wahrheit, Teil 1}\label{die-wahrheit-teil-1}}

\textbf{\uline{Die Wahrheit, Teil 1,}}

\textbf{\emph{\uline{Rätsel und Antworten}}}

13. Juni 1992.

Es war die letzte Schulwoche in Hogwarts, und Professor Quirrell lebte noch. Gerade so. Der Verteidigungsprofessor selbst würde an diesem Tag im Bett eines Heilers liegen, so wie er es schon fast die ganze letzte Woche getan hatte. Die Hogwarts-Tradition besagte, dass die Prüfungen in der ersten Juniwoche stattfanden, dass in der zweiten Woche die Prüfungsergebnisse veröffentlicht wurden und dass in der dritten Woche am Sonntag das Abschiedsfest stattfand und am Montag der Hogwarts-Express nach London fuhr.

Harry hatte sich vor langer Zeit, als er zum ersten Mal von diesem Zeitplan gelesen hatte, gefragt, was genau die Schüler während des Rests der zweiten Juniwoche machten, da \emph{"Auf die Prüfungsergebnisse warten"} nicht nach viel klang; und die Antwort hatte ihn überrascht, als er es herausgefunden hatte.

Aber nun war auch die zweite Juniwoche vorbei, und es war Samstag; es gab nichts mehr vom Jahr außer dem Abschiedsfest am 14. und der Fahrt mit dem Hogwarts-Express am 15. Und nichts war beantwortet worden. Nichts war geklärt worden.

Hermines Mörder war nicht gefunden worden. Irgendwie hatte Harry gedacht, dass am Ende des Schuljahres sicher die ganze Wahrheit ans Licht kommen würde; als wäre das das Ende eines Kriminalromans und die Lösung des Rätsels war ihm versprochen worden. Sicherlich musste sie bekannt sein, wenn der Verteidigungsprofessor… \emph{starb}, es konnte nicht sein, dass Professor Quirrell starb, ohne die Antwort zu kennen, ohne dass alles sauber aufgelöst war. Nicht die Examensnoten, schon gar nicht der Tod, nur die Wahrheit beendete eine Geschichte…

Aber solange man Draco Malfoys neuester Theorie nicht abkaufte, dass Professor Sprout zu der Zeit, als Hermine der Mordversuch angehängt wurde, weniger Hausaufgaben aufgegeben und benotet hatte, und damit bewies, dass Professor Sprout ihre Zeit damit verbracht hatte, den Mord zu planen, blieb die Wahrheit unentdeckt. Und stattdessen, als hätte die Welt Prioritäten, die eher der Denkweise anderer Menschen entsprachen, sollte das Jahr mit einem kulminierenden Quidditch-Match enden.

In der Luft über dem Stadion schwirrten und pirouettierten ferne Gestalten auf Besen und drehten sich umeinander. Der rot-purpurne Tetraederstumpf, der der Quaffel war, wurde gefangen, geworfen, geblockt und gelegentlich durch schwebende Reifen geworfen, begleitet von stadionerschütternden Schreien des Triumphs oder der Bestürzung. Blaue und grüne und gelbe und rotgesäumte Roben schrien mit dem Enthusiasmus, den Menschen so leicht empfinden, wenn von ihnen persönlich keine Handlung verlangt wird. Es war das erste Quidditchspiel, dem Harry in Hogwarts beiwohnte, und er hatte bereits beschlossen, dass es das letzte sein würde.

"Davies hat den Quaffel!", rief die verstärkte Stimme von Lee Jordan.

"Das sind weitere zehn Punkte für Ravenclaw in sieben … sechs … fünf … heiliger Strohsack, er hat es schon geschafft! Ein Schlag durch die Mitte des mittleren Reifens! Ich habe noch nie so eine Siegesserie gesehen - ich sage jetzt schon, dass Davies nächstes Jahr Kapitän wird, wenn Bortan zurücktritt -"

Lees Stimme brach abrupt ab und Professor McGonagalls eigene, verstärkte Stimme sagte: "Das ist die Angelegenheit des Ravenclaw-Teams, Mr. Jordan. Beschränken Sie sich bitte auf das Spiel."

"Und die Slytherins haben Quaffelbesitz - Flint übergibt den Quaffel an die schöne -"

"Mr. Jordan!"

"An die annehmbare Sharon Vizcaino, deren Haar wie ein Komet auf die Ravenclaw-Verteidigung zurast. Pucey ist Sharon auf den Fersen - was machst du da, Inglebee? - und sie weicht in der Luft aus - IST DAS DER SCHNATZ? LOS, CHO CHANG, LOS, HIGGS IST SCHON DA - WAS MACHT IHR BEIDEN DENN DA?"

"Beruhigen Sie sich, Mr. Jordan!"

"WIE SOLL ICH MICH DENN BERUHIGEN? DAS WAR DER SCHLIMMSTE FEHLSCHUSS, DEN ICH JE GESEHEN HABE! Und der Schnatz ist weg - vielleicht für immer weg, nachdem er so knapp verfehlt wurde - Pucey läuft auf die Torpfosten zu, Inglebee ist nicht in seiner Nähe -"

In einer fernen Ära der Geschichte, vielleicht in einer ganz anderen Welt, hatte Professor Quirrell dafür gesorgt, dass der Hauspokal entweder an Slytherin oder Ravenclaw gehen würde. Oder vielleicht auch beides, denn er hatte versprochen, dass drei Wünsche in Erfüllung gehen würden. Bis jetzt sah es bei zwei von drei Wünschen gut aus. Wenn man nur nach dem aktuellen Punktestand ging, führte Hufflepuff das Rennen um den Hauspokal mit etwa fünfhundert Punkten Vorsprung an, dank der Schüler von Hufflepuff, die ihre Hausaufgaben machten und sich von Ärger fernhielten. Es schien, als hätte Professor Snape den Hufflepuffs in den letzten sieben Jahren strategisch eine ganze Menge Punkte weggenommen. Haus Slytherin, der amtierende Champion der letzten sieben Jahre, hatte immer noch den Vorteil einer gewissen Großzügigkeit seines Hausoberhauptes bei der Punktevergabe; und das reichte aus, um mit Haus Ravenclaw, dem Haus der akademischen Leistungsträger, Kopf an Kopf zu liegen. Gryffindor lag weit abgeschlagen auf dem letzten Platz, wie es sich für das Haus der Nonkonformisten gehört; Gryffindor hatte das Profil von Slytherin, wenn es um Akademiker und Unfug ging, nur ohne den Vorteil von Professor Snape. Sogar Fred und George hatten das Jahr nur knapp im Plus überstanden. Sowohl das Haus Ravenclaw als auch das Haus Slytherin brauchten von irgendwoher eine Menge Punkte, wenn sie in den nächsten zwei Tagen mit Hufflepuff gleichziehen wollten.

Und so weit man wusste, hatte Professor Quirrell nichts getan, was zu dem offensichtlichen Ergebnis geführt hätte. Es geschah ganz von selbst, jetzt, wo ein einsamer Professor in Hogwarts einen Jahrgang in kreativen Problemlösungen unterrichtet hatte. Das letzte Quidditchspiel des Jahres fand zwischen Ravenclaw und Slytherin statt. Zu Beginn des Jahres war Gryffindors anfänglicher Vorsprung im Quidditch verschwunden, nachdem ihr neuer Sucher, Emmett Shear, während seines zweiten Spiels von einem möglicherweise defekten Besen gefallen war. Dies hatte auch eine eilige Neuansetzung der verbleibenden Spiele erforderlich gemacht. Dieses, das letzte Spiel des Jahres, würde erst enden, wenn der Schnatz gefangen wurde.

Quidditch-Ergebnisse gingen direkt in die Hauspunktzahl ein. Und heute schien es, als könnten sowohl die Slytherin- als auch die Ravenclaw-Sucher den… Schnatz… nicht… fangen.

"DER SCHNATZ WAR PRAKTISCH AUF DIR DRAUF, DU SCHWACHSINNIGER SCHWACHKOPF!"

"Beherrschen Sie Ihre Sprache, Mr. Jordan, oder ich nehme Sie aus dem Spiel! Obwohl es ein schreckliches Spiel war, das gebe ich zu."

Harry musste zugeben, dass Lee Jordan und Professor McGonagall eine wunderbare komödiantische Routine hatten, mit Jordan als Bananenmann und Professor McGonagall als Chef-Frau; Harry bedauerte jetzt ein wenig, dass er das bei den früheren Quidditchspielen verpasst hatte. Es war eine Seite von Professor McGonagall, die er bisher noch nicht gesehen hatte.

Ein paar Sitze weiter unten, wo Harry in der Hufflepuff-Sektion der Quidditch-Tribüne saß, lauerte die massige Gestalt von Cedric Diggory. Der Super-Hufflepuff hatte den jüngsten Beinahe-Zusammenstoß zwischen Cho Chang und Terence Higgs mit dem scharfen Auge eines Zauberers beobachtet, der selbst ein Sucher und ein Quidditch-Kapitän war.

"Der Ravenclaw-Sucher ist neu", sagte Cedric. "Aber Higgs ist in seinem siebten Jahr. Ich habe schon gegen ihn gespielt. Er ist besser als das."

"Meinst du, das ist eine Strategie?", fragte einer der Hufflepuffs, die neben Cedric saßen.

"Es würde Sinn machen, wenn Slytherin ein paar Extrapunkte bräuchte, um beim Quidditch-Cup in Führung zu gehen", sagte Cedric. "Aber Slytherin hat uns im Kampf um den Titel schon geschlagen. Was denken die sich nur? Da hätten sie doch gleich gewinnen können!"

Das Spiel hatte um sechs Uhr nachmittags begonnen. Ein typisches Spiel hätte bis sieben oder so gedauert, dann wäre es Zeit für das Abendessen gewesen.

Juni in Schottland bedeutete viel Tageslicht; die Sonne ging erst um zehn Uhr unter. Es war nach Harrys Uhr acht Uhr und sechs Minuten, als Slytherin gerade weitere zehn Punkte erzielt hatte, was den Spielstand auf 170:140 brachte, als Cedric Diggory von seinem Platz aufsprang und rief:

"Diese Bastarde!"

"Ja!", rief ein Junge neben ihm und sprang selbst auf die Beine. "Was denken die, wer sie sind, dass sie Punkte machen?"

"Das nicht!", rief Cedric Diggory. "Sie - sie versuchen, uns den Pokal zu stehlen!"

"Aber wir sind doch gar nicht mehr im Rennen um -"

"Nicht den Quidditch-Pokal! Den Hauspokal!"

Die Nachricht verbreitete sich, mit Schreien der Empörung.

Das war Harrys Stichwort. Harry fragte höflich eine Hufflepuff-Hexe, die neben ihm saß, und einen anderen Hufflepuff, der eine Reihe über ihm saß, ob sie zur Seite gehen könnten. Dann zog Harry aus seinem Beutel eine riesige Schriftrolle hervor und entrollte sie zu einem zwei Meter hohen Banner, das in der Luft hängen blieb.

Die Verzauberung war mit freundlicher Genehmigung eines Ravenclaw aus dem sechsten Jahr durchgeführt worden, der den Ruf hatte, weniger über Quidditch zu wissen als Harry.

In großen, leuchtenden lila Buchstaben stand auf dem Schild:

KA\textbf{UFT EINE UHR}

\textbf{2 : 06 : 47}

Darunter war ein Schnatz zu sehen, mit einem blinkenden roten X darüber.

Sekunde für Sekunde erhöhte sich der Zeitzähler. Je höher der Zähler stieg, desto mehr Hufflepuffs schienen beschlossen zu haben, dass sie neben Harrys Banner sitzen wollten. Als sich das Spiel über neun hinauszog, schienen auch viele Gryffindors da zu sein. Als die Sonne unterging und Harry anfing, Lumos zu benutzen, um seine Bücher zu lesen - das eigentliche Spiel hatte er schon vor langer Zeit aufgegeben - gab es eine auffällige Anzahl von Ravenclaws, die ihren Patriotismus für Vernunft verraten hatten. Und Professor Sinistra. Und Professor Vector. Und, als die Sterne zu leuchten begannen, Professor Flitwick.

Der Höhepunkt des letzten Quidditchspiels des Jahres zog sich in die Länge. Eines der Dinge, mit denen Harry nicht gerechnet hatte, als er beschlossen hatte, dies zu tun, war, dass er um - Harry blickte auf seine Uhr - elf Uhr nachts immer noch hier draußen sein würde. Harry las gerade ein Verwandlungslehrbuch aus dem sechsten Jahr; oder besser gesagt, er hatte das Buch aufgeschlagen, beleuchtet von einem Muggel-Leuchtstab, während er eine der Übungen machte. Letzte Woche, als die Ravenclaws der Abschlussklasse ihre U.T.Z.-Ergebnisse besprachen, hatte Harry zufällig mitbekommen, dass es bei den Verwandlungsübungen der Oberstufe einige

"Formungsübungen" gab, bei denen es mehr auf Kontrolle und präzises Denken als auf rohe Kraft ankam; und Harry hatte sich sofort daran gemacht, diese zu lernen, wobei er sich selbst hart auf die Stirn schlug, weil er nicht versucht hatte, alle Lehrbücher der Oberstufe früher zu lesen. Professor McGonagall hatte Harry eine Formgebungsübung genehmigt, bei der es darum ging, die Art und Weise zu kontrollieren, in der sich ein verwandelter Gegenstand seiner endgültigen Form näherte - zum Beispiel einen Federkiel so zu verändern, dass zuerst der Schaft herauswuchs, dann die Widerhaken. Harry machte eine analoge Übung mit Bleistiften, indem er zuerst die Mine herauswachsen ließ, sie dann mit Holz umgab und schließlich den Radiergummi obenauf setzen ließ. Wie Harry vermutet hatte, hatte sich die Konzentration seiner Aufmerksamkeit und Magie auf einen bestimmten Teil der fortschreitenden Verwandlung des Bleistifts als ähnlich erwiesen wie die mentale Disziplin, die bei der partiellen Verwandlung verwendet wurde - mit der man in der Tat denselben Effekt vortäuschen konnte, indem man nur die äußeren Schichten des Objekts partiell verwandelte. Dieser Weg erwies sich jedoch als viel einfacher.

Harry beendete seinen aktuellen Bleistift und schaute auf das Quidditchspiel, das, siehe da, immer noch fantastisch langweilig war.

Lee Jordan kommentierte in einem Ton dumpfer Abscheu: "Wieder zehn Punkte - juhu - und jetzt nimmt wieder jemand den Quaffel in Besitz - fragt mich, ob es mich interessiert."

Fast niemand auf der Tribüne achtete mehr darauf, denn alle, die im Stadion geblieben waren, schienen eine neue und interessantere Sportart entdeckt zu haben, nämlich die Debatte über die Änderung der Hauspokalregeln und/oder Quidditch.

Der Streit war so hitzig geworden, dass alle in der Nähe befindlichen Professoren kaum noch die Ordnung aufrechterhalten konnten, die über einen offenen Kampf hinausging. Dieser Streit hatte leider deutlich mehr als zwei Fraktionen. Einige verflixte Wichtigtuer schlugen vernünftig klingende Alternativen zur völligen Abschaffung des Schnatzes vor, und das drohte die Abstimmung zu spalten und den Schwung für die Reform zu ersticken.

\emph{Im Nachhinein} dachte Harry, \emph{wäre es schön gewesen, wenn Draco sein eigenes Banner auf der Slytherin-Seite entrollt hätte, auf dem stand:

"SCHNATZE SIND STARK", um die Polarität der Debatte zu bestimmen.}

Harry hatte vorhin zur Slytherin-Sektion hinübergeblinzelt, aber er hatte Draco nirgends auf der Tribüne entdecken können. Severus Snape, der ebenfalls sympathisch genug gewesen wäre, um den schurkischen Gegner zu spielen, war ebenfalls nirgends zu sehen.

"Mr. Potter?", sagte eine Stimme neben ihm.

Neben Harrys Platz stand ein kleiner, aber älterer Hufflepuff-Junge, jemand, der Harry noch nie aufgefallen war, und hielt ihm einen leeren Pergamentumschlag hin, auf dessen Vorderseite Wachs getropft war. Das Wachs war ebenfalls leer, ohne Abdruck.

"Was ist das?", fragte Harry.

"Ich bin es", sagte der Junge. "Mit dem Umschlag, den du mir gegeben hast. Ich weiß, du hast gesagt, ich soll nicht mit dir reden, aber -"

"Dann rede nicht mit mir", sagte Harry.

Der Junge warf Harry den Umschlag zu und ging mit beleidigtem Blick davon. Es ließ Harry ein wenig zusammenzucken, aber in Anbetracht der zeitlichen Probleme war es wahrscheinlich nicht die falsche Entscheidung gewesen… Dann brach Harry das unsignierte Wachssiegel und holte den Inhalt des Umschlags heraus. Es war Pergament statt des Muggelpapiers, das Harry erwartet hätte, aber die Schrift darauf war seine eigene Handschrift, wenn auch mit einem Federkiel statt einer Feder geschrieben. Auf dem Pergament stand:

\emph{Hüte dich vor der Konstellation und hilf dem Wächter der Sterne.}

\emph{Geh ungesehen an den Verbündeten der Lebensfresser vorbei, und an den Weisen und Wohlmeinenden.

Sechs, und sieben im Quadrat, an dem Ort, der verboten und verdammt dumm ist.}

Harry nahm es mit einem Blick auf, dann faltete er das Papier wieder zusammen und steckte es mit einem weiteren ausgeatmeten Seufzer zurück in seinen Mantel.

\emph{'Hüte dich vor der Konstellation'}, \emph{wirklich}? Harry hätte erwartet, dass ein Rätsel, das er sich selbst überlassen hatte, leichter zu interpretieren gewesen wäre… obwohl einige Teile offensichtlich genug waren. Offensichtlich hatte sich der Zukunfts-Harry Sorgen darüber gemacht, dass dieses Papier abgefangen werden könnte, und obwohl der Gegenwart-Harry normalerweise nicht an die hiesigen Auroren als diejenigen gedacht hätte, die mit den Dementoren von Askaban im Bunde standen, war das vielleicht die beste Art, "Auror" zu sagen, ohne möglicherweise irgendjemanden zu verraten, der das Pergament las und sein eigenes Bestes tat, es zu entschlüsseln.

Die Redewendung aus der Parselsprache, die er während des Vorfalls mit Askaban benutzt hatte, zurück zu übersetzen… das funktionierte, nahm Harry an. Die Notiz hatte besagt, dass Professor Quirrell Hilfe brauchte und dass, was auch immer vor sich ging, unbemerkt von den Auroren und von Dumbledore und McGonagall und Flitwick passieren musste. Da die Zeitumdrehung bereits involviert war, bestand die offensichtliche Lösung darin, auf die Toilette zu gehen, in der Zeit zurückzureisen und gleich nach dem Verlassen des Raumes zum Spiel zurückzukehren. Harry begann, sich von seinem Platz zu erheben, dann zögerte er.

Seine Hufflepuff-Seite bemerkte etwas darüber, dass er die Auroren-Eskorte zurückgelassen und Professor McGonagall nichts gesagt hatte, und er fragte sich, warum sein zukünftiges Ich dumm war.

Harry faltete das Pergament wieder auf und warf einen weiteren Blick auf den Inhalt. Bei näherer Betrachtung sagte das Rätselversum nicht, dass Harry niemanden mitbringen durfte.

\emph{Draco Malfoy}..\emph{. fehlte er beim Quidditchspiel, weil Zukunfts-Harry, Stunden in der Vergangenheit, Draco als Verstärkung mitgebracht hatte?}

\emph{Aber das machte keinen Sinn, die Sicherheit wurde nicht wesentlich erhöht, wenn man einen anderen Erstklässler mitbrachte…}

\emph{… Draco Malfoy wäre sicher dabei gewesen, unabhängig von seinen persönlichen Gefühlen gegenüber Quidditch, um zu sehen, wie Slytherin den Hauspokal gewinnt. War etwas mit ihm passiert?}

Plötzlich fühlte sich Harry nicht mehr so müde. Ein Rinnsal von Adrenalin begann in Harry aufzusteigen, \emph{aber nein, dies würde nicht wie der Troll sein.} Die Nachricht hatte Harry gesagt, wann er ankommen sollte. Harry würde nicht zu spät kommen, nicht dieses Mal. Harry schaute zu Cedric Diggory hinüber, der sichtlich hin- und hergerissen war zwischen einer Gruppe von Ravenclaws, die argumentierten, dass der Schnatz behalten werden müsse, weil es Tradition sei und Regeln Regeln seien, und einem Rudel Hufflepuffs, die sagten, es sei nicht fair, dass der Sucher wichtiger sei als die anderen Spieler. Cedric Diggory war ein hervorragender Duelllehrer für Harry und Neville gewesen, und Harry hatte gedacht, dass sie eine gute Beziehung aufgebaut hätten. Noch wichtiger war, dass ein Schüler, der buchstäblich alle Wahlfächer belegte vermutlich seinen eigenen Zeitumkehrer hatte. Vielleicht könnte Harry versuchen, Cedric dazu zu bringen, mit ihm in der Zeit zurückzureisen? Der Super-Hufflepuff schien ein guter Ersatz-Minion zu sein, den man in jeder Art von brenzliger Situation an seiner Seite haben wollte…

\textbf{Später, und früher:}

Harrys Uhr zeigte jetzt 11:45 Uhr, was nach einer Rückwärtsschleife von fünf Stunden 18:45 Uhr bedeutete.

"Es ist Zeit", murmelte Harry in die leere Luft und begann, den Korridor im dritten Stock über der großen Treppe auf der rechten Seite hinunterzugehen.

\emph{Der Ort, der verboten ist} würde normalerweise den Verbotenen Wald bedeuten; das war wahrscheinlich das, was jemand, der die Nachricht abfing, denken sollte.

Aber der Verbotene Wald war riesig, und es gab mehr als einen markanten Ort in ihm. Es gab keinen offensichtlichen Treffpunkt, an den man gehen sollte oder ein Ereignis finden konnte, das ein Eingreifen erforderte. Aber wenn man den Modifikator "\emph{verdammt dumm"} hinzufügte, gab es nur einen einzigen verbotenen Ort in Hogwarts, der passte.

\emph{Und so machte sich Harry auf zum verbotenen Korridor im dritten Stock, den, wenn die Gerüchte stimmten, alle Gryffindors des ersten Jahres zuvor gegangen waren.}

Der Korridor im 3. Stock, auf der rechten Seite. Eine geheimnisvolle Tür, die zu einer Reihe von Räumen führte, die mit gefährlichen und potenziell tödlichen Fallen gefüllt waren, durch die man unmöglich hindurchkommen konnte, vor allem, wenn man erst im ersten Jahr war.

Harry wusste selbst nicht, was für Fallen ihn erwarteten. Was, wenn er darüber nachdachte, bedeutete, dass die Schüler, die hindurchgegangen waren, erstaunlich gewissenhaft gewesen waren, um das Rätsel für andere nicht zu ruinieren. Vielleicht gab es dort unten ein Schild, auf dem stand: "\emph{Nicht verraten, nur als Gefallen für mich, aufrichtig Schulleiter Dumbledore}". Alles, was Harry bisher wusste, war, dass sich die äußere Tür nach Alohomora öffnete und dass sich im letzten Raum ein magischer Spiegel befand, der dein Spiegelbild in einer Situation zeigte, die du sehr ansprechend fandest, was anscheinend die große Belohnung war.

Der Korridor im dritten Stock wurde von einem schwachen blauen Licht erhellt, das aus dem Nichts zu kommen schien, und die Bögen waren mit Spinnweben bedeckt, als wäre der Korridor seit Jahrhunderten nicht mehr benutzt worden und nicht erst seit einem Jahr. Harrys Beutel war voll mit nützlichen Muggelsachen und nützlichen Zauberersachen und allem, was er gefunden hatte und was ein Questgegenstand sein könnte.

(Harry hatte Professor McGonagall gebeten, jemanden zu empfehlen, der das Fassungsvermögen des Beutels erweitern konnte, und sie hatte es einfach selbst getan.)

Harry hatte den Zauber angewandt, den er für Kämpfe gelernt hatte und der dafür sorgte, dass seine Brille an seinem Gesicht klebte, unabhängig davon, wie er seinen Kopf bewegte. Harry hatte die Verwandlungen, die er aufrechterhielt, aufgefrischt, sowohl das winzige Juwel in dem Ring an seiner Hand als auch das andere, für den Fall, dass er bewusstlos geschlagen wurde. Er war nicht buchstäblich auf alles vorbereitet, aber Harry war so bereit, wie er glaubte, sein zu können.

Die fünfseitigen Bodenfliesen knarrten unter Harrys Schuhen und verschwanden hinter ihm, als würde die Zukunft zur Vergangenheit.

Es war fast 19:49 Uhr, sieben und sieben im Quadrat. Offensichtlich, wenn man in Muggelmathematik dachte, ansonsten nicht so sehr. Gerade als Harry um eine weitere Ecke biegen wollte, kitzelte etwas in seinem Hinterkopf, und er hörte eine leise Stimme sprechen.

"… vernünftige Person… warte bis später… nachdem bestimmte Lehrkräfte gegangen sind…"

Harry blieb stehen, dann schlich er so leicht wie möglich vorwärts, ohne um die Ecke zu gehen, und versuchte, Professor Quirrells Stimme besser zu hören. Es kam ein lauteres Husten, und dann sprach die leise Stimme wieder um die Ecke.

"Aber wenn sie auch… selbst abreisen würden… zu dieser Zeit…", murmelte die Stimme, "könnten sie denken… dieses letzte Spiel… ist die beste Ablenkung… die in diesem Jahr noch übrig ist… eine vorhersehbare Ablenkung. Also habe ich nachgesehen… um zu sehen, welche wichtigen Leute… nicht bei dem Spiel waren… und ich sah, dass der Schulleiter fehlte… aber nach allem, was meine Magie mir sagen kann… könnte er in einem anderen… Reich der Existenz sein… Ich sah auch Ihre eigene Abwesenheit, also beschloss ich, dorthin zu gehen, wo Sie waren. Das ist es, was ich hier tue… und jetzt… was tust du hier?"

Harry atmete flach und lauschte.

"Und woher wussten Sie, wo ich bin?", erklang die Stimme von Severus Snape, so laut, dass Harry fast zusammenzuckte.

Ein kleines, hustendes Lachen.

"Überprüfen Sie Ihren Zauberstab … auf Spuren."

Severus sagte etwas in magischem Pseudolatein und dann:

"Du hast es gewagt, dich an meinem Zauberstab zu schaffen zu machen? Du hast es gewagt?!"

"Du bist ein Verdächtiger… genau wie ich… also ist deine falsche Empörung verschwendet… wie gut gespielt sie auch sein mag… jetzt sag mir… was tust du?"

"Ich beobachte diese Tür", sagte die Stimme von Professor Snape.

"Und ich werde Sie auffordern, sich davon zu entfernen!"

"Mit wessen Befugnis … befehlen Sie mir … mein Kollege Professor?"

Es gab eine Pause, dann: "Na, die des Schulleiters", kam die sanfte Stimme von Severus Snape. "Er hat mir befohlen, diese Tür während des Quidditchspiels zu bewachen, und als Professor muss ich seinen Launen gehorchen. Ich werde mich später mit dem Obersten Rat darüber unterhalten, aber im Moment tue ich, was ich tun muss. Und jetzt ab mit dir, wie der Schulleiter es wünscht."

"Was? Soll ich etwa glauben, dass du deine Slytherins im Stich gelassen hast, während ihres wichtigsten Spiels des Jahres, und auf Dumbledores Geheiß wie ein Hund aufgesprungen bist? Nun, das… ist durchaus plausibel, muss ich sagen. Und trotzdem… halte ich es für klug, wenn ich selbst über Sie wache, während Sie diese schöne Tür bewachen."

Es gab ein Geräusch von raschelndem Stoff und einen leisen Aufprall, als ob sich jemand hart auf den Boden gesetzt hätte oder vielleicht einfach gefallen wäre.

"Oh, bei der Liebe von Merlin -" Severus Snapes Stimme klang jetzt wütend. "Steh auf, du!"

"Ba-blu-a-bu-bluh -", sagte der Verteidigungsprofessor im Zombie-Modus und Sabber tropfte ihm aus dem offenen Mund.

"Steh auf!", sagte Severus Snape, und es gab einen leisen Aufprall.

\emph{Hilf dem Wächter der Sterne} - Harry trat um die Ecke, obwohl es möglich war, dass er dies auch ohne eine intertemporale Nachricht getan hätte.

\emph{Hatte Professor Snape gerade Professor Quirrell getreten? Das wäre leichtsinnig gewesen, wenn Professor Quirrell tot und begraben gewesen wäre.}

Eine runde Tür aus dunklem Holz war von einem steinernen Bogen eingerahmt, eingefasst in die staubigen Marmorziegel von Hogwarts. Wo ein Muggel einen Türknauf angebracht hätte, befand sich nur eine Klinke aus poliertem Metall; es gab keine sichtbaren Schlösser oder sichtbare Schlüssellöcher. An den Wänden auf beiden Seiten brannten zwei Fackeln, die ein unheilvolles orangefarbenes Licht verbreiteten.

Vor der Tür stand der Meister der Zaubertränke in seiner üblichen fleckigen Robe. Neben der Tür, links unter der orangefarbenen Fackel, sackte die Gestalt des Verteidigungsprofessors zusammen, mit dem Rücken an die Wand gelehnt, den Kopf auf die Umgebung starrend. Die Augen schienen zu flackern, als wären sie auf halbem Weg zwischen Bewusstsein und Leere.

"Was", sagte die hoch aufragende Gestalt des Tränkemeisters, "machst du hier, Potter?!".

Dem Gesichtsausdruck und dem Tonfall nach zu urteilen, war der Meister der Zaubertränke ziemlich wütend auf Harry; und ganz sicher war er nicht Harrys Mitverschwörer bei Beratungen, zu denen der Verteidigungsprofessor nie eingeladen worden war.

"Ich bin mir nicht sicher", sagte Harry.

Er war sich nicht sicher, welche Rolle er spielen sollte, und griff in seiner Verzweiflung auf einfache Ehrlichkeit zurück.

"Ich glaube, ich soll vielleicht ein Auge auf den Verteidigungsprofessor haben."

Der Meister der Zaubertränke starrte ihn kalt an.

"Wo ist deine Eskorte, Potter? Schüler dürfen nicht allein durch diese Hallen gehen!"

Harrys Verstand war wirklich leer. Das Spiel war im Gange, und niemand hatte ihm die Regeln erklärt. "Ich bin mir nicht sicher, wie ich das beantworten soll…"

Der kalte Ausdruck auf Professor Snapes Gesicht flackerte auf. "Vielleicht sollte ich die Auroren rufen", sagte er.

"Warten Sie!" platzte Harry heraus.

Die Hand des Zaubertränkemeisters schwebte über seinen Roben.

"Warum?", fragte der Zaubertrankmeister.

"Ich… ich denke nur, dass Sie sie wahrscheinlich nicht anrufen sollten…"

Blitzschnell war der Zauberstab des Meisters der Zaubertränke in seiner Hand.

"Nullus confundio!"

Ein schwarzer Strahl schoss hervor und traf Harry in die Richtung, in die Harry bereits auszuweichen begonnen hatte. Es folgten vier weitere Zaubersprüche, die Worte wie Polyfluis und Metamorphus enthielten; und für diese blieb Harry höflich stehen. Nachdem all diese Zauber keine Wirkung gezeigt hatten, starrte Severus Snape Harry mit einem dunklen Glitzern an, das nun echt zu sein schien.

"Ich schlage vor", sagte der Meister der Zaubertränke leise, "dass du dich erklärst, Potter."

"Ich kann mich nicht erklären", sagte Harry. "Ich habe nicht die Zeit dazu, noch nicht." Harry schaute dem Meister der Zaubertränke direkt in die Augen, als er die Worte "\emph{ich}" und "\emph{Zeit}" sagte, weitete seine eigenen Augen, um zu versuchen, die Schlüsselinformation zu vermitteln, und der Meister der Zaubertränke zögerte.

Harry versuchte krampfhaft herauszufinden, wer was vorgab zu sein. Da Professor Quirrell nicht in Dumbledores Verschwörung eingeweiht war, gab Severus vor, der böse Zaubertränkemeister von Hogwarts zu sein, der vom Schulleiter hierher geschickt worden war… oder vielleicht, vielleicht auch nicht, von Dumbledore hierher geschickt worden war… aber Professor Quirrell dachte entweder, oder gab vor zu denken, dass jemand ein Auge auf Professor Snape werfen müsse… und Harry selbst war von Zukunfts-Harry hierher geschickt worden und hatte keine Ahnung, warum… und warum standen sie alle überhaupt vor der verbotenen Tür des Schulleiters? Und dann… Von dort, wo Harry stand… kam das wachsende Geräusch einer weiteren Reihe von Schritten, schnell und vielschichtig. Professor Snape stach einmal mit seinem Zauberstab zu und erzeugte einen Schwall von Dunkelheit, der die Stelle, an der der Verteidigungsprofessor lag, einhüllte.

"Muffliato", zischte der Meister der Zaubertränke. "Mr. Potter, wenn Sie hier sein müssen, dann verstecken Sie sich! Ziehen Sie Ihren Unsichtbarkeitsumhang an! Meine Pflicht ist es, diese Tür zu bewachen, falls er herkommt. Und es hat - eine Störung gegeben, die den Schulleiter anlocken sollte, er denkt -"

"Wer -"

Severus machte einen langen Schritt nach vorne und schnippte seinen Zauberstab gegen die Seite von Harrys Kopf. Es gab ein tröpfelndes Gefühl, als hätte man ein Ei über ihm aufgeschlagen, das Gefühl eines Desillusionierungszaubers; und Harrys Hände verschwanden, gefolgt vom Rest von ihm. Die Dunkelheit, die eine Seite der Wand umhüllte, löste sich wie langsamer Nebel auf, und es war wieder die zusammengekauerte Gestalt des Verteidigungsprofessors zu sehen, der nichts sagte.

Harry schlich sich so leise wie möglich davon, dann drehte er sich um und sah zu. Die sich nähernden Schritte bogen um die Ecke -

"Was machen Sie hier?", kamen viele gleichzeitige Rufe.

In drei Slytherin-Grün und einem Hufflepuff-Gelb gekleidet standen Theodore Nott, Daphne Greengrass, Susan Bones und Tracey Davis.

"Wo", sagte Professor Snape mit wachsendem Zorn, "sind eure Begleitpersonen, Kinder? Erstklässler müssen immer von einem Schüler des sechsten oder siebten Jahres begleitet werden! Besonders ihr!"

Theodore Nott hob die Hand. "Wir sind, ähm", sagte Theodore Nott.

"Wir machen das, was die Chaoslegion eine teambildende Übung nennt … sehen Sie, uns ist gerade aufgefallen, dass noch keiner von uns die verbotene Kammer des Schulleiters ausprobiert hat, und es war nicht mehr viel Zeit übrig … und Harry Potter hat es genehmigt, Professor, er hat ausdrücklich gesagt, dass Sie sich nicht einmischen dürfen."

Severus Snape drehte sich um und blickte hinüber zu der Stelle, wo Harry Potter auf Zehenspitzen gegangen war; auf seiner Stirn schien sich ein Sturm zu sammeln, und in seinen Augen lag eine dunkle Wut.

\emph{Ich … vielleicht?}

Es war noch eine Stunde auf Harrys Zeitumkehrer übrig, also war es möglich.

"Harry Potter hat diese Befugnis nicht", sagte der Meister der Zaubertränke in einem trügerisch milden Ton. "Erklären Sie sich jetzt."

"Wirklich?", sagte die Gestalt von Susan Bones. "Wirklich? Du willst Professor Snape erzählen, dass Harry Potter die Mission autorisiert hat, das ist deine Vorstellung von einem Bluff?" Die junge Hufflepuff wandte sich an Professor Snape und sprach, ihre Stimme war seltsam fest.

"Professor, das ist die Wahrheit und es ist dringend. Draco Malfoy wird vermisst und wir glauben, dass er dort hinuntergegangen ist -"

"Wenn Mr. Malfoy vermisst wird", sagte Professor Snape, "warum sind die Auroren nicht benachrichtigt worden?"

"Wegen, wegen der Gründe!?", rief Daphne Greengrass. "Wir haben keine Zeit, du musst uns durchlassen!"

Professor Snapes Stimme war jetzt so sardonisch, wie Harry sie noch nie gehört hatte.

"Habt ihr vier Schwachköpfe den Eindruck, dass ihr euch auf irgendeinem Abenteuer befindet? Nun, da irrt ihr euch. Ich versichere euch, dass Mr. Malfoy nicht durch diese Tür gegangen ist."

"Wir glauben, Mr. Malfoy hat einen Unsichtbarkeitsumhang", sagte Susan Bones schnell. "Erinnern Sie sich, dass sich die Tür scheinbar ohne Grund geöffnet hat?"

"Nein", sagte der Zaubertränkemeister. "Und jetzt verschwindet von hier. Dieser Ort ist für heute tabu."

"Das ist der von Dumbledore verbotene Korridor", sagte Tracey. "Der Schulleiter selbst hat gesagt, dass niemand hierher kommen darf. Was glauben Sie, wer Sie sind, dass Sie das auch verbieten?"

"Miss Davis", sagte der Zaubertränkemeister, "Sie müssen aufhören, sich mit Gryffindors zu umgeben, besonders mit denen, die Lavender Brown heißen. Und wenn Sie in einer Minute immer noch hier sind, werde ich Papiere einreichen, um Ihre Versetzung in dieses Haus zu beantragen."

"Das würdest du nicht wagen!", kreischte Tracey.

"Hm", sagte Susan Bones, ihr Gesicht vor Konzentration verzogen. "Professor Snape, öffnen Sie gelegentlich selbst die Tür, um nachzusehen, was sich darin befindet?"

Professor Snape erstarrte auf der Stelle. Dann drehte er sich und legte seine rechte Hand auf den metallenen Türklopfer - Harry beobachtete die Hand auf dem Türklopfer, so dass er nicht bemerkte, was Professor Snape mit seiner linken Hand tat, bis er den plötzlichen Aufschrei hörte.

"Nein, in der Tat", sagte Professor Snape, der nun den erstickten Kopf von Draco Malfoy am Kragen festhielt, obwohl der Rest von Draco immer noch unter seinem Unsichtbarkeitsumhang steckte. "Trotzdem ein guter Versuch."

"Was?!", riefen Tracey und Daphne.

Susan Bones schlug sich an die Stirn.

"Ich kann nicht glauben, dass ich darauf reingefallen bin."

"Also, Mr. Malfoy", sagte Professor Snape. Seine Stimme hatte sich gesenkt.

"Sie haben Ihre Freunde mit einer List hierher geschickt… nur in der Hoffnung, dass Sie durch diese Tür gehen können? Warum sollten Sie das tun?"

"Ich denke, wir sollten ihm vertrauen", sagte Theodore Nott. "Mr. Malfoy, wir müssen ihm vertrauen, er ist der einzige Professor, der auf unserer Seite stehen würde!"

"Nein!", rief Dracos schwebender Kopf, von wo aus Professor Snape ihn immer noch am Kragen festhielt. "Du darfst nichts sagen! Halt!"

"Wir müssen das Risiko eingehen!", rief Theodore. "Professor Snape, Mr. Malfoy hat endlich herausgefunden, was das ganze Jahr über los war und warum - Dumbledore versucht, den Stein der Weisen von Nicholas Flamel zu stehlen! Denn Dumbledore ist der Meinung, dass niemand Unsterblichkeit haben sollte! Also hat Dumbledore versucht, Flamel davon zu überzeugen, dass der Dunkle Lord zurückkommt und den Stein braucht, um wiederzubeleben, und hat Flamel gebeten, ihn ihm zu geben, aber Flamel wollte nicht, und stattdessen hat Flamel den Stein in den Zauberspiegel gelegt, der da unten ist, und Dumbledore findet gerade heraus, wie er ihn bekommen kann, und dann wird er ihn holen, und wir müssen ihn zuerst kriegen! Dumbledore wird wirklich allmächtig sein, wenn er den Stein der Weisen bekommt!"

"Was?", sagte Tracey. "Das ist nicht das, was du vorhin gesagt hast!"

"Es -" sagte Daphne. Sie sah erschrocken, aber entschlossen aus. "Es spielt keine Rolle - Professor Snape, bitte, Sie müssen mir glauben. Ich habe mir die Bücher angesehen, die Hermine in der Bibliothek ausgeliehen hat, und sie hat über den Stein der Weisen geforscht, kurz bevor jemand sie umgebracht hat. In ihren Notizen stand, dass etwas Gefährliches passieren kann, wenn der Stein zu lange im Spiegel bleibt.

Wir müssen ihn sofort aus dem Schloss holen."

Susan Bones hatte nun beide Hände über ihrem Gesicht. "Ich gehöre nicht zu ihnen, ich bin nur mitgekommen, um zu verhindern, dass etwas noch Dümmeres passiert."

Severus Snape starrte auf Theodore Nott und die anderen. Dann drehte er den Kopf und sah Draco Malfoy an. "Mr. Malfoy", murmelte der Zaubertränkemeister.

"Wie sind Sie auf Dumbledores Komplott gekommen?"

"Ich habe es aus Beweisen abgeleitet!", sagte Draco Malfoys schwebender Kopf.

Professor Snapes Kopf schwenkte zurück zu Theodore Nott.

"Wie wollten Sie den Stein aus dem Inneren eines magischen Spiegels beschaffen, der angeblich selbst Dumbledore verblüffen konnte? Antworten Sie mir sofort!"

"Wir werden den ganzen Spiegel nehmen und ihn zu Flamel zurückschicken", sagte Theodore Nott. "Es ist ja nicht so, dass wir den Stein für uns selbst wollen, wir müssen nur Dumbledore davon abhalten, ihn zu stehlen."

Professor Snape nickte, als ob er etwas bestätigen wollte, und drehte seinen Kopf, um die anderen Schüler anzusehen.

"Sagt mir, hat einer von euch bemerkt, dass sich einer der anderen ungewöhnlich verhält? Besonders, wenn sie einen merkwürdigen Gegenstand in ihrem Besitz haben oder Zaubersprüche benutzen können, die ein Erstklässler nicht kennen sollte?"

Professor Snapes rechte Hand richtete nun seinen Zauberstab auf Susan Bones.

"Ich sehe, dass Miss Greengrass und Miss Davis versuchen, Sie nicht anzuschauen, Miss Bones. Wenn es eine weltliche Erklärung gibt, wäre es klug von Ihnen, diese sofort anzubieten."

Susan Bones' Haare färbten sich knallrot, aber ihr Gesicht veränderte sich nicht.

"Ich nehme an, es hat keinen Sinn, es noch länger zu verschweigen", sagte sie, "da ich sowieso in zwei Tagen meinen Abschluss mache."

"Doppelhexen dürfen sechs Jahre früher ihren Abschluss machen?", sagte Tracey Davis. "Das ist nicht fair!"

"Bones ist eine Doppelhexe?", rief Theodore.

"Nein, sie ist Nymphadora Tonks, ein Metamorphmagus", sagte Professor Snape.

"Sich als eine andere Schülerin auszugeben, ist äußerst vorschriftswidrig, wie Sie sicher wissen, Miss Tonks.

Es ist noch nicht zu spät, Sie zwei Tage vor Ihrem Abschluss von Hogwarts zu verweisen, was eine furchtbare Tragödie wäre - aus Ihrer Perspektive, meine ich.

Aus meiner Perspektive wäre es urkomisch. Und jetzt sag mir, was genau du hier tust."

"Das erklärt es", sagte Daphne Greengrass.

"Ähm, gibt es eigentlich eine Susan Bones, oder ist das Haus am Aussterben, sodass man Sie heimlich -"

Die rothaarige Gestalt von Susan Bones schlug eine Handfläche vor ihr Gesicht.

"Ja, Miss Greengrass, es gibt eine echte Susan Bones. Sie schickt mich nur rein, wenn ihr in lächerliche Schwierigkeiten geratet.

Professor Snape, ich bin hier, weil Draco Malfoy vermisst wird und diese Leute darauf bestanden haben, ihn zu suchen, anstatt die Auroren zu rufen.

Aus Gründen, von denen die echte Miss Bones sagte, sie hätten keine Zeit, sie mir zu erklären. Aber junge Schüler dürfen nie allein gehen, sie müssen immer von einem Sechst- oder Siebtklässler begleitet werden. Und jetzt haben wir Draco Malfoy gefunden. Wir können alle zurückgehen. Bitte? Bevor das hier noch lächerlicher wird?"

"Was in Merlins Namen ist hier los?"

"Ah", sagte Professor Snape, der immer noch den Zauberstab auf die rothaarige Gestalt von Susan Bones richtete, während seine andere Hand immer noch den Kragen unterhalb des körperlosen Kopfes von Draco Malfoy umklammerte, der neben der zerknitterten Gestalt des Verteidigungsprofessors stand.

"Professor Sprout. Was machen Sie hier?"

"Es ist nicht das, wonach es aussieht", meldete sich Tracey Davis.

Die kleine, plumpe Gestalt der Kräuterkundeprofessorin stürmte nach vorne. Sie hatte inzwischen ihren Zauberstab gezogen, obwohl sie ihn nicht auf jemanden richtete.

"Ich weiß nicht einmal, wonach das aussieht! Runter mit den Zauberstäben, ihr alle, sofort! Auch Sie, Professor!"

\emph{Ablenkung.}

Der Gedanke kam Harry mit plötzlicher Klarheit. Was immer er jetzt beobachtete, von dort, wo er unsichtbar und weit hinten im Geschehen stand, es war nicht das, was wirklich vor sich ging, es war nicht der wahre Faden der Geschichte, es war arrangiert worden. Die Ankunft von Professor Sprout hatte Harrys Unglaubwürdigkeit gebrochen; solche Dinge passierten nicht einfach aus komödiantischen Zufällen heraus.

\emph{Jemand verursacht absichtlich dieses ganze Chaos, aber was sollte das bringen?}

Harry hoffte wirklich, dass er nicht in der Zeit zurückgereist war und das getan hatte, denn es schien so, als ob \emph{er} so etwas tun würde.

Severus Snape ließ seinen Zauberstab sinken. Seine andere Hand löste Draco Malfoys Faust. "Professor Sprout", sagte der Zaubertrankmeister, "ich bin auf Anweisung des Schulleiters hier, um diese Tür zu bewachen. Alle anderen Anwesenden haben hier nichts zu suchen, und ich bitte Sie, dafür zu sorgen, dass sie sich entfernen."

"Eine unwahrscheinliche Geschichte", schnauzte Professor Sprout. "Warum sollte Dumbledore ausgerechnet Sie damit beauftragen, die Tür zu seinem Spielplatz zu bewachen? Es ist ja nicht so, dass er die Schüler draußen halten will, oh nein, sie sollen reingehen und sich in meiner Teufelsschlinge verfangen! Susan, Liebes, du hast doch einen Kommunikationsspiegel, oder? Benutze ihn, um die Auroren zu rufen."

Der beobachtende Harry nickte vor sich hin. Das war der Punkt. Die Auroren würden alle Anwesenden in dieser furchtbar verwirrenden Situation mitnehmen, keine Ausreden akzeptieren, und dann würde die Tür unbewacht sein. Aber sollte Harry selbst in den verbotenen Korridor gehen? Oder beobachten, um zu sehen, wer schließlich kam, wenn alle anderen weg waren? Ein lauter Hustenanfall ließ alle auf die Stelle blicken, wo der Verteidigungsprofessor lag.

"Snape - hören Sie -", sagte der Verteidigungsprofessor zwischen Hüsteln. "Warum - Sprout - hier -"

Der Meister der Zaubertränke sah zu Boden.

"Gedächtniszauber - impliziert - Professor -"

Der Verteidigungsprofessor begann wieder zu husten.

"Was?!"

Und die Logik entfaltete sich in Harrys Verstand in kristalliner Bestürzung, alle Schritte bereits vermutet, die schreckliche Erkenntnis kam als Wiederholung mit größerer Sicherheit. Jemand hatte Hermine einen Gedächtniszauber verpasst, damit sie glaubte, sie hätte versucht, Draco zu töten.

Nur ein Hogwarts-Professor hätte das unbemerkt tun können. Der wahre Drahtzieher brauchte also nur einen Hogwarts-Professor mit Legilimation oder Imperius zu versehen. Und der Letzte, den man verdächtigen würde, wäre der Leiter des Hauses Hufflepuff.

Snapes Kopf ruckte herum, als Professor Sprout ihren Zauberstab hob, und der Meister der Zaubertränke schaffte es, einen wortlosen durchsichtigen Schutzwall zwischen ihnen zu errichten. Doch der Blitz, der aus Professor Sprouts Zauberstab schoss, war dunkelbraun und löste in Harrys Geist eine Welle schrecklicher Angst aus; und der braune Blitz ließ Severus' Schild ausfahren, bevor sie sich berührten, und traf den rechten Arm des Zaubertrankmeisters, noch während er ausweichen konnte.

Professor Snape stieß einen dumpfen Schrei aus, seine Hand verkrampfte sich und ließ seinen Zauberstab fallen. Der nächste Blitz, der aus Sprouts Zauberstab kam, war ein leuchtendes Rot von der Farbe eines Betäubers und schien noch heller zu werden und sich schneller zu bewegen, während er ihren Zauberstab verließ, begleitet von einer weiteren Welle der Angst; und das schleuderte den Meister der Zaubertränke gegen die Tür und ließ ihn regungslos zu Boden fallen.

Zu diesem Zeitpunkt war die pinkhaarige Susan-Bones von einem facettenreichen blauen Dunst umgeben, und sie feuerte einen Fluch nach der anderen auf Professor Sprout ab. Professor Sprout ignorierte die Flüche, um Pflanzenranken zu beschwören, die die jüngeren Schüler verstrickten, als sie versuchten zu rennen, außer Draco Malfoy, der wieder unter seinem Unsichtbarkeitsumhang verschwunden war.

Nicht-Susan-Bones hörte auf zu zaubern. Sie richtete ihren Zauberstab auf, holte tief Luft und rief laut eine Beschwörungsformel, die goldene Lichtwürmer aussandte, die sich in den Schild um Professor Sprout fraßen. Daraufhin drehte sich die Kräuterkunde-Professorin zu Nicht-Susan um, ihr Gesichtsausdruck war leer, und hinter ihr erhoben sich neue Pflanzententakel in die Luft. Diese Stängel waren von einem dunkleren Grün und schienen ihre eigenen Schilde zu haben.

Harry Potter murmelte in die scheinbar leere Luft: "Greif Sprout an. Hilf Bones. Nur nicht tödlicher Eisnatz."

"Ja, mein Herr", flüsterte Lesath Lestrange unter Harrys Unsichtbarkeitsumhang, und die Präsenz des Slytherins im fünften Jahr bewegte sich auf den Kampf zu.

Harry blickte auf seine eigenen Hände hinunter und sah mit einem unangenehmen Schreck, dass sein Desillusionierungszauber nicht mehr so vollständig war wie zuvor.

Bei jeder Bewegung, die Harry machte, lag ein Hauch von Verzerrung in der Luft… Langsam ging Harry rückwärts, bis er zu einer Ecke kam und sich hinter einer Wand duckte. Dann holte er seinen Kommunikationsspiegel heraus… der leer und gestört war. \emph{Ja, natürlich}. Harry ließ den Spiegel schweben, damit er um die Ecke sehen konnte und das Ende der… \emph{Ablenkung} sehen konnte.

\emph{Was war los, warum?}

Professor Sprout und die Gestalt von Susan Bones duellierten sich mit Lichtblitzen und lebenden Blättern; und das lodernde Grün eines Großen Schildbrechers brach aus der Luft hervor und fraß sich halb durch die äußere Schicht von Professor Sprouts Schilden. Die Kräuterkundeprofessorin drehte sich um und feuerte einen breiten gelben Strahl auf die Stelle, von der der Brecher gekommen war, aber der Zauber schien nichts zu treffen. Gelbe Flammen, blaue Facetten, dunkelgrüne Pflanzenranken und wirbelnde lila Blütenblätter… Erst als Professor Sprout anfing, Bögen aus Karmesin in alle Richtungen zu feuern, fing eine der karmesinroten Klingen etwas in der Luft auf, wobei der Unsichtbarkeitsmantel nicht verbarg, wie der karmesinrote Bogen absorbiert wurde und ausblinzelte; und Lesaths Präsenz unter dem Unsichtbarkeitsmantel fiel zu Boden. Und das gab \emph{Nicht}-Susan-Bones Zeit genug, um stehen zu bleiben, Luft zu holen und etwas zu schreien, das in Harry eine weitere Woge des Schreckens auslöste; und der weiße Funke, der aufloderte, ging durch Professor Sprouts angeknackste Schilde und ihre Pflanzenrüstung und ließ sie fallen. \emph{Nicht}-Susan-Bones ging auf die Knie, keuchte, ihre Roben waren schweißgetränkt. Sie drehte den Kopf, um sich umzusehen, auf die Körper, die betäubt auf dem Boden lagen oder in Ranken eingewickelt waren.

"Was", sagte \emph{Nicht}-Susan. "Was? Was? Was?"

Es gab keine Antwort. Die Opfer, die sich in Professor Sprouts Lianen verfangen hatten, bewegten sich nicht, obwohl sie zu atmen schienen.

"Malfoy…", sagte die rosahaarige Gestalt von Susan, die immer noch nach Atem rang.

"Draco Malfoy, wo bist du? Bist du noch da? Ruf schon die Auroren! Merlin verdammt - Homenum Revelio!?"

Und Harry fand sich wieder sichtbar und starrte in seinem Spiegel auf die Gestalt von Draco Malfoy, der halb sichtbar unter einem schimmernden Umhang hinter Nicht-Susan stand und seinen Zauberstab auf eine Lücke in Nicht-Susans blauem Dunst richtete. Harrys Gedanken bewegten sich in Blitzen der Erkenntnis, zu langsam und doch zu schnell; selbst als Harrys Mund sich öffnete und er einatmete, um sich auf den Schrei vorzubereiten.

\emph{Hüte dich vor der Konstellation.}

\emph{Es gab eine Konstellation namens Draco, und wenn man einen Professor kontrollieren konnte, konnte man auch einen Schüler kontrollieren}

"Duck dich!" rief Harry, aber es war zu spät, ein Blitz aus rotem Licht traf aus nächster Nähe den Hinterkopf von Nicht-Susan und ließ sie zu Boden fallen.

Harry trat um die Ecke und sagte: "Somnium Somnium Somnium Somnium Somnium Somnium."

Draco Malfoys schimmernde Gestalt brach in einem Haufen zusammen.

Harry nahm sich einen Moment Zeit, um wieder zu Atem zu kommen. Dann sagte Harry: "Stupor!" und vergewisserte sich, dass, \emph{ja}, der Fluch tatsächlich Draco Malfoys Gestalt getroffen hatte.

(Man konnte sich irren, ob ein Somnium wirklich getroffen hatte. Harry hatte genug Horrorfilme gesehen, ganz zu schweigen von der Sache mit dem Sonnenschein Regiment, dass er diesen Fehler nicht noch einmal machen wollte.)

Nach einer weiteren Überlegung zauberte Harry eine weitere Betäubung auf die am Boden liegende Gestalt von Professor Sprout. Harry umklammerte seinen Zauberstab, starrte auf die Szene und atmete schwer vor Erschöpfung. Er hatte nicht mehr genug Magie übrig, um einen Boten-Patronus zu Dumbledore zu zaubern, und er hätte dieses Mal wirklich sofort an diese Möglichkeit denken sollen. Harry begann, dorthin zu greifen, wo sein Spiegel hingefallen war, um zu sehen, ob er nun nicht mehr gestört war. Und dann zögerte Harry. Seine Notiz an sich selbst hatte besagt, dass er sich vor den Auroren in Acht nehmen sollte, und Harry wusste immer noch nicht, was vor sich ging.

Die zerknitterte Gestalt von Professor Quirrell gab eine weitere Serie von quälenden Hustenstößen von sich, streckte eine Hand nach der Wand neben sich aus und zog sich langsam auf die Beine.

"Harry", krächzte Professor Quirrell. "Harry. Bist du da?"

Es war das erste Mal, dass Professor Quirrell ihn bei seinem Vornamen nannte.

"Ich bin hier", sagte Harry. Ohne einen bewussten Gedanken bewegten sich seine Füße vorwärts.

"Bitte", sagte Professor Quirrell. "Bitte, ich habe … nicht viel Zeit. Bitte bring mich … zum Spiegel … helf mir … den Stein zu holen."

"Den Stein der Weisen?" sagte Harry.

Er blickte sich bei den verstreuten Leichen um, aber er konnte Draco nicht mehr sehen, die Enthüllung war abgeklungen. "Glaubst du, Mr. Nott hatte recht? Ich glaube nicht, dass Dumbledore -"

"Nicht - Dumbledore", keuchte Professor Quirrell. "Weil - Sprout -"

"Ich verstehe", sagte Harry.

\emph{Wenn Dumbledore derjenige gewesen wäre, der hinter all dem steckte, hätte er es nicht nötig gehabt, einen Professor gedanklich zu kontrollieren, um Gedächtniszauber einzusetzen.}

"Spiegel… uraltes Relikt… könnte alles verbergen… Der Stein könnte da sein…. viele andere wollen den Stein… einer hat Sprout geschickt…"

Harry wiederholte schnell:

"Der Spiegel da unten ist ein uraltes Relikt, das man benutzen kann, um Dinge zu verstecken, und er wäre ein möglicher Ort, um den Stein der Weisen zu verstecken.

Wenn sich der Stein der Weisen in dem Spiegel befindet, dann könnten viele Leute daran interessiert sein, ihn zu bekommen. Einer von ihnen kontrolliert Sprout und das würde erklären, was ihr Ziel ist … nur … das erklärt nicht, warum Sprouts Kontrolleur hinter Hermine her war."

"Harry, bitte", sagte Professor Quirrell.

Seine Atmung war jetzt noch schwerer, seine Stimme kam mit quälender Langsamkeit.

"Es ist das Einzige… das mein Leben retten kann… und ich finde, jetzt… Ich will nicht sterben… bitte, hilf mir…"

Und irgendwie war das genug. Irgendwie war das ein bisschen zu viel. Das Gefühl der Traumartigkeit, das Harry überkommen hatte, als Professor Sprout gekommen war, die Ungläubigkeit, kehrte zurück; sein innerer Kritiker wog alles ab, als wäre es ein Theaterstück.

\emph{Das Timing, die Wahrscheinlichkeit, dass so viele Leute an derselben Tür auftauchen, die Verzweiflung des Verteidigungsprofessors}

… die ganze Situation fühlte sich nicht real an. Aber er könnte sie vielleicht lösen, wenn er sich einfach die Zeit nehmen würde, die Dinge im Voraus zu durchdenken, anstatt beim ersten Anruf des Abenteuers loszurennen. All die gesammelten Erfahrungen des letzten Jahres hatten sich schließlich zu so etwas wie einem Hauch von Kampfhärte herauskristallisiert. Ein Instinkt, der aus vergangenen Katastrophen geboren wurde, sagte Harry, dass er, wenn er einfach losstürmte, hinterher in einem traurigen Gespräch enden würde, in dem er feststellte, dass er dumm gewesen war.

\emph{Schon wieder.}

"Lass mich nachdenken", sagte Harry. "Lass Sie mich noch eine Minute nachdenken, bevor wir gehen."

Er wandte sich von dem Verteidigungsprofessor ab und betrachtete die bewusstlosen Körper, die in verschiedenen Formen auf dem Boden drapiert waren. Es waren schon so viele Puzzleteile gewesen, dieses letzte Jahr, vielleicht würde sich alles mit einem weiteren Teil zusammenfügen…

"Harry…", sagte der Verteidigungsprofessor mit stockender Stimme. "Harry, ich sterbe…"

\emph{Eine Minute mehr kann nicht den Unterschied ausmachen, er hatte das GANZE Jahr Zeit, um krank zu sein, es ist UNMÖGLICH, dass sein Leben gegen den Tod genau auf diese letzte Minute abgestimmt sein würde, egal was mit Hermine passiert -}

"Ich weiß!" sagte Harry. "Ich werde schnell nachdenken!"

Harry starrte auf die Leichen und versuchte zu denken. Es gab keine Zeit für Zweifel, für Vorbehalte, keine Bremsen oder Bedenken, einfach die ersten Gedanken nehmen und mit ihnen laufen - In Harrys Hinterkopf huschten Fragmente abstrakter Gedanken vorbei, Heuristiken der Problemlösung, für die keine Zeit war, sie in Worten zu proben. In wortlosen Blitzen schossen sie vorbei, um das Problem auf der Objektebene aufzustellen. -

\emph{Was fällt mir auf, was mich verwirrt -}

- der erste Ort, an dem man nach einem Problem sucht, ist derjenige Aspekt der Situation, der am unwahrscheinlichsten erscheint

- einfache Erklärungen sind wahrscheinlicher, eliminiere separate Unwahrscheinlichkeiten, die postuliert werden müssen

- Professor Snape war schon hier gewesen, dann war Professor Quirrell angekommen, dann war Harry angekommen (via Zeitdreher), dann war die Abenteurergruppe angekommen und Draco war aufgetaucht (Teil der Gruppe), dann war Professor Sprout aufgetaucht.

Zu viele Leute waren synchron aufgetaucht und das war zu viel Zufall, es war unwahrscheinlich, dass so viele verschiedene Parteien innerhalb eines Fünf-Minuten-Fensters am selben Ort auftauchen würden, es musste versteckte Verwicklungen geben.

\emph{Halte Sprouts Kontrolleur für das Superhirn, das Hermine mit dem Gedächtniszauber beauftragt hatte. Das Superhirn hatte Sprout geschickt.

Professor Snape hatte gesagt, dass der Schulleiter ihn geschickt hatte, um die Tür zu bewachen, nachdem es irgendeine Art von Störung gegeben hatte, wenn das Superhirn das als Ablenkung verursacht hatte, dann erklärte das auch Severus' Anwesenheit.}

Harry war sich nicht mehr sicher, dass Draco vom Superhirn gesteuert worden war, diese Hypothese war ihm spontan eingefallen, Draco könnte nur versucht haben, Nicht-Susan zu manipulieren damit er ungehindert in den Korridor gelangen konnte-

\emph{Nein, das war der falsche Weg zu denken, es umzudrehen, zu versuchen, die zeitliche Anwesenheit von Draco und seiner Abenteurergruppe zu erklären, keine Zeit für Selbstbefragung, mit der Hypothese loslegen, also sollte er annehmen, dass Sprouts Superhirn Draco geschickt oder sein Kommen ausgelöst hatte.}

Damit waren drei Ankünfte erklärt.

\emph{\hfill\break Harry war aufgetaucht, weil seine Notiz an sich selbst ihm dies befohlen hatte.

Das könnte man auf Zeitreisen zurückführen. Damit blieb nur noch der Verteidigungsprofessor übrig, der gesagt hatte, er würde Snape folgen, aber das schien nicht wirklich ein angemessener Grund für das Auftauchen von Professor Quirrell zu sein, was Harry nicht wirklich weniger verwirrte und so hatte das Superhirn vielleicht auch irgendwie den Zeitpunkt von Professor Quirrells Anwesenheit kontrolliert und sogar dafür gesorgt, dass Harry selbst in die} \emph{Zeitschleife eintrat.

}\strut 

Harrys Verstand stieß auf einen Stolperstein, er konnte nicht erkennen, wie er diese Überlegung weiterführen sollte. Aber es gab keine Zeit, um stumpfsinnig auf Stolpersteine zu starren. Ohne eine Pause oder ein Abbremsen griff Harrys Verstand das Problem aus einem neuen Blickwinkel an.

\emph{Professor Quirrell hatte einen kontrollierten Hogwarts-Professor aus der Notwendigkeit abgeleitet, Hermine mit einem Gedächtniszauber zu versehen, was bedeutete, dass der Kontrolleur von Professor Sprout Hermine reingelegt und dann ermordet hatte, was wiederum bedeutete, dass der Kontrolleur von Professor Sprout detaillierte Informationen über das Leben in Hogwarts hatte und vielleicht ein persönliches Interesse an dem Jungen-der-lebte und seinen Freunden hatte.

Dumbledore hatte gesagt, dass Lord Voldemorts stärkster Weg zum Leben hier in Hogwarts versteckt war, also war das Auferstehungswerkzeug der Stein der Weisen, der im Spiegel versteckt war, aber warum hatte Dumbledore den Spiegel in einen Korridor gestellt, durch den Erstklässler hindurchgehen konnten, nein, ignoriere diese Frage, sie ist jetzt nicht wichtig, und Professor Quirrell hatte gesagt, dass der Stein der Weisen große Heilkräfte besaß, also passte auch dieser Teil.}

\emph{Aber wenn es der Stein der Weisen war, der im Spiegel versteckt war, um ihn vom Dunklen Lord fernzuhalten, bedeutete das, dass der Spiegel auch das Einzige auf der Welt enthielt, das das Leben des Verteidigungsprofessors retten konnte} -

Harrys Verstand versuchte zu zögern, auszuweichen, er spürte eine plötzliche Befürchtung, wohin das führen würde. Aber für ein Zögern war keine Zeit. -

\emph{und das war auch viel zu viel Zufall, einfach zu viel Unwahrscheinlichkeit, wenn der Verstand es nicht als verblüffenden Plot Twist abtat, als wäre man in einer Geschichte. Könnte der vermeintliche Dunkle Lord auch Professor Quirrell manipulieren, so dass Professor Quirrell seine vermeintliche Rettung zum richtigen Zeitpunkt entdeckt, so dass Harry und Professor Quirrell das Auferstehungswerkzeug aus dem Spiegel holen, das vielleicht gar nicht der Stein der Weisen ist, und dann würde der Avatar des Dunklen Lords oder ein anderer Diener auftauchen und es ihnen wegnehmen, was all die Synchronitäten erklären und jeden Zufall negieren würde.}

Oder Professor Quirrell hatte von Anfang an gewusst, dass das eine Ding, das sein Leben retten konnte, in diesem Spiegel versteckt war und deshalb hatte er zugestimmt, Verteidigung in Hogwarts zu unterrichten und jetzt versuchte er endlich, ihn zu bekommen, aber warum dann warten, bis er so krank war, um es überhaupt zu versuchen und warum war Sprout zur gleichen Zeit wie Professor Quirrell aufgetaucht -

Harrys Gedanken stockten noch mehr. Sein inneres Auge schaute in eine Richtung, in die es sich normalerweise nicht zu schauen traute.

\emph{In der Notiz, die ich mir selbst geschickt habe, stand, dass ich dem Wächter der Sterne helfen soll. Ich würde mir selbst keine Notiz schicken, wenn ich nicht schon in der Zukunft herausgefunden hätte, dass es das Richtige war - vielleicht sagt mir die Notiz nur, dass ich weitermachen soll -}

Ein kleiner Anflug von Verwirrung wurde zu bewusster Aufmerksamkeit befördert.

\emph{Die verschlüsselte Nachricht auf dem Pergament… ein oder zwei Zeilen hatten nicht ganz richtig geklungen, hatten nicht wie der Code geklungen, den Harry selbst erwartet hätte…}

"Harry", flüsterte die sterbende Stimme von Professor Quirrell hinter ihm. "Harry, bitte."

"Ich bin fast fertig mit dem Denken", sagte Harrys Stimme laut, und Harry erkannte, als er die Worte sprach, dass sie wahr waren.

\emph{Dreh es um. Betrachte es aus der Perspektive des Feindes, von dort aus, wo der Feind seine eigenen intelligenten Planungen anstellt, irgendwo außerhalb deiner Sichtweite.}

Es gibt Auroren in Hogwarts, und dein Ziel Harry Potter ist jetzt voll auf der Hut. Harry Potter wird beim ersten Anzeichen von Ärger Auroren herbeirufen oder einen Patronus an Albus Dumbledore schicken. Wenn man das als Rätsel betrachtet, ist eine kreative Lösung, dass man -

- eine vermeintlich zeitumgekehrte Nachricht an Harry Potter von ihm selbst fälscht, in der er Harry Potter sagt, dass er nicht um Hilfe rufen soll, und ihm sagt, dass er an dem Ort und zu der Zeit sein soll, zu der du ihn haben willst. Du bringst das Ziel selbst dazu, alle von ihm eingerichteten Schutzmechanismen zu umgehen. Man umgeht sogar seinen Schutz der Skepsis mit der übergeordneten Autorität des Urteils seines eigenen zukünftigen Ichs. Es ist nicht einmal schwierig. Man kann einen beliebigen Schüler mit einem Gedächtniszauber dazu bringen, sich daran zu erinnern, dass Harry Potter ihm einen Umschlag überreicht hat, den er später selbst zurückgeben kann.

\emph{\hfill\break Du kannst diesen Schüler unter Gedächtniszauber setzen, weil du ein Hogwarts-Professor bist.}

Du machst dir nicht die Mühe, einen Bleistift und Muggelpapier aus Harry Potters Tasche zu stehlen. Stattdessen fälschst du Harry Potters Handschrift auf Zauberer-Pergament.

\emph{Du kannst Harry Potters Handschrift fälschen, weil du sie auf vom Ministerium angeordneten Prüfungen gesehen hast, die du benotet hast.}

Du nennst Draco Malfoy "die Konstellation", weil du weißt, dass Harry Potter sich für Astronomie interessiert und du bist ein Zauberer und hast Astronomie belegt und die Namen aller Sternbilder auswendig gelernt. Aber es ist nicht der natürliche Code, den Harry Potter verwenden würde, um Draco Malfoy vor sich selbst zu beschreiben, das wäre 'der Lehrling' gewesen.

Du nennst Professor Quirrell 'den Wächter der Sterne' und sagst Harry Potter, er solle ihm helfen.

Du weißt, dass man 'Lebensfresser' in Parsel für 'Dementor' sagt, und du erwartest, dass Harry Potter die Auroren als mit ihnen im Bunde sieht.

Du verschlüsselst 6:49 als 'sechs und sieben im Quadrat', weil du ein Muggel-Physikbuch gelesen hast, das Harry Potter dir gegeben hat.

\textbf{\emph{Wer bist du?}}

Harry bemerkte, dass sich sein Atem beschleunigt hatte, und mit einem Herzschlag verlangsamte Harry seinen Atem wieder, Professor Quirrell beobachtete ihn.

\emph{\hfill\break Was wäre, wenn, hypothetisch gesprochen, Professor Quirrell das Superhirn war und Harrys Nachricht gefälscht hatte, dann erklärte das, dass alle fünf Parteien auftauchten, die ganze synchrone Koordination der Komödie und dann wurde Professor Sprout nur kontrolliert, um Professor Quirrell eine Abstreitbarkeit zu geben, damit er jemand anderem die Schuld für den Falsche-Erinnerung-Zauber geben konnte, nachdem sich der Staub gelegt hatte. Aber warum sollte Professor Quirrell die zerbrechliche Allianz riskieren, die Harry mit Draco über den Mordversuch hatte.}

Warum sollte Professor Quirrell Hermine töten (wenn sein erster Versuch, sie zu beseitigen, nicht funktioniert hat)?

Wenn Professor Quirrell der Bösewicht war, könnte er über alles gelogen haben, was mit dem Horcrux zu tun hatte und vielleicht war es überhaupt kein Zufall, dass das Einzige, was sein Leben retten konnte, der Weg war, der den Dunklen Lord wiederauferstehen lassen konnte - was, wenn der Dunkle Lord das auch irgendwie arrangiert hatte?

(Eines Tages war David Monroe auf mysteriöse Weise verschwunden, vermutlich tot durch die Hand des Dunklen Lords.)

Eine schreckliche Eingebung war über Harry gekommen, etwas, das unabhängig von allen Überlegungen war, die er bisher angestellt hatte, eine Eingebung, die Harry nicht in Worte fassen konnte;

Außer, dass er und der Verteidigungsprofessor sich in gewisser Hinsicht sehr ähnlich waren und dass das Vortäuschen einer Zeitumkehr-Botschaft genau die Art von kreativer Methode war, die Harry selbst versucht hätte, um alle Schutzvorkehrungen eines Ziels zu umgehen -

\textbf{und das war der Moment, in dem Harry endlich begriff, was von Anfang an hätte offensichtlich sein müssen.}

\emph{Professor Quirrell war klug.

Professor Quirrell war auf die gleiche Weise klug wie Harry.

Professor Quirrell war genauso klug wie Harrys mysteriöse dunkle Seite.

Wenn man raten müsste, wann der Junge-der-lebte seine mysteriöse dunkle Seite erworben hatte, wäre die naheliegende Vermutung die Nacht des 31. Oktober 1981.}

Und und und Professor Quirrell kannte ein Passwort, von dem Bellatrix Black dachte, dass es den Dunklen Lord identifizierte, und seine Anwesenheit gab dem Jungen-der-lebte ein Gefühl von Untergang und seine Magie interagierte zerstörerisch mit Harrys und sein Lieblingszauber war Avada Kedavra und und und -

Die Erkenntnis schoss durch Harry wie ein riesiger Damm, der brach, alles Wasser herausließ und in einer unwiderstehlichen Flut durch seinen Geist brach, die alles mit sich riss.

Es gibt nur eine Realität, die alle Beobachtungen hervorbringt.

Wenn verschiedene Beobachtungen in unvereinbare Richtungen zu weisen scheinen, bedeutet das, dass die wahre Hypothese eine ist, an die man noch nicht gedacht hat.

Und in solchen Fällen, wenn du endlich an die richtige Hypothese denkst, reiht sich alles hinter ihr auf, jenseits von Leugnung oder Entsetzen, und reißt jeden Zweifel und jede Emotion weg, die ihr im Weg stehen könnten.

\emph{- und dann waren 'David Monroe' und 'Lord Voldemort' nur eine Person gewesen, die auf beiden Seiten des Zaubererkrieges spielte, und das war der Grund, warum die Familie Monroe getötet worden war, bevor sie 'David Monroe' treffen konnte, genau wie Moody es schon vermutet hatte -}

die Realität pendelte sich auf einen einzigen bekannten Zustand ein, einen kohärenten Zustand, der kompakt den Beobachtungssatz erzeugte.

Harry sprang nicht, änderte seine Atmung nicht, versuchte, kein einziges Zeichen des Schreckens und der Agonie zu zeigen, die seinen Verstand überfluteten.

\textbf{\emph{Der Feind war hinter ihm und beobachtete ihn.}}

"In Ordnung", sagte Harry laut, sobald er es wagte, seiner Stimme zu trauen, normal zu klingen. Er starrte weiter auf die Leichen und wandte den Blick von Professor Quirrell ab, weil Harry seinem eigenen Gesicht nicht traute. Harry hob einen Ärmel, um den Schweiß auf seiner Stirn wegzuwischen, und versuchte, die Geste lässig aussehen zu lassen; Harry konnte weder den Schweiß noch das schnelle Hämmern in seiner Brust kontrollieren.

"Lass uns den Stein der Weisen holen."

Alles, was Harry brauchte, war ein einziger Moment der Ablenkung irgendwo auf dem Weg, um seinen Zeitumkehrer zu benutzen.

Hinter ihm ertönte keine Antwort.

Die Stille dehnte sich aus.

Langsam drehte sich Harry um.

Professor Quirrell stand aufrecht da und lächelte. In der Hand des Verteidigungsprofessors war eine Form aus schwarzem Metall, die auf Harrys Zauberstabarm gerichtet war, gehalten mit dem sicheren Griff von jemandem, der genau wusste, wie man eine Pistole benutzt.

Harrys Mund war trocken, sogar seine Lippen zitterten vor Adrenalin, aber er schaffte es zu sprechen.

"Hallo, Lord Voldemort."

Professor Quirrell neigte anerkennend den Kopf und sagte:

"Hallo, Tom Riddle."

