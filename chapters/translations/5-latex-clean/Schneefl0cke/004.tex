

\hypertarget{die-hypothese-des-effizienten-marktes}{% \section{5. Die Hypothese des effizienten Marktes}\label{die-hypothese-des-effizienten-marktes}}

Zitat des Authors E.Y.: Wie schon andere bemerkt haben, scheinen die Romane in Bezug auf die scheinbare Kaufkraft einer Galeone inkonsistent zu sein; ich habe mich für einen einheitlichen Wert entschieden und bleibe dabei. Fünf Pfund Sterling für eine Galleone passen nicht zu sieben Galleonen für einen Zauberstab und Kindern, die gebrauchte Zauberstäbe benutzen. ~

*****

„Weltherrschaft„ ist so ein hässlicher Ausdruck. Ich ziehe es vor, es Weltoptimierung zu nennen.“ ~

*****

Haufenweise goldene Galleonen. Stapel von silbernen Sickeln. Haufen von bronzenen Knuts.

Harry stand da und starrte mit offenem Mund auf das Familienverlies. Er hatte so viele Fragen, dass er nicht wusste, wo er anfangen sollte. Von der Tür des Tresors aus beobachtete ihn Professor McGonagall, die scheinbar lässig an der Wand lehnte, aber ihre Augen waren aufmerksam.

Nun, das machte Sinn. Vor einen riesigen Haufen Goldmünzen geworfen zu werden, war ein Charaktertest, der so rein war, dass er archetypisch war.

„Sind diese Münzen aus ~reinem Metall?“ sagte Harry schließlich.

„Was?“, zischte der Kobold Griphook, der in der Nähe der Tür wartete. „Stellen Sie die Integrität von Gringotts in Frage, Mr~Potter-Evans-Verres?“

„Nein“, sagte Harry abwesend, „ganz und gar nicht, tut mir leid, wenn das falsch rüberkam, Sir. Ich habe nur überhaupt keine Ahnung, wie Ihr Finanzsystem funktioniert. Ich frage nur, ob Galleonen im Allgemeinen aus reinem Gold bestehen.“

„Natürlich“, sagte Griphook.

„Und kann sie jeder prägen, oder werden sie von einem Monopol herausgegeben, das dadurch Münzprägegewinn ~einnimmt?“

„Was?“, sagte Professor McGonagall.

Griphook grinste und zeigte scharfe Zähne. „Nur ein Narr würde irgendetwas anderen als Koboldmünzen trauen!“

„Mit anderen Worten“, sagte Harry, „die Münzen sollen nicht mehr wert sein als das Metall, aus dem sie bestehen?“

Griphook starrte Harry an. Professor McGonagall sah verwirrt aus.

„Ich meine, angenommen, ich käme mit einer Tonne Silber hierher. Könnte ich mir daraus eine Tonne Sickel anfertigen lassen?“

„Gegen eine Gebühr, Mr~Potter-Evans-Verres.“ Der Kobold sah ihn mit funkelnden Augen an. „Für ein gewisses Honorar. Wo würden Sie eine Tonne Silber finden, frage ich mich?“

„Ich habe nur hypothetisch gesprochen“, sagte Harry. Zumindest für den Moment. „Also…wie viel würden Sie an Gebühren verlangen, von einem Bruchteil des Gesamtgewichts?“

Griphooks Augen waren aufmerksam. „Ich müßte meine Vorgesetzten konsultieren…“

„Geben Sie mir eine wilde Vermutung. Ich werde Gringotts nicht darauf festnageln.“

„Ein zwanzigstel Teil des Metalls wäre für die Prägung ~angemessen.“

Harry nickte. „Vielen Dank, Mr~Griphook.“

Die Zaubererwirtschaft ist also nicht nur fast vollständig von der Muggelwirtschaft abgekoppelt, sondern niemand hier hat je etwas von der Ausnutzung von Zinsunterschieden gehört.

Die größere Muggelwirtschaft hatte eine schwankende Handelsspanne von Gold zu Silber, so dass jedes Mal, wenn das Muggel-Gold-zu-Silber-Verhältnis mehr als 5\% vom Gewicht von siebzehn Sickles zu einer Galeone abwich, entweder Gold oder Silber aus der Zaubererwirtschaft hätte abfließen müssen, bis es unmöglich wurde, den Wechselkurs aufrechtzuerhalten.

Bring eine Tonne Silber herein, tausche es in Sickles um (und zahle 5\%), tausche ~die Sickles in Galleonen um, bring das Gold in die Muggelwelt, tausche es in mehr Silber um, als du am Anfang hattest, und wiederhole das Ganze.

War das Verhältnis von Muggelgold zu Silber nicht irgendwo bei fünfzig zu eins? Harry glaubte jedenfalls nicht, dass es siebzehn war. Und es sah so aus, als wären die Silbermünzen tatsächlich kleiner als die Goldmünzen. Andererseits stand Harry in einer Bank, die ihr Geld buchstäblich in Tresoren voller Goldmünzen aufbewahrte, die von Drachen bewacht wurden, wo man hineingehen und Münzen aus dem Tresor holen musste, wenn man Geld ausgeben wollte.

Die Feinheiten des Ausnutzen von Marktineffizienzen könnten für sie unbekannt sein. Er war versucht, abfällige Bemerkungen über die Schlichtheit ihres Finanzsystems zu machen… Aber das Traurige ist, dass ihre Art wahrscheinlich besser ist. Andererseits könnte ein fähiger Aktienmanager innerhalb einer Woche die ganze Zaubererwelt besitzen.

Harry bewahrte diesen Gedanken für den Fall auf, dass ihm mal das Geld ausging oder er eine Woche frei hatte.

In der Zwischenzeit sollten die riesigen Haufen von Goldmünzen im Potter-Tresor seinen kurzfristigen Bedürfnissen entsprechen. Harry stolperte nach vorne und begann, mit einer Hand Goldmünzen aufzuheben und sie in die andere zu stecken. Als er zwanzig erreicht hatte, hustete Professor McGonagall.

„Ich denke, das wird mehr als genug sein, um Ihre Schulsachen zu bezahlen, Mr~Potter.“

„Hm?“ sagte Harry, mit seinen Gedanken ganz woanders. „Moment, ich mache gerade eine Fermi-Berechnung.“

„Eine was?“, sagte Professor McGonagall und klang dabei etwas beunruhigt.

„Das ist eine mathematische Sache. Benannt nach Enrico Fermi. Eine Methode, um schnell grobe Zahlen in den Kopf zu bekommen…“

Zwanzig Goldgalleonen wogen vielleicht ein Zehntel Kilogramm? Und Gold war, was, 10.000 britische Pfund pro Kilogramm wert? Also wäre eine Galeone etwa fünfzig Pfund wert… Die Goldmünzenhügel sahen so aus, als wären sie etwa sechzig Münzen hoch und zwanzig Münzen breit in jeder Dimension der Basis, und ein Hügel war pyramidenförmig, also würde er etwa ein Drittel eines Würfels ausmachen.

Achttausend Galleonen pro Hügel, grob geschätzt, und es gab etwa fünf Hügel dieser Größe, also vierzigtausend Galleonen oder 2 Millionen Pfund Sterling.

Nicht schlecht. Harry lächelte mit einer gewissen grimmigen Zufriedenheit. Es war zu schade, dass er gerade dabei war, die erstaunliche neue Welt der Magie zu entdecken, und sich keine Zeit nehmen konnte, die erstaunliche neue Welt des Reichseins zu erforschen, die laut einer schnellen Fermi-Schätzung etwa eine Milliarde Mal weniger interessant war.

Trotzdem, das war das letzte Mal, dass ich für ein lausiges Pfund den Rasen gemäht habe. Harry stand von dem riesigen Geldhaufen auf.

„Verzeihen Sie die Frage, Professor McGonagall, aber ich habe gehört, dass meine Eltern in ihren Zwanzigern waren, als sie starben. Ist es in der Zaubererwelt üblich, dass ein junges Paar so viel Geld in seinem Tresor hat?“

Wenn ja, dann kostete eine Tasse Tee wahrscheinlich fünftausend Pfund. Regel eins der Wirtschaftswissenschaften: Geld kann man nicht essen.

Professor McGonagall schüttelte den Kopf. „Ihr Vater war der letzte Erbe einer alten Familie, Mr~Potter. Es ist auch möglich…“

Die Hexe zögerte. „Ein Teil dieses Geldes könnte von Kopfgeldern stammen, die auf Du-weißt-schon-wen ausgesetzt wurden, zahlbar an seinen Mör- ah, an den, der ihn besiegen konnte. Oder diese Kopfgelder sind noch nicht eingetrieben worden. Ich bin mir nicht sicher.“

„Interessant…“ sagte Harry langsam. „Also gehört einiges davon in gewisser Weise wirklich mir. Das heißt, von mir verdient. Irgendwie. Möglicherweise. Auch wenn ich mich nicht mehr an das Ereignis erinnern kann.“

Harrys Finger klopften gegen sein Hosenbein. "Dann fühle ich mich weniger schuldig, weil ich einen winzigen Teil davon ausgegeben will!

Keine Panik, Professor McGonagall!"

„Mr~Potter! Sie sind minderjährig, und als solcher dürfen Sie nur vernünftige Ausgaben von…“

„Ich vernünftig! Ich ~verstehe steuerliche Vorsicht und Impulskontrolle! Aber ich habe auf dem Weg hierher ein paar Dinge gesehen, die vernünftige, erwachsene Anschaffungen darstellen würden…“

Harry schaute Professor McGonagall an und lieferte sich einen stummen Wettstreit der Blicke.

„Was zum Beispiel?“ sagte Professor McGonagall schließlich.

„Koffer, deren Inneres mehr enthält als ihr Äußeres?“

Professor McGonagalls Gesicht wurde ernst. „Die sind sehr teuer, Mr~Potter!“

„Ja, aber—“ flehte Harry. „Ich bin sicher, dass ich als Erwachsener einen haben möchte. Und ich kann mir einen leisten. Logischerweise würde es genauso viel Sinn machen, ihn jetzt zu kaufen, anstatt später, um ihn sofort zu benutzen. Es ist so oder so das gleiche Geld, richtig? Ich meine, ich würde einen guten haben wollen, mit viel Platz im Inneren, gut genug, dass ich später nicht einfach einen besseren kaufen muss…“

Harry brach hoffnungsvoll ab. Professor McGonagalls Blick wankte nicht.

„Und was genau würden Sie in so einem Koffer aufbewahren, Mr~Potter—“

„Bücher.“

„Natürlich“, seufzte Professor McGonagall.

„Sie hätten mir schon viel früher sagen sollen, dass es diese Art von magischen Gegenständen gibt! Und dass ich mir einen leisten kann! Jetzt werden mein Vater und ich die nächsten zwei Tage damit verbringen, verzweifelt alle Antiquariate nach alten Lehrbüchern abzuklappern, damit ich in Hogwarts eine anständige wissenschaftliche Bibliothek dabei habe - und vielleicht eine kleine Science-Fiction-Sammlung, wenn ich etwas Anständiges aus den Schnäppchenkisten zusammenstellen kann. Oder besser noch, ich mache Ihnen den Deal ein bisschen schmackhafter, okay? Lassen Sie mich einfach folgendes kaufen—“

„Mr~Potter! Sie glauben, Sie können mich bestechen?“

„Was? Nein! Nicht auf diese Weise! Ich will damit sagen, dass Hogwarts einige der Bücher, die ich mitbringe, behalten kann, wenn Sie meinen, dass eines davon eine gute Ergänzung für die Bibliothek wäre. Ich werde sie billig bekommen und ich möchte sie einfach irgendwo herumliegen haben. Es ist doch okay, Leute mit Büchern zu bestechen, oder? Das ist eine—“

„Familientradition.“

„Ja, genau.“

Professor McGonagalls Körper schien zusammenzusacken, die Schultern senkten sich in ihren schwarzen Roben.

„Ich kann den Sinn Ihrer Worte nicht leugnen, obwohl ich mir sehr wünsche, dass ich es könnte. Ich werde Ihnen erlauben, weitere hundert Galleonen abzuheben, Mr~Potter.“

Sie seufzte erneut. „Ich weiß, dass ich das bereuen werde, und ich tue es trotzdem.“

„Das ist die richtige Einstellung! Und tut ein 'Maulwurfsfell-Beutel' das, was ich denke, dass er tut?“

„Er kann nicht so viel wie ein Koffer“, sagte die Hexe mit sichtbarem Widerwillen, „aber…ein Maulwurfsfell Beutel mit einem Rückhol-Zauber und einem Unentdeckbarkeitszauber ~kann eine Reihe von Gegenständen aufbewahren, bis sie von demjenigen, der sie eingesetzt hat, hervorgeholt werden—“

„Ja! So einen brauche ich auch unbedingt! Es wäre wie der Super Rucksack der ultimativen Anwengung! Batmans toller Gürtel! Vergiss mein Schweizer Armeemesser, ich könnte ein ganzes Werkzeugset darin transportieren! Oder Bücher! Ich könnte die drei wichtigsten Bücher, die ich gerade lese, immer bei mir haben und überall einfach eines herausziehen! Ich müsste nie wieder eine Minute meines Lebens verschwenden! Was sagen Sie dazu, Professor McGonagall? Es ist zum Wohle von ~Kindern die Lesen wollen, die beste aller möglichen Ursachen.“

„… ich nehme an, Sie können noch zehn Galleonen drauflegen.“

Griphook warf Harry einen Blick zu, der von aufrichtigem Respekt, vielleicht sogar von offener Bewunderung geprägt war.

„Und ein bisschen Taschengeld, wie Sie vorhin erwähnten. Ich glaube, ich kann mich erinnern, noch ein oder zwei andere Dinge gesehen zu haben, die ich vielleicht in diesem Beutel aufbewahren möchte.“

„Übertreiben Sie es nicht, Mr~Potter.“

"Aber, oh, Professor McGonagall, warum wollen Sie mir in die Parade fahren? Heute ist doch ein glücklicher Tag, an dem ich zum ersten Mal alles über Zauberer erfahre! Warum die Rolle des mürrischen Erwachsenen spielen, wenn Sie stattdessen lächeln und sich an Ihre eigene unschuldige Kindheit erinnern könnten, wenn Sie den Ausdruck der Freude auf meinem jungen Gesicht sehen, wenn ich ein paar Spielsachen kaufe, mit einem unbedeutenden Bruchteil des Reichtums, den ich verdient habe, indem ich den schrecklichsten Zauberer besiegt habe, den Britannien je gekannt hat—

nicht, dass ich Ihnen Undankbarkeit vorwerfen würde oder so, aber was sind schon ein paar Spielsachen im Vergleich dazu?"

„Sie!“, knurrte Professor McGonagall.

Ihr Gesichtsausdruck war so furchteinflößend und schrecklich, dass Harry quietschend einen Schritt zurücktrat, dabei mit einem großen klirrenden Geräusch einen Stapel Goldmünzen umwarf und rückwärts in einen Geldhaufen kippte.

Griphook seufzte und legte eine Handfläche über sein Gesicht.

„Ich würde dem zaubernden Britannien einen großen Dienst erweisen, Mr~Potter, wenn ich Sie in diesen Tresor sperren und hier lassen würde. “

Und sie gingen ohne weitere Probleme.

