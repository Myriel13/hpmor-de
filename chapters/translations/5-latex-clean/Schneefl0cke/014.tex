

\hypertarget{gewissenhaftigkeit}{% \section{15. Gewissenhaftigkeit}\label{gewissenhaftigkeit}}

\textbf{Gewissenhaftigkeit}

„\emph{Ich bin sicher, ich werde die Zeit irgendwo finden.}“

„Frigideiro!“

Harry tauchte einen Finger in das Wasserglas auf seinem Schreibtisch. Es hätte kühl sein müssen.

Aber es war lauwarm, und lauwarm war es geblieben. Schon wieder.

Harry fühlte sich sehr, sehr betrogen.

Im Haushalt der Verres lagen Hunderte von Fantasy-Romanen verstreut. Harry hatte ziemlich viele davon gelesen.

Und es sah langsam so aus, als hätte er eine geheimnisvolle dunkle Seite. Nachdem sich das Glas Wasser die ersten paar Male geweigert hatte, zu kooperieren, hatte Harry sich im Zauberei-Klassenzimmer umgesehen, um sicherzugehen, dass niemand zusah, und dann tief durchgeatmet, sich konzentriert und sich selbst wütend gemacht.

Dachte über die Slytherins nach, die Neville schikanierten, und über das Spiel, bei dem jemand deine Bücher umwarf, wenn du sie wieder aufheben wolltest.

Dachte daran, was Draco Malfoy über das zehnjährige Lovegood-Mädchen gesagt hatte und wie das Zaubergamot wirklich funktionierte.

.. Und die Wut war in sein Blut übergegangen, er hatte seinen Zauberstab in einer vor Hass zitternden Hand ausgestreckt und in kaltem Ton gesagt:

„Frigideiro!“

und absolut nichts war passiert. Harry war reingelegt worden. Er wollte jemandem schreiben und eine Rückerstattung für seine dunkle Seite verlangen, die offensichtlich eine unwiderstehliche magische Kraft haben sollte, sich aber als defekt erwiesen hatte.

„Frigideiro!“, sagte Hermine wieder vom Schreibtisch neben ihm. Ihr Wasser war festes Eis und am Rand ihres Glases bildeten sich weiße Kristalle.

Sie schien völlig auf ihre eigene Arbeit konzentriert zu sein und sich der anderen Schüler, die sie mit hasserfüllten Augen anstarrten, überhaupt nicht bewusst zu sein, was entweder

a) eine gefährliche Unaufmerksamkeit ihr gegenüber oder

b) eine perfekt ausgefeilte Darbietung auf dem Niveau der hohen Kunst war.

„Oh, sehr gut, Miss~Granger!“, quietschte Filius Flitwick, ihr Zauberkunst-Professor und Leiter von Ravenclaw, ein winzig kleiner Mann ohne sichtbare Anzeichen, dass er ein früherer Duell-Champion war.

„Ausgezeichnet! Hervorragend!“

Harry hatte damit gerechnet, im schlimmsten Fall Zweiter hinter Hermine zu werden.

Harry hätte es natürlich lieber gesehen, wenn sie gegen ihn angetreten wäre, aber er hätte es auch andersherum akzeptieren können.

Am Montag befand sich Harry auf dem letzten Platz der Klasse, eine Position, um die er mit allen anderen muggelstämmigen Schülern außer Hermine konkurrierte.

Die war ganz allein und rivalitätslos an der Spitze, das arme Ding. Professor Flitwick stand über dem Pult einer der anderen Muggelgeborenen und justierte leise die Art, wie sie ihren Zauberstab hielt.

Harry schaute zu Hermine hinüber. Er schluckte schwer. Es war die offensichtliche Rolle für sie in dem Schema der Dinge…

„Hermine?“ sagte Harry zögernd. „Hast du eine Ahnung, was ich falsch machen könnte?“

Hermines Augen leuchteten in einem schrecklichen Licht der Hilfsbereitschaft auf und etwas in Harrys Hinterkopf schrie in verzweifelter Demütigung auf.

Fünf Minuten später schien Harrys Wasser tatsächlich merklich kühler als Raumtemperatur zu sein, und Hermine hatte ihm ein paar verbale Klapse auf den Kopf gegeben und ihm gesagt, er solle es das nächste Mal sorgfältiger aussprechen, und war gegangen, um jemand anderem zu helfen.

Professor Flitwick hatte ihr einen Hauspunkt gegeben, weil sie ihm geholfen hatte. Harry knirschte so sehr mit den Zähnen, dass ihm der Kiefer schmerzte, und das half seiner Aussprache nicht gerade.

\emph{Es ist mir egal, ob es ein unfairer Wettbewerb ist. Ich weiß genau, was ich mit den zwei zusätzlichen Stunden pro Tag mache. Ich werde in meinem Koffer sitzen und lernen, bis ich mit Hermine Granger mithalten kann.}

Später

„Verwandlung gehört zu den komplexesten und gefährlichsten Zaubern, die ihr in Hogwarts lernen werdet“, sagte Professor McGonagall.

Auf dem Gesicht der strengen alten Hexe war keine Spur von Leichtfertigkeit zu sehen.

"Jeder, der in meiner Klasse herumalbert, wird gehen und nicht wiederkommen.

Ihr wurdet gewarnt."

Ihr Zauberstab senkte sich und tippte auf ihr Pult, das sich in ein Schwein verwandelte.

Ein paar Muggelgeborene Schüler stießen ein leises Aufjaulen aus. Das Schwein sah sich um und schnaubte, schien verwirrt und wurde dann wieder zum Schreibtisch.

Die Professorin für Verwandlung sah sich im Klassenzimmer um, und dann blieb ihr Blick an einem Schüler hängen.

„Mr~Potter“, sagte Professor McGonagall. „Sie haben Ihre Schulbücher erst vor ein paar Tagen erhalten. Haben Sie schon angefangen, Ihr Verwandlungslehrbuch zu lesen?“

„Nein, tut mir leid, Professor“, sagte Harry.

„Sie brauchen sich nicht zu entschuldigen, Mr~Potter, wenn man von Ihnen verlangt hätte, vorzulesen, hätte man es Ihnen gesagt.“

McGonagalls Finger klopften auf das Pult vor ihr.

„Mr~Potter, möchten Sie raten, ob dies ein Schreibtisch ist, den ich in ein Schwein verwandelt habe, oder ob er als Schwein begann und ich die Verwandlung kurzzeitig entfernt habe? Wenn Sie das erste Kapitel Ihres Lehrbuchs gelesen hätten, wüssten Sie es.“

Harrys Augenbrauen runzelten sich leicht.

„Ich würde vermuten, dass es einfacher wäre, mit einem Schwein anzufangen, denn wenn es als Schreibtisch angefangen hat, weiß es vielleicht nicht, wie man aufsteht.“

Professor McGonagall schüttelte den Kopf.

"Kein Vorwurf an Sie, Mr~Potter, aber die richtige Antwort ist, dass man in \textbf{Verwandlung nicht raten darf.}

Falsche Antworten werden mit äußerster Strenge bewertet, nicht beantwortete Fragen werden mit großer Nachsicht bewertet.

Ihr müsst lernen, zu wissen, was ihr nicht wisst. Wenn ich euch eine Frage stelle, egal wie offensichtlich oder elementar, und ihr antwortet „\emph{Ich bin mir nicht sicher}“, werde ich euch das nicht übel nehmen und jeder, der lacht, verliert Hauspunkte. Können Sie mir sagen, warum es diese Regel gibt, Mr~Potter?"

\emph{Weil ein einziger Fehler in Verwandlung unglaublich gefährlich sein kann.}

„Nein.“

"Richtig. Verwandlung ist gefährlicher als Apparition, die man erst im sechsten Jahr lernt.

Leider muss Verwandlung in jungen Jahren erlernt und geübt werden, damit man als Erwachsener seine Fähigkeiten maximieren kann.

Es handelt sich also um ein gefährliches Fach, und ihr solltet euch davor hüten, irgendwelche Fehler zu machen, denn keiner meiner Schüler hat sich jemals dauerhaft verletzt, und ich wäre äußerst verärgert, wenn ihr die erste Klasse wärt, die mir diesen Rekord vermiest."

Mehrere Schüler schluckten.

Professor McGonagall stand auf und ging hinüber zur Wand hinter ihrem Schreibtisch, an der eine polierte Holztafel stand.

„Es gibt viele Gründe, warum Verwandlung gefährlich ist, aber ein Grund steht über all den anderen.“

Sie holte einen kurzen Federkiel mit einem dicken Ende hervor und skizzierte damit Buchstaben in Rot, die sie dann mit demselben Marker in Blau unterstrich: Verwandlung IST NICHT VON DAUER!

„Verwandlung ist nicht dauerhaft!“, sagte Professor McGonagall.

"\textbf{Verwandlung ist nicht dauerhaft!}

\textbf{\uline{Verwandlung ist nicht dauerhaft!}}

Mr~Potter, nehmen wir an, ein Schüler verwandelt einen Holzklotz in einen Becher mit Wasser, und Sie trinken ihn.

Was stellen Sie sich vor, was mit Ihnen passieren würde, wenn die Verwandlung nachlässt?"

Es gab eine Pause.

„Entschuldigen Sie, das hätte ich Sie nicht fragen sollen, Mr~Potter, ich vergaß, dass Sie mit einer ungewöhnlich pessimistischen Vorstellungskraft gesegnet sind—“

„Mir geht es gut“, sagte Harry und schluckte schwer.

„Also die erste Antwort ist, dass ich es nicht weiß“, der Professor nickte zustimmend,

„aber ich stelle mir vor, dass da… Holz in meinem Magen und in meinem Blutkreislauf sein könnte, und wenn irgendetwas von diesem Wasser in das Gewebe meines Körpers gelangt wäre - wäre es Holzbrei oder festes Holz oder…“

Harrys Verständnis von Magie ließ ihn im Stich. Er konnte nicht verstehen, wie sich Holz überhaupt in Wasser verwandeln konnte, also konnte er auch nicht verstehen, was passieren würde, nachdem die Wassermoleküle durch gewöhnliche thermische Bewegungen durcheinandergewirbelt wurden, die Magie nachließ und sich die Verwandlung umkehrte.

McGonagalls Gesicht war starr.

„Wie Mr~Potter richtig erkannt hat, würde er extrem krank werden und müsste sofort mit dem Flooing ins St. Mungo's Krankenhaus gebracht werden, wenn er eine Überlebenschance haben soll. Bitte schlagen Sie Ihre Lehrbücher auf Seite 5 auf.“

Selbst ohne Ton in dem bewegten Bild konnte man erkennen, dass die Frau mit der schrecklich verfärbten Haut schrie.

"Der Verbrecher, der ursprünglich Gold in Wein verwandelte und ihn dieser Frau zu trinken gab, \emph{'zur Begleichung der Schuld'}, wie er es ausdrückte, erhielt eine Strafe von zehn Jahren in Askaban.

Bitte blättern Sie auf Seite 6. Das ist ein Dementor. Sie sind die Wächter von Askaban. Sie saugen eure Magie aus, euer Leben und jeden glücklichen Gedanken, den ihr versucht zu haben.

Das Bild auf Seite 7 zeigt den Verbrecher 10 Jahre später, bei seiner Entlassung. Sie werden feststellen, dass er tot ist - ja, Mr~Potter?"

„Professor“, sagte Harry, „wenn in so einem Fall das Schlimmste passiert, gibt es dann eine Möglichkeit, die Verwandlung aufrechtzuerhalten?“

„Nein“, sagte Professor McGonagall schlicht und einfach.

"Eine Verwandlung aufrechtzuerhalten ist ein ständiger Entzug Ihrer Magie, der mit der Größe der Zielform skaliert.

Und Sie müssten das Ziel alle paar Stunden erneut mit dem Zauberstab berühren, was in einem Fall

wie diesem unmöglich ist.

\textbf{Solche Katastrophen sind nicht wiedergutzumachen!}"

Professor McGonagall beugte sich vor, ihr Gesicht war sehr hart.

"Ihr werdet unter keinen Umständen irgendetwas in eine Flüssigkeit oder ein Gas verwandeln. Kein Wasser, keine Luft.

Nichts wie Wasser, nichts wie Luft. Auch wenn es nicht zum Trinken gedacht ist. Flüssigkeit verdunstet, kleine Teile davon gelangen in die Luft.

Ihr werdet nichts verwandeln, was verbrannt werden soll. Es wird Rauch erzeugen und jemand könnte diesen Rauch einatmen!

Ihr werdet niemals etwas verwandeln, das auf irgendeine Weise in den Körper eines Menschen gelangen könnte.

Kein Essen. Nichts, was wie Essen aussieht. Nicht einmal als lustiger kleiner Streich, bei dem Du ihnen von Deinem Matschkuchen erzählen willst, bevor sie ihn tatsächlich essen.

Ihr werdet es niemals tun. Punkt. In diesem Klassenzimmer oder außerhalb oder sonst wo. Ist das jedem einzelnen Schüler klar?!"

„Ja“, sagten Harry, Hermine und ein paar andere.

Die anderen schienen sprachlos zu sein.

\textbf{„Ist das jedem einzelnen Schüler klar?!“}

„Ja“, sagten sie oder murmelten oder flüsterten.

"Wenn ihr gegen eine dieser Regeln verstoßt, werdet ihr während eures Aufenthalts in Hogwarts nicht weiter Verwandlung studieren.

Wiederhole mit mir zusammen.

Ich werde niemals etwas in eine Flüssigkeit oder ein Gas verwandeln."

„\emph{Ich werde niemals irgendetwas in eine Flüssigkeit oder ein Gas verwandeln}“, sagten die Schüler in zornigem Chor.

„\textbf{Noch mal! Lauter!} “

„Ich werde niemals etwas in eine Flüssigkeit oder ein Gas verwandeln."

„Ich werde nie etwas in eine Flüssigkeit oder ein Gas verwandeln."

„Ich werde nie etwas verwandeln, das wie Essen aussieht oder in einen menschlichen Körper passt.“ „Ich werde nie etwas verwandeln, das verbrannt werden soll, weil es Rauch erzeugen könnte.“

„Ich werde niemals etwas verwandeln, das wie Geld aussieht, auch nicht Muggelgeld“, sagte Professor McGonagall.

„Die Kobolde haben Möglichkeiten, herauszufinden, wer es getan hat. Nach anerkanntem Recht befindet sich die Koboldnation in einem permanenten \textbf{Kriegszustand} mit allen magischen Fälschern. Sie werden keine Auroren schicken. \textbf{Sie werden eine Armee schicken.}“

„\emph{Ich werde niemals etwas verwandeln, das wie Geld aussieht}“, wiederholten die Schüler.

„Und vor allem“, sagte Professor McGonagall,

"werdet ihr kein lebendes Subjekt verwandeln, schon gar nicht euch selbst.

Es wird euch sehr krank machen und vielleicht sogar sterben lassen, je nachdem, wie ihr euch verändert und wie lange die Veränderung aufrechterhalten wird."

Professor McGonagall hielt inne.

"Mr~Potter hält gerade seine Hand hoch, weil er eine Animagus-Verwandlung gesehen hat - genauer gesagt, einen Menschen, der sich in eine Katze verwandelt und wieder zurück.

Aber eine Animagus-Verwandlung ist \emph{keine freie Verwandlung.}"

Professor McGonagall holte ein kleines Stück Holz aus ihrer Tasche.

Mit einem Schlag ihres Zauberstabs wurde es zu einer Glaskugel. Und sie sagte:

„Crystferrium!“

und die Glaskugel wurde zu einer Stahlkugel.

Sie tippte sie ein letztes Mal mit ihrem Zauberstab an und die Stahlkugel wurde wieder zu einem Stück Holz.

"Crystferrium verwandelt einen Gegenstand aus massivem Glas in ein ähnlich geformtes Ziel aus massivem Stahl.

Es kann weder den umgekehrten Weg gehen, noch kann es einen Schreibtisch in ein Schwein verwandeln.

Die allgemeinste Form der Verwandlung - die freie Verwandlung, die Sie hier lernen werden - ist in der Lage, jeden Gegenstand in jedes Ziel zu verwandeln, zumindest was die physische Form betrifft.

Aus diesem Grund muss die freie Verwandlung wortlos durchgeführt werden. Die Verwendung von Zaubersprüchen würde für jede unterschiedliche Verwandlung zwischen Subjekt und Ziel unterschiedliche Worte erfordern."

Professor McGonagall warf ihren Schülern einen scharfen Blick zu.

"Manche Lehrer beginnen mit Verwandlungszaubern und gehen danach zur freien Verwandlung über.

Ja, das wäre am Anfang viel einfacher. Aber es kann euren Geist vernachlässigen lassen, was eure Fähigkeiten später beeinträchtigt.

Hier lernt ihr von Anfang an die freie Verwandlung, die erfordert, dass der Zauber wortlos gesprochen wird, indem man die Subjektform, die Zielform und die Verwandlung im eigenen Geist hältt."

"Und um die Frage von Mr.

Potter zu beantworten", fuhr Professor McGonagall fort,

"es ist die freie Verwandlung, die Sie niemals an einem lebenden Subjekt durchführen dürfen.

Es gibt Zaubersprüche und Tränke, die lebende Subjekte in begrenzter Weise sicher und reversibel verwandeln können.

Einem Animagus mit einer fehlenden Gliedmaße wird diese Gliedmaße zum Beispiel auch nach der Verwandlung noch fehlen.

\emph{Freie Verwandlung ist nicht sicher}. Ihr Körper wird sich verändern, während er verwandelt ist - das Atmen zum Beispiel führt zu einem ständigen Verlust von Körpermaterial an die Umgebungsluft.

Wenn die Verwandlung nachlässt und Ihr Körper versucht, in seine ursprüngliche Form zurückzukehren, wird er dazu nicht ganz in der Lage sein.

Wenn Sie Ihren Zauberstab an Ihren Körper drücken und sich vorstellen, dass Sie goldenes Haar haben, werden Ihnen danach die Haare ausfallen.

Wenn Sie sich als jemand mit reinerer Haut visualisieren, werden Sie einen langen Aufenthalt im St. Mungo's haben. Und wenn Sie sich in eine erwachsene Körperform wandeln, dann werden Sie, wenn die Verwandlung nachlässt, sterben."

Das erklärte, warum er solche Dinge wie dicke Jungen oder nicht ganz so hübsche Mädchen gesehen hatte. Oder alte Leute, was das betrifft. Das würde nicht passieren, wenn man sich einfach jeden Morgen verwandeln könnte.

.. Harry hob die Hand und versuchte, Professor McGonagall mit seinen Augen zu signalisieren.

„Ja, Mr~Potter?“

„Ist es möglich, ein lebendes Subjekt in ein Ziel zu verwandeln, das statisch ist, wie zum Beispiel eine Münze - nein, entschuldigen Sie, es tut mir schrecklich leid, sagen wir einfach eine Stahlkugel.“

Professor McGonagall schüttelte den Kopf.

"Mr~Potter, selbst unbelebte Gegenstände machen mit der Zeit kleine innere Veränderungen durch. Es gäbe danach keine sichtbaren Veränderungen an Ihrem Körper, und in der ersten Minute würden Sie nichts bemerken.

Aber in einer Stunde wären Sie krank, und in einem Tag wären Sie tot."

„Ähm, entschuldigen Sie, also wenn ich das erste Kapitel gelesen hätte, hätte ich erraten können, dass der Schreibtisch ursprünglich ein Schreibtisch war und kein Schwein“, sagte Harry,

„aber nur, wenn ich die weitere Annahme getroffen hätte, dass Sie das Schwein nicht töten wollten, was nicht sehr wahrscheinlich scheint, aber—“

„Ich kann absehen, dass das Benoten Ihrer Tests eine endlose Quelle der Freude für mich sein wird, Mr~Potter. Aber wenn Sie weitere Fragen haben, kann ich Sie bitten, bis nach dem Unterricht zu warten?“

„Keine weiteren Fragen, Professor.“

„Nun sprechen Sie mir nach“, sagte Professor McGonagall.

„Ich werde niemals versuchen, ein lebendes Subjekt zu verwandeln, schon gar nicht mich selbst, es sei denn, ich habe den ausdrücklichen Auftrag, dies mit einem speziellen Zauber oder Zaubertrank zu tun.“

"Wenn ich mir nicht sicher bin, ob eine Verwandlung sicher ist, werde ich sie nicht versuchen, bis ich Professor McGonagall oder Professor Flitwick oder Professor Snape oder den Schulleiter gefragt habe, die die einzigen anerkannten Autoritäten für Verwandlung in Hogwarts sind.

Einen anderen Schüler zu fragen, ist nicht akzeptabel, selbst wenn dieser sagt, dass er sich daran erinnert, die gleiche Frage gestellt zu haben."

„Selbst wenn der derzeitige Verteidigungsprofessor in Hogwarts mir sagt, dass eine Verwandlung sicher ist, und selbst wenn ich sehe, wie der Verteidigungsprofessor sie durchführt und nichts Schlimmes zu passieren scheint, werde ich sie nicht selbst ausprobieren.“

"Ich habe das absolute Recht, mich zu weigern, eine Verwandlung durchzuführen, bei der ich auch nur das kleinste bisschen nervös bin.

Da nicht einmal der Schulleiter von Hogwarts mir befehlen kann, etwas anderes zu tun, werde ich mit Sicherheit keinen solchen Befehl des Verteidigungsprofessors akzeptieren, selbst wenn der Verteidigungsprofessor damit droht, mir hundert Hauspunkte abzuziehen und mich von der Schule zu verweisen."

„Wenn ich gegen eine dieser Regeln verstoße, werde ich während meiner Zeit in Hogwarts nicht weiter Verwandlung studieren.“

„Wir werden diese Regeln den ersten Monat lang zu Beginn jeder Stunde wiederholen“, sagte Professor McGonagall.

„Und jetzt werden wir mit Streichhölzern als Gegenstand und Nadeln als Zielscheibe beginnen…legt eure Zauberstäbe weg, danke, mit '\emph{beginnen}' meinte ich, dass ihr anfangt, euch Notizen zu machen.“

Eine halbe Stunde vor Ende der Stunde teilte Professor McGonagall die Streichhölzer aus.

Am Ende der Stunde hatte Hermine ein silbrig glänzendes Streichholz und der gesamte Rest der Klasse, ob Muggelgeboren oder nicht, hatte genau das, womit sie angefangen hatte.

Professor McGonagall verlieh ihr einen weiteren Punkt für Ravenclaw. Nachdem die Verwandlungsklasse entlassen worden war, kam Hermine zu Harrys Schreibtisch hinüber, als dieser gerade seine Bücher in seinen Beutel einräumte.

„Weißt du“, sagte Hermine mit einem unschuldigen Gesichtsausdruck, „ich habe heute zwei Punkte für Ravenclaw verdient.“

„Das hast du“, sagte Harry kurz.

„Aber das war nicht so gut wie deine sieben Punkte“, sagte sie. „Ich schätze, ich bin einfach nicht so intelligent wie du.“

Harry stopfte seine Hausaufgaben in den Beutel und drehte sich mit zusammengekniffenen Augen zu Hermine um.

Das hatte er tatsächlich vergessen. Sie trommelte mit den Fingern auf seinem Tisch.

"Wir haben aber jeden Tag Unterricht. Ich frage mich, wie lange du brauchen wirst, um noch ein paar Hufflepuffs zu finden, die du retten kannst? Heute ist Montag.

Das heißt, du hast bis Donnerstag Zeit."

Die beiden starrten sich in die Augen, ohne zu blinzeln.

Harry sprach zuerst.

„Dir ist natürlich klar, dass das Krieg bedeutet.“

„Ich wusste nicht, dass wir im Frieden waren.“

Alle anderen Schüler sahen nun mit faszinierten Augen zu.

Alle anderen Schüler und, leider, auch Professor McGonagall.

„Oh, Mr~Potter“, rief Professor McGonagall von der anderen Seite des Raumes,

"ich habe gute Nachrichten für Sie. Madam Pomfrey hat Ihren Vorschlag zur Verhinderung von Brüchen in ihren Spimster-Wickets genehmigt, und es ist geplant, die Arbeit bis Ende nächster Woche zu erledigen.

Ich würde sagen, das verdient…sagen wir zehn Punkte für Ravenclaw."

Hermine's Gesicht klaffte vor Verrat und Schock.

Harry stellte sich vor, dass sein eigenes Gesicht nicht viel anders aussah.

„Professor…“ zischte Harry.

"Diese zehn Punkte sind zweifelsohne verdient, Mr~Potter. Ich würde Hauspunkte nicht aus einer Laune heraus vergeben.

Für Sie wäre es vielleicht eine einfache Sache gewesen, etwas Zerbrechliches zu sehen und eine Möglichkeit vorzuschlagen, es zu schützen, aber Spimster-Wickets sind teuer, und der Schulleiter war nicht erfreut, als das letzte Mal eines zerbrach. Professor McGonagall sah nachdenklich aus.

"Meine Güte, ich frage mich, ob jemals ein anderer Schüler siebzehn Hauspunkte an seinem ersten Unterrichtstag erreicht hat.

Ich muss es nachschlagen, aber ich vermute, das ist ein neuer Rekord. Vielleicht sollten wir zur Essenszeit eine Ankündigung machen?"

„PROFESSOR! Das ist unser Krieg! Hören Sie auf, sich einzumischen!“

„Jetzt haben Sie bis Donnerstag nächster Woche Zeit, Mr~Potter. Es sei denn, Sie begehen bis dahin irgendeinen Unfug und verlieren Hauspunkte. Einen Professor respektlos anzusprechen, zum Beispiel.“

Professor McGonagall legte einen Finger an ihre Wange und schaute nachdenklich.

„Ich gehe davon aus, dass Sie noch vor Ende des Freitags in den Minusbereich kommen werden.“

Harrys Mund schnappte zu.

Er schickte sein bestes Todesstarren an McGonagall, aber sie schien es nur amüsant zu finden.

„Ja, definitiv eine Ankündigung beim Abendessen“, überlegte Professor McGonagall.

"Aber es wäre nicht gut, die Slytherins zu beleidigen, also sollte die Ankündigung kurz sein. Nur die Anzahl der Punkte und die Tatsache des Rekords.

.. und wenn jemand zu Ihnen kommt und Sie um Hilfe bei den Schularbeiten bittet und enttäuscht ist, dass Sie nicht einmal angefangen haben, Ihre Lehrbücher zu lesen, können Sie ihn jederzeit an Miss~Granger verweisen."

„Professor!“, sagte Hermine mit ziemlich hoher Stimme. Professor McGonagall ignorierte sie.

„Meine Güte, ich frage mich, wie lange es wohl dauern wird, bis Miss~Granger etwas macht, das eine Ankündigung zur Abendbrotzeit verdient? Ich freue mich darauf, es zu sehen, was immer es auch sein mag.“

Harry und Hermine drehten sich im unausgesprochenen gegenseitigen Einverständnis um und stürmten aus dem Klassenzimmer. Sie wurden von einer Spur hypnotisierter Ravenclaws verfolgt.

„Ähm“, sagte Harry. „Sind wir nach dem Essen noch verabredet?“

„Natürlich“, sagte Hermine. „Ich möchte nicht, dass du mit dem Lernen noch weiter zurückfällst.“

„Oh, danke. Und ich muss sagen, so brillant, wie du jetzt schon bist, kann ich nicht umhin, mich zu fragen, wie du sein wirst, wenn du erst einmal eine Grundausbildung in Rationalität hast.“

„Ist es wirklich so nützlich? Es scheint dir nicht bei Zauberei oder Verwandlung geholfen zu haben.“

Es gab eine kleine Pause.

„Nun, ich habe meine Schulbücher erst vor vier Tagen bekommen. Deshalb musste ich die siebzehn Hauspunkte sammeln, ohne meinen Zauberstab zu benutzen.“

"Vor vier Tagen? Vielleicht schaffst du es nicht, acht Bücher in vier Tagen zu lesen, aber du könntest zumindest eines gelesen haben.

Wie viele Tage braucht man bei diesem Tempo, um fertig zu werden? Du kennst dich mit Mathematik aus, kannst du mir also sagen, was acht mal vier geteilt durch null ist?"

"Ich habe jetzt Unterricht, was du nicht hast, aber die Wochenenden sind frei, also.

.. Grenzwert von acht mal vier geteilt durch Epsilon, wenn Epsilon sich Null nähert plus… 10:47~Uhr am Sonntag."

„Ich habe es eigentlich in drei Tagen geschafft.“

„14:47~Uhr am Samstag ist es dann. Ich bin sicher, ich finde die Zeit irgendwo.“

Und es wurde Abend und es wurde Morgen, der erste Tag.

