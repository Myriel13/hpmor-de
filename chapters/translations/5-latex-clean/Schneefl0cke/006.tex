

\hypertarget{der-planungsirrtum}{% \section{7. Der Planungsirrtum}\label{der-planungsirrtum}}

Anmerkung des Autors: Der "Nachspiel"-Abschnitt dieses Kapitels ist Teil der Geschichte.

\emph{Du denkst, dein Tag war surreal? Versuch meinen}.

*****

Manche Kinder hätten bis nach ihrem ersten Ausflug zur Winkelgasse gewartet.

"Beutel des Elements 79",

sagte Harry und zog seine leere Hand aus dem Beutel zurück.

Die meisten Kinder hätten zumindest gewartet ihre Zauberstäbe zu bekommen.

"Beutel aus Okane", sagte Harry.

Der schwere Beutel mit Gold tauchte in seiner Hand auf. Harry zog den Beutel heraus und steckte ihn dann wieder in den Maulwurfsfellbeutel.

Er nahm die Hand heraus, steckte sie wieder hinein und sagte: "Beutel mit Wertmarken des wirtschaftlichen Austauschs."

Diesmal kam seine Hand leer heraus.

"Gib mir den Beutel zurück, den ich gerade hineingesteckt habe."

Der Beutel mit dem Gold kam wieder heraus. Harry James Potter-Evans-Verres hatte zumindest einen magischen Gegenstand in die Hände bekommen. Wozu noch warten?

"Professor McGonagall", sagte Harry zu der verwirrten Hexe, die neben ihm schlenderte,

"können Sie mir zwei Wörter nennen, ein Wort für Gold und ein Wort für etwas anderes, das kein Geld ist, in einer Sprache, die ich nicht kenne? Aber sagen Sie mir nicht, welches welches ist."

"Ahava und zahav", sagte Professor McGonagall. "Das ist Hebräisch, und das andere Wort bedeutet Liebe."

"Danke, Professor. Beutel mit Ahava."

Leer.

"Beutel mit Zahav."

Und er sprang ihm in die Hand.

"Zahav ist Gold?" fragte Harry, und Professor McGonagall nickte. Harry dachte über seine gesammelten Versuchsdaten nach. Es waren nur die gröbsten und vorläufigen Versuche, aber sie reichten aus, um zumindest eine Schlussfolgerung zu unterstützen:

"Aaaaaaarrrgh, das ergibt doch keinen Sinn!"

Die Hexe neben ihm hob eine hochgezogene Augenbraue.

"Probleme, Mr~Potter?"

"Ich habe gerade jede einzelne Hypothese falsifiziert, die ich hatte! Wie kann es wissen, dass 'Beutel mit 115 Galleonen' in Ordnung ist, aber nicht 'Beutel mit 90 plus 25 Galleonen'? Es kann zählen, aber es kann nicht addieren? Es kann Substantive verstehen, aber nicht einige Substantivphrasen, die das Gleiche bedeuten? Die Person, die das gemacht hat, spricht wahrscheinlich kein Japanisch, und ich spreche kein Hebräisch, also nutzt es nicht ihr Wissen, und es nutzt nicht mein Wissen -"

Harry winkte hilflos mit einer Hand.

"Die Regeln scheinen irgendwie konsistent zu sein, aber sie bedeuten nichts! Ich werde nicht einmal fragen, wie ein Beutel mit Stimmerkennung und natürlichem Sprachverständnis daherkommt, wenn die besten Programmierer für künstliche Intelligenz nach fünfunddreißig Jahren harter Arbeit nicht einmal die schnellsten Supercomputer dazu bringen können",

Harry schnappte nach Luft, "aber was ist los?"

"Magie", sagte Professor McGonagall.

"Das ist doch nur ein Wort! Selbst nachdem Sie mir das gesagt haben, kann ich keine neuen Vorhersagen machen! Das ist genau so, als würde man 'Phlogiston' oder 'Elan Vital' oder 'Emergenz' oder 'Komplexität' sagen!"

Die schwarzgewandete Hexe lachte laut auf. "Aber es ist Magie, Mr~Potter."

Harry sackte ein wenig in sich zusammen.

"Bei allem Respekt, Professor McGonagall, ich bin mir nicht ganz sicher, ob Sie verstehen, worauf ich hier hinaus will."

"Bei allem Respekt, Mr~Potter, ich bin mir ziemlich sicher, dass ich das nicht tue. Es sei denn - das ist nur eine Vermutung - Sie versuchen, die Welt zu erobern?"

"Nein! Ich meine ja - na ja, nein!?"

"Ich denke, es sollte mich vielleicht beunruhigen, dass Sie Schwierigkeiten haben, die Frage zu beantworten."

Harry dachte mürrisch an die Dartmouth-Konferenz über künstliche Intelligenz im Jahr 1956. Es war die erste Konferenz überhaupt zu diesem Thema gewesen, diejenige, die den Begriff "Künstliche Intelligenz" geprägt hatte. Sie hatten Schlüsselprobleme identifiziert, wie man Computer dazu bringt, Sprache zu verstehen, zu lernen und sich selbst zu verbessern. Sie hatten allen Ernstes behauptet, dass zehn Wissenschaftler, die zwei Monate lang zusammenarbeiten, bedeutende Fortschritte bei diesen Problemen machen könnten.

Nein. Kopf hoch. Du fängst gerade erst an, alle Geheimnisse der Magie zu enträtseln. Du weißt doch gar nicht, ob es in zwei Monaten nicht zu schwierig wird.

"Und Sie haben wirklich noch nie von anderen Zauberern gehört, die diese Art von Fragen stellen oder diese Art von wissenschaftlichen Experimenten durchführen?"

fragte Harry erneut. Es schien ihm so offensichtlich zu sein. Andererseits hatte es mehr als zweihundert Jahre nach der Erfindung der wissenschaftlichen Methode gedauert, bis irgendein Muggelwissenschaftler auf die Idee gekommen war, systematisch zu untersuchen, welche Sätze ein vierjähriger Mensch verstehen konnte und welche nicht.

Die Entwicklungspsychologie der Linguistik hätte im Prinzip schon im achtzehnten Jahrhundert entdeckt werden können, aber bis zum zwanzigsten Jahrhundert hatte niemand auch nur daran gedacht, danach zu suchen. Man konnte der viel kleineren Zaubererwelt also nicht wirklich vorwerfen, dass sie den Rückholzauber nicht erforscht hatte.

Professor McGonagall schürzte die Lippen, dann zuckte sie mit den Schultern.

"Ich bin mir immer noch nicht sicher, was Sie mit 'wissenschaftlichem Experimentieren' meinen, Mr~Potter. Wie gesagt, ich habe gesehen, wie muggelstämmige Schüler versucht haben, die Muggelwissenschaft in Hogwarts zum Laufen zu bringen, und jedes Jahr erfinden Leute neue Zaubersprüche und Zaubertränke."

Harry schüttelte den Kopf.

"Technologie ist überhaupt nicht dasselbe wie Wissenschaft. Und viele verschiedene Wege auszuprobieren, etwas zu tun, ist nicht dasselbe wie zu experimentieren, um die Regeln herauszufinden."

Es gab viele Leute, die versucht hatten, Flugmaschinen zu erfinden, indem sie viele Dinge ausprobierten - mit Flügeln, aber nur die Gebrüder Wright hatten einen Windkanal gebaut, um den Auftrieb zu messen…

"Ähm, wie viele Kinder aus Muggelfamilien kommen jedes Jahr nach Hogwarts?"

"Vielleicht zehn oder so?"

Harry verpasste einen Schritt und stolperte fast über seine eigenen Füße.

"Zehn?"

Die Muggelwelt hatte eine Bevölkerung von sechs Milliarden, Tendenz steigend. Wenn man einer von einer Million war, gab es sieben Muggel in London und weitere tausend in China. Es war unvermeidlich, dass die Muggelbevölkerung einige Elfjährige hervorbrachte, die rechnen konnten -

Harry wusste, dass er nicht der Einzige war. Er hatte andere Wunderkinder bei mathematischen Wettbewerben kennengelernt. Tatsächlich war er von Konkurrenten gründlich geschlagen worden, die wahrscheinlich buchstäblich den ganzen Tag mit dem Üben von Matheaufgaben verbracht und nie ein Science-Fiction-Buch gelesen hatten und die noch vor der Pubertät völlig ausgebrannt waren und es in ihrem späteren Leben zu nichts bringen würden, weil sie nur bekannte Techniken geübt hatten, anstatt zu lernen, kreativ zu denken. (Harry ein sehr schlechter Verlierer.)

Aber… in der Welt der Zauberer… Zehn muggelstämmige Kinder pro Jahr, die alle mit elf Jahren ihre Muggelausbildung beendet hatten? Und Professor McGonagall mochte voreingenommen sein, aber sie hatte behauptet, Hogwarts sei die größte und bedeutendste Zaubererschule der Welt… und sie bildete nur bis zum Alter von siebzehn Jahren aus. Professor McGonagall kannte zweifellos jedes kleinste Detail, wie man sich in eine Katze verwandelte. Aber sie schien buchstäblich noch nie etwas von der wissenschaftlichen Methode gehört zu haben. Für sie war es nur Muggelmagie.

Und sie schien nicht einmal neugierig darauf zu sein, welche Geheimnisse sich hinter dem natürlichen Sprachverständnis des Rückholzaubers verbergen könnten. Damit blieben eigentlich nur zwei Möglichkeiten.

Möglichkeit eins: Die Magie war so unglaublich undurchsichtig, verworren und undurchdringlich, dass Zauberer und Hexen, obwohl sie ihr Bestes versuchten, sie zu verstehen, wenig oder gar keine Fortschritte machten und schließlich aufgaben; und Harry würde es nicht besser machen.

Oder… Harry knackte entschlossen mit den Fingerknöcheln, aber sie machten nur ein leises Klickgeräusch, anstatt bedrohlich von den Wänden der Winkelgasse widerzuhallen.

Möglichkeit 2: Er würde die Welt erobern. Irgendwann. Vielleicht nicht sofort. So etwas dauerte manchmal länger als zwei Monate.

Die Muggelwissenschaft war nicht in der ersten Woche nach Galileo auf den Mond geflogen. Aber Harry konnte immer noch nicht das breite Lächeln unterdrücken, das seine Wangen so breit machte, dass sie anfingen zu schmerzen. Harry hatte immer Angst davor gehabt, als eines dieser Wunderkinder zu enden, aus denen nie etwas wurde und die den Rest ihres Lebens damit verbrachten, damit zu prahlen, wie weit sie im Alter von zehn Jahren schon gewesen waren.

Aber die meisten erwachsenen Genies haben es ja auch nie zu etwas gebracht. Es gab wahrscheinlich tausend Menschen, die so intelligent wie Einstein waren, für jeden tatsächlichen Einstein in der Geschichte. Denn diese anderen Genies hatten das eine, was man unbedingt braucht, um Größe zu erreichen, nicht in die Finger bekommen. Sie hatten nie ein wichtiges Problem gefunden.

\emph{Du gehörst jetzt mir},

dachte Harry mit Blick auf die Wände der Winkelgasse und all die Läden und Gegenstände und all die Ladenbesitzer und Kunden; und all die Länder und Menschen des zaubernden Britanniens und die ganze weitere zaubernde Welt; und das gesamte größere Universum, von dem Muggelwissenschaftler so viel weniger verstehen, als sie glauben.

\emph{Ich, Harry James Potter-Evans-Verres, beanspruche nun dieses Gebiet im Namen der Wissenschaft}.

Blitz und Donner blieben am wolkenlosen Himmel gänzlich aus.

"Worüber lächeln Sie?", erkundigte sich Professor McGonagall misstrauisch und müde.

"Ich frage mich, ob es einen Zauber gibt, der Blitze im Hintergrund aufblitzen lässt, wenn ich einen ominösen Vorsatz fasse", erklärte Harry.

Er prägte sich den genauen Wortlaut seines ominösen Vorsatzes sorgfältig ein, damit er in zukünftigen Geschichtsbüchern richtig wiedergegeben werden würde.

"Ich habe das deutliche Gefühl, dass ich etwas dagegen unternehmen sollte", seufzte Professor McGonagall.

"Ignorieren Sie es, dann geht es weg. Oh, was ist das!?"

Harry legte seine Gedanken an die Welteroberung vorübergehend auf Eis und hüpfte hinüber zu einem Laden mit einer offenen Auslage, und Professor McGonagall folgte ihm.

\textbf{Später}.

Harry hatte nun seine Zaubertränke-Zutaten und seinen Kessel gekauft, und, oh, noch ein paar Dinge mehr.

Dinge, die gut in Harrys Beutel (aka Maulwurfsfell Super Beutel QX31 mit Unauffindbarem Verlängerungszauber, Rückholzauber und größerer Öffnung) zu passen schienen. Kluge, sinnvolle Anschaffungen.

Harry verstand wirklich nicht, warum Professor McGonagall so misstrauisch schaute.

Im Moment befand sich Harry in einem Laden, der teuer genug war, um in der verwinkelten Hauptstraße der Winkelgasse einen Platz zu haben. Der Laden hatte eine offene Front mit Waren, die auf schrägen Holzreihen ausgelegt waren, bewacht nur von leichtem grauen Schein und einer jung aussehenden Verkäuferin in einer stark verkürzten Version von Hexenroben, die ihre Knie und Ellbogen entblößten.

Harry untersuchte das magische Äquivalent eines Erste-Hilfe-Kastens, das \emph{Notfall-Heiler-Paket Plus}. Darin befanden sich zwei selbstspannende Tourniquets. Eine Spritze mit etwas, das aussah wie flüssiges Feuer, das die Zirkulation in einem behandelten Bereich drastisch verlangsamen und gleichzeitig die Sauerstoffzufuhr des Blutes für bis zu drei Minuten aufrechterhalten sollte, wenn man verhindern wollte, dass sich ein Gift im Körper ausbreitete. Ein weißes Tuch, das man über einen Teil des Körpers wickeln konnte, um Schmerzen vorübergehend zu betäuben. Dazu kamen noch jede Menge anderer Gegenstände, die Harry überhaupt nicht verstand, wie die "Dementor-Expositionsbehandlung", die wie gewöhnliche Schokolade aussah und roch. Oder der "Bafflesnaffle Gegenzauber", der wie ein kleines, zitterndes Ei aussah und ein Schild trug, auf dem stand, wie man es jemandem ins Nasenloch stopfen konnte.

"Ein definitiver Kauf für fünf Galleonen, finden Sie nicht auch?" sagte Harry zu Professor McGonagall,

und die jugendliche Verkäuferin, die in der Nähe schwebte, nickte eifrig. Harry hatte erwartet, dass der Professor eine Art anerkennende Bemerkung über seine Umsicht und Vorbereitung machen würde. Was er stattdessen bekam, konnte man nur als den Bösen Blick bezeichnen.

"Und warum", sagte Professor McGonagall mit schwerer Skepsis, "glauben Sie, dass Sie einen Heilerkoffer brauchen, junger Mann?"

\emph{(Nach dem unglücklichen Vorfall im Zaubertränke-Laden versuchte Professor McGonagall zu vermeiden, "Mr~Potter" zu sagen, wenn jemand anderes in der Nähe war.)}

Harrys Mund öffnete und schloss sich.

"Ich erwarte nicht, dass ich es brauche! Es ist nur für den Fall!"

"Nur für den Fall von was?"

Harrys Augen weiteten sich.

"Sie denken, ich habe vor, etwas Gefährliches zu tun, und deshalb will ich einen Verbandskasten?"

Ein Blick aus grimmigem Misstrauen und ironischem Unglauben war die Antwort.

"Großer Schotte!„ \emph{(„Great Scott!“ von Dr~Brown aus dem Film Zurück in die Zukunft, anm. des Übersetzers)}, sagte Harry.

"Haben Sie das auch gedacht, als ich den Federfalltrank, das Algenkraut und die Flasche mit den Nahrungs- und Wasserpillen gekauft habe?"

"Ja."

Harry schüttelte erstaunt den Kopf. "Was glauben Sie denn, was für einen Plan ich hier habe?"

"Ich weiß es nicht", sagte Professor McGonagall düster, "aber er endet entweder damit, dass du eine Tonne Silber an Gringotts lieferst, oder mit der Weltherrschaft."

"Weltherrschaft ist so ein hässlicher Ausdruck. Ich ziehe es vor, es Weltoptimierung zu nennen."

Dieser urkomische Scherz konnte die Hexe, die ihm den einen Blick des Verderbens zuwarf, nicht beruhigen.

"Wow", sagte Harry, als er merkte, dass sie es ernst meinte. "Sie glauben es wirklich. Sie glauben wirklich, dass ich vorhabe, etwas Gefährliches zu tun."

"Ja."

"Als ob das der einzige Grund wäre, warum jemand einen Erste-Hilfe-Kasten kaufen würde? Verstehen Sie mich nicht falsch, Professor McGonagall, aber mit welcher Art von verrückten Kindern haben Sie es normalerweise zu tun?"

"Gryffindors",

spuckte Professor McGonagall, das Wort trug eine Art von Bitterkeit und Verzweiflung in sich, die wie ein ewiger Fluch auf allen jugendlichen Enthusiasmus und gute Laune fiel.

"Stellvertretende Schulleiterin Professor Minerva McGonagall", sagte Harry und stemmte die Hände streng in die Hüften. "Ich werde nicht in Gryffindor sein -"

An dieser Stelle warf die stellvertretende Schulleiterin etwas darüber ein, dass sie, wenn er es wäre, \emph{herausfinden würde, wie man einen Hut tötet}, eine seltsame Bemerkung, die Harry kommentarlos passieren ließ, obwohl die Verkäuferin einen plötzlichen Hustenanfall zu haben schien.

"- Ich werde in Ravenclaw sein. Und wenn Sie wirklich glauben, dass ich vorhabe, etwas Gefährliches zu tun, dann, um hier ehrlich zu sein, verstehen Sie mich überhaupt nicht. Ich mag keine Gefahr, sie ist beängstigend. Ich bin umsichtig. Ich bin behutsam. Ich bereite mich auf unvorhergesehene Eventualitäten vor. Wie meine Eltern mir immer vorgesungen haben:

Sei vorbereitet! Das ist das Marschlied der Pfadfinder! Sei bereit! Denn durch das Leben marschierst du mit! Sei nicht nervös, sei nicht aufgeregt, sei nicht ängstlich - sei vorbereitet!„ \emph{(„Be prepared“ von den „Boy Scouts“, eine Art Pfadfindertruppe,} \emph{https://genius.com/Tom-lehrer-be-prepared-lyrics} \emph{, anm. des Übersetzers)}

\emph{(Harrys Eltern hatten ihm tatsächlich nur diese eine Zeile des Tom-Lehrer-Liedes vorgesungen, den Rest kannte Harry nicht).}

Professor McGonagalls Haltung hatte sich etwas aufgeweicht - allerdings hauptsächlich, als Harry gesagt hatte, dass er nach Ravenclaw gehen würde.

"Auf welche Art von Eventualität glauben Sie, dass dieses Set Sie vorbereiten könnte, junger Mann?"

"Eine meiner Klassenkameradinnen wird von einem schrecklichen Monster gebissen, und während ich verzweifelt in meinem Beutel nach etwas krame, das ihr helfen könnte, sieht sie mich traurig an und sagt mit ihrem letzten Atemzug:

'Warum warst du nicht vorbereitet?'

Und dann stirbt sie, und ich weiß, als sich ihre Augen schließen, dass sie mir nie verzeihen wird -"

Harry hörte die Verkäuferin keuchen, und er schaute auf, um zu sehen, wie sie ihn mit zusammengepressten Lippen anstarrte. Dann wirbelte die junge Frau herum und flüchtete in die tieferen Nischen des Ladens. Was…?

Professor McGonagall griff nach unten und nahm Harrys Hand in ihre, sanft, aber bestimmt, und zog ihn aus der Hauptstraße der Winkelgasse heraus und führte ihn in eine Gasse zwischen zwei Geschäften, die mit schmutzigen Ziegeln gepflastert war und in einer Wand aus festem schwarzen Schmutz endete. Die große Hexe richtete ihren Zauberstab auf die Hauptstraße und sprach:

"Quietus",

und ein Schirm der Stille senkte sich um sie herum, der alle Straßengeräusche ausblendete.

\emph{Was habe ich nur falsch gemacht}… Professor McGonagall drehte sich um und betrachtete Harry. Sie hatte kein volles erwachsenes \emph{Du-hast-etwas-falsch-gemacht} Gesicht aber ihr Ausdruck war flach und kontrolliert.

"Sie müssen bedenken, Mr~Potter", sagte sie, "dass es in diesem Land vor nicht einmal zehn Jahren einen Krieg gab. Jeder hat jemanden verloren, und von Freunden zu sprechen, die in Ihren Armen sterben - das tut man nicht leichtfertig."

"Ich - ich hatte nicht die Absicht -"

Die Schlussfolgerung fiel wie ein Steinschlag in Harrys außergewöhnlich lebhafte Fantasie. Er hatte davon gesprochen, dass jemand seinen letzten Atemzug gehaucht hatte - und dann war die Verkäuferin weggelaufen - und der Krieg war vor zehn Jahren zu Ende gegangen, sodass das Mädchen höchstens acht oder neun Jahre alt gewesen sein konnte, als -

"Es tut mir leid, ich wollte nicht …"

Harry verschluckte sich und wandte sich ab, um dem Blick der älteren Hexe zu entgehen, aber eine Mauer aus Dreck versperrte ihm den Weg und er hatte seinen Zauberstab noch nicht dabei.

"Es tut mir leid, es tut mir leid, es tut mir leid!"

Ein schweres Seufzen kam von hinter ihm.

"Ich weiß, dass es das tut, Mr~Potter."

Harry wagte es, einen Blick hinter sich zu werfen. Professor McGonagall schien nur noch traurig zu sein.

"Es tut mir leid",

sagte Harry noch einmal und fühlte sich elendig.

"Ist so etwas mit Ihnen passiert -",

und dann schloss Harry seine Lippen und schlug sich sicherheitshalber eine Hand auf den Mund. Das Gesicht der älteren Hexe wurde noch ein wenig trauriger.

"Sie müssen lernen zu denken, bevor Sie sprechen, Mr~Potter, oder Sie gehen ohne viele Freunde durchs Leben. Das war das Schicksal vieler Ravenclaw, und ich hoffe, es wird nicht das Ihre sein."

Harry wollte einfach wegrennen. Er wollte einen Zauberstab zücken und die ganze Sache aus dem Gedächtnis von Professor McGonagall löschen, wieder bei ihr vor dem Laden sein, dafür sorgen, dass es nicht passiert -

"Aber um Ihre Frage zu beantworten, Mr~Potter, nein, so etwas ist mir noch nie passiert. Sicherlich habe ich ein- oder siebenmal einen Freund sein Leben aushauchen sehen. Aber nicht einer von ihnen hat mich jemals verflucht, als sie starben, und ich habe nie gedacht, dass sie mir nicht verzeihen würden. Warum würden Sie so etwas sagen, Mr~Potter? Warum sollten Sie es überhaupt denken?"

"Ich, ich, ich", schluckte Harry. "Es ist nur so, dass ich immer versuche, mir das Schlimmste vorzustellen, was passieren könnte",

und vielleicht hatte er auch ein bisschen herumgescherzt, aber er hätte sich lieber die eigene Zunge abgebissen, als das jetzt zu sagen.

"Was?", sagte Professor McGonagall. "Aber warum?"

"Damit ich es verhindern kann!"

"Mr~Potter …", die Stimme der älteren Hexe brach ab. Dann seufzte sie und kniete sich neben ihn.

"Mr~Potter", sagte sie, nun sanft,

"es ist nicht Ihre Aufgabe, sich um die Schüler in Hogwarts zu kümmern. Es ist meine. Ich werde nicht zulassen, dass Ihnen oder irgendjemand anderem etwas Schlimmes zustößt. Hogwarts ist der sicherste Ort für magische Kinder in der ganzen Zaubererwelt, und Madam Pomfrey hat ein komplettes Krankenhaus. Sie brauchen überhaupt keinen Heilerkoffer, geschweige denn einen für 5 Galleonen."

"Ich brauche ihn schon!" platzte Harry heraus. "Nirgendwo ist es vollkommen sicher! Und was ist, wenn meine Eltern einen Herzinfarkt haben oder in einen Unfall verwickelt werden, wenn ich über Weihnachten nach Hause fahre - Madam Pomfrey wird nicht da sein, ich werde selbst einen Heilerkoffer brauchen -"

"Was in Merlins Namen…"

sagte Professor McGonagall. Sie stand auf und schaute auf Harry herab, mit einem Ausdruck, der zwischen Verärgerung und Sorge schwankte.

"Es gibt keinen Grund, über solch schreckliche Dinge nachzudenken, Mr~Potter!"

Harrys Gesichtsausdruck verzog sich zu Bitterkeit, als er das hörte.

"Doch, das muss man! Wenn man nicht nachdenkt, tut man nicht nur sich selbst weh, sondern auch anderen Menschen!"

Professor McGonagall öffnete ihren Mund, dann schloss sie ihn wieder. Die Hexe rieb sich den Nasenrücken und sah nachdenklich aus.

"Mr~Potter … wenn ich Ihnen anbieten würde, Ihnen eine Weile zuzuhören … gibt es irgendetwas, worüber Sie mit mir reden möchten?"

"Worüber?"

"Darüber, warum Sie davon überzeugt sind, dass Sie immer auf der Hut sein müssen, dass Ihnen nichts Schreckliches passiert."

Harry starrte sie verwirrt an. Das war doch eigentlich ein selbstverständliches Axiom.

"Nun …"

Er versuchte, seine Gedanken zu ordnen. Wie sollte er sich einer Professor-Hexe erklären, wenn sie nicht einmal die Grundlagen kannte?

"Muggelforscher haben herausgefunden, dass Menschen immer sehr optimistisch sind, verglichen mit der Realität. Zum Beispiel sagen sie, dass etwas zwei Tage dauert und es dauert zehn Tage, oder sie sagen, dass es zwei Monate dauert und es dauert über fünfunddreißig Jahre.

In einem Experiment wurden Studenten nach Zeiten gefragt, zu denen sie 50 \%, 75 \% und 99 \% sicher waren, dass sie ihre Hausaufgaben erledigen würden, und nur 13 \%, 19 \% und 45 \% der Studenten wurden zu diesen Zeiten fertig.

Und sie fanden heraus, dass der Grund dafür war, dass sie, als sie eine Gruppe nach ihren Bester-Fall-Schätzungen fragten, wenn alles so gut wie möglich lief, und eine andere Gruppe nach ihren Durchschnitts-Schätzungen, wenn alles wie immer lief, Antworten zurückbekamen, die statistisch nicht unterscheidbar waren.

Sehen Sie, wenn Sie jemanden fragen, was er im Normalfall erwartet, visualisiert er das, was wie die Linie der maximal besten Wahrscheinlichkeit bei jedem Schritt auf dem Weg aussieht - alles läuft nach Plan, ohne Überraschungen.

Da aber mehr als die Hälfte der Studenten nicht zu dem Zeitpunkt fertig wurde, an dem sie sich zu 99 \% sicher waren, dass sie fertig werden würden, liefert die Realität in der Regel Ergebnisse, die etwas schlechter sind als das "Schlechtest-Mögliche-Szenario".

Das nennt man den Planungsirrtum, und der beste Weg, ihn zu beheben, ist, sich zu fragen, wie lange die Dinge beim letzten Mal gedauert haben, als man sie versucht hat.

Das nennt man die Außensicht statt der Innensicht. Aber wenn Sie etwas Neues machen und das nicht tun können, müssen Sie einfach sehr, sehr, sehr pessimistisch sein. So pessimistisch, dass die Realität im Schnitt besser ausfällt als die Schätzung.

Es ist wirklich sehr schwer, so pessimistisch zu sein, dass man eine gute Chance hat, das reale Leben zu unterschätzen.

Ich gebe mir zum Beispiel große Mühe, pessimistisch zu sein, und stelle mir vor, dass einer meiner Klassenkameraden gebissen wird, aber was tatsächlich passiert, ist, dass die überlebenden Todesser die ganze Schule angreifen, um mich zu erwischen.

Aber wenn wir schon dabei sind -"

"Stopp", sagte Professor McGonagall.

Harry blieb stehen. Er wollte gerade darauf hinweisen, dass sie wenigstens wussten, dass der Dunkle Lord nicht angreifen würde, da er ja tot war.

"Ich glaube, ich habe mich vielleicht nicht klar ausgedrückt",

sagte die Hexe, ihre präzise schottische Stimme klang noch vorsichtiger.

"Ist Ihnen persönlich etwas zugestoßen, das Sie erschreckt hat, Mr~Potter?"

"Was mir persönlich passiert ist, ist nur ein anekdotischer Beweis", erklärte Harry. "Es hat nicht das gleiche Gewicht wie ein replizierter, begutachteter Artikel über eine kontrollierte Studie mit zufälliger Zuordnung, vielen Probanden, großen Effektgrößen und starker statistischer Signifikanz."

Professor McGonagall kniff sich in den Nasenrücken, atmete ein und aus.

"Ich würde trotzdem gerne davon hören", sagte sie.

"Ähm …" sagte Harry.

Er holte tief Luft.

"Es gab ein paar Überfälle in unserer Nachbarschaft, und meine Mutter bat mich, eine Pfanne, die sie sich geliehen hatte, zu einem Nachbarn zwei Straßen weiter zurückzubringen, und ich sagte, ich wolle das nicht tun, weil ich überfallen werden könnte, und sie sagte:

\emph{'Harry, sag so etwas nicht!}'

Als ob es passieren würde, wenn ich darüber nachdenke, also wäre ich sicher, wenn ich nicht darüber reden würde. Ich versuchte zu erklären, warum ich nicht beruhigt war, und sie ließ mich die Pfanne trotzdem rübertragen. Ich war zu jung, um zu wissen, wie statistisch unwahrscheinlich es war, dass ein Straßenräuber es auf mich abgesehen hatte, aber ich war alt genug, um zu wissen, dass das Nicht-Denken an etwas nicht verhindert, dass es passiert, also hatte ich wirklich Angst."

"Sonst nichts?" sagte Professor McGonagall nach einer Pause, als klar wurde, dass Harry fertig war.

"Es ist sonst nichts weiter mit dir passiert?"

"Ich weiß, es hört sich nicht nach viel an", verteidigte sich Harry. "Aber es war einfach einer dieser kritischen Lebensmomente, verstehen Sie? Ich meine, ich wusste, dass das Nichtdenken an etwas nicht verhindert, dass es passiert, das wusste ich, aber ich konnte sehen, dass Mum wirklich so dachte."

Harry hielt inne und kämpfte mit der Wut, die wieder aufzusteigen begann, als er daran dachte.

"Sie wollte nicht zuhören. Ich habe versucht, es ihr zu sagen, ich habe sie angefleht, mich nicht wegzuschicken, und sie hat es weggelacht. Alles, was ich sagte, behandelte sie wie eine Art großen Witz …"

Harry zwang die schwarze Wut wieder hinunter.

"Da wurde mir klar, dass alle, die mich beschützen sollten, in Wirklichkeit verrückt waren, und dass sie mir nicht zuhören würden, egal wie sehr ich sie anflehen würde, und dass ich mich nie darauf verlassen konnte, dass sie irgendetwas richtig machen würden."

Manchmal reichten gute Absichten nicht aus, manchmal musste man zurechnungsfähig sein…

Es herrschte eine lange Stille.

Harry nahm sich die Zeit, tief durchzuatmen und sich zu beruhigen. Es hatte keinen Sinn, wütend zu werden. Es gab keinen Grund, wütend zu werden. Alle Eltern waren so, kein Erwachsener würde sich so weit herablassen, sich auf Augenhöhe mit einem Kind zu stellen und zuzuhören, seine genetischen Eltern wären nicht anders gewesen. Vernunft war ein winziger Funke in der Nacht, eine verschwindend seltene Ausnahme von der Regel des Wahnsinns, also gab es keinen Grund, wütend zu werden. Harry mochte sich selbst nicht, wenn er wütend war.

"Danke, dass Sie das mit mir teilen, Mr~Potter",

sagte Professor McGonagall nach einer Weile. Auf ihrem Gesicht lag ein seltsamer Ausdruck (fast genau derselbe Ausdruck, der auf Harrys eigenem Gesicht erschienen war, während er mit dem Beutel experimentierte, wenn Harry sich nur in einem Spiegel gesehen hätte, um das zu erkennen).

"Ich werde darüber nachdenken müssen."

Sie wandte sich der Gassenöffnung zu und hob ihren Zauberstab -

"Ähm", sagte Harry, "können wir jetzt den Heilerkoffer holen?"

Die Hexe hielt inne und schaute ihn wieder fest an.

"Und wenn ich nein sage - dass es zu teuer ist und du es nicht brauchen wirst - was dann?"

Harrys Gesicht verzog sich vor Bitterkeit.

"Genau das, was Sie denken, Professor McGonagall. Genau das, was Sie denken. Ich schließe daraus, dass Sie ein weiterer verrückter Erwachsener sind, mit dem ich nicht reden kann, und ich fange an zu planen, wie ich trotzdem einen Heilerkoffer in die Finger bekomme."

"Ich bin dein Vormund auf dieser Reise",

sagte Professor McGonagall mit einem Hauch von Gefahr in der Stimme.

"Ich werde nicht zulassen, dass Sie mich herumschubsen."

"Ich verstehe",

sagte Harry. Er hielt den Groll aus seiner Stimme heraus und sagte auch nichts von den anderen Dingen, die ihm in den Sinn kamen. Professor McGonagall hatte ihm gesagt, er solle nachdenken, bevor er sprach. Daran würde er sich morgen wahrscheinlich nicht mehr erinnern, aber er konnte sich zumindest fünf Minuten lang daran halten.

Der Zauberstab der Hexe machte einen leichten Kreis in ihrer Hand, und die Geräusche der Winkelgasse kamen zurück.

"In Ordnung, junger Mann", sagte sie. "Holen wir uns den Heilerkoffer."

Harrys Kinnlade fiel vor Überraschung herunter. Dann eilte er ihr hinterher und stolperte fast in seiner plötzlichen Eile.

\textbf{Später}.

Der Laden war noch genauso, wie sie ihn verlassen hatten, erkennbare und unerkennbare Artikel lagen noch immer auf der schrägen Holzauslage, der graue Schein schützte Sie noch immer und die Verkäuferin war wieder in ihrer alten Position.

Die Verkäuferin sah auf, als sie sich näherten, ihr Gesicht zeigte Überraschung.

"Es tut mir leid", sagte sie, als sie näher kamen, und Harry sprach fast im gleichen Moment: "Ich entschuldige mich für -"

Sie brachen ab und sahen sich an, und dann lachte die Verkäuferin ein wenig.

"Ich wollte Sie nicht in Schwierigkeiten mit Professor McGonagall bringen",

sagte sie. Ihre Stimme senkte sich verschwörerisch.

"Ich hoffe, sie war nicht zu furchtbar zu dir."

"Della!", sagte Professor McGonagall und klang empört.

"Beutel mit Gold", sagte Harry zu seinem Beutel und sah dann wieder zu der Verkäuferin auf, während er fünf Galleonen abzählte.

"Keine Sorge, ich verstehe, dass sie nur so schrecklich zu mir ist, weil sie mich liebt."

Er zählte der Verkäuferin fünf Galleonen ab, während Professor McGonagall etwas Unwichtiges murmelte.

"Ein Notfall-Heilpaket Plus, bitte."

Es war tatsächlich irgendwie zermürbend zu sehen, wie die sich öffnende Lippe den aktenkoffergroßen Arzneikasten verschluckte. Harry fragte sich, was passieren würde, wenn er versuchen würde, selbst in den Beutel zu klettern, da nur die Person, die etwas hineingesteckt hatte, es auch wieder herausnehmen konnte.

Als der Beutel damit fertig war, seinen hart erkämpften Kauf zu… fressen…, schwor Harry, dass er danach ein kleines Rülpsgeräusch hörte. Das musste mit Absicht hereingezaubert worden sein. Die alternative Hypothese war zu schrecklich, um sie in Betracht zu ziehen… tatsächlich konnte Harry nicht einmal an eine alternative Hypothese denken.

Harry blickte wieder zum Professor auf, als sie wieder durch die Winkelgasse gingen.

"Wohin als nächstes?"

Professor McGonagall zeigte auf einen Laden, der aussah, als wäre er aus Fleisch statt aus Ziegeln und mit Fell statt mit Farbe bedeckt.

"Kleine Haustiere sind in Hogwarts erlaubt - du könntest zum Beispiel eine Eule bekommen, um Briefe zu verschicken -"

"Kann ich einen Knut oder so bezahlen und eine Eule mieten, wenn ich Post verschicken muss?"

"Ja", sagte Professor McGonagall.

"Dann denke ich, mit Nachdruck nein."

Professor McGonagall nickte, als würde sie einen Punkt abhaken.

"Darf ich fragen, warum nicht?"

"Ich hatte mal einen Stein als Haustier. Er ist gestorben."

"Du glaubst nicht, dass du dich um ein Haustier kümmern könntest?"

"Ich könnte schon", sagte Harry, "aber ich würde mich den ganzen Tag damit beschäftigen, ob ich daran gedacht habe, ihn zu füttern, oder ob er langsam in seinem Käfig verhungert und sich fragt, wo sein Herrchen ist und warum es kein Futter gibt."

"Die arme Eule", sagte die ältere Hexe mit sanfter Stimme. "So allein gelassen. Ich frage mich, was sie tun würde."

"Nun, ich nehme an, sie würde richtig hungrig werden und versuchen, sich einen Weg aus dem Käfig oder der Kiste oder was auch immer zu beißen, obwohl sie wahrscheinlich nicht viel Glück dabei hätte -"

Harry hielt kurz inne. Die Hexe fuhr fort, immer noch mit dieser sanften Stimme.

"Und was würde danach mit ihm passieren?"

"Entschuldigen Sie",

sagte Harry und griff nach Professor McGonagall, um sie sanft, aber bestimmt bei der Hand zu nehmen und sie in eine weitere Gasse zu lenken;

nachdem er so vielen Gratulanten ausgewichen war, war der Vorgang fast unmerklich zur Routine geworden.

"Bitte sprechen Sie den Schweigezauber."

"Quietus."

Harrys Stimme zitterte. "Diese Eule repräsentiert mich nicht, meine Eltern haben mich nie in einen Schrank gesperrt und zum Verhungern zurückgelassen, ich habe keine Verlassensängste und mir gefällt der Trend Ihrer Gedanken nicht, Professor McGonagall!"

Die Hexe blickte ernst auf ihn herab.

"Und welche Gedanken sollen das sein, Mr~Potter?"

"Sie denken, ich war", Harry hatte Mühe, es auszusprechen, "ich wurde missbraucht?!"

"Wurden Sie?"

"Nein!" Harry schrie.

"Nein, das wurde ich nie! Hältst du mich für dumm? Ich weiß über Kindesmissbrauch Bescheid, ich weiß alles über unangemessene Berührungen und all das, und wenn so etwas passieren würde, würde ich die Polizei rufen! Und dem Schulleiter melden! Und den Sozialdienst im Telefonbuch suchen! Und es Opa und Oma und Mrs~Figg erzählen! Aber meine Eltern haben so etwas nie getan, niemals, niemals, niemals!

Wie kannst du es wagen, so etwas zu behaupten!"

Die ältere Hexe starrte ihn unverwandt an.

"Es ist meine Pflicht als stellvertretende Schulleiterin, möglichen Anzeichen von Missbrauch bei den Kindern unter meiner Obhut nachzugehen."

Harrys Zorn geriet außer Kontrolle und verwandelte sich in pure, schwarze Wut.

"Wage es nicht, auch nur ein Wort von diesen Andeutungen zu jemand anderem zu sagen! Niemandem, haben Sie mich verstanden, McGonagall? Eine solche Anschuldigung kann Menschen ruinieren und Familien zerstören, selbst wenn die Eltern völlig unschuldig sind! Ich habe davon in den Zeitungen gelesen!"

Harrys Stimme steigerte sich zu einem schrillen Schrei.

"Das System weiß nicht, wie es aufhören soll, es glaubt weder den Eltern noch den Kindern, wenn sie sagen, dass nichts passiert ist! Wagen Sie es nicht, meiner Familie damit zu drohen! Ich werde nicht zulassen, dass Sie mein Zuhause zerstören!"

"Harry",

sagte die ältere Hexe leise, und sie streckte eine Hand nach ihm aus -

Harry machte einen schnellen Schritt zurück, und seine Hand schnappte hoch und schlug ihre weg.

McGonagall erstarrte, dann zog sie ihre Hand zurück und machte einen Schritt rückwärts.

"Harry, es ist alles in Ordnung", sagte sie. "Ich glaube dir."

"Tust du das?", zischte Harry.

Die Wut rauschte immer noch durch sein Blut.

"Oder wartest du nur darauf, von mir wegzukommen, damit du die Papiere einreichen kannst?"

"Harry, ich habe dein Haus gesehen. Ich habe dich mit deinen Eltern gesehen. Sie lieben dich. Du liebst sie. Ich glaube dir, wenn du sagst, dass deine Eltern dich nicht missbrauchen. Aber ich musste fragen, denn hier ist etwas Seltsames am Werk."

Harry starrte sie an.

"Was zum Beispiel?"

"Harry, ich habe in meiner Zeit in Hogwarts viele missbrauchte Kinder gesehen, es würde dir das Herz brechen zu wissen, wie viele. Und wenn du glücklich bist, benimmst du dich nicht wie eines dieser Kinder, ganz und gar nicht. Du lächelst Fremde an, du umarmst Leute, ich habe meine Hand auf deine Schulter gelegt und du hast nicht gezuckt. Aber manchmal, nur manchmal, sagst oder tust du etwas, das sehr nach… jemandem aussieht, der seine ersten elf Jahre in einem Schrank eingesperrt war. Nicht die liebevolle Familie, die ich gesehen habe."

Professor McGonagall legte den Kopf schief, ihr Ausdruck wurde wieder rätselhaft. Harry nahm dies auf und verarbeitete es. Die schwarze Wut begann zu schwinden, als ihm dämmerte, dass man ihm respektvoll zuhörte und dass seine Familie nicht in Gefahr war.

"Und wie erklären Sie sich Ihre Beobachtungen, Professor McGonagall?"

"Ich weiß es nicht", sagte sie. "Aber es ist möglich, dass Ihnen etwas zugestoßen ist, an das Sie sich nicht erinnern."

Wut stieg wieder in Harry auf. Das klang nur zu sehr nach dem, was er in den Zeitungsberichten über zerrüttete Familien gelesen hatte.

"Verdrängte Erinnerung ist ein Haufen Pseudowissenschaft! Menschen verdrängen keine traumatischen Erinnerungen, sie erinnern sich für den Rest ihres Lebens nur zu gut an sie!"

„Nein, Mr~Potter. Es gibt einen Zauberspruch, der '\emph{Vergessen'} lässt.“

Harry erstarrte auf der Stelle.

"Ein Zauber, der Erinnerungen auslöscht?"

Die ältere Hexe nickte.

"Aber nicht alle Auswirkungen des Erlebten, wenn Sie verstehen, was ich meine, Mr~Potter."

Ein Schauer lief Harry über den Rücken. Diese Hypothese … konnte nicht einfach widerlegt werden.

"Aber meine Eltern konnten das nicht tun!"

"In der Tat nicht", sagte Professor McGonagall. "Dazu hätte es jemanden aus der Welt der Zauberer gebraucht. Es gibt … keine Möglichkeit, sicher zu sein, fürchte ich."

Harrys rationale Fähigkeiten begannen wieder hochzufahren.

"Professor McGonagall, wie sicher sind Sie sich bei Ihren Beobachtungen, und welche alternativen Erklärungen könnte es noch geben?"

Die Hexe öffnete ihre Hände, als wolle sie deren Leere zeigen.

"Sicher? Ich bin mir bei nichts sicher, Mr~Potter. In meinem ganzen Leben habe ich noch nie jemanden wie Sie getroffen. Manchmal scheinen Sie einfach nicht elf Jahre alt zu sein oder gar menschlich."

Harrys Augenbrauen hoben sich gen Himmel -

"Es tut mir leid!" sagte Professor McGonagall schnell.

"Es tut mir sehr leid, Mr~Potter. Ich habe versucht, etwas klarzustellen, und ich fürchte, das hat sich anders angehört, als ich es im Sinn hatte -"

"Ganz im Gegenteil, Professor McGonagall", sagte Harry und lächelte langsam. "Ich fasse es als ein großes Kompliment auf. Aber würde es Ihnen etwas ausmachen, wenn ich eine alternative Erklärung anbieten würde?"

"Bitte tun Sie das."

"Kinder sollten nicht viel klüger sein als ihre Eltern", sagte Harry. "Oder vielleicht vernünftiger -

mein Vater könnte mich wahrscheinlich überlisten, wenn er, Sie wissen schon, es tatsächlich versuchen würde, anstatt seine erwachsene Intelligenz hauptsächlich dazu zu benutzen, sich neue Gründe auszudenken, seine Meinung nicht zu ändern -"

Harry hielt inne.

"Ich bin zu klug, Professor. Zu normalen Kindern habe ich nichts zu sagen. Erwachsene respektieren mich nicht genug, um wirklich mit mir zu reden. Und ehrlich gesagt, selbst wenn sie es täten, würden sie nicht so klug klingen wie Richard Feynman, also könnte ich stattdessen auch etwas lesen, was Richard Feynman geschrieben hat. Ich bin isoliert, Professor McGonagall. Ich war mein ganzes Leben lang isoliert. Vielleicht hat das einige der gleichen Auswirkungen, wie in einem Schrank eingesperrt zu sein. Und ich bin zu intelligent, um zu meinen Eltern aufzuschauen, wie es Kinder tun sollten. Meine Eltern lieben mich, aber sie fühlen sich nicht verpflichtet, auf Vernunft zu reagieren, und manchmal habe ich das Gefühl, dass sie die Kinder sind - Kinder, die nicht zuhören wollen und die absolute Autorität über meine ganze Existenz haben. Ich versuche, nicht zu verbittert darüber zu sein, aber ich versuche auch, ehrlich zu mir selbst zu sein, also, ja, ich bin verbittert. Und ich habe auch ein Wutmanagement-Problem, aber ich arbeite daran. Das ist alles."

"Das ist alles?"

Harry nickte fest.

"Das ist alles. Sicherlich, Professor McGonagall, selbst im magischen Großbritannien ist die normale Erklärung immer eine Überlegung wert?"

*****

Es war später am Tag, die Sonne senkte sich am Sommerhimmel und die Käufer begannen, sich von den Straßen zu entfernen. Einige Geschäfte hatten bereits geschlossen; Harry und Professor McGonagall hatten seine Lehrbücher bei Flourish and Blotts kurz vor Ablauf der Frist gekauft. Mit einer kleinen Explosion, als Harry nach dem Stichwort "Arithmetik" suchte und feststellte, dass die Lehrbücher

für das siebte Jahr nichts mathematisch Fortgeschritteneres als Trigonometrie enthielten.

Doch in diesem Moment waren Träume von einfach zu erlangenden Forschungsergebnissen weit weg von Harrys Gedanken. In diesem Moment gingen die beiden aus dem Zauberstabladen '\emph{Ollivanders} und Harry starrte auf seinen Zauberstab.

Er hatte ihn geschwungen und dabei bunte Funken erzeugt, was ihn nach allem, was er bisher gesehen hatte, eigentlich nicht besonders schockieren sollte, aber irgendwie -

ich kann zaubern. Ich. Also ich persönlich. Ich bin magisch; ich bin ein Zauberer.

Er hatte gespürt, wie die Magie seinen Arm hinaufströmte, und in diesem Augenblick wurde ihm klar, dass er diesen Sinn schon immer gehabt hatte, dass er ihn sein ganzes Leben lang besessen hatte, den Sinn, der nicht Sehen oder Hören oder Riechen oder Schmecken oder Tasten war, sondern nur Magie.

Wie wenn man Augen hat, sie aber immer geschlossen hält, so dass man nicht einmal merkt, dass man die Dunkelheit sieht; und dann, eines Tages, öffnet sich das Auge und sieht die Welt.

Der Schock darüber war durch ihn hindurchgeflossen, hatte Teile von ihm selbst berührt, sie geweckt und war dann in Sekundenschnelle abgeklungen; zurück blieb nur das sichere Wissen, dass er jetzt ein Zauberer war, und dass er es immer gewesen war, und dass er es auf eine seltsame Weise sogar immer gewusst hatte. Und -

"Es ist in der Tat sehr merkwürdig, dass du für diesen Zauberstab bestimmt bist, wenn sein Bruder, nun, sein Bruder dir diese Narbe gegeben hat."

Das konnte unmöglich ein Zufall sein. Es waren Tausende von Zauberstäben in diesem Laden gewesen. Nun, okay, eigentlich konnte es Zufall sein, es gab sechs Milliarden Menschen auf der Welt und tausende Zufälle passierten jeden Tag.

Aber das Bayes'sche Theorem besagte, dass jede vernünftige Hypothese, die es wahrscheinlicher als tausend zu eins machte, dass er mit dem Bruder des Zauberstabs des Dunklen Lords enden würde, eher kein Zufall war.

Professor McGonagall hatte einfach gesagt, wie seltsam, und es dabei belassen, was Harry in einen Schockzustand angesichts der schieren, überwältigenden Nicht-Neugier von Zauberern und Hexen versetzt hatte.

In keiner vorstellbaren Welt hätte Harry einfach "\emph{Hm}" gesagt und wäre aus dem Laden gegangen, ohne auch nur zu versuchen, eine Hypothese für das, was vor sich ging, aufzustellen.

Seine linke Hand hob sich und berührte seine Narbe. Was … genau …

"Du bist jetzt ein vollwertiger Zauberer", sagte Professor McGonagall. "Herzlichen Glückwunsch."

Harry nickte.

"Und was hältst du von der Welt der Zauberer?", fragte sie.

"Es ist seltsam", sagte Harry. "Ich sollte über alles nachdenken, was ich von der Magie gesehen habe … alles, von dem ich jetzt weiß, dass es möglich ist, und alles, von dem ich jetzt weiß, dass es eine Lüge ist, und all die Arbeit, die noch vor mir liegt, um es zu verstehen. Und doch finde ich mich selbst abgelenkt durch relative Nebensächlichkeiten wie",

Harry senkte die Stimme,

"die ganze \emph{Junge-der-lebte-Sache}."

Es schien niemand in der Nähe zu sein, aber es hatte keinen Sinn, das Schicksal herauszufordern. Professor McGonagall sah ihn an.

"Wirklich? Was Sie nicht sagen."

Harry nickte.

"Ja. Es ist nur… merkwürdig. Herauszufinden, dass man Teil dieser großen Geschichte war, des Strebens, den großen und schrecklichen Dunklen Lord zu besiegen, und es ist bereits geschehen.

Beendet.

Völlig vorbei.

Als ob du Frodo Beutlin wärst und herausfindest, dass deine Eltern dich zum Schicksalsberg brachten und dich den Ring hineinwerfen ließen, als du ein Jahr alt warst, und du dich nicht einmal daran erinnern kannst."

Professor McGonagalls Lächeln war etwas starr geworden.

"Wissen Sie, wenn ich irgendjemand anderes wäre, überhaupt irgendjemand, würde ich mir wahrscheinlich ziemliche Sorgen machen, diesem Anfang gerecht zu werden.

\emph{Meine Güte, Harry, was habst du denn gemacht, seit du den Dunklen Lord besiegt haben? Deine eigene Buchhandlung? Das ist ja toll! Sag mal, wusstest du, dass ich mein Kind nach dir benannt habe?}

Aber ich habe die Hoffnung, dass sich das nicht als Problem erweisen wird."

Harry seufzte. "Trotzdem … es ist fast so, dass ich mir wünsche, es gäbe ein paar lose Enden der Aufgabe, nur damit ich sagen kann, dass ich wirklich, du weißt schon, irgendwie teilgenommen habe."

"Oh?", sagte Professor McGonagall in einem merkwürdigen Ton. "An was hast du denn gedacht?"

"Nun, zum Beispiel haben Sie erwähnt, dass meine Eltern verraten wurden. Wer hat sie verraten?"

"Sirius Black", sagte die Hexe und zischte den Namen fast. "Er ist in Askaban. Ein Zauberergefängnis."

"Wie wahrscheinlich ist es, dass Sirius Black aus dem Gefängnis ausbricht und ich ihn aufspüren und in einer Art spektakulärem Duell besiegen muss, oder noch besser, ein hohes Kopfgeld auf ihn aussetzen und mich in Australien verstecken muss, während ich auf die Ergebnisse warte?"

Professor McGonagall blinzelte. Zweimal. "Unwahrscheinlich. Es ist noch nie jemand aus Askaban entkommen, und ich bezweifle, dass er der Erste sein wird."

Harry war ein wenig skeptisch gegenüber diesem "\emph{Niemand ist jemals aus Askaban entkommen}". Trotzdem, vielleicht konnte man mit Magie tatsächlich an ein 100\% perfektes Gefängnis herankommen, besonders wenn man einen Zauberstab hatte und sie nicht. Der beste Weg, um herauszukommen, wäre, gar nicht erst dorthin zu gehen.

"Also gut", sagte Harry. "Hört sich an, als wäre alles in Butter."

Er seufzte und wischte sich mit der Handfläche über den Kopf.

"Oder vielleicht ist der Dunkle Lord in dieser Nacht nicht wirklich gestorben. Nicht vollständig. Sein Geist verweilt, flüstert den Menschen in Albträumen zu, die in die wache Welt übergehen, und sucht nach einem Weg zurück in die lebenden Länder, die er zu zerstören geschworen hat, und nun sind er und ich gemäß der alten Prophezeiung in ein tödliches Duell verwickelt, bei dem der Gewinner verlieren und der Verlierer gewinnen wird -"

Professor McGonagalls Kopf drehte sich, und ihre Augen huschten umher, als ob sie die Straße nach Zuhörern absuchen wollte.

"Das war ein Scherz, Professor", sagte Harry etwas verärgert. Pfff, warum nahm sie immer alles so ernst -

In Harrys Magengrube machte sich langsam ein flaues Gefühl breit.

Professor McGonagall sah Harry mit einem ruhigen Blick an. Einem sehr, sehr ruhigen Ausdruck. Dann wurde ein Lächeln aufgesetzt.

"Natürlich sind Sie das, Mr~Potter."

Ach, Mist.

Wenn Harry die wortlose Schlussfolgerung, die ihm gerade in den Sinn gekommen war, hätte formalisieren müssen, wäre etwas dabei herausgekommen wie:

„\emph{Wenn ich die Wahrscheinlichkeit, dass Professor McGonagall das getan hat, was ich gerade gesehen habe, als Ergebnis einer sorgfältigen Selbstkontrolle einschätze, gegen die Wahrscheinlichkeitsverteilung für all die Dinge, die sie natürlich tun würde, wenn ich einen schlechten Scherz mache, dann ist dieses Verhalten ein signifikanter Beweis dafür, dass sie etwas verbirgt.“}

Aber was Harry eigentlich dachte, war:

\emph{'Ach, Mist.'}

Harry drehte seinen eigenen Kopf, um die Straße zu scannen. Nein, niemand in der Nähe.

"Er ist doch nicht tot, oder?", seufzte Harry.

"Mr~Potter -"

"Der Dunkle Lord ist am Leben. Natürlich ist er am Leben. Es war ein Akt völligen Optimismus, dass ich überhaupt etwas anderes geträumt habe. Ich muss von allen guten Geistern verlassen sein, ich kann mir nicht vorstellen, was ich mir dabei gedacht habe. Nur weil jemand sagte, dass seine Leiche verbrannt aufgefunden wurde, kann ich mir nicht vorstellen, warum ich dachte, er sei tot. Offensichtlich muss ich noch viel über die Kunst des richtigen Pessimismus lernen."

"Mr~Potter -"

"Sagen Sie mir wenigstens, dass es nicht wirklich eine Prophezeiung gibt…"

Professor McGonagall schenkte ihm immer noch dieses strahlende, starre Lächeln.

"Oh, das kann nicht ihr ernst sein!"

"Mr~Potter, Sie sollten keine Dinge erfinden, um sich Sorgen zu machen -"

"Wollen Sie mir das wirklich sagen? Stellen Sie sich meine Reaktion später vor, wenn ich herausfinde, dass es doch einen Grund zur Sorge gab."

Ihr starres Lächeln erlahmte. Harrys Schultern sackten in sich zusammen.

"Ich habe eine ganze Welt voller Magie zu analysieren. Ich habe keine Zeit für so etwas."

Dann schwiegen beide, als ein Mann in wallenden orangefarbenen Roben auf der Straße erschien und langsam an ihnen vorbeiging; Professor McGonagalls Augen verfolgten ihn unauffällig.

Harrys Mund bewegte sich, als er kräftig auf seiner Lippe kaute, und jemand, der genau hinsah, hätte einen winzigen Blutfleck bemerkt. Als der Mann im orangefarbenen Gewand in der Ferne verschwunden war, sprach Harry wieder, mit leisem Murmeln.

"Werden Sie mir jetzt die Wahrheit sagen, Professor McGonagall? Und machen Sie sich nicht die Mühe, es abzustreiten, ich bin nicht dumm."

"Sie sind elf Jahre alt, Mr~Potter!", sagte sie in einem rauen Flüsterton.

"Und damit ein Untermensch. Tut mir leid … einen Moment lang habe ich das vergessen."

"Das sind furchtbare und wichtige Angelegenheiten! Sie sind geheim, Mr~Potter! Es ist eine Katastrophe, dass Sie, noch ein Kind, schon so viel wissen! Du darfst es niemandem erzählen, verstehst du? Absolut niemandem!"

Wie es manchmal geschah, wenn Harry hinreichend wütend wurde, wurde sein Blut kalt, statt heiß, und eine schreckliche dunkle Klarheit senkte sich über seinen Verstand, wobei er mögliche Taktiken entwarf und deren Folgen mit eisernem Realismus abschätzte.

\emph{Weise Sie darauf hin, dass du ein Recht hast, es zu wissen}: \textbf{Fehlschlag}. Elfjährige Kinder haben in den Augen von McGonagall kein Recht, irgendetwas zu wissen.

\emph{Sag, dass ihr keine Freunde mehr sein werdet:} \textbf{Fehlschlag}. Sie schätzt deine Freundschaft nicht genügend.

\emph{Weise Sie darauf hin, dass du in Gefahr bist, wenn du es nicht weißt:} \textbf{Fehlschlag}. Es wurden bereits Pläne gemacht, die auf deiner Unwissenheit basieren. Die sichere Unannehmlichkeit des Umdenkens wird Ihnen weitaus unangenehmer erscheinen als

die bloße Ungewissheit, dass du zu Schaden kommen könntest.

\emph{Gerechtigkeit und Vernunft werden beide versagen.}

\emph{Du musst entweder etwas finden, was Sie haben, was sie will, oder etwas finden, was du tun kannst, was sie fürchtet}… \textbf{Ah}.

"Nun denn, Professor",

sagte Harry in einem tiefen Ton,

"es klingt, als hätte ich etwas, das Sie wollen. Sie können, wenn Sie wollen, mir die Wahrheit sagen, die ganze Wahrheit, und im Gegenzug werde ich Ihre Geheimnisse bewahren. Oder Sie können versuchen, mich unwissend zu halten, damit Sie mich als Bauernopfer benutzen können, in diesem Fall schulde ich Ihnen nichts!"

McGonagall blieb kurz auf der Straße stehen. Ihre Augen loderten und ihre Stimme verstieg sich zu einem regelrechten Zischen.

"Wie können Sie es wagen!"

"Nein, wie können Sie es wagen!?", zischte er ihr zurück.

"Du würdest mich erpressen?"

Harrys Lippen verzogen sich. "Ich biete Ihnen einen Gefallen an. Ich gebe Ihnen die Chance, das kostbare Geheimnis zu schützen. Wenn Sie sich weigern, habe ich ein natürliches Motiv, mich anderweitig zu erkundigen, nicht um Sie zu ärgern, sondern weil ich es wissen muss!

Überwinden Sie Ihre sinnlose Wut auf ein Kind, von dem Sie glauben, dass es Ihnen gehorchen muss, und Sie werden erkennen, dass jeder vernünftige Erwachsene dasselbe tun würde!

Betrachten Sie es aus meiner Perspektive!

Wie würden Sie sich fühlen, wenn es SIE treffen würde?"

Harry beobachtete McGonagall, beobachtete ihr schweres Atmen. Es kam ihm in den Sinn, dass es an der Zeit war, den Druck abzubauen, sie eine Weile köcheln zu lassen.

"Sie müssen sich nicht sofort entscheiden", sagte Harry in einem normaleren Ton. "Ich verstehe, wenn Sie Zeit brauchen, um über mein Angebot nachzudenken … aber ich warne Sie vor einer Sache", sagte Harry,

seine Stimme wurde kälter.

"Versuchen Sie nicht diesen Vergessens-Zauber an mir. Vor einiger Zeit habe ich ein Signal ausgearbeitet, und ich habe dieses Signal bereits an mich selbst gesendet. Wenn ich dieses Signal finde und mich nicht daran erinnere, es gesendet zu haben…"

Harry ließ seine Stimme deutlich abebben. McGonagalls Gesicht arbeitete, während sich ihre Mimik veränderte.

"Ich… hatte nicht vor, Sie zu verhexen, Mr~Potter… aber warum hätten Sie ein solches Signal erfinden sollen, wenn Sie nicht wussten, dass -"

"Ich dachte daran, als ich ein Muggel-Science-Fiction-Buch las, und sagte mir, na ja, nur für den Fall…

Und nein, ich werde das Signal nicht verraten, ich bin ja nicht blöd."

"Ich hatte nicht vor zu fragen", sagte McGonagall.

Sie schien in sich zusammenzufalten und sah plötzlich sehr alt und sehr müde aus.

"Das war ein anstrengender Tag, Mr~Potter. Können wir Ihren Koffer holen und Sie nach Hause schicken? Ich vertraue darauf, dass Sie nichts zu dieser Angelegenheit sagen, bis ich Zeit zum Nachdenken hatte. Denken Sie daran, dass es auf der ganzen Welt nur 2 andere Menschen gibt, die von dieser Angelegenheit wissen, und das sind Schulleiter Albus Dumbledore und Professor Severus Snape."

So. Neue Informationen; das war ein Friedensangebot.

Harry nickte zustimmend und drehte den Kopf, um nach vorne zu schauen, und begann wieder zu gehen, während sein Blut langsam wieder warm wurde.

"Jetzt muss ich also einen Weg finden, einen unsterblichen dunklen Zauberer zu töten", sagte Harry und seufzte frustriert. "Ich wünschte wirklich, Sie hätten mir das gesagt, bevor ich mit dem Einkaufen angefangen habe."

\textbf{Später}.

Der Laden war reicher ausgestattet als jeder andere Laden, den Harry besucht hatte; die Vorhänge waren üppig und fein gemustert, der Boden und die Wände aus gebeiztem und poliertem Holz, und die Koffer nahmen Ehrenplätze auf polierten Elfenbeinpodesten ein.

Der Verkäufer war in eine Robe gekleidet, die nur einen Hauch unter der von Lucius Malfoy lag, und sprach mit erlesener, öliger Höflichkeit sowohl mit Harry als auch mit Professor McGonagall. Harry hatte seine Fragen gestellt und sich zu einem Koffer aus schwerem Holz hingezogen gefühlt, nicht poliert, aber warm und massiv, geschnitzt mit dem Muster eines Wächterdrachen, dessen Augen sich bewegten, um jeden anzusehen, der sich ihm näherte.

Ein Koffer, der so verzaubert war, dass er leicht war, auf Kommando schrumpfte, kleine krallenbewehrte Tentakel aus dem Boden sprießen ließ und sich hinter dem Besitzer herwinden konnte.

Mit zwei Schubladen auf jeder der vier Seiten, die sich herausschieben ließen, um Fächer zu enthüllen, die so tief waren wie der ganze Koffer. Ein Deckel mit vier Schlössern, von denen jedes einen anderen Raum im Inneren offenbaren würde. Und -

das war das Wichtigste - ein Griff an der Unterseite, der einen Rahmen herausschob, in dem sich eine Treppe befand, die hinunter in einen kleinen, beleuchteten Raum führte, der, so schätzte Harry, etwa zwölf Bücherregale beherbergen würde.

Wenn sie solche Gepäckstücke herstellten, wusste Harry nicht, warum sich jemand die Mühe machte, ein Haus zu besitzen.

Einhundertacht goldene Galleonen. Das war der Preis für einen guten, leicht gebrauchten Koffer. Bei etwa fünfzig britischen Pfund pro Galleone war das genug, um ein gebrauchtes Auto zu kaufen. Es wäre teurer als alles andere, was Harry in seinem ganzen Leben je gekauft hatte, alles zusammengenommen.

Siebenundneunzig Galleonen. So viel war in dem Beutel mit Gold übrig, den Harry aus Gringotts hatte mitnehmen dürfen. Professor McGonagall trug einen ärgerlichen Ausdruck auf ihrem Gesicht. Nach einem langen Einkaufstag hatte sie es nicht für nötig gehalten, Harry zu fragen, wie viel Gold noch in dem Beutel war, nachdem der Verkäufer seinen Preis genannt hatte, was bedeutete, dass die Professorin auch ohne Stift und Papier gut Kopfrechnen konnte.

Wieder einmal erinnerte sich Harry daran, dass wissenschaftlicher Analphabet keineswegs dasselbe war wie dumm.

"Es tut mir leid, junger Mann", sagte Professor McGonagall. "Das ist ganz allein meine Schuld. Ich würde Ihnen anbieten, Sie zurück zu Gringotts zu bringen, aber die Bank ist jetzt für alles außer Notfälle geschlossen."

Harry sah sie an und fragte sich…

"Nun", seufzte Professor McGonagall, während sie sich auf einen Absatz schwang, "ich nehme an wir können genauso gut gehen."

… sie hatte nicht völlig den Verstand verloren, als ein Kind es gewagt hatte, ihr zu trotzen. Sie war nicht glücklich gewesen, aber sie hatte nachgedacht, anstatt vor Wut zu explodieren. Vielleicht lag es einfach daran, dass es einen unsterblichen Dunklen Lord zu bekämpfen gab - dass sie Harrys Wohlwollen gebraucht hatte. Aber die meisten Erwachsenen wären nicht fähig gewesen, auch nur so viel zu denken; würden die zukünftigen Konsequenzen überhaupt nicht in Betracht ziehen, wenn jemand, der einen niedrigeren Status hatte, sich weigerte, ihnen zu gehorchen…

"Professor?" sagte Harry.

Die Hexe drehte sich um und sah ihn an. Harry nahm einen tiefen Atemzug. Er musste ein bisschen wütend sein für das, was er jetzt versuchen wollte, sonst hätte er nie den Mut dazu gehabt. Sie hat mir nicht zugehört, dachte er bei sich, ich hätte mehr Gold genommen, aber sie wollte nicht zuhören… Er konzentrierte seine ganze Welt auf McGonagall und das Bedürfnis, \emph{dieses Gespräch nach seinem Willen zu biegen}, und sprach.

"Professor, Sie dachten, hundert Galleonen wären mehr als genug für einen Koffer. Deshalb haben Sie sich auch nicht die Mühe gemacht, mich zu warnen, bevor es auf siebenundneunzig runterging. Das ist genau das, was die Forschungsstudien zeigen - das passiert, wenn die Leute denken, dass sie sich einen kleinen Fehlerspielraum lassen. Sie sind nicht pessimistisch genug. Wenn es nach mir gegangen wäre, hätte ich zweihundert Galleonen genommen, nur um sicher zu gehen. Es war genug Geld im Verließ, und ich hätte alles andere später zurücklegen können. Aber ich dachte, Sie würden es mir nicht erlauben. Ich dachte, Sie wären wütend auf mich, nur weil ich gefragt habe. Habe ich mich geirrt?"

"Ich muss wohl zugeben, dass Sie recht haben", sagte Professor McGonagall. "Aber, junger Mann -"

"Solche Dinge sind der Grund, warum ich Schwierigkeiten habe, Erwachsenen zu vertrauen."

Irgendwie konnte Harry seine Stimme ruhig halten.

"Weil sie wütend werden, wenn man auch nur versucht, mit ihnen zu diskutieren. Für sie ist es Trotz und Anmaßung und eine Herausforderung an ihren höheren Stammesstatus. Wenn man versucht, mit ihnen zu reden, werden sie wütend. Wenn ich also etwas wirklich Wichtiges zu tun hätte, wäre ich nicht in der Lage, Ihnen zu vertrauen. Selbst wenn Sie mit tiefer Besorgnis auf das hören würden, was ich sage -

denn auch das gehört zur Rolle eines besorgten Erwachsenen -

würden Sie Ihre Handlungen nicht ändern, Sie würden sich nicht wirklich anders verhalten, wegen irgendetwas, das ich sage."

Der Verkäufer beobachtete die beiden mit unverhohlener Faszination.

"Ich kann Ihren Standpunkt verstehen", sagte Professor McGonagall schließlich. "Wenn ich manchmal zu streng erscheine, denken Sie bitte daran, dass ich seit gefühlt mehreren tausend Jahren das Haus Gryffindor leite."

Harry nickte und fuhr fort.

"Also - angenommen, ich hätte eine Möglichkeit, mehr Galleonen aus meinem Verließ zu bekommen, ohne dass wir zurück nach Gringotts gehen müssten, aber dazu müsste ich die Rolle eines gehorsamen Kindes verletzen. Könnte ich Ihnen das anvertrauen, auch wenn Sie dafür aus Ihrer Rolle als Professor McGonagall heraustreten müssten?"

"Was?".

"Anders ausgedrückt: Wenn ich es schaffen könnte, dass der heutige Tag anders verlaufen wäre, so dass wir nicht zu wenig Geld mitgenommen hätten, wäre das in Ordnung, auch wenn das im Nachhinein eine Unverschämtheit eines Kindes gegenüber einem Erwachsenen bedeuten würde?"

"Ich … nehme an …", sagte die Hexe und sah ziemlich verwirrt aus.

Harry holte den Beutel heraus und sagte: "Elf Galleonen, die ursprünglich aus meinem Familienverließ stammen."

Und da war das Gold in Harrys Hand. Einen Moment lang klaffte Professor McGonagalls Mund weit auf, dann klappte ihre Kinnlade zu, ihre Augen verengten sich und die Hexe stieß hervor:

"Woher haben Sie das -"

"Aus meinem Familienverließ, wie ich schon sagte."

"Wie?"

"Magie."

"Das ist wohl kaum eine Antwort!", schnauzte Professor McGonagall, hielt dann inne und blinzelte.

"Nein, ist es nicht, oder? Ich sollte behaupten, dass es daran liegt, dass ich experimentell die wahren Geheimnisse der Funktionsweise des Beutels entdeckt habe und dass er tatsächlich Gegenstände von überall her holen kann, nicht nur aus seinem eigenen Inneren, wenn Sie die Anfrage richtig formulieren. Aber eigentlich kommt es daher, dass ich vorhin in den Goldhaufen gefallen bin und mir ein paar Galleonen in die Tasche gesteckt habe.

Jeder, der etwas von Pessimismus versteht, weiß, dass Geld etwas ist, das man schnell und ohne große Vorwarnung brauchen kann. Sind Sie nun wütend auf mich, weil ich mich Ihrer Autorität widersetzt habe? Oder froh, dass wir unsere wichtige Mission erfolgreich abgeschlossen haben?"

Die Augen des Verkäufers waren groß wie Untertassen. Und die große Hexe stand stumm da.

"Die Disziplin in Hogwarts muss durchgesetzt werden", sagte sie nach fast einer vollen Minute. "Zum Wohle aller Schüler. Und dazu gehören auch Höflichkeit und Gehorsam von Schülern gegenüber allen Professoren."

"Ich verstehe, Professor McGonagall."

"Gut. Jetzt lass uns den Koffer kaufen und nach Hause gehen."

Harry hatte das Gefühl, sich übergeben zu müssen, oder zu jubeln, oder in Ohnmacht zu fallen, oder so. Das war das erste Mal, dass seine vorsichtige Argumentation bei jemandem funktioniert hatte. Vielleicht, weil es auch das erste Mal war, dass er etwas wirklich Ernstes hatte, das ein Erwachsener von ihm brauchte, aber trotzdem - Minerva McGonagall, +1 Punkt.

Harry verbeugte sich und gab McGonagall den Beutel mit dem Gold und den zusätzlichen elf Galleonen in die Hand.

"Vielen Dank, Professor. Können Sie den Kauf für mich abschließen? Ich muss noch auf die Toilette."

Der Verkäufer, wieder salbungsvoll, deutete auf eine in die Wand eingelassene Tür mit einem goldfarbenen Knauf. Als Harry weggehen wollte, hörte er den Verkäufer mit seiner öligen Stimme fragen:

"Darf ich fragen, wer das war, Madam McGonagall? Ich nehme an, er ist aus Slytherin - vielleicht im dritten Jahr? - und aus einer prominenten Familie, aber ich habe nicht erkannt -"

Das Zuschlagen der Toilettentür schnitt ihm das Wort ab, und nachdem Harry das Schloss gefunden und eingedrückt hatte, griff er nach dem magischen, selbstreinigenden Handtuch und wischte sich mit zittrigen Händen die Feuchtigkeit von der Stirn. Harrys ganzer Körper war in Schweiß gehüllt, der seine Muggelkleidung durchdrungen hatte, aber wenigstens sah man es nicht durch die Roben hindurch.

\textbf{Später}.

Die Sonne ging unter und es war schon sehr spät, als sie wieder im Hof des Tropfenden Kessels standen, der stillen, mit Blättern bedeckten Schnittstelle zwischen dem magischen Britannien der Winkelgasse und der gesamten Muggelwelt.(Das war eine schrecklich entkoppelte Wirtschaft…)

Harry sollte zu einer Telefonzelle gehen und seinen Vater anrufen, sobald er auf der anderen Seite war. Er brauchte sich anscheinend keine Sorgen zu machen, dass sein Gepäck gestohlen werden könnte. Sein Koffer hatte den Status eines wichtigen magischen Gegenstandes, etwas, das die meisten Muggel nicht bemerken würden; das war ein Teil dessen, was man in der Zaubererwelt bekommen konnte, wenn man bereit war, den Preis eines Gebrauchtwagens zu zahlen.

"Hier trennen sich also unsere Wege, für eine gewisse Zeit", sagte Professor McGonagall.

Sie schüttelte verwundert den Kopf.

"Dies ist der seltsamste Tag in meinem Leben seit … vielen Jahren. Seit dem Tag, an dem ich erfahren habe, dass ein Kind Du-weißt-schon-wen besiegt hat. Ich frage mich jetzt, rückblickend, ob das der letzte vernünftige Tag der Welt war."

Oh, als ob sie irgendetwas hätte, worüber sie sich beschweren könnte. \emph{Du denkst, dein Tag war surreal? Versuch meinen.}

"Ich war heute sehr beeindruckt von Ihnen", sagte Harry zu ihr. "Ich hätte daran denken sollen, Ihnen laut ein Kompliment zu machen, in meinem Kopf habe ich Ihnen Punkte gegeben und so."

"Ich danke Ihnen, Mr~Potter", sagte Professor McGonagall. "Wären Sie bereits in ein Haus einsortiert worden, hätte ich Ihnen so viele Punkte abgezogen, dass Ihre Enkelkinder den Hauspokal noch verlieren würden."

"Danke, Professor."

Es war wahrscheinlich zu früh, sie \emph{Minnie} zu nennen. Diese Frau war vielleicht die vernünftigste Erwachsene, die Harry je getroffen hatte, trotz ihres fehlenden wissenschaftlichen Hintergrunds. Harry erwog sogar, ihr den zweiten Platz in der Gruppe anzubieten, die er zum Kampf gegen den Dunklen Lord bilden würde, obwohl er nicht so dumm war, das laut zu sagen.

Was wäre denn ein guter Name dafür…? \emph{Die Todesser-Esser}?

"Ich sehe Sie bald wieder, wenn die Schule beginnt", sagte Professor McGonagall. "Und, Mr~Potter, wegen Ihres Zauberstabs -"

"Ich weiß, was Sie fragen wollen", sagte Harry.

Er nahm seinen kostbaren Zauberstab heraus und drehte ihn mit einem tiefen Stich inneren Schmerzes in seiner Hand um und präsentierte ihr den Griff.

"Nehmen Sie ihn. Ich hatte nicht vor, irgendetwas zu tun, nicht das Geringste, aber ich möchte nicht, dass Sie Albträume davon bekommen, dass ich mein Haus in die Luft jage."

Professor McGonagall schüttelte schnell den Kopf.

"Oh nein, Mr~Potter! So wird das nicht gemacht. Ich wollte Sie nur davor warnen, Ihren Zauberstab zu Hause zu benutzen, da das Ministerium die Magie von Minderjährigen aufspüren kann und sie ohne Aufsicht verboten ist."

"Ah", sagte Harry. "Das klingt nach einer sehr vernünftigen Regel. Ich bin froh zu sehen, dass die Zaubererwelt so etwas ernst nimmt."

Professor McGonagall schaute ihn scharf an.

"Sie meinen das wirklich ernst."

"Ja", sagte Harry. "Ich habe es verstanden. Magie ist gefährlich und die Regeln sind aus guten Gründen da. Bestimmte andere Dinge sind auch gefährlich. Das verstehe ich auch. Vergessen Sie nicht, dass ich nicht dumm bin."

"Das werde ich wohl kaum je vergessen. Danke, Harry, das gibt mir ein besseres Gefühl, dir gewisse Dinge anzuvertrauen. Auf Wiedersehen für jetzt."

Harry wandte sich zum Gehen, in den Tropfenden Kessel und hinaus in die Muggelwelt. Als seine Hand den Griff der Hintertür berührte, hörte er ein letztes Flüstern hinter sich.

"Hermine Granger."

"Was?" sagte Harry, seine Hand immer noch an der Tür.

"Suchen Sie im Zug nach Hogwarts nach einer Erstklässlerin namens Hermine Granger."

"Wer ist sie?"

Es gab keine Antwort, und als Harry sich umdrehte, war Professor McGonagall weg.

\textbf{Nachspiel:}:

Schulleiter Albus Dumbledore lehnte sich über seinen Schreibtisch. Seine funkelnden Augen blickten Minerva an.

"Also, meine Liebe, wie haben Sie Harry gefunden?"

Minerva öffnete ihren Mund. Dann schloss sie den Mund. Dann öffnete sie den Mund wieder. Es kam kein Wort heraus.

"Ich verstehe", sagte Albus mit ernster Miene. "Ich danke dir für deinen Bericht, Minerva. Du kannst gehen."

Anmerkung von mir:

Ich hoffe mit der Formatierung ist es jetzt leserlicher als in den Kapiteln davor. Für Tipps und Anmerkungen bin ich immer dankbar.

