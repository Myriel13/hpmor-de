

\hypertarget{halt-die-klappe-und-tu-das-unmuxf6gliche-teil-2}{% \section{114. Halt die Klappe, und tu das Unmögliche, Teil 2}\label{halt-die-klappe-und-tu-das-unmuxf6gliche-teil-2}}

\uline{Halt die Klappe, und tu das unmögliche, Teil 2

}

Ein Traumartiger Zustand war über Harrys Geist gekommen. Der Gedanke an die Realität war teilweise von ihm abgefallen, teilweise blieb er bei ihm. Teile seines Verstandes waren betäubt, vielleicht absichtlich betäubt von einem Teil, der klug genug war, um vorauszusehen, was sonst passieren würde.

\emph{Was er gerade getan hatte}—

der Gedanke war abgeschaltet, machte Platz für ein Bewusstsein für andere Dinge.

Harry stand inmitten eines Friedhofs, dessen Grabsteine unordentlich verstreut waren. Bei Mond- und Sternenlicht konnte man erkennen, dass schwarze Roben den Boden übersäten, umgeben von Strukturen, die nicht zur umgebenden Friedhofserde passten, Nässe, die im Mondlicht rot gefärbt war. Einige Köpfe hatten sich aus den sie umgebenden Kapuzen der Roben gelöst und enthüllten langes oder kurzes, dunkles oder helles Haar, das alles war, was man im Mondlicht sehen konnte. Die silbernen Masken blieben auf, so dass alle Haare aus Schädeln stammten und nicht aus menschlichen Gesichtern—

der Gedanke wurde ausgeschaltet und machte Platz für die Wahrnehmung anderer Dinge.

Ein Mädchen in einer rotgeschmückten Hogwarts-Uniform schlief auf einem Altar. Neben dem Altar lagen Harrys Sachen auf einem Haufen. Auf dem Boden lag ein übergroßer, bleicher Mann mit unmenschlichem Gesicht, dem Blut aus den Stümpfen seiner Handgelenke floss.

\emph{Sobald der Dunkle Lord Voldemort erwacht, wird er alles zerstören, was du liebst. Dumbledore ist nicht mehr da, um ihn aufzuhalten. Er kann nicht gefangen gehalten werden, denn er kann seinen Körper jederzeit verlassen. Er kann nicht dauerhaft getötet werden, nicht ohne mehr als hundert Horkruxe zu zerstören, von denen einer die Schallplatte auf einer Raumsonde ist.}

\emph{Materialien: Ein Zauberstab, du darfst ihn diesmal bewegen und sprechen. Du hast in etwas fünf Minuten Zeit.}

\emph{Löse das Problem.}

Harry stolperte auf den Altar zu, kniete an seiner Seite nieder und hob seinen Beutel auf. Er ging auf die Stelle zu, wo Voldemort lag. Das Gefühl der Beklemmung hatte nachgelassen, nachdem Voldemort bewusstlos gehext worden war. Jetzt, als Harry sich näherte, stieg es zu einer erschreckenden Größe an, die sich auch in Schmerzen in seiner Narbe entlud. Harry ignorierte den inneren Aufschrei. Das war die letzte Erinnerung von Tom Riddle gewesen, die sich in Harrys Gehirn eingebrannt hatte, das letzte kognitive Muster, das in den Säugling überging, bevor Tom Riddle explodiert war: ein Gefühl von wachsendem Schrecken und Entsetzen, das mit der außer Kontrolle geratenen Resonanz verbunden war. Harry kannte jetzt die Bedeutung davon, dieses Gefühl der Beklemmung, und das machte es leichter, es zu ignorieren. Er hatte vermutet, dass die Wirkung der Resonanz hauptsächlich den Zaubernden traf, mit einer Kraft, die proportional zur Macht des Zaubernden war, und die Wette hatte sich ausgezahlt.

Harry blickte auf Voldemorts Körper und atmete tief ein - durch den Mund, denn der kupferartige Geruch, an die Harry nicht dachte, drang durch seine Nase ein. Harry kniete sich neben Voldemort, holte seinen Medizinkoffer aus der Tasche und legte einen selbstspannenden Druckverband um das linke Handgelenk des Körpers, dann einen weiteren Druckverband um das rechte. Es fühlte sich falsch an, Voldemort diese Sorge zu zeigen. Irgendein Teil von Harry war sich im Hinterkopf bewusst, dass einer Reihe von Menschen gerade etwas extrem Schlimmes zugestoßen war. Es wäre ein Ausgleich gewesen, es wäre Gerechtigkeit gewesen, wenn Voldemort ohne einen Augenblick mehr zu zögern das gleiche Schicksal erlitten hätte. Was Harry jetzt tat, fühlte sich an, als würde Batman mehr Fürsorge für den Joker zeigen als für dessen Opfer; es fühlte sich an wie ein Comic, in dem die Autoren endlos über die Moral des Tötens der großen, namentlich genannten Bösewichte ringen, während im Hintergrund weiter Unschuldige starben. Mehr Fürsorge für den Oberbösewicht zu zeigen als für seine Untergebenen, seinem Schicksal mehr Aufmerksamkeit zu schenken als dem Schicksal seiner Anhänger mit niedrigerem Status, war ein Fehler in der menschlichen Natur.

Deshalb fühlte es sich falsch an, als Harry sich neben der Leiche erhob, nachdem die Aderpressen an Voldemorts Handgelenken festgezogen worden waren; es fühlte sich an, als würde Harry etwas ethisch Ungeheuerliches tun. Auch wenn jedes vernünftige strategische Denken besagte, dass Voldemorts Körper nicht sterben durfte. Die Seele, die er für sich geschaffen hatte, musste in diesem Gehirn verankert sein, sie durfte nicht frei schweben.

Harry trat zurück, zurück von Voldemorts bewusstlosem Körper, und atmete tief durch den Mund. Er ging zu dem Stapel seiner Sachen, um seine Roben und andere Gegenstände anzulegen, und begann damit, den Zeitumkehrer um seinen Hals zu legen, um seine eigene Flucht und Rückkehr vorzubereiten, falls das erforderlich sein sollte…

Mehr als hundert Horcruxe. Das war wahnsinnig gewesen, es gab kein anderes Wort dafür, ein Zeichen für Voldemorts beschädigtes Denken über den Tod. Ein Muggel-Sicherheitsexperte hätte es Zaunpfahl-Sicherheit genannt, so wie man einen über hundert Meter hohen Zaunpfahl mitten in der Wüste baut. Nur ein sehr zuvorkommender Angreifer würde versuchen, den Zaunpfahl zu erklimmen. Jeder vernünftige Mensch würde einfach um den Zaunpfahl herumgehen, und den Zaunpfahl noch höher zu machen, würde das nicht verhindern. Wenn man erst einmal vergessen hatte, wie unmöglich das Problem sein sollte, war es nicht einmal schwierig, nicht im Vergleich zum vorherigen.

Nevilles Eltern, zum Beispiel, waren in den permanenten Wahnsinn gefoltert worden. Zweihundert verbesserte Horcruxe würden diesen Wahnsinn nicht verhindern, sie würden nur alle denselben beschädigten Geist widerspiegeln. Es wäre ein ethisch gerechtfertigter Einsatz des Cruciatus-Fluches, wenn das der einzige Weg wäre, Voldemort dauerhaft zu stoppen. Es wäre Gerechtigkeit, Ausgewogenheit, es würde zeigen, dass das Leben des Jokers nicht mehr wert war als sein fiesester Handlanger.

.. Alles, was Harry tun musste, war, den Patronus-Zauber zu sprechen, und ihn zu schicken an… Alastor Moody?…und ihm zu sagen, dass er herkommen soll. Nein, der Patronus-Zauber würde wohl nicht funktionieren, wenn er mit dieser Absicht gewirkt wurde. Oder seinen Zeitumkehrer benutzen, sobald er außerhalb der Reichweite von Voldemorts Wachzaubern ist. Und dann könnte Voldemort in den permanenten Wahnsinn gefoltert werden. Das war nicht mal das schlimmste Schicksal. Das wäre gewesen, Voldemorts Zauberstab in die Grube von Askaban zu werfen, wo der Zauberstab mit Voldemorts Leben und Magie verbunden blieb, egal wohin sein Geist zu fliehen versuchte.

Harry drehte sich zu der Stelle um, wo Voldemort lag. Er ging vorwärts, kontrollierte weiterhin seine Atmung und ignorierte das brennende Gefühl in seiner Kehle. Ein Teil von ihm wusste, dass Voldemort auch Professor Quirrell war, auch wenn sein Körper jetzt ein anderer war. Auch wenn der Persönlichkeitswechsel perfekt gewesen war und das bedeutete, dass Professor Quirrell nur eine weitere Maske gewesen war… Obwohl Voldemort nicht geplant hatte, Harry schmerzhaft zu töten. Er hatte nicht daran gedacht, Harry mit dem Cruciatus seiner Anhänger zu verhexen, als Harry vorher lästig war. Das hatte etwas zu bedeuten, wenn dein Gegner Voldemort war. Vielleicht hatte er doch noch einen Funken Mitgefühl für den anderen Tom Riddle übrig. Wäre es falsch, das zu berücksichtigen?

Harry blickte wieder zu den Sternen hinauf. Hier unter der Atmosphäre funkelten die Sterne, sie waren eingebettet in die falsche Kuppel des Nachthimmels, erstreckten sich über die Milchstraße, die wie ein langes Band leuchtete, als wären sie alle so nah, dass man auf einem Besen zu ihnen hinauffliegen und sie berühren könnte.

Was würden sie jetzt von ihm wollen, die Kinder der Kinder der Kinder?

Die Antwort darauf fühlte sich auch offensichtlich an, wenn nicht gerade der Teil von Harry, der sich immer noch um Professor Quirrell sorgte, das eigentliche Reden übernahm. Harry hatte das tun müssen, was er getan hatte, es hatte größeres Übel verhindert, Harry hätte Voldemort nicht aufhalten können, wenn die Todesser zuerst geschossen hätten. Aber diese Sache, die Harry getan hatte, war nicht etwas, das durch eine nicht notwendige Tragödie ausgeglichen werden konnte, die einem weiteren fühlenden Wesen widerfuhr, selbst wenn dieses Wesen Voldemort war. Es wäre nur ein weiteres Element des Leids der alten Erde vor so langer Zeit. Die Vergangenheit war Vergangenheit.

\emph{Du hast getan, was du tun musstest, und du hast nicht ein bisschen mehr Schaden angerichtet als das. Nicht einmal, um die Dinge auszugleichen und alles symmetrisch zu machen. Die Kinder der Kinder der Kinder würden nicht wollen, dass Voldemort stirbt, selbst wenn seine Schergen es täten.}

\emph{Sie würden nicht wollen, dass Voldemort verletzt wird, wenn es nichts bewirkte im Vergleich dazu, dass er nicht verletzt wurde.}

Harry atmete tief durch und ließ los - nicht seinen Hass - nicht ganz Hass, denn er hatte seinen Schöpfer nicht einmal ganz am Ende hassen können - aber trotzdem ließ Harry etwas los. Von dem Gefühl, dass er Voldemort hassen sollte, dass es ein Hass war, den er verpflichtet war zu fühlen, für die endlose Liste von Verbrechen, die Voldemort ohne guten Grund begangen hatte, nicht einmal für sein eigenes Glück…

\emph{Es ist in Ordnung,} flüsterten die Sterne zu ihm herab. \emph{Es ist in Ordnung, ihn nicht zu hassen. Das macht dich nicht zu einem schlechten Menschen.}

Am Ende gab es nur eine Möglichkeit, die er wählen würde, und da Harry das bereits wusste, war es sinnlos, sich darüber zu quälen. Ob es die beste Option war, würde nur die Zeit zeigen. Harry atmete tief durch und baute die Magie in sich selbst auf. Der Zauber, den er sprechen wollte, musste nicht präzise sein, aber es war trotzdem einer der mächtigsten Zauber, die er beherrschte. Harry dachte wieder daran, wie ungerecht es war, dass Voldemort nicht mit seinen Gefolgsleuten sterben konnte, spürte die leichte Spur von Kälte in seinem Blut, die mit dem Gedanken an Rücksichtslosigkeit einherging. Und dann ließ Harry es los, ließ alles unter dem Sternenlicht versickern, denn seine dunkle Seite war nie etwas anderes gewesen als ein ererbtes Erkenntnismuster, nur eine weitere schlechte Denkgewohnheit, die es zu durchbrechen galt. Stattdessen schaute Harry auf Hermines atmende Gestalt auf dem Altar und ließ die Tränen endlich aus seinen Augen laufen.

Was nun aus Hermine werden würde, welchen Weg sie danach einschlagen würde, konnte Harry nicht erahnen; aber sie würde eine Wahl haben, ihre Freundschaft würde ihre Existenz nicht zerstört haben. Er hatte nicht bemerkt, wie wackelig seine Hoffnung gewesen war, bis er gemerkt hatte, wie überrascht er gewesen war, nachdem die Hoffnung sich erfüllt hatte.

\emph{Manchmal liefen die Dinge doch besser als erwartet.}

Und Harry nahm auch diesen Gedanken auf und steckte ihn in die Magie, die er aufbaute. Die Kraft, die er aufgespeichert hatte, vibrierte in ihm, als wäre sein ganzer Körper ein Teil seines Zauberstabs. Und Harry dachte an die Form des Zaubers, den er sprechen würde, er hatte nicht viel Feinsteuerung, aber das Muster, das er brauchte, war einfach, es musste nur folgendes enthalten—

\emph{Alles, vergiss alles, Tom Riddle, Professor Quirrell, vergiss dein ganzes Leben, vergiss dein gesamtes episodisches Gedächtnis, vergiss die Enttäuschung und die Bitterkeit und die falschen Entscheidungen, vergiss Voldemort—}

Und im letzten Moment, bevor Harry den Zauber sprach, hatte er einen letzten Gedanken, eine Note der Gnade—

\emph{Aber wenn du jemals wirklich glückliche Erinnerungen hattest, nicht Menschen zu verletzen oder über ihren Schmerz zu lachen, sondern das warme Gefühl, jemandem zu helfen oder geholfen zu werden, davon wird es nicht viele geben, vielleicht nur, als du ein Kind warst, aber wenn du wirklich glückliche Erinnerungen hattest, dann behalte nur diese—}

Etwas Helles in ihm entfaltete sich bei der Entscheidung, er wusste, dass er die richtige Wahl getroffen hatte, und Harry drückte das auch in seinen Zauber—

„OBLIVIATE!“ Und alles strömte aus Harry heraus in den Zauberspruch.

Harry fiel auf die Seite, ließ seinen Zauberstab fallen, aus seiner Kehle kamen gepresste Schreie, seine Hände fuhren hilflos zu seiner Narbe, selbst als der plötzliche Schmerzstoß in seinem Kopf zu verblassen begann.

Nur schemenhaft sahen seine Augen, dass die Luft von glühenden Schneeflocken erfüllt war, die wie winzige Flecken eines Patronuszaubers in silbernem Licht umherschwebten. Nur einen Augenblick dauerte das silberne Licht, dann war es verschwunden.

\emph{Professor Quirrell war weg.}

\emph{Nichts übrig als ein Überbleibsel. Und dieser Geist, was von ihm übrig war, würde sich jetzt nicht mehr von Harrys eigenem unterscheiden.}

\emph{Die Prophezeiung war vollendet.}

\emph{Jeder hatte den anderen nach seinem Ebenbild neu erschaffen.}

Harry fing an zu schluchzen, von dort aus, wo er sich im Dreck zusammengerollt hatte. Er weinte eine ganze Weile. Und dann schließlich taumelte Harry auf die Beine und nahm seinen Zauberstab wieder in die Hand, denn die Arbeit des Tages war noch nicht ganz getan. Harry legte seinen Zauberstab direkt auf Voldemorts Handgelenksstumpf; es ließ seine Narbe mit einem anhaltenden Schmerz pochen, aber keiner von beiden explodierte. Und Harry begann eine Verwandlung. Langsam - wenn auch schneller, als Harry es beim letzten Mal geschafft hatte, Hermines Körper zu verwandeln - veränderte sich die betäubte Gestalt des Schlangenmenschen, formte sich neu. Je weiter die Verwandlung fortschritt, besonders als der Kopf des Schlangenmenschen glasig und geschrumpft wurde, desto mehr ließ der Schmerz in Harrys Narbe nach. Der Zauber würde aufrechterhalten werden, egal ob Harry wach war oder schlief; und später, wenn Harry älter und mächtiger war und vielleicht etwas Hilfe hatte, würde er den geistig verwirrten Tom Riddle ent-wandeln und seinen Körper mit der Kraft des Steins heilen. Nachdem Zukunfts-Harry herausgefunden hatte, was er mit einem fast völlig amnesischen Zauberer machen sollte, der immer noch viele schlechte Denkgewohnheiten und einige höchst negative emotionale Muster hatte - eine dunkle Seite sozusagen - plus eine große Menge an deklarativem und prozeduralem Wissen über mächtige Magie. Harry hatte sein Bestes gegeben, diesen Teil nicht vergessen zu lassen, weil er ihn vielleicht eines Tages brauchen würde. Und in der Zwischenzeit würden Voldemorts Horkruxe einen verwandelten Voldemort nicht als tot definieren und versuchen, ihn zurückzubringen, so wie die Magie ein verwandeltes Einhorn nicht als tot definiert hatte, um Zaubersprüche auszulösen. Das war zumindest die Hoffnung.

Harrys Narbe zuckte ein letztes Mal, als der Stahlring auf seinen kleinen Finger kam und den winzigen grünen Smaragd mit seiner Haut in Kontakt brachte. Dann ließ der Schmerz nach und die Narbe tat nicht mehr weh. Ein aufgetürmter Felsen diente Harry als Stuhl, als er darüber taumelte und sich regungslos hinsetzte, um sich auszuruhen und die Erschöpfung zurückzuschieben, die die Ecken seines Geistes bedrohte. Es war noch nicht getan, es gab noch mehr zu tun.

Harry holte noch einmal tief Luft, atmete immer noch durch den Mund ein, sagte

„Lumos“ und sah sich auf dem Friedhof um. Schwarze Roben und abgetrennte Totenkopfmasken, umgeben von Blutlachen - Hermine Granger, schlafend auf einem Altar. Voldemorts leere Roben und blutige Hände, die dort lagen, wo der Dunkle Lord gefallen war. Quirinus Quirrell mit seinen zerfetzten Roben, dort liegend, wo ihn der Tötungsfluch getroffen hatte. Harry stellte sich vor, wie jemand anderes diese Szene betrachtete und versuchte, sie zu verstehen, und schüttelte den Kopf, denn das würde nicht gehen, ganz und gar nicht.

Dann schob sich Harry von seinem Felsen hoch und schnitt eine Grimasse, als sein Geist, wenn auch nicht sein Körper, protestierte. Er war heute nicht besonders überanstrengt oder gefoltert worden, aber irgendwie schaffte es Harrys Körper, sich so anzufühlen, als hätte ihn der ganze Stress direkt getroffen. Harry taumelte zu der Stelle hinüber, an der Voldemort gefallen war, und hob Voldemorts linke Hand auf, wo sie auf dem Boden lag. Selbst in der linken Hand konnte man die schwache Spur von Schlangenschuppen erkennen; es war ganz eindeutig Voldemort. Das war gut so. Harry ging zum Altar, wo die schlafende Hermine lag, und legte die abgetrennte Hand sanft um Hermines Hals, wobei er die Finger vorsichtig bewegte, um ihre Kehle zu umklammern.

Es war schwer zu tun, Hermine wirkte so friedlich und unschuldig, wenn sie schlief, und Voldemorts abgetrennte Hand wirkte so hässlich; Harry überging stumpf, welcher Teil seines Verstandes auch immer das dachte, da es im Zusammenhang keinen Sinn ergab. Ein paar schwache Ausfransungen dienten dazu, den fast perfekten Schnitt, den die Nanofaser gemacht hatte, zu versauen, was kritisch war; es würde nicht reichen, wenn der Handstumpf wie der Halsstumpf aussähe. Die mehrfachen Diffindos verteilten kleine Stücke der Voldemort-Hand über Hermines Hemd, was, wie Harry sich erinnern musste, ebenfalls Teil des Plans war. Harry wiederholte dies mit der rechten Hand und ordnete sie symmetrisch mit der linken an. Harry benutzte ein Inflammare, um Voldemorts Roben dort zu versengen, wo sie lagen, und ordnete dann die angesengte Kleidung um Hermine herum an.

Voldemorts Waffe und sein Zauberstab kamen in Harrys Beutel. Harry steckte den Stein der Unsterblichkeit in seine normale Tasche, er war sich nicht sicher, was der Stein mit seinem Beutel anstellen würde. Der Haufen von Dingen aus dem Inneren von Quirrells Robe, ebenfalls in der Nähe des Altars, enthielten den Zauberstab, den der Verteidigungsprofessor benutzt hatte, als er Quirrell war. Harry ging zu der Stelle, an der Quirrell lag, richtete den Körper so gut er konnte auf und nahm Quirrells Zauberstab in die Hand. Tränen traten Harry vorhersehbar in die Augen, und Harry wischte sie an seinem Ärmel weg. Harry holte noch einmal tief Luft, atmete immer noch durch den Mund ein, sagte noch einmal „Lumos“ und sah sich noch einmal auf dem Friedhof um.

Schwarze Roben, abgetrennte Totenkopfmasken und Hermine Granger, die auf einem Altar lag, mit Voldemorts abgetrennten Händen um ihre Kehle geklammert, und Voldemorts versengte Kleidung um sie herum verstreut. Quirinus Quirrell lag tot da, seine Kleidung zerrissen und zerfetzt, seinen Zauberstab in der rechten Hand. Das würde genügen.

Es blieb das Problem, die Aufmerksamkeit auf sich zu ziehen. Harry hatte zu diesem Zeitpunkt schon fast keine Magie mehr. Aber er hatte noch genug übrig, um ein Blatt Papier in die entleerte Form eines drei Meter großen Wetterballons zu verwandeln.

In Harrys Tasche befanden sich eine Flasche mit Oxyacetylen, eine Stange Dynamit und eine Spule mit Zündschnur.

\emph{Sei vorbereitet, das ist das Marschlied der Pfadfinder, sei vorbereitet für ein Leben, das Bergtrolle und wer weiß, was noch alles beinhaltet.}

.. Harry blies den Wetterballon mit dem Oxyacetylen auf. Das würde einen sehr starken Überdruck erzeugen, wenn er detonierte, vielleicht so laut wie ein Überschallknall. Er befestigte die Dynamitstange daran- es war ziemlich viel für eine Detonation. Er befestigte eine 60-Sekunden-Lunte an der Dynamitstange, zündete sie aber noch nicht an. Harry zog seinen Unsichtbarkeitsumhang an, der zwischen den Sachen beim Opferaltar gelegen hatte. Er holte seinen Besen aus seiner Tasche und bestieg ihn. Harry legte einen Schweigezauber um Hermine Granger - es würde nicht den ganzen Lärm aufhalten, nicht einmal annähernd, und es war nicht so, dass sie dauerhaft verletzt würde, wenn ihre Trommelfelle platzten, aber es schien trotzdem höflich.

Und das war es dann auch. Der Schweigezauber hatte es geschafft. Harry war mindestens für die nächste Stunde ohne Magie. Harry bestieg den Besen und erhob sich langsam in die Luft, wobei er den mit Oxyacetylen gefüllten Wetterballon mit sich zog. Das Schloss Hogwarts kam in Sicht, das ein paar Kilometer entfernt im Mondlicht schimmerte, als Harry sich über die Bäume erhob; und Harry tat sein Bestes, um die Entfernung und den Winkel auszurechnen, wie man es von Hogwarts aus sehen würde. Als er hoch über dem Wald aufgestiegen war, benutzte Harry ein Feuerzeug, um die Lunte des Dynamits zu zünden, das an dem mit Oxyacetylen gefüllten Wetterballon befestigt war. Dann drehte Harry den Besen und sauste davon - allerdings nicht direkt auf das Schloss zu, das könnte ihn zu nahe an die Route führen, die Harry und Professor Quirrell in der Vergangenheit zurückgelegt hatten, es würde nicht gut gehen, wenn der Professor einen anderen Harry witterte - Harry fühlte einen bleiernen Stich der Traurigkeit und wies ihn zurück.

\emph{Einunddreißig eintausend, zweiunddreißig eintausend, dreiunddreißig eintausend…} Als Harry die vierzig erreicht hatte, schaute er auf seine Armbanduhr, notierte sich die genaue Zeit und drehte einmal an seinem Zeitumkehrer, um sein eigenes Trommelfell nicht zu gefährden.

