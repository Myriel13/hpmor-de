

\hypertarget{hedonische-bewusstheit}{% \section{87. Hedonische Bewusstheit}\label{hedonische-bewusstheit}}

\textbf{\uline{Hedonische Bewusstheit}}

Donnerstag, 16. April 1992.

Die Schule war jetzt fast menschenleer, neun Zehntel der Schüler waren über die Osterferien nach Hause gegangen, so gut wie jeder, den sie kannte, fehlte. Susan war zurückgeblieben, da ihre Großtante ziemlich beschäftigt war, ebenso wie Ron aus Gründen, die sie nicht kannte - vielleicht war die Familie Weasley arm genug, dass es eine spürbare Belastung gewesen wäre, alle Kinder eine Woche länger zu ernähren?

Es klappte alles ganz gut, denn Ron und Susan waren so ziemlich die Einzigen, die noch mit ihr redeten (zumindest mit denen, mit denen \emph{sie} reden wollte). Lavender war immer noch nett zu ihr, und Tracey war, ähm, Tracey, aber keine von beiden war wirklich entspannend, um eine freie Stunde mit ihr zu verbringen; und auf jeden Fall hatte keine der beiden in den Osterferien in Hogwarts übernachtet.) Wenn sie nicht nach Hause konnte - und sie durfte nicht nach Hause, ihre Eltern waren angelogen worden und hatten gesagt, sie hätte Glühpocken gehabt - dann war ein fast leeres Hogwarts das Nächstbeste. Sie konnte sogar die Bibliothek besuchen, ohne dass die Leute sie anstarrten, da kein Unterricht stattfand und niemand versuchte, Schularbeiten zu machen.

Es wäre ein Fehler zu denken, dass Hermine den ganzen Tag weinend durch die Korridore trieb. Oh, natürlich hatte sie in den ersten beiden Tagen viel geweint, aber zwei Tage waren genug gewesen. Es gab Stellen in Harrys geliehenen Büchern, die davon handelten, dass selbst Leute, die bei Autounfällen gelähmt wurden, sechs Monate später nicht annähernd so unglücklich waren, wie sie es erwartet hatten, genauso wie Lottogewinner nicht annähernd so glücklich waren, wie sie es erwartet hatten. Die Leute passten sich an, ihr Glückslevel ging zurück zu ihrem Glückssollwert, das Leben ging weiter.

Ein Schatten fiel über die Stelle, an der Hermine ihr aktuelles Buch las, und sie wirbelte herum, wobei der Zauberstab, den sie auf ihrem Schoß versteckt hielt, hochkam und direkt auf das überraschte Gesicht von—

„Entschuldigung!“ sagte Harry Potter und hielt hastig seine Handflächen nach oben, um zu zeigen, dass seine linke Hand leer war und seine rechte Hand ein kleines rotes Samttäschchen hielt. „Entschuldigung. Ich wollte dich nicht erschrecken.“

Es herrschte eine schreckliche Stille, ihr Herzschlag erhöhte sich und ihre Handflächen begannen zu schwitzen, als Harry Potter sie einfach nur ansah. Fast hätte sie mit ihm gesprochen, am ersten Morgen ihres neuen Lebens; aber als sie zum Frühstück gekommen war, hatte Harry Potter so schrecklich ausgesehen - also hatte sie sich nicht neben ihn an den Frühstückstisch gesetzt, sondern einfach still in ihrer eigenen kleinen Blase gegessen, in der niemand sonst neben ihr saß, und es war schrecklich gewesen, aber Harry war nicht zu ihr gekommen, und…sie hatte seitdem einfach nicht mehr mit ihm gesprochen.

(Es war nicht schwer, allen aus dem Weg zu gehen, wenn man sich aus dem Ravenclaw-Gemeinschaftsraum fernhielt und aus dem Unterricht rannte, bevor jemand mit einem reden konnte.)

Und seitdem fragte sie sich, was Harry jetzt von ihr dachte - ob er sie hasste, weil sie sein ganzes Geld verloren hatte - oder ob er wirklich in sie verliebt war und es deshalb getan hatte - oder ob er es aufgegeben hatte, dass sie mit ihm mithalten konnte, weil sie keine Dementoren erschrecken konnte - sie konnte ihm jetzt nicht gegenübertreten, sie konnte es einfach nicht, sie verbrachte schlaflose Nächte damit, sich zu fragen, was Harry jetzt von ihr dachte, und sie hatte Angst und ging dem Jungen aus dem Weg, der sein ganzes Geld ausgegeben hatte, um sie zu retten, und sie war ein schreckliches, undankbares Luder und ein furchtbarer Mensch und—

Dann blickte sie nach unten, um zu sehen, dass Harry in den roten Samtbeutel griff und ein herzförmiges, in rote Folie eingewickeltes Bonbon herausholte, und ihr Gehirn schmolz dahin wie Schokolade, die in der Sonne liegt.

„Ich wollte dir eigentlich mehr Freiraum geben“, sagte Harry Potter, „aber ich habe gerade Critchs Theorien über Hedonismus nachgelesen und darüber, wie man seine innere Gefühle trainiert und wie kleine unmittelbare positive und negative Rückmeldungen insgeheim das meiste kontrollieren, was wir tatsächlich tun, und da kam mir der Gedanke, dass du mich vielleicht meidest, weil du bei meinem Anblick an Dinge denkst, die sich wie negative Assoziationen anfühlen, und ich wollte das nicht länger durchgehen lassen, ohne etwas dagegen zu tun, also habe ich mir eine Tüte Pralinen von den Weasley-Zwillingen besorgt und werde dir einfach jedes Mal eine geben, wenn du mich siehst, als positive Verstärkung, wenn das für dich in Ordnung ist—“

„Atme, Harry“, sagte Hermine, ohne darüber nachzudenken.

Es war das erste Wort, das sie seit dem Tag des Prozesses mit ihm gesprochen hatte. Die beiden starrten sich gegenseitig an. Die Bücher starrten sie aus den umliegenden Regalen an. Sie starrten einander noch mehr an.

„Du sollst die Schokolade essen“, sagte Harry und hielt ihr die herzförmige Süßigkeit wie ein Valentinstagsgeschenk hin. „Es sei denn, es fühlt sich schon gut genug an, eine Schokolade geschenkt zu bekommen, um als positive Verstärkung zu zählen, in diesem Fall musst du sie wahrscheinlich in deine Tasche stecken oder so.“

Sie wusste, dass sie versagen würde, wenn sie noch einmal versuchte zu sprechen, also versuchte sie es nicht. Harrys Kopf sackte ein wenig ab.

„Hasst du mich jetzt?“

„Nein!“, sagte sie. „Nein, das solltest du nicht denken, Harry! Nur - nur - nur alles!“

Sie bemerkte, dass ihr Zauberstab immer noch auf Harry gerichtet war, und sie ließ ihn sinken. Sie bemühte sich sehr, nicht in Tränen auszubrechen.

„Alles!“, wiederholte sie und fand nichts Besseres zu sagen als das, obwohl sie sicher war, dass Harry ihr sagen wollte, sie solle genau sein.

„Ich glaube, ich verstehe“, sagte Harry behutsam. „Was liest du da?“

Bevor sie ihn davon abhalten konnte, beugte sich Harry über den Bibliothekstisch, um das Buch zu sehen, das sie las, und beugte seinen Kopf vor, bevor sie auf die Idee kommen konnte, das Buch wegzunehmen - Harry starrte auf die aufgeschlagene Seite.

„Die reichsten Zauberer der Welt und wie sie so wurden“, las Harry den Titel des Buches von oben ab. „Nummer fünfundsechzig, Sir Gareth, Besitzer eines Transportunternehmens, das die Schifffahrtskriege im 19. Jahrhundert gewonnen hat…Monopol auf Oh-Tee-Drei… Ich verstehe.“

„Du willst mir wohl sagen, dass ich mir um nichts Sorgen machen muss und du dich um alles kümmern wirst?“

Es klang schärfer, als sie es gewollt hätte, und sie fühlte einen weiteren Stich der Schuld, weil sie so ein schrecklicher Mensch war.

„Nein“, sagte Harry und klang dabei seltsam fröhlich. „Ich kann mich gut genug in dich hineinversetzen, um zu wissen, dass, wenn du einen Haufen Geld bezahlt hättest, um mich zu retten, ich versuchen würde, es zurückzuzahlen. Ich würde wissen, dass es auf einer gewissen Ebene dumm ist, und ich würde trotzdem versuchen, es ganz alleine zurückzuzahlen. Es ist unmöglich, dass ich das nicht verstehen würde, Hermine.“

Hermines Gesicht verzog sich und sie spürte Feuchtigkeit in ihren Augenwinkeln.

„Aber eine faire Warnung“, fuhr Harry fort, „ich könnte die Schulden bei Lucius Malfoy selbst einlösen, wenn ich einen Weg sehe, bevor du es tust, es ist wichtiger, das sofort zu regeln, als wer von uns beiden es regelt. Irgendetwas Interessantes bis jetzt?“

Drei Viertel von ihr liefen im Kreis und prallten gegen Bäume, während sie versuchte, die Implikationen von allem, was Harry gerade gesagt hatte, herauszufinden (respektierte er sie immer noch als Heldin? oder bedeutete das, dass er dachte, sie könnte es nicht allein schaffen?), und währenddessen blätterte ein viel vernünftigerer Teil von Hermine das Buch bis zur Seite 37 zurück, auf der der vielversprechendste Eintrag stand, den sie bisher gesehen hatte (obwohl sie es in ihrer Fantasie immer allein schaffte und Harry völlig überrumpelte)—

„Ich dachte, das wäre ganz interessant“, sagte ihre Stimme.

„Nummer vierzehn, 'Crozier', wahrer Name unbekannt“, las Harry. „Wow, das ist…das ist der knalligste karierte Zylinder, den ich je gesehen habe. Reichtum, mindestens sechshunderttausend Galleonen… also etwa dreißig Millionen Pfund, nicht genug, um einen Muggel berühmt zu machen, aber gut genug für die kleinere Zaubererpopulation, schätze ich. Es wird gemunkelt, dass er ein modernes Pseudonym des sechs Jahrhunderte alten Nicholas Flamel ist, dem einzigen bekannten Zauberer, dem das unglaublich schwierige alchemistische Verfahren zur Herstellung des Steins der Weisen gelungen ist, mit dem man unedle Metalle in Gold oder Silber umwandeln kann, sowie… das Elixier des Lebens, das die Jugend und Gesundheit des Benutzers auf unbestimmte Zeit verlängert… Ähm, Hermine, das scheint offensichtlich falsch zu sein.“

„Ich habe mehr Hinweise auf Nicholas Flamel gelesen“, sagte Hermine. „In Aufstieg und Fall der dunklen Künste steht, dass er Dumbledore heimlich trainiert hat, um sich gegen Grindelwald zu behaupten. Es gibt eine Menge Bücher, die die Geschichte ernst nehmen, nicht nur dieses eine…denkst du, es ist zu schön, um wahr zu sein?“

„Nein, natürlich nicht“, sagte Harry. Harry zog den Stuhl neben ihrem eigenen an dem kleinen Tisch hervor und setzte sich neben sie auf seinen gewohnten Platz zu ihrer Rechten, so als wäre er nie weg gewesen; sie musste ein Kratzen in ihrem Hals zurückwürgen. „Die Idee von 'zu gut, um wahr zu sein' ist keine kausale Schlussfolgerung, das Universum prüft nicht, ob der Ausgang der Gleichungen 'zu gut' oder 'zu schlecht' ist, bevor es ihn zulässt. Früher dachten die Menschen, dass Flugzeuge und Pockenimpfstoffe zu gut seien, um wahr zu sein. Muggel haben Wege gefunden, zu anderen Sternen zu reisen, ohne überhaupt Magie zu benutzen, und du und ich können mit unseren Zauberstäben Dinge tun, die Muggelphysiker für buchstäblich unmöglich halten. Ich kann mir nicht einmal vorstellen, wo wir die wahre Grenze von Magie setzen könnten.“

„Wo liegt dann das Problem?“ sagte Hermine. Ihre Stimme klang jetzt normaler, in ihren eigenen Ohren.

„Nun…“ sagte Harry.

Der Junge griff über ihren eigenen ausgestreckten Arm, wobei seine Robe die ihre streifte, und tippte auf die Illustration des Künstlers, die einen unheilvoll glühenden roten Stein zeigte, aus dem eine scharlachrote Flüssigkeit tropfte.

„Das Problem ist, dass es keinen logischen Grund gibt, warum dasselbe Artefakt in der Lage sein sollte, Blei in Gold umzuwandeln und ein Elixier herzustellen, das jemanden jung hält. Ich frage mich, ob es dafür in der Literatur einen offiziellen Namen gibt? Wie der “Auf-die-Elf-Effekt„ vielleicht? Wenn jeder eine Blume sehen kann, kommt man nicht damit durch, dass Blumen so groß wie Häuser sind. Aber bei einem UFO-Kult kann man, da sowieso niemand das außerirdische Mutterschiff sehen kann, sagen, es habe die Größe einer Stadt oder die Größe des Mondes. Beobachtbare Dinge müssen durch Beweise eingeschränkt werden, aber wenn sich jemand eine Geschichte ausdenkt, kann er die Geschichte so extrem machen, wie er will. So gibt einem der Stein der Weisen unbegrenztes Gold und ewiges Leben, nicht weil es eine einzige magische Entdeckung gibt, die diese beiden Effekte hervorbringt, sondern weil jemand eine Geschichte über ein super tolles Ding erfunden hat.“

„Harry, es gibt eine Menge Dinge in der Magie, die nicht vernünftig sind“, sagte sie.

„Zugegeben“, sagte Harry. „Aber Hermine, Problem zwei ist, dass nicht einmal Zauberer verrückt genug sind, um die Auswirkungen dieser Dinge beiläufig zu übersehen. Jeder würde versuchen, die Formel für den Stein der Weisen wiederzufinden, ganze Länder würden versuchen, den unsterblichen Zauberer zu fangen und ihm das Geheimnis zu entlocken—“

„Es ist kein Geheimnis.“ Hermine blätterte die Seite um und zeigte Harry die Diagramme. „Die Anleitung steht gleich auf der nächsten Seite. Es ist nur so schwierig, dass es nur Nicholas Flamel geschafft hat.“

„Also würden ganze Länder versuchen, Flamel zu entführen und ihn zu zwingen, mehr Steine herzustellen. Komm schon, Hermine, selbst Zauberer würden nicht von Unsterblichkeit und, und“, Harry Potter hielt inne, seine Beredsamkeit ließ ihn offenbar im Stich, „und einfach weitermachen. Menschen sind verrückt, aber so verrückt sind sie nicht!“

„Nicht jeder denkt so wie du, Harry.“

Da hatte Hermine nicht ganz unrecht, aber…wie viele verschiedene Anspielungen auf Nicholas Flamel waren ihr schon begegnet?

Neben „Die reichsten Zauberer der Welt“ und „Aufstieg und Fall der dunklen Künste“ gab es auch „Geschichten aus mäßig alten Zeiten“ und „Biografien berühmter Persönlichkeiten“…

„Also gut, Professor Quirrell hätte diesen Flamel entführt. Das würde ein böser Mensch oder ein guter Mensch oder einfach ein Egoist tun, wenn er vernünftig wäre. Der Verteidigungsprofessor kennt eine Menge Geheimnisse, und das würde er sich nicht entgehen lassen.“

Harry seufzte und sah auf; sie folgte seinem Blick, aber er betrachtete offenbar nur die größere Bibliothek, die Reihen und Reihen und Reihen von Bücherregalen.

„Ich will dir nicht in dein Projekt reinreden“, sagte Harry, „und ich will dich sicher nicht entmutigen, aber…Ehrlich gesagt, Hermine, bin ich mir nicht sicher, ob du in so einem Buch gute Ideen zum Geldverdienen finden wirst. Wie der alte Witz, dass wenn ein Ökonom einen 20-Pfund-Schein auf der Straße liegen sieht, er ihn nicht aufhebt, denn wenn er echt wäre, hätte ihn schon jemand anderes aufgehoben. Jede Möglichkeit, viel Geld zu verdienen, von der jeder weiß, bis zu dem Punkt, an dem sie in Büchern wie diesem steht… verstehst du was ich sagen will? Es kann nicht für jeden möglich sein, in drei einfachen Schritten tausend Galleonen im Monat zu verdienen, sonst würde es jeder tun.“

„Und? Das würde dich nicht aufhalten“, sagte Hermine, deren Stimme nun wieder rauer wurde. „Du machst ständig unmögliche Dinge, ich wette, du hast in der letzten Woche etwas Unmögliches getan und hast dir nicht die Mühe gemacht, es jemandem zu sagen.“

(Es entstand eine kleine Pause, die, wenn Miss~Granger es gewusst hätte, genau so lang gewesen wäre, wie wenn man nachrechnet das man genau acht Tage zuvor gegen Mad-Eye Moody gekämpft und gewonnen hätte.)

„Nicht in den letzten sieben Tagen, nein“, sagte Harry. „Sieh mal… ein Teil des Tricks, das Unmögliche zu tun, besteht darin, wählerisch zu sein, welche Unmöglichkeiten man herausfordert, und es nur zu versuchen, wenn man einen besonderen Vorteil hat. Wenn es in diesem Buch eine Methode zum Geldverdienen gibt, die für einen Zauberer schwierig klingt, aber einfach ist, wenn wir Dads alten Computer benutzen können, dann hätten wir einen Plan.“

„Das weiß ich, Harry“, sagte Hermine, ihre Stimme schwankte nur leicht. „Ich habe geschaut, ob es hier irgendetwas gibt, das ich herausfinden kann. Ich dachte, vielleicht ist das Schwierige an der Herstellung eines Steins der Weisen, dass der alchemistische Kreis superpräzise sein muss, und ich könnte es mit einem Muggelmikroskop hinbekommen—“

„Das ist genial, Hermine!“ Der Junge zog schnell seinen Zauberstab, sagte „Quietus“ und fuhr dann fort, nachdem die kleinen Geräusche der rüpelhaften Bücher verstummt waren. „Selbst wenn der Stein der Weisen nur ein Mythos ist, könnte derselbe Trick auch für andere schwierige Tränke funktionieren—“

„Nein, es kann nicht funktionieren“, sagte Hermine. Sie war quer durch die Bibliothek geflogen, um das einzige Buch über Alchemie nachzuschlagen, das nicht in der verbotenen Abteilung stand. Und dann - sie erinnerte sich an die niederschmetternde Enttäuschung, all die plötzliche Hoffnung, die sich wie Nebel auflöste. „Weil alle alchemistischen Kreise 'auf die Feinheit eines Kinderhaares' gezogen werden müssen, ist es für manche Alchemien nicht feiner als für andere. Und Zauberer haben ein Omniokular, und ich habe noch von keinem Zauber gehört, bei dem man das Omniokular benutzt, um Dinge zu vergrößern und sie genau zu machen. Das hätte ich merken müssen!“

„Hermine“, sagte Harry ernst, während er begann, wieder in dem roten Samtbeutel zu wühlen, „bestrafe dich nicht, wenn eine glänzende Idee nicht klappt. Du musst eine Menge fehlerhafter Ideen durchgehen, um eine zu finden, die funktionieren könnte. Und wenn du deinem Gehirn ein negatives Feedback sendest, indem du die Stirn runzelst, wenn dir eine fehlerhafte Idee einfällt, anstatt zu erkennen, dass das Vorschlagen von Ideen ein gutes Verhalten deines Gehirns ist, das es zu fördern gilt, wirst du bald überhaupt keine Ideen mehr haben.“

Harry legte zwei herzförmige Pralinen neben dem Buch ab.

„Hier, nimm noch eine Schokolade. Außer der von vorhin, meine ich. Diese hier soll dein Gehirn stärken, damit du eine gute Kandidatenstrategie entwickeln kannst.“

„Ich nehme an, du hast recht“, sagte Hermine mit leiser Stimme, aber sie rührte die Schokolade nicht an. Sie begann, die Seiten auf 167 zurückzublättern, wo sie gelesen hatte, bevor Harry hereingekommen war. (Hermine Granger brauchte natürlich keine Lesezeichen.)

Harry beugte sich leicht vor, sein Kopf berührte fast ihre Schulter und beobachtete die Seiten, während sie sie umblätterte, als könnte er aus einem kurzen Blick auf die Seite wertvolle Informationen herauslesen. Das Frühstück war noch nicht lange her, und sie konnte an dem schwachen Geruch seines Atems deutlich erkennen, dass Harry zum Nachtisch Bananenpudding gegessen hatte.

Harry sprach wieder. „Also nach all dem…und bitte fass das als positive Bestärkung auf…hast du wirklich versucht, einen Weg zur Massenproduktion von Unsterblichkeit zu erfinden, damit ich meine Schulden bei Lucius Malfoy abbezahlen kann?“

„Ja“, sagte sie mit noch leiserer Stimme. Selbst wenn sie versuchte, wie Harry zu denken, schien sie den Dreh noch nicht raus zu haben. „Und was hast du die ganze Zeit gemacht, Harry?“

Harry machte ein angewidertes Gesicht. „Versucht, Beweise für das ganze '\emph{Wer hat Hermine Granger reingelegt}'-Rätsel zu sammeln.“

„Ich…“ Hermine sah zu Harry auf. „Sollte ich… nicht versuchen, mein eigenes Rätsel zu lösen?“ Es war nicht ihr erster Gedanke gewesen, ihre erste Priorität, aber jetzt, wo Harry es erwähnte…

„Das würde in diesem Fall nicht funktionieren“, sagte Harry nüchtern. „Es gibt zu viele Leute, die mit mir reden werden und nicht mit dir…und ich muss leider auch sagen, dass einige von ihnen mir das Versprechen abverlangt haben, mit niemandem sonst zu reden. Tut mir leid, ich glaube nicht, dass du in dieser Sache viel helfen kannst.“

„Okay, schätze ich“, sagte Hermine träge. „Gut. Du machst alles. Du sammelst alle Hinweise und sprichst mit allen Verdächtigen, während ich nur hier in der Bibliothek sitze. Sag mir Bescheid, wenn sich herausstellt, dass es Professor Quirrell war, der es getan hat.“

„Hermine…“ sagte Harry. „Warum ist es so wichtig, wer was tut? Sollte es nicht wichtiger sein, dass alles aufgeklärt wird, als wer es aufklärt?“

„Ich schätze, du hast recht“, sagte Hermine. Sie hob ihre Hände und drückte sie an ihre Augen. „Ich schätze, es spielt keine Rolle mehr. Alle werden denken - ich weiß, es ist nicht deine Schuld, Harry, du warst - du warst gut, du warst ein perfekter Gentleman - aber egal, was ich jetzt tue, sie werden alle denken, dass ich nur - jemand bin, den du retten sollst.“

Sie hielt inne und sagte mit bebender Stimme: „Und vielleicht haben sie recht, Harry.“

„Whoa, whoa, warte mal eine Sekunde—“

„Ich kann Dementoren nicht erschrecken. Ich kriege zwar im Zauberunterricht Bestnoten, aber ich kann keine Dementoren erschrecken.“

„Ich habe eine geheimnisvolle dunkle Seite!“ zischte Harry, nachdem er seinen Kopf herumgedreht hatte, um die Bibliothek zu scannen. (Es gab einen Jungen in einer entfernten Ecke, der zwar gelegentlich in ihre Richtung schaute, aber er wäre zu

weit weg gewesen, um etwas zu hören, selbst ohne die Schweigebarriere.) „Ich habe eine dunkle Seite, die definitiv kein Kind ist, und wer weiß, was sonst noch für verrücktes magisches Zeug in meinem Kopf vor sich geht - Professor Quirrell hat behauptet, dass ich zu dem werde, für den ich mich halte - das ist alles Betrug, verstehst du nicht, Hermine? Es gibt eine Abmachung, die die Schulleitung getroffen hat, über die ich nicht sprechen darf, damit der Junge-der-lebte mehr Zeit zum Lernen hat, ich schummle und du schlägst mich immer noch in Zauberei. Ich bin - ich bin wahrscheinlich nicht - der Junge-der-lebte ist wahrscheinlich nicht einmal etwas, das man richtig als Kind bezeichnen könnte - und du konkurrierst immer noch damit. Ist dir nicht klar, dass du die mächtigste Hexe seit einem Jahrhundert wärst, wenn die Leute nicht auf mich achten würden? Wenn du allein gegen drei ältere Schüler antreten kannst und gewinnst?“

„Ich weiß es nicht“, sagte sie und presste ihre Hände wieder über die Augen, wobei ihre Stimme schwankte. „Alles, was ich weiß, ist - selbst wenn das alles wahr ist -, dass mich niemand mehr für mich selbst sehen wird, niemals.“

„In Ordnung“, sagte Harry nach einer Weile. „Ich verstehe, was du meinst. Statt des berühmten Forscherteams Potter-und-Granger wird es Harry Potter und seine Laborassistentin geben. Ähm… hier ist eine Idee. Wie wäre es, wenn ich mich eine Zeit lang nicht aufs Geldverdienen konzentriere? Ich meine, die Schulden werden erst fällig, wenn ich meinen Abschluss in Hogwarts mache. Also kannst du es selbst tun und der Welt zeigen, dass du es noch drauf hast. Und wenn du nebenbei zufällig das Geheimnis der Unsterblichkeit knackst, nennen wir es einfach einen Bonus.“

Der Gedanke, dass Harry sich darauf verließ, dass sie eine Lösung fand, schien…wie eine erdrückende Last der Verantwortung, die er einem armen, traumatisierten zwölfjährigen Mädchen aufbürdete, und sie wollte ihn umarmen, weil er ihr einen Weg anbot, ihre Selbstachtung als Heldin wiederherzustellen, und es war das, was sie verdiente, weil sie eine schreckliche Person war und die ganze Zeit scharf mit Harry gesprochen hatte, obwohl er die ganze Zeit ein wahrhaftiger Freund für sie gewesen war, als sie es je für ihn gewesen war, und es war gut, dass er ihr immer noch Dinge zutraute, und …

„Gibt es eine erstaunliche rationale Sache, die du tust, wenn dein Verstand in alle möglichen Richtungen läuft?“, schaffte sie es.

„Meine eigene Herangehensweise besteht normalerweise darin, die verschiedenen Wünsche zu identifizieren, ihnen Namen zu geben, sie als separate Individuen zu betrachten und sie in meinem Kopf ausdiskutieren zu lassen. Bis jetzt sind die hartnäckigsten meine Hufflepuff-, Ravenclaw-, Gryffindor- und Slytherin-Seite, mein innerer Kritiker und meine simulierten Kopien von dir, Neville, Draco, Professor McGonagall, Professor Flitwick, Professor Quirrell, Dad, Mum, Richard Feynman und Douglas Hofstadter.“

Hermine erwog, dies zu versuchen, bevor ihr gesunder Menschenverstand sie warnte, dass es gefährlich sein könnte, so zu tun, als ob.

„Es gibt eine Kopie von mir in deinem Kopf?“

„Natürlich gibt es eine!“ sagte Harry. Der Junge sah plötzlich ein bisschen verletzlicher aus. „Du meinst, es gibt keine Kopie von mir, die in deinem Kopf lebt?“

Es gab sie, stellte sie fest; und nicht nur das, sie sprach mit genau Harrys Stimme. „Das ist ziemlich beunruhigend, wenn ich so darüber nachdenke“, sagte Hermine. „Ich habe tatsächlich eine Kopie von dir in meinem Kopf leben. Es redet gerade mit mir, indem es deine Stimme benutzt und argumentiert, dass das völlig normal ist.“

„Gut“, sagte Harry ernst. „Ich meine, ich wüsste nicht, wie man ohne das befreundet sein könnte.“

Sie las weiter in ihrem Buch, und Harry schien sich damit zu begnügen, die Seiten über ihre Schulter zu betrachten. Sie war schon bei Nummer siebzig angelangt, bei Katherine Scott, die anscheinend eine Möglichkeit erfunden hatte, kleine Tiere in Zitronenkuchen zu verwandeln, als sie endlich den Mut aufbrachte, zu sprechen.

„Harry?“, sagte sie. (Sie lehnte sich jetzt ein wenig von ihm weg, obwohl sie es nicht merkte.) „Wenn eine Kopie von Draco Malfoy in deinem Kopf ist, heißt das, dass du mit Draco Malfoy befreundet bist?“

„Na ja…“ sagte Harry. Er seufzte. „Ja, ich hatte sowieso vor, mit dir darüber zu reden. Ich wünschte irgendwie, ich hätte schon früher mit dir geredet. Jedenfalls, wie soll ich das sagen… Ich habe ihn korrumpiert?“

„Was meinst du mit korrumpieren?“

„Ihn auf die helle Seite der Macht zu locken.“

Ihr Mund blieb einfach offen.

„Du weißt schon, wie der Imperator und Darth Vader, nur andersherum.“

„Draco Malfoy“, sagte sie. „Harry, hast du eine Ahnung—“

„Ja.“

„- was für Dinge Malfoy über mich gesagt hat? Was er gesagt hat, was er mit mir machen würde, sobald er die Gelegenheit dazu hätte? Ich weiß nicht, was er dir erzählt hat, aber Daphne Greengrass hat mir erzählt, was Malfoy sagt, wenn er in Slytherin ist. Es ist unaussprechlich, Harry! Es ist unaussprechlich im ganz wörtlichen Sinne, dass ich es nicht laut sagen kann!“

„Wann war das?“ sagte Harry. „Zu Beginn des Jahres? Hat Daphne gesagt, wann das war?“

„Nein“, sagte Hermine. „Weil es keine Rolle spielt, wann, Harry. Jeder, der so etwas sagt - wie Malfoy es gesagt hat - kann kein guter Mensch sein. Es spielt keine Rolle, wozu du ihn verleitet hast, er ist immer noch ein schlechter Mensch, denn egal was, ein guter Mensch würde niemals—“

„Du irrst dich.“ sagte Harry und sah ihr direkt in die Augen. „Ich kann mir denken, was Draco dir anzutun gedroht hat, denn als ich ihn das zweite Mal traf, sprach er davon, es mit einem zehnjährigen Mädchen zu tun. Aber verstehst du nicht, an dem Tag, als Draco Malfoy in Hogwarts ankam, hatte er sein ganzes vorheriges Leben damit verbracht, von Todessern aufgezogen zu werden. Es hätte eines übernatürlichen Eingriffs bedurft, damit er in Anbetracht seiner Umgebung deine Moral besitzt—“

Hermine schüttelte heftig den Kopf. „Nein, Harry. Niemand muss dir sagen, dass es falsch ist, Menschen zu verletzen, es ist nicht etwas, das du nicht tust, weil der Lehrer sagt, dass es nicht erlaubt ist, es ist etwas, das du nicht tust, weil - weil du sehen kannst, wenn Menschen verletzt werden, weißt du das nicht, Harry?“

Ihre Stimme zitterte jetzt.

„Das ist nicht - das ist keine Regel, an die man sich hält wie an die Regeln für Algebra! Wenn man es nicht sehen kann, wenn man es hier nicht spürt“, ihre Hand klatschte über die Mitte ihrer Brust, nicht ganz dort, wo sich ihr Herz befand, aber das spielte keine Rolle, denn es war sowieso alles nur im Gehirn, „dann hat man es einfach nicht!“

Da kam ihr der Gedanke, dass Harry es vielleicht nicht hatte.

„Es gibt Geschichtsbücher, die du nicht gelesen hast“, sagte Harry leise. „Es gibt Bücher, die du noch nicht gelesen hast, Hermine, und sie könnten dir eine ganz andere Perspektive geben. Ein paar Jahrhunderte früher - ich glaube, das gab es definitiv noch im siebzehnten Jahrhundert - war es eine beliebte Dorfunterhaltung, einen Weidenkorb oder ein Bündel mit einem Dutzend lebender Katzen darin zu nehmen und—“

„Stopp“, sagte sie.

„- sie über einem Lagerfeuer zu rösten. Eine ganz normale Feier. Ein guter, sauberer Spaß. Und das muss ich ihnen lassen, es war ein sauberer Spaß als das Verbrennen von Frauen, die sie für Hexen hielten. Denn die Art, wie Menschen gebaut sind, Hermine, die Art, wie Menschen gebaut sind, um sich innerlich zu fühlen—“ Harry legte eine Hand über sein eigenes Herz, in der anatomisch korrekten Position, dann hielt er inne und bewegte seine Hand nach oben, um in Richtung seines Kopfes zu zeigen, ungefähr auf Ohrhöhe, „- ist, dass es ihnen weh tut, wenn sie sehen, dass ihre Freunde leiden. Jemand, der sich in ihrem Umfeld befindet, ein Mitglied ihres eigenen Stammes. Dieses Gefühl hat einen Aus-Schalter, einen Aus-Schalter, der mit 'Feind' oder manchmal einfach nur 'Fremder' beschriftet ist. So sind die Menschen eben, wenn sie nicht anders lernen. Also, nein, es deutet nicht darauf hin, dass Draco Malfoy unmenschlich oder gar ungewöhnlich böse war, wenn er in dem Glauben aufgewachsen ist, dass es Spaß macht, seine Feinde zu verletzen—“

„Wenn man das glaubt“, sagte sie mit unsicherer Stimme, „wenn man das glauben kann, dann ist man böse. Menschen sind immer verantwortlich für das, was sie tun. Es spielt keine Rolle, was dir jemand sagt, was du tun sollst, du bist derjenige, der es tut. Jeder weiß, dass—“

„Nein, tun sie nicht! Du bist in einer Gesellschaft nach dem Zweiten Weltkrieg aufgewachsen, in der \emph{'Ich habe nur Befehle befolgt}' etwas ist, von dem jeder weiß, dass es die Bösen sagen. Im fünfzehnten Jahrhundert hätte man das '\emph{ehrenhafte Treue}' genannt.“

Harrys Stimme erhob sich.

„Glaubst du, du bist, du bist einfach genetisch besser als alle, die damals gelebt haben? Als ob du, wenn du als Baby in das London des fünfzehnten Jahrhunderts zurückversetzt worden wärst, ganz von allein erkannt hättest, dass Katzenverbrennung falsch ist, Hexenverbrennung falsch ist, Sklaverei falsch ist, dass jedes empfindungsfähige Wesen in deinen Sorgenkreis gehören sollte? Glaubst du, du würdest all das bis zu deinem ersten Tag in Hogwarts begreifen? Niemand hat Draco je gesagt, dass er persönlich dafür verantwortlich war, ethischer zu werden als die Gesellschaft, in der er aufwuchs. Und trotzdem hat er nur vier Monate gebraucht, um an den Punkt zu gelangen, an dem er eine Muggelgeborenen, die von einem Gebäude fällt, auffangen wollte.“

Harrys Augen waren so grimmig, wie sie ihn je gesehen hatte.

„Ich bin noch nicht fertig damit, Draco Malfoy zu korrumpieren, aber ich denke, er hat sich bis jetzt ziemlich gut geschlagen.“

Das Problem mit einem so guten Gedächtnis war, dass sie sich tatsächlich erinnerte. Sie erinnerte sich, dass Draco Malfoy ihr Handgelenk gepackt hatte, so fest, dass sie danach einen blauen Fleck hatte, als sie vom Dach von Hogwarts fiel. Sie erinnerte sich daran, wie Draco Malfoy ihr aufhalf, nachdem dieser mysteriöse Stolperzauber sie in den Teller des Slytherin-Quidditch-Kapitäns mit dem Essen hatte stolpern lassen. Und sie erinnerte sich - eigentlich war das der Grund, warum sie das Thema überhaupt angesprochen hatte - wie sie sich gefühlt hatte, als sie Draco Malfoys Aussage unter Veritaserum gehört hatte.

„Warum hast du mir nichts davon erzählt?“ sagte Hermine, und ihre Stimme stieg in der Tonlage. „Wenn ich das gewusst hätte—“

„Es war nicht mein Geheimnis, das ich dir erzählen konnte“, sagte Harry. „Draco ist derjenige, der in Gefahr gewesen wäre, wenn sein Vater es herausgefunden hätte.“

„Ich bin nicht dumm, Mr~Potter. Was ist der wahre Grund, warum du es mir nicht gesagt hast, und was hast du eigentlich mit Mr~Malfoy gemacht?“

„Ah. Na ja…“ Harry brach den Blickkontakt zu ihr ab und sah auf den Bibliothekstisch hinunter.

„Draco Malfoy hat den Auroren unter Veritaserum erzählt, dass er wissen wollte, ob er mich schlagen kann, also hat er mich zu einem Duell herausgefordert, um es \emph{empirisch zu testen}. Das waren seine genauen Worte laut der Abschrift.“

„Richtig“, sagte Harry und begegnete ihren Augen immer noch nicht.

\emph{Hermine Granger. Natürlich wird sie sich an den genauen Wortlaut erinnern. Es spielt keine Rolle, ob sie an ihren Stuhl gekettet ist und vor dem gesamten Zaubergamot wegen Mordes vor Gericht steht—}

„Was hast du wirklich mit Draco Malfoy gemacht?“

Harry zuckte zusammen und sagte: „Wahrscheinlich nicht ganz das, was du denkst, aber…“

Das Entsetzen kletterte in ihr hoch und runter und brach schließlich los.

„Du hast mit ihm WISSENSCHAFT betrieben????“

„Nun—“

„Du hast Wissenschaft mit ihm gemacht? Du solltest mit MIR Wissenschaft betreiben!“

„So war es nicht! Es ist nicht so, dass ich echte Wissenschaft mit ihm gemacht habe! Ich habe ihm nur, du weißt schon, ein paar harmlose Sachen aus der Muggelwissenschaft beigebracht, wie elementare Physik mit Algebra und so weiter - es ist nicht so, dass ich mit ihm originale magische Forschung betrieben habe, so wie ich es mit dir getan habe—“

„Und ich nehme an, du hast ihm auch nichts von mir erzählt?“

„Ähm, natürlich nicht?“ sagte Harry. „Ich habe seit Oktober mit ihm geforscht, und da war er noch nicht ganz bereit, von dir zu hören—“

Das unaussprechliche Gefühl des Verrats in ihr wogte und wogte, übernahm alles, ihre ansteigende Stimme, ihre glühenden Augen, ihre Nase, von der sie sicher war, dass sie zu laufen begann, das Brennen in ihrer Kehle. Sie stieß sich vom Tisch ab und trat einen Schritt zurück, um besser auf ihren Verräter herabsehen zu können, und ihre Stimme war fast kreischend, als sie schrie:

„Das ist nicht in Ordnung! Du kannst nicht mit zwei Leuten gleichzeitig Wissenschaft betreiben!“

„Äh—“

„Ich meine, du kannst nicht mit zwei verschiedenen Leuten Wissenschaft betreiben und ihnen nichts voneinander erzählen!“

„Ah…“ Harry sagte vorsichtig. „Daran habe ich gedacht, und ich war sehr vorsichtig, deine Forschung nicht mit dem zu vermischen, was ich mit ihm gemacht habe—“

„Du warst sehr vorsichtig.“

Sie hätte es gezischt, wenn es ein großes „S“ enthalten hätte.

Harry hob eine Hand und strich sich durch sein unordentliches Haar, und irgendwie brachte sie das dazu, ihn noch mehr anschreien zu wollen.

„Miss~Granger“, sagte Harry, „ich glaube, diese Unterhaltung ist auf einer Ebene metaphorisch geworden, die, ähm…“

„Was?“, kreischte sie ihn an, aus vollem Halse, innerhalb ihrer Barriere. Dann begriff sie und wurde so rot, dass ihre Haare spontan Feuer gefangen hätten, wenn sie ein erwachsenes Maß an magischer Kraft gehabt hätte.

Der einzige andere Besucher in der Bibliothek, der Ravenclaw-Junge, der in der gegenüberliegenden Ecke saß, starrte die beiden mit großen Augen an, während er einen ziemlich traurigen Versuch unternahm, dies zu verbergen, indem er ein Buch direkt unter seinem Gesicht hochhielt.

„Richtig“, sagte Harry mit einem kleinen Seufzer. „Also, wenn man sich vor Augen hält, dass es nur eine schlechte Metapher war und dass echte Wissenschaftler die ganze Zeit miteinander zusammenarbeiten, glaube ich nicht, dass ich geschummelt habe. Wissenschaftler schweigen oft über Projekte, an denen sie arbeiten. Du und ich forschen an etwas, das wir geheim halten, und es gab Gründe, vor allem Draco Malfoy nichts davon zu erzählen - er wäre anfangs gar nicht in meiner Nähe geblieben, wenn er gewusst hätte, dass ich dein Freund und nicht dein Rivale bin. Und Draco wäre derjenige gewesen, der in Gefahr gewesen wäre, wenn ich jemand anderem von ihm erzählt hätte—“

„Ist das wirklich alles?“, fragte sie. „Wirklich, Harry? Wolltest du nicht, dass wir beide uns besonders fühlen, als wären wir die Einzigen, mit denen du zusammen sein wolltest und die Einzigen, die mit dir zusammen sein dürfen?“

„Das war nicht der Grund, warum ich—“ Harry hielt inne. Harry sah sie an.

Das ganze Blut rauschte zurück in ihr Gesicht, wahrscheinlich hätte Dampf aus ihren Ohren kommen müssen, die wiederum mit dem flüssigen Fleisch, das ihr in den Nacken lief, von ihrem Kopf hätten schmelzen müssen, als sie realisierte, was sie gerade herausgeplatzt hatte.

Harry starrte sie in dämmerndem und völligem Entsetzen an.

„Nun…“, sagte sie mit ziemlich hoher Stimme, „es ist…oh, ich weiß nicht, Harry! Ist das nur eine Metapher? Wenn ein Junge hunderttausend Galleonen ausgibt, um ein Mädchen vor dem sicheren Untergang zu bewahren, dann hat sie doch ein Recht darauf, sich zu wundern, meinst du nicht? Es ist, als würde man Blumen kaufen, nur, weißt du, etwas mehr—“

Harry stieß sich vom Tisch ab und trat einen taumelnden Schritt zurück, auch wenn er die Arme hochhob, um verzweifelt zu winken. „Das ist nicht der Grund, warum ich es getan habe! Ich habe es getan, weil wir Freunde sind!“

„Nur Freunde?“ Harry Potters Atmung begann sich in Richtung Hyperventilation zu steigern.

„Sehr gute Freunde! Ganz besondere Freunde sogar! Beste Freunde für immer, möglicherweise! Aber nicht diese Art von Freunden!“

„Ist es wirklich so schrecklich, daran zu denken?“, fragte sie mit einem Haken in der Stimme.

„Ich meine - ich will nicht sagen, dass ich in dich verliebt bin, aber—“

„Oh, das bist du nicht? Gott sei Dank.“

Harry zog den Ärmel seines Mantels hoch und wischte sich über die Stirn.

„Hör zu, Hermine, bitte versteh mich nicht falsch, ich bin sicher, du bist ein wunderbarer Mensch—“

Sie trat einen schwankenden Schritt zurück.

„- aber - auch mit meiner dunklen Seite—“

„Ist es das, worum es hier geht?“, fragte Hermine.

„Aber ich - ich würde nicht - Nein, nein, ich meine, ich habe eine geheimnisvolle dunkle Seite und wahrscheinlich noch andere seltsame magische Dinge, die vor sich gehen, du weißt, ich bin kein normales Kind, nicht wirklich—“

„Es ist okay, nicht normal zu sein“, sagte sie und fühlte sich zunehmend verzweifelt und verwirrt. „Ich habe kein Problem damit—“

„Aber selbst mit all dem seltsamen magischen Zeug, das mich erwachsener sein lässt, als ich sein sollte, bin ich noch nicht in die Pubertät gekommen, und es gibt keine Hormone in meinem Blutkreislauf, und mein Gehirn ist physisch nicht in der Lage, sich in jemanden zu verlieben. Also bin ich nicht in dich verliebt! Ich kann unmöglich in dich verliebt sein! Soweit ich weiß, wird mein Gehirn vielleicht in sechs Monaten aufwachen und beschließen, sich in Professor Snape zu verlieben! Ähm, darf ich daraus schließen, dass du in die Pubertät gekommen bist?“

„Eep“, sagte Hermine mit einem hohen Ton.

Sie schwankte, wo sie stand, und einen Moment später eilte Harry zu ihr hinüber und half ihr, sich auf den Boden zu setzen, wobei er ihren Körper mit festen Händen stützte. Tatsache war, dass sie damals im Dezember zu Professor McGonagalls Büro getorkelt war, nicht völlig überrascht war weil sie ihre Lektüre gemacht hatte, aber doch ziemlich mulmig, und mit großer Erleichterung hatte sie erfahren, dass Hexen Zaubersprüche hatten, um mit den Unannehmlichkeiten fertig zu werden, und was tat Harry überhaupt, einem armen, unschuldigen Mädchen eine solche Frage zu stellen—

„Hör mal, es tut mir leid“, sagte Harry verzweifelt. „Ich habe das meiste davon wirklich nicht so gemeint, wie es sich angehört hat! Ich bin mir sicher, dass jeder, der die ganze Situation von außen betrachtet und Wetten darauf abschließt, wen ich letztendlich heiraten werde, dir eine höhere Wahrscheinlichkeit zuordnen würde als jedem anderen, der mir einfällt—“

Ihre Intelligenz, die gerade erst begonnen hatte, sich zusammenzureißen, explodierte prompt in Funken und Flammen.

„- wenn auch nicht unbedingt eine höhere Wahrscheinlichkeit als fünfzig Prozent, ich meine, von außen betrachtet gibt es eine Menge anderer Möglichkeiten, und wen ich mag, bevor ich in die Pubertät komme, ist wahrscheinlich keine allzu starke Diagnose dafür, mit wem ich sieben Jahre später zusammen sein werde - ich will nicht so klingen, als würde ich irgendetwas versprechen—“

Ihre Kehle gab irgendeine Art von hohen Tönen von sich, und sie hörte nicht genau hin, was. Ihr ganzes Universum hatte sich auf Harrys schreckliche, schreckliche Stimme verengt.

„- und außerdem habe ich über Evolutionspsychologie gelesen, und, na ja, es gibt all diese Hinweise darauf, dass ein Mann und eine Frau, die bis ans Ende ihrer Tage glücklich zusammenleben, eher die Ausnahme als die Regel sein könnten, und in Jäger- und Sammlerstämmen war es häufiger so, dass man nur zwei oder drei Jahre zusammenblieb, um ein Kind in seiner verletzlichsten Phase aufzuziehen - und, ich meine, wenn man bedenkt, wie viele Menschen in traditionellen Ehen schrecklich unglücklich enden, scheint das eine Sache zu sein, die einer cleveren Überarbeitung bedarf - besonders, wenn wir tatsächlich das Problem mit der Unsterblichkeit lösen—“

…

Tano Wolfe, Ravenclaw im fünften Jahr, stand langsam von seinem Bibliothekstisch auf, von dem aus er gerade beobachtet hatte, wie Granger schluchzend aus der Bibliothek floh. Er hatte den Streit nicht hören können, aber es war eindeutig einer von diesen Streits gewesen. Langsam und mit zitternden Knien näherte sich Tano dem Jungen-der-lebte, der in Richtung der Bibliothekstüren starrte, die noch immer von der Wucht des Zuschlagens vibrierten. Tano wollte das nicht unbedingt tun, aber Harry Potter war in Ravenclaw einsortiert worden. Der Junge-der-lebte war, technisch gesehen, sein Ravenclaw-Kollege. Und das bedeutete, dass es einen Kodex gab. Der Junge-der-lebte sagte nichts, als Tano auf ihn zukam, aber sein Blick war nicht freundlich.

Tano schluckte, legte Harry Potter eine Hand auf die Schulter und rezitierte mit leicht brüchiger Stimme die uralten traditionellen Worte: „Hexen! Nicht zu fassen, was?“

„\textbf{\emph{Nimm deine Hand weg, bevor ich dich in die äußere Finsternis werfe.}}“

Die Bibliothekstüren knallten im Zuge eines weiteren Abgangs noch einmal zu.

