

\hypertarget{rollen-nachspiel-vorsichtsmauxdfnahmen-teil-1}{% \section{99. Rollen, Nachspiel + Vorsichtsmaßnahmen, Teil 1}\label{rollen-nachspiel-vorsichtsmauxdfnahmen-teil-1}}

\textbf{\uline{Rollen, Nachspiel}}

10 Tage später wurde das erste tote Einhorn im Verbotenen Wald gefunden.

\textbf{\uline{Vorsichtsmaßnahmen, Teil 1}}

13. Mai 1992.

Argus Filchs Gesicht erschien im Licht der Öllampe, die er in der Hand hielt, verzerrt, Schatten tanzten über sein Gesicht. Hinter ihnen wichen die Türen von Hogwarts schnell zurück, und das dunkle Gelände rückte näher. Der Weg, den sie nun gingen, war schlammig und undeutlich. Die Bäume, deren Äste früher vom Winter kahl waren, waren noch nicht ganz vom Frühling umhüllt; ihre Zweige streckten sich wie magere Finger zum Himmel, Skelette waren inmitten des dünnen Laubes zu sehen. Der Mond war hell, aber Wolken, die über ihn hinweg zogen, warfen sie oft in die Dunkelheit, die nur von den schwachen Flammen von Filchs Lampe erhellt wurde.

Draco behielt seinen Zauberstab fest im Griff.

"Wo bringst du uns hin?", fragte Tracey Davis. Sie war zusammen mit Draco von Filch erwischt worden, als sie auf dem Weg zu einem geheimen Treffen der silbernen Slytherins nach der Sperrstunde waren, und hatte ebenfalls nachsitzen müssen.

"Du folgst mir einfach", sagte Argus Filch.

Draco fühlte sich von der ganzen Angelegenheit ziemlich genervt. Die silbernen Slytherins sollten als Schulvereinigung anerkannt werden. Es gab keinen Grund, warum eine geheime Verschwörung nicht die Erlaubnis haben sollte, sich nach der Ausgangssperre zu treffen, wenn es dem größeren Wohl von Hogwarts diente. Wenn das noch einmal passierte, würde er mit Daphne Greengrass reden und Daphne würde mit ihrem Vater reden und dann würde Filch die Weisheit lernen, wegzusehen, wenn es um die Malfoys ging.

Die Lichter des Hogwarts-Schlosses waren in der Ferne schwächer geworden, als Filch wieder sprach. "Ich wette, du wirst es dir zweimal überlegen, ob du noch einmal eine Schulregel brichst, nicht wahr?" Filch drehte seinen Kopf von der Lampe weg, damit er die vier Schüler, die ihm folgten, anstarren konnte. "Oh ja… harte Arbeit und Schmerz sind die besten Lehrer, wenn ihr mich fragt… Es ist nur schade, dass sie die alten Bestrafungen aussterben lassen… euch an den Handgelenken für ein paar Tage an die Decke hängen, ich habe die Ketten noch in meinem Büro, bewahre sie gut geölt auf, falls sie jemals gebraucht werden…"

"Hey!" sagte Tracey, ein Hauch von Empörung drang in ihre Stimme. "Ich bin zu jung, um so etwas zu hören - diese Art von - du weißt schon! Vor allem, wenn die Ketten gut geölt sind!"

Draco hatte nicht aufgepasst. Filch war einfach nicht in der Liga von Amycus Carrow. Hinter ihnen kicherte eine der beiden älteren Slytherins, die ihnen folgten, obwohl sie nichts sagte. Neben ihr war der andere, ein großer Junge mit einem slawischen Gesichtsausdruck, der immer noch mit Akzent sprach. Sie waren wegen irgendeines Vergehens erwischt worden, das irgendetwas mit der Art von Dingen zu tun hatte, von denen Tracey sprach, und sahen aus, als wären sie im dritten oder vierten Jahr.

"Pfeh", sagte der größere Junge. "In Durmstrang hängen sie dich kopfüber an den Zehen auf. An einem Zeh, wenn man frech ist. Hogwarts war früher schon weich."

Argus Filch schwieg etwa eine halbe Minute lang, als ob er sich eine passende Erwiderung überlegen wollte, und gluckste dann. "Wir werden sehen, was du dazu sagst … wenn du erfährst, was du heute Abend tun wirst! Ha!"

"Ich sagte doch, ich bin zu jung für so etwas!", sagte Tracey Davis. "Das muss warten, bis ich älter bin!"

Vor ihnen lag ein Häuschen mit erleuchteten Fenstern, doch die Proportionen schienen falsch zu sein. Filch pfiff, ein hoher, scharfer Ton, und ein Hund fing an zu bellen. Aus dem Häuschen trat eine Gestalt hervor, die die Bäume um sie herum zu kurz erscheinen ließ. Der Gestalt folgte ein Hund, der im Vergleich dazu wie ein Welpe wirkte, bis man ihn abgesehen von der größeren Silhouette betrachtete und erkannte, dass der Hund riesig war, eher wie ein Wolf. Dracos Augen verengten sich, bevor er sich wieder fing. Als Silberner Slytherin durfte er keine Vorurteile gegenüber einem fühlenden Wesen haben, schon gar nicht, wenn andere Menschen ihn sehen könnten.

"Wer ist das?", sagte die Gestalt mit der lauten, schroffen Stimme des Halbriesen. Sein Schirm leuchtete in einem weißen Schein, heller als Filchs schwache Lampe. In der anderen Hand hielt er eine Armbrust; ein Köcher mit kurzen Bolzen war an seinen Oberarm geschnallt.

"Schüler beim Nachsitzen", sagte Filch laut. "Sie werden dir helfen, den Wald nach … was auch immer sie gefressen hat, zu durchsuchen."

"Der Wald?", keuchte Tracey. "Wir dürfen da nachts nicht reingehen!"

"Das stimmt normalerweise", sagte Filch und wandte sich von Hagrid ab, um sie anzustarren. "Aber heute müsst in den Wald gehen, und ich glaube kaum, dass ihr alle heil wieder herauskommt."

"Aber -", sagte Tracey. "Da gibt es Werwölfe, habe ich gehört, und Vampire, und jeder weiß, was passiert, wenn ein Mädchen und ein Werwolf und ein Vampir gleichzeitig da sind!"

Der riesige Halbriese runzelte die Stirn. "Argus, ich hatte an dich und vielleicht ein paar Siebtklässler gedacht. Es hat nicht viel Sinn, Hilfe mitzubringen, wenn ich die ganze Zeit auf sie aufpassen soll."

Argus' Gesicht leuchtete mit grausamer Genugtuung. "Das ist ihr Problem, nicht wahr? Sie hätten an die Werwölfe denken sollen, bevor sie in Schwierigkeiten geraten, oder? Schick sie allein raus. Du solltest nicht zu freundlich zu ihnen sein, Hagrid. Immerhin sind sie hier, um bestraft zu werden."

Der Halbriese stieß einen gewaltigen Seufzer aus (es klang wie bei einem normalen Menschen, dem eine Knüppelhexe die ganze Luft aus den Lungen treibt). "Du hast deinen Teil getan. Ich übernehme ab hier."

"Ich bin im Morgengrauen wieder da", sagte Filch, "für das, was von ihnen übrig ist", fügte er böse hinzu, drehte sich um und ging zurück zum Schloss, wobei seine Lampe in der Dunkelheit vor sich hin dämmerte.

"Also gut", sagte Hagrid, "jetzt hört gut zu, denn es ist gefährlich, was wir heute Nacht tun werden, und ich will nicht, dass jemand ein Risiko eingeht. Folgt mir einen Moment hierher." Er führte sie an den Rand des Waldes. Er hielt seine Lampe hoch und zeigte auf einen schmalen, gewundenen Erdweg, der in den dichten schwarzen Bäumen verschwand. Eine leichte Brise wehte über Dracos Kopf, als er in den Wald blickte. "Da drinnen gibt es etwas, das Einhörner frisst", sagte der riesige Mann. Draco nickte; er erinnerte sich entfernt daran, vor ein paar Wochen, gegen Ende April, etwas in dieser Richtung gehört zu haben.

"Haben Sie uns gerufen, um eine Spur von silbrigem Blut zu einem verwundeten Einhorn zu finden?" sagte Tracey aufgeregt.

"Nein", sagte Draco, obwohl er es schaffte, das reflexartige Grinsen zu unterdrücken. "Filch hat uns heute Mittag den Zettel zum Nachsitzen gegeben, um 12 Uhr. Mr. Hagrid würde nicht so lange warten, um ein verwundetes Einhorn zu finden, und wenn wir nach so etwas suchen würden, würden wir am Tag suchen, wenn es hell ist. Also", Draco hielt einen Finger hoch, wie er es beim untersuchenden Auror Lesh in Theaterstücken gesehen hatte, "schließe ich daraus, dass wir nach etwas suchen, das nur nachts auftaucht."

"Aye", sagte der Halbriese und klang nachdenklich. "Du bist nicht das, was ich erwartet habe, Draco Malfoy. Ganz und gar nicht, was ich erwartet habe. Dann bist du also Tracey Davis. Ich habe schon von dir gehört. Eine aus der Truppe der armen Miss Granger." Rubeus Hagrid sah zu den beiden älteren Slytherins hinüber und musterte sie im Licht seines leuchtenden Schirms. "Und wer seid ihr noch mal? Ich kann mich nicht erinnern, dich oft gesehen zu haben, Junge."

"Cornelia Walt", sagte die Hexe, "und das ist Yuri Yuliy", und deutete auf den slawisch aussehenden Jungen, der von Durmstrang gesprochen hatte. "Seine Familie ist zu Besuch aus der Ukraine, und er ist nur für ein Jahr in Hogwarts."

Der ältere Junge nickte, mit einem leicht verächtlichen Ausdruck im Gesicht.

"Das ist Fang", sagte Hagrid und deutete auf den Hund. Zu fünft machten sie sich auf den Weg in den Wald.

"Was könnte die Einhörner umbringen?" sagte Draco, nachdem sie ein paar Minuten gelaufen waren. Draco wusste ein wenig über dunkle Kreaturen, aber er konnte sich an nichts erinnern, das Einhörner befallen haben sollte. "Was für eine Art von Kreatur tut das, weiß das jemand?"

"Werwölfe!", sagte Tracey.

"Miss Davis?" sagte Draco, und als sie ihn ansah, deutete er stumm mit einem Finger auf den Mond. Er war zunehmend im zunehmen, aber noch nicht voll.

"Oh, richtig", sagte Tracey.

"Es gibt keine Werwölfe im Wald", sagte Hagrid. "Es können auch keine Wölfe sein, die sind nicht annähernd schnell genug, um ein Einhorn zu fangen. Einhörner sind mächtige magische Geschöpfe, ich habe noch nie erlebt, dass eines verletzt wurde."

Draco hörte sich das an und dachte fast trotzig über das Rätsel nach. "Was ist dann schnell genug, um ein Einhorn zu fangen?"

"Das ist keine Frage der Geschwindigkeit", sagte Hagrid und warf Draco einen unverständlichen Blick zu. "Es gibt unendlich viele Möglichkeiten, wie Kreaturen jagen. Gift, Dunkelheit, Fallen. Kobolde, die man weder sehen, noch hören, noch sich merken kann, selbst wenn sie einen auffressen. Es gibt immer etwas Neues und Wunderbares zu lernen."

Eine Wolke zog über den Mond und warf den Wald in einen Schatten, der nur durch den Schein von Hagrids Schirm erhellt wurde. "Meself", fuhr Hagrid fort, "ich glaube, wir haben hier Pariser Hydra im Wald. Für einen Zauberer sind sie keine Bedrohung, ihr müsst sie nur lange genug aufhalten, dann könnt ihr auf keinen Fall verlieren. Ich meine buchstäblich, dass man nicht verlieren kann, solange man kämpft. Das Problem ist, gegen eine Pariser Hydra geben die meisten Kreaturen schon lange vorher auf. Es dauert eine Weile, bis man alle Köpfe abgehackt hat, versteht ihr?"

"Bah", sagte der ausländische Junge. "In Durmstrang lernen wir, wie man gegen Buchholz-Hydra kämpft. Unvorstellbar mühsamer zu kämpfen! Ich meine das wörtlich, könnt es euch nicht vorstellen. Die Erstklässler glauben uns nicht, wenn wir ihnen sagen, dass ein Sieg möglich ist! Der Ausbilder muss die zweite Befehl geben, wiederholen, bis sie es begreifen."

Sie liefen fast eine halbe Stunde lang, immer tiefer in den Wald hinein, bis es fast unmöglich wurde, dem Pfad zu folgen, weil die Bäume so dicht waren. Dann sah Draco es, dicke Spritzer an den Wurzeln der Bäume, die im Mondlicht in einer helleren Farbe schimmerten.

"Ist das -"

"Das Blut eines Einhorns", sagte Hagrid. Die Stimme des großen Mannes war traurig. Auf einer Lichtung vor ihnen, sichtbar durch die verworrenen Äste einer großen Eiche, sahen sie die gefallene Kreatur, schön und traurig auf dem Boden ausgebreitet, die Erde um sie herum glänzte mondsilbern von gesammeltem Blut. Das Einhorn war nicht weiß, sondern blassblau, oder es schien so unter dem Mond und dem Nachthimmel. Die schlanken Beine standen in seltsamen Winkeln ab, offensichtlich gebrochen, und die Mähne breitete sich über das dunkle Laub aus, grün-schwarz, aber mit einem Schimmer wie Perlen. Auf der Flanke befand sich ein kleines weißes Gebilde wie ein Sternenhimmel, ein Zentrum, das von acht geraden Strahlen umgeben war. Die Hälfte der Seite war weggerissen worden, die Ränder waren ausgefranst wie die Abdrücke von Zähnen, Knochen und innere Organe lagen frei. Ein seltsames Würgegefühl stieg in Dracos Kehle auf.

"Das ist sie", sagte Hagrid, sein trauriges Flüstern so laut wie die Stimme eines normalen Mannes. "Genau da, wo ich sie heute Morgen gefunden habe, tot wie ein toter Türknauf. Sie ist - war - das erste Einhorn, das ich in diesen Wäldern getroffen habe. Ich nannte es Alicorn, aber das spielt wohl keine Rolle mehr."

"Du hast ein Einhorn Alicorn genannt", sagte das ältere Mädchen. Ihre Stimme war ein bisschen trocken.

"Aber es hat keine Flügel", sagte Tracey.

"Alicorn ist das Horn eines Einhorns", sagte Hagrid, jetzt lauter. "Ich weiß nicht, wie ihr alle darauf kommt, dass es ein Einhorn mit Flügeln bedeutet, so etwas gibt es nicht, das habe ich noch nie gehört. Das ist genauso, als würde man einen Hund Fang nennen", und deutete auf den riesigen wolfsähnlichen Hund, der ihm kaum bis zu den Knien reichte. "Wie hättest du sie denn genannt? Hannah, oder so ähnlich? Ich habe ihr einen Namen gegeben, der für sie etwas bedeutet hätte. Höflichkeit, nenne ich es."

Niemand sagte etwas dazu, und nach einem weiteren Moment nickte der riesige Mann scharf. "Wir beginnen unsere Suche von hier aus, dem letzten Ort, an dem es zugeschlagen hat. Wir werden uns in zwei Gruppen aufteilen und der Spur in verschiedene Richtungen folgen. Ihr zwei, Walt und Yuliy - ihr geht da lang und nehmt Fang mit. Im Wald gibt es nichts, was euch schaden könnte, wenn ihr bei Fang seid. Schickt grüne Funken, wenn ihr was Interessantes findet, und rote Funken, wenn jemand Ärger macht. Davis, Malfoy, kommt mit mir."

Der Wald war schwarz und still. Rubeus Hagrid hatte das Licht seines Regenschirms gedimmt, nachdem sie sich auf den Weg gemacht hatten, so dass Draco und Tracey sich im Licht des Mondes orientieren mussten, nicht ohne gelegentlich zu stolpern und zu fallen. Sie liefen an einem bemoosten Baumstumpf vorbei, das Geräusch von fließendem Wasser sprach von einem Bach irgendwo in der Nähe. Ab und zu beleuchtete ein Strahl des Mondlichts durch die Äste oben einen silberblauen Blutfleck auf den gefallenen Blättern; sie folgten der Blutspur, dorthin, wo die Kreatur das Einhorn zuerst getroffen haben musste.

"Es gibt Gerüchte über dich", sagte Hagrid mit leiser Stimme, nachdem sie eine Weile gelaufen waren.

"Nun, sie sind alle wahr", sagte Tracey. "Alle von ihnen."

"Ich meinte nicht dich.", sagte Hagrid. "Hast du wirklich unter Veritaserum ausgesagt, dass du versucht hast, Miss Granger zu helfen, und das gleich dreimal?"

Draco wog seine Worte eine Weile ab und sagte schließlich: "Ja."

Es hätte nicht gut ausgesehen, zu eifrig zu erscheinen, um den Ruhm zu beanspruchen.

Der riesige Mann schüttelte den Kopf, seine großen Füße stapften immer noch lautlos durch den Wald. "Ich bin überrascht, um ehrlich zu sein. Und du auch, Davis, dass du versuchst, in den Hallen für Ordnung zu sorgen. Bist du sicher, dass der Sprechende Hut dich an den richtigen Platz gesetzt hat? Es gibt nicht eine einzige Hexe oder einen Zauberer, der böse wurde, der nicht in Slytherin war, so wurde es immer gesagt."

"Das ist nicht wahr", sagte Tracey. "Was ist mit Xiaonan Tong, dem Schwarzen Raben, Spencer of the Hill und Mister Kayvon?"

"Wer?", sagte Hagrid.

"Nur einige der besten dunklen Zauberer aus den letzten zwei Jahrhunderten", sagte Tracey. "Sie sind wahrscheinlich die besten aus Hogwarts, die nicht aus Slytherin waren." Ihre Stimme wurde leiser, verlor ihren Enthusiasmus. "Miss Granger hat mir immer gesagt, ich solle mich über alles informieren, was ich -"

"Wie auch immer", sagte Draco schnell, "das ist nicht wirklich relevant, Mr. Hagrid. Selbst wenn -" Draco arbeitete in seinem Kopf herum und versuchte, den Unterschied zwischen der Wahrscheinlichkeit, dass Slytherin einen dunklen Zauberer hervorbringt, und der Wahrscheinlichkeit, dass jeder Slytherin eine dunkler Zauberer ist, in eine unwissenschaftliche Sprache zu übersetzen.

"Selbst wenn die meisten Dunklen Zauberer aus Slytherin sind, sind nur sehr wenige Slytherins Dunkle Zauberer. Es gibt nicht \emph{so} viele Dunkle Zauberer, also können nicht alle Slytherins einer sein."

\emph{Oder wie Vater gesagt hatte: Zwar sollte jeder Malfoy einen Großteil der geheimen Überlieferungen kennen, aber die … kostspieligeren Rituale überließ man besser nützlichen Narren wie Amycus Carrow.}

"Du sagst also", sagte Hagrid, "dass die meisten dunklen Zauberer Slytherins sind, aber…"

"Aber die meisten Slytherins sind keine Dunklen Zauberer", sagte Draco.

Er hatte das müde Gefühl, dass sie noch eine Weile damit zu tun haben würden, aber wie beim Kampf gegen eine Hydra war es wichtig, nicht aufzugeben.

"So habe ich das noch nie gesehen", sagte der riesige Mann und klang ehrfürchtig. "Aber, na ja, wenn ihr nicht alle ein Haus von Schlangen seid, warum - geht hinter den Baum!" Hagrid packte Draco und Tracey und hievte sie vom Weg hinter eine hoch aufragende Eiche. Er zog einen Bolzen heraus, spannte ihn in seine Armbrust und hob sie schussbereit an.

Die drei lauschten. Etwas schlitterte über totes Laub in der Nähe: Es klang wie ein Umhang, der über den Boden schleift. Hagrid blinzelte den dunklen Pfad hinauf, aber nach ein paar Sekunden verklang das Geräusch.

"Ich wusste es", murmelte Hagrid. "Hier drin ist etwas, was nicht sein sollte."

Mit Hagrid an der Spitze und Tracey und Draco, die beide ihre Zauberstäbe bereit hielten, gingen sie der Stelle nach, von der das Rascheln gekommen war, aber sie fanden nichts, obwohl sie in einem immer größer werdenden Kreis suchten und ihre Ohren auf das leiseste Geräusch richteten. Sie liefen weiter durch die dichten, dunklen Bäume. Draco schaute immer wieder über seine Schulter, weil ihn das Gefühl beschlich, dass sie beobachtet wurden. Sie hatten gerade eine Biegung des Weges passiert, als Tracey schrie und auf etwas zeigte. In der Ferne erhellte ein Schauer aus roten Funken die Luft.

"Ihr beiden wartet hier!" Hagrid rief. "Bleibt, wo ihr seid, ich komme euch später holen!"

Bevor Draco ein Wort sagen konnte, drehte sich Hagrid und raste durch das Unterholz davon. Draco und Tracey standen da und sahen sich an, bis sie nichts mehr hörten außer dem Rascheln der Blätter um sie herum. Tracey sah verängstigt aus, versuchte aber, es zu verbergen. Draco war mehr verärgert als alles andere. Offenbar hatte Rubeus Hagrid, als er seine Pläne für heute Abend geschmiedet hatte, nicht einmal fünf Sekunden damit verbracht, sich die Konsequenzen vorzustellen, wenn tatsächlich etwas schiefgehen würde.

"Und was jetzt?", fragte Tracey, ihre Stimme etwas zu hoch.

"Wir warten darauf, dass Mr. Hagrid zurückkommt."

Die Minuten zogen sich hin. Dracos Ohren schienen schärfer zu sein als sonst, er nahm jedes Seufzen des Windes, jedes knackende Zweiglein auf. Tracey sah immer wieder zum Mond hinauf, als wolle sie sich vergewissern, dass er noch nicht voll war.

"Ich bin -" Tracey flüsterte. "Ich werde ein bisschen nervös, Mr. Malfoy."

Draco dachte ein wenig darüber nach. Um ehrlich zu sein, da war etwas … nun, es war nicht so, dass er ein Feigling war, oder gar, dass er Angst hatte. Aber es hatte einen Mord in Hogwarts gegeben, und wenn er sich selbst in einem Theaterstück gesehen hätte, nachdem er gerade von einem Halbriesen im Verbotenen Wald ausgesetzt worden war, würde er sich momentan danach fühlen, den Jungen auf der Bühne anzuschreien, dass er…

Draco griff in seinen Umhang und holte einen Spiegel heraus. Ein Tippen auf die Oberfläche zeigte einen Mann in roter Robe, der sofort die Stirn runzelte.

"Auror Captain Eneasz Brodski", sagte der Mann deutlich, was Tracey durch die Lautstärke im stillen Wald aufschrecken ließ. "Was gibt es, Draco Malfoy?"

"Fragen Sie alle 10 Minuten, ob ich noch antworte", sagte Draco. Er hatte beschlossen, sich nicht direkt über sein Nachsitzen zu beschweren. Er wollte nicht wie ein verzogener Bengel aussehen. "Wenn ich nicht reagiere, holen Sie mich ab. Ich bin im Verbotenen Wald."

Im Spiegel hoben sich die Brauen des Aurors. "Was machen Sie im Verbotenen Wald, Mr. Malfoy?"

"Ich suche mit Mr. Hagrid nach dem Einhornfresser", sagte Draco und klappte den Spiegel zu und steckte ihn zurück in seine Robe, bevor der Auror etwas über Nachsitzen fragen oder etwas sagen konnte, ohne sich zu beschweren.

Traceys Kopf drehte sich zu ihm um, obwohl es etwas zu schummrig war, um ihren Gesichtsausdruck zu lesen. "Ähm, danke", flüsterte sie.

Die wenigen Blätter an ihren Ästen raschelten, als eine weitere, kältere Brise durch den Wald wehte. Traceys Stimme war ein wenig lauter, als sie wieder sprach. "Das hättest du nicht tun müssen -", sagte sie und klang nun ein wenig schüchtern.

"Erwähne es nicht, Miss Davis."

Die dunkle Silhouette von Tracey legte die Hand an ihre Wange, als wolle sie ein Erröten verbergen, das ohnehin nicht sichtbar war. "Ich meine, nicht für mich -"

"Nein, wirklich", sagte Draco. "Erwähne es nicht. Überhaupt nicht."

Er hätte gedroht, den Spiegel herauszuholen und Captain Brodski zu befehlen, sie nicht zu retten, aber er hatte Angst, dass sie das als Flirten ansehen würde.

Traceys silhouettierter Kopf wandte sich von ihm ab, sah weg. Schließlich sagte sie mit leiserer Stimme: "Es ist noch zu früh, nicht wahr -"

Ein hoher Schrei hallte durch den Wald, ein nicht ganz menschlicher Laut, der Schrei von etwas wie einem Pferd; und Tracey kreischte und lief davon. "Nein, du Idiot!", schrie Draco und stürzte ihr hinterher. Das Geräusch war so unheimlich gewesen, dass Draco nicht sicher war, woher es kam - aber er dachte, dass Tracey Davis vielleicht tatsächlich direkt auf die Quelle dieses unheimlichen Schreis zulief. Brombeersträucher peitschten vor Dracos Augen, er musste eine Hand vor sein Gesicht halten, um sie abzuschirmen, und versuchte, Tracey nicht aus den Augen zu verlieren, denn es schien offensichtlich, dass, wenn dies ein Spiel war und sie getrennt wurden, einer von ihnen sterben würde. Draco dachte an den Spiegel, der in seinem Umhang befestigt war, aber er wusste irgendwie, dass, wenn er versuchen würde, ihn einhändig herauszunehmen, während er rannte, der Spiegel unweigerlich fallen und verloren gehen würde - vor ihnen hatte Tracey angehalten und Draco fühlte sich für einen Moment erleichtert, bevor er es sah.

Ein weiteres Einhorn lag auf dem Boden, umgeben von einer sich langsam ausweitenden Lache aus silbernem Blut, dessen Rand über den Boden kroch wie verschüttetes Quecksilber. Ihr Fell war violett, wie die Farbe des Nachthimmels, ihr Horn hatte genau die gleiche dämmrige Farbe wie ihre Haut, ihre sichtbare Flanke war von einem rosa Sternenfleck umgeben, der von weißen Flecken umgeben war. Der Anblick zerriss Draco das Herz, mehr noch als bei dem anderen Einhorn, denn die Augen dieses Einhorns starrten ihn glasig an, und da war eine - - verschwommene, sich windende Form -, die sich von einer offenen Wunde an der Seite des Einhorns ernährte, als würde sie daraus trinken - - Draco konnte nicht verstehen, konnte irgendwie nicht erkennen, was er sah - - es sah sie an.

Die verschwommene, brodelnde, unerkennbare Dunkelheit schien sich umzudrehen und sie zu betrachten. Ein Zischen kam von ihr, wie das Zischen der tödlichsten Schlange, die je existiert hatte, etwas, das bei weitem gefährlicher war als jeder Blaue Krait. Dann beugte es sich wieder über die Wunde des Einhorns und trank weiter. Der Spiegel lag in Dracos Hand, und er blieb leblos, während sein Finger mechanisch auf die Oberfläche klopfte, wieder und wieder.

Tracey hielt jetzt ihren Zauberstab in der Hand und sagte Dinge wie "Prismatis" und "Stupefy", aber nichts geschah. Dann erhob sich der brodelnde Umriss, wie ein Mann, der aufsteht, nur nicht so; und er schien nach vorne zu huschen, bewegte sich mit einem seltsamen halben Sprung über die Beine des sterbenden Einhorns und kam auf die beiden zu. Tracey zerrte an seinem Ärmel und drehte sich dann um, um zu rennen, zu rennen vor etwas, das Einhörner jagen konnte. Bevor sie drei Schritte machen konnte, kam ein weiteres schreckliches Zischen, das in seinen Ohren brannte, und Tracey fiel zu Boden und bewegte sich nicht. Irgendwo in seinem Hinterkopf wusste Draco, dass er gleich sterben würde. Selbst wenn der Auror in diesem Augenblick seinen Spiegel überprüfen würde, gab es keine Möglichkeit, dass jemand schnell genug hierher kommen konnte. Es war keine Zeit mehr. Weglaufen hatte nicht funktioniert. Magie hatte nicht funktioniert. Der brodelnde Umriss kam näher, während Draco in seinen letzten Momenten versuchte, das Rätsel zu lösen.

Dann stürzte ein gleißender silberner Lichtball aus dem Nachthimmel und blieb dort hängen, erhellte den Wald so hell wie das Tageslicht, und der brodelnde Umriss sprang zurück, als ob er sich vor dem Licht fürchtete. Vier Besen stürzten aus dem Himmel, drei Auroren mit leuchtenden, bunten Schilden und Harry Potter, der seinen Zauberstab in die Höhe hielt und hinter Professor McGonagall in einem größeren Schild saß.

"Raus hier!", brüllte Professor McGonagall - einen Augenblick bevor das brodelnde Ding ein weiteres schreckliches Zischen von sich gab und alle Schutzzauber erloschen. Die drei Auroren und Professor McGonagall stürzten von ihren Besen und fielen schwer auf den Waldboden, wo sie regungslos liegen blieben. Draco konnte nicht atmen, die intensivste Angst, die er je in seinem Leben gespürt hatte, griff in seine Brust und schickte Ranken um sein Herz. Harry Potter, der unberührt geblieben war, lenkte seinen Besen lautlos in Richtung Boden - - und sprang dann ab, um sich zwischen Draco und den brodelnden Umriss zu stellen und sich wie ein lebendiges Schild dazwischen zu stellen.

"Lauf!", sagte Harry Potter und drehte seinen Kopf halb nach hinten, um Draco anzusehen. Das silberne Mondlicht schimmerte auf seinem Gesicht. "Lauf, Draco! Ich werde es aufhalten!"

"Du kannst das Ding nicht allein bekämpfen!" schrie Draco laut auf.

Ein Brechreiz lag ihm im Magen, ein aufgewühltes Gefühl, das, wenn er sich zurückerinnert, einem Schuldgefühl sowohl ähnlich als auch unähnlich war, als hätte er die Empfindungen, aber nicht ganz das Gefühl.

"Ich muss", sagte Harry Potter grimmig. "Geh!"

"Harry, es - es tut mir leid, für alles - ich"

Obwohl Draco sich später, im Rückblick, nicht mehr genau erinnern konnte, wofür er sich hatte entschuldigen wollen, vielleicht war es, dass er Harrys Verschwörung zu Fall bringen wollte, vor all der Zeit.

Die brodelnde Gestalt, die jetzt noch schwärzer und schrecklicher wirkte, erhob sich in die Luft und schwebte über dem Boden.

"Los!", rief Harry.

Draco drehte sich um und floh kopfüber in den Wald, wobei ihm die Äste ins Gesicht peitschten. Hinter ihm hörte Draco ein weiteres schreckliches Zischen und Harrys Stimme, die sich erhob und etwas rief, das Draco aus der Entfernung nicht erkennen konnte; Draco drehte seinen Kopf nur für einen Augenblick, um sich umzudrehen, und in diesem Moment rannte er gegen etwas, schlug mit dem Kopf \textbf{HART} auf und wurde ohnmächtig.

Harry hielt seinen Zauberstab fest umklammert, eine prismatische Sphäre glühte um ihn herum. Er starrte die brodelnde, verschwommene Form vor ihm an und sagte: "Was in aller Welt tust du da?"

Die brodelnde, verschwommene Form löste sich auf, formte sich neu und entspannte sich wieder zu einer Kapuzenform. Was auch immer für eine Verschleierung am Werk gewesen war - eher ein Gerät als ein Zauber, vermutete Harry, da die Magie in der Lage gewesen war, ihn zu beeinflussen - hatte seinen Verstand daran gehindert, die Form zu erkennen oder sogar, dass die Form menschlich war. Aber es hatte Harry nicht daran gehindert, das scharfe Gefühl des Unheils zu erkennen.

Professor Quirrell stand aufrecht, das silberne Blut lief über die Vorderseite seines verhüllenden schwarzen Umhangs, er seufzte und betrachtete die gefallenen Gestalten der drei Auroren, Tracey Davis, Draco Malfoy und Professor McGonagall.

"Ich hatte ehrlich gesagt gedacht", murmelte Professor Quirrell, "dass ich den Spiegel unbemerkt blockiert habe. Was haben zwei Slytherins im ersten Jahr allein im Verbotenen Wald gemacht? Mr. Malfoy hat mehr Verstand als das… Was für ein Fiasko."

Harry antwortete nicht. Das Gefühl des Unheils war so stark, wie Harry sich nicht erinnern konnte, es jemals gefühlt zu haben, ein Gefühl der Macht in der Luft, das so groß war, dass es fast greifbar war. Ein Teil von ihm war immer noch zutiefst schockiert darüber, wie schnell die Schilde, die die Auroren umgaben, auseinandergerissen worden waren. Er war fast nicht in der Lage gewesen, die aufeinanderfolgenden Farbschläge zu sehen, die die Schilde wie Seidenpapier weggerissen hatten. Dagegen sah das Duell, das Professor Quirrell in Askaban gegen den Auror ausgefochten hatte, wie ein Hohn aus, wie ein Kinderspiel - obwohl Professor Quirrell damals behauptet hatte, dass der Auror in Sekundenschnelle tot gewesen wäre, wenn er ernsthaft gekämpft hätte; und Harry wusste jetzt, dass das stimmte. \emph{Wie hoch reichte die Machtleiter eigentlich?}

"Ich nehme an", sagte Harry und schaffte es, seine Stimme ruhig zu halten, "dass dein Essen von Einhörnern etwas damit zu tun hat, warum du von der Stelle als Verteidigungsprofessor gefeuert wirst. Ich nehme nicht an, dass du das in aller Ausführlichkeit erklären willst?"

Professor Quirrell sah ihn an. Das fast greifbare Gefühl der Macht in der Luft schien zu schwinden und zog sich in den Verteidigungsprofessor zurück.

"Ich werde mich in der Tat erklären", sagte der Verteidigungsprofessor. "Aber ich muss erst ein paar Gedächtniszauber wirken, und dann können wir gehen und es besprechen, denn es wäre nicht klug, wenn ich hier bliebe. Du wirst später in diese Zeit zurückkehren, wie ich weiß."

Harry wünschte sich, durch den Mantel, den er gemeistert hatte, sehen zu können; und er wusste, dass ein anderer Harry neben ihm stand, verborgen durch sein eigenes Heiligtum des Todes. Dann befahl Harry seinem Umhang, sich noch einmal vor sich selbst zu verbergen, und das tat er auch; sein zukünftiges Ich wahrnehmen zu können, bedeutete, die Erinnerung später abgleichen zu müssen. Harrys eigene Stimme sagte dann, in Gegenwart-Harrys Ohren seltsam klingend: "Er hat eine überraschend gute Erklärung."

Der gegenwärtige Harry erinnerte sich an die Worte, so gut er konnte. Es wurde nichts mehr zwischen ihnen gesagt.

Professor Quirrell ging zu Dracos Gestalt und sprach den Zauberspruch "Obliviate". Der Verteidigungsprofessor stand vielleicht eine Minute lang da, scheinbar für die Welt verloren. Harry hatte in den letzten Wochen Obliviationen studiert - obwohl er nicht bei den Zaubern hätte helfen können, es sei denn, er wäre bereit gewesen, sich fast vollständig zu erschöpfen, und aus irgendeinem Grund wollen, dass ein Auror jede einzelne Lebenserinnerung verliert, die mit der Farbe Blau zu tun hatte. Aber Harry hatte jetzt eine Vorstellung von der Konzentration, die der weitaus schwierigere Falsche-Erinnerungs-Zauber erforderte. Man musste versuchen, das gesamte Leben der anderen Person in seinem eigenen Kopf zu leben, zumindest wenn man die falschen Erinnerungen mit weniger als einer Verlangsamung von sechzehn zu eins erzeugen wollte, während man separat sechzehn große Erinnerungsspuren herstellte. Es mochte still sein, es mochte kein äußeres Zeichen geben; aber Harry wusste jetzt etwas über die Schwierigkeiten, und er wusste, dass er beeindruckt sein musste.

Professor Quirrell war fertig und ging weiter zu Tracey Davis, dann zu den drei Auroren und schließlich zu Professor McGonagall. Harry wartete, aber Zukunfts-Harry erhob keinen Protest. Es war möglich, dass sogar Professor McGonagall, wenn sie wach gewesen wäre, nicht protestiert hätte. Es waren noch nicht Mitte Mai, und offenbar würde es eine überraschend gute Erklärung geben.

Mit einer Geste wurde Dracos betäubter Körper angehoben und ein kurzes Stück in den Wald geschickt, bevor er vorsichtig auf dem Boden abgesetzt wurde. Dann riss Professor Quirrell mit einer letzten Geste ein riesiges Stück aus der Seite des Einhorns und hinterließ ausgefranste Ränder; das rohe Fleisch schwebte in der Luft, schwankte dann im Verschwinden und war weg.

"Erledigt", sagte Professor Quirrell. "Ich muss mich jetzt von diesem Ort entfernen, Mr. Potter. Komm mit mir, und bleib hier."

Professor Quirrell schritt davon, und Harry folgte und blieb zurück. Sie gingen eine Zeit lang schweigend durch den Wald, bevor Harry in der Ferne schwache Stimmen hörte. Vermutlich die nächste Gruppe von Auroren, nachdem die erste Gruppe den Kontakt abgebrochen hatte. Was sein zukünftiges Ich sagte, wusste Harry nicht.

"Sie werden uns nicht entdecken und unsere Sprache nicht hören", sagte Professor Quirrell. Das Gefühl der Macht und des Unheils, das den Verteidigungsprofessor umgab, war immer noch stark. Der Mann setzte sich auf einen Baumstumpf, auf den das Licht des fast vollen Mondes voll auf ihn fiel. "Wenn du in der Zukunft mit den Auroren sprichst, solltest du ihnen sagen, dass du die brodelnde Dunkelheit verscheucht hast, so wie du es mit dem Dementor getan hast. Das ist es, an was sich Mr. Malfoy erinnern wird." Professor Quirrell stieß einen kleinen Seufzer aus. "Es könnte eine gewisse Beunruhigung hervorrufen, wenn sie zu dem Schluss kommen, dass ein Schrecken, der mit den Dementoren verwandt und stark genug ist, um die Schilde der Auroren zu durchbrechen, im Verbotenen Wald frei herumläuft. Aber ich wüsste nicht, was ich sonst tun sollte. Wenn der Wald danach besser bewacht ist - aber mit etwas Glück habe ich schon, was ich brauche. Würdest du mir verraten, wie du so schnell hierher gekommen bist? Woher wusstest du, dass Mr. Malfoy in Schwierigkeiten war?"

Nachdem Kapitän Brodski erfahren hatte, dass Draco Malfoy sich im Verbotenen Wald aufhielt, anscheinend in Begleitung von Rubeus Hagrid, hatte Brodski begonnen, sich zu erkundigen, wer dies autorisiert hatte, und war immer noch nicht in der Lage gewesen, herauszufinden, dann hatte Draco Malfoy den Check-in nicht angenommen. Trotz Harrys Protesten hatte der Aurorenkapitän, der autorisiert war, über Zeitumkehrer Bescheid zu wissen, einen Einsatz vor der Zeit des verpassten Check-ins abgelehnt; es gab Standardverfahren, die mit der Zeit zu tun hatten. Aber Brodski hatte Harry einen schriftlichen Befehl gegeben, der ihm erlaubte, zurückzugehen und ein Aurorentrio einzusetzen, das eine Sekunde nach der verpassten Check-in-Zeit eintreffen sollte. Es hatte einen Patronus-Zauber gegeben, um Draco zu lokalisieren, den Harry erfolgreich dazu gebracht hatte, die Form eines Balls aus reinem, silbernem Licht anzunehmen, und der Flug der Auroren war auf die Sekunde pünktlich angekommen.

"Das kann ich leider nicht sagen", antwortete Harry gleichmütig. Professor Quirrell war immer noch ein Hauptverdächtiger, und es war gut für ihn, die Details nicht zu kennen. "Also, warum isst du Einhörner?"

"Ah", sagte Professor Quirrell. "Was das angeht …?" Der Mann zögerte. "Ich habe das Blut von Einhörnern getrunken, nicht gegessen. Das fehlende Fleisch, die zerlumpten Flecken auf dem Körper - das war, um den Fall zu verschleiern, um es wie ein anderes Raubtier aussehen zu lassen. Die Verwendung von Einhornblut ist allgemein bekannt."

"Ich weiß es nicht", sagte Harry.

"Ich weiß, dass du es nicht weißt", sagte der Verteidigungsprofessor scharf. "Sonst würdest du mich nicht damit belästigen. Die Kraft des Einhornblutes besteht darin, dein Leben eine Zeit lang zu erhalten, selbst wenn du kurz vor dem Tod stehst."

Es gab eine Zeitspanne, in der Harrys Gehirn behauptete, die Worte nicht verarbeiten zu können, was natürlich eine Lüge war, denn man konnte die Bedeutung, die man nicht verarbeiten durfte, nicht kennen, ohne sie bereits verarbeitet zu haben. Ein seltsames Gefühl der Leere überkam Harry, eine Abwesenheit von Reaktion, vielleicht war es das, was andere Leute fühlten, wenn jemand aus der Reihe tanzte und sie nichts sagen konnten oder ihnen nichts einfiel, was sie tun konnten.

\emph{Natürlich lag Professor Quirrell im Sterben, er war nicht nur vorübergehend krank. Professor Quirrell hatte gewusst, dass er im Sterben lag. Schließlich hatte er sich freiwillig für den Posten des Verteidigungsprofessors in Hogwarts gemeldet. Natürlich war es ihm das ganze Schuljahr über immer schlechter gegangen. Natürlich hatten Krankheiten, die immer schlimmer wurden, ein vorhersehbares Ziel am Ende.}

Harrys Gehirn hatte es sicher schon gewusst, irgendwo im sicheren Hinterkopf, wo er sich weigern konnte, Dinge zu verarbeiten, die er schon verarbeitet hatte.

\emph{Das war natürlich der Grund, warum Professor Quirrell nächstes Jahr nicht in der Lage sein würde, Kampfmagie zu unterrichten. Professor McGonagall bräuchte ihn nicht einmal zu feuern. Er würde einfach - - tot sein.}

"Nein", sagte Harry, seine Stimme zitterte ein wenig. "Es muss doch einen Weg geben -"

"Ich bin weder dumm noch besonders erpicht darauf, zu sterben. Ich habe bereits nach Möglichkeiten gesucht. Ich musste nur so weit gehen, weil ich weniger Zeit hatte, als ich gedacht hatte, und -" Der Kopf der dunklen, mondbeschienenen Gestalt wandte sich ab. "Ich glaube, ich will davon nichts mehr erzählen, Mr. Potter."

Harrys Atem stockte. Zu viele Emotionen sprudelten auf einmal in ihm hoch. Nach der Verleugnung kam die unglaubliche Wut. Und doch schien es erstaunlich angemessen zu sein.

"Und warum -" Harrys Atem stockte erneut. "Warum ist Einhornblut dann nicht Standard in Heiler-Kits? \textbf{Um jemanden am Leben zu erhalten, auch wenn er kurz davor ist, daran zu sterben, dass seine Beine gefressen werden?}"

"Weil es dauerhafte Nebenwirkungen hat", sagte Professor Quirrell leise.

"Nebenwirkungen? Nebenwirkungen?! Welche Art von Nebenwirkung ist medizinisch gesehen schlimmer als der \textbf{TOD?!}"

Harrys Stimme erhob sich bei dem letzten Wort, bis er schrie.

"Nicht jeder denkt so wie wir, Mr. Potter. Obwohl, um fair zu sein, das Blut muss von einem lebenden Einhorn stammen und das Einhorn muss beim Trinken sterben. Wäre ich sonst hier?"

Harry drehte sich um und starrte auf die umliegenden Bäume. "Halte eine Herde Einhörner in St. Mungos. Appariere die Patienten dorthin, oder benutze Portschlüssel."

"Ja, das würde funktionieren."

Harrys Gesicht straffte sich, das einzige äußere Zeichen hinter seinen zitternden Händen für all das, was in ihm hochkochte. Er musste schreien, brauchte ein Ventil, brauchte etwas, das er nicht benennen konnte, und schließlich richtete Harry seinen Zauberstab auf einen Baum und rief: "Diffindo!"

Es gab ein scharfes, reißendes Geräusch, und ein Schnitt erschien quer durch das Holz.

"Diffindo!" Noch ein Schnitt.

Harry hatte diesen Zauber erst zehn Tage zuvor gelernt, nachdem er angefangen hatte, sich ernsthaft mit Selbstverteidigung zu beschäftigen. Theoretisch war es ein Zauber aus dem zweiten Jahr, aber die Wut, die ihn durchströmte, schien keine Grenzen zu kennen, er wusste jetzt genug, um sich nicht zu erschöpfen, und er hatte noch immer Kraft.

"Diffindo!" Harry hatte diesmal einen Ast anvisiert, der mit einem Geräusch von Zweigen und Blättern zu Boden stürzte. Es schien keine Tränen in ihm zu geben, nur Druck, der kein Ventil fand.

"Ich werde dich in Ruhe lassen", sagte Professor Quirrell leise. Der Verteidigungsprofessor erhob sich von seinem Baumstumpf, das Blut des Einhorns schimmerte noch immer auf dem schwarzen Umhang, den er trug, und zog sich die Kapuze wieder über den Kopf.

