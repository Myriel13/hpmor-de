

\hypertarget{teil-1}{% \section{51. -\/-\/-\/-\/- Teil 1}\label{teil-1}}

\textbf{\uline{-\/-\/-\/-\/- Teil 1}}

\textbf{Samstag}.

Harry hatte am Freitagabend Probleme mit dem Einschlafen gehabt, was er vorausgesehen hatte, und so hatte er sich entschlossen, die offensichtliche Vorsichtsmaßnahme zu ergreifen und einen Schlaftrunk zu kaufen; und um zu verhindern, dass es ein sichtbares Zeichen dafür war, dass er nervös war, hatte er beschlossen, ihn ein paar Monate zuvor bei Fred und George zu kaufen.

(\emph{Sei bereit, das ist das Marschlied der Pfadfinder…})

So war Harry völlig ausgeruht, und sein Beutel enthielt fast alles, was er besaß und möglicherweise brauchen könnte. Harry war in der Tat an die Volumenbegrenzung des Beutels gestoßen; und da er bedachte, dass er eine große Schlange und wer-weiß-was-noch alles unterbringen musste, hatte er einige der sperrigeren Gegenstände, wie die Autobatterie, entfernt. Er war jetzt so weit, dass er etwas von der Größe einer Autobatterie in vier Minuten verwandeln konnte, also war das kein großer Verlust.

Harry hatte die Notfallfackeln und den Autogen-Schweißbrenner samt Brennstofftank behalten, da man Dinge, die verbrannt werden sollten, nicht einfach verwandeln durfte. (\emph{Sei vorbereitet, wie du durchs Leben marschierst…})

\textbf{Mary's Restaurant.}

Nachdem die Kellnerin ihre Bestellung aufgenommen, sich vor ihnen verbeugt und den Raum verlassen hatte, hatte Professor Quirrell nur vier Zauber durchgeführt, und dann hatten sie über nichts von großer Tragweite geredet, nur über Professor Quirrells komplexe These darüber, wie der Fluch des Dunklen Lords auf die Verteidigungsposition zum Niedergang des Duellierens geführt hatte und wie dies die gesellschaftlichen Sitten im magischen Britannien verändert hatte.

Harry hörte zu und nickte und sagte intelligente Dinge, während er versuchte, das Klopfen seines Herzens zu kontrollieren. Dann kam die Kellnerin wieder mit dem Essen herein, und diesmal, eine Minute nachdem die Kellnerin gegangen war, deutete Professor Quirrell mit einer Geste an, die Tür zu schließen und zu verriegeln, und begann, neunundzwanzig Sicherheitszauber zu sprechen, wobei einer der Zauber aus Mr. Besters Sequenz diesmal ausgelassen wurde, was Harry etwas verwirrte.

Professor Quirrell beendete seinen Zauberspruch - - stand von seinem Stuhl auf - - verwandelte sich in eine grüne Schlange mit blauen und weißen Bändern - - zischte:

"\textbf{\emph{Hungrig, Junge? Iss dich schnell satt, wir brauchen sowohl sStärke als auch Zeit.}}"

Harrys Augen wurden ein wenig groß, aber er zischte:

"\emph{Ich habe gut gefrühstückt}", und begann dann schnell, Nudeln in seinen Mund zu schieben.

Die Schlange beobachtete ihn einen Moment lang mit diesen flachen Augen und zischte dann: "\textbf{\emph{Will nicht, hier erklären. Möchte lieber ersssst einmal woanders sssein und den Raum verlassen, ohne ein Zeichen, dass wir ihn verlassen haben.}}"

"\emph{Damit uns niemand verfolgen kann}", zischte Harry.

"\textbf{\emph{Jawohl. Traust du mir so sehr, Junge? Denk nach, bevor du antwortesssst. Ich habe eine wichtige Anforderung an dich, die Vertrauen erfordert; wenn nicht soooo sssag jetzt}}."

Harry löste seinen Blick von den flachen Augen der Schlange und sah wieder auf seine mit Soße überzogenen Nudeln hinunter, und aß noch einen Bissen, dann noch einen, während er nachdachte.

Der Verteidigungsprofessor… war eine zwiespältige Gestalt, um es gelinde auszudrücken; Harry glaubte, einige seiner Ziele enträtselt zu haben, aber andere blieben rätselhaft. Aber Professor Quirrell hatte zweihundert Mädchen niedergeschlagen, um die zu stoppen, die Harry hinuntergezogen hatten.

Professor Quirrell hatte herausgefunden, dass der Dementor Harry durch seinen Zauberstab aussaugte.

Der Verteidigungsprofessor hatte Harrys Leben gerettet, zweimal, innerhalb von zwei Wochen. Was bedeuten könnte, dass der Verteidigungsprofessor Harry nur für später gerettet hatte, dass es Hintergedanken gab.

In der Tat war es sicher, dass es Hintergedanken gab. Professor Quirrell hat das nicht aus einer Laune heraus getan.

Aber dann hatte Professor Quirrell auch gefordert, dass Harry in Okklumentik unterrichtet wurde, er hatte Harry gelehrt, wie man verliert….

wenn der Verteidigungsprofessor Harry Potter benutzen wollte, war es eine Benutzung, die einen gestärkten Harry Potter erforderte, nicht einen geschwächten.

Das war es, was es bedeutete, von einem Freund benutzt zu werden, dass er wollte, dass man durch die Benutzung stärker und nicht schwächer wurde.

Und wenn es manchmal eine kalte Atmosphäre um den Verteidigungsprofessor gab, Bitterkeit in seiner Stimme oder Leere in seinem Blick, dann war Harry der Einzige, dem Professor Quirrell erlaubte, es zu sehen.

Harry wusste nicht recht, wie er das Gefühl der Verwandtschaft, das er mit Professor Quirrell empfand, in Worte fassen sollte, außer zu sagen, dass der Verteidigungsprofessor der einzige klar denkende Mensch war, den Harry in der Welt der Zauberer getroffen hatte.

Früher oder später fingen alle anderen an, Quidditch zu spielen, ihre Zeitmaschinen nicht mit Schutzhüllen zu versehen oder zu glauben, der Tod sei ihr Freund.

Es spielte keine Rolle, wie gut ihre Absichten waren. Früher oder später, und meistens früher, zeigten sie, dass etwas tief in ihrem Gehirn kaputt war.

Alle außer Professor Quirrell. Es war ein Band, das über alles hinausging, was sie sich schuldig waren, oder sogar über alles, was sie persönlich mochten, dass sie beide allein in der Welt der Zauberer waren. Und wenn der Verteidigungsprofessor gelegentlich ein wenig furchteinflößend oder ein wenig dunkel wirkte, nun, das war genau das, was manche Leute auch über Harry sagten.

"\emph{Ich vertraue dir}", zischte Harry.

Und die Schlange erklärte die erste Stufe des Plans. Harry nahm eine letzte Gabel voll Nudeln und kaute. Neben ihm aß Professor Quirrell, nun wieder in menschlicher Gestalt, seelenruhig seine Suppe, als ob nichts von besonderem Interesse vor sich ginge. Dann schluckte Harry und stand im selben Moment von seinem Stuhl auf, wobei er bereits spürte, wie sein Herz in seiner Brust zu hämmern begann.

Die Sicherheitsvorkehrungen, die sie getroffen hatten, waren buchstäblich die strengsten, die möglich waren…

"Sind Sie bereit, es zu testen, Mr. Potter?" sagte Professor Quirrell ruhig. Es war kein Test, aber das würde Professor Quirrell nicht sagen, nicht laut in menschlicher Sprache, auch nicht in diesem bis an die Grenze abgeschirmten Raum, den Professor Quirrell mit weiteren Zaubern gesichert hatte.

"Ja", sagte Harry so beiläufig, wie er konnte.

\textbf{Schritt eins}.

Harry sagte "Umhang"

zu seinem Beutel, zog den Unsichtbarkeitsumhang hervor, löste dann den Beutel von seinem Gürtel und warf ihn auf die andere Seite des Tisches.

Der Verteidigungsprofessor stand von seinem Platz auf, zog seinen Zauberstab, beugte sich hinunter, berührte den Beutel mit seinem Zauberstab und murmelte eine leise Beschwörungsformel.

Die neuen Zaubersprüche würden dafür sorgen, dass Professor Quirrell den Beutel allein in Schlangenform betreten und wieder verlassen konnte und dass er hören konnte, was draußen vor sich ging, während er im Beutel war.

\textbf{Schritt 2.}

Als Professor Quirrell sich von der Stelle erhob, an der er sich über den Beutel gebeugt hatte, und seinen Zauberstab wegsteckte, zeigte sein Zauberstab zufällig in Harrys Richtung, und es gab ein kurzes krabbelndes Gefühl auf Harrys Brust in der Nähe der Stelle, an der der Zeitdreher lag, als ob etwas Unheimliches ganz nah an ihm vorbeigegangen wäre, ohne ihn zu berühren.

\textbf{Schritt drei.}

Der Verteidigungsprofessor verwandelte sich wieder in eine Schlange, und das Gefühl des Unheils ließ nach; die Schlange kroch zu dem Beutel und in ihn hinein, das Maul des Beutels öffnete sich, um die grüne Gestalt aufzunehmen, und als sich das Maul hinter dem Schwanz wieder schloss, ließ das Gefühl des Unheils weiter nach.

\textbf{Schritt vier.}

Harry zog seinen Zauberstab und achtete darauf, dabei still zu stehen, damit sich der Zeitdreher nicht von der Stelle bewegte, an der Professor Quirrell die Sanduhr in der Schale in ihrer derzeitigen Ausrichtung verankert hatte. "Wingardium Leviosa", murmelte Harry, und der Beutel begann auf ihn zuzuschweben. Langsam, ganz langsam, wie Professor Quirrell es angeordnet hatte, begann der Beutel auf Harry zuzuschweben, der wachsam auf jedes Zeichen wartete, dass sich der Beutel öffnete, in diesem Fall sollte Harry den Schwebezauber benutzen, um ihn so schnell wie möglich von sich wegzuschieben. Als der Beutel bis auf einen Meter an Harry herankam, kehrte das Gefühl des Unheils zurück. Als Harry den Beutel wieder an seinem Gürtel befestigte, war das Gefühl des Unheils stärker als je zuvor, aber immer noch nicht überwältigend; es war erträglich. Sogar mit Professor Quirrells Animagus-Gestalt, die im erweiterten Raum des Beutels lag, der genau auf Harrys Hüfte ruhte.

\textbf{Schritt fünf.}

Harry steckte seinen Zauberstab in die Tasche. Seine andere Hand hielt immer noch den Umhang der Unsichtbarkeit und Harry zog diesen Umhang über sich.

\textbf{Schritt 6.}

Und so griff Harry in diesem gegen jede Art von Spionage abgeschirmten Raum, den Professor Quirrell persönlich noch weiter abgesichert hatte, erst, als er den echten Tarnumhang trug, unter sein Hemd und drehte einmal an der Außenhülle des Zeitdrehers. Die innere Sanduhr blieb verankert und unbeweglich, die Fassung drehte sich um sie herum - Das Essen verschwand vom Tisch, die Stühle sprangen zurück an ihren Platz, die Tür sprang auf.

Marys Zimmer war menschenleer, so wie es hätte sein sollen, denn Professor Quirrell hatte sich zuvor unter falschem Namen bei Mary gemeldet, um sich zu erkundigen, ob das Zimmer um diese Zeit noch frei sei - nicht um es zu reservieren, nicht um eine stornierte Reservierung zu vermerken, sondern nur um sich zu erkundigen.

\textbf{Schritt sieben.}

Unter dem Unsichtbarkeitsmantel bleibend, ging Harry durch die offene Tür. Er navigierte durch die gefliesten Gänge von Mary's Place zu der gut bestückten

Bar, die Neuankömmlinge begrüßte und von dem Besitzer Jake betreut wurde.

Es waren nur wenige Leute an der Bar, am Morgen vor der eigentlichen Mittagszeit, und Harry musste einige Minuten unsichtbar an der Tür warten, um dem Gemurmel der Gespräche und dem Glucksen des Alkohols zu lauschen, bevor sich die Tür öffnete, um einen großen, freundlichen Iren einzulassen, und Harry schlüpfte lautlos in seinem Gefolge hinaus.

\textbf{Schritt Acht.}

Harry ging eine Weile zu Fuß. Er war weit weg von Marys Place, als er von der Diagon Alley in eine kleinere Gasse abbog, an deren Ende ein Laden lag, der dunkel war, die Fenster in Schwärze verzaubert.

\textbf{Schritt neun.}

"Schwertfisch-Melonenfreund", sprach Harry das Passwort zum Schloss, und es klickte auf. Im Inneren des Ladens herrschte ebenfalls Dunkelheit, das Licht der offenen Tür erhellte ihn kurz und zeigte einen weiten, leeren Raum.

Der Möbelladen, der hier einmal betrieben worden war, war vor ein paar Monaten bankrott gegangen, so der Verteidigungsprofessor, und der Laden war zwar wieder in Besitz genommen, aber noch nicht wieder verkauft worden. Die Wände waren in einem schlichten Weiß gestrichen, der Holzboden zerkratzt und unpoliert, eine einzelne geschlossene Tür in der Rückwand; dies war einmal ein Ausstellungsraum gewesen, aber jetzt zeigte er nichts mehr. Die Tür klappte hinter Harry zu, und dann war es stockfinster.

\textbf{Schritt zehn.}

Harry zückte seinen Zauberstab und sagte "Lumos", was den Raum in weißes Licht tauchte; er nahm seinen Beutel von seinem Gürtel

(das Gefühl des Unheils wurde ein wenig schärfer, als er ihn mit den Fingern umfasste)

und warf ihn leicht auf die gegenüberliegende Seite des Raumes

(das Gefühl des Unheils verblasste fast vollständig).

Und dann begann er, den Unsichtbarkeitsumhang abzunehmen, während seine Stimme zischte: "Es ist vollbracht."

\textbf{Schritt elf.}

Aus dem Beutel stach ein grüner Kopf, kurz darauf folgte ein meterlanger grüner Körper, aus dem die Schlange herausschlüpfte. Einen Moment später verschwamm die Schlange zu Professor Quirrell.

\textbf{Schritt zwölf.}

Harry wartete schweigend, während der Verteidigungsprofessor dreißig Zaubersprüche aufsagte.

"In Ordnung", sagte Professor Quirrell ruhig, als er geendet hatte.

"Wenn uns jetzt noch jemand beobachtet, sind wir auf jeden Fall verloren, also werde ich Klartext reden und zwar in Menschengestalt.

Parsel passt nicht ganz zu mir, fürchte ich, denn ich bin weder ein Nachkomme von Salazar noch eine echte Schlange."

Harry nickte.

"Also, Mr. Potter", sagte Professor Quirrell. Sein Blick war abwartend, seine blassblauen Augen dunkel und schemenhaft im weißen Licht, das von Harrys Zauberstab ausging.

"Wir sind allein und unbeobachtet, und ich muss dir eine wichtige Frage stellen."

"Nur zu", sagte Harry, und sein Herz begann schneller zu schlagen.

"Was hältst du von der Regierung des magischen Britanniens?"

Das war nicht ganz das, was Harry erwartet hatte, aber es war nahe genug, also sagte Harry: "Basierend auf meinem begrenzten Wissen würde ich sagen, dass sowohl das Ministerium als auch das Zaubergamot dumm, korrupt und böse zu sein scheinen."

"Richtig", sagte Professor Quirrell. "Verstehst du, warum ich frage?"

Harry holte tief Luft und schaute Professor Quirrell direkt in die Augen, ohne mit der Wimper zu zucken. Harry hatte schließlich herausgefunden, dass man aus spärlichen Beweisen erstaunliche Schlüsse ziehen konnte, wenn man die Antwort im Voraus kannte, und er hatte diese Antwort schon vor einer ganzen Woche erraten.

Es brauchte nur noch eine kleine Anpassung…

"Sie sind dabei, mich einzuladen, einer geheimen Organisation beizutreten, die aus interessanten Leuten wie Ihnen besteht", sagte Harry, "eines ihrer Ziele ist es, die Regierung des magischen Britanniens zu reformieren oder zu stürzen, und ja, ich bin dabei."

Es gab eine kleine Pause.

"Ich fürchte, das ist nicht ganz das, worauf ich dieses Gespräch lenken wollte", sagte Professor Quirrell.

Die Winkel seiner Lippen zuckten leicht.

"Ich hatte lediglich vor, dich um Hilfe zu bitten, etwas äußerst Verräterisches und Illegales zu tun."

\emph{Verflixt}, dachte Harry. \emph{Immerhin hatte Professor Quirrell es nicht geleugnet…}

"Fahren Sie fort."

"Bevor ich es tue", sagte Professor Quirrell.

In seiner Stimme lag jetzt keine Leichtfertigkeit mehr.

"Bist du offen für solche Bitten, Mr. Potter? Ich wiederhole: Wenn du trotzdem nein sagen willst, musst du jetzt nein sagen. Wenn deine Neugierde dich zu etwas anderem drängt, dann unterdrücken sie."

"Verrat und Illegalität stören mich nicht", sagte Harry.

"Risiken stören mich, und der Einsatz müsste angemessen sein, aber ich kann mir nicht vorstellen, dass Sie leichtfertig Risiken eingehen."

Professor Quirrell nickte.

"Das würde ich nicht. Es wäre ein schrecklicher Missbrauch meiner Freundschaft zu dir und des Vertrauens, das in meine Lehrtätigkeit in Hogwarts gesetzt wird -"

"Diesen Teil können Sie auslassen", sagte Harry.

Die Lippen zuckten wieder und wurden dann flach.

"Dann werde ich ihn überspringen. Mr. Potter, Du machst manchmal ein Spiel aus Lügen und Wahrheiten, spielen mit Worten, um Ihre Bedeutungen zu verbergen, wenn sie offensichtlich sind. Auch ich bin dafür bekannt, dass ich das amüsant finde. Aber wenn ich dir sage, was ich hoffe, dass wir heute tun werden, Mr. Potter, wirst du lügen. Du wirst geradeheraus lügen, ohne Zögern, ohne Wortspiele oder Andeutungen, gegenüber jedem, der danach fragt, ob Feind oder bester Freund.

Du wirst Malfoy anlügen, Granger und McGonagall. Du wirst immer und ohne zu zögern so sprechen, wie du sprechen würdest, wenn du nichts wüsstest, ohne Rücksicht auf deine Ehre. So muss es sein."

Dann herrschte eine Zeit lang Schweigen. Das war ein Preis, der in einem Bruchteil von Harrys Seele gemessen wurde.

"Ohne es mir jetzt zu sagen …", sagte Harry. "Können Sie sagen, ob die Notlage verzweifelt ist?"

"Es gibt jemanden, der deine Hilfe am dringendsten braucht", sagte Professor Quirrell schlicht, "und es gibt niemanden, der ihm helfen kann, außer dir."

Wieder herrschte Schweigen, aber nicht lange.

"In Ordnung", sagte Harry leise. "Erzählen Sie mir von der Mission."

Die dunklen Roben des Verteidigungsprofessors schienen mit dem Schatten an der Wand zu verschwimmen, den seine Silhouette warf, die das weiße Licht von Harrys Zauberstab blockierte.

"Der gewöhnliche Patronus-Zauber, Mr. Potter, wehrt die Angst der Dementoren ab. Aber die Dementoren sehen trotzdem durch ihn hindurch, sie wissen, dass man da sind. Nur nicht dein Patronus-Zauber. Er blendet sie, oder mehr als blenden. Was ich unter dem Umhang gesehen habe, hat nicht einmal in unsere Richtung geschaut, als du es getötet hast; als hätte es unsere Existenz vergessen, selbst als es starb."

Harry nickte. Das war nicht überraschend, nicht, wenn man einem Dementor auf der Ebene seiner wahren Existenz gegenüberstand, jenseits des Anthropomorphismus.

Der Tod mochte der letzte Feind sein, aber er war kein fühlender Feind. Als die Menschheit die Pocken ausgerottet hatte, hatten die Pocken nicht zurückgeschlagen.

"Mr. Potter, die zentrale Zweigstelle von Gringotts wird von jedem hohen und niedrigen Zauber bewacht, den die Kobolde kennen. Trotzdem sind diese Tresore erfolgreich ausgeraubt worden; denn was die Zauberei vermag, kann die Zauberei wieder rückgängig machen. Und dennoch ist noch nie jemand aus Askaban entkommen. Keiner. Für jeden Zauber gibt es einen Gegenzauber, für jeden Schutzwall eine Umgehung. Wie kann es sein, dass noch nie jemand aus Askaban befreit wurde?"

"Weil Askaban etwas Unbesiegbares hat", sagte Harry. "Etwas, das so schrecklich ist, dass niemand es besiegen kann."

Das war der Grundpfeiler ihrer perfekten Sicherheit, das musste es sein, nichts Menschliches. Es war der Tod, der Askaban bewachte.

"Die Dementoren mögen es nicht, wenn man ihnen ihre Mahlzeiten wegnimmt", sagte Professor Quirrell.

Jetzt war Kälte in seine Stimme gekommen.

"Sie wissen, wenn es jemand versucht. Es sind mehr als hundert Dementoren dort, und sie sprechen auch mit den Wächtern. So einfach ist das, Mr. Potter. Wenn man ein mächtiger Zauberer ist, ist es nicht schwer, Askaban zu betreten, und es ist nicht schwer, es zu verlassen. Solange man nicht versucht, etwas zu entwenden, das den Dementoren gehört."

"Aber die Dementoren sind nicht unbesiegbar", sagte Harry.

Mit diesem Gedanken hätte er den Patronus-Zauber wirken können, in diesem Moment. "Ich habe das niemals geglaubt."

Professor Quirrells Stimme war sehr leise.

"Erinnerst du dich daran, wie es war, als du das erste Mal vor dem Dementor standest und versagt hast?"

"Ich erinnere mich."

Und dann, mit einem plötzlichen Übelkeitsgefühl im Magen, wusste Harry, worauf das hinauslief; er hätte es vorher sehen müssen.

"Es gibt eine unschuldige Person in Askaban", sagte Professor Quirrell. Harry nickte, es brannte in seiner Kehle, aber er weinte nicht.

"Derjenige, von dem ich spreche, stand nicht unter dem Imperius-Fluch", sagte der Verteidigungsprofessor, dessen dunkle Robe sich von einem größeren Schatten abhob. "Es gibt sicherere Wege, den Willen zu brechen als den Imperius, wenn man die Zeit für Folter und Legilimenz und Rituale hat, von denen ich nicht sprechen will.

Ich kann dir nicht sagen, woher ich das weiß, ich kann es dir nicht einmal andeuten, du wirst mir vertrauen müssen. Aber es gibt eine Person in Askaban, die sich nicht ein einziges Mal dafür entschieden hat, dem Dunklen Lord zu dienen, die Jahre damit verbracht hat, allein in der schrecklichsten Kälte und Dunkelheit zu leiden, die man sich vorstellen kann, und die keine einzige Minute davon verdient hat."

Harry sah es in einem einzigen Anflug von Intuition, sein Mund raste seinen Gedanken fast voraus. Es gab keine Andeutung, keine Warnung, er sprach was er dachte -

"Eine Person mit dem Namen Black", sagte Harry.

Es herrschte Stille. Stille, während die blassblauen Augen ihn anstarrten.

"Nun", sagte Professor Quirrell nach einer Weile.

"So viel dazu, dir den Namen erst zu sagen, nachdem du den Auftrag angenommen hast. Ich würde dich fragen, ob du meine Gedanken liest , aber das ist schlichtweg unmöglich."

Harry sagte nichts, aber es war einfach genug, wenn man an die Prozesse der modernen Demokratie glaubte. Die offensichtlichste Person in Askaban, die unschuldig war, war die, die keinen Prozess bekommen hatte -

"Ich bin wirklich beeindruckt, Mr. Potter", sagte Professor Quirrell. Sein Gesicht war ernst. "Aber dies ist eine ernste Angelegenheit, und wenn es eine Möglichkeit gibt, dass andere dieselbe Schlussfolgerung ziehen könnten, muss ich das wissen.

Also sage mir… Wie, im Namen von Merlin, von Atlantis und der Leere zwischen den Sternen, hast du erraten, dass ich von \textbf{\emph{Bellatrix}} gesprochen habe?"

