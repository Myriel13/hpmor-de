

\hypertarget{nuxfctzliche-priorituxe4ten}{% \section{48. Nützliche Prioritäten}\label{nuxfctzliche-priorituxe4ten}}

\textbf{\uline{Nützliche Prioritäten}}

Es war Samstag, der erste Morgen im Februar, und am Ravenclaw-Tisch stand ein Junge mit einem Frühstücksteller voller Gemüse und untersuchte nervös seine Portionen auf die kleinste Spur von Fleisch.

Vielleicht war es eine Überreaktion. Nachdem er den ersten Schock überwunden hatte, meldete sich Harrys gesunder Menschenverstand und stellte die Hypothese auf, dass es sich bei der "Parselzunge" wahrscheinlich nur um eine sprachliche Benutzeroberfläche zur Steuerung von Schlangen handelte…

\emph{… schließlich konnten Schlangen nicht wirklich auf menschlichem Niveau intelligent sein, das hätte schon jemand gemerkt.}

Die kleinsthirnigen Kreaturen, von denen Harry je gehört hatte, mit so etwas wie sprachlichen Fähigkeiten, waren die afrikanischen Graupapageien, die von Irene Pepperberg unterrichtet wurden. Und das war unstrukturierte Protosprache, bei einer Spezies, die komplexe Ehebruchsspiele spielte und andere Papageien nachahmen musste. Nach dem, woran sich Draco erinnern konnte, sprachen die Schlangen mit den Parselmündern in einer Sprache, die sich wie eine normale menschliche Sprache anhörte - d.h. mit einer ausgewachsenen rekursiven syntaktischen Grammatik. Das hatte bei Hominiden eine gewisse Zeit gedauert, um sich zu entwickeln, mit riesigen Gehirnen und starkem sozialen Selektionsdruck. Schlangen hatten überhaupt nicht viel Kultur oder Gesellschaften, von der Harry je gehört hatte. Und bei Tausenden und Abertausenden von Schlangenarten auf der ganzen Welt, wie konnten sie da alle dieselbe Version ihrer angeblichen Sprache, "Parsel", benutzen? Natürlich war das alles nur gesunder Menschenverstand, an den Harry allmählich den Glauben verlor. Aber Harry war sich sicher, dass er irgendwann einmal im Fernsehen das Zischen von Schlangen gehört hatte - schließlich wusste er von irgendwoher, wie sich das anhörte - und das hatte sich für ihn nicht wie eine Sprache angehört, was ihm um einiges beruhigender erschienen war… … anfangs.

Das Problem war, dass Draco auch behauptet hatte, dass Parselmünder Schlangen auf ausgedehnte, komplexe Missionen schicken konnten. Und wenn das stimmte, dann mussten die Parselmünder die Schlangen dauerhaft intelligent machen, indem sie mit ihnen sprachen. Im schlimmsten Fall würde das die Schlange sich Ihrer selbst bewusst machen, wie das, was Harry versehentlich mit dem Sprechenden Hut gemacht hatte. Und als Harry diese Hypothese vorgebracht hatte, hatte Draco behauptet, er könne sich an eine Geschichte erinnern - \emph{Harry hoffte bei Cthulhu, dass diese eine Geschichte nur ein Märchen war, sie hatte diesen Klang, aber es gab eine Geschichte -} über Salazar Slytherin, der eine tapfere junge Viper auf eine Mission schickte, um Informationen von anderen Schlangen zu sammeln. Wenn eine Schlange, mit der ein Parselmund gesprochen hatte, andere Schlangen dazu bringen konnte sich Ihrer selbst bewusst zu werden, indem sie mit ihnen sprach, \emph{dann…} \emph{dann…}

Harry wusste nicht einmal, warum er "\emph{dann… dann…}" dachte, wo er doch genau wusste, wie die exponentielle Entwicklung ablief, es war nur der schiere moralische Horror, der ihn umhaute. Und was, wenn jemand so einen Zauber erfunden hätte, um mit Kühen zu reden? Was, wenn es Geflügelmäuler gäbe? \emph{Oder was das betrifft…}

Harry erstarrte in plötzlicher Erkenntnis, gerade als die Gabel voll Karotten in seinen Mund wandern wollte.

\emph{Das konnte, konnte unmöglich wahr sein, kein Zauberer wäre so dumm, DAS zu tun…}

Und Harry wusste mit einem furchtbar mulmigen Gefühl, dass sie natürlich so dumm sein würden. Salazar Slytherin hatte wahrscheinlich nie auch nur eine Sekunde lang über die moralischen Implikationen der Schlangenintelligenz nachgedacht, so wie es Salazar auch nie in den Sinn gekommen war, dass Muggelgeborene intelligent genug waren, um Persönlichkeitsrechte zu verdienen. Die meisten Leute sahen moralische Probleme einfach nicht, es sei denn, jemand anderes wies sie darauf hin…

"Harry?", sagte Terry von nebenan und klang, als hätte er Angst, dass er die Frage bereuen würde. "Warum starrst du so auf deine Gabel?"

"Ich glaube langsam, dass Magie illegal sein sollte", sagte Harry. "Übrigens, hast du jemals Geschichten über Zauberer gehört, die mit Pflanzen sprechen konnten?"

Terry hatte noch nie von so etwas gehört. Genauso wenig wie alle Ravenclaws im siebten Jahr, die Harry gefragt hatte. Und nun war Harry zu seinem Platz zurückgekehrt, hatte sich aber noch nicht wieder hingesetzt, sondern starrte mit ausdruckslosem Blick auf seinen Gemüseteller. Er wurde immer hungriger, und später am Tag würde er Mary's Restaurant für eines ihrer unglaublich leckeren Gerichte besuchen… Harry geriet in die Versuchung, einfach wieder zu den gestrigen Essgewohnheiten zurückzukehren und es hinter sich zu lassen.

\emph{Irgendwas musst du doch essen,} sagte sein innerer Slytherin.\\ \emph{Und es ist nicht viel wahrscheinlicher, dass jemand Selbstbewusstsein auf Geflügel geniest hat als auf Pflanzen, also solange man so oder so Essen von fragwürdiger Empfindungsfähigkeit zu sich nimmt, warum nicht die leckeren frittierten Diracawl-Scheiben essen?}

\textbf{\emph{Ich bin mir nicht ganz sicher, ob das eine gültige utilitaristische Logik ist}} -

\emph{Oh, du willst utilitaristische Logik? Eine Portion utilitaristische Logik, kommt sofort: Selbst für den unwahrscheinlichen Fall, dass irgendein Schwachkopf es geschafft hat, Hühnern Empfindungsvermögen zu verleihen, ist es deine Forschung, die die beste Chance hat, diese Tatsache zu entdecken und etwas dagegen zu unternehmen. Wenn du deine Arbeit auch nur ein bisschen schneller abschließen kannst, indem du nicht mit deiner Diät herumspielst, dann ist es, so kontraintuitiv es auch erscheinen mag, das Beste, was du tun kannst, um die größte Anzahl von möglicherweise empfindungsfähigen Wer-weiß-was zu retten, keine Zeit mit wilden Vermutungen darüber zu verschwenden, was intelligent sein könnte. Es ist ja nicht so, dass die Hauselfen das Essen nicht schon vorbereitet hätten, egal, was man sich auf den Teller holt.}

Harry dachte einen Moment lang darüber nach. Es war eine ziemlich verführerische Argumentation -

\emph{Gut!} sagte Slytherin. \emph{Ich bin froh, dass du jetzt einsiehst, dass es das Moralischste ist, das Leben empfindungsfähiger Wesen für deine eigene Bequemlichkeit zu opfern, um deinen furchtbaren Appetit zu stillen, für das kranke Vergnügen, sie mit deinen Zähnen zu zerreißen} -

\textbf{\emph{Was?}} dachte Harry entrüstet. \textbf{\emph{Auf welcher Seite stehst du hier?}}

Die mentale Stimme seines inneren Slytherins war grimmig.\\ \emph{Auch du wirst eines Tages der Doktrin anhängen… dass der Zweck das Fleisch rechtfertigt.}\\ Darauf folgte ein mentales Kichern.

Seit Harry sich Gedanken darüber gemacht hatte, dass auch Pflanzen empfindungsfähig sein könnten, hatten seine Nicht-Ravenclaw-Komponenten Schwierigkeiten, seine moralische Vorsicht ernst zu nehmen.\\ Hufflepuff schrie jedes Mal "\emph{Kannibalismus}!", wenn Harry versuchte, an irgendein Nahrungsmittel zu denken, und Gryffindor stellte es sich schreiend vor, während er es aß, selbst wenn es, sagen wir, ein Sandwich war -

\emph{Kannibalismus}!\\ \emph{AHHHH; BITTE ESS MICH NICHT!-}\\ \emph{Ignoriere die Schreie, iss es trotzdem! Es ist ein sicherer Ort, um deine Ethik im Dienste höherer Ziele zu kompromittieren, alle anderen finden es in Ordnung, Sandwiches zu essen, also kannst du nicht deine übliche Rationalisierung über die geringe Wahrscheinlichkeit eines großen Nachteils verwenden, wenn du erwischt wirst -}

Harry stieß einen mentalen Seufzer aus und dachte:\\ \textbf{\emph{Solange es für dich in Ordnung ist, dass wir von riesigen Monstern gefressen werden, die nicht genug nachgeforscht haben, ob wir empfindungsfähig sind.}}

\emph{Ich habe kein Problem damit,} sagte der Slytherin. \emph{Ist jeder andere damit einverstanden?} (Internes mentales Nicken.) \emph{Toll, können wir jetzt wieder zu frittierten Diracawl-Scheiben übergehen?}

\textbf{\emph{Nicht, bis ich mehr darüber recherchiert habe, was empfindungsfähig ist und was nicht. Und jetzt halt die Klappe}}.

Und Harry wandte sich entschlossen von seinem Teller mit dem ach so verlockenden Gemüse ab, um in Richtung Bibliothek zu gehen -

\emph{Iss einfach die Schüler,} sagte Hufflepuff.\\ \emph{Es gibt keinen Zweifel daran, dass sie empfindungsfähig sind.}

\emph{Du weißt, dass du es willst,} sagte Gryffindor.\\ \emph{Ich wette, die jungen sind am leckersten.}

Harry begann sich zu fragen, ob der Dementor irgendwie seine imaginären Persönlichkeiten beschädigt hatte.

…\\ "Ehrlich gesagt", sagte Hermine. Die Stimme des jungen Mädchens klang ein wenig säuerlich, während ihr Blick die Bücherregale der Kräuterkunde in der Bibliothek von Hogwarts abtastete. Harry hatte ihr eine Nachricht hinterlassen, in der er sie fragte, ob sie nach dem Frühstück, das Harry übersprungen hatte, in die Bibliothek kommen könnte; aber als Harry dann das Thema des Tages vorstellte, wirkte sie ein wenig verblüfft.\\ "Weißt du, was dein Problem ist, Harry? Du hast keinen Sinn für Prioritäten. Du hast keinen Sinn für Prioritäten. Wenn du eine Idee hast, rennst du ihr sofort hinterher."

"Ich habe ein großartiges Gespür für Prioritäten", sagte Harry. Seine Hand griff nach "Cleveres Gemüse" von Casey McNamara und begann, die ersten Seiten durchzublättern, um das Inhaltsverzeichnis zu finden. "Deshalb will ich herausfinden, ob Pflanzen sprechen können, bevor ich meine Karotten esse."

"Meinst du nicht, dass wir beide vielleicht wichtigere Dinge zu erledigen haben?"

\emph{Du klingst genau wie Draco,} dachte Harry, sagte es aber natürlich nicht laut.\\ Laut sagte er: "Was könnte wohl wichtiger sein, als dass sich Pflanzen als empfindungsfähig erweisen?"

Es herrschte ein schwangeres Schweigen neben ihm, als Harrys Augen das Inhaltsverzeichnis hinuntergingen. Es gab tatsächlich ein Kapitel über Pflanzensprache, was Harrys Herz einen Schlag aussetzen ließ; und dann begannen seine Hände, schnell die Seiten umzublättern und die entsprechende Seitenzahl anzusteuern.

"Es gibt Tage", sagte Hermine Granger, "an denen ich wirklich, wirklich keine Ahnung habe, was in deinem Kopf vor sich geht."

"Schau, es ist eine Frage der Multiplikation, okay? Es gibt eine Menge Pflanzen auf der Welt, wenn sie nicht empfindungsfähig sind, dann sind sie nicht wichtig, aber wenn Pflanzen Menschen sind, dann haben sie mehr moralisches Gewicht als alle Menschen auf der Welt zusammen. Natürlich erkennt dein Gehirn das auf einer intuitiven Ebene nicht, aber das liegt daran, dass das Gehirn nicht multiplizieren kann. Wenn man zum Beispiel drei verschiedene Gruppen von kanadischen Haushalten fragen, wie viel sie zahlen würden, um zweitausend, zwanzigtausend oder zweihunderttausend Vögel vor dem Tod in Ölteichen zu bewahren, werden die drei Gruppen jeweils angeben, dass sie bereit sind, achtundsiebzig, achtundachtzig und achtzig Dollar zu zahlen. Kein Unterschied, mit anderen Worten. Das nennt man "\emph{Missachtung des Maßstabs}". Das Gehirn stellt sich einen einzelnen Vogel vor, der in einem Ölteich kämpft, und dieses Bild erzeugt eine gewisse Emotion, die deine Zahlungsbereitschaft bestimmt. Aber niemand kann sich auch nur zweitausend von irgendetwas vorstellen, also wird die Menge einfach zum Fenster hinausgeworfen. Versuch nun, diese Voreingenommenheit in Bezug auf hundert Billionen empfindungsfähiger Grashalme zu korrigieren, und du wirst erkennen, dass dies tausendmal wichtiger sein könnte, als wir früher glaubten, dass die gesamte menschliche Spezies … oh, Azathoth sei Dank, hier steht, dass nur Alraunen sprechen können und dass sie die normale menschliche Sprache laut aussprechen, nicht dass es einen Zauber gibt, mit dem man mit jeder Pflanze sprechen kann -"

"Ron kam gestern Morgen beim Frühstück zu mir", sagte Hermine. Jetzt klang ihre Stimme ein wenig leise, ein wenig traurig, vielleicht sogar ein wenig ängstlich.\\ "Er sagte, er sei furchtbar schockiert gewesen, als er sah, wie ich dich küsste. Dass das, was du gesagt hast, während du dement warst, mir hätte zeigen sollen, wie viel Böses du in dir versteckst. Und dass er sich nicht mehr sicher sei, ob er in meiner Armee sein wolle, wenn ich ein Anhänger eines dunklen Zauberers sein würde."

Harrys Hände hatten aufgehört, Seiten umzublättern. Es schien, dass Harrys Gehirn trotz all seines abstrakten Wissens immer noch nicht in der Lage war, Umfang auf einer echten emotionalen Ebene zu schätzen, denn es hatte gerade seine Aufmerksamkeit gewaltsam von Billionen möglicherweise empfindungsfähiger Grashalme, die vielleicht sogar leiden oder sterben würden, während sie sprachen, auf das Leben eines einzelnen Menschen gelenkt, der ihm zufällig näher und lieber war.

"Ron ist der gigantischste Trottel der Welt", sagte Harry. "Das werden sie in nächster Zeit nicht in der Zeitung drucken, denn es ist keine Neuigkeit. Also, nachdem du ihn gefeuert hast, wie viele seiner Arme und Beine hast du ihm gebrochen?"

"Ich habe versucht, ihm zu sagen, dass es nicht so war", fuhr Hermine mit der gleichen ruhigen Stimme fort. "Ich habe versucht, ihm zu sagen, dass du nicht so bist und dass es zwischen uns beiden nicht so war, aber das schien ihn nur noch mehr … noch mehr so zu machen, wie er war."

"Nun, ja", sagte Harry. Er war überrascht, dass er sich nicht noch wütender auf Ron Weasley fühlte, aber seine Sorge um Hermine schien das zu überlagern, für den Moment.\\ "Je mehr du versuchst, dich vor solchen Leuten zu rechtfertigen, desto mehr bestätigt es, dass sie das Recht haben, dich in Frage zu stellen. Es zeigt, dass du denkst, dass sie deine Inquisitoren sein dürfen, und sobald du jemandem diese Art von Macht über dich zugestehst, drängen sie nur noch mehr und mehr."

Das war eine von Draco Malfoys Lektionen, die Harry eigentlich für ziemlich klug gehalten hatte: Leute, die versuchten, sich zu verteidigen, wurden wegen jeder Kleinigkeit ausgefragt und konnten ihre Vernehmer nie zufriedenstellen; aber wenn man von Anfang an klarstellte, dass man eine Berühmtheit war und über den gesellschaftlichen Konventionen stand, würden sich die Leute nicht die Mühe machen, die meisten Verstöße zu verfolgen.

"Deshalb habe ich, als Ron zu mir rüberkam, als ich am Ravenclaw-Tisch saß, und mir sagte, ich solle mich von dir fernhalten, meine Hand über den Boden gestreckt und gesagt: '\emph{Siehst du, wie hoch ich meine Hand halte? Deine Intelligenz muss mindestens so groß sein, um mit mir zu reden}.' Dann beschuldigte er mich, Zitat, \emph{dich in die Dunkelheit zu saugen,} Zitat Ende, also schürzte ich meine Lippen und machte \emph{schluuuuurp}, und danach machte sein Mund immer noch diese Sprechgeräusche, also setzte ich einen Schweigezauber ein. Ich glaube nicht, dass er seine Vorträge noch einmal bei mir versuchen wird."

"Ich verstehe, warum du das getan hast", sagte Hermine mit fester Stimme, "ich wollte ihm auch eine Standpauke halten, aber ich wünschte wirklich, du hättest es nicht getan, das macht die Sache für mich noch schwieriger, Harry!"

Harry sah wieder von der Gemüseschnüffelei auf, er kam bei diesem Tempo nicht zum Lesen; und er sah, dass Hermine immer noch in dem Buch las, das sie in der Hand hatte, ohne zu ihm aufzuschauen. Ihre Hände blätterten eine weitere Seite um, selbst als er sie beobachtete.

"Ich glaube, du gehst den falschen Weg, wenn du versuchst, dich überhaupt zu verteidigen", sagte Harry. "Das denke ich wirklich. Du bist, wer du bist. Du bist befreundet, mit wem du willst. Sag jedem, der dich in Frage stellt, er soll es sich sonst wohin stecken."

Hermine schüttelte nur den Kopf und blätterte weiter.

"Option zwei", sagte Harry. "Geh zu Fred und George und sag ihnen, sie sollen sich mal mit ihrem missratenen Bruder unterhalten, die beiden sind echt gute Jungs -"

"Es ist nicht nur Ron", sagte Hermine fast im Flüsterton. "Viele Leute sagen es, Harry. Sogar Mandy wirft mir besorgte Blicke zu, wenn sie denkt, dass ich nicht hinschaue. Ist das nicht komisch? Ich mache mir ständig Sorgen, dass Professor Quirrell dich in die Dunkelheit saugt, und jetzt warnen mich die Leute genau so, wie ich dich zu warnen versuche."

"Nun, ja", sagte Harry. "Beruhigt dich das nicht ein wenig in Bezug auf mich und Professor Quirrell?"

"Mit einem Wort", sagte Hermine, "nein."

Es herrschte eine Stille, die lange genug dauerte, damit Hermine eine weitere Seite umblättern konnte, und dann sagte ihre Stimme, diesmal in einem richtigen Flüsterton: "Und, und Padma geht herum und erzählt allen, dass, da ich den P-Patronus-Zauber nicht wirken konnte, ich nur so tun muss, als wäre ich n-nett…"

"Padma hat es nicht einmal selbst versucht!" sagte Harry entrüstet. "Wenn du eine Dunkle Hexe wärst, die nur so tut, hättest du es nicht vor allen Leuten versucht, halten die dich für dumm?"

Hermine lächelte ein wenig und blinzelte ein paar Mal.

"Hey, ich muss mir Sorgen machen, dass ich tatsächlich böse werde. Hier ist das schlimmste Szenario, dass die Leute denken, du wärst böser als du wirklich bist. Wird dich das umbringen? Ich meine, ist das wirklich so schlimm?"

Das junge Mädchen nickte, ihr Gesicht fest verzogen.

"Sieh mal, Hermine… wenn du dich so sehr darum sorgst, was andere Leute denken, wenn du unglücklich bist, wenn andere Leute dich nicht genau so sehen, wie du dich selbst siehst, dann bist du schon dazu verdammt, immer unglücklich zu sein. Niemand denkt jemals genau so über uns, wie wir über uns selbst denken."

"Ich weiß nicht, wie ich es dir erklären soll", sagte Hermine mit trauriger, leiser Stimme. "Ich bin mir nicht sicher, ob du das jemals verstehen könntest, Harry. Das Einzige, was mir dazu einfällt, ist: Wie würdest du dich fühlen, wenn ich dich für böse halten würde?"

"Ähm…" Harry stellte es sich vor. "Ja, das würde wehtun. Sehr sogar. Aber du bist ein guter Mensch, der über solche Dinge intelligent nachdenkt, du hast dir diese Macht über mich verdient, es würde etwas bedeuten, wenn du denken würdest, ich hätte etwas falsch gemacht. Mir fällt kein einziger anderer Schüler ein, außer dir, um dessen Meinung ich mich genauso kümmern würde -"

"Du kannst so leben", flüsterte Hermine Granger. "Ich kann es nicht."

Das Mädchen hatte weitere drei Seiten schweigend durchgeblättert, und Harry hatte seine Augen wieder auf sein eigenes Buch gerichtet und versuchte, sich wieder zu konzentrieren, als Hermine schließlich mit leiser Stimme sagte:

"Bist du wirklich sicher, dass ich nicht wissen darf, wie man den Patronus-Zauber wirkt?"

"I…" Harry musste einen plötzlichen Kloß in seinem Hals herunterschlucken. Er sah sich plötzlich vor sich, wie er nicht wusste, warum der Patronus-Zauber bei ihm nicht funktionierte, wie er es Draco nicht zeigen konnte, wie ihm nur gesagt wurde, dass es einen Grund gab, und sonst nichts.

"Hermine, dein Patronus würde mit demselben Licht leuchten, aber er würde nicht normal sein, er würde nicht so aussehen, wie die Leute denken, dass Patronusse aussehen sollten, jeder, der ihn sehen würde, wüsste, dass etwas Seltsames vor sich geht. Selbst wenn ich dir das Geheimnis verraten würde, könntest du es niemandem zeigen, es sei denn, du würdest sie dazu bringen, in die andere Richtung zu schauen, so dass sie nur das Licht sehen könnten, und… und der wichtigste Teil eines jeden Geheimnisses ist das Wissen, dass ein Geheimnis existiert, du könntest es nur einem oder zwei Freunden zeigen, wenn du sie zur Verschwiegenheit verpflichtest…" Harrys Stimme verstummte hilflos.

"Ich nehme es an." Ihre Stimme war immer noch leise.

Es war sehr schwer, nicht einfach mit dem Geheimnis herauszuplatzen, genau hier in der Bibliothek.

"Ich, ich sollte nicht, ich sollte wirklich nicht, es ist gefährlich, Hermine, es könnte eine Menge Schaden anrichten, wenn dieses Geheimnis herauskommt! Kennst du nicht das Sprichwort: Drei können ein Geheimnis bewahren, wenn zwei tot sind? Dass es, wenn man es nur seinen engsten Freunden erzählt, dasselbe ist, wie wenn man es allen erzählt, weil man nicht nur ihnen vertraut, sondern allen, denen sie vertrauen? Es ist zu wichtig, ein zu großes Risiko, es ist nicht die Art von Entscheidung, die man treffen sollte, um den Ruf von jemandem in der Schule zu verbessern!"

"Okay", sagte Hermine. Sie klappte das Buch zu und stellte es zurück ins Regal. "Ich kann mich gerade nicht konzentrieren, Harry, es tut mir leid."

"Wenn ich noch etwas tun kann -"

"Sei netter zu allen."

Das Mädchen blickte nicht zurück, als sie aus dem Regal ging, was vielleicht gut war, denn der Junge war wie erstarrt, unbeweglich. Nach einer Weile begann der Junge wieder zu blättern.

