

\hypertarget{reflexionen}{% \section{108. Reflexionen}\label{reflexionen}}

\textbf{\uline{Reflexionen}}

Selbst das größte Artefakt kann von einem Gegenartefakt besiegt werden, das zwar kleiner, aber dafür spezialisiert ist. Das hatte der Verteidigungsprofessor Harry gesagt, nachdem er den Wahren Unsichtbarkeitsmantel fallen ließ, der sich in dicken Falten neben Harrys Schuhen sammelte. Der Spiegel der perfekten Reflexion hat Macht über das, was in ihm reflektiert wird, und diese Macht ist angeblich unanfechtbar. Aber da der Wahre Mantel der Unsichtbarkeit eine perfekte Abwesenheit des Bildes erzeugt, sollte er sich diesem Prinzip eher entziehen als es herausfordern. Es folgten eine Reihe von Fragen in Parsel, um festzustellen, dass Harry im Moment nicht vorhatte, irgendetwas Dummes zu tun oder zu versuchen, wegzulaufen, und weitere Hinweise darauf, dass Professor Quirrell ihn spüren konnte und Zaubersprüche hatte, um den Umhang aufzuspüren, und dass er Hunderte von Menschenleben plus Hermine als Geiseln hielt. Dann wurde Harry angewiesen, den Umhang anzulegen, die Tür zu öffnen, die jenseits der gelöschten Feuer lag, und durch die Tür in die letzte Kammer vorzudringen, während Professor Quirrell weit hinten, außerhalb der Sichtweite dieser Tür, stand.

Die letzte Kammer war in Lichtern von sanftem Gold erleuchtet, und die Steinwände waren von sanftem Weiß und mit Marmor verkleidet. In der Mitte des Raumes stand ein schlichter und schmuckloser goldener Rahmen, und innerhalb des Rahmens befand sich ein Portal zu einem weiteren goldbeleuchteten Raum, hinter dessen Tür eine weitere Zaubertrankkammer lag; das war es, was Harrys Gehirn ihm sagte. Die Lichtumwandlung des Spiegels war so perfekt, dass es bewusster Gedanken bedurfte, um zu folgern, dass der Raum im Inneren des Rahmens nur eine Reflexion und kein Portal war. (Obwohl es vielleicht einfacher gewesen wäre, es zu erahnen, wenn Harry nicht unsichtbar gewesen wäre, gerade in diesem Moment.) Der Spiegel berührte den Boden nicht; der goldene Rahmen hatte keine Füße. Er sah nicht aus, als würde er schweben; er sah aus, als wäre er an seinem Platz fixiert, fester und unbeweglicher als die Wände selbst, als wäre er an den Bezugsrahmen der Erdbewegung genagelt.

"Ist der Spiegel da? Bewegt er sich?", kam Professor Quirrells befehlende Stimme aus der Zaubertrankkammer.

"\emph{Ist da}", zischte Harry zurück. "\emph{Er bewegt sich nicht.}"

Wieder erklangen Befehlstöne.\\ "Geh auf die Rückseite des Spiegels."

Der goldene Rahmen wirkte von hinten fest und zeigte keine Spiegelungen, und Harry sagte das in Parsel.

"Nimm jetzt deinen Umhang ab", befahl Professor Quirrells Stimme, die immer noch aus dem Zaubertrankraum kam. "Melde dich sofort bei mir, wenn der Spiegel sich zu dir hin bewegt."

Harry nahm seinen Umhang ab. Der Spiegel blieb auf dem Bezugsrahmen der Erdbewegung festgenagelt; und Harry meldete dies.

Kurz darauf gab es ein Zischen und Brodeln, und der verfluchte Phönix schmolz durch die Marmorwand hinter Harry, wobei das Umgebungslicht im Raum eine rote Färbung annahm, als er eintrat. Professor Quirrell folgte ihm und trat aus dem neu geschaffenen Korridor heraus, seine schwarzen formellen Schuhe blieben von der rotglühenden geschmolzenen Oberfläche darunter unversehrt.

"Nun", sagte Professor Quirrell, "das ist eine mögliche Falle abgewendet. Und jetzt…" Professor Quirrell atmete aus. "Jetzt werden wir uns mögliche Strategien überlegen, um den Stein aus dem Spiegel zu holen, und du wirst sie ausprobieren; denn ich ziehe es vor, mein eigenes Bild nicht reflektieren zu lassen. Ich warne dich fairerweise, das ist der Teil, der sich als mühsam erweisen könnte."

"Ich nehme an, das ist kein Problem, das man mit verfluchtem Feuer lösen kann?"

"Ha", sagte Professor Quirrell und gestikulierte.\\ Der Phönix bewegte sich in einem Rausch aus purpurnem Schrecken vorwärts, wobei das rote Licht sich windende Schatten auf die restlichen Marmorwände warf. Harry sprang zurück, bevor er denken konnte. Die furchtbare dunkelrote Glut raste an Professor Quirrell vorbei, wogte in die goldene Rückseite des Spiegels und verschwand so schnell, wie sie das Gold berührte. Dann war das Feuer verschwunden und der Raum war nicht mehr scharlachrot gefärbt. Es gab keinen Kratzer auf der goldenen Oberfläche, kein Glühen, das die Absorption der Hitze markiert hätte. Der Spiegel war einfach an seinem Platz geblieben, unberührt.

Ein Schauer lief Harry über den Rücken. Hätte er Dungeons and Dragons gespielt und der Dungeonmaster hätte ihm dieses Ergebnis berichtet, hätte Harry eine mentale Illusion vermutet und hätte gewürfelt, um das nicht zu glauben.

Auf der Mitte des goldenen Rückens war eine Abfolge von Runen in einem unbekannten Alphabet erschienen, schwarze Abwesenheiten von Licht in kleinen Linien und Kurven, angeordnet in einer ebenen horizontalen Reihe. Harry kam der Gedanke, dass irgendeine kleine verdeckende Illusion im Feuer zerstört worden war, ein weitaus geringerer Zauber, der hinzugefügt worden war, um zu verhindern, dass Kinder diese Buchstaben sehen konnten …

"Wie alt ist dieser Spiegel?" sagte Harry fast flüsternd.

"Das weiß niemand, Mr. Potter." Der Verteidigungsprofessor streckte seine Finger nach den Runen aus, ein Ausdruck von so etwas wie Ehrfurcht auf seinem Gesicht; aber seine Finger berührten das Gold nicht. "Aber meine Vermutung ist dieselbe wie deine, glaube ich. In bestimmten Legenden, die vielleicht erfunden sind oder auch nicht, heißt es, dass dieser Spiegel sich selbst perfekt reflektiert und deshalb in seiner Existenz absolut stabil ist. So stabil, dass der Spiegel in der Lage war, zu überleben, als jeder andere Effekt von Atlantis rückgängig gemacht wurde, alle seine Folgen von der Zeit abgetrennt wurden. Du siehst, warum ich amüsiert war, als du das Feuer vorgeschlagen hast."\\ Der Verteidigungsprofessor ließ die Hand sinken.

Sogar inmitten des Geschehens spürte Harry die Ehrfurcht, wenn das wahr war. Der goldene Rahmen schimmerte nicht heller als zuvor, trotz der Offenbarung; aber man konnte sich vorstellen, dass er zurückging und zurückging, in eine Zivilisation, die dafür gemacht war, niemals zu sein…

"Was - tut der Spiegel genau?"

"Eine ausgezeichnete Frage", sagte Professor Quirrell. "Die Antwort steht in den Runen, die auf dem goldenen Rahmen des Spiegels geschrieben sind. Lies sie mir vor."

"Sie sind in keinem Alphabet, das ich kenne. Sie sehen aus wie willkürlich angeordnete Hühnerkratzer, gezeichnet von Tolkien-Elfen."

"Lies sie trotzdem. \textbf{\emph{Isst nicht gefährlich}}."

"Die Runen sagen, \emph{noitilov detalo partxe tnere hoc ruoy tu becafruoy ton wo hsi} -" Harry hielt inne und spürte ein weiteres Kribbeln an seiner Wirbelsäule. Harry wusste, was die Rune für noitilov bedeutete. Sie bedeutete noitilov. Und die nächsten Runen sagten, man solle den Noitilov bis zum Partxe auflösen, dann den Teil behalten, der sowohl tnere als auch hoc war. Dieser Glaube fühlte sich wie Wissen an, als hätte er mit sicherer Autorität "Ja" antworten können, wenn ihn jemand gefragt hätte, ob die Tonne wo ruoy oder becafruoy war. Es war nur so, dass Harry, wenn er versuchte, diese Begriffe mit anderen Begriffen in Verbindung zu bringen, ins Leere lief.

"Verstehst du, was Worte bedeuten, Junge?"

"\emph{Denke nicht}."

Professor Quirrell atmete leise aus, seine Augen verließen den goldenen Rahmen nicht. "Ich hatte mich gefragt, ob die Worte des falschen Verständnisses vielleicht für einen Muggelschüler verständlich sein könnten. Offenbar nicht."

"Vielleicht -" begann Harry.\\ \emph{Wirklich, Ravenclaw?} sagte Slytherin. \emph{Du ziehst das JETZT ab?}\\ "Vielleicht könnte ich noch einmal versuchen, die Worte zu verstehen, wenn ich mehr über den Spiegel wüsste?", sagte Harrys Ravenclaw-Teil, der die direkte Kontrolle übernommen hatte.

Professor Quirrells Lippen schürzten sich.\\ "Wie bei den meisten alten Dingen haben die Gelehrten so viele Lügen aufgeschrieben, dass es mittlerweile schwer ist, sich über irgendetwas sicher zu sein. Sicher ist, dass der Spiegel mindestens so alt ist wie Merlin, denn es ist bekannt, dass Merlin ihn als Werkzeug benutzt hat. Es ist auch bekannt, dass Merlin nach seinem Tod die schriftliche Anweisung hinterließ, dass der Spiegel nicht versiegelt werden muss, obwohl er bestimmte Kräfte besitzt, die normalerweise Anlass zur Sorge geben könnten. Er schrieb, dass, wenn man bedenkt, wie sorgfältig der Spiegel hergestellt wurde, um die Welt nicht zu zerstören, es einfacher wäre, die Welt mit einem Stück Käse zu zerstören."

Diese Aussage wirkte auf Harry nicht gerade beruhigend.

"Bestimmte andere Fakten über den Spiegel werden von berühmten Zauberern bezeugt, die einigermaßen skeptisch waren und deren Wort sich ansonsten als zuverlässig erwiesen hat. Die charakteristischste Kraft des Spiegels ist es, alternative Existenzbereiche zu erschaffen, obwohl diese Bereiche nur so groß sind wie das, was im Spiegel zu sehen ist; es ist bekannt, dass Menschen und andere Gegenstände darin gespeichert werden können. Es wird von mehreren Autoritäten behauptet, dass der Spiegel als einziger aller Zauber eine echte moralische Ausrichtung besitzt, obwohl ich mir nicht sicher bin, was das in praktischer Hinsicht bedeuten könnte. Ich würde erwarten, dass Moralisten den Cruciatus-Fluch als "böse" und den Patronus-Zauber als "gut" bezeichnen würden; ich kann mir nicht vorstellen, was ein Moralist noch moralischer finden würde als das. Aber es wird zum Beispiel behauptet, dass Phönixe aus einem Reich in unsere Welt kamen, das in diesem Spiegel beschworen wurde."

Worte wie "\emph{Wow}" und das, was seine Eltern als unangemessene Sprache bezeichnet hätten, schossen Harry durch den Kopf, keine sehr zusammenhängenden, während er auf die goldene Rückseite des Spiegels starrte.

"Ich bin durch die Welt gewandert und habe viele Geschichten kennengelernt, die man nicht oft hört", sagte Professor Quirrell. "Die meisten davon schienen mir Lügen zu sein, aber ein paar hatten eher den Klang von Geschichte als von Erzählung.\\ Auf einer Metallwand an einem Ort, an den seit Jahrhunderten niemand mehr gekommen war, fand ich die Behauptung geschrieben, dass einige Atlanter das Ende ihrer Welt voraussahen und versuchten, ein Gerät von großer Macht zu schmieden, um die unvermeidliche Katastrophe abzuwenden. Wäre dieses Gerät vollendet worden, so behauptete die Geschichte, wäre es zu einer absolut stabilen Existenz geworden, die der Kanalisierung unbegrenzter Magie standhalten konnte, um Wünsche zu erfüllen. Und außerdem - das soll die weitaus schwierigere Aufgabe gewesen sein - würde das Gerät irgendwie die unvermeidlichen Katastrophen abwenden, die jeder vernünftige Mensch unter dieser Prämisse erwarten würde. Der Aspekt, den ich interessant fand, war, dass laut der Geschichte, die auf diesen Metallplatten geschrieben stand, der Rest von Atlantis dieses Projekt ignorierte und seiner Wege ging. Es wurde manchmal als edles öffentliches Unterfangen gepriesen, aber fast alle anderen Atlanter hatten an einem beliebigen Tag Wichtigeres zu tun, als zu helfen. Sogar die atlantischen Adligen ignorierten die Aussicht, dass jemand anderes als sie selbst unanfechtbare Macht erlangen könnte, was ein weniger erfahrener Zyniker vielleicht erwarten würde, um ihre Aufmerksamkeit zu erregen. Mit relativ wenig Unterstützung mühte sich die winzige Handvoll Möchtegern-Hersteller dieses Geräts unter Arbeitsbedingungen ab, die nicht so sehr dramatisch beschwerlich, als vielmehr sinnlos nervig waren. Schließlich lief die Zeit ab und Atlantis wurde zerstört, während das Gerät noch lange nicht fertig war. Ich erkenne gewisse Anklänge an meine eigenen Erfahrungen, die man normalerweise nicht in bloßen Erzählungen erfunden sieht."\\ Eine Wendung in dem trockenen Lächeln.\\ "Aber vielleicht ist das nur meine eigene Vorliebe für ein Märchen unter hundert anderen Legenden. Du erkennst jedoch das Echo von Merlins Aussage, dass die Schöpfer des Spiegels ihn so gestaltet haben, dass er die Welt nicht zerstört. Am wichtigsten für unsere Zwecke ist, dass es vielleicht erklärt, warum der Spiegel die bisher unbekannte Fähigkeit besitzt, die Dumbledore oder Perenelle heraufbeschworen zu haben scheinen, nämlich jeder Person, die vor ihn tritt, die Illusion einer Welt zu zeigen, in der einer ihrer Wünsche erfüllt wurde. Es ist die Art von vernünftiger Vorsichtsmaßnahme, die man sich vorstellen kann, wenn jemand eine wunscherfüllende Schöpfung einbaut, die nicht furchtbar schief gehen soll."

"Wow", flüsterte Harry, und meinte es auch so. Das war Magie mit einem großen M, die Art von Magie, die in \emph{So You Want To Be A Wizard} auftauchte, nicht nur eine Sammlung von zufälligen, die Physik verletzenden Dingen, die man mit einem Zauberstab machen konnte.

Professor Quirrell gestikulierte auf den goldenen Rücken.\\ "Die letzte Eigenschaft, über die sich die meisten Erzählungen einig sind, ist, dass unabhängig von der unbekannten Art, den Spiegel zu befehligen - über diesen Schlüssel gibt es keine plausiblen Berichte - die Anweisungen des Spiegels nicht so geformt werden können, dass sie auf einzelne Personen reagieren. Es ist also nicht möglich, dass Perenelle dem Spiegel befiehlt: 'Gib den Stein nur Perenelle'. Dumbledore kann nicht sagen: 'Gib den Stein nur demjenigen, der ihn Nicholas Flamel geben will'. Es gibt im Spiegel eine Blindheit, wie sie Philosophen der idealen Gerechtigkeit zugeschrieben haben; er muss alle, die vor ihn kommen, nach derselben Regel behandeln, welche Regel auch immer in Kraft sein mag. Also muss es eine Regel geben, um das Versteck des Steins zu erreichen, auf die sich jeder berufen kann. Und jetzt siehst du, warum du, genannt der Junge-der-lebte, die Strategien umsetzen sollst, die wir beide uns ausdenken. Denn es wurde gesagt, dass dieses Ding eine moralische Ausrichtung besitzt, und es könnten ihm Befehle gegeben worden sein, die dasselbe widerspiegeln. Ich bin mir durchaus bewusst, dass du nach konventionellen Begriffen als gut gilst, so wie ich als böse gelte." Professor Quirrell lächelte, eher düster. "Lass uns also als ersten Versuch - wenn auch nicht als letzten, sei versichert - sehen, was dieser Spiegel von deinem Versuch hält, den Stein zurückzuholen, um das Leben von Hermine Granger und Hunderten Ihrer Mitschüler zu retten."

"Und die erste Version dieses Plans", sagte Harry, der endlich zu verstehen begann, "die, die du am Freitag in meiner ersten Woche in Hogwarts erfunden hast, sah vor, dass der Stein von Dumbledores goldenem Kind, dem Jungen-der-lebte, zurückgeholt werden sollte, der einen selbstlosen und edlen Versuch unternahm, das Leben seines sterbenden Verteidigungslehrers, Professor Quirrell, zu retten."

"Natürlich", sagte Professor Quirrell.

Es war eine poetische Art von Handlung, vermutete Harry, aber seine Wertschätzung dieser Eleganz wurde durch die umgebenden Umstände beeinträchtigt. Dann kam Harry ein anderer Gedanke.

"Ähm", sagte Harry. "Du glaubst, dass dieser Spiegel eine Falle für dich ist -"

"Es gibt keinen Weg unter dem Himmel, dass er nicht als Falle gedacht ist."

"Das heißt, dass es eine Falle für Lord Voldemort ist. Nur kann es keine Falle für ihn persönlich sein. Es muss eine allgemeine Regel dahinterstecken, irgendeine verallgemeinerbare Eigenschaft von Lord Voldemort, die sie auslöst."

Ohne sich dessen bewusst zu sein, runzelte Harry die Stirn, als er den goldenen Rücken des Spiegels betrachtete.

"Worauf willst du hinaus?", sagte Professor Quirrell, der angesichts von Harrys Stirnrunzeln die Stirn zu runzeln begann.\\ `

"Nun, am ersten Donnerstag dieses Jahres sagte mir der verrückte Schulleiter Dumbledore, den ich gerade gesehen hatte, wie er ein Huhn verbrannte, dass ich keinerlei Chance hätte, in seinen verbotenen Korridor zu gelangen, da ich den Zauberspruch Alohomora nicht kenne."

"Ich verstehe", sagte Professor Quirrell. "Oh, je. Ich wünschte, du hättest daran gedacht, mir das schon viel früher zu sagen."

Keiner von beiden brauchte das Offensichtliche laut auszusprechen, dass dieses bisschen umgekehrte Psychologie erfolgreich dafür gesorgt hatte, dass Harry sich verdammt noch mal von Dumbledores verbotenem Korridor fernhalten würde.

Harry konzentrierte sich immer noch.\\ "Glaubst du, Dumbledore vermutet, dass ich, wie er es nennt, ein Horcrux von Lord Voldemort bin, oder allgemeiner, dass einige Aspekte meiner Persönlichkeit von Lord Voldemort kopiert wurden?" Selbst als Harry dies laut fragte, wurde ihm klar, was für eine dumme Frage das war und wie viele völlig eklatante Beweise er bereits gesehen hatte, die -

"Dumbledore kann es unmöglich übersehen haben", sagte Professor Quirrell. "Es ist nicht gerade subtil. Was soll Dumbledore denn sonst denken, dass du ein Schauspieler in einem Stück bist, dessen dummer Autor noch nie einen echten Elfjährigen getroffen hat? Nur ein schwafelnder Dummkopf würde das glauben - ah, egal."

Die beiden starrten den Spiegel schweigend an.

Schließlich seufzte Professor Quirrell. "Ich habe mich selbst überlistet, fürchte ich. Weder du noch ich wagen es, sich in diesem Spiegel zu spiegeln. Ich muss wohl Professor Sprout befehlen, meine Verpflichtungen gegenüber Mr. Nott und Miss Greengrass rückgängig zu machen… Die andere große Schwierigkeit des Spiegels ist, dass die Regel, nach der er die Reflektierten behandelt, externe Kräfte außer Acht lässt, wie z.B. falsche Erinnerungen oder einen Confundus-Zauber. Der Spiegel reflektiert nur die Kräfte, die aus dem Inneren der Person selbst kommen, die Geisteszustände, die sie durch ihre eigenen Entscheidungen erreichen; so heißt es an mehreren Stellen. Deshalb hatte ich Mr. Nott und Miss Greengrass, die unterschiedliche Geschichten darüber glaubten, warum die Entnahme des Steins notwendig war, bereit, vor diesem Spiegel zu erscheinen."\\ Professor Quirrell rieb sich über den Nasenrücken.\\ "Ich konstruierte andere Geschichten für andere Schüler, die ich mit dem gewählten Auslöser in Gang setzen wollte … aber als dieser Tag näher rückte, begann ich, dem Projekt pessimistisch gegenüberzustehen. Solche wie Nott und Greengrass scheinen noch einen Versuch wert zu sein, wenn uns nichts Besseres einfällt. Aber ich frage mich, ob Dumbledore versucht hat, dieses Rätsel zu konstruieren, um sich gezielt gegen Voldemorts Gerissenheit zu wehren. Ich frage mich, ob ihm das gelungen sein könnte. Wenn du einen alternativen Plan entwirfst, den ich gut genug finde, um ihn auszuprobieren, dann verspreche ich dir, dass ich dem Bauern, den ich aussende, kein Leid zufügen werde, weder damals noch heute. Und ich erinnere dich nochmals an die Geiseln, die ich für den Fall meines Versagen halte, sowohl Miss Granger als auch alle anderen."

Wieder starrten sie schweigend in den Spiegel, der ältere Tom Riddle und der jüngere.

"Ich habe den Verdacht, Professor", sagte Harry nach einer Weile, "dass deine ganze Klasse von Hypothesen darüber, dass jemand den Stein für gute oder ehrliche Zwecke haben will, falsch ist. Der Schulleiter würde so eine Wiederbeschaffungsregel nicht aufstellen."

"Warum?"

"Weil Dumbledore weiß, wie leicht man in dem Glauben enden kann, das Richtige zu tun, wenn man es eigentlich nicht tut. Das wäre die erste Möglichkeit, die er sich vorstellen könnte."

"Ist es Wahrheit oder Täuschung, was ich höre?"

"\emph{Ich bin ehrlich}", sagte Harry.

Professor Quirrell nickte.\\ "Dann hast du recht."

"Ich bin mir nicht sicher, warum du glaubst, dass dieses Rätsel lösbar ist", sagte Harry. "Stell einfach eine Regel auf wie: deine linke Hand muss eine kleine blaue Pyramide und zwei große rote Pyramiden halten, und deine rechte Hand muss Mayonnaise auf einen Hamster pressen -"

"Nein", sagte Professor Quirrell. "Nein, ich glaube nicht. Die Legenden sind unklar darüber, welche Regeln gegeben werden können, aber ich denke, es muss etwas mit dem ursprünglichen Verwendungszweck des Spiegels zu tun haben - es muss etwas mit den tiefen Sehnsüchten und Wünschen zu tun haben, die aus dem\\ Inneren der Person kommen. Mayonnaise auf einen Hamster zu quetschen, wird für die meisten Menschen nicht dazu gehören."

"Mh", sagte Harry. "Vielleicht ist die Regel, dass die Person den Stein überhaupt nicht benutzen wollen muss - nein, das ist zu einfach, die Geschichte, die du Mr. Nott gegeben hast, löst das Problem."

"In mancher Hinsicht verstehst du Dumbledore vielleicht besser als ich", sagte Professor Quirrell. "Deshalb frage ich dich jetzt: Wie würde Dumbledore seine Vorstellung von der Akzeptanz des Todes nutzen, um diesen Stein zu bewachen?\\ Denn vor allem das, denkt er, kann ich nicht begreifen, und damit liegt er nicht weit daneben."

Harry dachte eine Weile darüber nach, erwog mehrere Ideen und verwarf sie wieder. Und dann, nachdem ihm etwas eingefallen war, überlegte Harry, ob er schweigen sollte… bevor er sich den offensichtlichen Teil des zukünftigen Gesprächs vorstellte, in dem Professor Quirrell ihm befahl, in Parsel zu sagen, ob ihm etwas eingefallen sei.

Zögernd sprach Harry. "Würde Dumbledore denken, dass dieser Spiegel das Jenseits erreichen könnte? Könnte er den Stein in etwas legen, das er für ein Jenseits hält, so dass nur Leute, die an ein Jenseits glauben, ihn sehen können?"

"Hm…" sagte Professor Quirrell. "Möglicherweise… ja, da ist eine gewisse Plausibilität dabei. Wenn man diese Einstellung des Spiegels benutzt, um den Leuten ihre Herzenswünsche zu zeigen… würde Albus Dumbledore sich mit seiner Familie wiedervereint sehen. Er würde sich mit ihnen im Tod vereint sehen und lieber selbst sterben wollen, als sich zu wünschen, dass sie ins Leben zurückkehren. Sein Bruder Aberforth, seine Schwester Ariana, seine Eltern Kendra und Percival… es wäre Aberforth, dem Dumbledore den Stein geben würde, denke ich. Würde der Spiegel erkennen, dass gerade Aberforth den Stein bekommen hat? Oder reicht auch ein toter Verwandter einer Person, wenn diese Person glaubt, der Geist ihres Verwandten würde ihr den Stein zurückgeben?"\\ Professor Quirrell schritt in einem kleinen Kreis umher und hielt dabei einen großen Abstand zu Harry und dem Spiegel.\\ "Aber das alles ist nur eine Idee. Lass uns eine andere ersinnen."

Harry begann, sich auf die Wange zu klopfen, dann hielt er abrupt inne, als ihm klar wurde, wo er diese Geste aufgeschnappt hatte.

"Was, wenn Perenelle diejenige ist, die den Stein hier hineingetan hat? Vielleicht hat sie den Spiegel so eingestellt, dass nur derjenige den Stein bekommt, der ihn ursprünglich hineingelegt hat."

"Perenelle hat so lange gelebt, weil sie ihre Grenzen kennt", sagte Professor Quirrell. "Sie überschätzt ihren eigenen Intellekt nicht, sie ist nicht hochmütig, wenn das so wäre, hätte sie den Stein schon längst verloren. Perenelle wird nicht versuchen, sich selbst eine gute Spiegelregel auszudenken, nicht, wenn Meister Flamel die Angelegenheit in Dumbledores weiseren Händen belassen kann… aber die Regel, den Stein nur demjenigen zurückzugeben, der sich an seine Platzierung erinnert, funktioniert auch, wenn Dumbledore selbst den Stein platziert hat. Es wäre schwer, diese Regel zu umgehen, da ich nicht einfach jemanden überlisten kann, der glaubt, dass er den Stein eingesetzt hat… Ich müsste einen falschen Stein und einen falschen Spiegel erschaffen und das Drama arrangieren…"\\ Professor Quirrell runzelte nun die Stirn.\\ "Aber es ist immer noch etwas, von dem Dumbledore sich vorstellen kann, dass Voldemort es arrangieren kann, wenn er Zeit hat. Wenn es irgend möglich ist, wird Dumbledore den Schlüssel zum Spiegel zu einem Geisteszustand machen wollen, von dem er glaubt, dass ich ihn nicht arrangieren kann - oder zu einer Regel, von der Dumbledore glaubt, dass Voldemort sie niemals begreifen kann, wie zum Beispiel eine Regel, die die Akzeptanz des eigenen Todes beinhaltet. Deshalb hielt ich deine vorherige Idee für plausibel."

Dann hatte Harry eine Idee. Er war sich nicht sicher, ob es eine gute Idee war. Aber es war nicht so, dass Harry hier eine große Wahl hatte. "Argumentieren wir so:", sagte Harry. "Wir sind uns nicht sicher, was notwendig ist, um den Stein wiederzubekommen. Aber eine hinreichende Bedingung sollte beinhalten, dass Albus Dumbledore oder vielleicht jemand anderes sich in einem Geisteszustand befindet, in dem er glaubt, dass der Dunkle Lord besiegt wurde, dass die Bedrohung vorbei ist und dass es an der Zeit ist, den Stein herauszunehmen und ihn Nicholas Flamel zurückzugeben. Wir sind uns nicht sicher, welcher Teil des Geisteszustandes dieser Person, sagen wir Dumbledores, der notwendige Teil sein wird, von dem er glaubt, dass Lord Voldemort ihn nicht verstehen oder duplizieren kann; aber unter diesen Bedingungen wird Dumbledores gesamter Geisteszustand ausreichend sein."

"Einleuchtend", sagte Professor Quirrell. "Und?"

"Die entsprechende Strategie", sagte Harry vorsichtig, "besteht darin, Dumbledores Geisteszustand unter diesen Bedingungen nachzuahmen, und zwar so detailliert wie möglich, während er vor dem Spiegel steht. Und dieser Geisteszustand muss durch innere Kräfte erzeugt worden sein, nicht durch äußere."

"Aber wie sollen wir das ohne Legilimenz oder den Konfundus-Zauber schaffen, die beide sicherlich extern wären - ha. Ich verstehe."\\ Professor Quirrells eisbleiche Augen waren plötzlich stechend.\\ "Du schlägst vor, dass ich mich selbst konfundiere, so wie du diesen Zauber an deinem ersten Tag in der Kampfmagie auf dich selbst gewirkt hast. Damit es eine innere Kraft ist und nicht eine äußere, ein Zustand, der nur durch meine eigenen Entscheidungen zustande kommt. Sag mir, ob du diesen Vorschlag in der Absicht gemacht hast, mich in eine Falle zu locken. Sag es mir in Parsel."

"\emph{Mein Verstand, den du für deine Strategie benutzt hast, mag vielleicht von einer solchen Absicht beeinflusst worden sein - wer weiß? Ich wußte, daß Du missstrauisch sein würdest, genau diese Frage stellssst. Die Entscheidung liegt bei dir, Herr Lehrer. Ich weiß nichts, was du nicht auch weißt und ob du damit in eine Falle gerätsssst. Nenne es nicht Verrat von mir, wenn du diess für dich selbst wählst, und es scheitert.}"

Harry spürte einen starken Impuls zu lächeln, unterdrückte ihn aber.

"Reizend", sagte Professor Quirrell und lächelte.\\ "Ich nehme an, es gibt einige Bedrohungen durch einen erfinderischen Geist, die selbst eine Befragung in Parsel nicht neutralisieren kann."

Harry legte auf Anweisung von Professor Quirrell den Unsichtbarkeitsumhang an, um zu verhindern, \emph{dass der Mann, der sich für sSchulleiter hält, dich sieht,} wie Professor Quirrell in Parsel sagte.\\ "Ob du den Umhang trägst oder nicht, du wirst selbst in Reichweite des Spiegels stehen", sagte Professor Quirrell. "Wenn ein Lavastrom herauskommt, wirst du ebenfalls verbrennen. Ich denke, dass diese Symmetrie gelten sollte."\\ Professor Quirrell zeigte auf eine Stelle rechts neben der Tür, durch die sie den Raum betreten hatten, vor dem Spiegel und weit dahinter.

Harry, der den Umhang trug, ging dorthin, wohin Professor Quirrell ihn gewiesen hatte, und widersprach nicht. Es war Harry zunehmend unklar, ob es eine schlechte Sache wäre, wenn beide Riddles hier sterben würden, selbst wenn Hunderte von anderen Schülergeiseln auf dem Spiel standen. Bei allen guten Absichten, die Harry hatte, hatte er sich bisher meistens als Idiot erwiesen, und der zurückgekehrte Lord Voldemort war eine Bedrohung für die ganze Welt.

(Obwohl, so oder so, Harry konnte sich nicht vorstellen, dass Dumbledore die Sache mit der Lava machen würde. Dumbledore war wahrscheinlich wütend genug auf Voldemort, um seine übliche Zurückhaltung abzulegen, aber Lava würde ein Wesen, das Dumbledore für eine verworfene Seele hielt, nicht dauerhaft aufhalten).

Dann deutete Professor Quirrell mit seinem Zauberstab, und ein schimmernder Kreis erschien um die Stelle, an der Harry auf dem Boden stand. Dieser, so sagte Professor Quirrell, würde bald zu einem Großen Kreis der Verhüllung werden, durch den nichts innerhalb dieses Kreises von außen gehört oder gesehen werden könnte. Harry würde sich dem falschen Dumbledore weder durch Abnehmen des Umhangs noch durch Schreien bemerkbar machen können.

"Du wirst diesen Kreis nicht durchqueren, sobald er aktiv ist", sagte Professor Quirrell. "Das würde dazu führen, dass du meine Magie berührst, und während ich abgelenkt bin, weiß ich vielleicht nicht mehr, wie ich die Resonanz aufhalten kann, die uns beide zerstören würde. Und außerdem, da ich nicht möchte, dass du mit Schuhen wirfst -" Professor Quirrell machte eine weitere Geste, und genau innerhalb des Großen Kreises der Abschirmung erschien ein leichter Schimmer in der Luft, eine kugelförmige Verzerrung. "Diese Barriere wird explodieren, wenn sie berührt wird, von dir oder einem anderen materiellen Ding. Die Resonanz könnte mich hinterher treffen, aber du wärst auch tot. Sag mir jetzt in Parsel, dass du nicht vorhast, diesen Kreis zu überqueren oder deinen Mantel abzulegen oder irgendetwas Impulsives oder Dummes zu tun. Sag mir, dass du hier unter dem Mantel ruhig warten wirst, bis das hier vorbei ist."

Dies wiederholte Harry.

Dann verwandelte sich Professor Quirrells Umhang in eine schwarze, goldfarbene Robe, wie sie auch Dumbledore bei einem feierlichen Anlass tragen könnte, und Professor Quirrell richtete seinen eigenen Zauberstab auf seinen Kopf. Professor Quirrell blieb lange Zeit regungslos sitzen, den Zauberstab immer noch an seinen Kopf haltend. Seine Augen waren in Konzentration geschlossen.

Und dann sagte Professor Quirrell: "Confundus."

Sofort änderte sich der Ausdruck des Mannes, der da stand; er blinzelte ein paar Mal wie verwirrt und ließ seinen Zauberstab sinken. Eine tiefe Müdigkeit breitete sich über das Gesicht von Professor Quirrell aus; ohne sichtbare Veränderung wirkten seine Augen älter, die wenigen Falten in seinem Gesicht machten auf sich aufmerksam. Seine Lippen waren zu einem traurigen Lächeln verzogen. Ohne jede Eile schritt der Mann ruhig zum Spiegel hinüber, als hätte er alle Zeit der Welt. Er trat in den Reflexionsbereich des Spiegels, ohne dass etwas geschah, und starrte in die Oberfläche. Was der Mann dort sehen mochte, konnte Harry nicht sagen; ihm schien, dass die flache, perfekte Oberfläche immer noch den Raum dahinter reflektierte, wie ein Portal zu einem anderen Ort.

"Ariana", hauchte der Mann. "Mutter, Vater. Und du, mein Bruder, es ist vollbracht." Der Mann stand still, als würde er zuhören. "Ja, es ist vollbracht", sagte der Mann. "Voldemort kam vor diesen Spiegel und wurde durch Merlins Methode gefangen. Er ist jetzt nur noch ein versiegelter Schrecken."

Wieder die lauschende Stille.

"Ich wünschte, ich könnte dir gehorchen, mein Bruder, aber es ist besser so."\\ Der Mann senkte den Kopf.\\ "Der Tod ist ihm verwehrt, für immer; diese Rache ist schrecklich genug."

Harry spürte einen Stich, als er das sah, ein Gefühl, dass dies nicht das war, was Dumbledore gesagt hätte, es schien eher ein Strohmann zu sein, ein oberflächliches Klischee… aber das war auch nicht der wirkliche Geist von Aberforth, das war das, was Professor Quirrell sich vorstellte, wie Dumbledore sich Aberforth vorstellte, und dieses doppelt gespiegelte Bild von Aberforth würde nichts bemerken, was nicht stimmt…\\ "Es ist an der Zeit, den Stein der Weisen zurückzugeben", sagte der Mann, der dachte, er sei Dumbledore. "Er muss jetzt zurück in Meister Flamels Obhut."\\ Lauschende Stille.\\ "Nein", sagte der Mann, "Meister Flamel hat ihn all die Jahre vor allen bewahrt, die nach Unsterblichkeit streben, und ich denke, in seinen Händen ist er am sichersten … nein, Aberforth, ich denke, seine Absichten sind gut."

Harry konnte die Spannung, die ihn wie ein stromführender Draht durchströmte, nicht kontrollieren; er hatte Mühe zu atmen. \emph{Unvollkommen}, Professor Quirrells Confundus-Zauber war unvollkommen gewesen. Die unterschwellige Persönlichkeit von Professor Quirrell sickerte durch und er sah die offensichtliche Frage, warum es für Nicholas Flamel selbst in Ordnung war, den Stein zu haben, wenn Unsterblichkeit so schrecklich war. Selbst wenn Professor Quirrell sich vorstellte, dass Dumbledore für diese Frage blind war, hatte Professor Quirrell keine Klausel in den Confundus aufgenommen, die besagte, dass Dumbledores Bild von Aberforth nicht daran denken würde; und all dies war letztlich eine Reflexion von Professor Quirrells eigenem Geist, ein Bild aus dem Inneren der Intelligenz von Tom Riddle…

"Es zerstören?", fragte der Mann. "Vielleicht. Ich bin mir nicht sicher, ob es zerstört werden kann, sonst hätte es Meister Flamel schon längst getan. Ich glaube, er hat es schon oft bereut, dass er es gemacht hat… Aberforth, ich habe es ihm versprochen, und wir sind selbst nicht so alt oder weise. Der Stein der Weisen muss zurück in die Obhut desjenigen, der ihn gemacht hat."

Und Harrys Atem stockte.

Der Mann hielt einen unregelmäßigen Brocken scharlachroten Glases in seiner linken Hand, vielleicht so groß wie Harrys Daumen vom Fingernagel bis zum ersten Gelenk. Die schimmernde Oberfläche des scharlachroten Glases ließ es feucht erscheinen; es sah aus wie Blut, das in der Zeit schwebte und eine zerklüftete Oberfläche bildete.

"Ich danke dir, mein Bruder", sagte der Mann leise. Ist es das, wie der Stein aussehen sollte? Weiß Professor Quirrell, wie der wahre Stein aussehen sollte? Wird der Spiegel unter diesen Bedingungen den echten Stein zurückgeben, oder eine Nachahmung herstellen und diese zurückgeben? Und dann -

"Nein, Ariana", sagte der Mann und lächelte sanft, "ich fürchte, ich muss jetzt gehen. Sei geduldig, meine Liebste, es wird früh genug sein, dass ich mich dir in Wahrheit anschließe… warum? Warum, ich bin mir nicht sicher, warum ich gehen muss… wenn ich den Stein in der Hand halte, soll ich den Spiegel verlassen und darauf warten, dass Meister Flamel mich kontaktiert, aber ich bin mir nicht sicher, warum ich den Spiegel verlassen muss, um das zu tun…"\\ Der Mann seufzte.\\ "Ah, ich werde alt. Es ist gut, dass dieser schreckliche Krieg zu Ende gegangen ist, als er es tat. Ich nehme an, es kann nicht schaden, wenn ich eine Zeit lang mit dir spreche, meine Liebste, wenn du es so wünschst."

Hinter Harrys Augen begannen Kopfschmerzen zu entstehen; irgendein Teil von Harry versuchte, eine Nachricht darüber zu senden, dass er schon eine Weile nicht mehr geatmet hatte, aber niemand hörte zu. \emph{Unvollkommen}, \emph{Professor Quirrells Konfundus-Zauber war unvollkommen gewesen, Professor Quirrells Abbild von Dumbledores Abbild von Ariana wollte mit Dumbledore sprechen und vielleicht nicht warten, weil Professor Quirrell auf irgendeiner Ebene wusste, dass es nicht wirklich ein Leben nach dem Tod gab, und der zuvor implantierte Impuls, nach dem Erhalt des Steins zu gehen, hielt Riddle-Arianas Argumenten nicht stand…}

Und dann spürte Harry, wie er ganz ruhig wurde. Er begann wieder zu atmen. Wie auch immer, es gab nicht viel, was Harry dagegen tun konnte. Professor Quirrell hatte Harry davon abgehalten, einzugreifen; nun, Professor Quirrell konnte gerne die Konsequenzen dieser Entscheidung ernten. Wenn die Konsequenzen auch Harry trafen, dann sei es so.

Der Mann, der sich für Dumbledore hielt, nickte meist geduldig, manchmal antwortete er seiner liebsten Schwester. Manchmal warf der Mann einen unruhigen Blick zur Seite; als ob er einen starken Impuls verspürte zu gehen, aber diesen Impuls mit der großen Geduld und Höflichkeit und Sorge um seine Schwester unterdrückte, die Professor Quirrell sich bei Albus Dumbledore vorstellte.

Harry sah es in dem Moment, als der Confundus nachließ und der Ausdruck des Mannes sich veränderte und wieder das Gesicht von Professor Quirrell wurde.

Und im selben Augenblick veränderte sich der Spiegel und zeigte Harry nicht mehr das Spiegelbild des Raumes, sondern die Gestalt des echten Albus Dumbledore, so als stünde er direkt hinter dem Spiegel und wäre durch ihn hindurch sichtbar.

Das Gesicht des echten Dumbledore war ernst und grimmig.

"Hallo, Tom", sagte Albus Dumbledore.

