

\hypertarget{was-sich-zu-beschuxfctzen-lohnt-hermine-granger}{% \section{121. Was sich zu beschützen lohnt: Hermine Granger}\label{was-sich-zu-beschuxfctzen-lohnt-hermine-granger}}

\textbf{\uline{Was sich zu beschützen lohnt: Hermine Granger}}

\emph{Anm. des Übersetzers:}

\emph{Dies ist das letzte Kapitel. Es war eine lange Reise, und hoffe das es jedem der bis hierher durchgehalten hat gefallen hat.}

Und es war Abend und es war Morgen, der letzte Tag. 15. Juni 1992.

Das beginnende Licht des Morgens, die Vor-Dämmerung vor dem Sonnenaufgang, erhellte den Himmel kaum. Östlich von Hogwarts, wo die Sonne aufgehen würde, machte dieser schwache Grauton den hügeligen Horizont jenseits der Quidditch-Tribüne kaum sichtbar. Die steinerne Terrassenplattform, auf der Harry jetzt saß, würde hoch genug sein, um die Morgendämmerung jenseits der Hügel zu sehen; das hatte er sich gewünscht, als er sein neues Büro beschrieben hatte. Harry saß gerade im Schneidersitz auf einem Kissen, eine kühle Morgenbrise strich über seine ungeschützten Hände und sein Gesicht. Er hatte den Hauselfen befohlen, den handgeschmückten Thron aus seinem früheren Büro als General Chaos heraufzuholen … und dann hatte er den Elfen gesagt, sie sollten ihn wieder zurückbringen, nachdem es Harry in den Sinn gekommen war, sich Gedanken darüber zu machen, woher sein Geschmack für Innendekorationen stammte und ob Voldemort einst einen ähnlichen Thron besessen hatte. Was an sich kein schlagendes Argument war - es war ja nicht so, dass das Sitzen auf einem glitzernden Thron, um die Ländereien unterhalb von Hogwarts zu überblicken, in irgendeiner Weise unethisch war, wie Harrys Moralphilosophie es erkennen konnte -, aber Harry hatte beschlossen, dass er sich Zeit nehmen und es durchdenken musste. In der Zwischenzeit würden einfache Kissen genügen.

Im Raum darunter, der durch eine einfache Holzleiter mit dem Dach verbunden war, befand sich Harrys neues Büro in Hogwarts. Ein großer Raum, der auf vier Seiten von Fenstern umgeben war, die das Sonnenlicht hereinließen, und der bis auf vier Stühle und einen Schreibtisch noch nicht eingerichtet war. Harry hatte Schulleiterin McGonagall gesagt, was er suchte, und Schulleiterin McGonagall hatte den Sprechenden Hut aufgesetzt und Harry dann die Reihe von Drehungen und Wendungen erklärt, die ihn dorthin bringen würden, wo er sein wollte. Hoch genug in Hogwarts, dass das Schloss nicht so hoch sein sollte, hoch genug in Hogwarts, dass niemand, der von außen hinsah, ein Stück des Schlosses sehen würde, das dem entsprach, wo Harry jetzt saß. Es schien eine elementare Vorsichtsmaßnahme gegen Scharfschützen zu sein, gegen die es keinen Grund gab, sie nicht zu treffen. Andererseits hatte Harry aber auch keine Ahnung, wo er sich im Moment wirklich befand. Wenn sein Büro vom Boden aus nicht zu sehen war, wie konnte Harry dann den Boden sehen, wie schafften es Photonen von der Landschaft zu ihm?

Auf der westlichen Seite des Horizonts glitzerten immer noch Sterne, klar in der Luft vor der Morgendämmerung. Waren diese Photonen die tatsächlichen Photonen, die von riesigen Plasmaöfen in der unvorstellbaren Ferne ausgestoßen worden waren? Oder saß Harry jetzt in einer träumerischen Vision des Schlosses Hogwarts? Oder war das alles, ohne weitere Erklärung, "nur Magie"? Er musste die Elektrizität dazu bringen, besser mit der Magie umzugehen, damit er mit dem Abwärts- und Aufwärtsstrahlen von Lasern experimentieren konnte.

Und ja, Harry hatte jetzt sein eigenes Büro auf Hogwarts. Er hatte zwar noch keinen offiziellen Titel, aber der Junge-der-lebte war jetzt ein fester Bestandteil der Hogwarts-Schule für Hexerei und Zauberei, der baldigen Heimat des Steins der Weisen und der einzigen höheren Bildungseinrichtung der Welt für Zauberer. Es war nicht vollständig gesichert, aber Professor Vector hatte einige vorläufige Zaubersprüche und Runen angebracht, um das Büro und sein Dach gegen Lauschangriffe abzuschirmen.

Harry saß auf seinem Kissen in der Nähe der Dachkante seines Büros und blickte auf Bäume, Seen und blühendes Gras hinunter. Weit unten saßen bewegungslos Kutschen, die noch nicht an die skelettierten Pferde angeschirrt waren. Kleine Boote lagen am Ufer, bereit, die jüngeren Schüler über den See zu bringen, wenn die Zeit gekommen war. Der Hogwarts-Express war über Nacht angekommen, und nun warteten die Waggons und die riesige altmodische Lokomotive auf der anderen Seite des südlichen Sees.Alles war bereit, um die Schüler nach dem Abschiedsfest am Morgen nach Hause zu bringen. Harry starrte über den See, auf die große altmodische Lokomotive, mit der er diesmal nicht nach Hause fahren würde. Schon wieder. Es lag eine seltsame Traurigkeit und Sorge in diesem Gedanken, als ob Harry bereits anfing, die verbindenden Erfahrungen mit den anderen Schülern seines Alters zu verpassen - wenn man das überhaupt sagen konnte, wo doch ein bedeutender Teil von Harry im Jahr 1926 geboren worden war. Gestern Abend im Ravenclaw-Gemeinschaftsraum hatte es sich für Harry so angefühlt, als hätte sich die Kluft zwischen ihm und den anderen Schülern sogar noch vergrößert. Obwohl das vielleicht nur an den Fragen lag, die sich Padma Patil und Anthony Goldstein aufgeregt über das Mädchen, das wiederbelebt wurde, gestellt hatten, an den schnellen Spekulationen, die von Ravenclaw zu Ravenclaw durch die Luft schossen. Harry hatte die Antworten gewusst, er hatte alle Antworten gewusst, und er war nicht in der Lage gewesen, sie auszusprechen. Ein Teil von Harry war versucht, mit dem Hogwarts-Express zu fahren und dann per Floo nach Hogwarts zurückzukommen. Aber wenn Harry sich vorstellte, fünf andere Schüler für sein Abteil zu finden und dann die nächsten acht Stunden damit zu verbringen, Geheimnisse vor Neville oder Padma oder Dean oder Tracey oder Lavender zu haben… das schien keine attraktive Aussicht zu sein. Harry hatte das Gefühl, dass er es aus Gründen der Geselligkeit mit den anderen Kindern tun sollte, aber er wollte es nicht tun. Er konnte sich mit allen zu Beginn des nächsten Schuljahres wieder treffen, wenn es andere Themen geben würde, über die er freier sprechen konnte.

Harry starrte nach Süden über den See, auf die riesige alte Lokomotive, und dachte über den Rest seines Lebens nach. Über die Zukunft. Die Prophezeiung, die Dumbledore in seinem Brief erwähnt hatte, dass er die Sterne am Himmel zerreißen würde … nun, das klang optimistisch. Dieser Teil hatte eine offensichtliche Interpretation für jeden, der mit der richtigen Art von Erziehung aufgewachsen war.

Er beschrieb eine Zukunft, in der die Menschheit gewonnen hatte, mehr oder weniger. Es war nicht das, woran Harry normalerweise dachte, wenn er die Sterne betrachtete, aber aus einer wirklich erwachsenen Perspektive betrachtet, waren die Sterne riesige Haufen wertvoller Rohstoffe, die unglücklicherweise Feuer gefangen hatten und nun verteilt und gelöscht werden mussten. Wenn man die riesigen Wasserstoff-Helium-Reservoirs nach Rohstoffen anzapfte, bedeutete das, dass die eigene Spezies erfolgreich erwachsen geworden war. Es sei denn, die Prophezeiung hatte sich auf etwas ganz anderes bezogen. Dumbledore könnte die Worte eines Sehers falsch interpretiert haben… aber seine Nachricht an Harry war so formuliert worden, als gäbe es eine Prophezeiung darüber, dass Harry persönlich in absehbarer Zeit Sterne zerreißen würde. Das schien potenziell besorgniserregender zu sein, obwohl es keineswegs sicher war, dass es wahr war, oder eine schlechte Sache, wenn es wahr war… Harry stieß einen Seufzer aus. In den langen Stunden, bevor ihn der Schlaf letzte Nacht ereilte, hatte er begonnen zu verstehen, was Dumbledores letzte Nachricht bedeutete.

Wenn er auf die Ereignisse des Schuljahres 1991-1992 in Hogwarts zurückblickte, war es geradezu erschreckend, jetzt, wo Harry verstand, was er da sah. Es war nicht nur so, dass Harry häufig in der Gesellschaft seines guten Freundes Lord Voldemort gewesen war. Es war nicht einmal hauptsächlich das. Es war die Vision einer schmalen Linie der Zeit, die Albus Dumbledore durch das enge Schlüsselloch des Schicksals gelenkt hatte, ein haarfeiner Strang der Möglichkeit, der durch ein Nadelöhr gefädelt war. Die Prophezeiungen hatten Dumbledore angewiesen, Tom Riddles Intelligenz in das Gehirn eines Zaubererkindes zu kopieren, das dann mit der Muggelwissenschaft aufwachsen sollte. Was sagte es über die wahrscheinliche Gestalt der Zukunft aus, wenn das die erste oder beste Strategie war, die die Seher finden konnten, die nicht zur Katastrophe führte? Harry konnte jetzt auf den unbrechbaren Schwur zurückblicken, den er geleistet hatte, und er konnte erahnen, dass ohne diesen Schwur die Katastrophe vielleicht schon gestern in Gang gesetzt worden wäre, als Harry das Internationale Geheimhaltungsstatut niederreißen wollte.

Was wiederum stark darauf hindeutete, dass die vielen Prophezeiungen, die Dumbledore gelesen hatte und deren Anweisungen er befolgt hatte, irgendwie dafür gesorgt hatten, dass Harry und Voldemort auf genau die richtige Weise aufeinandertreffen würden, um Voldemort dazu zu bringen, Harry zu zwingen, den Unbrechbaren Schwur abzulegen. Der Unbrechbare Schwur war Teil des engen Schlüssellochs der Zeit gewesen, eine der unwahrscheinlichen Voraussetzungen dafür, dass die Völker der Erde überleben konnten. Ein Schwur, dessen einziger Zweck es war, alle vor Harrys gegenwärtiger Dummheit zu schützen. Es war, als würde man sich ein Video von einem Beinahe-Unfall ansehen, bei dem man sich daran erinnerte, dass ein anderes Auto einen um Zentimeter verfehlt hatte, und das Video zeigte, dass auch jemand einen Kieselstein genau so geworfen hatte, dass ein riesiger Lastwagen diese Beinahe-Kollision verfehlte, und wenn man diesen Kieselstein nicht geworfen hätte, dann wärst du und deine ganze Familie im Auto und dein ganzer Planet von dem Lastwagen getroffen worden, der in der Metapher deine eigene schiere Vergesslichkeit darstellte. Harry war gewarnt worden, er hatte es auf irgendeiner Ebene gewusst, sonst hätte ihn der Schwur nicht aufgehalten, und trotzdem hätte er fast die falsche Entscheidung getroffen und die Welt zerstört. Harry konnte jetzt zurückblicken und sehen, dass, ja, der alternative Harry ohne Schwur Probleme gehabt hätte, die Argumentation zu akzeptieren, die besagte, dass man Muggeln nicht so schnell wie möglich magische Heilung zukommen lassen konnte.

Wenn der Ersatz-Harry die Gefahr überhaupt erkannt hätte, hätte er sie rationalisiert, versucht, einen cleveren Weg um das Problem herum zu finden und sich geweigert, zu akzeptieren, dass es ein paar Jahre länger dauert, und so wäre die Welt untergegangen. Selbst nach all den Warnungen, die Harry erhalten hatte, hätte es ohne den Unbrechbaren Schwur nicht funktioniert.

\emph{Ein winziger Strang der Zeit, der durch ein Nadelöhr gefädelt wurde.}

Harry wusste nicht, wie er mit dieser Enthüllung umgehen sollte. Es war keine Situation, für die menschliche Emotionen entwickelt worden waren. Alles, was Harry tun konnte, war, darauf zu starren, wie nahe er der Katastrophe gekommen war, wie nahe er der Katastrophe wieder kommen könnte, wenn dieser Schwur mehr als einmal ausgelöst werden sollte, und zu denken… denken… 'Ich will nicht, dass das noch einmal passiert', schien nicht der richtige Gedanke zu sein. Er hatte die Welt von vornherein nicht zerstören wollen. Harry fehlte es nicht an Schutzgefühlen gegenüber der intelligenten Bevölkerung der Erde, diese Schutzgefühle waren gewissermaßen das Problem. Was Harry fehlte, war eine klare Vision, die Bereitschaft, sich bewusst einzugestehen, was er tief im Inneren bereits wusste. Und die ganze Sache, dass Harry das letzte Jahr damit verbracht hatte, sich bei dem Verteidigungsprofessor einzuschmeicheln, sprach auch nicht gerade für seinen Intellekt. Es schien sogar auf das gleiche Problem hinzuweisen. Es gab Dinge, die Harry auf irgendeiner Ebene gewusst oder stark vermutet hatte, aber nie zu bewusster Aufmerksamkeit gebracht hatte. Und so hatte er versagt und war fast gestorben.

\emph{Ich muss das Niveau meines Spiels anheben.}

Das war der Gedanke, den Harry suchte. Er musste es besser machen als das, ein weniger dummer Mensch werden als das.

\emph{Ich muss das Niveau meines Spiels heben, oder ich werde versagen.}

Dumbledore hatte die Aufzeichnungen in der Halle der Prophezeiung vernichtet und veranlasst, dass keine weiteren Aufnahmen gemacht wurden. Offenbar gab es eine Prophezeiung, die besagte, dass Harry nicht in diese Prophezeiungen schauen darf.

Und der naheliegende nächste Gedanke, der wahr sein könnte oder auch nicht, war, dass die Rettung der Welt jenseits der Reichweite der prophetischen Anweisungen lag.´Dass für den Sieg Pläne nötig wären, die zu komplex für die Botschaften der Seher waren oder die die Weissagung irgendwie nicht sehen konnte. Wenn es einen Weg für Dumbledore gegeben hätte, die Welt selbst zu retten, dann hätte die Prophetie Dumbledore wahrscheinlich gesagt, wie er das tun sollte. Stattdessen hatten die Prophezeiungen Dumbledore gesagt, wie er die Voraussetzungen dafür schaffen konnte, dass eine bestimmte Art von Person existierte; eine Person vielleicht, die eine schwierigere Aufgabe lösen konnte, als die Prophezeiung direkt lösen konnte. Deshalb war Harry auf sich allein gestellt, um ohne prophetische Führung zu denken. Wenn Harry nur mysteriöse Befehle von Prophezeiungen befolgte, dann würde er nicht zu einer Person heranreifen, die diese unbekannte Aufgabe erfüllen könnte. Und im Moment war Harry James Potter-Evans-Verres immer noch eine wandelnde Katastrophe, die durch einen unbrechbaren Schwur daran gehindert werden musste, die Erde sofort auf einen unausweichlichen Kurs in Richtung Zerstörung zu bringen, obwohl er bereits davor gewarnt worden war. Das war buchstäblich gestern passiert, nur einen Tag nachdem er Voldemort geholfen hatte, den Planeten fast zu übernehmen. Eine bestimmte Zeile aus Tolkien ging Harry immer wieder durch den Kopf, die Stelle, an der Frodo am Schicksalsberg den Ring anlegt und Sauron plötzlich erkennt, was für ein kompletter Idiot er gewesen war.

\emph{'Und das Ausmaß seiner eigenen Torheit wurde endlich offengelegt',}

oder wie auch immer das gelaufen war. Es klaffte eine riesige Lücke zwischen dem, was Harry werden sollte, und dem, was er jetzt war. Und Harry glaubte nicht, dass Zeit, Lebenserfahrung und Pubertät das automatisch ausgleichen würden, obwohl sie vielleicht helfen würden. Obwohl, wenn Harry zu einem Erwachsenen heranwachsen könnte, der für dieses Ich das war, was ein normaler Erwachsener für einen normalen Elfjährigen war, würde das vielleicht ausreichen, um durch das enge Schlüsselloch der Zeit zu steuern…

Er musste erwachsen werden, irgendwie, und es gab keinen traditionellen Weg, der ihm dafür vorgezeichnet war. Da kam Harry der Gedanke an ein anderes belletristisches Werk, das noch obskurer war als Tolkien:

\emph{Du kannst nur zur Meisterschaft gelangen, indem du die Techniken, die du gelernt hast übst, dich Herausforderungen stellst und sie annimmst, indem du die Werkzeuge, die dir beigebracht wurden, bis zum Äußersten einsetzt, bis sie in deinen Händen zerbrechen und du inmitten absoluter Trümmer zurückbleibst… Ich kann keine Meister erschaffen. Ich habe nie gewusst, wie man Meister erschafft. Also geh und versage… Du wurdest zu etwas geformt, das aus den Trümmern auftauchen kann, entschlossen, deine Kunst neu zu erschaffen. Ich kann keine Meister erschaffen, aber wenn du nicht gelehrt worden wärst, wären deine Chancen geringer. Der höhere Weg beginnt, nachdem die Kunst dich zu enttäuschen scheint; obwohl die Realität sein wird, dass du es warst, der deine Kunst enttäuscht hat.}

Es war nicht so, dass Harry den falschen Weg eingeschlagen hatte, es war nicht so, dass der Weg zur Vernunft irgendwo außerhalb der Wissenschaft lag. Aber wissenschaftliche Abhandlungen zu lesen, hatte nicht gereicht. All die kognitionspsychologischen Abhandlungen über bekannte Fehler im menschlichen Gehirn und so weiter hatten geholfen, aber sie waren nicht ausreichend gewesen. Er hatte es nicht geschafft, das zu erreichen, was Harry allmählich als schockierend hohen Standard erkannte: so unglaublich rational zu sein, dass man anfing, Dinge richtig zu machen, im Gegensatz zu einer praktischen Sprache, mit der man hinterher alles beschreiben konnte, was man gerade falsch gemacht hatte. Harry konnte jetzt zurückblicken und Ideen wie "motivierte Kognition" anwenden, um zu sehen, wo er im letzten Jahr in die Irre gegangen war. Das zählte etwas, wenn es darum ging, in Zukunft vernünftiger zu sein. Das war besser, als keine Ahnung zu haben, was er falsch gemacht hatte. Aber das war noch nicht die Person, die durch das enge Schlüsselloch der Zeit gehen konnte, die erwachsene Form, deren Möglichkeit Dumbledore von Sehern angewiesen worden war, zu schaffen.

\emph{\hfill\break Ich muss schneller denken, schneller erwachsen werden…}

\emph{Wie allein bin ich, wie allein werde ich sein?}

\emph{Mache ich denselben Fehler, den ich bei Professor Quirrells erstem Kampf gemacht habe, als ich nicht erkannte, dass Hermine kompetente Offiziere hat? Den Fehler, den ich machte, als ich Dumbledore nichts von der Untergangsstimmung erzählte, als ich merkte, dass Dumbledore wahrscheinlich nicht verrückt oder böse ist? Es würde helfen, wenn Muggel Kurse für solche Dinge hätten, aber das haben sie nicht.}

Vielleicht könnte Harry Daniel Kahneman rekrutieren, seinen Tod vortäuschen, ihn mit dem Stein verjüngen und ihn damit beauftragen, bessere Trainingsmethoden zu erfinden…

Harry holte den Elderstab aus seinem Umhang, betrachtete erneut das dunkelgraue Holz, das Dumbledore ihm vererbt hatte. Harry hatte versucht, dieses Mal schneller zu denken, er hatte versucht, das Muster zu vervollständigen, das der Umhang der Unsichtbarkeit und der Stein der Auferstehung implizierten. Der Umhang der Unsichtbarkeit hatte die legendäre Macht, den Träger zu verbergen, und die verborgene Macht, dem Träger zu erlauben, sich vor dem Tod selbst in Form von Dementoren zu verstecken. Der Stein der Auferstehung hatte die legendäre Macht, ein Abbild der Toten zu beschwören, und dann hatte Voldemort ihn in sein Horkrux-System eingebaut, um seinem Geist zu erlauben, sich frei zu bewegen. Der zweite Heiligtum des Todes war ein potenzieller Bestandteil eines Systems wahrer Unsterblichkeit, das Cadmus Peverell nie vollendet hatte, vielleicht weil er Moralvorstellungen hatte. Und dann war da noch das dritte Heiligtum des Todes,

der Elderstab von Antioch Peverell, der der Legende nach von Zauberer zu stärkerem Zauberer weitergegeben wurde und seinen Besitzer unbesiegbar gegen gewöhnliche Angriffe machte; das war die bekannte und offenkundige Eigenschaft… Der Elderstab, der Dumbledore gehört hatte, der versucht hatte, den Tod der Welt selbst zu verhindern. Der Zweck des Elderstabs, der immer an den Sieger ging, könnte darin bestehen, den stärksten lebenden Zauberer zu finden und ihn noch weiter zu stärken, für den Fall, dass es eine Bedrohung für ihre gesamte Spezies gab; er könnte insgeheim ein Werkzeug sein, um den Tod in seiner Form als Zerstörer der Welten zu besiegen. Aber wenn es eine höhere Macht gab, die im Elderstab eingeschlossen war, so hatte sie sich Harry nicht auf Grund dieser Vermutung gezeigt. Harry hatte den Elderstab erhoben und mit ihm gesprochen, er hatte sich als Nachkomme Peverells bezeichnet, der die Aufgabe seiner Familie angenommen hatte; er hatte dem Elderstab versprochen, dass er sein Bestes tun würde, um die Welt vor dem Tod zu retten und Dumbledores Aufgabe zu übernehmen. Und der Elderstab hatte nicht stärker als zuvor auf seine Hand geantwortet und seinen Versuch abgelehnt, in der Geschichte weiterzuspringen. Vielleicht musste Harry seinen ersten richtigen Schlag gegen den Tod der Welten ausführen, bevor der Elderstab ihn anerkennen würde; so wie der Erbe von Ignotus Peverell bereits den Schatten des Todes besiegt hatte und der Erbe von Cadmus Peverell bereits den Tod seines Körpers überlebt hatte, als ihre jeweiligen Heiligtümer des Todes ihre Geheimnisse enthüllt hatten. Immerhin war es Harry gelungen, zu erraten, dass der Elderstab entgegen der Legende keinen Kern aus

"Thestralhaaren" enthielt. Harry hatte Thestrale gesehen, und sie waren skelettartige Pferde mit glatter Haut und ohne sichtbare Mähne auf ihren schädelähnlichen Köpfen und ohne Büschel an ihren knochigen Schwänzen. Aber welcher Kern sich wirklich im Inneren des Elderstabs befand, hatte Harry noch nicht zu wissen geglaubt; auch hatte er nirgendwo auf dem Elderstab die Kreis-Dreieck-Linie der Heiligtümer des Todes finden können, die vorhanden sein sollte.

"Ich nehme nicht an", murmelte Harry zum Elderstab, "dass du es mir einfach sagen könntest?"

Es kam keine Antwort von dem kugelförmigen Zauberstab zurück; nur ein Gefühl von Herrlichkeit und gebändigter Macht, das ihn skeptisch beobachtete.

Harry seufzte und steckte den mächtigsten Zauberstab der Welt zurück in seinen Schulmantel. Er würde es schon noch herausfinden, hoffentlich früher als später. Vermutlich früher wenn er jemanden hätte der ihm bei der Forschung helfen würde.

Harry war sich auf irgendeiner Ebene bewusst - \emph{nein, er musste aufhören, sich der Dinge auf irgendeiner Ebene bewusst zu sein und anfangen, sie einfach nur wahrzunehmen -} Harry war sich explizit und bewusst bewusst, dass er über die Zukunft nachdachte, hauptsächlich um sich von der bevorstehenden Ankunft von Hermine Granger abzulenken. Die, wenn sie an diesem Morgen sehr früh aufwachte, von St. Mungo's ein klares Gesundheitszeugnis erhalten würde und die dann mit Professor Flitwick per Floo nach Hogwarts zurückkehren würde. Woraufhin sie Professor Flitwick sagen würde, dass sie sofort mit Harry Potter sprechen müsse. Es hatte eine Notiz von Harry an sich selbst gegeben, als Harry an diesem Morgen, als die Sonne bereits aufgegangen war, im Ravenclaw-Schlafsaal aufgewacht war. Er hatte die Notiz gelesen und dann die Zeit zurückgedreht bis vor die Morgenstunde, in der Hermine Granger eintreffen würde.

\emph{Sie wird nicht wirklich böse auf mich sein … Ernsthaft. Hermine ist nicht diese Art von Mensch. Vielleicht war sie es am Anfang des Jahres, aber jetzt ist sie zu selbstbewusst, um darauf reinzufallen.}

\emph{…}

\emph{Was soll das heißen, "…"? Wenn du etwas zu sagen hast, innere Stimme, dann sag es einfach! Wir versuchen, uns unserer eigenen Gedankengänge bewusster zu werden, schon vergessen?}

Der Himmel hatte sich bereits blau-grau verfärbt, die Morgendämmerung stand kurz vor dem Sonnenaufgang, als Harry das Geräusch von Schritten hörte, die von der Treppe kamen, die in sein neues Büro führte.

Hastig stand Harry auf und begann, seine Roben abzustreifen; und dann, als ihm klar wurde, was er tat, stoppte er die nervösen Bewegungen. Er hatte gerade Voldemort besiegt, verdammt noch mal, er sollte nicht so nervös sein.

Der Kopf der jungen Hexe mit den kastanienbraunen Locken erschien in der Öffnung und spähte herum. Dann stieg sie höher, schien fast die Leitersprossen hinaufzulaufen, als ginge sie einen gewöhnlichen Bürgersteig entlang, aber senkrecht; Harry hätte blinzeln und es übersehen können, wie ihr einer Schuh auf der obersten Sprosse der Leiter landete und sie einen Augenblick später leichtfüßig auf das Dach sprang.

\emph{Hermine}.

Harrys Lippen bewegten sich um das Wort, gaben aber keinen Laut von sich. Es gab etwas, das Harry hatte sagen wollen, aber es war ihm gleich wieder entfallen. Vielleicht eine Viertelminute verging auf dem Dach, bevor Hermine Granger sprach. Sie trug jetzt eine Uniform mit blauem Rand und die blau-bronze-gestreifte Krawatte ihres richtigen Hauses.

"Harry", sagte Hermine Granger mit einer furchtbar vertrauten Stimme, die Harry fast die Tränen in die Augen trieb, "bevor ich dir all die Fragen stelle, möchte ich dir erst einmal vielen Dank sagen für, ähm, was auch immer du getan hast. Ich meine es wirklich ernst. Ich danke dir."

"Hermine", sagte Harry und schluckte. Der Satz "Darf ich dich umarmen?", den Harry sich für seine Eröffnungsrede vorgestellt hatte, konnte er offenbar unmöglich sagen. "Willkommen zurück. Warte, während ich ein paar Sichtschutzzauber aufbaue."

Harry holte den Elderstab aus seinem Umhang, holte ein Buch aus seiner Tasche, das er zu einem Lesezeichen aufschlug, und sprach dann vorsichtig "Homenum Revelio" aus, zusammen mit zwei anderen, kürzlich gelernten Sicherheitszaubern, die Harry kaum sprechen konnte, selbst wenn er den Elderstab benutzte. Es war nicht viel, aber es war eine geringfügig bessere Sicherheit, als sich nur auf Professor Vector zu verlassen.

"Du hast Dumbledores Zauberstab", sagte Hermine. Ihre Stimme war gedämpft und klang so laut wie eine Lawine in der noch dämmrigen Luft. "Und du kannst damit Zaubersprüche aus dem vierten Jahr wirken?"

Harry nickte und machte sich eine mentale Notiz, vorsichtiger zu sein, wer ihn sonst noch dabei sah.

"Ist es okay, wenn ich dich umarme?"

Hermine bewegte sich leicht zu ihm hinüber; ihre Bewegungen waren seltsam schnell, anmutiger als sie es zuvor gewesen waren. Ihre Bewegungen schienen einen Hauch von etwas Reinem und Unberührtem auszustrahlen, was Harry wieder daran erinnerte, wie friedlich Hermine ausgesehen hatte, als sie auf Voldemorts Altar schlief - die Erkenntnis traf Harry wie eine Tonne Ziegelsteine, oder zumindest wie ein Kilogramm Ziegelsteine. Und Harry umarmte Hermine und spürte, wie lebendig sie wirkte. Ihm war zum Weinen zumute, und er unterdrückte es, weil er nicht wusste, ob das nur ihre Aura war, die ihn beeinflusste oder nicht. Hermines Arme um ihn waren sanft, äußerst leicht in ihrem Druck, als ob sie bewusst darauf achtete, seinen Körper nicht wie einen gebrauchten Zahnstocher in zwei Hälften zu brechen.

"Also", sagte Hermine, als Harry sie losgelassen hatte. Ihr junges Gesicht sah sehr ernst, aber auch rein und unschuldig aus. "Ich habe den Auroren nicht gesagt, dass du da warst, oder dass es Professor Quirrell und nicht Du-weißt-schon-wer war, der alle Todesser getötet hat. Professor Flitwick ließ sie mir nur einen Tropfen Veritaserum geben, also musste ich es nicht sagen. Ich habe ihnen nur gesagt, dass der Troll das Letzte war, woran ich mich erinnern konnte."

"Ah", sagte Harry. Er hatte sich irgendwie dabei ertappt, dass er Hermines Nase anstarrte statt ihre Augen. "Was, glaubst du, ist genau passiert?"

"Nun", sagte Hermine Granger nachdenklich, "ich wurde von einem Troll gefressen, was ich ehrlich gesagt lieber nicht noch einmal erleben würde, und dann gab es einen wirklich lauten Knall und meine Beine waren wieder da, und ich lag auf einem steinernen Altar mitten auf einem Friedhof in einem dunklen, mondbeschienenen Friedhof, den ich noch nie zuvor gesehen hatte, und hatte die abgetrennten Hände von jemandem um meine Kehle gepackt. \emph{Sehen Sie, Mr~Potter,} als ich mich in einer so seltsamen, dunklen und unheimlichen Situation wiederfand, wollte ich nicht den gleichen Fehler wie beim letzten Mal mit Tracey machen. Ich wusste sofort, dass \emph{du} es warst."

Harry nickte. "Gut geraten."

"Ich habe deinen Namen gesagt, aber du hast nicht geantwortet", sagte Hermine. "Ich setzte mich auf, und eine der blutigen Hände glitt über mein Hemd hinunter und hinterließ kleine Fleischstückchen. Ich habe aber nicht geschrien, auch nicht, als ich mich umschaute und all die Köpfe und Leichen sah und merkte, was der Geruch war." Hermine hielt inne und holte noch einmal tief Luft. "Ich sah die Totenkopfmasken und erkannte, dass die toten Menschen Todesser gewesen waren. Ich wusste sofort, dass der Verteidigungsprofessor mit dir dort war und sie alle getötet hat, aber ich habe nicht bemerkt, dass Professor Quirrells Leiche auch dort lag. Ich habe nicht mal gemerkt, dass er es war, als ich sah, wie Professor Flitwick die Leiche untersuchte. Er sah … anders aus, als er tot war."

Hermine's Stimme wurde leiser. Sie sah irgendwie demütig aus, auf eine Art, die Harry nicht oft gesehen hatte.

"Sie sagten, David Monroe hätte sein Leben geopfert, um mich zurückzubringen, so wie deine Mutter sich für dich geopfert hat, damit der Dunkle Lord wieder explodiert, wenn er versucht, mich zu berühren. Ich bin mir ziemlich sicher, dass das nicht die ganze Wahrheit ist, aber… ich habe eine Menge böser Dinge über unseren Verteidigungsprofessor gedacht, die ich niemals hätte denken sollen."

"Ähm", sagte Harry.

Hermine nickte feierlich, die Hände wie zur Buße vor sich verschränkt.

"Ich weiß, dass du wahrscheinlich zu nett bist, um mir die Dinge zu sagen, die du jetzt sagen müsstest, also werde ich sie für dich sagen, Harry. Du hattest Recht, was Professor Quirrell angeht, und ich hatte Unrecht. Du hast es mir gesagt. David Monroe war ein bisschen dunkel und eine ganze Menge Slytherin, und es war kindisch von mir zu denken, dass das dasselbe ist wie böse zu sein."

"Ah…" sagte Harry. Es war sehr schwer, das zu sagen. "Eigentlich weiß der Rest der Welt diesen Teil nicht, nicht einmal die Schulleiterin. Aber in der Tat hattest du hundertzwölf Prozent recht damit, dass er böse ist, und ich werde mir für die Zukunft merken, dass, obwohl 'dunkel' und 'böse' technisch gesehen nicht dasselbe sind, es eine große statistische Korrelation gibt."

"Oh", sagte Hermine und verstummte wieder.

"Du willst doch nicht etwa sagen, dass du mir das gesagt hast?", fragte Harry.

Sein mentales Modell von Hermine schrie auf:

\emph{ICH HABE ES DIR GESAGT!}

\emph{HABE ICH ES IHNEN NICHT GESAGT, MR. POTTER? HABE ICH ES IHNEN NICHT GESAGT?}

\emph{PROFESSOR QUIRRELL IST BÖÖÖÖÖSE, HABE ICH GESAGT, ABER DU HAST MIR NICHT ZUGEHÖRT!}

Die echte Hermine schüttelte nur den Kopf. "Ich weiß, dass er dir sehr am Herzen lag", sagte sie leise. "Da ich ja schließlich recht hatte… Ich wusste, dass du wahrscheinlich sehr verletzt sein würdest, nachdem sich Professor Quirrell als böse herausgestellt hat, und dass es kein guter Zeitpunkt wäre, zu sagen, dass ich es dir gesagt habe. Ich meine, das habe ich beschlossen, als ich diesen Teil einige Monate zuvor durchdachte."

\emph{Danke, Miss~Granger.}

Harry war froh, dass sie das gesagt hatte, denn sonst hätte es sich nicht nach Hermine angefühlt.

"Also, Mr~Potter", sagte Hermine Granger und tippte mit den Fingern etwa auf Höhe der Oberschenkel auf ihren Bademantel. "Nachdem die Medizinhexe mir Blut abgenommen hatte, hörte es sofort auf, weh zu tun, und als ich das bisschen Blut an meinem Arm wegwischte, konnte ich nicht finden, wo mich die Nadel gestochen hatte. Ich habe etwas von dem Metall in meinem Bettgestell verbogen, ohne mich groß anzustrengen, und obwohl ich noch keine Gelegenheit hatte, es zu testen, habe ich das Gefühl, dass ich in der Lage sein sollte, wirklich schnell zu laufen. Meine Fingernägel sind perlweiß und glänzend, obwohl ich mich nicht daran erinnern kann, sie lackiert zu haben. Und meine Zähne sehen auch so aus, was mich, als Tochter von Zahnärzten, nervös macht. Es ist also nicht so, dass ich undankbar wäre, aber was genau hast du gemacht?"

"Ähm", sagte Harry. "Und ich nehme an, du fragst dich auch, warum du eine Aura der Reinheit und Unschuld ausstrahlst?"

"Ich tue WAS?"

"Der Teil war nicht meine Idee. Ehrlich gesagt." Harrys Stimme wurde leise. "Bitte bring mich nicht um."

Hermine Granger hob die Hände vor ihr Gesicht und starrte etwas schielend auf ihre Finger. "Harry, willst du damit sagen … Ich meine, meine strahlende Unschuld und dass ich ganz schnell und anmutig bin und dass meine Zähne perlweiß sind … ist es Alicorn, aus dem meine Fingernägel gemacht sind?"

"Alicorn?"

"Das ist die Bezeichnung für das Horn eines Einhorns, Mr~Potter."

Hermine Granger schien zu versuchen, an ihren Fingernägeln zu knabbern, und hatte nicht viel Glück. "Also, ich schätze, wenn man ein Mädchen von den Toten zurückholt, endet sie als, wie nannte Daphne es, eine funkelnde Einhornprinzessin?"

"Das ist nicht genau das, was passiert ist", sagte Harry, obwohl es erschreckend nah dran war. Hermine nahm ihren Finger aus dem Mund und betrachtete ihn stirnrunzelnd. "Ich kann da auch nicht durchbeißen. Mr~Potter, haben Sie an die Probleme gedacht, die sich daraus ergeben, dass es mir buchstäblich unmöglich ist, meine Finger- und Fußnägel zu kürzen?"

"Die Weasley-Zwillinge haben ein magisches Schwert, das funktionieren sollte", meldete sich Harry.

"Ich denke", sagte Hermine Granger entschieden, "dass ich gerne die ganze Geschichte hinter all dem kennen würde, Mr~Potter. Denn wie ich dich kenne und wie ich Professor Quirrell kenne, gab es eine Art Plan, der dahintersteckt."

Harry nahm einen tiefen Atemzug. Dann atmete er aus. "Tut mir leid, das ist… geheim. Ich könnte es dir sagen, wenn du Okklumentik studieren würdest, aber… willst du das?"

"Ob ich Okklumentik studieren will?" sagte Hermine und sah leicht überrascht aus. "Das ist doch mindestens auf der Stufe der sechsten Klasse, oder?"

"Ich habe es gelernt", sagte Harry. "Ich habe mit einer ungewöhnlichen Förderung angefangen, aber ich bezweifle, dass das auf lange Sicht wirklich von Bedeutung war. Ich meine, ich bin mir sicher, dass man Infinitesimalrechnung lernen kann, wenn man fleißig lernt, unabhängig davon, in welchem Alter Muggel es normalerweise lernen. Die Frage ist, ähm…" Harry musste seine Atmung kontrollieren. "Die Frage ist, willst du immer noch … diese Art von Dingen machen."

Hermine drehte sich um und schaute zu der Stelle, an der sich der Himmel im Osten aufhellte. "Du meinst", sagte sie leise, "ob ich immer noch ein Held sein will, jetzt, wo es mir das eine Mal einen schrecklichen Tod eingebracht hat."

Harry nickte, dann sagte er "Ja", weil Hermine sich nicht zu ihm umdrehte, obwohl sich das Wort in seiner Kehle blockiert anfühlte.

"Ich habe darüber nachgedacht", sagte Hermine. "Es war in der Tat ein außergewöhnlich grausamer und schmerzhafter Tod."

"Ich, ähm, ich habe ein paar Dinge vorbereitet, für den Fall, dass du immer noch ein Held sein willst. Es gab einige kurze Zeitfenster, in denen ich keine Zeit hatte, dich zu konsultieren, ich konnte dich nicht zu mir lassen, weil ich erwartete, dass du später Veritaserum bekommen würdest. Aber wenn es dir nicht gefällt, kann ich das meiste von dem, was ich getan habe, wieder rückgängig machen und du kannst den Rest einfach ignorieren."

Hermine nickte distanziert. "Zum Beispiel alle glauben zu lassen, dass ich… Harry, habe ich tatsächlich etwas mit Du-weißt-schon-wem gemacht?"

"Nein, das war alles ich, obwohl du das bitte niemandem erzählen solltest. Nur damit du's weißt, damals, als der Junge-der-lebte angeblich Voldemort besiegt hat, in der Halloween-Nacht 1981, das war Dumbledores Sieg und er hat alle glauben lassen, dass ich es war. Jetzt habe ich also einmal einen Dunklen Lord besiegt und bin einmal dafür belohnt worden. Irgendwann gleicht sich das alles aus, denke ich."

Hermine fuhr fort, nach Osten zu blicken.

"Ich fühle mich nicht wirklich wohl dabei", sagte sie nach einer Weile. "Die Leute denken, ich hätte den Dunklen Lord Voldemort besiegt, obwohl ich gar nichts getan habe … oh, das ist dasselbe, was du durchgemacht hast, nicht wahr?"

"Ja. Tut mir leid, dass ich dir das zugemutet habe. Ich habe … na ja, ich habe versucht, in den Köpfen der Leute eine eigene Identität für dich zu schaffen, denke ich. Es gab nur die eine Gelegenheit und alles war irgendwie überstürzt und… Im Nachhinein wurde mir klar, dass ich es vielleicht nicht hätte tun sollen, aber es war zu spät." Harry räusperte sich. "Obwohl, ähm… wenn du das Gefühl hast, dass du etwas tun willst, das der Art und Weise, wie die Leute über das Mädchen, das wiederbelebt wurde, denken, tatsächlich würdig ist, ähm… ich hätte da vielleicht eine Idee, was du tun könntest. Schon sehr bald, wenn du willst."

Hermine Granger warf ihm einen Blick zu.

"Aber das musst du nicht!" sagte Harry hastig. "Du kannst die ganze Sache einfach ignorieren und der beste Schüler in Ravenclaw sein! Wenn es das ist, was du vorziehst."

"Versuchst du etwa, umgekehrte Psychologie bei mir anzuwenden, Mr~Potter?"

"Nein! Ehrlich!" Harry holte tief Luft. "Ich versuche nicht, dein Leben für dich zu entscheiden. Ich dachte, ich hätte gestern gesehen, was als Nächstes auf dich zukommen könnte - aber dann erinnerte ich mich daran, wie viel Zeit ich in diesem Jahr damit verbracht habe, ein totaler Idiot zu sein. Ich dachte an ein paar Dinge, die Dumbledore zu mir gesagt hatte. Mir wurde klar, dass es mir wirklich nicht zusteht, das zu sagen. Dass man mit seinem Leben machen kann, was man will, und dass man vor allem die Wahl haben muss. Vielleicht willst du danach kein Held sein, vielleicht willst du eine große magische Forscherin werden, weil das das ist, was Hermine Granger die ganze Zeit über wirklich war, egal, woraus deine Fingernägel jetzt bestehen. Oder du gehst auf das Salemer Hexeninstitut in Amerika, statt nach Hogwarts. Ich werde nicht lügen und sagen, dass mir das gefallen würde, aber es liegt wirklich an dir." Harry wandte sich dem Horizont zu und machte eine weite Handbewegung, als wolle er auf die ganze Welt jenseits von Hogwarts hinweisen. "Von hier aus kannst du überall hingehen. Du kannst alles aus deinem Leben machen. Wenn du ein reicher, sechzigjähriger Wassermann sein willst, kann ich es möglich machen. Ich meine es ernst."

Hermine nickte langsam. "Ich bin neugierig, wie du das genau machen würdest, aber was ich will, ist, dass keine Dinge für mich erledigt werden."

"Ich verstehe. Ähm…" Harry zögerte. "Ich denke… wenn es dir vielleicht hilft… in meinem Fall wurde viel für mich arrangiert. Hauptsächlich von Dumbledore, aber auch von Professor Quirrell. Vielleicht ist die Macht, sich seinen Weg im Leben selbst zu verdienen, etwas, das man sich verdienen muss."

"Das klingt ja sehr weise", sagte Hermine. "Zum Beispiel, dass Eltern den Kindern das Studium bezahlen, damit ich Sie sich eines Tages einen eigenen Job suchen können. Dass Professor Quirrell mich als glitzernde Einhornprinzessin ins Leben zurückgebracht hat und dass du allen erzählt hast, dass ich den Dunklen Lord Voldemort ausgeschaltet habe, ist auch so eine Sache, wirklich."

"Es tut mir leid", sagte Harry. "Ich weiß, ich hätte es anders machen sollen, aber… ich hatte nicht viel Zeit zum Planen und ich war erschöpft und konnte nicht wirklich klar denken -"

"Ich bin dir dankbar, Harry", sagte Hermine, ihre Stimme war jetzt weicher. "Du bist zu hart zu dir selbst. Nimm es bitte nicht so ernst, wenn ich schnippisch zu dir bin. Ich will nicht die Art von Mädchen sein, die von den Toten zurückkommt und dann anfängt, sich darüber zu beschweren, welche Superkräfte sie bekommen hat und dass ihre Alicorn-Fingernägel die falsche Schattierung von Perlweiß haben." Hermine hatte sich umgedreht und starrte wieder gen Osten. "Aber, Mr~Potter … wenn ich mich entscheide, dass ein schrecklicher Tod nicht ausreicht, um meine Lebensentscheidungen zu überdenken … nicht, dass ich das jetzt schon sagen würde … was passiert dann?"

"Ich tue mein Bestes, um dich bei deinen Lebensentscheidungen zu unterstützen", sagte Harry fest. "Was auch immer sie sind."

"Ich nehme an, du hast bereits ein Abenteuer für mich geplant. Ein nettes, sicheres Abenteuer, bei dem es keine Chance gibt, dass ich wieder verletzt werde."

Harry rieb sich die Augen und fühlte sich innerlich müde. Es war, als könne er die Stimme von Albus Dumbledore in seinem Kopf hören.

\emph{Verzeih mir, Hermine Granger…}

"Es tut mir leid, Hermine. Wenn du diesen Weg weitergehst, muss ich dich wie Dumbledore behandeln und einige Dinge nicht sagen. Dich manipulieren, wenn auch nur für eine kurze Zeit. Ich glaube, dass du jetzt etwas tun kannst, etwas Echtes, etwas, das der Art, wie die Leute über das Mädchen, das wiederbelebt wurde, denken, würdig ist… dass du vielleicht sogar eine Bestimmung hast… aber das ist letztlich nur eine Vermutung, ich weiß viel weniger als Dumbledore. Bist du bereit, das Leben zu riskieren, das du gerade zurückbekommen hast?"

Hermine drehte sich um und sah ihn an, ihre Augen weiteten sich überrascht. "Mein Leben riskieren?"

Harry nickte nicht, denn das wäre eine glatte Lüge gewesen.

"Bist du bereit, das zu tun?" sagte Harry stattdessen. "Die Aufgabe, von der ich glaube, dass sie dein Schicksal sein könnte - und nein, ich kenne keine spezifischen Prophezeiungen, es ist nur eine Vermutung - beinhaltet buchstäblich einen Abstieg in die Hölle."

"Ich dachte…" sagte Hermine. Sie klang unsicher. "Ich war mir sicher, dass du und Professor McGonagall mich nach dieser Sache nie wieder etwas auch nur annähernd Gefährliches tun lassen würden."

Harry sagte nichts, er fühlte sich schuldig wegen des falschen Beziehungskredits, den er bekam. Es war in der Tat so, dass Hermine ihn mit enormer Genauigkeit modellierte, und dass, wenn Hermine nicht einen Horkrux hätte, die Oberfläche des Planeten Venus auf Bruchteil-Kelvin-Temperaturen gesunken wäre, bevor Harry dies versuchte.

"Auf einer Skala von null bis hundert, von wie wortwörtlich einem Abstieg in die Hölle reden wir hier?", fragte Hermine. Das Mädchen sah jetzt ein wenig besorgt aus.

Harry kalibrierte gedanklich seine Skala und erinnerte sich an Askaban.

"Ich würde sagen, vielleicht siebenundachtzig?"

"Das klingt nach etwas, das ich machen sollte, wenn ich älter bin, Harry. Es ist ein Unterschied, ob man ein Held ist oder ein totaler Wahnsinniger."

Harry schüttelte den Kopf. "Ich glaube nicht, dass sich das Risiko groß ändern würde", sagte Harry, wobei er die Frage beiseite ließ, wie viel Risiko das wirklich war, "und es ist die Art von Sache, die man besser früher macht, wenn man sie überhaupt macht."

"Und meine Eltern haben kein Stimmrecht", sagte Hermine. "Oder doch?"

Harry zuckte mit den Schultern. "Wir wissen beide, wie sie abstimmen würden, und du kannst das berücksichtigen, wenn du willst. Ähm, ich habe gesagt, dass Dr~und Dr~Granger noch nicht wissen sollen, dass du am Leben bist. Sie werden es herausfinden, nachdem du von deiner Mission zurückkommst, wenn du dich entscheiden, sie zu akzeptieren. Das scheint ein bisschen… schonender für die Nerven deiner Eltern zu sein, sie bekommen nur die eine angenehme Überraschung, anstatt sich um, ähm, Sachen sorgen zu müssen."

"Das ist ja sehr aufmerksam von dir", sagte Hermine. "Es ist schön, dass du dir so viele Gedanken über ihre Gefühle machst. Darf ich bitte ein paar Minuten darüber nachdenken?"

Harry gestikulierte in Richtung des Kissens, das er gegenüber von seinem eigenen hingelegt hatte, und Hermine bewegte sich mit fließender Anmut hinüber und setzte sich, um über den Rand des Schlosses zu blicken, während Sie immer noch Ruhe und Unschuld ausstrahlte.

\emph{Sie würden wirklich etwas dagegen tun müssen, vielleicht jemanden dafür bezahlen, einen Anti-Reinheitstrank zu erfinden.}

"Muss ich mich entscheiden, ohne zu wissen, was die Mission ist?" fragte Hermine.

"Oh verdammt, nein", sagte Harry und dachte an ein ähnliches Gespräch vor seiner eigenen Reise nach Askaban.

"Das ist eine Sache, die man frei wählen muss, wenn man sie überhaupt macht. Ich meine, das ist eine wirkliche Missionsanforderung. Wenn du sagst, dass du immer noch ein Held sein willst, werde ich dir nachher von der Mission erzählen - nachdem du etwas Zeit hattest, um zu essen und mit Leuten zu reden und dich ein bisschen zu erholen - und du wirst dann entscheiden, ob es etwas ist, was du tun willst. Und wir werden vorher testen, ob du nach deiner Rückkehr vom Tod den Zauber sprechen kannst, den normale Zauberer für unmöglich halten, bevor du gehst."

Hermine nickte und verfiel wieder in Schweigen.

Der Himmel hatte sich weiter aufgehellt, als Hermine wieder sprach.

"Ich habe Angst", sagte Hermine, fast im Flüsterton. "Nicht davor, wieder zu sterben, oder nicht nur das. Ich habe Angst, dass ich nicht gut genug sein werde. Ich hatte meine Chance, einen Troll zu besiegen, und stattdessen bin ich einfach gestorben -"

"Das war ein Troll, der von Voldemort als Waffe eingesetzt wurde, außerdem hat er alle deine magischen Gegenstände sabotiert, nur damit du es weißt."

"Ich bin gestorben. Und du hast den Troll getötet, irgendwie, ich glaube, ich erinnere mich an diesen Teil, es hat dich nicht einmal aufgehalten." Hermine weinte nicht, keine Tränen glitzerten auf ihren Wangen, sie starrte einfach in den hellen Himmel, wo die Sonne aufgehen würde. "Und dann hast du mich als funkelnde Einhornprinzessin von den Toten zurückgebracht. Ich weiß, dass ich das nicht hätte tun können. Ich fürchte, ich werde das nie können, egal, was die Leute über mich denken."

"In dieser Situation beginnt deine Reise, denke ich -" Harry hielt inne. "Entschuldigung, ich sollte nicht versuchen, deine Entscheidung zu beeinflussen."

"Nein", flüsterte Hermine, immer noch den Blick auf die Hügel unter ihr gerichtet. Sie erhob ihre Stimme. "Nein, Harry, ich will das hören."

"Okay. Ähm. Ich glaube, hier fängst du an. Alles, was bis jetzt passiert ist … es versetzt dich an denselben Ort, an dem ich im September angefangen habe, als ich mich vorher nur für ein Wunderkind gehalten habe. Wenn du dich nicht mit mir und meiner", \emph{erwachsene kognitive Muster von Tom Riddle kopiert,} "dunklen Seite vergleichen würdest… dann wärst du der hellste Stern von Ravenclaw, der seinen eigene Protest organisiert hat, um gegen Schultyrannen zu kämpfen und seinen Verstand unter dem Angriff von Voldemort bewahrt hat, und das alles, während du erst zwölf Jahre alt warst. Ich habe es nachgeschlagen, du hast bessere Noten als Dumbledore in seinem ersten Jahr." \emph{Die Note in Verteidigung mal beiseite gelassen, denn das war nur Voldemort, der Voldemort war.} "Jetzt hast du ein paar Kräfte und einen Ruf, dem du gerecht werden musst, und die Welt wird dir ein paar schwierige Aufgaben stellen. Da fängt für dich alles an, so wie es für mich angefangen hat. Verkauf dich nicht unter Wert." Und dann schloss Harry fest den Mund, denn er redete Hermine rein, und das war nicht richtig.

Er hatte es wenigstens geschafft, vor dem Teil aufzuhören, in dem er fragte, wenn sie mit all dem, was sie hatte, keine Heldin sein konnte, wer genau ihrer Meinung nach es dann tun würde.

"Weißt du", sagte Hermine zum Horizont, immer noch ohne Harry anzusehen, "ich hatte mal so ein Gespräch mit Professor Quirrell, über das Heldendasein. Er war natürlich auf der anderen Seite. Aber abgesehen davon fühlt es sich irgendwie so an wie damals, als er mit mir gestritten hat."

Harry hielt seine Lippen zusammengepresst. Es war schwer, Menschen ihre eigenen Entscheidungen treffen zu lassen, denn das bedeutete, dass sie auch die falschen treffen durften, aber es musste trotzdem getan werden.

Hermine sprach vorsichtig, die blauen Fransen ihrer Hogwarts-Uniform schienen sich jetzt heller von ihren schwarzen Roben abzuheben, während der Himmel um sie herum erhellt wurde; im Westen waren keine Sterne mehr zu sehen.

"Professor Quirrell hat es mir erzählt, er sagte, er sei einmal ein Held gewesen. Aber die Leute haben ihm nicht genug geholfen, also hat er aufgegeben und ist weggegangen, um etwas Interessanteres zu tun. Ich sagte Professor Quirrell, dass es nicht richtig gewesen sei, dass er das getan habe - was ich eigentlich sagte, war 'das ist schrecklich'. Professor Quirrell sagte, ja, vielleicht war er ein schrecklicher Mensch, aber was war dann mit all den anderen Leuten, die nie versucht hatten, ein Held zu sein? Waren die noch schlimmer als er? Und ich wusste nicht, was ich darauf erwidern sollte. Ich meine, es ist falsch zu sagen, dass nur Helden im Gryffindor-Stil gute Menschen sind - obwohl ich glaube, dass es aus Professor Quirrells Sicht eher so war, dass nur Leute mit großen Ambitionen ein Recht auf Atmen hatten. Und das habe ich nicht geglaubt. Aber es schien auch falsch zu sein, aufzuhören, ein Held zu sein, wegzugehen, wie er es getan hatte. Also stand ich einfach da und schaute dumm. Aber jetzt weiß ich, was ich ihm damals hätte sagen sollen."

Harry kontrollierte seine Atmung.

Hermine stand von ihrem Kissen auf und drehte sich zu Harry um.

"Ich bin fertig mit dem Versuch, eine Heldin zu sein", sagte Hermine Granger, während sich der östliche Himmel um sie herum aufhellte. "Ich hätte mich nie auf diese ganze Denkweise einlassen sollen. Es gibt einfach Menschen, die tun, was sie können, was auch immer sie können. Und es gibt auch Leute, die nicht einmal versuchen, das zu tun, was sie können, und ja, diese Leute machen etwas falsch.

Ich werde nie wieder versuchen, ein Held zu sein. Ich werde nicht in heldenhaften Begriffen denken, wenn ich es vermeiden kann. Aber ich werde nicht weniger tun als ich kann - oder nicht viel weniger, ich meine, ich bin auch nur ein Mensch."

Harry hatte nie verstanden, was an der Mona Lisa geheimnisvoll sein sollte, aber wenn er gerade ein Foto von Hermines resigniert-freudigem Lächeln hätte machen können, hatte er das Gefühl, dass er es stundenlang hätte betrachten können, ohne es zu verstehen, und dass Dumbledore es mit einem Blick hätte durchschauen können.

"Ich werde meine Lektion nicht lernen. Ich werde so dumm sein. Ich werde weiterhin versuchen, das meiste zu tun, was ich kann, oder zumindest einiges von dem, was ich kann - oh, du weißt, was ich meine. Selbst wenn das bedeutet, dass ich wieder mein Leben riskieren muss, solange es das Risiko wert ist und ich nicht, du weißt schon, wirklich dumm bin. Das ist meine Antwort."

Hermine holte tief Luft, ihr Gesicht war entschlossen.

"Also, gibt es etwas, das ich tun kann?"

Harrys Kehle war wie zugeschnürt. Er griff in seinen Beutel und schrieb UMHANG, da er nicht sprechen konnte, und holte den schwülstigen Klecks des Unsichtbarkeitsumhangs hervor und bot ihn Hermine ein letztes Mal an. Harry musste die Worte aus seiner Kehle zwingen.

"Dies ist der Wahre Umhang der Unsichtbarkeit", sagte Harry fast flüsternd, "das Heiligtum des Todes, das von Ignotus Peverell an seine Erben, die Potters, weitergegeben wurde. Und nun zu dir -"

"Harry!" sagte Hermine. Ihre Hände flogen über ihre Brust, als wolle sie sich vor dem angreifenden Geschenk schützen.

"Du musst das nicht tun!"

"Ich muss das tun. Ich habe den Teil des Weges verlassen, der es mir erlaubt, ein Held zu sein, ich kann es nicht riskieren, Abenteuer zu erleben, niemals. Und du … kannst es." Harry hob die Hand, die nicht den Umhang hielt, und wischte sich über die Augen. "Das wurde für dich gemacht, glaube ich. Für den Menschen, zu dem du werden wirst."

\emph{Eine Waffe, um den Tod zu bekämpfen, in seiner Form als der Schatten der Verzweiflung, der auf die Menschen fällt und ihnen die Hoffnung auf die Zukunft raubt; den wirst du bekämpfen, nehme ich an, in mehr Formen als nur Dementoren…

}\strut 

"Ich leihe dir meinen Umhang nicht, sondern schenke ihn dir Hermine Jean Granger. Beschütze sie auf ewig."

Langsam streckte Hermine die Hand aus und nahm den Umhang in die Hand, als würde sie selbst versuchen, nicht zu weinen.

"Ich danke dir", flüsterte sie. "Ich glaube… auch wenn ich mit der Vorstellung, ein Held zu sein, fertig bin… Ich glaube, du warst immer, von dem Tag an, an dem ich dich kennengelernt habe, mein geheimnisvoller alter Zauberer."

"Und ich glaube", sagte Harry, dem selbst die Kehle halb zugeschnürt war, "auch wenn du diese Denkweise jetzt abstreitest, ich glaube, dass du von Anfang an dazu bestimmt warst, der Held zu werden."

\emph{Wer muss Hermine Granger werden, welche Erwachsenenform muss sie annehmen, wenn sie erwachsen ist, um durch das enge Schlüsselloch der Zeit zu gehen? Ich kenne die Antwort darauf auch nicht, genauso wenig wie ich mir mein eigenes erwachsenes Ich vorstellen kann. Aber ihre nächsten Schritte scheinen klarer zu sein als meine…}

Harry ließ den Umhang los, und er wanderte von seinen Händen in ihre.

"Er singt", sagte Hermine. "Er singt für mich." Sie griff nach oben und wischte sich über die Augen. "Ich kann nicht glauben, dass du das getan hast, Harry."

Harrys andere Hand kam aus seinem Beutel und trug nun eine lange goldene Kette, an deren Ende eine geschlossene goldene Muschel baumelte.

"Und das ist deine persönliche Zeitmaschine."

Es gab eine Pause, in der sich der Planet Erde ein Stück weiter in seiner Umlaufbahn drehte.

"Was?", fragte Hermine tonlos.

"Ein Zeitumkehrer, so nennt man es. Hogwarts hat einen Bestand, den sie an einige Schüler verteilen, ich habe zu Beginn des Jahres einen bekommen, um meine Schlafstörung zu behandeln. Er lässt den Benutzer in der Zeit rückwärts gehen, in bis zu sechs Ein-Stunden-Schritten, was ich nutzte, um sechs zusätzliche Stunden am Tag zum Lernen zu bekommen. Und um aus der Zaubertränke-Klasse zu verschwinden und so weiter. Keine Sorge, ein Zeitumkehrer kann nicht die Geschichte verändern oder Paradoxa erzeugen, die das Universum zerstören."

"Du hast im Unterricht mit mir Schritt gehalten, indem du sechs Stunden am Tag zusätzlich mit einer Zeitmaschine gelernt hast."

Hermine Granger schien aus irgendeinem unerklärlichen Grund Schwierigkeiten mit diesem Konzept zu haben.

Harry verzog sein Gesicht zu einem verwirrten Blick.

"Ist daran etwas merkwürdig?"

Hermine streckte die Hand aus und nahm die goldene Halskette.

"Ich schätze, nicht nach Zauberer-Standards", sagte sie.

Aus irgendeinem Grund klang ihre Stimme ziemlich scharf.

Sie legte sich die Kette um den Hals und steckte die Sanduhr in ihr Hemd. "Ich fühle mich jetzt allerdings besser, wenn ich mit dir mithalten kann, also danke dafür."

Harry räusperte sich.

"Außerdem, da Voldemort das Haus Monroe ausgelöscht hat und du es dann, soweit alle glauben, gerächt hast, indem du Voldemort getötet hast, habe ich Amelia Bones dazu gebracht, eine Gesetzesvorlage durch das, was vom Zaubergamot übrig ist, zu schicken, die besagt, dass Granger jetzt ein Adelshaus von Britannien ist."

"Wie bitte?", sagte Hermine.

"Das macht dich auch zum einzigen Spross eines Adelshauses, was bedeutet, dass du, um die Volljährigkeit zu erlangen, nur noch die Gewöhnliche Zaubererprüfung bestehen musst, die ich für das Ende des Sommers angesetzt habe, damit wir vorher etwas Zeit zum Lernen haben. Wenn das für dich in Ordnung ist, meine ich."

Hermine Granger gab eine Art hochfrequentes Geräusch von sich, das bei einem weniger organischen Gerät auf eine Motorstörung hingedeutet hätte.

"Ich habe nur zwei Monate Zeit, um für meinen Z.A.G. zu lernen?"

"Hermine, es ist ein Test, der so konzipiert ist, dass die meisten Fünfzehnjährigen ihn bestehen können. Gewöhnliche Fünfzehnjährige. Wir können mit dem niedrigen Wissensniveau eines Drittklässlers bestehen, wenn wir die richtigen Zaubersprüche lernen, und das ist alles, was wir für das Bestehen brauchen. Allerdings musst du dich damit abfinden, dass du akzeptable Noten bekommst, statt der üblichen überragenden."

Die hohen Töne, die von Hermine Granger kamen, wurden immer lauter.

"Hier ist dein Zauberstab zurück."

Harry nahm ihn aus seinem Beutel.

"Und dein eigener magisch vergrößerter Beutel, ich habe dafür gesorgt, dass alles wieder da ist, was da war, als du gestorben bist."

Den Beutel zog Harry aus einer normalen Tasche seines Umhangs, da er ungern einen Beutel in einen Beutel steckte, obwohl es angeblich harmlos sein sollte, solange beide Geräte unter Beachtung aller Sicherheitsvorkehrungen hergestellt worden waren.

Hermine nahm ihren Zauberstab zurück und dann ihren Beutel, wobei sie es irgendwie schaffte, die Bewegungen anmutig aussehen zu lassen, obwohl ihre Finger etwas zittrig waren.

"Mal sehen, was noch… der Eid, den du vorhin dem Haus Potter geschworen hast, besagte nur, dass du bis 'zum Tag deines Todes' dienen musst, also bist du jetzt frei und unbelastet. Und gleich nach deinem Tod habe ich die Malfoys dazu gebracht, öffentlich zu erklären, dass du unschuldig bist, was den Mordversuch an Draco angeht."

"Vielen Dank noch mal, Harry", sagte Hermine Granger. "Das war sehr nett von dir, und von ihnen auch, denke ich."

Sie fuhr sich immer wieder mit den Fingern durch ihre kastanienbraunen Locken, als ob sie durch das Ordnen ihrer Haare wieder Vernunft in ihr Leben bringen könnte.

"Zu guter Letzt habe ich die Kobolde damit beauftragt, in Gringotts ein Konto für das Haus Granger anzulegen", sagte Harry. "Ich habe kein Geld hineingesteckt, denn das war etwas, womit ich warten und dich zuerst fragen konnte. Aber wenn du ein Superheld sein willst, der herumgeht und bestimmte Arten von Unrecht berichtigt, wird es sehr helfen, wenn die Leute dich als Teil der oberen sozialen Schicht betrachten und, ähm, ich denke, es könnte helfen, wenn sie wissen, dass du dir Anwälte leisten kannst. Ich kann so viel Gold in dein Verließ legen, wie du willst, denn nachdem Voldemort Nicholas Flamel getötet hat, bin ich in den Besitz des Steins der Weisen gelangt."

"Ich fühle mich, als müsste ich in Ohnmacht fallen", sagte Hermine mit hoher Stimme, "nur kann ich das nicht, wegen meiner Superkräfte und warum habe ich die wieder?"

"Wenn es dir recht ist, beginnt dein Okklumentikunterricht am Mittwoch bei Mr~Bester, er kann einmal am Tag mit dir arbeiten. Bis dahin halte ich es für besser, wenn der wahre Ursprung deiner Kräfte nicht bekannt wird, nur weil dir ein Legilimens in die Augen schaut. Ich meine, natürlich gibt es eine normale magische Erklärung, nichts Übernatürliches, aber die Leute neigen dazu, ihre eigene Unwissenheit zu verehren, und, nun ja, ich denke, das Mädchen, das wiederbelebt wurde, wird effektiver sein, wenn du geheimnisvoll bleibst. Sobald du Mr~Bester ausschalten und das Veritaserum besiegen kannst, erzähle ich dir die ganze Vorgeschichte, das verspreche ich dir, einschließlich all der Geheimnisse, die du niemandem sonst erzählen kannst."

"Das klingt wunderbar", sagte Hermine Granger. "Ich freue mich schon darauf."

"Allerdings musst du einen unbrechbaren Schwur ablegen, nichts zu tun, was die Welt zerstören könnte, bevor ich dir die gefährlicheren Teile der Geschichte erzählen kann. Ich meine, ich kann dir buchstäblich nichts anderes sagen, weil ich selbst einen unbrechbaren Schwur abgelegt habe. Ist das okay?"

"Sicher", sagte Hermine. "Warum sollte es nicht in Ordnung sein? Ich würde sowieso nicht die Welt zerstören wollen."

"Musst du dich wieder hinsetzen?" sagte Harry und fühlte sich beunruhigt durch die Art, wie Hermine leicht schwankte, als ob sie im Rhythmus der gesprochenen Worte schwankte.

Hermine Granger atmete mehrmals tief durch. "Nein, mir geht's prima", sagte sie.

"Gibt es sonst noch etwas, was ich wissen sollte?"

"Das war's. Ich bin fertig, zumindest für jetzt."

Harry hielt inne.

"Ich verstehe ja, dass du Dinge für dich selbst tun willst, nicht nur für dich tun lassen willst. Es ist nur … du wirst eine ernstere Art von Held sein, und die einzig vernünftige Wahl ist, dass ich dir alle Vorteile gebe, die ich erreichen kann -"

"Das verstehe ich sehr gut", sagte Hermine. "Jetzt, wo ich tatsächlich einen Kampf verloren habe und gestorben bin. Früher habe ich es nicht verstanden, aber jetzt verstehe ich es."

Eine Brise zerzauste Hermines kastanienbraunes Haar und brachte ihre Roben in Bewegung, was sie in der Morgenluft noch friedlicher aussehen ließ, während sie eine Hand hob und sie vorsichtig zu einer Faust ballte.

"Wenn ich das mache, dann richtig. Wir müssen messen, wie hart ich zuschlagen und wie hoch ich springen kann, und einen sicheren Weg finden, um zu testen, ob meine Fingernägel Lethifolds töten können wie das Horn eines echten Einhorns, und ich sollte üben, meine Schnelligkeit zu nutzen, um Zaubern auszuweichen, die mich nicht treffen dürfen, und … und es klingt, als könntest du vielleicht arrangieren, dass ich ein Aurorentraining bekomme, wie von dem, der Susan Bones unterrichtet hat."

Hermine lächelte jetzt wieder, ein seltsames Licht in ihren Augen, das Dumbledore stundenlang verwirrt hätte und das Harry sofort verstand, nicht ohne einen Stich der Besorgnis.

"Oh! Und ich möchte anfangen, Muggelwaffen zu tragen, vielleicht versteckt, damit niemand weiß, dass ich sie habe. Ich dachte an Brandgranaten, als ich gegen den Troll kämpfte, aber ich wusste, dass ich sie nicht schnell genug verwandeln konnte, selbst nachdem ich aufgehört hatte, mich um das Einhalten der Regeln zu kümmern."

"Ich habe das Gefühl", sagte Harry und imitierte Professor McGonagalls schottischen Akzent so gut er konnte, "dass ich etwas dagegen tun sollte."

"Oh, dafür ist es viel, viel, VIEL zu spät, Mr~Potter. Sag mal, kannst du mir eine Bazooka besorgen? Den Raketenwerfer, meine ich, nicht den Kaugummi? Ich wette, das werden sie von einem jungen Mädchen nicht erwarten, vor allem, wenn ich eine Aura der Unschuld und Reinheit ausstrahle."

"Also gut", sagte Harry ruhig, "jetzt fängst du an, mir Angst zu machen."

Hermine hielt inne, wo sie damit experimentierte, auf der Spitze ihres linken Schuhs zu balancieren, den Arm in die eine Richtung gestreckt und das rechte Bein in die andere, wie eine Balletttänzerin.

"Tue ich das? Ich habe nur gedacht, dass ich nicht einsehe dass ich nicht können sollte, was ein Ministeriumstrupp von Auroren nicht auch kann. Sie haben Besenstiele für die Beweglichkeit und Zaubersprüche, die härter treffen, als ich es möglicherweise könnte."

Sie senkte anmutig ihr Bein wieder ab.

"Ich meine, jetzt, wo ich ein paar Dinge ausprobieren kann, ohne mir Sorgen darüber zu machen, wer zuschaut, fange ich an zu glauben, dass ich es wirklich sehr mag, Superkräfte zu haben. Aber ich sehe immer noch nicht, wie ich einen Kampf gewinnen könnte, den Professor Flitwick nicht gewinnen könnte, es sei denn, es geht darum, dass ich einen dunklen Zauberer überrumple."

\emph{Du kannst Risiken eingehen, die andere nicht eingehen sollten, und es mit dem Wissen, was dich getötet hat, erneut versuchen. Man kann mit neuen Zaubern experimentieren, mehr als jeder andere es könnte, ohne sicher zu sterben.}

Aber Harry konnte noch nichts davon sagen, also sagte er stattdessen: "Ich denke, es ist in Ordnung, mehr an die Zukunft zu denken, nicht nur an das, was man in diesem Moment tun kann."

Hermine sprang hoch in die Luft, klickte auf dem Weg nach unten dreimal mit den Absätzen zusammen und landete auf den Zehenspitzen, perfekt posiert.

"Aber du hast gesagt, dass ich sofort etwas tun kann. Oder war das nur ein Test?"

"Dieser Teil ist ein Sonderfall", sagte Harry und spürte die Kälte der Morgenluft auf seiner Haut.

Er freute sich zunehmend weniger darauf, Super-Hermine zu sagen, dass ihre Prüfung darin bestehen würde, sich ihrem buchstäblich schlimmsten Albtraum zu stellen, und zwar unter Bedingungen, bei denen all ihre neu gewonnene körperliche Kraft nutzlos sein würde.

Hermine nickte, dann blickte sie nach Osten. Sofort ging sie zur Seite des Daches und setzte sich hin, wobei ihre Füße über den Dachvorsprung baumelten. Harry ging an ihre Seite und setzte sich ebenfalls, im Schneidersitz und weiter hinten an der Dachkante sitzend. In der Ferne erhob sich ein leuchtendes Rot über den Hügeln im Osten von Hogwarts. Beim Anblick der Spitze des Sonnenaufgangs fühlte sich Harry besser, irgendwie. Solange die Sonne am Himmel stand, war auf irgendeiner Ebene noch alles in Ordnung, so wie er die Sonne noch nicht zerstört hatte.

"Also", sagte Hermine. Ihre Stimme erhob sich ein wenig. "Wo wir gerade von der Zukunft sprechen, Harry. Ich hatte Zeit, über eine Menge Dinge nachzudenken, während ich in St. Mungo's gewartet habe, und … vielleicht ist es dumm von mir, aber es gibt eine Frage, auf die ich immer noch die Antwort wissen möchte.

Erinnerst du dich an die letzte Sache, über die wir zusammen gesprochen haben? Davor, meine ich?"

"Was?" sagte Harry ausdruckslos.

"Oh…" sagte Hermine. "Für dich war es vor zwei Monaten… Ich schätze, du erinnerst dich nicht mehr."

Und Harry erinnerte sich.

"Keine Panik!" sagte Hermine, als eine Art ersticktes Halbgurgeln aus Harrys Kehle kam. "Ich verspreche, egal was du sagst, ich werde nicht in Tränen ausbrechen und weglaufen und wieder von einem Troll gefressen werden! Ich weiß, dass es für mich noch keine zwei Tage her ist, aber ich denke, dass das Sterben viele Dinge, über die ich mich früher aufgeregt habe, viel unwichtiger erscheinen lässt im Vergleich zu dem, was ich durchgemacht habe!"

"Oh", sagte Harry, seine eigene Stimme jetzt hochtönend. "Das ist eine gute Verwendung für ein großes Trauma, schätze ich?"

"Nur, sieh mal, ich habe mich immer noch darüber gewundert, Harry, denn für mich ist es gar nicht so lange her seit unserem letzten Gespräch, und wir haben nicht zu Ende geredet, was zugegebenermaßen ganz allein meine Schuld war, weil ich die Kontrolle über meine Emotionen verloren habe und dann von einem Troll gefressen wurde, was ich definitiv nicht wieder tun werde. Ich habe mir überlegt, dass ich dir versichern sollte, dass das nicht jedes Mal passieren wird, wenn du das Falsche zu einem Mädchen sagst."

Hermine zappelte, lehnte sich von einer Seite zur anderen, wo sie saß, leicht hin und her.

"Aber, na ja, selbst die meisten Menschen, die verliebt sind, tun nicht einmal ein Hundertstel von dem, was du für mich getan hast. Also, Mr~Harry James Potter-Evans-Verres, wenn es keine Liebe ist, möchte ich genau wissen, was ich für dich bin. Das hast du nie gesagt."

"Das ist eine gute Frage", sagte Harry und kontrollierte die aufsteigende Panik. "Macht es dir etwas aus, wenn ich darüber nachdenke?"

Nach und nach wurde mehr von dem gleißend hellen Kreis hinter den Hügeln sichtbar.

"Hermine", sagte Harry, als die Sonne auf halber Höhe über dem Horizont stand,

"hast du jemals irgendwelche Hypothesen erfunden, um meine mysteriöse dunkle Seite zu erklären?"

"Nur die offensichtliche", sagte Hermine und strampelte mit den Beinen leicht über den Rand des Daches. "Ich dachte, vielleicht hat er, als Du-weißt-schon-wer direkt neben dir gestorben ist, einen Geist oder Teil davon zurückgelassen und etwas davon hat sich in dein Gehirn eingeprägt statt in den Boden. Aber das hat sich für mich nie richtig angefühlt, als wäre es nur eine clevere Erklärung, die nicht wirklich wahr ist, und es macht noch weniger Sinn, wenn Du-Weißt-Schon-Wer in dieser Nacht nicht wirklich gestorben ist."

"Gut genug", sagte Harry. "Stellen wir uns dieses Szenario doch einfach mal vor."

Sein innerer Rationalist verzog wieder das Gesicht darüber, wie er es geschafft hatte, nicht an Hypothesen wie diese zu denken. Es war nicht wahr, aber es war vernünftig, und Harry hatte sich nie ein so konkretes Kausalmodell überlegt, sondern nur vage einen Zusammenhang vermutet.

Hermine nickte.

"Du weißt das wahrscheinlich schon, aber ich dachte, ich sage es nur, um sicherzugehen: Du bist nicht Voldemort, Harry."

"Ich weiß. Und das bedeutest du auch für mich."

Harry holte tief Luft, da es ihm immer noch schwer fiel, es laut auszusprechen.

"Voldemort… er war kein glücklicher Mensch. Ich weiß nicht, ob er jemals glücklich war, nicht einen einzigen Tag in seinem Leben."

\emph{Er konnte nie den Patronus-Zauber wirken.}

"Das ist ein Grund, warum seine kognitiven Muster mich nicht übernommen haben, meine dunkle Seite fühlte sich nicht gut an, sie wurde nicht positiv verstärkt. Mit dir befreundet zu sein, bedeutet, dass mein Leben nicht so verlaufen muss wie das von Voldemort. Und ich war vor Hogwarts ziemlich einsam, auch wenn es mir damals nicht bewusst war, also… ja. Ich war vielleicht etwas verzweifelter, dich von den Toten zurückzuholen, als der Durchschnittsjunge meines Alters es gewesen wäre.

Allerdings behaupte ich auch, dass meine Entscheidung rein normativ-moralisch begründet war, und wenn andere Leute sich weniger um ihre Freunde kümmern, ist das ihr Problem, nicht meines."

"Ich verstehe", sagte Hermine leise. Sie zögerte. "Harry, versteh mich nicht falsch, aber ich fühle mich damit nicht hundertprozentig wohl. Es ist eine große Verantwortung, die ich mir nicht ausgesucht habe, und ich glaube nicht, dass es gesund ist, sie nur auf eine Person abzuwälzen."

Harry nickte.

"Ich weiß. Aber es gibt noch mehr, worauf ich hinaus will. Es gab eine Prophezeiung, dass ich Voldemort besiegen würde…"

"Eine Prophezeiung? Es gab eine Prophezeiung über dich? Ist das dein Ernst, Harry?"

"Ja ja, ich weiß. Wie auch immer, ein Teil davon lautete: 'Und der Dunkle Lord wird ihn als seinesgleichen kennzeichnen, aber er wird Macht haben, die der Dunkle Lord nicht kennt.' Was würdest du vermuten, was das bedeutet?"

"Hmmm", sagte Hermine. Ihre Finger klopften nachdenklich auf den Stein des Daches. "Deine geheimnisvolle dunkle Seite ist das Zeichen von Du-weißt-schon-wem, das dich ihm ebenbürtig gemacht hat. Die Macht, die er nicht kannte … war die wissenschaftliche Methode, richtig?"

Harry schüttelte den Kopf.

"Das dachte ich anfangs auch - dass es die Muggelwissenschaft sein würde, oder die Methoden der Rationalität. Aber …"

Harry atmete aus. Die Sonne war nun vollständig über den Hügeln aufgegangen. Es war ihm peinlich, das zu sagen, aber er wollte es trotzdem sagen.

"Professor Snape, der die Prophezeiung ursprünglich gehört hat - ja, auch das ist passiert -, Professor Snape hat gesagt, dass er nicht glaubt, dass es nur Wissenschaft sein kann, dass die 'Macht, die der Dunkle Lord nicht kennt', etwas sein muss, das Voldemort noch fremder ist als nur das. Selbst wenn ich es in Begriffen der Rationalität betrachte, nun, es stellt sich heraus, dass die Person, die Voldemort wirklich war",

\emph{warum, Professor Quirrell, warum,} der Gedanke stach immer noch krank in Harrys Herz,

"er hätte auch die Methoden der Rationalität lernen können, wenn er die gleichen wissenschaftlichen Abhandlungen gelesen hätte wie ich. Bis auf eine letzte Sache vielleicht …"

Harry holte tief Luft.

"Am Ende von all dem, während meines letzten Showdowns mit Voldemort, drohte er damit, meine Eltern und meine Freunde nach Askaban zu stecken. Es sei denn, ich käme mit interessanten Geheimnissen, die ich ihm erzählen könnte, eine Person pro Geheimnis gerettet. Aber ich wusste, ich konnte nicht genug Geheimnisse finden, um alle zu retten. Und in dem Moment, als ich keine Möglichkeit mehr sah, alle zu retten… da fing ich tatsächlich an, nachzudenken. Vielleicht zum ersten Mal in meinem Leben fing ich an zu denken. Ich dachte schneller als Voldemort, obwohl er älter war als ich und klüger, weil… weil ich einen Grund zum Denken hatte.

Voldemort hatte den Drang, unsterblich zu sein, er zog es stark vor, nicht zu sterben, aber das war kein positiver Wunsch, es war Angst, und Voldemort machte Fehler wegen dieser Angst. Ich glaube, die Macht, die Voldemort nicht kannte, war, dass ich etwas zu beschützen hatte."

"Oh, Harry", sagte Hermine sanft.

Sie zögerte.

"Ist es das, was ich für dich bin? Die Sache, die du beschützt?"

"Nein, ich meine, der ganze Grund, warum ich dir das erzähle, ist, dass Voldemort nicht gedroht hat, dich nach Askaban zu stecken. Selbst wenn er die ganze Welt übernommen hätte, wäre es dir gut gegangen. Er hatte bereits verbindlich versprochen, dir nichts anzutun, aus, ähm, aus Gründen. Also habe ich in meinem Moment der ultimativen Krise, als ich tief nach unten griff und die Macht fand, die Voldemort nicht kannte, es getan, um alle zu schützen, außer dich."

Hermine dachte darüber nach, ein langsames Lächeln breitete sich auf ihrem Gesicht aus. "Also, Harry", sagte sie. "Das ist das Unromantischste, was ich je gehört habe."

"Gern geschehen."

"Nein, wirklich, es hilft", sagte Hermine. "Ich meine, es macht die ganze Sache viel weniger stalkerhaft."

"Ich weiß, nicht wahr?"

Die beiden teilten ein kameradschaftliches Nicken, beide sahen jetzt entspannter aus, und betrachteten gemeinsam den Sonnenaufgang.

"Ich habe nachgedacht", sagte Harry, seine eigene Stimme wurde leise, "über den alternativen Harry Potter, die Person, die ich hätte sein können, wenn Voldemort meine Eltern nicht angegriffen hätte."

\emph{Wenn Tom Riddle nicht versucht hätte, sich auf mich zu kopieren.}

"Dieser andere Harry Potter wäre nicht so klug gewesen, schätze ich. Er hätte wahrscheinlich nicht viel Muggelwissenschaft studiert, auch wenn seine Mutter eine Muggelgeborene war. Aber dieser andere Harry Potter hätte… die Fähigkeit zur Wärme, die er von James Potter und Lily Evans geerbt hat, er hätte sich um andere Menschen gekümmert und versucht, seine Freunde zu retten, das weiß ich, denn das ist etwas, was Lord Voldemort nie getan hat, weißt du…"

Harrys Augen tränten.

"Also muss dieser Teil der Rest sein."

Die Sonne stand jetzt schon weit über dem Horizont, das goldene Licht beleuchtete sie beide und warf lange Schatten auf die andere Seite der Dachterrasse.

"Ich finde, du bist genau richtig, so wie du bist", sagte Hermine. "Ich meine, der andere Harry Potter wäre vielleicht ein netter Junge gewesen, aber es klingt so, als hätte ich das ganze Denken für ihn übernehmen müssen."

"Der Vererbung nach zu urteilen, wäre der andere Harry in Gryffindor gewesen, wie seine Eltern, und wir beide wären keine Freunde geworden. Obwohl James Potter und Lily Evans zu ihrer Zeit Schulsprecher von Hogwarts waren, also wäre er nicht so schlimm gewesen."

"Ich kann es mir gut vorstellen", sagte Hermine. "Harry James Potter, in Gryffindor einsortiert, aufstrebender Quidditchspieler -"

"Nein. Einfach nein."

"In die Geschichte eingegangen als der Handlanger von Hermine Jean Granger, die Mr~Potter losschickte, um für sie in Schwierigkeiten zu geraten, und dann das Rätsel aus der Bibliothek löste, indem sie Bücher las und ihr unglaubliches Gedächtnis benutzte."

"Du hast wirklich Spaß an diesem alternativen Universum, oder?"

"Vielleicht wäre er der beste Kumpel von Ron Weasley, dem klügsten Jungen in Gryffindor, und sie würden Seite an Seite in meiner Armee im Verteidigungsunterricht kämpfen und sich danach gegenseitig bei den Hausaufgaben helfen -"

"Okay, genug, das fängt an, mich zu gruseln."

"Tut mir leid", sagte Hermine, obwohl sie immer noch vor sich hin lächelte und in eine private Vision entrückt schien.

"Entschuldigung angenommen", sagte Harry trocken.

Die Sonne stieg ein wenig weiter am Himmel auf.

Nach einer Weile sprach Hermine.

"Meinst du, wir werden uns später ineinander verlieben?"

"Ich weiß es auch nicht besser als du, Hermine. Aber warum muss es darum gehen? Ernsthaft, warum muss es immer darum gehen? Vielleicht verlieben wir uns, wenn wir älter sind, und vielleicht auch nicht. Vielleicht bleiben wir verliebt, und vielleicht auch nicht."

Harry drehte den Kopf leicht, die Sonne brannte heiß auf seiner Wange und er trug keine Sonnencreme.

"Egal, wie es läuft, wir sollten nicht versuchen, unser Leben in ein Muster zu zwingen. Ich glaube, wenn Menschen versuchen, Muster zu erzwingen, dann werden sie am Ende unglücklich."

"Keine erzwungenen Muster?" sagte Hermine.

Ihre Augen hatten einen verschmitzten Blick angenommen.

"Das klingt wie eine kompliziertere Art zu sagen, keine Regeln. Was mir viel vernünftiger erscheint, als es noch zu Beginn des Jahres der Fall gewesen wäre. Wenn ich eine funkelnde Einhornprinzessin bin und meine eigene Zeitmaschine habe, kann ich genauso gut auf Regeln verzichten, nehme ich an."

"Ich sage ja nicht, dass Regeln immer schlecht sind, vor allem, wenn sie tatsächlich zu den Menschen passen, anstatt dass sie blindlings nachgeahmt werden wie beim Quidditch. Aber warst du nicht diejenige, die das 'Helden'-Muster abgelehnt hat, um einfach das zu tun, was sie konnte?"

"Ich nehme es an."

Hermine drehte den Kopf wieder, um auf das Gelände unterhalb von Hogwarts zu blicken, denn die Sonne war jetzt zu hell, um sie zu betrachten - obwohl, so dachte Harry, \emph{Hermines Netzhäute jetzt immer heilen würden, deshalb war es für sie allein sicher, direkt ins Licht zu schauen.}

"Du hast gesagt, Harry, dass du denkst, dass ich schon immer dazu bestimmt war, der Held zu sein. Ich habe darüber nachgedacht, und ich vermute, du liegst völlig falsch. Wenn es so bestimmt gewesen wäre, wäre alles viel einfacher gewesen.

Einfach das zu tun, was man kann - das muss man machen, das muss man wählen, immer und immer wieder."

"Das steht vielleicht nicht im Widerspruch dazu, dass du ein auserwählter Held bist", sagte Harry und dachte an compatibilistische Theorien über den freien Willen und Prophezeiungen, die man nicht erfüllen muss, um sie zu erfüllen.

"Aber darüber können wir später reden."

"Du musst es dir aussuchen", wiederholte Hermine.

Sie stieß sich auf den Händen ab, dann ließ sie sich rückwärts auf das Dach fallen und erhob sich mit einer geschmeidigen Bewegung auf ihre Füße.

"So wie ich mich dafür entscheide, das hier zu tun."

"Nicht küssen!" sagte Harry, rappelte sich auf und bereitete sich darauf vor, auszuweichen; allerdings kam ihm die Erkenntnis, dass das Mädchen, das wiederbelebt wurde, viel, viel schneller sein würde.

"Ich werde nicht noch einmal versuchen, dich zu küssen, Mr~Potter. Nicht, bis du mich fragst, falls du es jemals tust. Aber da sind all diese warmen Gefühle, die in mir hochkochen, und ich habe das Gefühl, dass ich platzen könnte, wenn ich nicht etwas tue, obwohl mir jetzt einfällt, dass es ungesund ist, wenn Mädchen keine andere Möglichkeit kennen, sich bei Jungen zu bedanken, als sie zu küssen."

Hermine holte ihren Zauberstab heraus und hielt ihn quer, in der Position, die sie benutzt hatte, um vor dem Zaubergamot ihren Treueeid auf das Haus Potter zu schwören.

"Oh verdammt, nein", sagte Harry. "Ist dir klar, was es gekostet hat, dich das letzte Mal aus dem Schwur herauszuholen -"

"Zieh keine voreiligen Schlüsse, Du. Ich hatte nicht vor, deinem Haus erneut die Treue zu schwören. Du musst anfangen, mir zu vertrauen, dass ich vernünftig bin, wenn du mein geheimnisvoller junger Zauberer sein willst. Jetzt halte bitte deinen Zauberstab heraus."

Langsam nahm Harry den Elderstab heraus und kreuzte ihn mit Hermines zehneinviertel Zentimeter langem Weinholz, wobei er eine letzte Sorge verdrängte, dass sie das Falsche gewählt hatte.

"Kannst du wenigstens nichts über 'bis dass der Tod mich holt' sagen, denn habe ich schon erwähnt, dass ich jetzt den Stein der Weisen habe? Oder irgendetwas über 'das Ende der Welt und ihre Magie'? Ich bin bei solchen Sätzen viel nervöser, als ich es früher war."

Auf einem Dach aus quadratischen Steinfliesen strahlt die strahlende Morgensonne auf zwei Nicht-mehr-wirklich-Kinder herab, beide in blau-gesäumten schwarzen Roben, die sich über gekreuzten Zauberstäben gegenüberstehen.

\emph{Die eine hat braune Augen unter chaotischen Kastanienlocken und strahlt eine Aura von Unschuld und Schönheit aus, die nicht nur magisch ist; die andere hat grüne Augen unter einer Brille, mit unordentlichem schwarzen Haar über einer kürzlich entzündeten Narbe. Unten streckt sich ein steinerner Turm, den niemand vom Boden aus zu sehen glaubt bis in das Grundgestein von Hogwarts herunter. Weit unter ihnen sind die grünen Hügel zu sehen, und der See. In der Ferne eine riesige rot-schwarze Reihe von Eisenbahnwaggons und eine Lokomotive, die aus dieser Höhe winzig erscheint, ein Zug, der weder muggelhaft noch vollständig magisch ist.Der Himmel ist fast unbewölkt, bis auf schwache orange-weiße Schattierungen, wo Feuchtigkeitsfetzen das Sonnenlicht reflektieren. Eine leichte Brise trägt die frische Kühle der Morgendämmerung und die Feuchtigkeit des Morgens mit sich; aber die riesige goldene Kugel ist jetzt hoch über den Horizont gestiegen, und ihre Glut wirft Wärme auf alles, was sie berührt.}

\emph{"Nun, vielleicht bist du danach weniger nervös", sagt die Heldin zu ihrem rätselhaften Zauberer. Sie weiß, dass sie nicht die ganze Geschichte kennt, aber das Fragment der Wahrheit, das sie besitzt, leuchtet hell wie Sonnenlicht in ihr, das Wärme auf ihr Inneres wirft, so wie die Sonne ihr Gesicht wärmt.}

\emph{\hfill\break "Ich entscheide mich für das hier, jetzt.

Bei meinem Leben und meiner Magie schwöre ich Harry Potter Freundschaft,}

\emph{ihm zu helfen und ihm zu vertrauen, ihm beizustehen und manchmal dorthin zu gehen, wo er nicht hingehen kann,}

\emph{bis zu dem Tag, an dem der Tod mich wirklich holt,}

\emph{falls er das jemals tut, meine ich,}

\emph{und falls die Welt oder ihre Magie endet, werden wir damit gemeinsam fertig."}

