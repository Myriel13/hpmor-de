

\hypertarget{rollen-teil-2}{% \section{91. Rollen, Teil 2}\label{rollen-teil-2}}

\textbf{\uline{Rollen, Teil 2}}

Kurz darauf klopfte es erneut an die Tür des Lagerraums.

"Wenn du dich tatsächlich um meine geistige Gesundheit sorgst", sagte der Junge, ohne aufzublicken, "dann gehst du weg, lässt mich in Ruhe und wartest, bis ich zum Essen runterkomme. Das ist nicht hilfreich."

Die Tür öffnete sich, und derjenige, der draußen gewartet hatte, trat ein.

"Ernsthaft?", sagte der Junge flach.

Die Tür schloss sich und klickte hinter Severus Snape. Der Zaubertränkemeister von Hogwarts trug nichts von seiner gewohnten Arroganz oder gar der leidenschaftslosen Miene, die er normalerweise im Büro des Schulleiters annahm; sein Blick war seltsam, als er auf den Jungen herabblickte, der die Tür bewachte; seine Gedanken waren unergründlich.

"Ich kann mir auch nicht vorstellen, was die stellvertretende Schulleiterin denkt", sagte der Zaubertränkemeister von Hogwarts. "Es sei denn, ich soll dir als Warnung dienen, wohin es dich führen wird, wenn du dich entscheidest, die Schuld an ihrem Tod auf dich zu nehmen."

Die Lippen des Jungen pressten sich zusammen.

"Na schön. Gehen wir einfach zum Ende dieser Unterhaltung über. Du hast gewonnen, Professor Snape. Ich gebe zu, dass du für Lily Potters Tod mehr verantwortlich warst als ich für Hermine Grangers Tod, und dass meine Schuld nicht mit deiner Schuld mithalten kann. Und dann bitte ich dich zu gehen und du sagst ihnen, dass es wahrscheinlich das Beste wäre, mich eine Weile in Ruhe zu lassen. Sind wir fertig?"

"Fast", sagte der Meister der Zaubertränke. "Ich bin derjenige, der Miss Granger die Zettel unter das Kopfkissen gelegt hat, auf denen stand, wo die Kämpfe zu finden sind, in die sie eingegriffen hat."

Der Junge reagierte darauf überhaupt nicht.

Schließlich sprach er.

"Weil du Mobbing nicht magst."

"Nicht nur das."

In der Stimme des Zaubertränkemeisters lag ein Ton des Schmerzes, der fremdartig klang; es war schwer vorstellbar, dass es dieselbe harte Stimme war, die Kinder anwies, sich nicht noch einmal zu rühren, sonst würden sie sich die Pulsadern aufschneiden.

"Ich hätte es erkennen müssen … sehr viel früher, nehme ich an, und doch habe ich es gar nicht gesehen, weil ich völlig in mich selbst vertieft war. Dass ich zum Oberhaupt von Slytherin ernannt wurde, bedeutet, dass Albus Dumbledore die Hoffnung völlig verloren hat, dass dem Haus Slytherin geholfen werden kann. Ich bin sicher, dass Dumbledore es versucht hat. Ich kann mir nicht vorstellen, dass er es nicht versucht hat, als er die Leitung von Hogwarts übernahm. Es muss ein schwerer Schlag für ihn gewesen sein, als danach so viele aus Slytherin dem Ruf des Dunklen Lords folgten … er hätte mir nicht die Autorität über dieses Haus übertragen und so gehandelt, wenn er nicht alle Hoffnung verloren hätte."

Die Schultern des Zaubertränkemeisters sanken unter seinem fleckigen Umhang. "Aber du und Miss Granger haben versucht, etwas zu tun, und ihr beide habt sogar geschafft, Mr. Malfoy und Miss Greengrass zu überzeugen, und vielleicht hätten diese beiden ein anderes Beispiel geben können … Ich nehme an, es war töricht von mir, das zu glauben. Der Schulleiter weiß nicht, was ich getan habe, und ich bitte dich, es ihm nicht zu sagen."

"Warum erzählst du mir das?"

"Die Sache ist viel zu ernst geworden, um sie nicht zu erzählen."

Severus Snape verzog die Lippen.

"Ich habe in meiner Amtszeit als Oberhaupt von Slytherin genug verhängnisvolle Verschwörungen gesehen, um zu wissen, wie das manchmal läuft. Wenn in Zukunft alles ans Licht kommen sollte - dann habe ich es dir wenigstens gesagt, und du darfst es auch sagen."

"Wunderbar", sagte der Junge. "Danke, dass du das geklärt hast. Ist das alles?"

"Willst du erklären, dass dein Leben jetzt eine Ruine ist und dass dir nichts anderes übrig bleibt als Rache?"

"Nein. Ich habe immer noch -"

Der Junge unterbrach sich.

"Dann gibt es nur wenig Rat, den ich dir geben kann", sagte Severus Snape.

Der Junge nickte distanziert.

"Im Namen von Hermine danke ich dir, dass du ihr mit den Tyrannen geholfen hast. Sie würde dir sagen, dass es das Richtige war. Und jetzt wäre ich dir sehr verbunden, wenn du ihnen sagen könntest, dass sie mich in Ruhe lassen sollen."

Der Meister der Zaubertränke wandte sich zur Tür, und als sein Gesicht nicht mehr zu sehen war, kam seine Stimme im Flüsterton.

"Es tut mir aufrichtig leid für deinen Verlust."

Severus Snape entfernte sich.

Der Junge starrte ihm hinterher und versuchte sich, so gut es auf diese Entfernung möglich war, an die Worte zu erinnern, die einige Zeit zuvor gesprochen worden waren.

\emph{Deine Bücher haben dich verraten, Potter. Sie haben dir die eine Sache nicht gesagt, die du wissen musstest. Du kannst nicht aus Büchern lernen, wie es ist, den zu verlieren, den du liebst. Das ist etwas, das man nie wissen kann, ohne es selbst erlebt zu haben.}

So ähnlich war es gewesen, dachte der Junge, wenn er sich richtig erinnerte.

Es waren nun Stunden vergangen, in der Krankenstation mit ihrer geschlossenen Tür und einem Körper, der dahinter lag. Harry starrte weiter auf seinen Zauberstab, wie er in seinem Schoß lag. Auf die winzigen Kratzer und Flecken auf den elf Zentimetern der Stechpalme, Fehler, die er noch nie genau genug betrachtet hatte, um sie zu bemerken. Eine schnelle gedankliche Berechnung ergab, dass es keinen Grund zur Sorge gab, denn wenn es sich hier um eine Ansammlung von Schäden aus sechs oder sieben Monaten handelte, dann würde ein normales Leben den Zauberstab nicht vollständig abnutzen.

Damals hätte er sich wahrscheinlich Sorgen darüber gemacht, dass sein eigener Zeitumkehrer weggenommen werden könnte, wenn er einfach offen in die Große Halle gerufen hätte: "Hat jemand einen Zeitumkehrer?", aber es wäre einfach genug gewesen, nach dem Mittagessen jemanden zu finden, der Professor Flitwick zwei Stunden früher eine Nachricht schickt, und dann hätte Professor Flitwick einfach direkt zu Hermine gehen oder ihr seinen Rabenpatronus schicken können, lange bevor der Troll in ihrer Nähe war. Oder hätte dieser alternative Harry bereits erfahren, dass es zu spät war - er erfuhr von Hermines Tod nach dem Mittagessen und bevor er irgendwelche Nachrichten erstellen konnte, die in der Zeit zurückgeschickt wurden? Vielleicht war eine grundlegende Richtlinie bei der Arbeit mit Zeitreisen, dass man nie riskieren sollte, zu spät zu erfahren, wenn man noch nicht rückwärts gereist war.

Am Ende seines Zauberstabs befand sich jetzt eine winzige Verätzung, vermutlich durch den Kontakt mit der Säure, in die er das Gehirn des Trolls teilweise verwandelt hatte, aber der Zauberstab schien robust zu sein, wenn es um den Verlust von kleinen Mengen Holz ging. Wirklich, das Konzept, dass ein '\emph{Zauberstab}' benötigt wurde, wurde immer seltsamer, je mehr man darüber nachdachte. Obwohl, wenn Zaubersprüche immer auf irgendeine mysteriöse Art und Weise erfunden wurden, neue Rituale als neue Hebel an der unbekannten Maschine geschnitzt wurden, könnte es sein, dass die Leute einfach immer wieder Rituale erfanden, die Zauberstäbe involvierten, so wie sie auch Ausdrücke wie "Wingardium Leviosa" erfanden. Es schien wirklich so, als müsste die Magie in gewisser Weise fast beliebig mächtig sein, und es wäre sicherlich praktisch, wenn Harry einfach die konzeptionelle Beschränkung umgehen könnte, die die Menschen daran hinderte, Zaubersprüche wie '\emph{Repariere einfach alles für immer}' zu erfinden, aber irgendwie war nichts jemals so einfach, wenn es um Magie ging.

Harry schaute wieder auf seine mechanische Uhr, aber es war immer noch nicht Zeit. Er hatte versucht, den Patronus-Zauber zu wirken, um seinem Patronus zu sagen, dass er zu Hermine Granger gehen sollte. Nur für den Fall, dass alles eine Lüge war, ein falscher Gedächtniszauber oder eine der wer-weiß-wie-vielen Arten, wie Zauberer dazu gebracht werden konnten, ihre Augen zu schließen und zu träumen. Nur für den Fall, dass die echte Hermine noch lebte und irgendwo festgehalten wurde, obwohl er spürte wie ihr Leben sie verließ. Nur für den Fall, dass es ein Leben nach dem Tod gab und der Wahre Patronus es erreichen konnte. Der Zauber hatte jedoch nicht funktioniert, also hatte dieser spezielle Test keine Beweise geliefert und ihn mit dem vorherigen, ungünstigen Prior zurückgelassen.

Die Zeit verging, und noch mehr Zeit.

Von außen hätte man nur einen Jungen gesehen, der da saß, mit abstraktem Blick auf seinen Zauberstab starrte und alle zwei Minuten oder so auf die Uhr schaute.

Die Tür zum Krankenzimmerbereich öffnete sich ein weiteres Mal. Der Junge, der dort saß, schaute mit einem tödlichen, abschreckenden Blick auf. Dann verzog sich das Gesicht des Jungen vor Entsetzen, und er rappelte sich auf.

"Harry", sagte der Mann im formellen Button-down-Hemd und einer darüber geworfenen schwarzen Weste. Seine Stimme war heiser. "Harry, was ist los? Der Direktor deiner Schule - er ist in diesen lächerlichen Roben in meinem Büro aufgetaucht und hat mir gesagt, dass Hermine Granger tot ist!"

Einen Moment später folgte eine Frau dem Mann in den Raum; sie schien weniger verwirrt als der Mann, weniger verwirrt und mehr verängstigt.

"Dad", sagte der Junge dünn. "Mum. Ja, sie ist tot. Sonst haben sie dir nichts gesagt?"

"Nein! Harry, was ist los?"

Es gab eine Pause.

Der Junge sackte zurück an die Wand.

"Ich k-k-kann nicht, ich kann nicht, ich kann das nicht tun."

"Was?"

"Ich kann nicht so tun, als wäre ich ein kleiner Junge, ich habe im Moment einfach nicht die Kraft dazu."

"Harry", sagte die Frau zögernd. "Harry -"

"Dad, du kennst doch diese Fantasy-Bücher, in denen der Held alles vor seinen Eltern verstecken muss, weil sie, sie würden es nicht verstehen, sie würden dumm reagieren und dem Helden in die Quere kommen? Das ist ein Plot-Device, richtig, damit der Held alles selbst lösen muss, anstatt es seinen Eltern zu erzählen. Bitte sei nicht dieser Plot-Device, Dad, oder du auch nicht, Mum. Spiel… spiel einfach nicht diese Rolle. Seid nicht die Eltern, die es nicht verstehen werden. Schrei mich nicht an und stellt mir elterliche Forderungen, die ich nicht erfüllen kann. Denn ich bin in einen verdammt blöden Fantasy-Roman hineingewandert und jetzt ist Hermine - ich habe einfach nicht die Energie, mich damit zu beschäftigen."

Langsam, als wären seine Gliedmaßen nur halb animiert, kniete der Mann in der schwarzen Weste dorthin, wo Harry stand, so dass seine Augen auf gleicher Höhe mit denen seines Sohnes waren.

"Harry", sagte der Mann. "Du musst mir alles erzählen, was passiert ist, und zwar sofort."

Der Junge holte tief Luft, schluckte.

"Sie sagen mir, dass der Dunkle Lord, den ich besiegt habe, vielleicht noch am Leben ist. Als ob das nicht der P-Plot von 100 beschissenen Büchern wäre, oder? Also, es könnte auch sein, dass der Direktor meiner Schule, der mächtigste Zauberer der Welt, verrückt geworden ist. Und, und Hermine wurde kurz vorher ein Mordversuch angehängt, nicht dass jemand ihren Eltern davon erzählt hätte oder so. Der Schüler, dem der Mordversuch angehängt wurde, war der Sohn von Lucius Malfoy, der der mächtigste Politiker im magischen Britannien ist und früher die Nummer Zwei des Dunklen Lords war. Die Position des Verteidigungsprofessors an dieser Schule ist mit einem Fluch belegt, niemand hält sich jemals länger als ein Jahr, sie haben ein Sprichwort, dass der Verteidigungsprofessor immer ein Verdächtiger ist. In diesem Jahr ist der Verteidigungsprofessor insgeheim ein mysteriöser Zauberer, der sich dem Dunklen Lord während des letzten Krieges widersetzt hat und vielleicht selbst böse ist, vielleicht auch nicht. Außerdem schmachtet der Meister der Zaubertränke schon seit Jahren nach Lily Potter und könnte aus irgendeinem verdrehten psychologischen Grund hinter der ganzen Sache stecken."

Die Lippen des Jungen pressten sich bitter zusammen.

"Ich glaube, das ist der größte Teil des verdammt dummen Plots."

Der Mann, der sich das alles schweigend angehört hatte, stand auf. Er legte dem Jungen sanft eine Hand auf die Schulter. "Das ist genug, Harry", sagte er. "Ich habe genug gehört. Wir werden diese Schule jetzt verlassen und dich mitnehmen."

Die Frau schaute den Jungen an, ihr Gesicht stellte eine Frage.

Der Junge blickte sie an und nickte. Die Stimme der Frau war dünn, als sie sprach. "Sie werden uns nicht lassen, Michael."

"Sie haben kein legales Recht, uns aufzuhalten -"

"Ihr seid Muggel", sagte der Junge. Er lächelte verzerrt. "Ihr habt im magischen britischen Rechtssystem genauso viel Ansehen wie Mäuse. Kein Zauberer wird sich für eure Argumente über Rechte, über Fairness interessieren, sie werden sich nicht einmal die Zeit nehmen, zuzuhören. Du hast keine Macht, also müssen sie sich nicht bemühen. Nein, Mum, ich lächle nicht so, weil ich mit ihrer Muggelpolitik einverstanden bin, ich lächle, weil ich mit eurer Kinderpolitik nicht einverstanden bin."

"Dann", sagte Professor Michael Verres-Evans fest, "werden wir sehen, was die echte Regierung dazu zu sagen hat. Ich kenne ein oder drei Abgeordnete -"

"Die werden sagen: Du bist verrückt, ich wünsche dir einen schönen Aufenthalt in dieser Anstalt. Vorausgesetzt, die Obliviatoren des Ministeriums erwischen dich nicht vorher und löschen dein Gedächtnis. Das machen sie oft mit Muggeln, habe ich gehört. Ich nehme an, die wahren Oberen unserer Regierung haben sich selbst ein paar gemütliche Abmachungen eingerichtet. Vielleicht bekommen sie ab und zu ein paar Heilzauber, wenn es jemand Wichtiges schafft, Krebs zu bekommen."

Der Junge schenkte wieder dieses verdrehte Lächeln.

"Und das ist die Situation, Dad, wie Mum schon weiß. Sie hätten dich nie hierher gebracht oder dir irgendetwas erzählt, wenn du auch nur das Geringste dagegen tun könntest."

Der Mund des Mannes öffnete sich, aber es kamen keine Worte heraus, als hätte er aus einem Skript gelesen, das beschrieb, was ein besorgter Elternteil in einer solchen Situation tun sollte, und dieses Skript war plötzlich an einer leeren Stelle angekommen.

"Harry", sagte die Frau zögernd. Der Junge schaute sie an. "Harry, ist etwas mit dir passiert? Du wirkst … anders …"

"Petunia!", sagte der Mann, und seine Zunge schien wieder zu arbeiten. "Sag so etwas nicht! Er steht unter Stress, das ist alles."

"Nun, Mum, du siehst -" Die Stimme des Jungen brach. "Bist du sicher, dass du das alles auf einmal willst, Mum?"

Die Frau nickte, obwohl sie nicht sprach.

"Ich habe … weißt du noch, wie der Schulpsychiater meinte, ich hätte Probleme mit meiner Wutbewältigung? Nun -" Der Junge hielt inne und schluckte. "Ich weiß nicht, wie ich dir das erklären soll, Mum. Es ist eher etwas Magisches. Wahrscheinlich hat es etwas mit dem zu tun, was in der Nacht passierte, als meine Eltern starben. Ich habe… na ja, ich nannte es eine mysteriöse dunkle Seite und ich weiß, es klingt wie ein Scherz und ich habe nachgefragt bei… bei einem uralten telepathischen Zauberhut, um sicherzugehen, dass meine Narbe nicht tatsächlich vom Geist des Dunklen Lords bewohnt wird und er sagte, dass nur eine Person unter der Krempe ist und ich glaube sowieso nicht, dass Zauberer wirkliche Seelen haben, da sie immer noch einen Hirnschaden haben können, nur -"

"Harry, nicht so schnell!", sagte der Mann.

"- nur, nur was auch immer es ist, es ist immer noch real, es ist etwas in mir, es gab mir Willenskraft, wenn es mir schlecht ging, ich konnte mich allem entgegenstellen, solange ich wütend war, Snape, Dumbledore, dem gesamten Zaubergamot, meine dunkle Seite hatte vor nichts Angst, außer vor Dementoren. Und ich war nicht dumm, ich wusste, dass es einen Preis für den Einsatz meiner dunklen Seite geben könnte und ich schaute weiter, um zu sehen, was der Preis sein könnte. Es veränderte meine Magie nicht, es schien keine permanente Ausrichtungsänderung zu verursachen, es versuchte nicht, mich von meinen Freunden wegzunehmen oder so etwas, also benutzte ich sie weiter, wann immer ich musste, und ich fand erst zu spät heraus, was der Preis wirklich war -"

Die Stimme des Jungen war fast zu einem Flüstern geworden.

"Ich habe es erst heute herausgefunden… jedes Mal, wenn ich es benutze… \emph{es verbraucht meine Kindheit}. Ich habe das Ding getötet, das Hermine geholt hat. Und es war nicht meine dunkle Seite, die das getan hat. Ich war es. Oh, Mum, Dad, es tut mir leid."

Es herrschte eine lange Stille, erfüllt vom Klang zerbrochener Masken.

"Harry", sagte der Mann und kniete sich wieder hin, "ich möchte, dass du noch einmal von vorne anfängst und das viel langsamer erklärst."

Der Junge sprach. Die Eltern hörten zu. Nach einiger Zeit stand der Vater auf. Der Junge blickte zu ihm auf und zog eine bittere Grimasse.

"Harry", sagte der Mann, "Petunia und ich werden dich so schnell wie möglich von hier wegbringen -"

"Nicht", sagte der Junge warnend. "Ich meine es ernst, Dad. Das Zaubereiministerium ist nichts, wogegen du dich wehren kannst. Tu so, als wären sie das Finanzamt oder der Dekan oder etwas anderes, das keine Herausforderung ihrer Dominanz duldet. Im magischen Britannien darf man sich nur an das erinnern, was die Regierung denkt, an dass man sich erinnern sollte, und sich an die Existenz von Magie zu erinnern oder dass man einen Sohn namens Harry hat, ist ein Privileg, kein Recht. Und wenn sie das täten, würde ich zusammenbrechen und das Ministerium in einen riesigen flammenden Krater verwandeln. Mum, du weißt, was Sache ist, du musst Dad unbedingt davon abhalten, etwas Dummes zu versuchen."

"Und Sohn -"

Der Mann rieb sich an den Schläfen.

"Vielleicht sollte ich das jetzt nicht sagen … aber bist du sicher, dass das, wovon du sprichst, wirklich eine magische dunkle Seite ist und nicht etwas Normales für einen Jungen in deinem Alter?"

"Normal", sagte der Junge betont geduldig. "Normal, wie genau? Ich könnte noch einmal nachsehen, aber ich bin mir ziemlich sicher, dass in \emph{Childcraft} nichts darüber stand. Meine dunkle Seite ist nicht nur ein emotionaler Zustand, sie macht mich klüger. Jedenfalls in gewisser Weise. Man kann sich nicht einfach schlauer stellen."

Der Mann rieb sich wieder am Kopf.

"Nun … es gibt ein gewisses bekanntes Phänomen, bei dem Kinder einen biologischen Prozess durchlaufen, der sie manchmal wütend und düster und grimmig macht, und dieser Prozess steigert auch ihre Intelligenz und ihre Körpergröße erheblich -"

Der Junge sackte zurück an die Wand.

"Nein, Dad, es ist nicht so, dass ich mich in einen Teenager verwandle. Ich habe mein Gehirn gecheckt und es denkt immer noch, dass Mädchen eklig sind. Aber wenn es das ist, was du vorgeben willst, dann gut. Vielleicht bin ich besser dran, wenn du mir nicht glaubst. Ich will nur -"

Die Stimme des Jungen erstickte.

"Ich konnte es einfach nicht ertragen, zu lügen."

"Die Adoleszenz funktioniert nicht unbedingt so, Harry. Es kann noch eine Weile dauern, bis du Mädchen bemerkst. Wenn du tatsächlich noch keine bemerkt hast -" und der Mann brach abrupt ab.

"Ich mochte Hermine nicht auf diese Weise", flüsterte der Junge.

"Warum denken immer alle, dass es nur darum gehen muss? Es ist respektlos ihr gegenüber, zu denken, jemand könnte sie nur auf diese Weise mögen."

Der Mann schluckte sichtlich.

"Wie auch immer, mein Sohn, du passt auf dich auf, während wir daran arbeiten, dich hier rauszuholen, ist das klar? Und denk bloß nicht, dass du dich der dunklen Seite zugewandt hast. Ich weiß, du hattest deine, wie ich es nannte, Ender-Wiggin-Momente -"

"Ich denke, wir sind jetzt weit über Ender hinaus und auf dem Weg zu Ender, nachdem die Mistkerle Valentine getötet haben."

"Harry, das ist genau das, was ich sage, dass du nicht glauben sollst", sagte Professor Verres-Evans fest. "Du sollst nicht glauben, dass du zum Bösen wirst. Du sollst niemanden verletzen, dich nicht in Gefahr bringen und auch nicht mit irgendeiner Art von schwarzer Magie herumspielen, während deine Mum und ich daran arbeiten, dich aus dieser Situation zu befreien. Ist das klar, mein Sohn?"

Der Junge schloss die Augen.

"Das wäre ein wunderbarer Rat, Dad, wenn ich nur in einem Comic wäre."

"Harry -", begann der Mann.

"Polizisten können das nicht tun. Soldaten können das nicht tun. Der mächtigste Zauberer der Welt kann das nicht, und er hat es versucht. Es ist den unschuldigen Umstehenden gegenüber nicht fair, Batman zu spielen, wenn man nicht wirklich jeden unter diesem Kodex beschützen kann. Und ich habe gerade bewiesen, dass ich es nicht kann."

Schweißperlen glitzerten auf der Stirn von Professor Michael Verres-Evans.

"Jetzt hör mir mal zu. Egal, was du in Büchern gelesen hast, du sollst niemanden beschützen! Oder dich auf irgendetwas Gefährliches einlassen! Absolut nichts Gefährliches, was auch immer! Halte dich einfach von allem fern, von jedem bisschen Verrücktheit, das in diesem Irrenhaus vor sich geht, während wir dich so schnell wie möglich von hier wegbringen!"

Der Junge schaute suchend zu seinem Vater, dann zu seiner Mutter. Dann schaute er wieder auf seine Armbanduhr.

"Ein guter Punkt", sagte der Junge. Der Junge marschierte zu der Tür, die nach draußen führte, und stieß sie auf. Die Tür flog mit einem Knall auf, der Minerva aufschrecken ließ, wo sie stand, und bevor sie Zeit zum Nachdenken hatte, marschierte Harry Potter aus dem Zimmer und starrte sie direkt an.

"Du hast meine Eltern hierher gebracht", sagte der Junge-der-lebte.

"Nach Hogwarts. Dorthin, wo Du-weißt-schon-wer oder jemand anderes herumlungert und es auf meine Freunde abgesehen hat. Was genau hast du dir dabei gedacht?"

Sie antwortete nicht, dass sie an Harry gedacht hatte, der vor der Tür zum Lagerraum mit Hermines Leiche saß und sich weigerte, sich zu bewegen.

"Wer weiß noch davon?" forderte Harry Potter. "Hat sie jemand bei dir gesehen?"

"Der Schulleiter hat sie hergebracht -"

"Ich will, dass sie sofort von hier verschwinden, bevor jemand anderes es bemerkt, vor allem Du-weißt-schon-wer, aber auch Professor Quirrell oder Professor Snape. Bitte schick deinen Patronus an den Schulleiter und sag ihm, dass er sie sofort zurückbringen muss. Erwähne meine Eltern nicht namentlich oder als Personen, falls jemand anderes zuhört."

"In der Tat", sagte Professor Verres-Evans und nickte streng dazu, von wo aus er direkt hinter dem Jungen stand, Petunia einen Schritt hinter ihm. Seine Hand ruhte fest auf Harrys Schulter. "Wir werden das Gespräch mit unserem Sohn zu Hause zu Ende führen."

"Einen Moment, bitte", sagte Minerva in reflexartiger Höflichkeit.

Ihr erster Versuch, den Patronus zu wirken, schlug fehl, ein Nachteil dieses Zaubers unter bestimmten Umständen. Es war nicht das erste Mal, dass sie es so gemacht hatte, aber sie schien den Dreh raus zu haben - Minerva schaltete den Gedanken aus und konzentrierte sich. Als die Nachricht abgeschickt war, wandte sie sich wieder an Professor Verres-Evans.

"Sir", sagte sie, "ich fürchte, dass Mr. Potter die Hogwarts-Schule nicht verlassen darf -"

Als Albus endlich ankam, gab es ein Geschrei, der Muggelmann hatte seine Würde aufgegeben. Zumindest gab es Geschrei auf einer Seite des Streits. Minervas Herz war nicht dabei. Die Wahrheit war, dass sie den Worten, die aus ihrem Mund kamen, nicht trauen konnte. Als der Professor sich umdrehte, um mit dem Schulleiter zu streiten, meldete sich Harry Potter zu Wort, der die ganze Zeit geschwiegen hatte.

"Nicht hier", sagte Harry. "Du kannst dich überall mit ihm streiten, nur nicht in Hogwarts, Dad. Mum, bitte, bitte sorge dafür, dass Dad nichts versucht, was ihn in Schwierigkeiten mit dem Ministerium bringen könnte."

Michael Verres-Evans' Gesicht verfinsterte sich. Er drehte sich um und sah Harry Potter an. Als seine Stimme herauskam, war sie heiser, begleitet von Wasser in seinen Augen. "Sohn - was tust du da?"

"Du weißt ganz genau, was ich tue", sagte Harry Potter. "Du hast diese Comics gelesen, lange bevor du sie mir gegeben hast. Ich habe einen Haufen Mist durchgemacht, bin ein bisschen gereift, und jetzt beschütze ich meine Verwandten. Eigentlich ist es noch einfacher, du weißt, was ich tue, denn du hast versucht, dasselbe zu tun. Ich lasse meine Angehörigen sofort aus Hogwarts holen, das tue ich. bevor Du-weißt-schon-wer ihre Anwesenheit entdeckt und sie für den Tod markiert."

Michael Verres-Evans begann, hektisch auf Harry zuzupreschen, und dann stoppte jede Bewegung, und der Muggelmann lehnte sich in seiner Flucht nach vorne.

"Es tut mir leid", sagte der Schulleiter leise. "Wir werden uns bald wieder unterhalten. Minerva, ich war bei den anderen, als du anriefst, sie warten in deinem Büro."

Der Schulleiter glitt wie im Flug vorwärts, bis er mitten in dem Raum stand, in dem der Mann und die Frau wie erstarrt standen, und es gab einen weiteren Flammenblitz. Die Bewegung setzte sich wieder fort.

Minerva sah Harry an. Worte kamen ihr nicht über die Lippen.

"Kluger Schachzug, sie hierher zu bringen", sagte Harry Potter. "Wahrscheinlich hat er unsere Beziehung dauerhaft beschädigt. Alles, was ich wollte, war, verdammt noch mal bis zum verdammten Abendessen in Ruhe gelassen zu werden. Was", der Junge schaute auf seine Armbanduhr, "es jetzt sowieso ist. Ich werde mich alleine von Hermine verabschieden, was, das verspreche ich, weniger als zwei Minuten dauern wird, und danach werde ich rauskommen und etwas essen gehen, wie ich es auch sonst getan hätte. Stört mich in diesen zwei verdammten Minuten nicht, sonst raste ich aus und versuche, jemanden zu töten, das meine ich ernst, Professor."

Der Junge drehte sich um und schritt in den kleinen Raum, öffnete die hintere Tür, in der Hermine Grangers Leiche aufbewahrt wurde, und schritt hinein, bevor sie daran denken konnte zu sprechen. Durch die Türöffnung sah sie einen Anblick aufblitzen, von dem sie wusste, dass kein Kind ihn sehen sollte - die Tür schlug zu.

Sie ging vorwärts, ohne nachzudenken. Auf halbem Weg zur Tür hielt sie inne. Ihr Verstand war immer noch langsam und schmerzhaft, und der Teil von ihr, den Harry Potter als das Bild einer strengen Zuchtmeisterin bezeichnet hätte, murmelte leblos Worte über unangemessenes Verhalten von Kindern. Der Rest von ihr hielt es für keine gute Idee, irgendein Kind, selbst Harry Potter, allein in einem Raum mit der blutigen Leiche seines besten Freundes zu lassen. Aber die Tür zu öffnen, oder irgendeine Art von Autorität auszuüben, erschien ihr nicht klug. Es gab nichts Richtiges zu tun und nichts Richtiges zu sagen; oder wenn es einen richtigen Weg gab, wusste sie ihn nicht.

Ganz langsam vergingen eineinhalb Minuten.

Als sich die Tür wieder öffnete, schien Harry sich verändert zu haben, als wären diese anderthalb Minuten im Laufe von Lebenszeiten vergangen.

"Versiegel den Raum", sagte Harry leise, "und dann gehen wir, Professor McGonagall."

Sie ging hinüber zur Tür des Lagerraums. Sie konnte sich nicht ganz davon abhalten, hineinzuschauen, und sah das getrocknete Blut, das Laken, das die untere Hälfte bedeckte, den wächsernen und puppenhaften Oberkörper und einen Blick auf Hermine Grangers geschlossene Augen. Etwas in ihr begann wieder zu weinen. Sie schloss die Tür. Ihre Finger bewegten sich auf ihrem Zauberstab, ihr Mund sprach Worte ohne Gedanken und Zauberstäbe, um den Raum gegen Eindringen zu versiegeln.

"Professor McGonagall", sagte Harry mit einer seltsamen Stimme, wie auswendig gelernt, "haben Sie den Stein? Den Stein, den mir der Schulleiter gegeben hat? Ich sollte ihn wieder in ein Juwel verwandeln, da er sich als nützlich erwiesen hat."

Automatisch wanderte ihr Blick zu dem Ring an Harrys linkem kleinen Finger und bemerkte die Leere der Fassung, wo der Edelstein hätte sein sollen.

"Ich werde es dem Schulleiter gegenüber erwähnen", erwiderte ihre Zunge.

"Ist das eine übliche Taktik, nebenbei bemerkt?" Harry sagte, die Stimme immer noch seltsam. "Etwas Großes zu tragen, das in etwas Kleines verwandelt wurde, um es als Waffe zu benutzen? Oder ist das eine übliche Übung für Verwandlungssübungen?"

Entfernt schüttelte sie den Kopf.

"Nun, dann lass uns gehen."

"Ich habe -", ihre Stimme stockte. "Ich fürchte, ich habe noch etwas anderes zu tun, und zwar jetzt. Kommen Sie allein zurecht, und versprechen Sie mir, direkt in die Große Halle zu gehen und etwas zu essen, Mr. Potter?

Der Junge versprach es (abgesehen von außergewöhnlichen und unvorhergesehenen Umständen, eine Klausel, mit der sie sich nicht auseinandersetzte) und ging dann aus dem Zimmer.

\emph{Was vor ihr lag … würde sicher nicht einfacher werden, und es könnte durchaus schwieriger werden.}

Minerva ging in zügigem Tempo zu ihrem Büro; nicht langsam, denn das wäre eine Unhöflichkeit gewesen. Professor McGonagall öffnete die Tür zu ihrem Büro.

"Madam Granger", sagte ihre Stimme,

"Mr. Granger, es tut mir so furchtbar leid für -"

