

\hypertarget{dominanzhierarchien}{% \section{18. Dominanzhierarchien}\label{dominanzhierarchien}}

\textbf{Dominanzhierarchien}

\emph{"Das klingt nach einer Sache, die ich tun würde, oder?"}

Es war Frühstückszeit am Freitagmorgen.

Harry nahm noch einen großen Bissen von seinem Toast und versuchte dann, sein Gehirn daran zu erinnern, dass das Verschlingen seines Frühstücks ihn nicht wirklich schneller in die Kerker bringen würde.

Immerhin hatten sie zwischen dem Frühstück und dem Beginn von Zaubertränke eine volle Stunde Lernzeit.

\emph{Aber Kerker! In Hogwarts!}

Harrys Vorstellungskraft skizzierte bereits die Abgründe, die schmalen Brücken, die fackelbeleuchteten Schornsteine und die Flecken aus glühendem Moos.

\emph{Würde es dort Ratten geben? Würde es Drachen geben?}

"Harry Potter", sagte eine leise Stimme hinter ihm.

Harry schaute über seine Schulter und erblickte Ernie Macmillan, der elegant in einen gelbgestreiften Umhang gekleidet war und etwas besorgt aussah.

"Neville dachte, ich sollte dich warnen", sagte Ernie mit leiser Stimme.

"Ich glaube, er hat recht.

Nehm dich vor dem Zaubertränkemeister in unserer heutigen Stunde in Acht. Die älteren Hufflepuffs haben uns erzählt, dass Professor Snape sehr böse zu Leuten sein kann, die er nicht mag, und er mag die meisten Leute nicht, die keine Slytherins sind.

Wenn du etwas Kluges zu ihm sagst, könnte das sehr schlecht für dich sein, habe ich gehört. Halte dich einfach bedeckt und gib ihm keinen Grund, dich zu bemerken."

Es gab eine Pause, als Harry das verarbeitete, und dann hob er die Augenbrauen.

(Harry wünschte, er könnte nur eine Augenbraue hochziehen, wie Spock, aber das hatte er noch nie gekonnt.)

"Danke", sagte Harry. "Du hast mir vielleicht gerade eine Menge Ärger erspart."

Ernie nickte und drehte sich um, um zurück zum Hufflepuff-Tisch zu gehen.

Harry aß weiter seinen Toast. Es war ungefähr nach vier Bissen, als jemand

"Verzeihung"

sagte und Harry sich umdrehte, um einen älteren Ravenclaw zu sehen, der ein wenig besorgt aussah -

Einige Zeit später beendete Harry seinen dritten Teller Krapfen.

(Er hatte gelernt, beim Frühstück viel zu essen. Er konnte immer leicht zu Mittag essen, wenn er nicht den Zeitumkehrer benutzen musste.)

Und wieder ertönte eine Stimme hinter ihm:

"Harry?"

"Ja", sagte Harry müde,

"ich werde versuchen, Professor Snapes Aufmerksamkeit nicht zu erregen -"

"Oh, das ist hoffnungslos", sagte Fred.

"Völlig hoffnungslos", sagte George.

"Also haben wir die Hauselfen gebeten, dir einen Kuchen zu backen", sagte Fred.

"Für jeden Punkt, den du für Ravenclaw verlierst, stellen wir eine Kerze drauf", sagte George.

"Und während des Mittagessens eine Party für dich am Gryffindor-Tisch veranstalten", sagte Fred.

"Wir hoffen, dass dich das nachher aufmuntert", schloss George.

Harry schluckte seinen letzten Bissen Rinderwurst hinunter und drehte sich um.

"In Ordnung", sagte Harry.

"Ich wollte das eigentlich nicht nach Professor Binns fragen, wirklich nicht, aber wenn Professor Snape so schrecklich ist, warum wurde er dann nicht gefeuert?"

"Gefeuert?", fragte Fred.

"Du meinst, entlassen?", fragte George.

"Ja", sagte Harry.

"Das macht man mit schlechten Lehrern.

Man feuert sie. Dann stellt man stattdessen einen besseren Lehrer ein. Hier gibt es keine Gewerkschaften und keine Festanstellung, richtig?"

Fred und George runzelten die Stirn, so wie die Stammesältesten der Jäger und Sammler die Stirn runzeln würden, wenn man versuchte, ihnen etwas über Infinitesimalrechnung zu erzählen.

"Ich weiß es nicht", sagte Fred nach einer Weile.

"Daran habe ich nie gedacht."

"Ich auch nicht", sagte George.

"Ja", sagte Harry,

"das höre ich oft. Wir sehen uns beim Mittagessen, Leute, und gebt mir nicht die Schuld, wenn keine Kerzen auf dem Kuchen sind."

Fred und George lachten beide, als ob Harry etwas Lustiges gesagt hätte, verbeugten sich vor ihm und gingen zurück in Richtung Gryffindor.

Harry wandte sich wieder dem Frühstückstisch zu und schnappte sich einen Muffin. Sein Magen fühlte sich schon voll an, aber er hatte das Gefühl, dass dieser Morgen eine Menge Kalorien verbrauchen könnte.

Während er sein Törtchen aß, dachte Harry an den schlimmsten Lehrer, dem er bisher begegnet war, Professor Binns von Geschichte.

Professor Binns war ein Geist. Nach dem, was Hermine über Geister gesagt hatte, schien es unwahrscheinlich, dass sie sich ihrer selbst voll bewusst waren.

Es gab keine berühmten Entdeckungen, die von Geistern gemacht wurden, oder viel von irgendwelchen originellen Arbeiten, egal wer sie im Leben gewesen waren.

Geister neigten dazu, sich nur schwer an das aktuelle Jahrhundert zu erinnern. Hermine hatte gesagt, sie seien wie zufällige Porträts, die durch einen Ausbruch von psychischer Energie, der den plötzlichen Tod eines Zauberers begleitete, in die umgebende Materie eingeprägt wurden.

Harry hatte bei seinen missglückten Ausflügen in die normale Muggelpädagogik schon einige dumme Lehrer kennengelernt -

sein Vater war natürlich viel wählerischer gewesen, wenn es um die Auswahl von Studenten als Tutoren ging -,

aber im Geschichtsunterricht war er zum ersten Mal einem Lehrer begegnet, der buchstäblich nicht empfindungsfähig war.

Und das zeigte sich auch. Harry hatte nach fünf Minuten aufgegeben und angefangen, ein Lehrbuch zu lesen. Als klar wurde, dass \emph{"Professor Binns"} nichts dagegen hatte, hatte Harry auch in seine Tasche gegriffen und sich Ohrstöpsel geholt.

\emph{Brauchten Geister kein Gehalt? War es das? Oder war es buchstäblich unmöglich, jemanden in Hogwarts zu feuern, selbst wenn er gestorben war?}

Jetzt schien es so, dass Professor Snape dabei war, zu jedem, der kein Slytherin war, absolut schrecklich zu sein, und es war niemandem in den Sinn gekommen, seinen Vertrag zu kündigen. Und der Schulleiter hatte ein Huhn angezündet.

"Entschuldigen Sie", kam eine besorgte Stimme von hinten.

"Ich schwöre", sagte Harry, ohne sich umzudrehen,

"dieser Ort ist fast achteinhalb Mal so schlimm wie das, was Dad über Oxford sagt."

Harry stapfte die steinernen Korridore hinunter und sah beleidigt, verärgert und wütend zugleich aus.

"Kerker!"

Harry zischte.

"Kerker! Das sind keine Kerker! Das ist ein Keller! Ein Keller!???"

Einige der Ravenclaw-Mädchen warfen ihm seltsame Blicke zu.

Die Jungs waren inzwischen alle an ihn gewöhnt. Es schien, dass das Stockwerk, in dem sich der Zaubertränke-Klassenraum befand, aus keinem besseren Grund als dem, dass es unter der Erde lag und etwas kälter war als das Hauptschloss, \emph{"Kerker"} genannt wurde.

\emph{In Hogwarts! In Hogwarts!}

Harry hatte sein ganzes Leben lang gewartet, und jetzt wartete er immer noch, und wenn es irgendwo auf der Welt anständige Kerker geben sollte, dann in Hogwarts!

Sollte Harry etwa sein eigenes Schloss bauen müssen, wenn er einen kleinen bodenlosen Abgrund sehen wollte?

Kurze Zeit später kamen sie zum eigentlichen Zaubertrank-Klassenzimmer und Harry munterte sich merklich auf.

Im Zaubertrank-Klassenzimmer gab es seltsame konservierte Kreaturen, die in riesigen Gläsern auf Regalen schwammen, die jeden Zentimeter Wandfläche zwischen den Schränken bedeckten.

Harry war in seiner Lektüre inzwischen so weit fortgeschritten, dass er einige der Kreaturen tatsächlich identifizieren konnte, wie zum Beispiel den Zabriskan Fontema.

Die fünfzig Zentimeter große Spinne sah zwar aus wie eine Acromantula, war aber zu klein, um eine zu sein.

Er hatte versucht, Hermine zu fragen, aber sie schien nicht sehr daran interessiert zu sein, auch nur in die Nähe der Stelle zu schauen, auf die er zeigte.

Harry schaute gerade auf eine große Staubkugel mit Augen und Füßen, als der \emph{Attentäter} in den Raum fegte.

Das war der erste Gedanke, der Harry in den Sinn kam, als er Professor Severus Snape sah. Es lag etwas Ruhiges und Tödliches in der Art, wie der Mann zwischen den Tischen der Kinder hindurchpirschte.

Seine Roben waren dreckig, sein Haar fleckig und fettig. Er hatte etwas an sich, das an Lucius erinnerte, obwohl die beiden sich nicht im Entferntesten ähnlich sahen, und man hatte den Eindruck, dass dort, wo Lucius einen mit makelloser Eleganz töten würde, dieser Mann einen einfach umbringen würde.

"Setzen Sie sich", sagte Professor Severus Snape. "Sofort."

Harry und ein paar andere Kinder, die herumstanden und sich unterhielten, drängelten sich um die Tische.

Harry hatte geplant, neben Hermine zu landen, aber irgendwie fand er sich auf dem nächstbesten leeren Tisch neben Justin Finch-Fletchley wieder

(es war eine Doppelstunde, Ravenclaw und Hufflepuff),

was ihn zwei Tische links von Hermine platzierte.

Severus setzte sich hinter das Lehrerpult und sagte ohne die geringste Überleitung oder Vorstellung:

"Hannah Abbott."

"Hier", sagte Hannah mit etwas zittriger Stimme.

"Susan Bones."

"Anwesend."

Und so ging es weiter, niemand wagte es, ein Wort zu sagen, bis..:

"Ah, ja.

Harry Potter. Unsere neue… Berühmtheit."

"Die Berühmtheit ist anwesend, Sir."

Die Hälfte der Klasse zuckte zusammen, und einige der Klügeren sahen plötzlich so aus, als wollten sie aus der Tür rennen, solange das Klassenzimmer noch da war.

Severus lächelte auf eine erwartungsvolle Art und Weise und rief den nächsten Namen auf seiner Liste auf.

Harry stieß einen mentalen Seufzer aus. Das war viel zu schnell passiert, als dass er etwas dagegen hätte tun können.

Nun gut. Offensichtlich mochte dieser Mann ihn bereits nicht, aus welchem Grund auch immer. Und wenn Harry darüber nachdachte, war es bei weitem besser, dass dieser Zaubertrank-Professor auf ihm herumhackte als, sagen wir, auf Neville oder Hermine.

Harry war viel besser in der Lage, sich zu verteidigen. Ja, wahrscheinlich war das alles zum Besten.

Als die Anwesenheitsliste vollständig war, ließ Severus seinen Blick über die ganze Klasse schweifen. Seine Augen waren so leer wie ein Nachthimmel ohne Sterne.

"Ihr seid hier", sagte Severus mit leiser Stimme, die die Schüler im Hintergrund angestrengt zu hören versuchten,

"um die subtile Wissenschaft und exakte Kunst der Zaubertrankherstellung zu erlernen. Da hier wenig mit Zauberstäben herumgefuchtelt wird, werden viele von euch kaum glauben, dass es sich um Magie handelt.

Ich erwarte nicht, dass ihr die Schönheit des sanft köchelnden Kessels mit seinen schimmernden Dämpfen wirklich verstehen werdet, die zarte Kraft der Flüssigkeiten, die durch die menschlichen Adern kriechen",

dies in einem eher schmeichelnden, hämischen Ton,

"die den Geist betören, die Sinne umgarnen",

\emph{dies wurde gerade immer gruseliger.}

"Ich kann euch beibringen, wie man Glück in Flaschen abfüllt, Ruhm zusammenbraut und sogar den Tod aufhält -

wenn ihr nicht so eine große Meute von Dummköpfen seid, wie ich es sonst zu lehren habe."

Severus schien irgendwie den skeptischen Blick auf Harrys Gesicht zu bemerken, oder zumindest sprang sein Blick plötzlich dorthin, wo Harry saß.

"Potter!", schnauzte der Zaubertränke-Professor.

"Was würde ich bekommen, wenn ich pulverisierte Asphodelwurzel zu einem Aufguss aus Wermut hinzufügen würde?"

Harry blinzelte.

"Steht das in zauberhafte Zaubertränke?", fragte er.

"Ich habe es gerade zu Ende gelesen, und ich kann mich an nichts erinnern, wo Wermut verwendet wurde -"

Hermine hob die Hand und Harry warf ihr einen Blick zu, der sie dazu veranlasste, ihre Hand noch höher zu heben.

"Tz, tz", sagte Severus seidenweich.

"Ruhm ist offensichtlich nicht alles."

"Wirklich?" sagte Harry.

"Aber Sie haben uns doch gerade gesagt, Sie würden uns beibringen, wie man Ruhm in Flaschen füllt. Wie funktioniert das genau? Man trinkt es und wird zur Berühmtheit?"

Drei Viertel der Klasse zuckten zusammen.

Hermines Hand sank langsam wieder nach unten. Nun, das war nicht überraschend. Sie mochte seine Rivalin sein, aber sie war nicht die Art von Mädchen, die mitspielen würde, nachdem klar wurde, dass der Professor ihn absichtlich zu demütigen versuchte.

Harry bemühte sich, sein Temperament unter Kontrolle zu halten. Die erste Erwiderung, die ihm in den Sinn kam, war "\emph{Abrakadabra}".

"Versuchen wir es noch einmal", sagte Severus.

"Potter, wo würdest du suchen, wenn ich dir sage, du sollst einen Bezoar finden?"

"Das steht auch nicht im Lehrbuch", sagte Harry,

"aber in einem Muggelbuch habe ich gelesen, dass ein Trichinobezoar eine Masse aus verfestigten Haaren ist, die man in einem menschlichen Magen findet, und die Muggel glaubten früher, dass es jedes Gift heilen würde -"

"Falsch", sagte Severus.

"Ein Bezoar befindet sich im Magen einer Ziege, er ist nicht aus Haar, und er heilt die meisten Gifte, aber nicht alle."

"Ich habe nicht gesagt, dass es das tut, ich habe gesagt, dass ich das in einem Muggelbuch gelesen habe -"

"Niemand hier interessiert sich für deine erbärmlichen Muggelbücher.

Letzter Versuch. Was ist der Unterschied, Potter, zwischen Mönchsblut und Wolfseisenhut?"

Das genügte.

"Wissen Sie", sagte Harry eisig,

"in einem meiner ziemlich faszinierenden Muggelbücher wird eine Studie beschrieben, in der es Menschen gelungen ist, sich selbst sehr klug aussehen zu lassen, indem sie Fragen über zufällige Fakten stellten, die nur sie kannten.

Anscheinend haben die Zuschauer nur bemerkt, dass die Fragesteller etwas wussten und die Antwortenden nicht, und haben es versäumt, die Unfairness des zugrunde liegenden Spiels zu bedenken.

Also, Professor, können Sie mir sagen, wie viele Elektronen sich im äußersten Orbital eines Kohlenstoffatoms befinden?"

Severus' Lächeln wurde breiter.

"Vier", sagte er. "Das ist allerdings eine nutzlose Tatsache, die sich niemand aufschreiben sollte.

Und zu deiner Information, Potter, Asphodel und Wermut ergeben einen Schlaftrunk, der so mächtig ist, dass er als der Trank des lebenden Todes bekannt ist.

Was den Wolfseisenhut und den Mönchsblut betrifft, so handelt es sich um dieselbe Pflanze, die auch unter dem Namen Eisenhut bekannt ist, wie Sie wissen würden, wenn Sie Tausend magische Kräuter und Pilze gelesen hätten.

Du dachtest wohl, du müsstest das Buch nicht aufschlagen, bevor du kommst, was, Potter? Alle anderen von euch sollten sich das abschreiben, damit ihr nicht so unwissend seid wie er."

Severus hielt inne und sah recht zufrieden mit sich selbst aus.

"Und das wären dann… fünf Punkte? Nein, machen wir doch gleich zehn Punkte von Ravenclaw für Widerrede."

Hermine keuchte auf, zusammen mit einer Reihe anderer.

"Professor Severus Snape", stieß Harry hervor.

"Ich weiß von nichts, was ich getan habe, um Ihre Feindschaft zu verdienen.

Wenn Sie ein Problem mit mir haben, von dem ich nichts weiß, dann schlage ich vor, dass wir -"

"Halt die Klappe, Potter.

Zehn weitere Punkte von Ravenclaw. Der Rest von euch, öffnet eure Bücher auf Seite 3."

Es gab nur ein leichtes, nur ein sehr schwaches Brennen hinten in Harrys Kehle und überhaupt keine Feuchtigkeit in seinen Augen.

Wenn Weinen keine effektive Strategie war, um diesen Zaubertrank-Professor zu vernichten, dann hatte es keinen Sinn, zu weinen.

Langsam setzte sich Harry ganz gerade auf. Sein ganzes Blut schien weggesaugt und durch flüssigen Stickstoff ersetzt worden zu sein.

Er wusste, dass er versucht hatte, sich zu beherrschen, aber er konnte sich nicht erinnern, warum.

"Harry", flüsterte Hermine zwei Tische weiter verzweifelt,

"hör auf, bitte, es ist alles in Ordnung, wir werden es nicht zählen -"

"Reden im Unterricht, Granger? Drei -"

"Also", sagte eine Stimme, die kälter war als null Kelvin,

"wie geht man vor, um eine formelle Beschwerde gegen einen beleidigenden Professor einzureichen? Spricht man mit der stellvertretenden Schulleiterin, schreibt man einen Brief an die Schulräte… Würden Sie uns bitte erklären, wie das funktioniert?"

In der Klasse herrschte absolute Stille.

"Nachsitzen für einen Monat, Potter", sagte Severus und lächelte noch breiter.

"Ich weigere mich, Ihre Autorität als Lehrer anzuerkennen, und ich werde kein Nachsitzen, das Sie mir auferlegen, absitzen."

Die Leute hörten auf zu atmen.

Severus' Lächeln verschwand.

"Dann werden Sie -"

seine Stimme stockte kurz.

"Rausgeschmissen, wollten Sie gerade sagen?" Harry hingegen lächelte nun dünn.

"Aber dann schienen Sie an Ihrer Fähigkeit zu zweifeln, die Drohung wahr zu machen, oder die Konsequenzen zu fürchten, wenn Sie es täten.

Ich hingegen habe weder Zweifel noch Angst davor, eine Schule mit weniger beleidigenden Professoren zu finden.

Oder vielleicht sollte ich, wie ich es gewohnt bin, Nachhilfelehrer engagieren und mich in meinem vollen Lerntempo unterrichten lassen.

Ich habe genug Geld in meinem Tresor. Irgendwas wegen Kopfgeldern auf einen Dunklen Lord, den ich besiegt habe.

Aber es gibt Lehrer in Hogwarts, die ich ziemlich mag, also denke ich, es wird einfacher sein, wenn ich stattdessen einen Weg finde, Sie loszuwerden."

"Mich loswerden?" sagte Severus und lächelte nun ebenfalls dünn.

"Was für eine amüsante Einbildung. Wie willst du das anstellen, Potter?"

"Ich habe gehört, dass es eine Reihe von Beschwerden von Schülern und ihren Eltern über Sie gegeben hat", \emph{eine Vermutung, aber eine sichere,}

"was nur die Frage offen lässt, warum Sie nicht schon längst weg sind.

Ist Hogwarts zu knapp bei Kasse, um sich einen echten Zaubertrank-Professor zu leisten? Ich könnte etwas beisteuern, falls ja.

Ich bin sicher, sie könnten einen besseren Lehrer finden, wenn sie das Doppelte Ihres jetzigen Gehalts bieten würden."

Zwei Eisstangen die Augen waren strahlten eisigen Winter über dem Klassenzimmer aus.

"Sie werden feststellen", sagte Severus leise,

"dass der Schulrat nicht das geringste Verständnis für Ihr Angebot hätte."

"Lucius …" sagte Harry.

"Deshalb sind Sie noch hier. Vielleicht sollte ich mich mit Lucius darüber unterhalten. Ich glaube, er wünscht, sich mit mir zu treffen. Ich frage mich, ob ich irgendetwas habe, was er will?"

Hermine schüttelte verzweifelt den Kopf.

Harry bemerkte es aus dem Augenwinkel, aber seine Aufmerksamkeit war ganz auf Severus gerichtet.

"Du bist ein sehr dummer Junge", sagte Severus.

Er lächelte jetzt überhaupt nicht mehr.

"Du hast nichts, was Lucius mehr schätzt als meine Freundschaft. Und wenn doch, dann habe ich andere Verbündete."

Seine Stimme wurde hart.

"Und ich finde es immer unwahrscheinlicher, dass du nicht nach Slytherin sortiert wurdest. Wie hast du es geschafft, nicht in mein Haus zu kommen? Ah, ja, weil der Sprechende Hut behauptet hat, es sei ein Scherz. Zum ersten Mal in der aufgezeichneten Geschichte. Worüber hast du wirklich mit dem Sprechenden Hut geplaudert, Potter? Hattest du etwas, was er wollte?"

Harry starrte in Severus' kalten Blick und erinnerte sich daran, dass der Sprechende Hut ihn gewarnt hatte, niemandem in die Augen zu sehen, während er darüber nachdachte - Harry ließ seinen Blick auf Severus' Schreibtisch fallen.

"Du scheinst mir seltsam unwillig zu sein, mir in die Augen zu sehen, Potter!"

Ein Schock plötzlichen Verstehens -

"Also warst du es, vor dem mich der Sprechende Hut gewarnt hat!"

"Was?", sagte Severus' Stimme und klang aufrichtig überrascht, obwohl Harry natürlich nicht in sein Gesicht sah.

Harry erhob sich von seinem Schreibtisch.

"Setz dich, Potter", sagte eine wütende Stimme von irgendwoher, wo er nicht hingesehen hatte.

Harry ignorierte sie und sah sich im Klassenzimmer um.

"Ich habe nicht vor, mir von einem unprofessionellen Lehrer meine Zeit in Hogwarts ruinieren zu lassen", sagte Harry mit tödlicher Ruhe.

"Ich denke, ich werde mich von dieser Klasse verabschieden und entweder einen Tutor anheuern, der mir Zaubertränke beibringt, während ich hier bin, oder, wenn das Kollegium wirklich so verschlossen ist, über den Sommer lernen.

Wenn jemand von euch beschließt, dass er sich von diesem Mann nicht schikanieren lassen will, stehen meine Stunden für euch offen."

\textbf{"Setz dich, Potter!"}

Harry schritt durch den Raum und griff nach dem Türknauf. Sie ließ sich nicht drehen. Harry drehte sich langsam um und erhaschte einen Blick auf Severus' fieses Lächeln, bevor er sich daran erinnerte, wegzusehen.

"Mach die Tür auf."

"Nein", sagte Severus.

"Du gibst mir das Gefühl, bedroht zu werden", sagte eine Stimme, die so eisig war, dass sie überhaupt nicht wie Harrys klang, "und das ist ein Fehler."

Severus' Stimme lachte.

"Was gedenkst du dagegen zu tun, kleiner Junge?"

Harry machte sechs lange Schritte vorwärts, weg von der Tür, bis er in der Nähe der hinteren Reihe der Tische stand.

Dann richtete sich Harry auf und hob seine rechte Hand in einer schrecklichen Bewegung, die Finger zum Schnippen bereit.

Neville schrie auf und tauchte unter sein Pult. Andere Kinder wichen zurück oder hoben instinktiv ihre Arme, um sich zu schützen.

"Harry, nicht!", kreischte Hermine. "Was auch immer du mit ihm vorhattest, tu es nicht!"

"\textbf{Seid ihr alle verrückt geworden?!}", bellte Severus' Stimme.

Langsam ließ Harry seine Hand sinken.

"Ich hatte nicht vor, ihn zu verletzen, Hermine", sagte Harry, seine Stimme etwas leiser.

"Ich wollte gerade die Tür in die Luft jagen."

Obwohl, jetzt, wo Harry sich daran erinnerte, durfte man keine Dinge verwandeln, die verbrannt werden sollten, was bedeutete, dass es vielleicht keine so gute Idee war, in der Zeit zurückzugehen und Fred oder George dazu zu bringen, eine sorgfältig bemessene Menge Sprengstoff zu verwandeln…

"Silencio", sagte Severus' Stimme.

Harry versuchte, "\emph{Was?"} zu sagen und stellte fest, dass kein Ton herauskam.

"Das ist lächerlich geworden. Ich denke, du hast dir für einen Tag genug Ärger eingehandelt, Potter.

Du bist der störendste und widerspenstigste Schüler, den ich je gesehen habe, und ich weiß nicht, wie viele Punkte Ravenclaw im Moment hat, aber ich bin sicher, dass ich es schaffen kann, sie alle auszulöschen.

10 Punkte von Ravenclaw.

10 Punkte von Ravenclaw.

10 Punkte von Ravenclaw!

\textbf{50 Punkte von Ravenclaw!}

Und jetzt setz dich hin und sieh zu, wie der Rest der Klasse ihren Unterricht absolviert!"

Harry steckte die Hand in seinen Beutel und versuchte, "\emph{Marker}" zu sagen, aber natürlich kamen keine Worte heraus.

Für einen kurzen Moment hielt ihn das auf; und dann fiel Harry ein, M-A-R-K-E-R mit Fingerbewegungen zu buchstabieren, was funktionierte.

B-L-O-C-K und er hatte einen Block Papier. Harry ging hinüber zu einem leeren Schreibtisch, nicht dem, an dem er sich ursprünglich hingesetzt hatte, und kritzelte eine kurze Nachricht.

Er riss das Blatt Papier ab, steckte den Marker und den Block in eine Tasche seines Umhangs, um schneller darauf zugreifen zu können, und hielt seine Nachricht hoch, nicht an Snape, sondern an den Rest der Klasse.

ICH GEHE JETZT, MUSS NOCH JEMAND RAUS?

"Du bist wahnsinnig, Potter", sagte Severus mit kalter Verachtung.

Abgesehen davon sprach niemand. Harry machte eine ironische Verbeugung vor dem Lehrerpult, ging zur Wand hinüber und riss mit einer geschmeidigen Bewegung eine Schranktür auf, trat hinein und schlug die Tür hinter sich zu.

Es gab das dumpfe Geräusch von jemandem, der mit den Fingern schnippte, und dann nichts mehr. Im Klassenzimmer sahen sich die Schüler verwirrt und ängstlich an.

Das Gesicht des Zaubertränkemeisters war nun völlig wütend. Mit schrecklichen Schritten durchquerte er den Raum und riss die Schranktür auf.

Der Schrank war leer.

\emph{Eine Stunde zuvor lauschte} Harry aus dem Inneren des geschlossenen Schranks.

Von draußen war kein Geräusch zu hören, und es hatte auch keinen Sinn, ein Risiko einzugehen. U-M-H-A-N-G, buchstabierten seine Finger.

Sobald er unsichtbar war, riss er ganz vorsichtig und langsam die Schranktür auf und spähte hinaus.

Es schien niemand im Klassenzimmer zu sein. Die Tür war nicht verschlossen. Erst als Harry außerhalb des gefährlichen Ortes und im Flur war, sicher unsichtbar, verflog etwas von der Wut und er realisierte, was er gerade getan hatte.

\textbf{\emph{Was er gerade getan hatte}}.

Harrys unsichtbares Gesicht war in absolutem Entsetzen erstarrt. Er hatte einen Lehrer um drei Größenordnungen mehr verärgert als alles, was er je zuvor geschafft hatte.

Er hatte gedroht, Hogwarts zu verlassen und würde es vielleicht auch tun. Er hatte alle Punkte von Ravenclaw verloren und dann hatte er den Zeitumkehrer benutzt… Seine Phantasie zeigte ihm, wie seine Eltern ihn anbrüllten, nachdem er von der Schule verwiesen worden war, wie Professor McGonagall von ihm enttäuscht war, und es war einfach zu schmerzhaft und er konnte es nicht ertragen und er konnte sich nicht vorstellen, wie er sich retten sollte -

Der Gedanke, den Harry sich erlaubte zu denken, war, dass, wenn das Wütendwerden ihn in all diese Schwierigkeiten gebracht hatte, er dann vielleicht, wenn er wütend war, einen Ausweg finden würde, die Dinge schienen irgendwie klarer, wenn er wütend war.

Und der Gedanke, den Harry nicht zuließ, war, dass er sich dieser Zukunft einfach nicht stellen konnte, wenn er nicht wütend war.

Also warf er seine Gedanken zurück und erinnerte sich an die brennende Demütigung -

\emph{Tut, tut. Ruhm ist offensichtlich nicht alles. Zehn Punkte von Ravenclaw für Widerrede.}

Die beruhigende Kälte schwappte durch seine Adern zurück wie eine Welle, die von irgendeinem Brecher reflektiert wurde und zurückkehrte, und Harry ließ seinen Atem aus.

Okay. Jetzt ist er wieder bei Verstand. Er fühlte sich tatsächlich ein bisschen enttäuscht von seinem nicht-ängstlichen Ich, weil er so zusammengebrochen war und nur noch aus dem Ärger heraus wollte.

Professor Severus Snape war jedermanns Problem. Normal-Harry hatte das vergessen und wünschte sich einen Weg, sich selbst zu schützen.

\emph{Und all die anderen Opfer hängen zu lassen?}

Die Frage war nicht, wie er sich schützen konnte, die Frage war, wie er diesen Zaubertrank-Professor vernichten konnte.

\emph{Das ist also meine dunkle Seite, ja? Ein etwas voreingenommener Begriff, meine helle Seite scheint eher egoistisch und feige zu sein, um nicht zu sagen, verwirrt und panisch.}

Und jetzt, wo er wieder klar denken konnte, war auch klar, was er als Nächstes tun sollte. Er hatte sich bereits eine zusätzliche Stunde zur Vorbereitung gegeben und könnte bei Bedarf bis zu fünf Stunden mehr bekommen…

Später.

Minerva McGonagall wartete im Büro des Schulleiters. Dumbledore saß in seinem gepolsterten Thron hinter seinem Schreibtisch, gekleidet in vier Lagen formeller lavendelfarbener Roben.

Minerva saß in einem Stuhl vor ihm, Severus gegenüber in einem anderen Stuhl. Den dreien gegenüber stand ein leerer Holzschemel.

Sie warteten auf Harry Potter.

\emph{Harry}, dachte Minerva verzweifelt, \emph{du hast versprochen, dass du keine Lehrer beißen würdest!}

Und in ihrem Kopf sah sie ganz deutlich die Antwort, Harrys wütendes Gesicht und seine empörte Reaktion: \emph{Ich sagte, ich würde niemanden beißen, der mich nicht zuerst beißt!}

Es klopfte an der Tür.

"Herein!" rief Dumbledore. Die Tür schwang auf, und Harry Potter trat ein.

Minerva keuchte fast laut auf. Der Junge sah kühl und gefasst aus und hatte sich vollkommen unter Kontrolle.

"Guten Mor-"

Harrys Stimme brach plötzlich ab. Seine Kinnlade fiel herunter. Minerva folgte Harrys Blick und sah, dass Harry auf Fawkes starrte, wo der Phönix auf seiner goldenen Stange saß.

Fawkes flatterte mit seinen leuchtend rotgoldenen Flügeln wie das Flackern einer Flamme und neigte den Kopf in einem gemessenen Nicken zu dem Jungen.

Harry drehte sich um und starrte Dumbledore an. Dumbledore zwinkerte ihm zu. Minerva hatte das Gefühl, dass sie etwas übersehen hatte.

Plötzliche Unsicherheit ging über Harrys Gesicht. Seine Kühle geriet ins Wanken. Furcht zeigte sich in seinen Augen, dann Zorn, und dann war der Junge wieder ruhig.

Ein Schauer lief Minerva den Rücken hinunter. Irgendetwas stimmte hier nicht.

"Bitte setz dich", sagte Dumbledore.

Sein Gesicht war nun wieder ernst. Harry setzte sich.

"Also, Harry", sagte Dumbledore.

"Ich habe einen Bericht über diesen Tag von Professor Snape gehört.

Würdest du mir mit deinen eigenen Worten erzählen, was passiert ist?"

Harrys Blick huschte abweisend zu Severus.

"Es ist nicht kompliziert", sagte der Junge und lächelte dünn.

"Er hat versucht, mich zu schikanieren, so wie er jeden Nicht-Slytherin in der Schule schikaniert hat, seit dem Tag, an dem Lucius ihn Ihnen untergeschoben hat.

Was die anderen Details angeht, so bitte ich um ein privates Gespräch mit Ihnen darüber. Von einem Schüler, der über missbräuchliches Verhalten eines Professors berichtet, kann man schließlich kaum erwarten, dass er vor eben diesem Professor offen spricht."

Diesmal konnte Minerva sich nicht zurückhalten, laut zu keuchen.

Severus lachte nur. Und das Gesicht des Schulleiters wurde ernst.

"Mr. Potter", sagte der Schulleiter,

"so spricht man nicht über einen Hogwarts-Professor. Ich fürchte, dass Sie unter einem schrecklichen Missverständnis leiden. Professor Severus Snape hat mein vollstes Vertrauen und dient Hogwarts auf mein eigenes Geheiß, nicht auf das von Lucius Malfoy."

Einige Augenblicke lang herrschte Schweigen.

Als der Junge wieder sprach, war seine Stimme eisig.

"Verpasse ich hier etwas?"

"Eine ganze Reihe von Dingen, Mr. Potter", sagte der Schulleiter.

"Du solltest zunächst verstehen, dass der Zweck dieses Treffens darin besteht, zu besprechen, wie du für die Ereignisse von heute Morgen diszipliniert werden kannst."

"Dieser Mann hat Ihre Schule über Jahre hinweg terrorisiert.

Ich habe mit Schülern gesprochen und Geschichten gesammelt, um sicherzustellen, dass es genug für eine Zeitungskampagne gibt, um die Eltern gegen ihn aufzubringen.

Einige der jüngeren Schüler weinten, als sie mir davon erzählten. Ich habe fast geweint, als ich sie hörte!

Sie haben diesem Schänder erlaubt, frei herumzulaufen? Sie haben das Ihren Schülern angetan? Warum?!"

Minerva schluckte einen Kloß in ihrem Hals hinunter.

"Mr. Potter", sagte der Schulleiter mit strenger Stimme,

"bei diesem Treffen geht es nicht um Professor Snape.

Es geht um dich und deine Missachtung der Schuldisziplin. Professor Snape hat vorgeschlagen, und ich habe zugestimmt, dass drei volle Monate Nachsitzen angemessen sind -"

"Abgelehnt", sagte Harry eisig.

Minerva war sprachlos.

"Das ist keine Bitte, Mr. Potter", sagte der Schulleiter.

Die ganze Kraft des Blicks des Zauberers war auf den Jungen gerichtet.

"Das ist deine Bestrafung…"

"Sie werden mir erklären, warum Sie diesem Mann erlaubt haben, den Kindern, die in Ihrer Obhut sind, etwas anzutun, und wenn Ihre Erklärung nicht ausreicht, werde ich meine Zeitungskampagne mit Ihnen als Zielperson beginnen."

Minervas Körper schwankte unter der Wucht dieses Schlages, unter der schier rohen Majestätsbeleidigung. Sogar Severus sah schockiert aus.

"Das, Harry, wäre äußerst unklug", sagte Dumbledore langsam.

"Ich bin die Hauptfigur, die Lucius auf dem Spielbrett gegenübersteht.

Wenn du so etwas tust, würde ihn das sehr stärken, und ich glaube nicht, dass das deine gewählte Seite ist."

Der Junge war einen langen Moment lang still.

"Dieses Gespräch wird privat", sagte Harry. Seine Hand schnippte in Severus' Richtung. "Schick ihn weg."

Dumbledore schüttelte den Kopf.

"Harry, habe ich dir nicht gesagt, dass Severus Snape mein vollstes Vertrauen genießt?"

Das Gesicht des Jungen zeigte den Schock darüber.

"Die Schikanen dieses Mannes machen Sie verwundbar! Ich bin nicht der Einzige, der eine Zeitungskampagne gegen Sie starten könnte! Das ist Wahnsinn! Warum tun Sie das?!"

Dumbledore seufzte.

"Es tut mir leid, Harry. Es hat mit Dingen zu tun, die du zu diesem Zeitpunkt noch nicht bereit bist zu hören."

Der Junge starrte Dumbledore an.

Dann drehte er sich um und sah Severus an. Dann wieder zu Dumbledore.

"Es ist Wahnsinn", sagte der Junge langsam.

"Sie haben ihn nicht im Zaum gehalten, weil Sie glauben, dass er Teil des Musters ist. Dass Hogwarts einen bösen Zaubertränkemeister braucht, um eine richtige Zauberschule zu sein, so wie es einen Geist braucht, um Geschichte zu unterrichten."

"Das hört sich doch so an, wie etwas das ich tun würde, oder?", sagte Dumbledore lächelnd.

"Inakzeptabel", sagte Harry barsch. Sein Blick war nun kalt und dunkel.

"Ich dulde weder Mobbing noch Missbrauch.

Ich habe viele Möglichkeiten in Betracht gezogen, mit diesem Problem umzugehen, aber ich werde es einfach machen.

\textbf{Entweder dieser Mann geht, oder ich tue es.}"

Minerva schnappte wieder nach Luft. Etwas Seltsames flackerte in Severus' Augen auf. Jetzt wurde auch Dumbledores Blick kalt.

"Der Schulverweis, Mr. Potter, ist die letzte Drohung, die gegen einen Schüler eingesetzt werden kann. Sie wird üblicherweise nicht als Drohung von Schülern gegen den Schulleiter eingesetzt. Dies ist die beste Zauberschule auf der ganzen Welt, und eine Ausbildung hier ist keine Chance, die jedem gegeben ist.

Hast du den Eindruck, dass Hogwarts ohne dich nicht auskommt?"

Und Harry saß da und lächelte dünn.

Plötzliches Entsetzen dämmerte Minerva auf. Sicherlich würde Harry nicht -

"Sie vergessen", sagte Harry,

"dass Sie nicht der Einzige sind, der Muster sehen kann.

Das wird privat. Jetzt schick ihn -"

Harry schnippte wieder mit der Hand nach Severus und hielt dann mitten im Satz und mitten in der Geste inne.

Minerva konnte es in Harrys Gesicht sehen, den Moment, als er sich erinnerte. Sie hatte es ihm schließlich gesagt.

"Mr. Potter", sagte der Schulleiter, "noch einmal, Severus Snape hat mein vollstes Vertrauen."

"Sie haben es ihm gesagt", flüsterte der Junge. "\textbf{Du Volltrottel!}"

Dumbledore reagierte nicht auf die Beleidigung. "Ihm was gesagt?"

"Dass der Dunkle Lord am Leben ist."

"Wovon in Merlins Namen sprichst du, Potter?!", rief Severus in einem Tonfall des blanken Erstaunens und der Empörung.

Harry blickte ihn kurz an und lächelte grimmig.

"Oh, wir sind also doch ein Slytherin", sagte Harry. "Ich habe mich schon gewundert."

Und dann herrschte Schweigen.

Schließlich sprach Dumbledore. Seine Stimme war sanft.

"Harry, wovon redest du?"

"Es tut mir leid, Albus", flüsterte Minerva.

Severus und Dumbledore drehten sich um und sahen sie an.

"Professor McGonagall hat es mir nicht gesagt", sagte Harrys Stimme, schnell und weniger ruhig als sie gewesen war.

"Ich habe es vermutet. Ich habe Ihnen gesagt, dass ich die Muster auch sehen kann. Ich habe es erraten, und sie hat ihre Reaktion genauso kontrolliert wie Severus.

Aber ihre Kontrolle war nicht ganz perfekt, und ich konnte erkennen, dass es Kontrolle war, nicht echt."

"Und ich habe ihm gesagt", sagte Minerva, ihre Stimme zitterte ein wenig,

"dass nur du, ich und Severus es wissen."

"Was sie als Zugeständnis gemacht hat, um zu verhindern, dass ich einfach herumlaufe und Fragen stelle, wie ich es angedroht habe, wenn sie nicht redet", sagte Harry.

Der Junge gluckste kurz.

"Ich hätte wirklich einen von euch allein erwischen und euch sagen sollen, dass sie mir alles erzählt hat, um zu sehen, ob ihr etwas verraten würdet.

Hätte wahrscheinlich nicht geklappt, aber einen Versuch wäre es wert gewesen."

Der Junge lächelte wieder.

"Die Drohung liegt immer noch auf dem Tisch und ich erwarte, dass ich irgendwann vollständig informiert werde."

Severus warf ihr einen Blick der völligen Verachtung zu.

Minerva hob ihr Kinn und ertrug ihn. Sie wusste, dass er verdient war.

Dumbledore lehnte sich in seinem gepolsterten Thron zurück.

\emph{Seine Augen waren so kalt wie alles, was Minerva seit dem Tag, an dem sein Bruder starb, von ihm gesehen hatte.}

"Und du drohst damit, uns Voldemort zu überlassen, wenn wir deinen Wünschen nicht nachkommen."

Harrys Stimme war rasiermesserscharf.

"Ich bedaure, Ihnen mitteilen zu müssen, dass Sie nicht das Zentrum des Universums sind. Ich drohe nicht damit, das magische Britannien zu verlassen. Ich drohe damit, Sie zu verlassen.

Ich bin kein sanftmütiger kleiner Frodo. Dies ist mein Abenteuer, und wenn Sie dabei sein wollen, werden Sie nach meinen Regeln spielen."

Dumbledores Gesicht war immer noch kalt.

"Ich fange an, an deiner Eignung als Held zu zweifeln, Mr. Potter."

Harrys erwidernder Blick war ebenso eisig.

"Ich fange an, an Ihrer Eignung als mein Gandalf zu zweifeln, Mr. Dumbledore. Boromir war zumindest ein plausibler Fehler. Was hat dieser Nazgul in meiner Gemeinschaft zu suchen?"

Minerva war völlig verwirrt. Sie schaute zu Severus, um zu sehen, ob er dem Ganzen folgte, und sie sah, dass Severus sein Gesicht von Harrys Blickfeld abgewandt hatte und lächelte.

"Ich nehme an", sagte Dumbledore langsam,

"dass es aus Ihrer Sicht eine berechtigte Frage ist. Also, Mr. Potter, wenn Professor Snape dich von nun an in Ruhe lassen soll, wird das das letzte Mal sein, dass dieses Thema aufkommt, oder werde ich dich jede Woche mit einer neuen Forderung hier finden?"

"Mich in Ruhe lassen?" Harrys Stimme war entrüstet.

"Ich bin nicht sein einziges Opfer und schon gar nicht das verletzlichste! Haben Sie vergessen, wie wehrlos Kinder sind? Wie verletzlich Sie sind?! Von nun an wird Severus jeden Schüler von Hogwarts mit angemessener und professioneller Höflichkeit behandeln, oder Sie werden einen anderen Zaubertränkemeister finden, oder Sie werden einen anderen Helden finden!"

Dumbledore begann zu lachen.

Aus voller Kehle, warmes, humorvolles Lachen, als ob Harry gerade einen komischen Tanz vor ihm aufgeführt hätte.

Minerva wagte nicht, sich zu bewegen. Ihre Augen flackerten und sie sah, dass Severus ebenso regungslos war.

Harrys Gesichtsausdruck wurde noch kälter.

"Sie verstehen mich falsch, Schulleiter, wenn Sie glauben, dass dies ein Scherz ist.

Das ist keine Bitte. Das ist Ihre Bestrafung."

"Mr. Potter -"

sagte Minerva. Sie wusste nicht einmal, was sie sagen wollte.

Das konnte sie einfach nicht über sich ergehen lassen. Harry machte eine schweigende Geste ihr gegenüber und sprach weiter mit Dumbledore.

"Und wenn Ihnen das unhöflich vorkommt", sagte Harry, seine Stimme war nun etwas weniger hart, "so kam es mir nicht weniger unhöflich vor, als Sie es zu mir sagten. Sie würden so etwas nicht zu jemandem sagen, den Sie für einen echten Menschen und nicht für ein untergeordnetes Kind halten, und ich werde Sie mit der gleichen Höflichkeit behandeln, mit der Sie mich behandeln -"

"Oh ja, in der Tat, das ist meine Strafe, wenn es je eine gab!

Natürlich bist du hier, um mich zu erpressen, um deine Mitschüler zu retten, nicht um dich selbst zu retten! Ich kann mir nicht vorstellen, warum ich etwas anderes gedacht hätte!"

Dumbledore lachte nun noch heftiger. Er hämmerte dreimal mit der Faust auf den Schreibtisch.

Harrys Blick wurde unsicher. Sein Gesicht wandte sich ihr zu und sprach sie zum ersten Mal an.

"Entschuldigen Sie", sagte Harry. Seine Stimme schien zu schwanken.

"Muss er seine Medikamente nehmen oder so?"

"Ah …"

Minerva hatte keine Ahnung, was sie sagen sollte.

"Nun", sagte Dumbledore. Er wischte sich die Tränen weg, die sich in seinen Augen gebildet hatten.

"Verzeih mir. Es tut mir leid, dass ich dich unterbrochen habe. Bitte fahre mit der Erpressung fort."

Harry öffnete den Mund, dann schloss er ihn wieder.

Er wirkte jetzt ein wenig unsicher. "Ah … er soll auch aufhören, die Gedanken der Schüler zu lesen."

"Minerva", sagte Severus, seine Stimme war tödlich, "du -"

"Der Sprechende Hut hat mich gewarnt", sagte Harry.

"Was?!"

"Ich kann nichts anderes sagen. Jedenfalls denke ich, das war's. Ich bin fertig."

Stille.

"Und was jetzt?"

Sagte Minerva, als es offensichtlich wurde, dass niemand sonst etwas sagen würde.

"Und was jetzt?" echote Dumbledore. "Nun, jetzt gewinnt der Held, natürlich."

"\textbf{Was?!}", sagten Severus, Minerva und Harry.

"Nun, er scheint uns wirklich in die Enge getrieben zu haben", sagte Dumbledore und lächelte glücklich.

"Aber Hogwarts braucht einen bösen Zaubertränkemeister, sonst wäre es ja keine richtige Zauberschule, oder? Wie wäre es also, wenn Professor Snape nur zu Schülern ab dem fünften Jahr böse ist?"

"\textbf{Was?!}", sagten wieder alle drei.

"Wenn es die verletzlichsten Opfer sind, um die du dir Sorgen machst, dann ist es vielleicht nicht so schlimm.

Vielleicht hast du recht, Harry. Vielleicht habe ich im Laufe der Jahrzehnte vergessen, wie es ist, ein Kind zu sein.

Also lass uns einen Kompromiss schließen. Severus wird weiterhin ungerecht Punkte an Slytherin vergeben und seinem Haus laxe Disziplin auferlegen.

Und er wird schrecklich zu Nicht-Slytherin-Schülern ab dem 5. Jahr sein. Zu anderen wird er furchterregend sein, aber nicht beleidigend.

Er wird versprechen, nur dann Gedanken zu lesen, wenn die Sicherheit eines Schülers es erfordert. Hogwarts wird seinen bösen Zaubertränkemeister haben, und die verletzlichsten Opfer, wie du es ausdrückst, werden sicher sein."

Minerva McGonagall war so schockiert, wie sie es noch nie in ihrem Leben gewesen war. Sie blickte unsicher zu Severus, dessen Gesicht völlig neutral geblieben war, als könne er sich nicht entscheiden, welchen Ausdruck er tragen sollte.

"Ich nehme an, das ist akzeptabel", sagte Harry. Seine Stimme klang ein wenig merkwürdig.

"Das kann nicht dein Ernst sein", sagte Severus, seine Stimme so ausdruckslos wie sein Gesicht.

"Ich bin sehr dafür", sagte Minerva langsam. Sie war so sehr dafür, dass ihr Herz unter ihren Roben wild pochte.

"Aber was könnten wir den Schülern sagen? Sie hätten das vielleicht nicht in Frage gestellt, als Severus noch … schrecklich zu allen war, aber -"

"Harry kann den anderen Schülern erzählen, dass er ein schreckliches Geheimnis von Severus entdeckt hat und ein bisschen erpresst wurde", sagte Dumbledore.

"Es ist doch wahr; er hat entdeckt, dass Severus Gedanken lesen kann, und er hat uns tatsächlich erpresst."

"Das ist Irrsinn!", explodierte Severus.

"\textbf{Bwah ha ha!}", sagte Dumbledore.

"Ah…", sagte Harry unsicher.

"Und wenn mich jemand fragt, warum die Fünftklässler und darüber den Kürzeren gezogen haben? Ich würde es ihnen nicht verübeln, wenn sie wütend wären, und dieser Teil war nicht gerade meine Idee -"

"Sag ihnen", sagte Dumbledore,

"dass es nicht du warst, der den Kompromiss vorgeschlagen hat, dass es alles war, was du bekommen konntest.

Und dann weiger dich noch mehr zu sagen. Auch das ist wahr. Das ist eine Kunst, die du mit etwas Übung lernen wirst."

Harry nickte langsam.

"Und die Punkte, die er Ravenclaw abgenommen hat?"

"Die dürfen nicht zurückgegeben werden."

Es war Minerva, die das sagte.

Harry sah sie an.

"Es tut mir leid, Mr. Potter", sagte sie. Es tat ihr leid, aber es musste getan werden.

"Es muss Konsequenzen für dein Fehlverhalten geben, oder diese Schule wird in Stücke fallen."

Harry zuckte mit den Schultern.

"Akzeptabel", sagte er schlicht.

"Aber in Zukunft wird Severus weder meine Hausverbindungen angreifen, indem er mir Punkte wegnimmt, noch wird er meine wertvolle Zeit mit Nachsitzen verschwenden. Sollte er das Gefühl haben, dass mein Verhalten einer Korrektur bedarf, kann er seine Bedenken Professor McGonagall mitteilen."

"Harry", sagte Minerva, "wirst du dich weiterhin der Schuldisziplin unterwerfen, oder sollst du jetzt über dem Gesetz stehen, so wie Severus es war?"

Harry sah sie an.

Etwas Warmes berührte seinen Blick, kurz bevor es unterdrückt wurde.

"Ich werde weiterhin ein ganz normaler Schüler für alle Mitglieder des Lehrkörpers sein, die nicht wahnsinnig oder böse sind, vorausgesetzt, sie werden nicht von anderen unter Druck gesetzt, die es sind."

Harry warf einen kurzen Blick auf Severus, dann wandte er sich wieder an Dumbledore.

"Lassen Sie Minerva in Ruhe, und ich werde in ihrer Gegenwart ein ganz normaler Hogwartsschüler sein. Keine besonderen Privilegien oder Immunitäten."

"Wunderbar", sagte Dumbledore aufrichtig. "Gesprochen wie ein wahrer Held."

"Und", sagte sie,

"Mr. Potter muss sich öffentlich für sein Verhalten von heute entschuldigen."

Harry warf ihr einen weiteren Blick zu. Dieser war ein wenig skeptisch.

"Die Disziplin der Schule ist durch Ihr Verhalten schwer verletzt worden, Mr. Potter", sagte Minerva.

"Sie muss wiederhergestellt werden."

"Ich denke, Professor McGonagall, dass Sie die Bedeutung dessen, was Sie Schuldisziplin nennen, erheblich überschätzen, im Vergleich dazu, dass Geschichte von einem lebenden Lehrer unterrichtet wird oder dass Sie Ihre Schüler nicht quälen.

Die Aufrechterhaltung der gegenwärtigen Statushierarchie und die Durchsetzung ihrer Regeln scheint viel weiser und moralischer und wichtiger zu sein, wenn man an der Spitze steht und die Durchsetzung vornimmt, als wenn man an der Basis steht, und ich kann bei Bedarf Studien dazu zitieren.

Ich könnte noch stundenlang über diesen Punkt sprechen, aber ich werde es dabei belassen."

Minerva schüttelte den Kopf.

"Mr. Potter, Sie unterschätzen die Bedeutung von Disziplin, weil Sie sie selbst nicht nötig haben -"

Sie hielt inne.

Das war nicht richtig rübergekommen, und Severus, Dumbledore und sogar Harry warfen ihr seltsame Blicke zu.

"Um zu lernen, meine ich. Nicht jedes Kind kann in Abwesenheit von Autorität lernen. Und es sind die anderen Kinder, die Schaden nehmen werden, Mr. Potter, wenn sie Ihr Beispiel als eines sehen, dem man folgen sollte."

Harrys Lippen verzogen sich zu einem schiefen Lächeln.

"Das erste und letzte Mittel ist die Wahrheit. Die Wahrheit ist, dass ich nicht hätte wütend werden dürfen, dass ich die Klasse nicht hätte stören dürfen, dass ich nicht hätte tun dürfen, was ich getan habe, und dass ich ein schlechtes Beispiel für alle gegeben habe.

Die Wahrheit ist auch, dass Severus Snape sich auf eine Art und Weise verhalten hat, die einem Hogwarts-Professor nicht angemessen ist, und dass er von nun an mehr Rücksicht auf die verletzten Gefühle von Schülern im vierten Jahr und darunter nehmen wird.

Wir beide könnten aufstehen und die Wahrheit sagen. Damit könnte ich leben."

"In deinen Träumen, Potter!", spuckte Severus.

"Immerhin", sagte Harry und lächelte grimmig,

"wenn die Schüler sehen, dass Regeln für alle da sind … auch für die Professoren, nicht nur für die armen, hilflosen Schüler, denen das System nichts als Leid bringt … sollten die positiven Auswirkungen auf die Schuldisziplin enorm sein."

Es gab eine kurze Pause, und dann kicherte Dumbledore.

"Minerva denkt, dass du mehr Recht hast, als dir zusteht."

Harrys Blick ruckte von Dumbledore weg, hinunter auf den Boden.

"Lesen Sie etwa ihre Gedanken?"

"Gesunder Menschenverstand wird oft mit Legilimenz verwechselt", sagte Dumbledore.

"Ich werde diese Angelegenheit mit Severus besprechen, und es wird keine Entschuldigung von dir verlangt werden, es sei denn, er entschuldigt sich ebenfalls.

Und nun erkläre ich diese Angelegenheit für abgeschlossen, zumindest bis zur Mittagspause."

Er hielt inne.

"Obwohl, Harry, ich fürchte, dass Minerva mit dir über eine weitere Angelegenheit sprechen möchte.

Und das ist nicht das Ergebnis von irgendeinem Druck meinerseits. Minerva, wenn Sie so freundlich wären?"

Minerva erhob sich von ihrem Stuhl und fiel fast hin.

Es war zu viel Adrenalin in ihrem Blut, ihr Herz schlug zu schnell.

"Fawkes", sagte Dumbledore, "begleiten Sie sie, bitte."

"Ich will nicht -", begann sie zu sagen.

Dumbledore warf ihr einen Blick zu, und sie verstummte. Der Phönix schwebte wie eine sanfte Flammenzunge durch den Raum und landete auf ihrer Schulter.

Sie spürte die Wärme durch ihre Roben, durch ihren ganzen Körper.

"Bitte folgen Sie mir, Mr. Potter", sagte sie jetzt mit fester Stimme, und sie gingen durch die Tür.

Sie standen auf der sich drehenden Treppe und stiegen schweigend hinunter. Minerva wusste nicht, was sie sagen sollte.

Sie kannte diese Person, die neben ihr stand, nicht. Und Fawkes begann zu singen. Es war zart und weich, wie ein Kamin klingen würde, wenn er eine Melodie hätte, und es umspülte Minervas Geist, beruhigte, besänftigte, was es berührte …

"Was ist das?" flüsterte Harry neben ihr. Seine Stimme war instabil, schwankte, wechselte die Tonlage.

"Das Lied des Phönix", sagte Minerva, ohne wirklich zu wissen, was sie sagte, ihre ganze Aufmerksamkeit galt dieser seltsamen, leisen Musik.

"Sie heilt."

Harry wandte sein Gesicht von ihr ab, aber sie erhaschte einen Blick auf etwas Gequältes.

Der Abstieg schien sehr lange zu dauern, oder vielleicht war es nur so, dass die Musik sehr lange zu dauern schien, und als sie durch die Lücke, wo ein Wasserspeier gewesen war, wieder herauskamen, hielt sie Harrys Hand fest in der ihren.

Als der Wasserspeier wieder an seinen Platz trat, verließ Fawkes ihre Schulter und schwebte vor Harry.

Harry starrte Fawkes an wie jemand, der vom ständig wechselnden Licht eines Feuers hypnotisiert ist.

"Was soll ich nur tun, Fawkes?", flüsterte Harry.

"Ich hätte sie nicht beschützen können, wenn ich nicht wütend gewesen wäre."

Die Flügel des Phönix schlugen weiter, er schwebte weiter an seinem Platz.

Es gab kein Geräusch außer dem Schlagen der Flügel. Dann gab es einen Blitz, wie ein Feuer, das aufflackert und erlischt, und Fawkes war weg.

Beide blinzelten, als wären sie aus einem Traum aufgewacht, oder vielleicht als wären sie wieder eingeschlafen.

Minerva blickte nach unten. Harry Potters helles, junges Gesicht blickte zu ihr auf.

"Sind Phönixe Menschen?", fragte Harry. "Ich meine, sind sie klug genug, um als Menschen zu gelten? Könnte ich mit ihnen reden, wenn ich wüsste, wie?"

Minerva blinzelte heftig. Dann blinzelte sie noch einmal.

"Nein", sagte Minerva, ihre Stimme schwankte.

"Phönixe sind Kreaturen mit mächtiger Magie. Diese Magie gibt ihrer Existenz ein Gewicht an Bedeutung, das kein einfaches Tier besitzen könnte. Sie sind Feuer, Licht, Heilung, Wiedergeburt. Aber letzten Endes nicht menschlich."

"Wo kann ich einen bekommen?"

Minerva beugte sich hinunter und umarmte ihn. Sie hatte es nicht vorgehabt, aber sie schien in dieser Angelegenheit keine Wahl zu haben. Als sie aufstand, fiel es ihr schwer zu sprechen. Aber sie musste fragen.

"Was ist heute passiert, Harry?"

"Ich kenne die Antworten auf die wichtigen Fragen auch nicht. Abgesehen davon möchte ich eine Weile nicht darüber nachdenken."

Minerva nahm wieder seine Hand in ihre, und sie gingen den Rest des Weges schweigend. Es war nur ein kurzer Weg, denn natürlich lag das Büro des Stellvertreters in der Nähe des Büros des Schulleiters. Minerva saß hinter ihrem Schreibtisch. Harry saß vor ihrem Schreibtisch.

"Also", flüsterte Minerva.

Sie hätte fast alles dafür gegeben, dies nicht zu tun, oder nicht diejenige zu sein, die es tun musste, oder dass es zu irgendeinem anderen Zeitpunkt als jetzt wäre.

"Es gibt da eine Angelegenheit der Schuldisziplin. Davon bist du nicht ausgenommen."

"Und zwar?", fragte Harry.

Er wusste es nicht. Er hatte es noch nicht herausgefunden. Sie spürte, wie sich ihre Kehle zusammenzog. Aber es gab Arbeit zu tun, und sie würde sich nicht davor drücken.

"Mr. Potter", sagte Professor McGonagall,

"ich muss Ihren Zeitumkehrer sehen, bitte."

Die ganze Ruhe des Phönix verschwand in einem Augenblick aus seinem Gesicht und Minerva fühlte sich, als hätte sie ihn gerade erstochen.

"Nein!" sagte Harry. Seine Stimme war panisch.

"Ich brauche ihn, ich werde nicht in der Lage sein, Hogwarts zu besuchen, ich werde nicht in der Lage sein zu schlafen!"

"Du wirst schlafen können", sagte sie.

"Das Ministerium hat die Schutzhülle für deinen Zeitumkehrer geliefert. Ich werde ihn so verzaubern, dass er sich nur zwischen 21 Uhr und Mitternacht öffnet."

Harrys Gesicht verzog sich.

"Aber - aber ich -"

"Mr. Potter, wie oft haben Sie den Zeitdreher seit Montag benutzt? Wie viele Stunden?"

"Ich…" sagte Harry. "Warten Sie, lassen Sie mich zusammenzählen -"

Er blickte auf seine Uhr hinunter.

Minerva spürte einen Anflug von Traurigkeit. Sie hatte es sich gedacht.

"Es waren also nicht nur zwei Stunden pro Tag.

Ich vermute, wenn ich Ihre Mitschüler fragen würde, würde ich feststellen, dass Sie Schwierigkeiten hatten, lange genug aufzubleiben, um zu einer vernünftigen Zeit schlafen zu gehen, und dass Sie jeden Morgen früher und früher aufwachten. Richtig?"

Harrys Gesicht sagte alles, was sie wissen musste.

"Mr. Potter", sagte sie sanft, "es gibt Schüler, denen man die Zeitdreher nicht anvertrauen kann, weil sie süchtig nach ihnen werden. Wir geben ihnen ein Mittel, das ihren Schlafzyklus um das nötige Maß verlängert, aber am Ende benutzen sie den Zeitumkehrer für mehr als nur die Teilnahme am Unterricht.

Und so müssen wir ihn zurücknehmen. Mr. Potter, Sie haben sich angewöhnt, den Zeitumkehrer als Ihre Lösung für alles zu benutzen, oft sehr töricht sogar.

Sie haben ihn benutzt, um ein Erinnermich zurückzubekommen. Sie sind auf eine für andere Schüler offensichtlich unmögliche Art und Weise aus einem Schrank verschwunden, anstatt zurückzugehen und mich oder jemand anderen zu holen, um die Tür zu öffnen."

Nach Harrys Gesichtsausdruck zu urteilen, hatte er daran nicht gedacht.

"Und was noch wichtiger ist", sagte sie,

"Sie hätten sich einfach in Professor Snapes Klasse setzen sollen. Und zusehen. Und am Ende der Stunde gehen sollen. So wie Sie es getan hätten, wenn Sie keinen Zeitumkehrer besessen hätten. Es gibt Schüler, denen man keinen Zeitumkehrer anvertrauen kann, Mr. Potter. Sie sind einer von ihnen.

Es tut mir leid."

"Aber ich brauche ihn!" platzte Harry heraus. "Was ist, wenn mich Slytherins bedrohen und ich fliehen muss? Es bringt mich in Sicherheit -"

"Jeder andere Schüler in diesem Schloss geht das gleiche Risiko ein, und ich versichere Ihnen, dass sie überleben werden. Seit 50 Jahren ist kein Schüler in diesem Schloss gestorben. Mr. Potter, Sie werden Ihren Zeitumkehrer aushändigen und zwar jetzt."

Harrys Gesicht verzog sich vor Schmerz, aber er zog den Zeitumkehrer unter seinen Roben hervor und gab ihn ihr. Von ihrem Schreibtisch holte Minerva eine der Schutzhüllen hervor, die nach Hogwarts geschickt worden waren.

Sie ließ die Hülle um die sich drehende Sanduhr des Zeitumkehrers einrasten und legte dann ihren Zauberstab auf die Hülle, um die Verzauberung zu vollenden.

"Das ist nicht fair!" kreischte Harry. "Ich habe heute halb Hogwarts vor Professor Snape gerettet, ist es richtig, dass ich dafür bestraft werde? Ich habe Ihren Gesichtsausdruck gesehen, Sie haben gehasst, was er getan hat!"

Minerva schwieg einige Augenblicke lang.

Sie war wie verzaubert. Als sie geendet hatte und aufblickte, wusste sie, dass ihr Gesicht streng war.

\emph{Vielleicht war es das Falsche. Und dann wiederum war es vielleicht das Richtige. Sie hatte ein eigensinniges Kind vor sich, und das bedeutete nicht, dass das Universum kaputt war.}

"Fair, Mr. Potter?", schnauzte sie.

"Ich musste an zwei aufeinanderfolgenden Tagen zwei Berichte über die öffentliche Benutzung eines Zeitumkehrers beim Ministerium einreichen!

Seien Sie äußerst dankbar, dass Sie den Zeitdreher überhaupt in eingeschränkter Form behalten durften! Der Schulleiter hat einen Termin beim Zaubereiminister gemacht, um ihn persönlich zu bitten, und wenn Sie nicht der Junge-der-lebte wären, hätte selbst das nicht gereicht!"

Harry starrte sie an. Sie wusste, dass er das wütende Gesicht von Professor McGonagall sah. Harrys Augen füllten sich mit Tränen.

"Es, es tut mir leid", flüsterte er, die Stimme nun erstickt und gebrochen.

"Es tut mir leid, dass ich Sie enttäuscht habe …"

"Es tut mir auch leid, Mr. Potter", sagte sie streng und reichte ihm den frisch gesperrten Zeitumkehrer.

"Sie können gehen."

Harry drehte sich um und flüchtete schluchzend aus ihrem Büro. Sie hörte, wie seine Füße den Flur entlang trampelten, und dann verstummte das Geräusch, als die Tür zuging.

"Es tut mir auch leid, Harry", flüsterte sie in den stillen Raum. "Mir tut es auch leid."

Fünfzehn Minuten nach der Mittagspause.

Keiner sprach mit Harry. Einige der Ravenclaws warfen ihm Blicke des Ärgers zu, andere des Mitgefühls, ein paar der jüngeren Schüler hatten sogar Blicke der Bewunderung, aber niemand sprach mit ihm.

Nicht einmal Hermine hatte versucht, zu ihm zu kommen. Fred und George waren zaghaft näher getreten.

Sie hatten nichts gesagt. Das Angebot war klar, und seine Optionalität. Harry hatte ihnen gesagt, dass er rüberkommen würde, wenn der Nachtisch begann, nicht früher.

Sie hatten genickt und waren schnell weggegangen. Wahrscheinlich war es der völlig ausdruckslose Blick auf Harrys Gesicht, der das bewirkt hatte.

Die anderen dachten wahrscheinlich, er würde seine Wut oder seine Bestürzung kontrollieren. Sie wussten, weil sie gesehen hatten, wie Flitwick ihn geholt hatte, dass er in das Büro des Schulleiters gerufen worden war.

Harry versuchte, nicht zu lächeln, denn wenn er lächelte, würde er anfangen zu lachen, und wenn er anfing zu lachen, würde er nicht aufhören, bis die netten Leute in weißen Jacken kamen, um ihn abzuholen.

Es war zu viel. Es war einfach alles zu viel. Harry war fast zur dunklen Seite übergetreten, seine dunkle Seite hatte Dinge getan, die im Nachhinein wahnsinnig erschienen, seine dunkle Seite hatte einen unmöglichen Sieg errungen, der vielleicht echt und vielleicht eine reine Laune eines verrückten Schulleiters gewesen war, seine dunkle Seite hatte seine Freunde beschützt.

Er konnte es einfach nicht mehr ertragen. Er brauchte Fawkes, der wieder für ihn sang. Er musste den Zeitumkehrer benutzen, um sich eine ruhige Stunde zu gönnen, um sich zu erholen, aber das war keine Option mehr, und der Verlust war wie ein Loch in seiner Existenz, aber daran konnte er nicht denken, denn dann würde er vielleicht anfangen zu lachen.

Zwanzig Minuten.

Alle Schüler, die zu Mittag essen wollten, waren eingetroffen, fast keiner fehlte. Das Klopfen eines Löffels schallte durch die Große Halle.

"Wenn ich um eure Aufmerksamkeit bitten dürfte", sagte Dumbledore.

"Harry Potter hat etwas, das er uns gerne mitteilen möchte."

Harry holte tief Luft und stand auf.

Er ging hinüber zum Haupttisch, wo alle Augen auf ihn starrten. Harry drehte sich um und schaute auf die vier Tische hinaus.

Es wurde immer schwieriger, nicht zu lächeln, aber Harry hielt sein Gesicht ausdruckslos, als er seine kurze und auswendig gelernte Rede hielt.

"Die Wahrheit ist heilig", sagte Harry tonlos.

"Einer meiner wertvollsten Besitztümer ist ein Knopf, auf dem steht: '\emph{Sprich die Wahrheit, auch wenn deine Stimme zittert}'.

Dies also ist die Wahrheit. Merkt euch das. Ich sage es nicht, weil ich dazu gezwungen werde, ich sage es, weil es wahr ist.

Was ich in Professor Snapes Unterricht getan habe, war töricht, dumm, kindisch und ein unverzeihlicher Verstoß gegen die Regeln von Hogwarts.

Ich habe den Unterricht gestört und meine Mitschüler ihrer unersetzlichen Lernzeit beraubt. Alles nur, weil ich mein Temperament nicht unter Kontrolle hatte.

Ich hoffe, dass kein einziger von euch je meinem Beispiel folgen wird. Ich werde auf jeden Fall versuchen, ihm nie wieder zu folgen."

Viele der Schüler, die Harry jetzt ansahen, hatten feierliche, unglückliche Blicke auf ihren Gesichtern, wie man sie bei einer Zeremonie zum Verlust eines gefallenen Champions sehen könnte.

Bei den Jüngeren am Gryffindor-Tisch war dieser Blick fast überall zu sehen.

Bis Harry seine Hand hob.

Er hob sie nicht hoch. Das hätte vorschnell wirken können. Er hob sie ganz sicher nicht in Richtung Severus.

Harry hob einfach die Hand auf Brusthöhe und schnippte leise mit den Fingern, eine Geste, die mehr gesehen als gehört wurde.

Es war möglich, dass die meisten am Kopftisch es überhaupt nicht sahen. Diese scheinbare Geste des Trotzes brachte den jüngeren Schülern und den Gryffindors ein plötzliches Lächeln auf das Gesicht, den Slytherins ein kaltes, überlegenes Hohngelächter und allen anderen Stirnrunzeln und besorgte Blicke.

Harry hielt sein Gesicht ausdruckslos.

"Danke", sagte er. "Das ist alles."

"Ich danke Ihnen, Mr.

Potter", sagte der Schulleiter. "Und jetzt hat Professor Snape auch noch etwas mit uns zu teilen."

Severus erhob sich geschmeidig von seinem Platz am Kopftisch.

"Man hat mich darauf aufmerksam gemacht", sagte er,

"dass mein eigenes Verhalten eine Rolle dabei gespielt hat, den zugegebenermaßen unentschuldbaren Zorn von Mr. Potter zu provozieren, und in der anschließenden Diskussion wurde mir klar, dass ich vergessen hatte, wie leicht die Gefühle von jungen und unreifen Menschen verletzt werden -"

\emph{Es ertönte das Geräusch vieler Menschen, die gleichzeitig ein dumpfes Würgen von sich gaben.}

Severus fuhr fort, als hätte er es nicht gehört.

"Das Zaubertränke-Klassenzimmer ist ein gefährlicher Ort, und ich bin immer noch der Meinung, dass strenge Disziplin notwendig ist, aber von nun an werde ich mir der… \emph{emotionalen Zerbrechlichkeit.}.. von Schülern im vierten Jahr und jünger stärker bewusst sein. Mein Punktabzug für Ravenclaw bleibt bestehen, aber ich werde Mr.

Potters Nachsitzen aufheben. Ich danke Ihnen."

Es gab ein einzelnes Klatschen aus Richtung Gryffindor und schneller als der Blitz war Severus' Zauberstab in der Hand und "\textbf{Quietus}!" brachte den Übeltäter zum Schweigen.

"Ich werde weiterhin Disziplin und Respekt in all meinen Klassen verlangen", sagte Severus kalt, "und jeder, der sich mit mir anlegt, wird es bereuen."

Er setzte sich hin.

"Ich danke auch Ihnen!" sagte Schulleiter Dumbledore fröhlich. "Macht weiter!"

Und Harry, immer noch ausdruckslos, machte sich auf den Weg zurück zu seinem Platz in Ravenclaw.

Es gab eine Explosion von Gesprächen. Zwei Worte waren am Anfang deutlich zu erkennen. Das erste war ein anfängliches

"Was -", mit dem viele verschiedene Sätze begannen, wie

"Was ist gerade passiert -"

und

"Was zum Teufel -"

Das zweite war

"Was soll das?",

während die Studenten das heruntergefallene Essen und die ausgespuckten Getränke von sich, der Tischdecke und voneinander aufräumten.

Einige Studenten weinten offen. Professor Sprout auch. Am Gryffindor-Tisch, wo eine Torte mit einundfünfzig unbeleuchteten Kerzen wartete, flüsterte Fred:

"Ich glaube der ist uns über, George."

Und von diesem Tag an, egal was Hermine versuchte, irgendjemandem zu erzählen, würde es eine Legende in Hogwarts sein, dass Harry Potter absolut alles mit einem Fingerschnippen geschehen lassen konnte.

