

\hypertarget{tabubruxfcche-vorspiel-schummeln}{% \section{78. Tabubrüche, Vorspiel: Schummeln}\label{tabubruxfcche-vorspiel-schummeln}}

\textbf{\uline{Tabubrüche, Vorspiel: Schummeln}}

Es war Samstag, der 4. April des Jahres 1992. Mr. und Mrs. Davis sahen ziemlich nervös aus, als sie in einem bestimmten Bereich der Hogwarts-Quidditch-Tribüne saßen - obwohl die gepolsterten Bänke heute nicht auf fliegende Besen blickten, sondern auf ein riesiges Quadrat aus so etwas wie Pergament; eine große weiße Leere, die bald mit Fenstern die Gras und Soldaten zeigten flackern würde. Für den Moment zeigte es nur die reflektierte stumpfe graue Farbe des umgebenden bedeckten Himmels. (Es sah ziemlich stürmisch aus, obwohl die Wettervorhersage versprochen hatten, dass der Regen nicht vor Einbruch der Nacht einsetzen würde.)

Normalerweise war es die uralte Tradition von Hogwarts, dass Eltern draußen bleiben sollten - aus demselben Grund, aus dem ungeduldige Kinder aufgefordert werden, sich aus der Küche zu entfernen und sich nicht in die Angelegenheiten des Kochs einzumischen. Der einzige Grund für eine Eltern-Lehrer-Konferenz war, wenn ein Lehrer das Gefühl hatte, dass ein Elternteil sich nicht ordentlich benahm. Es bedurfte eines außergewöhnlichen Umstandes, damit die Hogwarts-Verwaltung das Gefühl hatte, sich vor Ihnen rechtfertigen zu müssen. Im Allgemeinen hatte die Hogwarts-Verwaltung bei jeder Gelegenheit den Rückhalt von achthundert Jahren angesehener Geschichte und die Eltern nicht. Deshalb hatten Mr. und Mrs. Davis mit einigem Zögern auf einer Audienz bei der stellvertretenden Schulleiterin McGonagall bestanden. Es war schwer, ein angemessenes Gefühl der Empörung aufzubringen, wenn man derselben würdevollen Hexe gegenüberstand, die zwölf Jahre und vier Monate zuvor Sie beide zwei Wochen nachsitzen ließ, nachdem sie Sie auf frischer Tat bei der Zeugung von Tracey ertappt hatte. Andererseits wurde Mr. und Mrs. Davis' Mut dadurch gestärkt, dass sie wütend mit einem Exemplar vom Klitterer herumfuchtelten, dessen Schlagzeile in heller, fetter Schrift für alle Welt zu sehen war:

\textbf{PAKT MIT POTTER?}

\textbf{BONES, DAVIS, GRANGER IM LIEBESRECHTECK DER ANGST}

Und so hatten Mr. und Mrs. Davis in die Fakultätsloge der Hogwarts-Quidditch-Tribüne gestritten, wo sie sich nun mit einem ausgezeichneten Blick auf Professor Quirrells verzauberte Bildschirme niedergelassen hatten, so dass die beiden mit eigenen Augen sehen konnten, "\emph{was in dieser Schule für ein Schwachsinn vor sich geht, wenn Sie den Ausdruck verzeihen, stellvertretende Schulleiterin McGonagall!}" Links von Mr. Davis saß ein weiterer besorgter Elternteil, ein weißhaariger Mann in eleganten schwarzen Roben von unvergleichlicher Qualität, ein gewisser Lucius Malfoy, politischer Führer der stärksten Fraktion des Zaubergamot. Links von Lord Malfoy ein spöttischer, aristokratischer Mann mit vernarbtem Gesicht, der ihnen als Lord Jugson vorgestellt wurde. Dann ein älterer, aber scharfäugiger Bursche namens Charles Nott, von dem man munkelt, er sei fast so wohlhabend wie Lord Malfoy, der links von Lord Jugson saß. Zur Rechten von Mrs. Davis saßen die hübsche Lady und der noch hübschere Lord des edlen und uralten Hauses Greengrass. Sie waren jung, nach Zauberermaßstab, und trugen graue Seidengewänder mit winzigen dunklen Smaragden, die in die Form von Grashalmen gestickt waren. Die Lady Greengrass galt als entscheidende Stimme im Zaubergamot, nachdem ihre eigene Mutter sich überraschend schnell aus dem Gremium zurückgezogen hatte. Ihr charmanter Ehemann hatte, obwohl seine Familie an sich nicht adlig oder wohlhabend war, einen Sitz im Hogwarts-Gouverneursrat eingenommen. Zu ihrer Rechten saß eine kantige und unglaublich zäh aussehende alte Hexe, die Mr. und Mrs. Davis ohne den geringsten Anflug von Herablassung die Hand geschüttelt hatte. Das war Amelia Bones, Direktorin der Abteilung für magische Strafverfolgung. Zu Amelias Rechten saß eine ältere Frau, die die Modeszene des magischen Britanniens auf den Kopf gestellt hatte, indem sie einen lebenden Geier in ihren Hut integriert hatte, Augusta Longbottom. Obwohl sie nicht als Lady angesprochen wurde, würde Madam Longbottom die vollen Rechte der Familie Longbottom ausüben, solange deren letzter Spross noch nicht volljährig war, und sie galt als prominente Figur in einer Minderheitenfraktion des Zaubergamot. An der Seite von Madam Longbottom saß kein Geringerer als der Oberste Hexenmeister und Schulleiter Albus Percival Wulfric Brian Dumbledore, legendärer Bezwinger Grindelwalds, Beschützer Britanniens, Wiederentdecker der sagenumwobenen zwölf Verwendungen von Drachenblut, mächtigster Zauberer der Welt etc. etc.. Und schließlich, ganz rechts, fand man den rätselhaften Verteidigungsprofessor von Hogwarts, Quirinus Quirrell, der sich auf den gepolsterten Bänken zurücklehnte, als würde er sich ausruhen; Er schien sich in der Gesellschaft eines stimmberechtigten Quorums des Hogwarts-Gouverneursrats, der an diesem schönen Samstag vorbeigekommen war, um zu erfahren, was in Hogwarts im Allgemeinen und bei Draco Malfoy, Theodore Nott, Daphne Greengrass, Susan Bones und Neville Longbottom im Besonderen los war, ganz wohl zu fühlen.

Auch über den Namen Harry Potter war viel diskutiert worden. Oh, und natürlich durfte man Tracey Davis nicht vergessen. Direktor Bones' Augenbrauen waren interessiert in die Höhe geklettert, als sie das junge Paar als ihre Eltern vorgestellt bekam. Lord Jugson hatte ihnen einen kurzen, ungläubigen Blick zugeworfen, bevor er sie mit einem Schnauben abtat. Lucius Malfoy hatte sie höflich begrüßt, sein Lächeln enthielt einen Hauch von grimmiger Belustigung gemischt mit Mitleid. Mr. und Mrs. Davis, deren letzte Abstimmung über irgendetwas von Bedeutung darin bestanden hatte, ihre Zauberstäbe mit dem Namen von Minister Fudge zu berühren, die ganze dreihundert Galleonen in ihrem Gringotts-Tresor aufbewahrt hatten und die jeweils als Verkäufer von Kesseln in einem Zaubertränke-Laden und als Verzauberer von Omniokularen arbeiteten, saßen eng aneinandergepresst, starr aufrecht auf ihren gepolsterten Bänken und wünschten sich verzweifelt, sie hätten schönere Roben getragen.

Der Himmel über ihnen war eine feste, in dunklere und hellere Grautöne aufgelöste Wolkenmasse, düster mit der Verheißung zukünftiger Gewitter; doch noch flackerten keine Blitze, noch hallte fernes Donnergrollen, und nur ein paar bedrohliche Tropfen waren gefallen. Zu dem ihnen zugewiesenen Startplatz in einem bestimmten Wald marschierte das Sonnenschein Regiment, obwohl es eigentlich eher ein langsamer Spaziergang war; man wollte sich nicht ermüden, bevor die Schlacht überhaupt begonnen hatte, und die Brisen des Aprils waren ärgerlich feucht, wenn auch kühl. Vor ihnen wanderte eine gelbe Flamme langsam durch die Luft und lenkte sie je nach ihrem Tempo. Susan Bones warf dem Sonnenschein-General immer wieder besorgte Blicke zu, während sie durch den grau beleuchteten Wald marschierten.

Dass Professor Snape hinter Hermine her war, schien sie wirklich erschüttert zu haben. Hermine hatte sogar ihr offizielles Planungstreffen des Sonnenschein Regiments verpasst, was verständlich genug schien; aber als Susan ihr danach ihr Mitgefühl angeboten hatte, hatte Hermine gestammelt, dass sie die Zeit aus den Augen verloren hatte, was sie normalerweise nie sagte, und das Mädchen hatte erschöpft und verängstigt ausgesehen, als hätte sie gerade drei Tage mit einem Dementor in einer Toilettenkabine eingesperrt verbracht.

Selbst jetzt, wo die ganze Aufmerksamkeit des Generals auf die bevorstehende Schlacht hätte gerichtet sein sollen, huschte der Blick des Ravenclaw-Mädchens ständig in alle Richtungen, als erwarte sie, dass dunkle Zauberer aus dem Gebüsch springen und sie opfern würden.

"Das Verbot von Muggelartefakten schränkt unsere Möglichkeiten stark ein", sagte Anthony Goldstein in dem mürrischen Tonfall, mit dem der Junge absichtlichen Pessimismus ausdrückte. "Ich hatte die Idee, zu versuchen, Netze zu verwandeln, um sie auf die Leute zu werfen, aber -"

"Nicht gut", sagte Ernie Macmillan. Der Hufflepuff-Junge schüttelte den Kopf und sah noch ernster aus als Anthony. "Ich meine, das ist so, als würde man einen Fluch zaubern, sie würden ausweichen."

Anthony nickte. "Das habe ich mir auch gedacht. Hast du eine Idee, Seamus?"

Der ehemalige Chaotische Leutnant sah immer noch ein wenig nervös und deplatziert aus, als er mit seinen neuen Kameraden im Sonnenschein-Regiment mitmarschierte. "Tut mir leid", sagte der frischgebackene Captain Finnigan. "Ich bin eher der strategische Meistertyp."

"Ich bin der strategische Meistertyp", sagte Ron Weasley und klang verärgert.

"Es gibt drei Armeen", sagte der Sonnenschein-General bissig, "das heißt, wir kämpfen gegen zwei Armeen gleichzeitig, das heißt, wir brauchen mehr als einen Strategen, das heißt, halt die Klappe, Ron!"

Ron warf ihrem General einen überraschten und besorgten Blick zu.

"Hey", sagte der Gryffindor-Junge in einem beruhigenden Ton, "du solltest Snape nicht so sehr an dich heranlassen -"

"Was denkst du, was wir tun sollten, General?" sagte Susan sehr laut und schnell. "Ich meine, wir haben im Moment nicht wirklich einen Plan."

Ihre offizielle Planungssitzung war erstaunlicherweise gescheitert, weil Hermine nicht da war und sowohl Ron als auch Anthony dachten, sie hätten das Sagen.

"Brauchen wir wirklich einen Plan?", fragte der Sonnenschein-General und klang dabei ein wenig abgelenkt. "Wir haben dich und mich und Lavender und Parvati und Hannah und Daphne und Ron und Ernie und Anthony und Captain Finnigan."

"Das -", begann Anthony.

"Klingt nach einer ziemlich guten Strategie", sagte Ron mit einem zustimmenden Nicken. "Wir haben jetzt so viele starke Soldaten wie die beiden anderen Armeen zusammen. Das Chaos hat nur noch Potter und Longbottom und Nott - na ja, und Zabini auch, nehme ich an -"

"Und Tracey", sagte Hermine.

Mehrere Leute schluckten nervös.

"Ach, hör doch auf", sagte Susan scharf. "Sie ist nur ein kampferprobtes Mitglied von S.P.H.E.W., das ist alles, was General Sonnenschein meint."

"Trotzdem", sagte Ernie und drehte sich um, um Susan ernst anzuschauen, "ich denke, du solltest dich der Gruppe anschließen, die gegen Chaos kämpft, Captain Bones. Ich weiß, dass du deine doppelten magischen Kräfte nur einsetzen können, wenn Unschuldige in Gefahr sind, aber ich meine - nur für den Fall, dass Miss Davis, du weißt schon, außer Kontrolle gerät und versucht, jemandes Seele zu fressen -"

"Ich werde schon mit ihr fertig", sagte Susan, wobei ihre Stimme beruhigend klang. Zugegeben, Susan war im Moment nicht durch einen Metamorphmagus ersetzt worden, aber dann war Tracey wohl auch nicht der Vielsafttrankveränderte Dumbledore oder wer auch immer.

Captain Finnigan intonierte mit tiefer, irgendwie brummiger Stimme: "Ich finde deinen Mangel an Glauben beklagenswert."

Er hob seine Hand, wobei sich Daumen und Zeigefinger fast berührten, und zeigte auf Ernie. Aus irgendeinem Grund schien Anthony Goldstein einen plötzlichen Würgeanfall zu bekommen.

"Was soll das denn heißen?", fragte Ernie.

"Das ist nur etwas, was General Potter manchmal sagt", sagte Captain Finnigan. "Komisch, wenn man zum ersten Mal der Chaos Legion beitritt, scheint alles verrückt zu sein, und dann, nach ein paar Monaten, merkt man, dass eigentlich jeder, der nicht in der Chaos Legion ist, verrückt ist -"

"Ich sagte", sagte Ron laut, "es klingt nach einer guten Strategie. Wir verwandeln nichts, wir ermüden uns nicht, wir werden mit allem fertig, was sie uns entgegenwerfen, und dann überrennen wir sie einfach."

"Okay", sagte Hermine. "Dann machen wir das."

"Aber -", sagte Anthony und warf Ron einen finsteren Blick zu. "Aber General, Harry Potter hat noch sechzehn Leute in seiner Armee. Drache und wir haben jeweils achtundzwanzig. Harry weiß das, er weiß, dass er sich etwas Unglaubliches einfallen lassen muss -"

"Was zum Beispiel?", fragte Hermine und klang gestresst. "Wenn wir nicht wissen, was er vorhat, können wir unsere Magie genauso gut für den Einsatz von Massenfinte aufsparen. Wie wir es letztes Mal hätten tun sollen!"

Susan berührte Hermine sanft an der Schulter.

"General Granger?", sagte Susan. "Ich denke, du solltest vor der Schlacht eine kleine Pause einlegen."

Sie hatte erwartet, dass Hermine widersprechen würde, aber Hermine nickte nur und ging dann etwas schneller und entfernte sich von der Offiziersgruppe des Regiments, ihre Augen beobachteten immer noch den Wald und manchmal den Himmel.

Susan folgte ihr. Es würde nicht gut gehen, wenn es so aussah, als würde die Sonnenschein-Generalin aus ihrer eigenen Offiziersgruppe hinausgeworfen werden.

"Hermine?" sagte Susan leise, nachdem sie ein Stück weit gegangen waren. "Du musst dich konzentrieren. Professor Quirrell hat hier das Sagen, nicht Snape, und er wird nicht zulassen, dass dir oder anderen etwas Schlimmes zustößt."

"Du bist nicht hilfreich", sagte Hermine und klang zittrig. "Du bist überhaupt nicht hilfreich, Captain Bones."

Die beiden gingen schneller, umkreisten einige der anderen Soldaten, inspizierten die Marschrichtung und warfen einen Blick auf die umliegenden Bäume.

"Susan?" sagte Hermine mit leiser Stimme, als sie sich weiter von den anderen entfernt hatten. "Glaubst du, Daphne hat recht, dass Draco Malfoy etwas plant?"

"Ja", sagte Susan sofort, ohne überhaupt darüber nachzudenken. "Das sieht man daran, dass in seinem Namen die Buchstaben M-A-L-F-O und Y vorkommen."

Hermine sah sich um, als wolle sie sich vergewissern, dass niemand sie beobachtete, obwohl das natürlich eine wunderbare Methode war, um andere Leute dazu zu bringen, auf einen aufmerksam zu werden.

"Könnte Malfoy hinter dem stecken, was Snape getan hat?"

"Snape könnte hinter Malfoy stecken", sagte Susan nachdenklich und erinnerte sich an Tischgespräche, die sie bei ihrer Tante gehört hatte, "oder Lucius Malfoy könnte hinter beiden stecken."

Ein leichter Schauer lief Susan über den Rücken, als ihr dieser letzte Gedanke in den Sinn kam. Plötzlich erschien es ihr weniger vernünftig, Hermine zu sagen, sie solle sich einfach auf den kommenden Kampf konzentrieren.

"Warum, hast du irgendeinen Hinweis darauf gefunden?"

Hermine schüttelte den Kopf.

"Nein", sagte das Ravenclaw-Mädchen mit einer Stimme, die fast so klang, als würde sie gleich weinen.

"Ich habe - nur selbst darüber nachgedacht - das ist alles."

…

An seinem zugewiesenen Platz in einem Wald in der Nähe von Hogwarts warteten der Drachengeneral und die Krieger der Drachenarmee, wohin ihre rote Flamme sie geführt hatte, unter einem grauen Himmel. An Dracos rechter Seite stand Padma Patil, seine Stellvertreterin, die einmal die gesamte Drachenarmee angeführt hatte, nachdem Draco betäubt worden war. In Dracos Rücken stand Vincent, der Sohn von Crabbe, einer Familie, die den Malfoys bis in die Ferne der vergessenen Erinnerung gedient hatte; der muskulöse Junge war wachsam, wie er immer wachsam war, egal ob die Schlacht erklärt worden war oder nicht. Weiter hinten stand Gregory von den Goyles wartend neben einem der beiden Besen, die die Drachenarmee erhalten hatte; wenn die Goyles den Malfoys auch nicht so lange gedient hatten wie die Crabbes, so hatten sie doch nicht weniger gut gedient. Und an Dracos linker Seite stand nun ein gewisser Dean Thomas aus Gryffindor, ein Schlammblut oder möglicherweise Halbblut, der nichts von seinem Vater wusste.

Dean Thomas zur Drachenarmee zu schicken, war ein ganz bewusster Schachzug von Harry gewesen, da war sich Draco sicher. Drei andere ehemalige Chaoten waren ebenfalls zur Drachenarmee versetzt worden, und alle beobachteten Draco wie ein Falke, um zu sehen, ob er dem ehemaligen Leutnant auch nur die kleinste Beleidigung darbot. Manche hätten es vielleicht Sabotage genannt, aber Draco wusste es besser. Harry hatte auch Leutnant Finnigan zum Sonnenschein Regiment geschickt, obwohl Professor Quirrells Mandat nur verlangt hatte, dass Harry einen Leutnant abgab. Auch das war ein bewusster Schachzug gewesen, der jedem klar machte, dass Harry nicht einfach seine unbeliebtesten oder inkompetente Soldaten abgab.

In einer Hinsicht wäre es für Draco vielleicht einfacher gewesen, die wahre Loyalität seiner neuen Soldaten zu gewinnen, wenn sie gedacht hätten, dass Harry sie nicht gewollt hatte. In einer anderen Hinsicht… nun, das war nicht leicht in Worte zu fassen. Harry hatte ihm gute Soldaten mit intaktem Stolz gegeben, aber es war mehr als das. Harry hatte seinen Soldaten gegenüber Freundlichkeit gezeigt, aber es war mehr als das. Es war nicht nur, dass Harry fair gespielt hatte, es war etwas, das… das man nur mit der Art und Weise vergleichen konnte, wie das Spiel im Haus Slytherin gespielt wurde.

So hatte Draco Mr. Thomas nicht die geringste Beleidigung dargebracht, sondern ihn direkt an seine Seite gebracht, untergeordnet zu ihm und Padma, aber niemandem sonst. Es war ein Test, hatte Draco Mr. Thomas und allen anderen gesagt, keine Beförderung. Mr. Thomas würde sich innerhalb der Drachenarmee eines Ranges würdig erweisen müssen - aber er würde eine Chance bekommen, und die Chance würde fair sein. Mr. Thomas hatte bei der Zeremonie überrascht ausgesehen (nach dem, was Draco gehört hatte, legte die Chaoslegion keinen Wert auf Formalitäten), aber der Gryffindor-Junge hatte sich ein wenig aufrechter hingestellt und genickt. Und dann, nachdem Mr. Thomas in einer der Trainingseinheiten der Drachenarmee gut genug abgeschnitten hatte, war er zur Strategiesitzung in das riesige Militärbüro der Drachenarmee gebracht worden. Und nach ein paar Minuten der Sitzung hatte Padma zufällig gefragt - als wäre es eine ganz normale Frage - ob Mr. Thomas irgendwelche Ideen hätte, wie man die Chaos Legion besiegen könnte.

Der Gryffindor-Junge hatte fröhlich erzählt, dass Harry vorausgesagt hatte, dass General Malfoy einen seiner Soldaten dazu bringen würde, ihn das zu fragen, und dass Harry ihm die Botschaft gegeben hatte, dass General Malfoy sich fragen sollte, wo sein relativer Vorteil lag - was Draco Malfoy tun konnte oder was die Drachenarmee tun konnte, dem die Chaoslegion nicht gewachsen war - und dann versuchen sollte, ihn nach Kräften auszunutzen.

Dean Thomas konnte sich nicht vorstellen, was dieser Vorteil sein könnte, aber wenn ihm eine Idee einfiel, wie man das Chaos besiegen konnte, würde er sie mitteilen. Harry hatte es ihm schließlich befohlen.

\emph{Seufz}, hatte Draco gedacht, da er nicht wirklich laut seufzen konnte. Aber es war ein guter Rat, und Draco hatte ihn befolgt und saß mit Feder und Pergament an seinem Zimmertisch und listete alles auf, was einen relativen Vorteil darstellen könnte. Und, fast zu Dracos eigener Überraschung, hatte er eine Idee gehabt, eine richtige. Genau genommen hatte er sogar zwei.

Die hohle Glocke schallte durch den Wald und klang irgendwie bedrohlicher als je zuvor. Sofort riefen die beiden Piloten "Auf!" und sprangen auf ihre Besenstiele, um in den grauen Himmel zu steigen.

Mr. und Mrs. Davis waren inzwischen leicht in sich zusammengesackt, mehr aus schierer Muskelerschöpfung als aus einem Nachlassen der Spannung. Vor ihnen flimmerte das weite, weiße Pergament mit drei großen Fenstern, als wären Löcher in den Wald geschnitten worden, die drei Armeen auf dem Marsch zeigten. Kleinere Fenster zeigten die sechs Reiter auf ihren Besen, und die Ecke des Pergaments zeigte eine Ansicht des gesamten Waldes mit leuchtenden Punkten, die Armeen und Späher anzeigten. Das Fenster in Sonnenschein zeigte General Granger und ihre Hauptleute, die in der Mitte des Sonnenschein-Regiments marschierten, geschützt durch Contego-Schirme, zusammen mit einer Anzahl anderer junger Hexen.

Das Sonnenschein Regiment, so hatte der Verteidigungsprofessor bemerkt, wusste sehr wohl, dass es nun einen starken Vorteil an erfahrenen Soldaten erworben hatte, und es galt, diese Soldaten vor einem Überraschungsangriff zu schützen. Abgesehen davon bewegten sich die Sonnenschein-Soldaten in einem gleichmäßigen Marsch vorwärts, um ihre Kräfte zu schonen.

Die Soldaten in General Malfoys Armee, zumindest die mit den besseren Verwandlungsnoten, sammelten Blätter ein und verwandelten sie in irgendetwas…. naja, wenn man Padma Patil ansah, die mit ihrem fast fertig war, sah es so aus, als würde ihr Blatt zu einem linkshändigen Handschuh mit baumelndem Riemen werden. (Das Fenster hatte hineingezoomt, um dies zu zeigen.)

Lord Jugson betrachtete den Bildschirm mit flacher Miene; seine Stimme, wenn er sprach, schien vor Verachtung zu triefen und zu tropfen.

"Was macht Ihr Sohn, Lucius?"

Die fremdstämmige Hexe, die an Draco Malfoys rechter Seite stand, hatte die Verwandlung ihres Handschuhs beendet und brachte ihn nun wie ein Opfer vor den Drachengeneral.

"Ich weiß es nicht", sagte Lucius Malfoy, sein Tonfall war ruhig, wenn auch nicht weniger aristokratisch, "aber ich muss darauf vertrauen, dass er einen guten Grund dafür hat, es zu tun."

Die gesamte Drachenarmee hielt einen Moment inne, als Padma den Handschuh über ihre linke Hand schob, ihn festschnallte und ihn Draco Malfoy präsentierte; dieser blieb ebenfalls stehen, holte einige Male tief Luft, hob seinen Zauberstab, führte eine präzise Folge von acht Bewegungen aus und brüllte: "Colloportus!"

Die Drachenkriegerin hob ihre behandschuhte Hand, beugte sie und machte eine kleine Verbeugung vor Draco Malfoy, der sie etwas oberflächlicher erwiderte, obwohl der Drachengeneral leicht schwankte. Dann kehrte Padma auf ihren Platz an Dracos Seite zurück, und die Drachen begannen erneut zu marschieren.

"Nun", bemerkte Augusta Longbottom. "Ich nehme nicht an, dass jemand etwas erklären möchte?"

Amelia Bones runzelte leicht die Stirn, als sie auf den Bildschirm starrte.

"Aus irgendeinem Grund", sagte die amüsierte Stimme von Professor Quirrell, "scheint es, dass der Spross von Malfoy in der Lage ist, für einen Erstklässler überraschend starke Magie zu wirken. Das liegt natürlich an der Reinheit seines Blutes. Sicherlich hätte sich der gute Lord Malfoy nicht offen über die Gesetze der Minderjährigenmagie hinweggesetzt, indem er dafür gesorgt hätte, dass sein Sohn einen Zauberstab erhält, bevor er in Hogwarts aufgenommen wird."

"Ich schlage vor, Sie sind vorsichtig mit Ihren Andeutungen, Quirrell", sagte Lucius Malfoy kalt.

"Oh, das bin ich", sagte Professor Quirrell. "Ein Colloportus kann nicht durch Finite Incantatem aufgelöst werden; er erfordert einen Alohomora von gleicher Stärke. Bis dahin wird ein so verzauberter Handschuh geringeren materiellen Kräften widerstehen, die Schlafverhexung und die Betäubungsflüche abwehren. Und da weder Mr. Potter noch Miss Granger einen Gegenzauber wirken können, der stark genug ist, ist dieser Zauber auf diesem Schlachtfeld unbesiegbar. Es ist weder die ursprüngliche Absicht des Zaubers noch die Absicht dessen, der Mr. Malfoy einen Notfallzauber zum Ausweichen vor seinen Feinden beigebracht hat. Aber es scheint, dass Mr. Malfoy Kreativität gelernt hat."

Lucius Malfoy hatte sich aufgerichtet, als der Verteidigungsprofessor sprach; er saß nun aufrecht auf seiner gepolsterten Bank, den Kopf merklich höher erhoben als zuvor, und als er sprach, war es mit ruhigem Stolz. "Er wird der größte Lord Malfoy sein, der je gelebt hat."

"Ein schwaches Lob", sagte Augusta Longbottom unter ihrem Atem; Amelia Bones kicherte, ebenso wie Mr. Davis für einen winzigen, fatalen Bruchteil einer Sekunde, bevor er mit einem erstickten Gurgeln aufhörte.

"Ich bin ganz Ihrer Meinung", sagte Professor Quirrell, obwohl nicht klar war, zu wem er sprach. "Unglücklicherweise für Mr. Malfoy ist er noch neu in der Kunst der Kreativität, und so hat er einen klassischen Fehler von Ravenclaw begangen."

"Und was könnte das sein?", fragte Lucius Malfoy, dessen Stimme nun wieder kühl wurde.

Professor Quirrell hatte sich in seinem Sitz zurückgelehnt, die blassblauen Augen wurden kurz unscharf, als eines der Fenster seinen Blickwinkel innerhalb des großen Bildschirms verschob und heranzoomte, um den Schweiß zu zeigen, der nun auf Draco Malfoys Stirn stand.

"Es ist eine so schöne Idee, dass Mr. Malfoy ihre praktischen Schwierigkeiten völlig übersehen hat."

"Könnte mir das jemand erklären?", sagte Lady Greengrass. "Nicht alle von uns Anwesenden sind Experten in solchen … Angelegenheiten."

Amelia Bones sprach, die Stimme der alten Hexe etwas trocken.

"Es wird alle dazu verleiten, zu versuchen, Flüche zu fangen, denen sie besser einfach ausweichen sollten. Umso mehr, wenn sie wenig Übung darin haben, sie zu fangen. Und das Wirken von so vielen Zaubern wird ihren stärksten Krieger ermüden."

Professor Quirrell nickte der Direktorin halb anerkennend zu.

"Genau, Madam Bones. Für Mr. Malfoy ist es neu, Ideen zu haben, und wenn er eine hat, ist er stolz auf sich, weil er sie hat. Er hat noch nicht genug Ideen gehabt, um unbeirrt diejenigen zu verwerfen, die in einigen Aspekten schön und in anderen unpraktisch sind; er hat noch kein Vertrauen in seine eigene Fähigkeit erworben, sich bessere Ideen auszudenken, wenn er sie braucht. Was wir hier sehen, ist nicht Mr. Malfoys beste Idee, fürchte ich, sondern seine einzige."

Lord Malfoy wandte sich einfach wieder den Bildschirmen zu, als hätte der Verteidigungsprofessor seine Existenzberechtigung verwirkt.

"Aber -", sagte Lord Greengrass. "Aber was in Merlins Namen tut Harry Potter -"

Sechzehn verbliebene Soldaten der Chaoslegion - oder vielmehr fünfzehn plus Blaise Zabini - marschierten selbstbewusst durch den Wald, ihre Schuhe polterten über den noch trockenen Boden. Ihre Tarnuniformen fügten sich noch mehr als sonst in den Wald ein, alle Farben waren von den Tönen eines bedeckten Tages verwaschen.

Sechzehn Chaoslegionäre, gegen achtundzwanzig Drachenkrieger und achtundzwanzig Sonnenscheinsoldaten. Der allgemeine Konsens war, dass es bei so schlechten Chancen praktisch unmöglich war, dass sie verlieren würden. Schließlich musste sich General Chaos bei solchen Chancen etwas wirklich Spektakuläres einfallen lassen. Es hatte fast etwas Alptraumhaftes an sich, dass jeder zu erwarten schien, dass Harry bei Bedarf Wunder aus dem Hut zaubern würde, wann immer es nötig war. Es bedeutete, dass man, wenn man das Unmögliche nicht schaffte, seine Freunde enttäuschte und seinem Potenzial nicht gerecht wurde…

Harry hatte sich nicht die Mühe gemacht, sich bei Professor Quirrell über "\emph{zu viel Druck}" zu beschweren. Harrys mentales Modell des Verteidigungsprofessors hatte ihn mit einem stark verärgerten Gesichtsausdruck vorhergesagt, der etwas sagte wie: "\emph{Sie sind durchaus in der Lage, dieses Problem zu lösen, Mr. Potter; haben Sie es überhaupt versucht?}" und dann mehrere hundert Punkte abzog.

Von oben, wo zwei Besen ihren Marsch beobachteten, rief die hohe junge Stimme von Tess Walsh "Freund!" und nach einem weiteren Moment "Gingersnap!" Ein paar Sekunden später kehrte die Soldatin, die sich den Codenamen Gingersnap gegeben hatte, mit einer doppelten Handvoll Eicheln zurück, leicht schwitzend in der kühlen, aber feuchten Luft von dem Jogging, das sie zu der Eiche gebracht hatte, die Neville entdeckt hatte. Gingersnap näherte sich der Stelle, an der Shannon ein Uniformhemd in der Hand hielt, das am Hals zugebunden war, damit niemand eine Tasche verwandeln musste. Als Gingersnap ihre Hände nach vorne brachte, um zu versuchen, ihre Eicheln in das Halte-Shirt zu kippen, riss die chaotische Shannon kichernd das Shirt nach rechts, dann wieder nach links, als Gingersnap einen weiteren Versuch unternahm, die Eicheln zu kippen, bis ein scharfes "Miss Friedman!" von Lieutenant Nott Shannon dazu veranlasste, zu seufzen und das Shirt stillzuhalten. Gingersnap kippte ihre Eicheln zu den schon angesammelten und machte sich dann auf die Suche nach weiteren. Irgendwo im Hintergrund sang Ellie Knight ihre ganz eigene Version des Marschlieds der Chaos Legion, und etwa die Hälfte der anderen Soldaten versuchte, mitzusummen, obwohl sie die Melodie nicht kannten. In der Nähe beendete Nita Berdine, die eine hohe Punktzahl in Verwandlung hatte, die Herstellung einer weiteren grünen Sonnenbrille und reichte sie Adam Beringer, der die Sonnenbrille zusammenfaltete und sie in seine Uniformtasche steckte. Andere Soldaten trugen bereits ihre eigenen grünen Sonnenbrillen, trotz des bewölkten Tages.

\emph{Man könnte vermuten, dass es dafür eine unglaublich komplizierte und faszinierende Erklärung gab, und man hätte Recht.}

Zwei Tage zuvor hatte Harry inmitten seiner Bücherregale in dem bequemen Schaukelstuhl gesessen, den er sich für den Keller seines Koffers besorgt hatte, und in der ruhigen Zeit zwischen Unterricht und Abendessen still vor sich hin gegrübelt und über \emph{Macht} nachgedacht.

Damit sechzehn Chaoten achtundzwanzig Sonnen und achtundzwanzig Drachen besiegen konnten, brauchten sie einen Kraftverstärker. Es gab Grenzen, was man mit Manövern und Taktik machen konnte. Es musste eine Geheimwaffe geben, und die musste unbesiegbar sein, oder zumindest einigermaßen unaufhaltsam.

Muggel-Artefakte waren jetzt in den Scheinkämpfen von Hogwarts verboten, verboten durch einen Erlass des Ministeriums. Und das Problem, einen anderen cleveren und ungewöhnlichen Zauber zu finden, war, dass eine Armee, die doppelt so groß war wie man selbst, fast alles, was man versuchte, mit Gewalt beenden konnte.

Das Sonnenscheinregiment mochte diese Taktik mit dem verwandelten Kettenhemd übersehen haben, aber jetzt, wo Professor Quirrell darauf hingewiesen hatte, würde sie niemand mehr übersehen. Und Finite Incantatem war ein brachialer Gegenzauber, der mindestens so viel Magie erforderte wie der Zauber, der gebrochen werden sollte… was, wenn man zahlenmäßig stark unterlegen war, eine ganz neue Art von militärischer Herausforderung darstellte. Der Feind konnte alles beenden, was man versuchte, und hatte immer noch genug Magie für Schilde und Salven von Schlafverhexungen übrig. Es sei denn, man konnte irgendwie Potenzen beschwören, die über die normalen Kräfte von Hogwartsschülern im ersten Jahr hinausgingen, etwas, das zu mächtig war, als dass der Feind es hätte finiten können.

Also hatte Harry Neville gefragt, ob er jemals von kleinen, sicheren Opferritualen gehört hatte - und dann, nachdem das Geschrei und die Rufe abgeklungen waren, nachdem Harry aufgehört hatte, über unbrechbare Schwüre zu streiten und die ganze Sache einfach als unmöglich vom Standpunkt der Öffentlichkeitsarbeit aus aufgegeben hatte, war Harry klar geworden, dass er nicht einmal so weit hatte gehen müssen. In den normalen Hogwarts-Klassen wurde einem beigebracht, wie man Potenzen beschwört, die weit über die eigene Kraft hinausgehen.

\emph{Manchmal erkannte man, obwohl man gerade auf etwas schaute, nicht, worauf man schaute, bis man zufällig genau die richtige Frage stellte}.

Verteidigung. Zauberei. Verwandlung. Zaubertränke. Geschichte der Magie. Astronomie. Besenfliegen. Kräuterkunde…

"Feind!", schrie die Stimme von oben.

Gut, dass Neville Longbottom nicht die leiseste Ahnung hatte, dass seine Großmutter ihn beobachtete, sonst wäre er noch verlegener gewesen, wenn er aus voller Kehle furchterregende Kampfschreie gebrüllt und alle drei Sekunden Luminos gewirkt hätte, während er durch einen dichten Wald von Bäumen raste, dicht auf den Fersen von Gregory Goyle.

("Aber -" sagte Augusta Longbottom, wobei ihr Gesichtsausdruck fast so viel Erstaunen wie Sorge zeigte. "Aber Neville hat doch Höhenangst!")

("Nicht alle Ängste sind von Dauer", sagte Amelia Bones. Die alte Hexe musterte die große Leinwand vor ihnen mit einem messenden Blick.

"Oder vielleicht hat er Mut gefunden. Am Ende ist es das Gleiche.")

Ein roter Schimmer - Neville wich aus, fast in einen Baum, aber er wich aus; und dann schaffte es Neville irgendwie auch, fast allen Ästen auszuweichen, bevor sie ihm ins Gesicht schlugen. Jetzt zog Mr. Goyles Besen immer weiter weg - obwohl sie beide auf genau demselben Besen saßen und Mr. Goyle mehr wog, fiel Neville irgendwie immer noch zurück. Also verlangsamte Neville, zog zurück, winkelte aus dem Wald heraus und begann zu beschleunigen, zurück in Richtung der Stelle, wo die Chaoslegion noch immer marschierte.

Zwanzig Sekunden später - es war keine lange Verfolgungsjagd gewesen, nur eine aufregende - war Neville wieder unter seinen Chaotikerkollegen und stieg von seinem Besen ab, um ein wenig auf dem Boden zu laufen.

"Neville -", sagte General Potter. Harrys Stimme klang ein wenig distanziert, während er vorsichtig und stetig durch den Wald ging, seinen Zauberstab immer noch auf die fast fertige Form des Objekts angesetzt, das er langsam verwandelte. Neben ihm sah Blaise Zabini, der an einer kleineren Version derselben Verwandlung arbeitete, wie ein watschelnder Inferi aus, als er vorwärts stolperte.

"Ich habe dir gesagt - Neville - du musst nicht -"

"Doch, muss ich", sagte Neville. Er schaute nach unten, wo seine Finger den Besen umklammerten, und sah, dass nicht nur seine Hände, sondern seine ganzen Arme zitterten. Aber wenn nicht noch irgendjemand im Chaos jeden Tag eine Stunde lang mit Mr. Diggory das Duellieren geübt hatte und danach noch eine Stunde lang unter vier Augen das Zielen, war Neville wahrscheinlich der beste Schütze vom Besen aus, selbst wenn man berücksichtigt, dass er kein besonders guter Flieger war.

"Gute Show, Neville", sagte Theodore von dort aus, wo er vor ihnen allen herlief und die Chaoslegion nur mit seinem Unterhemd bekleidet durch den Wald anführte (Augusta Longbottom und Charles Nott tauschten kurz erstaunte Blicke aus und rissen dann ihre Blicke wie versteinert voneinander los).

Neville holte ein paar Mal tief Luft und versuchte, seine Hände zu beruhigen; Harry war nicht gut im tiefen strategischen Denken, während er mitten in einer ausgedehnten Verwandlung steckte.

"Leutnant Nott, haben Sie eine Ahnung, warum die Drachenarmee das gerade getan hat? Sie haben einen Besen verloren -"

Die Drachen hatten den Kampf mit einer Finte begonnen, um eine Ablenkung für Mr. Goyles Annäherung durch den Wald zu schaffen; Neville hatte erst fast zu spät bemerkt, dass es zwei Besen waren, die angriffen. Aber die Chaos Legion hatte den anderen Piloten erwischt. Das war der Grund, warum Besen normalerweise nicht angriffen, bevor Armeen aufeinander trafen, es bedeutete, dass eine ganze Armee das Feuer auf den Besen konzentrieren würde.

"Und die Dragons haben niemanden erwischt, oder?"

"Nö!" sagte Tracey Davis stolz. Auch sie marschierte jetzt an der Seite von General Potter, ihren Zauberstab tief und wachsam umklammert, während ihre Augen den umliegenden Wald abtasteten. "Ich habe einen prismatische Schild gezaubert, etwa den Bruchteil einer Sekunde, bevor Mr. Goyle Zabini verflucht hat, und so wie Mr. Goyle seinen anderen Arm ausgestreckt hatte, glaube ich, dass er auch den General niederschlagen wollte."

Die Slytherin-Hexe lächelte mit boshafter Zuversicht.

"Mr. Goyle versuchte es mit einer Brechbohrer-Verhexung, musste aber zu seinem Entsetzen feststellen, dass seine schwache Magie meinen neu entdeckten dunklen Kräften nicht gewachsen war, hahahaha!"

Einige Chaoten lachten mit ihr, aber in Nevilles Magen machte sich ein mulmiges Gefühl breit, als er erkannte, wie nahe die Chaos Legion einer völligen Katastrophe gekommen war. Wenn Mr. Goyle es geschafft hätte, beide Verwandlungen zu stören -

…

"Bericht!", schnappte der Drachengeneral und tat sein Bestes, um die Müdigkeit zu verbergen, die er nach siebzehn Sperrzaubern verspürte, von denen noch mehr kommen sollten. Auf Gregorys Stirn standen jetzt Schweißperlen.

"Der Feind hat Dylan Vaughan erwischt", sagte Gregory förmlich. "Harry Potter und Blaise Zabini haben jeweils etwas Dunkelgraues und Rundliches verwandelt, ich glaube nicht, dass es fertig war, aber es sah aus, als wäre es groß und hohl, eine Art Kessel. Der von Zabini war kleiner als der von Potter. Ich konnte keinen der beiden erwischen oder ihre Verwandlung stören, Tracey Davis blockierte mich. Neville Longbottom sitzt auf einem Besen und ist immer noch ein schrecklicher Flieger, aber er zielt wirklich gut."

Draco hörte zu und runzelte die Stirn, dann schaute er zu Padma und Dean Thomas, die beide den Kopf schüttelten und damit andeuteten, dass sie sich auch nicht vorstellen konnten, was groß und grau und wie ein Kessel geformt sein könnte.

"Sonst noch etwas?", fragte Draco. Wenn das alles war, hatten sie einen Besen umsonst verloren -

"Das einzige andere Seltsame, was ich gesehen habe", sagte Gregory und klang verwirrt, "war, dass einige Chaoten so eine Art Schutzbrille trugen…"

Draco dachte darüber nach und bemerkte nicht, dass er aufgehört hatte zu marschieren oder dass die gesamte Drachenarmee automatisch mit ihm stehen geblieben war. "War da irgendetwas Besonderes an den Brillen?" fragte Draco.

"Ähm …" sagte Gregory. "Sie waren … grünlich, vielleicht?"

"Okay", sagte Draco. Wieder ohne nachzudenken, begann er wieder zu gehen und seine Drachen folgten ihm. "Hier ist unsere neue Strategie. Wir werden nur noch elf Drachen gegen die Chaoslegion schicken, nicht vierzehn. Das sollte reichen, um sie zu besiegen, jetzt, wo wir ihren besonderen Vorteil neutralisieren können."

Es war ein Glücksspiel, aber man musste es manchmal riskieren, wenn man in einem Dreikampf als Erster ankommen wollte.

"Du hast den Plan von Chaos durchschaut, General Malfoy?", fragte Mr. Thomas mit beträchtlicher Überraschung.

"Was haben sie vor?", fragte Padma.

"Ich habe nicht die leiseste Ahnung", sagte Draco mit einem Grinsen von feinster Selbstgefälligkeit. "Wir werden einfach das Offensichtliche tun."

…

Harry, der inzwischen seinen Kessel fertiggestellt hatte, schöpfte vorsichtig Eicheln in den Behälter, während die Späher nach einer nahe gelegenen Wasserquelle suchten, die als flüssige Basis verwendet werden konnte. Sie waren im Wald schon häufig auf Senkgruben und Miniaturbäche gestoßen, es sollte also nicht lange dauern. Ein anderer Späher hatte einen geraden Stock mitgebracht, der als Rührgerät dienen würde, so dass Harry keinen verwandeln musste.

\emph{Manchmal erkannte man, obwohl man gerade auf etwas schaute, nicht, worauf man schaute, bis man zufällig genau die richtige Frage stellte…}

\emph{Wie kann ich magische Kräfte beschwören, die eigentlich außerhalb der Reichweite von Erstklässlern liegen sollten?}

Es gab eine warnende Geschichte, die der Meister der Zaubertränke ihnen erzählt hatte (mit viel Spott und Gelächter, um die Dummheit als unbedeutend erscheinen zu lassen, statt als gewagt und romantisch), über eine Hexe im zweiten Jahr in Beauxbatons, die ein paar extrem verbotene und teure Zutaten gestohlen hatte und versuchte, Vielsaft zu brauen, um sich die Gestalt eines anderen Mädchens für Zwecke auszuleihen, die besser unerwähnt blieben. Nur hatte sie es geschafft, den Trank mit Katzenhaaren zu verunreinigen, und anstatt sofort einen Heiler aufzusuchen, hatte sich die Hexe in einem Badezimmer versteckt, in der Hoffnung, die Wirkung würde einfach nachlassen; und als sie schließlich gefunden wurde, war es zu spät gewesen, um die Verwandlung vollständig rückgängig zu machen, was sie zu einem verzweifelten Leben als eine Art Katzen-Mädchen-Hybrid verdammte.

Harry hatte nicht begriffen, was das bedeutete, bis er die richtige Frage gestellt hatte - das bedeutete, dass ein junger Zauberer oder eine junge Hexe mit der Herstellung von Zaubertränken Dinge tun konnte, die sie mit Zaubern nicht einmal annähernd erreichen konnten. Vielsaft war einer der mächtigsten Zaubertränke, die man kannte … aber was Vielsaft zu einem Zaubertrank auf Prüfungsniveau machte, war anscheinend nicht das erforderliche Alter, bevor man genug magische Kraft hatte; es war, weil es schwierig war, den Zaubertrank präzise zu brauen und was mit einem passierte, wenn man es vermasselte.

Niemand in den Armeen hatte bis dahin versucht, einen Zaubertrank zu brauen. Aber Professor Quirrell würde einem fast alles durchgehen lassen, wenn es etwas war, was man auch in einem echten Krieg hätte tun können.

\emph{Schummeln ist Technik}, hatte der Verteidigungsprofessor ihnen einmal beigebracht. Oder besser gesagt, \emph{Schummeln ist das, was die Verlierer Betrug nennen, und ist bei erfolgreicher Ausführung zusätzliche Quirrell-Punkte wert.}

Im Prinzip war es nicht unrealistisch, ein paar Kessel zu verwandeln und Tränke aus allem zu brauen, was gerade zur Hand war, wenn man genug Zeit hatte, bevor die Armeen aufeinander trafen. Also hatte Harry sein Exemplar von "Magische Entwürfe und Zaubertränke" hervorgeholt und sich auf die Suche nach einem sicheren, aber nützlichen Trank gemacht, den er in den Minuten vor Beginn des Kampfes brauen konnte - ein Trank, der den Kampf zu schnell für Gegenzauber gewinnen oder Zaubereffekte erzeugen würde, die für Erstklässler zu stark waren, um sie zu beenden.

\emph{Manchmal, obwohl man gerade auf etwas schaute, merkte man nicht, worauf man schaute, bis man zufällig genau die richtige Frage stellte…}

\emph{Welchen Zaubertrank kann ich nur mit Komponenten brauen, die ich in einem gewöhnlichen Wald gesammelt habe?}

Jedes Rezept in Magische Entwürfe und Tränke verwendete mindestens eine Zutat aus einer magischen Pflanze oder einem magischen Tier. Was unglücklich war, denn alle magischen Pflanzen und Tiere befanden sich im Verbotenen Wald, nicht in den sichereren und kleineren Wäldern, in denen die Schlachten stattfanden. Jemand anderes hätte an dieser Stelle vielleicht aufgegeben. Harry blätterte von einem Rezept zum nächsten, überflog es immer schneller in der dämmernden Erkenntnis, die bestätigte, was er bereits gelesen hatte und nun zum ersten Mal sah.

\emph{Jedes einzelne Zaubertränke-Rezept schien mindestens eine magische Zutat zu verlangen, aber warum sollte das so sein? Zaubersprüche erforderten überhaupt keine materiellen Komponenten; man sagte nur die Worte und schwenkte den Zauberstab.}

Harry hatte sich die Herstellung von Zaubertränken im Wesentlichen analog vorgestellt: Anstatt dass die gesprochenen Silben ohne nachvollziehbaren Grund einen Zaubereffekt auslösten, sammelte man einen Haufen ekliger Zutaten und rührte viermal im Uhrzeigersinn, und das löste willkürlich einen Zaubereffekt aus. Wenn man bedenkt, dass die meisten Zaubertränke gewöhnliche Komponenten wie Stachelschweinkiele oder gedünstete Schnecken verwenden, würde man erwarten, dass es einige Tränke gibt, die nur gewöhnliche Komponenten verwenden. Aber stattdessen verlangte jedes einzelne Rezept in Magische Entwürfe und Tränke mindestens eine Komponente aus einer magischen Pflanze oder einem magischen Tier - eine Zutat wie Seide von einer Acromantula oder Blütenblätter von einer Venusfeuerfalle.

\emph{Manchmal, obwohl man gerade auf etwas schaute, wusste man nicht, worauf man schaute, bis man zufällig genau die richtige Frage stellte..}.

\emph{Wenn das Herstellen eines Zaubertranks wie das Wirken eines Zaubers ist, warum falle ich dann nicht vor Erschöpfung um, nachdem ich einen so mächtigen Trank gebraut habe?}

Am vorletzten Freitag hatte Harrys doppelter Zaubertrankstunde einen Siede-Heiltrank gebraut… obwohl selbst die trivialsten Heilzauber, wenn man sie mit Zauberstab und Beschwörung zu wirken versuchte, mindestens Zaubersprüche des vierten Jahres waren. Und danach hatten sie sich alle so gefühlt, wie sie sich normalerweise nach dem Zaubertrankunterricht fühlten, nämlich nicht nennenswert magisch erschöpft. Harry hatte sein Exemplar von Magische Entwürfe und Zaubertränke mit einem Schnalzen zugeklappt und war hinunter in den Ravenclaw-Gemeinschaftsraum geeilt. Harry hatte einen Ravenclaw aus dem siebten Jahr gefunden, der seine U.T.Z.-Hausaufgaben in Zaubertränke machte, und dem älteren Jungen eine Sickel bezahlt, damit er sich \emph{'Höchst potente Zaubertränke'} für fünf Minuten auslieh; denn Harry hatte nicht den ganzen Weg zur Bibliothek laufen wollen, um eine Bestätigung zu finden. Nachdem er fünf Rezepte in dem Buch aus dem siebten Schuljahr überflogen hatte, hatte Harry das sechste Rezept gelesen, für einen Zaubertrank mit Feueratem, für den Aschenwinder-Eier benötigt wurden… und das Buch warnte, dass das entstehende Feuer nicht heißer sein würde als das magische Feuer, das den Aschenwinder hervorgebracht hatte, der die Eier gelegt hatte.

Harry hatte mitten im Ravenclaw-Gemeinschaftsraum "Heureka!" geschrien und war von einem in der Nähe befindlichen Vertrauensschüler heftig zurechtgewiesen worden, der dachte, Mr. Potter versuche, einen Zauber zu sprechen. Niemand in der Zaubererwelt kannte oder interessierte sich für einen antiken Muggel namens Archimedes, noch für die Erkenntnis des Urphysikers, dass das aus einer Badewanne verdrängte Wasser dem Volumen des in die Badewanne eintretenden Objekts entspricht…

\emph{Erhaltungssätze}.

Sie waren die entscheidende Erkenntnis bei mehr Muggel-Entdeckungen, als Harry ohne weiteres zählen konnte.

In der Muggeltechnologie konnte man keine Feder einen Meter vom Boden abheben, ohne dass die Kraft von irgendwoher kam. Wenn man geschmolzene Lava betrachtete, die aus einem Vulkan sprudelte, und fragte, woher die Hitze kam, würde ein Physiker einem von radioaktiven Schwermetallen im Zentrum des geschmolzenen Erdkerns erzählen. Wenn man fragte, woher die Energie für die Radioaktivität kam, würde der Physiker auf eine Ära vor der Entstehung der Erde verweisen und auf eine Ur-Supernova in den frühen Tagen der Galaxie, die Atomkerne schwerer als das natürliche Limit gebacken hatte, wobei die Supernova Protonen und Neutronen zu einem engen, instabilen Paket komprimierte, das einen Teil der Energie der Supernova zurückgab, als es zerbrach.

Eine Glühbirne wurde durch Elektrizität angetrieben, durch ein Kernkraftwerk, durch eine Supernova… Man könnte das Spiel bis zurück zum Urknall spielen.

Magie schien nicht so zu funktionieren, um es gelinde auszudrücken. Die Haltung der Magie gegenüber Gesetzen wie dem Energieerhaltungssatz lag irgendwo zwischen einem riesigen ausgestreckten Mittelfinger und einem Achselzucken der totalen Gleichgültigkeit.

Aguamenti erschuf Wasser aus dem Nichts, soweit man wusste; es gab keinen bekannten See, dessen Wasserspiegel jedes Mal sank. Das war ein einfacher Zauber aus dem fünften Jahr, der von Zauberern nicht als beeindruckend angesehen wurde, denn ein einfaches Glas Wasser zu erschaffen, erschien ihnen nicht erstaunlich. Sie hatten nicht die verrückte Vorstellung, dass Masse konserviert werden müsste oder dass das Erzeugen von einem Gramm Masse irgendwie gleichbedeutend mit dem Erzeugen von 90.000.000.000.000 Joule Energie war. Es gab einen Oberstufenzauber, auf den Harry gestoßen war, dessen wörtliche Beschwörungsformel "\emph{Arresto Momentum!}" lautete, und als Harry gefragt hatte, ob das Momentum irgendwo anders hingeht, hatte er nur einen verwirrten Blick bekommen. Harry hatte immer verzweifelter nach einer Art Erhaltungsprinzip in der Magie Ausschau gehalten, egal wo…

… und die ganze Zeit über war es in jeder Zaubertränke-Stunde direkt vor ihm gewesen.

\emph{Die Herstellung von Zaubertränken erschafft keine Magie, sie erhält sie, deshalb braucht jeder Trank mindestens eine magische Zutat. Und wenn man Anweisungen wie "viermal gegen den Uhrzeigersinn und einmal im Uhrzeigersinn rühren" befolgte} - so hatte Harry vermutet -, \emph{tat man so etwas wie einen kleinen Zauber, der die Magie in den Zutaten umformte.}

(Und die physische Form auflöste, so dass sich Zutaten wie Stachelschweinkiele sanft in eine trinkbare Flüssigkeit auflösten; Harry vermutete stark, dass ein Muggel, der genau dasselbe Rezept befolgte, am Ende nichts als ein widerliches Durcheinander erhalten würde.)

Das war es, was die Herstellung von Zaubertränken wirklich ausmachte, die Kunst, vorhandene magische Essenzen zu transformieren. Deshalb war man nach dem Zaubertrank-Unterricht ein bisschen müde, aber nicht viel, denn man hat die Tränke nicht selbst hergestellt, sondern nur die Magie umgestaltet, die bereits vorhanden war. Und das war der Grund, warum eine Hexe im zweiten Jahr Vielsaft brauen konnte, oder zumindest nahe dran kam. Harry hatte das Buch weiter durchgeblättert, auf der Suche nach etwas, das seine glänzende neue Theorie widerlegen könnte. Nach fünf Minuten hatte er dem älteren Jungen eine weitere Münze zugeworfen (trotz seiner Proteste) und weitergemacht.

Der \emph{Trank der Riesenkraft} erforderte einen Re'em, um die pürierten Dugbogs zu zertreten, die man in den Trank rührte. Das war seltsam, hatte Harry nach einem Moment festgestellt, denn zerquetschte Dugbogs waren nicht selbst stark, sie waren nur… sehr, sehr zerquetscht, nachdem der Re'em mit ihnen fertig war. Ein anderes Rezept besagte, man solle den Trank "\emph{mit geschmiedeter Bronze berühren}", d.h. einen Knut in einer Zange fassen, damit man die Oberfläche des Trankes abschöpfen konnte; und wenn man den Knut ganz hineinfallen ließ, warnte das Buch, würde der Trank sofort überhitzen und den Kessel überkochen.

Harry hatte auf die Rezepte und ihre Warnungen gestarrt und eine zweite, noch seltsamere Hypothese aufgestellt.

\emph{Natürlich würde es nicht so einfach sein, Zaubertränke zu machen, indem man magische Potenziale in den Zutaten nutzte, so wie Muggelautos durch das Verbrennungspotenzial von Benzin angetrieben werden. Magie würde nie so} \emph{vernünftig sein…}

Und dann war Harry zu Professor Flitwick gegangen - da er sich Professor Snape nicht außerhalb des Unterrichts nähern wollte - und Harry hatte Professor Flitwick erzählt, dass er einen neuen Zaubertrank erfinden wollte, und er wusste, was die Zutaten sein sollten und was der Zaubertrank bewirken sollte, aber er wusste nicht, wie er das erforderliche Rührmuster ableiten sollte - nachdem Professor Flitwick aufgehört hatte, vor Entsetzen zu schreien und in kleinen Kreisen zu laufen, und Professor McGonagall in das anschließende heftige Verhör gerufen worden war, um Harry zu versprechen, dass es in diesem Fall sowohl akzeptabel als auch wichtig sei, dass er seine zugrundeliegende Theorie offenbare, hatte sich herausgestellt, dass Harry keine originelle magische Entdeckung gemacht hatte, sondern ein so altes Gesetz wiederentdeckt hatte, dass niemand wusste, wer es zuerst formuliert hatte:

\emph{Ein Trank gibt das aus, was in die Herstellung seiner Zutaten investiert wird. Die Hitze der Koboldschmieden, die den bronzenen Knut gegossen hatte, die Kraft von Re'em, die die Dugbogs zermalmt hatte, das magische Feuer, das den Aschenwinder hervorgebracht hatte: All diese Potenzen konnten durch den zauberähnlichen Prozess des Umrührens der Zutaten in exakten Mustern abgerufen, freigeschaltet und umstrukturiert werden.}

(Vom Muggel-Standpunkt aus gesehen war es einfach nur seltsam, eine gestörte Version der Thermodynamik, die von jemandem erfunden wurde, der dachte, das Leben sollte fair sein. Aus Muggelsicht war die beim Schmieden des Knuts aufgewendete Wärme nicht in die Bronze gegangen, die Wärme hatte sich in die Umgebung verflüchtigt und war dauerhaft weniger verfügbar. Energie war konserviert, konnte weder erzeugt noch zerstört werden; die Entropie nahm immer zu. Aber Zauberer dachten nicht so: Aus ihrer Sicht war es logisch, dass man, wenn man eine gewisse Menge an Arbeit in die Herstellung eines Knuts gesteckt hatte, genau die gleiche Arbeit wieder herausbekommen konnte. Harry hatte versucht zu erklären, warum das etwas seltsam klang, wenn man von Muggeln erzogen worden war, und Professor McGonagall hatte amüsiert gefragt, warum die Muggelperspektive besser sei als die der Zauberer).

Das Grundprinzip der Zaubertränke-Herstellung hatte keinen Namen und keine Standardformulierung, da man sonst in Versuchung geraten könnte, es aufzuschreiben. Und jemand, der nicht klug genug war, das Prinzip selbst herauszufinden, könnte es lesen. Und sie würden auf alle möglichen glänzenden Ideen kommen, um neue Tränke zu erfinden. Und dann würden sie in Katzenmädchen verwandelt werden.

Es war Harry sehr deutlich gemacht worden, dass er diese besondere Entdeckung nicht mit Neville teilen würde, und auch nicht mit Hermine nach der nächsten Schlacht der Armeen. Harry hatte versucht, etwas darüber zu sagen, dass Hermine in letzter Zeit wirklich schlecht drauf zu sein schien und dies genau die Art von Sache war, die sie aufmuntern könnte. Professor McGonagall hatte ganz lapidar gesagt, dass er nicht einmal daran denken solle, und Professor Flitwick hatte seine kleinen Hände gehoben und eine Geste gemacht, als würde er einen Zauberstab in zwei Hälften brechen. Obwohl die beiden Professoren so freundlich gewesen waren, vorzuschlagen, dass Mr. Potter, wenn er zu wissen glaubte, was die Zutaten des Zaubertranks sein sollten, in der Lage sein könnte, ein bereits existierendes Rezept zu finden, das dasselbe tat; und Professor Flitwick hatte mehrere Bände in der Hogwarts-Bibliothek erwähnt, die nützlich sein könnten…

…

Der riesige pergamentartige Bildschirm zeigte jetzt nur noch eine Luftansicht des Waldes, aus der man gerade noch die getarnten Formen dreier Armeen erkennen konnte, die, in je zwei Gruppen aufgeteilt, zu ihrer Dreierschlacht zusammenkamen. Die Bänke des Quidditch-Stadions füllten sich nun schnell mit der eher leicht gelangweilten Sorte von Zuschauern, die nur beim Endkampf dabei sein und alle langweiligen Punkte auf dem Weg dorthin auslassen wollten. (Wenn etwas an Professor Quirrells Schlachten nicht stimmte, so war man sich weitgehend einig, dann, dass seine Spektakel nicht annähernd so lange dauerten wie Quidditchspiele, wenn sie erst einmal begonnen hatten. Darauf hatte Professor Quirrell nur geantwortet: "So ist der Realismus", und das war's.) Durch das riesige Fenster - es war jetzt ein einziges Fenster, aus dem man aus großer Höhe beobachten konnte - kamen die vagen Ansammlungen winziger, getarnter Formen immer näher. Näher. Beinahe berührend - Das riesige weiße Pergamentfenster zeigte den ersten Hauch der Schlacht zwischen Sonnenschein und Chaos, eine schreiende Masse von rennenden Kindern mit Smiley-Gesichtern auf der Brust, die mit hochgehaltenen Contego-Schilden nach vorne stürmten und andere riefen "Somnium!" - Bis einer von ihnen mit erschrockener Stimme "Prismatis!" schrie und der ganze Ansturm vor der funkelnden Kraftwand, die vor ihnen aufgetaucht war, zum Stehen kam.

Tracey Davis war hinter den Bäumen hervorgetreten.

"Das stimmt", sagte Tracey, ihre Stimme tief und grimmig, während sie ihren Zauberstab auf die Barriere richtete. "Ihr solltet mich fürchten. Denn ich bin Tracey Davis, die Darke Lady! Das ist Darke Lady, buchstabiert D-A-R-K-E, mit einem E!"

(Amelia Bones, die Leiterin der Abteilung für magische Strafverfolgung, warf einen fragenden Blick auf Mr. und Mrs. Davis, die beide aussahen, als wären sie am liebsten auf der Stelle gestorben.)

Hinter der prismatischen Barriere fand eine Art gedämpfter Streit zwischen den Sonnenschein Soldaten statt, von denen insbesondere einer von einigen der anderen gescholten zu werden schien. Dann, einen Moment später, zuckte Tracey zusammen. Susan Bones hatte sich an die Spitze des Kontingents gestellt.

("Meine Güte", sagte Augusta Longbottom. "Was hat Ihre Großnichte wohl in Hogwarts gelernt?"

"Ich weiß es nicht", sagte Amelia Bones ruhig, "aber ich werde ihr einen Schokoladenfrosch und Anweisungen schicken, mehr darüber zu erfahren.")

Die prismatische Barriere verschwand. Die Sonnen stürmten wieder vorwärts. Tracey schrie mit hoher Stimme: "Inflammare!", und der Angriff der Sonnen kam zu einem weiteren plötzlichen Stillstand, als eine Feuerlinie zwischen ihnen im halbtrockenen Gras aufloderte und sich ausdehnte, um dem Weg von Traceys Zauberstab zu folgen, als sie ihn richtete; einen Augenblick später rief Susan Bones "Finite Incantatem!" und die Flammen verdunkelten sich, hellten sich auf, verdunkelten sich im Wettstreit ihres Willens, andere Soldaten hoben ihre Stäbe, um auf Tracey zu zielen; und das war, als Neville Longbottom schreiend vom Himmel stürzte.

…

Einer der Drachenkrieger, Raymond Arnold, machte ein Handzeichen, das nach vorne und schräg nach links zeigte; und es gab ein plötzliches gedämpftes Flüstern unter dem Kontingent der Drachenarmee, als sie sich alle leise in Richtung des Feindes neu orientierten. Die Sonnen wussten, dass sie dort waren, natürlich wussten es beide Armeen; aber irgendwie waren sie in diesem Moment alle instinktiv still geworden. Die Drachen krochen weiter vorwärts, und dann noch weiter, die dumpfen, getarnten Formen der Sonnen begannen zwischen den entfernten Bäumen aufzutauchen, und noch immer sprach niemand, niemand brüllte den Ruf zum Angriff. Draco stand jetzt an der Spitze seiner Soldaten, Vincent hinter ihm und Padma nur einen Schatten weiter hinten; wenn die drei den Schock der Besten aushalten konnten, hatte der Rest der Drachenarmee vielleicht eine Chance. Dann sah Draco eine Sonne, die ihn aus der Ferne anstarrte, in der Vorhut ihrer eigenen Armee; sie starrte ihn mit einem Blick voller Wut an - quer über das Waldschlachtfeld trafen sich ihre Augen. Draco hatte nur den Bruchteil einer Sekunde Zeit, um sich im Hinterkopf zu fragen, worüber Hermine Granger so wütend war, bevor der Ruf von ihren beiden Armeen ertönte; und sie rannten alle zum Angriff vor.

…

Die anderen Chaoten waren nun zwischen den Bäumen aufgetaucht, einige hatten sich aus den Bäumen fallen lassen, und der Kampf war nun in vollem Gange, jeder feuerte in jede Richtung auf alles, was wie ein Feind aussah. Dazu kamen einige Sonnen, die Neville Longbottom "Lumos!" zuriefen, während der Chaos-Hufflepuff sich drehte und auf Kursen durch die Luft schoss, die man nur als, in der Tat, "chaotisch" bezeichnen konnte - und es passierte, wie es nur ein einziges Mal von zwanzig Mal im Scheinluftkampf passierte, dass Neville Longbottoms Besen unter seinen geballten Händen hellrot glühte. Das hätte bedeuten müssen, dass Longbottom aus dem Spiel war. Dann, auf der Hogwarts-Tribüne, inmitten der zuschauenden Schülerschar, ging ein Schrei hoch -

\emph{Schlachtfelrrealismus}. Das war Professor Quirrells eine Hauptregel. Man konnte mit allem durchkommen, wenn es realistisch war, und im echten Leben verschwand ein Soldat nicht einfach, wenn sein Besen von einem Fluch getroffen wurde.

Neville stürzte zu Boden und schrie: "Chaotische Landung!" und die Chaoten rissen ihre Aufmerksamkeit von den Kämpfen ab, um den Schwebezauber zu wirken (und gleichzeitig zu rennen, damit sie keine leichte Beute waren), fast alle anderen blieben stehen und starrten - und Neville Longbottom knallte auf den laubbedeckten Waldboden, landete auf einem Knie, einem Fuß und beiden Händen, als würde er niederknien, um zum Ritter geschlagen zu werden. Alles blieb stehen. Selbst Tracey und Susan hielten in ihrem Zweikampf inne. Im Stadion verschwanden alle Publikumsgeräusche. Es herrschte eine allgemeine Stille, die sich aus Erstaunen, Besorgnis und schier sprachlosem Staunen zusammensetzte, während alle darauf warteten, was als nächstes passieren würde. Und dann stand Neville Longbottom langsam auf und richtete seinen Zauberstab auf die feindlichen Soldaten.

Obwohl es niemand auf dem Schlachtfeld hörte, hatte ein großer Teil des Publikums im Stadion begonnen, in stetig ansteigenden Tönen jedes Mal, wenn das Wort ausgesprochen wurde, "DOOM DOOM DOOM DOOM" zu skandieren, weil man das einfach nicht sehen konnte, ohne zu denken, dass es musikalische Begleitung brauchte.

"Die Menge jubelt deinem Enkel zu", sagte Amelia Bones.

Die alte Hexe warf einen prüfenden Blick auf den Bildschirm.

"Das tun sie", sagte Augusta Longbottom. "Einige, wenn ich richtig höre, jubeln: \emph{Unser Blut für Neville! Unsere Seelen für Neville}!"

"Durchaus", sagte Amelia und nahm einen Schluck aus einer Teetasse, die kurz zuvor noch nicht da gewesen war. "Das zeigt, dass der Junge Führungspotenzial hat."

"Diese Jubelrufe", fuhr Augusta fort, wobei ihre Stimme eine noch fassungslosere Qualität annahm, "scheinen von den Hufflepuff-Bänken zu kommen."

"Es ist das Haus der Loyalen, meine Liebe", sagte Amelia.

"Albus Percival Wulfric Brian Dumbledore! Was in Merlins Namen ist in dieser Schule passiert?"

Lucius Malfoy betrachtete die Bildschirme mit einem ironischen Lächeln, seine Finger klopften in einem nicht erkennbaren Muster auf seine Armlehne. "Ich weiß nicht, was beängstigender ist, der Gedanke, dass er einen verborgenen Plan hinter all dem hat, oder der Gedanke, dass er keinen hat."

"Sehen Sie!", rief der Lord von Greengrass. Der adrette junge Mann hatte sich halb aus seinem Stuhl erhoben und deutete mit dem Finger auf den Bildschirm. "Da ist sie!"

…

"Wir nehmen ihn beide sofort", flüsterte Daphne. Sie wusste, dass ein paar angsterfüllte Minuten echter Kampferfahrung, ein paar Mal in der Woche, nicht ausreichen würden, um Nevilles regelmäßiges Duelltraining mit Harry und Cedric Diggory im gleichen Zeitraum zu übertreffen. "Er ist zu viel für einen von uns, aber wir beide zusammen - ich setze meinen Charme ein, du versuchst einfach, ihn zu betäuben -"

Hannah neben ihr nickte, und dann schrien sie beide aus vollem Halse und stürmten vorwärts, wobei die Schwebezauber der beiden unterstützenden Sonnenschein Soldaten sie schneller und leichtfüßiger machten, wobei Daphne bereits "Tonare!" schrie, selbst als Hannah ein riesiges Contego-Schild vor ihnen in Bewegung hielt, und mit einem kurzen Extrahub sprangen sie über die Köpfe der vorderen Soldatenschar und landeten vor Neville mit hoch wehendem Haar -

(Fotos waren bei allen Hogwarts-Spielen strengstens verboten, aber irgendwie landete dieser Moment trotzdem auf der Titelseite des Quibbler vom nächsten Tag). -

und im selben Moment, weil der Kampf gegen die älteren Rabauken die geringste Spur von Zögern weggebrannt hatte, feuerte Hannah ihre erste Schlafverhexung auf Neville ab (sie hatte die Beschwörung begonnen, während sie noch in der Luft war), während Daphne, sich mehr auf Kraft als auf Geschwindigkeit konzentrierte. Aber Neville sprang hoch, nicht zur Seite, er sprang höher, als er hätte springen sollen, so dass ihr glühendes Schwert nur die Luft unter seinen Füßen durchschnitt. Irgendwie erkannte Daphne, was es bedeutete, dass Neville noch andere Chaoten über sich schweben hatte, und zwar rechtzeitig, um ihre Klinge über ihren Kopf zu heben, aber Neville fiel zu schnell, und als seine Klinge auf ihre prallte, war es, als würde sie von einem klatscher getroffen. Es warf Daphne von den Füßen und ließ sie rückwärts auf das Gras schleudern, wo sie hart auf dem Rücken aufschlug. Dann wäre es für sie vielleicht vorbei gewesen, wenn Neville nicht selbst zu hart gelandet wäre und mit einem schmerzhaften Keuchen in die Knie gegangen wäre. Und dann, bevor Neville seine glühende Klinge zu Boden bringen konnte, rief Hannah "Somnium!" und Neville taumelte hektisch zurück - obwohl natürlich kein Zauberspruch von Hannahs Zauberstab gekommen war, das Hufflepuff-Mädchen konnte nicht wirklich so schnell wieder feuern -, was Daphne eine Sekunde gab, um auf die Füße zu kommen und beide Hände wieder um ihren Zauberstab zu bekommen -

…

"Lieber Merlin", sagte Lady Greengrass. Ihre Stimme wirkte unsicher, die aristokratische Haltung war durchlöchert. "Meine Tochter kämpft mit dem Zauber der Uralten Klinge. In ihrem ersten Jahr. Ich wusste gar nicht, dass sie so ein außergewöhnliches Talent besitzt -"

"Ausgezeichnetes Blut", sagte Charles Nott anerkennend, was Augusta zu einem Schnauben veranlasste.

"Meine gute Frau", sagte Professor Quirrell und klang ernst. "Tun Sie Ihrer Tochter nicht so unrecht. Das ist nicht nur Talent, was Sie da sehen." Seine Stimme wurde ein wenig trockener. "Vielmehr ist es das, was passiert, wenn Kinder ihre wetteifernden Bemühungen in ein Spiel stecken, das tatsächliches Zaubern beinhaltet."

…

"Expelliarmus!", schrie Draco und versuchte, seine Stimme nicht schwanken zu lassen, während er gleichzeitig dem feuerroten Betäuber auswich, den Hermine Granger auf ihn abgefeuert hatte, wobei sich seine Muskeln verkrampften, weil er in die falsche Richtung ausweichen musste - sie hatte nach links gedeutet und dann mit einem geheimnisvollen Ruck nach rechts gefeuert -, Hermine wich dem schnell fliegenden Duellfluch aus und schrie mit kaum einem weiteren Moment des Innehaltens: "Steleus!", eine Weitwinkelverhexung, der Draco nicht ausweichen konnte, aber er schaffte es, seinen Zauberstab auf sein eigenes Gesicht zu richten und "Quiescus!" zu rufen, bevor der plötzliche Drang zum Einatmen in einen Niesanfall übergehen konnte, der den Kampf beendet hätte.

Draco Malfoy war schon halb erschöpft von all den Schließzaubern und Verwandlungen vorhin, aber seine Verwirrung wich langsam dem Gefühl, dass sein eigenes Blut kochte, er wusste nicht, warum Granger ihn plötzlich so wütend angriff, aber wenn sie einen Kampf wollte, würde er ihr einen geben -

(Die Drachen und Sonnen hielten nicht inne, um das Duell ihrer Generäle zu beobachten, die Drachen waren zu diszipliniert, um innezuhalten und zuzusehen, und das bedeutete, dass die Sonnen auch weiterkämpfen mussten; aber die gaffenden Zuschauer auf der Hogwarts-Quidditch-Tribüne wurden sogar von Nevilles und Daphnes Spektakel abgelenkt und richteten ihre Augen auf das Duell der beiden Generäle, als Malfoy und Granger einen Fluch nach dem anderen und einen Zauber nach dem anderen aufeinander abfeuerten, wobei sie schneller zauberten, als jeder andere Schüler ihres Jahrgangs es hätte tun können, Der geübte Duell-Tanz des Drachen-Generals wurde von der rasenden Energie des Sonnenschein-Generals übertroffen. Der Kampf zwischen ihnen begann, einem Duell von Erwachsenen zu ähneln, als die beiden magiebegabtesten Erstklässler zu exotischeren Zaubern als der üblichen Schlafverhexung griffen.) -

obwohl Draco allmählich erkannte, dass er, Harry und Professor Quirrell Miss Granger zwar als so tötungswillig wie eine Schale nasser Weintrauben abgetan hatten, sie Sie aber bis jetzt nie wütend gesehen hatten.

…

Daphne schlug mit ihrer uralten Klinge zu, wobei sie wieder nicht versuchte, hart zuzuschlagen, sondern die Klinge nur so schnell wie möglich bewegte, während Hannah gleichzeitig "Somnium!" rief und Neville sprang wieder zurück, aber es war ein weiterer Bluff gewesen und Hannah bewegte sich, um einen echten Zauber fast aus nächster Nähe abzufeuern - und Neville Longbottom tat genau das, was, er würde es später erklären, Cedric Diggory ihm beigebracht hatte, wenn er gegen Bellatrix Black kämpfen würde, nämlich sich herumzudrehen und Hannah richtig hart in die Magengrube zu treten.

Das Hufflepuff-Mädchen gab einen traurigen Laut von sich, einen keuchenden Schmerzensschrei, als sie von den Füßen gestoßen wurde, weil der harte Schuh mit der Wucht von Nevilles ganzem Körper in ihren Unterleib eindrang. Einen Augenblick lang stand das Schlachtfeld still, alles hielt inne, außer Hannahs fallender Gestalt. Dann verzog sich Nevilles Gesicht zu absoluter Bestürzung und er senkte seinen Zauberstab, der Chaotische Leutnant ging instinktiv auf seine Hausgenossin zu, während er mit der anderen Hand nach ihr griff - selbst als Hannah ihren Sturz in eine Rolle verwandelte und mit erhobenem Zauberstab auf ihn zustürmte. Einen Sekundenbruchteil später versenkte Daphne, die ebenfalls nicht gezögert hatte, ihre uralte Klinge direkt in Nevilles Rücken, wodurch die Muskeln des chaotischen Leutnants durch die betäubende Magie, die sich in ihm entlud, krampfhaft zuckten, selbst als Hannahs Schlafverhexung ihre Wirkung zeigte, und dann lag der letzte Spross der Longbottoms noch immer auf dem Boden, mit einem Ausdruck völliger Überraschung im Gesicht.

…

"Heute hat Mr. Longbottom eine wertvolle Lektion über Gefühle wie Mitleid und Reue gelernt", sagte Professor Quirrell.

"Und über Ritterlichkeit", sagte Amelia und nippte wieder an ihrem Tee.

…

"Geht es dir gut?", flüsterte Daphne, als sie schützend über der Stelle stand, an der Hannah auf dem Boden lag und sich den Bauch hielt. Das Mädchen erwiderte nichts, außer weiteren Würgegeräuschen, die sich anhörten, als würde Hannah versuchen, sich nicht zu übergeben, während sie versuchte, nicht zu weinen. Irgendwie, auch wenn es vielleicht keine gute Taktik war - es wäre besser gewesen, wenn Hannah direkt verhext worden wäre, als dass andere Soldaten gefesselt wurden, um sie zu beschützen - schienen einige Sonnen mit ihren Zauberstäben fest umklammert vor Hannah zu stehen und starrten die Chaoten wütend an. Jemand hatte eine prismatische Barriere zwischen den beiden Gruppen errichtet, Daphne konnte nicht sehen, wer. Und aus irgendeinem Grund schienen die Chaos Legionäre den Angriff nicht zu forcieren. Sogar Tracey hatte ihren grimmigen Gesichtsausdruck völlig abgelegt und verlagerte ihr Gewicht nervös von einem Fuß auf den anderen, als hätte sie Schwierigkeiten, sich zu erinnern, auf welcher Seite sie stand -

"Halt!", rief eine Stimme. "Pausiert die Schlacht!"

Es war zwar keine große Schlacht im Gange, aber sie machten Pause. General Potter, der so aussah wie der Junge, der lebte, schritt aus den Bäumen hervor und hielt etwas Großes in Tarnkleidung unter einem Arm.

"Atmet Miss Abbott noch gut?" schrie General Potter.

Daphne drehte sich nicht um. Sie vertraute nicht darauf, dass es sich nicht um eine Falle handelte - es war absolut sicher, dass, wenn die Chaoten die Gelegenheit zum Angriff nutzten, Professor Quirrell es nicht nur für legal erklären, sondern ihnen hinterher auch Extrapunkte geben würde. Aber Daphne konnte die Antwort mit ihren Ohren gut genug hören, es war ja nicht so, als würde Hannah versuchen, leise zu atmen, und so sagte sie: "Irgendwie schon."

"Sie sollte hier raus und zu jemandem gehen, der Heilzauber anwenden kann", sagte Harry. "Nur für den Fall, dass das etwas kaputt gemacht hat."

Von hinter Daphne sagte eine kleine keuchende Stimme:

"Ich - kann - noch - kämpfen -"

"Miss Abbott, nicht -" sagte Harry, gerade als hinter Daphne das Geräusch von jemandem ertönte, der auf den Rasen zurücksank, nachdem er versucht hatte, auf die Beine zu kommen, und es nicht geschafft hatte. Alle zuckten zusammen, aber Daphne drehte Harry nicht den Rücken zu.

"Warum haben die Lehrer den Kampf nicht beendet?", fragte Susan mit wütender Stimme.

"Ich nehme an, weil Miss Abbott nicht in Gefahr ist, bleibende Schäden zu erleiden und Professor Quirrell denkt, dass wir wertvolle Lektionen lernen", sagte Harry mit harter Stimme. "Hören Sie, Miss Abbott, wenn Sie gehen, wird sich auch Tracey aus dem Kampf zurückziehen. Sie sind bereits in der Überzahl, also ist das ein sehr guter Deal für Ihre Seite. Bitte nehmen Sie es an."

"Hannah, geh einfach!", sagte Daphne. "Ich meine, sag einfach, dass du raus bist!" Als Daphne einen Blick zurückwarf, sah sie, dass Hannah den Kopf schüttelte, immer noch zu einem Ball zusammengerollt im Gras lag.

"Ach, was soll's", sagte Harry. "Chaoten! Je schneller wir sie betäuben, desto schneller ist sie hier weg! Wir werden das sehr schnell erledigen, auch wenn wir Verluste in Kauf nehmen! Waffenstillstand beenden! TUNFISCH!"

Daphnes politisches Hinterhirn hatte nur einen Augenblick Zeit, um zu bewundern, wie Harrys wenige Worte die Chaoten gerade zu den Guten gemacht hatten, und dann tauchten die Chaoten in fast perfektem Gleichklang ihre Hände in die Taschen ihrer Uniformen und zogen grüne Sonnenbrillen in einem ungewohnten Stil hervor. Nicht wie etwas, das man am Strand tragen würde, eher wie eine Schutzbrille für fortgeschrittene Zaubertränke - Dann erkannte Daphne, was gleich passieren würde, und hob ihre andere Hand, um ihre Augen zu schützen, gerade als Harry das Tuch vom Kessel riss.

Die Flüssigkeit, die herausspritzte, als Harry Potter den Inhalt des Kessels in die Luft warf, war zu hell, um gesehen zu werden, zu grell, um es sich vorzustellen, glühend wie die Sonne in dutzendfacher Vergrößerung - (was es auch war, es war das Sonnenlicht, das investiert worden war, um die Eicheln zu erschaffen, die helle Energie, die einen Baum angetrieben hatte, der sich aus dem nackten Dreck erhob, lodernd hell, ein glühendes Purpur, die Farbe der gemischten blauen und roten Wellenlängen, die das Chlorophyll absorbierte mit fast keinem der grünen Wellenlängen, die das Chlorophyll reflektierte, um die grüne Farbe der Blätter zu erzeugen was die Farbe der Sonnenbrille der Chaos Legion war, die gemacht wurde, um grüne Wellenlängen durchzulassen, rote und blaue blockierte und selbst das glühendste violette Blendlicht auf etwas Erträgliches reduzierte) -

das grelle Licht loderte weiter und weiter, Daphne versuchte, ihren Arm von den Augen zu nehmen, aber sie stellte fest, dass sie nirgends direkt hinsehen konnte, selbst das sekundäre violette Blendlicht war so hell, dass sie blinzeln musste; und sie hatte nur Zeit, ein Finite Incantatem zu rufen, das nicht funktionierte, bevor ein Schlafzauber sie traf. Was von dem Kampf übrig blieb, dauerte danach nicht mehr sehr lange.

…

"JETZT!", brüllte Blaise Zabini, ehemals Sonnenschein, jetzt Kommandeur eines Trupps von Chaoslegionären. "Ich meine, Tunfisch!"

Die Hand des Slytherin-Jungen griff nach dem Tuch, das den Kessel vor der auslösenden Berührung des Tageslichts abschirmte, und begann bereits, es zur Seite zu schieben.

"JETZT!", brüllte Dean Thomas, ehemals Chaos, und kommandierte eine Abordnung von Drachenkriegern. "TUT, WAS SIE TUN!"

Die Chaoten von Zabinis Truppe steckten die Hände in die Taschen ihrer Uniformen und kamen mit grünen Sonnenbrillen heraus -- die Drachen taten das gleiche

(Wie General Malfoy erklärt hatte, musste man nicht wissen, warum die Chaos-Legion grün gefärbte Zaubertrankbrillen trug, um ein paar Exemplare zu verwandeln, wenn Mr. Goyle davon berichtete.)

"DAS IST SCHUMMELN!", kreischte Blaise Zabini.

"DAS IST TECHNIK!" brüllte Dean zurück. "DRACHEN, ANGRIFF!"

("Verzeihung", sagte die Lady Greengrass. "Könnten Sie aufhören, so zu lachen, Mr. Quirrell? Es ist nervtötend.")

"SCHIE?T AUF IHRE BRILLEN!", rief Blaise Zabini, als die beiden Armeen durch das allgegenwärtige, augenbetäubende Blendlicht kopflos aufeinander zu rannten.

"WIR KÖNNEN NOCH GEWINNEN!"

"Ihr habt ihn gehört!" brüllte Dean. "HOLT IHRE BRILLEN!"

Blaise Zabinis Antwort darauf war nicht gerade wortgewandt.

Der Kampf dauerte noch viel länger an.

…

"Stupor!", schrie der Sonnenschein-General.

Draco wich nicht aus, er konterte nicht, er hatte für beides nicht mehr genug Energie übrig, alles, was er tun konnte, war, seine linke Hand in Position zu peitschen und zu hoffen - Der rote Fluch verpuffte wieder an Dracos kolloportiertem Handschuh, den er genauso wie den Rest der Drachenarmee verwandelt und mit einem Zauber belegt hatte. Das war alles, was ihn jetzt noch rettete, dieser Schild.

Es hätte Zeit für einen Gegenangriff sein sollen, aber Draco konnte nur den Atem anhalten, während die beiden in den nicht enden wollenden Bewegungen ihres Duells unter den Bäumen hin und her tanzten. Ihm gegenüber keuchte General Granger schwer, das Gesicht des jungen Mädchens glitzerte vor Schweiß wie Tau, ihr kastanienbraunes Haar war zu braunen Zöpfen geflochten. Ihre Tarnuniform war mit feuchten Flecken befleckt, ihre Schultern zitterten sichtlich vor Erschöpfung, aber ihr Zauberstab war immer noch stahlhart, wo er während all ihrer Bewegungen auf Draco gerichtet blieb. Ihre Augen funkelten, ihre Wangen waren vor Wut gerötet.

\emph{Also, kleines Mädchen, warum tust du heute so, als würdest du wie ein Erwachsener kämpfen?} Der Spott kam ihm in den Sinn, aber er glaubte nicht wirklich, dass er Granger noch wütender machen musste; also sagte Draco stattdessen einfach - obwohl er seine eigene Stimme schwanken hören konnte -

"Gibt es einen Grund, warum du wütend auf mich bist, Granger?"

Das Mädchen schnappte selbst nach Luft, ihre eigene Stimme schwankte, als sie sprach. "Ich weiß, was du vorhast", sagte Hermine Granger, ihre Stimme erhob sich. "Ich weiß, was du und Snape vorhast, Malfoy, und ich weiß, wer dahintersteckt!"

"Hä?" Sagte Draco, ohne darüber nachzudenken.

Das schien Grangers Wut nur noch zu steigern, und ihre Finger wurden weiß um den Zauberstab, den sie auf ihn richtete.

Und als Draco es begriff, kochte sein eigenes Blut in seinen Adern. \emph{Sogar sie dachte, dass er heimlich ein Komplott gegen sie schmiedete -}

"Du auch?!" brüllte Draco. "Ich habe dir geholfen, du bucklige Tussi! Du, du, du" - stotternd an allen dunklen Flüchen vorbei, die ihm in den Sinn kamen, bis er etwas fand, das er ihr tatsächlich entgegenschleudern konnte - "DENSAUGEO!"

Aber Granger blitzte auf und wirbelte um den zahnverlängernde Verhexung herum, und dann kam ihr eigener Zauberstab und zielte fast aus nächster Nähe, selbst als Draco seine linke Hand wie einen Schild hochhob und den magisch verschlossenen Handschuh zwischen sich und das, was sie abfeuern wollte, legte, und die Stimme des Sonnenschein-Generals erhob sich zu einem Schrei, der über das ganze Schlachtfeld zu hören war - "ALOHOMORA!"

Die Zeit hätte innehalten müssen. Aber das tat sie nicht. Stattdessen klickte der Handschuh als der Kolloportus gebrochen wurde.

\emph{Einfach so.}

Die Bildschirme zeigten alles sehr deutlich, für das gesamte zuschauende Hogwarts-Stadion. Und die totenstille Stille, die sich über jede Bank auf der Tribüne legte, sagte, dass jeder ganz genau verstand, was es bedeutete, dass der Spross des Hauses Malfoy soeben von einem Muggelgeborenen mit seiner Magie überwältigt worden war.

Hermine Granger hielt in ihrem Kampf nicht inne, gab kein Zeichen, dass sie überhaupt wusste, was sie getan hatte; stattdessen holte sie mit ihrem Fuß zu einem Tritt im Muggelstil aus, der Draco den Zauberstab sauber aus der Hand schlug, wobei sich sein geschockter Geist und Körper ein wenig zu langsam bewegten. Draco stürzte nach seinem Zauberstab und krabbelte verzweifelt auf dem Boden herum, aber von hinten sagte eine Mädchenstimme "Somnium!" und Draco Malfoy fiel hin und stand nicht wieder auf.

Es gab einen weiteren Moment der eisigen Stille. Die Sonnenschein-Generalin schwankte auf ihren Füßen und sah aus, als würde sie in Ohnmacht fallen. Dann schrien die Drachenkrieger aus vollem Halse und stürmten vor, um ihren gefallenen Kommandanten zu rächen.

…

Mr. und Mrs. Davis zitterten, als sie von den bequemen Sesseln der Quidditch-Loge der Fakultät aufstanden; sie konnten sich beim Gehen nicht ganz umklammern, aber sie hielten sich fest an den Händen und gaben sich Mühe, unsichtbar zu sein. Wären sie Kinder gewesen, die jung genug für versehentliche Zauberei waren, hätten sie sich wahrscheinlich spontan selbst desillusioniert. Der ältere Charles Nott sagte nichts, als er von seinem Stuhl aufstand. Der vernarbte Lord Jugson sagte nichts, als er von seinem eigenen Stuhl aufstand. Lucius Malfoy sagte nichts, als er aufstand. Alle drei drehten sich ohne Pause um und schritten auf die Treppe der erhöhten Tribüne zu, bewegten sich im unheimlichen Gleichklang wie ein Auroren-Trio -

"Lord Malfoy", sagte der Verteidigungsprofessor in mildem Ton. Der Mann saß immer noch in seinem eigenen Stuhl und blickte auf seine pergamentartigen Bildschirme, die Arme schlaff an der Seite, als ob er aus irgendeinem Grund keine Lust hätte, sich zu bewegen. Der weißhaarige Mann blieb kurz vor dem Ausgangstor stehen, und der ältere Mann und der vernarbte Mann blieben ebenfalls stehen und flankierten ihn.

Lord Malfoys Kopf drehte sich, zu leicht, um eine Form der Anerkennung zu sein, aber in die Richtung des Verteidigungsprofessors.

"Ihr Sohn hat sich heute außergewöhnlich gut geschlagen", sagte Professor Quirrell. "Ich muss gestehen, dass ich ihn unterschätzt habe. Und er hat sich die Loyalität seiner Armee verdient, wie Sie gesehen haben."

Die Stimme des Verteidigungsprofessors war immer noch sehr mild.

"Als Lehrer Ihres Sohnes bin ich der Meinung, dass es ihm nicht gut tun wird, wenn Sie sich in seine -"

Lord Malfoy und seine Mitstreiter verschwanden die Treppe hinunter.

"Ein guter Versuch, Quirinus", sagte Dumbledore leise. Auf dem Gesicht des alten Zauberers zeichneten sich kleine Sorgenfalten ab; auch er hatte sich nicht von seinem Platz erhoben und starrte auf die Pergamentschirme, als wären sie noch aktiv. "Glauben Sie, er wird zuhören?"

Die Schultern des Verteidigungsprofessors zuckten leicht, die einzige Bewegung, die sie seit dem Ende des Kampfes gezeigt hatten.

"Nun", sagte die Lady Greengrass, als sie sich erhob, mit den Fingerknöcheln knackte und sich streckte, ihr Mann schwieg neben ihr. "Ich muss sagen, das war ziemlich … interessant …"

Amelia Bones hatte sich ohne Umschweife von ihrem eigenen gepolsterten Sitz erhoben. "In der Tat interessant", sagte Direktor Bones. "Ich muss gestehen, dass mich die Geschicklichkeit, mit der diese Kinder gegeneinander gekämpft haben, beunruhigt."

"Geschicklichkeit?" sagte Lord Greengrass. "Ihre Zaubersprüche schienen mir nicht sonderlich beeindruckend zu sein. Abgesehen von dem von Daphne natürlich."

Die alte Hexe rührte ihre Augen nicht von der Stelle, wo sie den kahlen Kopf des Verteidigungsprofessors anstarrte.

"Der Betäubungsfluch ist kein Zauber den Erstklässler normalerweise beherrschen, Lord Greengrass, aber das ist nicht die Fähigkeit, die ich im Sinn hatte. Sie haben sich mit diesen einfachen Zaubern gegenseitig unterstützt, sie haben schnell auf Überraschungen reagiert …"

Die Direktorin der magischen Strafverfolgung hielt inne, als ob Sie nach Worten suchte, die ein einfacher Zivilist verstehen konnte.

"Mitten im Kampf", sagte sie schließlich, "mit Zaubersprüchen, die in alle Richtungen flogen … schienen sich diese Kinder wie zu Hause zu fühlen."

"In der Tat, Direktor Bones", sagte der Verteidigungsprofessor. "Manche Künste werden am besten in der Jugend gelehrt."

Die Augen der alten Hexe verengten sich.

"Sie bereiten sie darauf vor, eine militärische Streitmacht zu werden, Professor. Zu welchem Zweck?"

"Jetzt warte Sie mal!", warf Lord Greengrass ein. "Es gibt viele Schulen, in denen sie im ersten Jahr Duellieren lehren!"

"Duellieren?", fragte der Verteidigungsprofessor. Von hinten war nicht zu erkennen, ob das blasse Gesicht lächelte. "Das ist nichts, Lord Greengrass, im Vergleich zu dem, was meine Schüler gelernt haben. Sie haben gelernt, im Angesicht von Hinterhalten und stärkeren Gegnern nicht zu zögern. Sie haben gelernt, sich anzupassen, wenn sich die Kampfbedingungen ändern und wieder ändern. Sie haben gelernt, ihre Verbündeten zu beschützen, diejenigen, die wertvoller sind, mehr zu schützen und Soldaten aufzugeben, die nicht gerettet werden können. Sie haben gelernt, dass sie Befehle befolgen müssen, um zu überleben. Einige haben sogar ein wenig Kreativität gelernt. Oh nein, Lord Greengrass, diese Zauberer und Hexen werden sich nicht in ihren Häusern verstecken und darauf warten, beschützt zu werden, wenn die nächste Bedrohung kommt. Sie werden wissen, wie man kämpft!"

Augusta Longbottom klatschte lautstark dreimal in die Hände.

…

\emph{Wir haben gewonnen.}

Es war das Erste, was Draco hörte, als er auf dem Schlachtfeld aufwachte, Padma, die ihm erzählte, wie sich seine Soldaten nach seinem Sturz wieder aufgerafft hatten. Wie Sie, dank der Weitsicht des Drachengenerals, seine Truppe zum Sieg über das Chaos geführt hatte. Wie General Potter den Teil des Sonnenschein Regiments besiegt hatte, der mit ihm kämpfte. Wie Mr. Thomas' Drachenkrieger mit ihren eigenen Schutzbrillen und den Sonnenbrillen der besiegten Chaoten wieder zum Hauptteil der Soldaten stießen. Wie nur Augenblicke später General Potters verbliebenes Kontingent die beiden anderen Armeen mit einem Trank angegriffen hatte, der sengendes Licht ausstrahlte.

Aber Drache hatte sowohl gegenüber Sonnenschein als auch gegenüber dem Chaos den zahlenmäßigen Vorteil und genügend Sonnenbrillen für ihre Krieger gehabt; und so war es Padma gelungen, ihre geerbte Armee zum Sieg zu führen.

Dem Leuchten in Padmas Augen und ihrem arroganten Lächeln nach zu urteilen, das auch einem Malfoy zur Ehre gereicht hätte, erwartete sie Glückwünsche.

Draco schaffte es, irgendeine Form des Lobes zwischen seinen zusammengebissenen Zähnen hervorzukräuseln, und hätte hinterher nicht sagen können, was es war. Die ausländische Hexe, so schien es, hatte keine Ahnung, was passiert war oder was es bedeutete.

\emph{Ich habe verloren.} Die Drachen stapften unter grauem Himmel zurück nach Hogwarts, und kalte Tropfen landeten schwer auf Dracos Haut, einer nach dem anderen. Während er noch fassungslos gewesen war, hatte es begonnen, der lang versprochene Regen begann endlich zu fallen. Jetzt blieb Draco nur noch eine Möglichkeit. Ein erzwungener Zug, wie Mr. MacNair, der Draco Schach beigebracht hatte, es genannt hätte. Harry Potter würde es wahrscheinlich nicht gefallen, wenn er wirklich in Granger verliebt war, wie alle sagten. Aber der erzwungene Zug, wie Mr. MacNair ihn definiert hatte, war einer, den man machen musste, wenn man wollte, dass das Spiel überhaupt weiterging. Es spielte sich immer wieder in Dracos Kopf ab, selbst als er wie ein Automat durch die riesigen Portale von Hogwarts schritt, Vincent und Gregory mit zwei scharfen Worten wegschickte und allein in seinem privaten Schlafzimmer war, auf seinem Bett saß und die Wand über seinem Schreibtisch anstarrte.

Die Erinnerung füllte seinen Geist, als hätte man ihn mit einem Dementor eingesperrt. \emph{Sein Handschuh klickte und fiel weg} - Draco wusste, er wusste, was er falsch gemacht hatte. Er war so müde gewesen, nachdem er siebenundzwanzig Sperrzauber für alle anderen Drachenkrieger gewirkt hatte. Weniger als eine Minute war nicht genug Zeit, um sich nach jedem Zauber zu erholen. Und so hatte er einfach den Kolloportus auf seinen eigenen verschlossenen Handschuh gewirkt, einfach den Zauber gewirkt, nicht seine ganze Kraft eingesetzt, um ihn stärker zu binden, als Harry Potter oder Hermine Granger es rückgängig machen könnten. Aber das würde niemand glauben, selbst wenn es wahr wäre. Selbst in Slytherin würde das niemand glauben. Es klang wie eine Ausrede, und eine Ausrede war alles, was jemand hören wollte. \emph{Granger wirbelte herum und schrie "ALOHOMORA!"} - Dracos Verstand spielte es immer wieder ab, während sich der Groll aufbaute. \emph{Er hatte Granger geholfen - mit ihr zusammengearbeitet, um Verräter zu verbannen - ihre Hand gehalten, als sie vom Dach baumelte - einen Aufstand um sie herum in der Großen Halle verhindert - hatte sie eine Ahnung, was er riskiert hatte, was er wahrscheinlich schon verloren hatte, was es für den Erben des Hauses Malfoy bedeutete, so etwas} \emph{für ein Schlammblut zu tun} - Und jetzt war nur noch ein Zug übrig, und die Sache mit einem erzwungenen Zug war, dass man ihn machen musste, selbst wenn es bedeutete, nachsitzen zu müssen und Hauspunkte zu verlieren. Professor Snape würde es wissen und verstehen, aber es gab Grenzen (Vater hatte ihn gewarnt), über die der Meister der Zaubertränke hinwegsehen würde.

Granger zu einem Zaubererduell herausfordern, unter offener Missachtung der Hogwarts-Regeln. Sie direkt anzugreifen, wenn sie sich weigern würde.

Sie im Zweikampf besiegen, in aller Öffentlichkeit, nicht durch geschickte Duelltechnik, sondern durch Überwältigung ihrer Magie. Schlag sie solide, vollständig, zerquetsche sie, so wie der Dunkle Lord selbst seine Feinde zerquetscht hat. Mach allen klar, damit niemand daran zweifeln kann, dass Draco einfach nur erschöpft war, weil er den Zauber so oft gewirkt hatte. Beweise, dass das Blut der Malfoys stärker war als das jedes Schlammbluts -

\emph{Nur ist es das nicht}, flüsterte Harry Potters Stimme in Dracos Kopf. \emph{Man vergisst leicht, was wirklich wahr ist, Draco, sobald man versucht, in der Politik zu gewinnen. Aber in Wirklichkeit gibt es nur eine Sache, die dich zu einem Zauberer macht, weißt du noch?}

Draco wusste also, er kannte den Grund für das Unbehagen in seinem Hinterkopf, als er auf die leere Wand über seinem Schreibtisch starrte und über seinen erzwungenen Schritt nachdachte. Es hätte einfach sein sollen - wenn man nur einen Zug hatte, musste man ihn machen - aber - \emph{Granger wirbelte herum, schweißnasse Haare flogen um sie herum, Flüche flogen von ihrem Zauberstab so schnell wie seine eigenen, Zauber und Gegenzauber, glühende Fledermäuse flogen auf sein Gesicht zu, und durch all das hindurch der Blick der Wut in Grangers Augen} - Es gab einen Teil von ihm, der das bewundert hatte, bevor alles schief gegangen war, der Grangers Wut und Kraft bewundert hatte; ein Teil von ihm, der im ersten richtigen Kampf, den er je bestritten hatte, gejubelt hatte, gegen \emph{… einen ebenbürtigen Gegner.}

Wenn er Granger herausforderte und verlor… \emph{Das sollte nicht möglich sein}. Draco hatte seinen Zauberstab 2 volle Jahre vor allen anderen in seiner Hogwarts-Klasse bekommen. Nur gab es einen Grund, warum man Neunjährigen normalerweise keine Zauberstäbe schenkte. Das Alter zählte auch, es ging nicht nur darum, wie lange man einen Zauberstab besessen hatte. Grangers Geburtstag war erst ein paar Tage her, als Harry ihr diesen Beutel gekauft hatte. Das bedeutete, dass sie jetzt zwölf war, dass sie fast seit dem Beginn von Hogwarts zwölf gewesen war. Und die Wahrheit war, dass Draco außerhalb des Unterrichts nicht viel geübt hatte, wahrscheinlich nicht annähernd so viel wie Hermine Granger aus Ravenclaw. Draco hatte nicht gedacht, dass er noch mehr Übung brauchte, um vorne zu bleiben…

\emph{Und Granger war auch erschöpft,} flüsterte die Stimme des Gegenbeweises in ihm. \emph{Granger musste erschöpft sein von all den Betäubungsflüchen, und selbst in diesem Zustand hatte sie seinen Sperrzauber aufheben können.}

Und Draco konnte es sich nicht leisten, Granger öffentlich herauszufordern, eins-gegen-eins, ohne Ausreden, und zu verlieren. Draco wusste, was man in so einer Situation tun musste.

\emph{Man sollte schummeln. Aber wenn jemand entdeckte, dass Draco schummelte, wäre das katastrophal, perfektes Erpressungsmaterial, selbst wenn es nie öffentlich herauskäme, und jeder Slytherin, der zusah, würde das wissen, sie würden hinsehen…}

Und dann, wenn man aufgepasst hätte, hätte man gesehen, wie Draco Malfoy von seinem Bett aufstand, zu seinem Schreibtisch ging und ein Blatt feinstes Schafspergament und ein perlengeschnitztes Tintenfass herausnahm, das mit grünlich-silberner Tinte gefüllt war, die aus echtem Silber und zerstoßenen Smaragden hergestellt worden war. Aus der großen Truhe am Fußende seines Bettes zog der Slytherin ein ebenfalls in Silber und Smaragde gebundenes Buch mit dem Titel \emph{Die Etikette der Häuser von Britannien} hervor. Und mit einem neuen, sauberen Federkiel begann Draco Malfoy zu schreiben, wobei er immer wieder auf das Buch schaute, in dem es aufgeschlagen lag, um nachzuschlagen. Ein grimmiges Lächeln lag auf dem Gesicht des Jungen und ließ den jungen Malfoy seinem Vater sehr ähnlich sehen, während er jeden Buchstaben sorgfältig zeichnete, als wäre er ein eigenes Kunstwerk.

\emph{Von Draco,}

\emph{Sohn von Lucius, Sohn von Abraxis, Herr des edlen und uralten Hauses Malfoy,}

\emph{Sohn auch von Narcissa, Tochter von Druella, Herrin des edlen und uralten Hauses Black, Spross und Erbe des edlen und uralten Hauses Malfoy:}

\emph{An Hermine, die erste Granger:}

(Diese Form mag vor langer Zeit, als sie erfunden wurde höflich klingen; heutzutage, nach Jahrhunderten der Anrede von Schlammblütern, trägt sie einen lieblichen Hauch von raffiniertem Gift in sich.)

\emph{Ich, Draco, aus einem adligen Haus, verlange Wiedergutmachung,}

dann hielt Draco inne und schob die Schreibfeder vorsichtig zur Seite, damit sie nicht tropfte. Er brauchte einen Vorwand, zumindest wenn er die Bedingungen für das Duell durchsetzen wollte. Der Herausgeforderte hatte die Wahl der Bedingungen, es sei denn, er hatte ein Adelshaus beleidigt. Er musste es so aussehen lassen, als hätte Granger ihn beleidigt… \emph{Was dachte er sich?! Granger} \textbf{\emph{hatte}} \emph{ihn beleidigt}.

Draco blätterte im Buch auf die Seite mit den Standardformeln und fand eine, die ihm passend erschien.

\emph{Ich, Draco, aus einem adligen Haus, verlange Wiedergutmachung dafür, dass ich dir dreimal geholfen und dir nur mein Wohlwollen angeboten habe und du mich im Gegenzug fälschlicherweise beschuldigt hast, ein Komplott gegen dich zu schmieden},

jetzt musste Draco innehalten und durchatmen, um die brodelnde Wut zu unterdrücken; er begann jetzt, die Beleidigung wirklich zu spüren, und er hatte den letzten Satz einfach ausgeschrieben und unterstrichen, ohne nachzudenken, als wäre es ein gewöhnlicher Brief. Nach kurzem Nachdenken entschied er sich, es stehen zu lassen; es war vielleicht nicht die exakte formale Formulierung, aber es hatte einen rauen, wütenden Ton, der angemessen schien. welche Beleidigung Sie vor den Augen Britanniens begangen hatte.

\emph{So zwinge ich, Draco, dich, Hermine, durch den Brauch, durch das Gesetz, durch}

"Die siebzehnte Entscheidung des einunddreißigsten Zaubergamots", sagte Draco laut, ohne aufzusehen, eine Zeile, die in vielen Theaterstücken vorgetragen wird; er saß gerader, als er es sagte, und fühlte jeden Puls des edlen Blutes in seinen Adern.

\emph{So zwinge ich, Draco, dich, Hermine, durch den Brauch, durch das Gesetz, durch die 17. Verordnung des 31. Zaubergamot, mich im Zaubererduell mit folgenden Bedingungen zu treffen:}

\emph{Dass jeder von uns allein und in aller Stille kommt, mit niemandem vorher oder nachher spricht,}

Wenn das Duell schlecht ausging, konnte Draco einfach nichts sagen und es dabei belassen. Und wenn er Granger besiegte, hätte er experimentell gelernt, dass er sie in einer öffentlichen Herausforderung wieder schlagen könnte. Es war kein Betrug, aber es war Wissenschaft, was fast genauso gut war. Ein Wettkampf allein durch Magie, ohne Tod oder bleibende Verletzungen, … \emph{wo}? Man hatte Draco von einem Raum in Hogwarts erzählt, der sich gut für Duelle eignete, wo alles Wertvolle bereits durch Zauber geschützt war und es keine Porträts gab, die einen verpetzen konnten … \emph{welcher war es noch mal gewesen} … im Trophäenraum des Schlosses der Hogwarts-Schule für Hexerei und Zauberei, Und ihr zweites und öffentliches Duell musste besser bald stattfinden, wie morgen, es würde nur sehr wenig Zeit brauchen, bis sein Ruf in Slytherin unwiederbringlich zu Nichts verkommen würde. Er musste heute Nacht zum ersten Mal gegen Granger kämpfen. Schlag Mitternacht musste es sein.

Draco, aus dem edlen und uralten Haus der Malfoys. Draco unterschrieb das formale Pergament und holte dann sein normales Pergament und seine normale Tinte für sein Postskriptum hervor:

\emph{Wenn du nicht weißt, wie die Regeln funktionieren, Granger, hier ist es. Du hast ein sehr altes Haus beleidigt, und ich habe das Recht, dich herauszufordern. Und wenn du gegen die Bedingungen des Duells verstößt, etwa indem du Flitwick im Trophäenraum auftauchen lässt, oder auch nur irgendjemandem davon erzählst, wird mein Vater dich und deine falsche Ehre direkt zum Zaubergamot bringen.}

\emph{Draco Malfoy}

Beim letzten Buchstaben drückte seine Feder so heftig auf das Pergament, dass die Feder abbrach und einen Tintenfleck und einen kleinen Riss im Pergament hinterließ, der nach Dracos Meinung auch angemessen aussah.

…

An diesem Abend zur Essenszeit kam Susan Bones zu Harry Potter und erzählte ihm, dass sie glaubte, Draco Malfoy würde seinen Plan gegen Hermine sehr bald ausführen. Sie hatte alle Mitglieder von S.P.H.E.W. gewarnt, und sie hatte Professor Sprout gewarnt, und sie hatte Professor Flitwick gewarnt, und sie wollte heute Abend einen Brief an ihre Tante schicken, und nun warnte sie auch Harry Potter. Nur konnten sie mit Padma nicht so recht darüber reden - sagte Susan und sah dabei sehr ernst aus -, denn Padma fühlte sich hin- und hergerissen zwischen ihrer Loyalität zu Hermine und ihrer Loyalität zu ihrem General.

Harry James Potter-Evans-Verres, der sich zu diesem Zeitpunkt mehr von der ganzen Situation frustriert fühlte als von etwas wirklich Produktivem, schnauzte sie an, dass ja, er wisse, dass etwas getan werden müsse.

Nachdem Susan Bones gegangen war, schaute Harry zum anderen Ende des Ravenclaw-Tisches hinüber, wo Hermine sich abseits von ihm oder Padma oder Anthony oder einem ihrer anderen Freunde hingesetzt hatte. Aber Hermine sah nicht so aus, als wäre sie in einer Stimmung, in der es gut ankäme, wenn jemand zu ihr rübergehen und sie stören würde.

Später, wenn er zurückblickte, dachte Harry daran, wie in seinen SF- und Fantasy-Romanen die Menschen ihre großen, wichtigen Entscheidungen immer aus großen, wichtigen Gründen trafen. Hari Seldon hatte seine Stiftung gegründet, um die Asche des Galaktischen Imperiums wieder aufzubauen, nicht weil er wichtiger aussehen würde, wenn er seine eigene Forschungsgruppe leiten könnte. Raistlin Majere hatte sich von seinem Bruder getrennt, weil er ein Gott werden wollte, nicht weil er in persönlichen Beziehungen inkompetent und nicht bereit war, um Rat zu fragen, wie er es besser machen könnte. Frodo Beutlin hatte den Ring genommen, weil er ein Held war, der Mittelerde retten wollte, und nicht, weil es zu peinlich gewesen wäre, es nicht zu tun. Wenn irgendjemand jemals eine wahre Geschichte der Welt schreiben würde - nicht, dass das jemals jemand könnte oder würde -, würden sich wahrscheinlich 97\% aller Schlüsselmomente des Schicksals als aus Lügen und Seidenpapier und trivialen kleinen Gedanken konstruiert herausstellen, die jemand anderes genauso gut hätte denken können.

Harry James Potter-Evans-Verres schaute zu Hermine Granger, die sich am anderen Ende des Tisches hingesetzt hatte, und verspürte einen gewissen Widerwillen, sie zu stören, wenn sie so aussah, als hätte sie bereits schlechte Laune. Dann dachte Harry, dass es wahrscheinlich sinnvoller war, zuerst mit Draco Malfoy zu sprechen, nur damit er Hermine absolut sicher versichern konnte, dass Draco wirklich keine Verschwörung gegen sie anzettelte. Und später nach dem Abendessen, als Harry in den Slytherin-Keller hinunterging und ihm von Vincent gesagt wurde, dass der Boss nicht gestört werden dürfe… da dachte Harry, dass er vielleicht schauen sollte, ob Hermine gleich mit ihm reden würde. Dass er einfach anfangen sollte, den ganzen Schlamassel zu entwirren, bevor er sich noch weiter ausweitet. Harry fragte sich, ob er vielleicht nur prokrastinierte, ob sein Verstand nur eine clevere Ausrede gefunden hatte, um etwas Unangenehmes, aber Notwendiges aufzuschieben. Das dachte er tatsächlich. Und dann beschloss Harry James Potter-Evans-Verres, dass er stattdessen einfach am nächsten Morgen, nach dem Sonntagsfrühstück, mit Draco Malfoy reden würde und dann mit Hermine.

\emph{Menschen machten so etwas andauernd.}

….

Es war Sonntagmorgen, am 5. April 1992, und der simulierte Himmel über der Großen Halle von Hogwarts zeigte große Regenströme, die in einer solchen Dichte niedergingen, dass die Blitze abgeschwächt wurden und sich in kleine Impulse weißen Lichts auflösten, die manchmal die Tische der Häuser beschienen, ihre Gesichter bleich machten und alle Schüler kurzzeitig wie Geister erscheinen ließen.

Harry saß am Ravenclaw-Tisch, aß müde eine Waffel und wartete darauf, dass Draco auftauchte, damit er anfangen konnte, diese ganze Sache zu klären. Es wurde ein Klitterer herumgereicht, der irgendwie mit Hannah und Daphne auf der ersten Seite gelandet war, aber er war noch nicht bei ihm angekommen. Ein paar Minuten später aß Harry seine Waffel zu Ende und sah sich dann noch einmal um, um zu sehen, ob Draco schon zum Frühstück am Slytherin-Tisch erschienen war.

\emph{Es war merkwürdig. Draco Malfoy war fast nie zu spät.}

Da Harry in die Richtung des Slytherin-Tisches schaute, sah er nicht, wie Hermine Granger durch die großen Türen der Großen Halle eintrat. Daher war er ziemlich erschrocken, als er sich umdrehte und Hermine entdeckte, die sich direkt neben ihn an den Ravenclaw-Tisch setzte, als hätte sie das seit über einer Woche nicht mehr getan.

"Hi, Harry", sagte Hermine, ihre Stimme klang fast ganz normal. Sie begann, Toast auf ihren Teller zu legen und eine Auswahl an gesundem Obst und Gemüse.

"Wie geht es dir?"

"Normal innerhalb einer Standardabweichung meines eigenen merkwürdigen kleinen Durchschnitts", antwortete Harry automatisch. "Und wie geht es dir? Hast du gut geschlafen?"

Unter Hermine Grangers Augen waren dunkle Säcke zu sehen.

"Aber ja, mir geht es gut", sagte Hermine Granger.

"Ähm", sagte Harry.

Er nahm ein Stück Kuchen auf seinen Teller (da sein Gehirn mit anderen Dingen beschäftigt war, nahm Harrys Hand einfach das Leckerste in Reichweite, ohne komplexe Konzepte zu bewerten, wie zum Beispiel, ob er bereit war, einen Nachtisch zu essen).

"Ähm, Hermine, ich muss heute später mit dir reden, ist das okay?"

"Sicher", sagte Hermine. "Warum sollte es das nicht sein?"

"Weil -" sagte Harry. "Ich meine - du und ich haben - in den letzten paar Tagen -"

\emph{Halt die Klappe}, schlug ein innerer Teil von Harry vor, der anscheinend seit kurzem für die Regelung von Hermine-bezogenen Themen zuständig war.

Hermine Granger sah jedenfalls nicht so aus, als ob sie ihm viel Aufmerksamkeit schenken würde. Sie starrte nur auf ihren Teller hinunter und begann dann, nach etwa zehn Sekunden peinlichen Schweigens, ihre Tomatenscheiben zu essen, eine nach der anderen, ohne Pause. Harry wandte den Blick von ihr ab und begann, ein Stück Kuchen zu essen, das sich, wie er feststellte, irgendwie auf seinem Teller materialisiert hatte.

"So!" sagte Hermine Granger plötzlich, nachdem sie den größten Teil ihres Tellers schweigend verputzt hatte. "Ist heute irgendetwas passiert?"

"Ähm …" sagte Harry. Er schaute sich hektisch um, als ob er etwas finden wollte, das er als Gesprächsstoff verwenden konnte. Und so war Harry einer der ersten, der es sah und wortlos darauf hinwies, obwohl die plötzliche Woge von Geflüster, die durch die Große Halle schwappte, zeigte, dass auch eine Reihe anderer Leute es gesehen hatten.

Die unverwechselbare karmesinrote Färbung der Roben wäre überall zu erkennen gewesen, aber Harrys Gehirn brauchte trotzdem ein paar Augenblicke, um die Gesichter zuzuordnen.

Ein asiatisch aussehender Mann, feierlich und heute eher grimmig dreinblickend. Ein Mann mit einem durchdringenden Blick, der über den Raum schweifte, sein langes schwarzes Haar wehte hinter ihm in einem Pferdeschwanz. Ein Mann, dünn und blass und unrasiert, mit einem Gesicht, das so leer war, dass es wie Stein wirkte. Harry brauchte einige Augenblicke, um die Gesichter zuzuordnen und sich an die Namen zu erinnern, von jenem lang zurückliegenden Tag im Januar, als der Dementor nach Hogwarts gekommen war: Komodo, Butnaru, Goryanof.

"Ein Auroren-Trio?" sagte Hermine mit einer seltsam hellen Stimme.

"Ich frage mich, was die wohl hier wollen."

Dumbledore befand sich ebenfalls in ihrer Gesellschaft und sah so besorgt aus, wie Harry ihn noch nie gesehen hatte; und nach einer kurzen Pause, in der die Augen des alten Zauberers die Große Halle und die über ihrem Frühstück flüsternden Schüler abtasteten, deutete er - - \emph{direkt auf Harry}.

"Oh, was jetzt", sagte Harry leise. Seine inneren Gedanken waren viel panischer als das, denn er fragte sich verzweifelt, ob ihn jemand irgendwie mit dem Ausbruch aus Askaban in Verbindung gebracht hatte. Er schaute auf den Lehrertisch, versuchte, den Blick beiläufig zu machen, und stellte fest, dass Professor Quirrell an diesem Morgen nirgends zu sehen war -

die Auroren kamen mit schnellen Schritten auf ihn zu, Auror Goryanof näherte sich von der anderen Seite des Ravenclaw-Tisches, als ob er jede Flucht in diese Richtung verhindern wollte, Auror Komodo und Auror Butnaru näherten sich von Harrys Seite, der Schulleiter folgte Komodo direkt auf den Fersen. Alle Unterhaltungen waren völlig verstummt.

Die Auroren erreichten Harrys Platz am Tisch und umringten ihn aus drei Richtungen.

"Ja?" sagte Harry so normal, wie er konnte. "Was gibt es?"

"Hermine Granger", sagte Auror Komodo mit tonloser Stimme, "Sie sind verhaftet wegen des versuchten Mordes an Draco Malfoy."

