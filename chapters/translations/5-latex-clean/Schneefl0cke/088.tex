

\hypertarget{zeitdruck-teil-2}{% \section{89. Zeitdruck, Teil 2}\label{zeitdruck-teil-2}}

\textbf{\uline{Zeitdruck, Teil 2}}

Kühle blaue Feuer klammerten sich in kleinen Massen an den Boden und umgaben ein loderndes Becken, das mit einem tödlicheren, heißeren Blau zu brennen schien. In einem engen Kreis waren die Marmorfliesen versengt und zersplittert von einem explosiven Zauber, den nur die mächtigste Hexe im ersten Jahr mit dem letzten Rest ihrer Kraft hätte wirken können. Auf der Terrasse, die sich immer noch im offenen Sonnenlicht bewegte, stand eine große, klumpige Kreatur von mattem Granitgrau.

Der Körper glich einem Felsbrocken, auf dem ein kleiner kahler Kopf wie ein Stein thronte, die kurzen Beine waren dick wie Baumstämme mit flachen, hornigen Füßen. In der einen Hand hielt er eine gewaltige steinerne Keule, so lang und breit wie ein erwachsener Mensch, und in der anderen Hand hielt er—

\emph{Die Weasley-Zwillinge schrien. Harrys Patronus zerbrach.}

Der Troll schnaubte und drehte sich zu ihnen um, ließ etwas in die rote Lache fallen, die sich unter seinen Füßen ausgebreitet hatte, und hob seine Keule hoch.

Dann rief ein Weasley eine Beschwörung und die Keule wurde dem Troll aus der Hand gerissen, schlug ihm so hart ins Gesicht, dass er einen Schritt zurückwich, ein Schlag, der einen Muggel hätte töten können. Der Troll brüllte vor Wut, seine Nase war zerquetscht und blutbespritzt, und dann richtete sich die Nase wieder gerade, regenerierte sich. Der Troll griff mit beiden Händen nach der Keule, die durch die Luft schoss, aber nur knapp dem Griff auswich.

„Lockt es weg, haltet es mir vom Leib“, sagte eine Stimme.

Die schwebende Keule bewegte sich rückwärts vom Troll, von der Terrasse auf den weiten Boden unter der Decke; und der Troll machte einen großen, gewaltigen Sprung, der die Keule fast in seine Hände brachte. Dann machte der Troll einen weiteren großen Sprung, als sich die Keule zur Seite bewegte; und der Besen bewegte sich nach vorne, und Harry sprang ab und rannte auf die Stelle zu, wo Hermine Granger in einer Lache ihres eigenen Blutes lag und ihre Beine bis zu den Oberschenkeln weggefressen waren.

Harrys Hände rissen das Heilerkit aus der Tasche, schnappten sich einen der selbstspannenden Tourniquets, wickelten sie um einen zerfetzten, befleckten Stumpf, seine Hände glitten kurz in das Blut, sie zitterten nicht, \emph{es war nicht erlaubt, dass seine Hände zitterten}. Als die Aderpresse eine vollständige Schlinge bildete, zog sie sich fest zusammen, und es trat noch mehr Blut aus, aber dann stoppte die Blutung an diesem Schenkelstumpf, und Harry wandte sich dem anderen zu. Ein Teil seines Verstandes \emph{schrie, schrie, schrie} und sogar der Teil von ihm, der die andere, sich selbst festziehende Aderpresse aufhob, hörte es, \emph{aber auch das war nicht erlaubt.}

Die beiden Weasley-Zwillinge riefen Zaubersprüche, einen nach dem anderen in einem Schnellfeuer, das Harry in sechzig Sekunden bewusstlos gemacht hätte, manchmal riefen die Zwillinge zwei Zaubersprüche gleichzeitig in perfekter Koordination, aber die meisten der Zauber lösten sich in harmlosen Funkenschauern auf der Haut des Trolls auf. Als sich die andere Aderpresse in einem weiteren Blutstoß zusammenzog, blickte Harry zu einem „Diffindo!“ / „Reducto!“, das die verletzlichen Augen des Trolls in einem doppelten Schauer glasigen Schleims explodieren ließ, aber der Troll brüllte nur noch einmal, seine Augen formten sich bereits neu.

„Feuer und Säure!“ schrie Harry. „Benutzt Feuer oder Säure!“

„Fuego!“ / „Incendio!“

Harry hörte es, aber er sah nicht hin, er griff nach der Spritze mit der leuchtend orangefarbenen Flüssigkeit, die der sauerstoffhaltige Trank war, und stieß sie in Hermines Hals, an der Stelle, von der Harry hoffte, dass es die Halsschlagader war, um ihr Gehirn am Leben zu erhalten, selbst wenn ihre Lungen oder ihr Herz versagten, \emph{Solange ihr Gehirn intakt blieb, konnte alles andere repariert werden, es musste möglich sein, dass Magie es reparierte, es musste möglich sein, dass Magie es reparierte, es musste möglich sein, dass Magie es reparierte}, und Harry drückte den Kolben der Spritze ganz nach unten, wodurch ein schwaches Glühen unter der blassen Haut ihres Halses entstand. Harry drückte dann auf ihren Brustkorb, wo ihr Herz sein sollte, harte Kompressionen, von denen er hoffte, dass sie das sauerstoffreiche Blut dorthin brachten, wo es ihr Gehirn erreichen konnte, auch wenn ihr Herz vielleicht aufgehört hatte zu schlagen, er hatte nicht daran gedacht, ihren Puls zu überprüfen.

Dann starrte Harry auf die anderen Dinge in seinem Verbandskasten, sein Verstand wurde leer, als er versuchte, herauszufinden, was von dem, was dort war, wenn überhaupt etwas, er gebrauchen konnte.

Das Schreien in dieser entfernten Ecke seines Geistes wurde lauter, viel lauter, jetzt, da seine Hände ihre hektischen Bewegungen eingestellt hatten. Er war sich plötzlich des flüssigen Gefühls bewusst, wo das Blut seine Robe und die Knie seiner Hose durchtränkt hatte. Hinter Harry ertönte ein weiteres Brüllen des Trolls, und er hörte, wie einer der Weasley-Zwillinge „Deligitor prodeas!“ rief und dann „HILFE! Tu etwas!“

Harry drehte den Kopf zurück, um nachzusehen, und sah, dass einer der Weasley-Zwillinge jetzt irgendwie den Sprechenden Hut auf dem Kopf trug und dem Troll gegenüberstand, der die riesige steinerne Keule in beiden Händen hielt, die jetzt etwas verbrannt aussah und eine oder zwei rauchende Narben über den Armen hatte, aber immer noch intakt war.

Und dann brüllte die Stimme des Hutes mit einer so lauten Stimme, dass sie die Wände zu erschüttern schien: „\textbf{GRYFFINDOR}!“

Ein Puls der Macht verbrannte die Luft, die Magie fühlte sich selbst für Harrys junge Sinne fast greifbar an, der Troll sprang mit einem Schnauben der Überraschung einen Schritt zurück. Fred oder George, mit einem seltsamen Gesichtsausdruck, fegte den Hut mit einer Bewegung vom Kopf, die so geschmeidig war wie ein Zaubertrick, und griff mit einer Hand hinein und zog einen Griff hervor, dessen Knauf ein glühender Rubin war, gefolgt von einer breiten Parierstange aus glänzendem weißen Metall und einer Klinge, die so lang war wie ein großes Kind. Als das Schwert enthüllt wurde, schien sich die Luft mit einem stummen Schrei der Wut zu füllen. Auf der Klinge stand in goldener Schrift geschrieben: \textbf{\emph{nihil supernum}}.

Dann hob der Weasley-Zwilling das Schwert in die Höhe, als würde die riesige Klinge nichts wiegen, und stürmte mit einem Schrei auf den Troll zu. Harrys Lippen öffneten sich, um etwas zu sagen, irgendeinen langen Satz wie: \emph{Nein, hör auf, du hast keine Ahnung, wie man ein Schwert benutzt,} aber nicht einmal eine einzige Silbe verließ seine Lippen, bevor das Schwert den rechten Arm des Trolls am Ellbogen abtrennte und Haut und Fleisch und Knochen wie Gelee durchtrennte; gerade als der bereits schwingende Bogen der steinernen Keule in den angreifenden Weasley-Zwilling einschlug und ihn durch die Luft über den Marmorboden fliegen ließ, über die Lücke, aus der sie auf dem Besenstiel gestiegen waren, bis dieser Weasley auf der gegenüberliegenden Seite gegen die Wand prallte und dann zu einem unbeweglichen Haufen zusammenbrach. Das Schwert verschwand in der Öffnung im Boden und klapperte weithin hörbar, als es fiel.

„Fred!“, schrie George Weasley, und dann „VENTUS!“

Ein unsichtbarer Schlag erwischte den Troll und schleuderte ihn seitwärts durch die Luft.

„VENTUS!“

Der Troll wurde erneut getroffen und an den Rand der Platform und den nach unten führenden Spalt geschleudert.

„VENTUS!“

Aber der Troll hatte nach unten gegriffen und griff nach dem Boden, seine verbliebene Hand drückte durch den Marmor, um einen festen Halt zu finden. Der dritte Schlag schickte den Körper des Trolls über den Spalt; aber die Hand blieb an der Kante. Und dann zog sich der Troll eigenhändig und brüllend wieder nach oben.

George Weasley taumelte, fiel fast hin, seine Hand fiel auf seine Seite. „Harry—“ , sagte der Weasley-Zwilling mit angestrengter Stimme, „Lauf—“

Der verbliebene Weasley-Zwilling machte einen Schritt zur Seite, sackte gegen die Wand und glitt zu Boden.

Die Zeit war in Harrys Kopf zerbrochen, die Welt um ihn herum schien sich langsam zu bewegen, verzerrt, oder vielleicht war es sein eigener Verstand, der sich verdrehte und faltete. Er hätte sich bewegen, etwas tun sollen, aber eine seltsame Lähmung schien alle seine Muskeln, alle seine Bewegungen zu stoppen. Ohne Zeit für Worte kamen die Gedanken in Form von Gedankenblitzen: dass, wenn Harry weglief, der Troll sowohl die Weasley-Zwillinge als auch Hermine fressen würde, dass, wenn Klatscher keine Zauberer töteten, Fred noch am Leben sein müsste, dass die Weasley-Zwillinge mächtigere Zauberer waren als er und sie den Troll nicht hatten aufhalten können, dass er keine Zeit hatte, etwas zu verwandeln, was er nicht schon besaß, dass der Troll zu agil schien, um über den Rand der Terrasse gelockt zu werden und von den Seiten des Hogwarts-Schlosses zu stürzen, dass jemand den Troll gegen Sonnenlicht verzaubert hatte, bevor er ihn als Mordwaffe einsetzte, und ihn vielleicht auch auf andere Weise gestärkt hatte.

Und dann ein geistiges Bild von Hermine, wie sie vor dem Troll wegrannte, wie sie dem Sonnenlicht nachrannte, wie sie schließlich die helle Terrasse erreichte, mit dem Troll auf den Fersen, nur um festzustellen, dass jemand anderes auch an diese Möglichkeit gedacht hatte.

\emph{Der schreiende Horror in seinem Kopf wurde von einer anderen Emotion übertönt.}

Harry stand auf. Auf der anderen Seite des Raumes hatte sich \emph{der Feind} ebenfalls erhoben, der nicht regenerierende Stumpf eines mit dem Schwert abgetrennten Arms immer noch blutig.

\textbf{\emph{Fokus auf töten}}

Der Troll ergriff seine gefallene Keule in der verbliebenen Hand und stieß einen gewaltigen Schrei aus, der die Keule auf den Boden schlug und Marmorsplitter umherfliegen ließ. Der Troll begann, auf die Stelle zuzutrampeln, an der George gefallen war, und ein dünner Sabberfaden rann ihm von den Lippen.

Harry machte fünf Schritte nach vorne, und \emph{der Feind} brüllte erneut und wandte sich von George ab, wobei seine Augen direkt auf ihn gerichtet waren.

\textbf{\emph{TÖTEN}}

Harrys linke Hand hielt bereits den verwandelten Diamanten aus seinem Ring, seine rechte Hand hielt bereits seinen Zauberstab.

„Wingardium Leviosa.“

Harrys Zauberstab lenkte das winzige Juwel in das Maul des Trolls.

„Finite Incantatem.“

Der Kopf des Trolls platzte von der Wirbelsäule, als sich der Felsen wieder zu seiner alten Form ausdehnte, und Harry trat zur Seite, als der Körper des Feindes dort zusammenbrach, wo er gestanden hatte. Der Kopf des Feindes begann sich bereits zu regenerieren, der zerfetzte Stumpf des Kiefers und der Wirbelsäule glättete sich, der Mund vervollständigte sich und ersetzte die Zähne.

Harry bückte sich und hob den Kopf des Trolls an seinem linken Ohr auf. Sein Zauberstab stach durch das linke Auge des Trolls, tauchte durch das geleeartige Material und ging durch die breite Augenhöhle im Knochen. Harry visualisierte einen ein Millimeter breiten Querschnitt durch das Gehirn des Feindes und verwandelte es zu Schwefelsäure. Der Feind hörte auf, sich zu regenerieren.

Harry warf die Leiche über den Rand der Terrasse und drehte sich wieder zu Hermine um. Ihre Augen bewegten sich und waren auf ihn gerichtet. Harry kletterte neben ihr hinunter und ignorierte das Blut, das noch mehr von seinen bereits durchnässten Roben durchtränkte.

\emph{Du wirst schon wieder}, formte sein Gehirn den Satz, aber seine Lippen wollten sich nicht bewegen. \emph{Du wirst wieder gesund, wir werden irgendeinen Zauber finden, der das alles wieder in Ordnung bringt, dich wieder normal werden lässt, halt einfach durch, nicht—}

Hermines Lippen bewegten sich, nur ein winziges bisschen, aber sie bewegten sich.

„Deine… Schuld…“ Die Zeit stand still. Harry hätte ihr sagen sollen, sie solle nicht reden, um sich den Atem zu sparen, nur er konnte seine Lippen nicht lösen.

Hermine holte noch einmal tief Luft, und ihre Lippen flüsterten:

„Nicht deine Schuld.“

Dann atmete sie aus und schloss die Augen.

Harry starrte sie mit halb geöffnetem Mund an, der Atem blieb ihm im Hals stecken. „\emph{Bitte nicht…}“, sagte seine Stimme.

\emph{Er war nur zwei Minuten zu spät gekommen.}

Hermine zuckte plötzlich zusammen, ihre Arme zuckten in die Luft, als würden sie nach etwas greifen, und ihre Augen flogen wieder auf. Es gab einen Ausbruch von etwas, das Magie und noch mehr war, ein Schrei, lauter als ein Erdbeben und mit tausend Büchern, tausend Bibliotheken, alles gesprochen in einem einzigen Schrei, der Hermine war; zu gewaltig, um verstanden zu werden, außer dass Harry plötzlich wusste, dass Hermine den Schmerz ausgeblendet hatte, und froh war, nicht allein zu sterben. Für einen Moment schien es, als würde der Ausfluss der Magie anhalten, sich im Stein des Schlosses verwurzeln; aber dann endete der Ausfluss und die Magie verblasste, ihr Körper hörte auf, sich zu bewegen, und alle Bewegungen kamen zum Stillstand, als Hermine Jean Granger aufhörte zu existieren—

\textbf{\emph{Nein}}.

Harry stand von dem Körper auf und schwankte.

\textbf{\emph{Nein}}.

Es gab eine Flammenexplosion und Dumbledore stand mit Fawkes da, seine Augen waren voller Entsetzen. „Ich habe einen Schüler sterben fühlen! Was—“

Die Augen des alten Zauberers sahen, was auf dem Boden lag.

„Oh, nein“, flüsterte Albus Dumbledore.

Fawkes gab ein trauriges, klagendes Krächzen von sich.

„Bring sie zurück.“

Es herrschte Stille auf der Terrasse. Fred Weasley hatte sich auf eine Geste von Dumbledores Zauberstab hin in die Luft erhoben und schwebte auf sie zu, umgeben von einem beruhigenden rosa Schein.

„Harry—“ , begann der alte Zauberer. Seine Stimme überschlug sich. „Harry—“

„Fawkes soll sich bei ihr ausweinen oder so. Beeil dich.“

Die Stimme, die sprach, klang vollkommen ruhig.

„Ich - ich kann nicht, Harry, es ist zu spät, sie ist tot—“

„Ich will nichts davon hören. Wenn ich da liegen würde, würdest du irgendein tolles Kaninchen aus dem Hut ziehen und mich retten, \emph{richtig}, denn der Held darf nicht sterben, bevor die Geschichte zu Ende ist. Sie ist auch der Held, also was auch immer du dir für diesen besonderen Anlass aufgespart hast, mach weiter und benutze es jetzt. Ich verspreche, ich zahle es dir zurück.“

„Es gibt nichts, was ich tun kann! Ihre Seele ist von uns gegangen, sie ist von uns gegangen!“

Harry öffnete den Mund, um seine ganze Wut herauszuschreien, und schloss ihn dann wieder. \emph{Es hatte keinen Sinn, zu schreien, es würde nichts bewirke}n. Der unerträgliche Druck, der in ihm aufstieg, konnte auf diese Weise nicht herausgelassen werden. Harry wandte sich von Dumbledore ab und sah auf die Überreste von Hermine Granger hinunter, die in einer Blutlache lagen. Ein Teil seines Verstandes hämmerte auf die Welt um ihn herum ein, versuchte, sie verschwinden zu lassen, aus dem Albtraum aufzuwachen und sich wieder in seinem Ravenclaw-Schlafsaal zu befinden, wo die Morgensonne durch die Vorhänge schien. Aber das Blut blieb und Harry wachte nicht auf, und ein anderer Teil von ihm wusste bereits, dass dieses Ereignis real war, Teil derselben fehlerhaften Welt, zu der Askaban und die Kammer des Zaubergamot gehörten und \textbf{\emph{Nein}}.

Mit einem zerbrechenden Gefühl, als ob die Zeit um ihn herum immer noch in Stücke gerissen wäre, wandte Harry sich von Dumbledore ab und sah auf die Überreste von Hermine Granger hinunter, die in einer Blutlache lag, mit zwei Tourniquets um ihre Oberschenkelstümpfe gebunden, und beschloss…

\textbf{\emph{Nein. Ich akzeptiere das nicht. Es gibt keinen Grund, es zu akzeptieren, nicht, wenn es Magie in der Welt gibt.}}

\textbf{\emph{Harry würde lernen, was immer er lernen musste, erfinden, was immer er erfinden musste, das Wissen von Salazar Slytherin aus dem Geist des Dunklen Lords reißen, das Geheimnis von Atlantis entdecken, jedes Tor öffnen und jedes Siegel brechen, das nötig war, seinen Weg zur Wurzel aller Magie finden und sie neu programmieren. Er würde die Grundlagen der Realität und alles brechen, um Hermine Granger zurückzubekommen.}}

„Die Krise ist vorbei“, sagte der Verteidigungsprofessor. „Sie können absteigen, Madam.“

Trelawney, die hinter ihm auf dem Zwei-Personen-Besen gesessen hatte, der gerade durch Hogwarts gerast war und dabei alle Wände und Böden in ihrem Weg verbrannt hatte, stieg hastig ab und setzte sich dann hart auf den Boden, einen Schritt entfernt von den rotglühenden Rändern einer neu entstandenen Lücke in der Wand. Die Frau atmete immer noch keuchend und beugte sich über sich selbst, als wäre sie kurz davor, etwas Größeres als sie selbst auszukotzen.

Der Verteidigungsprofessor hatte das Entsetzen des Jungen gespürt, durch die Verbindung, die zwischen ihnen beiden bestand, die Resonanz in ihrer Magie; und er hatte erkannt, dass der Junge den Troll gesucht und gefunden hatte. Der Verteidigungsprofessor hatte versucht, einen Impuls zu senden, sich zurückzuziehen, den Tarnumhang anzulegen und zu fliehen; aber er war nie in der Lage gewesen, den Jungen durch die Resonanz zu beeinflussen, und es war ihm auch dieses Mal nicht gelungen. Er hatte gespürt, wie der Junge sich ganz der Tötungsabsicht hingab. Das war der Zeitpunkt, an dem der Verteidigungsprofessor begonnen hatte, sich durch die Substanz von Hogwarts zu brennen und zu versuchen, den Kampf rechtzeitig zu erreichen.

Er hatte gespürt, wie der Junge seinen Feind in Sekundenschnelle auslöschte.

Er hatte die Bestürzung des Jungen gespürt, als einer seiner Freunde starb.

Er spürte die Wut, die der Junge auf ein Ärgernis gerichtet hatte, die wahrscheinlich Dumbledore war, gefolgt von einer unbekannten Entschlossenheit dessen unnachgiebige Härte selbst er als angemessen empfand. Mit etwas Glück hatte der Junge soeben seinen törichten kleinen Widerwillen abgelegt.

Unbemerkt von allen, verzogen sich die Lippen des Verteidigungsprofessors zu einem dünnen Lächeln.

\emph{Trotz seiner kleinen Höhen und Tiefen war dies im Großen und Ganzen ein erstaunlich guter Tag gewesen—}

\textbf{\hfill\break "ER IST HIER!}

\textbf{ER, DER SELBST DIE STERNE AM HIMMEL ZERREISSEN WIRD!}

\textbf{ER IST HIER!}

\textbf{ER IST DAS ENDE DER WELT!"}

