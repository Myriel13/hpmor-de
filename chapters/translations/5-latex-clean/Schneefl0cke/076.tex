

\hypertarget{oberfluxe4chliche-erscheinungen}{% \section{77. Oberflächliche Erscheinungen}\label{oberfluxe4chliche-erscheinungen}}

\textbf{\uline{Oberflächliche Erscheinungen}}

\textbf{Nachspiel}: \textbf{Albus Dumbledore und -}\\ Der alte Zauberer saß allein an seinem Schreibtisch, in der Stille des Schulleiterbüros, inmitten der unzähligen und unbemerkten Geräte; seine Robe war von einem sanften Gelb, aus weichem Stoff, nicht solche Kleidung, wie er sie gewöhnlich vor anderen trug. In seiner faltigen Hand hielt er einen Federkiel, mit dem er auf einem offiziell aussehenden Pergament herumkratzte. Wenn man irgendwie dabei gewesen wäre, um sein gesäumtes Gesicht zu beobachten, hätte man nicht mehr über den Mann selbst herausfinden können, als man von den rätselhaften Geräten versteht. Man hätte vielleicht bemerkt, dass das Gesicht ein wenig traurig aussah, ein wenig müde, aber so sah Albus Dumbledore immer aus, wenn er allein war.

Im Kamin war nur noch verstreute Asche ohne einen Hauch von Flamme, eine magische Tür, die so fest verschlossen worden war, dass sie nicht mehr existierte. Auf der materiellen Ebene war die große Eichentür zum Büro geschlossen und verriegelt worden; jenseits dieser Tür blieb die Endlose Treppe unbeweglich; am Fuß dieser Treppe flossen die Wasserspeier, die den Eingang versperrten nicht, ihr Pseudoleben hatte sich zurückgezogen, um festen Fels zu hinterlassen.

Dann, gerade als die Feder ein Wort schreiben wollte, gerade als sie einen Buchstaben kratzen wollte - schoss der alte Zauberer mit einer Geschwindigkeit auf die Beine, die jeden Beobachter schockiert hätte, und ließ die Feder mitten im Schreiben auf das Pergament fallen; wie ein Blitz drehte er sich auf der Eichentür, seine gelben Roben wirbelten um ihn herum und ein Zauberstab von furchtbarer Macht sprang in seine Hand -

und ebenso abrupt hielt der alte Zauberer inne, hielt seine Bewegung an, als der Zauberstab in Kampfhaltung war. Eine Hand schlug gegen die Eichentür, klopfte dreimal.

Langsamer, jetzt, ging der grimmige Zauberstab zurück in das Duellierholster, das unter dem Ärmel des alten Zauberers steckte. Der alte Mann trat ein paar Schritte vor, richtete sich in eine förmlichere Haltung auf und verzog das Gesicht. In der Nähe auf dem Schreibtisch bewegte sich der Federkiel zur Seite des Pergaments, als wäre er dort sorgfältig platziert worden und nicht in Eile fallen gelassen worden; und das Pergament selbst klappte um und zeigte Leere. Mit einem leisen Zucken seines Willens schwang die Eichentür auf.

Hart wie Stein starrten ihn die grünen Augen an.

"Ich gebe zu, dass ich beeindruckt bin, Harry", sagte der alte Zauberer leise.\\ "Der Unsichtbarkeitsmantel hätte dich meinen schwächeren Sichtmitteln entgehen lassen; aber ich habe weder gespürt, wie meine Golems zur Seite traten, noch wie sich die Treppe drehte. Wie bist du hierher gekommen?"

Der Junge ging in das Büro, einen Schritt nach dem anderen, bis sich die Tür sanft hinter ihm schloss.

"Ich kann gehen, wohin ich will, mit oder ohne Erlaubnis", sagte der Junge. Seine Stimme wirkte ruhig; zu ruhig vielleicht. "Ich bin in Ihrem Büro, weil ich beschlossen habe, hier zu sein, und zum Teufel mit Passwörtern. Sie irren sich gewaltig, Schulleiter Dumbledore, wenn Sie glauben, dass ich in dieser Schule bleibe, weil ich hier ein Gefangener bin. Ich habe mich einfach noch nicht entschieden, zu gehen. Wenn Sie das bedenken, warum haben Sie dann Ihrem Agenten, Professor Snape, befohlen, die Vereinbarung zu brechen, die wir in diesem Büro getroffen haben, dass er keine Schüler im vierten Jahr oder darunter quälen darf?"

Der alte Zauberer sah den wütenden jungen Helden einen langen Moment lang an. Dann, langsam genug, um den Jungen nicht zu erschrecken, zogen diese verhutzelten Finger eine der vielen Schubladen des Schreibtisches auf, hoben ein Blatt Pergament heraus und legten es auf den Schreibtisch.

"Vierzehn", sagte der alte Zauberer. "Das ist nicht die Zahl aller Eulen, die letzte Nacht geschickt wurden. Nur die Eulen, die an Familien mit einem Sitz im Zaubergamot geschickt wurden, oder Familien mit großem Reichtum, oder Familien, die bereits mit unseren Feinden verbündet sind. Oder, im Fall von Robert Jugson, alles drei; denn sein Vater, Lord Jugson, ist ein Todesser und sein Großvater ein Todesser, der durch Alastor Moodys Zauberstab starb. Was in den Briefen stand, weiß ich nicht, aber ich kann es mir denken. Hast du es immer noch nicht verstanden, Harry Potter? Jedes Mal, wenn Hermine Granger gewonnen hat, wie du sagst, ist die Gefahr für sie aus Slytherin wieder gewachsen, und noch einmal. Aber jetzt haben die Slytherins über sie triumphiert, leicht und sicher, ohne Gewalt oder bleibenden Schaden. Sie haben gewonnen und brauchen nicht mehr zu kämpfen…"\\ Der alte Zauberer seufzte.\\ "So hatte ich es geplant. So hatte ich es gehofft. So wäre es auch gewesen, wenn der Verteidigungsprofessor sich nicht eingemischt hätte. Jetzt geht der Streit vor den Obersten Rat, wo Severus den Verteidigungsprofessor scheinbar besiegen wird; aber das wird sich für die Slytherins nicht so anfühlen, es wird nicht in einem Augenblick zu ihrer Zufriedenheit vorbei und beendet gewesen sein."

Der Junge trat weiter in den Raum vor, sein Kopf neigte sich weiter nach hinten, um zu den halbmondförmigen Gläsern hinaufzuschauen; und irgendwie war es, als würde der Junge eher auf den Schulleiter hinunterschauen als hinauf.

"Dieser Lord Jugson ist also ein Todesser?", fragte der Junge leise.\\ "Gut. Dann ist sein Leben bereits gekauft und bezahlt, und ich kann mit ihm machen, was ich will, ohne ethische Probleme -"

"Harry!"

Die Stimme des Jungen war klar wie Eis, gefroren wie reinstes Wasser aus einer unberührten Quelle.\\ "Sie scheinen zu denken, dass das Licht in Angst vor der Dunkelheit leben sollte. Ich sage, es sollte andersherum sein. Ich würde es vorziehen, diesen Lord Jugson nicht zu töten, selbst wenn er ein Todesser ist. Aber eine Stunde Brainstorming mit dem Verteidigungsprofessor wäre genug Zeit, um einen kreativen Weg zu finden, ihn finanziell zu ruinieren, oder ihn aus dem magischen Britannien zu verbannen. Das würde der Sache genügen, denke ich."

"Ich gestehe", sagte der alte Zauberer langsam, "dass mir der Gedanke, ein fünfhundert Jahre altes Haus zu ruinieren und einen Todesser wegen eines Handgemenges in einem Hogwarts-Flur zum Krieg bis zum Ende herauszufordern, nicht in den Sinn gekommen ist, Harry."\\ Der alte Zauberer hob einen Finger, um seine Halbmondbrille zurückzuschieben, die ihm bei seiner plötzlichen Bewegung vorhin ein wenig auf die Nase gerutscht war. "Ich wage zu behaupten, dass es auch Miss Granger nicht in den Sinn gekommen wäre, ebenso wenig Professor McGonagall und Fred und George."

Der Junge zuckte mit den Schultern.\\ "Es würde nicht um die Flure gehen", sagte der Junge. "Es ginge um Gerechtigkeit für seine vergangenen Verbrechen, und ich würde es nur tun, wenn Jugson den ersten Schritt machen würde. Es geht schließlich nicht darum, den Leuten Angst vor mir als Joker zu machen. Es geht darum, ihnen beizubringen, dass Neutrale vor mir absolut sicher sind und dass es unglaublich gefährlich ist, mich mit einem Stock zu stoßen." Der Junge lächelte auf eine Weise, die seine Augen nicht erreichte.\\ "Vielleicht kaufe ich eine Anzeige im Tagespropheten, in der ich sage, dass jeder, der diesen Streit mit mir weiterführen will, die wahre Bedeutung von Chaos lernen wird, aber jeder, der mich in Ruhe lässt, wird in Ruhe gelassen."

"Nein", sagte der alte Zauberer. Seine Stimme war jetzt tiefer und zeigte etwas von seinem wahren Alter und seiner Macht.\\ "Nein, Harry, das darf nicht sein. Du hast die Bedeutung des Kämpfens noch nicht gelernt, was wirklich passiert, wenn Feinde im Kampf aufeinandertreffen. Und so träumst du, wie kleine Jungs es tun, deine Feinde zu lehren, dich zu fürchten. Es erschreckt mich, dass du in deinem viel zu jungen Alter schon genug Macht haben könntest, um einen Teil deiner Träume in die Realität umzusetzen. Es gibt keine Abzweigung dieser Straße, die nicht in die Dunkelheit führt, Harry, \emph{keine!} Das ist der Weg eines Dunklen Lords, ganz sicher."

Der Junge zögerte, dann flackerten seine Augen zu der leeren goldenen Plattform, auf der Fawkes manchmal seine Flügel ausruhte. Es war eine Geste, die nur wenige verstanden hätten, aber der alte Zauberer kannte sie sehr gut.

"In Ordnung, vergiss den Teil, dass ich ihnen beibringe, mich zu fürchten", sagte der Junge dann. Seine Stimme war nicht weniger hart, aber etwas von der Kälte war aus ihr gewichen. "Ich denke immer noch, dass man nicht zulassen sollte, dass Kinder aus Angst davor, was jemand wie Lord Jugson tun könnte, verletzt werden. Sie zu beschützen ist der Sinn Ihres Jobs. Wenn Lord Jugson wirklich versucht, sich Ihnen in den Weg zu stellen, dann tun Sie alles, was nötig ist, um ihn aufzuhalten. Geben Sie mir vollen Zugang zu meinen Tresoren, und ich übernehme persönlich die Verantwortung für die Folgen des Verbots von Tyrannen in Hogwarts, ganz gleich, ob es sich um Lord Jugson oder jemand anderen handelt."

Langsam schüttelte der alte Zauberer den Kopf.\\ "Du scheinst zu glauben, Harry, dass ich nur meine volle Kraft einsetzen muss, und alle Feinde werden beiseite gefegt werden. Du irrst dich. Lucius Malfoy kontrolliert Minister Fudge, über den Tagespropheten beeinflusst er ganz Großbritannien, nur ganz knapp kontrolliert er nicht genug vom Obersten Rat, um mich aus Hogwarts zu vertreiben. Amelia Bones und Bartemius Crouch sind Verbündete, aber selbst sie würden zur Seite treten, wenn sie uns mutwillig handeln sehen. Die Welt, die dich umgibt, ist zerbrechlicher, als du zu glauben scheinst, und wir müssen vorsichtiger sein. Der alte Zaubererkrieg hat nie geendet, Harry, er ging nur in anderer Form weiter; der schwarze König schlief und Lucius Malfoy bewegte eine Zeit lang seine Schachfiguren. Glaubst du, Lucius Malfoy würde dir leichtfertig erlauben, einen Bauern seiner Farbe zu nehmen?"

Der Junge lächelte, jetzt wieder mit einem Hauch von Kälte.\\ "Okay, ich werde mir etwas einfallen lassen, um es so zu arrangieren, dass es so aussieht, als hätte Lord Jugson seine eigene Seite verraten."

"Harry -"

"Hindernisse bedeuten, dass man kreativ wird, Schulleiter. Das heißt nicht, dass man die Kinder, die man beschützen soll, im Stich lässt. Lassen Sie das Licht gewinnen, und wenn es Ärger gibt -"\\ Der Junge zuckte mit den Schultern.\\ "Lass das Licht wieder siegen."

"So könnten Phönixe sprechen, wenn sie Worte hätten", sagte der alte Zauberer. "Aber du kennst nicht den \emph{Preis des Phönix}."

Die letzten beiden Worte wurden mit einer seltsam klaren Stimme gesprochen, die im ganzen Büro widerzuhallen schien, und dann schien ein gewaltiges Grollen von überall her zu kommen. Zwischen dem uralten Schild an der Wand und der Hutablage des Sprechenden Hutes begann der Stein der Wände zu fließen und sich zu bewegen, goss sich in zwei rahmende Säulen und gab eine Lücke zwischen ihnen frei, eine Öffnung, die eine Reihe von Steintreppen zeigte, die nach oben in die Dunkelheit führten. Der alte Zauberer drehte sich um und schritt auf diese Treppe zu, dann blickte er zurück zu Harry Potter.\\ "Komm!", sagte der alte Zauberer. In seinen blauen Augen war jetzt kein Funkeln mehr zu sehen.\\ "Da du schon so weit gegangen bist, dich uneingeladen hierher zu drängen, kannst du genauso gut weitergehen."

Es gab kein Geländer auf diesen Steinstufen, und nach den ersten paar Schritten zog Harry seinen Zauberstab und zauberte Lumos. Der Schulleiter blickte nicht zurück, schien nicht nach unten zu schauen, als wäre er die Stufen oft genug hinaufgestiegen, um keine Augen mehr zu brauchen. Der Junge wusste, dass er neugierig hätte sein sollen, oder ängstlich, aber dafür hatte er keine Gehirnkapazität übrig. Es kostete ihn all seine Beherrschung, die in ihm brodelnde Wut nicht noch weiter überkochen zu lassen, als sie es ohnehin schon tat. Die Treppe ging nur ein kurzes Stück weiter, eine gerade, ansteigende Treppe ohne Kurven und Wendungen.

Oben war eine Tür aus massivem Metall, die im blauen Licht von Harrys Zauberstab schwarz aussah, was bedeutete, dass das Metall selbst entweder schwarz oder vielleicht rot war. Albus Dumbledore hob seinen langen Zauberstab wie ein geschwungenes Symbol und sprach wieder mit dieser seltsamen Stimme, die in Harrys Ohren zu hallen schien, als hätte sie sich in sein Gedächtnis eingebrannt:\\ "\emph{Schicksal des Phönix.}"

Die letzte Tür öffnete sich, und Harry folgte Dumbledore hinein. Der Raum dahinter schien aus schwarzem Metall zu sein, wie die Tür, die zu ihm führte. Die Wände waren schwarz, der Boden war schwarz. Die Decke darüber war schwarz, bis auf eine einzelne Kristallkugel, die an einer weißen Kette von der Decke herabhing und in einem brillanten, silbernen Licht leuchtete, das aussah, als wäre es in Nachahmung des Patronus-Lichts gegossen worden, obwohl man sehen konnte, dass es nicht das echte war. Im Raum standen Sockel aus schwarzem Metall, jeder trug ein sich bewegendes Bild oder einen aufrechten Zylinder, der halb mit einer schwach glänzenden silbernen Flüssigkeit gefüllt war, oder einen einzelnen kleinen Gegenstand; eine versengte silberne Halskette, einen zerdrückten Hut, einen unberührten goldenen Ehering. Viele Sockel trugen alle drei, das bewegte Bild und die silberne Flüssigkeit und den Gegenstand. Auf diesen Sockeln schienen eine ganze Reihe von Zauberstäben zu stehen, und viele dieser Stäbe waren zerbrochen oder verbrannt oder sahen aus, als sei das Holz irgendwie geschmolzen.

Es dauerte lange, bis Harry begriff, was er da sah, und dann schnürte es ihm plötzlich die Kehle zu; es war, als hätte die Wut in ihm einen Hammerschlag bekommen, vielleicht den härtesten Hammerschlag seiner ganzen Existenz.

"Das sind nicht alle Gefallenen aus all meinen Kriegen", sagte Albus Dumbledore. Sein Rücken war Harry zugewandt, nur seine grauen Locken und die gelblichen Roben zeigten ihn. "Nicht einmal annähernd alle von ihnen. Nur meine engsten Freunde, und diejenigen, die durch meine schlimmsten Entscheidungen gestorben sind, nur von Ihnen ist etwas hier. Diejenigen, die ich am meisten bedaure, das ist ihr Platz."

\emph{Harry konnte nicht zählen, wie viele Podeste im Raum standen.}

\emph{Es vielleicht um die hundert .}

Der Raum aus schwarzem Metall war nicht klein, und es war eindeutig noch mehr Platz darin für zukünftige Sockel. Albus Dumbledore drehte sich um und betrachtete Harry, die tiefblauen Augen wie Stahl in die Stirn gelegt, aber seine Stimme, als er sprach, war ruhig.

"Es scheint mir, dass du nichts über den Preis des Phönix weißt", sagte Albus Dumbledore leise. "Es scheint mir, dass du kein böser Mensch bist, sondern furchtbar unwissend und selbstbewusst in deiner Unwissenheit; so wie ich es einst war, vor langer Zeit. Dennoch habe ich Fawkes nie so deutlich gehört, wie du es an jenem Tag zu tun schienst. Vielleicht war ich schon zu alt und voller Kummer, als mein Phönix zu mir kam. Wenn es etwas gibt, was ich nicht verstehe, wie bereit ich sein sollte, zu kämpfen, dann erzähle mir von dieser Weisheit."\\ Es lag kein Zorn in der Stimme des alten Zauberers; der Aufprall, der einem den Atem raubte, als würde man von einem Besenstiel fallen, lag in den verbrannten und zerbrochenen Zauberstäben, die im silbernen Licht sanft in ihrem Tod schimmerten.\\ "Oder du drehst dich um und gehst von diesem Ort, aber dann will ich nichts mehr davon hören."

Harry wusste nicht, was er sagen sollte. So etwas hatte es in seinem eigenen Leben noch nicht gegeben, und alle Worte schienen ihm zu entfallen. Er würde etwas zu sagen finden, wenn er suchte, aber er konnte in diesem Moment nicht glauben, dass die Worte sinnvoll sein würden. Man sollte nicht in der Lage sein, jedes mögliche Argument zu gewinnen, nur weil Menschen durch seine Entscheidungen gestorben waren, und doch fühlte es sich so an, als gäbe es nichts zu sagen. Dass es nichts gab, was Harry zu sagen hatte. Und fast hätte Harry sich umgedreht und wäre von diesem Ort gegangen, bis zu der Erkenntnis, die ihm dann kam: dass es wahrscheinlich einen Teil von Albus Dumbledore gab, der immer an diesem Ort stand, immer, egal wo er war. Und dass man, wenn man an einem solchen Ort stand, alles tun, alles verlieren konnte, wenn es bedeutete, dass man kein weiteres Mal kämpfen musste.

Einer der Sockel fiel Harry ins Auge; das Foto darauf bewegte sich nicht, lächelte oder winkte nicht, es war ein Muggelfoto von einer Frau, die ernst in die Kamera blickte, ihr braunes Haar zu Zöpfen geflochten in einem gewöhnlichen Muggelstil, den Harry noch an keiner Hexe gesehen hatte. Neben dem Foto stand ein Zylinder mit silbriger Flüssigkeit, aber kein Gegenstand; kein geschmolzener Ring oder zerbrochener Zauberstab.

Harry ging vorwärts, langsam, bis er vor dem Sockel stand.\\ "Wer war sie?" sagte Harry, und seine Stimme klang seltsam in seinen eigenen Ohren.

"Ihr Name war Tricia Glasswell", sagte Dumbledore. "Die Mutter einer Muggelgeborenen, die von den Todessern getötet wurde. Sie war eine Detektivin der Muggelregierung und hat danach den Orden des Phönix mit Informationen der Muggelbehörden gefüttert, bis sie - \emph{verraten} - in die Hände von Voldemort fiel."

Es lag ein Haken in der Stimme des alten Zauberers.\\ "Sie ist nicht gut gestorben, Harry."

"Hat sie Leben gerettet?" fragte Harry.

"Ja", sagte der Zauberer leise. "Das hat sie."

Harry hob seinen Blick von dem Podest und sah Dumbledore an.\\ "Wäre die Welt ein besserer Ort, wenn sie nicht gekämpft hätte?"

"Nein, wäre sie nicht", sagte der alte Zauberer.\\ Seine Stimme war müde und traurig. Er wirkte jetzt noch gebeugter, als würde er in sich zusammensinken.\\ "Ich sehe, dass du immer noch nicht verstehst. Ich denke, du wirst es nicht verstehen, bis zu dem Tag, an dem du - oh, Harry. Vor so langer Zeit, als ich nicht viel älter war als du jetzt, lernte ich das wahre Gesicht der Gewalt kennen und ihren Preis. Die Luft mit tödlichen Flüchen zu erfüllen - aus welchem Grund auch immer, Harry - ist eine kranke Sache, und ihre Natur ist korrumpiert, so schrecklich wie die dunkelsten Rituale. Gewalt, einmal begonnen, wird wie ein unaufhaltsamer Wahnsinn, der jedes Leben in seiner Nähe angreift. Ich … würde dir diese Lektion ersparen, so wie ich sie gelernt habe, Harry."

Harry wandte den Blick von den blauen Augen ab, warf seinen Blick auf das schwarze Metall des Bodens. Der Schulleiter wollte ihm etwas Wichtiges sagen, das war klar; und es war auch nichts, was Harry für dumm hielt.

"Es gab einmal einen Muggel namens Mohandas Gandhi", sagte Harry zum Boden. "Er war der Meinung, dass die Regierung des Muggelstaates Großbritannien nicht über sein Land herrschen sollte. Und er weigerte sich, zu kämpfen. Er überzeugte sein ganzes Land, nicht zu kämpfen. Stattdessen forderte er sein Volk auf, auf die britischen Soldaten zuzugehen und sich niederschlagen zu lassen, ohne Widerstand zu leisten, und als Großbritannien das nicht mehr ertragen konnte, haben wir sein Land befreit. Ich fand das eine sehr schöne Sache, als ich darüber las, dachte ich, das sei etwas Höheres als alle Kriege, die jemals mit Gewehren oder Schwertern geführt worden waren. Dass sie das wirklich getan hatten, und dass es tatsächlich funktioniert hatte." Harry holte noch einmal tief Luft. "Erst dann habe ich herausgefunden, dass Gandhi seinen Leuten während des Zweiten Weltkriegs gesagt hat, dass sie, wenn die Nazis einmarschieren, auch gewaltlosen Widerstand gegen sie leisten sollten. Aber die Nazis hätten einfach jeden in Sichtweite erschossen. Und vielleicht hatte Winston Churchill immer das Gefühl, dass es einen besseren Weg hätte geben müssen, irgendeinen klugen Weg, um zu gewinnen, ohne jemanden verletzen zu müssen; aber er hat ihn nie gefunden, und deshalb musste er kämpfen."\\ Harry sah zu dem Schulleiter auf, der ihn anstarrte.\\ "Winston Churchill war derjenige, der versucht hat, die britische Regierung davon zu überzeugen, die Tschechoslowakei nicht an Hitler als Gegenleistung für einen Friedensvertrag zu übergeben, dass sie sofort kämpfen sollten -"

"Ich erkenne den Namen, Harry", sagte Dumbledore.\\ Die Lippen des alten Zauberers zuckten nach oben.\\ "Obwohl ich der Ehrlichkeit halber sagen muss, dass der liebe Winston nie ein Freund von Gewissensbissen war, auch nicht nach einem Dutzend Gläsern Feuerwhiskey." …\\ "Der Punkt ist", sagte Harry nach einer kurzen Pause, um sich daran zu erinnern, mit wem er sprach, und um das plötzlich wiederkehrende Gefühl zu bekämpfen, dass er ein unwissendes Kind war, das vor Dreistigkeit verrückt geworden war und kein Recht hatte, in diesem Raum zu sein und Albus Dumbledore über irgendetwas zu befragen, "der Punkt ist, dass zu sagen, Gewalt sei böse, keine Antwort ist. Es sagt nicht, wann man kämpfen soll und wann nicht. Es ist eine schwierige Frage und Gandhi hat sich geweigert, sich damit auseinanderzusetzen, und deshalb habe ich einen Teil meines Respekts vor ihm verloren."

"Und deine eigene Antwort, Harry?" sagte Dumbledore leise.

"Eine Antwort ist, dass man niemals Gewalt anwenden sollte, außer um Gewalt zu stoppen", sagte Harry. "Man sollte nie das Leben von jemandem riskieren, außer um noch mehr Leben zu retten. Das klingt gut, wenn man es so sagt. Das Problem ist nur, dass wenn ein Polizist einen Einbrecher sieht, der ein Haus ausraubt, der Polizist versuchen sollte, den Einbrecher zu stoppen, auch wenn der Einbrecher sich wehren könnte und jemand verletzt oder sogar getötet werden könnte. Selbst wenn der Einbrecher nur versucht, Schmuck zu stehlen, was nur eine Sache ist. Denn wenn niemand Einbrecher auch nur belästigt, wird es mehr Einbrecher geben, und noch mehr Einbrecher. Und selbst wenn sie jedes Mal nur etwas stehlen würden, würde es - das Gefüge der Gesellschaft -" Harry hielt inne.\\ Seine Gedanken waren nicht so geordnet, wie sie es normalerweise vorgaben, in diesem Raum zu sein. Er hätte in der Lage sein müssen, eine vollkommen logische Erklärung im Sinne der Spieltheorie zu geben, hätte es zumindest so sehen müssen, aber es entging ihm. - "Verstehen Sie nicht, wenn böse Menschen bereit sind, Gewalt zu riskieren, um zu bekommen, was sie wollen, und gute Menschen immer einen Rückzieher machen, weil Gewalt zu schrecklich ist, um sie zu riskieren, dann ist es - es ist keine gute Gesellschaft, in der man leben kann, Herr Direktor! Ist Ihnen nicht klar, was dieses ganze Mobbing in Hogwarts anrichtet, vor allem im Haus Slytherin?"

"Krieg ist zu schrecklich, um ihn zu riskieren", sagte der alte Zauberer. "Und doch wird er kommen. Voldemort kehrt zurück. Die schwarzen Schachfiguren versammeln sich. Severus ist eine der wichtigsten Figuren, die unsere eigene Seite in diesem Krieg besitzt. Aber unser böser Zaubertränkemeister muss, wie man so schön sagt, den Schein wahren. Wenn Severus diesen Schein wahren kann, indem er die Gefühle von Kindern verletzt, und zwar nur ihre Gefühle, Harry", die Stimme des alten Zauberers war sehr sanft, "dann muss man schon furchtbar unschuldig sein, um zu glauben, dass er ein schlechtes Geschäft gemacht hat. Schwere Entscheidungen sehen nicht so aus, Harry. Sie sehen - \emph{so} aus."\\ Der alte Zauberer machte keine Geste. Er blieb einfach stehen, wo er war, zwischen den Sockeln.

"Sie sollten nicht Schulleiter sein", sagte Harry durch das Brennen in seiner Kehle. "Es tut mir leid, es tut mir so leid, aber Sie sollten nicht versuchen, ein Schulleiter zu sein und gleichzeitig einen Krieg zu führen. Hogwarts sollte nicht daran beteiligt sein."

"Die Kinder werden überleben", sagte der alte Zauberer mit müden alten Augen. "Sie würden Voldemort nicht überleben. Hast du dich gefragt, warum die Kinder von Hogwarts nicht viel von ihren Eltern sprechen, Harry? Das liegt daran, dass es immer jemanden in Hörweite gibt, der seine Mutter oder seinen Vater oder beide verloren hat. Das hat Voldemort hinterlassen, als er das letzte Mal hier war. Nichts ist es wert, dass dieser Krieg auch nur einen Tag früher beginnt, als er muss, oder einen Tag länger dauert, als er muss."\\ Der alte Zauberer machte jetzt eine Geste, als wolle er auf all die zerbrochenen Zauberstäbe hinweisen.\\ "Wir haben nicht gekämpft, weil es uns gerecht erschien, dies zu tun! Wir haben gekämpft, als wir es mussten, als es keinen anderen Weg mehr gab. Das war \emph{unsere} Antwort."

"Hast du deshalb so lange damit gewartet, Grindelwald zu konfrontieren?" Harry hatte die Frage geäußert, ohne richtig nachzudenken -

es gab eine langsame Zeit, während die blauen Augen ihn absuchten.\\ "Mit wem hast du geredet, Harry?", fragte der alte Zauberer. "Nein, antworte nicht. Ich weiß es schon."\\ Dumbledore seufzte.\\ "Viele haben mir diese Frage gestellt, und immer habe ich sie abgewiesen. Doch mit der Zeit musst du die volle Wahrheit über diese Angelegenheit erfahren. Schwörst du, niemals einem anderen davon zu erzählen, bis ich dir die Erlaubnis gebe?"

Harry hätte es Draco gerne gesagt, aber -\\ "Ich schwöre", sagte Harry.

"Grindelwald besaß ein uraltes und schreckliches Artefakt", sagte Dumbledore. "Solange er es besaß, konnte ich seine Verteidigung nicht brechen. In unserem Duell konnte ich nicht gewinnen, nur stundenlang gegen ihn kämpfen, bis er vor Erschöpfung fiel; und ich wäre danach daran gestorben, wenn Fawkes nicht gewesen wäre. Aber während seine Muggelverbündeten noch Blutopfer brachten, um ihn zu unterstützen, konnte Grindelwald nicht fallen. Er war in dieser Zeit wahrlich unbesiegbar. Von dem grimmigen Artefakt, das Grindelwald besaß, darf niemand etwas wissen, niemand etwas ahnen, es darf nicht die geringste Andeutung geben. Und deshalb darfst du nicht darüber sprechen. Und ich werde vorerst nichts mehr sagen. Das ist alles, Harry. Es gibt keine Moral darin und keine Weisheit. Mehr gibt es nicht."

Harry nickte langsam. Es war nicht ganz unplausibel, nach den Maßstäben der Magie…

"Und dann", fuhr Dumbledores Stimme fort, noch leiser, fast so, als spräche er zu sich selbst, "da ich es war, der ihn fällte, gehorchten sie mir, als ich sagte, er solle nicht sterben, obwohl sie zu Tausenden nach seinem Blut schrien. So wurde er in Nurmengard eingekerkert, in dem Gefängnis, das er gebaut hat, und dort ist er bis zum heutigen Tag. Ich bin zu diesem Duell gegangen, ohne die Absicht, ihn zu töten, Harry. Denn, weißt du, ich hatte schon einmal versucht, Grindelwald zu töten, vor langer Zeit, und das… das war… es erwies sich als… ein Fehler, Harry…"\\ Der alte Zauberer starrte jetzt auf seinen langen, dunkelgrauen Zauberstab, den er in beiden Händen hielt, als wäre er eine Kristallkugel aus der Muggelphantasie, ein Wahrsagebecken, in dem man Antworten finden konnte.\\ "Und ich dachte damals… Ich dachte, dass ich niemals töten sollte. Und dann kam Voldemort."\\ Der alte Zauberer sah wieder zu Harry auf und sagte mit heiserer Stimme:\\ "Er ist nicht wie Grindelwald, Harry. Es ist nichts Menschliches mehr in ihm. Ihn musst du vernichten. Du darfst nicht zögern, wenn die Zeit gekommen ist. Bei ihm allein, von allen Kreaturen dieser Welt, darfst du keine Gnade walten lassen; und wenn du fertig bist, musst du es vergessen, vergessen, dass du so etwas jemals getan hast, und wieder zum Leben zurückkehren. Spare dir deine Wut dafür auf, und nur dafür."

In diesem Büro herrschte Stille. Sie dauerte viele, viele Sekunden und wurde schließlich von einer einzigen Frage durchbrochen.

"Gibt es Dementoren in Nurmengard?"

"Was?", sagte der alte Zauberer. "Nein! Das hätte ich ihm nicht angetan -"\\ Der alte Zauberer starrte den Jungen an, der sich aufgerichtet hatte, und sein Gesicht veränderte sich.

"Mit anderen Worten", sagte der Junge, als spräche er mit sich selbst, ohne dass andere Leute im Raum waren, "man weiß schon, wie man mächtige Dunkle Zauberer im Gefängnis halten kann, ohne Dementoren zu benutzen. Die Leute wissen, dass sie das wissen."

"Harry …?"

"Nein", sagte der Junge. Der Junge sah auf, und seine Augen loderten wie grünes Feuer. "Ich akzeptiere Ihre Antwort nicht, Schulleiter. Fawkes hat mir einen Auftrag gegeben, und ich weiß jetzt, warum Fawkes diesen Auftrag mir gegeben hat und nicht Ihnen. Sie sind bereit, ein Gleichgewicht der Kräfte zu akzeptieren, bei dem die Bösen am Ende gewinnen. Ich bin es nicht."

"Auch das ist keine Antwort", sagte der alte Zauberer; sein Gesicht zeigte nichts von seiner Verletzung, er hatte lange Übung darin, Schmerz zu verbergen. "Sich zu weigern, etwas zu akzeptieren, ändert es nicht. Ich frage mich jetzt, ob du einfach zu jung bist, um diese Sache zu verstehen, Harry, trotz deiner Äußerlichkeiten; nur in Kinderphantasien können alle Schlachten gewonnen und kein einziges Übel geduldet werden."

"Und das ist der Grund, warum ich Dementoren vernichten kann und du nicht", sagte der Junge. "Weil ich daran glaube, dass die Dunkelheit gebrochen werden kann."

Dem alten Zauberer stockte der Atem in der Kehle. "Der Preis des Phönix ist nicht unausweichlich", sagte der Junge. "Es ist nicht Teil eines tiefen Gleichgewichts, das in das Universum eingebaut ist. Es sind nur die Teile des Problems, bei denen man noch nicht herausgefunden hat, wie man sie lösen kann."

Die Lippen des alten Zauberers spalteten sich, aber es kamen keine Worte heraus. Silbernes Licht fällt auf zerbrochene Zauberstäbe.

"Fawkes gab mir einen Auftrag", wiederholte der Junge, "und ich werde diesen Auftrag ausführen, und wenn ich dafür das gesamte Ministerium brechen muss. Das ist der Teil der Antwort, den Sie übersehen. Man hört nicht auf und sagst, na ja, ich schätze, ich kann unmöglich einen Weg finden, um das Mobbing in Hogwarts zu stoppen, und belässt es dabei. Man sucht einfach weiter, bis man herausgefunden hat, wie man es machen kann. Wenn das erfordert, Lucius Malfoys gesamte Verschwörung zu zerschlagen, schön."

"Und der wahre Kampf, der Kampf gegen Voldemort?", fragte der alte Zauberer mit unsicherer Stimme. "Was wirst du tun, um ihn zu gewinnen, Harry? Wirst du die ganze Welt zerstören? Selbst wenn du eines Tages eine solche Macht erlangst, bist du noch nicht über den Preis hinaus, und vielleicht wirst du es auch nie sein! Dass du dich jetzt so verhältst, ist nichts weniger als Wahnsinn!"

"Ich habe Professor Quirrell gefragt, warum er gelacht hat", sagte der Junge gleichmütig, "nachdem er Hermine diese hundert Punkte verliehen hatte. Und Professor Quirrell sagte, das sind nicht seine genauen Worte, aber es ist ziemlich genau das, was er sagte, dass er es ungeheuer amüsant fand, dass der große und gute Albus Dumbledore da saß und nichts tat, während dieses arme unschuldige Mädchen um Hilfe bettelte, während er derjenige war, der sie verteidigte. Und er erzählte mir damals, dass gute und moralische Menschen, wenn sie fertig damit waren, sich in Knoten zu verstricken, für gewöhnlich nichts taten; oder wenn sie handelten, konnte man sie kaum von den Leuten unterscheiden, die man als schlecht bezeichnete. Er hingegen konnte unschuldigen Mädchen helfen, wann immer ihm danach war, denn er ist kein guter Mensch. Und das sollte ich mir merken, wenn ich daran dachte, gut zu werden."

Der alte Zauberer ließ sich die Wucht des Schlages nicht anmerken. Nur ein leichtes Aufreißen seiner Augen hätte es verraten, wenn man ihn genau beobachtet hätte.

"Machen Sie sich keine Sorgen, Herr Direktor", sagte der Junge. "Ich habe mich nicht geändert. Ich weiß, dass ich das Gute von Hermine und Fawkes lernen soll, nicht von Professor Quirrell und Ihnen. Was mich zu dem eigentlichen Grund bringt, warum ich hierher gekommen bin. Hermine's Zeit ist zu wertvoll, um sie mit Nachsitzen zu vergeuden. Professor Snape wird sie widerrufen und behaupten, ich hätte ihn erpresst."

Nach einem Zögern nickte der alte Zauberer mit dem Kopf, der silberne Bart wogte langsam darunter. "Das wäre nicht das Beste für sie, Harry", sagte der alte Zauberer. "Das Nachsitzen kann bei Professor Binns eingetragen werden, und du kannst mit ihr zusammen in seinem Klassenzimmer lernen."

"Gut", sagte der Junge. "Ich glaube, das war dann auch schon alles, was wir zusammen zu tun hatten. Sie können davon ausgehen, dass ich das nächste Mal, wenn Sie scheinbar auf der Seite der Bösen arbeiten oder sie gewinnen lassen, alles tun werde, was ich glaube, dass Fawkes mir sagen würde, egal, wie viel Ärger dabei herauskommt. Ich hoffe, wir sind uns beide darüber im Klaren."

Ohne ein weiteres Wort drehte sich der Junge um und ging aus dem Zimmer, durch die offene Tür aus schwarzem Metall, die Worte "Lumos!" und das Licht seines Zauberstabs folgten einen Moment später.

Der alte Zauberer stand stumm da, stumm inmitten der Erinnerungen an die Toten, die sein eigenes Leben hinterlassen hatte. Seine faltige Hand hob sich zitternd, um seine Halbmondbrille zu berühren - der Junge steckte den Kopf wieder herein.

"Würden Sie bitte die Treppe einschalten, Herr Direktor? Ich möchte mir nicht noch einmal die ganze Arbeit machen, um auf demselben Weg zu gehen, auf dem ich gekommen bin."

"Geh, Harry Potter", sagte der alte Zauberer. "Die Treppe wird dich tragen."

(Einige Zeit später folgte eine frühere Version von Harry, der seit 21 Uhr unsichtbar neben den Wasserspeiern gewartet hatte, der stellvertretenden Schulleiterin durch die Öffnung, die sich für sie öffnete, stellte sich leise hinter sie auf die sich drehende Treppe, bis sie oben ankamen, und drehte dann, immer noch unter dem Umhang, seinen Zeitdreher dreimal.)

\textbf{Nachspiel: Professor Quirrell und -}\\ Auf einer schattigen Lichtung wartete der Verteidigungsprofessor, den Rücken nachlässig an die raue graue Rinde einer hoch aufragenden Buche gelehnt, die in den späten Märztagen noch nicht belaubt war, so dass ihr Stamm und ihre Krone wie ein bleicher Arm wirkten, der aus dem Boden ragte und in eine Hand mit tausend Fingern explodierte. Um den Verteidigungsprofessor herum und über ihm waren die Äste so dicht, dass man selbst im frühesten Frühling, wenn nur wenige Bäume so etwas wie Knospen trugen, vom Boden aus kaum den Himmel hätte sehen können. Die Stränge des hölzernen Netzes kreuzten und wucherten so oft, dass man, wenn man oben auf einem Besenstiel saß und unten nach jemandem suchte, eher den Ohren als den Augen hätte folgen können. Es hätte auch nicht geholfen, dass es inmitten des verbotenen Waldes schon fast dunkel war, die unsichtbare Sonne fast untergegangen, so dass nur ein paar Schimmer des verblassenden Sonnenlichts die Wipfel der höchsten Bäume beleuchteten. Dann kam das leiseste Geräusch von Schritten, fast unhörbar selbst auf dem Waldboden; der Gang eines Mannes, der es gewohnt war, ungesehen vorbeizugehen. Kein Zweig knackte, kein Blatt raschelte -

"Guten Tag", sagte Professor Quirrell. Der Verteidigungsprofessor machte sich nicht die Mühe, seine Augen oder seine Hände zu bewegen, die nachlässig an seiner Seite ruhten. Eine in einen schwarzen Umhang gekleidete Gestalt schimmerte ins Dasein, sein Kopf drehte sich, um nach links und dann nach rechts zu schauen. In der rechten Hand der Gestalt lag, tief gegriffen, ein Stab aus Holz, der so grau war, dass er fast silbern wirkte.

"Ich weiß nicht, warum Sie sich ausgerechnet hier treffen wollten", sagte Severus Snape, seine Stimme war kühl.

"Oh", sagte Professor Quirrell müßig, als wäre die ganze Angelegenheit von geringster Bedeutung, "ich dachte, Sie würden die Privatsphäre vorziehen. Die Wände von Hogwarts haben Ohren, und Sie möchten doch nicht, dass der Schulleiter von Ihrer Rolle in der gestrigen Affäre erfährt, oder?"

Die Märzkälte schien tiefer zu werden, die Temperatur weiter zu sinken.\\ "Ich weiß nicht, wovon Sie reden", sagte der Zaubertränkemeister eisig.

"Sie wissen sehr wohl, wovon wir reden", sagte Professor Quirrell mit amüsierter Stimme. "Wirklich, mein guter Professor, Sie sollten sich nicht in die Angelegenheiten von Idioten einmischen, es sei denn, Sie sind bereit, sich auf der Stelle gegen ihre Gewalt zu verteidigen."

(Die Hände des Verteidigungsprofessors lagen immer noch entspannt und offen an seiner Seite.)\\ "Und doch scheint sich keiner dieser Idioten an Ihren Sturz zu erinnern, noch erinnern sich die jungen Damen an Ihre Anwesenheit. Was die faszinierende Frage aufwirft, warum Sie sich die außerordentliche Mühe, ich wage zu sagen, die verzweifelte Mühe machen, zweiundfünfzig Gedächtniszauber zu wirken."\\ Professor Quirrell legte den Kopf schief.\\ "Würden Sie sich so sehr vor der Meinung von einfachen Schülern fürchten? Ich glaube nicht. Würden Sie befürchten, dass Ihr guter Freund, Lord Malfoy, davon erfährt? Aber diese Narren haben an Ort und Stelle eine recht zufriedenstellende Entschuldigung für Ihre Anwesenheit erfunden. Nein, es gibt nur eine Person, die so viel Macht über Sie hat und die höchst beunruhigt wäre, wenn Sie ohne sein Wissen ein Komplott schmiedeen. Dein wahrer und verborgener Meister, Albus Dumbledore."

"Was?!", zischte der Meister der Zaubertränke, der Zorn stand ihm ins Gesicht geschrieben.

"Aber jetzt, so scheint es, sind Sie auf eigene Faust unterwegs; und deshalb bin ich höchst neugierig, was Sie tun könnten, und warum."\\ Der Verteidigungsprofessor betrachtete die schwarz gekleidete Silhouette des Zaubertränkemeisters mit der Aufmerksamkeit, die ein Mann einem außergewöhnlich interessanten Käfer schenken würde, auch wenn es letztlich nur ein Käfer war.

"Ich bin kein Diener von Dumbledore", sagte der Tränkemeister kalt.

"Wirklich? Was für eine erstaunliche Nachricht."\\ Der Verteidigungsprofessor lächelte leicht.\\ "Erzählen Sie mir alles darüber."

Es gab eine lange Pause. Von irgendeinem Baum schrie eine Eule, das Geräusch war gewaltig in der Stille; keiner der beiden Männer schreckte auf oder zuckte zurück. "Sie wollen mich nicht zum Feind haben, Quirrell", sagte Severus Snape, seine Stimme war sehr sanft.

"Will ich nicht?", fragte Professor Quirrell. "Woher wollen Sie das wissen?"

"Andererseits", fuhr der Meister der Zaubertränke mit immer noch weicher Stimme fort, "genießen meine Freunde viele Vorteile."

Der Mann, der an der grauen Rinde lehnte, hob die Augenbrauen. "Zum Beispiel?"

"Es gibt vieles, was ich von dieser Schule weiß", sagte der Meister der Zaubertränke. "Dinge, von denen Sie vielleicht nicht glauben, dass ich sie weiß."

Es entstand eine erwartungsvolle Pause.

"Wie unglaublich faszinierend", sagte Professor Quirrell. Der Mann untersuchte seine Fingernägel mit einem gelangweilten Blick. "Fahren Sie fort."

"Ich weiß, dass Sie … den Korridor im dritten Stock untersucht haben …"

"Sie wissen nichts dergleichen."\\ Der Rücken des Mannes richtete sich gegen das Holz.\\ "Bluffen Sie nicht gegen mich, Severus Snape; ich finde es ärgerlich, und Sie sind nicht in der Position, mich zu ärgern. Ein einziger Blick würde jedem kompetenten Zauberer verraten, dass der Schulleiter diesen Korridor mit einer lächerlichen Menge an Schutzzaubern und Fallen, Auslösern und Stolpersteinen verzaubert hat. Und mehr noch: Dort liegen Zaubersprüche von uralter Macht, magische Konstrukte, von denen ich nicht einmal Gerüchte gehört habe, Techniken, die aus den gehorteten Überlieferungen von Flamel selbst stammen müssen. Selbst Er-der-nicht-genannt-werden-darf hätte Schwierigkeiten gehabt, diese unbemerkt zu passieren."\\ Professor Quirrell tippte mit einem Finger nachdenklich auf seine Wange.\\ "Und was das eigentliche Schloss betrifft, so ist es ein Colloportus, der auf einen gewöhnlichen Türknauf gelegt wurde und so schwach ist, dass er Miss Granger an dem Tag, als sie Hogwarts betrat, nicht hätte aussperren können. Noch nie in meinem Leben bin ich auf eine so offensichtliche Falle gestoßen."\\ Jetzt verengte der Verteidigungsprofessor seine Augen.\\ "Ich kenne niemanden mehr auf der Welt, gegen den solche fantastischen Erkennungsleistungen irgendeinen nützlichen Zweck erfüllen würden. Wenn es einen Zauberer gibt, der über uralte Überlieferungen verfügt, von denen ich nichts weiß, und gegen den diese Falle aufgestellt ist - Sie können diese Information gegen so viel Schweigen eintauschen, wie Sie wollen, mein lieber Professor, und eine gute Portion meiner Gunst, die danach übrig bleibt."

Man hätte schwören können, dass Professor Quirrell Severus Snape mit lebhaftem Interesse beobachtete. Nicht die leiseste Spur eines Lächelns kam über die Lippen des Mannes. Wieder herrschte langes Schweigen auf der Lichtung.

"Ich weiß nicht, wen Dumbledore fürchtet", sagte Snape. "Aber ich weiß, welchen Köder er ausgelegt hat, und etwas davon, wie er wirklich bewacht wird -"

"Was das angeht", sagte Professor Quirrell und klang wieder gelangweilt, "habe ich es vor Monaten gestohlen und eine Fälschung an seiner Stelle hinterlassen. Aber vielen Dank für die Frage."

"Sie lügen", sagte Severus Snape nach einer Pause.

"Ja, das tue ich." Professor Quirrell lehnte sich wieder an den grauen Wald, sein Blick schweifte hinauf zu dem dichten Netz aus Ästen, zwischen denen die hereinbrechende Nacht kaum zu erkennen war.\\ "Ich wollte nur wissen, ob Sie mich darauf ansprechen würden, da Sie vorgeben, so wenig zu wissen."\\ Der Verteidigungsprofessor lächelte vor sich hin.

Der Meister der Zaubertränke sah aus, als würde er gleich an seiner eigenen Wut ersticken. "Was wollen Sie?!"

"Eigentlich nichts.", sagte der Verteidigungsprofessor und starrte weiter an die Walddecke. "Ich war nur neugierig. Ich nehme an, ich werde einfach zusehen, wohin Ihre Verschwörungen führen, und in der Zwischenzeit werde ich dem Schulleiter nichts sagen - solange Sie bereit sind, mir ab und zu einen Gefallen zu tun, natürlich." Ein trockenes Lächeln ging über sein Gesicht.\\ "Sie sind für den Moment entlassen, Severus Snape. Allerdings hätte ich nichts dagegen, wenn wir uns bald noch einmal unterhalten, wenn Sie bereit sind, mir ehrlich zu sagen, wo Ihre Loyalitäten liegen. Und ich meine \emph{ehrlich}, nicht die falschen Gesichter, die Sie heute gezeigt haben. Sie könnten feststellen, dass Sie mehr Verbündete haben, als Sie dachten. Nehmen Sie sich etwas Zeit, darüber nachzudenken, mein Freund."

\textbf{Nachwirkungen: Draco Malfoy und -}\\ Eine Regenbogenhalbkugel, eine Kuppel aus festem Glas mit wenig eigener Farbigkeit, die das eindringende Licht in zersplitterten Reflexen zurückschickte, schillernd in vielen Farben, während sie den Glanz der vielfarbigen Kronleuchter des Slytherin-Gemeinschaftsraumes brach. Unter der regenbogenfarbenen Halbkugel lag das verängstigte Gesicht einer jungen Hexe, die sich noch nie gegen Tyrannen gewehrt hatte, die sich keiner von Professor Quirrells Armeen angeschlossen hatte, die in ihrem Verteidigungskurs bestenfalls akzeptable Noten bekam und die nicht einmal eine prismatische Barriere hätte zaubern können, um ihr eigenes Leben zu retten.

"Ach, hört auf", sagte Draco Malfoy und ließ seine Stimme trotz des Schweißes, der unter seinen Roben ausgebrochen war, gelangweilt klingen, während er seinen Zauberstab auf die Barriere richtete, die Millicent Bulstrode schützte. Er konnte sich nicht daran erinnern, die Entscheidung getroffen zu haben, die beiden älteren Jungen waren gerade dabei, Millicent zu verhexen, der Gemeinschaftsraum starrte schweigend vor sich hin, und dann hatte Dracos Hand gerade seinen Zauberstab gezogen und die Barriere gewirkt und sein Herz mit schockiertem Adrenalin vollgepumpt, während sein armes, trauriges Gehirn verzweifelt nach Erklärungen suchte -\\ Die beiden älteren Jungen richteten sich von der Stelle auf, an der sie Millicent bedrängt hatten, drehten sich zu Draco um und sahen ihn mit einer Mischung aus Schock und Wut an.

Gregory und Vincent neben ihm hatten bereits ihre eigenen Zauberstäbe gezogen, richteten sie aber nicht auf ihn. Alle drei zusammen hätten sowieso nicht gewinnen können. Aber die älteren Jungen würden ihn nicht verhexen. Niemand konnte so dumm sein, den nächsten Lord Malfoy zu verhexen. Es war nicht die Angst, verhext zu werden, die Draco unter seinem Umhang schwitzen ließ, während er verzweifelt hoffte, dass die Wasserperlen auf seiner Stirn nicht sichtbar waren. Draco schwitzte wegen der dämmernden und unangenehmen Gewissheit, dass, selbst wenn er jetzt damit durchkäme, wenn er diesen Weg weiterverfolgte, irgendwann alles zusammenbrechen würde; und dann wäre er vielleicht nicht mehr der nächste Lord Malfoy.

"Mr. Malfoy", sagte der älteste Junge. "Warum beschützen Sie sie?"

"Ihr habt also die Urheberin der Verschwörung ausfindig gemacht", sagte Draco mit einem Nummer-zwei-Grinsen, "und es ist, damit ich das jetzt richtig verstehe, ein Mädchen aus dem ersten Jahr namens Millicent Bulstrode. Sie ist nur ein Bauernopfer, ihr Idioten!"

"Und?", fragte der ältere Junge. "Sie hat ihnen trotzdem geholfen!"

Draco hob seinen Zauberstab und die prismatische Sphäre erlosch. Immer noch mit gelangweilter Stimme fragte Draco: "Wussten Sie, was Sie da tun, Miss Bulstrode?"

"N-nein", stammelte Millicent von dort, wo sie immer noch an ihrem Schreibtisch saß. "Wussten Sie, wohin die Slytherin-Botschaften, die Sie weitergeben wollten, gehen?"

"Nein!", sagte Millicent.

"Danke", sagte Draco. "Ihr alle lasst sie bitte in Ruhe, sie ist nur ein Spielball. Miss Bulstrode, Sie dürfen den Gefallen, den Sie mir im Februar getan haben, als zurückgezahlt betrachten."

Und Draco wandte sich wieder seinen Hausaufgaben in Zaubertränke zu und hoffte bei Merlin und wieder zurück, dass Millicent nicht irgendetwas unglaublich Dummes sagte wie: \emph{"Welcher Gefallen?"} -

"Warum dann", sagte eine Stimme deutlich vom anderen Ende des Raumes, "sind diese Hexen dorthin gegangen, wo ein Zettel von Millicent sie hinwies?"

Noch mehr schwitzend hob Draco den Kopf und schaute zu der Stelle, an der Randolph Lee gesprochen hatte.\\ "Was genau stand auf dem gefälschten Zettel?", fragte Draco.\\ "War es 'Ich befehle euch, im Namen der Dunklen Lady Bulstrode hinauszugehen' oder 'Bitte trefft Sie mich hier, aufrichtig Millicent?'"

Randolph Lee öffnete den Mund, zögerte für den Bruchteil einer Sekunde -

"Das dachte ich mir", sagte Draco. "Das war kein sehr guter Test, Mr. Lee, es - es kann -" Ein hektischer, nervenaufreibender Moment, während er überlegte, wie er es sagen sollte, ohne Harry-Wörter wie "Fehlschluss" \emph{('false positiv, Anm. des Übersetzers)} zu benutzen.\\ "Es kann Hexen dazu bringen, dorthin zu gehen, wenn eine von ihnen nur mit Millicent befreundet ist."\\ Als wäre die Angelegenheit völlig geklärt, blickte Draco wieder auf seine Zaubertrank-Hausaufgaben hinunter und ignorierte (abgesehen von dem Gefühl des kranken Grauens in seinem Magen) das Geflüster aus dem Raum. Nur aus dem Augenwinkel heraus bemerkte er, wie Gregory ihn anstarrte.\\ Dracos Augen ruhten auf seinen Astronomie-Hausaufgaben, aber er konnte seine Gedanken nicht darauf konzentrieren.

Wenn man versuchte, nicht an die Dinge zu denken, die Harry Potter gesagt hatte, war so ziemlich das Schlimmste, was man tun konnte, sich die Bilder des Nachthimmels in seinem Lehrbuch anzusehen und zu versuchen, sich an das zu erinnern, was man nicht über die Wanderung der Planeten wissen sollte. Astronomie, eine edle und angesehene Kunst, ein Zeichen von Gelehrsamkeit und Wissen; aber Muggel besaßen geheime moderne Artefakte, die das eine Million Milliarden Mal besser konnten, und zwar mit Methoden, die Harry zu erklären versucht hatte und die Draco immer noch nicht ansatzweise verstehen konnte, außer dass es anscheinend nicht einmal Magie brauchte, um Dingen Arithmetik beizubringen.

\emph{Draco sah sich die Bilder von Sternbildern an und fragte sich, ob es in den anderen Häusern auch so war, ob sich die Leute in Ravenclaw immer gegenseitig bedrohten.}

Harry Potter hatte ihm einmal gesagt, dass Soldaten auf einem Schlachtfeld nicht wirklich für ihr Land kämpften. Patriotismus brachte sie vielleicht überhaupt erst auf das Schlachtfeld, aber wenn sie einmal dort waren, kämpften sie, um sich gegenseitig zu beschützen, die Freunde, mit denen sie trainiert hatten und die direkt vor ihnen standen. Und Harry hatte beobachtet, und Draco wusste, dass es stimmte, dass man die Loyalität zu einem Anführer nicht dazu benutzen konnte, einen Patronus-Zauber zu wirken, es war nicht die richtige Art von warmen und glücklichen Gedanken. Aber der Gedanke, jemanden neben sich zu beschützen würde funktionieren - das, hatte Harry Potter nachdenklich gesagt, war wahrscheinlich der Grund, warum die Todesser in dem Moment auseinandergefallen waren, als der Dunkle Lord gefallen war. Sie waren nicht herzlich genug zueinander gewesen.\\ Man konnte eine Gruppe rekrutieren, zu der neben Lord Malfoy und Mr. MacNair auch Bellatrix Black und Amycus Carrow gehörten, und sie mit dem Cruciatus-Fluch auf Linie halten. Aber in dem Moment, in dem der Meister des Dunklen Mal weg war, hatte man keine Armee mehr, sondern einen Bekanntenkreis.

\emph{Deshalb hatte Vater versagt. Es war nicht einmal wirklich seine Schuld gewesen. Es gab nichts, was Vater hätte tun können, nachdem er Todesser geerbt hatte, die nicht wirklich miteinander befreundet waren.}

Und obwohl es das Haus Slytherin war, das er verteidigen sollte - das Haus Slytherin, zu dessen Rettung er und Harry einen Pakt geschlossen hatten -, konnte Draco manchmal nicht anders, als zu denken, dass es einfach weniger ermüdend war, wenn er Armeeübungen leitete. Wenn er mit Schülern aus den anderen drei Häusern arbeitete, die nicht zu Slytherin gehörten.

\emph{Wenn man die Probleme einmal gesehen und benannt hatte, konnte man nicht mehr aufhören, sie zu sehen, es wurde nur jeden Tag nerviger}.

"Mr. Malfoy?", ertönte die Stimme von Gregory Goyle, der in dem kleinen, aber privaten Zimmer neben Dracos Schreibtisch auf dem Boden lag; Gregory machte gerade seine Hausaufgaben in Verwandlung, bei denen er oft Hilfe brauchte. Jede Ablenkung war zu diesem Zeitpunkt willkommen.

"Ja?", sagte Draco.

"Du hast dich gar nicht wirklich gegen Granger verschworen", sagte Gregory. "Stimmt's?"

Das Gefühl, das sich in Dracos Magen ausbreitete, fühlte sich genau so an, wie Gregorys Stimme klang: angewidert und ängstlich.

"Du hast Granger tatsächlich geholfen, an dem Tag, als du sie vom Boden aufhobst", sagte Gregory. "Und vorher, als du sie davor bewahrt hast, vom Dach zu fallen. Du hast einem Schlammblut geholfen -"

"Ja, richtig", sagte Draco sarkastisch, ohne das geringste Zögern oder Zaudern, und blickte wieder auf seine Astronomie-Hausaufgaben hinunter, als wäre er nicht im Geringsten nervös.

\emph{Es geschah alles so, wie Draco es befürchtet hatte, aber das bedeutete zumindest, dass er dieses Gespräch in seinem Kopf immer und immer wieder durchgespielt und sich den richtigen Eröffnungsgag ausgedacht hatte.}

"Komm schon, Gregory, du hast dich mit General Granger duelliert, du weißt, wie stark ihre Zaubersprüche sind. Als ob ein echter Muggel-Sprössling mächtiger wäre als du, mächtiger als Theodore, mächtiger als jeder einzelne Reinblüter in unserem ganzen Schuljahr außer mir? Glaubst du eigentlich an irgendetwas von dem was Vater sagt? Sie ist adoptiert. Ihre Eltern starben im Krieg und jemand steckte sie zu ein paar Muggeln, um sie zu verstecken. Auf keinen Fall ist General Granger ein echtes Schlammblut."

Ein langsamer Puls der Stille durch Dracos Schlafzimmer. Draco wollte es wissen, musste wissen, welcher Ausdruck auf Gregors Gesicht lag. Aber er konnte nicht von seinem Schreibtisch aufschauen, noch nicht, nicht bevor Gregory das Wort ergriff. Und dann -

"Ist es das, was Harry Potter zu dir gesagt hat?", sagte Gregory.\\ Die Stimme schwankte und brach. Als Draco von seinen Hausaufgaben aufblickte, sah er, dass Tränen aus Gregorys Augen liefen. \emph{Offenbar hatte das nicht geklappt}.

"Ich weiß nicht, was ich tun soll", sagte Gregory flüsternd. "Ich weiß nicht, was ich jetzt tun soll, Mr. Malfoy. Dein Vater wird - wenn er es erfährt - nicht begeistert sein, Mr. Malfoy!"

\emph{Es ist nicht deine Aufgabe, zu entscheiden, was Vater gefallen wird, Goyle} - Draco konnte die Worte in seinem Kopf hören; sie klangen in Vaters Stimme, mit der gleichen Strenge. Es war die Art von Dingen, die Vater ihm gesagt hatte, die er sagen sollte, wenn Vincent oder Gregory ihn jemals in Frage stellten; und wenn das nicht funktionierte, sollte er sie verfluchen. \emph{Sie waren keine gleichberechtigten Freunde,} hatte Vater gesagt, \emph{und das sollte er nie vergessen.} \emph{Draco hatte das Sagen, sie waren seine Diener, und wenn Draco sich nicht daran halten konnte, dann war er nicht geeignet, das Haus Malfoy zu erben.}..

"Ist schon gut, Gregory", sagte Draco, so sanft wie er konnte. "Du musst dich nur darum kümmern, mich zu beschützen. Niemand wird dir einen Vorwurf machen, wenn du meine Befehle befolgst, nicht mein Vater und nicht deiner."\\ Er legte so viel Wärme wie möglich in seine Stimme, als wollte er einen Patronus-Zauber wirken.\\ "Und überhaupt, der nächste Krieg wird nicht so sein wie der letzte. Das Haus Malfoy gab es schon lange vor dem Dunklen Lord, und nicht jeder Lord Malfoy macht das Gleiche. Vater weiß das."

"Tut er das?", fragte Gregory mit zitternder Stimme. "Weiß er das wirklich?"

Draco nickte.\\ "Professor Quirrell weiß es auch", sagte Draco. "Darum geht es ja bei den Armeen. Der Verteidigungsprofessor hat recht, wenn der nächste Krieg kommt, wird Vater nicht in der Lage sein, das ganze Land zu vereinen, sie werden sich an den letzten Krieg erinnern. Aber jeder, der in Professor Quirrells Armeen gekämpft hat, wird sich erinnern, wer die stärksten Generäle waren, sie werden wissen, wer würdig ist, sie anzuführen. Sie werden Harry Potter zu ihrem Herrn erklären. Und ich bin seine rechte Hand. Und das Haus Malfoy wird siegen, wie immer. Vielleicht wenden sich die Leute sogar an mich, wenn Potter nicht da ist, solange sie mich für vertrauenswürdig halten. Das ist es, was ich jetzt vorhabe. Vater wird das verstehen."

Gregory griff nach oben, wischte sich über die Augen und sah wieder auf seine Verwandlungs-Hausaufgaben hinunter.\\ "In Ordnung", sagte Gregory mit zittriger Stimme. "Wenn du es sagst, Mr. Malfoy."

Draco nickte wieder, ignorierte das hohle Gefühl in seinem Inneren über die Lügen, die er seinem Freund gerade erzählt hatte, und wandte sich wieder den Sternen zu.

\textbf{Nachspiel: Hermine Granger und -}\\ Unsichtbar zu sein hätte interessanter sein sollen als das hier, die Korridore von Hogwarts hätten in seltsamen Farben umrissen werden sollen oder so. Aber eigentlich, dachte Hermine, war es unter Harrys Unsichtbarkeitsumhang genau so, als wäre man nicht unter einem Unsichtbarkeitsumhang, bis auf den Teil mit dem Umhang. Wenn man den Schleier aus weichem schwarzem Stoff von der Kapuze herunter und über das Gesicht zog, konnte man nicht einmal sehen, wie er sich vor einem ausbreitete, und auch danach schien er das Atmen nicht zu behindern. Und die Welt sah genauso aus, nur dass man, wenn man an Dingen aus Metall vorbeiging, keine kleinen Spiegelungen von sich selbst sah. Porträts sahen einen nie an, sondern taten nur, was auch immer für seltsame Dinge sie taten, wenn sie allein waren. Hermine hatte noch nicht versucht, an einem Spiegel vorbeizugehen, sie war sich nicht sicher, ob sie das wollte. Vor allem aber gab es kein Du mehr, wenn man herumlief, keine Hände, keine Füße, nur einen wechselnden Blickwinkel. Es war ein beunruhigendes Gefühl, nicht so sehr, unsichtbar zu sein, als vielmehr nicht zu existieren.

Harry hatte sie überhaupt nicht befragt, sie hatte nur das Wort "Unsichtbarkeits-" herausgebracht und dann zog Harry seinen Unsichtbarkeitsumhang aus seinem Beutel. Man hatte ihr nicht einmal die Chance gegeben, von ihrem äußerst geheimen Treffen mit Daphne und Millicent Bulstrode zu erzählen, oder dass sie dachte, es würde helfen, die anderen Mädchen zu schützen, Harry hatte ihr einfach etwas übergeben, was wahrscheinlich ein Heiligtum des Todes war.

Wenn man fair war, und sie versuchte, fair zu sein, musste sie zugeben, dass Harry manchmal ein sehr echter, wahrer Freund sein konnte. Das geheime Treffen selbst war ein großer Fehlschlag gewesen. Millicent hatte behauptet, eine Seherin zu sein. Hermine hatte Millicent und Daphne sorgfältig und ausführlich erklärt, dass dies unmöglich wahr sein konnte. Sie und Harry hatten schon früh in ihren Nachforschungen über Wahrsagerei nachgeschlagen; Harry hatte darauf bestanden, dass sie alles über Prophezeiungen lesen sollten, was nicht in der verbotenen Abteilung stand. Wie Harry bemerkt hatte, würde es eine Menge Mühe ersparen, wenn sie einfach einen Seher dazu bringen könnten, alles zu prophezeien, was sie fünfunddreißig Jahre später herausfinden würden.

(Oder, um es in Harrys Worten auszudrücken, jede Möglichkeit, Informationen zu erhalten, die aus der fernen Zukunft übermittelt wurden, war potenziell eine sofortige globale Siegbedingung.)

Aber, wie Hermine Millicent erklärt hatte, war das Prophezeien nicht kontrollierbar, es gab keine Möglichkeit, um eine Prophezeiung über etwas Bestimmtes zu bitten. Stattdessen (so stand es in den Büchern) gab es eine Art Druck, der sich in der Zeit aufbaute, wenn irgendein großes Ereignis eintreten wollte oder sich selbst davon abhalten wollte. Und Seher waren wie Schwachstellen, die den Druck abließen, wenn der richtige Zuhörer in der Nähe war. Prophezeiungen betrafen also nur große, wichtige Dinge, denn nur das erzeugte genug Druck; und man bekam fast nie mehr als einen Seher, der das Gleiche sagte, denn danach war der Druck weg. Und, wie Hermine Millicent weiter erklärt hatte, erinnerten sich die Seher selbst nicht an ihre Prophezeiungen, weil die Botschaft nicht für sie bestimmt war. Und die Botschaften würden in Rätseln herauskommen, und nur jemand, der die Prophezeiung mit der Originalstimme des Sehers hörte, würde die ganze Bedeutung, die in dem Rätsel steckte, verstehen. Es gab keine Möglichkeit, dass Millicent einfach so eine Prophezeiung über Schultyrannen aussprechen konnte, wann immer sie wollte, und dass sie sich dann daran erinnerte, und wenn sie es getan hätte, wäre es als \emph{"das Skelett ist der Schlüssel"} herausgekommen und nicht als \emph{"Susan Bones muss dort sein"}.

Millicent hatte zu diesem Zeitpunkt ziemlich verängstigt ausgesehen, also hatte Hermine ihre Fäuste dort entspannt, wo sie in ihre Hüften geklemmt waren, sich selbst beruhigt und vorsichtig erklärt, dass sie froh war, dass Millicent ihnen geholfen hatte, aber sie waren manchmal in Fallen getappt, wenn sie dem gefolgt waren, was Millicent gesagt hatte, und deshalb wollte Hermine wirklich wissen, woher die Botschaften tatsächlich kamen. Und Millicent hatte mit leiser Stimme gesagt: \emph{Aber, aber sie hat mir gesagt, dass sie eine Seherin ist…}

Hermine hatte Daphne gesagt, sie solle keinen Druck machen, nachdem Millicent sich geweigert hatte, ihre Quelle preiszugeben. Es war nicht nur so, dass Hermine sich schrecklich gefühlt hatte wegen des verängstigten Blicks auf Millicents Gesicht. Es war, dass Hermine das starke Gefühl hatte, dass, wenn sie die Person finden würden, die Millicent Dinge erzählt hatte, sich herausstellen würde, dass sie am Morgen nur Umschläge unter ihrem Kopfkissen finden würden. Sie bekam dasselbe verzweifelte Gefühl, das sie in der Schlacht vor Weihnachten bekommen hatte, als sie Zabinis Diagramm mit all den farbigen Linien und Kästchen betrachtete und… und sie hatte erst jetzt begriffen, was es bedeutete, dass Zabini derjenige gewesen war, der ihr dieses Diagramm gezeigt hatte.

\emph{Sogar für eine Ravenclaw,} dachte sie, \emph{gab es so etwas wie ein übermäßig kompliziertes Leben}.

Hermine begann, eine kurze Spirale aus gelben Marmorstufen hinaufzusteigen, die von einem zentralen Rückgrat ausgingen, eine schlecht gehütete "\emph{geheime}" Treppe, die in Wirklichkeit einer der schnellsten Wege von den Slytherin-Kerkern zum Ravenclaw-Turm hinauf war, die aber nur Hexen durchqueren konnten.

(Warum gerade Mädchen einen schnellen Weg brauchten, um von Ravenclaw nach Slytherin und zurück zu gelangen, war Hermine ein wenig rätselhaft.)

Am oberen Ende der Treppe, jetzt, wo sie von den Slytherin-Kerkern weg und zurück in den Hauptteil von Hogwarts war, blieb Hermine stehen und nahm Harrys Unsichtbarkeitsumhang ab. Nachdem ihr Beutel den Umhang verschluckt hatte, wandte sich Hermine nach rechts und begann, einen kurzen Gang hinunterzugehen, wobei sie nun automatisch in alle Richtungen Ausschau hielt, ohne wirklich darüber nachzudenken, und ihre ständig suchenden Augen blickten in eine schattige Nische -

\emph{(flüchtige Desorientierung)}

- und dann traf sie ein Ansturm von Schock und Angst wie ein betäubender Fluch an ihren ganzen Körper, sie stellte fest, dass ohne jeden Gedanken oder irgendeine bewusste Entscheidung ihr Zauberstab in ihre Hand gesprungen war und bereits auf etwas gerichtet war -

… einen schwarzen Mantel, der so weit und wogend war, dass es unmöglich war, festzustellen, ob die Gestalt darunter männlich oder weiblich war, und oben auf dem Mantel einen breitkrempigen schwarzen Hut; und ein schwarzer Nebel schien sich darunter zu sammeln und das Gesicht von wem oder was auch immer darunter liegen mochte, zu verdecken.

"Hallo noch mal, Hermine", flüsterte eine zischende Stimme unter dem schwarzen Hut, hinter dem schwarzen Nebel.

Hermines Herz hämmerte bereits gewaltig in ihrer Brust, ihr Hexengewand fühlte sich bereits schweißnass auf ihrer Haut an, ein Geschmack von Angst lag bereits in ihrem Mund; sie wusste nicht, warum sie so plötzlich von Adrenalin erfüllt war, aber ihre Hand griff fester nach ihrem Zauberstab.

"Wer sind Sie?" forderte Hermine.

Der Hut neigte sich leicht; die flüsternde Stimme, als sie aus dem schwarzen Nebel hervortrat, klang trocken wie Staub.\\ "Der letzte Verbündete", sprach das zischende Flüstern. "Derjenige, der endlich antwortet, wenn kein anderer dir antworten will. Ich bin vielleicht der einzige wahre Freund, den du in ganz Hogwarts hast, Hermine. Denn du hast nun gesehen, wie die anderen geschwiegen haben, als du in Not warst -"

"Wie ist dein Name?"

Der schwarze Umhang drehte sich leicht hin und her, es sah nicht wie ein Schulterzucken aus, aber es vermittelte ein Zucken.\\ „Das ist das Rätsel, junger Ravenclaw \emph{(„This is the Riddle“, Anm. des Übersetzers)}. Bis du es gelöst hast, darfst du mich nennen, wie du willst."

Sie spürte, wie ihre Handfläche bereits schwitzte, und war dankbar für die spiralförmigen Rillen auf ihrem Zauberstab, die ihrer Hand halfen, einen festen Griff auf dem Holz zu behalten.\\ "Nun, Mister Unglaublich Verdächtige Person", sagte Hermine, "was wollen Sie von mir?"

"Das ist die falsche Frage", kam es flüsternd aus dem schwarzen Nebel. "Du solltest eher fragen, was ich dir anbieten kann."

"Nein", sagte das junge Mädchen ganz ruhig, "ich glaube, das sollte ich eigentlich nicht fragen."

Ein hochtöniges Glucksen hinter dem schwarzen Nebel.\\ "Nicht Macht", flüsterte die Stimme, "nicht Reichtum, um solche Dinge scherst du dich wenig, nicht wahr, junge Ravenclaw? Wissen. Das ist es, was ich besitze. Ich weiß, was sich in dieser Schule abspielt, all die verborgenen Pläne und Akteure, die Antworten auf das Rätsel. Ich kenne den wahren Grund für die Kälte, die du in Harry Potters Augen siehst. Ich kenne die wahre Natur von Professor Quirrells mysteriöser Krankheit. Ich weiß, wen Dumbledore wirklich fürchtet."

"Schön für dich", sagte Hermine Granger. "Aber weißt du, wie oft man an einem Lutscher lecken muss bis er weg ist?"

Der schwarze Nebel schien sich leicht zu verdunkeln, die Stimme klang tiefer, als er sprach, enttäuscht.\\ "Du bist also nicht einmal neugierig, junger Ravenclaw, auf die Wahrheiten hinter den Lügen?"

"Einhundertsiebenundachtzig", sagte sie. "Ich habe es einmal versucht, und so viele sind es gewesen."\\ Ihre Hand rutschte fast auf ihrem Zauberstab ab, \emph{es war ein Gefühl der Ermüdung in ihren Fingern, als hätte sie den Zauberstab schon seit Stunden statt Minuten gehalten}

- Die Stimme zischte: "Professor Snape ist heimlich ein Todesser."

Hermine ließ fast ihren Zauberstab fallen.

"Ah", flüsterte die Stimme befriedigt. "Ich dachte, das würde dich interessieren. Also, Hermine. Gibt es sonst noch etwas, das du über deine Feinde wissen möchtest, oder über die, die du Freunde nennst?"

Sie starrte hinauf in den schwarzen Nebel, der den hoch aufragenden schwarzen Umhang umhüllte, und versuchte krampfhaft, ihre Gedanken zu ordnen.

\emph{Professor Snape war ein Todesser? Wer würde ihr so etwas sagen, warum, was war hier los?}\\ "Das ist -" sagte Hermine. Ihre Stimme war zittrig. "Das ist eine sehr ernste Angelegenheit, wenn es wirklich wahr ist. Warum erzählst du so etwas mir und nicht dem Schulleiter Dumbledore?"

"Dumbledore hat nichts getan, um Snape aufzuhalten", flüsterte der schwarze Nebel. "Du hast es gesehen, Hermine. Die Fäulnis in Hogwarts beginnt an der Spitze. Alles, was an dieser Schule falsch ist, beginnt mit dem verrückten Schulleiter. Du allein hast es gewagt, ihn darauf anzusprechen - und deshalb spreche ich zu dir."

"Und hast du dann auch mit Harry Potter gesprochen?" sagte Hermine und hielt ihre Stimme so gleichmäßig, wie sie konnte. \emph{Wenn das sein hilfreicher Geist war} -

Der schwarze Nebel verdunkelte und lichtete sich, wie ein Kopfschütteln.\\ "Ich habe Angst vor Harry Potter", flüsterte es. "Vor der Kälte in seinen Augen, vor der Dunkelheit, die hinter ihnen wächst. Harry Potter ist ein Mörder, und jeder, der sich ihm in den Weg stellt, wird sterben. Selbst du, Hermine Granger, wenn du es wagst, dich ihm wirklich entgegenzustellen, wird die Dunkelheit hinter seinen Augen nach dir greifen und dich vernichten. Das weiß ich."

"Dann weißt du nicht einmal die Hälfte von dem, was du zu wissen vorgibst", sagte Hermine, ihre Stimme wurde etwas fester. "Ich habe auch Angst vor Harry. Aber nicht vor dem, was er mir jemals antun könnte. Ich habe Angst davor, was er tun könnte, um mich zu beschützen -"

"Falsch." Das Flüstern war flach und hart, als ob es keine Möglichkeit des Leugnens geben sollte. "Harry Potter wird sich mit der Zeit gegen dich wenden, Hermine, wenn die Dunkelheit ihn ganz einnimmt. Er wird keine Träne vergießen, er wird es nicht einmal bemerken, an dem Tag, an dem seine Schritte dich endgültig unter sich zermalmen."

"Doppelt falsch!", gab sie mit erhobener Stimme zurück, obwohl ihr ein Schauer über den Rücken lief. Einer von Harrys Sätzen kam ihr in den Sinn.\\ "Was glaubst du zu wissen, und woher glaubst du es überhaupt zu wissen?"

"Zeit -" Die Stimme schien sich zu fangen. "Dafür ist später noch genug Zeit. Für jetzt, für heute, ist Harry Potter tatsächlich nicht dein Feind. Und doch bist du in größter Gefahr."

"Das kann ich glauben", sagte Hermine Granger.\\ Verzweifelt wollte sie ihren Zauberstab in die andere Hand legen, \emph{sie hatte das Gefühl, ihren rechten Arm festhalten zu müssen, nur um ihn oben zu halten, ihr Kopf schmerzte, als hätte sie tagelang in den schwarzen Nebel gestarrt; sie wusste nicht, warum sie so schnell müde geworden war.}

"Lucius Malfoy ist auf dich aufmerksam geworden, Hermine."\\ Das Flüstern war lauter geworden, hatte sich von seiner Tonlosigkeit gelöst und einen Ton hörbarer Besorgnis angenommen.\\ "Du hast das Haus Slytherin gedemütigt, du hast seinen Sohn im Kampf besiegt. Schon vorher warst du eine Blamage für alle, die auf der Seite der Todesser stehen; denn du bist eine Muggelgeborene und besitzt doch eine Zauberkraft, die größer ist als die jedes Reinbluts. Und nun wirst du bekannt, die Augen der Welt sind auf dich gerichtet. Lucius Malfoy will dich vernichten, Hermine, dir wehtun und dich vielleicht sogar töten, und er hat die Mittel dazu!"\\ Das Flüstern war eindringlich geworden.

Es gab eine Pause.

"Ist das alles?" sagte Hermine.\\ Wenn sie Zabini oder Harry Potter wäre, würde sie wahrscheinlich kluge Fragen stellen, um mehr Informationen zu sammeln; aber ihr Verstand fühlte sich langsam und müde an. \emph{Sie musste wirklich von hier weg und sich eine Weile hinlegen.}

"Du glaubst mir nicht", sagte das Flüstern, jetzt weicher und trauriger.\\ "Warum nicht, Hermine? Ich versuche doch, dir zu helfen."

Hermine machte einen Schritt rückwärts, weg von der schattigen Nische.

"Warum nicht, Hermine?!", fragte die Stimme, die sich zu einem Zischen steigerte. "So viel bist du mir schuldig! Sag es mir, und dann -"\\ Die Stimme verfing sich und wurde wieder leiser.\\ "Und dann kannst du gehen, nehme ich an. Sag mir nur - warum -"

Vielleicht hätte sie nicht antworten sollen; vielleicht hätte sie sich einfach umdrehen und fliehen sollen, oder noch besser, erst eine Schildwand zaubern und dann lauthals schreien, während sie rannte; aber es war der Ton von echtem Schmerz in der Stimme, der sie ergriff, und so antwortete sie.\\ "Weil du unglaublich düster und unheimlich und verdächtig aussiehst", sagte Hermine mit höflicher Stimme, während ihr Zauberstab auf den hoch aufragenden schwarzen Mantel und den gesichtslosen schwarzen Nebel gerichtet blieb.

"Das ist alles?!", flüsterte die Stimme ungläubig.\\ Traurigkeit schien sie zu durchdringen.\\ "Ich hatte mir mehr von dir erhofft, Hermine. Sicherlich weiß eine Ravenclaw wie du, die intelligenteste Ravenclaw, die Hogwarts seit einer Generation ziert, dass der Schein trügen kann."

"Oh, das weiß ich", sagte Hermine.\\ Sie trat einen weiteren Schritt zurück, ihre müden Finger verkrampften sich um den Zauberstab.\\ "Aber was die Leute manchmal vergessen, ist, dass, obwohl der Schein trügen kann, er es normalerweise nicht tut."

Es gab eine Pause.

"Du bist die Kluge", sagte die Stimme, und der schwarze Nebel verflüchtigte sich, verdunkelte nicht länger\emph{; sie sah das Gesicht darunter, und das Wiedererkennen schickte einen Schock von erschrecktem Adrenalin durch sie hindurch} -

\emph{(flüchtige Desorientierung)}\\ \emph{\hfill\break } - und dann traf sie ein Ansturm von Schock und Angst wie ein betäubender Fluch an ihren ganzen Körper, sie stellte fest, dass ohne jeden Gedanken oder irgendeine bewusste Entscheidung ihr Zauberstab in ihre Hand gesprungen war und bereits auf etwas gerichtet war -

… eine leuchtende Dame, deren langes weißes Kleid wie von unsichtbaren Winden umweht war; weder ihre Hände noch ihre Füße waren zu sehen, ihr Gesicht war unter einem weißen Schleier verborgen; und sie glühte am ganzen Körper, nicht wie ein Geist, nicht durchsichtig, nur von sanftem weißen Licht umgeben.

Hermine starrte mit offenem Mund auf den sanften Anblick und fragte sich, warum ihr Herz bereits hämmerte und warum sie sich so verängstigt fühlte.

"Hallo noch mal, Hermine", flüsterte es freundlich aus dem weißen Schein hinter dem Schleier. "Ich bin geschickt worden, um dir zu helfen, also hab bitte keine Angst. Ich bin dein Diener in allen Dingen; denn du, Mylady, bist die Trägerin eines höchst wunderbaren Schicksals -"

.\\ ..\\ …\\ … …\\ … … …\\ \emph{(flüchtige Desorientierung)}\\ \emph{… … …}\\ \emph{… …}\\ \emph{…}\\ \emph{..}\\ \emph{.}

