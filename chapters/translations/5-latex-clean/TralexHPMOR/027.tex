

\hypertarget{empathie}{% \section{6. Empathie}\label{empathie}}

\emph{Nach einer langen Pause möchte ich das Projekt wieder einmal weiterführen. Dabei gibt es zwei Ankündigungen. Erstens ich habe mir die Unterstützung einer "Lektorin„ gesucht, die sich doch einiges besser in der deutschen Sprache ausdrücken kann. Außerdem möchte ich das Projekt etwas umstrukturieren, also zunächst jene Teile übersetzen die zur eigentlich Haupthandlung gehören, sodass sie Chance steigt, die Geschichte jemals fertigzustellen. Über Unterstützung bei Übersetzung bzw. Hörbuch (Youtube) würde ich mich natürlich freuen; Schreibt in diesem Fall doch etwas auf diesem Discord \url{https://discord.gg/wV5J9xu} ~}

Kapitel 26

Empathie

Man konnte nicht jeden Tag Harry Potter flehen sehen.

„\emph{Biiitte,“ winselte Harry Potter.}

Fred und George schüttelten lächelnd ihre Köpfe.

Harrys Gesicht zeigte einen gequälten Ausdruck. „Aber ich habe euch \emph{erzählt, was ich mit Kevin Entwhistles Katze gemacht habe, und mit Hermine und der verschwundenen Limo. Und ich \emph{kann} euch \emph{nichts} vom Sprechenden Hut oder dem Erinnermich oder Professor Snape…“}

Fred und George zuckten mit den Schultern und wandten sich ab.

„Falls du es jemals herausfindest,“ sagte einer der Weasley-Zwillinge, „lass es uns wissen.“

„\emph{Ihr seid böse! Ihr seid beide böse!“}

Fred und George schlossen energisch die Tür zum leeren Klassenzimmer hinter sich. Dabei hielten ihr Grinsen noch eine Weile aufrecht. Nur für den Fall, das Harry durch Türen sehen konnte…

Dann gingen sie um eine Ecke - und ihre Gesichtszüge entglitten.

„Ich gehe davon aus, dass auch dir Harrys Vermutungen -------- nicht geholfen haben?“ fragten sie beide gleichzeitig, und dann blickten sie noch ein wenig enttäuschter drein.

Ihre letzte relevante Erinnerung war von Flume. Er hatte es abgelehnt, ihnen zu helfen. Aber \emph{was hatten sie bloß von ihm verlangt? Daran erinnerten sie sich einfach nicht.}

Doch sie mussten wohl woanders geschaut und \emph{\emph{irgendjemand} gefunden haben, der ihnen dabei geholfen hatte \emph{etwas} Illegales zu machen. Sonst hätten sie niemals zugestimmt, ihre Erinnerungen gelöscht zu bekommen.}

Und dann wäre da noch die wichtigste Frage: Wie hatten sie \emph{jemals in der Lage sein können, all das mit gerade mal vierzig Galleonen zu bewerkstelligen?}

Das Einzige was sie jetzt noch tröstete war der Umstand, dass Harry nicht wusste, dass sie es nicht wussten.

Aber zurück zu den aktuellen Fakten. Zunächst hatten sie befürchtet, sie wären derart geschickt vorgegangen, dass Harry schließlich \emph{wirklich Ginny heiraten würde. Aber darum hatten sie sich anscheinend auch gekümmert. Die Berichte des Zauberergamots waren nochmalszurück zum ursprünglichen Zustand verändert worden. Damit war der gefälschte Verlobungsvertrag aus dem mit Drachen beschützen Verlies in Gringotts verschwunden, und so weiter und so fort.}

Alles in allem war es eigentlich ziemlich unheimlich. Viele Menschen glaubten inzwischen, der \emph{Tagesprophet hätte sich aus unbekannten Gründen alles nur ausgedacht und der Klitterer wäre dann auf diesen Zug aufgesprungen. Mit der Überschrift „HARRY POTTER HEIMLICH VERLOBT MIT LUNA LOVEGOOD“hätte er am nächsten Tag den Finger in die Wunde gelegt.}

Wenn auch immer sie beauftragt hatten, er würde ihnen, so hofften sie verzweifelt, schließlich davon erzählen, sobald die Verjährungsfrist abgelaufen war. Aber bis dahin war es kaum zu ertragen. Sie hatten ihren besten Streich aller Zeiten gespielt, vielleicht der beste Streich in der Geschichte des Streichspielens, und konnten \emph{sich nicht erinnern wie. Sie waren doch beim ersten Mal darauf gekommen, wieso konnten sie es jetzt nicht mehr erkennen, obwohl sie sogar alles wussten, was sie getan hatten? Es war einfach verrückt!}

Sie beschlossen, noch einmal alle Personen durchzugehen, die etwas von dem Streich wissen konnten.

Da war zunächst ihre Mutter. Nicht einmal sie hatte die Zwillinge nach den Vorfällen gefragt, trotz eindeutigen Verbindungen zu den Weasleys.\\ Doch was auch getan worden war, es war weit außerhalb der Reichweite eines Hogwarts Schülers… außer vielleicht \emph{einem, der, solange ein paar Gerüchte stimmten, dies bewerkstelligt haben könnte indem er mit den Fingern schnipste. \emph{Harry} war unter dem Einfluss von Veritaserum danach gefragt worden, das hatte er ihnen erzählt… und zwar mit Dumbledore an seiner Seite, der den Auroren furchteinflössende Blicke zugeworfen hatte. Die Auroren hatten Harry gerade genug gefragt, um sicherzustellen, dass er den Streich nicht selbst ausgeführt oder dafür gesorgt hatte, dass jemand verschwand und waren dann schleunigst aus Hogwarts geflohen.}

Fred und George fragten sich, ob sie sich beleidigt fühlen sollten, weil Harry für \emph{ihren} Streich von Auroren befragt wurde. Aber allein für Harrys Blick, vermutlich aus genau jenem Grund, hatte es sich gelohnt.

Wenig überraschend waren Rita Kimmkorn und der Redakteur des Tagespropheten verschwunden und vermutlich inzwischen in einem anderen Land. Sie \emph{hätten ihrer Familie gerne von diesem Teil erzählt. Dad hätte ihnen wahrscheinlich gratuliert, nachdem Mum sie umgebracht und Ginny ihre Überreste verbrannt hatte.}

Aber alles war noch in Ordnung, sie würden es Dad eines Tages erzählen, und bis dahin…

… bis dahin hatte es sich zutragen, dass Dumbledore nieste, als er in den Gängen an ihnen vorbeiging. Dabei war ein kleines Paket unabsichtlich aus einer seiner Taschen gefallen und darin waren zwei Fluchbrecher-Monokel von \emph{unglaublicher} Qualität. Die Weasley-Zwillinge hatten ihre neuen Monokel am „Verbotenen“ Korridor im dritten Stock ausprobiert, mit einer kurze Tour zum Spiegel und wieder zurück. Der Effekt war mehr als zufriedenstellend. Sie hatten zwar nicht alle Aufspürnetze sehen können, aber die Monokel hatte ihnen \emph{deutlich} mehr gezeigt als beim ersten Mal.

Natürlich mussten sie sehr vorsichtig sein, um nicht mit den Monokeln in ihrem Besitz gefasst zu werden. Sonst würden sie im Büro des Schulleiters landen und seine strenge Rede anhören müssen oder vielleicht sogar mit dem Rauswurf bedroht werden.

Es war gut zu wissen, dass nicht jeder, der ins Haus Gryffindor sortiert wurde, sich in Professor McGonagall verwandelte.

-\/-\/-

Harry saß in einem weißen, fensterlosen, konturlosen Raum vor einem Tisch, einem ausdruckslosen Mann in formellen, tiefschwarzen Umhängen gegenüber.

Der Raum war gegen Aufspürung geschützt, und der Mann hatte exakt siebenundzwanzig Zauber ausgeführt, bevor er auch nur so viel wie „Hallo, Mr. Potter.“ sagte.

Es war merkwürdig angemessen, dass der Mann in Schwarz bald versuchen würde, Harrys Gedanken zu lesen.

„Bereiten Sie sich vor,“ sagte der Mann tonlos.

Ein menschlicher Verstand, dass hatte Harrys Okklumentikbuch gesagt, war einem Legilimens nur auf bestimmten \emph{Oberflächen ausgesetzt. Falls jemand daran scheiterte, seine Oberflächen zu verteidigen, würde ein Legilimens \emph{durch} sie schreiten und auf jeden Teile zugreifen können, den sein eigener Verstand verstehen konnte…}

… was üblicherweise nicht besonders viel war. Anscheinend war der menschliche Verstand außerhalb der oberflächlichsten Ebene für Menschen nur schwer verständlich. Harry hatte sich gefragt, ob seine umfangreichen Kenntnisse der Kognitionswissenschaft ihn zu einem unglaublich mächtigen Legilimens machen würden. Doch die wiederholte Erfahrung hatte in \emph{schließlich} gelehrt, dass er sich in Anbetracht solcher Dinge ein wenig weniger aufregen sollte. Es war ja nicht so, dass irgendein Kognitionswissenschaftler die Menschen so gut genug verstand, um einen zu bauen.

Der erste Schritt, um Okklumentik zu lernen, bestand darin, sich vorzustellen, eine andere Person zu sein. Dabei musste man sich so fest wie möglich auf diese Aufgabe konzentrieren, um völlig in die alternative Rolle einzutauchen. Das war nicht immer und nicht auf Dauer nötig, aber zu Beginn war es unverzichtbar, um seine Oberflächen kennen zu lernen. Der Legilimens würde versuchen, die Gedanken zu lesen und man konnte es spüren, wie er einzudringen versuchte, wenn man gut genug aufpasste. Und die Aufgabe des Okklumens war es, sicherzustellen, dass stets nicht die echte, sondern nur die vorgestellte Person berührt wurde.

Nachdem man gut genug darin war, konnte man sich vorstellen, eine sehr \emph{einfache} Art von Person zu sein, zum Beispiel ein Stein, und sich daran gewöhnen, die Täuschung an der Stelle der Oberflächen unbegrenzt zu erhalten. Das war eine Standard-Okklumentik-Barriere. Sich vorzustellen, ein Stein zu sein, war zwar schwer zu erlernen, aber danach leicht aufrechtzuerhalten und die enthüllte Oberfläche des Verstandes war sehr viel einfältiger als das Innere, sodass man die Barriere mit genügend Übung im Hintergrund fortführen konnte ohne ihr bewusst Aufmerksamkeit zu widmen.

Oder, im Falle eines perfekten Okklumens, hatte man noch eine andere Option. Dann konnte man an den Sondierungen \emph{vorbei}ziehen und Anfragen in dem Augenblick beantworten, in dem sie gefragt wurden. Wenn der Legilimens dann die Oberfläche durchdringen würde, würde er einen Verstand erblicken, der von der vorgegebenen Person nicht zu unterscheiden war.

Sogar die besten Legilimens konnten so getäuscht werden. Falls ein perfekter Okklumens behauptete, er habe seine Okkumentik-Barriere fallengelassen, gab es keine Möglichkeit, das zu überprüfen. Und noch schlimmer war es natürlich, wenn man nicht einmal wusste, dass man es mit einem perfekten Okklumens zu tun hatte. Sie waren selten, aber allein der Fakt, dass sie existierten, war essentiell. Man konnte auf Legilimentik bei \emph{niemandem} vertrauen.

Es war eine traurige Tatsache, wie wenig menschliche Wesen einander verstanden und wie wenig wiederum die Zauberer von den Tiefen begriffen, die unter der Oberfläche des Verstandes lagen. Dennoch konnte man den besten menschlichen Telepathen täuschen, indem man vorgab, jemand anderes zu sein.

Allerdings verstanden sich menschliche Wesen ohnehin nur mithilfe von Täuschungen. Man stellte keine Vorhersagen über Menschen an, indem man die hundert Trillionen Synapsen in ihrem Gehirn als separate Objekt modellierte. Trug man dem besten Gedankenleser der Welt auf, eine künstliche Intelligenz von Grund auf zu errichten, würde dieser nur einen dummen Blick zurückgeben. Man verfasste Vorhersagen über Menschen, indem man sein Gehirn anwies, genauso zu handeln wie ihres. \emph{Man versetzte sich in ihre Lage hinein.} Falls man wissen wollte, was eine wütende Person machen würde, aktivierte man im eigenen Gehirn den Wutschaltkreis und machte eine entsprechende Vorhersage. Wie der neurale Schaltkreis für Wut tatsächlich innen aussah? Wer konnte das schon sagen? Der beste Gedankenleser der Welt wüsste vielleicht nicht einmal was Neuronen \emph{waren} \emph{-}und genauso wenig wusste es der beste Legilimens.

Alles was der Legilimens \emph{verstehen} konnte, konnte ein Okklumens \emph{vortäuschen.} Es war der gleiche Trick in beiden Fällen ---- wahrscheinlich von den gleichen neuralen Schaltkreisen implementiert, eine bestimmte Menge an Kontrollschaltkreisen zum Rekonfigurieren des eigenen Gehirns um als das Model eines Anderen zu handeln.

Und so war der Wettstreit zwischen telepathischem Angriff und telepathischer Verteidigung ein entschiedener Sieg für die Verteidigung gewesen. Andernfalls wäre die gesamte Zaubererwelt, möglicherweise sogar die gesamte Welt eine ganz andere gewesen…

Harry holte tief Luft und konzentrierte sich. Es zeigte sich ein leichtes Lächeln auf seinen Zügen. Dieses Mal, \emph{nur dieses eine Mal}, war Harry im Bereich mysteriöser Begabungen nicht übergangen worden.

Nach fast einem Monat mit Übung, und dabei mehr aus einer Laune als aus einer Ahnung heraus, hatte er beschlossen sich in den kalt-zornigen Zustand zurückzuversetzen, und dann die Okklumentik-Lektionen wieder zu versuchen. Zu diesem Zeitpunkt hatte er es größtenteils aufgeben sich solche Fähigkeiten zu erhoffen, aber einen Versuch sollte es wert sein ----

Er war die schwersten Aufgaben innerhalb von zwei Stunden durchgegangen und am nächsten Tag hatte er Professor Quirrell aufgesucht und ihm erklärt, er wäre bereit.

Seine dunkle Seite war, so hatte es sich gezeigt, \emph{äußerst} gut darin vorzugeben jemand Anderes zu sein.

Harry erinnerte sich an seinen Standard-Auslöser, vom ersten Mal als er völlig zu seiner dunklen Seite gewechselt war…

Severus hielt inne und sah sehr zufrieden mit sich aus. „Das macht dann … fünf Punkte? Nein, sagen wir zehn Punkte Abzug von Ravenclaw wegen Widerworten.“

Harrys Lächeln wurde kälter, und er betrachtete den schwarz-gewandeten Mann, der glaubte bald Harrys Gedanken zu lesen.

Und dann verwandelte Harry sich in jemand völlig anderen, jemand der ihm angemessen für den Anlass erschien.

… in einem weißen Raum, fensterlos, konturlos, vor einem Tisch sitzend, einem ausdruckslosen Mann in formellen Umhängen von tiefem schwarz gegenüber.

Kimball Kinnison betrachtete den Schwarz-Gewandeten, der glaubte bald die Gedanken eines Lensman der zweiten Stufe der galaktischen Patrouille zu lesen.

Zu sagen, dass Kimball Kinnison darauf vertraute das Ergebnis zu kennen, war eine Untertreibung. Er war vom Mentor von Arisia trainiert worden, dem mächtigsten Verstand dieses Universums, sowie allen anderen Universen, und der einfache Zauberer, der ihm gegenüber saß, würde genau das sehen, was der Graue Lensman wollte…

..den Verstand des Jungen, für den er sich zurzeit ausgab, ein unschuldiges Kind namens Harry Potter.

„Ich bin bereit,“ sagte Kimball Kinnison mit einem nervösen Unterton, der genau passend für einen elfjährigen Jungen war.

„\emph{Legilimens,“ sagte der schwarz-gewandeten Zauberer.}

Es entstand eine Pause.

Der schwarz-gewandete Zauberer blinzelte, als ob es etwas derart Schockierendes gesehen hatte, dass ausreichte \emph{seine Lider zu bewegen. Seine Stimme war nicht mehr so tonlos wie zuvor, als er fragte: „Der-Junge-Der-Überlebte hat \emph{eine mysteriöse dunkle Seite?“}}

Die Hitze stieg langsam in Harrys Wangen.

„Also,“ sagte der Mann. Sein Gesichtsausdruck war nun wieder zu perfekter Ruhe zurückgekehrt. „Entschuldigung. Mr. Potter, es ist gut ihre Vorzüge zu kennen, aber das ist nicht das Gleiche wie völlig vermessen auf sie zu vertrauen. Sie könnten tatsächlich in der Lage sein Okklumentik mit elf Jahren zu lernen. Das verblüfft mich. Ich hatte angenommen, Dumbledore würde wieder vorgeben verrückt zu sein. Ihr dissoziatives Talent ist dermaßen stark, dass ich überrascht bin keine sonstigen Zeichen von Kindesmisshandlung zu finden, und Sie könnten mit genügend Zeit ein perfekter Okklumens werden. Aber da bleibt noch ein gewaltiger Unterschied zwischen dem und der Erwartung eine erfolgreiche Okklumentik-Barriere beim ersten Versuch zu errichten. Das ist schlicht lächerlich. Haben Sie etwas gespürt als ich Ihre Gedanken gelesen habe?“

Harry schüttelte seinen Kopf, feuerrot anlaufend.

„Dann achten Sie beim nächsten Mal besser darauf. Das Ziel ist es nicht eine perfektes Abbild an ihrem ersten Unterrichtstag zu erstellen. Das Ziel ist es zu lernen, wo ihre Oberflächen liegen. Bereiten Sie sich vor.“

Harry versuchte wieder vorzugeben Kimball Kinnison zu sein, versuchte besser Acht zu geben, aber seine Gedanken waren ein wenig zerstreut und er war sich plötzlich aller derer Dinge bewusst, über die er nicht nachdenken sollte…

Oh, das würde unangenehm werden.

Harry biss die Zähne zusammen. Zumindest würde auf den Ausbilder danach ein Gedächtniszauber gewirkt werden.

„\emph{Legilimens.“}

Es entstand eine Pause ----

… in einem weißen Raum, fensterlos, konturlos, vor einem Tisch sitzend, einem ausdruckslosen Mann in formellen Umhängen von tiefem schwarz gegenüber.

Es war ihr vierter Tag, an einem Sonntagabend. Wenn man so viel bezahlte, bekam man\\ zu jeder Zeit Unterricht an der man auch wollte, unbeeinflusst vom Konzept von Wochenenden.

„Hallo, Mr. Potter.“ sagte der Telepath tonlos, nachdem er die gesamte Spektrum an Privatspärenzauber gewirkt hatte.

„Hallo, Mr. Bester,“ sagte Harry erschöpft. „Dann sollten wir wohl den anfänglichen Schock hinter uns bringen, oder?

„Sie haben mich überrascht?“ sagte der Mann, jetzt mit einem etwas interessiert klingenden Tonfall. „Also gut.“ Er richtete seinen Zauberstab auf Harry und starrte ihm in die Augen. „\emph{Legilimens.“}

Es entstand eine Pause, und dann zuckte der schwarz-gewandete Zauberer zurück, als ob jemand ihn mit einem Viehtreiber geschockt hätte.

„Der Dunkle Lord ist am \emph{Leben?“, schluckte er. Seine Augen wurden plötzlich wild. \emph{Dumbledore macht sich selbst unsichtbar und schleicht sich die Mädchenschlafsäle?“}}

Harry seufzte und sah auf seine Uhr. In ungefähr drei Sekunden…

„Also,“ sagte der Mann. Er hatte seine Tonlosigkeit noch nicht völlig zurückgewonnen. „Sie glauben ernsthaft Sie werden die geheimen Regeln der Magie entdecken und allmächtig werden.“

„Das ist richtig,“ sagte Harry gleichgültig, noch immer auf seine Uhr blickend. „Ich bin \emph{derart vermessen.“}

„Ich bin neugierig. Wie es scheint, glaubt der Sprechende Hut Sie werden der nächste Dunkle Lord sein.“

„Und \emph{Sie wissen, dass ich ziemlich stark versuche das nicht zu werden, und Sie haben gesehen, dass wir bereits eine längere Diskussion darüber hatten, ob Sie dazu bereit wären mir Okklumentik beizubringen, und Sie sich zum Schluss dafür entschieden haben, also können diesen Teil einfach überspringen?“}

„In Ordnung,“ sagte der Mann exakt sechs Sekunden später, genau wie beim letzten Mal. „Bereiten Sie sich vor.“ Er machte eine Pause, und sagte dann mit wehmutsvoller Stimme, „Jedoch wünschte ich, ich könnte mich an den Trick mit dem Silber und Gold erinnern.“

Harry fühlte sich ziemlich davon verstört, wie reproduzierbar menschlich Gedanken waren, wenn man die Leute in die gleiche Initialposition zurückversetzte und sie den gleichen Impulse aussetzte. Es vertrieb die Illusionen, die ein guter Reduktionist in erster Linie gar nicht haben sollte.

Harry war in einer eher schlechten Stimmung als er am nächsten Montagmorgen aus Kräuterkunde stampfte.

Hermine brodelte vor Wut neben ihm.

Die anderen Kinder waren noch drinnen, ein bisschen langsamer darin ihre Sachen einzusammeln, da sie noch aufgeregt über Ravenclaws Sieg beim zweiten Quidditch Spiel des Jahres quatschten.

Wie es schien, war letzte Nacht nach dem Abendessen ein Mädchen für dreißig Minuten auf einem Besen herumgeflogen und dann eine Art gigantischen Moskito gefangen. Es gab noch andere Fakten von den Vorgängen dieses Spiels, aber sie waren irrelevant.

Harry hatte dieses aufregende Sportereignis wegen seiner Okklumentikstunde verpasst und außerdem weil er ein Leben hatte.

Er hatte danach sämtliche Unterhaltungen im Ravenclaw-Schlafsaal vermieden; Waren der Quietus-Zauber und magische Koffer nicht etwas Wunderbares? Er hatte am Gryffindor-Tisch gefrühstückt.

Aber er konnte Kräuterkunde nicht entgehen, und die Ravenclaws hatten darüber vor dem Unterricht, nach dem Unterricht und \emph{während des Unterrichts, bis Harry von seinem Baby-Pelziger, dessen Windel er gerade wechselte, aufsah, und laut verkündete, dass einzelne von ihnen versuchten etwas über \emph{Pflanzen} zu lernen und das Schnatze nicht auf Bäumen wachsen sie also \emph{bitte} aufhören könnten über Quidditch zu reden. Jeder der Anwesenden hatte ihm einen schockierten Blick zugeworfen, außer Hermine, die aussah als ob sie ihm applaudieren wollte, und Professor Sprout, die ihm einen Punkt für Ravenclaw verliehen hatte.}

Ein Punkt für Ravenclaw.

\emph{Ein Punkt.}

\emph{Die sieben Idioten auf ihren sieben idiotischen Besen ihr idiotisches Spiel spielend hatten einhundert und neunzig Punkte für Ravenclaw verdient.}

\emph{Anscheinend wurden Quidditch-Punkte direkt zu den Hauspunkten hinzu addiert. Mit anderen Worten waren das Fangen eines goldenen Moskitos 150 Hauspunkte wert.}

\emph{Harry konnte sich nicht einmal vorstellen, was er tun müsste um hundert und fünfzig Hauspunkte zu verdienen.}

\emph{Außer vielleicht, man könnte es sich eigentlich denken, einhundert und fünfzig Hufflepuffs zu retten, oder auf fünfzehn so gute Ideen wie die Schutzhüllen für Zeitmaschinen zu kommen, oder sich eintausend fünfhundert kreative Wege jemand umzubringen auszudenken, oder Hermine Granger das gesamte Jahr über zu sein.}

\emph{„Wir sollten sie ermorden,“ sagte Harry zu Hermine, die mit dem gleichen gereizten Ausstrahlung neben ihm her lief.}

„Wenn?“ sagte Hermine. „Das Quidditch Team?“

„Ich dachte eigentlich an alle, die in irgendeiner Form mit Quidditch zu tun haben, aber das Ravenclaw Team wäre ein Anfang, genau.“

Hermines Lippen waren missbilligend geschürzt. „Du weißt eigentlich, dass Menschen ermorden falsch ist, oder?“

„Ja,“ antwortete Harry.

„Okay, ich wollte nur sichergehen.“, sagte Hermine. „Lass uns zuerst den Sucher nehmen. Ich habe einige Agatha Christie Romane gelesen, weißt du wie wir sie auf einen Zug kriegen können?“

„Zwei Mord planende Schüler,“ sagte eine trockene Stimme. „Wie schockierend.“

Von der nächsten Kurve schlenderte ihnen ein Mann in leicht befleckten Umhängen entgegen, sein fettiges Haar fiel ihm lang und ungekämmt bis auf die Schultern. Tödliche Gefahr schien von ihm auszustrahlen und erfüllte die Korridore mit dem Gedanken an unpassend gemischten Zaubertränken, und unabsichtlichen Stürzen und Menschen die in ihrem Bett starben, an etwas das die Auroren als natürliche Umstände beschreiben würden.

Ohne auch nur darüber nachzudenken, trat Harry vor Hermine. Jemand zog scharf die Luft ein und einen Moment später, eilte Hermine an ihm vorbei und trat vor \emph{ihn. „Lauf, Harry!“ rief Hermine. „Jungen sollten nicht in Gefahr sein.“}

Severus Snape lächelte freudlos. „Unterhaltsam. Ich beantrage einen Teil ihrer Zeit, Potter, falls Sie sich von ihren Flirtereien mit Miss Granger losreißen können.“

Plötzlich erschien ein besorgter Ausdruck auf Hermines Gesicht. Sie drehte sich zu Harry um und öffnete ihren Mund, hielt dann betrübt blickend inne.

„Oh, machen Sie sich keine Sorgen, Miss Granger,“ sagte Severus seidige Stimme. „Ich verspreche ihren Galan unversehrt zurückzubringen.“ Sein Lächeln verschwand. „Jetzt werden Potter und ich gleich davongehen und eine private Unterhaltung führen, nur wir zwei. Ich hoffe es ist klar, dass Sie nicht eingeladen sind, aber nur für den Fall, nehmen Sie es als den Befehl eines Hogwarts Professors. Ich bin mir sicher eine brave, junge Dame wie Sie, wird sich dem nicht widersetzen.“

Und Severus drehte sich um ging wieder um die Kurve. „Kommen Sie, Potter?“ erklang seine Stimme.

„Um,“ sagte Harry zu Hermine. „Könnte ich ihm jetzt einfach folgen und es dir überlassen sich auszudenken was ich sagen sollte um sicherzustellen, dass du überhaupt nicht besorgt oder verärgert bist?“

„Nein,“ erwiderte Hermine mit zitternder Stimme.

Severus Gelächter schallte von der Kurve des Korridors entgegen.

Harry senkte seinen Kopf. „Entschuldigung,“ sagte er leise, „ernsthaft,“ und er folgte dem Zaubertränkelehrer.

„Also,“ sagte Harry. Es war nun völlig still, außer zwei Paar Beine, ein Langes und ein Kurzes, die durch einen steinernen Korridor tappten. Der Zaubertränkelehrer schritt zügig voran, sodass Harry gerade noch mithalten konnte, und soweit Harry das Konzept von Richtungen auf Hogwarts anwenden konnte, entfernten sie sich von den häufig besuchten Bereichen. „um was geht es?“

„Ich vermute nicht Sie könnten mir erklären,“ sagte Severus trocken, „warum Sie beide den Mord an Cho Chang planten?“

„Ich vermute nicht \emph{Sie könnten mir erklären,“ sagte Harry ebenfalls trocken, „in ihrer Funktion als Fakultätsmitglied des Schulsystems von Hogwarts, wieso das Fangen eines goldenen Moskitos als eine akademische Errungenschaft im Wert von einhundert und fünfzig Hauspunkte gesehen wird?“}

Ein Lächeln strich über Severus Lippen. „Wahrhaftig, und ich hatte gedacht, sie sollten eigentlich scharfsinnig sein. Sind Sie wirklich derart unfähig ihre Klassenkameraden zu verstehen, Potter, und verabscheuen Sie zu sehr um es zu versuchen? Falls Quidditchpunkte nicht für den Hauspokal zählen würden, dann würde sich keiner von ihnen überhaupt für Hauspunkte interessieren. Es wäre nur einer obskurer Wettbewerb für Schüler wie Sie und Miss Granger.“

Es war eine schockierend gute Antwort.

Und dieser Schock weckte Harrys Verstand vollends auf.

Im Nachhinein sollte es eigentlich nicht überraschend sein, dass Severus seine Schüler verstand, sie \emph{wirklich gut verstand. Immerhin hatte er ihre Gedanken gelesen.}

Und…

… dem Buch zufolge war ein erfolgreicher Legilimens extrem selten, seltener als ein perfekter Okklumens, weil fast niemand genügend mentale Disziplin hatte.

\emph{Mentale Disziplin?}

Harry hatte Geschichten über einen Mann gesammelt, der regelmäßig im Unterricht seine Beherrschung verlor und die Schüler zusammenstauchte.

… aber der gleiche Mann hatte, als Harry davon gesprochen hatte, dass der Dunklen Lord noch Leben war, unverzüglich und perfekt geantwortet ---- er hatte genau so reagiert wie ein völlig Unwissender reagieren würde.

Der Mann schritt durch Hogwarts mit der Aura eines Assassinen, \emph{Gefahr ausstrahlend…}

… was das genaue Gegenteil zu dem war was ein echter Assassine tun sollte. Echte Assassinen sollten wie kleine schwache Buchhalter wirken bis sie zuschlugen.

Er war der Hauslehrer vom stolzen und aristokratischen Slytherin, und trug einen Umhang befleckt von Stückchen von Zaubertränken und Zutaten, die zwei Minuten Magie entfernen könnten.

Harry bemerkte, dass er verwirrt war.

Und seine Gefahreneinschätzung des \emph{Hauslehrer\emph{s von} \emph{Slytherin} schoss ins Astronomische.}

Dumbledore schien zu glauben, Severus wäre einer der seinen, und es war noch nichts vorgefallen um dem zu widersprechen; der Zaubertränkelehrer war „furchteinflößend gewesen, hatte seine Position aber nicht missbraucht“ wie versprochen. Also hatte Harry zuvor gefolgert, dass dies eine Mitgliedschaftsangelegenheit war. Falls Severus plante ihm zu schaden, würde er sicherlich nicht Harry vor Hermine, einer Zeugin, abpassen, wenn er doch einfach warten konnte bis Harry allein war…

Harry biss sich auf die Unterlippe.

„Ich kannte einst einen Jungen, der Quidditch wahrhaftig geliebt hatte,“ sagte Severus Snape. „Er war ein äußerster Schwachkopf. Genau wie wir zwei, Sie und ich es erwarten würden.“

„Was \emph{ist} das hier?, fragte Harry zögerlich.

„Geduld, Potter.“

Severus drehte seinen Kopf, und glitt mit dem Gebaren eines Assassinen in eine nahe Öffnung in einer Wand des Korridors, ein schmalerer und engerer Gang zweigte hier ab.

Sie bogen ab und bogen wieder ab, und landeten in einer Sackgasse, einer einfachen glatten Wand. Falls Hogwarts wirklich gebaut worden wäre, anstatt beschwört, herbeigerufen, geboren oder wer weiß was zu werden, hätte Harry ein paar harte Worte für den Architekten gehabt, wenn man Personen dafür bezahlte Gänge ins Nichts zu bauen.

„Quietus,“ sagte Severus, und einige weitere Zauber danach.

Harry lehnte sich zurück, verschränkte seine Arme vor seiner Brust, und betrachtete Severus Gesicht.

„Sie schauen mir in die Augen, Potter?“ fragte Severus Snape. „Deine Okklumentikstunden können noch nicht derart weit vorangeschritten sein, sodass Sie Legilimentik verhindern könnten. Aber vielleicht sind sie bereits derart weit, dass Sie es bemerken können. Da ich nichts anderes wissen kann, werde ich es nicht riskieren es zu versuchen.“ Der Man lächelte dünn. „Und das Gleiche wird für Dumbledore gelten, schätze ich. Was der Grund ist, warum wir \emph{jetzt} dieses Gespräch führen.“

Harrys Augen weiteten sich unwillkürlich.

„Zum Beginn,“ sagte Severus mit funkelnden Augen. „Ich hätte gerne ihr Versprechen niemanden etwas von diesen Unterhaltungen zu erzählen. Soweit es die Schule betrifft, besprechen wir ihre Zaubertränkehausaufgaben. Ob sie es glauben oder nicht nicht ist unwichtig. Soweit es Dumbledore und McGonagall betrifft, verletze ich Draco Malfoys Vertrauen in mich, und keiner von uns glaubt, dass es angemessen ist weiter über die Einzelheiten zu sprechen.“

Harrys Gehirn versuchte die Auswirkungen und Implikationen davon zu berechnen, aber es ging ihm der Swap Space aus.

„Also?“ fragte der Zaubertränkelehrer.

„Also gut,“ sagte Harry zögerlich. Es war schwer abzusehen, wie eine Unterhaltung zu führen und nichts davon erzählen zu können, einschränkender sein sollte als sie \emph{nicht} zu führen, da man in diesem Fall \emph{ebenfalls} nichts von deren Inhalt weitergeben konnte. „Ich verspreche es.“

Severus beobachtete Harry aufmerksam. „Sie haben einmal im Büro des Schulleiters gesagt, dass Sie Schikanieren und Missbrauch nicht tolerieren würden. Und So frage ich mich, Harry Potter. Wie sehr ähneln sie ihrem Vater?“

„Solange wir nicht über Michael Verres-Evans reden,“ sagte Harry, „ist die Antwort, dass ich sehr wenig über James Potter weiß.“

Severus nickte, wie zu sich selbst. „Es gibt da einen Fünftklässler-Slytherin. Ein Junge namens Lesath Lestrange. Er wird von Gryffindors tyrannisiert. Ich bin… in meinen Möglichkeiten in solchen Situationen einzugreifen ziemlich einschränkt. \emph{Sie} könnten vielleicht ihm helfen. Falls Sie möchten; Ich erbitte hiermit keinen Gefallen, und werde Ihnen auch keinen Schulden. Es ist schlicht eine Gelegenheit zu tun was Sie auch tun werden.“

Harry starrte Severus an, angestrengt nachdenkend.

„Fragen Sie sich ob es eine Falle ist?“ sagte Severus, ein angedeutetes Lächeln spielte über seine Lippen. „Es ist keine. Es \emph{ist} ein Test. Nennen Sie meine eigene Neugierde. Aber Lesath Probleme sind echt, so wie meine Schwierigkeiten dabei einzuschreiten.“

Das war das Problem wenn andere Leute wusste, dass man ein guter Mensch ist. Selbst wenn man selbst wusste, dass sie es wussten, konnte man den Köder nicht ignorieren.

Und sein Vater hatte ebenfalls Schüler vor Schlägern beschützt… es machte keinen Unterschied warum Severus ihm davon erzählt hatte. Er fühlte sich deswegen stolz, warm ums Herz und machte es unmöglich sich davon abzuwenden.

„Gut,“ sagte Harry. „Erzählen Sie mir von Lesath. Warum wird er schikaniert?“

Severus Gesicht verlor sein schwaches Lächeln. „Sie denken es gibt \emph{Gründe}, Potter?“

„Vielleicht nicht,“ erwiderte Harry leise. „Aber mir war der Gedanke gekommen, dass er möglicherweise ein unbedeutendes Schlammblutmädchen die Treppen heruntergeschubst haben könnte.

„Lesath Lestrange,“ sagte Severus, seine Stimme kalt, „ist der Sohn von Bellatrix Black, die fanatischste und böseste Dienerin des dunklen Lords. Lesath ist der anerkannte Bastard von Rabastan Lestrange. Kurz nach dem Tod des dunklen Lords, wurden Bellatrix, Rabastan und Rabastans Bruder Rodolphus erfasst, während sie Alice und Frank Longbottom folgerten. Alle drei haben lebenslänglich in Askaban. Die Longbottoms wurden noch wiederholte Cruciatus-Flüche in den Wahnsinn getrieben und verbleiben in St. Mungos Station für Unheilbare. Ist etwas hiervon ein guter Grund, um ihn zu tyrannisieren?“

„Es ist überhaupt kein Grund,“ sagte Harry, immer noch leise. „Und Lesath selbst hat soweit Sie es wissen nichts Böses getan?“

Ein unscheinbares Lächeln schlich sich wieder auf Severus Lippen. „Er ist so wenig ein Heiliger wie jeder andere. Aber er hat kein Muggelmädchen die Treppe heruntergestoßen, nicht soweit ich es weiß.“

„Oder in seinen Gedanken gelesen haben,“ sagte Harry.

Severus Gesichtsausdruck wurde kalt. „Ich bin nicht in seine Privatsphäre eingedrungen, Potter. Ich habe stattdessen die Gryffindors betrachtet. Er ist einfach ein angenehmes Ziel für ihre eigenen kleinen Befriedigungen.“

Ein kalter Schwall Ärger lief Harry den Rücken herunter, und er musste sich daran erinnern, dass Severus keine vertrauenswürdige Informationsquelle sein könnte.

„Und Sie glauben,“ sagte Harry, „dass eine Intervention von Harry Potter, dem Junge-Der-Überlebte, sich als effektiv herausstellen könnte.“

„Allerdings,“ sagte Severus Snape und erzählte Harry wo und wann die Gryffindors ihr nächstes kleines Spiel planten.

Durch die Mitte von Hogwarts zweitem Stock verläuft auf die Nord-Süd-Achse ein Hauptkorridor, und nahe der Mitte dieses Korridors, öffnet sich ein weiterer kurzer Korridor der sich nur ein Dutzend Schritte nach hinten erstreckt, bis er in einem rechten Winkel abknickt und ein L formt, worauf er nach einem weiteren dutzend Schritten an einem hellen, breiten Fenster endet von dem man aus dem dritten Stock blickend die Lichtstrahlen auf den östlichen Ländereien von Hogwarts beobachteten konnte. Am Fenster kann man nichts vom Hauptkorridor hören, und niemand im Korridor würde etwas von den Geschehnissen am Fenster hören. Falls jemand dies merkwürdig findet, war dieser jemand offensichtlich noch nicht lange in Hogwarts.

Vier Jungen in rot-besetzen Umhängen lachten, und ein Junge in einem grün-besetzen Umhang schrie, während er verzweifelt versuchte die Fensterkante mit seinen Händen zu erreichen, als die vier Jungen so taten als ob sie ihn hinausschubsen wollten. Es ist natürlich nur ein Streich, und außerdem, ein Sturz aus dieser Höhe würde einen Zauberer nicht töten. Alles nur Spaß. Falls jemand dies merkwürdig vorkommt ----

„\emph{Was macht ihr da?“ sagte die Stimme eines sechsten Jungens.}

Die vier Jungen in rot-besetzenden Umhängen fuhren in einer plötzlichen Bewegung herum, und der Junge im grün-besetzen Umhang zog sich hektisch weg vom Fenster und fiel mit tränenüberströmtem Gesicht zu Boden.

„Oh,“ sagte der attraktivste der Jungen in rot-besetzen Umhängen erleichtert, „\emph{du bist's. Hey Lessy, weißt du wer das ist?“}

Es kam keine Antwort vom Jungen am Boden, der versuchte sein Schiefen unter Kontrolle zu bringen und der Junge im rot-besetzen Umhang, holte mit seinem Fuß einen Tritt aus ----

„\emph{Lass dass!“ rief der sechste Junge.}

Der Junge im rot-besetzen Umhang schwankte als er seinen Tritt abbrach. „Ähm,“ sagte er, „weißt du wer das ist?“

Das Atmen des sechsten Jungens klingt eigenartig. „Lesath Lestrange,“ sagte er, sein Atem stoßweise, „und \emph{er hat meinen Eltern nicht angetan, er war fünf Jahre alt.“}

Neville Longbottom starrte die vier riesigen Fünftklässler vor sich an und bemühte sich sein Zittern unter Kontrolle zu halten.

Er hatte Harry Potter absagen sollen.

„Wieso verteidigst du ihn?“ sagte der Attraktive, langsam, verwirrt und mit ersten Anklängen von Ärger. „Er ist ein \emph{Slytherin. Und ein \emph{Lestrange.“}}

„Er ist ein Junge der seine Eltern verloren hat,“ sagte Neville Longbottom. „Ich weiß wie das ist.“ Er wusste nicht, woher die Worte kamen. Es klang zu cool, wie etwas das Harry Potter sagen würde.

Das Zittern ließ jedoch nicht nach.

„\emph{Wer denkst du, dass du bist?“ sagte der Attraktive, nun mit ärgerlicher Stimme.}

\emph{Ich bin Neville, der letzte Erbe des Noblen und Uralten Hauses Longbottoms ----}

Neville konnte es nicht sagen.

\emph{„Ich glaube er ist ein Verräter,“ sagte einer der anderen Gryffindors, und Neville spürte wie ihm sein Herz in die Hose rutschte.}

Er hatte es gewusst, er hatte es gewusst. Harry Potter lag schließlich doch falsch. Solche Schläger würden nicht aufhören, nur weil Neville Longbottom es ihnen auftrug.

Der Attraktive machte einen Schritt auf ihn zu, und die drei Anderen folgten ihm.

„So ist es also für euch,“ sagte Neville etwas fasziniert davon wie beständig seine Stimme war. „Es kümmert euch nicht ob es Lesath Lestrange oder Neville Longbottom ist.“

Lesath Lestrange entfuhr ein plötzliches Keuchen, von seinen Position am Boden.

„Böse ist böse,“ knurrte der gleich Junge, der bereits zuvor gesprochen hatte, „und wenn du mit dem Bösen befreundet bist, bist du ebenfalls böse.“

Die Vier traten einen weiteren Schritt auf Neville zu.

Lesath stand schwankend auf. Seine Gesicht war aschfahl, und er trat einige Schritte nach vorn, und lehnte sich an die Wand, und sagte kein Wort. Seine Auge waren auf die Ecke des Korridors fokussiert, dem Weg ins Freie.

„Freunde,“ sagte Neville. Nun erhöhte sich seine Stimme. „Ja, ich habe Freunde. Einer von ihnen ist der Junge-Der-Überlebte.

Ein Teil der Gryffindors sah plötzlich besorgt aus. Der Attraktive schreckte nicht zurück. „Harry Potter ist nicht hier,“ sagte er, seine Stimme fest, „und selbst wenn, glaube ich nicht, dass es ihm gefallen würde einen Longbottom einen Lestrange verteidigen zu sehen.“

Und die Gryffindors traten einen weiteren großen Schritt auf ihn zu, während hinter ihnen Lesath an der Wand entlang kroch, auf seine Gelegenheit wartend.

Neville schluckte, und hob seine rechte Hand mit Zeigefinger und Daumen aufeinander gepresst in die Luft.

Er schloss seine Augen, weil er Harry Potter versprochen hatte, dass er nicht linsen würde.

Falls es nicht klappte, würde er nie wieder irgendjemandem vertrauen.

Seine Stimme erklang unter diesen Umstanden überraschend klar.

„Harry James Potter-Evans-Verres. Harry James Potter-Evans-Verres. Harry James Potter-Evans-Verres. Bei der dem was du mir schuldest und der Macht deines wahren Namens beschwöre ich dich, ich öffne den Weg für dich, ich rufe dich an dich vor mir zu materialisieren.“

Neville schnipste mit seinen Fingern.

Und dann öffnete Neville seine Augen.

Lesath Lestrange starrte ihn an.

Die vier Gryffindors starrten ihn an.

Der Attraktive fing an zu lachen, und das löste auch das Lachen der Anderen aus.

„Sollte Harry eigentlich gerade um die Ecke kommen oder sowas?“, sagte der Attraktive. „Ohhh, wie scheint wurdest du versetzt.“

Der Attraktive machte noch einen weiteren bedrohlichen Schritt auf Neville zu.

Die anderen drei folgten im Gleichschritt.

„Ahem,“ sagte Harry Potter hinter ihnen, an die Wand neben dem Fenster gelehnt, in der Sackgasse des Korridors, die niemand ungesehen erreichen hätte können.

Wenn es sich immer so gut anfühlte Menschen schreien zu sehen, könnte Neville fast verstehen wieso manche Menschen Schläger wurden.

Harry Potter schritt nach vorne und positionierte sich zwischen Lesath Lestrange und den Anderen. Er ließ seinen eiskalten Blick über die Jungen in rot-besetzen Umhängen wandern, bis sie auf dem Attraktiven verharrten, dem Anführer. „Mr. Carl Sloper,“ sagte Harry Potter. „Ich nehme an, ich habe die Situation komplett verstanden. Falls Lesath Lestrange jemals selbst Unheil angerichtet hat, ohne von den falschen Eltern geboren zu werden in Betracht zu ziehen, dann ist dieser Fakt \emph{Ihnen} nicht bekannt. Falls ich mich hierbei irre, Mr. Sloper, schlage ich vor, Sie informieren mich sofort davon.“

Neville sah die Angst und die Ehrfurcht auf den Gesichtern der anderen Jungen. Er fühlte sie selbst. Harry hatte \emph{vorgegeben es würde alles nur ein Trick seinen, aber wie konnte das stimmen?}

„Aber, er ist ein \emph{Lestrange,“ sagte der Anführer.}

„Er ist ein Junge, der \emph{seine Eltern verloren hat,“ sagte Harry Potter, während seine Stimme stetig kälter wurde.}

Dieses Mal schreckten alle drei der anderen Gryffindors zurück.

„So,“ sagte Harry Potter. „Sie haben gesehen, dass Neville nicht wollte, dass Sie einen unschuldigen Jungen im Namen der Longbottoms quälen. Dies hat Sie nicht gekümmert. Wenn ich Ihnen nun mitteile, der Junge-Der-Überlebte glaubt \emph{ebenfalls , dass Sie falsch lägen und dass Sie einen schrecklichen Fehler begangen haben, verändert dies etwas?“}

Der Anführer trat einen Schritt auf Harry.

Die Anderen folgten im \emph{nicht.}

„Carl,“ sagte einer von ihnen, schluckend. „Vielleicht sollten wir gehen.“

„Sie sagen du würdest der nächste Dunkle Lord werden,“ sagte der Anführer, den Blick auf Harry fixiert.

Ein Lächeln huschte über über Harrys Gesicht. „Sie sagen auch, ich wäre heimlich mit Ginevra Weasley verlobt und es gäbe eine Prophezeiung in der wir Frankreich erobern.“ Das Lächeln verschwand. „Da Sie entschlossen scheinen, dies zu vertiefen, Mr. Carl Sloper, lassen Sie mich es klar stellen. \emph{Lass Lesath in Ruhe. Ich werde es wissen falls nicht.“}

„Also hat Lessy gepetzt,“ sagte der Anführer kalt.

„Natürlich,“ sagte Harry Potter trocken, „er hat mir auch erzählt, was du nach dem Ende des Zauberkunstunterrichts mit einem bestimmten Hufflepuff-Mädchen mit einer weißen Schleife im Haar an einem privaten, abgeschieden Ort gemacht hast ----

Die Kinnlade des Anführers klappte schockiert herunter.

„Iiiep,“ sagte einer der anderen Gryffindors in einer äußerst hohen Tonlage, fuhr herum und rannte um die Ecke. Seine Schritte entfernten sich hektisch und waren schließlich nicht mehr zu hören.

Und dann waren es nur noch sechs.

„Ah,“ sagte Harry Potter, „da entkommt ein leidlich intelligenter junger Mann. Ihr könntet wirklich etwas von Bertram Kirkes Beispiel lernen, bevor ihr, sagen wir, in Schwierigkeiten geratet.“

„Drohst du damit uns zu verpetzen?“ sagte der attraktive Gryffindor, der bewusst versuchte verärgert zu klingen, aber nicht das Schwanken seiner Stimme verhindern konnte. „Schlimme Dinge passieren mit Petzen.“

Die anderen Gryffindors kamen langsam wieder näher.

Harry Potter fing an zu lachen. „Oh, das hast du gerade nicht gesagt, oder? Versuchst du \emph{wirklich mich einzuschüchtern? \emph{Mich?} Jetzt mal im Ernst, glaubst du wirklich du wärst angsteinflößender als Peregrine Derrick, Severus Snape und übrigens auch Du-weißt-schon-wer?“}

Sogar der Rädelsführer schreckte nun zurück.

Harry Potter hob seine Hand, seine Finger bereit, und alle drei Gryffindors sprangen zurück, und einer von ihnen rief „Nein----!“

„Schaut,“ sagte Harry Potter, „dies ist der Moment in dem ich mit den Fingern schnipse und ihr Teil einer irrsinnig komischen Geschichte werdet, die man sich begleitet von viel nervösem Gelächter beim Abendessen erzählen wird. Der Punkt ist aber, dass Leute, denen ich vertraue, mir raten es zu lassen. Professor McGonagall erklärte mir, ich würde immer nur den einfachen Weg nehmen und Professor Quirrell sagte, ich müsse lerne zu verlieren. Also, erinnert ihr euch an die Geschichte, in der ich es zugelassen habe von älteren Slytherins verprügelt zu werden? Etwas derartiges könnten wir machen. Ihr schikaniert mich für eine gewisse Zeit und ich könnte es zulassen. Aber erinnert ihr euch auch an den Teil am Ende, in dem ich meinen vielen, vielen Freunden innerhalb dieser Schule nicht deshalb zu unternehmen? Dieses Mal werden ich das auslassen. Also, macht nur. Greift mich an.“

Harry Potter schritt nach vorn, seine Arme einladend geöffnet.

Die drei Gryffindors wichen zurück und rannten los, sodass Neville schnell aus dem Weg springen musste, um nicht zertrampelt zu werden.

Es wurde still, nach dem ihre Schritte verklungen waren und blieb auch einige Zeit danach noch still.

Und dann waren es nur noch drei.

Harry Potter hole tief Luft und atmete dann wieder aus. „Puh,“ sagte er. „Wie fühlst du dich, Neville?“

Neville Stimme folgte in einem schrillen Quieken. „Okay, \emph{das war wirklich cool.“}

Ein Lächeln erschien auf Harry Potters Gesicht. „\emph{Du warst auch ziemlich cool, weißt du.“}

Neville wusste, dass Harry Potter dies nur sagte, damit er sich besser fühlte und trotzdem erzeugte ein warmes Glimmen in seiner Brust.

Harry wandte sich zu Lesath Lestrange ----

„Wie geht es dir, Lestrange?“ sagte Neville bevor Harry seinen Mund öffnen konnte.

Das war etwas, von dem er nie erwartet hätte es jemals zu sagen.

Lesath Lestrange drehte sich langsam, und starrte Neville an, sein Gesicht angespannt. Ich weinte nicht mehr, aber seine Tränen glitzerten, während sie trockneten.

„Du glaubst, du weißt wie es ist?“ fragte Lesath mit hoher, zitternder Stimme. „\emph{Du glaubst du weißt wie es ist? Meine Eltern sind in \emph{Askaban}, ich versuche nicht daran zu denken, aber sie erinnern mich immer daran, sie meinen es wäre großartig meine Mutter dort im Kalten und Dunklen zu festzuhalten, während Dementoren ihr Leben aussaugen, ich wünschte mir es wäre bei mir wie bei Harry Potter, wenigstens leiden seine Eltern nicht, meine Eltern leiden immer, in jeder Sekunde des Tages, ich wünschte mir es wäre wie bei dir, zumindest kannst du deine Eltern manchmal sehen, wenigstens weißt du, dass sie dich geliebt haben. Falls meine Mutter mich jemals geliebt hat, haben die Dementoren den Gedanken inzwischen aufgegessen.}

Neville Augen öffneten sich schockiert. Das hatte er nicht erwartet.

Lesath wandte sich Harry zu, dessen Augen von purem Horror gefüllt waren.

Lesath warf sich vor Harry Potter auf die Knie, berührte mit der Stirn den Boden, und flüsterte, „Helft mir, Lord.“

Es entstand eine schreckliche Stille. Neville fiel nichts ein das er sagen konnte, und von dem unverhüllten Schock auf Harrys Gesicht nach zu urteilen, wusste er auch nichts.

„Sie sagen du kannst alles, bitte, bitte mein Herr, holt meine Eltern aus Askaban, ich werde für immer Euer loyaler Diener sein, mein Leben wird Eures sein und mein Tod ebenso, nur bitte ----“

„Lesath,“ sagte Harry mit brüchiger Stimme, „Lesath, ich kann nicht wirklich solche Sachen machen, es waren immer nur blöde Tricks.“

„Das ist \emph{nicht wahr!“ sagte Lesath mit hoher, verzweifelter Stimme. „Ich \emph{habe es gesehen,} die Geschichten sind wahr, du \emph{kannst} es!“}

Harry schluckte. „Lesath, ich habe alles vorher mit Neville arrangiert, wir haben alles im voraus geplant, frag ihn!“

Das stimmte, obwohl Harry nie erklärt hatte \emph{wie er irgendetwas davon machen würde…}

Als Lesath vom Boden aufsah, war ein sein Gesicht zornig verzerrt und seine Stimme war nur noch ein Kreischen, dass in Nevilles Ohren schmerzte. „\emph{Du Sohn eines Schlammbluts! Du könntest sie befreien, du machst es nur nicht! Ich bin auf die Knie gefallen und habe dich angefleht und du willst noch immer nicht helfen! Ich hätte es wissen sollen, du bist der Junge-Der-Überlebte, du glaubst sie gehört \emph{dorthin!“}}

„Ich \emph{kann es nicht!“ sagte Harry genauso verzweifelt wie Lesath. „Es ist nicht die Frage was ich will, ich habe nicht die \emph{Macht} dafür!“}

Lesath erhob sich, spuckte vor Harry auf den Boden, wandte sich um und ging davon. Als er um die Ecke verschwunden war, beschleunigten sich seine Schritte, und als sie verklangen, hörte Neville eine vereinzeltes Schluchzen.

Und dann waren es nur noch zwei.

Neville schaute Harry an.

Harry schaute Neville an.

„Wow,“ sagte Neville leise. „Er schien nicht gerade dankbar dafür gerettet zu werden.“

„Er dachte, ich könnte ihm helfen,“ erwiderte Harry heiser. „Er hatte zum ersten Mal seit Jahren wieder Hoffnung.“

Neville schluckte, und sprach es aus. „Es tut mir leid.“

„Was?“, rief Harry, jetzt völlig verwirrt.

„Ich war auch nicht dankbar, als du mir geholfen hast ----“

„Wirklich alles was du gesagt hast, war komplett richtig,“ sagte der Junge-Der-Überlebte.

„Nein,“ sagte Neville, „war es nicht.“

Sie lächelten einander gleichzeitig für einen kurzen Moment traurig an, beide ein wenig herablassend.

„Ich weiß, dass das nicht echt was,“ sagte Neville, „ich weiß, ich hätte nichts erreichen können, wenn du nicht da gewesen wärst, aber danke, dass ich es vorgeben durfte.“

„Nun mach mal halblang!“ sagte Harry.

Harry hatte sich von Neville abgewandt, und starrte aus dem Fenster an die grauen Wolken.

Ein komplett lächerlicher Gedanke fiel Neville ein. „Fühlst du dich schuldig, weil du Lesaths Eltern nicht aus Askaban befreien kannst?“

„Nein,“ antwortete Harry.

Einige Sekunden verstrichen.

„Ja,“ sagte Harry.

„Du bist dumm,“ sagte Neville.

„Ich bin mir dessen bewusst,“ sagte Harry.

„Musst du buchstäblich \emph{alles machen, um das dich jemand bittet?“}

Der Junge-Der-Überlebte wandte sich wieder zu Neville um. „\emph{Müssen? Nein. Sich schuldig fühlen? Ja.“}

Neville hatte Schwierigkeiten Worte zu finden. „Nachdem der Dunkle Lord gestorben ist, war Bellatrix Black buchstäblich die bösartige Person in der gesamten Welt und das \emph{bevor sie nach Askaban gebracht wurde. Sie hat meine Mutter und meinen Vater bis zum Wahnsinn gefoltert, weil sie herausfinden wollte, was mit dem dunklen Lord geschehen war----“}

„Ich weiß,“ sagte Harry tonlos. „Ich verstehe schon, aber----“

„Nein! Du \emph{verstehst es nicht! Sie hatte einen \emph{Grund} für diese Tat und meine Eltern waren beide Auroren! Es ist nicht einmal \emph{nahe dran} am Schlimmsten, dass sie jemals getan hat.“ Nevilles Stimme zitterte.}

„Und trotzdem,“ sagte der Junge-Der-Überlebte, sein Blick abwesend, als ob er einen fernen, andersartigen Ort anstarrte, den Neville sich nicht einmal vorstellen konnte.

„Es konnte eine unglaublich clevere Lösung geben, die es ermöglicht alle zu retten, damit sie glücklich bis ans Ende ihrer Tage leben können, und wenn ich nur schlau genug wäre, wäre sie mir inzwischen schon eingefallen----“

„Du hast Probleme,“ sagte Neville. „Du glaubst du müsstest genau der sein, den Lesath Lestrange in dir sieht.“

„Ja,“ erwiderte der Junge-Der-Überlebte, „das trifft es ganz gut. Jedes Mal, wenn jemand im Gebet um Hilfe anfleht und ich ihn nicht erhören kann, fühle ich mich schuldig nicht Gott zu sein.“

Neville verstand es nicht vollends, aber… „Das klingt nicht gut.“

Harry seufzte. „Ich verstehe, dass ich ein Problem habe, und ich weiß was ich tun muss, um es zu lösen, in Ordnung? Ich arbeite daran.“

Harry beobachtete, wie Neville davonging.

Natürlich hatte Harry nicht gesagt war die Lösung war.

Die Lösung war, offensichtlicherweise, sich zu beeilen und Gott zu werden.

Neville Schritte entfernten sich und waren bald nicht mehr wahrzunehmen.

Und dann war es nur noch einer.

„Ähem,“ sagte Severus Snapes Stimme von direkt hinter ihm.

Harry stieß einen kurzen Schrei aus und hasste sich sofort dafür.

Langsam, wandte Harry sich um.

Der große, schmierige Mann in befleckten Umhängen lehnte an der gleichen Stelle an die Wand, die zuvor Harry besetzt hatte.

„Eine schöner Tarnumhang, Potter,“ sagte der Zaubertränkelehrer gedehnt. „Das erklärt vieles.“

Gott verdammt.

„Und vielleicht war ich zu lang in Dumbledores Umfeld,“ sagte Severus, „aber nicht kann nicht umhin mich zu fragen, ob dies \emph{der Tarnumhang ist.“}

Harry verwandelte sich sofort in jemanden, der niemals zuvor von \emph{dem Tarnumhang gehört hatte und der \emph{exakt} so clever war, wie Harry glaubte, dass Severus es von ihm vermuten würde.}

„Oh, möglicherweise,“ antwortete Harry. „Ich nehme an, sie sind sich der Schlussfolgerungen bewusst, wenn dem so wäre?“

Severus Stimme war herablassend. „Du hast keine Ahnung, wovon ich rede, nicht wahr, Potter? Eine ziemlich unbeholfener Versuch Informationen zu sammeln.“

(Professor Quirrell hatte beim Mittagessen angemerkt, Harry müsse wirklich seinen Geisteszustand bessern verschleiern, anstatt immer wenn ein gefährliches Thema anschnitten wird ein ausdrucksloses Gesicht aufzusetzen, und hatte dann Täuschungen erster Stufe, zweiter Stufe und so weiter erklärt. Also vermutete Severus entweder in Harry wirklich einen Spieler erster Stufe, sodass er selbst zur zweiten Stufe gehörte, und Harrys Zug dritter Stufe erfolgreich gewesen war; oder Severus spielte auf der vierten Stufe und wollte, dass Harry \emph{glaubte die Täuschung wäre erfolgreich gewesen. Harry hatte lächelnd Professor Quirrell gefragt, auf welchem Level er spielte, und Professor Quirrell hatte ebenfalls lächelnd geantwortet \emph{Ein Level höher als du.})}

„Sie haben demnach die ganze Zeit zugeschaut,“ sagte Harry. „Desillusionierung, nennt man es glaube ich.“

Ein dünnes Lächeln. „Es wäre töricht jegliches Risiko einzugehen, dass Sie zu schaden kommen.“

„Und außerdem wollte die Ergebnisse aus erster Hand sehen,“ sagte Harry. „Also, bin ich wie mein Vater?“

Ein merkwürdiger Ausdruck kam über den Mann, einer, der auf seinem Gesicht fremdartig aussah. „Ich würde eher sagen, Harry Potter, dass sie jemand anderem ähneln…“

Severus brach ab.

Er starrte Harry an.

„Lestrange nannte Sie Sohn eines Schlammbluts,“ sagte Severus zögernd. „Es schien Ihnen nicht auszumachen.“

Harry runzelte die Stirn. „Nicht unter diesen Umständen, nein.“

„Sie hatten ihm gerade geholfen,“ sagte Severus. Seine Augen waren aufmerksam auf Harry gerichtet. „Und er hat es Ihnen um die Ohren gehauen. Sicherlich ist dies nichts, dass Sie einfach vergeben können?“

„Er hatte gerade eine ziemlich erschütternde Erfahrung hinter sich gebracht,“ sagte Harry. „und ich glaube kaum, das von Erstklässlern gerettet zu werden seinem Stolz geholfen hat.“

„Ich vermute, es war leicht zu vergeben,“ sagte Severus mit eigenartiger Stimme, „da Lestrange Ihnen nichts bedeutet hat. Nur ein fremde Slytherin. Wenn er ein Freund gewesen wäre, dann hätten seine Worte Sie deutlich mehr verletzt.“

„Wenn er ein Freund wäre,“ sagte Harry, „gibt noch viel mehr Gründe ihm zu verzeihen.“

Eine lange Pause entstand zwischen ihnen. Harry fühlte, und er hätte nicht sagen können warum oder woher, dass sich die Luft mit einer schrecklichen Spannung füllt, wie stetig ansteigendes Wasser, steigend und weiter steigend.

Dann lächelte Severus, plötzlich viel gelassener aussehend, und die Spannung verschwand.

„Sie sind eine sehr nachsichtige Person,“ sagte Severus, noch immer lächelnd. „Ich vermute ihr Stiefvater, Michael Verres-Evans, war derjenige, der es ihnen beigebracht hat.“

„Vielmehr Dads Science-Fiction und Fantasy Kollektionen,“ sagte Harry. „wenn man so will mein fünftes Elternteil. Ich habe das Leben all der Charaktere in diesen Büchern gelebt, und ihre gewaltige Weisheit donnert in meinem Kopf. Irgendwo dabei war jemand wie Lesath, vermute ich, obwohl ich nicht sagen könnte wer. Es war nicht schwer mich in seine Situation zu versetzen. Und es waren meine Bücher, die es bei gebracht haben. Die Guten vergeben.“

Severus ließ ein helles, amüsiertes Lachen klingen. „Ich fürchte, ich weiß nicht viel von den Taten der Guten.“

Harry musterte ihn. Das war eigentlich ziemlich traurig. „Ich könnte Ihnen einige Bücher mit guten Menschen leihen, wenn Sie möchten.“

„Ich würde Sie gerne um ihren Rat wegen etwas fragen,“ sagte Severus, seine Stimme lässig. „Ich weiß von einem anderen Slytherin-Fünftklässler, der von Gryffindors schikaniert wird. Er warb um\\ ein hübsches muggelstämmiges Mädchen, die vorbeikam als er drangsaliert würde und versuchte ihn zu retten. Und er nannte sie Schlammblut, und das war das Ende für die beiden. Er entschuldigte sich, viele Male, aber Sie hat ihm nie verziehen. Haben Sie eine Idee, was er hätte sagen oder tun können, um von ihr die gleiche Versöhnlichkeit zu gewinnen, die Sie Lestrange gewährt haben?“

„Ähm,“ sagte Harry, „basierend nur auf diesen Informationen, bin ich mir nicht sicher ob \emph{er das eigentliche Problem war. Ich hätte ihm geraten nicht mit jemand auszugehen, der derart unfähig war zu verzeihen. Nehmen wir an sie hätten geheiratet, können Sie sich das Leben in diesem Haushalt vorstellen?}

Es entstand eine Pause.

„Oh, aber sie konnte vergeben,“ sagte Severus etwas amüsiert. „Denn, danach wandte sich von ihm ab und wurde die Freundin des Schlägers. Erklären Sie mir, warum würde sie dem Schläger vergeben und nicht dem Opfer?“

Harry zuckte mit den Schultern. „Wenn ich raten müsste, weil der Schläger jemand \emph{anders sehr verletzt hatte, und das Opfer des Schlägers \emph{sie} nur ein bisschen verletzt hat, und sich das für sie irgendwie viel unverzeihlicher anfühlte. Oder, wenn ich es ganz direkt formuliere, war der Schläger attraktiv? Oder, vielleicht reich?“}

Es entstand wieder eine Pause.

„Beides,“ sagte Severus.

„Und da haben Sie es,“ sagte Harry. „Nicht, dass ich selbst schon das Verhalten von Teenagern kennengelernt habe, aber meine Bücher habe mir beigebracht, dass eine bestimmte Art von Mädchen auf eine einzige Kränkung eines einfachen oder armen Jungen wutentbrannt reagieren würde, während sie trotzdem einen Platz in ihrem Herzen finden, um dem reichen und attraktiven Jungen seine Anmassungen zu vergeben. In anderen Worten, sie war oberflächlich. Erklären Sie wer es auch war, dass sie seiner nicht würdig gewesen war und er über sie hinweg kommen müsste und in Zukunft sich lieber mit Mädchen verabreden sollte, die eher tiefgründig statt hübsch sind.“

Severus starrte Harry still an, seine Augen funkelnd. Das Lächeln war verschwunden und obwohl Severus Gesicht zuckte, tauchte es nicht wieder auf.

Harry begann sich nervös zu fühlen. „Um, es natürlich nicht so, dass irgendwie Erfahrung auf diesem Gebiet habe, aber ich glaube es ist dem ähnlich, was ein weiser Berater in meinen Büchern sagen würde.“

Andauernde Stille und weiteres Funkeln der Augen.

Es war wahrscheinlich an der Zeit das Thema zu wechseln.

„Also,“ sagte Harry. „Habe ich denn Test bestanden, was er auch war?“

„Ich glaube,“ sagte Severus, „wir sollten keine weiteren Unterhaltungen führen, Potter, und es sei Ihnen angeraten niemals von dieser hier zu sprechen.“

Harry blinzelte. „Würde es Ihnen etwas ausmachen, mir zu sagen was ich falsch gemacht habe?“

„Sie haben mich erzürnt,“ sagte Severus. „Und ich glaube nicht länger an ihre Gerissenheit.“

Harry starrte Severus ziemlich bestürzt an.

„Aber Sie haben mir einen gut gemeinten Rat gegeben,“ sagte Severus Snape, „und daher auch Ihnen im Gegenzug einen echten Ratschlag erteilen.“ Seine Stimme war beinahe völlig ruhig. Wie ein fast horizontal gespannte Schnur, trotz eines massiven Gewichts in der Mitte, da mehrere Millionen Tonnen an Spannkraft an beiden Seiten zogen. „Sie sind heute fast gestorben, Potter. In Zukunft, teile sie ihre Weisheit nicht mit Anderen, solange Sie nicht exakt wissen, worüber beide reden.“

Harrys Verstand erstellte schließlich die Verbindung.

„\emph{Sie waren der----“}

Harry Mund klappte zu als das \emph{fast gestorben eingesunken war, zwei Sekunden später.}

„Ja,“ sagte Severus, „das war ich.“

Und die schreckliche Spannung floss in den Raum zurück, wie Wasser mit dem Druck, den man ihn am Boden des Ozeans auffinden würde.

Harry konnte nicht atmen.

\emph{Verliere. Jetzt.}

„Ich wusste es nicht,“ flüsterte Harry. „Es tut mir----“

„Nein,“ sagte Severus. Nur dieses eine Wort.

Harry stand dort ohne einen Ton von sich zugeben, in Gedanken hektisch nach Optionen suchend. Severus stand zwischen ihm und dem Fenster, was wirklich schade war, denn ein Fall aus dieser Höhe würde einen Zauberer nicht töten.

„Ihre Bücher haben Sie betrogen, Potter,“ sagte Severus, weiterhin mit gepresster Stimme, auf die scheinbar mehrere Tonnen Druck wirkten. „Sie haben Ihnen nicht die einzige Sache beigebracht, die Sie wissen müssen. Sie können aus Geschichten nicht lernen, wie es ist einen geliebten Menschen zu verlieren. Das kann man nicht verstehen, ohne es selbst zu erleben.

„Mein Vater,“ flüsterte Harry. Es war seine beste Vermutung, die einzige Sache, die ihn vielleicht noch retten könnte. „Mein Vater hat versucht, Sie von den Schlägern zu beschützen.“

Ein grauenhaftes Lächeln erstreckte sich über Severus Mund, und der Mann bewegte sich auf Harry zu.

Und an ihm vorbei.

„Lebewohl, Potter,“ sagte Severus, ohne nochmal zurückzublicken. „Wir sollten in Zukunft einander wenig zu sagen haben.“

Und an der Ecke hielt der Mann an, und sprach, ohne sich umzudrehen, ein letztes Mal.

„Dein Vater war der Schläger,“ sagte Severus Snape, „und was deine Mutter in ihm sah, war etwas, dass ich nie begriffen habe, bis heute.“

Er verschwand.

Harry drehte sich um und ging zum Fenster. Seine zitternden Hände platzierte er auf den Sims. \emph{Gebe niemals jemand deinen weisen Rat, \emph{solange Sie nicht exakt wissen, worüber beide reden.} \emph{Verstanden.}}\\ Harry blickte eine Weile nach draußen auf die Wolken und Sonnenstrahlen. Das Fenster zeigte ihm die östlichen Ländereien, und es war Nachmittag, sodass wenn Sonne durch die Wolken überhaupt sichtbar war, Harry sie nicht sehen konnte.

Seine Hände hatten aufgehört zu zittern. Aber Harry hatte ein einengendes Gefühl in seiner Brust als ob sie von Metallbändern zusammengequetscht würde.

Also war sein Vater ein Schläger gewesen.

Und seine Mutter war oberflächlich gewesen.

Vielleicht waren sie später erwachsen geworden. Gute Menschen wie Professor McGonagall schienen viel von ihnen zu halten, und es der Grund war vielleicht nicht \emph{nur, dass sie heroische Märtyrer waren.}

Auf jeden Fall war es ein geringer Trost für einen Elfjährigen, der sich fragte welche Art von Teenager er wohl werden würde.

So richtig schrecklich.

So richtig traurig.

Was Harry für ein schreckliches Leben führte.

Zu Lernen, dass seine genetische Eltern nicht perfekt gewesen waren… Natürlich, er sollte einige Zeit damit verbringen, Trübsal zu blasen und sich selbst zu bemitleiden.

Vielleicht könnte er sich bei Lesath Lestrange beklagen.

Harry hatte von Dementoren gelesen. Kälte und Dunkelheit umgab sie, außerdem die Angst, da sie sämtliche glücklichen Gedanken aufsogen und in ihrer Abwesenheit all die schlimmsten Erinnerungen an die Oberfläche drängten.

Er könnte sich in Lesath hineinversetzen, er wusste, dass seine Eltern lebenslänglich in Askaban waren, der Ort, dem niemand jemals entkommen war.

Und Lesath würde sich in seine Mutter hineinversetzen, in die Kälte und die Dunkelheit und die Angst, allein mit all den schlimmsten Erinnerungen, sogar in ihren Träumen, jede Sekunde des Tages.

Für einen Augenblick stellte Harry sich vor, seine eigene Mutter und sein eigener Vater wären in Askaban, mit Dementoren, die ihnen ihr Leben aussaugen, ihnen die glücklichen Erinnerungen an ihre Liebe für ihn entzogen. Jedoch nur für einen Augenblick, bevor in seiner Vorstellungskraft eine Sicherung durchbrannte und eine Notfallabschaltung in Gang setzte und im auftrug niemals wieder sich sowas vorzustellen.

War es rechtens dies jemandem anzutun, auch wenn es die zweitschlimmste Person auf Erden ist?

\emph{Nein, sagte die Weisheit von Harrys Büchern, nicht wenn es eine andere Möglichkeit gibt, irgendeine andere Möglichkeit.}

Und solange das Justizsystem der Zauberer nicht genauso perfekt war wie ihre Gefängnisse----und das klang, alles in allem, ziemlich unwahrscheinlich----war irgendwo in Askaban eine völlig unschuldige Person, und wahrscheinlich mehr als eine.

Harry schnürte etwas die Kehle zu und Feuchtigkeit sammelte sich in seinen Augen, und er wollte alle Insassen von Askaban in Sicherheit teleportieren und Feuer vom Himmel herabbeschwören und diesen schrecklichen Ort bis zum Grundgestein niederbrennen. Aber er konnte es nicht, denn er war nicht Gott.

Und Harry erinnerte sich an die Worte von Professor Quirrell, gesprochen unter Sternenlicht:\\ Manchmal, wenn diese makelbehaftete Welt mir ungewöhnlich hasserfüllt scheint, dann frage ich mich, ob es irgendeinen anderen Ort geben könnte, weit entfernt, an dem ich sein sollte… Aber die Sterne sind so weit, weit entfernt… Und ich frage mich, wovon ich träumen würde, wenn ich eine lange, lange Zeit schliefe.

Gerade erschien ihm diese makelbehaftete Welt ungewöhnlich hasserfüllt.

Und Harry konnte Professor Quirrells Worte nicht verstehen, es hätten die Worte eines Außerirdischens gewesen sein können, oder die einer künstlichen Intelligenz, etwas dessen Bauplan sich so von Harrys unterschied, dass sein Gehirn nicht dazu bringen konnte, in diesem Modus zu arbeiten.

Man seinen Heimatplaneten nicht verlassen, wenn er einen Ort wie Askaban enthielt.

Man müsste bleiben und kämpfen.

