

\hypertarget{machiavellische-intelligenz-hypothese}{% \section{3. Machiavellische Intelligenz-Hypothese}\label{machiavellische-intelligenz-hypothese}}

\textbf{Kapitel 3}

\textbf{Machiavellische Intelligenz-}Hypothese

Act 3:

Draco wartete mit aufgewühltem Magen in einer Alkove mit kleinem Fenster, die er in der Nähe der Großen Halle gefunden hatte.

Das wird Kosten haben, und sie würden nicht gering werden. Draco wusste das, seitdem er aufgewacht war und verstanden hatte, dass er sich nicht traute die Große Halle fürs Frühstück zu betreten, weil er vielleicht Harry Potter dort sehen würde und Draco nicht wusste, was danach passieren wird.

Schritte näherten sich.

„Hier ist er,“ sagte Vincents Stimme. „Aber, der Boss ist heute nicht guter Stimmung, pass also auf wo du hintrittst.“

Draco würde diesem Idioten die Haut abziehen und den gehäuteteten Körper, wie eine tote Rennmaus, mit einer Anfrage nach einem intelligenterem Bediensteten zurücksenden.

Ein Teil der Fußschritte entfernte sich, während die Restlichen näher kamen.

Das aufwühlende Gefühl in Dracos Magen verschlimmerte sich.

Harry bewegte sich in sein Sichtfeld. Sein Gesicht war angestrengt neutral, aber seine blau-besetzten Umhänge sahen merkwürdig schief aus, als ob sie nicht richtig angezogen worden wären \later

„\emph{Deine Hand,“} sagte Draco ohne auch nur darüber nachzudenken.

Harry hob seinen linken Arm, so als ob er ihn selbst betrachten wollte.

Die Hand baumelte schlaff an ihm, wie etwas Totes.

„Madam Pomfrey hat gesagt es sei nicht permanent,“ sagte Harry leise. „Sie sagte, sie sollte fast vollständig wiederhergestellt sein, wenn der Unterricht morgen wieder beginnt.“

Für einen kurzen Moment kam die Neuigkeit als Erleichterung.

Und dann verstand Draco es.

„Du bist zu Madam Pomfrey gegangen,“ flüsterte Draco.

„Natürlich bin ich das,“ sagte Harry, als ob er nur das Offentsichtliche aussprach. „Meine Hand hat nicht funktioniert.

Es fiel Draco langsam auf was für ein \emph{vollkommender} Idiot er doch gewesen war, viel schlimmer als die älteren Slytherins, die er zusammengestaucht hatte.

Er hatte es einfach als gegeben angenommen, dass niemand die Obrigkeiten verständigen würde, wenn ein Malfoy ihnen etwas angetan hatte. Dass niemand Lucius Malfoys Auge jemals auf sich haben wollte.

Aber Harry Potter war kein verängstigter kleiner Hufflepuff, der versuchte sich aus dem Spiel herauzuhalten. Er nahm bereits teil und Vaters Auge war schon auf ihm.

„Was hat Madam Pomfrey sonst noch gesagt?“ fragte Draco, das Herz ihm bis zum Hals schlagend.

„Professor Flitwick hat gesagt, der Zauber, der auf meine Hand eingesetzt wurde, wäre ein dunkler Folterzauber und eine extrem ernste Angelegenheit und die Weigerung den Täter zu sagen, wäre völlig inakzeptabel.“

Es entstand eine längere Pause.

„Und dann“ sagte Draco mit schwankender Stimme.

Harry Potter lächelte etwas. „Ich habe mich vielmals entschuldigt, sodass Professor Flitwick sehr streng geschaut hat, worauf ich dann ihm erzählt habe, diese ganzen Sache sei in der Tat eine extrem ernste, geheime, \emph{vertrauliche} Angelegenheit und dass ich den Schulleiter schon über das Projekt informiert hätte.“

Draco keuchte. „Nein! Flitwick wird das nicht einfach akzeptieren. Er wird es mit Dumbledore überprüfen.“

Draco zitterte jetzt. Wenn Dumbledore Harry Potter vor den Zauberergamot brachte, willentlich oder nicht, und den Jungen-der-überlebte unter dem Einfluss von Veritaserum gestehen ließ, Draco habe ihn gefoltert… zu viele Leute liebten Harry Potter. Vater könnte die Abstimmung verlieren…

Vater wäre vielleicht in der Lage Dumbledore zu überzeugen das nicht zu tun, das würde aber etwas kosten. Schrecklich viel kosten. Das Spiel hatte nun Regeln, man konnte nicht mehr einfach jemand beliebigen bedrohen. Doch Draco war durch seinen eigenen Willen in Dumbledores Hände gefallen. Und Draco war eine wertvolle Geisel.

Obwohl er nicht so viel wert war wie Vater dachte, da er jetzt nie ein Todesser werden konnte.

Dieser Gedanke riss an seinem Herz wie ein Schneidezauber.

„Und dann?“ flüsterte Draco.

„Dumbledore folgerte sofort, dass du es warst. Er wusste, dass wir zusammenhängen.“

Das schlechtest mögliche Szenario. Wenn Dumbledore nicht erraten hatte, wer es getan hätte, hätte er möglicherweise es nicht riskiert Legilimentik einzusetzen nur um es herauszufinden… aber wenn er es \emph{wusste…}

„Und?“ Draco zwang sich das Wort auszusprechen.

„Wir haben ein kleine Gespräch geführt.“

„Und?“

Harry Potter grinste. „Und ich habe ihm erklärt, es wäre in seinem besten Interesse nichts zu tun.“

Dracos Verstand prallte direkt in eine Ziegelmauer und spritzte in alle Richtungen. Er starrte Harry Potter nur noch an, wie ein Idiot mit nachlässig offenstehendem Mund.

Er brauchte einen Moment, um sich zu erinnern.

Harry kannte Dumbledores mysteriöses Geheimnis, jenes, welches auch Snape als Druckmittel verwendete.

Draco konnte es jetzt vor sich sehen. Dumbledore mit sehr ernstem Blick, seinen Eifer mühsam verschleiernd, während er Harry erklärte, was für eine ernste Angelegenheit dies war.

Und Harry, wie er höflich Dumbledore verdeutlichte, er solle die Klappe halten, wenn er nur wüsste was gut für ihn wäre.

Vater hatte ihn vor dieser Art von Menschen gewarnt, jemand, der dich vollständig ruinieren konnte, währendessen jedoch so sympathisch war und es daher erschwerte ihn richtig zu hassen.

„Nachdem,“ sagte Harry, „der Schulleiter Professor Flitwick erzählte, dies wäre in der Tat eine geheime und empfindliche Angelegenheit über die er bereits informiert wurde und dass, er nicht glaube es würde mir oder irgendjemand helfen es zu diesem Zeitpunkt weiter zu bedrängen. Professor Flitwick fing dann an etwas darüber sagen, dass die Pläne des Schulleiters zu weit gingen, sodass ich ihn unterbrechen müsste und ihm darlegte, dies wäre meine \emph{eigene} Idee gewesen, also nichts wozu der Schulleiter mich gezwungen hatte, sodass Professor Flitwick herumfuhr und schließlich begann \emph{mich} zu belehren, sodass der Schulleiter \emph{ihn} unterbrach und sagte, als Junge-der-überlebte wäre ich ohnehin dazu verdammt merkwürdige und gefährliche Abenteuer zu erleben, ich also sicherer wäre, in diese mit Absicht zu geraten, statt zu warten bis sie ausversehenpassieren, und dann warf Professor Flitwick seine kleinen Hände ihn die Höhe und begannnun an uns \emph{beide} gewandt in einer äußerst hohen Stimme zu kreischen, es würde ihn nicht kümmern, was wir hier vorbereiten, aber dass dies nie wieder vorzukommen habe, solange ich in Ravenclaw wäre, weil er mich sonst rauswerfen würde, damit ich nach Gryffindor gehen könne, wohin all dieses „dumbledoren“ gehöre \later“

Harry machte es \emph{sehr} schwer für Draco ihn zu hassen.

„Jedenfalls,“ sagte Harry „wollte ich nicht aus Ravenclaw herausgeworfen werden, sodass ich Professor Flitwick versprach, nichts dergleichen würde wieder passieren und falls doch, ich ihm dann schon sagen würde, wer es getan hatte.“

Harrys Augen hätten eiskalt sein müssen. Waren sie aber nicht. Die Stimme hätte eine tödliche Drohung aussprechen müssen. Machte sie aber nicht.

Und Draco erkannte die Frage, die eigentlich naheliegend sein sollte, und es verdarb die Stimmung sofort.

„Warum… hast du es ihnen nicht gesagt?“

Harry ging herüber zum Fenster, hinein in den kleinen Strahl Sonnenlicht, der in die Alkove schien und wandte den Kopf nach draußen, zu den grünen Ländereien von Hogwarts. Die Helligkeit strahlte auf ihn, auf seinen Umhang, auf sein Gesicht.

„Warum habe ich es ihnen nicht gesagt…?“ fragte Harry. Seine Stimme fing sich wieder. „Wahrscheinlich weil ich nicht wütend auf dich werden konnte. Ich wusste, ich hatte dich zuerst verletzt. Ich würde es nicht einmal fair nennen, da meine Taten schlimmer waren als deine.“

Es war als ob er wieder in eine Ziegelmauer gelaufen wäre. Harry hätte Altgriechisch sprechen können, so viel verstand Draco vom Gesagten.

Dracos Verstand wühlte nach Strukturen und fand keine. Diese Aussage war ein Zugeständnis außerhalb Harrys eigener Interessen. Es war nicht einmal etwas, was Harry sagen würde, um Draco zu einem noch loyaleren Gefolgsmann zu machen, jetzt da Harry Macht über ihn erlangt hatte. Dafür sollte er betonen wie freundlich er gewesen war, nicht wie sehr er Draco verletzt hatte.

„Trotzdem,“ sagte Harry, und seine Stimme war diesmal leiser, fast ein Flüstern, „bitte mach das nie wieder, Draco. Es hat weh getan und ich bin mir nicht sicher, ob ich dir noch ein zweites Mal vergeben konnte. Ich bin mir nicht sicher, dass ich noch in der Lage wäre es zu wollen.“

Draco kapierte es schlicht nicht.

Versuchte Harry ein \emph{Freund} von ihm zu sein?

Es konnte doch nicht sein, dass Harry Potter dumm genug sein könnte, zu glauben dies wäre nach dem was er getan hatte noch möglich.

Man konnte jemandes Freund und Verbündeter sein, so wie Draco es mit Harry versucht hatte, oder man konnte das Leben des Anderen zerstören und ihm keine Optionen mehr eröffnen.

Nicht beides.

Aber dann verstand Draco nicht, was Harry sonst probieren \emph{könnte}.

Und ein merkwürdiger Gedanke fiel Draco ein, etwas über das Harry gestern die ganze Zeit gesprochen hatte.

Der Gedanke lautete: \emph{Teste es}.

\emph{Du bist jetzt als Wissenschaftler erwacht,} hatte Harry gesagt, \emph{und selbst wenn du nie lernst deine Kräfte einzusetzen, wirst du immer, nach Wegen suchen, deine Überzeugungen, zu überprüfen…}

Diese unheilvollen Worte, gesprochen in von Qual durchsetztem Keuchen, hatten sich immer wieder in Dracos Kopf wiederholt.

Wenn Harry \emph{wirklich} vorgab ein reumütiger Freund zu sein, der ohne Absicht jemand verletzt hatte…

„Du hast es \emph{geplant} gehabt!“ sagte Draco, wobei er es schaffte einen Anklang von Anklage in seine Stimme zu bringen. „Du hast es nicht getan, weil du wütend warst, nur weil du es \emph{gewollt} hattest!“

\emph{Idiot, würde Harry Potter sagen,} natürlich habe ich es geplant, und jetzt bist du mein\later

\emph{Harry drehte sich zurück zu Draco. „Was gestern passiert ist,} war nicht der Plan,“ sagte Harry, die Stimme scheinbar in seiner Kehle feststeckend. „Der Plan war dir beizubringen, wieso es stets besser ist die Wahrheit zu kennen und dann gemeinsam zu versuchen die Wahrheit über Blut zu entdecken und dann jegliche Antwort zu akzeptieren. Gestern habe ich… uns gehetzt.“

\emph{„}Stets besser die Wahrheit zu kennen,“ sagte Draco kalt. „Als ob du mir einen Gefallen getan hättest.“

\emph{Harry nickte, womit er Draco vollig umhaute, und sagte, „Was wenn Lucius die gleiche Idee einfällt, die ich auch hatte, nämlich dass stärkere Zauberer weniger Kinder haben? Er würde vielleicht eine Kampagne starten, die die mächtigsten Reinblüter bezahlt, damit die mehr Kinder haben. Tatsächlich sollte Lucius es tun, wenn das Reinblütertum richtig} wäre -- das Problem auf seiner Seite angreifen, wo er Einfluss nehmen kann. Aktuell, Draco, warst du der einzige Freund, den Lucius hat, der versuchen würde ihn aufzuhalten diesen Aufwand zu verschwenden, da du der einzige bist, der die richtige Wahrheit kennt und damit die richtigen Resultate vorhersagen kann.“

Draco fiel ein, dass Harry in einem so merkwürdigen Umwelt aufgewachsen war, er also eher ein magisches Geschöpf war, anstatt ein Zauberer. Draco konnte schlicht nicht erraten, was Harry als nächstes sagen würde.

„\emph{Warum?“} sagte Draco. Schmerz und Verrat in seiner Stimme unterzubringen war überhaupt nicht schwer. „Warum \emph{hast} du mir das angetan? Was \emph{war} dein Plan?“

„Also,“ sagte Harry, „du bist Lucius' Erbe, und glaube es oder nicht, Dumbledore denkt ich gehöre zu ihm. So könnten wir aufwachsen und ihre Kämpfe untereinander austragen. \emph{Oder} wir machen etwas anders.“

Langsam, verarbeitete Dracos Verstand dies. „Du willst einen endgültigen Kampf der beiden provozieren, dann die Macht ergreifen, während sie noch geschwächt sind.“ Draco fühlte die kalte Furcht in seiner Brust. Er würde versuchen müssen dies zu verhindern, unabhängig seiner eigenen Kosten.

Aber Harry schüttelte seinen Kopf. „Sterne im Himmel, \emph{nein}!“

„Nein…?“

„Da würdest du nicht mitmachen und ich auch nicht,“ sagte Harry. „Dies ist \emph{unsere} Welt, wir wollen sie nicht zerstören. Aber stell es dir vor, sagen wir, Lucius nimmt an die Verschwörung wäre dein Werkzeug und ich auf seiner Seite, Lucius also denkt, du hättest mich gedreht, während Dumbledore glaubt die Verschwörung gehöre zu mir, Dumbledore also denkt, ich hätte dich gedreht und so sie beide uns helfen würde, aber nur in solcher Art, die der andere nicht bemerkt.“

Draco musste nicht mal vorgeben sprachlos zu sein.

Vater hatte ihn einmal zu einem Theaterstück mitgenommen, namens \emph{Die Tragödie des Lichts}, über den \emph{unglaublicher} cleveren Slytherin, Light, der auszog die Welt von allem Bösen zu reinigen, mithilfe eines uralten Ringes, der jeden töten konnte, dessen Name und Gesicht man kannte, dem ein anderer unglaublich cleverer Slytherin, ein Bösewicht namens Lawliet, der eine Tarnung trug um sein wahres Gesicht zu verbergen; und Draco hatte an den richtigen Stellen geschrieen und gejubelt, besonders in der Mitte; und dann hatte das Stück traurig geendet und Draco war sehr enttäuscht gewesen und Vater hatte ihn freundlich darauf hingewiesen, dass das Wort „Tragödie“ direkt dort im Namen stünde.

Anschließend, hatte sein Vater Draco gefragt, ob er verstanden hatte warum sie dieses Stück gesehen hatten.

Draco hatte geantwortet, es wäre um ihm beizubringen so listig zu sein wie Light und Lawliet, wenn er erwachsen war.

Vater hatte gemeint, Draco hätte nicht falscher liegen können, und ihn darauf hingewiesen, dass Lawliet zwar geschickterweise sein Gesicht verborgen hatte, es jedoch keinen guten Grund für ihn gab Light seinen \emph{Namen} zu verraten. Vater war fortgefahren, indem er fast jeden Teil des Stückes vernichtete, während Draco mit immer größer werdenden Augen zuhörte. Und schließlich schloss sein Vater mit der Aussage, Stücke jener Art wären \emph{stets} unrealistisch, da falls der Bühnenautor gewusst hätte, was jemand so schlaues wie Light tun würde, der Bühnenautor selbst versucht hätte die Welt zu beherrschen, statt Theaterstücke darüber zu schreiben.

Dann hatte Vater Draco von der „Regel der Drei“ erzählt, nämlich dass jeglicher Plan, der das Geschehen von mehr als drei verschiedene Dinge erforderte, im realen Leben niemals funktionieren würde.

Vater hatte \emph{außerdem} erklärt, da nur ein Narr jemals einen Plan probieren würde, der so \emph{kompliziert wie möglich} war, das richtige Limit 2 wäre.

Draco konnte nicht einmal Worte finden, um die schier gewaltige \emph{Unausführbarkeit} von Harry Masterplan zu beschreiben.

Aber es wäre \emph{genau} die Art von Fehler, die man machen würde, wenn man keine Mentoren hätte, dächte man wäre clever und alles über das Pläne schmieden aus Theaterstücken gelernt hätte.

„Also,“ sagte Harry, „was denkst du von diesem Plan?“

„Er ist clever…“ entgegnete Draco langsam. \emph{Brillant!} zu schreien und vor Bewunderung zu keuchen, würde zu auffällig aussehen. „Harry, kann ich dich etwas fragen?“

„Natürlich,“ bestätigte Harry.

„Warum hast du Granger diesen teuren Beutel gekauft?“

„Um keine Abneigung zu zeigen,“ erwiderte Harry sofort. „Außerdem erwarte ich, dass sie sich unbehaglich damit fühlen würde, kleinere Gefallen, die ich über die nächsten Monate erbitte, auszuschlagen.

Und in diesem Moment bemerkte Draco, dass Harry wirklich versuchte sein Freund zu sein.

Harrys Zug gegen Granger war schlau gewesen, vielleicht sogar brilliant. Sorge dafür, dass dich dein Feind nicht verdächtigt, \emph{und} setze sie mit deiner freundschaftlicher Art in deine Schuld, sodass du sie Position steuern kannst, indem du \emph{einfach fragst}. Draco wäre damit nicht davongekommen, sein Ziel wäre argwöhnisch geworden, doch der Junge-der-überlebte \emph{konnte} es. Also war der erste Schritt von Harrys Plan dem Feind ein teures Geschenk zu geben, Draco wäre das nicht eingefallen, doch es könnte \emph{funktionieren…}

Wenn du Harrys Feind warst, könnte es zunächst schwierig sein, seine Pläne zu durchschauen; sie könnten sogar dumm wirken, aber sein Begründungen würden schließlich \emph{Sinn ergeben,} nachdem du sie verstanden hast; du würdest begreifen, dass er versuchte dir zu schaden.

Die Art in der Harry sich jetzt vor ihm aufführte ergab \emph{keinen} Sinn.

Denn wenn du Harrys Freund bist, dann würde Harry versuchen dein Freund zu sein in der fremdartigen, unverständlichen Weise, die er bei den Muggeln gelernt hatte, sogar wenn es bedeutete dein gesamtes Leben zu zerstören.

Die Stille streckte sich.

„Ich weiß, ich habe unsere Freundschaft schrecklich missbraucht,“ sagte Harry letztendlich. „Aber verstehe bitte, Draco, ich wollte uns beide gemeinsam die Wahrheit ergründen lassen. Kannst du mir das verzeihen?

Eine Gabelung mit zwei Wegen, doch nur ein Pfad, bei dem man leicht umkehren konnte, sollte Draco seine Meinung jemals ändern…

„Ich glaube ich verstehe, was du probiert hast,“ log Draco. „also, ja.“

Harrys Augen erhellten sich. „Es freut mich das zu hören, Draco,“ sagte er sanft.

Die zwei Schüler standen in der Alkove, Harry noch immer in den einzelnen Sonnenstrahl getaucht, Draco im Schatten.

Und Draco bemerkte mit einem Anklang von Horror und Verzweifelung, dass obwohl es in der Tat ein angsteinflössendes Schicksal war Harrys Freund zu sein, er jetzt so viele verschiedene Wege hatte, um Draco zu bedrohen, dass sein Feind zu sein noch \emph{schlimmer} sein könnte.

Wahrscheinlich.

Vielleicht.

Naja, er konnte immer noch die Seite wechseln.

Er war geliefert.

„So,“ sagte Draco. „Was jetzt?“

„Wir lernen nächsten Samstag weiter?“

„Es wird hoffentlich nicht so verlaufen wie das letzte Mal \later“

„Keine Sorge, das wird es nicht,“ sagte Harry. „Noch einige Samstage wie \emph{dieser} und du wärst weiter als \emph{ich}.“

Harry lachte. Draco nicht.

„Oh, und bevor du gehst,“ sagte Harry, verlegen grinsend. „Ich weiß, dies ist vielleicht ein schlechter Zeitpunkt, aber ich wollte dich eigentlich nach deinem Rat über etwas fragen.“

„Okay,“ sagte Draco, noch immer ein wenig von dieser letzten Aussage abgelenkt.

Harrys Augen wurden entschlossen. „Diesen Beutel für Granger zu kaufen hat den Großteil des Geldes verbraucht, das ich aus Gringotts gestohen habe \later“

Was.

„\laterund McGonagall hat den Verliesschlüssel, oder vielleicht hat ihn auch Dumbledore. Und ich war gerade dabei einen Plan zu starten, der möglicherweise etwas Geld benötigte, also habe ich mich gefragt, ob du weißst, wie ich Zugriff auf mein Verließ erreich \later“

„Ich leihe dir das Geld,“ sagte Dracos Mund in schierem, existenziellem Reflex.

Harry sah überrascht aus, aber in einer erfreuten Weise. „Draco, du musst das nicht \later“

„Wie viel?“

Harry nannte die Menge und Draco konnte nicht ganz den Schock auf seinem Gesicht verbergen. Das war fast sein gesamtes Taschengeld, welches Vater ihm für das ganze Jahr gegeben hatte, Draco würde nur noch einige Galleonen übrig haben \later

Dann kickte sich Draco innerlich. Er musste nur Vater schreiben und erklären, das Geld sei weg, da es geschafft hatte es \emph{Harry Potter zu leihen}, und Vater würde ihm eine, in goldener Tinte geschriebene, besondere Glückwunschkarte, einen riesigen Schokofrosch, den er zwei Wochen lang essen würde, sowie zehnmal so viele Galleonen schicken, nur für den Fall, dass Harry einen weiteren Kredit benötigte.

„Es ist viel zu viel, oder?“ sagte Harry. „Es tut mir leid, ich hätte nicht fragen sollen \later“

„Entschuldige mal, ich \emph{bin} ein Malfoy, weißst du,“ sagte Draco. „Mich hat nur überrascht wie viel du \emph{wolltest.“}

\emph{„}Keine Sorge,“ sagte Harry Potter fröhlich. „Es wird nicht die Interesse deiner Familie bedrohen, es ist nur ich, wie ich böse bin.“

Draco nickte. „Dann also, kein Problem. Möchtest du jetzt gleich haben?“

„Klar,“ sagte Harry.

Als sie die Alkove verließen und auf die Kerker zusteuerten, konnte Draco sich die Frage nicht verkneifen: „Also, \emph{kannst} du mit sagen, für welchen Plan das ist?“

„Rita Kimmkorn.“

Draco fluchte innerlich, doch es war zu spät nein zu sagen.

Als sie die Kerker erreicht hatten, hatte Draco seine Gedanken wieder gesammelt.

Er \emph{hatte} Schwierigkeiten Harry Potter zu hassen. Harry \emph{hatte} versucht freundlich zu sein, war jedoch schlicht verrückt.

Und das würde Dracos Rache nicht aufhalten oder auch nur verlangsamen. „Also,“ sagte Draco, nachdem er sich umgeschaut hatte, um sicher zu stellen, dass niemand in der Nähe war. Ihre Stimmen würden natürlich beide undeutlich klingen, doch es schadete nie extra vorsichtig zu sein.

„Ich habe nachgedacht. Wenn wir neue Rekruten in die Verschwörung bringen, müssen sie \emph{glauben} wir wären gleichrangig? Sonst bräuchte man nur \emph{einen,} um den Plan Vater zu verraten. Daran hast schon gedacht, oder?

„Selbstverständlich,“ antwortete Harry.

„\emph{Werden} wir gleichrangig sein?“ fragte Draco.

„Ich fürchte eher nicht,“ erwiderte Harry. Es zeigte eindeutig, wie Harry versuchte sanft zu klingen, aber auch, dass er versuchte eine beträchtige Menge Herablassung zu unterdrücken, es jedoch nicht ganz schaffte. „Es tut mir leid, Draco, aber du weißst zurzeit nicht einmal für war das Wort Bayes'che in Bayes'che Verschwörung steht. Du wirst noch einige Monate lernen müssen, bis jemand anderen aufnehmen, nur damit eine \emph{gute Fassade} aufsetzen kannst.“

„Weil ich noch nicht genügend Wissenschaften kenne,“ sagte Draco, vorsichtig seine Stimme neutral haltend.

Harry schüttelte seinen Kopf. „Das Problem ist nicht dein Unwissen über spezifische Wissenschaftssachen wie Desoxyribonukleinsäure. \emph{Das} würde dich nicht daran hindern, gleichrangig zu sein. Das Problem ist vielmehr dein fehlendes Training in den Methoden des rationalen Denkens, das tiefliegende Geheimnis, wie all diese Entdeckungen gemacht wurden. Ich werde \emph{probieren} dir diese beizubringen, doch sie sind sehr viel schwieriger zu lernen. Erinnere dich an gestern, Draco. Ja, du hast mitgearbeitet. Aber ich hatte die Kontrolle. Du hast einige Fragen beantwortet. Ich habe sie alle gefragt. Du hast bei der Ausführung geholfen. Ich allein habe sie gelenkt. Und ohne die Methoden des rationalen Denkens, Draco, kannst auf keinen Fall die Verschwörung in die richtige Richtung lenken.“

„Ich verstehe,“ sagte Draco mit enttäuschter Stimme.

Harrys Stimme versuchte noch sanfter zu werden. „Ich werde probieren deine Expertise zu respektieren, zum Beispiel beim Zeugmit anderen Leuten. Aber auch du musst meine Expertise respektieren, und es gibt keine Möglichkeit in der du gleichrangig wärst, im Bezug auf das Lenken der Verschwörung. Du bist erst seit \emph{einemTag} ein Wissenschaftler, du kennst \emph{ein} Geheimnis über Desoxyribonukleinsäure, und hast noch kein Trainung in den Methoden des rationen Denkens gehabt.“

„Ich verstehe,“ sagte Draco.

Und das tat er.

\emph{Zeug mit anderen Leuten, hatte Harry gesagt. Die Kontrolle über die Verschwörung zu übernehmen würde wahrscheinlich nicht einmal schwierig werden. Und danach, würde er Harry Potter ermorden nur um sicher zu gehen \later}

Die Erinnerung, wie krank es sich letzte Nacht angefühlt hatte, von Harrys Schreien zu wissen, jedoch nichts zu tun, stieg in ihm auf.

Draco fluchte wieder innerlich.

\emph{Gut. Er} würde Harry nicht ermorden. Harry war bei Muggeln aufgewachsen, es war nicht sein Fehler, dass er verrückt war.

Stattdessen, würde Harry weiter leben, nur damit Draco im erzählte konnte, es wäre alles zum Besten von Harry selbst; ganz ehrlich, er sollte dankbar sein \later

\emph{Und mit einem plötzlichen Zucken überraschteter Freude, bemerkte Draco, dass es} wirklich zu Harrys Bestem war. Falls Harry versuchte seinen Plan auszuführen, versuchte Vater und Dumbledore auszutricksen, würde er sterben.

Das machte es perfekt.

Draco würde Harry all seine Träume entreißen, genauso wie er es ihm angetan hatte.

Draco würde ihm erzählen, es wäre alles zu seinem Besten gewesen, und es würde völlig richtig sein.

Draco würde die Veschwörung und die Macht der Wissenschaft führen, um die Zaubererwelt zu säubern und Vater würde so stolz auf ihn sein, als ob er ein Todesser gewesen wäre.

\emph{Harry Potters} böse Pläne würden vereitelt werden und die Kräfte des Guten würden bestehen.

\emph{D}ie perfekte Rache.

Es sei denn…

\emph{„Gebe nur vor, vorzugeben ein Wissenschaftler zu sein.“hatte Harry ihm gesagt.}

Draco konnte nicht in Worte fassen, was in Harrys Verstand falsch lief \later

\emph{(da er noch nie den Begriff} Tiefe der Rekursion gehört hatte)

\emph{\later doch er konnte sich vorstellen, welcher Art von Plänen e}s implizierte.

\emph{… es sei denn, dies wäre genau das, was Harry im Zuge eines noch} größeren Plans bei Draco erreichen wollte, er also in Harrys Hände spielen würde, wenn er versuchte diesen zu vereiteln; vielleicht wusste Harry sogar, dass sein Plan unausführbar war; dass er keinen Zweck hatte, außer Draco zu ködern ihn zu durchkreuzen \later

\emph{Nein. Das war unprofessioneller} Wahnsinn. Es musste eine Grenze geben. Der Dunkle Lord selbst war nicht so gewieft gewesen. Diese Art von Dingen passierte nicht im wahren Leben, nur in Vaters

albernen Gutenachtgeschichten über törichte Wasserspeier, die schließlich jedes Mal den Plan des Helden unterstützen, wenn sie versuchten ihn aufzuhalten.

\emph{Und neben Draco lief Harry her mit einem Lächeln auf seinem Gesicht, über die ev}olutionären Ursprünge der menschlichen Intelligenz nachdenkend.

\emph{Zu Beginn, bevor die Menschen genau verstanden hatte, wie Evolution funktionierte,} waren sie mit verrückten Ideen daher gekommen wie die menschliche Intelligenz entwickelte sich, damit wir bessere Werkzeuge erfinden konnten.

\emph{Der Grund warum dies verrückt war, war dass nur eine Person eines Stammes dieses Werkzeug erfinden musste, dann es aber jeder andere verwenden würde und es sich auch zu anderen Stämmen verbreiten würde und noch immer von ihren Nachfahren hunderte Jahre später verwendet werden würde. Dies war aus der Sicht des wissenschaftlichen Fortschritte exzellent, unter evolutionären Gesichtspunkte}n bedeutete es jedoch, dass der Erfinder selbst kaum einen Fitniss-Vorteil hatte, also nicht viel mehr Kinder hatte, als jeder andere. Nur relative Fitness-Vorteile konnten die relative Häufigkeit eines Genes in der Population erhöhen, und somit eine einzelne Mutation zu dem Punkt führen, in der sie universell war und jeder sie hatte. Und brillante Erfindungen waren schlicht nicht häufig genug, um den durchgehenden Selektionsdruck zu bieten, der nötig war eine Mutation voranzubringen. Es war eine natürliche Vermutung, die Intelligenz wäre vorhanden um die Technologie zu entwicklen, wenn man die Menschen mit ihren Gewehren, Panzern und nuklearen Waffen betrachtete und sie mit den Schimpansen verglich. Eine natürliche Vermutung, wenn auch falsch.

\emph{B}evor die Menschen genau verstanden hatte, wie Evolution funktionierte, waren sie mit verrückten Ideen daher gekommen wie das Klima änderte sich und die Stämme mussten fortziehen und die Menschen mussten schlauer werden, um die neuartigen Probleme zu lösen.

Menschen hatten aber viermal so große Gehirne wie die Schimpansen. 20\% der menschlichen Stoffwechselenergie wurde genutzt um das Gehirn zu unterhalten. Menschen waren \emph{lächerlich} viel schlauer als alle anderen Gattungen. Diese Art an Auffälligkeit trat nicht auf, weil die Umwelt die Schwierigkeit seiner Probleme ein wenig erhöhte. Dann würden die Organismen nur ein wenig schlauer werden müssen um sie zu lösen. Um mit so einem gigantischen, überdimensionierten Gehirn zu enden, musste es einen \emph{unaufhaltbaren} evolutionären Prozess gegeben haben, etwas das ohne Grenzen das Wachstum des Gehirns antrieb und wieder antrieb.

Und heutige Wissenschaftler hatten ziemlich gute Vermutungen, was dieser unaufhaltbare evolutionäre Prozess gewesen war.

Harry hatte einst ein berühmtes Buch mit Namen C\emph{himpanzee} \emph{Politics} gelesen. Das Buch hatte beschreiben wie ein erwachsener Schimpanse namens Luit dem alternen Alpha, Yereon, entgegengetreten war, mit der Hilfe eines jungen, kurzlich gereiften Schimpansen namens Nikkie. Nikkie hatte nicht direkt in die Kämpfe von Luit und Yereon eingegriffen, aber Yereons andere Unterstützer daran gehindert, ihm zur Hilfe zu kommen, indem er sie ablenkte, jedes Mal wenn sich eine Konfrontation zwischen Luit und Yereon entwickelte. Und mit der Zeit hatte Luit gewonnen, und war der neue Alpha geworden, mit Nikkie als den zweitmächtigsten…

… obwohl es nicht lange gedauert hatte, bis Nikkie eine Allianz mit dem besiegten Yereon gebildet hatte, Luit gestürzt hatte und er selbst der \emph{neue} neue Alpha geworden war.

Man musste hier wirklich würdigen, was Millionen von Jahren in denen Hominoiden versuchten einander zu überlisten \later ein evolutionäres Wettrüsten ohne Grenze \later im Bezug auf Vergrößerung der mentalen Kapazität zur Folge hatte.

Da natürlich, ein Mensch dies restlos vorhergesehen hätte.

Und neben Harry lief Draco, ein Lächeln unterdrückend, als er an seine Rache dachte.

Eines Tages, vielleicht erst in einigen Jahren, würde Harry Potter lernen was es bedeutete einen Malfoy zu unterschätzen.

Draco war schon nach einem Tag als Wissenschaftler erwacht. Harry hatte gesagt, das sollte in den ersten Monaten nicht passieren.

Wenn du aber natürlich ein Malfoy warst, würdest du ein mächtigerer Wissenschaftler sein, als jeder der keiner war.

Also würde Draco alle von Harrys Methoden des rationalen Denkens erlernen, und wenn die Zeit reif war \later

