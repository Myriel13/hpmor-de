

\hypertarget{verwirrung-bemerken}{% \section{5. Verwirrung bemerken}\label{verwirrung-bemerken}}

Kapitel 5

Verwirrung bemerken

Professor Quirrells Sprechstunden bestanden aus 11:40 bis 11:55 am Donnerstag. Das galt für alle seine Schüler in allen Schuljahren. Es kostete eine Quirrellpunkt auch nur an die Tür zu klopfen und wenn er glaubte der Grund wäre seine Zeit nicht wert gewesen, verlor man noch einmal fünfzig.

Harry klopfte an die Tür.

Es gab eine Pause. Dann sagte eine beißende Stimme, „Ich nehme an, Sie könnten auch gleich eintreten, Mr. Potter.“

Und bevor Harry den Türknauf berühren konnte, knallte die Tür auf, auf die Wand treffend mit einem so eindringlichen Knacken, das klang als ob vielleicht etwas im Holz, im Stein oder in beiden zerstört worden wäre.

Professor Quirrell saß in seinem Stuhl zurückgelehnt und ein verdächtig altes Buch lesend, gebunden in nachtblauem Leder mit silbernen Runen auf dem Einband. Seine Augen hatten sich nicht von den Seiten bewegt. „Ich bin nicht in guter Stimmung, Mr. Potter. Und wenn ich nicht guter Stimmung bin, bin ich kein angenehmer Zeitgenosse. Um Ihrer selbst Willen, führen Sie ihr Anliegen schnell durch und entfernen Sie sich.“

Ein kalter Schauer drang aus dem Raum, als ob er etwas beinhaltete, dass Dunkelheit in der Art ausstrahlte, wie Lampen Licht ausstrahlen und nicht vollständig verborgen war.

Harry war ein bisschen verblüfft. \emph{Nicht guter Stimmung} schien das nicht ganz abzudecken. Was könnte Professor Quirrell derart belasten…?

Also, man ließ nicht einfach seine Freunde im Stich, wenn sie sich niedergeschlagen fühlten. Harry drang vorsichtig in den Raum vor. „Gibt es etwas, dass ich tun kann \later“

„Nein,“ sagte Professor Quirrell, weiterhin nicht von seinem Buch aufblickend.

„Ich meine, wenn Sie mit Idioten zu tun hatten und jemand geistig Gesundes als Gesellschaft wünschen…“

Es entstand eine überraschend lange Pause.

Professor Quirrells schlug das Buch zu und es verschwand mit einem wispernden Geräusch. Er schaute auf und Harry zuckte zurück.

„Ich vermute eine intelligente Konversation würde zu diesem Zeitpunkt angenehm für \emph{mich} sein,“ sagte Professor Quirrell im gleichen beißenden Ton, der Harry dazu gebeten hatte, einzutreten. „Wahrscheinlich werden Sie es nicht angenehm finden, seien Sie gewarnt.“

Harry holte tief Luft. „Ich verspreche ich werde nicht dagegen haben, wenn Sie mich anblaffen.

Was ist passiert?“

Die Kälte im Raum schien sich zu verschlimmern. „Ein Sechstklässler aus Gryffindor hat einen Fluch auf einen meiner vielversprechenden Schüler eingesetzt, ein Sechstklässler aus Slytherin.“

Harry schluckte. „Welche… Art von Fluch?“

Und die Wut auf Professor Quirrells Gesicht war nicht länger kontrolliert. „Warum sich damit aufhalten so eine unwichtige Frage zu stellen, Mr. Potter? Unser Freund der Sechstklässler-Gryffindor meinte nicht es wäre wichtig!“

„Ist das Ihr \emph{Ernst?“} rief Harry bevor sich selbst daran hindern konnte.

„Nein, ich bin ohne bestimmten Grund heute schrecklicher Stimmung. \emph{Ja, es ist mein Ernst,} \emph{du Narr.} Er wusste es nicht. Er \emph{wusste es wirklich nicht}. Ich habe es nicht geglaubt, bis es die Auroren unter Veritaserum bestätigt hatten. Er ist in seinem \emph{sechsten Schuljahr in Hogwarts} und hat einen dunklen Fluch höheren Levels eingesetzt, \emph{ohne zu wissen was er bewirkte}.“

„Sie meinen doch nicht,“ sagte Harry, „dass er sich \emph{geirrt} hatte, über was er bewirkte, dass er die falsche Zauberbeschreibung gelesen hatte \later“

„Alles was er wusste war, dass er dafür gedacht war auf Feinde eingesetzt zu werden. Er \emph{wusste}, dass dies alles war, was er wusste.“

Und das hatte ausgereicht den Zauber einzusetzen. „Ich verstehe nicht, wie etwas mit einem so kleinen Gehirn aufrecht gehen kann.“

„Gewiss, Mr. Potter,“ sagte Professor Quirrell.

Es gab eine Pause. Professor Quirrell lehnte sich vor und nahm das silberne Tintenfass von seinem Schreibtisch, drehte es in seinen Händen herum, es anstarrend, als ob er darüber nachdachte, wie man es anstellte ein Tintenfass zu Tode zu foltern.

„Wurde der Slytherin-Sechstklässler schwer verletzt?“ fragte Harry.

„Ja.“

„Wurde der Gryffindor-Sechstklässler von Muggeln aufgezogen?“

„\emph{Ja}“

„Weigert Dumbledore sich den Jungen hinauszuwerfen, weil der arme Junge es nicht wusste?“

Professor Quirrells Hände wurden weiß, das Tintenfass festhaltend. „Haben ihre Fragen einen Zweck, oder sprechen Sie nur das Offensichtliche aus?“

„Professor Quirrell,“ sagte Harry ernst, „alle von Muggeln aufgezogenen Schüler in Hogwarts benötigen einen Sicherheits-Vortrag, indem sie die Sachen erklärt bekommen, die derart lächerlich offensichtlich sind, das kein Zaubererstämmiger jemals daran denken würde sie zu erwähnen. Setze keine Flüche ein, falls du nicht weißt was sie tun; Wenn du etwas Gefährliches entdeckst, erzähl es nicht der ganzen Welt; Braue keine fortgeschritten Zaubertränke in einem Badezimmer; Der Grund wieso es die Gesetze Zauberei Minderjähriger gibt, all das Grundlegende.“

„Warum?“ erwiderte Professor Quirrell. „Lass die Dummen sterben, vorher sie sich vermehren.“

„Wenn es Ihnen nichts ausmacht ein paar Slytherin-Sechstklässler mit ihnen zu verlieren.“

Das Tintenfass in Professor Quirrells Händen fing Feuer und brannte mit schrecklicher Langsamkeit, abscheuliche schwarz-orange Flammen, die am Metall zehrten und scheinbar kleine Bissen davon nahmen, das Silber verdrehte sich, während es schmolz, als ob es versuchte und scheiterte zu fliehen. Es entstand eine leises, kreischendes Geräusch, als ob das Metall schreien würde.

„Ich vermute, Sie haben Recht,“ sagte Professor Quirrell mit einem resigniertem Lächeln. „Ich werde wohl einen Vortrag entwerfen, um sicher zu stellen, dass Muggelstämmige, die zu blöd zum Leben sind, nicht jemand Wertvollen mitnehmen wenn sie von uns gehen.“

Das Tintenfass in Professor Quirrells Händen schrie und brannte weiter, sodass kleine Tropfen Metall, noch immer brennend, auf den Tisch tropften, wie wenn das Tintenfass weinen würde.

„Sie rennen nicht weg,“ bemerkte Professor Quirrell.

Harry öffnete seinen Mund \later

„Wenn Sie kurz davor waren zu sagen, Sie hätten keine Angst vor mir,“ sagte Professor Quirrell, \emph{„lassen Sie es.“}

„Sie sind die beängstigendste Person, die ich kenne,“ sagte Harry, „und eine der besten Gründe dafür ist ihre Kontrolle. Ich kann mir schlicht nicht vorstellen zu hören, sie hätten jemanden verletzt, bei dem sie nicht die vorsätzliche Entscheidung zum Verletzten getroffen hätten.“

Das Feuer in Professor Quirrells Händen erlöscht und er platzierte das zerstörte Tintenfass vorsichtig auf seinen Tisch. „Sie sagen die nettesten Dinge, Mr. Potter. Haben Sie Unterricht in Schmeichelei genommen? Von, zum Beispiel, Mr. Malfoy.“

Harry hielt sein Gesicht ungerührt, und bemerkte eine Sekunde zu spät, dass es genauso gut auch ein unterschriebenes Geständnis gewesen sein könnte. Professor Quirrell kümmerte sich nicht darum, welchen Gesichtsausdruck man hatte, ihn kümmerte welche Gemütszustände diesen wahrscheinlich machten.

„Ich verstehe,“ sagte Professor Quirrell. „Mr. Malfoy ist ein nützlicher Freund, Mr. Potter, und es gibt viel, was er ihnen beibringen könnte, aber ich hoffe sie haben nicht den Fehler begangen ihm zu viele Geheimnisse anzuvertrauen.“

„Er weiß nichts, bei dem ich fürchte es würde bekannt werden,“ sagte Harry.

„Sehr gut,“ sagte Professor Quirrell, leicht lächelnd. „Also, was war ihr eigentliches Anliegen hier?“

„Ich glaube, ich bin mit den vorausgehenden Aufgaben in Okklumentik fertig und bereit für den Tutor.“

Professor Quirrell nickte. „Ich werde sie diesen Sonntag zu Gringotts geleiten.“ Er machte eine Pause, sah Harry an, und lächelte. „Und wir können daraus sogar einen kleinen Ausflug machen, wenn Sie wollen. Ich hatte gerade einen angenehmen Gedanken.“

Harry nickte, zurücklächelnd.

Als Harry das Büro verließ, hörte er, wie Professor Quirrell eine kleine Melodie pfiff.

Harry war froh es geschafft zu haben, ihn aufzumuntern.

An diesem Sonntag schien es eine ziemlich große Anzahl an Leuten zu geben, die in den Korridoren miteinander flüsterten, jedenfalls wenn Harry Potter an ihnen vorbei ging.

Und viele auf ihn deutende Finger.

Und eine große Menge Mädchengekicher.

Es begann beim Frühstück, als jemand Harry gefragt hatte, ob die Neuigkeiten gehört hatte, und Harry ihn schnell unterbrochen hatte und gesagt hatte, dass falls die Neuigkeiten von Rita Kimmkorn geschrieben worden wären, er dann nichts davon \emph{hören} wolle, da er es selbst in der Zeitung lesen wolle.

Es hatte sich dann gezeigt, dass nicht viele Schüler in Hogwarts Exemplare des \emph{Tagespropheten} bekamen, und dass die Exemplare, die nicht bereits von ihrem Besitzer weiterverkauft worden waren, in irgendeiner Art komplizierter Reihenfolge herumgegeben wurden und niemand wirklich wusste, wer zurzeit eine hatte…

Also hatte Harry den Quietus-Zauber eingesetzt und war fortgefahren sein Frühstück zu essen, hatte es seinen Sitznachbarn überlassen die vielen, vielen Fragestellenden abzuweisen, und sein Bestes gegeben die Ungläubigkeit, das Gelächter, das beglückwünschende Lächeln, die mitleidigen oder angsterfüllten Blicke, und die fallengelassenen Teller wenn neue Schüler zum Frühstück herunterkamen und davon hörten, zu ignorieren.

Harry fühlte sich \emph{ziemlich} neugierig, aber es wäre \emph{wirklich} unpassend gewesen, sich das Kunstwerk verderben lassen, indem er es aus zweiter Hand hörte.

Er hatte in der Sicherheit seines Koffers die nächsten Stunden damit verbracht, seine Hausaufgaben zu machen, nachdem er seinen Schlafsaalsgenossen aufgetragen hatte, ihn zu holen, sollte jemand ein Original der Zeitung finden.

Harry war noch immer unwissend, als er um 10 Uhr in einer Kutsche Hogwarts verließ, mit Professor Quirrell vorne rechts im Wagen und zurzeit im Zombie-Modus. Harry saß im diagonal gegenüber, so weit weg wie es die Kutsche erlaubte, hinten links. Dennoch hatte Harry ein konstantes Gefühl des Verderbens während die Kutsche über einen kleinen Pfad durch einen Teil des nicht-Verbotenen Waldes holperte. Es machte es etwas schwer zu lesen, besonders da das Material schwierig war, und Harry sich plötzlich wünschte, stattdessen eines der Science-Fiction-Bücher seiner Kindheit zu lesen \later

„Wir sind außerhalb der Schutzzauber, Mr. Potter,“ sagte Professor Quirrells Stimme von vorne. „Zeit zu gehen.“

Professor Quirrell stieg vorsichtig aus der Kutsche, sich wappnend bevor hinaustrat. Harry, auf seiner Seite, sprang herunter.

Harry fragte sich, wie genau sie dorthin gelangen würden, als Professor Quirrell „Fang“ rief und ihm einen bronzenen Knut zu warf, und Harry fing ihn ohne nachzudenken.

Ein gigantischer, immaterieller Haken verfing sich an Harrys Bauchnabel und zerrte ihn zurück, heftig, nur ohne jegliches Gefühl der Beschleunigung, und einen Moment später stand Harry in der Mitte der Winkelgasse.

(\emph{Entschuldige mal, was? fragte} sein Gehirn.)

(\emph{Wir sind gerade teleportiert,} erklärte Harry.)

(\emph{Das ist in der guten, alten Zeit nicht vorgekommen,} beschwerte sich Harrys Gehirn, und desorientierte ihn.)

Harry stolperte als seine Füße sich an das Gestein der Straße anstelle der Erde des Waldlandes die sie zuvor überquert hatten, gewöhnten. Er richtete sich auf, noch immer benommen, sodass die umherwuselden Hexen und Zauberer etwas zu schwanken schienen, und das Geschrei der Ladenbesitzer sich scheinbar in seinem Gehör herumbewegte, während sein Gehirn versuchte eine Welt einzuordnen in der es sich befinden konnte.

Einige Zeit später, gab es eine Art schlürfendes Geräusch ein paar Schritte hinter Harry, und als er sich danach umdrehte, war Professor Quirrell da.

„Würde es Ihnen etwas ausmachen \later“ sagte Harry, im gleichen Moment als Professor Quirrell sagte, „Ich fürchte ich \later“

Harry stoppte, Professor Quirrell nicht.

„ muss sie kurz verlassen und etwas in Bewegung setzen, Mr. Potter. Da es mir umfassend erläutert wurde, dass Ich verantwortlich bin für was auch immer Ihnen zustößt, werde ich Sie bei \later“

„Zeitungsladen,“ sagte Harry.

„Wie bitte?“

„Oder irgendwo anders, wo Ich ein Exemplar des \emph{Tagespropheten} kaufen kann. Bringen Sie mich dorthin und ich bin zufrieden.“

Kurz darauf, wurde Harry in einem Bücherladen abgesetzt, begleitet von einigen leise gesprochenen, mehrdeutigen Drohungen. Und der Ladenbesitzer hatte \emph{weniger} mehrdeutige Drohungen erhalten, danach zu urteilen, wie er zusammengezuckt war und wie sein Blick nun stets zwischen Harry und dem Eingang hin und her huschten.

Falls der Bücherladen niederbrannte, würde Harry in der Mitte des Feuers verharren, bis Professor Quirrell zurückkehrte.

Unterdessen \later

Harry schaute sich rasch um.

Der Bücherladen schien eher klein und schäbig, mit gerade mal vier sichtbaren Bücherregalen, und die nächste Ablage, auf die Harrys Blick gefallen war, schien sich mit schmalen, billig gebundenen Büchern mit grimmigen Titeln wie \emph{Das Massaker von Albanien im Fünfzehnten Jahrhundert} zu beschäftigen.

Das Wichtigste zuerst. Harry trat zum Verkaufstresen.

„Entschuldigen Sie,“ sagte Harry, „Eine Kopie des \emph{Tagespropheten}, bitte.“

„Fünf Sickel,“ sagte der Ladenbesitzer. „Tut mir Leid, Junge, ich habe nur noch drei übrig.“

Fünf Sickel fielen auf den Tresen. Harry hatte das Gefühl, er hätte ihn ein wenig herunterhandeln können, aber in diesem Moment kümmerte ihn das nicht mehr so richtig.

Die Augen des Ladenbesitzers weiteten sich und schienen zum ersten Mal Harry wirklich zu bemerken. „\emph{Du!}“

„\emph{Ich!}“

„\emph{Ist es wahr? Bist du wirklich \later“}

\emph{„Seien Sie still!} Entschuldigung, ich habe den \emph{ganzen Tag} darauf gewartet, dies im Original zu lesen, statt es aus zweiter Hand zu hören, also bitte \emph{geben Sie es einfach herüber,} in Ordnung?“

Der Ladenbesitzer starrte Harry einen Moment an, griff dann wortlos unter den Tresen und reichte ihm ein gefaltetes Exemplar des \emph{Tagespropheten.}

Die Überschrift lautete:

HARRY POTTER

HEIMLICH VERLOBT

MIT GINERVRA WEASLEY

Harry starrte sie an.

Er hob die Zeitung vom Tresen, sanft, ehrfürchtig, als ob er ein Original Escher Kunstwerk in Händen halten würde, und entfaltete sie um zu lesen…

… über die Indizien, die Rita Kimmkorn überzeugt hatten.

… und einige weitere interessante Details.

… und noch mehr Indizien.

Fred und George hatten das doch sicherlich vorher mit ihrer Schwester abgesprochen, oder? Ja, natürlich, hatten sie das. Es gab da ein Bild von Ginevra Weasley, wie sie sich sehnend seufzend über, etwas beugte, das soweit Harry es erkennen konnte, ein Foto von ihm selbst war. Das musste

gestellt gewesen sein.

Aber \emph{wie um Himmels Willen…?}

Harry saß in einem billigen Klappstuhl, die Zeitung zum vierten Male lesend, als die Tür sich leise flüsternd öffnete und Professor Quirrell in den Laden zurückkehrte.

„Ich bitte um Entschuldigung für \later was in Merlins Namen lesen Sie da?“

„Wie es scheint,“ antwortete Harry, Ehrfurcht in seiner Stimme, „wurde jemand namens Mr. Arthur Weasley durch einen Todesser unter den Einfluss des Imperius-Fluch gesetzt, denn mein Vater getötet hat, sodass eine Schuld zum Hause Potter entstanden ist, für die mein Vater meine Verlobung mit der, vor Kurzem geborenen, Ginevra Weasley verlangte. Machen die Leute hier tatsächlich so was in dieser Art?“

„Wie könnte Miss Kimmkorn \emph{jemals} ein derartiger Narr sein, dies zu glauben \later“

Und Professor Quirrell Stimme brach ab.

Harry hatte die Zeitung vertikal und aufgeschlagen gehalten, sodass Professor Quirrell, von wo er stand, den Text unter der Überschrift lesen konnte.

Der Ausdruck von Schock auf Professor Quirrells Gesicht war ein Kunstwerk fast ebenbürtig mit der Zeitung selbst.

„Keine Sorge,“ sagte Harry fröhlich, „es ist alles gefälscht.“

Von irgendwo anders im Laden, hörte er den Ladenbesitzer keuchen. Es entstand das Geräusch von einem umgefallenen Bücherstapel.

„Mr. Potter…“ sagte Professor Quirrell langsam. „Sind Sie sich dabei \emph{sicher}?“

„Ziemlich sicher. Sollen wir gehen?“

Professor Quirrell nickte, ziemlich entrückt wirkend, und Harry faltete die Zeitung zusammen, und folgte ihm aus der Tür.

Aus irgendeinem Grund schien Harry jetzt nichts vom Lärm der Straße zu hören.

Sie gingen dreißig Sekunden still weiter, bevor Professor Quirrell zu sprechen anfing. „Miss Kimmkorn hat die originalen Berichte der limitieren Zauberergamot-Sitzung eingesehen.“

„Ja.“

„Die \emph{originalen Berichte des} \emph{Zauberergamots.“}

\emph{„}Ja.“

„\emph{Ich} hätte Schwierigkeiten das umzusetzen.“

„Wirklich?“ sagte Harry. „Denn wenn mein Verdacht richtig ist, wurde dies von einem Hogwarts Schüler gemacht.“

„Das ist mehr als unmöglich,“ sagte Professor Quirrell kategorisch. „Mr. Potter… es missfällt mir Ihnen das zu sagen, aber diese junge Dame erwartet sie zu heiraten.“

„Aber \emph{das} ist unwahrscheinlich,“ sagte Harry. „Um Douglas Adams zu zitieren, hat das Unmögliche häufig eine Art Integrität, die dem lediglich Unwahrscheinlichen abgeht.“

„Ich verstehe Ihren Punkt,“ sagte Professor Quirrell langsam. „Aber… nein, Mr.Potter. Es könnte unmöglich sein, aber dennoch kann ich mir \emph{vorstellen} die Berichte des Zauberergamots zu manipulieren. Es ist \emph{unvorstellbar,} dass der Grand Manager von Gringotts das Siegel seines Büros als Zeuge eines falschen Verlobungsvertrags beifügen würde und Miss Kimmkorn hat das Siegel persönlich verifiziert.“

„Gewiss,“ sagte Harry, „man würde erwarten, dass der Grand Manager von Gringotts beteiligt war, wenn so viel Geld den Besitzer wechselt. Wie es scheint war Mr. Weasley hoch verschuldet, und forderte eine zusätzliche Bezahlung von zehntausend Galleonen \later“

„\emph{Zehntausend Galleonen} für eine \emph{Weasley?} Man könnte sich die Tochter eines Noblen Hauses dafür kaufen!“

„Entschuldigen Sie,“ sagte Harry. „Ich muss in diesem Moment wirklich noch einmal nachfragen, machen die Leute so etwas wirklich hier \later“

„Selten,“ sagte Professor Quirrell, das Gesicht verziehend. „Und gar nicht mehr, vermute ich, seitdem der Dunkle Lord verschwand. Ich nehme an, dass gemäß der Zeitung Ihr Vater es einfach bezahlt hat?“

„Er hatte keine Wahl,“ sagte Harry. „Nicht, wenn er die Bedingungen der Prophezeiung erfüllen wollte.“

„\emph{Geben Sie mir das,“} sagte Professor Quirrell, und die Zeitung sprang aus Harrys Händen so schnell, dass er sich einen Papierschnitt zuzog.

Harry steckte sich automatisch den Finger in den Mund um daran zu lutschen, ziemlich schockiert und wandte sich wieder Professor Quirrell zu, um sich bei ihm zu beschweren \later

Professor Quirrell hatte plötzlich mitten auf der Straße angehalten und seine Augen flackerten schnell vor und zurück, während eine unsichtbare Kraft die Zeitung vor ihm herabhängend hielt.

Harry sah zu, in aufrichtiger Ehrfurcht starrend, wie sich die Zeitung öffnete, um Seite zwei und drei zu offenbaren. Und nicht viel später, vier und fünf. Es war als ob der Mann seine vorgetäuschte Sterblichkeit abgeschüttelt hatte.

Und nach einer beunruhigend kurzen Zeit, faltete sich die Zeitung wieder hübsch zusammen. Professor Quirrell pflückte sie aus der Luft und warf sie Harry zu, der sie völlig im Reflex auffing; und dann begann Professor Quirrell wieder an zu gehen, und Harry stolperte ihm automatisch nach.

„Nein,“ sagte Professor Quirrell, „diese Prophezeiung klang für mich auch nicht ganz richtig.“

Harry nickte, noch immer betäubt.

„Die Zentauren könnten unter dem \emph{Imperius} stehen,“ sagte Professor Quirrell, mit finsterem Blick, „\emph{das} scheint verständlich zu sein. Was Magie erschaffen kann, kann Magie auch verderben, und es nicht unvorstellbar, dass das Große Siegel von Gringotts derart verdorben werden könnte, sodass sie einem anderen folgt. Der Unsägliche könnte durch Vielsafttrank imitiert werden, genauso der bayerische Seher. Und mit \emph{genügend} Einsatz könnte es möglich sein die Berichte des Zauberergamots zu verfälschen. Haben Sie irgendeine Idee, wie dies umgesetzt wurde?“

„Ich habe nicht eine einzige plausible Hypothese,“ sagte Harry. „Ich weiß nur, dass es vollständig mit einem Budget von vierzig Galleonen durchgeführt wurde.“

Professor Quirrell hielt plötzlich an und fuhr zu Harry herum. Sein Gesichtsausdruck war völlig ungläubig. „Vierzig Galleonen würden einen kompetenten Fluchbrecher bezahlen, um einen Weg in das Zuhause zu ermöglichen, von dem sie etwas stehlen möchten. \emph{Vierzigtausend} Galleonen \emph{könnten} ausreichen ein Team der größten professionellen Kriminellen der Welt zu bezahlen, die Berichte des Zauberergamots zu manipulieren!“

Harry zuckte hilflos mit den Schultern. „Ich werde daran denken, wenn ich das nächste Mal neununddreißig-tausend-neun-hundert-und-sechzig Galleonen sparen möchte, indem die den richtigen Vertragspartner finde.“

„Ich sage dies nicht häufig,“ erwiderte Professor Quirrell. „Ich bin beeindruckt.“

„Ebenso,“ sagte Harry.

„Und wer ist dieser unglaubliche Hogwartsschüler?“

„Ich fürchte, ich kann es nicht sagen.“

Zu Harrys großer Überraschung, erhob Professor Quirrell keine Einwände hierzu.

Sie gingen in Richtung des Gringottsgebäude, nachdenkend, da keiner von ihnen die Sorte von Person war, die bei einem Problem aufgeben würden, ohne es für mindestens fünf Minuten zu erwägen.

„Ich habe das Gefühl,“ sagte Harry schließlich, „dass wir das Problem aus der falschen Richtung angehen. Es gibt da eine Geschichte, die ich einst gehört habe, über einige Schüler in einem Physik-Kurs. Der Lehrer zeigte ihnen eine große Metallplatte nahe dem Feuer. Sie trug ihnen auf, das Metall anzufassen, und sie bemerkten, dass das Metall nahe dem Feuer kühler und das Metall weiter weg wärmer war. Und sie trug ihnen auf, ihre Vermutung aufzuschreiben warum das passiert. Also schrieben manche Schüler 'wegen der Art, wie das Metall Wärme leitet', und manche schrieben 'wegen der Art, wie die Luft sich bewegt', und niemand sagte 'dies scheint schlicht unmöglich', und die richtige Antwort war, dass der Lehrer, bevor die Schüler in den Raum kamen, die Platte gedreht hatte.“

„Interessant,“ sagte Professor Quirrell. „Dies klingt durchaus ähnlich. Gibt es eine Moral?“

„Das deine Stärke als Rationalist deine Fähigkeit ist, verwirrter von Fiktion zu sein, als von der Wirklichkeit,“ sagte Harry. „Wenn man genauso gut darin ist jede Art von Resultat zu erklären, hast man keinerlei Wissen. Die Schüler dachten, sie könnten Worte wie 'wegen Wärmeleitung' nutzen um alles zu erklären, sogar warum eine Platte auf der Seite des Feuers kühler war. Also bemerkten sie nicht, wie verwirrt sie waren, und konnten damit nicht verwirrter von der Unwahrheit als von der Wahrheit sein. Wenn Sie mir sagen, die Zentauren wären unter dem Imperius-Fluch gewesen, habe ich noch immer das Gefühl, etwas wäre nicht ganz richtig ist. Ich bemerke, dass ich weiterhin verwirrt bin, sogar nachdem ich Ihre Erklärung gehört habe.“

„Hm,“ sagte Professor Quirrell.

Sie gingen weiter.

„Ich nehme nicht an,“ sagte Harry, „dass es möglich ist tatsächlich Leute in ein Paralleluniversium zu bringen? So was wie, dies ist nicht unsere eigene Rita Kimmkorn, oder sie haben sie temporär irgendwo anders hingeschickt?“

„Wenn \emph{das} möglich wäre,“ sagte Professor Quirrell, mit eher trockener Stimme, „wäre ich dann noch \emph{hier?“}

Und gerade als sie die riesige weiße Fassade von Gringotts erreicht hatten, sagte Professor Quirrell:

„Ah, \emph{natürlich.} Jetzt verstehe ich. Lassen Sie mich raten, die Weasley-Zwillinge?“

„\emph{Was?“} rief Harry, mit einer um eine Oktave höheren Stimme. „\emph{Wie?“}

„Ich fürchte, ich kann es nicht sagen.“

„… Das ist \emph{nicht} fair.“

„Ich denke, es ist extrem fair,“ sagte Professor Quirrell, und sie traten durch die bronzenen Türen ein.

Es war kurz vor 12 Uhr und Harry und Professor Quirrell saßen am Kopf- und Fußende eine breiten, langen, glatten Tisches in einem luxuriöses ausgestattetem privaten Raum mit sorgfältig gepolsterten Sofas und Stühlen an den Wänden und zarten Umhängen überall.

Sie waren kurz davor ihr Mittagessen in Mary's Place zu sich zunehmen, von dem Professor Quirrell gesagt hatte, es sei ihm als eines der besten Restaurants in der Winkelgasse bekannt, insbesondere für \later und seine Stimme hatte sich bedeutungsvoll gesenkt \later \emph{spezielle Zwecke.}

Es war das edelste Restaurant, indem Harry jemals gewesen war und es zerrte an ihm, dass Professor Quirrell \emph{ihn} mit der Mahlzeit verwöhnte.

Der erste Teil ihrer Mission, einen Okklumentik Lehrer zu finden, war erfolgreich gewesen. Professor Quirrell, böse lächelnd, hatte Griphook aufgefordert, den Besten zu empfehlen den er kannte, und sich nicht um die Kosten zu kümmern, weil Dumbledore dafür zahlte; und der Kobold hatte zurückgelächelt. Es gab vielleicht auch auf Seiten von Harry das eine oder andere Lächeln.

Der zweite Teil des Plan war eine kompletter Fehlschlag gewesen.

Harry war es nicht gestattet, Geld aus seinem Verlies zu nehmen, wenn nicht Schulleiter Dumbledore oder ein anderer Amtsträger der Schule anwesend war, und Professor Quirrell war der Verließschlüssel nicht gegeben worden. Harrys Muggeleltern konnten es auch nicht autorisieren, da sie Muggel waren, und Muggel ungefähr den gleichen Rechtsstand hatten wie Kinder oder Kätzchen: sie waren süß, sodass man jemanden einsperren konnte wenn er sie in der Öffentlichkeit folterte, aber sie waren keine \emph{Menschen}. Einige widerstrebende Zugeständnisse wurden umgesetzt, um die Eltern von Muggelstämmigen als Menschen in einem begrenzten Sinne anzuerkennen, aber Harrys Adoptiveltern fielen nicht in diese Rechtsklasse.

Es schien, als ob Harry im Prinzip eine Waise in den Augen der Zaubererwelt war. Daher waren der Schulleiter von Hogwarts, oder einer seiner Beauftragten \emph{innerhalb} des Schulsystems, Harrys Vormund bis er die Schule abschloss. Harry \emph{durfte} ohne Dumbledores Erlaubnis atmen, aber nur solange der Schulleiter es nicht direkt verboten hatte.

Harry hatte dann gefragt, ob er Griphook \emph{einfach} sagen konnte, wie er seine Investitionen, über Stapel von Goldmünzen in seinem Verlies hinaus, streuen sollte.

Griphook hatte in mit leeren Blick angestarrt und gefragt was 'Investitionen streuen' bedeutete.

Banken, so schien es, legten kein Geld an. Banken bewahrten deine Goldmünzen in sicheren Verliesen für eine jährliche Gebühr.

Die Zaubererwelt kannte nicht das Konzept von Aktien. Oder Marktwert. Oder Konzernen. Unternehmen würden von Familien aus ihrem eigenen Verlies heraus geführt.

Kredite wurden von reichen Leuten ausgegeben, nicht von Banken. Obwohl Gringotts den Vertrag, gegen eine Gebühr, bezeugen würde und, gegen eine sehr viel höhere Gebühr, das Eintreiben des Geldes durchführen würde.

Gute reiche Leute liehen ihren Freunden Geld und lassen Sie es irgendwann zurückbezahlen. \emph{Böse} reiche Leute verlangten \emph{Zinsen.}

Es gab keine Sekundärmarkt für Kredite.

Böse reiche Leute verlangten jährliche Zinssätze von mindestens 20\%.

Harry war aufgestanden, hatte sich abgewandt und seinen Kopf an die Wand gelehnt.

Harry hatte gefragt, ob er die Erlaubnis des Schulleiters benötigte, um eine Bank zu eröffnen.

Professor Quirrell hatte ihn in diesem Moment unterbrochen und gesagt, es wäre Zeit fürs Mittagessen, und den rauchenden Harry rasch aus den bronzenen Türen von Gringotts, durch die Winkelgasse, zu einem feinen Restaurant namens Mary's Place verfrachtet, in dem ein Raum für sie reserviert war. Der Besitzer hatte schockiert ausgesehen, als er Professor Quirrell begleitet von Harry erblickte, sie jedoch ohne Beanstandung zu dem Raum gebracht.

Und Professor Quirrell hatte sehr bewusst verkündet, er würde die Rechnung bezahlen, scheinbar ziemlich Harrys Gesichtsausdruck genießend.

„Nein,“ sagte Professor Quirrell zu der Kellnerin, „wir werden keine Speisekarten benötigen. Ich hätte gerne das Tagesmenü, und dazu eine Flasche Chianti und Mr. Potter nimmt die Diricawl Suppe als Vorspeise, gefolgt von einem Teller Roopo Kugeln, und Sirup-Pudding als Nachspeise.“

Die Kellnerin, bekleidet mit einem unüblich kurzem, aber dennoch ernsten und formellen Umhang, verbeugte sich respektvoll, verließ den Raum und schloss die Tür hinter sich.

Professor Quirrell wedelte eine Hand in Richtung der Tür, und ein Riegel glitt zu. „Bemerken Sie den Riegel auf der Innenseite. Dieser Raum, Mr. Potter, ist bekannt als Mary's Zimmer. \emph{Zufällig} ist er gegen sämtliche Formen des Abhörens gesichert, und ich meine \emph{sämtliche;} Dumbledore selbst könnte nichts von dem wahrnehmen, was hier passiert. Mary's Zimmer wird von zwei Arten von Personen genutzt. Die erste Sorte ist in gesetzwidrigen Spielereien verwickelt. Und die zweite Sorte führt interessante Leben.“

„\emph{Wirklich}“, sagte Harry.

Professor Quirrell nickte.

Harrys Lippen waren vor Erwartung geöffnet. „Es wäre dann eine Verschwendung nur hier zu sitzen und Mittagessen zu essen, ohne etwas Besonderes zu machen.“

Professor Quirrell lächelte, nahm dann seinen Zauberstab heraus und schnickte ihn in Richtung der Tür. „Natürlich,“ sagte er, „vollführen solche, die interessante Leben führen, gründlichere Sicherheitsvorkehrungen als die Dilettanten. Ich habe uns gerade hier eingeschlossen. Nichts wird mehr in oder aus diesem Raum gelangen \later durch den Spalt unter der Tür zum Beispiel. Und…“

Professor Quirrell sprach dann nicht weniger als vier verschiedene Zaubersprüche, von denen Harry keinen erkannte.

„Sogar das reicht nicht \emph{wirklich} aus,“ sagte Professor Quirrell. „Falls wir etwas von wahrhaftig großer Wichtigkeit machen würden, wäre es nötig weitere dreiundzwanzig Zauber auszuführen. Wenn, zum Beispiel, Rita Kimmkorn wusste oder erraten hätte, dass wir hierher kommen würden, wäre es möglich, dass sie in diesem Raum sein könnte, den wahren Tarnumhang tragend. Oder sie könnte vielleicht ein Animagus mit einer kleinen Form sein. Es gibt Tests um solch seltene Vorkommnisse auszuschließen, aber alle davon durchzuführen wäre mühselig. Dennoch frage ich mich, ob ich sie nicht trotzdem alle ausführen sollte, nur um ihnen keine schlechten Angewohnheiten beizubringen?“ Und Professor Quirrell tippte sich mit einem Finger an die Wange, ins Leere blickend.

„Es ist schon in Ordnung,“ sagte Harry, „ich verstehe es, und werde mich daran erinnern.“ Obwohl er durchaus ein bisschen enttäuscht war, dass nicht von wahrhaftig großer Wichtigkeit machen würden.

„Also gut,“ sagte Professor Quirrell. Er lehnte sich in seinem Stuhl zurück, breit grinsend. „Sie haben recht gute Arbeit geleistet, Mr. Potter. Die grundsätzliche Idee war sicherlich Ihre, selbst wenn sie die Ausführung delegiert haben. Ich glaube nicht, dass wir noch viel von Rita Kimmkorn hören werden. Lucius Malfoy wird von ihrem Versagen nicht angetan sein. Falls sie clever ist, wird sie das Land verlassen, sobald sie bemerkt, dass sie getäuscht worden ist.“

In Harrys Magen erwachte ein ungutes Gefühl. „Lucius Malfoy steckte hinter Rita Kimmkorn…?

„Oh, das wussten Sie nicht?“ sagte Professor Quirrell.

Harry hatte nicht darüber nachgedacht, was mit Rita Kimmkorn danach passieren würde.

Überhaupt nicht.

Nicht im Mindesten.

Aber sie würde wohl gefeuert werden, \emph{natürlich} würde sie gefeuert werden, soweit Harry es wusste, könnte sie Kinder in Hogwarts haben, und jetzt war es schlimmer, viel schlimmer…

„Wird Lucius sie töten lassen?“ sagte Harry in einer kaum hörbaren Stimme. Irgendwo im Hinterkopf, schrie in der Sprechende Hut an.

Professor Quirrell lächelte trocken. „Falls Sie noch nicht mit Journalisten zu tun hatten, glauben Sie mir, die Welt wird ein bisschen heller, jedes Mal wenn einer von ihnen stirbt.“

Harry sprang mit einer verkrampften Bewegung aus seinem Stuhl auf, er musste Rita Kimmkorn finden und sie warnen bevor es zu spät war \later

„\emph{Setzen Sie sich,“} sagte Professor Quirrell scharf. „Nein, Lucius wird sie nicht ermorden. Aber Lucius macht das Leben \emph{extrem} unangenehm, für jene, die ihm einen schlechten Dienst erweisen.

Miss Kimmkorn wird fliehen und ihr ein neues Leben mit einem anderen Namen beginnen. \emph{Setzen Sie sich}, Mr. Potter; es gibt nichts was Sie tun können, und Sie haben eine Lektion zu lernen.“

Harry setzte sich, zögerlich. Professor Quirrells Gesicht zeigte einen enttäuschten, entnervten Ausdruck, der mehr Erfolg hatte ihn aufzuhalten als die Worte.

„Es gibt Zeiten,“ sagte Professor Quirrell mit schneidender Stimme, „wenn ich befürchte ihr brillanter Slytherinverstand ist bei Ihnen schlicht verschwendet. Sprechen Sie mir nach. Rita Kimmkorn war eine bösartige, widerwärtige Frau.“

„Rita Kimmkorn war eine bösartige, widerwärtige Frau,“ sagte Harry. Er fühlte sich unbehaglich es auszusprechen, aber es schien keine anderen möglichen Aktionen zu geben.

„Rita Kimmkorn hat versucht meinen Ruf zu zerstören, aber ich habe einen ausgeklügelten Plan ausgeführt und \emph{ihren} Ruf zuerst zerstört.“

„Rita Kimmkorn hat mich herausgefordert. Sie hat das Spiel verloren, ich habe gewonnen.“

„Rita Kimmkorn war ein Hindernis für meine zukünftigen Pläne. Ich hatte keine Option, außer mich um sie zu kümmern, wenn ich meine Pläne umsetzen wollte.“

„Rita Kimmkorn war mein Gegner.“

„Ich kann unmöglich im Leben etwas erreichen, falls ich nicht gewillt bin meine Gegner zu besiegen.“

„Ich habe heute einen meiner Gegner besiegt.“

„Ich bin ein guter Junge.“

„Ich verdiene eine außergewöhnliche Belohnung.“

„Ah,“ sagte Professor Quirrell, der während der letzten Zeilen wohlwollend gelächelt hatte, „Ich merke, ich hatte Erfolg Ihre Aufmerksamkeit zu gewinnen.“

Das war richtig. Und obwohl Harry das Gefühl hatte zu etwas gedrängt worden zu sein \later nein, es war nicht nur ein Gefühl, es \emph{wurde} ihm aufgedrängt \later er konnte nicht leugnen, dass er sich besser fühlte, als er diese Sachen sagte und Professor Quirrell lächeln sah.

Professor Quirrell griff in seinen Umhang, die Geste langsam und bewusst bedeutungsvoll, und zog…

\emph{ein Buch hervor.}

\emph{Es war völlig anders als alle Bücher, die Harry jemals gesehen hatte, die Kanten und Ecken sichtbar unförmig;} ungehobelt war der naheliegende Begriff, als ob es aus einer Büchermine gehauen wurde.

„Was ist es?“ keuchte Harry.

„Ein Tagebuch,“ sagte Professor Quirrell.

„Wessen?“

„Von jemandem Bekannten.“ Professor Quirrell lächelte breit.

„Okay…“

\emph{Professor Quirrells Gesichtsausdruck wurde ernster. „Mr. Potter, eine der Voraussetzungen, um ein mächtiger Zauberer zu werden, ist ein} exzellentes Gedächtnis. Der Schlüssel zu einem Rätsel ist häufig etwas, dass man zwanzig Jahre zuvor in einer alten Schriftrolle gelesen hat, oder ein merkwürdiger Ring, den man am Finger eines Mannes gesehen hatte, den man nur einmal getroffen hat. Ich erwähne dies, um zu erklären, wie ich es schaffte mich an dieses Objekts und die daran befestigte Beschreibung zu erinnern, nachdem ich Sie einige Zeit später traf. Verstehen Sie, Mr. Potter, im Laufe meines Lebens, habe ich viele Privatkollektionen im Besitz von Einzelnen gesehen, die, vielleicht, nicht ganz das verdienten was sie besaßen \later“

\emph{„Sie haben es} gestohlen“, rief Harry ungläubig.

„Das ist richtig,“ sagte Professor Quirrell. „Erst kürzlich, um genauer zu sein. Ich glaube, Sie werden diesen bestimmten Gegenstand sehr viel mehr schätzen, als der niederträchtige, kleine Mann, der es allein zu dem Zweck besaß, seine ebenso niederträchtigen Freunde mit seiner Seltenheit zu beeindrucken.“

Harry glotzte nun nur noch.

\emph{„Aber wenn Sie der Meinung sind, meine Handlungen wäre}n falsch, Mr. Potter, müssen Sie, so vermute ich mal, Ihr besonderes Geschenk nicht annehmen. Obwohl ich natürlich nicht den Aufwand auf mich nehmen werde, es zurück zu stehlen. Also, was soll es sein?“

Professor Quirrell warf das Buch von einer Hand zu anderen, was Harry dazu brachte, unwillkürlich mit einem Blick der Bestürzung danach zu greifen.

„Oh,“ sagte Professor Quirrell, „sorgen Sie sich nicht um die unsanfte Behandlung. Man könnte dieses Tagebuch in den Kamin werfen und es würde unversehrt wieder zum Vorschein kommen. Dennoch, ich erwarte Ihre Entscheidung.“

Professor Quirrell warf das Buch lässig in die Luft und fing es lächelnd wieder auf.

\emph{Nein, sagten Gryffindor und Hufflepuff.}

\emph{Ja}, sagte Ravenclaw. Welchen Teil des Wortes 'Buch' habt ihr beide nicht verstanden?

\emph{Den Teil über Diebstahl, sagte Hufflepuff.}

\emph{Ach, komm schon, sagte Ravenclaw,} du kannst uns nicht ernsthaft darum bitten nein zu sagen und den Rest unseres Lebens damit zu verbringen, sich zu fragen was es gewesen war.

\emph{Es klingt nach einem positiven Netto-Effekt aus utilitaristischer Sicht, sagte Slytherin.} Stell es dir als eine ökonomische Transaktion vor, die Gewinn durch Handel erzielt, nur ohne den Teil mit dem Handel. Außerdem, haben wir es nicht gestohlen und es würde niemandem helfen, wenn Professor Quirrell es behalten würde.

\emph{Er versucht dich auf die Dunkle Seite zu ziehen, kreischte Gryffindor, und Hufflepuff nickte bestimmt.}

\emph{Sei kein naiver kleiner Junge, sagte Slytherin,} er versucht dir Slytherin beizubringen.

\emph{Genau, sagte Ravenclaw.} Wer auch immer das Buch vorher besessen hatte, war wahrscheinlich ein Todesser oder so. Es gehört zu uns.

Harry öffnete seinen Mund, und verharrte dann so, mit einer gequälten Miene auf seinem Gesicht.

Professor Quirrell schien sich ziemlich zu amüsieren. Er balancierte das Buch auf einer Ecke, auf einem Finger, und hielt es aufrecht, während er eine kleine Melodie summte.

Jemand klopfte an der Tür.

Das Buch verschwand wieder in Professor Quirrells Umhang, und er erhob sich von seinem Stuhl. Professor Quirrell begann zur Tür hinüber zu gehen \later

\later und strauchelte, sodass er plötzlich gegen die Wand taumelte.

„Es ist alles gut,“ sagte Professor Quirrells Stimme, die plötzlich um einiges schwächer klang als sonst. „Setzen Sie sich, Mr. Potter, es ist nur ein Schwindelanfall. Setzen Sie sich.“

Harrys Finger klammerten sich an die Kanten seines Stuhls, er war sich unsicher was er tun sollte, was tun \emph{konnte.} Harry konnte sich nicht einmal Professor Quirrell zu sehr nähern, solange er nicht diesem Gefühl des Verderbens trotzen wollte \later

Professor Quirrell richtete sich auf, und öffnete dann, mit anscheinend schwerem Atmen, die Tür.

Die Kellnerin trat mit einer Platte voll Essen ein; und während sie die Teller verteilte, ging Professor Quirrell langsam zurück zum Tisch.

Doch, in dem Moment als die Kellnerin sich ihren Weg heraus verbeugt hatte, saß Professor Quirrell aufrecht und lächelte wieder.

Dennoch, dieser kurze Vorfall von was-auch-immer-es-war hatte Harry zu einer Entscheidung gebracht. Er konnte es nicht ablehnen, nicht nachdem Professor Quirrell so viel auf sich genommen hatte.

„Ja“, sagte Harry.

Professor Quirrell hielt einen warnenden Finger hoch, zog dann seinen Zauberstab hervor, schloss wieder die Tür, und wiederholte drei der gleichen Zauber, die er schon früher gesprochen hatte.

Dann nahm Professor Quirrell das Buch wieder aus seinem Umhang und warf es Harry zu, der es beinahe in seine Suppe fallen ließ.

Harry warf Professor Quirrell einen Blick hilfloser Empörung zu. Man machte so etwas nicht mit Büchern, verzaubert oder nicht.

Harry öffnete das Buch mit tiefsitzender, instinktiver Vorsicht. Die Seiten schienen zu dick, mit einer Beschaffenheit anderes als Muggelpapier oder auch dem Pergament der Zauberer. Und der Inhalt war…

… leer?

„Sollte Ich etwas sehen \later“

„Schauen näher am Anfang nach,“ sagte Professor Quirrell, und Harry blätterte (wieder mit hilfloser, tief verwurzelter Umsicht) einige Seiten zurück.

Die Schrift war offensichtlich handgeschrieben, und sehr schwer zu lesen, aber Harry vermutete die Worte wären Latein.

„Was ist das?“ sagte Harry.

„Das,“ sagte Professor Quirrell, „sind die Aufzeichnungen der magischen Untersuchungen eines Muggelstämmigen, der niemals nach Hogwarts kam. Er lehnte seinen Brief ab, und vollführte seine eigenen kleinen Nachforschungen, die, ohne einen Zauberstab, nie besonders weit kamen. Von der Beschreibung auf dem Schild ausgehend, erwartete ich, dass der Name für Sie weitaus mehr Bedeutung hat als für mich. Das, Mr. Potter, ist das Tagebuch von Roger Bacon.“

Harry fiel fast in Ohnmacht.

Klebend an der Wand, an der Stelle wo Professor Quirrell gestrauchelt war, glänzten die zerquetschten Überreste eines hübschen blauen Käfers.

