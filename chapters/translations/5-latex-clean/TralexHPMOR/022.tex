

\hypertarget{die-wissenschaftliche-methode}{% \section{1. Die wissenschaftliche Methode}\label{die-wissenschaftliche-methode}}

\emph{Nach einer langen Pause möchte ich das Projekt wieder einmal weiterführen. Dabei gibt es zwei Ankündigungen. Erstens ich habe mir die Unterstützung einer "Lektorin„ gesucht, die sich doch einiges besser in der deutschen Sprache ausdrücken kann. Außerdem möchte ich das Projekt etwas umstrukturieren, also zunächst jene Teile übersetzen die zur eigentlich Haupthandlung gehören, sodass sie Chance steigt, die Geschichte jemals fertigzustellen. Über Unterstützung bei Übersetzung bzw. Hörbuch (Youtube) würde ich mich natürlich freuen; Schreibt in diesem Fall doch etwas auf diesem Discord \url{https://discord.gg/wV5J9xu} ~}

Harry Potter\\ und die Methoden des rationalen Denkens

von Eliezer Yudkowski\\ übersetzt von TR\\ Zweites Buch: Harry James Potter-Evans-Verres und die Spiele des Professor

Kapitel 1\\ Die wissenschaftliche Methode

Ein kleines Studierzimmer, in Nähe von, aber nicht im Ravenclaw-Schlafsaal, einer der vielen ungenutzten Räume von Hogwarts. Böden aus grauem Stein, Wände aus roten Ziegeln, die Decke aus dunkel verfärbtem Holz, vier leuchtene Glaskugeln, eingelassen in den vier Wänden des Raumes. Ein kreisrunder Tisch, der wirkte wie eine breite Scheibe aus schwarzen Marmor gestützt von schwarzen Marmorsäulen, der aber, wie es sich herausstellte doch sehr leicht war und daher leicht anzuheben und zu bewegen wäre, sollte das nötig sein. Die zwei komfortabel ausgepolsterte Sessel, die scheinbar an unangenehmen Plätzen am Boden befestigt wurden, würden, so hatten die beiden schließlich herausgefunden, sofort zu jeder Person flitzen, die Anstalten machte sich hinzusetzen.

Es scheinen auch etliche Fledermäuse durch das Zimmer zu flattern.

Dies war der Ort, wie es zukünftige Historiker einmal festhalten werden, an dem -- wenn das Projekt jemals wirklich etwas ergab -- die wissenschaftliche Erforschung der Magie begonnen hatte, mit zwei jungen Hogwarts-Erstklässern.

Harry James Potter-Evans-Verres, Theoretiker.

Und Hermine Jean Granger, Experimentartorin und Testsubjekt.

Harry hatte inszwischen mehr Erfolg in seinen Schulfächern, zumindest in denen, die er als wichtig erachtete. Er hatte mehr Bücher gelesen, und zwar nicht Bücher für Elfjährige. Er hatte Verwandlung wieder und wieder in einer seiner Zusatzstunden geübt und eine weitere Stunden zur Übung von Okklumentik verwendet. Er nahm die lohnenswerten Fächer ernst, nicht nur einfache Abgabe der erforderten Hausaufgaben, sondern beschäftigte sich mit über den Unterrichtsstoff herausgehender Lektüre, um das Thema zu meistern, statt nur einzelne Antworten zu lernen; um - zu brillieren. Sowas war unüblich außerhalb von Ravenclaw. Selbst innerhalb von Ravenclaw, blieben als seine letzten Wettstreiter nur Padma Patil (deren Eltern nicht aus einer englischsprachigen Kultur kamen, sodass sie mit einer wirklichen Arbeitsethik erzogen wurde), Anthony Goldstein (aus einer bestimmten kleinen etnischen Gruppe, die 25\% der Nobelpreise gewonnen hat) und natürlich, weit überlegen, wie ein Titan, der durch eine Haufen Hundewelpen schreitet, Hermine Granger.

Um jenes spezifische Experiment auszuführen, musste das Testsubjekt sechzehn neue Zauber erlernen, auf sich alleingestellt, ohne Hilfe oder Korrektur. Das bedeutet das Testsubjekt war Hermine. Punkt.

Man sollte jetzt einmal erwähnen, dass die Fledermäuse zurzeit nicht leuchteten.

Harry hatte Schwierigkeiten mit den Implikationen fertig zu werden.

„Oogley boogely!“, sagte Hermine wieder.

Abermals, erschein an der Spitze ihres Zauberstabs abrupt und übergangslos das Erscheinungsbild einer Fledermaus. In einem Moment leere Luft, im Nächsten, Fledermaus.\\ Ihre Flügel schienen bereits in Bewegung zu sein, sobald sie auch nur erschienen war.

Und sie leuchtete noch immer nicht.

„Kann ich jetzt aufhören?“, fragte Hermine.

„Bist du sicher, “ sagte Harry mit einem Kloß im Hals, „dass du nicht mit etwas mehr Übung, sie auch zum Leuchten bringen könntest?“.

Er verletzte hiermit seine zuvor eigens angefertigte Experimentierverfahrensweise, eine Sünde, und zwar weil er die Ergebnisse nicht anerkennen wollte, eine Todsünde, die dich bis in die Wissenschaftshölle bringen könnte, aber es schien alles sowieso keinen Unterschied mehr zu machen.

„Was hast du dieses Mal verändert?“, sagte Hermine, mit leicht genervter Stimme.

„Die Dauer der oo, eh und ee Laute. Es sollte im Verhältnis 3 zu 2 zu 2 nicht 3 zu 1 zu 1 sein.

„Oogley boogely!“, sagte Hermine.

„3 zu 2 zu 1.“

„Oogley boogely!“

Dieses Mal hatte die Fledermaus nicht einmal Flügel und fiel wie eine tote Maus.

„3 zu 1 zu 2.“

Und siehe da, die Fledermaus erschien und sie flog direkt zur Decke, völlig gesund und in grüner Farbe hell leuchtend.

Hermine nickte zufrieden gestellt. „Okay, was jetzt?“

Stille trat ein.

„Ernsthaft? Du musst ernsthaft die Laute von Oogley boogely, also oo {[}u{]}, eh {[}e{]} und ee {[}i{]} mit einem Verhältnis von 3 zu 1 zu 2 sagen, oder die Fledermaus leuchtet nicht? Warum? Warum? Im Namen von allem was heilig ist, warum?

„Warum nicht“

„AAAAAAAAARRRRRRGHHHH“

=========================

In den Kerkern von Slytherin.

Ein ungenutztes Klassenzimmer, erleuchted mit gespenstig grünen Licht, dieses Mal sehr viel heller, drang von einer zeitweise verzauberten kleinen Kristallkugel hervor. Trotzdem war es immer noch gespenstiges grünes Licht, dass merkwürdige Schatten warf.

Zwei Figuren, so groß wie Jungen, gekleidet in graue Umhänge (keine Masken) hatten den Raum still betreten, und setzten sich gegenüber auf zwei Stühle an den gleichen Tisch.

Es was das zweite Treffen der Bayes'chen Verschwörung.

Draco Malfoy war sich nicht sicher gewesen ob er sich hierauf freuen sollte.

Harry Potter, seinem Gesichtsausdruck nach zu schließen, schein keine Zweifel für die angemessene Stimmung zu haben.

Harry Potter sah aus als ob er bereit wäre zu töten.

„Hermine Granger.“ sagte Harry Potter, gerade als Draco seinen Mund öffnete. „Frag nicht.“

„\emph{Es} \emph{könnte doch nicht ein weiteres Date gegeben haben, oder?“} dachte Draco, aber sagte\\ nichts.

„Harry.“ sagte Draco „Es tut mir Leid, dass ich dich das fragen muss, aber hast du wirklich für das Schlammblutmädchen einen teuren Eselsfellbeutel als Geburtstaggeschenk bestellt?“

„Ja, das habe ich. Du hast natürlich bereits herausgefunden warum.“

Draco griff nach oben und raufte sich frustiert die Haare, dabei streichte seine Kapuze die Rückseite seiner Hand. Er war sich \emph{nicht} sicher gewesen warum, aber jetzt konnte er das nicht mehr zugeben. Und Slytherin wusste, dass er um Harry bemüht war, nach der Verteidigungsstunde sollte er es offentsichtlich genug gemacht haben. „Harry,“ sagte Draco, „alle wissen das wir Freunde sind, sie wissen zwar nichts von der Verschwörung, aber sie wissen, dass wir befreundet sind, und es stellt mich schlecht dar wenn du solche Sachen machst.“

Harry Potters Gesicht wurde härter. „Jeder in Slytherin, der das Konzept des freundlich auf jene wirken, die man eigentlich nicht leiden kann, nicht versteht, sollte festgehalten und an Schlangen verfüttert werden.“

„Es gibt viele in Slytherin, die daran scheitern“, sagte Draco, seine Stimme ernst. „Viele Leute sind zu beschränkt, und trotzdem muss man sich vor ihnen gut aufführen.“ Harry Potter \emph{musste} das einfach verstehen, wenn er jemals irgendwas in seinem Leben erreichen wollte.

„Was kümmert es dich was die Anderen denken? Wirst du wirklich dein Leben so leben, immerzu alles was du tust erkärend, sodass selbst der bescheuertste Slytherin mitkommt, das \emph{sie dich} dafür urteilen? Es tut mir Leid, Draco, aber ich werde meine raffinierten Pläne nicht auf das Level des dümmsten Slytherin bringen, nur weil es dich vielleicht schlecht aussehen ließe. Nicht einmal deine Freundschaft ist das wert. Es würde den ganzen Spaß aus dem Leben nehmen. Sag mir du hättest noch nicht das Gleiche gedacht, wenn ein Slytherin sich mal wieder zu blöd zum Atmen verhält, dass es unter der Würde eines Malfoys ist, sich bei solchen Leuten beliebt zu machen.“

Draco hatte das ernsthaft noch nicht. Jemals. Sich bei Idioten beliebt zu machen war vergleichbar zum Atmen, man machte es ohne darüber nachzudenken.

„Harry,“ sagte Draco schließlich. „Stets nur das zu tun was du willst, ohne darüber nachzudenken wie es aussieht, ist nicht klug. Selbst der Dunkle Lord achtete auf sein Erscheinungsbild. Er war gefürchet und gehasst und er wusste \emph{genau} welche Art von Furcht und Hass er hervorrufen wollte. \emph{Jeder} muss sich darum sorgen, was die Anderen denken.“

Die umhangbehangende Figur zuckte. „Vielleicht. Erinnere mich, dir irgendwann etwas vom\\ Konformitätsexperiment von Asch zu erzählen. Du könntest es ganz amüsant finden. Jetzt werde ich erstmal nur anmerken, dass es gefährlich ist, sich Sorgen über das zu machen, was andere Leute instinktiv denken, weil wenn es dich wirklich kümmert, es keine Frage kaltblütiger Berechnung mehr ist. Denk daran, ich wurde von älteren Slytherins für fünfzehn Minuten geschlagen und beschimpft, und danach stand ich auf und verzieh ihnen gnädig. Genau wie es der gute und tugendhafte Junge-der-überlebt-hattun sollte. Aber meine kaltblütigen Berechnungen sagen mir, Draco, dass ich keinen Nutzen für die dümmsten Idioten in Slytherin habe, schlicht da ich keine Schlange als Haustier besitze. Also sehe ichauch keinen Grund zu beachteten, wie sie über die Art meines Duells mit Hermine Granger denken.“

Draco ballte seine Hände frustiert zu Fäusten. „Sie ist doch nur ein Schlammblut,“ sagte Draco, seine Stimme kontrollierend, anstatt zu schreien. „Wenn du sie nicht leiden kannst, schubse sie die Treppe runter.“

„Die Ravenclaws würden wissen ----“

„Lass Pansy Parkinson sie die Treppe runter schubsen. Dafür müsstest du sie nicht einmal manipulieren, biete ihr einen Sickel und sie würde es machen!“

„Ich würde es wissen! Hermine hat mich in einem Buchlese-Wettschreit besiegt, sie bekommt bessere Noten als ich, ich muss sie mithilfe meines Gehirns schlagen, sonst zählt es nicht.

„\emph{Sie ist doch nur ein Schlammblut! Wieso respektierst du sie so sehr?“}

\emph{„Sie es eine Kraft unter Ravenclaws! Warum kümmert es dich was irgendein machtloser Idiot in Slytherin denkt?“}

\emph{„Das nennt man Politik! Und wenn du dort nicht mitspielen kannst, kannst du auch auch keine Macht erlangen!“}

\emph{„Auf dem} \emph{Mond} \emph{zu laufen ist Macht. Ein großer Zauberer zu sein ist Macht! Es existieren Arten von Macht, die mich nicht dazu zwingen, den Rest meines Lebens mit dem} \emph{Einschmeicheln bei Trotteln zu verbringen.“}

Beide stoppten und beinahe zeitgleich begannen sie tief Atemzüge zu nehmen, um sich zu beruhigen.

„Entschuldigung,“ sagte Harry nach einer Weile, den Schweiß von der Stirn wischend. „Entschuldigung, Draco. Du besitzt viel politische Macht und es ist sinnvoll diese zu erhalten. Du sollst einberechnen was Slytherin denkt. Es ist ein wichtiges Spiel, das ich nicht beleidigen sollte. Trotz allem kannst du nicht verlangen, dass ich mein Level des Spiels in Ravenclaw senke, nur damit du nicht schlecht aussiehst, wenn du dich mir mir abgibst. Erzähl Slytherin, dass du es zähneknischend aushälst, um vorzugeben mein Freund zu sein.“

Das war genau was Draco Slytherin erzählt \emph{hatte,} und er war sich noch sicher, ob es richtig war.

„Trotzdem,“ sagte Draco. „Wenn wir schon von deinem Erscheinungsbild reden. Ich fürchte ich habe schlechte Nachrichten. Rita Kimmkorn hat einige der Geschichten über dich gehört und stellt Fragen.“

Harry hob seine Augenbrauen. „Wer?“

„Sie schreibt für den \emph{Tagespropheten},“ sagte Draco. Er versuchte die Besorgnis aus seiner Stimme herauszuhalten. Der \emph{Tagesprophet} war eines von Vaters bevorzugten Werkzeugen, er verwendete ihn wie ein Zauberer einen Zauberstab. „Das ist die Zeitung, die die Leute wirklich beachten. Rita Kimmkorn schreibt über Prominente, und nutzt ihre Feder, wie sie es ausdrückt, um deren aufgeblassenen Ruf aufzustechen. Wenn sie keine Gerüchte findet, wird sie einfach eigene \emph{erfinden}.“

„Ich verstehe,“ sagte Harry Potter. Sein grün-erleuchtetes Gesicht sah unter dem Umhang sehr gedankenverloren aus.

Draco zögerte bevor er sagte, was er sagten musste. Gewiss hatte jemand inszwischen Vater in Kenntnis gesetzt, dass Draco sich um Harry bemühte, und Vater würde auch wissen, dass Draco darüber nichts geschrieben hatte, und Vater würde auch verstehen, dass Draco nicht glaubte, dass er es verheimlichen könnte, was die klare Nachricht vermittelte, dass Draco nun sein eigenes Spiel ausübte, dabei aber immernoch auf Vaters Seite war, denn er würde falsche Berichte senden, falls er zum Seitenwechsel verleitet worden wäre.

Sein Vater würde dementsprechend vermutlich auch voraussehen, was Draco als nächstes sagen würde.

Das Spiel mit Vater wirklich zu spielen war eine eher anstrengende Erfahrung. Sogar, wenn sie auf der gleichen Seite waren. Es war einerseits belebend, aber Draco wusste bereits, dass zum Schluss herauskommen würde, dass Vater besser gespielt hatte. Es gab keine Alternative zu diesem Ausgang.

„Harry,“ sagte Draco schließlich. „Das ist kein Vorschlag. Das ist nicht mein Vorschlag. Das ist einfach so. Mein Vater kann höchstwahrscheinlich den Artikel unterdrücken, aber es würde dich etwas kosten.

Den Umstand, dass Vater erwartete, dass Draco Harry genau dies sagte, war nichts was er aussprechen würde. Harry Potter würde es selbst herausfinden, oder eben nicht.

Aber stattdessen schüttelte Harry seinen Kopf, unter dem Umhand lächelnd. „Ich habe nicht vor mich von Rita Kimmkorn unterdrücken zu lassen.“

Draco versuchte erst gar nicht die Ungläubigkeit aus seine Stimme herauszuhalten. „Du \emph{kannst} mir nicht erzählen, dass dich nicht einmal die Meinung der \emph{Zeitung} kümmert!“

„Sie kümmert mich weniger als du vielleicht annimmst,“ sagte Harry. „Doch ich habe andere Wege um mit jemandem wie Kimmkorn umzugehen. Lucius' Hilfe benötige ich nicht.

Ein besorgter Blick stahl sich auf Dracos Gesicht, bevor er es zurückhalten konnte.\\ Was auch immer Harry Potter als nächstes tun würde, war es in jedem Fall etwas Unerwartetes, sogar für seinen Vater, und Draco fühlte sich sehr nervös, über die Richtung, in die das wohl führen könnte.

Draco bemerkte außerdem, wie er unter der Kutte seines Umhangs zu schwitzen begann.\\ Er hatte zuvor noch nie wirklich einen getragen und nicht gemerkt, dass die Todesser vermutlich soetwas wie Kühlzauber auf ihren Umhängen hatten.

Harry Potter wischte sich den Schweiß wieder von der Stirn, verzog das Gesicht, zog seinen Zauberstab, zielte nach oben, atmete tief ein und sagte „Frigideiro!“

Kurze Zeit später fühlte Draco den kalten Lufthauch.

„Frigideiro! Frigideiro! Frigideiro! Frigideiro! Frigideiro!“

Dann senkte Harry seinen Zauberstab, auch wenn seine Hand ein wenig zittrig schien, und steckte ihn zurück in seinen Umhang.

Der gesamte Raum schien merklich kühler. Draco hätte das auch vollbringen können, aber trotzdem, nicht schlecht.

„So,“ sagte Draco. „Wissenschaft. Du wolltest mir etwas über Blut erzählen.

„Wir werden etwas über Blut \emph{herausfinden,}“ sagte Harry Potter. „Mithilfe von Experimenten“

„Alles klar“, sagte Draco. „Welche Art von Experimenten?“

Harry Potter lächelte böse unter seiner Kutte und sprach: „Sagt du es mir.“

Draco hatte bereits von so etwas wie der Sokratischen Methode gehört, nämlich das Unterrichten mithilfe von Fragen stellen (benannt nach einem antiken Philisophen, der zu schlau war um ein Muggel zu sein und daher ein getarnter reinblütiger Zauberer). Einer seiner Mentoren hatte Sokratisches Lehren viel benutzt. Es war anstrengend, aber effektiv.

Und es gab auch die Potterische Methode, die war einfach verrückt.

Fairerweise, das musste Draco zugeben, hatte Harry Potter zunächst die Sokratische Methode probiert und das hatte nicht allzu gut funktioniert.

Harry Potter hatte Draco gefragt, wie er denn versuchen würde die Reinblüter Hypothese, dass heutige Zauberer nicht mehr das schöne Zeug vollbringen konnten, das sie vor acht Jahrhunderten noch schafften, weil sie sich mit Muggelstämmigen und Squibs gekreuzt hatten, zu widerlegen.

Draco hatte gesagt, dass er nicht verstehe, wie Harry Potter dort mit ernstem Gesicht sitzen konnte, während er behauptete das sei keine Falle.

Harry Potter hatte geantwortet, immernoch mit ernstem Gesicht, das falls dies eine Falle wäre, sie so lächerlich offentsichtlich wäre, dass er erfasst und an Schlangen verfüttert werden solle, und dass es nicht so wäre, dass es schlicht eine Verhaltensregel der Wissenschaftler wäre, stets zu versuchen seine eigenen Theorien zu widerlegen und dass wenn man nach ehrlicher Anstrengeng keinen Erfolg habe, das der Sieg war.

Draco hatte versucht die schiere Dummheit davon zu verdeutlichen, indem er vorschlug, dass es der Schüssel zum Überleben eines Duells wäre, Avada Kedavra auf seinen eigenen Fuß zu wirken, aber nicht zu treffen.

Harry Potter hatte genickt.

Draco hatte seinen Kopf geschüttelt.

Harry Potter hatte dann den Ansatz vorgestellt, dass Wissenschaftler zusahen wie Ideen kämpften, um zu sehen wer gewinnt und dass man nicht ohne Gegner kämpfen könne, Draco sich also Gegner zum Reinblütertum ausdenken müsste, damit es gewinnen könne, was Draco ein wenig besser verstand, obgleich Harry Potter es mit einem Abneigung widerspiegelden Blick sagte.

So als ob es klar wäre, dass wenn Reinblütertum wirklich die Art war wie die Welt funktionierte, der Himmel blau seinen müsste. Jedoch falls eine anderes Theorem stimmte, der Himmel grün sein müsste; und dass niemand bisher den Himmel gesehen hat; und dass wenn man raus ging und nachschaute, die Reinblüter gewannen; und nachdem das sechs mal in Folge passierte, die Leute den Trend bemerken würden.

Harry hatte stets behauptet, seine erdachten Gegner seien zu schwach, sodass Reinblütertum keine Anerkennung für einen Sieg gewinnen würden, da die Kämpfe nicht eindrucksvoll genug waren. Draco hatte das auch verstanden. \emph{Zauberer sind schwächer geworden, da Hauselfen die Magie stehlen,} hatte sich auch für ihn nicht überzeugend angehört.

(Obwohl Harry Potter gesagt hatte, dass dies zumindest testbar war, da sie nachschauen konnten, ob Hauselfen stärker geworden waren und hatte sogar zwei Grafiken erstellt, die die steigende Stärke der Hauselfen, beziehungsweise die fallende Stärke der Zauberer repräsentierten und dass falls die beiden Bilder passten, dass dies auf die Hauselfen deuten würde, alles in einem so völlig ernsten Ton, dass Draco schon dem Impuls fühlte, Dobby ein paar bestimmte Fragen unter Veritaserum zu stellen, bevor er sich zusammenriss.)

Und Harry Potter hatte schließlich gesagt, dass Draco den Kampf nicht manipulieren durfte, Wissenschafter wären nicht dumm, es wäre offentsichtlich, wenn der Kampf manipuliert worden wäre, es musste ein richtiger Kampf sein, zwischen zwei unterschiedlichen Theorien, die beide wirklich wahr sein konnten, mit einem Test, den nur die wahre Hypothese gewinnen würde, etwas das tatsächlich anders herauskommen würde, je nachdem welche Hypothese tatsächlich richtig war, und Wissenschaftler würden zuschauen, um sicher zu stellen, dass das exakt so passierte. Harry Potter hatte behauptet, er wolle nur wissen \emph{wie Blut wahrhaftig funktionierte}, und dass er dafür sehen müsse wie Reinblütertum wirklich gewann, und dass Draco ihn nicht mit Theorien reinlegen würde, die nur dafür da waren um verworfen zu werden.

Sogar nachdem Draco diesen Punkt eingesehen hatte, hatte er keinen Erfolg beim Erfinden von „plausiblen Alternativen“ gehabt, wie Harry Potter es formuliert hatte, zu der Idee, dass Zauberer weniger mächtig wurden, weil sie Blut mit Schlamm vermischten. Es war \emph{zu} offentsichtlich richtig.

Es war in dem Moment, als Harry Potter, ziemlich frustriert, gesagt hatte, er könne sich nicht vorstellen, wie Draco \emph{wirklich} so schlecht im Einbeziehen anderer Blickwinkel sein könne, \emph{sicherlich} habe es doch bereits Todesser gegeben, die sich als Gegner der Reinblüter aufgeführt haben und plausibler-klingende Argumente gegen die eigene Seite vorgebracht hatten, als das was Draco anbot. Wenn Draco probiert hätte, sich als Anhänger von Dumbledores Fraktion auszugeben, und er mit der Hauselfen-Hypothese aufgetaucht wäre, hätte er niemanden auch nur eine Sekunde getäuscht.

Draco war gezwungen gewesen zuzugeben, dass da etwas dran war.

Darum die Potter-Methode.

„Bitte, Dr. Malfoy“, jammerte Harry Potter, „wieso wollen Sie mein nicht Paper akzeptieren?“

Harry Potter hatte es dreimal wiederholen müssen, bevor Draco es verstanden hatte. „Gebe nur vor, vorzugeben ein Wissenschaftler zu sein.“

In diesem Moment hatte Draco erkannt, dass etwas in Harry Potters Gehirn gewaltig \emph{falsch} lief, und dass jeder, der Legilimentik auf es versuchte, niemals wieder herauskommen würde.

Harry Potter hatte es dann noch einiges detaillierte erklärt: Draco sollte so tun als sei er ein Todesser, der vorgab Dr. Malfoy, der Herausgeber eines wissenschaftlichen Journal, zu sein, der das Paper seines Feindes Dr. Potter „Die Vererbung der magischen Fähigkeit“ ablehnen wollte, jedoch falls er sich nicht verhielt wie ein richtiger Wissenschaftler, er als Todesser enttarnt und hingerichtet werden würde, während er gleichzeitig noch von seinen eigenen Rivalen beobachtet wurde und er so \emph{wirken} musste, als ob er Dr. Potters Paper aus neutralen wissenschaftlichen Gründen ablehnte, weil er sonst seine Stelle als Journal-Herausgeber verlieren würde.

Es war ein Wunder, dass der Sprechende Hut nun \emph{nicht} verrückt vor sich herstammelnd im St. Mungos lebte.

Es war außerdem das Komplizierteste, das irgendjemand \emph{jemals} von ihm verlangt hatte vorzugeben und so gab keine Möglichkeit der Challenge zu widerstehen.

Gerade jetzt waren sie, wie Harry Potter es genannt hatte, beim in Stimmung kommen.

„Mir scheint, Dr. Potter, dass Sie das in der falschen Tintenfarbe geschrieben haben.“, sagte Draco. „Der Nächste!“

Dr. Potters Gesicht führte eine exzellente Darbietung auf, wie es in Verzweiflung zerfloss, und Draco konnte nicht verhindern etwas von Dr. Malfoys Schadenfreude zu fühlen, obwohl der Todesser nur vorgab dieser zu sein.

Dieser Teil war \emph{lustig}. Er könnte sowas den ganzen Tag machen.

Dr. Potter stand von seinem Stuhl auf, sackte vor Bestürzung zusammen, und stapfte davon, und verwandelte sich in Harry Potter, der Draco einen Daumen-nach-oben gab, sich wieder in Dr. Potter verwandelte und mit eifrigem Lächeln auf hin zu kam.

Dr. Potter setzte sich und präsentierte Dr. Malfoy ein Stück Pergament, auf dem Stand:

\emph{Die Vererbung der magischen Fähigkeit}

\emph{Dr. H.J. Potter-Evans-Verres, Institut für ausreichend fortgeschrittene Wissenschaft}

\emph{Meine Beobachtung:}

\emph{Heutige Zauberer können nicht mehr so eindrucksvolle Dinge vollbringen,}\\ \emph{wie es Zauberer vor 800 Jahren tun konnten.}

\emph{Meine Schlussfolgerung:}

\emph{Die Zaubererschaft wurde schwächer, indem sie ihr Blut mit dem von Muggelstämmigen und Squibs mischten.}

„Dr. Malfoy,“, sagte Dr. Potter mit einem hoffnungsvollen Blick, „Ich habe mich gefragt ob das \emph{Journal der nicht reproduzierbaren Resultate} in Betracht ziehen könnte, mein Paper namens „Die Vererbung der magischen Fähigkeit“ zu veröffentlichen.“

Draco sah sich das Pergament lächelnd an, während er mögliche Zurückweisungen erwägte.\\ Wenn er ein Professor wäre, würde es die Abhandlung als zu kurz ablehnen, also --

„Es ist zu lang, Dr. Potter,“ sagte Dr. Malfoy.

Einen Moment war ehrliche Ungläubigkeit auf Dr. Potter's Gesicht zu erkennen.

„Ah…“ sagte Dr. Potter. „Was wenn ich die getrennten Zeilen für Beobachtung und Schlussfolgerung entferne, und ich stattdessen ein \emph{daher ----“}

\emph{„}Dann wäre es zu kurz. Der nächste!“

Dr. Potter trottete davon.

„Also gut,“ sagte Harry Potter, „du wirst zu gut darin. Zweimal noch zur Übung, und dann beim dritten Mal richtig, keine Unterbrechungen dazwischen, Ich werde einfach direkt auf dich zu kommen und dieses Mal wirst du das Paper aufgrund seines wirklichen Inhalts ablehnen, denk daran, deine wissenschaftlichen Rivalen schauen zu.“

Dr. Potters nächstes Paper war perfekt in jeder Hinsicht, musste jedoch leider abgelehnt werden, da Dr. Malfoys Journal Probleme mit dem Buchstaben E hatte. Dr. Potter bot an es ohne solche Wörter nochmal zu schreiben, und Dr. Malfoy erklärte, dass es doch eher ein Vokalproblem wäre.

Das Paper danach wurde ablehnt, weil es Dienstag war.

Es war, in Wirklichkeit, Samstag.

Dr. Potter versuchte das hervorzuheben und bekam „Der Nächste!“ zu hören.

(Draco begann zu verstehen, warum Snape seinen Einfluss über Dumbledore genutzt hatte, nur um eine Position zu erlangen, die es ihm erlaubte schrecklich zu Schülern zu sein.)

Und dann ----

Dr. Potter näherte sich mit einem überlegenen Grinsen auf dem Gesicht.

„Dies ist mein jüngstes Paper, \emph{Die Vererbung der magischen Fähigkeit,“} verkündete Dr. Potter selbstbewusst, und zog das Pergament hervor. „Ich habe mich dazu entschlossen, ihrem Journal zu gestatten es zu veröffentlichen und habe es unter vollständiger Einhaltung ihrer Richtlinien erstellt, auf dass Sie es schnell veröffentlichen können.“

Der Todesser entschied, Dr. Potter ausfindig zu machen und zu ermorden nachdem seine Mission abgeschlossen war. Dr. Malfoy behielt ein höfliches Lächeln auf seinem Gesicht, weil seine Rivalen zusahen, und sagte…

(Die Pause streckte sich, begleited von ungeduldigen Blicken von Dr. Potter.)

…„Lassen Sie mich bitte einen Blick auf das werfen.“

Dr. Malfoy nahm das Pergament und prüfte es sorgfältig.

Der Todesser begann nervös zu werden, wegen der Tatsache, dass er kein echter Wissenschaftler war und Draco versuchte sich zu erinnern wie Harry Potter redete.

„Sie müssen andere mögliche Erklärungen für Ihre Beobachtungen in Betracht ziehen, abseits dieser einen ----“

„Ernsthaft?“ unterbrach Dr. Potter. „Was denn zum Bespiel? \emph{Hauselfen stehlen unsere Magie?} Meine Daten erlauben nur eine mögliche Schlussfolgerung, Dr. Malfoy. Es gibt keine anderen plausiblen Hypothesen.“

Draco versuchte aufgebracht seine Gedanken zu ordnen, was würde er sagen, wenn er sich als Mitglied von Dumbledores Fraktion ausgab, was gaben die als Erklärung für die Schwächung der Zauberer an, Draco hatte sich nie darum gekümmert, wirklich zu fragen.

„Wenn Ihnen keine Alternative einfällt, die meine Daten erklärt, werden Sie wohl mein Paper veröffentlichen müssen, Dr. Malfoy.“

Es war der Spott auf Dr. Potters Gesicht, der es auslöste.

„Ist es so?“, gab Dr. Malfoy zurück. „Woher wissen Sie, dass nicht die Magie selbst verblasst?“

Die Zeit stoppte.

Draco und Harry Potter tauschten Blicke des entsetzten Horrors.

Dann rief Harry Potter etwas, dass wahrscheinlich ein extrem schlimmes Word war, wenn man bei Muggeln aufgewachsen wäre. „\emph{Daran habe ich nicht gedacht!“} sagte Harry Potter. „Und das hätte ich tun sollen. Die Magie verschwindet. Verdammt, Verdammt, Verdammt!“

Die Furcht in Harry Potters Stimme war ansteckend. Ohne auch nur darüber nachzudenken, griff Dracos Hand unter seinen Umhang und umklammerte seinen Zauberstab. Er hatte gedacht das Haus Malfoy wäre \emph{sicher,} dennsolange man nur in Familien heirate, die Ihre Blutbahnen vier Generationen zurückverfolgen konnten, sollten man doch \emph{sicher} sein, es war ihm nie in den Sinn gekommen, dass es vielleicht nichts gab, was irgendjemand tun könnte, um das Ende der Magie zu verhindern. „Harry, was machen wir jetzt?“ Dracos Stimme schwoll in Panik an. \emph{„Was machen wir jetzt?“}

\emph{„Lass mich nachdenken!“}

\emph{Nach kurzer Zeit, griff Harry sich vom nahen Schreibtisch die selbe Feder und} Pergamentrolle, die er für sein vorgetäuschtes Paper verwendet hatte und begann etwas darauf zu kritzeln.

„Wir werden es herausfinden,“ sagte Harry, seine Stimme gefasst, „falls die Magie aus der Welt verblasst, werden wir herausfinden wie schnell sie verblasst und wie viel Zeit wir übrig haben um zu reagieren und dann werden wir herausfinden warum sie verblasst, und dann werden wir etwas deswegen tun. Draco, schwächen sich die Zauberkräfte stetig oder gab es scharfe Abbrüche?“

„Ich… Ich weiß nicht…“

„Du hast mir gesagt, niemand wäre auf dem Level der vier Gründer von Hogwarts. So läuft es schon seit acht Jahrhunderten, oder? Du kannst dich nicht daran erinnern, jemals etwas über das plötzliche Erscheinen dieser Probleme fünf Jahrhunderte zuvor und etwas Ähnliches gehört zu haben?“

Draco versuchte verzweifelte nachzudenken. „Ich habe immer gehört, dass niemand so mächtig wie Merlin war und danach dass niemand so mächtig wie die Gründer von Hogwarts waren.“

„Ok,“ sagte Harry. Er schrieb noch immer. „Weil es vor drei Jahrhunderten war, als die Muggel anfingen nicht mehr an Magie zu glauben, was möglicherweise etwas damit zu tun hatte.\\ Und ungefähr vor anderthalb Jahrhunderten begannen die Muggel jene Technologie zu verwenden, die im Umfeld von Magie nicht funktioniert und ich hätte mir vorstellen können, dass es auch andersherum passiert.“

Draco fuhr von seinem Stuhl hoch, so wütend, dass er kaum reden konnte. „Es sind die Muggel ----“

„Verdammt noch mal!“, brüllte Harry. „Hörst du überhaupt \emph{dir selbst} zu? Es findet seit mindestens acht Jahrhunderten statt und damals machten die Muggel nichts Spannendes! \emph{Wir müssen die Wahrheit} \emph{ergründen!} Die Muggel \emph{könnten} etwas damit zu tun haben, aber falls nicht und du es trotzdem alles auf sie schiebst, dann behinderst du damit uns beim Herausfinden was wirklich vor sich geht und eines Tages wirst du aufwachen und bemerken, dass dein Zauberstab nur noch ein Stück Holz ist!“

Dracos Atem stockte ihm in der Kehle. Sein Vater verwendete oft in seinen Reden \emph{unsere Zauberstäbe werden uns in den Händen zerbrechen}, aber Draco hatte nie darüber nachgedacht, was das bedeutete, \emph{ihm} würde das ja nicht passieren. Doch jetzt wirkte es plötzlich sehr real. \emph{Nur ein Stück Holz.} Draco konnte sich genau vorstellen, wie es wohl wäre, wenn er seinen Zauberstab hervorzog und versuchte einen Zauber zu wirken und festzustellen, dass nicht passierte…\\ Das könnte \emph{allen} passieren.

Es würde keine Zauberer mehr geben, keine Magie mehr, jemals. Nur Muggel, die noch einige Legenden über die Fähigkeiten ihre Vorfahren parat hatten. Einige dieser Muggel würden Malfoy heißen und das wäre das Einzige, das von dem Namen übrig bliebe.

Zum ersten Mal in seinem Leben, realizierte Draco warum es Todesser gab.

Er hatte es stets als gegeben angenommen, dass Todesser werden einfach etwas war das man machte, wenn man älter wird. Jetzt verstand Draco, er wusste warum sein Vater und Vaters Freunde geschworen hatten ihr Leben zu geben, um diesen Alptraum zu verhindern, es gab Dinge bei denen man nicht daneben stehen und zusehen konnte. Aber was wenn es trotzdem passieren würde, was wenn all die Opfer, all die Freunde, die sie an Dumbledore verloren hatten, die Familie die sie verloren hatten, was wenn das alles für \emph{nichts} gewesen wäre…

„Die Magie kann nicht verblassen,“ sagte Draco. Seine Stimme stockte. „Es wäre nicht \emph{fair}.“

Harry hörte auf zu schreiben und sah hoch. Sein Gesicht zeigte einen gereizten Ausdruck.\\ „Hat dein Vater dir nie erklärt, dass das Leben nicht fair ist?“

Sein Vater hatte das jedes Mal gesagt, wenn er das Wort benutzte. „Aber, aber, es ist zu schrecklich, um geglaubt zu werden ----“

„Draco, lass mich dir die etwas zeigen, ich nenne es die Litanei von Tarski. Sie verändert sich jedes Mal, wenn du sie verwendest. In dieser Situation geht sie so: \emph{Falls Magie aus der Welt verblasst, möchte ich glauben, dass Magie aus der Welt verblasst. Falls Magie nicht aus der Welt verblasst, möchte ich glauben, dass Magie nicht aus der Welt verblasst. Lass mich nicht einem Glauben zugetan werden, den Ich nicht möchte.} Wenn wir in einer Welt leben in der die Magie verblasst, \emph{dann ist es, was wir zu glauben haben,} wir müssen wissen, was kommt, damit wir es aufhalten können, oder um im schlimmsten Fall, vorbereitet zu sein alles Mögliche zu tun, solange wir noch Zeit übrig haben. Nicht daran glauben, wird es nicht aufhalten. Also ist die einzig zu stellende Frage, ob die Magie wirklich verblasst, und falls das die Welt ist in der wir leben, dann ist das genau das, was wir glauben möchten. Litanei von Gendlin: „\emph{Was wahr ist, ist bereits so, es anzuerkennen, macht es nicht schlimmer.} Verstanden, Draco? Ich werde dich später dazu bringen, dir das einzuprägen. Es ist, was man sich selbst sagt, wann immer man überlegt etwas nicht Wahres zu glauben. \emph{Eigentlich, möchte ich, dass du es sofort sagst.} Sag es.“

„\emph{Was wahr ist, ist bereits so,“} sagte Draco, seine Stimme zitternd, „\emph{es anzuerkennen, macht es nicht schlimmer.“}

\emph{„Falls Magie aus der Welt verblasst, möchte ich glauben, dass Magie aus der Welt verblasst. Falls Magie nicht aus der Welt verblasst, möchte ich glauben, dass Magie nicht aus der Welt verblasst.} \emph{Sag es.“}

Draco wiederholte die Worte, während ein ungutes Gefühl seinen Magen aufwühlte.

„Gut,“, sagte Harry, „denk dran, es könnte gerade gar nicht passieren und dann wirst auch nicht daran glauben müssen. Zunächst wollen wir wissen, was wirklich passiert, in welcher Welt wir leben.“ Harry wandte sich wieder seiner Arbeit zu, ergänzte etwas und hielt dann das Pergament Draco hin. Draco beugte sich über den Tisch und Harry schob das grüne Licht näher.

Beobachtung:\\ \emph{Die Zaubererschaft ist nicht mehr so stark, wie zu der Zeit, in der Hogwarts gegründet wurde.}\\

Hypothesen:\\ \emph{1. Die Magie selbst verblasst.}\\ \emph{2. Zauberer} \emph{kreuzen sich mit Muggeln und Squibs.}\\ \emph{3. Das Wissen um mächtige Zaubersprüche geht verloren.}\\ \emph{4. Zauberer essen während ihre Kindheit das falsche Essen, oder irgendwas anderes außer ihrem Blut lass sie schwächer heranwachsen.}\\ \emph{5. Die Muggeltechnologie beeinträchtigt die Magie. (Seit 800 Jahren?)}\\ \emph{6. Stärkere Zauberer haben weniger Kinder (Draco = Einzelkind? Finde heraus, ob drei mächtige Zauberer, Quirrell / Dumbledore / Dunkler Lord, Kinder hatten.)}

Tests:

„Also,“, sagte Harry. Seine Atemzüge klangen ein wenig ruhiger. „Immer wenn man sich mit einem verwirrenden Problem beschäftigt und keine Ahnung hat, was vor sich geht, dann es ist clever sich ein paar wirklich einfache Tests auszudenken, Sachen, die man sich gleich ansehen kann. Wir brauchen schneller Tests, die zwischen diesen Hypothesen unterscheiden. Observationen, die sich für mindestens eine anders verhalten im Vergleich zu allen anderen.“

Draco starrte die Liste schockiert an. Er merkte plötzlich, dass eine schrecklich große Menge an Reinblüter kannte, die Einzelkinder waren. Er selbst, Gregory, Vincent, im Prinzip \emph{jeder}. Die zwei stärksten Zauberer, über die jeder sprach, waren Dumbledore und der Dunkle Lord und keiner von ihnen hatte Kinder, genau wie Harry es vermutete…

„Es wird sehr schwierig sein zwischen 2 und 6 zu unterscheiden,“ sagte Harry, „in beiden Fällen ist es im Blut, man müsste probieren, den Niedergang der Zaubererschaft festzuhalten und mit der Anzahl an Kindern verschiedener Zauberer vergleichen und die Fähigkeiten von Muggelstämmigen und Reinblütern messen und…“ Harrys Finger trommelten nervös auf den Tisch. „Lass uns einfach 6 mit 2 zusammenfassen und sie zunächst die Blut-Hypothese nennen. 4 ist unwahrscheinlich, da man einen abrupten Einbruch bemerken würde, wenn die Zauberer ihre Essensgewohnheiten geändert haben, es ist schwer vorstellbar, was sich stetig über 800 Jahre ändern würde. 5 ist unwahrscheinlich, wegen dem gleichen Grund, kein abrupter Einbruch, die Muggel haben vor 800 Jahren gar nichts gemacht. Sowieso ähnelt 4 2 und 5 1. Das heißt wir sollten uns darauf konzentrieren zwischen 1,2 und 3 zu entscheiden“ Harry drehte das Pergament in seine Richtung, zeichnete eine Ellipse um jene drei Zahlen, und drehte es zurück. „Magie verblasst, Blut wird schwächer, Wissen geht verloren. Welcher Test verhält sich unterschiedlich je nachdem welche Hypothese wahr ist. Was könnten wir sehen, sodass eine der Hypothesen falsch sein muss?“

„\emph{Ich} weiß nicht!“ platzte es aus Draco heraus. „Warum fragst du mich? Du bist doch der Wissenschaftler!“

„Draco,“ sagte Harry, mit einer Note flehende Verzweiflung in seiner Stimme, „Ich weiß nur was die Muggelwissenschaftler wissen! Du bist in der Zaubererwelt aufgewachsen, ich nicht! Du weißst mehr Magie als ich und weißst mehr \emph{über} Magie als ich und außerdem hast du selbst an diese ganze Idee gedacht, also fang an wie ein Wissenschaftler zu denken und es zu lösen!“

Draco schluckte schwer und starrte auf das Paper.

Magie verblasst… Zauberer kreuzen sich mit Muggeln… Wissen geht verloren…

„Wie sieht die Welt aus, wenn Magie verblasst?“ fragte Harry Potter. „Du weißst mehr über Magie, du sollst die Vorschläge machen, nicht ich! Stell dir vor du erzählst eine Geschichte darüber, was passiert in dieser Geschichte?“

Draco stellte es sich vor. „Zauber verlieren ihre Wirkung.“ \emph{Zauberer wachen auf und bemerken, dass ihr Zauberstab nur ein Stück Holz ist…}

\emph{„}Wie sieht die Welt aus, wenn das Zaubererblut schwächer wird?“

„Die Leute können nicht mehr das, was ihre Vorfahren tun konnten.“

„Wie sieht die Welt aus, wenn Wissen verloren geht?“

„Die Leute wissen nicht einmal wie man den Zauber wirkt…“ sagte Draco. Er stockte, überrascht über sich selbst. „Das ist ein Test, oder?“

Harry nickte entschieden. „Das ist einer.“ Er schrieb es auf das Pergament unter Tests:

\emph{A. Gibt es Zauber, die wir kennen, aber nicht benutzen können (1 oder 2) oder sind die Zauber unbekannt (3)?}

„So, das unterscheidet zwischen 1 und 2 auf der einen Seite, und 3 auf der anderen Seite,“ sagte Harry. „Jetzt brauchen wir noch einen Weg um zwischen 1 und 2 zu unterscheiden. Magie verblasst, Blut wird schwächer, wie können wir den Unterschied erkennen?“

„Welche Art Zaubersprüche haben Schüler üblicherweise in ihrem ersten Jahr in Hogwarts geschafft?“ schlug Draco vor. „Wenn sie einst mächtigere Zauber wirken konnten, war ihr Blut stärker ----“

Harry Potter schüttelte seinen Kopf. „Oder die Magie selbst war stärker. Wir müssen einen Weg zum \emph{unterscheiden} finden.“ Harry stand von seinem Stuhl auf und begann nervös im Klassenzimmer auf und ab zu gehen. „Nein, warte, das könnte trotzdem gehen. Nehmen wir an, verschiedene Zauber benötigen unterschiedlich viel magische Energie. Dann würden, wenn sich die umgebenende Magie schwächt, die mächtigen Zauber zuerst aussterben, aber die Erstklässler zauber würden gleich bleiben…“ Harrys nervöses auf-und-ab-gehen beschleunigte sich. „Es nicht gerade ein guter Test, jemandes Blut könnte zu schwach für mächtige Beschwörungen sein, aber stark genug für schwache Zauber… Draco, weißst du, ob mächtige Zauberer einer einzigen Epoche, zum Beispiel nur Zauberer aus diesem Jahrhundert, auch als Kind mächtiger waren? Wenn der Dunkle Lord als Elfjähriger versucht hätte den Kühlzauber einzusetzen, hätte er dann den ganzen Raum einfrieren können?“

Dracos Gesicht warf sich in Falten, als er angestrengt versucht sich zu erinnern. „Ich kann mich nicht erinnern, jemals etwas über den Dunklen Lord gehört zu haben, aber ich glaube Dumbledore soll etwas Herausragendes in seinem Verwandlung Z.A.G. im fünften Jahr… Ich glaube andere mächtige Zauberer waren auch gut in Hogwarts…“

Harry verzog sein Gesicht, noch immer durch das Zimmer schreitend. „Vielleicht haben sie auch fleißig geübt und gelernt. Trotzdem, wenn Erstklässler dieselben Zauber lernten und gleich stark wirkten, könnten wir es ein schwaches Indiz für 1 im Vergleich zu 2 nennen… warte, warte.“ Harry hielt an, wo er stand. „Ich habe noch einen anderen Test, der vielleicht zwischen 1 und 2 unterscheidet. Es würde eine Weile dauern, es dir zu erklären, es verwendet Sachen, die die Wissenschaftler über Blut und Vererbung wissen. Und wenn wir deinen und meinen Test vereinen und beide das gleiche Ergebnis liefern, dann ist das schon ein gutes Indiz für die Antwort.“ Harry rannte fast zurück, nahm das Pergament und schrieb:

\emph{B. Konnten frühere Erstklässler die gleiche Art an Zaubern, mit derselben} \emph{Stärke wie heute} \emph{wirken} \emph{(Schwaches Indiz für 1 im Vergleich zu 2, aber Blut könnte dennoch nur mächtige Beschwörungen betreffen.)}\\ \emph{C. Zusätzlicher Test, der zwischen 1 und 2 unterscheidet unter Verwendung der Wissenschaft des Blutes, werde ich später erklären.}

„Okay,“ sagte Harry, „wir können wenigstens versuchen den Unterschied zwischen 1 und 2 und 3 zu finden, also lass uns damit beginnen, danach können wir uns im Notfall noch weitere Tests einfallen lassen. Weil es ein wenig seltsam aussehen würde, Draco Malfoy und Harry Potter zusammen Fragen stellen zu sehen, schlage ich folgendes vor. Du gehst durch Hogwarts und suchst alte Portraits und fragst sie nach den Zaubern, die sie während ihres ersten Schuljahres gelernt haben. Sie sind Portraits, demnach werden sie nichts überraschendes daran finden, dass Draco Malfoy soetwas macht. Ich werde neurere Portraits und lebende Menschen nach Zaubern fragen, die wir zwar kennen, aber nicht verwenden können; niemand wird etwas ungewöhnliches bemerken,\\ wenn Harry Potter merkwürdige Fragen stellt. Und ich muss noch komplexe Recherche über vergessene Zauber anstellen, also möchte ich, dass du die Daten für meine eigene wissenschaftliche Frage einholst. Es ist eine simple Frage und du solltest in der Lage sein, Antworten von Portraits zu bekommen. Möglicherweise solltest du es aufschreiben, bereit?

Draco setzte sich und wühlte in seiner Schultasche nach Pergament und Feder. Als beides auf dem Tisch lag, hob er den Kopf mit entschlossendem Gesicht. „Schieß los.“

„Suche nach Portraits, die verheiratete Squibpaare kannten ---- mach nicht so ein Gesicht, Draco, das ist eine wichtige Information. Frag nur aktuelle Portraits, die Gryffindors waren oder sowas. Suche nach Portraits, die ein verheiratetes Squibpaar gut genug kannten, sodass sie die Namen der Kinder kennen. Schreib den Namen von jedem Kind auf und ob es ein Zauberer, ein Squib oder ein Muggel war. Falls sie nicht wissen ob ein Kind ein Squib oder Muggel war, schreib „Nicht-Zauberer“. Schreib das für \emph{Jedes} Kind des Paares auf, lass keine aus. Wenn das Portrait nur die Namen der Zaubererkinder kennt, notiere überhaupt keine Daten von diesem Paar . Es sehr wichtig, dass du mir nur Daten von solchen bringst, die \emph{alle} Kinder eines Squibpaares mit Namen kannten. Probiere, wenn möglich mindestens vierzig Namen insgesamt zu sammeln, und wenn du noch Zeit für mehr übrig hast; noch besser. Hast du das alles?

„Wiederhole es nochmal,“ sagte Draco, nachdem er zu schreiben aufgehört hatte und Harry wiederholte es.

„Fertig,“ sagte Draco, „aber warum ----“

„Es hat einem der Geheimnisse des Blutes zu tun, die die Wissenschaftler schon entdeckt haben. Ich werde es dir erklären, wenn du zurück kommt. Wir sollten uns aufteilen und dann in einer Stunde wieder hier treffen, 18:22 wäre das dann. Sind wir bereit anzufangen?“

Draco nickte entschieden. Es was alles sehr gehetzt, doch er wurde schon lange darin ausgebildet sich zu beeilen.

„Dann los“ sagte Harry Potter und warf seinen Kapuzenumhang ab, stopfte ihn in seinen Beutel, der ihn zu essen begann und, ohne auch nur seinem Beutel die Zeit zum Aufessen zu geben, fuhr er herum und fing an sich mit schnellen Schritten zur Klassenzimmertür zu begeben, stieß aber in seiner Hast gegen einen Tisch und fiel dabei fast um.

Als Draco es geschafft hatte, seinen eigenen Umhang auszuziehen und in seiner Schultasche zu verstauen, war Harry Potter bereits weg.

Draco rannte beinahe aus dem Raum.\\

