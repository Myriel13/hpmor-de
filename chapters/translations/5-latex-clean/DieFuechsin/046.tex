

\hypertarget{humanismus-teil-4}{% \section{14. Humanismus Teil 4}\label{humanismus-teil-4}}

-\/-\/-\/-\/- Kapitel 46: Humanismus, Teil 4 -\/-\/-\/-\/-

Der letzte Zipfel der Sonne versank unter dem Horizont, das rote Licht verblasste von den Baumkronen, nur der blaue Himmel erhellte die sechs Menschen, die auf dem wintergetrockneten und schneebedeckten Gras standen, in der Nähe eines leeren Käfigs, auf dessen Boden ein leerer, zerfetzter Umhang lag.

Harry fühlte sich … nun, wieder \emph{normal}. Wie ein normaler Mensch. Der Zauber hatte den Tag und seinen Schaden nicht rückgängig gemacht, die Verletzungen nicht verschwinden lassen, als wären sie nie gewesen, aber seine Schmerzen waren … verbunden, gemildert? Es war schwer zu beschreiben.

Dumbledore sah auch gesünder aus, wenn auch nicht völlig wiederhergestellt. Der alte Zauberer drehte sich einen Moment um, tauscht einen Blick mit Professor Quirrell aus und sah dann zu Harry zurück. »Harry«, sagte Dumbledore, »brichst du bald vor Erschöpfung zusammen und stirbst womöglich?«

»Nein, komischerweise«, sagte Harry. »Das hat mich etwas mitgenommen, aber viel weniger, als ich dachte.« \emph{Oder vielleicht hat es mir auch etwas zurückgegeben, ebenso wie es mir etwas genommen hat.} »Ehrlich gesagt, erwartete ich, dass mein Körper jetzt mit einem Rumms auf den Boden aufschlägt.«

Es gab ein klar vernehmliches Geräusch der Sorte Körper-trifft-den-Boden-mit-einem-Rumms.

»Danke, dass du dich darum gekümmert hast, Quirinus«, sagte Dumbledore zu Professor Quirrell, der nun über und hinter den unbewussten Formen der drei Auroren stand. »Ich gestehe, ich fühle mich immer noch etwas kränklich. Aber ich werde mich selbst um die Erinnerungszauber kümmern.«

Professor Quirrell neigte den Kopf und schaute dann Harry an. »Ich werde eine Menge nutzloser Ungläubigkeit auslassen«, sagte Professor Quirrell, »Bemerkungen wie, dass Merlin selbst dies nicht geschafft hat, etc. Gehen wir direkt zur wichtigen Frage über. Was zur süßen, schlängelnden Schlange war \emph{das}?«

»Der Patronus-Zauber«, sagte Harry. »Version 2.0.«

»Ich freue mich, dass du wieder dein gewohntes Selbst bist«, sagte Dumbledore. »Aber du gehst \emph{nirgendwohin}, junger Ravenclaw, bevor du mir nicht sagst, was genau dieser warme, glückliche Gedanke war.«

»Hm …«, sagte Harry. Er klopfte sich mit einem besinnlichen Finger auf die Wange. »Ich frage mich, ob ich das tun sollte.«

Professor Quirrell grinste plötzlich.

»Bitte?«, sagte der Direktor. »Bitte, bitte, mit Zucker obendrauf?«

Harry fühlte einen Impuls und beschloss, sich darauf einzulassen. Es war gefährlich, aber es könnte bis zum Ende der Zeit keine bessere Gelegenheit geben.

»3 Limonaden«, sagte Harry zu seinem Beutel, dann sah er zum Verteidigungsprofessor und dem Direktor von Hogwarts auf. »Meine Herren«, sagte Harry, »diese Limonaden habe ich bei meinem ersten Besuch auf Plattform Neun und drei viertel gekauft, am Tag meiner Ankunft in Hogwarts. Ich habe sie für besondere Anlässe aufbewahrt; es gibt eine kleine Verzauberung auf ihnen, um sicherzustellen, dass sie zur richtigen Zeit getrunken werden. Dies sind meine letzten Vorräte, aber ich glaube nicht, dass es jemals eine schönere Gelegenheit geben wird. Wollen wir?«

Dumbledore nahm Harry eine Dose ab und Harry warf Professor Quirrell eine andere zu. Die beiden älteren Männer murmelten jeweils den gleichen Zauber über ihren Dose und runzelten kurz die Stirn über das Ergebnis. Harry seinerseits ploppte einfach den Deckel auf und trank.

Der Verteidigungsprofessor und der Schulleiter von Hogwarts folgten höflich dem Beispiel.

Harry sagte: »Ich dachte an meine absolute Ablehnung des Todes als die natürliche Ordnung.«

Es ist vielleicht nicht die richtige Art von Wärme, die man für einen Patronus-Zauber braucht, aber er kam trotzdem in Harrys Top 10.

Die Blicke des Verteidigungsprofessors und des Schulleiters machten Harry kurz nervös, als der verschüttete Poin-Tee verschwand; aber dann sahen die beiden sich gegenseitig an und beide entschieden anscheinend, dass sie nicht damit durchkommen konnten, Harry in der Gegenwart des anderen etwas wirklich Furchtbares anzutun.

»Mr~Potter«, sagte Professor Quirrell, »selbst \emph{ich} weiß, dass die Dinge nicht so laufen sollten.«

»In der Tat«, sagte Dumbledore. »Erklären Sie das.«

Harry öffnete den Mund und als ihm die Erkenntnis kam, schnappte er schnell wieder zu. Godric hatte es niemandem erzählt, auch nicht Rowena, wenn sie es gewusst hätte; es könnte eine ganze Reihe von Zauberern gegeben haben, die es herausgefunden und den Mund gehalten hatten. Du könntest nicht vergessen, wenn du \emph{wüsstest}, dass es das ist, was du versuchst; wenn du erst einmal erkannt hast, \emph{wie} es funktioniert, würde die Tierform des Patronuszaubers nie wieder für dich funktionieren -- und die meisten Zauberer hatten nicht die richtige Erziehung, um Dementoren anzugreifen und zu vernichten --

»Ähm, tut mir leid«, sagte Harry. »Aber ich habe gerade in diesem Moment erkannt, dass es eine \emph{ausgesprochen} schlechte Idee wäre, es zu erklären, bis Sie ein paar Dinge selbst herausgefunden haben.«

»Ist das die Wahrheit, Harry?«, fragte Dumbledore langsam. »Oder gibst du nur vor, weise zu sein?«

»Herr \emph{Direktor}!«, sagte Professor Quirrell und klang dabei wirklich schockiert. »Mr~Potter hat Ihnen gesagt, dass dieser Zauber nicht mit denen besprochen wird, die ihn nicht wirken können! In solchen Dingen drängt man keinen Zauberer!«

»Wenn ich Ihnen sagen würde --«, begann Harry.

»Nein«, sagte Professor Quirrell und klang ziemlich streng. »Sie sagen uns nicht \emph{warum}, Mr~Potter, Sie sagen uns nur, dass wir es nicht wissen sollen. Wenn Sie eine Andeutung machen wollen, tun Sie das vorsichtig, in aller Ruhe, nicht mitten im Gespräch.«

Harry nickte.

»Aber«, sagte der Schulleiter. »Aber«, sagte der Direktor. »Aber, was soll ich dem Ministerium sagen? Man kann nicht einfach einen Dementor \emph{verlieren}!«

»Sagen Sie Ihnen, ich hätte ihn gegessen«, sagte Professor Quirrell, und Harry verschluckte sich am Soda, das er unbewusst an seine Lippen gehoben hatte. »Das macht mir nichts aus. Sollen wir zurückgehen, Mr~Potter?«

Die beiden begannen, den Feldweg zurück nach Hogwarts zu gehen, wobei sie Albus Dumbledore zurückließen der verloren auf den leeren Käfig und die drei schlafenden Auroren blickte, die auf ihre Gedächtniszauber warteten.

͙⃰⁎

\emph{Nachspiel, Harry Potter und Professor Quirrell:}

Sie liefen eine Weile, bevor Professor Quirrell sprach, und alle Hintergrundgeräusche verstummten, als er sprach.

»Du bist außergewöhnlich gut darin, Dinge zu töten, mein Schüler«, sagte Professor Quirrell.

»Danke«, sagte Harry aufrichtig.

»Ich bin nicht neugierig«, sagte Professor Quirrell, »aber für den unwahrscheinlichen Fall, dass es \emph{nur} der Schulleiter war, dem du das Geheimnis nicht anvertraust …?«

Harry zog das in Betracht. Professor Quirrell konnte den Patronuszauber jetzt schon nicht wirken.

Aber ein Geheimnis konnte man nicht wieder ungesagt machen, und Harry lernte schnell genug, um zu erkennen, dass er zumindest eine Weile \emph{nachdenken} sollte, bevor er dieses Geheimnis auf die Welt losließ.

Harry schüttelte den Kopf, und Professor Quirrell nickte zustimmend.

»Nur aus Neugierde, Professor Quirrell«, sagte Harry, »wenn Sie den Dementor für ein böses Komplott nach Hogwarts gebracht hätten, was wäre dann das Ziel gewesen?«

»Dumbledore zu ermorden, während er geschwächt wäre«, sagte Professor Quirrell ohne zu zögern. »Hm. Der Schulleiter hat dir gesagt, dass er mir misstraut hat?«

Harry sagte eine Sekunde lang nichts, während er versuchte, sich eine Antwort auszudenken, und gab dann auf, als er merkte, dass er schon geantwortet hatte.

»Interessant …«, sagte Professor Quirrell. »Mr~Potter, es ist nicht ausgeschlossen, dass heute ein Komplott im Gange \emph{war}. Dass Ihr Zauberstab so nahe am Käfig des Dementors gelandet ist, \emph{könnte} ein Unfall gewesen sein. Oder einer der Auroren könnte unter dem Einfluss des Imperiusfluchs oder des Verwirrungszaubers oder eines Legilimentikers gestanden haben, um Einfluss auszuüben. Flitwick und ich sollten bei Ihren Überlegungen nicht als Verdächtige ausgeschlossen werden. Man bemerkt, dass Professor Snape heute alle seine Kurse abgesagt hat, und ich vermute, dass er mächtig genug ist, um sich selbst zu desillusionieren; die Auroren haben schon früh Erkennungszauber gewirkt, aber sie haben sie nicht unmittelbar bevor Sie dran waren wiederholt. Aber am einfachsten, Mr~Potter, könnte die Tat von Dumbledore selbst geplant worden sein; und wenn er es \emph{getan hat}, könnte er auch im Voraus Schritte unternehmen, um Ihren Verdacht auf andere zu lenken.«

Sie gingen ein paar Schritte weiter.

»Aber warum \emph{sollte} er das tun?«, sagte Harry.

Der Verteidigungsprofessor blieb einen Moment still und sagte dann: »Mr~Potter, welche Schritte haben Sie unternommen, um den Charakter des Schulleiters zu untersuchen?«

»Nicht viele«, sagte Harry. Ihm war erst kürzlich klar geworden … »Nicht annähernd genug.«

»Dann werde ich feststellen«, sagte Professor Quirrell, »dass man nicht alles über einen Mann erfährt, wenn man nur seine Freunde fragt.«

Nun war Harry an der Reihe, ein paar Schritte schweigend auf dem leicht ausgetretenen Feldweg zu gehen, der zurück nach Hogwarts führte. Er sollte es eigentlich schon besser wissen. Bestätigungsfehler* war der Fachbegriff; er bedeutete unter anderem, dass man bei der Auswahl der Informationsquellen eine bemerkenswerte Tendenz dazu hatte, Informationsquellen zu wählen, die mit der eigenen aktuellen Meinung übereinstimmten.

»Danke«, sagte Harry. »Eigentlich … ich habe das nicht schon früher gesagt, oder? Ich danke Ihnen für \emph{alles}. Sollte Sie je ein anderer Dementor bedrohen oder gar leicht verärgern, lassen Sie es mich wissen und ich stelle ihm Herrn Glühende Person vor. Ich mag es nicht, wenn Dementoren meine Freunde leicht verärgern.«

Das brachte ihm einen unleserlichen Blick von Professor Quirrell ein. »Sie haben den Dementor vernichtet, weil er mich bedroht hat?«

»Ähm«, sagte Harry, »Ich hatte mich schon vorher dafür entschieden, aber ja, das wäre schon ein Grund gewesen.«

»Ich verstehe«, sagte Professor Quirrell. »Und was hätten Sie gegen die Bedrohung getan, wenn Ihr Zauber zur Zerstörung des Dementors \emph{nicht} gewirkt hätte?«

»Plan B«, sagte Harry. »Den Dementor in dichtes Metall mit hohem Schmelzpunkt hüllen, wahrscheinlich Wolfram, in einen aktiven Vulkan fallen lassen und hoffen, dass er im Erdmantel landet. Ah, der ganze Planet ist unter seiner Oberfläche mit geschmolzener Lava gefüllt --«

»Ja«, sagte Professor Quirrell. »Ich weiß.« Der Verteidigungsprofessor trug ein sehr seltsames Lächeln. »Daran hätte ich wirklich selbst denken sollen, alles in allem. Sagen Sie mir, Mr~Potter, wenn Sie etwas verlieren wollten, wo niemand es je wiederfinden würde, wo würden Sie es hinlegen?«

Harry bedachte diese Frage. »Ich sollte wohl nicht fragen, \emph{was} Sie gefunden haben, das man verlieren muss --«

»Durchaus«, sagte Professor Quirrell, wie Harry erwartet hatte; und dann: »Vielleicht erfahren Sie es, wenn Sie älter sind«, was Harry nicht erwartet hatte.

»Nun«, sagte Harry, »außer zu versuchen, es in den geschmolzenen Kern des Planeten zu bekommen, könnte man es einen Kilometer unter der Erde an einem zufällig ausgewählten Ort in festem Gestein vergraben -- vielleicht könnte man es hineinteleportieren, wenn es eine Möglichkeit gibt, das blind zu tun, oder ein Loch bohren und das Loch danach reparieren; das Wichtigste wäre, keine Spuren zu hinterlassen, die dorthin führen, damit es nur ein anonymer Kubikmeter irgendwo in der Erdkruste ist. Man könnte es in den Marianengraben fallen lassen, das ist die tiefste Meerestiefe des Planeten -- oder einfach einen anderen Ozeangraben auswählen, um es weniger offensichtlich zu machen. Wenn man ihm genug Auftrieb geben könnte und unsichtbar, dann könnte man es in die Stratosphäre werfen. Oder im Idealfall würde man ihn ins All schießen, mit einer Tarnung gegen Entdeckung und einem zufällig schwankenden Beschleunigungsfaktor, der ihn aus dem Sonnensystem herausbringt. Und danach würden Sie Ihre Erinnerungen mit Obliviate verändern, so dass selbst Sie nicht genau wissen, wo es ist.«

Der Verteidigungsprofessor lachte, und es klang noch seltsamer als sein Lächeln.

»Professor Quirrell?«, sagte Harry.

»Alles ausgezeichnete Vorschläge«, sagte Professor Quirrell. »Aber sagen Sie mir, Mr~Potter, warum genau diese fünf?«

»Hm?«, sagte Harry. »Sie schienen mir nur die offensichtlichen Ideen zu sein.«

»Oh?«, sagte Professor Quirrell. »Aber dabei gibt ein interessantes Muster, wissen Sie. Man könnte sagen, es klingt wie eine Art Rätsel. Ich muss zugeben, Mr~Potter, obwohl er im Großen und Ganzen seine Höhen und Tiefen hatte, war dies ein überraschend guter Tag.«

Und sie gingen weiter den Weg entlang, der zu den Toren von Hogwarts führte, ziemlich weit voneinander entfernt; denn Harry blieb, ohne darüber nachzudenken, automatisch weit genug vom Verteidigungsprofessor entfernt, um nicht dieses Gefühl des Untergangs auszulösen, das aus irgendeinem Grund gerade jetzt ungewöhnlich stark schien.

͙⃰⁎

\emph{Nachspiel, Daphne Greengrass:}

Hermine hatte sich geweigert, irgendwelche Fragen zu beantworten, und sobald sie die Abzweigung, die zu den Verliesen von Slytherin führte, passiert hatten, waren Daphne und Tracey sofort abgehauen, so schnell sie konnten. Gerüchte verbreiteten sich schnell in Hogwarts, so daß sie sofort in die Verliese gehen mußten, wenn sie die Geschichte allen als erste erzählen wollten.

»Denk dran«, sagte Daphne, »plappere nicht einfach was über den Kuss heraus, sobald wir hineingehen, okay? Es funktioniert besser, wenn wir die ganze Geschichte der Reihe nach erzählen.«

Tracey nickte aufgeregt.

Und sobald sie in den Slytherin-Gemeinschaftsraum platzten, atmete Tracey Davis tief durch und rief: »\emph{Alle herhören! Harry Potter konnte den Patronuszauber nicht aussprechen und der Dementor hätte ihn fast gefressen und Professor Quirrell hat ihn gerettet, aber dann war Potter ganz böse, bis Granger ihn mit einem Kuss zurückgebracht hat! Das ist wahre Liebe, ganz sicher!«}

Es war eine Art von Geschichtenerzählen, vermutete Daphne .

Die Nachricht hatte nicht die erwartete Reaktion hervorgerufen. Die meisten Mädchen sahen hinüber und blieben dann in ihren Sofas, oder die Jungen lasen einfach weiter in ihren Stühlen.

»Ja«, sagte Pansy säuerlich, von dort, wo sie mit Gregorys Füßen im Schoß saß, sich zurücklehnte und etwas las, das wie ein Malbuch aussah, »hat Millicent uns schon erzählt.«

\emph{Wie --}

»Warum hast \emph{du} ihn nicht zuerst geküsst, Tracey?«, sagten Flora und Hestia Carrow von ihren eigenen Stühlen aus. »Jetzt wird Potter ein Schlammblut-Mädchen heiraten! \emph{Du} hättest seine wahre Liebe sein können und in ein reiches Adelshaus kommen können, wenn du ihn vorher geküsst hättest!«

Traceys Gesicht war ein Bild von verblüffter Erkenntnis.

»\emph{Was?«}, kreischte Daphne. »So funktioniert Liebe nicht!«

»Natürlich tut sie das«, sagte Millicent, von wo aus sie eine Art von Zauber übte, während sie aus dem Fenster auf das wirbelnde Wasser des Hogwarts-Sees schaute. »Der erste Kuss bekommt den Prinzen.«

»\emph{Es war nicht ihr erster Kuss!«,} rief Daphne. »Hermine war \emph{schon} seine wahre Liebe! Deshalb konnte \emph{sie} ihn zurückbringen!« Dann merkte Daphne, was sie gerade gesagt hatte, und zuckte innerlich zusammen, aber wie es so schön heißt, musste man die Zunge an das Ohr anpassen.

»Whoa, whoa, whoa, was?«, sagte Gregory, und schwang die Füße von Pansys Schoß. »Was war das? Fräulein Bulstrode hat diesen Teil nicht erzählt.«

Alle anderen sahen jetzt auch Daphne an.

»Oh, ja«, sagte Daphne, »Harry schubste sie weg und rief: 'Ich sagte doch, kein Küssen! Dann schrie Harry, als würde er sterben, und Fawkes fing an, ihm etwas vorzusingen -- ich weiß nicht, was davon zuerst passiert ist --«

»Das klingt für \emph{mich} nicht nach wahrer Liebe«, sagten die Carrow-Zwillinge. »Das klingt, als hätte die \emph{falsche Person} ihn geküsst.«

»\emph{Ich} hätte es sein sollen«, flüsterte Tracey. Ihr Gesicht war immer noch fassungslos. »\emph{Ich} sollte seine wahre Liebe sein. Harry Potter war \emph{mein} General. Ich hätte, ich hätte mit Granger um ihn kämpfen sollen --«

Daphne drehte sich erbost zu Tracey um. »\emph{Du?} Harry von Hermine stehlen?«

»Ja!«, sagte Tracey. »Ich!«

»Du bist verrückt«, sagte Daphne mit Überzeugung. »Selbst wenn du ihn zuerst geküsst \emph{hättest}, weißt du, was das aus dir machen würde? Das traurige kleine Liebeskranke Mädchen, das am Ende des zweiten Aktes stirbt.«

»\emph{Das nimmst du zurück!«,} schrie Tracey.

Währenddessen hatte Gregory den Raum durchquert, dorthin wo Vincent seine Hausaufgaben gemacht hatte. »Mr~Crabbe«, sagte Gregory mit leiser Stimme, »ich glaube, Mr~Malfoy muss das wissen.«

͙⃰⁎

\emph{Nachspiel, Hermine Granger:}

Hermine starrte auf das wachsversiegelte Papier, auf dessen Oberfläche lediglich die Zahl 42 stand.

\emph{Ich habe herausgefunden, warum wir den Patronuszauber nicht wirken konnten, Hermine, es hat nichts damit zu tun, dass wir nicht glücklich genug waren. Aber ich kann's dir nicht sagen. Ich konnte es nicht mal dem Schulleiter sagen. Es muss noch geheimer sein als die partielle Verwandlung, jedenfalls vorerst. Aber wenn du jemals gegen Dementoren kämpfen musst, das Geheimnis steht hier, verschlüsselt, so dass, wenn jemand nicht weiß, dass es um Dementoren und den Patronuszauber geht, sie nicht wissen werden was es bedeutet} …

Sie hatte Harry erzählt, dass sie ihn sterben sah, dass ihre Eltern und Freunde starben, alle starben. Sie hatte ihm nicht von ihrer Todesangst davor allein zu sterben erzählt, das war irgendwie noch zu schmerzhaft.

Harry hatte ihr erzählt, dass er sich an den Tod seiner Eltern erinnerte und dass er es lustig fand.

\emph{Es gibt kein Licht an dem Ort, wo der Dementor dich hinbringt, Hermine. Keine Wärme. Keine Fürsorge. Es ist ein Ort, an dem du nicht einmal das Glück verstehen kannst. Es gibt Schmerz und Angst, und die können dich immer noch antreiben. Du kannst hassen und dich daran erfreuen, zu zerstören, was du hasst. Du kannst lachen, wenn du siehst, wie andere Menschen leiden. Aber du kannst niemals glücklich sein, du kannst dich nicht einmal daran erinnern, was es ist, das nicht mehr da ist … Ich glaube nicht, dass ich jemals erklären kann, wovor du mich gerettet hast. Normalerweise schäme ich mich, Leute in Schwierigkeiten zu bringen, ich kann es normalerweise nicht ertragen, wenn Menschen für mich Opfer bringen, aber dieses eine Mal werde ich sagen, dass, egal was es dich am Ende kostet, mich geküsst zu haben, zweifle nie auch nur eine Sekunde daran, dass es das Richtige war.}

Hermine hatte nicht realisiert, wie \emph{wenig} der Dementor sie berührt hatte, wie klein und oberflächlich die Dunkelheit war, in die er sie mitgenommen hatte.

Sie hatte gesehen, wie alle gestorben waren, und das hatte ihr noch wehtun können.

Hermine steckte das Papier wieder in ihren Beutel, wie es ein gutes Mädchen tun sollte.

Aber sie wollte sie unbedingt lesen.

Sie hatte Angst vor den Dementoren.

͙⃰⁎

\emph{Nachspiel, Minerva McGonagall:}

Sie fühlte sich wie erstarrt, sie hätte nicht so schockiert sein sollen, es hätte nicht so hart sein sollen Harry gegenüberzutreten, aber nach dem, was er durchgemacht hatte … Sie hatte den kleinen Jungen vor ihr nach Anzeichen eines Dementorangriffs untersucht und sie nicht gefunden. Aber etwas an der Ruhe, mit der er eine so vorahnungsvolle Frage gestellt hatte, schien zutiefst besorgniserregend. »Mr~Potter, ich kann unmöglich ohne die Erlaubnis des Schulleiters über solche Dinge sprechen!«

Der Junge in ihrem Büro nahm das auf, ohne seinen Ausdruck zu ändern. »Ich würde es vorziehen den Schulleiter mit dieser Angelegenheit nicht zu stören«, sagte Harry Potter ruhig. »Ich \emph{bestehe} darauf, ihn nicht zu stören, und Sie haben versprochen, dass unser Gespräch vertraulich bleibt. Also lassen Sie es mich so sagen. Ich weiß, dass es in der Tat eine Prophezeiung gab. Ich weiß, dass Sie diejenige sind, die sie ursprünglich von Professor Trelawney gehört hat. Ich weiß, dass die Prophezeiung das Kind von James und Lily als gefährlich für den Dunklen Lord identifiziert hat. Und ich weiß, wer ich bin, in der Tat weiß jetzt jeder, wer ich bin, also enthüllen Sie nichts Neues oder Gefährliches, wenn Sie mir nur dies sagen: Wie war der \emph{genaue Wortlaut}, der \emph{mich}, das Kind von James und Lily, identifizierte?«

Trelawneys hohle Stimme hallte in ihrem Kopf wider --

GEBOREN VON DENEN, DIE IHM DREIMAL GETROTZT HABEN,

GEBOREN ALS DER SIEBTE MONAT STIRBT …

»Harry«, sagte Professor McGonagall, »das kann ich Ihnen unmöglich sagen!« Es fror sie bis auf die Knochen bei dem Gedanken, dass Harry schon so viel wusste, dass sie sich nicht vorstellen konnte, wie Harry gelernt hat --

Der Junge sah sie mit seltsamen, traurigen Augen an. »Können Sie nicht ohne Erlaubnis des Schulleiters niesen, Professor McGonagall? Denn ich verspreche Ihnen, dass ich guten Grund habe, zu fragen, und guten Grund, die Frage geheim zu halten.«

»Bitte nicht, Harry«, flüsterte sie.

»In Ordnung«, sagte Harry. »Eine einfache Frage. Bitte. Wurde die Familie Potter \emph{namentlich} erwähnt? Sagt die Prophezeiung wörtlich 'Potter'?«

Sie starrte Harry eine Weile an. Sie hätte nicht sagen können, warum oder woher sie das Gefühl hatte, dass dies ein kritischer Punkt war, dass sie die Bitte nicht leichtfertig ablehnen oder ihr leichtfertig zustimmen konnte --

»Nein«, sagte sie schließlich. »Bitte, Harry, frag nicht weiter.«

Der Junge lächelte, ein wenig traurig wie es schien, und sagte: »Danke, Minerva. Sie sind eine gute Frau und wahrhaftig.«

Und während ihr Mund noch vom Schock offen stand, erhob sich Harry Potter und verließ das Büro; und erst dann merkte sie, dass Harry ihre Weigerung als Antwort aufgefasst hatte, und die wahre Antwort noch dazu --

Harry schloss die Tür hinter sich.

Die Logik hatte sich mit einer seltsamen diamantartigen Klarheit präsentiert. Harry hätte nicht sagen können, ob es ihm während Fawkes' Gesang eingefallen war, oder vielleicht sogar schon davor.

Lord Voldemort hatte James Potter getötet. Er hatte es vorgezogen, Lily Potters Leben zu verschonen. Folglich hatte er seinen Angriff mit dem einzigen Ziel fortgesetzt, ihr Kleinkind zu töten.

Dunkle Lords hatten normalerweise keine Angst vor Kleinkindern.

Es gab also eine Prophezeiung, dass Harry Potter für Lord Voldemort gefährlich sein würde, und Lord Voldemort hatte diese Prophezeiung gekannt.

\emph{»Ich gebe dir die seltene Gelegenheit zu fliehen. Aber ich werde mich nicht bemühen, dich zu unterwerfen, und dein Tod hier wird dein Kind nicht retten. Tritt zur Seite, törichtes Weib, wenn du überhaupt Verstand in dir hast!«}

War es eine Laune, ihr diese Chance zu geben? Aber dann hätte Lord Voldemort nicht versucht, sie zu überreden. Hatte die Prophezeiung Lord Voldemort davor gewarnt, Lily Potter zu töten? Dann \emph{hätte} sich Lord Voldemort die Mühe gemacht, sie zu unterwerfen. Lord Voldemort war \emph{leicht} geneigt gewesen, Lily Potter nicht zu töten. Diese Präferenz war stärker als eine Laune, aber nicht so stark wie eine Warnung.

Nehmen wir also an, dass jemand, den Lord Voldemort als nützlichen, aber nicht unentbehrlichen Verbündeten oder Diener ansah, den Dunklen Lord angefleht hatte, Lilys Leben zu verschonen. Lilys, aber nicht das von James.

Diese Person hatte gewusst, dass Lord Voldemort das Haus der Potters angreifen würde. Er hatte sowohl die Prophezeiung als auch die Tatsache gekannt, dass der Dunkle Lord sie kannte. Sonst hätte er nicht um Lilys Leben gebettelt.

Laut Professor McGonagall waren neben ihr die beiden anderen, die von der Prophezeiung wussten, Albus Dumbledore und Severus Snape.

Severus Snape, der Lily geliebt hatte, bevor sie Lily Potter war, und der James gehasst hatte.

Severus hatte also von der Prophezeiung erfahren und sie dem Dunklen Lord erzählt. Was er getan hatte, weil die Prophezeiung die Potters nicht namentlich genannt hatte. Es war ein Rätsel gewesen und Severus hatte das Rätsel erst zu spät gelöst.

Aber wenn Severus als \emph{erster} die Prophezeiung gehört und sie dem Dunklen Lord erzählt hätte, warum sollte er dann auch Dumbledore oder Professor McGonagall davon erzählt haben?

Also hatten Dumbledore oder Professor McGonagall sie zuerst gehört.

Der Schulleiter von Hogwarts hatte keinen offensichtlichen Grund, der Professorin für Verwandlung von einer äußerst heiklen und entscheidenden Prophezeiung zu erzählen. Aber die Professorin für Verwandlung hatte allen Grund, es dem Direktor zu erzählen.

Es schien also wahrscheinlich, dass Professor McGonagall sie als Erste gehört hatte.

Die vorherigen Wahrscheinlichkeiten sagten, dass es Professor Trelawney war, die ortsansässige Seherin von Hogwarts. Seher waren selten, wenn man also die Sekunden, die Professor McGonagall im Laufe ihres Lebens in der Gegenwart eines Sehers verbracht hatte, grob zusammenzählte, dann waren die meisten dieser Seher-Sekunden Trelawney-Sekunden.

Professor McGonagall hatte es Dumbledore gesagt und hätte ohne Erlaubnis niemandem sonst von der Prophezeiung erzählt.

Daher war es Albus Dumbledore, der dafür gesorgt hatte, dass Severus Snape irgendwie von der Prophezeiung erfuhr. Und Dumbledore selbst hatte das Rätsel erfolgreich gelöst, sonst hätte er nicht \emph{Severus}, der einst Lily geliebt hatte, als Vermittler ausgewählt.

Dumbledore hatte absichtlich dafür gesorgt, dass Lord Voldemort von der Prophezeiung erfuhr, in der Hoffnung, ihn in den Tod zu locken. Vielleicht hatte Dumbledore dafür gesorgt, dass Severus nur \emph{einen Teil} der Prophezeiung erfuhr, oder es gab andere Prophezeiungen, von denen Severus nichts wusste … irgendwie hatte Dumbledore gewusst, dass ein \emph{sofortiger} Angriff auf die Potters immer noch zu Lord Voldemorts \emph{sofortiger} Niederlage führen würde, obwohl Lord Voldemort selbst dies nicht geglaubt hatte. Oder vielleicht war das nur ein Glücksfall von Dumbledores Wahnsinn, seiner Vorliebe für bizarre Komplotte.

Severus hatte danach Dumbledore gedient; vielleicht würden die Todesser Severus nicht freundlich gesinnt sein, wenn Dumbledore seine Rolle bei ihrer Niederlage enthüllen würde.

Dumbledore hatte versucht, Harrys Mutter zu retten. Aber dieser Teil seines Plans war gescheitert. Und er hatte James Potter wissentlich zum Tode verurteilt.

Dumbledore war für den Tod von Harrys Eltern verantwortlich. \emph{Wenn} die ganze logische Kette richtig war. Harry konnte nicht zu Recht sagen, dass die erfolgreiche Beendigung des Zaubererkrieges nicht als mildernde Umstände gelten würde. Aber irgendwie … \emph{störte ihn das trotzdem sehr}.

Und es war an der Zeit, Draco Malfoy zu fragen, was die \emph{andere} Seite dieses Krieges über die Figur des Albus Percival Wulfric Brian Dumbledore zu sagen hatte.

*Bestätigungsfehler -- engl.: Confirmation bias.

Einer der Gründe für die Bildung sogenannter Filterblasen in sozialen Netzwerken.

