

\hypertarget{personenstatus-theorie}{% \section{15. Personenstatus-Theorie}\label{personenstatus-theorie}}

-\/-\/-\/-\/- Kapitel 47: Personenstatus-Theorie -\/-\/-\/-\/-

Es gibt in jeder Intrige einen Punkt, an dem das Opfer anfängt, Verdacht zu schöpfen; und wenn es zurückblickt, sieht es eine Spur von Ereignissen, die alle in eine einzige Richtung weisen. Und wenn dieser Punkt kommt, so hatte Vater erklärt, kann die Aussicht auf den Verlust so unerträglich und zuzugeben, das man selbst ausgetrickst wurde so demütigend erscheinen, dass das Opfer die Verschwörung dennoch leugnen wird und das Spiel noch lange danach weitergehen kann.

Vater hatte Draco gewarnt, das nicht noch einmal zu tun.

Doch zuerst ließ er Mr. Avery all die Kekse aufessen, die er Draco abgeluchst hatte, während Draco zusah und weinte. Die ganze schöne Dose Kekse, die Vater ihm nur ein paar Stunden zuvor gegeben hatte, denn Draco hatte sie alle bis auf den letzten an Mr. Avery verloren.

Es war also ein vertrautes Gefühl, das Draco in der Magengrube gefühlt hatte, als Gregory ihm von dem Kuss erzählte.

Manchmal schaute man zurück und sah Dinge…

(In einem lichtlosen Klassenzimmer - man konnte es nicht mehr wirklich \emph{unbenutzt} nennen, da es in den letzten Monaten wöchentlich benutzt wurde - saß ein Junge in eine Kutte mit Kapuze gehüllt, mit einer unbeleuchteten Kristallkugel auf dem Schreibtisch vor sich. Er dachte in der Stille nach, dachte in der Dunkelheit und wartete auf eine sich öffnende Tür, um das Licht hereinzulassen).

Harry hatte Granger weggestoßen und gesagt: \emph{Ich habe dir gesagt, kein Küssen!}

Harry würde wahrscheinlich so etwas sagen wie: \emph{Sie hat es beim letzten Mal nur getan, um mich zu ärgern, so wie sie mich zu dieser Verabredung gezwungen hat.}

Aber die bestätigte Geschichte war, dass Granger bereit gewesen war, dem Dementor erneut gegenüberzutreten, um Harry zu helfen; dass sie Harry weinend geküsst hatte, als er in den Tiefen der Dementation verloren war; und dass ihr Kuss ihn zurückgebracht hatte.

Das klang nicht nach Rivalität, nicht einmal nach freundschaftlicher Rivalität.

Das klang nach der Art von Freundschaft, die man normalerweise nicht einmal in Theaterstücken sieht.

Aber warum hatte Harry dann seine Freundin dazu gebracht, die eisigen Wände von Hogwarts zu erklimmen?

Weil Harry Potter so etwas mit seinen Freunden machte?

Vater hatte Draco gesagt, dass, um eine seltsame Handlung zu ergründen, eine Technik darin bestand, sich anzusehen was \emph{letztendlich} passiert ist, anzunehmen dass es das \emph{beabsichtigte} Ergebnis war und zu fragen, wer davon profitiert hat.

Was am Ende als Ergebnis des gemeinsamen Kampfes von Draco und Granger gegen Harry Potter herausgekommen war… war, dass Draco begonnen hatte sich viel freundlicher gegenüber Granger zu fühlen.

Wer profitierte davon, dass sich der Spross von Malfoy mit einer Schlammbluthexe anfreundete?

Wer profitierte davon, der für genau diese Art von Komplott berühmt war?

Wer profitierte davon, der möglicherweise Harry Potters Fäden in der Hand hatte?

Dumbledore.

Und wenn das wahr war, dann \emph{musste} Draco zu Vater gehen und ihm alles erzählen, egal, was danach passierte, Draco konnte sich nicht vorstellen, was danach passieren würde, es war schrecklich jenseits aller Vorstellungskraft. Deshalb wollte er sich verzweifelt an den letzten Rest Hoffnung klammern, dass nicht alles so war, wie es aussah…

… Draco erinnerte sich auch daran aus Mr. Averys Unterricht.

Draco hatte noch nicht geplant, Harry damit zu konfrontieren. Er versuchte noch sich einen experimentellen Test zu überlegen, etwas, das Harry nicht durchschauen und vortäuschen würde. Aber dann kam Vincent mit der Nachricht, dass Harry sich diese Woche eher treffen wollte, am Freitag statt am Samstag.

Und so stand Draco hier, in einem dunklen Klassenzimmer, mit einer unbeleuchteten Kristallkugel auf seinem Schreibtisch und wartete.

Minuten vergingen.

Schritte näherten sich.

Die Tür knarrte leicht, als sie in das Klassenzimmer aufschwang und Harry Potter mit seiner eigenen Kapuze und Kutte enthüllte; Harry trat nach vorne in das dunkle Klassenzimmer, und die stabile Tür schloss sich hinter ihm mit einem leisen Klicken.

Draco klopfte an die Kristallkugel, und das Klassenzimmer erhellte sich mit hellgrünem Licht. Grünes Licht projizierte die Schatten der Tische auf den Boden und strahlte von den geschwungenen Stuhllehnen zurück, wobei die Photonen so vom Holz abprallten, dass der Einfallswinkel dem Reflexionswinkel entsprach.

Zumindest \emph{so} viel von dem, was er gelernt hatte, war wahrscheinlich keine Lüge.

Harry war zusammengezuckt, als das Licht aufleuchtete, hielt einen Moment inne und setzte dann seinen Weg fort. „Hallo, Draco“, sagte Harry leise und zog seine Kapuze zurück, als er zu Dracos Schreibtisch kam. „Danke, dass du gekommen bist. Ich weiß, es ist nicht unsere übliche Zeit…"

„Gern geschehen“, sagte Draco rundheraus.

Harry zog einen der Stühle heran, um Draco am Schreibtisch gegenüberzusitzen, die Beine machten ein leicht kreischendes Geräusch auf dem Boden. Er drehte den Stuhl so, dass er in die falsche Richtung zeigte, und setzte sich breitbeinig darauf, die Arme über der Stuhllehne gefaltet. Das Gesicht des Jungen war nachdenklich, finster, ernst und sah selbst für Harry Potter sehr erwachsen aus.

„Ich muss dir eine wichtige Frage stellen“, sagte Harry, „aber vorher müssen wir noch etwas anderes tun."

Draco sagte nichts, er fühlte eine gewisse Müdigkeit. Ein Teil von ihm wollte einfach nur, dass es schon vorbei war.

„Sag mir, Draco“, sagte Harry. „Wieso lassen Muggel keine Geister zurück, wenn sie sterben?"

„Weil Muggel offensichtlich keine Seelen haben“, sagte Draco. Erst nachdem er es gesagt hatte, merkte er, dass es Harrys Politik widersprechen könnte, aber es war ihm egal. Außerdem \emph{war} es offensichtlich.

Harrys Gesicht zeigte keine Überraschung. „Bevor ich meine wichtige Frage stelle, will ich sehen, ob du den Patronus-Zauber lernen kannst.„

Einen Moment lang war Draco durch diesen Gedankensprung überfordert. Der gute alte, unmöglich-vorherzusehende-oder-zu-verstehende Harry Potter. Es gab Zeiten, in denen Draco sich fragte, ob Harry diese Desorientierung absichtlich als Taktik nutzte.

Dann verstand Draco es und schob sich mit einer einzigen wütenden Bewegung von seinem Schreibtisch weg. Das war's. Es war vorbei. „Wie \emph{Dumbledores} Diener“, spuckte er.

„Wie Salazar Slytherin“, sagte Harry ruhig.

Draco wäre beim ersten Schritt zur Tür fast über seine eigenen Füße gestolpert.

Langsam drehte sich Draco wieder zu Harry um.

„Ich weiß nicht, wie du darauf gekommen bist“, sagte Draco, „aber es ist falsch. Jeder weiß, dass der Patronus-Zauber ein Gryffindor-Zauber ist…"

„Salazar Slytherin konnte einen körperlichen Patronuszauber wirken“, sagte Harry. Harrys Hand huschte in seine Robe und holte ein Buch hervor, dessen Titel weiß auf grün geschrieben war und so fast unmöglich in dem grünen Licht zu lesen war; aber es sah alt aus. „Ich entdeckte das, als ich früher den Patronuszauber recherchierte. Und ich fand die Originalreferenz und habe das Buch aus der Bibliothek ausgeliehen, nur für den Fall, dass du mir nicht glaubst. Der Autor dieses Buches glaubt auch nicht, daß etwas \emph{Ungewöhnliches} daran ist, daß Salazar einen Patronus auslösen konnte; der Glaube, daß Slytherins das nicht können, muss neueren Datums sein. Und als weitere historische Anmerkung: Obwohl ich das Buch nicht bei mir habe, Godric Gryffindor konnte es nie.„

Nachdem Draco die ersten sechs Mal versucht hatte, Harry beim Bluffen zu überführen, bei sechs immer lächerlicher werdenden Gelegenheiten, hatte er erkannt, dass Harry einfach \emph{nicht} über etwas log, das in Büchern geschrieben stand. Trotzdem, als Harry das Buch aufschlug und es an der Stelle eines Lesezeichens öffnete, lehnte sich Draco hinüber und studierte die Stelle, auf die Harry mit dem Finger zeigte.

\emph{Dann fielen die Feuer von Ravenclaw auf die Dunkelheit, die den linken Flügel von Fürst Fauls Armee verhüllt hatte, und zerbrachen sie und es wurde enthüllt, dass Lord Gryffindor die Wahrheit gesagt hatte; die Angst, die sie alle gefühlt hatten, war nicht natürlich in ihrer Quelle, sondern kam von drei Dutzend Dementoren, denen die Seelen der Besiegten versprochen worden waren. Sofort brachten Lady Hufflepuff und Lord Slytherin ihre Schutzpatrone hervor, einen riesigen zornigen Dachs und eine helle silberne Schlange, und die Verteidiger hoben ihre Köpfe, als der Schatten aus ihren Herzen wich. Und Lady Ravenclaw lachte und bemerkte, dass Fürst Faul ein großer Narr sei, denn nun würde sein eigenes Heer der Furcht ausgesetzt sein, aber nicht die Verteidiger von Hogwarts. Und doch sagte Lord Slytherin: „Kein Narr ist er, so viel weiß ich.“ Und der Lord Gryffindor neben ihm studierte das Schlachtfeld mit einem Stirnrunzeln auf dem Gesicht…}

Draco schaute wieder auf. „Und?"

Harry schloss das Buch und steckte es in seinen Beutel. „Chaos und Sonnenschein haben beide Soldaten, die körperliche Patronuszauber wirken können. Körperliche Patronusse können dazu benutzt werden, um Botschaften zu übermitteln. Wenn du den Zauber nicht lernen kannst, ist die Drachenarmee militärisch im Nachteil -"

Das war Draco im Moment egal und er sagte es Harry. Seine Stimme war schärfer, als sie wohl hätte sein sollen.

Harry blinzelte nicht. „Dann fordere ich den Gefallen ein, den du mir schuldest, als ich einen Aufstand an unserem ersten Tag in der Besenflugstunde verhinderte. Ich werde versuchen, dir den Patronuszauber beizubringen und für meinem Gefallen sollst du dein wahrhaft Bestes tun, um ihn zu lernen und zu wirken. Ich vertraue auf die Ehre des Hauses Malfoy, dass du das tun wirst."

Draco verspürte wieder diese gewisse Müdigkeit. Hätte Harry zu einem anderen Zeitpunkt gefragt, wäre es eine faire Revanche für den geschuldeten Gefallen gewesen, da es eigentlich kein Gryffindor-Zauber war. Aber…

„\emph{Wieso?}“, sagte Draco.

„Um herauszufinden, ob du das kannst, was Salazar Slytherin konnte“, sagte Harry. „Dies ist ein experimenteller Test und ich sage dir erst, was er bedeutet, wenn du ihn durchgeführt hast. Wirst du das tun?"

… Es \emph{war} wohl eine gute Idee, sich für etwas Unschädliches von dieser Gefälligkeit zu entbinden, zumal wenn es Zeit war, mit Harry Potter zu brechen. „Na gut.„

Harry zog einen Zauberstab aus seinem Gewand und legte ihn gegen die Kugel. „Nicht gerade die beste Farbe, um den Patronuszauber zu lernen“, sagte Harry. „Grünes Licht ist genau der Farbton des tödlichen Fluchs, meine ich. Aber Silber ist doch auch eine Slytherin-Farbe oder nicht? \emph{Dulak}.“ Das Licht ging aus, und Harry flüsterte die ersten beiden Sätze der Dauerlichtverzauberung, wobei er diesen Teil umformulierte, obwohl keiner von ihnen das Ganze allein hätte zaubern können. Dann klopfte Harry wieder an die Kugel, und der Raum erhellte sich mit einem silbernen Glanz, brillant und doch weich und sanft. Die Farbe kehrte auf die Schreibtische und Stühle zurück und auf Harrys leicht verschwitztes Gesicht unter seinem schwarzen Haarschopf.

So lange dauerte es, bis Draco die Andeutung erkannte. „Du hast seit unserem letzten Treffen einen \emph{tödlichen Fluch} gesehen? Wann - wie -"

„Sprich den Patronuszauber“, sagte Harry, der ernster aussah als je zuvor, „und ich sage es dir."

Draco drückte seine Hände an seine Augen und schloss das silberne Licht aus. „Weißt du, ich sollte mir wirklich merken, dass du zu \emph{schräg} für \emph{normale} Pläne bist!"

In seiner selbst auferlegten Dunkelheit hörte er das Kichern von Harry.

Harry beobachtete genau, wie Draco seinen letzten Durchlauf der vorbereitenden Gesten beendete, den Teil des Zaubers, der schwer zu lernen war; der endgültige Schwung und die Aussprache mussten nicht präzise sein. Alle drei letzten Durchläufe waren perfekt gewesen, soweit Harry sehen konnte. Harry hatte auch einen merkwürdigen Impuls verspürt, Dinge zu korrigieren, über die Mr. Lupin nichts gesagt hatte, wie den Winkel von Dracos Ellbogen oder die Richtung, in die sein Fuß zeigte; es hätte seine eigene Fantasie sein können und war es wahrscheinlich auch, aber Harry hatte sich entschieden, für alle Fälle mitzumachen.

„In Ordnung“, sagte Harry leise. Es gab eine Spannung in seiner Brust, die das Sprechen etwas schwieriger machte. „Wir haben hier keinen Dementor, aber das ist schon in Ordnung. Wir werden keinen brauchen. Draco, als dein Vater am Bahnhof zu mir sprach, sagte er, dass du das Einzige auf der Welt wärst, das ihm etwas bedeutet. Er drohte, alle seine anderen Pläne aufzugeben, um sich an mir zu rächen, falls du jemals zu Schaden kommen solltest."

„Er… was?“ Dracos Stimme hakte und ein seltsamen Ausdruck erschien auf seinem Gesicht. „Warum erzählst du mir \emph{das}? „

„Warum sollte ich nicht?“ Harry ließ seinen Gesichtsausdruck sich nicht ändern, obwohl er ahnte, was Draco dachte; dass Harry geplant hatte, Draco von seinem Vater zu trennen, und dass er nichts sagen sollte, was sie einander näher bringen würde. „Es gab immer nur eine Person, die dir am meisten bedeutet hat, und ich weiß genau, welcher warme und glückliche Gedanke dich den Patronuszauber wirken lässt. Du hast es mir am Bahnhof vor dem ersten Schultag gesagt. Einmal bist du von einem Besenstiel gefallen und hast dir die Rippen gebrochen. Es tat mehr weh als alles, was du je gefühlt hast, und du dachtest, du müsstest sterben. Tu so, als käme die Angst von einem Dementor, der vor dir steht, einen zerfetzten schwarzen Umhang trägt und wie ein totes Ding aussieht, das im Wasser liegt. Und dann sprich den Patronuszauber und wenn du den Zauberstab schwingst, um den Dementor zu vertreiben, denk daran, wie dein Vater deine Hand gehalten hat, damit du keine Angst hast; und dann denk daran, wie sehr er dich liebt und wie sehr du ihn liebst, und lege alles in deine Stimme, wenn du \emph{Expecto Patronum} sagst. Für die Ehre des Hauses Malfoy und nicht nur, weil du mir einen Gefallen versprochen hast. Zeig mir, dass du mich an dem Tag im Bahnhof nicht angelogen hast, als du sagtest, Lucius sei ein guter Vater. Zeig mir, dass du tun kannst, was Salazar Slytherin tun konnte."

Und Harry trat rückwärts, hinter Draco, aus Dracos Blickfeld, so dass Draco nur noch vor dem verstaubten alten Lehrertisch und der Tafel vorn in dem unbenutzten Klassenzimmer stand.

Draco warf einen Blick hinter sich, diesen seltsamen Ausdruck noch immer auf seinem Gesicht und wandte sich dann nach vorne. Harry sah das Ausatmen, das Einatmen. Der Zauberstab zuckte einmal, zweimal, dreimal und viermal. Dracos Finger glitten am Zauberstab entlang, genau in den richtigen Abständen -

Draco senkte seinen Zauberstab.

„Das ist zu…“ sagte Draco „Ich kann nicht richtig \emph{denken}, während du zusiehst -„

Harry drehte sich um und ging auf die Tür zu. „Ich komme in einer Minute wieder“, sagte Harry. „Halte deinen glücklichen Gedanken fest und der Patronus wird bleiben."

Von hinter Draco kam das Geräusch der sich wieder öffnenden Tür.

Draco hörte Harrys Schritte das Klassenzimmer betreten, aber Draco drehte sich nicht um, um nachzusehen.

Harry sagte auch nichts. Die Stille dehnte sich aus.

Schließlich…

„Was \emph{bedeutet} das?“ fragte Draco. Seine Stimme schwankte ein wenig.

„Es bedeutet, dass du deinen Vater liebst“, sagte Harrys Stimme. Das war genau das, was Draco gedacht hatte. Er versuchte, nicht vor Harry zu weinen. Es war zu richtig, einfach zu richtig -

Vor Draco, auf dem Boden, lag die glänzende Form einer Schlange, die Draco erkannte; ein Blauer Krait, eine Schlange, die zuerst von Lord Abraxas Malfoy nach einem Besuch in einem fernen Land zu ihrem Anwesen gebracht wurde und Vater hatte seitdem einen Blauen Krait im Ophidiarium aufbewahrt. Die Sache mit dem Blauen Krait war, dass der Biss nicht sehr schmerzhaft war. Vater hatte das gesagt und Draco gesagt, dass er die Schlange \emph{nie} streicheln dürfe, egal wer zuschaute. Das Gift tötete deine Nerven so schnell, dass du keine Zeit hattest, Schmerzen zu empfinden, wenn das Gift sich ausbreitete. Du könntest sogar nach der Anwendung von Heilzaubern daran sterben. Sie fraß andere Schlangen. Es war so Slytherin, wie eine Kreatur nur sein konnte.

Deshalb war der Kopf eines Blauen Kraits in den Griff von Vaters Stock geschmiedet worden.

Die helle Schlange ließ ihre Zunge, die ebenfalls silbern war, hervorschnellen und schien irgendwie zu \emph{lächeln}, auf eine wärmere Art und Weise, als es irgendein Reptil tun sollte.

Und dann wurde Draco klar…

„Aber“, sagte Draco und starrte immer noch auf die wundervoll strahlende Schlange, „\emph{du} kannst den Patronuszauber nicht wirken.“ Jetzt, wo Draco ihn selbst gezaubert hatte, verstand er, warum das wichtig war. Man konnte böse sein, wie Dumbledore und trotzdem den Patronuszauber wirken, solange man noch \emph{etwas} Leuchtendes in sich trug.

Aber wenn Harry Potter nicht einen einzigen Gedanken in sich hatte, der so glänzte -

„Der Patronuszauber ist komplizierter, als du denkst, Draco“, sagte Harry ernsthaft. „Nicht jeder, der beim Zaubern versagt, ist ein schlechter Mensch oder gar unglücklich. Aber wie auch immer, ich \emph{kann} ihn zaubern. Ich habe es beim zweiten Versuch getan, nachdem mir klar wurde, was ich beim ersten Mal vor dem Dementor falsch gemacht hatte. Aber, na ja, mein Leben ist manchmal etwas seltsam, und mein Patronus kam seltsam heraus, und ich halte es vorerst noch geheim -"

"Soll ich das einfach so \emph{glauben}?"

„Du kannst Professor Quirrell fragen, wenn du mir nicht glaubst“, sagte Harry. „Frag ihn, ob Harry Potter einen körperlichen Patronus zaubern kann, und sag ihm ich hätte dir gesagt du sollst fragen. Er wird wissen, dass die Bitte von mir kam, niemand sonst weiß es."

Oh, und jetzt sollte Draco \emph{Professor Quirrell} vertrauen? So wie er Harry kannte, könnte es trotzdem wahr sein. Und Professor Quirrell würde nicht aus trivialen Gründen lügen.

Die glühende Schlange drehte ihren Kopf hin und her, als ob sie eine Beute suchte, die nicht da war, und wickelte sich dann zu einem Kreis, als ob sie sich ausruhen wollte.

„Ich frage mich“, sagte Harry leise, „wann es war, welches Jahr, welche Generation, dass die Slytherins aufhörten, den Patronuszauber zu lernen. Wann es war, dass die Leute anfingen zu denken, dass Slytherins selbst anfingen zu denken, dass List und Ehrgeiz dasselbe sei wie Kälte und Unglück. Und wenn Salazar wüsste, dass seine Schüler sich nicht einmal mehr die Mühe machten zu erscheinen um den Patronuszauber zu lernen, frage ich mich, ob er sich wünschte, er wäre nie geboren worden. Ich frage mich, wie das alles schief ging, wann Slytherins Haus kaputt ging."

Das leuchtende Wesen verschwand, der Aufruhr in Draco machte es unmöglich, den Zauber aufrechtzuerhalten. Draco drehte sich zu Harry, er musste sich beherrschen, um seinen Zauberstab nicht zu erheben. „Was weißt \emph{du} über Haus Slytherin \emph{oder} Salazar Slytherin? \emph{Du} wurdest nie in mein Haus sortiert, was gibt dir das Recht…"

Und \emph{da} wurde es Draco \emph{endlich} klar.

„\emph{Du wurdest nach Slytherin sortiert!}“ sagte Draco. „Das \emph{wurdest} du und danach hast du, du hast irgendwie, du \emph{hast mit den Fingern geschnippt} -“ Draco hatte Vater einmal gefragt, ob es nicht klüger wäre, sich in ein anderes Haus sortieren zu lassen, damit ihm alle vertrauen würden, und Vater hatte gelächelt und gesagt, dass er in Dracos Alter auch daran gedacht hatte, aber es gab keine Möglichkeit, den Sprechenden Hut zu täuschen…

… nicht bevor \emph{Harry Potter} auftauchte.

Wie konnte er auch nur \emph{eine Minute} lang glauben, dass \emph{Harry} ein \emph{Ravenclaw} war?

„Eine interessante Hypothese“, sagte Harry gleichförmig. „Weißt du, du bist die zweite Person in Hogwarts, die mit einer solchen Theorie aufwartet. Zumindest bist du der Zweite, der mir das ins Gesicht gesagt hat."

„Snape“, sagte Draco mit Sicherheit. Sein Hauslehrer war kein Narr.

„Professor Quirrell, \emph{natürlich}“, sagte Harry. „Aber wenn ich so darüber nachdenke, Severus fragte mich, wie ich es schaffte, seinem Haus fernzubleiben und ob ich etwas hätte, das der Sprechende Hut wollte. Ich nehme an, man könnte sagen, dass du die Nummer drei bist. Oh, aber Professor Quirrells Theorie war etwas anders als deine. Darf ich dein Wort haben, sie nicht zu wiederholen?"

Draco nickte, ohne wirklich darüber nachzudenken. Was hätte er tun sollen, nein sagen?

"Professor Quirrell war der Meinung, dass Dumbledore mit der Wahl des Hutes für den Jungen der lebte nicht glücklich war."

Und in dem Moment, als Harry es sagte, \emph{wusste} Draco, er \emph{wusste} dass es stimmte, es war einfach \emph{offensichtlich}. Wen hatte Dumbledore überhaupt zum Narren gehalten?

… gut, abgesehen von jeder einzelnen anderen Person in Hogwarts, außer Snape und Quirrell, Harry glaubte es vielleicht sogar \emph{selbst}…

Draco stolperte wie betäubt zu seinem Schreibtisch zurück und setzte sich so schwer hin, dass es weh tat. So etwas passierte etwa einmal im Monat mit Harry und im Januar war es noch nicht passiert, also war es Zeit.

Sein Slytherinkollege, der sich für einen Ravenclaw halten könnte oder auch nicht, setzte sich wieder quer auf den Stuhl, den er schon zuvor benutzt hatte und schaute Draco aufmerksam an.

Draco wusste nicht, \emph{was} er jetzt tun sollte, ob er versuchen sollte, den verlorenen Slytherin-Jungen davon zu überzeugen, dass er eigentlich \emph{kein} Ravenclaw \emph{war}… oder versuchen herauszufinden, ob Harry mit Dumbledore im Bunde war, obwohl das plötzlich weniger wahrscheinlich erschien… aber \emph{warum} hatte Harry dann die ganze Sache mit ihm und Granger eingefädelt…

Er hätte sich erinnern \emph{müssen}, dass Harry zu schräg für normale Pläne war.

„Harry“, sagte Draco. „Hast du mich und General Sunshine absichtlich verärgert, damit wir zusammen gegen dich arbeiten?"

Harry nickte ohne zu zögern, als wäre es die normalste Sache der Welt und nichts, wofür man sich schämen müsste.

"Die ganze Sache mit den Handschuhen und das Heraufklettern an den Wänden von Hogwarts, der \emph{einzige Punkt} war, dass ich und Granger uns mögen. Und sogar schon vorher. Du hast es schon sehr lange geplant. Von \emph{Anfang an}."

Wieder das Nicken.

"\emph{WHYYYYYYY?}"

Harry hob kurz die Augenbrauen, die einzige Reaktion, auf Dracos lauten Schrei im geschlossenen Klassenzimmer, der ihm in seinen eigenen Ohren wehtat. WARUM, WARUM, WARUM hatte Harry Potter das \emph{GETAN}?

Dann sagte Harry: „Damit die Slytherins wieder den Patronuszauber wirken können."

„\emph{Das… macht… keinen… SINN!}“ Draco war klar, dass er die Kontrolle über seine Stimme verlor, aber er schien nicht in der Lage zu sein, sich selbst aufzuhalten. „\emph{Was hat das mit Granger zu tun?}"

„Muster“, sagte Harry. Sein Gesicht war jetzt sehr ernst und sehr finster. „Wie ein Viertel der Kinder von Squib-Paaren, die als Zauberer geboren werden. Ein einfaches, unverkennbares Muster, das man sofort erkennen würde, wenn man wüsste, was man da sieht; doch wenn man es nicht wüsste, würde man nicht einmal merken, dass es ein Hinweis ist. Das Gift im Hause Slytherin ist etwas, das man schon in der Muggelwelt gesehen hat. Das ist eine \emph{Voraussage}, Draco, ich hätte es dir schon vor unserem ersten Schultag aufschreiben können, nachdem ich dich in der King's Cross Station reden gehört habe. Lass mich dir ein paar wirklich erbärmliche Leute beschreiben, die bei den politischen Kundgebungen deines Vaters herumhängen, reinblütige Familien, die niemals zum Essen in Malfoy Manor eingeladen werden würden. Wenn man bedenkt, dass \emph{ich} sie nie getroffen \emph{habe}, sage ich es nur voraus, weil ich das Muster dessen, was mit Haus Slytherin passiert, erkannt habe -"

Und Harry Potter fuhr fort, die Parkinsons und Montagues und Boles mit einer ruhigen, schneidenden Genauigkeit zu beschreiben, die Draco nicht zu \emph{denken} gewagt hätte, falls ein Legilimens in der Nähe wäre. Es war \emph{mehr} als beleidigend, sie würden Harry \emph{töten}, wenn sie jemals hören würden…

„Zusammenfassend“, schloss Harry ab, „haben sie selbst keine Macht. Sie haben selbst keinen Reichtum. Wenn sie keine Muggelgeborenen hätten, die sie hassen könnten, wenn alle Muggelgeborenen verschwinden würden, wie sie es angeblich \emph{wollen}, würden sie eines Morgens aufwachen und feststellen, dass sie \emph{nichts} haben. Aber solange sie sagen können, dass Reinblütige überlegen sind, können sie sich selbst überlegen fühlen, können sie sich als Teil der Oberklasse fühlen. Auch wenn dein Vater nicht im Traum daran denken würde, sie zum Essen einzuladen, obwohl es nicht eine einzige Galeone in ihren Gewölben gibt, auch wenn sie auf ihren OWLs schlechter abgeschnitten haben als der schlimmste Muggelgeborene in Hogwarts. Auch wenn sie den Patronuszauber nicht mehr ausüben können. Die Muggelgeborenen sind für sie an allem schuld, sie haben außer sich selbst noch jemanden, der für ihr eigenes Versagen verantwortlich ist und das macht sie noch schwächer. Genau das wird aus Slytherin House, \emph{erbärmlich} und die Wurzel des Problems ist der Hass auf die Muggelgeborenen."

„Salazar Slytherin selbst hat gesagt, dass Schlammblüter ausgetrieben werden müssen! Dass sie unser Blut schwächen…“ Dracos Stimme war zu einem Schrei erwacht.

"\emph{Salazar hat sich geirrt, das ist eine einfache Tatsache!} Das \emph{weißt} du, Draco! Und dieser \emph{Hass} vergiftet dein ganzes Haus. Mit so einem Gedanken könntest du den Patronuszauber nicht wirken!"

"Warum konnte \emph{Salazar Slytherin} dann den Patronuszauber wirken?"

Harry wischte sich den Schweiß von der Stirn. „Weil sich die Dinge zwischen damals und heute \emph{geändert} haben! Hör zu, Draco, vor dreihundert Jahren gab es große Wissenschaftler, auf ihre Art so groß wie Salazar, die dir gesagt hätten, dass einige andere Muggel wegen ihrer Hautfarbe minderwertig wären -"

"\emph{Hautfarbe?}"sagte Draco.

"Ich weiß, Hautfarbe statt etwas Wichtigem wie Blutreinheit, ist das nicht lächerlich? Aber dann hat sich etwas in der Welt verändert, und \emph{jetzt} findet man keine großen Wissenschaftler mehr, die immer noch denken, dass die Hautfarbe eine Rolle spielen sollte, nur noch Verlierer wie die, die ich dir beschrieben habe. Salazar Slytherin machte den Fehler, als alle anderen ihn machten, weil er mit dem Glauben daran aufgewachsen war, nicht weil er \emph{verzweifelt jemanden zum Hassen suchte}. Es gab ein paar Leute, die es besser machten als alle anderen um sie herum, und \emph{sie} waren außergewöhnlich gut. Aber diejenigen, die einfach akzeptierten, was alle anderen dachten, waren nicht \emph{außergewöhnlich} schlecht. Die traurige Tatsache ist, dass die meisten Menschen ein moralisches Problem überhaupt nicht bemerken, es sei denn, jemand anderes weist sie darauf hin; und wenn sie einmal so alt sind wie Salazar war, als er Godric traf, haben sie die Fähigkeit verloren, ihre Meinung zu ändern. Erst \emph{dann} wurde Hogwarts gebaut, und Hogwarts begann, wie Godric es wollte, Aufnahmeschreiben an Muggelgeborene zu schicken, und immer mehr Menschen begannen zu bemerken, dass Muggelgeborene \emph{nicht} anders waren. Jetzt ist es ein großes politisches Thema, statt etwas, das jeder einfach glaubt, ohne darüber nachzudenken. Und die \emph{richtige} Antwort ist, dass Muggelgeborene \emph{nicht} schwächer sind als Reinblütige. Die Leute, die sich \emph{jetzt} auf Salazars Seite schlagen, sind entweder in einer sehr geschlossenen, reinblütigen Umgebung aufgewachsen wie du, \emph{oder} sie sind selbst so armselig, dass sie verzweifelt nach jemandem suchen, dem sie sich überlegen fühlen können, nach Leuten, die es lieben zu hassen."

„Das klingt nicht… das klingt nicht richtig…“ sagte Dracos Stimme. Seine Ohren hörten zu und wunderten sich, dass ihm nichts Besseres zu sagen einfiel.

„Tut es das nicht? Draco, du \emph{weißt} jetzt, dass mit Hermine Granger alles in Ordnung ist. Du hattest Probleme, sie vom Dach fallen zu lassen, wie ich hörte. Obwohl du wusstest, dass sie einen Federfalltrank genommen hatte, obwohl du wusstest, dass sie sicher war. Was glaubst du, welche Sorte Mensch sie \emph{töten} will, nicht für das, was sie ihnen angetan hat, sondern nur weil sie eine Muggelgeborene ist?

Auch wenn sie, sie ist nur ein junges Mädchen, das ihnen in einer Sekunde bei den Hausaufgaben helfen würde, wenn sie sie jemals fragen würden“, Harrys Stimme brach, „was für ein Mensch will, dass sie \emph{stirbt}? „

\emph{Vater} -

Draco fühlte sich zweigeteilt, er schien ein Problem mit der Doppelsicht zu haben, \emph{Granger ist ein Schlammblut, sie sollte sterben} und ein Mädchen, das an seiner Hand auf dem Dach hängt, es war wie doppelt zu sehen, doppelt sehen -

"Und wer \emph{nicht} will, dass Hermine Granger stirbt, der will auch nicht mit solchen Leuten rumhängen, die es \emph{wollen}! Das ist alles, was die Leute denken, was Slytherin jetzt \emph{ist}, keine kluge Planung, kein Versuch, Größe zu erlangen, nur der Hass auf Muggelgeborene! Ich bezahlte Morag einen Sickel, um Padma zu fragen, warum sie nicht nach Slytherin gegangen ist, wir beide wissen, dass sie die Möglichkeit dazu hatte. Und Morag erzählte mir, dass Padma sie nur \emph{anschaute} und sagte, dass sie nicht Pansy Parkinson sei. Siehst du? Die \emph{besten} Schüler mit den Tugenden von mehr als einem Haus, die Schüler mit \emph{Wahlmöglichkeiten}, sie gehen unter den Hut und denken \emph{alles außer Slytherin}, und jemand wie Padma landet in Ravenclaw. Und… Ich denke, der Sprechende Hut versucht, die Balance in der Sortierung zu halten, also füllt er die Reihen Slytherins mit jedem, der \emph{nicht} von all dem Hass abgestoßen ist. Statt Padma Patil bekommt Slytherin also Pansy Parkinson. Sie ist nicht sehr gerissen und nicht sehr ehrgeizig, aber sie ist die Art von Person, die sich nicht darum kümmert, was aus Slytherin wird. Und je mehr Padmas nach Ravenclaw gehen und je mehr Pansies nach Slytherin gehen, desto mehr beschleunigt sich der Prozess. \emph{Es zerstört das Haus Slytherin, Draco!}"

Es klang schrecklich nach der Wahrheit, Padma \emph{hatte} in Slytherin gehört… und stattdessen bekam Slytherin Pansy… Vater versammelte kleinere Familien wie die Parkinsons, weil sie bequeme Unterstützungsquellen waren, aber Vater hatte nicht erkannt, welche \emph{Konsequenzen} es hatte, Slytherins Namen mit ihnen in Verbindung zu bringen…

„Ich kann nicht…“ sagte Draco, aber er war sich nicht einmal sicher, was er nicht tun konnte - „Was \emph{willst} du von mir?"

„Ich bin mir nicht sicher, wie man Haus Slytherin heilen kann“, sagte Harry langsam. „Aber ich weiß, dass es etwas ist, das du und ich am Ende tun müssen. Es dauerte Jahrhunderte, bis die Wissenschaft über der Muggelwelt anbrach, es geschah nur langsam, aber je stärker die Wissenschaft wurde, desto weiter zog sich diese Art von Hass zurück.“ Harrys Stimme war jetzt ruhig. „Ich weiß nicht genau, warum es so funktioniert hat, aber so ist es historisch gesehen passiert. Als gäbe es in der Wissenschaft so etwas wie den Glanz des Patronuszaubers, der alle Arten von Dunkelheit und Wahnsinn zurückdrängt, nicht sofort, aber er scheint überall dorthin zu folgen, wo die Wissenschaft hingeht. Die Aufklärung*, so wurde sie in der Muggelwelt genannt. Es hat etwas mit der Suche nach der Wahrheit zu tun, denke ich… damit, dass man seine Meinung von dem, woran man aufgewachsen ist, ändern kann… damit, dass man \emph{logisch} denkt und erkennt, dass es keinen \emph{Grund} gibt, jemanden zu hassen, weil seine Haut eine andere Farbe hat, genauso wie es keinen Grund gibt, Hermine Granger zu hassen… oder vielleicht ist da etwas dran, das selbst ich nicht verstehe. Aber die Aufklärung ist etwas, zu dem wir beide gehören, wir beide. Haus Slytherin wiederherzustellen ist nur eines der Dinge, die wir tun müssen."

„Lass mich nachdenken“, sagte Draco, seine Stimme krächzte, „bitte“, und er legte seinen Kopf in seine Hände und dachte nach.

Draco dachte eine Weile nach, mit den Handflächen über den Augen, um die Welt auszuschließen, nur sein und Harrys Atmen waren zu hören. All die überzeugende Vernunft dessen, was Harry sagte, die offensichtlichen Körnchen Wahrheit, die es enthielt; und dagegen das Offensichtliche, die vollkommene und völlig offensichtliche Hypothese darüber, was \emph{wirklich} vor sich ging…

Nach einer Weile hob Draco schließlich den Kopf.

„Es klingt richtig“, sagte Draco leise.

Ein riesiges Lächeln brach auf Harrys Gesicht los.

„Also“, fuhr Draco fort, „bringst du mich jetzt zu Dumbledore, um es offiziell zu machen?"

Er hielt seine Stimme sehr lässig, als er das sagte.

„Oh ja“, sagte Harry. „Das war genau das, was ich dich fragen wollte -"

Dracos Blut gefror in seinen Adern, erstarrte fest und zerbrach -

"Professor Quirrell sagte etwas zu mir, das mich zum Nachdenken anregte, und, nun, egal wie du diese Frage beantwortest, bin ich schon dumm, weil ich dich nicht schon viel früher gefragt habe. Jeder in Gryffindor hält Dumbledore für einen Heiligen, die Hufflepuffs halten ihn für verrückt, die Ravenclaws sind alle stolz darauf, dass sie herausgefunden haben, dass er nur so tut, als sei er verrückt, aber ich habe nie jemanden aus Slytherin gefragt. Ich sollte es besser wissen, als so einen Fehler zu machen. Aber wenn selbst \emph{du} denkst, Dumbledore ist jemand mit dem man sich zur Wiederherstellung von Haus Slytherin verschwören kann, habe ich wohl nichts Wichtiges verpasst."

…

…

…

„Weißt du“, sagte Draco, seine Stimme war bemerkenswert ruhig, alles in allem, „jedes Mal, wenn ich mich frage, ob du so etwas nur tust, um mich zu ärgern, sage ich mir, dass es ein Versehen sein \emph{muss}, \emph{niemand} könnte so etwas mit Absicht tun, selbst wenn er es versucht hätte, bis Blut aus seinen Ohren ränne. Das ist der einzige Grund, warum ich dich jetzt nicht erwürgen werde."

"Hä?"

Und dann würde er \emph{sich selbst} erwürgen, denn Harry \emph{war} mit Muggeln aufgewachsen, und dann hatte Dumbledore ihn problemlos von Slytherin nach Ravenclaw umgeleitet, also war es durchaus plausibel, dass Harry vielleicht gar \emph{nichts} wusste, und Draco hatte nie daran gedacht, es \emph{ihm zu sagen}.

Oder Harry hatte geahnt, dass Draco sich nicht so leicht mit Dumbledore verbünden würde, und das war nur der nächste Schritt in Dumbledores Plan…

Aber wenn Harry \emph{wirklich} nichts von Dumbledore wusste, dann hatte die Warnung für ihn Vorrang vor \emph{allem anderen}.

„Na schön“, sagte Draco, nachdem er seine Gedanken ordnen konnte. „Ich weiß nicht, wo ich anfangen soll, also fange ich einfach irgendwo an.“ Draco holte tief Luft. Das sollte eine Weile dauern. „Dumbledore hat seine kleine Schwester ermordet und kam davon, weil sein Bruder nicht gegen ihn aussagen wollte -„

Harry hörte mit zunehmender Sorge und Bestürzung zu. Harry war, so dachte er, bereit gewesen, die Reinblüterseite der Geschichte mit einem Körnchen Salz zu betrachten. Das Problem war, dass selbst nachdem man eine enorme Menge Salz hinzugefügt hatte, es \emph{immer} noch nicht gut klang.

Dumbledores Vater war verurteilt worden, weil er unverzeihliche Flüche auf Kinder angewandt hatte, und starb in Askaban. Das war keine Sünde von Dumbledore, aber es wäre eine Sache des staatlichen Archivs. Harry konnte diesen Teil überprüfen und feststellen, ob das alles von den Reinblütern aus der Luft gegriffen war.

Dumbledores Mutter war auf mysteriöse Weise gestorben, kurz bevor seine jüngere Schwester an dem, was die Auroren als Mord deklariert hatten, starb. Angeblich war diese Schwester von Muggeln brutal misshandelt worden und hatte danach nie wieder gesprochen; was, wie Draco betonte, bemerkenswert nach einem verpfuschten Gedächtniszauber klang.

Nach Harrys ersten Unterbrechungen schien Draco das allgemeine Prinzip zu begreifen und präsentierte nun zuerst die Beobachtungen und danach die Schlussfolgerungen.

“- also musst du mir nicht glauben“, sagte Draco, „du kannst es doch \emph{sehen}, oder? Jeder in Slytherin kann es sehen. Dumbledore wartete mit seinem Duell mit Grindelwald genau bis zu dem Moment, in dem es für Dumbledore am besten aussehen würde, \emph{nachdem} Grindelwald fast ganz Europa ruiniert und sich einen Ruf als der schrecklichste Dunkle Zauberer der Geschichte aufgebaut hatte, und gerade als Grindelwald die Gold- und Blutopfer, die er von seinen Muggelbauern erhielt, verloren hatte, ging es mit ihm bergab. Wenn Dumbledore wirklich der edle Zauberer wäre, der er vorgab zu sein, hätte er schon lange vorher gegen Grindelwald gekämpft. Wahrscheinlich \emph{wollte} Dumbledore Europa ruinieren, es war wahrscheinlich Teil ihres gemeinsamen Plans, er hat Grindelwald erst angegriffen, nachdem seine Marionette ihn \emph{im Stich gelassen} hatte. Und dieses große, auffällige Duell war nicht echt, es ist unmöglich, dass zwei Zauberer so genau zusammenpassen, dass sie zwanzig ganze Stunden lang kämpfen, bis einer von ihnen vor Erschöpfung umfällt, das war nur Dumbledore, der es noch spektakulärer aussehen ließ“. Hier wurde Dracos Stimme noch empörter. „Und so wurde Dumbledore zum \emph{Obersten Hexenmeister des Zaubergamots} ernannt! Die Linie von Merlin ungebrochen, nach 1.500 Jahren korrumpiert! Und \emph{dann} wurde er auch noch Oberster Mugwump, und er \emph{hatte bereits} Hogwarts als unbesiegbare Festung - Schulleiter \emph{und} Oberster Hexenmeister \emph{und} Oberster Mugwump, kein normaler Mensch würde versuchen, all das auf einmal zu tun, \emph{wie kann man nicht sehen, dass Dumbledore versucht, die Welt zu erobern?}"

„Pause“, sagte Harry und schloss die Augen, um nachzudenken.

Es war nicht schlimmer als das, was man über den Westen in Stalins Russland gehört hätte, und nichts davon wäre wahr gewesen. Obwohl die Reinblüter nicht damit durchkommen würden, sich etwas einfach auszudenken… oder doch? Der \emph{Tagesprophet} hatte eine ausgeprägte Neigung gezeigt, sich etwas auszudenken… aber dann wiederum, als sie ihren Hals zu weit in die Weasley-Verlobung steckten, \emph{wurden} sie entlarvt und sie \emph{wurden} in Verlegenheit gebracht…

Harry öffnete die Augen und sah, dass Draco ihn mit einem ruhigen, abwartenden Blick beobachtete.

"Als du mich gefragt hast, ob es Zeit ist, zu Dumbledore zu gehen, war das nur ein Test."

Draco nickte.

"Und vorher, als du sagtest, es klang richtig…"

„Es \emph{klingt} richtig“, sagte Draco. „Aber ich weiß nicht, ob ich dir vertrauen kann. Willst du dich darüber beschweren, dass ich dich \emph{getestet habe}? Willst du sagen, dass ich dich \emph{getäuscht} habe? Dass ich \emph{dich verführt habe}? „

Harry wusste, dass er lächeln sollte wie ein guter Sportsmann, aber er konnte es nicht wirklich, es war zu viel der Enttäuschung.

„Du hast Recht, es ist fair, ich kann mich nicht beklagen“, sagte Harry stattdessen. „Und was ist mit Ihm, dessen Name nicht genannt werden darf? Nicht so schlimm, wie er dargestellt wurde?"

Draco sah daraufhin verbittert aus. „Du glaubst also, dass es nur darum geht, Vaters Seite gut und Dumbledores Seite schlecht aussehen zu lassen und das ich das alles einfach glaube weil Vater es mir erzählt hat."

„Das ist eine Möglichkeit, die ich in Betracht ziehe“, sagte Harry ruhig.

Dracos Stimme war leise und intensiv. „Sie \emph{wussten} es. Mein Vater wusste es, seine Freunde wussten es. Sie \emph{wussten}, dass der Dunkle Lord böse war. \emph{Aber er war die einzige Chance, die sie gegen Dumbledore hatten!} Der einzige Zauberer überhaupt, der mächtig genug war, gegen ihn zu kämpfen!

Einige der anderen Todesser waren wirklich böse, wie Bellatrix Black - Vater ist nicht so - aber Vater und seine Freunde \emph{mussten} es tun, Harry, sie \emph{mussten} es tun, Dumbledore übernahm alles, der Dunkle Lord war die einzige Hoffnung, die es noch gab!

Draco starrte Harry fest an. Harry traf den Blick, versuchte nachzudenken. Niemand hielt sich je für den Bösewicht seiner eigenen Geschichte - vielleicht tat es Lord Voldemort, vielleicht Bellatrix, aber Draco ganz sicher nicht. Dass die Todesser Bösewichte waren, stand nicht in Frage. Die Frage war, ob sie \emph{die} Bösen waren, ob es in der Geschichte \emph{einen} oder \emph{zwei} Bösewichte gab…

„Du bist nicht überzeugt“, sagte Draco. Er sah besorgt aus und ein wenig wütend. Was Harry nicht überraschte. Er war sich ziemlich sicher, dass Draco selbst das alles glaubte.

„\emph{Sollte} ich überzeugt sein?“ sagte Harry. Er sah nicht weg. „Nur weil du es glaubst? Bist du jetzt ein so starker Rationalist, dass dein Glaube ein starker Beweis für mich ist, denn du würdest es wohl kaum glauben, wenn es nicht wahr wäre? Als ich dich traf, warst du nicht so stark. Hast du alles, was du mir erzählt hast, nach deinem Erwachen als Wissenschaftler überdacht, oder ist es nur etwas, woran du als Kind geglaubt hast? Kannst du mir in die Augen sehen und mir bei der Ehre des Hauses Malfoy schwören, dass du es bemerkt hättest, wenn eine Unwahrheit in deinen Worten steckt, eine Sache, die hinzugefügt wurde, nur um Dumbledore ein wenig schlechter aussehen zu lassen?"

Draco fing an, den Mund aufzumachen, und Harry sagte: „Lass es. Beflecke die Ehre des Hauses Malfoy nicht. So stark bist du noch \emph{nicht} und das solltest du wissen. Hör zu, Draco, mir sind auch ein paar beunruhigende Dinge aufgefallen. Aber es gibt nichts \emph{Eindeutiges}, nichts \emph{Gewisses}, es sind alles nur Schlussfolgerungen und Hypothesen und unzuverlässige Zeugen… Und in deiner Geschichte gibt es auch nichts Sicheres. Dumbledore hätte vielleicht noch einen anderen guten Grund gehabt, sich Jahre zuvor nicht gegen Grindelwald zu wehren - obwohl es eine ziemlich gute Entschuldigung sein \emph{müsste}, besonders wenn man bedenkt, was auf der Muggel-Seite der Dinge geschah… aber trotzdem. Gibt es eine eindeutig böse Sache, die Dumbledore ganz \emph{sicher} getan hat, damit ich mir darüber keine Gedanken machen muss?„

Dracos Atem war streng. „Also gut“, sagte Draco mit unruhiger Stimme, „ich sage dir, was Dumbledore getan hat.“ Aus Dracos Gewand kam ein Zauberstab, und Draco sagte: „Quietus“, dann wieder „Quietus“, aber er hatte sich beide Male bei der Aussprache geirrt und schließlich holte Harry seinen eigenen Zauberstab raus und tat es.

„So“, sagte Draco heiser, „es war einmal ein Mädchen, und ihr Name war Narzissa, und sie war das hübscheste, klügste, gerissenste Mädchen, das je nach Slytherin sortiert wurde, und mein Vater liebte sie, und sie heirateten, und sie war keine Todesserin, sie war keine Kämpferin, \emph{alles was sie je getan hatte, war Vater zu lieben} -“ Draco stoppte dort, weil er weinte.

Harry fühlte sich total schlecht. Draco hatte nie über seine \emph{Mutter} geredet, nicht ein einziges Mal, das hätte er früher merken müssen. „Sie… kam einem Fluch in die Quere?"

Dracos Stimme kam in einem Schrei heraus. „\emph{Dumbledore hat sie in ihrem eigenen Schlafzimmer verbrannt!}"

In einem Klassenzimmer, das mit weichem, silbernen Licht gefüllt ist, starrt ein Junge einen anderen Jungen an, der schluchzt und sich mit den Ärmeln seiner Robe verzweifelt die Augen abwischt.

Es war schwer für Harry, das Gleichgewicht zu halten, das Urteil zurückzuhalten, es war zu emotional, es gab etwas, das entweder aus Sympathie für Draco Tränen aus seinen eigenen Augen hervorrufen wollte, oder \emph{wissen} wollte, dass es nicht wahr war…

\emph{Dumbledore hat sie in ihrem eigenen Schlafzimmer verbrannt!}

Das…

… klang nicht nach Dumbledore…

… aber man konnte diesen Gedanken nur soundso oft denken, bevor man sich über die Vertrauenswürdigkeit dieses `Stil'-Konzepts wunderte.

„Es muss furchtbar wehgetan haben“, sagte Draco, seine Stimme zitterte, „Vater spricht nie darüber, man spricht nie in seiner Gegenwart darüber, aber Mr. Macnair hat mir gesagt, dass im ganzen Schlafzimmer Brandspuren zu sehen waren, davon wie Mutter sich gewehrt haben muss, als Dumbledore sie \emph{lebendig verbrannt hat}. Das ist die Schuld, die Dumbledore dem Hause Malfoy zahlen muss, und \emph{dafür bekommen wir sein Leben!}"

„Draco“, sagte Harry, er ließ all die Heiserkeit in seine eigene Stimme einfließen, es wäre \emph{falsch}, ruhig zu klingen, „Es tut mir leid, es tut mir so leid, dass ich gefragt habe, aber ich \emph{muss} wissen, \emph{woher} du weißt, dass es Dumble-"

„Dumbledore hat \emph{gesagt}, das er es war. Er hat Vater gesagt es war eine \emph{Warnung}! Und Vater konnte nicht unter Veritaserum aussagen, weil er ein Okklumentiker war, er konnte Dumbledore nicht einmal vor Gericht bringen, Vaters eigene Verbündete glaubten ihm nicht, nachdem Dumbledore einfach alles öffentlich abgestritten hatte, aber wir wissen es, die Todesser wissen es, Vater hätte keinen Grund, deshalb zu lügen, Vater würde wollen, dass wir uns an der \emph{richtigen Person} rächen, siehst du das nicht, Harry?“ Dracos Stimme war wild.

\emph{Es sei denn, Lucius hat es selbst getan und fand es bequemer, Dumbledore die Schuld zu geben.}

Obwohl… es schien auch nicht \emph{Lucius'} Stil zu sein. Und hätte er Narzissa ermordet, wäre es klüger gewesen, die Schuld einem leichteren Opfer zuzuschieben, statt politisches Kapital und Glaubwürdigkeit zu verlieren…

Mit der Zeit hörte Draco auf zu weinen und sah Harry an. „\emph{Und?}“ sagte Draco. Es klang, als wollte er die Worte ausspucken. „Ist das \emph{böse} genug für Sie, Mr. Potter?„

Harry sah hinunter, wo seine Arme auf der Stuhllehne lagen. Er konnte Draco nicht mehr in die Augen sehen, der Schmerz in ihnen war zu roh. „Das hätte ich nicht erwartet“, sagte Harry leise. „Ich weiß nicht mehr, was ich denken soll."

„Du \emph{weißt es nicht?}“ Dracos Stimme erhob sich zu einem Schrei und er stand plötzlich von seinem Schreibtisch auf -

„Ich erinnerte mich daran, dass der Dunkle Lord meine Eltern getötet hat“, sagte Harry. „Als ich das erste Mal vor den Dementor trat, war es das an das ich mich erinnerte, die schlimmste Erinnerung. Auch wenn es schon so lange her ist. Ich hörte, wie sie starben. Meine Mutter flehte den Dunklen Lord an, mich nicht zu töten, \emph{nicht Harry, bitte nicht, nimm mich, töte mich stattdessen!} Das ist, was sie sagte. Und der Dunkle Lord verspottete sie und lachte. Dann, ich erinnere mich, der Blitz aus grünem Licht -"

Harry schaute zu Draco auf.

„Wir können kämpfen“, sagte Harry, „wir können einfach mit demselben Kampf weitermachen. Du könntest mir sagen, dass es richtig war, dass meine Mutter starb, denn sie war die Frau von James, der einen Todesser getötet hat. Aber es war schlecht für \emph{deine} Mutter zu sterben, denn \emph{sie} war unschuldig. Und ich könnte dir sagen, dass es richtig war, dass deine Mutter gestorben ist, dass Dumbledore einen \emph{Grund} gehabt haben muss, damit es \emph{okay} war sie in ihrem eigenen Schlafzimmer lebendig zu verbrennen; aber schlecht für \emph{meine} Mutter, dass sie gestorben ist. Aber weißt du, Draco, wäre es nicht \emph{offensichtlich}, dass wir nur voreingenommen waren? Denn die Regel, die besagt, dass es falsch ist, Unschuldige zu töten, kann nicht für meine Mutter ein- und für deine Mutter ausgeschaltet werden. Und sie kann nicht für deine Mutter ein- und für meine ausgeschaltet werden. Wenn du mir sagst, dass Lily ein Feind der Todesser war und es richtig ist, deine Feinde zu töten, dann sagt die gleiche Regel, dass Dumbledore Recht hatte, Narzissa zu töten, da sie \emph{seine} Feindin war.“ Harrys Stimme wurde heiser. „Wenn wir uns auf irgendwas einigen können, dann dass der Tod von \emph{keiner} von beiden richtig war und dass \emph{niemandes} Mutter sterben sollte."

Die Wut, die in Draco brodelte, war so groß, dass er sich kaum davon abhalten konnte, aus dem Raum zu stürmen; alles, was ihn aufhielt, war das Erkennen eines kritischen Moments; und ein kleiner Rest von Freundschaft, ein winzige aufblitzen von Sympathie, denn er hatte vergessen, er hatte \emph{vergessen}, dass Harrys Mutter \emph{und} Vater durch die Hand des Dunklen Lords gestorben waren.

Die Stille dehnte sich aus.

„Du kannst reden“, sagte Harry, „Draco, rede mit mir, ich werde nicht wütend - denkst du, ich weiß nicht, dass Narzissas Tod viel schlimmer war als Lilys Tod? Dass es für mich falsch ist, diesen Vergleich überhaupt anzustellen?"

„Ich schätze, ich war auch dumm“, sagte Draco. „Die ganze Zeit, die ganze Zeit habe ich vergessen, dass du die Todesser hassen musst, weil sie deine Eltern getötet haben, die Todesser so sehr hassen, wie ich Dumbledore hasse.“ Und Harry hatte nie etwas gesagt, nie reagiert, wenn Draco von den Todessern sprach, hat es \emph{verschwiegen} - Draco war ein Narr.

„Nein“, sagte Harry. „Es ist nicht - so ist es nicht, Draco, ich, ich weiß nicht mal, wie ich es dir erklären soll, außer, dass so ein Gedanke nicht“, Harrys Stimme klang erstickt, „du würdest ihn nie benutzen können, um den Patronuszauber zu wirken…"

Draco spürte einen plötzlichen Schmerz in seinem Herzen, unerwünscht, aber er spürte ihn. „Tust du so, als würdest du deine eigenen Eltern einfach \emph{vergessen}? Willst du sagen, ich soll Mutter einfach \emph{vergessen}?"

„Dann \emph{müssen} wir beide also Feinde sein?“ Jetzt wurde Harrys Stimme genauso wild. „Was haben \emph{wir uns} je angetan, das bedeutet, dass wir Feinde sein müssen? Ich weigere mich, so gefangen zu sein! Gerechtigkeit kann nicht bedeuten, dass wir \emph{beide} uns \emph{gegenseitig} angreifen sollen, das macht keinen Sinn!“ Harry blieb stehen, atmete tief durch, fuhr mit den Fingern durch das absichtliche Durcheinander seiner Haare - die Finger waren danach verschwitzt, Draco konnte es sehen. „Draco, hör zu, wir können nicht erwarten, uns sofort über alles einig zu sein, du und ich. Also werde ich dich nicht bitten zu sagen, dass es \emph{falsch} war, dass der Dunkle Lord meine Mutter getötet hat, sage einfach, dass es… \emph{traurig} war. Wir werden nicht darüber reden, ob es \emph{notwendig} war oder nicht, ob es \emph{gerechtfertigt} war. Ich werde dich nur bitten, zu sagen, dass es traurig war, dass es passiert ist, dass das Leben meiner Mutter auch wertvoll war, sag das einfach vorerst. Und ich sage, dass es traurig war, dass Narzissa starb, weil ihr Leben auch etwas wert war. Wir können nicht erwarten, dass wir uns sofort über alles einig werden, aber wenn wir damit anfangen, dass jedes Leben wertvoll ist, dass es traurig ist, wenn \emph{jemand} stirbt, dann weiß ich, dass wir uns eines Tages treffen werden. Das ist es, was ich von dir hören möchte. Nicht wer Recht hatte. Nicht, wer falsch lag. Nur, dass es traurig war, als deine Mutter starb, und traurig, als meine Mutter starb, und es wäre traurig, wenn Hermine Granger sterben würde, jedes Leben ist kostbar, können wir uns darauf einigen und den Rest vorerst ruhen lassen, reicht es, wenn wir uns nur darauf einigen? Können wir das, Draco? Das scheint… eher ein Gedanke zu sein, mit dem man den Patronuszauber wirken kann.„

Da waren Tränen in Harrys Augen.

Und Draco wurde wieder wütend. „Dumbledore hat Mutter \emph{getötet}. Es reicht nicht, nur zu sagen, dass es \emph{traurig} ist. Ich verstehe nicht, was \emph{du} glaubst, dass du tun musst, aber die Malfoys \emph{müssen} sich rächen!“ Den Tod eines Familienmitglieds nicht zu rächen, ging über Schwäche, über Schande \emph{hinaus}, du könntest genauso gut gar nicht \emph{existieren}.

„Das bestreite ich nicht“, sagte Harry leise. „Aber wirst du sagen, dass Lily Potters Tod traurig war? Sag nur diese eine Sache?"

„Das ist…“ Draco fiel es schwer, wieder Worte zu finden. „Ich weiß, ich weiß, wie du dich fühlst, aber siehst du denn nicht, Harry, selbst wenn ich sage, dass Lily Potters Tod \emph{traurig} war, das geht \emph{schon} gegen die Todesser!"

"Draco, du musst sagen \emph{können}, dass die Todesser sich in einigen Dingen geirrt haben! Du \emph{musst}, du kannst dich als Wissenschaftler nicht weiterentwickeln, sonst gibt es eine Straßensperre auf deinem Weg, eine Autorität, der du nicht widersprechen kannst. Nicht jede Veränderung ist eine Verbesserung, aber jede Verbesserung ist eine Veränderung, man kann nichts \emph{besser} machen, wenn man es nicht schafft, es \emph{anders} zu machen, \emph{du musst dich selbst es besser machen lassen als andere Menschen!} Sogar dein Vater, Draco, sogar er. Du musst auf etwas zeigen können, das dein Vater getan hat, und sagen, dass es falsch war, weil er nicht \emph{perfekt} war, und wenn du das nicht sagen kannst, kannst du es nicht besser machen."

Vater hatte ihn einen Monat lang jede Nacht vor dem Schlafengehen gewarnt, bevor er nach Hogwarts ging, dass es Menschen mit diesem Ziel geben würde.

"Du versuchst, mich von Vater loszureißen."

„Ich versuche, einen \emph{Teil} von dir loszureißen“, sagte Harry. „Ich versuche, dich Dinge richten zu lassen, die dein Vater falsch gemacht hat. Ich versuche, dich etwas \emph{besser} machen zu lassen. Aber nicht… deinen \emph{Patronus} brechen! „Harrys Stimme wurde leiser. „So etwas Leuchtendes möchte ich nicht kaputtmachen. Wer weiß, \emph{das} könnte auch nötig sein, um Haus Slytherin zu reparieren…"

Es kam an Draco heran, das war das Dumme, trotz allem kam es an ihn heran, man musste bei Harry sehr vorsichtig sein, weil seine Argumente so überzeugend klangen, \emph{selbst wenn er sich irrte}. „Du gibst \emph{nicht} zu, dass Dumbledore gesagt hat, du könntest den Tod deiner Eltern rächen, indem du den Sohn von Lord Malfoy von ihm nimmst."

„\emph{Nein}. Nein. Dieser Teil ist einfach falsch.“ Harry hat tief durchgeatmet. „Ich wusste nicht, wer Dumbledore war, oder wer der Dunkle Lord war, oder wer die Todesser waren, oder wie meine Eltern starben, bis 3 Tage, bevor ich nach Hogwarts kam. An dem Tag, an dem wir uns im Kleiderladen zum ersten Mal begegnet sind, habe ich es erfahren. Und Dumbledore \emph{mag} die Muggelwissenschaft nicht mal, oder er sagt, er mag sie nicht, ich konnte ihn erst einmal darauf testen. Der Gedanke, mich durch dich an den Todessern zu rächen, ist mir \emph{nie} in den Sinn gekommen, nicht ein \emph{einziges} Mal, bis jetzt. Ich wusste nicht, wer die Malfoys waren, als ich dich im Kleiderladen traf, und dann \emph{mochte} ich dich."

Es herrschte eine lange Stille.

„Ich wünschte, ich könnte dir vertrauen“, sagte Draco. Seine Stimme zitterte. „Wenn ich nur wüsste, dass du die Wahrheit sagst, wäre alles viel einfacher -"

Und dann kam Draco plötzlich ein Einfall.

Der Weg, um zu wissen, ob Harry Potter wirklich alles meinte, was er sagte, darüber, dass er Haus Slytherin wiederherstellen wollte, darüber, dass er traurig war, dass Mutter gestorben war.

Es wäre illegal, und da er es ohne Vaters Hilfe tun müsste, wäre es \emph{gefährlich}, er könnte nicht mal Harry Potter vertrauen ihm zu helfen, aber…

„Na schön“, sagte Draco. „Ich habe an ein endgültiges Experiment gedacht."

"Was ist es?"

„Ich möchte dir einen Tropfen Veritaserum geben“, sagte Draco. „Nur einen Tropfen, damit du nicht lügen kannst, aber nicht genug, um dich zu einer Antwort zu \emph{zwingen}. Ich weiß nicht, wo ich es herkriege, aber ich sorge dafür, dass es \emph{sicher} ist, -"

„Ähm“, sagte Harry. Da war ein hilfloser Blick in seinem Gesicht. „Draco, ähm…"

„Sag es nicht“, sagte Draco. Seine Stimme war fest und ruhig. „Wenn du nein sagst, ist das mein Versuchsergebnis."

"Draco, ich bin ein Okklumentiker -"

"\emph{OH, DAS IST SO EINE LÜGE -}"

"Ich wurde von Mr. Bester ausgebildet. Professor Quirrell hat es eingerichtet. Hören zu, Draco, ich \emph{nehme} einen Tropfen Veritaserum, wenn du es besorgen kannst. Ich \emph{warne} dich nur, ich bin ein Okklumentiker. Kein perfekter Okklumentiker, aber Mr. Bester sagte, ich würde einen kompletten Block aufstellen, und ich könnte wahrscheinlich Veritaserum schlagen."

"\emph{Du bist in deinem ersten Jahr in Hogwarts! Das ist einfach verrückt!}"

"Kennst du einen Legilimentiker, dem du vertrauen kannst? Ich demonstriere es dir gerne - schau Draco, es tut mir leid, aber zählt nicht die Tatsache, dass ich dir \emph{gesagt} habe etwas? Ich \emph{hätte} es dich einfach machen lassen können, weißt du."

\emph{"WARUM? Warum bist du immer so, Harry? Warum musst du immer alles vermasseln, auch wenn es UNMÖGLICH ist? Und hör auf zu grinsen, das ist nicht lustig! „}

"Tut mir leid, tut mir leid, ich \emph{weiß}, es ist nicht witzig, ich…"

Es dauerte eine Weile, bis Draco sich unter Kontrolle hatte.

Aber Harry hatte recht. Harry \emph{hätte} Draco einfach das Veritaserum verabreichen lassen können. \emph{Wenn} er wirklich ein Okklumentiker war… wusste Draco nicht, wen er bitten konnte, es mit Legilimentik zu versuchen, aber er hätte zumindest Professor Quirrell fragen können ob es stimmte… Könnte Draco \emph{Professor Quirrell} vertrauen? Vielleicht würde Professor Quirrell einfach alles sagen, worum Harry ihn bat.

Dann erinnerte sich Draco an die andere Sache, die Harry ihm gesagt hatte, er solle Professor Quirrell fragen, und er dachte an einen anderen Test.

„\emph{Weißt} du“, sagte Draco. „Du \emph{weißt}, was es mich kostet, wenn ich zustimme, dass das Gift in Haus Slytherin ist Muggelgeborene zu hassen und sage, dass Lily Potters Tod traurig war. Und das ist \emph{Teil deines Plans}, sag mir nicht, dass es nicht so ist."

Harry sagte nichts, was klug von ihm war.

„Es gibt etwas, das ich als Gegenleistung von dir will“, sagte Draco. „Und vorher möchte ich einen experimentellen Test durchführen…„

Draco drückte die Tür auf, zu der die Porträts sie geführt hatten, und dieses Mal war es die richtige Tür. Vor ihnen lag ein kleiner leerer Steinplatz unterm Nachthimmel. Kein Dach wie das, von dem er Harry runtergeworfen hatte, sondern ein winziger und ordentlicher Hof, weit über dem Boden. Mit einem richtigen Geländer, kunstvollen Steinzeichnungen, die sich nahtlos in den Steinboden übergingen… Wie so viel \emph{Kunstfertigkeit} in die Erschaffung von Hogwarts eingeflossen war, war etwas das Draco jedes Mal, wenn er darüber nachdachte in Ehrfurcht versetzte. Es musste einen Weg gegeben haben, alles auf einmal zu machen, niemand hätte so viele Details Stück für Stück ausarbeiten können, das Schloss \emph{veränderte} sich und jedes neue Stück war so. Es war so weit jenseits der Zauberei dieser verblassenden Tage, dass niemand es geglaubt hätte, wenn er nicht den Beweis in Hogwarts selbst gesehen hätte.

Wolkenlos und kalt, der winterliche Nachthimmel; es wurde lange vor der Ausgangssperre der Studenten dunkel in den letzten Januartagen.

Die Sterne leuchteten hell, in der klaren Luft.

Harry hatte gesagt, unter den Sternen zu sein, würde ihm helfen.

Draco berührte seine Brust mit seinem Zauberstab, ließ die Fingern in einer geübten Bewegung gleiten und sagte: „\emph{Thermos}.“ Eine Wärme breitete sich in ihm aus, die von seinem Herzen ausging; der Wind wehte weiter auf sein Gesicht, aber ihm war nicht mehr kalt.

„\emph{Thermos}“, sagte Harrys Stimme hinter ihm.

Sie gingen zusammen zum Geländer, um weit nach unten auf den Boden zu schauen. Draco versuchte herauszufinden, ob sie sich in einem der Türme befanden, die man von außen sehen konnte, und stellte fest, dass er sich im Moment nicht ganz vorstellen konnte, wie Hogwarts von außen aussah. Aber der Boden unten war immer derselbe; er konnte den Verbotenen Wald als vagen Umriss sehen und das Mondlicht vom Hogwarts-See aus glitzern sehen.

„Weißt du“, sagte Harrys Stimme leise neben ihm, wo seine Arme auf dem Geländer neben Dracos lehnten, „eines der Dinge, die Muggel wirklich falsch machen, ist, dass sie nachts nicht alle Lichter ausschalten. Nicht einmal für eine Stunde im Monat, nicht einmal für fünfzehn Minuten einmal im Jahr. Die Photonen streuen sich in der Atmosphäre und waschen alle bis auf die hellsten Sterne aus, und der Nachthimmel sieht überhaupt nicht mehr gleich aus, es sei denn, man geht weit weg von irgendwelchen Städten. Wenn man einmal in den Himmel über Hogwarts geblickt hat, kann man sich kaum vorstellen, in einer Muggelstadt zu leben, wo man die Sterne nicht sehen könnte. Du würdest sicher nicht dein ganzes Leben in Muggelstädten verbringen wollen, wenn du einmal den Nachthimmel über Hogwarts gesehen hast."

Draco warf einen Blick auf Harry und fand heraus, dass Harry seinen Hals streckte, um dorthin zu starren, wo die Milchstraße sich durch die Dunkelheit wölbte.

„Natürlich“, fuhr Harry fort, „kann man von der \emph{Erde} aus die Sterne auch nie richtig sehen, die Luft ist immer im Weg. Man muss von woanders hinsehen, wenn man die Wirklichkeit sehen will, die Sterne, die hart und hell brennen, wie ihr wahres Selbst. Hast du dir jemals gewünscht, dass du dich einfach im Nu in den Nachthimmel hinaufbewegen könntest, Draco, und dir anschauen, was es um andere Sonnen herum zu sehen gibt als unsere? Wenn es keine Grenzen für deine Magie gäbe, wäre das eine Sache die du tun würdest, wenn du alles tun könntest?„

Es herrschte Stille, und dann wurde Draco klar, dass man von ihm eine Antwort erwartete. „Daran habe ich vorher nicht gedacht“, sagte Draco. Ohne eine bewusste Entscheidung kam seine Stimme so sanft und leise heraus wie die von Harry. „Glaubst du wirklich, dass irgendjemand dazu in der Lage wäre?"

„Ich glaube nicht, dass es so einfach sein wird“, sagte Harry. „Aber ich weiß, dass ich nicht mein ganzes Leben auf der Erde verbringen will."

Es wäre lustig gewesen, wenn Draco nicht gewusst hätte, dass einige Muggel die Erde schon verlassen hatten, ohne Magie einzusetzen.

„Um deinen Test zu bestehen“, sagte Harry, „muss ich sagen, was \emph{mir} dieser Gedanke bedeutet, das Ganze, nicht die kürzere Version, die ich dir vorher zu erklären versuchte. Aber du solltest sehen, dass es die gleiche Idee ist, nur allgemeiner. \emph{Meine} Version des Gedankens, Draco, ist, dass wir, wenn wir zu den Sternen fliegen, dort andere Leute finden könnten. Und wenn das so ist, werden sie sicher nicht so aussehen wie wir. Vielleicht gibt es da draußen Dinge, die aus Kristall gewachsen sind, oder große, pulsierende Klumpen… oder sie könnten aus Magie bestehen, wenn ich so darüber nachdenke. Also, bei all der Seltsamkeit, wie erkennt man eine \emph{Person}? Nicht an der Form, nicht daran, wie viele Arme oder Beine sie hat. Nicht an der Substanz, aus der sie besteht, ob es Fleisch oder Kristall ist oder etwas, das ich mir nicht vorstellen kann. Man müsste sie wegen ihres \emph{Verstandes} als Personen erkennen. Und selbst ihr Verstand würde nicht so funktionieren wie unserer. Aber alles, was lebt und denkt und sich selbst kennt und nicht sterben will, es ist traurig, Draco, es ist traurig, wenn diese Person sterben muss, weil sie es nicht will. Verglichen mit dem, was da draußen sein könnte, jeder Mensch, der jemals gelebt hat, wir sind alle wie Brüder und Schwestern, man könnte uns kaum unterscheiden. Diejenigen da draußen, die \emph{uns} kennenlernen, würden keine Briten oder Franzosen sehen, sie würden nicht in der Lage sein, den Unterschied zu erkennen, sie würden nur einen Menschen sehen. Menschen, die lieben und hassen und lachen und weinen können; und für \emph{sie}, die da draußen, wären wir dann alle so gleich wie Erbsen in einer Schote. Aber \emph{sie} wären anders. \emph{Wirklich} anders. Aber das würde uns nicht aufhalten, und es würde sie nicht aufhalten, wenn wir beide Freunde sein wollten."

Harry erhob seinen Zauberstab und Draco drehte sich um und schaute weg, wie er es versprochen hatte; er schaute zu dem Steinboden und der Steinmauer, in die die Tür eingesetzt war. Denn Draco hatte versprochen, nicht hinzuschauen und niemandem zu erzählen, was Harry gesagt hatte, oder überhaupt irgendetwas von dem, was hier in dieser Nacht geschah, obwohl er nicht wusste, warum es so geheim sein sollte.

„Ich habe einen Traum“, sagte Harrys Stimme, „dass eines Tages empfindungsfähige Wesen nach den Mustern ihres Geistes beurteilt werden, und nicht nach ihrer Farbe oder ihrer Form oder dem Stoff, aus dem sie sind, oder wer ihre Eltern waren. Denn wenn wir eines Tages mit Kristalldingern zurechtkommen, wie dumm wäre es dann, nicht mit den Muggelgeborenen zurechtkommen, die wie wir geformt sind und wie wir denken, so ähnlich wie Erbsen in einer Schote? Die Kristalldinger würden nicht einmal in der Lage sein, den Unterschied zu erkennen. Wie unmöglich ist es sich vorzustellen, dass der Hass, der das Haus Slytherin vergiftet, es wert wäre, mit uns zu den Sternen gebracht zu werden? Jedes Leben ist kostbar, alles, was denkt und sich selbst kennt und nicht sterben will. Lily Potters Leben war kostbar, und Narzissa Malfoys Leben war kostbar, auch wenn es für sie jetzt zu spät ist, es war traurig, als sie starben. Aber es gibt andere Leben, die noch leben, um die man kämpfen muss. Dein Leben und mein Leben und das Leben von Hermine Granger, alle Leben der Erde und alle Leben jenseits davon, die verteidigt und beschützt werden müssen, \emph{EXPECTO PATRONUM!}"

Und es ward Licht.

Alles verwandelte sich in diesem Licht in Silber, der Steinboden, die Steinmauer, die Tür, das Geländer, so blendend allein in der Spiegelung, dass man sie kaum noch sehen konnte, sogar die Luft schien zu leuchten, und das Licht wurde heller und heller und heller -

Als das Licht ausging, war es wie ein Schock, Dracos Hand ging automatisch in seine Robe, um ein Taschentuch herauszuholen und erst dann merkte er, dass er weinte.

„Da ist dein Versuchsergebnis“, sagte Harrys Stimme leise. „Ich habe es wirklich so gemeint, diesen Gedanken."

Draco drehte sich langsam zu Harry um, der nun seinen Zauberstab gesenkt hatte.

„Das, das muss doch ein Trick sein, oder?“ fragte Draco. Er konnte nicht mehr viele dieser Schocks ertragen. „Dein Patronus - so hell \emph{kann} er doch nicht sein -“ Und doch \emph{war} es Patronuslicht gewesen, sobald man wusste, was man sah, konnte man es nicht mehr mit etwas anderem verwechseln.

„Das war die \emph{wahre} Form des Patronuszaubers“, sagte Harry. „Etwas, das dich all deine Kraft in den Patronus stecken lässt, ohne dass du dich selbst daran hinderst. Und bevor du fragst, ich hab's nicht von Dumbledore gelernt. Er kennt das Geheimnis nicht und könnte auch nicht die wahre Gestalt zaubern, wenn er es wüsste. Ich habe das Rätsel für mich selbst gelöst. Und ich wusste, sobald ich es verstanden hatte, dass man über diesen Zauber nicht sprechen darf. Um deinetwillen habe ich deine Prüfung abgelegt. Aber du darfst nicht darüber sprechen, Draco."

Draco wusste nichts mehr, er wusste nicht, wo die wahre Stärke lag, oder welche Dinge richtig waren. Doppeltsehen, Doppeltsehen. Draco wollte Harrys Ideale als Schwäche bezeichnen, als Hufflepuff-Torheit, als die Art von Lügen, die die Herrscher erzählten, um die Bevölkerung zu besänftigen, und die Harry dumm genug war selbst zu glauben, ernst genommene Torheiten, die in wahnsinnige Höhen erhoben und auf die Sterne selbst projiziert wurden -

Etwas Schönes und Verborgenes, Mysteriöses und Helles -

„Werde ich“, flüsterte Draco, „eines Tages in der Lage sein, so einen Patronus zu zaubern?"

„Wenn du immer die Wahrheit suchst“, sagte Harry, „und wenn du die warmen Gedanken nicht ablehnst, wenn du sie findest, dann wirst du es sicher können. Ich glaube, ein Mensch könnte überall hinkommen, wenn er nur lange genug weiter macht, sogar zu den Sternen."

Draco wischte sich wieder mit seinem Taschentuch die Augen ab.

„Wir sollten wieder reingehen“, sagte Draco mit unruhiger Stimme, „jemand hätte es sehen können, das ganze Licht -"

Harry nickte und ging zur und durch die Tür; und Draco schaute ein letztes Mal in den Nachthimmel, bevor er ihm folgte.

Wer war der Junge, der lebte, dass er bereits ein Okklumentiker war und die wahre Form des Patronuszaubers wirken und noch andere seltsame Dinge tun konnte? Was war Harrys Patronus, warum musste er ungesehen bleiben?

Draco hatte keine dieser Fragen gestellt, denn Harry hätte \emph{antworten} können, und Draco konnte heute einfach keine weiteren Schocks verkraften. Er konnte es einfach \emph{nicht}. Noch ein weiterer Schock und sein Kopf würde einfach von seinen Schultern fallen und die Korridore von Hogwarts hinunter hüpfen - hüpf, hüpf.

Sie hatten sich auf Dracos Bitte hin in eine kleine Nische geduckt, anstatt den ganzen Weg zurück ins Klassenzimmer zu gehen; er war zu nervös, um es noch länger hinauszuzögern.

Draco stellte eine Schallbarriere auf und schaute Harry dann in stummer Frage an.

„Ich habe darüber nachgedacht“, sagte Harry. „Ich werde es tun, aber es gibt fünf Bedingungen -"

"\emph{Fünf?} „

"Ja, fünf. Hör mal Draco, so ein Schwur \emph{bettelt} nur darum schief zu gehen. Du weißt es würde schief gehen, wenn das ein Theaterstück wäre…"

„Nun, das ist es nicht!“ antwortete Draco. „Dumbledore hat Mutter getötet. Er ist böse. Das ist eine der Sachen, über die man spricht, die \emph{nicht} kompliziert sein müssen."

„Draco“, sagte Harry, „ich \emph{weiß} nur, dass \emph{du} sagst, \emph{Lucius} sagt, \emph{Dumbledore} sagt, er hätte Narzissa getötet. Um das zu glauben, muss ich dir \emph{und} Lucius \emph{und} Dumbledore vertrauen. Darum gibt es, wie gesagt, Bedingungen. Die erste ist, dass \emph{du} mich jederzeit von dem Versprechen entbinden kannst, wenn es nicht mehr sinnvoll erscheint. Es muss eine bewusste und beabsichtigte Entscheidung deinerseits sein, natürlich keine Wortklauberei oder so was."

„Ok“, sagte Draco. Das hörte sich sicher genug an.

"Bedingung zwei: Ich verspreche, denjenigen zum Feind zu nehmen wer auch immer Narzissa umgebracht hat und den ich mit bestem Wissen und Gewissen mit meinen Fähigkeiten als Rationalist ermittelt habe. Ob das nun Dumbledore ist oder jemand anders. Und du hast mein Wort, dass ich mein Bestes als Rationalist geben werde, um dieses Beurteilung ehrlich als eine Frage nach dieser simplen Tatsache zu halten. Einverstanden?"

„Das gefällt mir nicht“, sagte Draco. Das tat es nicht. Es ging ihm darum, dass Harry nie auf der Seite von Dumbledore stehen würde. Aber wenn Harry ehrlich \emph{war}, würde er Dumbledore schon früh genug erwischen; und wenn er unehrlich war, hatte er sein Wort bereits gebrochen… „Aber ich stimme dir zu."

"Bedingung drei: Narcissa muss \emph{lebendig verbrannt} worden sein. Wenn sich dieser Teil der Geschichte als etwas übertrieben herausstellt, nur um es ein wenig schlimmer klingen zu lassen, dann kann ich selbst entscheiden, ob ich das Versprechen noch einhalten will oder nicht. Gute Menschen müssen manchmal töten. Aber sie foltern niemanden zu Tode. Weil Narcissa \emph{lebendig verbrannt} wurde, weiß ich, dass derjenige, der das getan hat, böse war."

Draco beherrschte sich, grade so.

"Bedingung vier: Wenn Narcissa sich die Hände schmutzig machte und z. B. das Kind von jemandem mit einem \emph{Cuciatus-Fluch} in den Wahnsinn trieb und diese Person Narcissa aus Rache verbrannte, ist der Deal vielleicht wieder gestrichen. Denn dann war es immer noch falsch, sie zu verbrennen, sie hätten sie immer noch ohne Schmerzen einfach töten sollen; aber es war nicht so \emph{böse}, als wäre sie nur Lucius' Liebe, die selbst nie etwas getan hat, wie du gesagt hast. Bedingung fünf ist: Wenn derjenige, der Narcissa getötet hat, irgendwie ausgetrickst wurde, es zu tun, dann ist mein Feind derjenige, der sie ausgetrickst hat, nicht die Person, die ausgetrickst wurde."

"Das klingt alles so, als ob du dich \emph{wirklich} davor drücken willst, -"

"Draco, ich werde mir keinen guten Menschen zum Feind machen, weder für dich noch für sonst jemanden. Ich muss wirklich glauben, dass sie im Unrecht sind. Aber ich habe darüber nachgedacht, und es scheint mir, dass, wenn Narcissa nichts Böses mit ihren eigenen Händen getan hat, sich einfach in Lucius verliebt hat und sich dafür entschieden hat, seine Frau zu bleiben, dann wird derjenige, der sie lebendig in ihrem eigenen Schlafzimmer verbrannt hat, wahrscheinlich kein guter Kerl sein. Und ich verspreche, denjenigen, der das getan hat, zum Feind zu nehmen, egal ob es Dumbledore oder sonst wer war, es sei denn, du entbindest mich absichtlich von diesem Versprechen. Hoffentlich geht \emph{das} nicht so schief, als wenn das ein Theaterstück wäre."

„Ich bin nicht glücklich“, sagte Draco. „Aber ok. Du schwörst, den Mörder meiner Mutter zum Feind zu nehmen, und ich werde -"

Harry wartete, mit einem geduldigen Gesichtsausdruck, während Draco versuchte, seine Stimme wieder zum Funktionieren zu bringen.

„Ich helfe dir, das Problem zu lösen, das Haus Slytherin Muggelgeborene hasst“, beendete Draco im Flüsterton. „Und ich werde sagen, dass es traurig war, dass Lily Potter gestorben ist."

„So soll es sein“, sagte Harry.

Und es war getan.

Der Bruch, so wusste Draco, hatte sich gerade noch ein wenig erweitert. Nein, nicht ein bisschen, \emph{sehr viel}. Es war ein Gefühl des Wegdriftens, des Verlorenseins, immer weiter weg vom Ufer, immer weiter weg von zu Hause…

„Entschuldige mich“, sagte Draco. Er wandte sich von Harry ab und versuchte dann, sich zu beruhigen, er musste diesen Test machen, und er wollte ihn nicht aus Nervosität oder Scham vermasseln.

Draco hob seinen Zauberstab in die Startposition für den Patronuszauber.

Er erinnerte sich daran, wie er von seinem Besenstiel fiel, der Schmerz, die Angst, er stellte sich vor, dass es von einer großen Gestalt in einem Umhang kam, die wie ein totes Ding aussah, das im Wasser zurückgelassen wurde.

Und dann schloss Draco die Augen, um sich besser daran zu erinnern, dass Vater seine kleinen, kalten Hände in seiner eigenen warmen Kraft hielt.

\emph{Hab keine Angst, mein Sohn, ich bin ja da…}

Der Zauberstab schwang sich in einem weiten Schwenk nach oben, um die Angst zu vertreiben, und Draco war überrascht über dessen Stärke; und er erinnerte sich in diesem Moment daran, dass \emph{Vater} nicht verloren war, nie verloren sein würde, immer da und stark in seiner eigenen Person sein würde, egal was mit Draco geschah, und seine Stimme rief „\emph{Expecto Patronum!"}

Draco öffnete seine Augen.

Eine leuchtende Schlange schaute zu ihm zurück, nicht weniger hell als zuvor.

Hinter sich hörte er \emph{Harry} ausatmen, als wäre er erleichtert.

Draco blickte in das weiße Licht. Es schien, als wäre er nach alledem doch nicht völlig verloren.

„Das erinnert mich an etwas“, sagte Harry nach einer Weile. „Können wir meine Hypothese testen, wie man mit einem Patronus Botschaften senden kann?"

„Wird es mich überraschen?“, sagte Draco. „Ich will heute keine Überraschungen mehr erleben."

Harry hatte behauptet, dass die Idee gar nicht so seltsam sei und er wüsste nicht, wie es Draco in irgendeiner Weise schockieren könnte, was Draco irgendwie noch nervöser machte; aber Draco konnte sehen, wie wichtig es war, eine Möglichkeit zu haben, in Notfällen Nachrichten zu versenden.

Der Trick - so Harrys Hypothese - bestand darin, die gute Nachricht verbreiten zu wollen, damit der Empfänger die Wahrheit über den glücklichen Gedanken erführe, mit dem man den Patronus-Zauber gewirkt hat. Doch statt es dem Empfänger in Worten zu sagen, war der Patronus selbst die Botschaft. Indem man wollte, dass sie das sehen, ginge der Patronus zu ihnen.

„Sag Harry“, sagte Draco zu der leuchtenden Schlange, obwohl Harry nur wenige Schritte entfernt auf der anderen Seite des Raumes stand, „sich, ähm, vor dem grünen Affen zu hüten“, dies war ein Zeichen aus einem Stück, das Draco einmal gesehen hatte.

Und dann, genau wie am Bahnhof von King's Cross, wollte Draco Harry wissen lassen, dass Vater sich immer um ihn gekümmert hatte; nur versuchte er es diesmal nicht in Worten zu sagen, sondern wollte es mit dem glücklichen Gedanken selbst sagen.

Die helle Schlange glitt durch den Raum und sah eher so aus, als würde sie durch die Luft gleiten als auf dem Stein; sie erreichte Harry, nachdem sie die kurze Strecke zurückgelegt hatte -

- und sagte zu Harry, mit einer seltsamen Stimme, die Draco als das erkannte, wie er selbst wahrscheinlich für andere Menschen klang: „Hüte dich vor dem grünen Affen“.

„\emph{Hsssss ssss sshsshssssss}“, sagte Harry.

Die Schlange glitt zurück über den Boden zu Draco.

„Harry sagt, die Nachricht wurde empfangen und bestätigt“, sagte der strahlende blaue Krait in Dracos Stimme.

„Hm“, sagte Harry. „Es fühlt sich komisch an, mit Patronussen zu reden."

…

…

…

…

„Warum siehst du mich so an?“, sagte der Erbe Slytherins.

Nachspiel:

Harry starrte Draco an.

"Du meinst, nur \emph{magische} Schlangen, oder?"

„N-nein“, sagte Draco. Er sah ziemlich blass aus und stotterte immer noch, aber er hatte zumindest damit aufgehört unzusammenhängenden Geräusche von sich zu geben. „Du bist ein Parselmund, du kannst Parsel sprechen, es ist die Sprache aller Schlangen überall. Du kannst jede Schlange verstehen, wenn sie spricht, und sie verstehen, wenn du mit ihnen sprichst… Harry, du kannst doch \emph{unmöglich} glauben, dass du nach Ravenclaw sortiert wurdest! \emph{Du bist der Erbe Slytherins!} „

…

…

…

…

…

"SCHLANGEN SIND VERNUNFTBEGABT?„

*Aufklärung - in Englisch “Age of Enlightenment", also das Zeitalter der Erleuchtung passt noch besser zum Bild des leuchtenden Patronus.

