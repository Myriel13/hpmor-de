

\hypertarget{das-stanford-prison-experiment-teil-2}{% \section{20. Das Stanford-Prison-Experiment, Teil 2}\label{das-stanford-prison-experiment-teil-2}}

—\/-\/-\/-\/- Kapitel 52: Das Stanford-Prison-Experiment, Teil 2 -\/-\/-\/-\/-

Das Adrenalin floss bereits durch Harrys Venen, sein Herz hämmerte bereits in seiner Brust, dort in diesem verdunkelten und bankrotten Laden. Professor Quirrell hatte die Erklärung beendet, und in einer Hand hielt Harry einen winzigen Holzzweig, der der Schlüssel sein würde. Dies war er, dies war der Tag und der Moment, an dem Harry anfing, die Rolle zu spielen. Sein erstes echtes Abenteuer, ein Kerker, in den eingedrungen werden musste, eine böse Regierung, der man trotzen musste, eine Jungfer in Nöten, die gerettet werden musste. Harry hätte ängstlicher und widerwilliger sein sollen, aber stattdessen fühlte er nur, dass es an der Zeit war, zu den Menschen zu werden, über die er in seinen Büchern gelesen hatte; seine Reise zu dem zu beginnen, wovon er immer gewusst hatte, dass er dazu bestimmt war, ein Held zu sein. Den ersten Schritt auf dem Weg zu machen, der zu Kimball Kinnison und Captain Picard und Liono von Thundera und ganz sicher \emph{nicht} zu Raistlin Majere führte. Soweit Harrys Gehirn aus frühmorgendlichen Zeichentrickfilmen wusste, erlangte man, wenn man erwachsen wurde, erstaunliche Kräfte und rettete das Universum, das war was Harrys Gehirn bei Erwachsenen gesehen und als sein Vorbild für den Reifungsprozess übernommen hatte, und Harry wollte unbedingt anfangen, erwachsen zu werden.

Und wenn das Muster der Geschichte erforderte, dass der Held als Ergebnis seines ersten Abenteuers einen Teil seiner Unschuld verlor, dann schien es zumindest jetzt, in diesem immer noch unschuldigen Moment, für ihn an der Zeit zu sein, diesen Schmerz zu erfahren. Wie das Ablegen von Kleidungsstücken, die zu klein für ihn waren, oder wie der endgültige Aufstieg in die nächste Phase des Spiels, nachdem er elf Jahre lang auf Welt 3, Ebene 2 von Super Mario Brothers feststeckte.

Harry hatte genug Romane gelesen, um zu ahnen, dass er danach nicht mehr so enthusiastisch sein würde, also genoss er es, solange es andauerte.

Es gab ein knallendes Geräusch, als etwas in der Nähe von Harry disapparierte, und dann gab es keine Zeit mehr für heroisches Grübeln.

Harrys Hand zerbrach den kleinen Holzzweig.

Ein Haken riss regungslos an Harrys Unterleib, als der Portschlüssel betätigt wurde, und es fühlte sich diesmal wie ein viel stärkerer Zug an als bei den kleineren Transporten zwischen dem Hogwarts-Gelände und der Winkelgasse—

—und ließ ihn mitten in eine riesige Donnerwolke fallen, die sich auflöste, kalter Regen peitschte ihm übers Gesicht, das Wasser haftete an Harrys Brille und ließ ihn in einem Augenblick erblinden und verwandelte die Welt in ein verschwommenes Bild, während er sich den tobenden Wellen des Ozeans weit unten zu nähern begann.

Er war hoch, hoch, hoch über der leeren Nordsee angekommen.

Durch den Schock des heftigen Sturms ließ Harry fast den Besenstiel los, den ihm Professor Quirrell gegeben hatte, was keine gute Idee gewesen wäre. Es dauerte fast eine ganze Sekunde, bis Harry seinen Verstand wieder beisammenhatte und seinen Besenstiel mit einem leichten Schwung wieder hochholte.

„Ich bin hier“, sagte eine ungewohnte Stimme aus der leeren Luft über ihm; tief und rau, die Stimme des fahlen, schlaksigen, bärtigen Mannes, in den Professor Quirrell sich mit Vielsafttrank verwandelt hatte, bevor er sich und seinen Besenstiel desillusionierte.

„Ich bin hier“, sagte Harry unter dem Mantel der Unsichtbarkeit. Er selbst hatte keinen Vielsafttrank benutzt. Einen anderen Körper zu tragen, behinderte deine Magie, und Harry könnte sein gesamtes bisschen Magie benötigen; daher sah der Plan vor, dass Harry fast immer unsichtbar bleiben sollte, anstatt Vielsaft zu verwenden.

(Keiner von beiden hatte den Namen des anderen ausgesprochen. Du benutzt einfach zu keinem Zeitpunkt während einer illegalen Mission deinen Namen, selbst wenn du unsichtbar über einem anonymen Wasserfleck in der Nordsee schwebst. Du tust es einfach nicht. Das wäre dumm).

Während der Regen und der Wind um ihn herum heulten, hielt Harry den Besenstiel mit einer Hand vorsichtig im Griff, hob seinen Zauberstab in einem ebenso vorsichtigen Griff an und imperviuste seine Brille.

Dann, als die Gläser klar waren, sah sich Harry um.

Er war von Wind und Regen umgeben, mit etwas Glück waren es vielleicht fünf Grad Celsius; er hatte bereits einen Wärmezauber auf sich gewirkt, einfach weil er im Februar draußen war, aber der hielt den umhertreibenden kalten Tröpfchen nicht stand. Schlimmer als Schnee, durchnässte der Regen jede exponierte Fläche. Der Unsichtbarkeitsumhang machte dich unsichtbar, aber er \emph{bedeckte} dich nicht komplett, und das bedeutete, dass er dich nicht komplett vor Regen schützte. Harrys Gesicht war der vollen Wucht des stürmischen Wassers ausgesetzt, und es fuhr direkt in seinen Nacken und sickerte in sein Hemd, auch in die Ärmel seiner Robe und seine Hosenaufschläge und seine Schuhe, das Wasser nahm jedes Stück Stoff als Weg, um sich einzuschleichen.

„Hier entlang“, sagte die Vielsaft-Stimme, und ein grüner Lichtfunke leuchtete vor Harrys Besenstiel auf, und dann huschte er in eine Richtung, die Harry wie jede andere schien.

Durch den dichten Regen folgte Harry. Manchmal verlor er ihn, diesen kleinen grünen Funken, und jedes Mal rief Harry laut, und der Funke erschien ein paar Sekunden später wieder vor ihm.

Als Harry den Trick raus hatte, dem Funken zu folgen, beschleunigte dieser, und Harry brachte den Besenstiel auf Touren und folgte ihm. Der Regen peitschte ihn härter, er fühlte sich so, wie es sich anfühlen musste, ein Gesicht voller Schrotkugeln zu bekommen, aber seine Brille blieb klar und schützte seine Augen.

Nur wenige Minuten später erhaschte Harry bei voller Fahrt einen Blick durch den Regen, auf einen riesigen Schatten, der weit über das Wasser ragte.

Und er fühlte ein fernes, hohles Echo der Leere, das von dort ausstrahlte, wo der Tod wartete, und das Harrys Geist umspülte und sich um ihn herum teilte, wie eine Welle, die auf Stein bricht. Diesmal kannte Harry seinen Feind, und sein Wille war aus Stahl und allem Licht.

„Ich kann die Dementoren schon spüren“, sagte die rauhe Stimme des vielgesafteten Quirrells. „Das habe ich nicht erwartet, nicht so bald.“

„Denken Sie an die Sterne“, sagte Harry über ein fernes Donnergrollen hinweg. „Lassen Sie keine Wut in sich aufkommen, nichts Negatives, denken Sie einfach an die Sterne, wie es sich anfühlt, sich selbst zu vergessen und körperlos durchs All zu fallen. Halten Sie an diesem Gedanken fest wie an einer Okklumentik-Barriere in Ihrem ganzen Geist. Die Dementoren werden einige Schwierigkeiten haben, darüber hinaus zu gelangen“.

Es herrschte einen Moment lang Stille, dann: „Interessant.“

Der grüne Funke hob sich, und Harry neigte seinen Besenstiel leicht nach oben, um zu folgen, selbst als er sie in eine Nebelbank lenkte, eine Wolke, die tief auf dem Wasser schwebte.

Bald schwebten sie über und leicht schräg über dem riesigen dreiseitigen Metallgebäude, das weit unten aufragte. Das Stahldreieck war hohl, nicht massiv, es war ein Gebäude mit drei dicken massiven Wänden und ohne Zentrum. Die Auroren auf Wache waren im obersten Stockwerk und an der Südseite des Gebäudes untergebracht, wie Professor Quirrell gesagt hatte, geschützt durch ihre Patronus-Amulette. Der legale Eingang nach Askaban befand sich auf dem Dach der südwestlichen Ecke des Gebäudes. Den die beiden natürlich nicht benutzen würden. Stattdessen würden sie einen Korridor benutzen, der direkt unter der nördlichen Ecke des Gebäudes verlief. Professor Quirrell würde als erster hinuntergehen und direkt an der Nordspitze ein Loch in das Dach und seine Schutzzauber bohren und eine Illusion hinterlassen, um die Lücke zu überdecken.

Die Gefangenen wurden in der Seite des Gebäudes in Ebenen untergebracht, die ihren Verbrechen entsprachen. Und ganz unten, im äußersten Zentrum und in der Tiefe von Askaban, lag ein Nest mit mehr als hundert Dementoren. Gelegentlich wurde viel Schmutz hineingeworfen, um das Niveau zu halten, da die den Dementoren direkt ausgesetzte Materie in Schlamm und Nichts zerfiel…

„Warten Sie eine Minute“, sagte die raue Stimme, "folgen Sie mir schnell und fliegen Sie vorsichtig hindurch.

„Verstanden“, sagte Harry leise.

\emph{Der Funke verschwand, und Harry begann zu zählen, \emph{eintausendundeins, eintausendundzwei, eintausendunddrei, eintausend…}}

…\emph{eintausendeinundsechzig} , und Harry tauchte ab, wobei der Wind um ihn herum kreischte, während er auf die riesige Metallstruktur hinabtauchte, hinunter, wo er die Schatten des Todes spüren konnte, die auf ihn warteten, Licht einsaugten und Leere ausstrahlten, während die Metallstruktur immer größer und größer wurde. Schlicht und eigenschaftslos ragte die riesige graue Form auf, doch in der südwestlichen Ecke befand sich eine einzelne erhöhte, kastenartige Struktur. Die Nordecke war einfach leer, Professor Quirrells Loch war nicht zu erkennen.

Harry zog scharf nach oben, als er sich der Nordecke näherte, und gab sich damit mehr Sicherheitsspielraum, als er in Flugklassen für nötig befunden hätte, aber nicht zu viel. Sobald er zum Stillstand gekommen war, begann er, seinen Besenstiel langsam wieder abzusenken, in Richtung dessen, was wie das solide Dach der Spitze der Nordecke aussah.

Es war eine seltsame Erfahrung, unsichtbar durch das illusorische Dach hinabzusteigen, und dann fand sich Harry in einem Metallgang wieder, der mit einem schwach orangefarbenen Licht beleuchtet war - das, wie Harry nach einem erschreckten Blick erkannte, von einer altmodischen Mantelgaslampe kam…

… denn Magie würde versagen, nach einiger Zeit in Anwesenheit der Dementoren aufgesaugt sein.

Harry stieg von seinem Besen ab.

Der Sog der Leere war nun stärker, als er sich teilte und um Harry herumfloss, ohne ihn zu berühren. Sie waren weit entfernt, aber es waren viele, die Wunden in der Welt; Harry hätte mit geschlossenen Augen auf sie zeigen können.

„\emph{Rufe deinen Patronuss}“, zischte eine Schlange vom Boden, die in dem schwach orangefarbenen Licht mehr verfärbt als grün aussah.

Sogar in Parsel kamen Anzeichen von Stress durch. Harry war überrascht; Professor Quirrell hatte gesagt, dass Animagi in ihren Animagus-Formen viel weniger anfällig für Dementoren seien. (Harry nahm an aus demselben Grund aus dem Patronusse Tiere waren.) Wenn Professor Quirrell in seiner Schlangenform in so großer Bedrängnis war, was war dann mit ihm geschehen, während er in menschlicher Form war, die ihm erlaubte, seine Magie zu benutzen…?

Harrys Zauberstab hob sich bereits in seiner Hand.

Dies würde der Anfang sein.

Auch wenn es nur eine Person war, nur eine Person, die er aus der Dunkelheit retten konnte, auch wenn er noch nicht mächtig genug war, \emph{alle} Gefangenen Askabans in Sicherheit zu teleportieren und die dreieckige Hölle bis auf die Grundmauern niederzubrennen…

Trotzdem war es ein Start, es war ein Anfang, es war eine Anzahlung auf alles, was Harry mit seinem Leben erreichen wollte. Kein Warten mehr, kein Hoffen, kein bloßes Versprechen, es würde alles hier beginnen. Hier und \emph{jetzt}.

Harrys Zauberstab schnitt auf den Punkt herab, an dem die Dementoren weit unten warteten.

„\emph{Expecto Patronum!} “

Die glühende humanoide Figur erschien aufflammend. Es war nicht das sonnenhelle Ding, das es vorher gewesen war… wahrscheinlich, weil Harry sich nicht ganz davon abhalten konnte, an all die \emph{anderen} Gefangenen in ihren Zellen zu denken, an diejenigen, die zu retten er \emph{nicht} hier war.

Aber vielleicht war es so am besten. Harry müsste diesen Patronus noch eine Weile aufrechterhalten, und es wäre vielleicht besser, wenn er nicht ganz so hell wäre.

Bei diesem Gedanken dimmte sich der Patronus ein wenig weiter; und dann noch weiter, als Harry versuchte, etwas weniger Kraft hineinzulegen, bis schließlich die brillante humanoide Gestalt nur noch geringfügig heller leuchtete als der hellste Tierpatronus, und Harry fühlte, dass er sie nicht weiter dimmen konnte, ohne Gefahr zu laufen, sie ganz zu verlieren.

Und dann: „Ess isst stabil“, zischte Harry und begann, seinen Besenstiel in seinen Beutel zu stecken. Sein Zauberstab blieb in seiner Hand, und ein leichter, anhaltender Strom von ihm ersetzte die leichten Verluste seines Patronus.

Die Schlange verschwamm zur Form eines schlaksigen, fahlen Mannes, der Professor Quirrells Zauberstab in der einen und einen Besenstiel in der anderen Hand hielt. Der schmächtige Mann taumelte, als er wieder ins Leben zurückkehrte, und lehnte sich für einen Moment an die Wand.

„Gut gemacht, wenn auch vielleicht ein wenig langsam“, murmelte die rauhe Stimme. Professor Quirrells Trockenheit steckte darin, auch wenn es weder zur Stimme passte, ebenso wie der ernste Blick auf dem dickbärtigen Gesicht. „Ich spüre sie jetzt überhaupt nicht mehr.“

Einen Augenblick später verschwand der Besenstiel unter den Roben des Mannes. Dann erhob sich der Zauberstab des Mannes und klopfte auf seinen Kopf, und mit einem Geräusch wie eine knackende Eierschale verschwand er wieder.

In der Luft erblühte ein schwacher grüner Funke, und Harry, immer noch in den Mantel der Unsichtbarkeit gehüllt, folgte ihm nach.

Hätte man von außen zugesehen, hätte man nichts gesehen als einen kleinen grünen Funken, der durch die Luft schwebte, und einen glänzend silbernen Humanoiden, der ihm nachlief.

Sie gingen hinunter und hinunter und hinunter, vorbei an einer Gaslampe nach der anderen und an der gelegentlich riesigen Metalltür, und stiegen in einer scheinbar vollkommenen Stille nach Askaban hinab. Professor Quirrell hatte eine Art Barriere errichtet, durch die er hören konnte, was in der Nähe vor sich ging, aber es konnten keine Geräusche nach außen dringen, und keine Geräusche konnten Harry erreichen.

Harry war nicht ganz in der Lage gewesen, seinen Geist davon abzuhalten, sich zu fragen, \emph{warum} diese Stille herrschte, oder seinen Geist davon abzuhalten, die Antwort zu geben. Die Antwort hatte er bereits auf einer vorbewussten Ebene erkannt, die ihn dazu veranlasst hatte, vergeblich zu versuchen, nicht darüber nachzudenken.

Irgendwo hinter diesen riesigen Metalltüren schrien Menschen.

Die silberne humanoide Gestalt flackerte, wurde heller und dunkler, jedes Mal, wenn Harry daran dachte.

Harry war angewiesen worden, sich selbst mit einem Kopfblasenzauber zu verzaubern. Um zu verhindern, dass er etwas riechen konnte.

Der ganze Enthusiasmus und Heldentum hatten bereits nachgelassen, wie Harry es geahnt hatte, es dauerte selbst nach seinen Maßstäben nicht lange, der Prozess war abgeschlossen, als sie das erste Mal an einer dieser Metalltüren vorbei gingen. Jede Metalltür war mit einem riesigen Schloss verschlossen, einem Schloss aus einfachem, nicht-magischem Metall, das einen Hogwarts-Studenten im ersten Jahr nicht aufgehalten hätte - wenn man noch einen Zauberstab hätte, wenn man noch seine Magie hätte, was die Gefangenen nicht hatten. Diese Metalltüren waren nicht die Türen einzelner Zellen, hatte Professor Quirrell gesagt, jede einzelne öffnete sich zu einem Korridor, in dem sich eine Gruppe von Zellen befand. Irgendwie half das ein wenig, nicht daran zu denken, dass jede Tür direkt einem Gefangenen entsprach, der direkt hinter ihr wartete. Stattdessen könnte es \emph{mehr} als einen Gefangenen geben, was die emotionale Wirkung verminderte; genau wie die Studie die zeigte, dass Menschen mehr beitrugen, wenn ihnen gesagt wurde, dass ein bestimmter Geldbetrag erforderlich war, um das Leben eines Kindes zu retten, als wenn ihnen gesagt wurde, dass der gleiche Gesamtbetrag erforderlich war, um acht Kinder zu retten…

Harry fiel es immer schwerer, nicht daran zu denken, und jedes Mal, wenn er es tat, flackerte das Licht seines Patronus.

Sie kamen zu der Stelle, an der der Durchgang nach links abbog, an der Ecke des dreieckigen Gebäudes. Wieder einmal gab es absteigende Metalltreppen, eine weitere Treppe; wieder einmal gingen sie hinunter.

Bloße Mörder wurden nicht in die unterste Zelle gesteckt. Es gab immer einen niedrigeren Ort, wo man hingehen konnte, eine noch schlimmere Bestrafung, die man fürchten musste. Ganz gleich, wie tief man bereits gesunken war, die Regierung des magischen Britannien hatte noch eine Drohung gegen dich, wenn du etwas noch schlimmeres machst.

Aber Bellatrix Black war die Todesserin gewesen, die mehr Furcht auslöste als jeder andere außer Lord Voldemort selbst, eine schöne und tödliche Zauberin, die ihrem Meister absolut treu war; sie war, wenn so etwas möglich war, sogar noch sadistischer und bösartiger als Du-weißt-schon-wer, als ob sie versuchte, ihren Meister zu übertrumpfen…

… das war es, was die Welt von ihr wusste, was die Welt von ihr glaubte.

Doch davor, so hatte Professor Quirrell Harry erzählt, vor dem Debüt der schrecklichsten Dienerin des Dunklen Lords, hatte es in Slytherin ein Mädchen gegeben, das still gewesen sei, meistens für sich geblieben und niemandem etwas getan habe. Danach hatte man erfundene Geschichten über sie erzählt, wobei sich die Erinnerungen im Nachhinein verändert hatten (Harry kannte die Forschungen dazu gut). Aber zu dieser Zeit, als sie noch die Schule besuchte, war die talentierteste Hexe in Hogwarts als sanftmütiges Mädchen bekannt gewesen (hatte Professor Quirrell gesagt). Ihre wenigen Freunde waren überrascht gewesen, als sie sich den Todessern angeschlossen hatte, und noch mehr überrascht, dass sie so viel Dunkelheit hinter diesem traurigen, wehmütigen Lächeln versteckt hatte.

So war Bellatrix einst gewesen, die vielversprechendste Hexe ihrer eigenen Generation, bevor der Dunkle Lord sie stahl und zerbrach, sie zerschmetterte und umgestaltete und sie auf einer tieferen Ebene und mit dunkleren Künsten als jeder Imperius an sich band.

Zehn Jahre lang hatte Bellatrix dem Dunklen Lord gedient, tötete, wen er sie töten ließ, und folterte, wen er sie foltern ließ.

Und dann war der Dunkle Lord endlich besiegt worden.

Und Bellatrix' Albtraum ging weiter.

Irgendwo in Bellatrix könnte es etwas geben, das immer noch schrie, das die ganze Zeit geschrien hatte, etwas, das ein psychiatrischer Heiler zurückbringen könnte; oder auch nicht, Professor Quirrell konnte es nicht wissen. Aber so oder so, sie könnten…

… sie könnten sie wenigstens aus Askaban herausholen…

Bellatrix Black war in die unterste Ebene von Askaban gesteckt worden.

Harry konnte sich nicht vorstellen, was er sehen würde, wenn sie in ihre Zelle kämen. Bellatrix musste am Anfang fast keine Angst vor dem Tod gehabt haben, wenn sie überhaupt noch am Leben war.

Sie stiegen eine weitere Treppe hinunter und kamen dem Tod und Bellatrix so viel näher, dass das Klappern ihrer unsichtbaren Schuhe das einzige Geräusch war, das Harry hören konnte. Das schwache orangefarbene Licht, das von den Gaslichtern ausging, der schwache grüne Funke, der durch die Luft schwebte, die leuchtende Gestalt, die mit ihrem silbernen Licht folgte, von Zeit zu Zeit flackernd.

Nachdem sie viele Male hinabgestiegen waren, kamen sie rechtzeitig zu einem Korridor, der nicht in einer Treppe endete, und zu einer letzten Metalltür, und der grüne Funke hielt vor ihr an.

Harrys Herz hatte sich ein wenig beruhigt, als sie weit in die Tiefen von Askaban hinabstiegen, ohne dass etwas geschah. Aber jetzt hämmerte es wieder in seiner Brust. Sie waren ganz unten, und die Schatten des Todes waren ganz nah.

Ein weiches, metallisches Klicken kam von dem Schloss, als Professor Quirrell den Weg öffnete.

Harry holte tief Luft und erinnerte sich an alles, was Professor Quirrell ihm gesagt hatte. Der schwierige Teil würde nicht nur darin bestehen, die vorgetäuschte Persönlichkeit so weit zu bringen, dass sie Bellatrix Black selbst täuschen könnte, der schwierige Teil würde darin bestehen, gleichzeitig seinen Patronus am Laufen zu halten…

Der grüne Funke ging aus, und einen Augenblick später schimmerte eine meterhohe Schlange auf, die nicht mehr unsichtbar war.

Die Metalltür bewegte sich mit einem langsamen, knarrenden Geräusch, als Harry sie mit seiner unsichtbaren Hand schob, sie nur einen Spalt öffnete und durch sie hindurch schaute.

Er sah einen geraden Korridor, der in massivem Stein endete. Dort gab es kein Licht, außer dem, was von Harrys Patronus hereinschlich. Das war hell genug, um die äußeren Gitterstäbe der acht Zellen zu sehen, die in den Korridor eingelassen waren, aber er konnte das Innere nicht sehen; noch wichtiger war jedoch, dass er im Korridor selbst niemanden sah.

„\emph{Ich ssehe nichtss}“, zischte Harry.

Die Schlange huschte voraus und schlängelte sich schnell über den Boden.

Einen Augenblick später—

„\emph{Ssie isst allein}“, zischte die Schlange.

\emph{Bleib}, dachte Harry an seinen Patronus, der eine Position auf einer Seite der Tür einnahm, als ob er sie bewachen würde; und dann schob Harry die Tür weiter auf und folgte nach innen.

Die erste Zelle, die Harry betrachtete, enthielt eine ausgetrocknete Leiche, deren Haut grau und fleckig wurde, deren Fleisch stellenweise soweit verrottet war, dass die Knochen darunter frei lagen, keine Augen—

Harry schloss seine Augen. Das konnte er immer noch tun, er war immer noch unsichtbar, er verriet nichts, indem er die Augen schloss.

Er hatte es schon gewusst, er hatte es auf Seite sechs seines Transfigurationsbuches gelesen, dass du in Askaban bliebst waren, bis deine Haftzeit vorüber war. Wenn du vorher gestorben bist, behielten sie dich dort, bis sie deine Leiche freigaben. Wenn deine Haftzeit lebenslänglich war, ließen sie die Leiche einfach in der Zelle, bis die Zelle gebraucht wurde, und warfen deine Leiche dann in die Grube der Dementoren. Aber es war immer noch ein Schock zu sehen, dass es sich bei der Leiche um eine \emph{Person} handelte, die einfach dort \emph{gelassen} worden war—

Das Licht im Raum flackerte.

\emph{Ruhig}, dachte Harry in seinem Innersten. Es wäre nicht gut für Professor Quirrell, wenn sich dieser Patronus wegen seinen traurigen Gedanken verabschieden würde. So nahe bei den Dementoren könnte der Verteidigungsprofessor dort, wo er stand, einfach tot umfallen. \emph{Ruhig, Harry James Potter-Evans-Verres, ruhig!}

Mit diesem Gedanken öffnete Harry wieder die Augen, es gab keine Zeit zu verlieren.

Die zweite Zelle, in die er schaute, enthielt nur ein Skelett.

Und hinter den Gittern der dritten Zelle sah er Bellatrix Black.

Etwas Kostbares und Unersetzliches in Harry verdorrte wie trockenes Gras.

Man konnte erkennen, dass die Frau kein Skelett war, dass ihr Kopf kein Totenschädel war, denn die Textur der Haut unterschied sich noch immer von der Textur der Knochen, egal wie weiß und bleich sie geworden war, vom allein im Dunkeln warten. Entweder gaben sie ihr nicht viel zu essen, oder was sie aß wurde von den Schatten des Todes von ihr abgesaugt; denn ihre Augen schienen unter den Lidern zusammengeschrumpft, ihre Lippen sahen zu verschrumpelt aus, um sich über den Zähnen zu schließen. Die Farbe schien aus der schwarzen Kleidung, die sie im Gefängnis getragen hatte, ausgelaugt zu sein, als ob die Dementoren auch diese ausgelaugt hatten. Diese Kleidung war dazu bestimmt, gewagt auszusehen, und jetzt lag sie lose über einem Skelett und enthüllte verschrumpelte Haut.

\emph{Ich bin hier, um sie zu retten, ich bin hier, um sie zu retten, ich bin hier, um sie zu retten}, dachte Harry bei sich selbst, verzweifelt, immer und immer wieder mit einer Anstrengung wie bei Okklumentik, seinem Patronus befehlend, nicht fort zu gehen, zu bleiben und \emph{Bellatrix vor den Dementoren zu beschützen}—

In seinem Herzen, in seinem Innersten, hielt Harry an all seinem Mitleid und seinem Mitgefühl fest, an seinem Willen, sie aus der Dunkelheit zu retten; der silberne Glanz, der durch die offene Tür hereinkam, erhellte sich, selbst als er es dachte.

Und in einem anderen Teil von ihm, als ob er gerade einen anderen Teil seines Geistes eine Gewohnheit ausüben ließe, ohne ihr große Aufmerksamkeit zu schenken…

Ein kalter Ausdruck kam über Harrys Gesicht, unsichtbar unter der Kaputze.

„Hallo, meine liebe Bella“, sagte ein kühles Flüstern. „Hast du mich vermisst?“

