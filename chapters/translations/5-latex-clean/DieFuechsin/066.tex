

\hypertarget{selbstverwirklichung-teil-1}{% \section{34. Selbstverwirklichung, Teil 1}\label{selbstverwirklichung-teil-1}}

—\/-\/-\/-\/- Kapitel 66: Selbstverwirklichung, Teil 1 -\/-\/-\/-\/-

\emph{\emph{Zzögern} \emph{isst immer leicht,} \emph{sseltennützzlich}.}

So hatte es ihm der Verteidigungsprofessor gesagt; und während man über die Details des Sprichworts streiten konnte, verstand Harry die Schwächen der Ravenclaws gut genug, um zu wissen, dass man versuchen musste, seinen eigenen Spitzfindigkeiten letztlich auch zu \emph{beantworten}. Verlangten manche Pläne nach Warten? Ja, viele Pläne erforderten ein \emph{verzögertes Handeln;} aber das war nicht dasselbe wie das \emph{Zögern} sich zu \emph{entscheiden}. Nicht das Zögern, weil man den richtigen Zeitpunkt abpassen wollte, sondern das Zögern, weil man sich nicht entscheiden konnte - es gab keinen cleveren Plan, der das erforderte.

Brauchte man manchmal mehr Informationen, um sich zu entscheiden? Ja, aber das konnte auch zu einer Ausrede für das Zögern werden; und es wäre \emph{verlockend}, zu zögern, wenn man vor der Wahl zwischen zwei schmerzhaften Alternativen stand und das \emph{Nichtwählen} den psychischen Schmerz für eine Weile vermeiden würde. Man würde sich einfach eine Information heraussuchen, die man nicht so einfach herausbekommen konnte, und behaupten, dass man ohne diese Information unmöglich eine Entscheidung treffen könnte; das wäre die Ausrede. Obwohl, wenn man wusste, \emph{welche} Informationen man benötigte, \emph{wann} und \emph{wie} man diese Informationen erhalten würden und was man in Abhängigkeit von jeder möglichen Beobachtung \emph{tun} würde, dann wäre das als Ausrede fürs Zögern weniger verdächtig.

Wenn man \emph{nicht} nur zögerte, sollte man in der Lage sein, im \emph{Voraus} zu entscheiden, was man tun würde, sobald man die zusätzlichen Informationen hatte, die man angeblich benötigte.

Wenn der Dunkle Lord \emph{wirklich} da draußen wäre, wäre es dann klug, auf Professor Quirrells Plan einzugehen, jemanden zu beauftragen, sich für den Dunklen Lord auszugeben?

Nein. Eindeutig nein. Auf gar keinen Fall.

Und wenn Harry \emph{genau} wüsste, dass der Dunkle Lord \emph{nicht} wirklich da draußen war… In \emph{dem} Fall…

Das Büro des Verteidigungsprofessors war ein kleiner Raum, zumindest heute; es hatte sich verändert, seit Harry es das letzte Mal gesehen hatte, der Stein des Raumes war nun dunkler, polierter. Hinter dem Schreibtisch des Verteidigungsprofessors stand das einzige leere Bücherregal, das den Raum immer zierte, ein hohes Bücherregal, das fast vom Boden bis zur Decke reichte, mit sieben leeren Holzregalen. Harry hatte Professor Quirrell nur ein einziges Mal gesehen, wie er ein Buch aus diesen leeren Regalen nahm, und nie, wie er ein Buch zurückstellte.

Die grüne Schlange wiegte sich über der Sitzfläche des Stuhls hinter dem Schreibtisch des Verteidigungsprofessors, die lidlosen Augen begegneten Harry Blick auf Augenhöhe.

Sie waren jetzt mit zweiundzwanzig Zaubern geschützt, alle, die man in Hogwarts wirken konnte, ohne die Aufmerksamkeit des Schulleiters zu erregen.

\emph{„Nein“,} zischte Harry.

Die grüne Schlange legte den Kopf schief und neigte ihn leicht; die Geste vermittelte keine Emotion, jedenfalls nichts dass Harrys Parselmund-Talent hätte entschlüsseln können. \emph{„Warum nicht?} “, erwiderte die grüne Schlange.

„\emph{Zu} \emph{risskant}“, sagte Harry schlicht. Das stimmte, ob der Dunkle Lord nun da draußen war oder nicht. Der Zwang, sich im Voraus zu entscheiden, hatte ihn erkennen lassen, dass er die unbeantwortete Frage nur als Vorwand zum Zögern benutzt hatte; die vernünftige Entscheidung war so oder so dieselbe.

\emph{Einen Moment lang schienen die dunklen, bodenlosen Augen schwarz zu glänzen, einen Moment lang klaffte das geschuppte Maul auf, um die Reißzähne zu entblößen. „\emph{Ich denke, du} \emph{hasst} \emph{aus früheren} \emph{Missserfolgen} \emph{die} \emph{falsche Lektion gelernt, mein Junge. Meine Pläne} \emph{scheitern} \emph{üblicherweissse} \emph{nicht, und der letzte wäre einwandfrei verlaufen, wenn nicht deine eigene Dummheit gewesen wäre. Die richtige Lektion isst es,} \emph{Sschritten} \emph{zu folgen, die von älteren und weiseren} \emph{Sslytherin} \emph{für dich entworfen wurden,} \emph{zähme} \emph{deine wilden} \emph{Impulsse}.}“

„\emph{Lektion für mich ist,} \emph{nicht Pläne zu schmieden, bei den Mädchen-Freund mich für Böse oder Junge-Freund mich für dumm hält.}“ schnauzte Harry zurück. Er hatte eine etwas ausweichendere Antwort geplant, aber irgendwie waren ihm die Worte einfach herausgerutscht.

\emph{Der sssss-Laut, das von der Schlange kam, hörte Harry nicht als Worte, sondern nur als pure Wut. Einen Moment später: „\emph{Du hast ihnen} \emph{erzzählt} \emph{—}}“

\emph{„\emph{Natürlich nicht! Aber ich weiß,} \emph{wassssiessagen} \emph{würden}.}“

Es gab eine lange Pause, während der Schlangenkopf schwankte und Harry anstarrte; wieder kam keine erkennbare Emotion durch, und Harry fragte sich, für welche Probleme Professor Quirrell so lange zum Nachdenken brauchen könnte.

\emph{„\emph{Interesssiert} \emph{es dich ernsthaft, was die beiden denken? }“, ertönte das Zischen der Schlange. „\emph{Die beiden} \emph{sind} \emph{noch wahrhaft jung, nicht wie} \emph{du. Sie können keine} \emph{Erwachsenensachenabwägen}.}“

\emph{„\emph{Hätten besser gehandelt} \emph{als ich}“, zischte Harry. „\emph{Junge-Freund hätten nach geheimen Motiven gefragt, bevor er sich bereit erklärt hätte, Frau zu retten—} }“

\emph{\emph{„Gut}, \emph{dasss} \emph{du} \emph{dasss} \emph{jetzt verstehst}“, zischte die Schlange kalt. \emph{"Frag immer nach den Vorteilendeines Gegenübers. Lerne als} \emph{Nächsstess, immer auf} \emph{deinen} \emph{eigenen Vorteil zu achten. Wenn} \emph{dir} \emph{mein Plan} \emph{nicht schmeckt,} \emph{wass} \emph{isst dann deiner?}}

„\emph{Wenn} \emph{ess} \emph{notwendig ist - 6 Jahre in der Schule bleiben und lernen. Hogwarts scheint ein guter Ort zum Verweilen zu sein. Bücher, Freunde,} \emph{sseltsamess} \emph{aber} \emph{leckeressEsssen.}“ Harry wollte kichern, aber es gab keine Geste in Parseltongue für die Art von Lachen, die er ausdrücken wollte.

\emph{Die Augenhöhlen der Schlange schienen fast schwarz zu sein. „\emph{Das} \emph{lässst} \emph{sich} \emph{jetzt} \emph{leicht sagen.} \emph{Ssolche} \emph{wie du und ich, wir dulden keine Gefangenschaft. Du wirst die Geduld lange vor dem} \emph{ssiebten} \emph{Jahr verlieren, vielleicht} \emph{ssogar} \emph{vor dem Ende diesenJahress. Ich werde entsprechend planen.}}“

Und bevor Harry ein weiteres Wort in Parseltongue zischen konnte, saß die Menschengestalt von Professor Quirrell wieder auf seinem Stuhl. „Also, Mr~Potter“, sagte der Verteidigungsprofessor, seine Stimme so ruhig, als ob sie nichts Wichtiges besprochen hätten, als ob das ganze Gespräch gar nicht stattgefunden hätte, „ich habe gehört, dass Sie angefangen haben, sich im Duellieren zu üben. Nicht die wertlose Sorte mit \emph{Regeln}, hoffe ich?“

Hannah Abbott sah so entnervt aus, wie Hermine sie noch nie gesehen hatte (außer am Tag des Phönix, dem Tag, an dem Bellatrix Black entkommen war, was aber für niemanden zählen sollte). Das Hufflepuff-Mädchen war während des Abendessens an den Ravenclaw-Tisch gekommen, hatte Hermine auf die Schulter getippt und sie fast weggezerrt.

„Neville und Harry Potter lernen Duellieren bei Mr~Diggory!“ platzte Hannah heraus, sobald sie ein paar Schritte vom Tisch entfernt waren.

„Bei wem?“, fragte Hermine.

„\emph{Cedric} \emph{Diggory!}“, sagte Hannah. „Er ist der Kapitän unserer Quidditch-Mannschaft und General einer Armee, und er belegt \emph{alle} Wahlfächer und bekommt bessere Noten als alle anderen, und ich habe gehört, dass er in den Sommerferien von professionellen Tutoren das Duellieren lernt und einmal \emph{zwei} Siebtklässler besiegt hat, und sogar einige Lehrer nennen ihn den Super-Hufflepuff, und Professor Sprout sagt, wir sollten ihn alle emi…, äh, emidieren oder so ähnlich, und—“

Nachdem Hannah endlich nach Luft geschnappt hatte (die Liste ging noch eine Weile weiter), gelang es Hermine, sie zu unterbrechen.

„Sonnenschein-Soldat Abbott!“, sagte Hermine. „\emph{Beruhige} dich. Wir werden doch nicht gegen General Diggory kämpfen, oder? Sicher, Neville lernt etwas, um uns zu schlagen, aber wir können auch lernen—“

„\emph{Verstehst} du denn nicht? “, kreischte Hannah und erhob ihre Stimme viel lauter, als es nötig gewesen wäre, vor allem wenn sie das Gespräch vor all den anderen Ravenclaws geheim halten wollten. „Neville lernt nicht, um \emph{uns} zu schlagen! Er übt, damit er gegen \emph{Bellatrix Black} kämpfen kann! Sie werden uns zerfetzen wie ein Klatscher einen Stapel Pfannkuchen!“

Die Sonnenschein-Generalin warf ihrem Soldaten einen Blick zu. „Hör mal“, sagte Hermine, „ich glaube nicht, dass ein paar Wochen Training jemanden zu einem unbesiegbaren Kämpfer machen. Außerdem \emph{wissen} wir bereits, wie man mit unbesiegbaren Kämpfern umgeht. Wir werden das Feuer auf sie konzentrieren und sie werden genau wie Draco zu Boden gehen.“

Das Hufflepuff-Mädchen sah sie mit einer Mischung aus Bewunderung und Skepsis an. „Machst du dir gar keine, du weißt schon, \emph{Sorgen}? “

„Also, \emph{ehrlich}! “, sagte Hermine. Manchmal war es schwer, die einzige vernünftige Person im ganzen Schuljahr zu sein. „Hast du noch nie den Spruch gehört, dass das Einzige, was wir fürchten müssen, die Angst selbst ist?“

„\emph{Was?}“, sagte Hannah. „Das ist verrückt, was ist mit Lethifolds, die in der Dunkelheit lauern, und mit dem Imperius-Fluch, und schrecklichen Verwandlungsunfällen und—“

„Ich \emph{meine}“, sagte Hermine, und Verärgerung sickerte in ihre nun erhobene Stimme. Sie hörte solche Dinge schon \emph{die ganze Woche}, „wie wäre es, wenn wir warten, bis die Chaos-Legion uns tatsächlich \emph{vernichtet haben}, um so viel Angst vor ihnen zu haben, und \emph{hast du gerade} \emph{'Gryffindors' geflüstert?}“

Ein paar Augenblicke später ging Hermine zurück zu ihrem Platz am Tisch, wobei ein süßes Lächeln ihr junges Gesicht zierte; es war nicht der schrecklich kalte Blick von Harrys dunkler Seite, aber es war das furchterregendste Gesicht, das sie bewerkstelligen konnte.

Harry Potter \emph{würde} zu Grunde gehen.

„Das ist Wahnsinn“, keuchte Neville, mit der winzigen Menge an Atem, die er noch erübrigen konnte, weil er völlig außer Atem war.

„Das ist \emph{brillant}! “, meinte Cedric Diggory. Die Augen des Super-Hufflepuffs leuchteten vor manischem Enthusiasmus und glänzten wie der Schweiß auf seiner Stirn, als er mit den Füßen durch eine seiner wie ein Tanz anmutenden Duellstellungen stapfte. Seine normalerweise leichten Schritte hatten sich in schwerere Stampfbewegungen verwandelt, was vielleicht etwas mit den verwandelten Metallgewichten zu tun hatte, die sie alle an ihren Armen und Beinen befestigt und über die Brust geschnallt hatten. „Woher \emph{haben} Sie diese Ideen, Mr~Potter?“

„Ein seltsamer alter Laden… in Oxford… und ich werde dort nie… wieder einkaufen…“ \emph{Rumms}.

