

\hypertarget{humanismus-teil-3}{% \section{13. Humanismus, Teil 3}\label{humanismus-teil-3}}

-\/-\/-\/-\/- Kapitel 45: Humanismus, Teil 3 -\/-\/-\/-\/-

Fawkes' Lied klang langsam aus.

Harry setzte sich auf, von wo aus er im vertrockneten Wintergras gelegen hatte, Fawkes saß immer noch auf seiner Schulter.\\ Überall um ihn herum war Einatmen zu hören.\\ „Harry“, sagte Seamus mit schwankender Stimme, „bist du in Ordnung?"

Die Ruhe des Phönix war noch immer in ihm und die Wärme, von dort wo Fawkes saß. Wärme, die sich in ihm ausbreitete, und die Erinnerung an das Lied, das in der Gegenwart des Phönix noch lebendig war. Es waren schreckliche Dinge, die ihm passiert waren, schreckliche Gedanken, die durch ihn hindurchgegangen waren. Er hatte eine unmögliche Erinnerung wiedererlangt, trotz dessen der Dementor ihn dazu gebracht hatte, sie zu entweihen. Ein seltsames Wort hallte immer wieder in seinem Kopf wider. Und all das konnte für später aufgeschoben werden, während der Phönix noch rot und gold unter der untergehenden Sonne leuchtete.

Fawkes krächzte ihn an.\\ „Muss ich irgendwas tun?“ sagte Harry zu Fawkes. „Was?“.\\ Fawkes wippte mit dem Kopf in Richtung des Dementors.\\ Harry blickte auf das unsehbare Grauen noch immer in seinem Käfig, dann zurück zum Phönix, verwirrt.

„Mr. Potter?“, sagte Minerva McGonagalls Stimme hinter ihm. „\emph{Geht} es Ihnen gut?"\\ Harry stellte sich auf seine Füße und drehte sich um.

Minerva McGonagall schaute ihn sehr besorgt an, Albus Dumbledore neben ihr studierte ihn sorgfältig, Filius Flitwick schien ungeheuer erleichtert, und alle Schüler starrten ihn einfach nur an.

„Ich glaube schon, Professor McGonagall“, sagte Harry ruhig. Er hatte fast schon \emph{Minerva} gesagt, bevor er sich selbst gestoppt hatte. Zumindest solange Fawkes auf seiner Schulter war, ging es Harry gut; es könnte sein, dass er einen Moment nach Fawkes' Abflug zusammenbrach, aber irgendwie schienen solche Gedanken nicht wichtig zu sein. „Ich glaube, es geht mir gut.„

Es hätte Jubel oder erleichterte Seufzer oder so etwas geben sollen, aber niemand schien zu wissen, was er sagen sollte, überhaupt niemand.

Der Frieden des Phönix blieb bestehen.

Harry drehte sich um. „Hermine?“, sagte er.\\ Jeder, der auch nur einen Funken Romantik im Herzen hatte, hielt den Atem an.

„Ich weiß nicht, wie ich mich freundlich bedanken soll“, sagte Harry leise, „so wenig wie ich weiß, wie ich man sich entschuldigt. Ich kann nur sagen, falls du dich fragst, ob es das Richtige war, es war richtig."

Der Junge und das Mädchen sahen sich in die Augen.\\ „Entschuldigung“, sagte Harry. „Dafür was als nächstes passiert. Wenn ich etwas tun kann -"\\ „Nein“, antwortete Hermine. „Es gibt nichts. Aber es ist schon in Ordnung.“ Dann wandte sie sich von Harry ab und ging weg, in Richtung des Pfades, der zu den Toren von Hogwarts führte.

Einige Mädchen warfen Harry verwirrte Blicke zu und folgten ihr dann. Als sie gingen, hörte man wie die aufgeregten Fragen begannen.\\ Harry sah sie an, als sie gingen, drehte sich um und sah die anderen SchülerInnen an. Sie hatten ihn am Boden liegen sehen, schreiend, und...

Fawkes streichelte seine Wange, kurz.

...und das würde ihnen eines Tages helfen, zu verstehen, dass der Junge, der lebte, auch verletzt werden könnte, dass es ihm elend gehen könnte. Wenn sie selbst verletzt waren, erinnerten sie sich daran, wie Harry sich am Boden wand und wussten, dass ihr eigener Schmerz und ihre eigenen Probleme nicht bedeuteten, dass sie es nie zu etwas bringen würden. Hatte der Schulleiter das mit einberechnet, als er die anderen Schüler bleiben und zusehen ließ?

Harrys Augen kehrten fast geistesabwesend zu dem großen, zerfetzten Umhang zurück, und ohne sich wirklich bewusst zu sein, was er aussprach, sagte Harry: „Es sollte nicht existieren."

„Ah“, sagte eine trockene, präzise Stimme. „Ich dachte mir, dass Sie das sagen würden. Es tut mir sehr leid, Ihnen sagen zu müssen, Mr. Potter, dass Dementoren nicht getötet werden können. Viele haben es versucht."

„Wirklich?“ sagte Harry, immer noch geistesabwesend. „Was haben sie versucht?"

„Es gibt einen äußerst gefährlichen und zerstörerischen Zauber,“ sagte Professor Quirrell, „den ich hier nicht benennen werde; ein Zauber aus verfluchtem Feuer. Er ist das, was man benutzen würde, um ein so altes Instrument wie den Sprechenden Hut zu zerstören. Er hat keine Wirkung auf Dementoren. Sie sind unsterblich."

„Sie sind nicht unsterblich“, sagte der Schulleiter. Die Worte mild, der Blick scharf. „Sie besitzen kein ewiges Leben. Sie sind Wunden in der Welt, und ein Angriff auf eine Wunde macht sie nur größer."

„Hm“, sagte Harry. „Angenommen man würde sie in die Sonne werfen? Würden sie zerstört werden?"

„In die \emph{Sonne geworfen}?“ quietschte Professor Flitwick und sah aus, als wolle er ohnmächtig werden.

„Das scheint unwahrscheinlich, Mr. Potter“, sagte Professor Quirrell trocken. „Die Sonne ist doch sehr groß, ich bezweifle, dass der Dementor einen Effekt auf sie hätte. Aber das ist kein Test, den ich ausprobieren würde, Mr. Potter, nur für den Fall."

„Ich verstehe“, sagte Harry.

Fawkes krächzte ein letztes Mal, legte seine Flügel um Harrys Kopf und schwang sich dann von Harry auf. Er stürzte direkt auf den Dementor zu und schrie einen durchdringenden Schrei des Trotzes, der über das Feld hallte. Und bevor irgendjemand darauf reagieren konnte, gab es einen Feuerblitz und Fawkes war weg.

Der Frieden verblasste, ein wenig.\\ Die Wärme verblasste, ein wenig.\\ Harry atmete tief ein und ließ den Atem wieder heraus.\\ „Ja“, sagte Harry. „Noch lebendig."

Wieder diese Stille, wieder das Fehlen von Jubel; niemand schien zu wissen, wie er reagieren sollte -\\ „Es ist gut zu wissen, dass Sie vollständig genesen sind, Mr. Potter“, sagte Professor Quirrell entschieden, als ob er jede andere Möglichkeit verneinen wollte. „Nun, ich glaube, Miss Ransom war die Nächste?"

Das löste einen Streit aus, in dem Professor Quirrell Recht hatte und alle anderen Unrecht. Der Verteidigungsprofessor wies darauf hin, dass trotz der verständlichen Emotionen aller Beteiligten die Wahrscheinlichkeit, dass ein ähnliches Missgeschick bei einem anderen Studenten auftrat, an den Rand des Infinitesimalen grenzte; zumal sie nun wussten, wie sie Missgeschicke mit Zauberstäben vermieden. Und in der Zwischenzeit gab es andere Schüler, die ihre eigene beste Chance nutzen mussten, einen körperlichen Patronuszauber zu wirken, oder aber das Gefühl eines Dementors kennenlernen mussten, um zu fliehen und ihren eigenen Grad an Verletzlichkeit zu entdecken...

Am Ende stellte sich heraus, dass Dean Thomas und Ron Weasley von Gryffindor die einzigen waren, die noch bereit waren, sich in die Nähe des Dementors zu begeben, was die Argumentation vereinfachte.

Harry warf einen Blick in die Richtung des Dementors. Das Wort hallte wieder in seinem Kopf wider.

\emph{Also gut}, dachte Harry sich, \emph{wenn der Dementor ein Rätsel ist, was ist dann die Antwort?}

Und einfach so war es offensichtlich.

Harry sah sich den angelaufenen, leicht korrodierten Käfig an.

Er sah, was unter dem hohen, zerfetzten Umhang lag.

Das war es also.

Professor McGonagall kam und sprach mit Harry. Sie hatte das Schlimmste nicht gesehen, also war nur ein leichtes Glitzern in ihren Augen zu sehen. Harry sagte ihr, dass er danach mit ihr reden und eine Frage stellen müsse, die er eine Weile aufgeschoben hatte, aber das musste nicht sofort geschehen, wenn sie beschäftigt war. Da war etwas an ihrem Äußeren, das darauf hinwies, dass sie von etwas Wichtigem weggezogen worden war; und Harry bemerkte dies ihr gegenüber und sagte, dass sie sich ehrlich gesagt nicht schuldig fühlen müsse zu gehen. Das brachte ihm einen scharfen Blick ein, aber dann verließ sie sie, schnell, mit dem Versprechen, dass sie später miteinander reden würden.

Dean Thomas beschwor wieder seinen weißen Bären, sogar in der Gegenwart des Dementors; und Ron Weasley erschuf ein angemessenes Schild aus funkelndem Nebel. Damit endete der Tag, soweit es alle anderen betraf, und Professor Flitwick begann, die Studenten zurück nach Hogwarts zu treiben. Als klar war, dass Harry zurückbleiben wollte, schaute Professor Flitwick ihn fragend an; und Harry seinerseits warf einen bedeutenden Blick auf Dumbledore. Harry wusste nicht, was Professor Flitwick daraus machte, aber nach einem scharfen Blick der Warnung ging sein Hauslehrer weg.

Und so blieben nur Harry, Professor Quirrell, Schulleiter Dumbledore und ein Auror-Trio übrig.

Es wäre besser gewesen, das Trio vorher loszuwerden, aber Harry fiel kein guter Weg ein, das zu tun.

„Also gut“, sagte Auror Komodo, „bringen wir es zurück."

„Entschuldigung“, sagte Harry. „Ich würde es gern noch mal bei dem Dementor versuchen.„

-\/-\/-\/-\/-\/-\/-\/-\/-\/-\/-\/-\/-\/-\/-\/-\/-\/-\/-\/-\/-\/-\/-\/-\/-\/-\/-\/-\/-\/-\/-\/-\/-\/-\/-\/-\/-\/-\/-\/-\/-\/-\/-\/-\/-

Harrys Bitte stieß auf eine gewisse Ablehnung der „\emph{Du bist völlig verrückt}“-Sorte, obwohl nur Auror Butnaru das tatsächlich laut aussprach.

„Fawkes hat es mir befohlen“, sagte Harry.

Das hatte nicht alle Widerstände überwunden, trotz des schockierten Gesichtsausdrucks von Dumbledore. Die Diskussion ging weiter und begann, die Ränder des verbliebenen Phönixfriedens abzunutzen, was Harry, wenn auch nur ein wenig, ärgerte.

„Hört zu“, sagte Harry, „ich weiß, was ich vorhin falsch gemacht habe. Es gibt eine Art von Mensch, der eine andere Art von warmen und glücklichen Gedanken benutzen muss. Lasst es mich einfach versuchen, okay?"

Das hatte sich auch nicht als überzeugend erwiesen.

„Ich denke“, sagte Professor Quirrell schließlich und starrte Harry mit zusammengekniffenen Augen an, „wenn wir ihm nicht erlauben, dies unter Aufsicht zu tun, wird er sich vielleicht irgendwann davonschleichen und selbst nach einem Dementor suchen. Oder beschuldige ich Sie zu Unrecht, Mr. Potter?"

Daraufhin gab es eine entsetzte Pause. Es schien ein guter Zeitpunkt zu sein, seinen Trumpf auszuspielen.

„Es macht mir nichts aus, wenn der Schulleiter seinen eigenen Patronus aufrechterhält“, sagte Harry. \emph{Denn ich werde trotzdem in der Gegenwart eines Dementors sein, Patronus hin oder her.}

Daraufhin herrschte Verwirrung, sogar Professor Quirrell sah verwirrt aus; aber der Schulleiter willigte schließlich ein, da es unwahrscheinlich schien, dass Harry durch vier Patronusse hindurch verletzt werden könnte.

\emph{Wenn der Dementor dich nicht auf irgendeiner Ebene durch deinen Patronus hindurch erreichen könnte, Albus Dumbledore, würdest du keinen nackten Mann sehen, der schmerzhaft anzusehen ist...}

Harry hat es nicht laut ausgesprochen, aus offensichtlichen Gründen.

Und sie begannen, auf den Dementor zuzugehen.

„Schulleiter“, sagte Harry, „angenommen, die Ravenclaw-Tür würde Ihnen folgendes Rätsel stellen: Was ist der Kern eines Dementors? Was würden Sie sagen?"

„Furcht“, sagte der Schulleiter.

Es war ein ganz einfacher Fehler. Der Dementor näherte sich, und die Angst überkam dich. Die Angst tat weh, du fühltest, wie die Angst dich schwächte, du wolltest, dass die Angst verschwindet.

Es war natürlich, dass man dachte, die Angst sei das Problem.

Also waren sie zu dem Schluss gekommen, dass der Dementor ein Wesen aus purer Angst war, dass es nichts zu fürchten gab, außer der Angst selbst... dass der Dementor dir nicht wehtun konnte, wenn du keine Angst hattest...

Aber...

\emph{Was ist der Kern eines Dementors?}

\emph{Die Angst.}

\emph{Was ist so schrecklich, dass der Verstand sich weigert es zu sehen?}

\emph{Die Angst.}

\emph{Was ist unmöglich zu töten?}

\emph{Die Angst.}

...es passte nicht ganz, wenn man darüber nachgedachte.

Obwohl es klar genug war, warum Leute zögerten über die erste Antwort hinauszublicken.

Die Menschen \emph{verstanden} Angst.

Die Menschen wussten, was sie gegen die Angst \emph{tun} sollten.

Angesichts eines Dementors wäre es also nicht gerade tröstlich zu fragen: „Was ist, wenn die Angst nur eine Nebenwirkung und nicht das Hauptproblem ist?

Sie waren sehr nahe an den Käfig des Dementors herangekommen, der von vier Patronussen bewacht wurde, als die drei Auroren und Professor Quirrell scharf einatmeten. Die Gesichter aller drehten sich um, um den Dementor zu betrachten, und schienen zuzuhören; auf Auror Goryanofs Gesicht lag Entsetzen.

Dann hob Professor Quirrell seinen Kopf, das Gesicht hart und spuckte dem Dementor entgegen.

„Es mag es wohl nicht, wenn man ihm seine Beute wegnimmt“, sagte Dumbledore leise. „Nun ja. Wenn es notwendig wird, Quirinus, wird es in Hogwarts immer eine Zuflucht für dich geben."

„Was hat es gesagt?“, sagte Harry.

Jeder Kopf schwang herum, um ihn anzustarren.

„Du hast es nicht gehört...?“ fragte Dumbledore.

Harry schüttelte den Kopf.

„Es sagte zu mir“, sagte Professor Quirrell, „dass es mich kennen würde und mich eines Tages jagen wird, wo auch immer ich mich verstecken will.“ Sein Gesicht war starr und zeigte keine Angst.

„Ah“, sagte Harry. „Darüber würde ich mir keine Sorgen machen, Professor Quirrell.“ \emph{Es ist nicht so als ob Dementoren wirklich reden} \emph{ode} \emph{denken können, die} \emph{Struktur} \emph{die sie besitzen ist deinem eigenen Verstand und deinen Erwartungen entlehnt...}

Jetzt warfen ihm alle \emph{sehr} seltsame Blicke zu. Die Auroren sahen sich nervös an, den Dementor, Harry.

Und sie standen direkt vor dem Käfig des Dementors.

„Es sind Wunden in der Welt“, sagte Harry. „Das ist nur eine wilde Vermutung, aber ich vermute, derjenige, der das gesagt hat, war Godric Gryffindor."

„Ja...“, sagte Dumbledore. „Woher wusstest du das?"

\emph{Es ist ein weitverbreiteter Irrtum,} dachte Harry, \emph{dass die besten Rationalisten nach Ravenclaw einsortiert werden und keiner für andere Häuser} \emph{übrig bleibt. Das ist nicht wahr. Nach Ravenclaw sortiert zu sein, zeigt, dass deine stärkste Tugend Neugierde ist, dass du dich etwas fragst und die wahre Antwort wissen willst. Und das ist nicht die} einzige \emph{Tugend, die ein Rationalist braucht. Manchmal muss man hart an einem Problem arbeiten und eine Weile daran festhalten. Manchmal braucht man einen klugen Plan, um es herauszufinden. Und manchmal braucht man mehr als alles andere, um eine Antwort zu sehen, den Mut, sich ihr zu stellen...}

Harrys Blick glitt auf das, was unter dem Mantel lag, der Schrecken weitaus schlimmer als jede verwesende Mumie. Rowena Ravenclaw hätte es auch wissen können, denn das Rätsel war offensichtlich genug, sobald man es als Rätsel sah.

Und es war auch klar, warum die Patronusse Tiere waren. Die Tiere wussten es nicht, und so wurden sie vor der Angst beschützt.

Aber Harry wusste es und würde es immer wissen und niemals vergessen können. Er hatte versucht, sich selbst beizubringen, der Realität ins Auge zu sehen, ohne mit der Wimper zu zucken, und obwohl Harry diese Kunst noch nicht beherrschte, waren in seinem Kopf immer noch diese Rillen eingeprägt, der erlernte Reflex, dem schmerzhaften Gedanken \emph{entgegen z}u sehen, statt wegzuschauen. Harry würde nie in der Lage sein, zu vergessen, indem er warme, glückliche Gedanken hatte, und deshalb hatte der Zauber für ihn nicht funktioniert.

Also würde Harry einen warmen, glücklichen Gedanken denken, der \emph{nicht} von etwas anderem handelte.

Harry zog seinen Zauberstab, den Professor Flitwick ihm zurückgegeben hatte, und setzte seine Füße in die Anfangsstellung für den Patronuszauber.

In seinem Geist warf Harry die letzten Reste des Phönixfriedens ab, legte die Ruhe, den traumhaften Zustand beiseite, erinnerte sich stattdessen an Fawkes' durchdringenden Schrei und machte sich bereit für den Kampf. Er rief alle Teile und Elemente seiner selbst auf, um zu erwachen. Rief in sich selbst alle Kraft auf, auf die der Patronus-Zauber jemals zurückgreifen konnte, um sich in die richtige Stimmung für den letzten warmen und glücklichen Gedanken zu bringen; er erinnerte sich an alle hellen Dinge.

Die Bücher, die sein Vater ihm gekauft hatte.

Mamas Lächeln, als Harry ihr eine Muttertagskarte handgefertigt hatte, ein aufwendiges Ding, für das er ein halbes Pfund Elektronik-Ersatzteile aus der Garage verwendet hatte, um Lichter blinken zu lassen und ein kleines Liedchen zu piepen, und für dessen Herstellung er drei Tage gebraucht hatte.

Professor McGonagall wie sie ihm erzählte, dass seine Eltern gut gestorben waren, um ihn zu beschützen. So wie sie es getan hatten.

Die Erkenntnis, dass Hermine mit ihm Schritt halten und sogar schneller laufen würde, dass sie echte Rivalen und Freunde sein könnten.

Draco aus der Dunkelheit herauszulocken und zu beobachten, wie er sich langsam dem Licht näherte.

Neville und Seamus und Lavender und Dean und alle anderen, die zu ihm aufsahen, alle, für deren Schutz er gekämpft hätte, wenn Hogwarts durch irgendetwas bedroht würde.

Alles, was das Leben lebenswert machte.

Sein Zauberstab stieg in die Ausgangsposition für den Patronuszauber.

Harry dachte an die Sterne, das Bild, das den Dementor auch ohne Patronus fast aufgehalten hätte. Nur dieses Mal fügte Harry die fehlende Zutat hinzu, er hatte es zwar nie wirklich gesehen, aber er hatte die Bilder und das Video gesehen. Die Erde, strahlend blau und weiß mit reflektiertem Sonnenlicht, wie sie im Weltraum hing, inmitten der schwarzen Leere und den brillanten Lichtpunkten.

Sie gehörte dorthin, in dieses Bild, weil sie es war, was allem anderen seine Bedeutung gab. Die Erde war das, was die Sterne bedeutsam machte, sie zu mehr machte als unkontrollierte Fusionsreaktionen, denn es war die Erde, die eines Tages die Galaxie kolonisieren und das Versprechen des Nachthimmels erfüllen würde.

Würden sie immer noch von Dementoren geplagt werden, die Kinder ihrer Kindeskinder, die fernen Nachkommen der Menschheit, die von Stern zu Stern schritten? Nein. Natürlich nicht. Die Dementoren waren nur kleine Plagegeister, die sich im Licht dieses Versprechens in Nichts auflösen; nicht unsterblich, nicht unbesiegbar, nicht einmal annähernd. Man musste kleine Plagegeister ertragen, wenn man zu den wenigen Glücklichen und Unglücklichen gehörte, die auf der Erde geboren wurden; auf der Alten Erde wie man sich eines Tages daran erinnern würde. Auch das war ein Teil dessen, was es bedeutete, am Leben zu sein, wenn man zu der winzigen Handvoll fühlender Wesen gehörte, die am Anfang aller Dinge geboren wurden, bevor das intelligente Leben vollständig zu seiner Macht gekommen war. Dass die viel größere Zukunft davon abhing, was man hier und jetzt, in den frühesten Tagen der Dämmerung tat, als es noch so viel Dunkelheit zu bekämpfen gab und vorübergehende Plagegeister wie Dementoren.

Mama und Papa, die Freundschaft von Hermine und Dracos Reise, Neville und Seamus und Lavendel und Dean, der blaue Himmel und die strahlende Sonne und alle hellen Dinge, die Erde, die Sterne, das Versprechen, alles, was die Menschheit war und alles, was sie werden würde...

Auf dem Zauberstab bewegten sich Harrys Finger in ihre Ausgangsposition; er war nun bereit, die richtige Art von warmen und glücklichen Gedanken zu denken.

Und Harrys Augen starrten direkt auf das, was unter dem zerfetzten Umhang lag, sahen direkt auf das, was Dementor genannt worden war. Das Nichts, die Leere, das Loch im Universum, die Abwesenheit von Farbe und Raum, der offene Abfluss, durch den die Wärme aus der Welt strömte.

Die Angst, die es ausstrahlte, stahl alle glücklichen Gedanken weg, seine Nähe entzog dir Kraft und Stärke, sein Kuss würde alles zerstören, was du warst.

\emph{Ich kenne dich jetzt}, dachte Harry, als sein Zauberstab ein-, zwei-, drei- und viermal zuckte, als seine Finger genau über die richtigen Abstände glitten, \emph{ich verstehe dein Wesen, du symbolisierst den Tod, durch irgendein Gesetz der Magie bist du ein Schatten, den der Tod in die Welt wirft.}

\emph{Und den Tod werde ich nie akzeptieren.}\\ \emph{Es ist nur eine kindische Sache, über das die menschliche Spezies noch nicht hinausgewachsen ist.}\\ \emph{Und eines Tages...}\\ \emph{Werden wir darüber hinwegkommen...}\\ \emph{Und die Menschen werden sich nicht mehr verabschieden müssen...}

Der Zauberstab erhob sich und zeigte direkt auf den Dementor.

"\emph{EXPECTO PATRONUM!}"

Der Gedanke explodierte aus ihm wie ein brechender Damm, stürzte seinen Arm hinunter in den Zauberstab und brach als glühendes weißes Licht aus ihm hervor. Licht, das körperlich wurde, Form und Substanz annahm.

Eine Gestalt mit zwei Armen, zwei Beinen und einem Kopf, aufrecht stehend; das Tier \emph{Homo sapiens}, die Gestalt eines Menschen.

Immer heller und heller glühend, als Harry all seine Kraft in seinen Zauber fließen ließ, lodernd mit weißglühendem Licht, das heller als der verblassende Sonnenuntergang war, die Auroren und Professor Quirrell schirmten ihre Augen schockiert ab -

\emph{Und eines Tages, wenn sich die Nachkommen der Menschheit von Stern zu Stern ausgebreitet haben, werden sie den Kindern erst dann von der Geschichte der Alten Erde erzählen, wenn sie alt genug sind, sie zu ertragen; und wenn sie davon erfahren, werden sie darüber weinen, dass es so etwas wie den Tod jemals gegeben hat!}

Die Menschengestalt leuchtete jetzt noch brillanter als die Mittagssonne, so strahlend, dass Harry die Wärme auf seiner Haut spüren konnte; und Harry sandte seinen ganzen Trotz gegen den Schatten des Todes aus und öffnete alle Schleusen in seinem Inneren, um diese helle Gestalt noch heller und heller auflodern zu lassen.

\emph{Du bist nicht unbesiegbar, und eines Tages wird die menschliche Spezies dir ein Ende bereiten.}\\ \emph{Ich werde dich vernichten, wenn ich kann, durch die Macht des Geistes, der Magie und der Wissenschaft.}\\ \emph{Ich werde mich nicht in Angst vor dem Tod verkriechen, nicht solange ich eine Chance habe, zu gewinnen.}\\ \emph{Ich lasse nicht zu, dass der Tod mich berührt. Ich lasse den Tod nicht die berühren, die ich liebe.}\\ \emph{Und selbst wenn du mich erledigst, bevor ich dich erledigen kann,}\\ \emph{Wird ein anderer wird meinen Platz einnehmen, und wieder ein anderer,}\\ \emph{Bis die Wunde in der Welt endlich geheilt ist...}\\ Harry senkte seinen Zauberstab und die helle Gestalt eines Menschen verschwand.\\ Langsam atmete er aus.

Als ob er aus einem Traum erwachte, als ob er seine Augen nach dem Schlaf öffnete, bewegte sich Harrys Blick vom Käfig weg, er sah sich um und sah, dass alle ihn anstarrten.

Albus Dumbledore starrte ihn an.\\ Professor Quirrell starrte ihn an.\\ Das Auror-Trio starrte ihn an.\\ Sie starrten ihn an, als hätten sie gerade gesehen, wie er einen Dementor vernichtet hatte.

Der zerfetzte Umhang lag leer im Käfig.

