

\hypertarget{utilitaristische-priorituxe4ten}{% \section{16. Utilitaristische Prioritäten}\label{utilitaristische-priorituxe4ten}}

—\/-\/-\/-\/- Kapitel 48: Utilitaristische Prioritäten -\/-\/-\/-\/-

Es war Samstag, der erste Morgen im Februar, und am Ravenclaw-Tisch inspizierte ein Junge, der einen mit Gemüse befülltes Frühstückstablett trug, nervös seine Portionen auf die geringste Spur von Fleisch.

Es \emph{könnte} eine Überreaktion gewesen sein. Nachdem er den ursprünglichen Schock überwunden hatte, war Harrys gesunder Menschenverstand aufgewacht und stellte die Hypothese auf, dass „Parsel“ wahrscheinlich nur eine linguistische Benutzeroberfläche zur Kontrolle von Schlangen war…

… schließlich könnten Schlangen nicht \emph{wirklich} auf menschlicher Ebene intelligent sein, \emph{irgendjemand} hätte das inzwischen bemerkt. Die Kreaturen mit dem kleinsten Gehirn, von denen Harry je gehört hatte, die so etwas wie sprachliche Fähigkeiten hatten, waren die afrikanischen Graupapageien, die von Irene Pepperberg unterrichtet wurden. Und das war eine unstrukturierte Ursprache, bei einer Spezies, die komplexe Ehebruchspiele spielte und andere Papageien nachahmen musste. Während nach dem, woran sich Draco erinnern konnte, Schlangen zu Parselmündern in etwas, was sich nach normaler menschlicher Sprache anhörte - d.h. ~ausgewachsene rekursive syntaktische Grammatik. Es hatte einige \emph{Zeit} gekostet ~bis Hominiden diese entwickelt hatten, mit großen Gehirnen und einem starken sozialen Selektionsdruck. Schlangen hatten überhaupt nicht viel Gesellschaft, soweit Harry wusste. Und bei Tausenden und Abertausenden von verschiedenen Schlangenarten auf der ganzen Welt, wie konnten sie alle \emph{dieselbe} Version ihrer angeblichen Sprache „Parsel“ verwenden?

Natürlich war das alles nur gesunder Menschenverstand, an den Harry den Glauben völlig zu verlieren begann.

Aber Harry war sich sicher, dass er irgendwann im Fernsehen Schlangen zischen gehört hatte - schließlich wusste er \emph{irgendwoher} wie das klang - und \emph{das} hatte sich für ihn nicht wie eine Sprache angehört, was ihm viel beruhigender erschienen war…

… anfangs. Das Problem war, dass Draco auch behauptet hatte, dass Parselmünder Schlangen auf ausgedehnte komplexe Missionen schicken konnten. Und wenn das stimmte, dann mussten Parselmünder die Schlangen durch Gespräche mit ihnen \emph{längerfristig intelligent machen}. Im schlimmsten Fall würde das die Schlange sich ihrer selbst bewusst machen, wie Harry es versehentlich mit dem Sprechenden Hut getan hatte.

Und als Harry \emph{diese} Hypothese angeboten hatte, hatte Draco behauptet, er könne sich an eine Geschichte erinnern - Harry hoffte bei Cthulhu, dass \emph{diese eine} Geschichte nur ein Märchen sei, sie klang zumindest so, aber es \emph{gab} eine Geschichte - über Salazar Slytherin, der eine tapfere junge Viper auf eine Mission schickte, um \emph{Informationen von anderen Schlangen zu sammeln}.

Wenn eine Schlange, mit der ein Parselmund gesprochen hatte, \emph{andere} Schlangen durch Gespräche mit \emph{ihnen} sich ihrer selbst bewusst machen konnte, dann…

Dann…

Harry wusste nicht einmal, warum sein Verstand „dann… dann…“ machte, wenn er ganz genau wusste, wie die exponentielle Progression funktionierte, es war einfach der schiere moralische Horror, der ihn umhaute.

Und was wäre, wenn jemand einen solchen Zauber erfunden hätte, um mit Kühen zu sprechen?

Was wäre, wenn es Geflügelmäuler gäbe?

Oder was das betraf …

Harry erstarrte in plötzlicher Erkenntnis, gerade als er die Gabel voller Karotten in seinen Mund steckte.

\emph{Das konnte, konnte unmöglich wahr sein, sicherlich wäre kein Zauberer dumm genug, DAS zu tun…}

Und Harry wusste mit einem schrecklichen, bangen Gefühl, dass sie \emph{natürlich} so dumm sein würden. Salazar Slytherin hatte wahrscheinlich nie auch nur eine Sekunde über die moralischen Auswirkungen der Schlangenintelligenz nachgedacht, genau wie es Salazar nie in den Sinn gekommen war, dass \emph{Muggelgeborene}intelligent genug waren, um Persönlichkeitsrechte zu verdienen. Die meisten Menschen sahen moralische Probleme einfach nicht, es sei denn, jemand anderes wies sie darauf hin…

„Harry?“, sagte Terry von nebenan, und klang so, als hätte er Angst, dass er die Frage bereuen würde. „Warum starrst du so auf deine Gabel?“

„Ich fange an zu glauben, dass Magie illegal sein sollte“, sagte Harry. „Übrigens, hast du jemals Geschichten über Zauberer gehört, die mit Pflanzen sprechen konnten?“

Terry hatte von so etwas noch nie gehört.

Auch keiner der Ravenclaws aus dem siebten Jahr, die Harry gefragt hatte.

Und nun war Harry an seinen Platz zurückgekehrt, hatte sich aber noch nicht wieder hingesetzt und starrte mit verzweifeltem Blick auf seinen Gemüseteller. Er wurde immer hungriger, und später am Tag würde er Marys Zimmer für eines ihrer unglaublich schmackhaften Gerichte besuchen… Harry fand sich schmerzlich versucht, einfach zu seinen gestrigen Essgewohnheiten zurückzukehren und damit abzuschließen.

\emph{Du musst etwas essen}, sagte sein innerer Slytherin. \emph{Und es ist nicht} viel \emph{wahrscheinlicher, dass jemand seine Selbstwahrnehmung auf Geflügel geniest hat als auf Pflanzen, also, wenn man schon Lebensmittel mit fragwürdiger Empfindung isst, warum nicht die leckeren frittierten} \emph{Diracawl-Scheiben essen?}

\emph{Ich bin mir nicht ganz sicher, ob das korrekte, utilitaristische Logik ist,—}

\emph{Oh, du willst utilitaristische Logik? Eine Portion utilitaristische Logik kommt sofort: Selbst in dem unwahrscheinlichen Fall, dass irgendein Schwachkopf es geschafft} hat\emph{, den Hühnern ein Bewusstsein zu verleihen, ist es deine Forschung, die die besten Chancen hat, diese Tatsache zu entdecken und etwas dagegen zu unternehmen.}

\emph{Wenn du deine Arbeit sogar noch etwas schneller erledigen könntest, indem du} nicht \emph{an deiner Ernährung herumpfuschst, dann ist es, so kontraintuitiv es auch erscheinen mag, das Beste, was du tun kannst, um die größte Anzahl möglicherweise empfindungsfähiger Wer-weiß-was zu retten, keine Zeit mit wilden Vermutungen darüber zu verschwenden, was intelligent sein könnte. Es ist ja nicht so, dass die Hauselfen das Essen nicht schon vorbereitet hätten, unabhängig davon, was du auf deinen Teller legst.}

Harry überlegte dies einen Moment lang. Es war eine ziemlich verführerische Argumentation—

\emph{Gut!} sagte Slytherin. \emph{Ich bin froh, dass du jetzt siehst, dass es das moralischste ist, das Leben von fühlenden Wesen für deine eigene Bequemlichkeit zu opfern, um deine schrecklichen Gelüste zu stillen, für das kranke Vergnügen, sie mit den Zähnen zu zerreißen—}

\emph{Was?} dachte Harry empört. \emph{Auf welcher Seite stehst du eigentlich?}

Seine innere mentale Slytherin-Stimme war grimmig. \emph{Auch du wirst dir eines Tages die Doktrin zu eigen machen} \emph{…} \emph{dass der Zweck die Schnitzel} \emph{heiligt}. Es folgte ein geistiges Kichern.

Seit Harry anfing sich zu sorgen, dass auch Pflanzen empfindungsfähig sein könnten, hatten seine Nicht-Ravenclaw-Komponenten Schwierigkeiten, seine moralische Vorsicht ernst zu nehmen. Hufflepuff schrie jedes Mal \emph{Kannibalismus!}, wenn Harry versuchte, an irgendein Nahrungsmittel zu denken, und Gryffindor stellte sich vor, wie es schrie, während er es aß, selbst wenn es, sagen wir, ein Butterbrot war—

\emph{Kannibalismus!}

WAAAAHHHH ISS MICH NICHT—

\emph{Ignoriere die Schreie,} \emph{iss} \emph{es trotzdem! Es ist ein sicherer Ort, um deine Ethik im Dienste höherer Ziele zu kompromittieren, alle anderen finden es in Ordnung, Butterbrote zu essen, so dass du nicht deine übliche Rationalisierung über eine kleine Wahrscheinlichkeit eines großen Nachteils nutzen kannst, wenn du erwischt wirst—}

Harry seufzte auf und dachte: \emph{Solange du damit einverstanden bist, dass wir von riesigen Monstern gefressen werden, die nicht genügend untersucht haben, ob} wir \emph{empfindungsfähig sind.}

\emph{Damit kann ich leben}, sagte Slytherin. \emph{Sind alle anderen damit einverstanden?} (Inneres mentales nicken.) \emph{Toll, können wir jetzt wieder zu den frittierten} \emph{Diracawl-Scheiben übergehen?}

\emph{Nicht, bevor ich mehr darüber recherchiert habe, was empfindungsfähig ist und was nicht. Jetzt halt die Klappe.} Und Harry wandte sich standhaft von seinem Teller voller oh-so-verlockendem Gemüse ab, um zur Bibliothek zu gehen—

\emph{Iss einfach die Schüler}, sagte Hufflepuff. \emph{Es gibt keinen Zweifel daran, dass} sie \emph{empfindungsfähig sind.}

\emph{Du weißt, dass du es willst}, sagte Gryffindor. \emph{Ich wette, die jungen sind die schmackhaftesten.}

Harry begann sich zu fragen, ob der Dementor seine imaginären Persönlichkeiten irgendwie beschädigt hatte.

„\emph{Ganz ehrlich}“, sagte Hermine. Die Stimme des jungen Mädchens klang ein wenig bitter, als ihr Blick die Bücherregale der Kräuterkunde-Magazine in der Hogwarts-Bibliothek überflog. Harry hatte ihr eine Nachricht hinterlassen, mit der Bitte nach dem Frühstück, das Harry ausgelassen hatte, in die Bibliothek zu kommen; aber dann, als Harry das Thema des Tages vorgestellt hatte, schien sie etwas verwirrt zu sein. „Weißt du, was dein Problem ist, Harry? Du hast kein Gespür für Prioritäten. Eine Idee kommt dir in den Kopf und du rennst ihr einfach hinterher.“

„Ich habe ein \emph{großartiges} Gespür für Prioritäten“, sagte Harry. Er streckte seine Hand aus und griff nach \emph{Gerissenes Gemüse} von Casey McNamara und begann, die ersten Seiten durchzublättern und nach dem Inhaltsverzeichnis zu suchen. „Deshalb will ich herausfinden, ob Pflanzen sprechen können, \emph{bevor} ich meine Karotten esse.“

„Meinst du nicht, dass wir beide vielleicht \emph{wichtigere} Dinge haben, über die wir uns Sorgen machen müssen?“

\emph{Du klingst genau wie Draco}, dachte Harry, aber natürlich sprach er es nicht laut aus. Laut sagte er: „Was könnte wohl \emph{wichtiger} sein, als Pflanzen, die sich als empfindungsfähig erweisen?“

Neben ihm herrschte ein bedeutungsschwangeres Schweigen, während Harrys Augen das Inhaltsverzeichnis durchsuchten. Es gab tatsächlich ein Kapitel über Pflanzensprache, was Harrys Puls kurz aussetzen ließ; und dann begannen seine Hände schnell die Seiten umzublättern, um die entsprechende Seitenzahl aufzuschlagen.

„Es gibt Tage“, sagte Hermine Granger, „an denen ich wirklich und wahrhaftig, absolut keine Ahnung habe, was in deinem Kopf vorgeht“.

„Schau, es ist eine Frage der Multiplikation, okay? Es gibt \emph{viele} Pflanzen auf der Welt, wenn sie \emph{nicht} empfindungsfähig sind, dann sind sie nicht wichtig, aber wenn Pflanzen wie Leute sind, dann haben sie mehr moralisches Gewicht als alle Menschen auf der Welt zusammen. Natürlich erkennt dein Gehirn das nicht auf intuitiver Ebene, aber das liegt daran, dass das Gehirn nicht multiplitzieren kann. Wenn man beispielsweise drei verschiedene Gruppen kanadischer Haushalte fragt, wie viel sie zahlen würden, um zweitausend, zwanzigtausend oder zweihunderttausend Vögel vor dem Tod in den Öltümpeln zu retten, werden die drei Gruppen jeweils angeben, dass sie bereit sind, achtundsiebzig, achtundachtzig und achtzig Dollar zu zahlen. Mit anderen Worten: kein Unterschied. Man nennt das “Missachtung des Maßstabs". Dein Gehirn stellt sich einen einzelnen Vogel vor, der sich in einer Ölpfütze quält, und dieses Bild erzeugt ein gewisses Maß an Emotionen, das deine Zahlungsbereitschaft bestimmt. Aber niemand kann sich auch nur zweitausend von irgendetwas vorstellen, also wird die \emph{Menge} einfach aus dem Fenster geworfen.

Versuche nun, diese Verzerrung in Bezug auf \emph{hundert Billionen} empfindungsfähiger Grashalme zu \emph{korrigieren}, und du wirst erkennen, dass dies tausendmal wichtiger sein könnte, als die gesamte menschliche Spezies wie wir früher dachten… oh dank sei Azathoth, hier steht, dass es nur Alraunen sind, die sprechen können und sie sprechen die normale menschliche Sprache laut aus, nicht dass es einen Zauber gibt, mit dem man mit \emph{jeder} Pflanze sprechen kann -".

„Ron kam gestern Morgen beim Frühstück zu mir“, sagte Hermine. Nun klang ihre Stimme ein bisschen leise, ein wenig traurig, vielleicht sogar ein wenig verängstigt. "Er sagte, er sei schrecklich schockiert gewesen, als er sah, wie ich dich küsste. Das, was du gesagt hast, als du dementiert warst, hätte mir zeigen sollen, wie viel Böses du in dir versteckst. Und dass, wenn ich einem dunklen Zauberer folgen würde, er sich nicht sicher wäre, ob er noch in meiner Armee sein wollte.

Harrys Hände hatten aufgehört, umzublättern. Es schien, dass Harrys Gehirn trotz seines abstrakten Wissens immer noch unfähig war, Tragweite auf einer echten emotionalen Ebene zu erkennen, weil es seine Aufmerksamkeit gerade zwangsweise von den Billionen möglicherweise empfindungsfähiger Grashalme, die vielleicht sogar während ihres Gesprächs litten oder starben, auf das Leben eines einzelnen Menschen gelenkt hatte, der ihm zufällig näher und lieber war.

„Ron ist der gigantischste Trottel der Welt“, sagte Harry. „Das werden sie in nächster Zeit nicht in der Zeitung drucken, weil es keine Nachrichten sind. Wie viele seiner Arme und Beine hast du ihm gebrochen nachdem du ihn gefeurt hast ?“

„Ich habe versucht, ihm zu sagen, dass es nicht so war“, fuhr Hermine mit derselben leisen Stimme fort. „Ich habe versucht, ihm zu sagen, dass \emph{du} nicht so bist und dass es zwischen uns beiden nicht so ist, aber es schien ihn nur noch mehr… so zu machen, wie er war.“

„Nun, ja“, sagte Harry. Er war überrascht, dass er nicht wütender auf Captain Weasley war, aber seine Besorgnis um Hermine schien das vorerst zu überwiegen. „Je mehr man versucht, sich vor solchen Leuten zu rechtfertigen, desto mehr erkennt man es an, dass sie das \emph{Recht} haben, dich in Frage zu stellen. Es zeigt, dass man glaubt, sie könnten dein Inquisitor sein, und sobald du jemandem diese Art von Macht über dich zugestehst, drängen sie dich einfach immer mehr.“ Dies war eine von Draco Malfoys Lektionen, die Harry eigentlich für ziemlich klug gehalten hatte: Leute, die sich zu verteidigen \emph{versuchten}, wurden über jede Kleinigkeit befragt und konnten ihre Befrager nie zufrieden stellen; aber wenn man von Anfang an klar machte, dass man eine Berühmtheit ist und über den gesellschaftlichen Konventionen steht, würde sich der Verstand der Leute überhaupt nicht die Mühe machen, die meisten Verstöße zu verfolgen. „Als Ron zu \emph{mir} kam, als ich mich am Ravenclaw-Tisch setzte und mir sagte, ich solle mich von dir fernhalten, hielt ich deshalb meine Hand über den Boden und sagte: `Siehst du, wie hoch ich meine Hand halte? Deine Intelligenz muss mindestens so hoch sein, um mit mir zu sprechen.' Dann beschuldigte er mich, Zitat: “dich in die Dunkelheit zu saugen„, Zitat Ende, und ich schürzte meine Lippen und machte \emph{schlüüüüürf}, und danach machte sein Mund immer noch diese Sprechgeräusche, so dass ich einen Stillezauber sprach. Ich glaube nicht, dass er seine Vorträge an mir noch einmal versuchen wird.“

„Ich verstehe, warum du das getan hast“, sagte Hermine mit fester Stimme, "ich \emph{wollte} ihn auch verpetzen, aber ich wünschte wirklich, du hättest es nicht getan, das macht die Sache für \emph{mich} noch schwieriger, Harry!

Harry schaute wieder von \emph{Gerissenes Gemüse} auf, er konnte so nicht ordentlich lesen; und er sah, dass Hermine immer noch das Buch las, das sie hatte, und nicht zu ihm aufschaute. Ihre Hände blätterten noch eine weitere Seite um, während er zusah.

„Ich glaube, du gehst den falschen Weg, indem du versuchst, dich überhaupt zu verteidigen“, sagte Harry. „Das glaube ich wirklich. Du bist, wer du bist. Du freundest dich mit wem auch immer du willst an. Sag jedem, der dich fragt, er soll sich das sonst wo hinschieben.“

Hermine schüttelte nur den Kopf und blätterte eine weitere Seite um.

„Option zwei“, sagte Harry. „Geh zu Fred und George und sag ihnen, sie sollen mit ihrem launischen Bruder reden, \emph{die} beiden sind wirklich gute Jungs—“

„Es ist nicht nur Ron“, sagte Hermine fast flüsternd. "Viele Leute sagen es, Harry. Sogar Mandy wirft mir besorgte Blicke zu, wenn sie denkt, dass ich nicht hinsehe. Ist das nicht komisch? Ich mache mir ständig Sorgen, dass Professor Quirrell \emph{dich} in die Dunkelheit zieht, und jetzt warnen mich die Leute genau so, wie ich versuche, dich zu warnen.

„Nun, \emph{ja}“, sagte Harry. „Beruhigt dich das nicht ein wenig bei mir und Professor Quirrell?“

„Mit einem Wort“, sagte Hermine, „nein.“

Es herrschte eine Stille, die lange genug dauerte, damit Hermine eine weitere Seite umblättern konnte, und dann kam ihre Stimme, diesmal im richtigen Flüsterton: „Und, und Padma geht herum und erzählt allen, dass ich, da ich den P-Patronus-Zauber nicht aussprechen konnte, nur vorgebe, n-nett zu sein…“

„Padma hat es nicht einmal selbst \emph{versucht}!“ sagte Harry empört. „Wenn du eine dunkle Hexe \emph{wärst}, die nur so tut, hättest du es nicht vor allen Leuten \emph{versucht}, halten die dich für \emph{dumm}? “

Hermine lächelte ein wenig und blinzelte ein paar Mal.

„Hey, \emph{ich} muss mir Sorgen machen, ob ich \emph{wirklich} böse werde. Hierbei halten dich die Leute im schlimmsten Fall für böser, als du wirklich bist. Wird dich das umbringen? ~Ich meine, ist das alles \emph{so} schlimm?“

Das junge Mädchen nickte, ihr Mund war fest zusammengepresst.

„Hör zu, Hermine… wenn du dich so sehr darum sorgst, was andere Leute denken, wenn du unglücklich bist, wenn andere Leute dich nicht genauso sehen, wie du dich selbst siehst, dann verdammt \emph{bereits} das dich dazu, immer unglücklich zu sein. Niemand denkt je genauso über uns, wie wir über uns selbst denken“.

„Ich weiß nicht, wie ich dir das erklären soll“, sagte Hermine mit traurig leiser Stimme. „Ich bin nicht sicher, ob du das je verstehen könntest, Harry. Mir fällt nur ein, wie würdest du dich fühlen, wenn \emph{ich} dich für böse halten würde?“

„Ähm…“ Harry stellte es sich vor. „Ja, das \emph{würde} weh tun. Sehr weh. Aber du bist ein guter Mensch, der über solche Dinge intelligent nachdenkt, du hast dir diese Macht über mich \emph{verdient}, es würde etwas \emph{bedeuten}, wenn du dächtest, dass ich etwas falsch gemacht hätte. Mir fällt außer dir kein einziger anderer Schüler ein, dessen Meinung mir genauso am Herzen läge—“

„Du kannst so leben“, flüsterte Hermine Granger. „Ich kann es nicht.“

Das Mädchen war schweigend weitere drei Seiten durchgegangen, und Harry hatte seine Augen wieder auf sein eigenes Buch gerichtet und versuchte, sich wieder zu konzentrieren, als Hermine schließlich mit leiser Stimme sagte: "Bist du wirklich sicher, dass ich nicht wissen darf, wie man den Patronuszauber ausübt?

„I…“ Harry musste einen plötzlichen Kloß im Hals runterschlucken. Er sah sich plötzlich selbst \emph{nicht} wissend, warum der Patronuszauber bei ihm nicht wirkte, \emph{nicht} in der Lage, es Draco zu zeigen, einfach gesagt zu bekommen, dass es einen Grund dafür gab, und mehr nicht. „Hermine, dein Patronus würde im selben Licht leuchten, aber er wäre nicht \emph{normal}, er würde nicht so aussehen, wie die Leute denken, dass Patronusse aussehen sollten, jeder, der ihn sieht, würde wissen, dass etwas Seltsames vor sich geht. Selbst wenn ich dir das Geheimnis verraten würde, könntest du das niemandem \emph{zeigen}, es sei denn, du würdest sie in die andere Richtung schauen lassen, damit sie nur das Licht sehen könnten, und… und der wichtigste Teil eines jeden Geheimnisses ist das Wissen, dass ein Geheimnis existiert, du könntest es nur ein oder zwei Freunden zeigen, wenn du sie zur Geheimhaltung verpflichtest…“. Harrys Stimme verstummte hilflos.

„Ich nehme es.“ Ihre Stimme war noch leise.

Es war sehr schwer, das Geheimnis nicht einfach herauszuposaunen, direkt hier in der Bibliothek.

„Ich, ich sollte nicht, ich sollte \emph{wirklich} nicht, es ist \emph{gefährlich}, Hermine, es könnte viel Schaden anrichten, wenn das Geheimnis nach außen dringen würde! Kennst du nicht die Redensart, dass drei ein Geheimnis bewahren können, wenn zwei tot sind? Dass es das gleiche ist, es seinen Freunden zu erzählen, wie es allen zu sagen, denn man vertraut nicht nur ihnen, sondern allen, denen sie vertrauen? Es ist zu wichtig, zu riskant, es ist nicht die Art von Entscheidung, die man treffen sollte, um den Ruf von jemandem in der Schule zu reparieren!“

„Okay“, sagte Hermine. Sie schloss das Buch und stellte es wieder ins Regal. „Ich kann mich jetzt nicht konzentrieren, Harry, tut mir leid.“

„Wenn ich \emph{irgendetwas} anderes tun kann,—“

„Sei netter zu allen.“

Das Mädchen schaute nicht zurück, als sie zwischen den Magazinen heraustrat, was vielleicht gut gewesen wäre, denn der Junge war unbeweglich an Ort und Stelle eingefroren.

Nach einer Weile begann der Junge wieder Seiten umzublättern.

