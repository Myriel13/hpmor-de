

\hypertarget{das-stanford-prison-experiment-teil-8-eingeschruxe4nkte-wahrnehmung}{% \section{26. Das Stanford-Prison-Experiment, Teil 8, Eingeschränkte Wahrnehmung}\label{das-stanford-prison-experiment-teil-8-eingeschruxe4nkte-wahrnehmung}}

-\/-\/-\/-\/- Kapitel 58: Das Stanford-Prison-Experiment, Teil 8, Eingeschränkte Wahrnehmung -\/-\/-\/-\/-

Anmerkung der Übersetzerin: Ein Filmtrailer für \emph{Armee der Finsternis}, der dem ähnelt, den Harry gesehen hat, ist Ls6\_fdzRpKw auf YouTube.

Das Schlüsselzitat lautet wie folgt, gesprochen von einem Mann der Neuzeit an Zuhörer aus dem Mittelalter:

"Okay ihr stumpfsinnigen Blechköpfe, jetzt hört mal zu. Seht ihr das hier? DAS ist mein \emph{Zauberstab}*!„

(*Im Original: „boomstick“, was sich so ähnlich anhört wie „Broomstick“, also Besen.

Da mir nichts Adäquates im Deutschen einfällt, entfällt dieses Wortspiel leider in der Übersetzung.)

-\/-\/-\/-\/-\/-\/-\/-\/-\/-\/-\/-\/-\/-\/-\/-\/-\/-\/-\/-\/-\/-\/-\/-\/-\/-\/-\/-\/-\/-\/-\/-\/-\/-\/-\/-\/-\/-\/-\/-\/-\/-\/-\/-\/-\/-\/-\/-\/-\/-\/-\/-\/-\/-\/-\/-\/-\/-\/-\/-\/-\/-\/-\/-\/-\/-\/-\/-\/-\/-\/-\/-\/-\/-\/-\/-\/-\/-\/-\/-\/-\/-\/-\/-\/-\/-\/-\/-\/-\/-\/-\/-\/-\/-\/-\/-\/-\/-\/-\/-\/-\/-\/-\/-\/-\/-\/-\/-\/-\/-\/-\/-\/-\/-\/-\/-\/-\/-\/-\/-\/-\/-\/-\/-\/-\/-\/-\/-\/-\/-\/-\/-\/-\/-\/-\/-\/-\/-\/-\/-\/-\/-\/-\/-\/-\/-\/-\/-\/-\/-\/-\/-\/-\/-\/-\/-\/-\/-

In absoluter Dunkelheit stand ein Junge, der seinen Zauberstab an die massive Metallwand von Askaban hielt und einen Zauber ausführte, den nur drei andere Menschen auf der Welt für möglich gehalten hätten und den keiner außer ihm allein ausüben konnte.

Natürlich hätte ein mächtiger Zauberer die Wand in Sekundenschnelle durchschneiden können, mit einer Geste und einem Wort.

Für einen durchschnittlichen Erwachsenen hätte es vielleicht nur ein paar Minuten gedauert, und danach wäre er außer Atem gewesen.

Aber um dasselbe als Hogwarts-Schüler im ersten Jahr zu erreichen, musste man \emph{effizient} sein.

Zum Glück - nun, nicht zum \emph{Glück}, Glück hatte damit nichts zu tun - Harry hatte \emph{gewissenhaft} jeden Tag eine Stunde lang Verwandlungen geübt, bis zu dem Punkt, an dem er sogar Hermine in dieser einen Klasse übertraf; er hatte teilweise Verwandlungen geübt, bis zu dem Punkt, an dem seine Gedanken begonnen hatten, das wahre Universum als selbstverständlich zu nehmen, so dass es nur noch etwas mehr Anstrengung erforderte, seine zeitlose Quantennatur im Gedächtnis zu behalten, selbst wenn er eine feste mentale Trennung zwischen dem Konzept der Form und dem Konzept der Substanz bewahrte.

Und das \emph{Problem}, dass diese Kunst so zur Routine geworden war...

...war, dass Harry dabei an andere Dinge denken konnte.

Irgendwie hatten es seine Gedanken geschafft, nicht dorthin zu gehen, sich nicht mit dem Offensichtlichen zu konfrontieren, bis er vor der Aussicht stand, \emph{es in nur wenigen Minuten wirklich zu tun}.

Was Harry im Begriff war zu tun...

...war gefährlich.

Wirklich gefährlich.

Jemand-könnte-tatsächlich-wirklich-getötet-werden gefährlich.

Zwölf Dementoren ohne Patronuszauber zu besiegen, war \emph{beängstigend}, aber nur beängstigend. Harry hätte den Patronuszauber wirken können, \emph{hätte} ihn gewirkt, sobald er dachte, dass er Gefahr lief, dies nicht mehr tun zu können, sobald er spürte, dass sein Widerstand zu versagen begann. Und selbst wenn das nicht funktioniert hätte... außer, wenn die Dementoren angewiesen worden wären, jeden zu küssen, den sie fanden, hätte das Scheitern nicht \emph{tödlich} sein dürfen.

Dies war anders.

Das verwandelte Muggelgerät konnte explodieren und sie töten.

Die Schnittstelle zwischen der Technologie und der Magie könnte auf vielerlei Weise versagen und sie töten.

Die Auroren könnten einen Glückstreffer landen.

Es war einfach, na ja...

\emph{Ernsthaft} gefährlich.

Harry hatte seinen Verstand erwischt, als er versuchte, sich selbst weiszumachen, dass es sicher sei.

Und sicher, das Ganze \emph{könnte} funktionieren, aber...

Aber selbst, wenn man außer Acht ließe, dass Rationalisten sich niemals selbst Dinge schönreden durften, wusste Harry, dass er sich unmöglich selbst dazu gebracht haben konnte, mit einer Wahrscheinlichkeit von weniger als 20\% zu schätzen, dass er sterben würde.

\emph{Verliere}, sagte Hufflepuff.

\emph{Verliere}, sagte die Stimme von Professor Quirrell in seinem Kopf.

\emph{Verliere}“, sagten sein mentales Modell von Hermine und Professor McGonagall und Professor Flitwick und Neville Longbottom und, nun ja, im Grunde genommen allen, die Harry kannte, außer Fred und George, die ohne zu zögern mitgemacht hätten.

Er sollte einfach zu Dumbledore gehen und sich stellen. Er sollte, er sollte es wirklich, es war das einzig \emph{Vernünftige}, was er zu diesem Zeitpunkt tun konnte.

Und wenn nur Harry auf dieser Mission gewesen wäre, wenn nur sein eigenes Leben auf dem Spiel gestanden hätte, hätte er es getan; er hätte es sicherlich getan.

Der Teil, der ihn fast seine Konzentration bei der teilweisen Verwandlung, die er ausführte, verlieren ließ, der Teil, der ihn für die Dementoren zu öffnen drohte...

...war Professor Quirrell, immer noch bewusstlos, immer noch eine Schlange.

Wenn Professor Quirrell wegen seiner Rolle bei der Flucht nach Askaban gehen würde, würde er sterben. Er würde wahrscheinlich nicht einmal eine Woche überleben. So empfindlich war er.

So einfach war das.

Wenn Harry hier \emph{verlieren} würde...

...verlor er Professor Quirrell.

\emph{\emph{Auch wenn er wahrscheinlich böse ist}, sagte der Hufflepuff-Teil von ihm leise. \emph{Auch dann?}}

Es war keine Entscheidung, die Harry auf irgendeiner bewussten Ebene getroffen hatte. Er konnte es einfach nicht tun. Verlieren war für Hauspunkte, nicht bei \emph{Menschen}.

\emph{Wenn du glaubst, dein eigenes Leben sei wertvoll genug, dass du nicht bereit bist, eine achtzigprozentige Sterbewahrscheinlichkeit auf dich zu nehmen, um alle Gefangenen in Askaban zu schützen}, beobachtete seine Slytherin-Seite, \emph{kannstdu} \emph{es nicht rechtfertigen, ein zwanzigprozentiges Risiko für} \emph{dein} \emph{Leben einzugehen, um Bellatrix und Professor Quirrell zu retten. Die Rechnung geht nicht auf, man kann hier keinen} \emph{konsistentenNutzen} \emph{über Ergebnisse stellen}.

Die logische Seite von ihm bemerkte, dass Slytherin das Argument gerade gewonnen hatte.

Harry behielt die Form im Gedächtnis und hielt weiterhin den Zauberspruch aufrecht. Er konnte die Mission jederzeit einfach abbrechen, wenn er mit der Verwandlung \emph{fertig} war, er wollte nicht dass die Mühe, die er bereits investiert hatte, umsonst war.

Und dann dachte Harry an etwas anderes, das es ihm plötzlich sehr schwer machte, die Magie am Laufen zu halten, sehr schwer, seinen Widerstand gegen die Dementoren aufrechtzuerhalten.

\emph{\emph{Was ist, wenn der Portschlüssel uns nicht dorthin bringt, wo Professor Quirrell gesagt hat?}}

Rückblickend war es in dem Moment, als er darüber nachdachte, offensichtlich.

Selbst wenn die geplante Flucht völlig richtig verlief, selbst wenn das Muggelgerät funktionierte und \emph{nicht} explodierte und die Interaktion mit dem angepassten magischen Gegenstand \emph{nicht} schief ging, selbst wenn die Auroren keinen Glückstreffer landeten, selbst wenn Harry weit genug von Askaban entfernt war, um den Portschlüssel zu benutzen...

... könnte es am Ende vielleicht gar keinen psychiatrischen Heiler geben.

Das hatte Harry geglaubt, als er Professor Quirrell vertraut hatte, und er hatte vergessen, es neu zu bewerten, nachdem er Professor Quirrell nicht mehr trauen konnte.

\emph{\emph{Das} \emph{kannst du} \emph{nicht tun,} sagte Hufflepuff\emph{. An diesem Punkt sprechen wir von reiner Dummheit.}}

Die Kälte schien sich im Raum auszubreiten, aber Harry hielt die Verwandlung aufrecht, selbst als sein Widerstand gegen die Dementoren ins Schwanken geriet.

\emph{\emph{Ich darf Professor Quirrell nicht verlieren.}}

\emph{Er hat versucht, einen Polizisten zu töten}, sagte Hufflepuff. \emph{Du hast ihn in diesem Moment bereits verloren. Bellatrix ist wahrscheinlich genau das, was alle von ihr denken. Nimm einfach deinen Umhang zurück, geh zu Dumbledore und sag ihm, du wurdest reingelegt.}

\emph{\emph{Nein}, dachte Harry verzweifelt, \emph{nicht ohne mit Professor Quirrell zu sprechen, es könnte eine Erklärung geben, ich weiß nicht, vielleicht stand er weit genug von meinem Patronus entfernt, dass die Dementoren ihn erwischt haben... Ich verstehe das nicht, es ergibt bei} \emph{keiner} \emph{Hypothese} \emph{Sinn, warum er das tun würde... Ich kann doch nicht einfach...}}

Harry wandte seinen Verstand von diesem Gedankengang ab, bevor er seinen Widerstand gegen die Angst vollständig brach, denn er konnte nicht daran denken, Professor Quirrell an die Dementoren zu verfüttern, während er entschlossen dem Tod widerstand, es war eine kognitive Unmöglichkeit.

\emph{Deine Argumentation ist künstlich beeinträchtigt}, beobachtete der logische Teil von ihm ruhig, \emph{finde einen Weg, wie sie nicht beeinträchtigt ist.}

\emph{In Ordnung lass uns einfach Alternativen} \emph{durchgehen}, dachte Harry. \emph{Nicht wählen, nicht abwägen, schon gar nicht festlegen... denk einfach darüber nach, was ich außer dem ursprünglichen Plan noch tun könnte.}

Und Harry fuhr fort, das Loch in die Wand zu schneiden. Er benutzte die partielle Verwandlung bei einem dünnen Zylinder aus Metall, zwei Meter im Durchmesser und einen halben Millimeter dick, der ganz durch die Wand verlief. Er verwandelte diese einen halben Millimeter dicke Metallschale in Motoröl. Motoröl war eine Flüssigkeit, und man durfte keine Flüssigkeiten verwandeln, weil sie verdampfen könnten, aber er, Bellatrix und die Schlange hatten alle Kopfblasenzauber. Und Harry zauberte sofort danach Finite auf das Öl und löste seine eigene Verwandlung auf...

...sobald der abgetrennte und geölte Metallklumpen aus der Wand auf den Boden ihrer Zelle glitt, er hatte ihn angeschrägt, so dass die Schwerkraft ihn nach der Verwandlung hineinziehen würde.

Wenn Harry und Bellatrix \emph{nicht} auf seinem Besenstiel durch das entstandene Loch in der Wand herausflogen...

Harrys Gehirn schlug vor, dass er versuchen könnte, eine Oberflächenabdeckung über dem Loch in der Wand zu verwandeln, so dass Bellatrix und Professor Quirrell mit dem Tarnumhang einen Platz zum Verstecken hätten, während Harry sich stellte. Und Professor Quirrell würde schließlich aufwachen, und er und Bellatrix könnten versuchen, herauszufinden, wie sie Askaban allein verlassen könnten.

Das war erstens eine dumme Idee, und zweitens würde immer noch ein riesiger Brocken Metall auf dem Boden der Zelle liegen, der es verraten würde.

Und dann sah Harrys Gehirn das Offensichtliche.

\emph{Lass Bellatrix und Professor Quirrell den von dir erfundenen Fluchtweg benutzen. Du bleibst zurück und stellst dich}.

Bellatrix und Professor Quirrell waren diejenigen, deren Leben auf dem Spiel stand.

Sie gewannen, und verloren nicht, weil sie das Risiko auf sich nahmen.

Und es gab keinen Grund, überhaupt keinen vernünftigen Grund, dass Harry mit ihnen ging.

Eine Ruhe überkam Harry, als er das dachte, die Kälte und die Dunkelheit, die an den Rändern seines Geistes genagt hatten, zogen sich zurück. Ja, das war es, das war der kreative Weg über den Tellerrand, das war die versteckte dritte Alternative. Die Falschheit des Dilemmas war im Rückblick offensichtlich. Wenn Harry sich stellte, musste er \emph{nicht} Bellatrix und Professor Quirrell anzeigen. Wenn Bellatrix und Professor Quirrell einen gefährlichen Fluchtweg einschlugen, brauchte Harry \emph{nicht} mit ihnen zu gehen.

Harry brauchte sich nicht einmal der Peinlichkeit zu stellen, zuzugeben, dass er ausgetrickst worden war, wenn er Bellatrix befahl, die Erinnerung zu entfernen. Jeder würde einfach annehmen, er sei entführt worden, auch Harry selbst. Zugegeben, es gab keinen plausiblen Grund, warum der Dunkle Lord Bellatrix jemals darum bitten sollte; aber Harry konnte einfach lächeln und Bellatrix sagen, dass sie es nicht wissen durfte, und das wäre es dann...

Ihr Auror-Team hatte etwa drei Viertel des Weges Askaban herunter zurückgelegt, ebenso wie die beiden anderen Teams auf den beiden anderen Spiralen. Amelia fühlte sich bereits angespannter, obwohl sie darauf setzte, dass die Kriminellen sich auf der zweitniedrigsten Etage versteckten. Ein Teil von ihr wünschte sich, Dumbledore hätte daran gedacht, diese bestimmte Etage sorgfältiger zu überprüfen, und ein Teil war froh, dass er das nicht getan hatte.

Und dann gab es ein entferntes Geräusch, wie ein winziges „Tink“-Geräusch, das von weit her kam. Wie ein sehr lautes Geräusch, das zum Beispiel aus dem zweituntersten Stockwerk kam.

Amelia schaute Dumbledore an, bevor sie es bemerkte, bevor sie es schaffte, sich selbst aufzuhalten.

Der alte Zauberer zuckte die Achseln, schenkte ihr ein kleines Lächeln und sagte: „Wenn du so fragst, Amelia“, und ging wieder weg.

„\emph{Finite} \emph{Incantatem}“, sagte Harry zu dem Öl, mit dem der riesige Metallklumpen auf dem Boden bedeckt war. Er hörte sich selbst kaum sprechen, seine Ohren klingelten immer noch von dem gigantischen Aufschlag des massiven Metalls, das aus der Wand glitt und herunterfiel. (Im Nachhinein betrachtet hätte er einen Stillezauber wirken sollen, obwohl das den Lärm nicht davon abgehalten hätte, sich durch den massiven Metallboden auszubreiten). Und dann sagte Harry noch einmal „\emph{Finite} \emph{Incantatem}“ zu dem Öl, mit dem das zwei Meter große Loch in der Wand überzogen war, und ließ den Effekt großflächig wirken; es war seine eigene Magie, die Harry aufhob, was den Zauber leicht machte. Harry fühlte sich jetzt ein wenig müde, aber das war der letzte Einsatz von Magie, den er brauchen würde. Er hatte es eigentlich gar nicht \emph{nötig} gehabt, aber Harry wollte die verwandelte Flüssigkeit nicht herumliegen lassen, und er wollte auch nicht das Geheimnis der teilweisen Verwandlung verraten.

Es schien sehr... \emph{einladend}, dieses zwei Meter große Loch, das in die Freiheit führte.

Das Licht, das von außen hereinkam,... war nicht gerade die Sonne, die auf sein Gesicht schien, aber es war heller als alles, was im Inneren Askabans zu sehen war.

Harry \emph{war} versucht, einfach mit Bellatrix und der Schlange auf den Besenstiel zu hüpfen. Die Chancen standen \emph{gut}, dass sie sicher herauskommen würden. Und \emph{wenn} sie sicher herauskämen und Harry mitkäme, dann könnten er und Professor Quirrell in der Zeit zurückgehen und völlig unschuldig aussehen, alles könnte wieder normal werden.

Wenn Harry zurückblieb und sich selbst stellte... dann, selbst wenn alle annahmen, Harry sei eine Geisel gewesen, angenommen, Harry habe Professor McGonagalls Patronus wegen eines auf ihn gerichteten Zauberstabes angelogen... selbst wenn Harry mit einem blauen Auge davonkam, nun...

Es war unwahrscheinlich, dass der Verteidigungsprofessor seine Lehrtätigkeit in Hogwarts fortsetzen würde.

Professor Quirrell hätte das prädestinierte Ende seiner Karriere erreicht, nämlich im Februar des Schuljahres.

Und ja, Professor McGonagall würde Harry töten, und ja, es würde langsam und schmerzhaft sein.

Aber zurückzubleiben war das Sinnvollste, Sicherste, \emph{Vernünftigste}, was er tun konnte, und Harry fühlte eher entspannt als bedauernd.

Harry wandte sich an Bellatrix; er öffnete den Mund, um sie ein letztes Mal anzuweisen -

Und da war ein Zischen, ein schwaches Zischen, ein Zischen, das langsam und verwirrt klang, und das Zischen sagte,

"Wass war... dieser Krach? „

Der alte Zauberer schritt durch den Korridor. Er kam zu einer Metalltür und öffnete sie, da er bereits aus dem Gedächtnis wusste, dass die Zellen darin leer waren.

Sieben mächtige und anspruchsvolle Beschwörungsformeln sprach der Zauberer dann, bevor er weiterging; es wäre insgesamt wenig anstrengend, da so wenige Zellen zur Überprüfung übrigblieben.

„\emph{Lehrer}“, zischte Harry. So viele Emotionen sprudelten in ihm hoch, alle auf einmal. Er wusste, obwohl er nicht sehen konnte, dass die grüne Schlange um Bellatrix' Schultern langsam den Kopf hob, um sich umzusehen. „\emph{Geht} \emph{ess} \emph{dir... gut, Lehrer?} „

„\emph{Lehrer?} “ kam das schwache, verwirrte Zischen. „\emph{Wo} \emph{ssind} \emph{wir hier?}“

„\emph{Gefängniss}“, zischte Harry, „\emph{das} \emph{Gefängniss} \emph{mit den Lebensfressern, wir wollten eine Frau retten, du und ich. Du} \emph{hasst} \emph{versucht einen Wächter zu töten, ich habe deinen tödlichen Fluch geblockt, es gab eine} \emph{Ressonanz} \emph{zwischen} \emph{unss... Du bist in Ohnmacht gefallen, ich musste den Wächter selbst besiegen... mein Beschützerzauber wurde aufgelöst, die Lebensfresser konnten den Wächtern sagen, dass die Frau entkommen war. Es gibt hier jemanden, der meinen Beschützerzauber erspüren kann, wahrscheinlich der Schulleiter... also musste ich meinen Beschützerzauber auflösen, einen anderen Weg finden, um dich und die Frau vor den} \emph{Lebensfressern} \emph{ohne Beschützerzauber zu verstecken, lernen, mich selbst ohne} \emph{Beschützerzauber zu schützen, die} \emph{Lebensfresser} \emph{ohne Beschützerzauber abschrecken, dann einen neuen Fluchtplan für dich und die Frau entwerfen und schließlich ein Loch in eine dicke Metallwand} \emph{einessGefängnissess} \emph{schneiden, obwohl ich Erstklässler bin. Keine Zeit für Erklärungen, ihr müsst jetzt gehen. Wenn wir} \emph{unss} \emph{nie wiedersehen, Lehrer, dann war ich eine Zeit lang froh, dich zu kennen, auch wenn du wahrscheinlich böse bist.} \emph{Ess} \emph{isst gut, die Gelegenheit zu haben,} \emph{diess} \emph{zu sagen: Auf Wiedersehen}."

Und Harry nahm den Besenstiel und überreichte ihn Bellatrix und sagte einfach: „Steig auf."

Er hatte beschlossen, die Erinnerungen zu behalten. Zum einen waren sie wichtig. Zum anderen hatten er und der Verteidigungsprofessor vor einer Woche mit der Planung begonnen, und Harry hatte nicht vor, die ganze letzte Woche auszulöschen \emph{oder} Bellatrix zu erklären, was genau vermieden werden musste. Harry könnte wahrscheinlich Veritaserum täuschen, und wenn Dumbledore darauf bestand, dass Harry seine Okklumentik-Schilde für eine eingehendere Untersuchung fallen ließ... nun, Harry hatte sich durchweg heldenhaft verhalten.

„\emph{Sstopp}! „sagte die Schlange. Ihre Stimme war jetzt stärker. „\emph{Sstopp,} \emph{sstopp,} \emph{sstopp,} \emph{sstopp!} \emph{Wass} \emph{meinst du mit „Auf Wiedersehen}“? „

„\emph{Der Fluchtplan isst} \emph{risskant}“, sagte Harry. „\emph{Mein Leben steht nicht auf dem Spiel, nur deins und ihrs. Also bleibe ich hier und stelle mich} -"

„\emph{Nein}!“ sagte die Schlange. Das Zischen war heftig. „\emph{Darfsst} \emph{nicht! Isst nicht erlaubt}! „

Bellatrix bestieg den Besenstiel; Harry konnte spüren (aber nie sehen), wie sich ihr Kopf drehte, um ihn anzusehen, sie sagte kein Wort. Vielleicht wartete sie auf ihn, oder sie wartete nur auf seine Befehle.

„\emph{Ich vertraue dir nicht mehr}“, sagte Harry einfach. „\emph{Nicht seit du versucht hast, den Wächter zu töten.}"

Und die Schlange zischte: „\emph{Ich habe nicht danach getrachtet, den Wächter zu töten! Bist du ein Narr, Junge? Ihn zu töten, würde keinen Sinn machen, böse oder nicht}! „

Die Erde hörte auf, sich um ihre Achse zu drehen und hielt in ihrer Umlaufbahn um die Sonne inne.

Das Zischen der Schlange war nun wütender als alles, was Harry je von dem menschlichen Professor Quirrell gehört hatte. „\emph{Ihn töten? Hätte ich verssucht, ihn zu töten, wäre er in} \emph{Ssekundenschnelle} \emph{tot gewesen, dummer Junge, er war mir nicht gewachsen! Ich wollte ihn unterwerfen, dominieren, ihn zwingen, Schilde um} \emph{sseinen} \emph{Verstand fallen zu lassen, ich musste ihn lesen, wissen, wer auf seine Antwort wartete,} \emph{Detailss} \emph{für den Gedächtniszauber lernen} -"

"\emph{Du sprachst den Todesfluch}! „

"\emph{Ich wusste, er würde ausweichen}! „

"\emph{War} \emph{sein Leben so wenig wert? Was wäre, wenn er nicht} \emph{ausgewichen wäre}? „

"\emph{Hätte ihn mit eigener Magie aus dem Weg geschubst, dummer Junge}! „

Wieder die Pause in der Drehung des Planeten. Daran hatte Harry nicht gedacht.

„\emph{Geistloser Schwachkopf von einem Verschwörer}“, zischte die Schlange so wütend, dass sich die Zischlaute zu überschneiden schienen und übereinander zu gleiten schienen, „\emph{kluger Dummkopf, listiger Idiot, Narr eines untrainierten} \emph{Sslytherin, dein} \emph{fehlplatziertess} \emph{Misstrauen hat alles ruiniert} -"

„\emph{Dies ist nicht der richtige Zeitpunkt, um zu streiten}“, bemerkte Harry milde. Die Welle der Erleichterung, die ihn zu durchfluten versuchte, wurde durch die erhöhte Spannung zunichte gemacht. „\emph{Da ich mich nicht richtig über dich ärgern kann, ohne mich für Lebensfresser zu öffnen. Wir müssen uns beeilen, vielleicht hat jemand den Lärm gehört} -"

„\emph{Erkläre den Fluchtplan}“, sagte die Schlange gebieterisch. „\emph{Rasch}! „

Harry erklärte. Parsel hatte keine Worte für die Muggel-Technologie, aber Harry beschrieb die Funktion und Professor Quirrell schien sie zu verstehen.

Es gab ein paar kurze Zischlaute, das schlangenartige Äquivalent eines überraschten Gelächters, und dann zackige Befehle. „\emph{Sag der Frau,} \emph{ssiessoll} \emph{wegschauen, zaubere Zauber der Stille,} \emph{ssetze} \emph{den Wächterzauber vor die Tür. Ich werde mich verwandeln, ein paar schnelle Verbesserungen an deiner Erfindung vornehmen, der Frau einen Notfalltrank geben, damit} \emph{ssie} \emph{uns beschützen kann, mich zurückverwandeln, bevor du den Zauber aufhebst. Der Plan wird dann} \emph{ssicherer} \emph{sein.}"

„\emph{Und soll ich glauben}“, zischte Harry, „\emph{dass wirklich ein Heiler für die Frau auf uns wartet}? „

"\emph{Nutze deinen Verstand, Junge! Nimm an, ich bin böse. Dich hier zu vernichten, ist offensichtlich nicht das, was ich vorhatte.} \emph{Misssion} \emph{ist das Ziel von Gelegenheit, erfunden, nachdem ich deinen Beschützerzauber gesehen habe, die ganze Angelegenheit sollte} \emph{unbemerkt bleiben, versteckt, als wir} \emph{den Essplatz verließen. Offensichtlich} \emph{wirst du} \emph{bei} \emph{unserer} \emph{Ankunft} \emph{eine} \emph{Persson sehen,} \emph{die} \emph{sich als Heiler ausgibt! Gehen danach zum Essplatz zurück, Originalplan} \emph{geht} \emph{ungestört weiter}! „

Harry starrte die unsichtbare Schlange an.

Einerseits fühlte sich Harry, wenn er das so sagte, ziemlich dumm.

Und auf der anderen Seite war es nicht gerade beruhigend.

„\emph{Also}“, zischte Harry, „\emph{wie} \emph{ssieht} \emph{dein Plan für mich aus,} \emph{ganzz} \emph{genau?} „

„\emph{Du hast gesagt keine Zeit}“, zischte die Schlange, „\emph{aber der Plan ist offensichtlich,} \emph{dasss} \emph{du} \emph{dass} \emph{Land regieren sollst, das hat sogar dein junger adliger Freund inzwischen verstanden, und wenn du} \emph{willsst, kannst du ihn bei der Rückkehr fragen. Will jetzt} \emph{nichtss} \emph{mehr} \emph{ssagen,} \emph{ess} \emph{ist} \emph{Zzeitzzu} \emph{fliegen, nicht} \emph{zzu} \emph{reden}.„

Der alte Zauberer streckte die Hand nach einer anderen Metalltür aus, hinter der ein endloses, totes Gemurmel erscholl: „Das ist nicht mein Ernst, das ist nicht mein Ernst, das ist nicht mein Ernst...“. Der rot-goldene Phönix auf seiner Schulter schrie bereits eindringlich, und der alte Zauberer zuckte bereits zusammen, als -

Ein weiterer Schrei durchdrang den Korridor, phönixartig, aber nicht der Ruf des wahren Phönix.

Der Kopf des Zauberers drehte sich um, blickte auf die glühende silberne Kreatur auf seiner anderen Schulter, selbst als flüchtige und substanzlose Krallen das Zauberwesen in die Luft beförderten.

Der falsche Phönix flog den Korridor hinunter.

Der alte Zauberer rannte hinterher, die Beine flogen wie bei einem rüstigen jungen Mann von sechzig Jahren.

Der wahre Phönix schrie einmal, zweimal und ein drittes Mal und schwebte vor der Metalltür; und dann, als klar wurde, dass sein Meister nicht wegen seiner Rufe zurückkehren würde, flog er widerwillig hinterher.

Professor Quirrell hatte diesmal seine wahre Gestalt angenommen - Vielsafttrank dauerte ohne eine neue Dosis nur eine Stunde an - und obwohl der Verteidigungsprofessor blass war und an den Metallgittern der nächstgelegenen Zelle lehnte, war seine Magie stark genug, um seinen Zauberstab ohne ein Wort zu ergreifen, selbst als Bellatrix den Umhang ablegte und ihn gehorsam in Harrys wartende Hand legte. Das Gefühl des Untergangs baute sich erneut auf, wenn auch nicht in voller Stärke, als die Macht des Verteidigungsprofessors zurückkehrte und die Ränder seiner gewaltigen Kraft mit Harrys leicht kindlicher Aura kollidierten.

Harry beschrieb sein Muggelgerät für den beobachtenden Zauberer, und dann verwandelte ein Finite von Harry all seine harte Arbeit wieder in einen Eiswürfel. Professor Quirrell konnte nichts verzaubern, das Harry verwandelt hatte, denn das wäre eine, wenn auch geringfügige, Wechselwirkung zwischen ihren Magien, aber --

Drei Sekunden später hielt Professor Quirrell seine eigene verwandelte Version des Muggelgeräts in der Hand. Ein einzelnes gebelltes Wort und ein Schwenken seines Zauberstabs, und der Kleberückstand war von dem magischen Gegenstand verschwunden; drei weitere Beschwörungsformeln später waren das Magische und das Technologische wie zu einem einzigen Ding verschmolzen, und die Zauber der Unzerbrechlichkeit und der makellosen Funktion waren auf das Muggelgerät geworfen worden.

(Harry fühlte sich viel besser dabei, dies unter Aufsicht eines Erwachsenen zu tun).

Bellatrix wurde ein Zaubertrank zugeworfen, und Professor Quirrell und Harry befahlen beide: „Trink“, als ob sie mit derselben Stimme sprechen würden. Die abgemagerte Frau hatte ihn bereits an ihre Lippen gehoben, ohne abzuwarten; denn es war für jedermann offensichtlich, dass dieser Schlangenanimagus ein Diener des Dunklen Lords war, und zwar ein mächtiger und vertrauenswürdiger.

Harry zog sich die Kapuze des Tarnumhangs über den Kopf.

Ein kurzer und schrecklicher Zauber ging aus dem Zauberstab des Verteidigungsprofessors hervor, der das Loch in der Wand verbrannte und den riesigen Metallklumpen in der Mitte des Raumes vernarbte; Harry hatte darum gebeten und gesagt, dass die von ihm angewandte Methode ihn identifizieren könnte.

„Linker Handschuh“, sagte Harry zu seinem Beutel, zog ihn heraus und zog ihn an.

Eine Geste des Verteidigungsprofessors ließ auf Bellatrix' Schultern ein Geschirr erscheinen und ein weiteres, kleineres Ding aus Stoff an ihrer Hand und so etwas wie Handschellen an ihren Handgelenken, gerade als die Frau den Trank ausgetrunken hatte.

Eine seltsame, ungesunde Farbe schien Bellatrix' blasses Gesicht zu überkommen, sie richtete sich auf, ihre eingesunkenen Augen schienen heller und viel gefährlicher...

...aus ihren Ohren strömten kleine Dampfschwaden...

(Harry beschloss, nicht über diesen Teil nachzudenken.)

...und Bellatrix Black lachte, ein plötzliches verrücktes Lachen, das viel zu laut inmitten der kleinen Gefängniszellen von Askaban ertönte.

(Sehr bald, so hatte der Verteidigungsprofessor gesagt, würde Bellatrix bewusstlos werden und eine ganze Weile so bleiben, der Preis für den Trank, den sie genommen hatte; aber für nur wenige Augenblicke würde sie vielleicht einen zwanzigstel Teil der Macht, die sie einst hatte, wiedererlangen).

Der Verteidigungsprofessor warf seinen Zauberstab in Richtung Bellatrix, und einen Augenblick später verschwamm er zu einer grünen Schlange.

Einen Augenblick \emph{danach} kehrte die Dementorenangst in den Raum zurück.

Bellatrix zuckte nur leicht zusammen, fing den Zauberstab auf und gestikulierte ohne ein Wort; die Schlange flog hoch und wurde in das Geschirr auf ihrem Rücken eingeführt.

Harry sagte „Hoch!“ zum Besenstiel.

Bellatrix befestigte den Zauberstab am Halfter an ihrer Hand.

Harry sprang in den Fahrersitz des Zwei-Personen-Besenstiels.

Bellatrix folgte ihm hinterher, nahm die manschettenartigen Vorrichtungen an ihren Handgelenken und kettete ihre Hände an die Griffe des Besenstiels, als Harrys rechte Hand den Zauberstab in seinen Beutel schob.

Und die drei schossen vorwärts durch das Loch in der Wand -

- hinaus unter freien Himmel, direkt über der Grube der Dementoren, im Inneren des riesigen dreieckigen Prismas, das Askaban war, der blaue Himmel, der jetzt deutlich über ihnen zu sehen war und sein Tageslicht nach unten strahlte.

Harry winkelte den Besenstiel an und begann zu beschleunigen, nach oben und zur Mitte des dreieckigen Raumes hin. Seine linke Hand, mit Handschuhen bekleidet, um den direkten Kontakt zwischen seiner Haut und etwas, das Professor Quirrell verwandelt hatte, zu verhindern, hielt den Schalter der Steuerung am Muggelgerät.

Weit über ihnen ertönten entfernte Rufe.

\emph{Okay ihr stumpfsinnigen Blechköpfe}!

Auroren auf schnell fliegenden Besenstielen, die aus dem Himmel geradewegs auf sie hinabstürzten, schwache Lichtfunken, die bereits nach unten flammten, als die ersten Schüsse abgegeben wurden.

\emph{\emph{Jetzt hört mal zu!}}

„Protego Maximus!“, rief Bellatrix mit mächtiger, spröder Stimme, gefolgt von einem gackernden Lachen, als ein schimmerndes blaues Feld sie umgab.

\emph{\emph{Seht ihr das hier?}}

Aus der verwesenden Grube im Zentrum Askabans erhoben sich über hundert Dementoren in die Luft, die aussahen wie eine große Masse von Leichen, wie ein fliegender Friedhof; einem anderen erschienen sie als ein Konglomerat von Leeren, die einen riesigen Riss in der Welt zu bilden schienen, während sie nach oben glitten.

\emph{Dies}...

Die Stimme eines alten und mächtigen Zauberers brüllte eine schreckliche Beschwörungsformel, und ein großer weißgoldener Feuerstoß schoss aus dem Loch in der Wand Askabans, der einen Moment lang unförmig war, bevor er begann, Flügel zu bilden.

\emph{Ist}...

Und die Auroren aktivierten den Anti-Anti-Schwerkraft-Zauber, der in die Schutzzauber Askabans eingebaut worden war, und deaktivierten alle Flugzauber, deren Zauber nicht mit der kürzlich geänderten Passphrase verzaubert worden waren.

Der Auftrieb von Harrys Besen schaltete sich aus.

Die Schwerkraft hingegen blieb an.

Der Aufstieg ihres Besens verlangsamte sich, verlangsamte sich und begann, sich in einen Sturz zu verwandeln.

\emph{Mein}...

Aber die Verzauberungen, die den Besen in eine Richtung zeigen ließen und das Lenken ermöglichten, die Verzauberungen, die die Reiter an ihm festhielten und sie ein wenig vor der Beschleunigung schützten, \emph{diese} Verzauberungen funktionierten immer noch.

\emph{\emph{BESEN!}}

Harry drückte den Zündschalter an der von General Technics hergestellten, Modell \emph{Berserker} \emph{PFRC}, N-Klasse, Ammoniumperchlorat-Verbundtreibstoff, Feststoffrakete, die mit seinem Nimbus X200 Zwei-Personen-Besenstiel zusammengebaut worden war.

Und es ward Lärm.

