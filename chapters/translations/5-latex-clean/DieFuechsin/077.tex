

\hypertarget{selbstverwirklichung-nachspiel-auuxdfenwirkung}{% \section{45. Selbstverwirklichung Nachspiel: Außenwirkung}\label{selbstverwirklichung-nachspiel-auuxdfenwirkung}}

-\/-\/-\/-\/- Kapitel 77: Selbstverwirklichung Nachspiel: Außenwirkung -\/-\/-\/-\/-

\emph{Nachspiel: Albus Dumbledore und -}

Der alte Zauberer saß allein an seinem Schreibtisch, in der Nicht-Stille des Schulleiterbüros, inmitten der unzähligen und unbemerkten Geräte; seine Roben waren in einem sanften Gelb gehalten, aus weichem Stoff, nicht wie er sie gewöhnlich vor anderen trug. In seiner faltigen Hand hielt er einen Federkiel, mit dem er auf einem offiziell aussehenden Pergament herumkratzte. Wenn du irgendwie dabei gewesen wärst, um sein gefurchtes Gesicht zu beobachten, hättest du nicht mehr über den Mann selbst herausfinden können, als du von den rätselhaften Geräten verstanden hättest. Du hättest vielleicht bemerkt, dass das Gesicht ein wenig traurig aussah, ein wenig müde, aber so sah Albus Dumbledore immer aus, wenn er allein war.

Im Floo-Kamin lag nur noch verstreute Asche ohne den Hauch einer Flamme, eine magische Tür, die so fest verschlossen worden war, dass sie nicht mehr existierte. Auf der materiellen Ebene war die große Eichentür zum Büro verschlossen und verriegelt worden; jenseits dieser Tür blieb die Endlose Treppe unbeweglich; die Gargoyles am Fuß dieser Treppe, die den Eingang versperrten, bewegten sich nicht, ihr Pseudoleben wurde ihnen entzogen und ließ nur festen Fels zurück.

Dann, gerade als die Feder ein Wort schreiben wollte, gerade als sie einen Buchstaben kratzen wollte -

Der alte Zauberer schoss mit einer Geschwindigkeit auf die Beine, die jeden Beobachter schockiert hätte, und ließ die Feder mitten im Schreiben auf das Pergament fallen; wie ein Blitz wirbelte er zu der Eichentür herum, seine gelben Gewänder wirbelten um ihn herum und ein Zauberstab von furchtbarer Macht sprang in seine Hand -

Und ebenso abrupt hielt der alte Zauberer inne, stoppte seine Bewegung grade als der Zauberstab zum Einsatz kam.

Eine Hand schlug gegen die Eichentür, klopfte dreimal.

Der grimme Zauberstab wanderte, langsamer jetzt, zurück in das Duellierholster, das unter dem Ärmel des alten Zauberers steckte. Der alte Mann trat ein paar Schritte vor, nahm eine förmlichere Haltung an und setzte ein gefasstes Gesicht auf. In der Nähe auf dem Schreibtisch bewegte sich der Federkiel auf die Seite des Pergaments, als wäre er dort sorgfältig platziert worden und nicht in Eile fallen gelassen worden; und das Pergament selbst klappte um und zeigte Leere.

Mit einem leisen Zucken seines Willens schwang die Eichentür auf.

Hart wie Stein starrten ihn die grünen Augen an.

"Ich gebe zu, dass ich beeindruckt bin, Harry", sagte der alte Zauberer leise. "Der Unsichtbarkeitsumhang hätte dich meinen schwächeren Warnmeldern entgehen lassen; aber ich habe weder gespürt, wie meine Golems zur Seite traten, noch wie sich die Treppe drehte. Wie bist du hierher gekommen?"

Der Junge ging in das Büro, einen Schritt nach dem anderen, bis sich die Tür sanft hinter ihm schloss. "Ich kann gehen, wohin ich will, mit oder ohne Erlaubnis", sagte der Junge. Seine Stimme wirkte ruhig; zu ruhig vielleicht. "Ich bin in Ihrem Büro, weil ich beschlossen habe, hier zu sein, und zum Teufel mit Passwörtern. Sie irren sich gewaltig, Schulleiter Dumbledore, wenn Sie glauben, dass ich in dieser Schule bleibe, weil ich hier ein Gefangener bin. Ich habe mich einfach\emph{noch}nicht entschieden, zu gehen. Wenn Sie das im Hinterkopf behalten, warum haben Sie dann Ihrem Agenten, Professor Snape, befohlen, die Vereinbarung zu brechen, die wir in diesem Büro getroffen haben, dass er keine Schüler im vierten Jahr oder darunter quälen darf?"

Der alte Zauberer sah den wütenden jungen Helden einen langen Moment lang an. Dann, langsam genug, um den Jungen nicht zu erschrecken, zogen diese verhutzelten Finger eine der vielen Schubladen des Schreibtisches auf, hoben ein Blatt Pergament heraus und legten es auf den Schreibtisch. "Vierzehn", sagte der alte Zauberer. "Das ist nicht die Zahl aller Eulen, die letzte Nacht geschickt wurden. Nur die Eulen, die an Familien mit einem Sitz im Zaubergamot geschickt wurden, oder Familien mit großem Reichtum, oder Familien, die bereits mit deinen Feinden verbündet sind. Oder, im Fall von Robert Jugson, alles drei. Denn sein Vater, Lord Jugson, ist ein Todesser. Und sein Großvater war ein Todesser, der durch Alastor Moodys Zauberstab starb. Was in den Briefen stand, weiß ich nicht, aber ich kann es mir denken. Hast du es\emph{immer}noch nicht verstanden, Harry Potter? Jedes Mal, wenn Hermine Granger\emph{gewonnen}hat, wie du sagst, ist die Gefahr für sie durch Slytherin wieder und wieder gewachsen. Aber jetzt haben die Slytherins über sie triumphiert, leicht und sicher, ohne Gewalt oder bleibenden Schaden. Sie haben gewonnen und brauchen nicht mehr zu kämpfen…" Der alte Zauberer seufzte. "So hatte ich es geplant. So hatte ich es gehofft. So wäre es auch gewesen, wenn der Verteidigungsprofessor sich nicht eingemischt hätte. Jetzt geht der Streit vor den Obersten Rat, wo Severus den Verteidigungsprofessor scheinbar besiegen wird; aber das wird sich für die Slytherins nicht so anfühlen, es wird nicht in einem Augenblick zu ihrer Zufriedenheit vorbei und beendet sein."

Der Junge trat weiter in den Raum vor, sein Kopf neigte sich weiter nach hinten, um zu den halbmondförmigen Gläsern hinaufzuschauen; und irgendwie war es, als würde der Junge eher auf den Schulleiter hinunterschauen als hinauf. "Dieser Lord Jugson ist also ein Todesser?", fragte der Junge leise. "Gut. Dann ist sein Leben bereits gekauft und bezahlt, und ich kann mit ihm machen, was ich will, ohne ethische Probleme -"

"\emph{Harry!}"

Die Stimme des Jungen war klar wie Eis, gefroren aus reinstem Wasser aus einer unberührten Quelle. "Sie scheinen zu denken, dass das Licht in Angst vor der Dunkelheit leben sollte. Ich sage, es sollte andersherum sein. Ich würde es vorziehen, diesen Lord Jugson nicht zu töten, selbst wenn er ein Todesser ist. Aber eine Stunde Brainstorming mit dem Verteidigungsprofessor wäre genug Zeit, um einen kreativen Weg zu finden, ihn finanziell zu ruinieren, oder ihn aus dem magischen Britannien zu verbannen. Das würde unseren Standpunkt klarmachen, denke ich."

"Ich gestehe", sagte der alte Zauberer langsam, "dass mir der Gedanke, ein fünfhundert Jahre altes Haus zu ruinieren und einen Todesser wegen eines Handgemenges in einem Hogwarts-Flur zum Krieg bis zum Ende herauszufordern, nicht in den Sinn gekommen ist, Harry." Der alte Zauberer hob einen Finger, um seine Halbmondbrille zurückzuschieben, die ihm bei seiner plötzlichen Bewegung vorhin ein wenig auf die Nase gerutscht war. "Ich wage zu behaupten, dass es auch Miss Granger nicht in den Sinn gekommen wäre, ebenso wenig Professor McGonagall und Fred und George."

Der Junge zuckte mit den Schultern. "Es würde\emph{nicht}um die Flure gehen", sagte der Junge. "Es ginge um Gerechtigkeit für seine vergangenen Verbrechen, und ich würde es nur tun, wenn Jugson den ersten Schritt machen würde. Es geht schließlich nicht darum, den Leuten Angst vor mir als Joker zu machen. Es geht darum, ihnen beizubringen, dass Neutrale vor mir absolut sicher sind und dass es unglaublich gefährlich ist, mich mit einem Stock zu pieken." Der Junge lächelte auf eine Weise, die seine Augen nicht erreichte. "Vielleicht kaufe ich eine Anzeige im Tagespropheten, in derich sage, dass jeder, der diesen Streit mit mir weiterführen will, die wahre Bedeutung von Chaos lernen wird, aber jeder, der mich in Ruhe lässt, sicher ist."

"\emph{Nein}", sagte der alte Zauberer. Seine Stimme war jetzt tiefer und zeigte etwas von seinem wahren Alter und seiner Macht. "Nein, Harry, das darf nicht sein. Du hast die Bedeutung des Kämpfens noch nicht gelernt, was wirklich passiert, wenn Feinde im Kampf aufeinandertreffen. Und so träumst du, wie kleine Jungs es tun, deine Feinde zu lehren, dich zu fürchten. Es erschreckt mich, dass du in deinem viel zu jungen Alter schon genug Macht haben könntest, um einen Teil deiner Träume in die Realität umzusetzen. Es gibt\emph{keine}Abzweigung dieser Straße, die nicht in die Dunkelheit führt, Harry, keine. Das ist der Weg eines Dunklen Lords, ganz sicher."

Der Junge zögerte, dann flackerten seine Augen zu der leeren goldenen Plattform, auf der Fawkes manchmal seine Flügel ausruhte. Es war eine Geste, die nur wenige verstanden hätten, aber der alte Zauberer kannte sie sehr gut.

"Also gut, vergessen Sie den Teil, mit dem sie lehren, mich zu fürchten", sagte der Junge dann. Seine Stimme war nicht weniger hart, aber etwas von der Kälte war aus ihr gewichen. "Ich denke immer noch, dass man nicht zulassen sollte, dass Kinder aus Angst davor, was jemand wie Lord Jugson tun\emph{könnte}, verletzt werden. Sie zu beschützen ist der Sinn Ihres Jobs. Wenn Lord Jugson wirklich versucht, sich Ihnen in den Weg zu stellen, dann tun Sie alles, was nötig ist, um ihn aufzuhalten. Geben Sie mir vollen Zugang zu meinem Verlies, und\emph{ich}übernehme persönlich die Verantwortung für die Folgen des Verbots von Mobbing in Hogwarts, ganz gleich, ob es sich um Lord Jugson oder jemand anderen handelt."

Langsam schüttelte der alte Zauberer den Kopf. "Du scheinst zu glauben, Harry, dass ich nur meine volle Kraft einsetzen muss, und alle Feinde werden beiseite gefegt werden. Du irrst dich. Lucius Malfoy kontrolliert Minister Fudge, über den\emph{Tagespropheten}beeinflusst er ganz Großbritannien, nur ganz knapp kontrolliert er nicht genug vom Obersten Rat, um mich aus Hogwarts zu vertreiben. Amelia Bones und Bartemius Crouch sind Verbündete, aber selbst sie würden zur Seite treten, wenn sie uns mutwillig handeln sähen. Die Welt, die dich umgibt, ist zerbrechlicher, als du zu glauben scheinst, und wir müssen vorsichtiger sein. Der alte Zaubererkrieg hat nie geendet, Harry, er ging nur in anderer Form weiter; der schwarze König schlief und Lucius Malfoy bewegte eine Zeit lang seine Schachfiguren. Glaubst du, Lucius Malfoy würde dir leichtfertig erlauben, einen Bauern seiner Farbe zu schlagen?"

Der Junge lächelte, jetzt wieder mit einem Hauch von Kälte. "Okay, ich werde mir etwas einfallen lassen, um es so zu arrangieren, dass es so aussieht, als hätte Lord Jugson seine eigene Seite verraten."

"Harry -"

"Hindernisse bedeuten, dass man\emph{kreativ wird}, Schulleiter. Das heißt nicht, dass man die Kinder, die man beschützen soll, im Stich lässt. Lassen Sie das Licht gewinnen, und wenn es Ärger gibt -" Der Junge zuckte mit den Schultern. "Lassen Sie das Licht wieder siegen."

"So würden Phönixe sprechen, wenn sie Worte hätten", sagte der alte Zauberer. "Aber du verstehst den\emph{Preis des Phönix}nicht."

Die letzten Worte wurden mit einer seltsam klaren Stimme gesprochen, die im ganzen Büro widerzuhallen schien, und dann schien ein gewaltiges Grollen von überall her zu kommen.

Zwischen dem uralten Schild an der Wand und der Ablage des Sprechenden Hutes begann der Stein der Wände zu zerfließen und sich zu bewegen, goss sich in zwei Säulen und gab eine Lücke zwischen ihnen frei, eine Öffnung, die eine Reihe von Steintreppen zeigte, die nach oben in die Dunkelheit führten.

Der alte Zauberer drehte sich um und schritt auf diese Treppe zu, dann blickte er zurück zu Harry Potter. "Komm!", sagte der alte Zauberer. In seinen blauen Augen war jetzt kein Funkeln mehr zu sehen. "Da du schon so weit gegangen bist, uneingeladen hier einzudringen, kannst du genauso gut weitergehen."

Auf diesen Steinstufen gab es kein Geländer, und nach den ersten paar Stufen zog Harry seinen Zauberstab und zauberte\emph{Lumos}. Der Schulleiter blickte nicht zurück, schien nicht nach unten zu schauen, als wäre er die Stufen oft genug hinaufgestiegen, um den Weg blind zu finden.

Der Junge wusste, dass er neugierig hätte sein sollen, oder ängstlich, aber dafür hatte er keine Gehirnkapazität übrig. Es kostete ihn all seine Beherrschung, die in ihm brodelnde Wut nicht noch weiter überkochen zu lassen, als sie es ohnehin schon tat.

Die Treppe ging nur ein kurzes Stück weiter, eine gerade, ansteigende Treppe ohne Kurven und Wendungen.

Oben war eine Tür aus massivem Metall, die im blauen Licht von Harrys Zauberstab schwarz aussah, was bedeutete, dass das Metall selbst entweder schwarz oder vielleicht rot war.

Albus Dumbledore hob seinen langen Zauberstab wie ein geschwungenes Symbol und sprach wieder mit dieser seltsamen Stimme, die in Harrys Ohren zu hallen schien, als hätte sie sich in sein Gedächtnis gebrannt: "\emph{Schicksal des Phönix}."

Die letzte Tür öffnete sich, und Harry folgte Dumbledore hinein.

Der Raum dahinter schien aus schwarzem Metall zu sein, wie die Tür, die zu ihm führte. Die Wände waren schwarz, der Boden war schwarz. Die Decke darüber war schwarz, bis auf eine einzelne Kristallkugel, die an einer weißen Kette von der Decke herabhing und in einem brillanten silbernen Licht leuchtete, das aussah, als wäre es in Nachahmung des Patronus-Lichts gezaubert worden, obwohl man sehen konnte, dass es nicht echt war.

Im Raum standen Sockel aus schwarzem Metall, jeder trug ein sich bewegendes Bild oder einen aufrechten Zylinder, der halb mit einer schwach glänzenden silbernen Flüssigkeit gefüllt war, oder einen einzelnen kleinen Gegenstand; eine versengte silberne Halskette, einen zerdrückten Hut, einen unberührten goldenen Ehering. Viele Sockel trugen alle drei, das bewegte Bild und die silberne Flüssigkeit und den Gegenstand. Auf diesen Sockeln schienen eine ganze Reihe von Zauberstäben zu stehen, und viele dieser Stäbe waren zerbrochen oder verbrannt oder sahen aus, als sei das Holz irgendwie geschmolzen.

Es dauerte so lange, bis Harry begriff, was er da sah, und dann schnürte es ihm plötzlich die Kehle zu; es war, als hätte die Wut in ihm einen Hammerschlag bekommen, vielleicht den härtesten Hammerschlag seiner ganzen Existenz.

"Das sind nicht alle Gefallenen aus all meinen Kriegen", sagte Albus Dumbledore. Sein Rücken war Harry zugewandt, nur seine grauen Locken und die gelblichen Roben zeigten ihn. "Nicht einmal annähernd alle. Nur meine engsten Freunde, und diejenigen, die durch meine schlimmstenEntscheidungen gestorben sind, von denen ist etwas hier. Diejenigen, die ich am meisten bedaure, das ist ihr Platz."

Harry konnte nicht zählen, wie viele Podeste im Raum standen. Es mögen um die hundert gewesen sein. Der Raum aus schwarzem Metall war nicht klein, und es war eindeutig noch mehr Platz darin für zukünftige Sockel.

Albus Dumbledore drehte sich um und betrachtete Harry, die tiefblauen Augen schauten stählern unter seinen Brauen hervor, aber als er sprach war seine Stimme ruhig. "Es scheint mir, dass du nichts über den Preis des Phönix weißt", sagte Albus Dumbledore leise. "Es scheint mir, dass du kein böser Mensch bist, sondern furchtbar unwissend und selbstbewusst in deiner Unwissenheit; so wie ich es einst war, vor langer Zeit. Dennoch habe ich Fawkes nie so deutlich gehört, wie du es an jenem Tag scheinbar getan hast. Vielleicht war ich schon zu alt und voller Kummer, als mein Phönix zu mir kam. Wenn es etwas gibt, was ich nicht verstehe, wie bereit ich sein sollte, zu kämpfen, dann erzähle mir von dieser Weisheit." Es lag kein Zorn in der Stimme des alten Zauberers; der Aufprall, der einem den Atem raubte, als würde man von einem Besenstiel fallen, lag in den verbrannten und zerbrochenen Zauberstäben, die im silbernen Licht sanft in ihrem Tod schimmerten. "Oder du drehst dich um und verlässt diesen Ort, aber dann will ich nichts mehr davon hören."

Harry wusste nicht, was er sagen sollte. So etwas hatte es in seinem eigenen Leben noch nicht gegeben, und alle Worte schienen ihm zu entfallen. Er würde etwas zu sagen finden, wenn er suchte, aber er konnte in diesem Moment nicht glauben, dass die Worte sinnvoll sein würden. Man sollte nicht in der Lage sein, jede mögliche Diskussion zu gewinnen, nur weil Menschen durch seine Entscheidungen gestorben waren, und doch fühlte es sich so an, als gäbe es nichts zu sagen. Dass es nichts gab, was Harry zu sagen hatte.

Und fast hätte Harry sich umgedreht und wäre von diesem Ort gegangen, bis zu der Erkenntnis, die ihm dann kam: dass es wahrscheinlich einen Teil von Albus Dumbledore gab, der immer an diesem Ort stand, immer, egal wo er war. Und dass man, wenn man an einem solchen Ort stand, alles tun, alles\emph{verlieren}konnte, wenn es bedeutete, dass man kein weiteres Mal kämpfen musste.

Einer der Sockel fiel Harry ins Auge; das Foto darauf bewegte sich nicht, lächelte oder winkte nicht, es war ein Muggelfoto von einer Frau, die ernst in die Kamera blickte, ihr braunes Haar zu Zöpfen in einem gewöhnlichen Muggelstil geflochten, den Harry noch an keiner Hexe gesehen hatte. Neben dem Foto stand ein Zylinder mit silbriger Flüssigkeit, aber kein Gegenstand; kein geschmolzener Ring oder zerbrochener Zauberstab.

Harry ging langsam vorwärts, bis er vor dem Sockel stand. "Wer war sie?" sagte Harry, und seine Stimme klang seltsam in seinen eigenen Ohren.

"Ihr Name war Tricia Glasswell", sagte Dumbledore. "Die Mutter einer muggelstämmigen Tochter, die von den Todessern getötet wurde. Sie war eine Detektivin der Muggelregierung und hat danach den Orden des Phönix mit Informationen der Muggelbehörden gefüttert, bis sie - verraten - in die Hände von Voldemort fiel." Die Stimme des alten Zauberers stockte kurz. "Sie ist nicht gut gestorben, Harry."

"Hat sie Leben gerettet?" fragte Harry.

"Ja", sagte der Zauberer leise. "Das hat sie."

Harry hob seinen Blick von dem Podest und sah Dumbledore an. "Wäre die Welt ein besserer Ort, wenn sie nicht gekämpft hätte?"

"Nein, wäre sie nicht", sagte der alte Zauberer. Seine Stimme war müde und traurig. Er wirkte jetzt noch gebeugter, als würde er in sich zusammensinken. "Ich sehe, dass du immer noch nicht verstehst. Ich denke, du wirst es nicht verstehen, bis zu dem Tag, an dem du - oh, Harry. Vor so langer Zeit, als ich nicht viel älter war als du jetzt, lernte ich das wahre Gesicht der Gewalt kennen und ihren Preis. Die Luft mit tödlichen Flüchen zu erfüllen - aus\emph{welchem Grund auch immer}, Harry - ist eine kranke Sache, und ihre Natur ist korrumpiert, so schrecklich wie die dunkelsten Rituale. Einmal begonnen wird Gewalt wie ein Lethifold, das jedes Leben in seiner Nähe angreift. Ich … würde gerne vermeiden, dass du diese Lektion so lernen musst, wie ich sie gelernt habe, Harry."

Harry wandte den Blick von den blauen Augen ab und betrachtete das schwarze Metall des Bodens. Der Schulleiter wollte ihm etwas Wichtiges sagen, das war klar; und es war auch nichts, was Harry für dumm hielt.

„Es gab einmal einen Muggel namens Mohandas Gandhi“ {[}Anm. d. Übers.: Ehrenname Mahatma{]}, sagte Harry zum Fußboden. "Er war der Meinung, dass die Regierung des Muggelstaates Großbritannien nicht über sein Land herrschen sollte. Und er weigerte sich, zu kämpfen. Er überzeugte sein ganzes Land, nicht zu kämpfen. Stattdessen forderte er sein Volk auf, auf die britischen Soldaten zuzugehen und sich niederschlagen zu lassen, ohne Widerstand zu leisten, und als Großbritannien das nicht mehr ertragen konnte, haben wir sein Land befreit. Ich fand das eine sehr schöne Sache, als ich darüber las, dachte ich, das sei etwas Höheres als alle Kriege, die jemals mit Gewehren oder Schwertern geführt worden waren. Dass sie das wirklich getan hatten, und dass es\emph{tatsächlich funktioniert}hatte." Harry holte noch einmal tief Luft. "Erst später habe ich herausgefunden, dass Gandhi seinen Leuten während des Zweiten Weltkriegs gesagt hat, dass sie, wenn die Nazis einmarschieren, auch gewaltlosen Widerstand gegen sie leisten sollten. Aber die Nazis hätten einfach jeden in Sichtweite erschossen. Und vielleicht hatte Winston Churchill immer das Gefühl, dass es einen besseren Weg hätte geben müssen, irgendeinen klugen Weg, um zu gewinnen, ohne jemanden verletzen zu müssen; aber er hat ihn nie gefunden, und deshalb musste er kämpfen." Harry sah zu dem Schulleiter auf, der ihn anstarrte. "Winston Churchill war derjenige, der versucht hat, die britische Regierung davon zu überzeugen, die Tschechoslowakei\emph{nicht}an Hitler als Gegenleistung für einen Friedensvertrag zu übergeben, dass sie sofort kämpfen sollten -"

"Ich erkenne den Namen, Harry", sagte Dumbledore. Die Lippen des alten Zauberers zuckten nach oben. "Obwohl mich die Ehrlichkeit zwingt, zu sagen, dass der liebe Winston nie einer war, der Gewissensbisse hatte, selbst nach einem Dutzend Gläsern Feuerwhiskey."

"Der Punkt ist", sagte Harry nach einer kurzen Pause, um sich daran zu erinnern, mit wem er eigentlich sprach, und um das plötzlich wiederkehrende Gefühl zu bekämpfen, dass er ein unwissendes, vor Dreistigkeit verrückt gewordenes Kind war, das kein Recht hatte, in diesem Raum zu sein, und kein Recht, Albus Dumbledore über irgendetwas zu befragen, "der Punkt ist, zu sagen, Gewalt sei böse, ist keine\emph{Antwort}. Es beantwortet nicht, wann man kämpfen soll und wann nicht. Es ist eine schwierige Frage und Gandhi hat sich geweigert, sich damit auseinanderzusetzen, und deshalb habe ich einen Teil meines Respekts vor ihm verloren."

"Und deine eigene Antwort, Harry?" Dumbledore sagte leise.

"Eine Antwort ist, dass man niemals Gewalt anwenden sollte, außer um Gewalt zu stoppen", sagte Harry. "Man sollte nie das Leben von jemandem riskieren, außer um noch mehr Leben zu retten. Das\emph{klingt}gut, wenn man es so sagt. Das Problem ist nur, dass wenn ein Polizist einen Einbrecher sieht, der ein Haus ausraubt, der Polizist versuchen\emph{sollte}, den Einbrecher zu stoppen, auch wennder Einbrecher sich wehren könnte und jemand verletzt oder sogar getötet werden könnte. Selbst wenn der Einbrecher nur versucht, Schmuck zu stehlen, was nur eine\emph{Sache}ist. Denn wenn es keine Hindernisse für Einbrecher gibt, wird es\emph{mehr}Einbrecher geben, und noch\emph{mehr}Einbrecher. Und selbst wenn sie jedes Mal nur\emph{Dinge}stehlen würden, würde es - das Gefüge der Gesellschaft -" Harry hielt inne. Seine Gedanken waren nicht so geordnet, wie sie es normalerweise vorgaben, in diesem Raum zu sein. Er hätte in der Lage sein müssen, eine vollkommen logische Erklärung im Sinne der Spieltheorie zu geben, hätte es zumindest so\emph{sehen}müssen, aber es entglitt ihm. Falken und Tauben - "Verstehen Sie nicht, wenn böse Menschen bereit sind, Gewalt zu riskieren, um zu bekommen, was sie wollen, und gute Menschen immer einen Rückzieher machen, weil Gewalt zu schrecklich ist, um sie zu riskieren, dann ist es - es ist keine gute Gesellschaft, in der man leben kann, Herr Direktor! Ist Ihnen nicht klar, was dieses ganze Mobbing in Hogwarts anrichtet, vor allem im Haus Slytherin?"

"\emph{Krieg}ist zu schrecklich, um ihn zu riskieren", sagte der alte Zauberer. "Und doch wird er kommen. Voldemort kehrt zurück. Die schwarzen Schachfiguren versammeln sich. Severus ist eine der wichtigsten Figuren, die unsere eigene Seite in diesem Krieg besitzt. Aber unser böser Zaubertränkemeister muss, wie man so schön sagt, den Schein wahren. Wenn Severus diesen Schein wahren kann, indem er die Gefühle von Kindern verletzt, und zwar nur ihre Gefühle, Harry", die Stimme des alten Zauberers war sehr sanft, "dann muss man schon furchtbar naiv in Bezug auf Kriege sein, um zu glauben, dass er ein schlechtes Geschäft gemacht hat. Schwere Entscheidungen sehen nicht\emph{so}aus, Harry. Sie sehen - so aus." Der alte Zauberer machte keine Geste. Er blieb einfach stehen, wo er war, zwischen den Sockeln.

"Sie sollten nicht Schulleiter sein", sagte Harry durch das Brennen in seiner Kehle. "Es tut mir leid, wirklich, aber Sie sollten nicht versuchen, ein Schulleiter zu sein und gleichzeitig einen Krieg zu führen. Hogwarts sollte nicht daran beteiligt sein."

"Die Kinder werden es überleben", sagte der alte Zauberer mit müden alten Augen. "Sie würden Voldemort nicht überleben. Hast du dich gefragt, warum die Kinder von Hogwarts nicht viel von ihren Eltern erzählen, Harry? Das liegt daran, dass es immer jemanden in Hörweite gibt, der seine Mutter oder seinen Vater oder beide verloren hat. Das hat Voldemort hinterlassen, als er das letzte Mal hier war.\emph{Nichts}ist es wert, dass dieser Krieg auch nur einen Tag früher beginnt, als er muss, oder einen Tag länger dauert, als er muss." Der alte Zauberer machte jetzt eine Geste, als wolle er auf all die zerbrochenen Zauberstäbe hinweisen. "Wir haben nicht gekämpft, weil es uns gerecht erschien, dies zu tun! Wir haben gekämpft, als wir es mussten, als es keinen anderen Weg mehr gab. Das war unsere Antwort."

"Haben Sie deshalb so lange damit gewartet, Grindelwald zu konfrontieren?"

Harry hatte die Frage gestellt, ohne richtig nachzudenken -

Die Zeit verlangsamte sich, während die blauen Augen ihn studierten.

"Mit wem hast du geredet, Harry?", fragte der alte Zauberer. "Nein, antworte nicht. Ich weiß es schon." Dumbledore seufzte. "Viele haben mir diese Frage gestellt, und immer habe ich sie abgewiesen. Doch mit der Zeit musst du die volle Wahrheit über diese Angelegenheit erfahren. Schwörst du, niemals mit jemand anderem darüber zu sprechen, bis ich dir die Erlaubnis gebe?"

Harry hätte es Draco gerne gesagt, aber - "Ich schwöre", sagte Harry.

"Grindelwald besaß ein uraltes und schreckliches Gerät", sagte Dumbledore. "Solange er es besaß, konnte ich seine Verteidigung nicht brechen. In unserem Duell konnte ich nicht gewinnen, nurstundenlang gegen ihn kämpfen, bis er vor Erschöpfung umfiel; und ich wäre danach daran gestorben, wenn Fawkes nicht gewesen wäre. Aber während seine Muggelverbündeten noch Blutopfer brachten, um ihn zu unterstützen, wäre Grindelwald\emph{nicht}gefallen. Er war in dieser Zeit buchstäblich unbesiegbar. Von dem fürchterlichen Gerät, das Grindelwald besaß, darf niemand etwas wissen, niemand etwas ahnen, es darf keine Andeutung geben. Und deshalb darfst du nicht darüber sprechen. Und ich werde vorerst nichts weiter sagen. Das ist alles, Harry. Es gibt keine Moral darin und keine Weisheit. Mehr gibt es da nicht."

Harry nickte langsam. Es war nicht ganz unplausibel, nach den Maßstäben der Magie…

"Und dann", fuhr Dumbledores Stimme fort, noch leiser, fast so, als spräche er zu sich selbst, "da ich es war, der ihn schlug, gehorchten sie mir, als ich sagte, er solle nicht sterben, obwohl sie zu Tausenden nach seinem Blut schrien. So wurde er in Nurmengard eingekerkert, in dem Gefängnis, das er gebaut hat, und dort sitzt er bis zum heutigen Tag. Ich bin zu diesem Duell gegangen, ohne die Absicht, ihn zu töten, Harry. Denn, weißt du, ich hatte schon einmal versucht, Grindelwald zu töten, vor langer Zeit, und das… das war… es erwies sich als… ein Fehler, Harry…" Der alte Zauberer starrte jetzt auf seinen langen, dunkelgrauen Zauberstab, den er in beiden Händen hielt, als wäre er eine Kristallkugel aus einem Muggelmärchen, ein Wahrsagebecken in dem man Antworten finden konnte. "Und ich dachte damals… Ich dachte, dass ich niemals töten sollte. Und dann kam Voldemort."

Der alte Zauberer blickte wieder zu Harry auf und sagte mit heiserer Stimme: "Er ist nicht wie Grindelwald, Harry. Es ist nichts Menschliches mehr in ihm.\emph{Ihn}musst du vernichten. Du darfst nicht zögern, wenn die Zeit gekommen ist. Bei ihm allein, von allen Kreaturen dieser Welt, darfst du keine Gnade walten lassen; und wenn du fertig bist, musst du es vergessen, vergessen, dass du so etwas je getan hast, und wieder zum normalen Leben übergehen. Spare dir deine Ihre Wut dafür auf, und nur dafür."

Im Büro herrschte Stille.

Sie dauerte viele, viele Sekunden und wurde schließlich von einer einzigen Frage durchbrochen.

"Gibt es Dementoren in Nurmengard?"

"Was?", sagte der alte Zauberer. "Nein! Das hätte ich selbst ihm nicht angetan -"

Der alte Zauberer starrte den Jungen an, der sich aufgerichtet hatte, und dessen Gesicht sich veränderte.

"Mit anderen Worten", sagte der Junge, als spräche er mit sich selbst, ohne dass andere Leute im Raum waren, "es ist schon bekannt, wie man mächtige Dunkle Zauberer gefangen halten kann, ohne Dementoren zu benutzen. Die Leute\emph{wissen}, dass sie das wissen."

"Harry …?"

"Nein", sagte der Junge. Der Junge sah auf, und seine Augen loderten wie grünes Feuer. "Ich akzeptiere Ihre Antwort nicht, Herr Direktor. Fawkes hat mir einen Auftrag gegeben, und ich weiß jetzt, warum Fawkes diesen Auftrag mir gegeben hat und nicht Ihnen. Sie sind bereit, ein Gleichgewicht der Kräfte zu akzeptieren, bei dem die Bösen am Ende gewinnen. Ich bin es nicht."

"Auch das ist keine Antwort", sagte der alte Zauberer; sein Gesicht zeigte nichts von seiner Verletzung, er hatte lange Übung darin, Schmerz zu verbergen. "Sich zu weigern, etwas zu akzeptieren, ändert es nicht. Ich frage mich jetzt, ob du einfach zu jung bist, um dies zu verstehen, Harry, auch wenn du dich erwachsen gibst. Nur in Kinderphantasien können alle Schlachten gewonnen und kein einziges Übel geduldet werden."

"Und das ist der Grund, warum ich Dementoren vernichten kann und Sie nicht", sagte der Junge. "Weil ich daran glaube, dass die Dunkelheit durchbrochen werden kann."

Dem alten Zauberer stockte der Atem in der Kehle.

"Der Preis des Phönix ist nicht unausweichlich", sagte der Junge. "Er ist nicht Teil eines tiefen Gleichgewichts, das in das Universum eingebaut ist. Es sind nur die Teile des Problems, bei denen man noch nicht herausgefunden hat, wie man sie überlisten kann."

Die Lippen des alten Zauberers öffneten sich, aber es kamen keine Worte heraus.

Silbernes Licht fiel auf zerbrochene Zauberstäbe.

"Fawkes gab mir einen Auftrag", wiederholte der Junge, "und ich werde diesen Auftrag ausführen, und wenn ich dafür das gesamte Ministerium zerstören muss. Das ist der Teil der Antwort, den Sie übersehen. Man hört nicht auf und sagt,\emph{na ja, ich schätze, ich kann unmöglich einen Weg finden, um das Mobbing in Hogwarts zu beenden}, und\emph{belässt}es dabei. Man sucht einfach weiter, bis man herausgefunden hat, wie man es machen kann. Wenn das erfordert, Lucius Malfoys gesamte Verschwörung zu zerschlagen,\emph{schön}."

"Und der wahre Kampf, der Kampf gegen Voldemort?", fragte der alte Zauberer mit unsicherer Stimme. "Was wirst du tun, um\emph{den}zu gewinnen, Harry? Wirst du die ganze Welt zerstören? Selbst wenn du eines Tages eine solche Macht erlangst, bist du noch nicht darüber hinaus, dass alles einen Preis hat, und vielleicht wirst du es auch nie sein! Dass du dich\emph{jetzt}so verhältst, ist nichts weniger als Wahnsinn!"

"Ich habe Professor Quirrell gefragt, warum er gelacht hat", sagte der Junge gleichmütig, "nachdem er Hermine diese hundert Punkte verliehen hatte. Und Professor Quirrell sagte, das sind nicht seine exakten Worte, aber es ist ziemlich genau das, was er sagte, dass er es ungeheuer amüsant fand, dass der große und gute Albus Dumbledore da saß und nichts tat, während dieses arme unschuldige Mädchen um Hilfe bettelte, während\emph{er}derjenige war, der sie verteidigte. Und er erzählte mir dann, dass gute und moralische Menschen, wenn sie fertig damit waren, sich in Knoten zu verstricken, für gewöhnlich nichts taten; oder wenn sie handelten, konnte man sie kaum von den Leuten unterscheiden, die man als schlecht bezeichnete.\emph{Er}hingegen konnte unschuldigen Mädchen helfen, wann immer ihm danach war, denn er war kein guter Mensch. Und das sollte ich mir merken, wenn ich daran dachte, gut zu werden."

Der alte Zauberer ließ sich die Wucht des Schlages nicht anmerken. Nur ein leichtes Aufreißen seiner Augen hätte es verraten, wenn man ihn genau beobachtet hätte.

"Machen Sie sich keine Sorgen, Schulleiter", sagte der Junge. "Ich habe mich nicht verirrt. Ich weiß, dass ich das Gute von Hermine und Fawkes lernen soll, nicht von Professor Quirrell und Ihnen. Was mich zu dem eigentlichen Grund bringt, warum ich hierhergekommen bin. Hermines Zeit ist zu wertvoll, um sie mit Nachsitzen zu vergeuden. Professor Snape wird sie widerrufen und behaupten, ich hätte ihn erpresst."

Nach einem Zögern nickte der alte Zauberer mit dem Kopf, der silberne Bart wogte langsam darunter. "Das wäre nicht das Beste für\emph{sie}, Harry", sagte der alte Zauberer. "Aber das Nachsitzen kann bei Professor Binns abgeleistet werden, und du kannst mit ihr zusammen in seinem Klassenzimmer lernen."

"Gut", sagte der Junge. "Ich glaube, das war dann auch schon alles, was wir zu besprechen hatten. Sie können davon ausgehen, dass ich das nächste Mal, wenn Sie scheinbar auf der Seite der Bösen arbeiten oder sie gewinnen lassen, alles tun werde, was ich glaube, dass Fawkes mir sagen würde, egal, wie viel Ärger dabei herauskommt. Ich hoffe, wir sind uns beide darüber im Klaren."

Ohne ein weiteres Wort drehte sich der Junge um und ging aus dem Raum, durch die offene Tür aus schwarzem Metall, die Worte "\emph{Lumos!}" und das Licht seines Zauberstabs folgten einen Moment später.

Der alte Zauberer stand stumm da, stumm inmitten der ruinierten Schicksale, die sein eigenes Leben hinterlassen hatte. Seine faltige Hand hob sich, zitternd, um seine Halbmondbrille zu berühren -

Der Junge steckte seinen Kopf wieder herein. "Würden Sie bitte die Treppe einschalten, Schulleiter? Ich möchte mir nicht noch einmal die ganze Arbeit machen, um auf demselben Weg zu gehen, auf dem ich gekommen bin."

"Geh, Harry Potter", sagte der alte Zauberer. "Die Treppe wird dich empfangen."

(Einige Zeit später folgte eine frühere Version von Harry, der seit 21 Uhr unsichtbar neben den Wasserspeiern gewartet hatte, der stellvertretenden Schulleiterin durch die Öffnung, die sich für sie öffnete, stellte sich leise hinter sie auf die sich drehende Treppe, bis sie oben ankamen, und drehte dann, immer noch unter dem Umhang, dreimal seinen Zeitumkehrer).

\emph{Nachspiel: Professor Quirrell und -}

Auf einer schattigen Lichtung wartete der Verteidigungsprofessor, den Rücken nachlässig an die raue, graue Rinde einer hoch aufragenden Buche gelehnt, die in den späten Märztagen noch nicht belaubt war, so dass ihr Stamm und ihre Krone wie ein bleicher Arm wirkten, der aus dem Boden ragte und in eine Hand mit tausend Fingern explodierte. Um den Verteidigungsprofessor herum und über ihm waren die Äste so dicht, dass man selbst im frühesten Frühling, wenn nur wenige Bäume so etwas wie Knospen trugen, vom Boden aus kaum den Himmel hätte sehen können. Die Stränge des hölzernen Netzes wucherten und kreuzten sich so oft, dass man, wenn man oben auf einem Besenstiel saß und unten nach jemandem suchte, eher den Ohren als den Augen hätte folgen können. Es hätte auch nicht geholfen, dass es inmitten des verbotenen Waldes schon fast dunkelund die unsichtbare Sonne fast untergegangen war, so dass nur ein paar Schimmer des verblassenden Sonnenlichts die Wipfel der höchsten Bäume beleuchteten.

Dann war das leise Geräusch von Schritten zu vernehmen, fast unhörbar auf dem Waldboden; der Gang eines Mannes, der es gewohnt war, ungesehen vorbeizugehen. Kein Zweig knackte, kein Blatt raschelte -

"Guten Tag", sagte Professor Quirrell. Der Verteidigungsprofessor machte sich nicht die Mühe, seine Augen oder seine Hände zu bewegen, die nachlässig an seiner Seite ruhten.

Eine in einen schwarzen Umhang gekleidete Gestalt erschien, sein Kopf drehte sich, um nach links und dann nach rechts zu schauen. In der rechten Hand der Gestalt lag ein Stab aus Holz, der so grau war, dass er fast silbern wirkte.

"Ich weiß nicht, warum Sie sich ausgerechnet\emph{hier}treffen wollten", sagte Severus Snape, seine Stimme war kühl.

"Oh", sagte Professor Quirrell beiläufig, als wäre die ganze Angelegenheit kaum von Bedeutung, "ich dachte, Sie würden Privatsphäre vorziehen. Die Wände von Hogwarts haben Ohren, und Sie möchten doch nicht, dass der Schulleiter von Ihrer Rolle in der gestrigen Affäre erfährt, oder?"

Die Märzkälte schien sich zu intensivieren, die Temperatur weiter zu sinken. "Ich weiß nicht, wovon Sie reden", sagte der Zaubertränkemeister eisig.

"Sie wissen sehr wohl, wovon wir reden", sagte Professor Quirrell mit amüsierter Stimme. "Wirklich, mein guter Professor, Sie sollten sich nicht in die Angelegenheiten von Idioten einmischen, es sei denn, Sie sind bereit, sich auf der Stelle gegen all ihre Gewalt zu verteidigen." (Die Hände des Verteidigungsprofessors lagen immer noch entspannt und offen an seiner Seite.) "Und doch scheint sich keiner dieser Idioten an Ihren Sturz zu erinnern, noch erinnern sich die jungen Damen an Ihre Anwesenheit. Was die faszinierende Frage aufwirft, warum Sie sich die außerordentliche Mühe, ich wage zu sagen, die\emph{verzweifelte}Mühe machen,\emph{zweiundfünfzig}Gedächtniszauber zu wirken." Professor Quirrell legte den Kopf schief. "Sorgen Sie sich so sehr um die Meinung von einfachen Schülern? Ich glaube nicht. Befürchten Sie, dass Ihr guter Freund, Lord Malfoy, davon erfährt? Aber diese Narren haben an Ort und Stelle eine recht zufriedenstellende Entschuldigung für Ihre Anwesenheit erfunden. Nein, es gibt nur eine Person, die so viel Macht über Sie hat und die höchst beunruhigt wäre, wenn Sie ohne sein Wissen ein Komplott schmieden. Ihr wahrer und verborgener Meister, Albus Dumbledore."

"\emph{Was?}", zischte der Zaubertränkemeister, die Wut stand ihm ins Gesicht geschrieben.

"Aber jetzt scheinen Sie auf eigene Faust zu handeln, und deshalb bin ich höchst neugierig, was Sie\emph{eigentlich}tun und warum." Der Verteidigungsprofessor betrachtete die schwarz gekleidete Silhouette des Zaubertränkemeisters mit der Aufmerksamkeit, die ein Mann einem außergewöhnlich interessanten Käfer schenken würde, auch wenn es letztlich nur ein Käfer war.

"Ich bin nicht Dumbledores Diener ", sagte der Tränkemeister kalt.

"Wirklich? Was für eine erstaunliche Nachricht." Der Verteidigungsprofessor lächelte leicht. "Erzählen Sie mir alles darüber."

Es gab eine lange Pause. Von irgendeinem Baum schrie eine Eule, das Geräusch war weithin hörbar in der Stille; keiner der beiden Männer erschrak oder zuckte auch nur.

"Sie wollen mich nicht zum Feind haben, Quirrell", sagte Severus Snape, seine Stimme war sehr sanft.

"Will ich nicht?", fragte Professor Quirrell. "Woher wollen Sie das wissen?"

"Andererseits", fuhr der Meister der Zaubertränke mit immer noch sanfter Stimme fort, "genießen meine Freunde viele Vorteile."

Der Mann, der an der grauen Rinde lehnte, hob die Augenbrauen. "Zum Beispiel?"

"Es gibt vieles, was ich über diese Schule weiß", sagte der Meister der Zaubertränke. "Dinge, von denen Sie vielleicht nicht glauben, dass ich sie weiß."

Es entstand eine erwartungsvolle Pause.

"Wie unglaublich faszinierend", sagte Professor Quirrell. Der Mann untersuchte seine Fingernägel mit einem gelangweilten Blick. "Fahren Sie fort."

"Ich weiß, dass Sie den Korridor im dritten Stock…\emph{untersucht…}haben -"

"Sie wissen nichts dergleichen." Der Rücken des Mannes richtete am Baumstamm auf. "Bluffen Sie nicht bei mir, Severus Snape; ich finde es ärgerlich, und Sie sind in keiner Position, mich zu ärgern. Ein einziger Blick würde jedem kompetenten Zauberer verraten, dass der Schulleiter diesen Korridor mit einer lächerlichen Menge an Schutzzaubern und Spinnweben, Auslösern und Stolpersteinen versehen hat. Und mehr noch: Dort liegen Zaubersprüche von uralter Macht, magische Konstrukte, von denen ich nicht einmal Gerüchte gehört habe, Techniken, die aus den gehorteten Überlieferungen von Flamel selbst stammen müssen. Selbst Er-dessen-Name-nicht-genannt-werden-darf hätte Schwierigkeiten gehabt, diese unbemerkt zu passieren." Professor Quirrell tippte mit einem Finger nachdenklich auf seine Wange. "Und was das eigentliche Schloss betrifft, so ist es ein\emph{Colloportus}, der auf einen gewöhnlichen Türknauf gelegt wurde und so schwach ist, dass er Miss Granger an dem Tag, als sie Hogwarts betrat, nicht hätte aussperren können. Noch nie in meinem Leben bin ich auf eine so offensichtliche Falle gestoßen." Jetzt verengte der Verteidigungsprofessor seine Augen. "Ich kenne niemanden mehr auf der Welt, gegen den solche fantastischen Aufspürzauber irgendeinen nützlichen Zweck erfüllen würden. Wenn es einen Zauberer gibt, der über uralte Überlieferungen verfügt, von denen ich nichts weiß, und gegen den diese Falle aufgestellt wurde - Sie können\emph{diese}Information gegen so viel Schweigen eintauschen, wie Sie wollen, mein lieber Professor, und eine gute Portion meiner Gunst, die danach übrig bleibt."

Man hätte schwören können, dass Professor Quirrell Severus Snape mit lebhaftem Interesse beobachtete. Nicht die leiseste Spur eines Lächelns kam über die Lippen des Mannes.

Wieder herrschte langes Schweigen auf der Lichtung.

"Ich weiß nicht,\emph{wen}Dumbledore fürchtet", sagte Snape. "Aber ich weiß, welchen Köder er ausgelegt hat, und etwas davon, wie er wirklich bewacht wird -"

"Was das angeht", sagte Professor Quirrell und klang wieder gelangweilt, "habe ich ihn vor Monaten gestohlen und eine Fälschung an seiner Stelle hinterlassen. Aber vielen Dank für die Frage."

"Sie lügen", sagte Severus Snape nach einer Pause.

"Ja, das tue ich." Professor Quirrell lehnte sich wieder an den grauen Stamm, sein Blick schweifte hinauf zu dem dichten Netz aus Ästen, zwischen denen die hereinbrechende Nacht kaum zu erkennen war. "Ich wollte nur wissen, ob Sie mich darauf ansprechen würden, da Sie vorgeben, so wenig zu wissen." Der Verteidigungsprofessor lächelte vor sich hin.

Der Meister der Zaubertränke sah aus, als würde er gleich an seiner eigenen Wut ersticken. "\emph{Was wollen Sie?}"

"Eigentlich nichts", sagte der Verteidigungsprofessor und starrte weiter in die Äste über sich. "Ich war nur neugierig. Ich nehme an, ich werde einfach zusehen, wohin Ihre Verschwörungen führen, und in der Zwischenzeit werde ich dem Schulleiter nichts sagen - solange Sie bereit sind, mir ab und zu einen Gefallen zu tun, natürlich." Ein trockenes Lächeln huschte über sein Gesicht. "Sie sind für den Moment entlassen, Severus Snape. Allerdings hätte ich nichts dagegen, wenn wir uns bald noch einmal unterhalten, wenn Sie bereit sind, mir ehrlich zu sagen, wo Ihre Loyalitäten liegen. Und ich meine\emph{ehrlich}, nicht die Fassaden, die Sie heute gezeigt haben. Sie könnten feststellen, dass Sie mehr Verbündete haben, als Sie dachten. Nehmen Sie sich etwas Zeit, um darüber nachzudenken, mein Freund."

\emph{Nachspiel: Draco Malfoy und -}

Eine Regenbogenhalbkugel, eine Kuppel aus fester Kraft mit wenig eigener Farbigkeit, die das einfallende Licht zersplitterte und in vielfarbig schillernden Reflexen zurückschickte, während sie den Glanz der prachtvollen Kronleuchter des Slytherin-Gemeinschaftsraums brach.

Unter der regenbogenfarbenen Halbkugel lag das verängstigte Gesicht einer jungen Hexe, die noch nie gegen Schläger gekämpft hatte, die sich keiner von Professor Quirrells Armeen angeschlossen hatte, die in ihrem Verteidigungskurs bestenfalls akzeptable Noten bekam und die nicht einmal eine prismatische Barriere hätte zaubern können, um ihr eigenes Leben zu retten.

"Ach, hört doch auf", sagte Draco Malfoy und ließ seine Stimme trotz des Schweißes, der unter seinen Roben ausgebrochen war, gelangweilt klingen, während er seinen Zauberstab auf die Barriere richtete, die Millicent Bulstrode schützte.

Er konnte sich nicht daran erinnern, die Entscheidung bewusst getroffen zu haben, da waren gerade die beiden älteren Jungen gewesen, die Millicent verhexen wollten, der Gemeinschaftsraum schaute schweigend zu, und dann hatte Dracos Hand gerade seinen Zauberstab gezogen und die Barriere gewirkt, während sein Herz sich mit geschocktem Adrenalin vollpumpte und sein armes, trauriges Gehirn verzweifelt nach Erklärungen suchte -

Die beiden älteren Jungen richteten sich von der Stelle auf, an der sie Millicent bedrängt hatten, drehten sich zu Draco um und sahen ihn mit einer Mischung aus Schock und Wut an. Gregory und Vincent neben ihm hatten bereits ihre eigenen Zauberstäbe gezogen, richteten sie aber nicht auf sie. Alle drei zusammen hätten sowieso nicht gewinnen können.

Aber die älteren Jungen würden ihn nicht verhexen. Niemand konnte so dumm sein, den nächsten Lord Malfoy zu verhexen.

Es war nicht die Angst, verhext zu werden, die Draco unter seinem Umhang schwitzen ließ, während er verzweifelt hoffte, dass die Wasserperlen auf seiner Stirn nicht sichtbar waren.

Draco schwitzte wegen der dämmernden unangenehmen Gewissheit, dass, selbst wenn er jetzt damit durchkäme, wenn er diesen Weg weiterverfolgte, irgendwann alles zusammenbrechen würde; und dann wäre er vielleicht nicht mehr der nächste Lord Malfoy.

"Mr. Malfoy", sagte der älteste Junge. "Warum beschützen Sie sie?"

"Sie haben also den Kopf der Verschwörung ausfindig gemacht", sagte Draco mit einem Nummer-zwei-Grinsen, "und es ist, damit ich das jetzt richtig verstehe, eine Erstklässlerin namens Millicent Bulstrode. Sie ist nur ein\emph{Bote}, du\emph{Dummkopf}! "

"Und?", fragte der ältere Junge. "Sie hat ihnen trotzdem geholfen!"

Draco hob seinen Zauberstab und die prismatische Sphäre erlosch. Immer noch mit gelangweilter Stimme sagte Draco: "Wussten Sie,\emph{was}Sie da tun, Miss Bulstrode?"

"N-nein", stammelte Millicent von ihrem Schreibtisch aus.

"Wussten Sie, wohin die Slytherin-Botschaften, die Sie weitergaben, gingen?"

"Nein!", sagte Millicent.

"Danke", sagte Draco. "Ihr alle lasst sie bitte in Ruhe, sie ist nur ein Spielball. Miss Bulstrode, Sie dürfen den Gefallen, den Sie mir im Februar getan haben, als zurückgezahlt betrachten." Und Draco wandte sich wieder seinen Zaubertrank-Hausaufgaben zu und hoffte bei Merlin und wieder zurück, dass Millicent nicht irgendetwas unglaublich Dummes sagte wie "Welcher Gefallen?" -

"Warum", sagte eine Stimme deutlich von der anderen Seite des Raumes, "sind diese Hexen dann dorthin gegangen, wohin ein Zettel von Millicent sie schickte?"

Noch mehr schwitzend hob Draco den Kopf und schaute zu Randolph Lee, der gesprochen hatte. "Was genau stand auf dem gefälschten Zettel?", fragte Draco. "War es 'Ich befehle euch, im Namen der Dunklen Lady Bulstrode dort zu erscheinen' oder 'Bitte treffen Sie mich hier, mit freundlichen Grüßen, Millicent?'"

Randolph Lee öffnete den Mund und zögerte für den Bruchteil einer Sekunde -

"Das dachte ich mir schon", sagte Draco. "Das war kein sehr guter Test, Mr. Lee, es - es kann -" Ein hektischer, nervenaufreibender Moment, während er überlegte, wie er es sagen sollte, ohne Harry-Wörter wie "\emph{falsch positiv}" zu benutzen. "Es kann die Hexen dazu bringen, dorthin zu gehen, wenn eine von ihnen mit Millicent befreundet ist."

Als wäre die Angelegenheit völlig geklärt, blickte Draco wieder auf seine Zaubertrank-Hausaufgaben hinunter und ignorierte (abgesehen von dem Gefühl des schlimmen Grauens in seinem Magen) das Geflüster aus dem Raum.

Nur aus dem Augenwinkel heraus bemerkte er, wie Gregory ihn anstarrte.

Dracos Augen ruhten auf seinen Astronomie-Hausaufgaben, aber er konnte seine Gedanken nicht darauf konzentrieren. Wenn man versuchte, nicht an die Dinge zu denken, die Harry Potter gesagt hatte, war so ziemlich das Schlimmste, was man tun konnte, sich die Bilder des Nachthimmels in seinem Lehrbuch anzusehen und zu versuchen, sich an das zu erinnern, was man\emph{nicht}über die Wanderung der Planeten\emph{wissen sollte}. Astronomie, eine edle und angesehene Kunst, ein Zeichen von Gelehrsamkeit und Wissen. Bloß besaßen Muggel geheime moderne Artefakte, die das eine Million Milliarden Mal besser konnten, und zwar mit Methoden, die Harry zu erklären versucht hatte und die Draco immer noch nicht ansatzweise verstehen konnte, außer dass es anscheinend nicht einmal\emph{Magie}brauchte, um\emph{DingeArithmantik}tun zu lassen.

Draco sah sich die Bilder von Sternbildern an und fragte sich, ob es in den anderen Häusern auch so war, ob sich die Leute in Ravenclaw immer gegenseitig bedrohten.

Harry Potter hatte ihm einmal gesagt, dass Soldaten auf einem Schlachtfeld nicht wirklich für ihr Land kämpften. Patriotismus brachte sie vielleicht überhaupt erst auf das Schlachtfeld, aber wennsie einmal dort waren, kämpften sie, um\emph{sich gegenseitig}zu beschützen, die Freunde, mit denen sie trainiert hatten und die direkt vor ihnen standen. Und Harry hatte beobachtet, und Draco wusste, dass es stimmte, dass man die Loyalität zu einem Anführer nicht dazu benutzen konnte, einen Patronus-Zauber zu wirken, es war nicht die\emph{richtige}Art von warmen und glücklichen Gedanken. Aber der Gedanke, jemanden an seiner Seite zu beschützen -

Das, hatte Harry Potter nachdenklich gesagt, war wahrscheinlich der Grund, warum die Todesser in dem Moment auseinandergefallen waren, als der Dunkle Lord verschwand. Sie hatten nichts\emph{für einander}empfunden.

Man konnte eine Gruppe rekrutieren, zu der neben Lord Malfoy und Mr. MacNair auch Bellatrix Black und Amycus Carrow gehörten, und sie mit dem Cruciatus-Fluch auf Linie halten. Aber in dem Moment, in dem der Meister des Dunklen Malsweg war, hatte man keine Armee mehr, sondern einen Bekanntenkreis. Deshalb hatte Vater versagt. Es war nicht einmal wirklich seine Schuld gewesen. Es gab nichts, was Vater hätte tun\emph{können}, nachdem er Todesser geerbt hatte, die nicht wirklich miteinander\emph{befreundet}waren.

Und obwohl es das Haus Slytherin war, das er verteidigen sollte - das Haus Slytherin, zu dessen\emph{Rettung}er und Harry einen Pakt geschlossen hatten -,mussteDraco manchmalfeststellen, dass es einfach weniger\emph{ermüdend}war, wenn er Armeeübungen leitete. Wenn er mit Schülern aus den anderen drei Häusern arbeitete, die nicht zu Slytherin gehörten. Wenn man die Probleme einmal gesehen und benannt hatte, konnte man nicht mehr\emph{aufhören}, sie zu sehen, es wurde nur jeden Tag\emph{nerviger}.

"Mr. Malfoy?", ertönte die Stimme von Gregory Goyle, der in dem kleinen, aber privaten Schlafzimmer neben Dracos Schreibtisch auf dem Boden lag; Gregory machte gerade seine Hausaufgaben inVerwandlung, bei denen er oft Hilfe brauchte.

Jede Ablenkung war zu diesem Zeitpunkt willkommen. "Ja?", sagte Draco.

"Du hast garkeinen Plangegen Grangerausgeheckt", sagte Gregory. "Stimmt's?"

Das Gefühl, das sich in Dracos Magen ausbreitete, fühlte sich genau so an, wie Gregorys Stimme klang: angewidert und ängstlich.

"Du hast Granger tatsächlich geholfen, an dem Tag, als du sie vom Boden aufhobst", sagte Gregory. "Und davor, als du sie davor bewahrt hast, vom Dach zu fallen. Du hast einem\emph{Schlammblutgeholfen}-"

"Ja, genau", sagte Draco sarkastisch, ohne das geringste Zögern oder Zaudern, und blickte wieder auf seine Astronomie-Hausaufgaben hinunter, als wäre er nicht im Geringsten nervös. Es geschah alles so, wie Draco es befürchtet hatte, aber das bedeutete zumindest, dass er dieses Gespräch in seinem Kopf immer und immer wieder durchgespielt und sich den richtigen Eröffnungsgag ausgedacht hatte. "Komm schon, Gregory, du hast dich mit General Granger duelliert, du\emph{weißt}, wie stark ihre Zaubersprüche sind. Als ob ein echter Muggel-Sprössling mächtiger wäre als du, mächtiger als Theodore, mächtiger als jeder einzelne Reinblüter in unserem ganzen Schuljahr außer mir?\emph{Glaubst}du eigentlich an alles, was Vater sagt? Sie ist\emph{adoptiert}. Ihre Eltern starben im Krieg und jemand steckte sie zu ein paar Muggeln, um sie zu verstecken. Auf\emph{keinen Fall}ist General Granger ein echtes Schlammblut."

DieStillepulsierte langsamdurch Dracos Schlafzimmer. Draco wollte es wissen, musste wissen, welcher Ausdruck auf Gregors Gesicht lag. Aber er\emph{konnte}nicht von seinem Schreibtisch aufschauen, nicht bevorGregory zuerst sprach.

Und dann -

"Ist es\emph{das}, was Harry Potter direrzählthat?", fragte Gregory.

Die Stimme schwankte und brach. Als Draco von seinen Hausaufgaben aufblickte, sah er, dass Tränen aus Gregorys Augen liefen.

Offenbar hatte das nicht geklappt.

"Ich weiß nicht, was ich tun soll", sagte Gregory flüsternd. "Ich weiß nicht, was ich jetzt tun soll, Mr. Malfoy.DeinVater wird - wenn er es erfährt - nicht begeistert sein, Mr. Malfoy!"

\emph{Es ist nicht}deine\emph{Aufgabe, zu entscheiden, was Vater gefallen wird, Goyle}-

Draco konnte die Worte in seinem Kopf hören; sie klangenwieVaters Stimme, mit der gleichen Strenge. Es war die Art von Dingen, die Vater ihmerklärthatte, die er sagen sollte, wenn Vincent oder Gregory ihn jemals in Frage stellten; und wenn das nicht funktionierte, sollte er sie verhexen. Sie waren\emph{keine}gleichberechtigten Freunde, hatte Vater gesagt, und das sollte er nie vergessen. Draco hatte das Sagen, sie waren seine Diener, und wenn Draco sich nicht daran halten konnte, dann war er nicht geeignet, das Haus Malfoy zu erben…

"Ist schon gut, Gregory", sagte Draco, so sanft wie er konnte. "Du musst dich nur darum kümmern, mich zu beschützen. Niemand wird dir einen Vorwurf machen, wenn du meine Befehle befolgst, nicht mein Vater und nicht deiner." Er legte so viel Wärme wie möglich in seine Stimme, als wollte er einen Patronuszauber wirken. "Und überhaupt, der nächste Krieg wird nicht so sein wie der letzte. Das Haus Malfoy gab es schon lange vor dem Dunklen Lord, und nicht jeder Lord Malfoy macht das Gleiche. Vater weiß das."

"Tut er das?", fragte Gregory mit zitternder Stimme. "Tut er das\emph{wirklich}? "

Draco nickte. "Professor Quirrell weiß es auch", sagte Draco. "Darum geht es ja bei den Armeen. Der Verteidigungsprofessor hat recht, wenn der nächste Krieg kommt, wird Vater nicht in der Lage sein, das ganze Land zu vereinen, sie werden sich an den\emph{letzten}Krieg erinnern. Aber jeder, der in Professor Quirrells Armeen gekämpft hat, wird sich erinnern, wer die stärksten Generäle waren, sie werden wissen, wer würdig ist, sie anzuführen. Sie werden Harry Potter zu ihremHerrscherernennen. Und ichwerdeseine rechte Hand. Und das Haus Malfoy wird siegen, wie immer. Vielleicht wenden sich die Leute sogar an\emph{mich}, wenn Potter nicht da ist, solange sie mich für vertrauenswürdig halten. Das ist es, was ich jetzt vorhabe. Vater wird das verstehen."

Gregory griff nach oben, wischte sich über die Augen und sah wieder auf seineVerwandlungshausaufgaben hinunter. "Okay", sagte Gregory mit zittriger Stimme. "Wenndues sagst, Mr. Malfoy."

Draco nickte wieder, ignorierte das hohle Gefühl in seinem Inneren über die Lügen, die er seinem Freund gerade erzählt hatte, und wandte sich wieder den Sternen zu.

\emph{Nachspiel: Hermine Granger und}-

Unsichtbar zu sein hätte\emph{interessanter}sein sollen als das hier, die Korridore von Hogwarts hätten sich in seltsamen Farben abzeichnen sollen oder so. Aber eigentlich, dachte Hermine, war es unter Harrys Unsichtbarkeitsumhang genau so, als wäre man\emph{nicht}unter einem Unsichtbarkeitsumhang, bis auf den Teil mit dem Umhang. Wenn man den Schleier aus weichem, schwarzem Stoff von der Kapuze herunter und über das Gesicht zog, konnte manihnnicht einmaldirekt vor seinen Augensehen, und danach schien erauch dasAtmen nicht zubeeinträchtigen. Und die Welt sah genauso aus, nur dassdu, wennduan Dingen aus Metall vorbeigingst, keine kleinen Spiegelungen vondirselbst sahst. Porträts sahen einen nie an, sondern taten nur, was auch immer für seltsame Dinge sie taten, wenn sie allein waren. Hermine hatte noch nicht versucht, an einem Spiegel vorbeizugehen, sie war sich nicht sicher, ob sie das\emph{wollte}. Vor allem aber gab es kein\emph{Du}mehr, wenn man herumlief,ohneHändeundFüße, nurmiteinemwechselnden Blickwinkel. Es war ein beunruhigendes Gefühl, nicht so sehr,\emph{unsichtbar}zu sein, als vielmehr\emph{nicht zu existieren}.

Harry hatte sie überhaupt nichtsgefragt, sie hatte nur das Wort 'Unsichtbarkeit' herausgebracht und dann hatte Harry seinen Unsichtbarkeitsumhang aus seiner Tasche gezogen. Sie hatte nicht einmal die Chance bekommen, ihr extrem geheimes Treffen mit Daphne und Millicent Bulstrode zu erklären, oder dass sie dachte, es würde helfen, die anderen Mädchen zu schützen, Harry hatte ihr einfach etwas übergeben,das wahrscheinlich ein Heiligtum des Todes war. Wenn man fair war, und sie versuchte, fair zu sein, musste sie zugeben, dass Harry manchmal ein echter, wahrer Freund sein konnte.

Das geheime Treffen selbst war ein großer Fehlschlag gewesen.

Millicent hatte behauptet, eine Seherin zu sein.

Hermine hatte Millicent und Daphne sorgfältig und ausführlich erklärt, dass dies unmöglichwahrsein konnte.

Sie und Harry hatten schon früh in ihren Nachforschungen Wahrsagerei nachgeschlagen; Harry hatte darauf bestanden, dass sie alles über Prophezeiungen lesen sollten, was nicht in der verbotenen Abteilung stand. Wie Harry bemerkt hatte, würde es eine Menge Mühe ersparen, wenn sie einfach einen Seher dazu bringen könnten, alles zu prophezeien, was sie fünfunddreißig Jahre später herausfinden würden. (Oder, um es in Harrys Worten auszudrücken, jede Möglichkeit, Informationen zu erhalten, die aus der fernen Zukunft übermittelt wurden, war potenziell eine sofortige globale Siegbedingung.)

Aber, wie Hermine Millicent erklärt hatte, war das Prophezeien nicht kontrollierbar, es gab keine Möglichkeit, um eine Prophezeiung über etwas Bestimmtes zu\emph{bitten}. Stattdessen (so stand es in den Büchern) gab es eine Art\emph{Druck}, der sich in der Zeit aufbaute, wenn irgendein großes Ereignis eintreten wollte oder sich selbst davon abhalten wollte. Und Seher waren wie Schwachstellen, die den Druck abließen, wenn der richtige Zuhörer in der Nähe war. Prophezeiungen betrafen also nur große, wichtige Dinge, denn nur das erzeugte genug Druck; und man bekam fast nie mehr als einen Seher, der das Gleiche sagte, denn danach war der Druck weg. Und, wie Hermine Millicent weiter erklärt hatte, erinnerten sich die Seher selbst nicht an ihre Prophezeiungen, weil die Botschaft nicht für\emph{sie}bestimmt war. Und die Botschaften würden in Rätseln herauskommen, und nur jemand, der die Prophezeiung mit der Originalstimme des Sehers hörte, würde die ganze Bedeutung, die in dem Rätsel steckte, verstehen. Es gab\emph{keine}Möglichkeit, dass Millicent einfach so eine Prophezeiung über\emph{Schultyrannen}aussprechen konnte,\emph{wann immer sie wollte}, und dass sie sich dann daran\emph{erinnerte}, und wenn sie es getan\emph{hätte}, wäre es als "das Skelett ist der Schlüssel" herausgekommen und nicht als "Susan Bones muss dort sein".

Millicent hatte zu diesem Zeitpunkt ziemlich verängstigt ausgesehen, also hatte Hermine ihre Fäuste entspannt,diesie in ihre Hüftengestemmthatte, sich selbst beruhigt und vorsichtig erklärt, dass sie froh war, dass Millicent ihnen geholfen hatte, aber sie\emph{waren}manchmal in Fallen getappt, wenn sie dem gefolgt waren, was Millicent gesagt hatte, und deshalb wollte Hermine wirklich wissen, woher die Botschaften\emph{eigentlich}kamen.

Und Millicent hatte mit leiser Stimme gesagt:

\emph{Aber, aber sie hat}mir\emph{gesagt, dass sie eine Seherin ist}…

Hermine hatte Daphne gesagt, sie solleihrkeinen Druck machen, nachdem Millicent sich geweigert hatte, ihre Quelle zu verraten. Es war nicht nur so, dass Hermine sich schrecklich gefühlt hatte wegen des verängstigten Blicks auf Millicents Gesicht. Herminehatte außerdemdas starke Gefühl, dass, wenn sie die Person finden\emph{würden}, die Millicent Dinge erzählt hatte, sich herausstellen würde, dass\emph{sie}am Morgen nur Umschläge unter ihrenKopfkissen finden würden.

Sie bekam dasselbe verzweifelte Gefühl, das sie in der Schlacht vor Weihnachten bekommen hatte, als sie Zabinis Diagramm mit all den farbigen Linien und Kästchen betrachtete und… und sie hatte erst jetzt begriffen, was es bedeutete, dass\emph{Zabini}derjenige gewesen war, der ihr dieses Diagramm gezeigt hatte.

Selbst für eine Ravenclaw, so fand sie, gab es so etwas wie ein Leben, dasvielzu kompliziert wurde.

Hermine begann eine kurze Spirale aus gelben Marmorstufen zu erklimmen, eine wenig versteckte "geheime" Treppe, die einer der schnellsten Wege von den Verliesen Slytherins zum Turm von Ravenclaw war, den aber nur Hexen nehmen konnten. (Warum vor allem Mädchen einen schnellen Weg von Ravenclaw nach Slytherin und zurück benötigten, fand Hermine etwas rätselhaft).Hermine blieb amoberen Ende der Treppestehen, nun, da sie nun weit genug von den Slytherins entfernt und zurück im Hauptteil vonHogwarts war, und zog Harrys Unsichtbarkeitsumhang aus.

Nachdem ihr Beutel den Umhang verschluckt hatte, wandte sie sich nach rechts und begann einen kurzen Gang entlang zu gehen, wobei sie nun automatisch in alle Richtungen Ausschau hielt, ohne wirklich darüber nachzudenken, und ihre ständigprüfendenAugen blickten in eine schattige Nische -

(\emph{kurze Desorientierung})

- und dann durchfuhreinSchwall von Schock und Angst wie ein Betäubungszauber ihren ganzen Körper. Sie stellte fest, dass ihr Zauberstab ohne jeden Gedanken oder bewusste Entscheidung in ihre Hand gesprungen war und bereits aufetwas gerichtet war…

… einenschwarzenUmhang, der so breit und wallend war, dass es unmöglich zu bestimmen war, ob die Figur darunter männlich oder weiblich war. Auf dem Mantel befand sich ein schwarzer Hut mit breiter Krempe; und ein schwarzer Nebel schien sich darunter zu sammeln und das Gesicht von wem oder was auch immersichdarunterbefindenmochte, zu verschleiern.

"Noch einmal hallo, Hermine", flüsterte eine zischende Stimme unter dem schwarzen Hut, hinter dem schwarzen Nebel.

Hermines Herz klopfte bereits gewaltig in ihrer Brust, ihre Hexenumhang fühlte sich bereits schweißgetränkt auf ihrer Haut an,sie schmeckte bereitseinenGeschmack von Angst in ihremMund; sie wusste nicht, warum sie so plötzlichvollerAdrenalin war, aber ihre Hand griff fester nach ihrem Zauberstab. "Wer sind Sie?" verlangte Herminezu wissen.

Der Hut neigte sich leicht; die flüsternde Stimme klang staubtrocken, als sie aus dem schwarzen Nebel kam. "Der letzte Verbündete", sprach das zischende Flüstern. "Derjenige, deram Endeantwortet, wenn kein andererdirantworten will. Ich bin vielleicht der einzig\emph{wahre}Freund, denduin ganz Hogwarts hast, Hermine. Denn jetzt hast du gesehen, wie die anderen geschwiegen haben, als du in Not warst -"

"Wie ist Ihr\emph{Name?} "

Der schwarze Umhang drehte sich leicht hin und her, es\emph{sah}nichtganz so\emph{aus}, aber es vermittelteden EindruckeinesAchselzuckens. "Das ist das Rätsel, junge Ravenclaw. Bis du es gelöst hast, kannst du mich nennen, wie du willst."

Sie spürtebereits Schweiß inihrerHandfläche, und war dankbar für die spiralförmigen Rillen auf ihrem Zauberstab, die ihrer Hand halfen das Holz fest im Griff zu behalten. "Nun, MisterUnglaublich-Verdächtige-Person", sagte Hermine, "was wollen Sie von mir?

"Das ist die falsche Frage", erklang das Flüsternaus dem schwarzen Nebel. "Du solltest lieber fragen, was\emph{ich dir}anbieten kann."

"Nein", sagte das junge Mädchenweiterhin fest, „ich glaube\emph{nicht}, dass ich das fragen sollte.“

Ein hochfrequentes Kichernkamaus dem schwarzen Nebel. "Nicht Macht", flüsterte die Stimme, "noch Reichtum, das interessiert dich wenig, nicht wahr, junge Ravenclaw?\emph{Wissen}. Das ist es, was ich besitze. Ich weiß, was sich in dieser Schule abspielt, all die verborgenen Pläne und Akteure, die Antworten auf das Rätsel. Ich kenne den wahren Grund für die Kälte, die du in Harry Potters Augen siehst. Ich kenne die wahre Natur von Professor Quirrells mysteriöser Krankheit. Ich weiß, wen Dumbledore wirklich fürchtet."

"Schön für Sie", sagte Hermine Granger. „Aber wissen Sie, wie oft man lecken muss, um zum Schokoladenkern eines Tootsie-Pops zu gelangen?“ {[}Anm. d.Übers.: Süßigkeit in den USA{]}

Der schwarze Nebel schien sich leicht zu verdunkeln, die Stimme klang beim Sprechen tiefer und enttäuscht. "Du bist also gar nicht neugierig, junge Ravenclaw, auf die Wahrheiten hinter den Lügen?"

"Hundertsiebenundachtzig", sagte sie. "Ich habe es einmal versucht, und so viele waren es." Ihre Hand rutschte beinahe vom Zauberstab ab, ihre Finger fühlten sich müde, als hätte sie den Zauberstab schon seit Stunden statt Minuten gehalten -

Die Stimme zischte: "Professor Snape ist ein getarnter Todesser".

Hermine ließ fast ihren Zauberstab fallen.

"Ah", flüsterte die Stimme zufrieden. "Ich dachte mir, das könnte dich interessieren. Also, Hermine. Gibt es sonst noch etwas, was du gerne über deine Feinde oder diejenigen, die du Freundenennst, wissen möchtest?"

Sie starrte auf den schwarzen Nebel, der den hoch aufragenden schwarzen Umhang bedeckte, und versuchte verzweifelt, ihre Gedanken zu ordnen. Professor Snape war ein Todesser? Wer würde\emph{ihr}so etwas sagen?\emph{Warum?Was}ging da vor sich? "Das ist…" sagte Hermine. Ihre Stimme zitterte. "Das ist eine äußerst ernste Angelegenheit, wenn es wirklich wahr ist. Warum erzählen Sie so etwas\emph{mir}und nicht Schulleiter Dumbledore?"

"Dumbledore hat nichts unternommen, um Snape aufzuhalten", flüsterte der schwarze Nebel. "Du hast es gesehen, Hermine. Die Fäulnis in Hogwarts beginnt ganz oben. Alles, was an dieser Schule falsch ist, das beginnt mit dem verrückten Schulleiter. Du allein hast es gewagt, ihn deshalb zur Rede zur Stellen - und deshalb spreche ich zu dir."

"Und haben Sie auch mit Harry Potter gesprochen?" sagte Hermine und hielt ihre Stimme so gleichmäßig wie möglich. Wenn\emph{dies}sein hilfreicher Geist war -

Der schwarze Nebel wurde dunkler und heller, wie ein Kopfschütteln. "Ich habe Angst vor Harry Potter", flüsterte es. "Vor der Kälte in seinen Augen, vor der Dunkelheit, die hinter seinen Augen wächst. Harry Potter ist ein Mörder, und jeder, der ihm im Weg steht, wird sterben. Selbst du, Hermine Granger, wenn du es wagst, dich ihm wirklich zu widersetzen, wird die Dunkelheit hinter seinen Augen nach dir greifen und dich vernichten. Das weiß ich."

"Dann wissen Sie nicht einmal die Hälfte von dem, was Sie vorgeben zu wissen", sagte Hermine, ihre Stimme etwas fester. "Ich habe auch Angst vor Harry. Aber nicht wegen dem, was er\emph{mir}jemals antun könnte. Ich habe Angst davor, was er tun könnte, um mich zu\emph{beschützen}-"

"Falsch." Das Flüstern war flach und hart, als ob es keine andere Möglichkeit gäbe. "Harry Potter\emph{wird}sich mit der Zeit gegen dich wenden, Hermine, wenn die Dunkelheit ihn völlig einnimmt. Er wird keine Träne vergießen, er wird es nicht einmal bemerken, an dem Tag, an dem seine Schritte dich schließlich unter ihm begraben.

"\emph{Doppelt}falsch", sagte sie mit nachdrücklicher Stimme, obwohl sie eine Gänsehaut bekam. Einer von Harrys Phrasen kam ihr in den Sinn. "Was glauben Sie zu wissen, und warum glauben Sie es zu wissen?"

"Zeit -" Die Stimme schien sich zu fangen. "das hat Zeit für später. Denn jetzt, für heute, ist Harry Potter in der Tat nicht dein Feind. Und doch bist du in größter Gefahr."

"\emph{Das}kann ich glauben", sagte Hermine Granger. Sie wollte verzweifelt ihren Zauberstab in die andere Hand legen, sie hatte das Gefühl, ihren rechten Arm greifen zu müssen, nur um ihn aufrecht zu halten, ihr Kopf schmerzte, als hätte sie seit Tagen auf den schwarzen Nebel gestarrt; sie wusste nicht, warum sie so schnell müde geworden war.

"Lucius Malfoy hat dich bemerkt, Hermine." Das Flüstern wurde höher, die Ausdruckslosigkeit war gewichen, und hatte einen Ton von hörbarer Besorgnis angenommen. „Du hast durch deinen Erfolg das Haus Slytherin gedemütigt. Schon vorher warst du eine Schande für alle, die für die Todesser stehen; denn du bist muggelstämmig, und doch besitzt du eine Zauberkraft, die größer ist als die von jedem mit reinem Blut. Und jetzt wirst du dafür bekannt, die Augen der Welt richten sich auf dich. Lucius Malfoy versucht, dich zu zermalmen, Hermine, dich zu verletzen und vielleicht sogar zu töten, und er hat die Mittel, das zu tun!“ Das Flüstern war eindringlich geworden.

Es entstand eine Pause.

"Ist das alles?" sagte Hermine. Wenn sieEx-OberstZabini oder Harry Potter wäre, würde sie wahrscheinlichclevereFragen stellen, um mehr Informationen herauszufinden; aber ihr Geist fühlte sichschwerfälligund müde an. Sie musste unbedingt hier raus und sich eine Weile hinlegen.

"Du glaubst mir nicht", sagte das Flüstern, das jetzt leiser und trauriger war. "Warum nicht, Hermine? Ich\emph{versuche}, dir zu helfen."

Hermine machte einen Schritt zurück, weg von der schattigen Nische.

"\emph{Warum nicht, Hermine?}"forderte die Stimme, die sich zu einem Zischen erhob. "So viel schuldest du mir! Sag es mir, und dann -" Die Stimme fingsichund kamnunleiser zurück. "Und dann kannst dugehen, nehme ich an. Sag mir nur -\emph{warum}-"

Vielleicht hätte sie nicht antworten sollen; vielleicht hätte sie sich einfach umdrehen und fliehen sollen, oder besser noch, zuerst eine Prismatische Barriere beschwören und dann aus voller Kehle schreien sollen, während sie rannte; aber es war der Tonfall des echten Schmerzes in dieser Stimme, der sie erreichte, und so antwortete sie.

"Weil Sie unglaublich dunkel und furchterregend und verdächtig aussehen", sagte Hermine und hielt ihre Stimme höflich, während ihr Zauberstab auf denhoch aufragenden schwarzen Umhang und dem gesichtslosen schwarzen Nebelgerichtetblieb.

"Ist das\emph{alles?}"flüsterte die Stimme ungläubig. Traurigkeit schien sie zu durchdringen. „Ich hatte mir Besseres von dir erhofft, Hermine. Sicherlich weiß ein Ravenclaw wie du, die intelligenteste Ravenclaw, die Hogwarts seit einer Generation besucht hat, dass der Schein trügen kann.“

"Oh,dasweißich", sagte Hermine. Sie trat einen weiteren Schritt zurück, ihre müden Finger verkrampften sichum ihrenZauberstab. „Aber die Sache, die die Menschen manchmal vergessen, ist, dass der Schein zwar irreführend sein\emph{kann}, es aber normalerweise\emph{nicht}ist.“

Es entstand eine Pause.

"Du\emph{bist}sehrclever", sagte die Stimme, und der schwarze Nebel verflüchtigte sich, er verdunkelte nicht mehr; sie erkannte das Gesicht darunter, und es schickte einen heftigenStoß Adrenalin durch ihren Körper -

(\emph{kurze Desorientierung})

- und dann durchfuhreinSchwall von Schock und Angst wie ein Betäubungszauber ihren ganzen Körper. Sie stellte fest, dass ihr Zauberstab ohne jeden Gedanken oder bewusste Entscheidung in ihre Hand gesprungen war und bereits auf jemandengerichtet war…

… eine strahlende Dame, ihr langes weißes Kleid wogte um sie herum wie in einem unsichtbaren Wind; weder ihre Hände noch ihre Füße waren sichtbar, ihr Gesichtwarunter einem weißen Schleier verborgen; und sie leuchtete überall, nicht wie ein Geist, nicht durchsichtig, nur von weichem weißen Licht umgeben.

Hermine starrte mit offenem Mund auf den sanften Anblick und fragte sich, warum ihr Herz bereits hämmerte und warum sie sich so verängstigt fühlte.

"Nochmals hallo, Hermine", flüsterte eine freundliche Stimme, die von dem weißen Schein hinter dem Schleier ausging. „Ich bin geschickt worden, um dir zu helfen, also hab bitte keine Angst. Ich bin deine Dienerin; denn dir junge Dame, wird ein höchst wunderbares Schicksal zuteil -“

…

…

…

