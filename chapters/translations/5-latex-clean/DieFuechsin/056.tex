

\hypertarget{das-stanford-prison-experiment-teil-6-eingeschruxe4nkte-optimierungen}{% \section{24. Das Stanford-Prison-Experiment, Teil 6, Eingeschränkte Optimierungen}\label{das-stanford-prison-experiment-teil-6-eingeschruxe4nkte-optimierungen}}

-\/-\/-\/-\/- Kapitel 56: Das Stanford-Prison-Experiment, Teil 6, Eingeschränkte Optimierungen -\/-\/-\/-\/-

Still, Gott sei Dank war sie still, die Metalltür auf der nächsttieferen Ebene. Entweder war dahinter niemand, oder sie litten leise, oder vielleicht schrien sie und ihre Stimmen waren bereits verstummt, oder sie murmelten nur leise vor sich hin in der Dunkelheit…

\emph{Ich bin mir nicht sicher, ob ich das tun kann}, dachte Harry, und er konnte den verzweifelten Gedanken auch nicht auf die Dementoren schieben. Es wäre besser, weiter unten zu sein, sicherer weiter unten zu sein, die Umsetzung seines Plans würde Zeit brauchen, und die Auroren arbeiteten sich wahrscheinlich schon nach unten. Aber wenn Harry noch mehr dieser Metalltüren passieren musste, während er schwieg und seinen Atem vollkommen gleichmäßig hielt, könnte er verrückt werden; wenn er an jeder dieser Türen ein Stück von sich zurücklassen müsste, wäre bald nichts mehr von ihm übrig -

Eine leuchtende mondbeschienene Katze sprang ins Leben und landete vor Harrys Patronus. Harry schrie beinahe, was seinem Ansehen bei Bellatrix nicht gerade zuträglich gewesen wäre.

"Harry!" sagte die Stimme von Professor McGonagall, die so alarmiert klang, wie Harry es noch nie von ihr gehört hatte. "Wo bist du? Geht es dir gut? Das ist mein Patronus, antworte mir!"

Mit einer krampfhaften Anstrengung klärte Harry seinen Geist, räusperte sich, erzwang Ruhe, wechselte in eine andere Persönlichkeit wie eine okklumenische Barriere. Es dauerte ein paar Sekunden, und er hoffte sehr, dass Professor McGonagall dank der Kommunikationsverzögerung kein Problem bemerkte, genauso wie er hoffte, dass Patronusse nicht über ihre Umgebung berichteten.

Die unschuldige Stimme eines kleinen Jungen sagte: "Ich bin in Marys Platz, Professor, in der Winkelgasse. Ich gehe gerade auf die Toilette. Was ist los?"

Die Katze sprang fort, und Bellatrix fing an, leise zu kichern, ein staubiges anerkennendes Lachen, aber sie brach abrupt durch ein Zischen von Harry ab.

Einen Augenblick später kehrte die Katze zurück und sagte mit Professor McGonagalls Stimme: „Ich komme Sie jetzt sofort abholen. Gehen Sie \emph{nirgendwo} hin, wenn Sie nicht in der Nähe des Verteidigungsprofessors sind, gehen Sie nicht zu ihm zurück, sagen Sie niemandem etwas, ich komme so schnell ich kann!“

Und die helle Katze wurde unscharf und verschwand.

Harry blickte auf seine Uhr und notierte sich die Zeit, so dass er, nachdem er alle hier rausgeholt hatte und Professor Quirrell den Zeitumkehrer wieder verankert hatte, zurückgehen und zur gegebenen Zeit auf der Toilette von Marys Platz sein konnte…

\emph{Weißt du}, sagte der problemlösende Teil seines Gehirns, \emph{es gibt eine Grenze, wie viele Einschränkungen man einem Problem hinzufügen kann, bevor es wirklich unmöglich ist, weißt du das?}

Es hätte keine Rolle spielen sollen, und das tat es auch nicht wirklich, es war nicht zu vergleichen mit dem Leiden eines einzigen Gefangenen in Askaban, und wusste Harry, wenn sein Plan nicht damit enden würde, dass er von Marys Platz abgeholt würde, so als ob er ihn nie verlassen hätte, und der Verteidigungsprofessor völlig unschuldig aussah, dass Professor McGonagall \emph{ihn umbringen} würde.

Während sich ihr Team darauf vorbereitete, noch einen weiteren Bereich der C-Spirale abzuschirmen und zu scannen, bevor die vorherige Abschirmung weiter hinten aufgelöst wurde, klopfte Amelia mit den Fingern auf ihre Hüfte und fragte sich, ob sie den offensichtlichen Experten zu Rate ziehen sollte. Wenn er nur nicht so wäre -

Amelia hörte das vertraute Knistern von Feuer und wusste, was sie sehen würde, wenn sie sich umdrehte.

Ein Drittel ihrer Auroren drehte sich um und legte ihre Zauberstäbe auf den alten Zauberer mit halbmondförmiger Brille und einem langen silbernen Bart an, der direkt in ihrer Mitte erschienen war, einen leuchtend rot-goldenen Phönix auf der Schulter.

"Nicht schießen!" Vielsafttrank machte es leicht, das Gesicht zu fälschen, aber die Phönix-Reise vorzutäuschen wäre sehr viel schwieriger gewesen - die Schutzzauber erlaubten es als einen der schnellen Wege nach Askaban hinein, obwohl es keine schnellen Wege nach draußen gab.

Die alte Hexe und der alte Zauberer starrten sich einen langen Moment lang an.

(Amelia fragte sich insgeheim, wer von ihren Auroren die Nachricht geschickt hatte, es gab mehrere ehemalige Mitglieder des Ordens des Phönix in ihrem Team; sie versuchte sich zu erinnern, ob sie Emmelines Spatz oder Andys Katze gesehen hatte, die in der Schar der leuchtenden Geschöpfe fehlten; aber sie wusste, dass es aussichtslos war. Vielleicht war es nicht einmal einer ihrer Leute, denn der alte Wichtigtuer wusste oft Dinge, die er überhaupt nicht wissen konnte).

Albus Dumbledore neigte seinen Kopf in einer höflichen Geste zu Amelia. "Ich hoffe, ich bin hier nicht unwillkommen", sagte der Zauberer ruhig. "Wir sind doch alle auf der gleichen Seite, nicht wahr?"

"Das kommt darauf an", sagte Amelia mit harter Stimme. "Sind Sie hier, um uns zu helfen, Kriminelle zu fangen, oder um sie vor den Folgen ihrer Handlungen zu schützen? \emph{Wirst du versuchen, die Mörderin meines Bruders daran zu hindern, ihren wohlverdienten Kuss zu bekommen, alter} \emph{Besserwisser?} Nach dem, was Amelia gehört hatte, war Dumbledore gegen Ende des Krieges klüger geworden, was vor allem auf Mad-Eyes pausenloses Nörgeln zurückzuführen war; aber er war in dem Augenblick, als Voldemorts Leiche gefunden wurde, in seine törichten Gnadenhandlungen zurückgefallen.

Ein Dutzend kleiner weiß-silberner Punkte, Reflexionen leuchtender Tiere, schimmerten auf der Halbmondbrille des alten Zauberers, als er sprach. "Ich möchte noch weniger als Sie sehen, dass Bellatrix Black befreit wird", sagte der alte Zauberer. "Sie \emph{darf} dieses Gefängnis nicht lebend verlassen, Amelia."

Bevor Amelia erneut sprechen konnte, sogar um ihre überraschte Genugtuung auszudrücken, gestikulierte der alte Zauberer mit seinem langen schwarzen Zauberstab, und ein glühender silberner Phönix erwachte zum Leben, vielleicht heller als all ihre anderen Patronusse zusammen. Es war das erste Mal, dass sie diesen Zauber wortlos ausgeführt sah. "Befehlen Sie all ihren Auroren, ihre Patronuszauber für zehn Sekunden abzubrechen", sagte der alte Zauberer. "Was die Dunkelheit nicht finden kann, wird vielleicht das Licht finden."

Amelia schickte den Befehl an den Kommunikationsoffizier, der alle Auroren über ihre Spiegel benachrichtigen würde.

Das dauerte einige Augenblicke, eine Periode entsetzlichen Schweigens, keiner der Auroren wagte zu sprechen, während Amelia versuchte, ihre eigenen Gedanken abzuwägen. \emph{Sie darf dieses Gefängnis nicht lebend verlassen}… Albus Dumbledore würde sich nicht ohne triftigen Grund in Bartemius Crouch verwandeln. Wenn er ihr hätte sagen wollen, \emph{warum}, hätte er es bereits getan; aber es war sicher kein positives Zeichen.

Trotzdem war es gut zu wissen, dass sie in der Lage sein würden, in dieser Sache zusammenzuarbeiten.

"Jetzt", sagte ein Chor von Spiegeln, und alle Patronuszauber gingen aus, bis auf den glühenden silbernen Phönix.

"Ist noch ein weiterer Patronus anwesend", sagte der alte Zauberer deutlich zu dem leuchtenden Wesen.

Das helle Geschöpf senkte den Kopf in einem Nicken.

"Kannst du ihn finden?"

Der silberne Kopf nickte erneut.

"Wirst du dich an ihn erinnern, wenn er weggeht und wiederkommt?"

Ein letztes Nicken des lodernden Phönix.

"Es ist vollbracht", sagte Dumbledore.

"Vorbei", sagten alle Spiegel einen Augenblick später, und Amelia hob ihren Zauberstab und begann, ihren eigenen Patronus wieder herbei zu zaubern. (Obwohl es etwas zusätzlicher Konzentration bedurfte, mit diesem wölfischen Lächeln auf ihrem Gesicht, um an das erste Mal zu denken, als Susan ihre Wange geküsst hatte, anstatt sich mit dem drohenden Schicksal von Bellatrix Black zu beschäftigen. Dieser andere Kuss war in der Tat ein glücklicher Gedanke, aber nicht ganz die richtige Art für den Patronuszauber).

Sie waren noch nicht einmal am Ende dieses Korridors angekommen, bevor Harrys Patronus höflich, wie in einem Klassenzimmer, die Hand hob.

Harry dachte schnell. Die Frage war, wie - nein, das war auch klar.

"Es scheint", sagte Harry mit kühler, amüsierter Stimme, "dass jemand diesen Patronus angewiesen hat, seine Botschaft nur mir zu überbringen". Er kicherte. "Nun denn. Verzeih mir, liebe Bella. \emph{Quietus}."

Sofort sagte der silberne Humanoide mit Harrys eigener Stimme: "Es gibt noch einen Patronus, der diesen Patronus sucht."

"\emph{Was?}", sagte Harry. Und dann, ohne innezuhalten, um darüber nachzudenken, was geschah: "Kannst du ihn blockieren? Ihn daran hindern, dich zu finden?"

Der silberne Humanoide schüttelte den Kopf.

Kaum hatten Amelia und die anderen Auroren ihre Patronusse erneut gezaubert, da -

Der flammende silberne Phönix flog davon, und der wahre rot-goldene Phönix folgte ihm, und der alte Zauberer ging ruhig hinter den beiden her und hielt seinen langen Zauberstand gesenkt.

Die Schilde um ihr Territorium teilten sich wie Wasser um den alten Zauberer und schlossen sich hinter ihm mit kaum einer Kräuselung.

"Albus!"rief Amelia. "Was glaubst du, was du da tust?"

Aber sie wusste es bereits.

"Folgt mir nicht", sagte die Stimme des alten Zauberers streng. "Ich kann mich selbst schützen, andere kann ich nicht schützen."

Der Fluch, den Amelia ihm nachschrie, ließ sogar ihre eigenen Auroren zucken.

\emph{Das ist nicht fair, ist nicht fair, ist nicht fair! Es gibt eine Grenze, wie viele Einschränkungen man einem Problem hinzufügen kann, bevor es wirklich unmöglich ist!}

Harry blockte die nutzlosen Gedanken ab, ignorierte die Müdigkeit, die er fühlte, und zwang seinen Verstand, sich den neuen Anforderungen zu stellen. Er musste \emph{schnell} denken, das Adrenalin dazu verwenden, den Ketten der Logik schnell und ohne Zögern zu folgen, anstatt es in Verzweiflung zu vergeuden.

Damit die Mission gelingen konnte,

(1) Harry müsste seinen Patronuszauber beenden.

(2) Bellatrix musste vor den Dementoren versteckt werden.

(3) Harry musste dem Einfluss der Dementoren widerstehen, nachdem sein Patronus weg war.

…

\emph{\emph{Wenn ich dieses Problem löse,} sagte Harrys Gehirn, \emph{dann will ich danach einen Keks, und wenn} \emph{du} \emph{das Problem noch schwieriger machst, ich meine, auch nur ein} \emph{winziges} \emph{bisschen schwieriger, dann klettere ich aus} \emph{deinem} \emph{Schädel und fahre nach Tahiti.}}

Harry und sein Gehirn überlegten sich das Problem.

Askaban war jahrhundertelang unbesiegbar gewesen und hatte sich auf die Unmöglichkeit verlassen, dem Blick der Dementoren auszuweichen. Wenn Harry also \emph{einen anderen} Weg fand, Bellatrix vor den Dementoren zu verstecken, würde er sich entweder auf sein wissenschaftliches Wissen oder auf seine Erkenntnis verlassen, dass die Dementoren der Tod waren.

Harrys Gehirn schlug vor, dass ein offensichtlicher Weg, die Dementoren davon abzuhalten, Bellatrix zu sehen, darin bestünde, ihre Existenz zu beenden, d. h. sie zu töten.

Harry beglückwünschte sein Gehirn dazu, dass es über den Tellerrand hinausschaute, und sagte ihm, es solle weitersuchen.

\emph{\emph{Töte sie und bringe sie dann zurück,} kam der nächste Vorschlag. \emph{Verwende} \emph{Frigideiro, um Bellatrix bis zu dem Punkt abzukühlen, an dem ihre Hirnaktivität aufhört, und} \emph{wärmesie} \emph{danach mit einer} \emph{Thermos} \emph{auf, so wie Menschen, die in sehr kaltes Wasser fallen} \emph{und} \emph{eine halbe Stunde später ohne merkliche Hirnschäden erfolgreich wiederbelebt werden können.}}

Harry zog dies in Betracht. Bellatrix könnte in ihrem geschwächten Zustand nicht überleben. \emph{Und} es könnte den Tod nicht davon abhalten, sie zu sehen. \emph{Und} er hätte Schwierigkeiten, eine kalte, bewusstlose Bellatrix sehr weit zu tragen. \emph{Und} Harry konnte sich nicht an die Forschungen erinnern, welche Körpertemperatur genau nicht tödlich, sondern nur vorübergehend hirnverzögernd sein sollte.

Es war eine weitere gute, unkonventionelle Idee, aber Harry sagte seinem Gehirn, es solle weiter daran denken…

…\emph{wie man sich vor dem Tod verstecken kann}…

Ein Stirnrunzeln zog über Harrys Gesicht. Irgendwo hatte er etwas darüber gehört.

\emph{\emph{Eine der Voraussetzungen, um ein mächtiger Zauberer zu werden, sei ein ausgezeichnetes Gedächtnis,} hatte Professor Quirrell gesagt. \emph{Der Schlüssel zu einem Rätsel ist oft etwas, das man vor zwanzig Jahren in einer alten Schriftrolle las, oder ein seltsamer Ring, den man am Finger eines Mannes sah, den man nur einmal} \emph{traf…}}

Harry konzentrierte sich so sehr er konnte, aber er konnte sich nicht erinnern, es lag ihm auf der Zunge, aber er konnte sich nicht erinnern; also sagte er seinem Unterbewusstsein, es solle weiter versuchen, sich daran zu erinnern, und konzentrierte seine Aufmerksamkeit wieder auf die andere Hälfte des Problems.

\emph{\emph{Wie kann ich mich selbst ohne einen} \emph{Patronuszauber} \emph{vor den Dementoren schützen?}}

Der Schulleiter war einen ganzen Tag lang wiederholt einem Dementor aus ein paar Schritten Entfernung ausgesetzt gewesen, immer und immer wieder, und er kam nur müde daraus hervor. Wie hatte der Schulleiter das gemacht? Konnte Harry es auch tun?

Es könnte einfach irgendeine zufällige genetische Sache sein, und in diesem Fall war Harry am Arsch. Aber angenommen, das Problem wäre \emph{lösbar}…

Dann war die Antwort klar: Dumbledore hatte keine Angst vor dem Tod.

Dumbledore hatte \emph{wirklich} keine Angst vor dem Tod. Dumbledore hatte wirklich, ehrlich geglaubt, dass der Tod das nächste große Abenteuer sei. Er glaubte es in seinem Innersten, nicht nur als bequeme Worte, die dazu dienen, kognitive Dissonanzen zu unterdrücken, und nicht nur so zu tun, als ob er weise wäre. Dumbledore hatte beschlossen, dass der Tod die natürliche und normative Ordnung sei, und was auch immer für eine winzige Angst noch in ihm steckte, es dauerte lange und benötigte wiederholten Kontakt, bis der Dementor ihn durch diese kleine Schwachstelle ausgesaugt hatte.

Dieser Weg war für Harry versperrt.

Und dann dachte Harry an die Kehrseite der Medaille, an die offensichtliche umgekehrte Frage:

\emph{Warum bin ich so viel verletzlicher als der Durchschnitt? Andere Studenten fielen nicht um, wenn sie dem Dementor gegenüberstanden.}

Harry wollte den Tod vernichten, ihn beenden, wenn er konnte. Er wollte ewig leben, wenn er konnte; er hatte Hoffnung darauf, der Gedanke an den Tod brachte ihm kein Gefühl der Verzweiflung oder Unvermeidbarkeit. Er hing nicht blind an seinem eigenen Leben; in der Tat hatte es Anstrengung benötigt, \emph{nicht} sein ganzes Leben zu verbrennen um andere vor dem Tod zu schützen. Warum hatten die Schatten des Todes solche Macht über Harry? Er hätte sich selbst nicht für so ängstlich gehalten.

War es Harry, der die ganze Zeit rationalisiert hatte? Wer hatte insgeheim so viel Angst vor dem Tod, dass er seine eigenen Gedanken verdrehte, wie Harry Dumbledore beschuldigt hatte?

Harry dachte darüber nach und verhinderte, dass er vor sich selbst zurückschreckte. Es war ihm unangenehm, aber…

Aber…

Aber unbehagliche Gedanken waren nicht immer \emph{wahr}, und dieser Gedanke klang nicht ganz richtig. Als ob es ein Körnchen Wahrheit gäbe, das sich aber nicht \emph{dort} verbarg, wo es laut der Hypothese sein sollte -

Und da wurde Harry klar.

\emph{Oh.}

\emph{Oh, jetzt verstehe ich.}

\emph{Derjenige, der Angst hat, ist…}

Harry fragte seine dunkle Seite, was sie vom Tod hielt.

Und Harrys Patronus schwankte, verdunkelte sich, ging fast auf den Augenblick aus, denn diese verzweifelte, schluchzende, schreiende Angst, eine unaussprechliche Furcht, die alles tun würde, um nicht zu sterben, alles beiseite werfen würde, um nicht zu sterben, die nicht klar denken oder fühlen konnte in der Gegenwart dieses absoluten Grauens, die nicht in den Abgrund der Nichtexistenz blicken konnte, genauso wenig wie sie direkt in die Sonne hätte starren können, ein blindes, verängstigtes Ding, das nur eine dunkle Ecke finden und sich verstecken wollte um nicht mehr daran denken zu müssen -

Die Silberfigur hatte sich zu Mondlicht verdunkelt, flackerte wie eine brennende Kerze -

\emph{Ist schon gut}, dachte Harry, \emph{ist schon gut}.

Er stellte sich vor, wie er seine dunkle Seite wie ein verängstigtes Kind in seinen Armen wiegte.

\emph{\emph{Es ist recht und billig, entsetzt zu sein, denn der Tod ist schrecklich.} \emph{Du} \emph{musstdein Entsetzen nicht verbergen,} \emph{du musst dich dafür nicht schämen,} \emph{du kannst} \emph{es als Ehrenzeichen tragen, offen in der Sonne.}}

Es war seltsam, sich so in zwei Hälften gespalten zu fühlen, die Spur seiner Gedanken, die den Trost spendete, die Spur seiner Gedanken, die dem Unverständnis seiner dunklen Seite gegenüber der Fremdheit der Gedanken des gewöhnlichen Harry folgte; von all den Dingen, die seine dunkle Seite mit ihrer eigenen Angst vor dem Tod verband, war das Einzige, was sie niemals erwartet oder sich vorgestellt hatte finden zu können, Akzeptanz und Lob und Hilfe…

\emph{Man muss nicht allein kämpfen}, sagte Harry schweigend zu seiner dunklen Seite. \emph{Der Rest von mir wird dich in dieser Sache unterstützen. ~Ich werde mich nicht sterben lassen, und ich werde auch meine Freunde nicht sterben lassen. Nicht du/Ich, nicht Hermine, nicht Mama oder Papa, nicht Neville oder Draco oder sonst jemand, dies ist der Wille zu beschützen}… Er stellte sich Flügel aus Sonnenlicht vor, wie die Flügel des Patronus, die er ausgebreitet hatte, um diesem verängstigten Kind Schutz zu gewähren.

Der Patronus erhellte sich wieder, die Welt drehte sich um Harry oder war es sein eigener Verstand, der sich drehte?

\emph{Nimm meine Hand}, dachte Harry und stellte es sich vor, \emph{komm mit mir, und wir werden diese Sache} \emph{gemeinsamdurchziehen}…

In Harrys Verstand gab es einen Ruck, als hätte sein Gehirn einen Schritt nach links gemacht, oder als hätte das Universum einen Schritt nach rechts gemacht.

Und in einem hell erleuchteten Korridor in Askaban, in dem die schummrigen Gaslichter vom gleichmäßigen und unerschütterlichen Licht eines Patronus in Menschengestalt weit überstrahlt wurden, stand ein unsichtbarer Junge mit einem seltsamen kleinen Lächeln auf dem Gesicht und zitterte nur leicht.

Harry wusste irgendwie, dass er gerade etwas Bedeutendes getan hatte, etwas, das über die bloße Stärkung seines Widerstands gegen Dementoren hinausging.

Und mehr als das, er hatte sich \emph{erinnert}. Den Tod als eine anthropomorphe Gestalt zu sehen, hatte den Zweck erfüllt, ironischerweise. Jetzt konnte Harry sich daran erinnern, was angeblich jemanden vor den Blicken des Todes selbst verbergen sollte…

In einem Korridor in Askaban kamen die weit ausschreitenden Beine eines Zauberers abrupt zum Stillstand; denn das helle silberne Ding, das ihn führte, war in der Luft stehen geblieben und hatte unglücklich mit den Flügeln geflattert. Der strahlend weiße Phönix hob seinen Kopf, blickte verwirrt nach hinten und nach vorn als ob er verwirrt wäre, drehte sich dann zu seinem Meister um und schüttelte den Kopf, um sich zu entschuldigen.

Ohne ein weiteres Wort drehte sich der alte Zauberer um und ging den Weg zurück, den er gekommen war.

Harry stand gerade und aufrecht und fühlte, wie die Angst über ihn und um ihn herum schwappte. Ein winziger Teil von ihm mochte von den Wellen der Leere, die ständig an seinem unbeweglichen Stein zerbrachen, ein wenig erodiert worden sein, aber seine Glieder waren nicht kalt, und seine Magie war mit ihm. Mit der Zeit könnten diese Wellen ihn zerfressen und verzehren und sich zu dem winzigen Teil von ihm durchschleichen, der noch vor dem Tod niederkauerte, anstatt seine Angst zu nutzen, um sich für den Kampf zu energetisieren. Aber dieser Untergang würde Zeit brauchen, da die Schatten des Todes weit entfernt waren und keine Notiz von ihm nahmen. Der Makel, der Riss, die Bruchlinie, die in ihm war, war repariert worden, und die Sterne leuchteten hell in seinem Geist, riesig und furchtlos und strahlend inmitten von Kälte und Dunkelheit.

Für jeden anderen hätte es so ausgesehen, als stünde der Junge allein in dem schwach beleuchteten Metallgang und trüge dieses seltsame Lächeln.

Denn Bellatrix Black und die um ihre Schultern drapierte Schlange waren durch den Unsichtbarkeitsmantel verborgen, einer der drei Heiligtümer des Todes, der angeblich seinen Träger vor dem Blick des Todes selbst verbergen sollte. Das Rätsel, dessen Antwort verloren gegangen war und das Harry neu gefunden hatte.

Und Harry wusste nun, dass die Verschleierung des Umhangs mehr war als die bloße Transparenz der Desillusionierung, dass der Umhang dich ~\emph{verborgen} hielt und nicht nur unsichtbar, genau so nicht zu sehen wie die Thestrale für die Unwissenden. Und Harry wusste auch, dass es Thestralblut war, das das Symbol der Heiligtümer des Todes auf die Innenseite des Umhangs malte, das den Teil der Macht des Todes in den Umhang einband und es dem Umhang ermöglichte, den Dementoren auf ihrer eigenen Ebene entgegenzutreten und sie zu blockieren. Es fühlte sich an wie eine Vermutung, und doch eine sichere Vermutung, das Wissen wurde ihm im Augenblick der Lösung des Rätsels zuteil.

Bellatrix war innerhalb des Mantels immer noch durchsichtig, aber für Harry war sie nicht länger verborgen, er wusste, dass sie da war, für ihn so offensichtlich wie ein Thestral. Denn Harry hatte seinen Umhang nur geliehen, nicht geschenkt, und er hatte das Heiligtum des Todes verstanden und gemeistert das durch die Potter-Linie überliefert worden war.

Harry blickte die unsichtbare Frau direkt an und sagte: "Können die Dementoren dich erreichen, Bella?

"Nein", sagte die Frau mit leiser, staunender Stimme. Dann sagte sie: "Aber mein Herr… \emph{Ihr}…"

"Wenn du etwas Dummes sagst, wird mich das ärgern", sagte Harry kalt. "Oder hast du den Eindruck, dass ich mich für dich opfern würde?"

"Nein, mein Lord", antwortete die Dienerin des Dunklen Lords, klang verwirrt und vielleicht sogar ehrfürchtig.

"Folge mir", sprach Harrys kaltes Flüstern.

Und sie setzten ihre Reise nach unten fort, als der Dunkle Lord in seinen Beutel griff, einen Keks nahm und ihn aß. Wenn Bellatrix gefragt hätte, hätte Harry behauptet, es sei wegen der Schokolade, aber sie fragte nicht.

Der alte Zauberer schritt zurück in die Mitte der Auroren, der silberne und der rot-goldene Phönix folgten nun hinter ihm.

"\emph{Du} -" Amelia begann zu brüllen.

"Sie haben ihren Patronus entlassen", sagte Dumbledore. Der alte Zauberer schien seine Stimme nicht zu erheben, aber seine ruhigen Worte übertönten irgendwie ihre eigenen. "Ich kann sie jetzt nicht finden."

Amelia knirschte mit den Zähnen, legte eine Reihe von beleidigenden Bemerkungen auf Eis und wandte sich an den Kommunikationsoffizier. "Das Dienstzimmer soll die Dementoren \emph{noch einmal} fragen, ob sie Bellatrix Black aufspüren können."

Die Kommunikationsspezialistin sprach einen Moment lang mit ihrem Spiegel, und ein paar Sekunden später schaute sie überrascht auf. "Nein -"

Amelia fluchte bereits heftig in Gedanken.

"- aber sie können auf den unteren Ebenen jemand anderen sehen, der kein Gefangener ist."

"Gut!", fauchte Amelia. "Sagen Sie dem Dementor, dass ein Dutzend seiner Art befugt sind, Askaban zu betreten und denjenigen zu ergreifen, der es ist, und jeden in seiner Gesellschaft! Und wenn sie Bellatrix Black sehen, sollen sie ihr sofort den Kuss geben!"

Amelia drehte sich um und blickte Dumbledore an und forderte ihn heraus zu diskutieren; aber der alte Zauberer sah sie nur ein wenig traurig an und schwieg.

Auror McCusker beendete das Gespräch mit der Leiche, die vor dem Fenster schwebte, er hatte die Befehle der Direktorin übermittelt.

Die Leiche schenkte ihm ein tödliches Lächeln, das seine Glieder zum Schlottern brachte und schwebte dann nach unten.

Bald darauf erhob sich ein Dutzend Dementoren von dort, wo sie in die zentrale Grube von Askaban getrieben hatten, und begab sich nach außen, zu den Wänden der riesigen Metallstruktur, die über ihnen aufragte.

Durch Löcher, die in den Sockel von Askaban gebohrt worden waren, traten die dunkelsten aller Kreaturen ihren Schreckensmarsch an.

