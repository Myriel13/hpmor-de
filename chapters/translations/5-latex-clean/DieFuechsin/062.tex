

\hypertarget{das-stanford-prison-experiment-finale}{% \section{30. Das Stanford-Prison-Experiment, Finale}\label{das-stanford-prison-experiment-finale}}

-\/-\/-\/-\/-\/- Kapitel Zweiundsechzig: Das Stanford-Prison-Experiment, Finale -\/-\/-\/-\/-\/-

Minerva blickte auf die Uhr, die goldenen Zeiger und silbernen Ziffern, die ruckartigen Bewegungen. Muggel hatten das erfunden, und bis dahin hatten sich Zauberer nicht darum gekümmert, die Zeit zu messen. Durch Sanduhr gesteuerte Glocken hatten Hogwarts zum Zeitpunkt ihrer Erbauung für seine Schulstunden gedient. Das war eines der Dinge, von denen sich die Blutpuristen wünschten, dass sie nicht wahr wären, und deshalb wusste Minerva das.

Sie hatte für ihren Muggelstudien U.T.Z.* ein ‚Ohnegleichen` erhalten, was ihr jetzt ein Zeichen der Schande zu sein schien, wenn man bedachte, wie wenig sie wusste. Ihr jüngeres Ich hatte schon damals erkannt, dass die Klasse eine Täuschung war, die von einem Reinblüter unterrichtet wurde, angeblich weil Muggelgeborene nicht verstehen konnten, was Zauberergeborenen beigebracht wurde, und weil der Schulrat Muggel überhaupt nicht zugelassen hatte. Aber als sie siebzehn Jahre alt war, war die herausragende Note die Hauptsache, die ihr wichtig war, und sie erinnerte sich mit Trauer daran…

\emph{\emph{Wenn Harry Potter und Voldemort ihren Krieg mit Muggelwaffen führen, wird von der Welt nichts anderes übrigbleiben als Feuer!}}

Sie konnte sich das nicht vorstellen, und der Grund dafür war, dass sie sich nicht vorstellen konnte, dass Harry gegen Du-weißt-schon-wen kämpfte.

Sie war dem Dunklen Lord vier Mal begegnet und hatte jedes Mal überlebt, dreimal mit Albus als Schutzschild und einmal mit Moody an ihrer Seite. Sie erinnerte sich an das lädierte, schlangenartige Gesicht, die schwach sichtbaren grünen Schuppen, die über die Haut verstreut waren, die glühend roten Augen, die Stimme, die in einem hohen Zischen lachte und nichts als Grausamkeit und Qualen versprach: ein Monster in Reinkultur.

Harry Potter konnte sie sich auch leicht vorstellen, der helle Ausdruck auf dem Gesicht eines kleinen Jungen, der dazwischen schwankte, das Lächerliche ernst zu nehmen und das Ernste lächerlich zu machen.

Doch der Gedanke an die beiden, wie sie sich mit ihren Zauberstäben gegenüberstanden, war zu schmerzhaft, um es sich vorzustellen.

Sie hatten kein Recht, überhaupt kein Recht, dies einem elfjährigen Jungen aufzubürden. Sie wusste, was der Schulleiter an diesem Tag für ihn entschieden hatte, denn man hatte ihr gesagt, sie solle die Vorkehrungen treffen; und wäre es um sie selbst im gleichen Alter gegangen, hätte sie wochenlang gewütet und geschrien und geweint und wäre untröstlich gewesen, und…

\emph{\emph{Er ist kein gewöhnlicher Erstklässler,} hatte Albus gesagt. \emph{Er ist als dem Dunklen Lord ebenbürtig gekennzeichnet, und er hat Macht, die der Dunkle Lord nicht kennt.}}

Die schreckliche hohle Stimme, die aus Sybill Trelawneys Kehle dröhnte, die wahre und ursprüngliche Prophezeiung, hallte noch einmal durch ihren Verstand. Sie hatte das Gefühl, dass es nicht das bedeutete, was der Schulleiter dachte, aber es gab keine Möglichkeit, den Unterschied in Worte zu fassen.

Und dennoch schien es immer noch wahr zu sein, dass, wenn es auf der ganzen Erde einen Elfjährigen gäbe, der diese Last tragen könnte, dieser Junge jetzt auf dem Weg zu ihrem Büro war. Und wenn sie ihm irgendetwas etwas wie ‚armer Harry` ins Gesicht sagte… nun, es würde ihm nicht gefallen.

\emph{Jetzt muss ich also einen Weg finden, um einen unsterblichen dunklen Zauberer zu töten}, hatte Harry an dem Tag gesagt, an dem er es zum ersten Mal erfahren hatte. \emph{Ich wünschte wirklich, Sie hätten mir das gesagt, bevor ich eingekauft habe}…

Sie war lange genug Leiterin des Hauses Gryffindor gewesen, sie hatte genug Freunde sterben sehen, um zu wissen, dass es Menschen gab, die man nicht davor retten konnte, Helden zu werden.

Es klopfte an der Tür, und Professor McGonagall sagte: "Treten Sie ein."

Als Harry hereinkam, hatte sein Gesicht denselben kalten, wachen Blick, den sie schon in Mary's Platz gesehen hatte; und sie fragte sich einen Augenblick lang, ob er den ganzen Tag dieselbe Maske getragen hatte, dasselbe Selbst.

Der kleine Junge setzte sich auf den Stuhl vor ihrem Schreibtisch und sagte: „Ist es jetzt an der Zeit, mir zu sagen, was vor sich geht?“ Die Worte waren neutral und hatten nicht die Schärfe, die mit dem Gesichtsausdruck hätte einhergehen sollen.

Professor McGonagalls Augen erhoben sich überrascht, bevor sie sie stoppen konnte, und sie sagte: "Der Schulleiter hat Ihnen nichts gesagt, Mr. Potter?

Der Junge schüttelte den Kopf. "Nur, dass er eine Warnung erhalten hatte, dass ich in Gefahr sein könnte, aber jetzt wäre ich in Sicherheit."

Minerva hatte Schwierigkeiten, seinem Blick zu begegnen. Wie konnten sie ihm das antun, wie konnten sie das einem elfjährigen Jungen \emph{antun}, diesen Krieg, dieses Schicksal, diese Prophezeiung… und sie \emph{vertrauten} ihm nicht einmal…

Sie zwang sich, Harry direkt anzuschauen, und sah, dass seine grünen Augen ruhig waren, während sie auf ihr ruhten.

"Professor McGonagall?", sagte der Junge leise.

"Mr. Potter", sagte Professor McGonagall, „ich fürchte, es steht mir nicht zu, dies zu erklären, aber wenn der Schulleiter Ihnen hiernach \emph{immer} \emph{noch} nichts sagt, können Sie zu mir zurückkommen, und ich werde ihn für Sie anschreien.“

Die Augen des Jungen weiteten sich, etwas von dem echten Harry zeigte sich durch den Riss, bevor die kühle Maske wieder aufgesetzt wurde.

"Auf jeden Fall", sagte Professor McGonagall rasch. „Entschuldigen Sie die Unannehmlichkeiten, Mr. Potter, aber ich muss Sie bitten, mit Ihrem Zeitumkehrer sechs Stunden bis drei Uhr zurückzugehen und Professor Flitwick die folgende Nachricht zu übermitteln: ‚Silber auf dem Baum`. Bitten Sie den Professor, die Uhrzeit zu notieren, zu der Sie ihm diese Nachricht gegeben haben. Danach möchte sich der Schulleiter mit Ihnen treffen, wann immer es Ihnen passt.“

Es gab eine Pause.

Dann sagte der Junge: „Also werde ich verdächtigt, meinen Zeitumkehrer missbraucht zu haben?“

"Nicht von \emph{mir}!" sagte Professor McGonagall hastig. "Ich \emph{entschuldige} mich für die Unannehmlichkeiten, Mr. Potter."

Es gab eine weitere Pause, und dann zuckte der Junge mit den Achseln. „Es wird meinen Schlafplan durcheinanderbringen, aber ich nehme an, es lässt sich nicht ändern. Bitte lassen Sie die Hauselfen wissen, dass, wenn ich morgen früh um, sagen wir, drei Uhr um ein frühes Frühstück bitte, ich es auch kriege.“

"Natürlich, Mr. Potter", sagte sie. "Danke für Ihr Verständnis."

Der Junge erhob sich von seinem Stuhl und nickte ihr förmlich zu, dann schlüpfte er zur Tür hinaus mit der Hand unter dem Hemd, wo sein Zeitumkehrer wartete; und sie rief beinahe \emph{Harry!} nur hätte sie nicht gewusst, was sie danach sagen sollte.

Stattdessen wartete sie, ihre Augen auf die Uhr gerichtet.

Wie lange musste sie warten, bis Harry Potter in der Zeit zurückging?

Eigentlich brauchte sie überhaupt nicht zu warten; wenn er es getan hatte, dann war es schon geschehen…

Minerva wusste also, dass sie zögerte, weil sie nervös war, und die Erkenntnis machte sie traurig. Unfug, ja, unaussprechlicher, unvorstellbarer Unfug mit all der Vorsicht und Weitsicht eines Steinschlags - sie wusste nicht, wie der Junge den Hut dazu gebracht hatte, ihn nicht nach Gryffindor zu sortieren, wo er offensichtlich hingehörte - aber nichts Dunkles oder Schädliches, niemals. Unter diesem Unfug steckte seine Güte so tief und so wahrhaftig wie die der Weasley-Zwillinge, obwohl nicht einmal der Cruciatus-Fluch sie dazu hätte bringen können, dies laut auszusprechen.

"\emph{Expecto Patronum}", sagte sie, und dann: „Gehe zu Professor Flitwick, und bring mir seine Antwort zurück, nachdem du ihn Folgendes gefragt hast: 'Hat Mr. Potter Ihnen eine Nachricht von mir gegeben, was war das für eine Nachricht, und wann haben Sie sie erhalten?`\,“

Eine Stunde zuvor hatte Harry, nachdem er die letzte verbleibende Drehung seines Zeitumkehrers nach dem Anlegen des Tarnumhangs benutzt hatte, die Sanduhr wieder in sein Hemd gesteckt.

Und er machte sich auf den Weg zu den Verliesen von Slytherin, wobei er so schnell ausschritt, wie es seine unsichtbaren Beine schafften, ohne zu rennen. Glücklicherweise befand sich das Büro der stellvertretenden Schulleiterin bereits in einer unteren Etage von Hogwarts…

Ein paar Treppenhäuser später, zwei, aber nicht drei Stufen auf einmal nehmend, blieb Harry an einem Korridor stehen, hinter dessen letzter Biegung der Eingang zu den Schlafsälen Slytherins lag.

Harry nahm ein Stück Pergament (nicht Papier) aus seinem Beutel, nahm eine flinke Feder (keinen Stift) aus seinem Beutel und sagte der Feder: "Schreib diese Briefe genau so, wie ich sie sage: Z-P-G-B-S-Y, Leerzeichen, F-V-Y-I-R-E-B-A-G-U-R-G-E-R-R".

Es gab zwei Arten von Codes in der Kryptographie, Codes, die deinen kleinen Bruder davon abhielten, deine Botschaft zu lesen, und Codes, die große Regierungen davon abhielten, deine Botschaft zu lesen, und dies war die erstere Art von Code, aber besser als nichts. Theoretisch sollte ihn sowieso niemand lesen; aber selbst wenn sie es täten, würden sie sich an nichts Interessantes erinnern, wenn sie nicht zuerst Kryptographie anwenden würden.

Harry steckte dieses Pergament dann in einen Pergamentumschlag und schmolz mit seinem Zauberstab etwas grünes Wachs, um es zu versiegeln.

Im Prinzip hätte Harry all das natürlich schon Stunden früher tun können, aber irgendwie schien das Warten, bis \emph{nachdem} er die Botschaft von Professor McGonagalls eigenen Lippen hörte, weniger Sich Mit Der Zeit Anlegen zu sein.

Harry steckte diesen Umschlag dann in einen anderen Umschlag, der bereits ein weiteres Blatt Papier mit anderen Anweisungen und fünf silberne Sickel enthielt.

Er schloss diesen Umschlag (auf dessen Außenseite bereits ein Name geschrieben stand), versiegelte ihn mit mehr grünem Wachs und drückte einen letzten Sickel in dieses Siegel.

Dann steckte Harry \emph{diesen} Umschlag in den allerletzten Umschlag, auf dem in großen Buchstaben der Name "Merry Tavington" stand.

Und Harry spähte um die Biegung herum, wo das finstere Porträt wartete, das als Tür zu den Schlafsälen Slytherins diente; und da er nicht wollte, dass sich das Porträt daran erinnerte, jemand unsichtbaren nicht-gesehen zu haben, benutzte Harry den Schwebezauber, um den Umschlag zu dem finsteren Mann zu schweben und ihn gegen ihn klopfen zu lassen.

Der finster dreinblickende Mann blickte durch ein Monokel auf den Umschlag, seufzte und drehte sich um, um sich dem Inneren des Slytherin-Schlafsaals zuzuwenden, und rief: „Botschaft für Merry Tavington!“

Der Umschlag durfte dann auf den Boden fallen.

Wenige Augenblicke später öffnete sich die Porträttür, und Merry schnappte sich den Umschlag vom Boden.

Sie öffnete ihn und fand einen Sickel und einen Umschlag, der an eine Studentin im vierten Jahr namens Margaret Bulstrode adressiert war.

(Slytherins machten so etwas ständig, und ein Sickel war definitiv ein Eilauftrag).

Margaret würde \emph{ihren} Umschlag öffnen und fünf Sickel zusammen mit einem Umschlag finden, der in einem unbenutzten Klassenzimmer abgegeben werden sollte…

…\emph{nachdem} sie mit ihrem Zeitumkehrer fünf Stunden zurück gegangen war,…

… woraufhin sie weitere fünf Sickel vorfinden würde, die auf sie warteten, wenn sie schnell dorthin kam.

Und ein unsichtbarer Harry Potter wartete in diesem Klassenzimmer von 15.00 Uhr bis 15.30 Uhr, nur für den Fall, dass jemand den offensichtlichen Test versuchte.

Für Professor Quirrell war es jedenfalls offensichtlich gewesen.

Professor Quirrell war auch klar gewesen, dass (a) Margaret Bulstrode einen Zeitumkehrer hatte und (b) dass sie nicht sehr genau darauf achtete, wie sie ihn benutzte, z.B. indem sie ihrer jüngeren Schwester wirklich gute Klatschgeschichten erzählte, "bevor" jemand anderes sie gehört hatte.

Ein Teil der Spannung entwich aus Harry, als er von der Porträttür wegging, immer noch unsichtbar. Irgendwie hatte es sein Verstand geschafft, sich über den Plan Gedanken zu machen, obwohl er \emph{wusste}, dass er bereits erfolgreich war. Jetzt blieb nur noch die Konfrontation mit Dumbledore, und dann war er für den Tag fertig… er würde um 21 Uhr zu den Wasserspeiern des Schulleiters gehen, da es verdächtiger erschien es um 20 Uhr zu tun. Auf diese Weise konnte er behaupten, er habe einfach missverstanden, was Professor McGonagall mit "danach" gemeint hatte…

Ein obskurer Schmerz durchzuckte Harrys Herz, als er an Professor McGonagall dachte.

Also zog sich Harry ein wenig weiter in seine dunkle Seite zurück, die den ruhigen Gesichtsausdruck getragen und die Müdigkeit von seinem Gesicht ferngehalten hatte, und ging weiter.

Die Abrechnung dafür würde kommen, aber manchmal musste man einen Tag komplett auf Pump leben und konnte erst am nächsten Tag bezahlen.

Sogar Harrys dunkle Seite fühlte die Erschöpfung, als die Wendeltreppe ihn zur großen Eichentür gebracht hatte, die das letzte Tor zu Dumbledores Büro darstellte; aber da Harry nun \emph{rechtlich gesehen} vier Stunden über seiner natürlichen Schlafenszeit war, war es ungefährlich, etwas von der Müdigkeit zu zeigen, wenn nicht von der emotionalen, so doch von der körperlichen.

Die eichene Tür schwang auf -

Harrys Augen waren bereits auf den großen Schreibtisch und den Thron dahinter gerichtet; daher dauerte es einen Moment, bis er bemerkte, dass der Thron verwaist war, der Schreibtisch leer, bis auf ein einziges ledergebundenes Buch; und dann ließ Harry seinen Blick schweifen, um den Zauberer zu finden, der zwischen seinen ausgetüftelten Dingen stand, den vielen geheimnisvollen, unbekannten Geräten. Fawkes und der Sprechende Hut besetzten ihre jeweiligen Sitzstangen, ein helles, fröhliches Feuer knisterte in einer Ecke, von der Harry vorher nicht bemerkt hatte, dass es sich um einen Kamin handelte, und da waren die zwei Regenschirme und drei rote Pantoffeln für linke Füße. Alle Dinge an ihrem Platz und in ihrer üblichen Erscheinung, außer dem alten Zauberer selbst, der aufrecht stand und in Gewänder in förmlichstem Schwarz gekleidet war. Es war ein Schock für die Augen, diese Gewänder an dieser Person, es war, als hätte Harry seinen Vater in einem Geschäftsanzug gesehen.

Die Erscheinung von Albus Dumbledore war sehr alt und traurig.

"Hallo, Harry„, sagte der alte Zauberer.

Aus einem alternativen Selbst heraus, dass er wie eine Okklumentik-Barriere aufrechterhielt, nickte grüßend ein unschuldiger Harry, der absolut keine Ahnung hatte, was passiert war, und sagte in kaltem Tonfall: „Schulleiter. Ich vermute, dass Sie mittlerweile von der stellvertretenden Schulleiterin McGonagall gehört haben. Wenn Sie nichts dagegen haben, würde ich jetzt \emph{wirklich} gerne wissen was los ist.“

"Ja", sagte der alte Zauberer, "es ist Zeit, Harry Potter." Der Rücken richtete sich auf, nur ein wenig, denn der Zauberer hatte bereits geradegestanden; aber irgendwie ließ selbst diese kleine Veränderung den Zauberer einen Fuß größer und stärker, wenn nicht sogar jünger erscheinen, furchterregend, wenn auch nicht gefährlich, seine Macht umgab ihn wie eine Kutte. Mit klarer Stimme sprach er: "An diesem Tag hat dein Krieg gegen Voldemort begonnen."

"Was?", sagte der äußere Harry, der nichts wusste, während etwas, das von innen zuschaute, das Gleiche dachte, nur mit viel mehr Profanität.

"Bellatrix Black ist aus Askaban befreit worden, sie ist aus einem Gefängnis geflohen, aus dem man nicht fliehen kann", sagte der alte Zauberer. „Es ist eine Tat, die Voldemorts Handschrift trägt, da bin ich mir sicher; und sie, seine treueste Dienerin, ist eine der drei Requisiten, die er erlangen muss, um in einem neuen Körper wieder auferstehen zu können. Nach zehn Jahren ist der Feind, den du einst besiegt hast, zurückgekehrt, wie es vorhergesagt wurde.“

Keinem der beiden Teile Harrys fiel etwas ein, was er erwidern könnte, zumindest nicht in den wenigen Sekunden, bevor der alte Zauberer weitersprach.

"Für dich wird sich vorerst wenig verändern", sagte der alte Zauberer. „Ich habe begonnen, den Orden des Phönix, der dir helfen wird, wiederherzustellen, ich habe die wenigen Seelen, die es verstehen können und sollen, gewarnt: Amelia Bones, Alastor Moody, Bartemius Crouch und einige andere. Von der Prophezeiung - ja, es gibt eine Prophezeiung - habe ich ihnen nichts gesagt, aber sie wissen, dass Voldemort zurückgekehrt ist, und sie wissen, dass du eine wichtige Rolle spielen wirst. Sie und ich werden deinen Krieg in seinen ersten Anfängen kämpfen, während du hier in Hogwarts stärker und vielleicht auch weiser wirst.“ Die Hand des alten Zauberers erhob sich fast flehentlich. "Für dich gibt es nur eine einzige Veränderung, und ich bitte dich inständig, ihre Notwendigkeit zu verstehen. Erkennst du das Buch auf meinem Schreibtisch, Harry?"

Der innere Teil von Harry schrie und schlug mit dem Kopf gegen imaginäre Wände, während der äußere Harry sich umdrehte und auf das starrte, was sich als -

Es gab eine ziemlich lange Pause.

Dann sagte Harry: „Das ist eine Kopie von \emph{Der Herr der Ringe} von J. R. R. Tolkien.“

"Du hast ein Zitat aus diesem Buch erkannt", sagte Dumbledore, einen entschlossenen Blick in seinen Augen, "also nehme ich an, dass du dich gut daran erinnerst. Korrigiere mich, wenn ich mich irre."

Harry starrte ihn nur an.

"Es ist wichtig zu verstehen", sagte Dumbledore, "dass dieses Buch keine realistische Darstellung eines Zaubererkrieges ist. John Tolkien hat nie gegen Voldemort gekämpft. Dein Krieg wird nicht wie die Bücher sein, die du gelesen hast. Das wirkliche Leben ist nicht wie Geschichten. Verstehst du das, Harry?"

Harry nickte ziemlich langsam ja, und dann schüttelte er den Kopf nein.

"Insbesondere", so Dumbledore, "hat Gandalf im ersten Buch etwas sehr Dummes getan. Er macht viele Fehler, dieser Zauberer von Tolkien; aber dieser eine Fehler ist der unverzeihlichste. Dieser Fehler ist: Als Gandalf zum ersten Mal, und sei es auch nur für einen Moment, vermutete, dass Frodo den Einen Ring hält, hätte er Frodo \emph{sofort} nach Bruchtal versetzen müssen. Es wäre ihm, diesem alten Zauberer, vielleicht peinlich gewesen, wenn sich sein Verdacht als falsch erwiesen hätte. Es wäre ihm vielleicht unangenehm gewesen, Frodo so herum zu kommandieren, und Frodo hätte große Unannehmlichkeiten gehabt, weil er viele andere Pläne und Hobbies hätte absagen müssen. Aber ein wenig Peinlichkeit und Unbehagen und Unannehmlichkeiten sind nichts im Vergleich zu dem Verlust deines ganzen Krieges, wenn die neun Nazguls über das Auenland herfallen, und den Ring sofort an sich nehmen, während du alte Schriftrollen in Minas Tirith liest. Und es ist nicht Frodo allein, der verletzt worden wäre; ganz Mittelerde wäre versklavt worden. Wenn es \emph{nicht} nur eine Geschichte gewesen wäre, Harry, hätten sie ihren Krieg verloren. Verstehst du, was ich sage?"

"Äh…" sagte Harry, "nicht genau…" Dumbledore hatte etwas an sich, wenn er so war, dass es schwer machte, richtig kalt zu bleiben; seine dunkle Seite hatte Probleme mit dem Seltsamen.

"Dann werde ich es aussprechen", sagte der alte Zauberer. Seine Stimme war streng, seine Augen waren traurig. „Frodo hätte von Gandalf selbst sofort nach Bruchtal gebracht werden müssen - und Frodo hätte Bruchtal niemals ohne Wache verlassen dürfen. Es hätte keine Nacht des Schreckens in Bree geben dürfen, keine Hünengräber, keine Wetterspitze, auf der Frodo verwundet wurde, sie hätten jederzeit ihren ganzen Krieg verlieren können, für Gandalfs Torheit! Verstehst du nun, was ich dir sage, Sohn von Michael und Petunia?“

Und der Harry, der nichts wusste, hatte verstanden.

Und der Harry, der nichts wusste, sah, dass es das Kluge, das Weise, das Intelligente und Vernünftige war, das einzig \emph{Richtige}.

Und der Harry, der nichts wusste, sagte genau das, was ein unschuldiger Harry gesagt \emph{hätte}, während der stille Wächter verwirrt und qualvoll schrie.

"Sie sagen", sagte Harry, seine Stimme zitterte, als die Emotionen im Innern die äußere Ruhe durchbrannten, "dass ich zu Ostern nicht nach Hause zu meinen Eltern gehen werde".

"Du \emph{wirst} sie wiedersehen", sagte der alte Zauberer schnell. "Ich werde sie bitten, hierher zu kommen, um bei dir zu sein, ich werde ihnen bei ihren Besuchen jede Höflichkeit erweisen. Aber du gehst zu Ostern nicht nach Hause, Harry. Du fährst den Sommer über nicht nach Hause. Du wirst nicht mehr in der Winkelgasse zu Mittag essen, auch nicht mit Professor Quirrell als Aufpasser. Dein Blut ist das Zweite, das Voldemort braucht, um so stark wie zuvor zurückzukehren. Du wirst also nie wieder die Grenzen von Hogwarts' Schutzzaubern ohne einen lebenswichtigen Grund verlassen, sowie eine Wache, die stark genug ist, jeden Angriff lange genug abzuwehren, um dich in Sicherheit zu bringen. "

In den Augenwinkeln von Harry begann das Wasser zu fließen. "Ist das eine Bitte?", sagte seine zitternde Stimme. "Oder ein Befehl?"

"Tut mir Leid, Harry", sagte der alte Zauberer leise. "Deine Eltern werden die Notwendigkeit erkennen, hoffe ich; aber wenn nicht… Ich fürchte, sie haben keinen Rechtsbehelf; das Gesetz erkennt sie, so falsch es auch sein mag, nicht als deine Vormünder an. Es tut mir leid, Harry, und ich verstehe, wenn du mich dafür verachtest, aber es muss getan werden."

Harry wirbelte herum und schaute zur Tür, er konnte Dumbledore nicht mehr ansehen, konnte seinem eigenen Gesicht nicht mehr trauen.

\emph{Das ist der Preis,} \emph{den es dich kostet}, sagte Hufflepuff in seinen Gedanken, \emph{so wie} \emph{du anderen Kosten auferlegt hast. Wird das} \emph{deine} \emph{ganze Sicht auf die Sache ändern, so wie Professor Quirrell es sich vorstellt?}

Automatisch sagte die Maske des unschuldigen Harry genau das, was sie gesagt hätte: "Sind meine Eltern in Gefahr? Müssen \emph{sie} hierhergebracht werden?"

"Nein", sagte die Stimme des alten Zauberers. „Das glaube ich nicht. Die Todesser lernten gegen Ende des Krieges, die Familien des Ordens nicht anzugreifen. Und wenn Voldemort jetzt ohne seine früheren Gefährten handelt, weiß er immer noch, dass ich es bin, der die Entscheidungen für den Augenblick trifft, und er weiß, dass ich ihm nichts für die Bedrohung deiner Familie geben würde. Ich habe ihm beigebracht, dass ich einer Erpressung nicht nachgebe, und deshalb wird er es auch nicht versuchen.“

Harry drehte sich daraufhin um und sah eine Kälte auf dem Gesicht des alten Zauberers, die seiner veränderten Stimme entsprach, Dumbledores blaue Augen, die hinter der Brille hart wie Stahl wurden, sie passten nicht zu der Person, aber sie passte zu den formellen schwarzen Gewändern.

"Ist das dann alles?", sagte Harrys zitternde Stimme. Später würde er darüber nachdenken, später würde ihm eine listige Gegenmaßnahme einfallen, später würde er Professor Quirrell fragen, ob es eine Möglichkeit gäbe, den Schulleiter davon zu überzeugen, dass er sich geirrt habe. Im Augenblick nahm das Aufrechterhalten der Maske Harrys ganze Aufmerksamkeit in Anspruch.

"Voldemort benutzte ein Muggelartefakt, um Askaban zu entkommen", sagte der alte Zauberer. "Er beobachtet dich und lernt von dir, Harry Potter. Bald wird ein Mann namens Arthur Weasley im Ministerium ein Edikt erlassen, dass jegliche Verwendung von Muggelartefakten in den Kämpfen des Verteidigungsprofessors eingestellt werden muss. In Zukunft, wenn du eine gute Idee hast, behalte sie für dich".

Das schien im Vergleich dazu nicht wichtig zu sein. Harry nickte nur und sagte erneut: "Ist das alles?"

Es gab eine Pause.

"Bitte", sagte der alte Zauberer im Flüsterton. „Ich habe kein Recht, dich um Verzeihung zu bitten, Harry James Potter-Evans-Verres, aber bitte, sag wenigstens, dass du verstehst, warum.“ Da waren Tränen in den Augen des alten Zauberers.

"Ich verstehe", sagte die Stimme des äußeren Harry, der verstand, "ich meine… ich habe sowieso irgendwie darüber nachgedacht… und mich gefragt, ob ich Sie und meine Eltern dazu bringen könnte, mich im Sommer in Hogwarts bleiben zu lassen, wie die Waisenkinder, damit ich die Bibliothek hier lesen kann, es ist in Hogwarts sowieso interessanter…"

Aus der Kehle von Albus Dumbledore kam ein erstickendes Geräusch.

Harry drehte sich wieder zur Tür. Es war keine unversehrte Flucht, aber es war Flucht.

Er machte einen Schritt nach vorn.

Seine Hand griff nach dem Türgriff.

Ein durchdringender Schrei spaltete die Luft -

Wie in Zeitlupe, als Harry sich drehte, sah er, wie der Phönix bereits durch die Luft schoss und auf ihn zuschwebte.

Vom wahren Harry, der seine eigene Schuld kannte, kam blitzartig Panik, daran hatte er nicht gedacht, nicht damit gerechnet, er hatte sich auf Dumbledore vorbereitet, aber er hatte \emph{Fawkes} vergessen -

Flügelschlag, Flügelschlag und Flügelschlag, dreimal schlugen die Flügel des Phönix wie das Aufflammen und Abklingen eines Feuers, die Zeit schien zu langsam zu vergehen, als Fawkes über die mysteriösen Vorrichtungen in Richtung von Harry schwebte.

Und der rot-goldene Vogel schwebte mit sanften Flügelschlägen vor ihm und wippte in der Luft wie eine Kerzenflamme.

"Was ist los, Fawkes", sagte der falsche Harry verwirrt und sah dem Phönix in die Augen, wie er es tun würde, wenn er unschuldig wäre. Der echte Harry fühlte sich innerlich genauso schrecklich krank wie damals, als Professor McGonagall ihm ihr Vertrauen ausgesprochen hatte, und dachte: "\emph{Bin ich heute böse geworden, Fawkes? Ich dachte nicht, dass ich böse} \emph{war}… \emph{Hasst du mich jetzt? Wenn ich etwas geworden bin, das ein Phönix hasst, sollte ich vielleicht jetzt einfach aufgeben, jetzt alles aufgeben und gestehen} -

Fawkes schrie, der schrecklichste Schrei, den Harry je gehört hatte, ein Schrei, der alle Geräte in Schwingung versetzte und alle schlafenden Figuren in ihren Porträts zum Vorschein brachte.

Er drang durch Harrys gesamte Verteidigung wie ein weißglühendes Schwert durch Butter, zerbrach alle seine Schichten wie ein durchgestochener Luftballon, der zerplatzte, und änderte seine Prioritäten in diesem einen Augenblick, als er sich an das Wichtigste erinnerte; die Tränen flossen frei aus Harrys Augen, über seine Wangen, seine Stimme erstickte, als die Worte aus seiner Kehle kamen, als hustete er Lava aus -

"Fawkes sagt", sagte Harrys Stimme, "er will, dass ich etwas wegen der Gefangenen in Askaban tue -"

"Fawkes, \emph{nein!}" sagte der alte Zauberer. Dumbledore schritt vorwärts und streckte dem Phönix eine flehende Hand entgegen. Die Stimme des alten Zauberers war fast so verzweifelt wie der Schrei des Phönix. "Das kannst du nicht von ihm verlangen, Fawkes, er ist noch ein Kind!"

\emph{"Sie sind nach Askaban gegangen", flüsterte Harry, "Sie haben Fawkes mitgenommen, er sah -- \emph{Sie} sahen -- Sie waren \emph{dort}, Sie sahen - WARUM HABEN SIE NICHTS GETAN? WARUM HABEN SIE SIE NICHT RAUSGELASSEN?"}

Als die Instrumente aufhörten zu vibrieren, erkannte Harry, dass Fawkes zur gleichen Zeit wie er selbst geschrien hatte, dass der Phönix nun neben Harry flog und Dumbledore an seiner Seite gegenüberstand, den rot-goldenen Kopf auf gleicher Höhe mit seinem eigenen.

"Kannst du", flüsterte der alte Zauberer, „kannst du die Stimme des Phönix wirklich so deutlich hören?“

Harry schluchzte fast zu heftig, um sprechen zu können, denn all die Metalltüren, an denen er vorbeigegangen war, die Stimmen, die er gehört hatte, die schlimmsten Erinnerungen, das verzweifelte Betteln beim Weggehen, all das war über ihn hereingebrochen wie Feuer beim Schrei des Phönix, all die inneren Bollwerke waren zerstört. Harry wusste nicht, ob er die Stimme des Phönix wirklich so deutlich hören konnte, ob er Fawkes verstanden hätte, ohne es bereits zu wissen. Alles, was Harry wusste, war, dass er eine plausible Entschuldigung dafür hatte, die Dinge zu sagen, die Professor Quirrell ihm verboten hatte \emph{jemals} wieder in Gesprächen zu erwähnen; denn das war genau das, was ein unschuldiger Harry gesagt hätte, getan hätte, wenn er so klar gehört \emph{hätte}. "Sie leiden - wir müssen ihnen helfen -"

"Ich \emph{kann nicht!}" rief Albus Dumbledore. "Harry, Fawkes, ich \emph{kann nicht}, ich kann nichts tun!"

Noch ein durchdringender Schrei.

"WARUM NICHT? GEHEN SIE EINFACH REIN UND HOLEN SIE SIE RAUS!"

Der alte Zauberer riss seinen Blick vom Phönix los, stattdessen trafen seine Augen auf die von Harry. "Harry, erkläre es Fawkes für mich! Sag ihm, dass es nicht so einfach ist! Phönixe sind keine normalen Tiere, aber sie sind Tiere, Harry, sie können nicht verstehen, -"

"Ich verstehe es auch nicht", sagte Harry mit zitternder Stimme. "Ich verstehe nicht, warum Sie \emph{Menschen an Dementoren verfüttern! Askaban ist kein Gefängnis, es ist eine Folterkammer, und} \emph{Sie foltern diese Menschen zu Tode!"}

"Percival", sagte der alte Zauberer heiser, "Percival Dumbledore, mein eigener Vater, Harry, mein eigener Vater starb in Askaban! Ich weiß, ich weiß, es ist ein Horror! \emph{Aber was willst du von mir?} Askaban mit Gewalt zerstören? Willst du, dass ich einen offenen Aufstand gegen das Ministerium anzettle?"

KRÄCHZ!

Es gab eine Pause, und Harrys zitternde Stimme sagte: "Fawkes weiß nichts über Regierungen, er will nur, dass Sie - die Gefangenen aus ihren Zellen holen - und er wird Ihnen beim Kampf helfen, wenn sich Ihnen jemand in den Weg stellt - und - und das werde ich auch, Schulleiter! Ich werde mit Ihnen gehen und jeden Dementor vernichten, der sich Ihnen nähert! Wir kümmern uns später um die politischen Folgen. Ich wette, dass wir beide zusammen damit durchkommen -"

"Harry", flüsterte der alte Zauberer, „Phönixe verstehen nicht, wie ein Sieg in einer Schlacht einen Krieg verlieren kann.“ Dem alten Zauberer liefen Tränen über die Wangen, die in seinen silbernen Bart tropften. "Die Schlacht ist alles, was sie kennen. Sie sind gut, aber nicht weise. Deshalb wählen sie Zauberer als ihre Meister."

"Können Sie die Dementoren dorthin bringen, wo ich sie erreichen kann?" Harrys Stimme bettelte geradezu. "Bringen Sie sie in Fünfzehnergruppen heraus - ich glaube, ich könnte so viele auf einmal vernichten, ohne mich zu verletzen -"

Der alte Zauberer schüttelte den Kopf. "Es war schwer genug, den Verlust eines einzelnen durchgehen zu lassen - sie mögen mir vielleicht noch einen geben, aber niemals zwei - sie gelten als Nationalbesitz, Harry, Waffen im Kriegsfall -"

Wut loderte in Harry auf, loderte wie Feuer, sie könnte von dort gekommen sein, wo jetzt ein Phönix auf seiner Schulter ruhte, oder sie könnte von seiner eigenen dunklen Seite gekommen sein, und die beiden Wutausbrüche vermischten sich in ihm, die kalte und die heiße, und es war eine seltsame Stimme, die aus seiner Kehle sagte: „Erklären Sie mir etwas. Was muss eine Regierung tun, was müssen die Wähler mit ihrer Demokratie tun, was müssen die \emph{Menschen} in einem \emph{Land} tun, bevor ich entscheiden sollte, dass ich nicht mehr auf ihrer Seite bin?“

Die Augen des alten Zauberers weiteten sich, als er den Jungen mit dem Phönix auf der Schulter anstarrte. "Harry… sind das deine Worte, oder die des Verteidigungsprofessors -"

"Denn \emph{irgendwo} muss es doch eine Grenze geben, nicht wahr? Und wenn es nicht Askaban ist, wo ist es dann?"

"Harry, hör zu, bitte, hör mich an! Zauberer könnten nicht zusammenleben, wenn jeder von ihnen zur Rebellion gegen die Gemeinschaft aufruft, jedes Mal, wenn sie eine unterschiedliche Meinung haben! Es wird immer etwas geben -"

"\emph{Askaban ist nicht einfach irgendetwas! Es ist böse!}"

„Ja, sogar böse! Sogar manche Übel, Harry, denn Zauberer sind nicht vollkommen gut! Und doch ist es besser, dass wir in Frieden leben, als im Chaos; und für dich und mich wäre es der Beginn des \emph{Chaos}, wenn wir Askaban mit Gewalt zerstörten, siehst du das nicht?“ Die Stimme des alten Zauberers flehte. „Und es ist möglich, sich dem Willen deiner Mitmenschen offen oder im Verborgenen zu widersetzen, ohne sie zu \emph{hassen}, ohne sie zum Bösen oder Feind zu erklären! Ich glaube nicht, dass die Menschen dieses Landes das von dir verdienen, Harry! Und selbst wenn einige von ihnen es täten - was ist mit den Kindern, was mit den Schülern in Hogwarts, was mit den vielen guten Menschen, die sich mit den schlechten vermischt haben?“

Harry schaute auf seine Schulter, wo Fawkes sich hingehockt hatte, sah die Augen des Phönix zurückblicken, sie glühten nicht, und doch loderten sie, rote Flammen in einem Meer aus goldenem Feuer.

\emph{\emph{Was denkst du, Fawkes?}}

"Krächz?" sagte der Phönix.

Fawkes verstand das Gespräch nicht.

Der Junge blickte den alten Zauberer an und sagte mit belegter Stimme: "Oder vielleicht sind die Phönixe weiser als wir, klüger als wir, vielleicht folgen sie uns überall hin und hoffen, dass wir ihnen eines Tages \emph{zuhören}, eines Tages werden wir \emph{verstehen}, eines Tages werden wir die Gefangenen einfach \emph{aus} ihren \emph{Zellen} \emph{herausholen} -"

Harry drehte sich und zog die Tür aus Eichenholz auf, trat auf die Treppe und schlug die Tür hinter sich zu.

Die Treppe begann sich zu drehen und Harry herab zu befördern und er legte sein Gesicht in die Hände und begann zu weinen.

Erst als er halb unten war, bemerkte er das etwas anders war, bemerkte die Wärme, die sich noch immer in ihm ausbreitete, und erkannte, dass -

"Fawkes?" flüsterte Harry.

- der Phönix immer noch auf seiner Schulter saß, wie er ihn schon einige Male bei Dumbledore gesehen hatte.

Harry sah wieder in die Augen, rote Flammen in goldenem Feuer.

"Du bist jetzt nicht mein Phönix… oder?"

Krächzen!

"Oh", sagte Harry, seine Stimme zitterte ein wenig, "Ich bin froh, das zu hören, Fawkes, denn ich glaube nicht - der Schulleiter - ich glaube nicht, dass er es verdient hat -"

Harry blieb stehen, holte tief Luft.

"Ich glaube nicht, dass er das verdient hat, Fawkes, er hat versucht, das Richtige zu tun…"

Krächzen!

"Aber du bist wütend auf ihn und versuchst, deinen Standpunkt deutlich zu machen. Ich verstehe dich."

Der Phönix schmiegte seinen Kopf an Harrys Schulter, und der steinerne Wasserspeier ging sanft zur Seite, um Harry passieren zu lassen - zurück in die Korridore von Hogwarts.

-\/-\/-\/-\/-\/-\/-\/-\/-\/-\/-\/-\/-\/-\/-\/-\/-\/-\/-\/-\/-\/-\/-\/-\/-\/-\/-\/-\/-\/-\/-\/-\/-\/-\/-\/-\/-\/-\/-\/-\/-\/-\/-\/-\/-\/-\/-\/-\/-\/-\/-\/-\/-\/-\/-\/-\/-\/-\/-\/-\/-\/-\/-\/-\/-\/-\/-\/-\/-\/-\/-\/-\/-\/-\/-\/-\/-\/-\/-\/-\/-\/-\/-\/-\/-\/-\/-\/-\\ * U.T.Z. steht für den Unheimlich toller Zauberer-Grad\\ im englischen Original ist es N.E.W.T. Nastily Exhausting Wizarding Test, wobei Newt auch noch ein Wortspiel (= Molch) ist.

