

\hypertarget{zwischenspiel-die-grenze-uxfcberschreiten}{% \section{5. Zwischenspiel: Die Grenze überschreiten}\label{zwischenspiel-die-grenze-uxfcberschreiten}}

—\/-\/-\/-\/- Kapitel 37: Zwischenspiel: Die Grenze überschreiten -\/-\/-\/-\/-

Es war fast schon Mitternacht.

Lange aufzubleiben war für Harry ein Kinderspiel. Er hatte einfach den Zeitumkehrer nicht benutzt. Harry folgte damit einer Tradition seinen Schlafzyklus genau abzustimmen, um sicherzustellen dass er wach war wenn aus Heiligabend der erste Weihnachtsfeiertag wurde; denn obwohl er nie jung genug gewesen war, um an den Weihnachtsmann \emph{zu glauben}, war er einmal jung genug gewesen, um Zweifel zu hegen.

Es wäre schön gewesen, wenn es eine mysteriöse Gestalt gegeben \emph{hätte}, die in der Nacht in dein Haus kam und dir Geschenke brachte…

Auf einmal lief Harry ein Schauder über den Rücken.

Eine Ahnung von etwas Entsetzlichem, das sich näherte.

Ein schleichender Schrecken.

Ein Gefühl von drohendem Unheil.

Blitzschnell saß Harry aufrecht im Bett.

Er schaute zum Fenster.

„\emph{Professor Quirrell?}“, Harry schrie sehr leise.

Professor Quirrell machte eine Geste, als würde er etwas anheben, und Harrys Fenster schien sich in seinen Rahmen zusammenzufalten. Sofort blies ein kalter Windstoß durch die Öffnung in den Raum, begleitet von ein paar Schneeflocken aus einem Himmel, der mit grauen Nachtwolken gesprenkelt schien, inmitten der Schwärze und den Sternen.

„Keine Angst, Mr~Potter“, sagte der Verteidigungsprofessor in normalem Tonfall. „Ich habe Ihre Eltern in Schlaf versetzt; sie werden nicht aufwachen, bis ich gegangen bin.“

„Niemand sollte wissen, wo ich bin!“ sagte Harry und hielt seine schrille Stimme nach wie vor gedämpft. „Sogar Eulen sollen meine Post nach Hogwarts bringen, nicht hierher!“ Harry hatte dem bereitwillig zugestimmt; es wäre dumm, wenn irgendein Todesser den ganzen Krieg jederzeit gewinnen könnte, indem er ihm einfach eine magisch ausgelöste Handgranate per Eule sandte.

Professor Quirrell grinste ihn aus dem Garten hinter dem Fenster, in dem er stand, an. „Oh, ich würde mir an Ihrer Stelle keine Sorgen machen, Mr~Potter. Sie sind gut gut gegen Aufspürzauber geschützt und kein Reinblut-Fanatiker wird auf die Idee kommen, in ein Telefonbuch zu schauen.“ Sein Grinsen wurde breiter. „Und es bedurfte erheblicher Anstrengungen, um die Schutzzauber zu überwinden, die der Schulleiter um dieses Haus eingerichtet hat—obwohl natürlich jeder, der Ihre Adresse kennt, einfach draußen warten und Sie das nächste Mal angreifen könnte, wenn sie das Haus verlassen würden.“

Harry starrte Professor Quirrell eine Weile lang an. „Was \emph{machen} Sie hier?“ sagte Harry schließlich.

Das Lächeln verschwand von Professor Quirrells Gesicht. „Ich bin gekommen, um mich zu entschuldigen, Mr~Potter“, sagte der Verteidigungsprofessor leise. „Ich hätte nicht so barsch mit Ihnen reden sollen wie ich—“

„Nicht“, sagte Harry. Er blickte auf die Decke, die er um seinen Pyjama legte. „Lassen Sie es einfach.“

„Habe ich Sie derart schwer beleidigt?“, fragte Professor Quirrells leise Stimme.

„Nein“, sagte Harry. „Aber das werden Sie, wenn Sie sich entschuldigen.“

„Ich verstehe“, sagte Professor Quirrell, und sofort wurde seine Stimme ernst. „Wenn ich Sie also als gleichgestellt behandeln soll, Mr~Potter, muss ich sagen, dass Sie die Etikette, die zwischen sich freundschaftlich gesinnten Slytherins gilt, ernsthaft verletzt haben. Wenn Sie nicht gerade das Spiel gegen jemanden spielen, \emph{dürfen} Sie sich nicht in seine Pläne einmischen, nicht ohne ihn \emph{vorher} zu fragen. Denn Sie wissen weder, was sein eigentlicher Plan sein, noch welche Einsätze er verlieren könnte. So etwas würde Sie zu seinem Feind machen, Mr~Potter.“

„Es tut mir leid“, sagte Harry, in genau dem gleichen ruhigen Ton, wie Professor Quirrell ihn verwendet hatte.

„Entschuldigung akzeptiert“, sagte Professor Quirrell.

„Aber“, sagte Harry, immer noch leise, „Sie und ich müssen irgendwann wirklich nochmal über Politik sprechen.“

Professor Quirrell seufzte. „Ich weiß, dass Sie keine Herablassung mögen, Mr~Potter—“

Das war eine ziemliche Untertreibung.

„Aber es wäre noch herablassender“, sagte Professor Quirrell, „wenn ich es nicht klar sagen würde. Ihnen fehlt es noch an Lebenserfahrung, Mr~Potter.“

„Und stimmen dann alle, die über genügend Lebenserfahrung verfügen, mit Ihnen überein?“, sagte Harry ruhig.

„Was nützt Lebenserfahrung dem, der Quidditch spielt?“, fragte Professor Quirrell und zuckte mit den Achseln. „Ich denke, Sie werden Ihre Meinung mit der Zeit noch ändern, nachdem Sie von allem, worin Sie Vertrauen hatten, enttäuscht wurden und zynisch geworden sind.“

Der Verteidigungsprofessor, vom Fenster umrahmt gegen die Schwärze und die Sterne und den bewölkten Himmel, sagte es, als wäre es die gewöhnlichste Aussage überhaupt, während einige winzige Schneeflocken in der beißenden Winterluft an ihm vorbeiflogen.

„Dabei fällt mir ein“, sagte Harry. „Frohe Weihnachten.“

„So scheint es“, sagte Professor Quirrell. „Schließlich, wenn es \emph{keine} Entschuldigung ist, dann muss es wohl ein Weihnachtsgeschenk sein. Tatsächlich das allererste, das ich je gemacht habe.“

Harry hatte noch nicht einmal damit begonnen, Latein zu lernen, um das Forschungstagebuch von Roger Bacon lesen zu können; und er wagte es kaum, seinen Mund zu öffnen, um zu fragen.

„Ziehen Sie Ihren Wintermantel an“, sagte Professor Quirrell, „oder nehmen Sie einen Wärmetrank, wenn Sie einen dahaben und treffen Sie mich draußen, unter freiem Himmel. Ich werde sehen, ob ich es diesmal etwas länger aufrechterhalten kann.“

Harry brauchte einen Moment die Worte zu begreifen und dann stürzte er schon auf den Kleiderschrank zu.

Professor Quirrell hielt den Zauber des Sternenlichts mehr als eine Stunde lang aufrecht, obwohl sein Gesicht sich verzerrte und er sich nach einer Weile setzen musste. Harry protestierte nur einmal und wurde sofort zum Schweigen gebracht.

Sie erlebten den Ausgang des Heiligabends und das Kommen des ersten Weihnachtsfeiertages in dieser zeitlos scheinenden Leere, in der die Rotation der Erde bedeutungslos war, die einzig wirklich immerwährende Stille Nacht.

Und wie versprochen, schliefen Harrys Eltern während all dem laut schnarchend, bis Harry wieder sicher in seinem Zimmer und der Verteidigungsprofessor gegangen war.

