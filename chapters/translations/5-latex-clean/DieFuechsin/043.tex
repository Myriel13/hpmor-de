

\hypertarget{humanismus-teil-1}{% \section{11. Humanismus, Teil 1}\label{humanismus-teil-1}}

-\/-\/-\/-\/- Kapitel 43: Humanismus, Teil 1 -\/-\/-\/-\/-

Die sanfte Januarsonne strahlte auf die kalten Felder vor Hogwarts.

Für einige der Schüler war es eine Studienstunde, andere waren aus dem Unterricht entlassen worden. Die Erstklässler, die sich dafür entschieden hatten, übten einen bestimmten Zauber, einen Zauber, der am vorteilhaftesten im Freien, unter der strahlenden Sonne und einem klaren blauen Himmel gelernt wurde, und nicht innerhalb der Grenzen eines Klassenzimmers. Kekse und Limonade wurden ebenfalls als hilfreich angesehen.

Die ersten Gesten des Zaubers waren komplex und präzise; du schwingst deinen Zauberstab ein-, zwei-, drei- und viermal mit kleinen Neigungen in genau den richtigen relativen Winkeln, du hast deinen Zeigefinger und Daumen genau in den richtigen Abständen verschoben…

Das Ministerium dachte, das bedeutete, dass es sinnlos war, zu versuchen, irgendwem den Zauber vor seinem fünften Jahr beizubringen. Es hatte einige wenige bekannte Fälle von jüngeren Kindern gegeben, die es gelernt hatten, und dies war als "Genie" abgetan worden.

Es mag keine sehr höfliche Art gewesen sein, es auszudrücken, aber Harry fing an zu verstehen, warum Professor Quirrell behauptet hatte, dass der Lehrplanausschuss des Ministeriums von größerem Nutzen für die Zaubererschaft gewesen wäre, wenn sie als Deponie genutzt worden wären.

Die Gesten waren also kompliziert und grazil. Das hält dich aber nicht davon ab, es zu lernen, wenn du elf Jahre alt bist. Das bedeutete, dass man besonders vorsichtig sein und jedes Teil viel länger als sonst üben musste, das war alles.

Die meisten Zauber, die nur von älteren Schülern gelernt werden konnten, waren so, weil sie mehr magische Kraft benötigten, als jeder junge Schüler aufbringen konnte. Aber der Patronuszauber war nicht so, es war nicht so schwierig, weil es zu viel Magie brauchte, es war schwierig, weil es mehr als nur Magie brauchte.

Es bedurfte der warmen, glücklichen Gefühle, die du in deinem Herzen festgehalten hast, die liebevollen Erinnerungen, einer anderen Art von Kraft, die du nicht für gewöhnliche Zauber brauchtest.

Harry schwenkte ein-, zwei-, drei- und viermal mit dem Zauberstab, verschob seine Finger genau um die richtigen Abstände…

\emph{\emph{"Viel Glück in der Schule, Harry. Glaubst du, ich habe dir genug Bücher gekauft?"}}

\emph{\emph{"Man kann nie genug Bücher haben… aber du hat es} \emph{definitiv} \emph{versucht, es war ein wirklich, wirklich, wirklich guter Versuch…"}}

Es hatte ihm Tränen in die Augen getrieben, als Harry sich das erste Mal daran erinnert hatte und versuchte, es in den Zauber zu packen.

Harry brachte den Stab hoch und herum und schwang ihn, eine Geste, die nicht präzise sein musste, sondern nur mutig und trotzig.

"\emph{Expecto Patronum!} " rief Harry.

Nichts ist passiert.

Nicht ein einziges Lichtflackern.

Als Harry aufblickte, studierte Remus Lupin noch immer den Zauberstab, ein ziemlich beunruhigter Blick auf seinem schwach vernarbten Gesicht.

Schließlich schüttelte Remus den Kopf. "Es tut mir leid, Harry", sagte der Mann leise. "Deine Stabführung war genau richtig."

Und es gab auch nirgendwo sonst ein Lichtflackern, denn alle anderen Erstklässler, die ihre Patronuszauber praktizieren sollten, hatten stattdessen aus den Augenwinkeln auf Harry geschaut.

Die Tränen drohten, in Harrys Augen zurückzukehren, und es waren keine Freudentränen. Ausgerechnet das hatte Harry nie erwartet.

Es gab etwas schrecklich Erniedrigendes daran, informiert zu werden, dass man nicht glücklich genug war.

Was hatte Anthony Goldstein in sich, was Harry nicht hatte, dass Antonys Zauberstab mit diesem hellen Licht zum Leuchten brachte?

Liebte Anthony seinen eigenen Vater mehr?

"Mit welchem Gedanken hast du den Zauber versucht?", sagte Remus.

"Mein Vater", sagte Harry, seine Stimme zitterte. "Ich bat ihn, mir ein paar Bücher zu kaufen, bevor ich nach Hogwarts kam, und das tat er, und sie waren teuer, und dann fragte er mich, ob es genug seien -"

Harry versuchte nicht, das Motto der Familie Verres zu erklären.

"Ruh dich aus, bevor du einen anderen Gedanken probierst, Harry", sagte Remus. Er deutete dort hin, wo einige andere Schüler auf dem Boden saßen und enttäuscht oder verlegen oder bedauernd aussahen. "Du wirst nicht in der Lage sein, einen Patronuszauber auszuüben, während du dich schämst, nicht dankbar genug zu sein." In Mr. Lupins Stimme lag ein sanftes Mitgefühl, und für einen Moment hatte Harry Lust, etwas zu schlagen.

Stattdessen drehte sich Harry um und stolzierte dorthin, wo die anderen Misserfolge lagen. Die anderen Studenten, deren Stabarbeit ebenfalls als perfekt erklärt worden war und die nun nach glücklicheren Gedanken suchen sollten; nach ihrem Aussehen kamen sie nicht gut voran. Dort gab es viele Gewänder, die in dunkelblau gekleidet waren, und eine Handvoll Rot, und ein einsames Hufflepuff-Mädchen, das immer noch weinte. Die Slytherins hatten sich nicht einmal die Mühe gemacht, aufzutauchen, außer Daphne Greengrass und Tracey Davis, die immer noch versuchten, die Gesten zu verstehen.

Harry stürzte sich auf das kalte, tote Wintergras, neben dem Studenten, dessen Misserfolg ihn am meisten überrascht hatte.

"Also konntest du es auch nicht tun", sagte Hermine. Sie war zuerst vom Feld geflohen, aber sie war danach zurückgekommen, und man musste genau auf ihre geröteten Augen schauen um zu sehen, dass sie geweint hatte.

"Ich", sagte Harry, "Ich, ich würde mich wahrscheinlich viel schlechter fühlen, wenn du nicht versagt hättest, du bist die netteste Person, die ich kenne, die ich je getroffen habe, Hermine, und wenn \emph{du} es auch nicht kannst, bedeutet das, dass ich es vielleicht noch bin, gut bin….".

"Ich hätte zu Gryffindor gehen sollen", flüsterte Hermine. Sie blinzelte ein paar Mal kräftig, aber sie wischte sich nicht die Augen ab.

Der Junge und das Mädchen gingen zusammen vorwärts, definitiv nicht Händchen haltend, sondern jeder zog eine Art Kraft aus der Anwesenheit des anderen, etwas, das sie das Flüstern ihrer Jahrgangskameraden ignorieren ließ, als sie durch den Flur gingen und sich den großen Türen von Hogwarts näherten.

Harry war es nicht gelungen, den Patronuszauber zu wirken, egal was für einen glücklichen Gedanken er versuchte. Die Menschen schienen davon nicht überrascht zu sein, was es noch schlimmer machte. Hermine war auch nicht in der Lage gewesen, es zu tun. Die Leute waren \emph{sehr} überrascht davon gewesen, und Harry hatte gesehen, wie sie die gleichen verstohlenen Blicke bekam wie er. Die anderen Ravenclaws, die versagt hatten, bekamen diese Blicke nicht. Aber Hermine war der Sonnenschein-General, und ihre Fans behandelten es, als hätte sie ihnen gegenüber versagt, irgendwie, als hätte sie ein Versprechen verraten, das sie nie gegeben hatte.

Die beiden waren in die Bibliothek gegangen, um den Patronuszauber zu recherchieren, was Hermines Weg war, mit Kummer umzugehen, ebenso wie es manchmal auch Harrys war. Studieren, lernen, versuchen zu verstehen, \emph{warum}…

Die Bücher hatten bestätigt, was der Schulleiter Harry gesagt hatte; oft konnten Zauberer, die den Patrinuszauber in der Praxis nicht aussprechen konnten, dies in Anwesenheit eines echten Dementors tun, was vom totalen Versagen bis hin zu einem vollständigen körperlichen Patronus ging. Es widersetzte sich jeder Logik, die Aura der Angst des Dementors sollte es \emph{erschweren}, einen glücklichen Gedanken zu fassen; aber so war es.

Also wollten die beiden einen letzten Versuch wagen, davon würde sie niemand abhalten.

Es war der Tag, an dem der Dementor nach Hogwarts kam.

Zuvor hatte Harry den Stein seines Vaters zurückverwandelt, von dort aus wo er normalerweise in Form eines winzigen Diamanten in einem Ring auf seinem kleinen Finger ruhte, und den riesigen grauen Stein wieder in seinen Beutel gelegt. Nur für den Fall, dass Harrys Magie völlig fehlschlug, wenn er der dunkelsten aller Kreaturen gegenüberstand.

Harry hatte bereits begonnen sich pessimistisch zu fühlen und er stand noch nicht einmal vor einem Dementor.

"Ich wette, du kannst es tun und ich kann es nicht", sagte Harry flüsternd. "Ich wette das ist was passiert."

"Es fühlte sich für mich falsch an", sagte Hermine, ihre Stimme noch leiser als seine. "Ich habe es heute Morgen versucht und es wurde mir klar. Als ich das Schwingen am Ende machte, noch bevor ich die Worte sagte, fühlte es sich falsch an."

Harry hat nichts gesagt. Er hatte von Anfang an dasselbe gespürt, obwohl es weitere fünf Versuche mit fünf weiteren glücklichen Gedanken gebraucht hatte, bis er es sich selbst eingestehen konnte. Jedes Mal, wenn er versuchte, seinen Stab zu schwingen, hatte er sich hohl angefühlt; der Zauber, den er zu lernen versuchte, passte nicht zu ihm.

"Das bedeutet nicht, dass wir Dunkle Magier werden", sagte Harry. "Viele Menschen, die den Patronuszauber nicht ausüben können, sind keine Dunklen Zauberer. Godric Gryffindor war kein Dunkler Zauberer…"

Godric hatte Dunkle Lords besiegt, kämpfte um Bürger vor noblen Häusern und Muggel vor Zauberern zu schützen. Er hatte viele gute und wahre Freunde gehabt und nicht mehr als die Hälfte von ihnen für die eine oder andere gute Sache verloren. Er hatte den Schreien der Verwundeten zugehört, in den Armeen, die er zur Verteidigung der Unschuldigen aufgestellt hatte; junge Zauberer mit Mut hatten sich seinem Aufruf angeschlossen, und er hatte sie danach begraben. Bis schließlich, als seine Zauberei gerade erst begonnen hatte, ihn im Alter zu verlassen, er die drei anderen mächtigsten Zauberer seiner Zeit zusammengeführt hatte, um Hogwarts aus der nackten Erde zu erschaffen; die eine große Errungenschaft zu Godrics Namen, die nichts mit Krieg zu tun hatte, irgendeine Art von Krieg, egal wie gerecht. Es war Salazar und nicht Godric, der die erste Klasse von Hogwarts in Kampfmagie unterrichtet hatte. Godric hatte die erste Klasse von Hogwarts in Kräuterkunde unterrichtet, die Magie des grün wachsenden Lebens.

Bis zu seinem letzten Tag war es ihm nie gelungen, den Patronuszauber zu wirken.

Godric Gryffindor war ein guter Mann gewesen, kein glücklicher.

Harry glaubte nicht an Existenzangst, er konnte es nicht ertragen, über weinerliche Helden zu lesen, er wusste, dass eine Milliarde anderer Menschen auf der Welt alles gegeben hätte, um mit ihm zu tauschen, und…

Und auf seinem Sterbebett hatte Godric Helga erzählt (denn Salazar hatte ihn verlassen, und Rowena war vorher gestorben), dass er nichts davon bereute, und er warnte seine Schüler \emph{nicht} davor, in seine Fußstapfen zu treten, niemand sollte \emph{jemals} sagen, dass er jemandem gesagt hatte, er solle nicht in seine Fußstapfen treten. Wenn es das Richtige für \emph{ihn} gewesen war, dann würde er niemandem sonst sagen, dass er sich falsch entscheiden solle, nicht einmal dem jüngsten Schüler in Hogwarts. Und doch hoffte er, dass diejenigen, \emph{die} in seine Fußstapfen traten, sich daran erinnern würden, dass Gryffindor seinem Haus gesagt hatte, dass es in Ordnung sei, wenn sie glücklicher wären als er. Dass Rot und Gold von nun an leuchtend warme Farben wären.

Und Helga hatte ihm weinend versprochen, dass sie, wenn sie Direktorin war, dafür sorgen würde.

Daraufhin war Godric gestorben und ließ keinen Geist hinterlassen; und Harry hatte das Buch zurück zu Hermine geschoben und ging ein wenig weg, damit sie ihn nicht weinen sah.

Man würde nicht vermuten, dass ein Buch mit einem unschuldigen Titel wie "Der Patronuszauber: Zauberer die ihn konnten und die ihn nicht konnten" das traurigste Buch war, das Harry je gelesen hatte.

Harry…

Harry wollte das nicht.

In diesem Buch sein.

Harry wollte das nicht.

Der Rest der Schule schien nur zu denken, dass \emph{kein Patronus} schlicht und einfach \emph{schlechte Person} bedeutete. Irgendwie schien sich die Tatsache, dass Godric Gryffindor auch nicht in der Lage gewesen war, den Patronuszauber zu wirken, nicht weitergegeben zu werden. Vielleicht haben die Leute nicht darüber gesprochen, um seinen letzten Wunsch zu respektieren, Fred und George wussten es wahrscheinlich nicht und Harry würde es ihnen sicherlich nicht sagen. Oder vielleicht haben die anderen Misserfolge es nicht erwähnt, weil es weniger beschämend war, der kleinere Verlust von Stolz und Status, eher dunkel als als unglücklich zu gelten.

Harry sah, dass Hermine neben ihm stark blinzelte; und er fragte sich, ob sie an Rowena Ravenclaw dachte, die auch Bücher geliebt hatte.

"Okay", flüsterte Harry. "Glücklichere Gedanken. Wenn du eine vollen, körperlichen Patronus wirkst, was denkst du, was dein Tier sein wird?"

"Ein Otter", sagte Hermine sofort.

"Ein \emph{Otter}? " flüsterte Harry ungläubig.

"Ja, ein Otter", sagte Hermine. "Was ist mit deinem?"

"Wanderfalke", sagte Harry ohne zu zögern. "Er kann schneller als dreihundert Kilometer pro Stunde tauchen, er ist das schnellste Lebewesen, das es gibt." Der Wanderfalke war seit jeher Harrys Lieblingstier. Harry war entschlossen, eines Tages ein Animagus zu werden, nur um das als seine Form zu bekommen, und mit der Kraft seiner eigenen Flügel zu fliegen, und das Land unten mit schärferen Augen zu sehen… "Aber warum ein Otter? "

Hermine lächelte, sagte aber nichts.

Und die riesigen Türen von Hogwarts schwangen auf.

Sie gingen eine Zeit lang, die Kinder, über einen Weg, der in Richtung des unverbotenen Waldes führte, und fuhren weiter durch den Wald selbst. Die Sonne senkte sich bis in die Nähe des Horizontes, die Schatten waren lang, das Sonnenlicht durchdrang die nackten Äste der Winterbäume; denn es war Januar, und die Erstklässler die letzten, die an diesem Tag lernen sollten.

Dann wandte sich der Weg und nahm eine neue Richtung, und sie alle sahen es in der Ferne, die Lichtung im Wald und die märchenhaften Wintergründe, vergilbtes getrocknetes Gras, das von ein paar kleinen Schneeresten weiß gemacht wurde.

Die menschlichen Figuren waren in dieser Entfernung noch klein. Die beiden Flecken dämmrigen weißen Lichts von den Patronussen der Auroren und der hellere Fleck silbernen Lichts vom Schulleiter, neben etwas…

Harry blinzelte.

Etwas…

Es muss Harrys Phantasie gewesen sein, denn es hätte für einen Dementor keine Möglichkeit geben dürfen, an drei körperlichen Patronussen vorbeizukommen, aber er dachte, er könnte einen Hauch von Leere spüren, der seinen Verstand streifte ihn direkt in der weichen Mitte seines Selbst streifte, ohne Rücksicht auf die Okklumentik-Barrieren.

Seamus Finnigan war aschfahl und zitternd, als er sich den Schülern anschloss, die auf dem verwelkten und schneebedeckten Gras herumliefen. Seamus' Patronuszauber war erfolgreich gewesen, aber es gab immer noch dieses Intervall zwischen dem Zeitpunkt, an dem der Schulleiter seinen eigenen Patronus entließ und dem Zeitpunkt, an dem man seinen eigenen zauber sollte, in dem man sich der Angst des Dementors ungeschützt stellte.

Bis zu zwanzig Sekunden Exposition in fünf Schritten Entfernung waren natürlich sicher, selbst für einen elfjährigen Zauberer mit schwachem Widerstand und einem noch reifenden Gehirn. Es gab viele Unterschiede in der Art und Weise, wie hart die Macht des Dementors die Menschen traf, was eine andere Sache war, die nicht ganz verstanden wurde; aber zwanzig Sekunden waren definitiv sicher.

Vierzig Sekunden Dementor-Exposition in fünf Schritten könnten \emph{möglicherweise} ausreichen, um dauerhafte Schäden zu verursachen, wenn auch nur bei den empfindlichsten Personen.

Es war hartes Training sogar nach den Maßstäben von Hogwarts, wo die Art und Weise, wie man lernte auf einem Hippogreif zu fliegen, darin bestand, dass man auf einen geworfen wurde und gesagt wurde, man solle loslegen. Harry war kein Fan von Überfürsorge, und wenn man sich den Reifegradunterschied zwischen einem Viertklässler in Hogwarts und einem vierzehnjährigen Muggel ansieht, war klar, dass Muggel ihre Kinder verhätschelten… aber selbst Harry hatte angefangen, sich zu fragen, ob das zu viel war. Nicht jeder Schmerz konnte im Nachhinein geheilt werden.

Aber wenn du den Zauber unter diesen Bedingungen nicht wirken konntest, bedeutete das, dass du dich nicht darauf verlassen konntest, den Patronuszauber zur Verteidigung zu benutzen; Selbstüberschätzung war für Zauberer noch gefährlicher als für Muggel. Dementoren konnten deine Magie und deine körperliche Vitalität, nicht nur deine glücklichen Gedanken absaugen, was bedeutete, dass du vielleicht \emph{nicht} in der Lage warst zu apparieren wenn du zu lange gewartet, oder wenn du die nahende Angst nicht erkannt hast, bis der Dementor in Reichweite für seinen Angriff war. (Während seiner Recherchen hatte Harry mit großem Entsetzen entdeckt, dass einige Bücher behaupteten, der Kuss des Dementors würde \emph{deine Seele fressen} und dass dies der Grund für das permanente geistlose Koma war, in das er die Opfer versetzte. Und diese Zauberer, \emph{die das glaubten}, hatten den Kuss des Dementors bewusst benutzt, um \emph{Kriminelle hinzurichten}. Es war eine Gewissheit, dass einige der sogenannten Kriminellen unschuldig waren, und selbst wenn sie es nicht waren, \emph{ihre Seelen zerstörten}? Wenn Harry an Seelen geglaubt hätte, hätte er… nichts zu sagen gewusst, er konnte sich einfach keine angemessene Antwort darauf vorstellen.)

Der Direktor nahm die Sicherheit ernst, ebenso wie die drei Wache stehenden Auroren. Ihr Anführer war ein asiatisch aussehender Mann, feierlich, ohne grimmig zu sein, Auror Komodo, dessen Stab nie seine Hand verließ. Sein Patronus, ein Orang-Utan aus festem Mondlicht, schritt auf und ab zwischen dem Dementor und den Erstklässlern, die darauf warteten, dass sie dran waren; neben dem Orang-Utan bewegte sich der leuchtend weiße Panther von Auror Butnaru, einem Mann mit einem durchdringenden Blick, langen schwarzen Haaren in einem Pferdeschwanz und einem lang geflochtenen Spitzbart. Diese beiden Auroren und ihre beiden Patronusse beobachteten alle den Dementor. Auf der anderen Seite der Schüler ruhte sich Auror Goryanof aus, groß und dünn und blass und unrasiert, der sich auf einem Stuhl zurücklehnte, den er ohne Wort und Stab beschworen hatte, und der ein geistesabwesendes Pokerface beibehielt, als er die gesamte Szene beobachtete. Professor Quirrell war aufgetaucht kurz nachdem die Erstklässler ihre Versuche begonnen hatten, und seine Augen waren nie weit von Harry entfernt. Der winzige Professor Flitwick, der ein Meister Duellant war, fummelte abwesend an seinem Stab herum; und \emph{seine} Augen, die aus dem Inneren des riesigen, bauschigen Bartes, der ihm als Gesicht diente, blickten, blieben auf Professor Quirrell konzentriert.

Und es muss Harrys Fantasie gewesen sein, aber Professor Quirrell schien jedes Mal, wenn der Patronus des Schulleiters zwinkerte, um den nächsten Schüler zu testen, leicht zusammen zu zucken. Vielleicht stellte sich Professor Quirrell den gleichen Placebo-Effekt wie Harry vor, diesen Rückstrom der Leere, der seinen Geist berührte.

"Anthony Goldstein", rief die Stimme des Schulleiters.

Harry ging leise auf Seamus zu, während Anthony anfing, sich dem glänzenden silbernen Phönix zu nähern, und…. was auch immer das unter dem zerfetzten Umhang war.

"Was hast du gesehen?" fragte Harry Seamus mit leiser Stimme.

Viele Studenten hatten Harry nicht geantwortet, als er versucht hatte, die Daten zu sammeln; aber Seamus war Finnigan von Chaos, einer von Harrys Leutnants. Vielleicht war das nicht fair, aber…

"Tot", sagte Seamus flüsternd, "gräulich und schleimig… tot und eine Weile im Wasser liegen gelassen… "

Harry nickte. "Das ist es, was viele Leute sehen", sagte Harry. Er strahlte Vertrauen aus, obwohl es falsch war, weil Seamus es brauchte. "Geh und iss etwas Schokolade, dann wirst du dich besser fühlen."

Seamus nickte und stolperte auf den Tisch der heilenden Süßigkeiten zu.

"\emph{Expecto Patronum!} " rief die Stimme eines kleinen Jungen.

Es gab ein Aufkeuchen, sogar von den Auroren.

Harry drehte sich um, um nachzusehen -

Zwischen Anthony Goldstein und dem Käfig stand ein strahlender, silberner Vogel. Der Vogel erhob seinen Kopf und ließ einen Schrei los, und der Schrei war auch silbern, so hell und hart und schön wie Metall.

Und etwas in Harrys Hintergedanken sagte: \emph{Wenn das ein Wanderfalke ist, werde ich ihn im Schlaf erwürgen.}

\emph{Halt die Klappe}, sagte Harry zu dem Gedanken: \emph{Willst du, dass wir ein Dunkler Zauberer werden}?

\emph{\emph{Wozu das Ganze? Du wirst letztendlich sowieso als einer enden.}}

Das… war nichts, was Harry normalerweise gedacht hätte…

\emph{Es ist ein Placebo-Effekt}, sagte Harry sich wieder. \emph{Der Dementor kann mich nicht wirklich durch drei physische} \emph{Patronusse} \emph{erreichen, ich stelle mir nur vor, wie es meiner Meinung nach} \emph{sein wird. Wenn ich dem Dementor} \emph{wirklich} \emph{gegenüber stehe, wird es sich völlig anders anfühlen, und dann werde ich wissen, dass ich vorher nur dumm} \emph{gewesen bin.}

Ein leichter Schauder lief über Harrys Rücken, weil er das Gefühl hatte, dass es ganz anders sein \emph{würde}, und nicht auf eine positive Art.

Der lodernde silberne Phönix sprang aus dem Zauberstab des Schulleiters zurück, der kleinere Vogel verschwand; und Anthony Goldstein begann zurückzugehen.

Der Schulleiter kam mit Anthony, anstatt den nächsten Namen zu rufen, der Patronus wartete hinter ihm, um den Dementor zu bewachen.

Harry blickte hinüber zu Hermine, direkt hinter dem glühenden Panther. Hermine wäre als nächstes an der Reihe gewesen, musste nun aber anscheinend warten.

Sie sah gestresst aus.

Vorhin hatte sie Harry höflich gebeten, bitte nicht mehr zu versuchen, sie zu entspannen.

Dumbledore lächelte leicht, als er Anthony zurück zu den anderen eskortierte; lächelte nur leicht, weil der Schulleiter sehr, sehr müde aussah.

"Unglaublich", sagte Dumbledore mit einer Stimme, die viel schwächer klang als sein gewohntes Dröhnen. "Ein körperlicher Patronus, in seinem ersten Jahr. Und eine erstaunliche Anzahl von Erfolgen unter den anderen jungen Studenten. Quirinus, ich muss zugeben, dass Sie Ihren Standpunkt bewiesen haben."

Professor Quirrell neigte seinen Kopf. "Eine einfache Vermutung, sollte ich mir denken. Ein Dementor greift durch Angst an, und Kinder haben weniger Angst."

"\emph{Weniger} Angst?" sagte Auror Goryanof, von wo aus er saß.

"Das sagte ich auch", sagte Dumbledore. "Und Professor Quirrell wies darauf hin, dass Erwachsene mehr Mut hatten, nicht weniger Angst, was mir, wie ich gestehe, noch nie in den Sinn gekommen war."

"Das war nicht meine \emph{genaue} Formulierung", sagte Professor Quirrell trocken, "aber es wird genügen. Und der Rest unserer Vereinbarung, Direktor?"

"Wie Sie meinen", sagte Dumbledore widerstrebend. "Ich gebe zu, ich hatte nicht erwartet, diese Wette zu verlieren, Quirinus, aber Sie haben Ihre Weisheit bewiesen."

Alle Studenten sahen sie verwirrt an; außer Hermine, die in Richtung des Käfigs und der hohen, zerfallenden Roben starrte; und Harry, der alle beobachtete, da er sich vorstellte, sich paranoid zu fühlen.

Professor Quirrell sagte in Tönen, die nicht zu weiteren Kommentaren einluden: "Ich darf den Tötungsfluch an Schüler weitergeben, die ihn lernen wollen. Das macht sie wesentlich sicherer vor Dunklen Zauberern und anderen Schädlingen, und es ist töricht zu glauben, dass sie sonst keine tödlichen Magien kennen werden." Professor Quirrell hielt inne, seine Augen verengten sich. "Direktor, ich stelle respektvoll fest, dass Sie nicht gut aussehen. Ich schlage vor, den Rest des Tages Professor Flitwick zu überlassen."

Dumbledore schüttelte den Kopf. "Wir sind für heute fast fertig, Quirinus. Ich werde durchhalten."

Hermine hatte sich Anthony genähert. "Captain Goldstein", sagte sie, und ihre Stimme zitterte nur ein wenig, "Können Sie mir einen Rat geben?"

"Hab keine Angst", sagte Anthony fest. "Denke an nichts, worüber es dich nachdenken lassen will. Du hältst nicht nur den Stab vor dir als Schild gegen die Angst hoch, du \emph{schwingst} deinen Stab, um die Angst wegzutreiben, so machst du aus einem glücklichen Gedanken etwas Solides…" Anthony zuckte hilflos mit den Achseln. "Ich meine, ich habe das alles schon mal \emph{gehört}, aber…"

Andere Studenten begannen, sich um Anthony zu versammeln, mit ihren eigenen Fragen.

"Miss Granger?", sagte der Schulleiter. Seine Stimme könnte sanft oder nur geschwächt gewesen sein.

Hermine streckte ihre Schultern und folgte ihm.

"Was hast du unter dem Umhang gesehen?" fragte Harry Anthony.

Anthony sah Harry überrascht an und antwortete dann: "Ein sehr großer Mann, der tot war, ich meine, irgendwie tot geformt und todfarben… es tat weh, ihn zu sehen, und ich wusste, dass das der Dementor war, der versuchte mich dranzukriegen."

Harry blickte wieder dahin, wo Hermine mit dem Käfig und dem Umhang konfrontiert war.

Hermine hob ihren Zauberstab in Position für die ersten Gesten.

Der Phönix des Schulleiters verschwand in einem Blinzeln aus der Realität.

Und Hermine gab einen winzigen, erbärmlichen Schrei von sich, zuckte -

- trat einen Schritt zurück, Harry konnte sehen, wie sich ihr Stab bewegte, und dann schwang sie ihn und sagte "Expecto Patronum"!

Nichts passierte.

Hermine drehte sich um und rannte.

"\emph{Expecto Patronum!}"sagte die tiefere Stimme des Schulleiters, und der silberne Phönix erwachte lodernd wieder zum Leben.

Das junge Mädchen stolperte und lief weiter, seltsame Geräusche begannen aus ihrem Hals zu kommen.

"\emph{Hermine}! "Susan schrie es, und Hannah, und Daphne, und Ernie, und sie alle begannen, auf sie zuzulaufen; grade als Harry, der immer einen Schritt voraus dachte, sich umdrehte und zum Tisch mit der Schokolade rannte.

Sogar nachdem Harry die Schokolade in Hermines Mund geschoben hatte und sie gekaut und geschluckt hatte, atmete sie immer noch keuchend und weinte, ihre Augen schienen noch unfokussiert zu sein.

\emph{Sie kann nicht dauerhaft dement sein}, dachte Harry verzweifelt wegen der Verwirrung in ihm, die schreckliche Angst und die tödliche Wut, die begannen, sich umeinander zu drehen, \emph{sie kann nicht, sie war nicht einmal zehn Sekunden lang ausgesetzt, geschweige denn vierzig} -

Aber sie konnte \emph{vorübergehend} dement sein, wie Harry in diesem Moment erkannte, es gab keine Regel, dass man nicht \emph{vorübergehend} von einem Dementor in nur zehn Sekunden verletzt werden konnte, wenn man empfindlich genug war.

Dann schienen sich Hermines Augen zu fokussieren, und sie huschten herum und richteten sich auf ihn.

"Harry", keuchte sie, und die anderen Schüler schwiegen. "Harry, nicht. \emph{Nicht}! "

Harry hatte plötzlich Angst zu fragen, was er nicht tun sollte, war \emph{er} in ihren schlimmsten Erinnerungen oder der Alptraum eines Schlafens, den sie jetzt im realen Leben durchlebte?

"\emph{Geh nicht in die Nähe davon!}" sagte Hermine. Ihre Hand streckte sich aus, packte ihn am Revers seiner Robe. "Du darfst dich ihm nicht nähern, Harry! \emph{Es sprach zu mir, Harry, es kennt dich, es weiß, dass du hier bist!}"

" Was -" sagte Harry und verfluchte sich dann selbst, weil er fragte.

"\emph{Der Dementor!}" sagte Hermine. Ihre Stimme erhob sich zu einem Schrei. "\emph{Professor Quirrell will, dass es dich frisst!}"

In der plötzlichen Stille trat Professor Quirrell ein paar Schritte vor; aber er kam nicht näher (Harry war schließlich da). "Miss Granger", sagte er, und seine Stimme war ernst, "Ich denke, Sie sollten noch etwas Schokolade nehmen."

"\emph{Professor Flitwick, lassen Sie Harry es nicht versuchen, schicken Sie ihn zurück!}"

Der Direktor war bis dahin angekommen, und er und Professor Flitwick tauschten besorgte Blicke aus.

"Ich habe den Dementor nicht sprechen hören", sagte der Schulleiter. " Trotzdem…"

"Fragen Sie einfach", sagte Professor Quirrell und klang ein wenig müde.

"Hat der Dementor gesagt, \emph{wie} er zu Harry kommen würde?", sagte der Direktor.

"Zuerst alle seine leckersten Teile", sagte Hermine, "es würde - es würde essen -".

Hermine blinzelte. Etwas Verstand schien in ihre Augen zurückzukehren.

Dann fing sie an zu weinen.

"Du warst zu tapfer, Hermine Granger", sagte der Schulleiter. Seine Stimme war sanft und deutlich hörbar. "Sehr viel mutiger, als ich vermutet habe. Du hättest dich umdrehen und weglaufen sollen, statt es zu ertragen und zu versuchen deinen Zauber zu vervollständigen. Wenn Sie älter und stärker sind, Miss Granger, weiß ich, dass Sie es erneut versuchen werden, und ich weiß, dass Sie Erfolg haben werden."

"Es tut mir leid", sagte Hermine keuchend, "Es tut mir leid, es tut mir leid, es tut mir leid…. Es tut mir leid, Harry, ich kann dir nicht sagen, was ich gesehen habe, ich habe es nicht angeguckt, ich habe es nicht gewagt, es anzusehen, ich wusste, dass es zu schrecklich war, um jemals gesehen zu werden…"

Es hätte Harry sein sollen, aber er hatte gezögert, weil seine Hände alle schokoladig waren; und dann waren Ernie und Susan da und halfen Hermine, von wo aus sie auf das Gras gefallen war, und führten sie zum Snacktisch.

Fünf Tafeln Schokolade später schien Hermine wieder in Ordnung zu sein, und sie ging hinüber und entschuldigte sich bei Professor Quirrell; aber sie beobachtete Harry die ganze Zeit, jedes Mal, wenn er in ihre Richtung blickte. Er war nur einmal auf sie zugegangen und hielt an, als sie wegging. Ihre Augen hatten sich stillschweigend entschuldigt und ihn stillschweigend gebeten, sie in Ruhe zu lassen.

Neville Longbottom hatte etwas Totes und halb Aufgelöstes gesehen, herausquellend und mit einem Gesicht wie ein zerquetschter Schwamm.

Es war das Schlimmste, was bisher jemand beschrieben hatte gesehen zu haben. Neville war es zuvor gelungen, aus seinem Stab ein kleines Lichtflackern zu erzeugen, aber er hatte sich intelligent und mit großer Geistesgegenwart umgedreht und war weggelaufen, anstatt zu versuchen, seinen eigenen Patronuszauber zu wirken.

(Der Schulleiter hatte den anderen Studenten nichts gesagt, niemand anderem gesagt, er solle weniger mutig sein; aber Professor Quirrell hatte ruhig festgestellt, dass, wenn man den Fehler machte, \emph{nachdem} man gewarnt worden war, dass das der Zeitpunkt war an dem Unwissenheit zu Dummheit wurde).

"Professor Quirrell?" sagte Harry mit leiser Stimme, nachdem er dem Verteidigungsprofessor so nahegekommen war, wie er es wagte. "Was sehen Sie, wenn Sie -"

"Frag nicht." Die Stimme war sehr matt.

Harry nickte respektvoll. "Was war Ihre \emph{ursprüngliche} Formulierung an den Schulleiter, wenn ich fragen darf?"

Trocken. "Unsere schlimmsten Erinnerungen können mit zunehmendem Alter nur noch schlimmer werden."

"Ah", sagte Harry. " Logisch."

Etwas Seltsames flackerte in den Augen von Professor Quirrell, als er Harry ansah. "Hoffen wir," sagte Professor Quirrell, "dass Sie mit diesem Versuch Erfolg haben werden, Mr. Potter. Denn wenn Sie es tun, könnte der Schulleiter Ihnen seinen Trick beibringen, einen Patronus zu benutzen, um Nachrichten zu senden, die nicht gefälscht oder abgefangen werden können, und die militärische Bedeutung dessen ist unmöglich zu überschätzen. Es wäre ein enormer Vorteil für die Chaos-Legion, und eines Tages, so vermute ich, für das ganze Land. Aber wenn Sie es \emph{nicht} schaffen, Mr. Potter… nun, \emph{ich} werde es verstehen."

Morag MacDougal hatte mit schwankender Stimme "Autsch" gesagt, und Dumbledore hatte seinen Patronus sofort neu beschworen.

Parvati Patil hatte einen körperlichen Patronus in Form eines Tigers hervorgebracht, der größer als Dumbledores Phönix war, wenn auch nicht annähernd so hell. Es hatte einen großen Applaus von allen Zuschauern gegeben, wenn auch nicht den gleichen Schock wie als Anthony es getan hatte.

Und dann war Harry an der Reihe.

Der Direktor nannte den Namen Harry Potter, und Harry hatte Angst.

Harry wusste, er wusste, dass er scheitern würde, und er wusste, dass es wehtun würde.

Aber er musste es trotzdem versuchen, denn manchmal, in Gegenwart eines Dementors, schaffte es ein Zauberer ohne ein Lichtflimmern zu einem vollen, körperlichen Patronus, und niemand verstand warum.

Und weil, wenn Harry sich nicht gegen Dementoren verteidigen \emph{konnte}, er in der Lage sein musste, ihre Annäherung zu erkennen, das Gefühl von ihnen mit seinem Verstand zu erkennen und zu laufen, bevor es zu spät war.

\emph{\emph{Was ist meine schlimmste Erinnerung…?}}

Harry hatte erwartet, dass der Schulleiter ihm einen besorgten Blick, einen hoffnungsvollen Blick oder einen tiefgründigen Rat geben würde; stattdessen beobachtete Albus Dumbledore ihn nur mit stiller Ruhe.

\emph{Er denkt, dass ich scheitern werde, aber er wird mich nicht sabotieren, indem er es mir sagt}, dachte Harry, \emph{wenn er} \emph{etwas wahrhaft Ermutigendes zu sagen hätte, würde er es sagen…}

Der Käfig kam näher. Er war bereits angelaufen, aber nicht zu einem nichts weggerostet, noch nicht.

Der Umhang kam näher. Er war ausgefranst und mit ungeflickten Löchern durchsetzt; er war an diesem Morgen neu gewesen, hatte Auror Goryanof gesagt.

"Schulleiter?" sagte Harry. "Was sehen Sie da?"

Auch die Stimme des Schulleiters war ruhig. "Die Dementoren sind Geschöpfe der Angst, und so wie eure Angst vor dem Dementor abnimmt, so nimmt auch die Furcht vor seiner Form ab. Ich sehe einen großen, dünnen, nackten Mann. Er zerfällt nicht. Er ist nur leicht schmerzhaft anzusehen. Das ist alles. Was siehst du, Harry?"

… Harry konnte unter dem Umhang nichts sehen.

Oder das war nicht richtig, es war, dass sein Verstand sich \emph{weigerte} zu sehen, was unter dem Umhang war…

Nein, sein Verstand versuchte, das \emph{Falsche} unter dem Umhang zu sehen, Harry konnte es spüren, seine Augen versuchten, einen Fehler zu erzwingen. Aber Harry hatte sein Bestes getan, um sich selbst zu trainieren, um dieses winzige Gefühl der Verwirrung zu bemerken, um automatisch vor dem Erfinden von Sachen zurückzuschrecken; und jedes Mal, wenn sein Verstand versuchte, eine Lüge über das, was unter dem Umhang war, zu erfinden, war dieser Reflex schnell genug, um ihn zu beenden.

Harry sah unter den Umhang und sah…

Eine offene Frage. Harry ließ nicht zu, dass sein Verstand etwas Falsches sah, und so sah er nichts, als wenn der Teil seines visuellen Kortex, der dieses Signal erhielt, gerade aufhörte zu existieren. Es gab einen blinden Fleck unter dem Umhang. Harry konnte nicht wissen, was da drunter war.

Nur, dass es viel schlimmer war als jede verrottende Mumie.

Der unsichtbare Schrecken unter dem Umhang war jetzt sehr nah, aber der lodernde Vogel aus Mondlicht, der weiße Phönix, lag noch zwischen ihnen.

Harry wollte weglaufen, wie einige der anderen Schüler. Die Hälfte derjenigen, die kein Glück mit ihren Patronuszaubern gehabt hatten, waren heute einfach gar nicht erst aufgetaucht. Von den übrigen war die Hälfte geflohen, bevor der Schulleiter überhaupt seinen eigenen Patronus entlassen hatte, und niemand hatte ein Wort gesagt. Es hatte ein kleines Lachen gegeben, als Terry sich umgedreht hatte und noch vor seinem eigenen Versuch zurückging; und Susan und Hannah, die vorher gegangen waren, hatten alle angeschrien, die Klappe zu halten.

Aber Harry war der Junge, der lebte, und er würde viel Respekt verlieren, wenn man sehen würde, dass er aufgab, ohne es überhaupt zu versuchen…

Stolz und Rollen schienen sich zu schwinden und wegzufallen, in Gegenwart von was auch immer unter dem Umhang lag.

\emph{\emph{Warum bin ich immer noch hier?}}

Es war nicht die Schande, dass andere ihn für feige hielten, die Harrys Füße an Ort und Stelle hielt.

Es war nicht die Hoffnung, seinen Ruf wiederherzustellen, die seinen Zauberstab hob.

Es war nicht der Wunsch, den Patronuszauber als Magie zu beherrschen, der seine Finger in die Ausgangsposition brachte.

Es war etwas anderes, etwas, das dem entgegenwirken \emph{musste}, was unter dem Umhang lag, dies war die wahre Dunkelheit, und Harry musste herausfinden, ob es in ihm lag, die Macht es zurückzutreiben.

Harry hatte geplant, ein letztes Mal zu versuchen, an seine Buchkauftour mit seinem Vater zu denken, aber stattdessen kam ihm in letzter Minute, als er Dementor gegenüberstand, eine andere Erinnerung in den Sinn, etwas, das er noch nie zuvor ausprobiert hatte; ein Gedanke, der nicht warm und glücklich auf die normale Art und Weise war, aber sich irgendwie richtiger anfühlte.

Und Harry erinnerte sich an die Sterne, erinnerte sich an sie, die in der Stillen Nacht schrecklich hell und unerschütterlich brannten; er ließ sich von diesem Bild erfüllen, ihn erfüllen wie eine Okklumentik Barriere durch seinen gesamten Verstand, wurde wieder einmal zum körperlosen Bewusstsein der Leere.

Der helle, silbern glänzende Phönix verschwand.

Und der Dementor schlug in seinen Verstand wie die Faust Gottes.

\textbf{\textbf{ANGST / KÄLTE / DUNKELHEIT}}

Es gab einen Moment, in dem die beiden Kräfte aufeinanderprallten, in dem die friedliche sternenklare Erinnerung gegen die Angst stand, während Harrys Finger mit den Zauberstabbewegungen begannen, die er so lange geübt hatte, bis sie automatisch kamen. Sie waren nicht warm und glücklich, diese lodernden Lichtpunkte in perfekter Schwärze; aber es war ein Bild, das der Dementor nicht leicht durchdringen konnte. Denn die stillen brennenden Sterne waren riesig und furchtlos, und in der Kälte und Dunkelheit zu leuchten, war ihr natürlicher Zustand.

Aber es gab einen Fehler, einen Riss, eine Verwerfungslinie in dem unbeweglichen Objekt, das versuchte dieser unwiderstehlichen Kraft zu widerstehen. Harry fühlte einen Anflug von Wut auf den Dementor, weil der versucht hatte, sich von ihm zu ernähren und es war, als würde er auf nassem Eis rutschen. Harrys Verstand begann seitwärts zu rutschen, in Bitterkeit, schwarze Wut, tödlichen Hass -

Harrys Zauberstab kam zum endgültigen Schwung nach oben.

Es fühlte sich falsch an.

"Expecto Patronum", sprach seine Stimme, die Worte hohl und sinnlos.

Und Harry fiel in seine dunkle Seite, fiel in seine dunkle Seite weiter und schneller und tiefer als je zuvor, nach unten, nach unten, nach unten als die Rutsche beschleunigte, als der Dementor sich an den freiliegenden und empfindlichen Teilen festhielt und sich von ihnen ernährte und das Licht verzehrte. Ein verblassender Reflex krabbelte zur Wärme, aber selbst als ein Bild von Hermine zu ihm kam, oder ein Bild von Mama und Papa, verdrehte der Dementor es, zeigte ihm Hermine, die tot auf dem Boden lag, die Leichen seiner Mutter und seines Vaters, und dann wurde sogar das weggesaugt.

In das Vakuum stieg die Erinnerung, die schrecklichste Erinnerung, etwas, das vor so langer Zeit vergessen wurde, dass die neuronalen Muster nicht mehr hätten existieren dürfen.

"\emph{Lily, nimm Harry und geh! Er ist es!" rief die Stimme eines Mannes. " Los! Lauf! Ich werde} \emph{ihn aufhalten!}"

\emph{\emph{Und Harry konnte nicht anders, als in den leeren Tiefen seiner dunklen Seite darüber nachzudenken, wie lächerlich übertrieben selbstbewusst James Potter gewesen war. Lord Voldemort aufhalten? Womit?}}

\emph{\emph{Dann sprach die andere Stimme, schrill wie das Zischen einer Teekanne, und es war wie Trockeneis, das auf Harrys sämtliche Nerven gelegt wurde, wie ein} \emph{Brandzeichen} \emph{aus Metall, das auf} \emph{die Temperatur von} \emph{flüssigem} \emph{Helium gekühlt} \emph{war} \emph{und auf jeden Teil von ihm gelegt wurde. Und die Stimme sagte:}}

\emph{\emph{" Avadakedavra."}}

(Der Stab flog aus den nervenlosen Fingern des Jungen, als sein Körper sich zu verkrampfen und zu fallen begann, die Augen des Schulleiters weiteten sich in Alarmbereitschaft, als er seinen eigenen Patronuszauber begann.)

\emph{\emph{"Nicht Harry, nicht Harry, bitte nicht Harry!" schrie die Stimme der Frau.}}

\emph{\emph{Was von Harry übrig blieb, hörte mit all dem Licht, das aus ihm herausgeflossen war, in der toten Leere seines Herzens, und fragte sich, ob sie dachte, dass Lord Voldemort aufhören würde, weil sie höflich fragte.}}

\emph{\emph{"Geh zur Seite, Frau!" sagte die schrille Stimme von brennender Kälte. "Wegen dir bin ich nicht gekommen, nur der Junge."}}

\emph{\emph{"Nicht Harry! Bitte… haben Sie Mitleid… haben Sie Mitleid…"}}

\emph{\emph{Lily Potter, dachte Harry, schien nicht zu verstehen, welche Art von Menschen überhaupt zu Dunklen Lords wurden; und wenn dies die beste Strategie war, die sie sich vorstellen konnte, um das Leben ihres Kindes zu retten, war das ihr letztes Versagen als Mutter.}}

\emph{\emph{"Ich gebe dir diese seltene Chance zu fliehen", sagte die schrille Stimme. "Aber ich werde mich nicht bemühen, dich zu unterwerfen, und dein Tod hier wird dein Kind nicht retten. Tritt beiseite, dumme Frau, wenn du überhaupt Verstand in dir hast!"}}

\emph{\emph{"Nicht Harry, bitte nicht, nimm mich, töte mich stattdessen!"}}

\emph{\emph{Die leere Sache, die Harry war, fragte sich, ob Lily Potter sich ernsthaft vorstellte, dass Lord Voldemort ja sagen würde, sie töten und dann abreisen würde, ohne ihren Sohn zu verletzen.}}

\emph{\emph{"Nun gut", sagte die Stimme des Todes, die jetzt kühl amüsiert klang, "ich akzeptiere den Handel. Du selbst wirst sterben, und das Kind wird leben. Jetzt lass deinen Zauberstab fallen, damit ich dich töten kann."}}

\emph{\emph{Es herrschte eine abscheuliche Stille.}}

\emph{\emph{Lord Voldemort begann zu lachen, ein schreckliches, verächtliches Lachen.}}

\emph{\emph{Und dann, endlich, schrie Lily Potters Stimme vor verzweifeltem Hass, "Avada ke-\/-".}}

\emph{\emph{Die tödliche Stimme beendete ihn zuerst, der Fluch schnell und präzise.}}

\emph{\emph{"Avadakedavra."}}

\emph{\emph{Ein blendendes grünes Licht markierte das Ende von Lily Potter.}}

\emph{\emph{Und der Junge in der Krippe sah sie, die Augen, diese beiden purpurroten Augen, die hellrot zu leuchten schienen, wie Miniatursonnen zu lodern und Harrys ganze Vision zu erfüllen, als sie seine eigenen erfassten -}}

Die anderen Kinder sahen Harry Potter fallen, sie hörten Harry Potter schreien, einen dünnen, hohen Schrei, der ihre Ohren wie Messer zu durchbohren schien.

Es gab einen brillanten Silberblitz, als der Schulleiter "\emph{Expecto Patronum!}" brüllte und der lodernde Phönix ins Sein zurückkehrte.

Aber Harry Potters schrecklicher Schrei ging immer weiter und weiter, selbst als der Schulleiter den Jungen in die Arme nahm und ihn vom Dementor entfernte, auch als Neville Longbottom und Professor Flitwick beide gleichzeitig Schokolade holen gingen und -

Hermine wusste es, sie wusste es, als sie es sah, sie wusste, dass ihr Albtraum echt war, er wurde wahr, irgendwie wurde er wahr.

"Holt ihm Schokolade!" verlangte die Stimme von Professor Quirrell, sinnloserweise, denn Professor Flitwicks winzige Form war bereits wie eine Kanonenkugel auf dem Weg dorthin, wo der Schulleiter auf die Studenten zusteuerte.

Hermine ging selbst voran, obwohl sie nicht wusste, was sie sonst noch tun wollte -

"\emph{Ruft eure Patronusse!} "rief der Schulleiter, als er Harry hinter die Auroren brachte. "\emph{Jeder, der es kann! Bringt sie zwischen Harry und den Dementor! Er} \emph{ernährt sich immer noch von ihm!}"

Es gab einen Moment gefrorenen Schreckens.

"\emph{Expecto Patronum!}" rief Professor Flitwick und Auror Goryanof, und dann Anthony Goldstein, aber er scheiterte beim ersten Mal, und dann Parvati Patil, die es schaffte, und dann versuchte Anthony es wieder und sein silberner Vogel spreizte seine Flügel und schrie den Dementor an, und Dean Thomas brüllte die Worte, als wären sie in Feuerbuchstaben geschrieben worden, und sein Stab brachte einen hoch aufragenden weißen Bären zur Welt; Es gab acht lodernde Patronusse, alle in einer Reihe zwischen Harry und dem Dementor, und Harry schrie und schrie weiter, als der Schulleiter ihn auf das trockene Gras legte.

Hermine konnte keinen Patronuszauber aussprechen, also rannte sie zu Harry. In ihrem Kopf versuchte zu erraten, wie lange es schon gedauert hatte. Waren es zwanzig Sekunden? Mehr?

Auf dem Gesicht von Albus Dumbledore herrschten schreckliche Qual und Verwirrung. Sein langer schwarzer Stab war in seiner Hand, aber er sprach keine Zaubersprüche, sah nur entsetzt auf Harrys sich verkrampfenden Körper herab -

Hermine wusste nicht, was sie tun sollte, sie wusste nicht, was sie tun sollte, sie verstand nicht, was geschah, und der mächtigste Zauberer der Welt schien ebenso ratlos zu sein.

"\emph{Benutzen Sie Ihren Phönix!}" brüllte Professor Quirrell. "\emph{Bringen Sie ihn weit weg von diesem Dementor!"}

Ohne ein einziges Wort nahm der Schulleiter Harry in die Arme und verschwand in einem Feuerblitz zusammen mit dem plötzlich erscheinenden Fawkes; und der Patronus des Schulleiters verschwand von wo er den Dementor bewacht hatte.

Entsetzen und Verwirrung und plötzliches Geplapper.

"Mr. Potter sollte sich erholen", sagte Professor Quirrell und erhob seine Stimme, aber sein Tonfall war jetzt wieder ruhig: "Ich glaube, es waren nur etwas mehr als 20 Sekunden."

Dann erschien der leuchtend weiße Phönix wieder, als würde er von anderswo vor ihnen herfliegen, zu Hermine Granger kam das Geschöpf des Mondlichts, und es schrie zu ihr in Albus Dumbledores Stimme:

"\emph{Es ernährt sich immer noch von ihm, sogar hier! Wie? Wenn du es weißt, Hermine Granger, musst du es mir sagen! Sag es mir!}"

Der ältere Auror drehte sich um, um sie anzustarren, und viele Studenten auch. Professor Flitwick drehte sich nicht um, er richtete jetzt seinen Stab auf Professor Quirrell, der die deutlich leeren Hände ausstreckte.

Die Sekunden vergingen, ungezählt.

Sie konnte sich nicht daran erinnern, sie konnte sich nicht klar an den Alptraum erinnern, sie konnte sich nicht erinnern, warum sie es für möglich gehalten hatte, warum sie Angst hatte -

Hermine erkannte dann, was sie tun sollte, und es war die schwerste Entscheidung ihres Lebens.

Was wäre, wenn das, was mit Harry passiert wäre, auch mit ihr passiert wäre?

Alle ihre Gliedmaßen waren kalt wie der Tod, ihre Sicht wurde dunkel, die Angst überwältigte alles; sie hatte Harry sterben sehen, Mama und Papa starben, alle ihre Freunde starben, alle starben, so dass sie am Ende, wenn sie starb, allein sein würde. Das war ihr geheimer Alptraum, über den sie nie mit jemandem gesprochen hatte, der dem Dementor seine Macht über sie gegeben hatte, die einsamste Sache war allein zu sterben.

Sie wollte nicht wieder an diesen Ort gehen, sie, sie wollte nicht, sie wollte nicht für immer dort bleiben -

\emph{Du hast Mut genug für Gryffindor}, sagte die ruhige Stimme des Sprechenden Hutes in ihrem Gedächtnis\emph{, aber du wirst in jedem Haus das ich dir gebe tun, was richtig ist. Du wirst lernen, du wirst zu deinen Freunden stehen in jedem Haus, das du wählst. Also keine Angst, Hermine Granger, entscheide dich einfach, wo du hingehörst…}

Es blieb keine Zeit für eine Entscheidung, Harry lag im Sterben.

"Ich kann mich jetzt nicht mehr erinnern", sagte Hermine, ihre Stimme brach, "aber warte einfach, ich gehe wieder vor den Dementor…"

Sie begann, auf den Dementor zuzurennen.\\ "Miss Granger" quietschte Professor Flitwick, aber er tat nichts, um sie aufzuhalten, hielt nur seinen Zauberstab weiter auf Professor Quirrell gerichtet.

"\emph{Alle zusammen!}" rief Auror Komodo mit einer Stimme des Militärkommandos. "\emph{Bringt} \emph{eure} \emph{Patronusse} \emph{aus} \emph{ihrem} \emph{Weg!}"

"\emph{FLITWICK!}" brüllte Professor Quirrell. "\emph{RUF POTTERS ZAUBERSTAB}"

Grade als Hermine es verstand, rief Professor Flitwick bereits "\emph{Accio!}", und sie sah, wie der Holzstab von dort, wo er gelegen hatte, aufstieg und fast den Käfig des Dementors berührte.

Die Augen öffneten sich, tot und leer.

"\emph{Harry!}" keuchte eine Stimme in der farblosen Welt. "\emph{Harry! Sprich mit mir!"}

Das Gesicht von Albus Dumbledore lehnte sich in das Sichtfeld, das von einer entfernten Marmordecke eingenommen worden war.

"Du bist nervig", sagte die leere Stimme. "Du solltest sterben."

