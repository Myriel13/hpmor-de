

\hypertarget{zwischenspiel-mit-dem-beichtvater-versunkene-kosten}{% \section{44. Zwischenspiel mit dem Beichtvater: Versunkene Kosten}\label{zwischenspiel-mit-dem-beichtvater-versunkene-kosten}}

\textbf{-\/-\/-\/-\/- Kapitel 76: Zwischenspiel mit dem Beichtvater: Versunkene Kosten -\/-\/-\/-\/-}

Rianna Felthorne stieg die Treppe aus rauhem Stein und grobem Mörtel hinunter, wobei sie ein \emph{Lumos} in den Abständen zwischen den Wandlampen leuchten ließ und ihren Zauberstab in den Lücken von Licht zu Licht emporhielt.

Sie kam zu einer leeren, von vielen dunklen Öffnungen durchbrochenen Felshöhle, die von einer altertümlichen Fackel beleuchtet wurde, die bei ihrem Eintreten aufloderte.

Es war noch niemand da, und nach langen Minuten nervösen Herumstehens begann sie den Zauber, um ein gepolstertes Sofa zu verwandeln, das groß genug war, dass zwei Leute darauf sitzen oder vielleicht sogar liegen konnten. Ein einfacher Holzschemel wäre einfacher gewesen, das hätte sie in fünfzehn Sekunden schaffen können, aber - nun ja -

Selbst als das Sofa vollständig herbeigezaubert war, war Professor Snape immer noch nicht da, und sie setzte sich auf die linke Seite des Sofas, wobei ihr Puls in ihrem Hals hämmerte. Irgendwie wurde sie nur noch nervöser, nicht weniger, je länger die Verzögerung andauerte.

Sie wusste, dass dies das letzte Mal war.

Das letzte Mal, bevor all diese Erinnerungen verschwanden und Rianna Felthorne sich in einer mysteriösen Höhle wiederfand und sich fragte, was hier vor sich ging.

Es hatte etwas an sich, das sich wie Sterben anfühlte.

In den Büchern stand, dass ein richtig durchgeführter Gedächtniszauber nicht schädlich war. Die Leute vergaßen ständig Dinge. Die Leute träumten und wachten dann auf, ohne sich an ihre Träume zu erinnern. Gedächtniszauber beinhalteten nicht einmal sehr viele Unterbrechungen, nur einen kurzen Moment der Desorientierung; es war so, als würde man von einem lauten Geräusch abgelenkt werden und einen Gedanken verlieren, an den man sich danach nicht mehr erinnern konnte. So stand es in den Büchern, und deshalb waren Gedächtniszauber vom Ministerium für alle autorisierten Regierungszwecke vollständig zugelassen.

Aber trotzdem, \emph{diese Gedanken}, die Gedanken, die sie jetzt gerade dachte; bald würde sie niemand mehr haben. Wenn sie in die Zukunft blickte, gab es niemanden, der die Gedanken vervollständigen konnte, die sie noch nicht zu Ende gedacht hatte. Selbst wenn sie es schaffte, in der nächsten Minute alle losen Enden in ihren Gedanken zu verknüpfen, würde danach nichts mehr davon übrig sein. War es nicht genau das, worüber man sich Gedanken machen würde, wenn man in der nächsten Minute sterben würde?

Da ertönte das Geräusch gedämpfter Schritte…

Severus Snape tauchte in der Höhle auf.

Sein Blick wanderte zu ihr, wie sie auf dem Sofa saß, und ein seltsamer Ausdruck ging über sein Gesicht; seltsam, weil er weder sardonisch noch wütend oder kalt war.

"Danke, Miss Felthorne", sagte Snape leise, "das war sehr rücksichtsvoll von Ihnen." Der Meister der Zaubertränke zückte seinen Zauberstab und führte die üblichen Geheimhaltungszauber aus, dann ging er auf sie zu und setzte sich schwerfällig neben sie auf das verwandelte Sofa.

Ihr Puls pochte jetzt aus einem ganz anderen Grund.

Sie drehte sich langsam zu Professor Snape um und sah, dass er den Kopf gegen das Sofa gelehnt hatte und seine Augen geschlossen waren. Er schlief jedoch nicht. Sein Gesicht wirkte verkrampft, angespannt und voller Schmerz.

Sie wusste -- da war sie sich plötzlich sicher -, dass sie diesen Anblick nur sehen durfte, weil sie sich danach nicht mehr daran erinnern würde; und dass niemand vor ihr ihn je hatte sehen dürfen.

\emph{Das hektische Gespräch, das in Rianna Felthornes Kopf ablief, hörte sich ungefähr so an: \emph{Ich könnte mich einfach vorbeugen und ihn küssen.}}

\emph{\emph{Du bist völlig von Sinnen.}}

\emph{\emph{Seine Augen sind geschlossen,ich wette, er würde mich nicht rechtzeitig aufhalten.}}

\emph{Ich wette, es würde Jahre dauern, bis jemand deine Leiche findet -}

Aber dann öffnete Professor Snape seine Augen (zu ihrer inneren Enttäuschung und Erleichterung) und sagte mit normalerer Stimme: "Ihre Bezahlung, Miss Felthorne." Aus seiner Robe nahm er einen Rubin, geschliffen nach Gringotts-Standard, und hielt ihn ihr entgegen. "Fünfzig Facetten. Ich habe nichts dagegen, wenn Sie sie zählen."

Sie streckte eine zitternde Hand aus, in der Hoffnung, dass Snape den Rubin in ihre Finger drücken würde, dass sie eine Berührung seiner Haut an ihrer spüren würde -

Aber stattdessen hob Snape seine Hand leicht an und ließ den Rubin in ihre Hand fallen, dann lehnte er sich zurück an die Couch. "Sie werden sich daran erinnern, dass Sie ihn auf dem Boden dieser Höhle gefunden haben, wo Sie auf Erkundungstour waren", sagte Snape. "Und da niemand außer Ihnen das wirklich glauben wird, werden Sie sich daran erinnern, dass Sie dachten, es wäre weniger lästig, wenn Sie das Geld in einem separaten Fach in Gringotts deponieren würden."

Eine Zeit lang war nur das schwache Knistern der Fackel zu hören.

"Warum -" sagte Rianna Felthorne. \emph{Er weiß, dass ich mich nicht erinnern werde.} "Warum haben Sie es getan? Ich meine - Sie sagten, ich solle Ihnen sagen, wo Schläger sein würden und wer sie sein würden, aber nicht, ob Granger dort sein würde. Und ich weiß, so wie ein Zeitumkehrer funktioniert, wenn man Granger \emph{veranlassen} will dort zu sein, darf einem nicht gesagt werden, ob es schon passiert ist. Daher habe ich geschlussfolgert, dass \emph{wir} diejenigen waren, die ihr sagten, wohin sie gehen soll. Das waren wir, nicht wahr?"

Snape nickte wortlos. Er hatte die Augen wieder geschlossen.

"Aber", sagte Rianna, "ich habe nicht verstanden, \emph{warum} Sie ihr geholfen haben. Und jetzt - nach dem, was Sie Granger in der Großen Halle angetan haben - verstehe ich überhaupt nichts mehr." Rianna hatte sich selbst nie für besonders nett gehalten. Sie hatte von der Kontroverse um den Sonnenscheingeneral wenig mitbekommen. Aber Granger im Kampf gegen Rowdies zu \emph{helfen}, hatte etwas … nun, sie hatte sich angewöhnt, das als die gute Seite zu betrachten und \emph{sich selbst} als auf der guten Seite stehend zu sehen. Und sie hatte festgestellt, dass es ihr sogar gefiel. Es war schwer, das einfach loszulassen. "Warum haben Sie das getan, Professor Snape?"

Snape schüttelte den Kopf, sein Gesicht straffte sich.

"Gibt es -" sagte Rianna zögernd. "Ich meine - wo wir schon mal hier sind - gibt es irgendetwas, worüber Sie sprechen möchten?" Es gab etwas, das \emph{sie} sagen wollte, aber sie konnte die Worte nicht über ihre eigenen Lippen bringen.

"Mir fällt da eine Sache ein", sagte Snape nach einer Pause. "Wenn Sie interessiert sind, Miss Felthorne."

Snapes Augen waren immer noch geschlossen, also konnte sie nicht einfach nicken. Ihre Stimme brach fast, als sie sich zwang, "Ja" zu sagen.

"Es gibt einen gewissen Jungen in Ihrer Klasse, der Sie mag, Miss Felthorne", sagte Snape hinter seinen geschlossenen Augen. "Ich werde seinen Namen nicht sagen. Aber er beobachtet Sie jedes Mal, wenn Sie durch den Raum gehen, wenn er denkt, dass Sie nicht hinsehen. Er träumt von Ihnen und wünscht sich, Sie zu besitzen, aber er hat Sie noch nie auch nur um einen Kuss gebeten."

Ihr Herz fing an, noch heftiger zu hämmern.

"Bitte sagen Sie mir die ehrliche Wahrheit, Miss Felthorne. Was halten Sie von diesem Jungen?"

"Nun -", sagte sie. Sie stolperte über ihre Worte. "Ich denke - niemals auch nur um einen Kuss zu bitten - wäre -"

\emph{\emph{Traurig.}}

\emph{\emph{Einfach erbärmlich.}}

"Schwach", sagte sie, und ihre Stimme zitterte.

"Ich stimme zu", sagte Snape. "Aber nehmen wir an, der Junge hätte Ihnen geholfen. Würden Sie denken, dass Sie ihm einen Kuss schulden, wenn er Sie darum bittet?"

Sie atmete scharf ein -

"Oder würden Sie denken", fuhr Snape fort, die Augen immer noch geschlossen, "dass er nur lästig wäre?"

Die Worte stachen wie ein Messer in sie hinein und sie konnte nicht anders, als laut zu keuchen.

Snapes Augen flogen auf und sein Blick traf den ihren auf der anderen Seite des Sofas.

Dann begann der Meister der Zaubertränke zu lachen, ein kleines, trauriges Kichern.

"Nein, nicht \emph{Sie}, Miss Felthorne!" sagte Snape. "Nicht \emph{Sie}! Wir reden hier \emph{wirklich} von einem Jungen. Einer, der in Ihre Zaubertränkeklasse geht."

"Oh", sagte sie. Sie versuchte sich daran zu erinnern, was Snape vorhin gesagt hatte, und fühlte sich jetzt ziemlich entnervt, als sie an einen Jungen dachte, der sie beobachtete, der sie immer schweigend beobachtete. "Nun, ähm, in diesem Fall. Das ist irgendwie \emph{unheimlich}, ehrlich gesagt. Wer ist es?"

Der Meister der Zaubertränke schüttelte den Kopf. "Das spielt keine Rolle", sagte Snape. "Nur so aus Neugierde: Was würden Sie davon halten, wenn dieser Junge Jahre später immer noch in Sie verliebt wäre?"

"Ähm", sagte sie, etwas verwirrt, "das wäre doch total erbärmlich?"

Die Fackel knisterte ein wenig in der Höhle.

"Es ist seltsam", sagte Snape leise. "Ich hatte im Laufe meiner Jahre zwei Mentoren. Beide waren außerordentlich scharfsinnig, und keiner von beiden hat mir je die Dinge gesagt, die ich nicht gesehen habe. Es ist klar genug, warum der erste nichts gesagt hat, aber der zweite …" Snapes Gesicht straffte sich. "Ich nehme an, ich müsste naiv sein, um zu fragen, warum er geschwiegen hat."

Die Stille dehnte sich, während Rianna krampfhaft versuchte, sich etwas einfallen zu lassen, was sie sagen könnte.

"Es ist wirklich merkwürdig", sagte Snape, seine Stimme noch leiser, "nach nur zweiunddreißig Jahren zurückzublicken und sich zu fragen, wann dein Leben ohne Hoffnung auf Rettung ruiniert wurde. War es schon vorherbestimmt, als der Sprechende Hut 'Slytherin!' für mich rief? Es scheint ungerecht, da ich keine Wahl hatte; der Sprechende Hut sprach in dem Moment, als er meinen Kopf berührte. Dennoch kann ich nicht behaupten, dass er mich unrechtmäßig einsortiert hat. Ich habe das Wissen nie um seiner selbst willen geschätzt. Ich war der einen Person, die ich Freund nannte, gegenüber nicht loyal. Ich war nie jemand für gerechten Zorn, weder damals noch heute. Tapferkeit? Es ist nicht mutig, ein Leben zu riskieren, das bereits ruiniert ist. Meine kleinen Ängste haben mich immer beherrscht, und ich bin wegen dieser kleinen Ängste nie von einem der Wege abgewichen, die ich beschritten habe. Nein, der Sprechende Hut hätte mich niemals in ihr Haus stecken können. Vielleicht stand mein endgültiger Verlust schon damals fest. Ist das gerecht, frage ich, selbst wenn der Sprechende Hut die Wahrheit sagt? Ist es gerecht, dass einige Kinder mehr Mut besitzen als andere und dass so über das Leben eines Menschen geurteilt wird?"

Rianna Felthorne wurde allmählich klar, dass sie nicht die geringste Ahnung davon hatte, wer ihr Zaubertränkemeister im Inneren war, und leider halfen ihr all diese dunklen, verborgenen Tiefen nicht bei ihrem Problem.

"Aber nein", sagte Snape. "Ich weiß, wo es endgültig schiefgegangen ist. Ich könnte genau auf den Tag und die Stunde zeigen, an dem ich meine letzte Chance verpasst habe. Miss Felthorne, hat der Sprechende Hut Ihnen Ravenclaw angeboten?"

"J-ja", sagte sie, ohne nachzudenken.

"Waren Sie jemals gut im Rätseln?"

"Ja", sagte sie wieder, denn was immer Professor Snape sagen wollte, sie würde es nicht hören, wenn sie \emph{nein} sagte.

"Ich bin schrecklich im Rätseln", sagte Snape mit distanzierter Stimme. "Mir wurde einmal ein Rätsel gegeben, das ich lösen sollte, und ich habe nicht einmal den einfachsten Teil verstanden, bis es zu spät war. Ich habe nicht einmal erkannt, dass das Rätsel für \emph{mich} bestimmt war, bis es zu spät war. Ich dachte, ich hätte es nur zufällig mitgehört, aber in Wahrheit war ich es, bei dem mitgehört wurde. Also verkaufte ich das Rätsel an einen anderen, und das war der Zeitpunkt, an dem die Trümmer meines Lebens nicht mehr zu retten waren." Snapes Stimme war immer noch distanziert und klang mehr abstrakt als traurig. "Und selbst jetzt verstehe ich nichts von Bedeutung. Sagen Sie mir, Miss Felthorne, nehmen wir an, ein Mann trüge ein Messer bei sich, er stolperte über ein Baby und erstäche sich selbst. Würden Sie sagen, dass das Baby", Snapes Stimme senkte sich, als würde er eine noch tiefere Stimme imitieren, "DIE MACHT hatte, ihn zu VERNICHTEN?"

"Ähm … nein?", sagte sie zögernd.

"Was bedeutet es \emph{dann}, die Macht zu haben, jemanden vernichten?"

Rianna dachte über das Rätsel nach. (Sie wünschte sich, nicht zum ersten Mal in ihrem Leben, sie hätte Ravenclaw gewählt und zur Hölle mit dem Missfallen ihrer Eltern; aber der Sprechende Hut hatte ihr nie Gryffindor angeboten.) "Nun …" sagte Rianna. Es fiel ihr schwer, ihre Gedanken in Worte zu fassen. "Es bedeutet, dass Sie die \emph{Macht} haben, aber Sie es nicht tun \emph{müssen}. Es bedeutet, dass Sie es tun könnten, wenn Sie es versuchen -"

"Eine Wahl", sagte der Meister der Zaubertränke mit derselben weit entfernten Stimme, als würde er gar nicht wirklich mit ihr sprechen. "Es wird eine Wahl geben. Das ist es, was das Rätsel anzudeuten scheint. Und diese Wahl ist für den Wählenden nicht von vornherein klar, denn das Rätsel sagt nicht, \emph{wird} \emph{vernichten}, sondern \emph{die Macht zu} \emph{vernichten}. Wie würde ein erwachsener Mann ein Baby als seinesgleichen kennzeichnen?"

"Was?", sagte Rianna. Das verstand sie überhaupt nicht.

"Ein Baby zu \emph{markieren} ist einfach. Jeder starke dunkle Fluch würde eine bleibende Narbe hinterlassen. Aber das kann man jedem Kind antun. Welches Zeichen würde bedeuten, dass ein Baby dir \emph{ebenbürtig} ist? "

Sie antwortete mit dem ersten Gedanken, der ihr in den Sinn kam. "Wenn Sie einen Verlobungsvertrag unterschrieben hätten, würde das bedeuten, dass Sie eines Tages mit ihm gleichgestellt wären, wenn es erwachsen ist und Sie heiraten."

"Das …", sagte Snape. "Das ist es wahrscheinlich nicht, Miss Felthorne, aber danke, dass Sie es versucht haben." Die langen, zarten Finger, die durch das Rühren von Zaubertränken auf unvorstellbar feine Toleranzen geschliffen waren, griffen nach oben und rieben an den Schläfen der Stirn des Mannes. "Es ist genug, um mich in den Wahnsinn zu treiben, so viel hängt von solch zerbrechlichen Worten ab. Macht, die er nicht kennt … es \emph{muss} mehr sein als ein unbekannter Zauberspruch. Nicht etwas, das \emph{er} einfach durch Übung und Studium erlangen könnte. Ein angeborenes Talent? Keiner kann lernen, ein Metamorphmagus zu sein… und doch scheint das kaum eine Macht zu sein, die er \emph{nicht kennt}. Ich kann auch nicht erkennen, wie jeder der beiden den anderen bis auf einen Rest zerstören könnte; ich sehe es in der einen Richtung, aber nicht umgekehrt…" Der Meister der Zaubertränke seufzte. "Und nichts von alledem bedeutet Ihnen etwas, nicht wahr, Miss Felthorne? Die Worte sind nichts. Die Worte sind Schatten. Es ist ihre \emph{Intonation}, die die Bedeutung trägt, und das ist etwas, das ich nie konnte…"

Der Zaubertränkemeister brach ab, und Rianna starrte ihn an.

"Eine \emph{Prophezeiung?} ", sagte Rianna mit einem hohen Quietschen. "Sie haben eine \emph{Prophezeiung} gehört? "Sie hatte ein paar Monate lang Wahrsagerei belegt, bevor sie es angewidert abbrach, aber sie wusste wenigstens wie es funktionierte.

"Ich werde noch eine letzte Sache versuchen", sagte Snape. "Etwas, das ich bisher noch nicht versucht habe. Miss Felthorne, hören Sie auf den \emph{Klang} meiner Stimme, auf die Art, wie ich es sage, nicht auf die Worte selbst, und sagen Sie mir, was Sie denken, das es bedeutet. Können Sie das tun? Gut", sagte Snape, als sie gehorsam nickte, obwohl sie sich überhaupt nicht sicher war, was sie tun sollte.

Severus Snape holte tief Luft und intonierte: "DENN JENE ZWEI UNTERSCHIEDLICHEN GEISTER KÖNNEN NICHT IN DERSELBEN WELT EXISTIEREN."

Es jagte ihr einen Schauer über den Rücken, umso mehr, weil sie wusste, dass die hohlen Worte in Nachahmung einer wahren Prophezeiung gesprochen worden waren. Entnervt platzte sie mit dem Ersten heraus, was ihr in den Sinn kam, was vielleicht durch ihre gegenwärtige Gesellschaft beeinflusst war. "Diese beiden unterschiedlichen Zutaten können nicht in demselben Kessel existieren?"

"Aber warum \emph{nicht}, Miss Felthorne? Was ist die \emph{Bedeutung} einer solchen Aussage? Was wird uns da wirklich gesagt?"

"Äh …", wagte sie zu sagen. "Wenn sich die beiden Zutaten vermischen, werden sie Feuer fangen und den Kessel verbrennen?"

Snapes Gesicht veränderte seinen Ausdruck nicht im Geringsten.

"Vielleicht", sagte Snape schließlich, nachdem sie minutenlang in entsetzlichem Schweigen auf dem Sofa gesessen hatten. "Das würde das Wort \emph{'müssen'} erklären. Ich danke Ihnen, Miss Felthorne. Sie sind wieder einmal sehr hilfreich gewesen."

"Ich -", sagte sie, "ich bin froh, dass ich -" und die Worte blieben ihr im Hals stecken. Der Meister der Zaubertränke hatte sich mit einem Ton der Endgültigkeit bei ihr bedankt, und sie wusste, dass die Zeit der Rianna Felthorne, die sich an diese Momente erinnerte, sich dem Ende zuneigte. "Ich wünschte, ich müsste das nicht vergessen, Professor Snape!"

"Ich wünschte", sagte Severus Snape in einem Flüsterton, der so leise war, dass sie ihn kaum hören konnte, "dass alles anders gewesen wäre …"

Der Meister der Zaubertränke stand vom Sofa auf, das Gewicht seiner Anwesenheit neben ihr verschwand. Er drehte sich um, zog seinen Zauberstab aus seinem Gewand und richtete ihn auf sie.

"Warten Sie -", sagte sie. "Bevor das -"

Irgendwie war es unglaublich schwer, den ersten Schritt von der Fantasie zur Realität zu machen, von der Vorstellung zum Tun. Selbst wenn es nur ein Schritt war und nie weiter gehen würde. Die Kluft erstreckte sich wie die Entfernung zwischen zwei Bergen.

Der Sprechende Hut hatte ihr nie Gryffindor angeboten…

… war es fair, dass so über das Leben einer Frau geurteilt wurde?

\emph{\emph{Wenn} \emph{du} \emph{es jetzt nicht sagen kann, wenn} \emph{du dich hinterher nicht einmal mehr daran erinnern wirst} \emph{- wenn von diesem Moment an nichts mehr so sein wird, als würdest du} \emph{sterben -, wann wirst du} \emph{es dann jemals sagen, zu irgendjemandem?}}

"Kann ich zuerst einen Kuss bekommen?", fragte Rianna Felthorne.

Snapes schwarze Augen studierten sie so intensiv, dass ihr die Röte zur Brust reichte, und sie fragte sich, ob er genau wusste, dass sie immer noch schwach war und es kein Kuss war, den sie wirklich wollte.

"Warum nicht", sagte der Meister der Zaubertränke leise, und er beugte seinen Kopf über das Sofa und küsste sie.

Es war nicht so, wie sie es sich vorgestellt hatte. In ihren Fantasien waren Snapes Küsse heftig, von ihr genommen, aber das hier war - eigentlich war es nur \emph{unbeholfen}. Snapes Lippen pressten sich zu fest auf ihre, zwangen sie gegen ihre Zähne, und der Winkel stimmte nicht und ihre Nasen bogen sich irgendwie und seine Lippen waren zu \emph{fest} und -

Erst als der Meister der Zaubertränke sich wieder aufrichtete und seinen Zauberstab erneut hob, wurde es ihr bewusst.

"Das war nicht -", sagte sie mit verwunderter Stimme und sah zu ihm auf. "Das war nicht - war das - Ihr erster -"

Rianna Felthorne blinzelte in der steinernen Höhle, die sie entdeckt hatte, und hielt immer noch den außergewöhnlichen Rubin in der Hand, den sie im Dreck in einer Ecke gefunden hatte. Es war ein unglaublicher Glücksfall, und sie wusste nicht, warum sie sich beim Anblick des Rubins so traurig fühlte, als hätte sie etwas vergessen, etwas, das ihr sehr wichtig gewesen war.

