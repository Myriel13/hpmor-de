

\hypertarget{selbstverwirklichung-finale-verantwortung}{% \section{43. Selbstverwirklichung, Finale -- Verantwortung}\label{selbstverwirklichung-finale-verantwortung}}

\textbf{-\/-\/-\/-\/- Kapitel 75: Selbstverwirklichung, Finale -- Verantwortung \textbf{-\/-\/-\/-\/-}}

Es war eine sich schlängelnde, mäandernde Allee inmitten von Hogwarts, die wie eine verirrte Haarlocke umherwanderte; manchmal kreuzte sie sich selbst, so schien es, aber man konnte niemals bis zum Ende gelangen, wenn man der Versuchung scheinbarer Abkürzungen nachgab.

Am Ende des Gewirrs lehnten sechs Schüler an rauen Steinen, die schwarzen Roben waren grün gesäumt und hoben sich von den grauen Wänden ab, die Augen huschten von einem zum anderen. Fackeln brannten in den Wandleuchten und spendeten Licht gegen die Dunkelheit und Wärme gegen die Kälte der Slytherin-Kerker.

„Ich bin \emph{sicher}“, schnauzte Reese Belka, „absolut \emph{sicher}, dass das kein richtiges Ritual war. Kleine Erstlingshexen können diese Art von Magie nicht, und selbst wenn sie es könnten, wer hat je von einem dunklen Ritual gehört, bei dem ein versiegeltes Grauen \emph{geopfert} wird - für - \emph{das?} „

„Warst du -“, sagte Lucian Bole. „Ich meine - nachdem das Mädchen mit den Fingern geschnippt hat -„

Belkas Blick hätte ihn zum Schmelzen bringen müssen. „Nein“, spuckte sie, „das war ich \emph{nicht}."

„Das heißt, sie war nicht nackt“, murmelte Marcus Flint, seine breiten Schultern lehnten sich scheinbar entspannt gegen die unebene Steinoberfläche. „Bedeckt mit Schokoladenglasur, ja, aber nicht nackt."

„Potter hat unsere Häuser heute aufs Übelste beleidigt“, sagte die grimmige Stimme von Jaime Astorga.

„Ja, ähm, es tut mir leid, dass ich so unverblümt bin“, sagte Randolph Lee gleichmütig. Der Duellant im siebten Jahr rieb sich am Kinn, wo er einen schwachen Bartflaum hatte wachsen lassen. „Aber wenn dich jemand an die Decke klebt, ist das eine Botschaft, Astorga. Es ist eine Botschaft, die besagt: Ich bin ein unglaublich mächtiger Dunkler Zauberer, der mit dir alles machen kann, was mir verdammt noch mal gefällt, und es ist mir egal, ob dein Haus beleidigt ist."

Robert Jugson III. stieß daraufhin ein leises, tiefes Lachen aus, ein Kichern, das einigen einen Schauer über den Rücken jagte. „Da fragt man sich schon, ob man sich die falsche Seite ausgesucht hat, nicht wahr? Ich habe Geschichten über solche \emph{Botschaften} gehört, die auf Geheiß des alten Dunklen Lords geschickt wurden..."

„Ich bin noch nicht bereit, vor Potter auf die Knie zu gehen“, sagte Astorga und starrte Jugson fest in die Augen.

„Ich auch nicht“, sagte Belka.

Jugson hielt seinen Zauberstab in der Hand und drehte ihn müßig in seinen Fingern hin und her, wobei er mal nach oben und dann wieder nach unten zeigte. „Bist du ein Gryffindor oder ein Slytherin?“, fragte Jugson. „Jeder hat einen Preis. Jeder, der klug ist."

Diese Aussage erzeugte einen Moment der Stille.

„Sollte Malfoy nicht hier sein?“ sagte Bole zaghaft.

Flint wedelte abweisend mit den Fingern. „Was auch immer Malfoy vorhat, er will den Anschein der Unschuld erwecken. Er darf nicht zur gleichen Zeit wie wir vermisst werden."

„Aber das \emph{weiß} doch schon jeder“, sagte Bole. „Auch in den anderen Häusern."

„Ja, sehr ungeschickt“, sagte Belka. Sie schnaubte. „Malfoy hin oder her, er ist nur ein kleiner Erstie und wir brauchen ihn hier nicht."

„Ich werde meinem Vater eine Eule schicken“, sagte Jugson leise, „und \emph{er} wird selbst mit Lord Malfoy sprechen -“ Abrupt hörte Jugson auf zu sprechen.

„Ich weiß nicht, wie es \emph{euch} geht, meine Lieben“, sagte Belka gespielt liebevoll, „aber ich habe nicht vor, vor einem falschen Ritual davonzulaufen, und \emph{ich} bin noch nicht fertig mit Potter und seinem Lieblingsschlammblut."

Niemand antwortete. Alle Blicke richteten sich an ihr vorbei.

Langsam drehte sich Belka um, um zu sehen, wohin die anderen starrten.

„Du wirst \emph{nichts} tun“, zischte ihr Hausoberhaupt. Severus Snapes Gesicht war wütend, als er sprach, flogen kleine Speichelflecken aus seinem Mund, die seine ohnehin schon schmutzigen Roben noch mehr befleckten. „Ihr Narren habt \emph{genug} getan! Ihr habt mein Haus in Verlegenheit gebracht - gegen Erstklässler \emph{verloren} - und jetzt wollt ihr die edlen Herren des Zaubergamot in eure \emph{erbärmlichen}, kindischen Streitereien verwickeln? \emph{Ich} werde mich um diese Angelegenheit kümmern. \emph{Ihr} werdet dieses Haus nicht noch einmal in Verlegenheit bringen. Ihr werdet nicht \emph{riskieren}, dieses Haus noch einmal in Verlegenheit zu bringen! Gegen Hexen zu kämpfen ist für euch \emph{erledigt}, und wenn ich etwas anderes höre -"

\hfill\break Wer dachte, dass sie danach beim Abendessen nebeneinander sitzen würden, der irrte sich gewaltig.

„Was \emph{will} sie von mir?“, kam der klagende Aufschrei eines Jungen, der trotz seiner umfangreichen Lektüre in wissenschaftlicher Literatur immer noch ein wenig naiv war, was gewisse Dinge anging. „\emph{Wollte} sie verprügelt werden?"

Die Ravenclaw-Jungen aus der Oberstufe, die sich neben ihn an den Esstisch gesetzt hatten, tauschten flüchtige Blicke miteinander aus, bis nach einem unausgesprochenen Protokoll der erfahrenste von ihnen das Wort ergriff.

„Hör zu“, sagte Arty Grey, der Siebtklässler, der in ihrem Wettbewerb mit drei Hexen und einem Verteidigungsprofessor in Führung lag, „die Sache, die du verstehen musst, ist, nur weil sie \emph{wütend} ist, bedeutet das nicht, dass du Punkte verloren hast. Miss Granger ist wütend, weil sie sich erschrocken hat und du \emph{bist dafür verantwortlich}, verstehst du? Aber gleichzeitig, auch wenn sie es nicht zugeben wird, ist sie gerührt, dass ihr Freund so lächerliche und offen gesagt verrückte Anstrengungen unternommen hat, um sie zu beschützen."

„Hier geht es nicht um \emph{Punkte}“, stieß Harry Potter hervor, wobei die Worte fast sichtbar zwischen seinen zusammengebissenen Zähnen hervorkamen. Das Essen stand unbeachtet vor ihm auf dem Tisch. „Hier geht es um \emph{Gerechtigkeit}. Und \emph{ich}. \emph{Bin. Nicht. Ihr}. \emph{Freund}. „

Dies wurde von allen Anwesenden mit einem gewissen Gekicher quittiert.

„Naja“, sagte ein Ravenclaw-Junge aus dem sechsten Jahr, „ich denke, nachdem sie dich geküsst hat, um dich aus der Dementation zu holen, und du vierundvierzig Schläger für sie an die Decke geklebt hast, sind wir weit über 'sie ist nicht meine Freundin, wirklich' hinaus und bei der Frage angelangt, wie eure Kinder sein werden. Wow, das ist ein beängstigender Gedanke...“ Der Ravenclaw brach ab und sagte dann mit leiserer Stimme: „Bitte sieh mich nicht so an."

„Hör mal“, sagte Arty Grey, „es tut mir leid, dass ich so offen bin, aber du kannst entweder Gerechtigkeit oder Mädchen haben, di kannst nicht beides gleichzeitig haben.“ Er klopfte Harry Potter kameradschaftlich auf die Schulter. „Du hast Potenzial, Junge, mehr Potenzial als jeder Zauberer, den ich je gesehen habe, aber du musst lernen, es zu \emph{nutzen}, weißt du? Sei ein bisschen netter zu ihnen, lerne ein paar Zaubersprüche, um das Chaos, das du Haare nennst, zu recht zu machen. Vor allem musst du deine Bösartigkeit besser verstecken - nicht \emph{zu} gut, aber besser. Nette, gepflegte Jungs bekommen Mädchen, und Dunkle Zauberer bekommen auch Mädchen, aber nette, gepflegte Jungs, die im Verdacht stehen, \emph{heimlich} Dunkel zu sein, bekommen mehr Mädchen, als du dir vorstellen kannst -"

„Kein Interesse“, sagte Harry flach, während er die Hand des Jungen von seiner Schulter nahm und sie kurzerhand fallen ließ.

„Ach das kommt noch“, sagte Arty Grey, seine Stimme tief und ahnungsvoll. „Ja, das kommt noch!"

Anderswo am selben Tisch -

„\emph{Romantisch?}“, kreischte Hermine Granger, so laut, dass einige der Mädchen neben ihr zusammenzuckten. „\emph{Welcher Teil davon war romantisch?} Er hat nicht \emph{gefragt}! Er \emph{fragt} nie! Er schickt einfach Geister hinter den Leuten her und klebt sie an Decken und macht mit \emph{meinem} Leben, was er will!"

„Aber verstehst du denn nicht?“, sagte eine Hexe aus dem vierten Jahr. „Das heißt, obwohl er böse ist, \emph{liebt} er dich!"

„Du bist nicht hilfreich“, sagte Penelope Clearwater ein Stück weiter unten am Tisch, aber sie wurde ignoriert. Einige ältere Hexen hatten sich auf Hermine zubewegt, nachdem sie sich an das äußerste Ende des Tisches gegenüber von Harry Potter gesetzt hatte, aber dann hatte eine schnellere Wolke von jüngeren Mädchen Hermine mit einer undurchdringlichen Barriere umgeben.

„Jungen“, sagte Hermine Granger, „sollte es nicht erlaubt sein, Mädchen zu lieben, ohne sie vorher zu fragen! Das gilt in vielerlei Hinsicht und vor allem, wenn es darum geht, Menschen an die Decke zu kleben!„

Auch dies wurde ignoriert. „Das ist ja wie im Theater!“ seufzte ein Mädchen aus dem dritten Jahr.

„Im Theater?“, fragte Hermine. „Ich würde gerne das Stück sehen, in dem \emph{so etwas} passiert!"

„Oh“, sagte die Drittklässlerin, „ich dachte an dieses wirklich \emph{romantische} Stück, in dem es diesen sehr netten, süßen Jungen gibt, der einen Floo-Ruf macht, nur dass er sein Ziel falsch ausspricht und in diesen Raum voller Dunkler Zauberer stolpert, die ein verbotenes Ritual durchführen, das für immer in der Zeit verloren bleiben sollte, und sie opfern sieben Opfer, um dieses uralte Grauen zu entsiegeln, das jemandem einen Wunsch erfüllen soll, wenn es befreit wird, und natürlich unterbricht die Anwesenheit des Jungen das Ritual, und während das Grauen alle Dunklen Zauberer frisst und alle sterben, ist der letzte Gedanke des Jungen, dass er sich wünscht, er hätte eine Freundin gehabt, und das nächste, was man sieht, ist, dass der Junge im Schoß dieser schönen Frau liegt, deren Augen in einem schrecklichen Licht brennen, nur versteht sie nichts davon, ein Mensch zu sein, also muss der Junge sie immer davon abhalten, Menschen zu fressen. Das ist genau wie in dem Stück, nur dass du der Junge bist und Harry Potter das Mädchen!"

„Das...“ sagte Hermine, die ziemlich überrascht war. „Das klingt \emph{tatsächlich} so ähnlich wie -"

"\emph{Wirklich?} „, platzte eine Zweitklässlerin heraus, die am anderen Ende des Tisches saß und sich nun nach vorne lehnte, entsetzt und noch faszinierter dreinblickend.

„Nein!“, sagte Hermine. „Ich meine - \emph{er ist nicht mein Freund!} „

Zwei Sekunden später vernahmen Hermines Ohren, was ihre Lippen gerade gesagt hatten.

Die Hexe im vierten Jahr legte ihre Hand auf Hermines Schulter und drückte sie tröstend. „Miss Granger“, sagte sie mit beruhigender Stimme, „ich denke, wenn du wirklich ehrlich zu dir selbst bist, wirst du zugeben, dass der wahre Grund, warum du wütend auf deinen Dunklen Meister bist, der ist, dass er seine unaussprechlichen Kräfte durch Tracey Davis kanalisiert hat statt durch dich."

Hermines Mund öffnete sich, aber ihre Kehle schnürte sich zu, bevor die Worte herauskamen, was wahrscheinlich gut war, denn wenn sie tatsächlich so laut geschrien hätte, wäre etwas kaputt gegangen.

„Wie ist das eigentlich möglich?“, fragte die Drittklässlerin. „Ich meine, dass Harry Potter durch ein anderes Mädchen wirken kann, obwohl er sich an dich gebunden hat? Habt ihr drei etwa so eine, du weißt schon, Abmachung?"

„\emph{Haaaaack}“, sagte Hermine Granger, ihre Kehle war immer noch verschlossen, ihr Gehirn hatte angehalten, und ihre Stimmbänder machten spontan ein Geräusch, als würde sie ein Yak aushusten.

\hfill\break \emph{\emph{(Später.)}}

„Ich verstehe nicht, warum du so \emph{unvernünftig} bist“, sagte eine andere Hexe aus dem zweiten Jahr, die das Mädchen aus dem dritten Jahr ersetzt hatte, nachdem Hermine gedroht hatte, Tracey zu bitten, ihre Seele zu essen. „Ich meine, wirklich, wenn jemand wie Harry Potter \emph{mich} gerettet hätte, würde ich - ihm Dankeskarten schicken und ihn umarmen und“, das Gesicht des Mädchens war etwas rot, „na ja, ihn küssen, hoffe ich."

„Ja!“, sagte die andere Hexe aus dem zweiten Jahr. „Ich habe nie verstanden, warum Mädchen in Theaterstücken \emph{wütend} werden, wenn die Hauptfigur sich Mühe gibt, nett zu ihnen zu sein. \emph{Ich} würde mich nicht so verhalten, wenn der Held \emph{mich} mögen würde."

Hermione Granger hatte den Kopf auf den Esstisch sinken lassen, ihre Hände zogen langsam an ihren Haaren.

„Du verstehst einfach nichts von männlicher Psychologie“, sagte die Hexe im vierten Jahr mit autoritärer Stimme. „Granger muss es so \emph{aussehen} lassen, als könne sie auf geheimnisvolle Weise seinem verführerischen Charme widerstehen."

\hfill\break \emph{\emph{(Noch} \emph{später.)}}

Und so hatte sich Hermine Granger schon bald an die einzige Person gewandt, mit der sie noch reden konnte, die einzige Person, die ihren Standpunkt garantiert verstehen würde -

„Sie sind alle verrückt“, sagte Hermine Granger, während sie energisch in Richtung Ravenclaw-Turm schritt, nachdem sie das Abendessen etwas früher verlassen hatte. „Alle außer dir und mir, Harry, ich meine \emph{alle} außer uns in dieser ganzen Hogwarts Schule, sie sind alle völlig \emph{verrückt}. Und die Ravenclaw-Mädchen sind die \emph{schlimmsten}, ich weiß nicht, \emph{was} Ravenclaw-Mädchen lesen, wenn sie älter werden, aber ich bin mir sicher, dass sie es nicht lesen sollten. Eine Hexe hat mich gefragt, ob wir beide seelenverbunden seien, was ich heute Abend in der Bibliothek nachschlagen werde, aber ich bin mir ziemlich sicher, dass das noch nie passiert ist -"

„Ich kenne nicht einmal einen \emph{Namen} für diese Art von Trugschlüssen“, sagte Harry Potter. Der Junge ging normal, was bedeutete, dass er oft ein paar Schritte vorwärts hüpfen musste, um mit ihrer eigenen, von Empörung getragenen Geschwindigkeit mitzuhalten. „Ich glaube ernsthaft, wenn es nach \emph{denen} ginge, würden sie uns in dieser Minute abschleppen, um unsere Namen in Potter-Evans-Verres-Granger ändern zu lassen... Wenn ich das laut ausspreche, wird mir klar, wie furchtbar das klingt."

„Du meinst, \emph{dein} Name wäre Potter-Evans-Verres-Granger und \emph{meiner} wäre Granger-Potter-Evans-Verres“, sagte Hermine. „Das ist zu schrecklich, um es sich vorzustellen."

„Nein“, sagte der Junge, „Haus Potter ist ein Adelshaus, also denke ich, der Name bleibt vorne -"

„\emph{Was?}“, sagte sie entrüstet. „Wer sagt denn, dass \emph{wir} -"

Es herrschte plötzlich eine betretene Stille, die nur durch das Tappen ihrer Schuhe unterbrochen wurde.

„\emph{Wie auch immer}“, sagte Hermine hastig, „einige der verrückten Dinge, die sie beim Abendessen gesagt haben, haben mich zum Nachdenken gebracht, also möchte ich dir nur sagen, Harry, dass ich dir wirklich dankbar bin, dass du mich und alle anderen davor bewahrt hast, verprügelt zu werden, und auch wenn mich einige Teile des heutigen Nachmittags verärgert haben, bin ich sicher, dass wir in Ruhe darüber reden können."

„Äh...“ Harry sagte mit einem schwachen und zaghaften Lächeln, seine Augen zeigten eine Mischung aus Verwirrung und Besorgnis, „das ist... gut, denke ich?"

Um genau zu sein, war da die Hexe aus dem vierten Jahr, die erklärte, dass, da Harry der böse Zauberer war, der sich in Hermine verliebt hatte, und Hermine das reine und unschuldige Mädchen, das ihn entweder erlösen oder selbst von den Dunklen Künsten verführt werden würde, daraus folgte, dass Hermine ständig über alles, was Harry tat, entrüstet sein \emph{musste}, selbst wenn er sie heldenhaft vor dem sicheren Untergang rettete, nur damit sich ihre Romanze nicht vor dem Ende des vierten Aktes auflöste. Und \emph{dann} hatte Penelope Clearwater, die Hermine eigentlich für klüger gehalten hatte, mit lauter Stimme bemerkt, dass es für Hermine aus denselben Gründen \emph{unmöglich} war, einfach rüberzugehen und vernünftig mit Harry darüber zu reden, warum sie sich verletzt fühlte, und außerdem fühlten sich Dunkle Zauberer von leidenschaftlichem Trotz bei einer Frau angezogen, nicht von Logik. Das war der Punkt, an dem Hermine sich von den Bänken hochgeschoben hatte, wütend zu Harry hinüberstapfte und ihn mit vernünftiger Stimme fragte, ob sie beide spazieren gehen und die Dinge klären könnten.

„Also mit anderen Worten“, sagte Hermine mit ihrer ruhigsten Stimme überhaupt, „du hast nicht wirklich Ärger mit mir, ich rede immer noch mit dir, wir sind immer noch Freunde, und wir lernen immer noch zusammen. Wir haben \emph{keinen} Streit. Richtig?„

Irgendwie schien dies Harry Potters Befürchtungen nur noch zu verstärken. „Richtig“, sagte der Junge, der lebte.

„Großartig!“, sagte Hermine. „Also, hast du herausgefunden, warum ich mich aufgeregt habe, Mr. Potter?„

Es gab eine Pause. „Du wolltest, dass ich mich aus deinen Angelegenheiten heraushalte?“ sagte Harry vorsichtig. „Ich meine - ich weiß, du wolltest die Dinge selbst in die Hand nehmen. Und ich \emph{wollte} dir aus dem Weg gehen, bis ich hörte, dass du von drei Junior Todessern in einen Hinterhalt gelockt wurdest, und damit hatte ich, ehrlich gesagt, nicht gerechnet. \emph{Professor Quirrell} hat das auch nicht erwartet. Ich dachte ihr wärt überfordert und dann, nichts für ungut, Hermine, 44 Tyrannen in einem Hinterhalt, \emph{niemand} könnte das ohne Hilfe bewältigen. Deshalb dachte ich, du bräuchtest wirklich Hilfe, nur dieses eine Mal -"

„Nein, der Teil ist in Ordnung“, sagte Hermine. „Wir \emph{waren} überfordert, ganz ehrlich. Bitte rate noch einmal, Mr. Potter."

„Ähm“, sagte Harry. „Was Tracey getan hat... hat dich beunruhigt?"

„Mich beunruhigt, Mr. Potter?“ Da war vielleicht ein Hauch von Säure in ihrer Stimme. „Nein, Mr. Potter, ich habe mich \emph{erschrocken}. Ich habe mich \emph{gefürchtet}. Ich würde nicht zugeben wollen, dass ich mich nur vor \emph{Drachen} oder so fürchte, die Leute könnten mich für \emph{feige} halten, aber wenn man in der Ferne Stimmen hört, die 'Tekeli-li! Tekeli-li!' rufen und unter allen Türen Blutlachen hervorquellen, dann ist es okay, Angst zu haben."

„Es \emph{tut} mir leid“, sagte Harry mit etwas, das wie echtes Bedauern klang. „Ich dachte, du würdest erkennen, dass ich es war."

„Und der \emph{Grund}, warum wir uns alle so erschreckt haben, Mr. Potter, war, dass \emph{du nicht zuerst gefragt hast!}“ Trotz ihrer Absichten stellte Hermine fest, dass ihre Stimme wieder anstieg. „Du hättest mich \emph{fragen} müssen, bevor du so etwas tust, Harry! Du hättest ganz konkret sagen sollen: 'Hermine, kann ich dafür sorgen, dass Blut unter den Türen hervorkommt?' Es ist wichtig, genau zu sein, wenn man nach so etwas fragt!"

Der Junge rieb sich im Gehen den Nacken. „Ich... ehrlich gesagt, dachte ich, du \emph{müsstest} dann Nein sagen."

"Ja, Mr. Potter, \emph{ich hätte nein sagen können}. Das ist doch \emph{der Sinn, wenn man vorher fragt}, Mr. Potter!"

"Nein, ich meine, du hättest nein sagen \emph{müssen}, egal ob es das war, was du \emph{wirklich} wolltest oder nicht. Und dann wärt ihr alle verprügelt worden und es wäre \emph{meine} Schuld gewesen, weil ich zuerst gefragt hätte.„

Hermines Augenbrauen gingen überrascht nach oben, und sie ging ein paar Schritte weiter, während sie versuchte, das zu verstehen. „Was?“, fragte sie.

„Nun...“, sagte der Junge etwas langsam. „Ich meine... du bist doch der Sonnenschein-General, nicht wahr? Du \emph{konntest} nicht ja sagen, wenn ich Leuten Angst mache, nicht einmal Schlägern, nicht einmal, um deine Freunde davor zu bewahren, verprügelt zu werden. Du hättest nein sagen \emph{müssen}, und dann wärst du verletzt worden. Auf diese Weise kannst du den Leuten ehrlich sagen, dass du keine Ahnung hattest und dass es nicht deine Schuld war. Deshalb habe ich dich auch nicht gewarnt."

Hermine blieb stehen und wandte sich Harry ganz zu, anstatt nur den Kopf zu drehen. Ihre Stimme war vorsichtig gleichmäßig, als sie sagte: „Harry, du \emph{musst} aufhören, dir schlaue Gründe für dumme Dinge auszudenken."

Harrys Augenbrauen flogen in die Höhe. Nach einem Moment sagte er: „Hör mal... Ich weiß natürlich, was du meinst, aber da ist immer noch die Frage, ob es tatsächlich eine gute Idee \emph{ist}, nicht nur eine clevere -"

„Ich verstehe, warum du getan hast, was du heute getan hast“, sagte Hermine. „Aber ich möchte, dass du mir versprichst, dass du mich von jetzt an immer zuerst fragst, auch wenn dir ein Grund einfällt, warum du es nicht tun solltest."

Es gab eine Pause, die sich in die Länge zog, und Hermine konnte spüren, wie ihr Herz sank.

„Hermine -“ begann Harry zu sagen.

„\emph{Warum?}“ Die Frustration brach in ihrer Stimme hervor. „\emph{Warum ist es so furchtbar? Du brauchst doch nur zu fragen!}"

Harrys Augen waren sehr ernst. „Wen in B. E.L. F.E. R. versuchst du am meisten zu verteidigen, Hermine? Um wen hast du am meisten Angst, wenn du kämpfst?"

„Hannah Abbott“, sagte Hermine, ohne darüber nachzudenken, und fühlte sich dann ein bisschen schlecht, denn Hannah gab sich große Mühe, und sie \emph{hatte} sich sehr verbessert -

"Wäre es für dich in Ordnung, jemand anderem, wie Tracey, die \emph{finale} Verantwortung für den Schutz von Hannah anzuvertrauen? Wenn du wüsstest, dass Hannah in einen Hinterhalt laufen würde, und du dir einen Plan ausdenken würdest, um sie zu beschützen, würdest du dich gut dabei fühlen, Tracey sagen zu lassen, ob du es tun darfst oder nicht?"

„Nun... nein?“, sagte Hermine verwirrt.

Die grünen Augen des Jungen, der lebte waren fest auf ihre gerichtet. „Würdest du Hannah zutrauen, das letzte Wort darüber zu haben, ob sie beschützt werden muss?"

„Ich -“, sagte Hermine und hielt dann inne. Es war seltsam, sie wusste die richtige Antwort und sie wusste auch, dass die richtige Antwort eigentlich nicht stimmte. Hannah bemühte sich so sehr, zu beweisen, dass sie keine Angst hatte, obwohl sie ängstlich \emph{war}, und es war leicht zu erkennen, dass das Hufflepuff-Mädchen sich \emph{zu} sehr bemühen könnte -

Dann erkannte Hermine die Andeutung. „Du denkst, ich bin wie \emph{Hannah?}"

„Nicht... genau...“ Harry fuhr sich mit den Händen durch sein wirres Haar. „Hör zu, Hermine, was hättest \emph{du} vorgeschlagen zu tun, wenn ich dich vor einem Hinterhalt von vierundvierzig Tyrannen gewarnt hätte?"

„Ich hätte das \emph{Verantwortungsvolle} getan und es \emph{Professor} \emph{McGonagall} gesagt und \emph{sie} sich darum kümmern lassen“, sagte Hermine prompt. „Und \emph{dann} hätte es keine Dunkelheit und schreiende Menschen und schreckliches blaues Licht gegeben -"

Aber Harry schüttelte nur den Kopf. „So etwas ist \emph{nicht} verantwortungsvoll, Hermine. Es ist das, was jemand, der die \emph{Rolle} eines verantwortungsvollen Mädchens spielt, tun würde. \emph{Ja}, ich dachte daran, zu Professor McGonagall zu gehen. Aber sie hätte die Katastrophe nur \emph{einmal} aufgehalten. Wahrscheinlich, bevor es überhaupt zu einer Störung kam, z. B. indem sie den Schlägern sagte, dass sie Bescheid weiß. Wenn die Rowdies bestraft würden, nur weil sie sich verschworen hatten, wäre es der Verlust von Hauspunkten oder schlimmstenfalls ein Tag Nachsitzen, nichts, was ihnen wirklich Angst machen würde. Und dann hätten die Tyrannen es \emph{wieder versucht}. Weniger von ihnen, mit besserer operativer Sicherheit, so dass ich nichts davon mitbekommen hätte. Sie würden wahrscheinlich \emph{einem} von euch auflauern, allein. Professor McGonagall hat nicht die \emph{Autorität}, etwas zu tun, das beängstigend genug ist, um euch zu schützen - und \emph{sie} hätte ihre Autorität nicht überschritten, weil sie nicht wirklich verantwortlich ist."

„\emph{Professor} \emph{McGonagall} ist nicht verantwortlich?“ sagte Hermine ungläubig. Sie stemmte die Hände in die Hüften und starrte ihn nun offen an. „Bist du \emph{verrückt?}„

Der Junge blinzelte nicht. „Man könnte es vielleicht heldenhafte Verantwortung nennen“, sagte Harry Potter. „Nicht so wie die übliche Art. Es bedeutet, dass, was auch immer passiert, egal was, es \emph{immer} deine Schuld ist. Selbst wenn du es Professor McGonagall sagst, ist sie nicht verantwortlich für das, was passiert, sondern \emph{du}. Die Schulregeln zu befolgen ist keine Ausrede, jemand anderes zu sein ist keine Ausrede, sogar sein Bestes zu versuchen ist keine Ausrede. Es gibt einfach keine Ausreden, du musst \emph{den Job erledigen, egal was passiert}.“ Harrys Gesicht straffte sich. „Deshalb sage ich, dass du nicht verantwortungsbewusst denkst, Hermine. Zu denken, dass dein Job erledigt ist, wenn du es Professor McGonagall sagst - das ist kein Heldinnendenken. Als ob es dann \emph{okay} ist, dass Hannah verprügelt wird, weil es nicht mehr \emph{deine Schuld} ist. Eine Heldin zu sein bedeutet, dass dein Job erst dann erledigt ist, wenn du \emph{alles getan hast, was nötig ist}, um die anderen Mädchen zu schützen, und zwar \emph{dauerhaft}.“ In Harrys Stimme lag ein Hauch des Stahls, den er sich seit dem Tag angeeignet hatte, an dem er Fawkes auf der Schulter gehabt hatte. „Du darfst nicht so tun, als ob das bloße Befolgen der Regeln bedeutet, dass du deine Pflicht getan hast."

„Ich denke“, sagte Hermine ruhig, „dass du und ich in einigen Dingen nicht einer Meinung sind, Mr. Potter. Zum Beispiel darüber, ob du oder Professor McGonagall \emph{verantwortungsbewusster} seid, und ob \emph{verantwortungsbewusst} zu sein normalerweise bedeutet, dass Leute herumrennen und schreien, und wie sehr es eine gute Idee ist, die Schulregeln zu befolgen. Und nur weil wir uns nicht einig sind, Mr. Potter, heißt das nicht, dass \emph{du} das letzte Wort hast."

„Nun“, sagte Harry, „Du hast gefragt, was so schrecklich daran ist, dich zuerst fragen zu müssen, und das war eine überraschend gute Frage, also habe ich meinen Verstand untersucht, und das ist, was ich gefunden habe. Ich glaube, meine wirkliche Angst ist, dass, wenn Hannah in Schwierigkeiten ist und ich mir einen Weg ausdenke, sie zu retten, der seltsam oder dunkel oder so erscheint, du die Konsequenzen für Hannah nicht abwägen könntest. Du könntest die Verantwortung der Heldin nicht akzeptieren, sich \emph{irgendeinen} Weg einfallen zu lassen, sie zu retten, irgendwie, egal wie. Stattdessen würdest du einfach die \emph{Rolle} von Hermine Granger, dem vernünftigen Ravenclaw-Mädchen, spielen; und die Rolle von Hermine Granger sagt automatisch nein, egal ob sie einen besseren Plan hat oder nicht. Und dann werden vierundvierzig Tyrannen abwechselnd Hannah Abbott verprügeln, und es wird alles meine Schuld sein, weil ich \emph{Bescheid wusste}, auch wenn ich nicht wollte, dass die Realität so ist, ich wusste, dass es so laufen würde. Ich bin mir ziemlich sicher, dass das meine geheime, wortlose, unsagbare Angst war.„

Die Frustration baute sich wieder in ihr auf. „Es ist \emph{mein} Leben!“ platzte Hermine heraus. Sie konnte sich vorstellen, wie es sein würde, wenn Harry sich die ganze Zeit mit ihr anlegte, ständig Rechtfertigungen erfand, um sie nicht zuerst zu fragen und ihre Einwände nicht zu hören. Sie sollte keine \emph{Diskussion gewinnen} müssen, nur um - „Es wird \emph{immer} einen Grund geben, du kannst \emph{immer} sagen, dass ich nicht richtig denke! Ich will mein \emph{eigenes Leben}! Sonst gehe ich weg, das werde ich wirklich, ich meine es ernst, Harry.„

Harry seufzte. „Das ist genau das, wie ich nicht wollte, dass die Dinge enden, und hier sind wir nun. Du hast vor genau demselben Angst wie ich, nicht wahr? Angst, dass wir abstürzen, wenn \emph{du} das Steuer loslässt.“ Seine Lippenwinkel verzogen sich, aber es sah nicht wie ein echtes Lächeln aus. „Das ist etwas, das ich verstehen kann."

"Ich glaube, du verstehst \emph{überhaupt nichts}! „, sagte Hermine barsch. „Du hast gesagt, wir würden \emph{Partner} sein, Harry!"

Das hielt ihn auf, sie konnte sehen, wie es ihn stoppte.

„Wie wäre es damit?“ sagte Harry schließlich. „Ich verspreche, dich zuerst zu fragen, bevor ich irgendetwas tue, was als Einmischung in deine Angelegenheiten ausgelegt werden könnte. \emph{Du} musst \emph{mir} nur versprechen, dass du vernünftig bist, Hermine. Ich meine \emph{wirklich}, aufrichtig, halte erst mal inne und denke zwanzig Sekunden nach, behandle es als eine echte Entscheidung. Die Art von Vernunft, bei der dir klar ist, dass ich dir eine Möglichkeit anbiete, die anderen Mädchen zu schützen, und dass es, wenn du automatisch \emph{nein} sagst, ohne es richtig zu bedenken, die \emph{tatsächliche Konsequenz} gibt, dass Hannah Abbott im Krankenhaus landet."

Hermine starrte Harry an, als der seinen Vortrag beendete.

„Und?“, sagte Harry.

„Ich sollte keine Versprechungen machen müssen“, sagte sie, „nur um über \emph{mein eigenes Leben befragt} zu werden.“ Sie wandte sich von Harry ab und begann in Richtung des Ravenclaw-Turms zu gehen, ohne ihn anzusehen. „Aber ich werde trotzdem darüber nachdenken."

Sie hörte Harry seufzen, und danach gingen sie eine Weile schweigend weiter, durch einen Torbogen aus irgendeinem rötlichen Metall, das wie Kupfer aussah, in einen Korridor, der genauso aussah wie der, den sie verlassen hatten, nur dass er mit Fünfecken statt mit Quadraten gefliest war.

„Hermine...“, sagte Harry. „Ich habe dich beobachtet und nachgedacht, seit dem Tag, an dem du gesagt hast, du würdest eine Heldin werden. Du \emph{hast} den Mut dazu. Du kämpfst für das, was richtig ist, selbst im Angesicht von Feinden, die andere Leute abschrecken würden. Du hast die nötige Intelligenz dafür und bist im Innern wahrscheinlich ein besserer Mensch als ich. Aber trotzdem... Aber, um ehrlich zu sein, Hermine. Ich kann mir nicht vorstellen, dass du in Dumbledores Fußstapfen trittst und den Kampf des magischen Britanniens gegen Du-weißt-schon-wen führst. Jedenfalls noch nicht."

Hermine hatte den Kopf gedreht und starrte Harry an, der einfach weiterging, als sei er in Gedanken versunken. \emph{Diese} Schuhe ausfüllen? Sie hatte nie versucht, sich selbst auf diese Weise vorzustellen. Sie hatte sich nie \emph{vorgestellt}, sich selbst so vorzustellen.

„Und vielleicht irre ich mich“, sagte Harry, als sie weitergingen. „Vielleicht habe ich einfach zu viele Geschichten gelesen, in denen die Helden nie das Vernünftige tun und die Regeln befolgen und es ihren Professor McGonagalls sagen, deshalb hält mein Gehirn dich nicht für einen richtigen Märchenhelden. Vielleicht bist du die Vernünftige, Hermine, und ich bin nur dumm. Aber jedes Mal, wenn du davon sprichst, Regeln zu befolgen oder dich auf Lehrer zu verlassen, habe ich das gleiche Gefühl, als ob es mit dieser einen letzten Sache zusammenhängt, die dich aufhält, einer letzten Sache, die dein Spieler-Charakter ~einschläfert und dich wieder in einen Nicht-Spieler-Charakter verwandelt...“ Harry stieß einen Seufzer aus. „Vielleicht hat Dumbledore deshalb gesagt, ich sollte böse Stiefeltern haben."

"Er hat \emph{was} gesagt? „

Harry nickte. „Ich weiß immer noch nicht, ob der Schulleiter einen Scherz gemacht hat oder... die Sache ist, dass er in gewisser Weise \emph{recht} hatte. Ich \emph{hatte} liebevolle Eltern, aber ich hatte nie das Gefühl, dass ich ihren Entscheidungen trauen konnte, sie waren nicht \emph{vernünftig} genug. Ich wusste immer, wenn ich die Dinge nicht selbst durchdenke, könnte ich verletzt werden. Professor McGonagall wird alles tun, was nötig ist, um den Job zu erledigen, wenn \emph{ich} da bin, um sie zu nerven, sie bricht keine Regeln auf eigene Faust ohne heldenhafte Aufsicht. Professor Quirrell \emph{ist} wirklich jemand, der die Dinge erledigt, egal was passiert, und er ist die einzige andere Person, die ich kenne, die Dinge wie den Schnatz bemerkt, der Quidditch ruiniert. Aber \emph{ihm} kann ich nicht vertrauen, dass er ein \emph{guter Mensch} ist. Auch wenn es traurig ist, denke ich, dass das ein Teil der Umgebung ist, die das hervorbringt, was Dumbledore einen Helden nennt - Leute, die niemanden haben, auf den sie die finale Verantwortung schieben können, und deshalb bilden sie die geistige Gewohnheit, alles selbst zu verfolgen."

Hermine sagte nichts dazu, aber sie dachte an etwas, das Godric Gryffindor am Ende seiner sehr kurzen Autobiografie geschrieben hatte. Kurz und ohne jede Erklärung, denn die Schriftrolle war dazu bestimmt gewesen, von Hand kopiert zu werden, Jahrhunderte bevor die Muggel-Druckerpresse Zauberer dazu inspiriert hatte, die Flinke Feder zu erfinden.

\emph{\emph{Keinen Retter hat der Retter}, hatte Godric Gryffindor geschrieben. \emph{Keinen Herrn} \emph{hat derMeister, keine Mutter und kein Vater, nur das Nichts darüber.}}

Wenn \emph{das} der Preis dafür war, ein Held zu sein, war Hermine nicht sicher, ob sie ihn zahlen wollte. Oder vielleicht - obwohl es nicht das war, was sie gedacht hätte, bevor sie anfing, mit Harry herumzuhängen - vielleicht hatte Godric Gryffindor sich \emph{geirrt}.

„Vertraust du \emph{Dumbledore?}“ sagte Hermine. „Ich meine, er ist genau hier in unserer Schule und er ist der legendärste Held der ganzen Welt -"

„Er \emph{war} der legendärste Held“, sagte Harry. „Jetzt setzt er Hühner in Brand. Ganz ehrlich, kommt \emph{dir} Dumbledore verlässlich vor? „

Hermine antwortete nicht.

Seite an Seite begannen die beiden, eine riesige breite Wendeltreppe hinaufzusteigen, deren Stufen abwechselnd aus bronzenem Metall und blauem Stein bestanden; der letzte Weg zu dem Ort, an dem das Ravenclaw-Porträt darauf wartete, ihren Schlafsaal mit dummen Rätseln zu bewachen.

„Oh, und mir ist gerade etwas eingefallen, was ich dir sagen sollte“, sagte Harry, als sie etwa auf halbem Weg nach oben waren. „Da es dein Leben betrifft und so. Sieh es als eine Art Anzahlung -"

„Was ist es?“, fragte Hermine.

"Ich sage voraus, dass B. E.L. F.E. R. in den Ruhestand gehen wird."

"In den \emph{Ruhestand}? „, sagte Hermine und stolperte fast über eine der Stufen.

„Ja“, sagte Harry. „Ich meine, ich könnte mich irren, aber ich vermute, dass die Lehrer bald hart gegen Kämpfe in den Gängen vorgehen werden.“ Harry grinste, während er sprach, ein Glitzern in seinen Augen hinter der Brille deutete auf geheimes Wissen hin. „Neue Zaubersprüche einsetzen, um Angriffszauber aufzuspüren, oder damit beginnen, Berichte über Mobbing mit Veritaserum zu überprüfen - mir fallen da mehrere Möglichkeiten ein, wie sie das unterbinden könnten. Aber wenn ich recht habe, ist das ein Grund zum Feiern, Hermine, du und ihr alle. Ihr habt so viel öffentlichen Aufruhr verursacht, dass ihr sie dazu gebracht habt, tatsächlich etwas gegen das Mobbing zu \emph{unternehmen}. Das \emph{ganze} Mobbing.„

Langsam schlich sich ein Lächeln auf ihre Lippen, und als sie das obere Ende der Treppe erreichte und auf das Ravenclaw-Porträt zuging, um ihr Rätsel zu lösen, fühlte sich Hermine etwas leichter auf den Beinen, ein wunderbares, erhebendes Gefühl breitete sich in ihr aus, als hätte man sie mit Helium vollgepumpt.

Irgendwie hatte sie trotz all der Anstrengungen, die sie acht auf sich genommen hatten, nicht erwartet, dass es tatsächlich \emph{funktionieren} würde.

Sie hatten einen \emph{Unterschied} gemacht...

\hfill\break Es war das Ende der Frühstückszeit am nächsten Morgen.

Die Schüler aller Jahrgänge saßen ganz still in ihren Bänken, alle Köpfe in die gleiche Richtung gedreht, zum Haupttisch, vor dem ein einziges Erstklässler-Mädchen starr und regungslos stand, den Kopf nach hinten geneigt, um zum Hausoberhaupt Slytherins hinaufzustarren.

Professor Snapes Gesicht war vor Wut und Triumph verzerrt, rachsüchtig wie das Gemälde eines dunklen Zauberers; und hinter ihm saßen die anderen Professoren am Lehrertisch und sahen mit Gesichtern zu, die wie aus Stein gemeißelt waren.

“- dauerhaft aufgelöst“, spuckte der Meister der Zaubertränke. „Ihr selbsternanntes Bündnis ist in Hogwarts \emph{geächtet}, und zwar durch meine Entscheidung als Professor! Wenn Ihre Gesellschaft oder ein Mitglied von ihr noch einmal bei einer Schlägerei auf den Fluren entdeckt wird, Granger, werden Sie \emph{persönlich} dafür verantwortlich gemacht und von mir von der Hogwarts-Schule für Hexerei und Zauberei verwiesen!"

Die Erstklässlerin stand da, vor dem Lehrertisch, an den sie zuvor nur gerufen worden war, um Belobigungen und ein Lächeln zu erhalten; sie stand aufrecht da, die Krümmung ihrer Wirbelsäule sah wie der Bogen eines Zentauren aus und sie gab dem Feind nichts.

Diese Hexe im ersten Jahr stand da, alle Tränen und alle Wut weggeschlossen, ihr Gesicht unbewegt, nichts änderte sich an ihrer äußeren Erscheinung, während langsam etwas in ihr zerbrach, sie konnte fühlen, wie es brach.

Es zerbrach weiter, als Professor Snape ihr zwei Wochen Nachsitzen für das Verbrechen der Gewalttätigkeit in der Schule gab, mit dem verächtlichen Gesicht, das er ihnen allen am ersten Tag im Zaubertränkeunterricht gezeigt hatte, und mit einem schiefen Lächeln, das sagte, dass der Meister der Zaubertränke genau wusste, wie unfair er war.

Was-auch-immer-es-war in ihr, es knackte von oben bis unten, als Professor Snape Ravenclaw hundert Punkte abzog.

Dann war Schluss, und Snape sagte ihr, dass sie entlassen sei.

Sie drehte sich um und sah, dass am Ravenclaw-Tisch Harry Potter still auf seinem Platz saß, sie konnte seinen Gesichtsausdruck von hier aus nicht sehen, sie sah seine Fäuste auf dem Tisch, aber sie konnte nicht erkennen, ob sie weiß geballt waren wie ihre eigenen. Sie hatte ihm zugeflüstert, als Professor Snape sie gerufen hatte, dass er nichts tun solle, ohne vorher zu fragen.

Hermine drehte sich wieder um und blickte auf den Kopftisch, gerade als Snape sich von ihr abwandte, um seinen Platz wieder einzunehmen.

„Ich sagte, du kannst gehen, Mädchen“, sagte die höhnische Stimme, aber es lag ein zufriedenes Lächeln auf Snapes Gesicht, als ob er darauf wartete, dass sie etwas tat -

Hermine schritt weitere fünf Schritte auf den Lehrertisch zu und sagte mit brechender Stimme: „Schulleiter?"

Völlige Stille erfüllte die Große Halle.

Schulleiter Dumbledore sagte nichts, bewegte sich nicht. Es war, als wäre auch er gerade aus Stein gemeißelt worden.

Hermine wandte ihren Blick zu Professor Flitwick, dessen Kopf, kaum sichtbar über dem Tisch, in seinen Schoß zu starren schien. Neben ihm war das Gesicht von Professor Sprout sehr angespannt, sie schien sich zum Zuschauen zu zwingen, und ihre Lippen zitterten, aber sie sagte nichts.

Der Stuhl von Professor McGonagall war leer, die stellvertretende Schulleiterin war an diesem Morgen nicht zum Frühstück erschienen.

„Warum sagt keiner von Ihnen etwas?“, fragte Hermine Granger. Ihre Stimme zitterte vom letzten Rest ihrer Hoffnung, dem letzten verzweifelten Griff nach Hilfe von diesem Ort in ihrem Inneren. „Sie \emph{wissen}, dass das, was er tut, falsch ist!"

„Zwei weitere Wochen Nachsitzen, wegen Unverschämtheit“, sagte Snape seidenweich.

Es zerbrach.

Sie schaute noch ein paar Sekunden länger auf den Lehrertisch, auf Professor Flitwick und Professor Sprout und den leeren Platz, wo Professor McGonagall hätte sitzen sollen. Dann drehte sich Hermine Granger um und begann, auf den Ravenclaw-Tisch zuzugehen.

Ein Stimmengewirr setzte ein, als die Schüler sich von ihren Plätzen erhoben.

Und dann, als sie fast am Ravenclaw-Tisch war.

Die trockene Stimme von Professor Quirrell durchbrach alles, und diese Stimme sagte: „Hundert Punkte für Miss Granger dafür, dass sie tut, was richtig ist."

Hermine fiel fast über ihre eigenen Füße; und dann ging sie weiter, auch als Snape etwas Wütendes rief, auch als Professor Quirrell sich in seinem Stuhl zurücklehnte und zu lachen begann, auch als Dumbledores Stimme etwas sagte, das sie nicht verstand, und dann saß sie wieder am Ravenclaw-Tisch neben Harry Potter.

Harry Potter war wie erstarrt neben ihr, er sah aus wie jemand, der sich nicht zu bewegen wagte.

„Es ist in Ordnung“, sagte sie zu ihm, automatisch, ohne dass eine Wahl oder ein Gedanke im Spiel gewesen wären, obwohl es eigentlich gar nicht in Ordnung war. „Aber kannst du gucken, ob du mich aus Snapes Nachsitzen herausholen kannst, so wie du es damals selbst getan hast?„

Harry Potter nickte, eine einzige ruckartige Bewegung seines Kopfes. „Ich -“, sagte Harry. „Ich - es tut mir leid, das - das ist alles meine Schuld -"

„Mach dich nicht \emph{lächerlich}, Harry.“ Es war merkwürdig, wie ihre Stimme ganz normal herauskam, ohne dass sie darüber nachdachte, was sie sagen sollte. Hermine schaute auf ihren Frühstücksteller hinunter, aber an Essen war grade nicht zu denken, in ihrem Magen herrschte ein Aufruhr der darauf hindeutete, dass sie bereits kurz davor war, sich zu übergeben, was seltsam war, denn sie hätte schwören können, dass sich ihr ganzer Körper taub anfühlte, als würde sie gar nichts mehr fühlen, beides zur gleichen Zeit.

„Und“, sagte ihre Stimme, „wenn du gegen die Schulregeln verstoßen willst oder so, kannst du mich fragen, ich verspreche, dass ich nicht einfach nein sagen werde."

\hfill\break

Non est salvatori salvator,

neque defensori dominus,

nec pater nec mater,

nihil supernum.

- Godric Gryffindor,

1202 n. Chr.

