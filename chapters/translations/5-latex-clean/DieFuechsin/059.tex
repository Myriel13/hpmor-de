

\hypertarget{das-stanford-prison-experiment-teil-9-neugier}{% \section{27. Das Stanford-Prison-Experiment, Teil 9, Neugier}\label{das-stanford-prison-experiment-teil-9-neugier}}

—\/-\/-\/-\/- Kapitel 59: Das Stanford-Prison-Experiment, Teil 9, Neugier -\/-\/-\/-\/-

Besenstiele wurden in einer Zeit erfunden, die ein Muggel als das finstere Mittelalter bezeichnet hätte, vermutlich von einer legendären Hexe namens Celestria Relevo, angeblich der Ur-Urenkelin Merlins.

Celestria Relevo, oder welche Person oder Gruppe auch immer diese Verzauberungen wirklich erfunden hatte, hatte nicht die geringste Ahnung von Newtonscher Mechanik.

Besenstiele funktionierten daher nach der Aristotelischen Physik.

Sie flogen dorthin, wohin man sie zeigen ließ.

Wenn man sich geradeaus bewegen wollte, richtete man sie geradeaus; man sorgte sich nicht darum, einen Teil des Schubs nach unten zu lenken, um die Wirkung der Schwerkraft aufzuheben.

Wenn man einen Besenstiel drehte, verlagerte sich seine ganze Geschwindigkeit in die neue Richtung, in die er zeigte; sie blieb nicht seitwärts, basierend auf seinem alten Trägheitsmoment.

Besenstiele hatten maximale Geschwindigkeiten, nicht maximale Beschleunigungen. Nicht wegen irgendwas, das mit Luftwiderstand zu tun hatte, sondern weil ein Besenstiel einen maximalen aristotelischen Impuls hatte, den seine Verzauberungen ausüben konnten.

Harry hatte das nie zuvor ausdrücklich \emph{bemerkt}, obwohl er geschickt genug war, um die besten Noten in der Flugklasse zu erhalten. Besenstiele funktionierten so sehr, wie der menschliche Verstand \emph{instinktiv von ihnen erwartete}, dass sein Gehirn es geschafft hatte, ihre physikalische Absurdität völlig zu übersehen. Harry war an seinem ersten Donnerstag im Besenstielunterricht durch interessanter erscheinende Phänomene abgelenkt worden, auf Papier geschriebene Worte und einen rot leuchtenden Ball. Sein Gehirn hatte also einfach seinen Unglauben eingestellt, die Realität der Besenstiele als akzeptiert markiert und seinen Spaß gehabt, ohne auch nur \emph{einmal an die Frage} zu denken, deren Antwort offensichtlich gewesen wäre. Denn es ist eine traurige Tatsache, dass wir immer nur über einen winzigen Bruchteil all der Phänomene \emph{nachdenken}, denen wir begegnen…

Das ist die Geschichte, wie Harry James Potter-Evans-Verres durch seinen eigenen Mangel an Neugier fast getötet wurde.

Weil Raketen \emph{nicht} nach der aristotelischen Physik funktionierten.

Raketen funktionierten \emph{nicht} so, wie ein menschlicher Verstand instinktiv dachte, dass ein fliegendes Ding funktionieren sollte.

Ein raketengestützter Besenstiel bewegte sich daher \emph{nicht} wie die magischen Besenstiele, auf denen Harry ein sehr guter Flieger war.

Nichts davon ging Harry damals tatsächlich durch den Kopf.

Zum einen hinderte ihn das lauteste Geräusch, das er je in seinem Leben gehört hatte, daran, sich selbst denken zu hören.

Zum anderen bedeutete die Beschleunigung nach oben mit 4g, dass er insgesamt etwa zweieinhalb Sekunden Zeit hatte, um vom Boden von Askaban zum Dach zu gelangen.

Und selbst wenn es zweieinhalb der \emph{längsten} Sekunden in der Geschichte der Zeit waren, war das nicht genug, um viel nachzudenken.

Die Zeit reichte nur, um die Lichter der Flüche der Auroren auf ihn herabfliegen zu sehen, den Besenstiel leicht anzuwinkeln, um ihnen auszuweichen, zu erkennen, dass der Besenstiel einfach mit fast gleichem Trägheitsmoment weiterflog, anstatt in die Richtung zu fliegen, in die er ihn richtete, und wortlos die Begriffe zu denken

*Scheiße*

und

*Newton*

Daraufhin winkelte Harry den Besenstiel viel stärker an, und dann begannen sie sich sehr schnell der Wand zu nähern, so dass er ihn in die andere Richtung zurückwendete, und es kamen noch mehr Lichter herunter, und die Dementoren glitten zusammen mit einer Art riesiger geflügelter Kreatur aus weiß-goldenen Flammen geschmeidig auf sie zu, also riss Harry den Besenstiel zurück in Richtung Himmel, aber jetzt glitt er immer noch auf eine andere Wand zu, also kippte er den Besen leicht an und hörte auf, sich zu nähern, aber er war zu nah, also kippte er ihn wieder an, und dann waren die fernen Auroren auf ihren Besenstielen gar nicht mehr weit entfernt, und er würde in diese Frau krachen, also drehte er seinen Besenstiel direkt von ihr weg, und in einem weiteren Augenblick erkannte er, dass seine Rakete ein extrem starker Flammenwerfer war und im Bruchteil einer Sekunde direkt auf die Aurorin gerichtet war. Er drehte den Besenstiel seitwärts, während er weiter aufstieg, und er konnte sich nicht erinnern, ob der Besen jetzt auf einen Auror gerichtet war, aber zumindest war er nicht auf \emph{sie} gerichtet.

Harry verfehlte einen weiteren Auror um etwa einen Meter und schoss mit einem seitwärts gerichteten Flammenwerfer an ihm vorbei, der sich mit, wie Harry später vermutete, etwa 300 Stundenkilometern nach oben bewegte.

Wenn es Schreie von gerösteten Auroren gab, hörte er sie nicht, aber das war kein Beweis dafür oder dagegen, denn alles, was Harry in diesem Moment hörte, war ein extrem lautes Geräusch.

Ein paar \emph{ruhigere, wenn nicht sogar leisere} Sekunden später schien es keine Auroren oder Dementoren oder riesige geflügelte Flammenwesen um sie herum zu geben, und das riesige und schreckliche Bauwerk von Askaban sah aus dieser Höhe erstaunlich winzig aus.

Harry richtete den Besenstiel auf die Sonne, die durch die Wolken nur schwach sichtbar war. Zu dieser Tageszeit und in den Wintermonaten war sie nicht hoch am Himmel, und der Besenstiel beschleunigte für weitere zwei Sekunden in diese Richtung und nahm sehr schnell eine erstaunliche Geschwindigkeit auf, bevor die Feststoffrakete ausbrannte.

Danach, sobald Harry sich wieder denken hören konnte, als nur noch der heulende Fahrtwind ihrer lächerlich hohen Geschwindigkeit zu hören war, und Harrys von der Verzauberung unterstützte Finger, die den Besenstiel griffen, sich lediglich dem verlangsamenden Zug widersetzten, weil er sich weit schneller als die Endgeschwindigkeit bewegte, \emph{das} war der Zeitpunkt, an dem Harry tatsächlich all das Zeug über Newtonsche Mechanik und aristotelische Physik und Besenstiele und Raketentechnik und die Bedeutung der Neugier dachte und wie er nie wieder etwas so Gryffindorhaftes machen würde, zumindest nicht, bis er das Geheimnis der Unsterblichkeit des Dunklen Lords gelernt hatte, und \emph{warum} hatte er Professor Quirinus „\emph{Ich versichere Ihnen, dass ich das nicht versuchen würde, wenn ich nicht mit meinem Überleben rechnen würde}“ Quirrel zugehört anstelle von Professor Michael „Sohn, wenn du alleine irgendetwas mit Raketen versuchst, ich meine \emph{egal was}, ohne ausgebildete professionelle Beobachter, wirst du sterben, und das wird Mama traurig machen“ Verres-Evans.

„WAS?“, kreischte Amelia in den Spiegel.

Der Wind war auf ein erträgliches Maß abgeflaut, als der Luftwiderstand sie verlangsamte, was Harry reichlich Gelegenheit gab, dem summenden, klingenden Geräusch zu lauschen, das sein ganzes Gehirn zu füllen schien.

Professor Quirrell hatte den Raketenauspuff mit einem Stillezauber versehen sollen… anscheinend gab es Grenzen für das, was Stillezauber bewirken konnten… rückblickend hätte Harry ein Paar Ohrstöpsel verwandeln und nicht nur dem Stillezauber vertrauen sollen, obwohl das wahrscheinlich auch nicht gereicht hätte…

Nun, magische Heilung hatte wahrscheinlich etwas, um bleibende Hörschäden zu behandeln.

Nein, wirklich, magische Heilung hatte wahrscheinlich etwas, um das zu behandeln. Er hatte gesehen, wie Schüler mit Verletzungen zu Madam Pomfrey gingen, die viel schlimmer klangen…

\emph{\emph{Gibt es eine Möglichkeit, eine imaginäre Persönlichkeit in den Kopf eines anderen Menschen zu verpflanzen?} fragte Hufflepuff. \emph{Ich will nicht mehr in deinem leben.}}

Harry schob das alles für den Moment beiseite, er konnte im Moment wirklich nichts deswegen tun. Gab es etwas, worüber er sich Sorgen machen \emph{sollte}—

Dann warf Harry einen Blick hinter sich und erinnerte sich zum ersten Mal daran, zu prüfen, ob Bellatrix oder Professor Quirrell vom Besenstiel gepustet worden waren.

Aber die grüne Schlange steckte immer noch in ihrem Geschirr, und die abgemagerte Frau klammerte sich immer noch an den Besenstiel, ihr Gesicht war immer noch von ungesunder Farbe und ihre Augen immer noch hell und gefährlich. Ihre Schultern zitterten, als ob sie hysterisch lachen würde, und ihre Lippen bewegten sich, als ob sie schrie, aber es kam kein Ton heraus—

Ach ja, richtig.

Harry nahm die Kapuze seines Umhangs ab und klopfte sich auf die Ohren, um sie wissen zu lassen, dass er nicht hören konnte.

Daraufhin ergriff Bellatrix ihren Zauberstab, richtete ihn auf Harry, und plötzlich wurde das Klingeln in seinen Ohren leiser, er konnte sie hören.

Einen Augenblick später bedauerte er es; die Verwünschungen, die sie gegen Askaban, Dementoren, Auroren, Dumbledore, Lucius, Bartemius Crouch, etwas, das sich Orden des Phönix nannte, und allen, die sich ihrem Dunklen Herrn in den Weg stellten, usw., schrie, waren für jüngere und empfindlichere Zuhörer nicht geeignet; und ihr Lachen schmerzte in seinen neu geheilten Ohren.

„Genug, Bella“, sagte Harry schließlich, und ihre Stimme verstummte auf der Stelle.

Es gab eine Pause. Harry zog sich den Umhang wieder über den Kopf, nur aus Prinzip; und im selben Augenblick, in dem ihm klar wurde, dass sie dort unten Teleskope oder so etwas haben könnten, war es im Nachhinein betrachtet unglaublich dumm gewesen, auch nur für einen Moment die Kapuze herunterzuziehen, und er hoffte, dass die ganze Mission nicht wegen dieses einen Fehlers scheitern würde…

\emph{Wir sind nicht wirklich dafür geschaffen, oder?} beobachtete Slytherin.

\emph{Hey}, wandte Hufflepuff reflexartig ein, \emph{wir können nicht erwarten, dass wir beim ersten Mal alles perfekt machen, wir brauchen wahrscheinlich nur mehr Übung, VERGESST, DAS ICH DAS GESAGT HABE!}

Harry schaute sich noch einmal um, sah Bellatrix mit einem verwirrten, staunenden Gesichtsausdruck um sich blicken. Ihr Kopf drehte sich weiter und weiter.

Und schließlich sagte Bellatrix, ihre Stimme nun leiser, „Mein Herr, wo sind wir?“.

\emph{Was meinst du?} wollte Harry sagen, aber der Dunkle Lord würde niemals zugeben, dass er nichts verstand, also antwortete Harry trocken: „Wir sind auf einem Besenstiel.“

\emph{Glaubt sie, dass sie tot ist, dass dies der Himmel ist?}

Bellatrixs Hände waren immer noch an den Besenstiel gekettet, so dass nur ein Finger hochkam und zeigte, als sie sagte: „Was ist \emph{das}? “

Harry folgte der Richtung ihres Fingers und sah… eigentlich nichts Besonderes…

Dann wurde es Harry klar. Nachdem sie hoch genug hinaufgeflogen waren, hatte es keine Wolken mehr gegeben, die es verdeckt hätten.

„Das ist die Sonne, liebe Bella.“

Es kam bemerkenswert kontrolliert heraus, der Dunkle Lord klang vollkommen ruhig und vielleicht ein wenig ungeduldig mit ihr, selbst als die Tränen über Harrys Wangen flossen.

In der endlosen Kälte, in der pechschwarzen Dunkelheit, wäre die Sonne sicher…

Eine glückliche Erinnerung…

Bellatrix' Kopf drehte sich weiter.

„Und die flauschigen Dinger?“, sagte sie.

„Wolken.“

Es gab eine Pause, und dann sagte Bellatrix: „Aber was \emph{sind} sie?“

Harry antwortete ihr nicht, seine Stimme hätte auf keinen Fall ruhig sein können, alles, was er tun konnte, war seinen Atem vollkommen gleichmäßig zu halten, während er weinte.

Nach einer Weile atmete Bellatrix, so leise, dass Harry es fast nicht hörte: „Hübsch…“.

Ihr Gesicht entspannte sich langsam, die Farbe verließ ihre Blässe fast so schnell, wie sie gekommen war.

Ihr skelettartiger Körper sackte auf dem Besenstiel zusammen.

Der geliehene Zauberstab baumelte leblos an dem Riemen, der an ihrer unbeweglichen Hand befestigt war.

DAS SOLL DOCH WOHL EIN WITZ SEIN—

Harrys Verstand erinnerte sich dann, dass der Notfalltrank seinen Preis hatte; Bellatrix würde \emph{eine beträchtliche Zzeit lang sschlafen}, hatte Professor Quirrell gesagt.

Und im selben Augenblick war ein anderer Teil von Harry völlig überzeugt, als er auf die kreideweiße abgemagerte Frau zurückblickte, die im hellen Sonnenlicht toter schien als alles, was Harry je lebend gesehen hatte, dass sie tot war, dass sie gerade ihr letztes Wort gesprochen hatte, dass Professor Quirrell die Dosis falsch eingeschätzt hatte—

—oder absichtlich Bellatrix geopfert hatte, um ihre eigene Flucht zu sichern—

\emph{Atmet sie?}

Harry konnte nicht sehen, ob sie atmete.

Auf dem Besenstiel gab es keine Möglichkeit, nach hinten zu greifen und ihren Puls zu messen.

Harry schaute nach vorne, um sicherzugehen, dass sie nicht auf irgendwelche fliegenden Steine stießen, lenkte den Besenstiel weiter auf die Sonne zu, der unsichtbare Junge und die möglicherweise tote Frau, die in den Nachmittag hineinritten, während seine Finger das Holz so fest ergriffen, dass sie weiß wurden.

Er konnte nicht nach hinten greifen und Wiederbelebungsmaßnahmen durchführen.

Er konnte nichts aus seinem Heiler-Kit benutzen.

\emph{Vertraust du Professor Quirrell, dass er sie nicht in Gefahr gebracht hat?}

Seltsam, ja es war schon seltsam, dass selbst der aufrichtige Glaube, dass Professor Quirrell den Auror nicht hatte töten wollen (denn das wäre dumm gewesen), und der Gedanke an die Zusicherungen des Verteidigungsprofessors nicht mehr beruhigend wirkten.

Dann fiel Harry ein, was er noch überprüfen musste,—

Harry blickte zurück und zischte: „\emph{Lehrer?}“

Die Schlange rührte sich in ihrem Geschirr nicht und sagte kein Wort.

… vielleicht war die Schlange, da sie kein wirklicher Reiter war, nicht vor der Beschleunigung geschützt gewesen. Oder vielleicht hatte das Annähern an die Dementoren ohne Schild, auch nur für einen Moment in Animagus-Form, den Verteidigungsprofessor außer Gefecht gesetzt.

Das war nicht gut.

Es sollte Professor Quirrell sein, der Harry sagte, wann es sicher sei, den Portschlüssel zu benutzen.

Harry lenkte den Besenstiel mit weiß-verkrampften Fingern und dachte nach, er dachte sehr angestrengt nach für eine kleine, nicht bemessene Zeitspanne, während der Bellatrix vielleicht atmete oder nicht atmete, während der Professor Quirrell selbst vielleicht schon eine Weile nicht mehr geatmet hatte.

Und Harry entschied, dass es zwar möglich sei, sich von dem Fehler zu erholen, den in seinem Besitz befindlichen Portschlüssel zu verschwenden, dass es aber nicht möglich sei, sich von dem Fehler zu erholen, ein Gehirn zu lange ohne Sauerstoff zu lassen.

Also nahm Harry den nächsten Portschlüssel in der Sequenz aus seinem Beutel, während er seinen Besenstiel in der strahlend blauen Luft zum Stillstand brachte (Harry wusste nicht, ob die Fähigkeit eines Portschlüssels, sich an die Erdrotation anzupassen, auch die Fähigkeit einschloss, die Geschwindigkeit im Allgemeinen an die neue Umgebung anzupassen), berührte den Portschlüssel mit dem Besenstiel und…

Harry hielt inne und hielt immer noch den Zweig fest, dessen Gegenstück er vor gefühlten zwei Wochen gebrochen hatte. Er verspürte eine plötzliche Abneigung; sein Gehirn schien durch einen rein neuronalen Prozess der operanten Konditionierung die Regel gelernt zu haben, dass Zweige Zu Zerbrechen Eine Schlechte Idee War.

Aber das war eigentlich nicht logisch, also brach Harry den Zweig trotzdem durch.

Hinter der nahe gelegenen Metalltür gab es einen donnernden Knall, der Amelia veranlasste, den Spiegel, den sie in der Hand hielt, fallen zu lassen und sich mit dem Zauberstab in der Hand herumzudrehen, und dann brach die Tür auf und enthüllte Albus Dumbledore, der dort vor einem großen rauchenden Loch in der Gefängniswand stand.

„Amelia“, sagte der alte Zauberer. Von seiner gewohnten Leichtfertigkeit war keine Spur zu sehen, seine Augen waren hart wie Saphire unter seiner Halbmondbrille. „Ich muss Askaban verlassen, und zwar \emph{sofort}. Gibt es einen schnelleren Weg als einen Besenstiel, um über die Schutzzauber hinauszukommen?“

„Nein…“

„Dann brauche ich deinen schnellsten Besenstiel, und zwar sofort!“

Der Ort, an dem Amelia sein \emph{wollte}, war bei dem Auror, der von diesem Dämonsfeuer oder was auch immer das gewesen war, verletzt worden war.

Was sie tun \emph{musste}, war herauszufinden, was Dumbledore wusste.

„Ihr da!“ bellte die alte Hexe das Team um sie herum an. „Räumt die Korridore weiter frei, bis ihr unten seid, vielleicht sind noch nicht alle entkommen!“ Und dann zu dem alten Zauberer: „Zwei Besenstiele. Du kannst mich einweisen, sobald wir in der Luft sind.“

Es gab einen abschätzenden Blickkontakt, aber keinen langen.

Ein ekelerregend harter Ruck durchfuhr Harrys Unterleib, wesentlich härter als der Ruck, der ihn nach Askaban transportiert hatte, und diesmal war die zurückgelegte Entfernung groß genug, dass er sich einen Augenblick der Stille hören konnte, den unsichtbaren Raum zwischen den Räumen, in dem Spalt zwischen einem Ort und einem anderen, beobachten konnte.

Die Sonne, die nur kurz auf die beiden geschienen hatte, wurde schnell von einer Regenwolke verdeckt, als sie von Askaban weg schossen, in Windrichtung und schneller als der Wind.

„Wer steckt dahinter?“, rief Amelia zum Besenstiel der einen Schritt neben ihr flog.

„Eine von zwei Personen“, sagte Dumbledore zurück, „ich weiß in diesem Augenblick nicht, welche. Falls es der erstere, dann stecken wir in Schwierigkeiten. Wenn es der zweite ist, dann stecken wir alle in weit größeren Schwierigkeiten.“

Amelia erübrigte keinen Atemzug für Seufzer. „Wann wirst du es wissen?“

Die Stimme des alten Zauberers war grimmig, leise und erhob sich doch irgendwie über den Wind. „Drei Dinge brauchen sie für die Vollkommenheit, wenn es der eine ist: Das Fleisch des treuesten Dieners des Dunklen Lords, das Blut des größten Feindes des Dunklen Lords und den Zugang zu einem bestimmten Grab. Ich hatte Harry Potter für sicher gehalten, da ihr Angriff auf Askaban so gut wie gescheitert war - obwohl ich trotzdem Schutzzauber auf ihm platziert habe -, aber jetzt habe ich tatsächlich Angst. Sie haben Zugang zur Zeit, jemand mit einem Zeitumkehrer schickt ihnen Nachrichten; und ich vermute, dass der Entführungsversuch auf Harry Potter bereits vor einigen Stunden stattgefunden hat. Deshalb haben \emph{wir} nichts davon gehört, da wir uns in Askaban befinden, wo die Zeit sich nicht selbst verknoten kann. Diese Vergangenheit kam nach unserer eigenen Zukunft, verstehst du?“

„Und wenn es der andere ist?“, rief Amelia. Was sie gehört hatte, war schon beunruhigend genug; das klang wie das dunkelste aller dunklen Rituale und konzentrierte sich auf den toten Dunklen Lord selbst.

Der alte Zauberer, dessen Gesicht jetzt noch grimmiger war, sagte nichts, schüttelte nur den Kopf.

Als der Zug des Portschlüssels nachgelassen hatte, schaute die Sonne gerade erst über den Horizont und sah eher nach Morgendämmerung als nach Sonnenuntergang aus, als ihr Besenstiel tief über einer schmalen Fläche aus dunkelorangefarbenem Fels und Sand schwebte, die zu klumpigen Hügeln angeordnet waren, als hätte jemand das Landes wie Teig ein paar Mal geknetet und dann vergessen, es flach zu rollen. In der Nähe rollten die Wellen in der endlosen Weite des Wassers vorbei, obwohl der Boden, über dem der Besenstiel schwebte, höchstens einige Meter über dem Meeresspiegel lag.

Harry blinzelte wegen der Farben der Morgendämmerung, und dann wurde ihm klar, dass der Portschlüssel international gewesen war.

„Oy!“ kam ein forscher, weiblicher Ruf von hinten, und Harry drehte den Besenstiel, um nachzusehen. Eine Dame mittleren Alters hielt in einer scheinbar rufenden Geste eine Hand an den Mund und eilte vorwärts. Ihre freundlichen Gesichtszüge, ihre schmalen Augen und ihre dunkle Haut kennzeichneten ein Volk, das Harry nicht kannte; sie war in leuchtend purpurne Gewänder gekleidet, wie Harry sie noch nie zuvor gesehen hatte; und als sich ihre Lippen wieder öffneten, sprach sie mit einem Akzent, den Harry nicht zuordnen konnte, denn er war nicht weit gereist. „Wo wart ihr? Ihr kommt zwei Stunden zu spät! Ich hätte euch alle fast aufgegeben… Hallo?“

Es gab eine kurze Pause. Harrys Gedanken schienen sich seltsam zu bewegen, zu langsam, alles fühlte sich weit entfernt an, als wäre eine dicke Glasscheibe zwischen ihm und der Welt und eine weitere dicke Glasscheibe zwischen ihm und seinen Gefühlen, so dass er sehen, aber nicht berühren konnte. Es war über ihn gekommen, als er das Licht der Morgendämmerung und die freundliche Hexe sah und dachte, dass das alles wie ein richtiges Ende des Abenteuers schien.

Dann eilte die Hexe vorwärts und zog ihren Zauberstab; ein gemurmeltes Wort löste die Fesseln, die die abgemagerte Frau an den Besenstiel banden, und Bellatrix wurde auf den Sandstein hinunterschweben gelassen, wobei ihre skelettartigen Arme und bleichen Beine wie leblos baumelten. „Oh, Merlin“, flüsterte die Hexe, „Merlin, Merlin, Merlin…“

\emph{Sie wirkt besorgt}, dachte ein abstraktes, weit entferntes Ding zwischen zwei Glasscheiben. \emph{Ist es das, was ein echter Heiler sagen würde, oder ist es das, was jemand sagen würde, der eine Vorstellung abliefern sollte?}

Als wäre es nicht Harry, der sprach, sondern ein anderer Teil von ihm hinter einer weiteren Glasscheibe, kam ein Flüstern von seinen Lippen. „Die grüne Schlange auf ihrem Rücken ist ein Animagus.“ Nicht hoch das Flüstern, nicht kalt, nur leise. „Er ist bewusstlos.“

Der Kopf der Hexe zuckte hoch, um zu sehen, wo diese Stimme aus der leeren Luft zu sprechen schien, und blickte dann wieder zu Bellatrix hinunter. „Du bist nicht Herr Jaffe.“

„Das wäre der Animagus“, flüsterten Harrys Lippen. \emph{Oh}, dachte der Harry hinter Glas und lauschte dem Klang seiner eigenen Lippen, \emph{das ergibt Sinn; Professor Quirrell muss einen anderen Namen verwendet haben.}

„Seit wann ist \emph{er} ein - bah, vergiss es.“ Die Hexe legte ihren Zauberstab einen Moment lang auf die Nase der Schlange und schüttelte dann scharf den Kopf. „Ihm fehlt nichts, was ein Tag Ruhe nicht heilen könnte. \emph{Ihr}…“

„Können Sie ihn jetzt aufwecken?“, flüsterte Harry's Lippen. \emph{Ist das eine gute Idee?} dachte Harry, aber seine Lippen schienen definitiv so zu denken.

Wieder das scharfe Kopfschütteln. „Wenn ein Rennervate bei ihm nicht gewirkt hat—“ begann die Hexe.

„Ich habe es nicht versucht“, flüsterte Harrys Lippen.

„Was? Warum - oh, egal. \emph{Rennervate}.“

Es gab eine Pause, und dann kroch langsam eine Schlange aus ihrem Geschirr. Langsam tauchte der grüne Kopf auf, sah sich um.

Ein kurzes Verschwimmen später stand Professor Quirrell vor ihnen, und einen Moment später ging er in die Knie.

„Leg dich hin“, sagte die Hexe, ohne von Bellatrix aufzublicken. „Bist du das da drin, Jeremy?“

„Ja“, sagte der Verteidigungsprofessor ziemlich heiser, als er sich vorsichtig auf einem relativ flachen Fleck orangefarbenen Sandsteins niederlegte. Er war nicht so blass wie Bellatrix, aber sein Gesicht war im dämmrigen Morgenlicht blutleer. „Ich grüße Sie, Miss~Camblebunker.“

„Ich habe Ihnen gesagt“, sagte die Hexe mit Schärfe in der Stimme und mit einem leichten Lächeln im Gesicht, „nennen Sie mich Crystal, dies ist nicht Großbritannien, und wir wollen hier keine Ihrer Formalitäten. \emph{Und} es heißt jetzt Doktor, nicht Miss.“

„Entschuldigen Sie, Dr~Camblebunker.“ Es folgte ein trockenes Glucksen.

Das Lächeln der Hexe wurde ein wenig breiter, ihre Stimme so viel schärfer. „Wer ist dein Freund?“

„Das brauchst du nicht zu wissen.“ Die Augen des Verteidigungsprofessors waren geschlossen, als er auf dem Boden lag.

„Wie verkehrt ist es gelaufen?“

Wirklich sehr trocken: „Du kannst morgen in jeder Zeitung mit internationalem Feuilleton darüber lesen.“

Der Zauberstab der Hexe klopfte hier und da, stocherte und stupste am ganzen Körper von Bellatrix herum. „Du hast mir gefehlt, Jeremy.“

„Wirklich?“, sagte der Verteidigungsprofessor und klang leicht überrascht.

„Nicht mal ein kleines bisschen. Wenn ich dir nichts schulden würde…“

Der Verteidigungsprofessor fing an zu lachen, und dann wurde es eher zu einem Hustenanfall.

\emph{Was denkst du?} sagte Slytherin zum Inneren Kritiker, während Harry hinter den Glaswänden zuhörte. \emph{Schauspiel oder Realität?}

\emph{Kann ich nicht sagen}, sagte Harrys Innerer Kritiker. \emph{Ich bin im Moment nicht in bester Verfassung als Kritiker.}

\emph{Fällt jemandem eine gute Probe ein, um mehr Informationen zu sammeln?} sagte Ravenclaw.

Wieder dieses Flüstern aus der leeren Luft über dem Besenstiel: „Wie groß ist die Chance, all das ungeschehen zu machen, was man ihr angetan hat?“

„Oh, lass mal sehen. Legilimentik und unbekannte dunkle Rituale, zehn Jahre, bis das in Kraft tritt, gefolgt von zehn Jahren Dementor-Exposition? \emph{Das} rückgängig machen? Sie haben den Verstand verloren, Mister Wer-auch-immer. Die Frage ist, ob noch etwas \emph{übrig} ist, und ich würde sagen, dass die Chance dafür vielleicht eins zu drei ist—“ Die Hexe biss sich plötzlich auf die Zunge. Ihre Stimme war leiser, als sie wieder sprach. „Wenn du ihr Freund warst, bevor… dann nein, du wirst sie nie wieder zurückbekommen. Am besten verstehst du das jetzt.“

\emph{Ich stimme dafür, dass dies eine Vorführung ist}, sagte der innere Kritiker. \emph{Sie würde nicht einfach so auf eine Frage hin mit all dem herausplatzen, es sei denn, sie suchte nach einer Gelegenheit.}

\emph{Zur Kenntnis genommen, aber ich setze wenig Vertrauen darauf}, sagte Ravenclaw. \emph{Es ist sehr schwer, sich nicht von seinen Verdächtigungen leiten zu lassen, wenn man versucht, so subtile Beweise abzuwägen.}

„Welchen Trank hast du ihr gegeben?“, sagte die Hexe, nachdem sie Bellatrix' Mund geöffnet und hineingeblickt hatte, wobei ihr Zauberstab in verschiedenen Farben aufblitzte.

Der Mann, der ruhig auf dem Boden lag, sagte: „Notfall—“

„\emph{Bist du verrückt geworden?} “

Wieder das hustende Lachen.

„Wenn sie Glück hat, schläft sie eine Woche lang“, sagte die Hexe und schnalzte mit der Zunge. „Ich werde dir eine Eule schicken, wenn sie ihre Augen öffnet, ich nehme an, damit du zurückkommen und sie zu diesem Unbrechbaren Schwur überreden kannst. Hast du etwas, um sie davon abzuhalten, mich auf der Stelle zu töten, falls sie es überhaupt schafft, sich im nächsten Monat zu bewegen?“

Der Verteidigungsprofessor nahm, noch immer mit geschlossenen Augen, ein Blatt Papier aus seiner Robe; einen Augenblick später begannen darauf Worte zu erscheinen, begleitet von winzigen Rauchfahnen. Als kein Rauch mehr aufstieg, schwebte das Papier auf die Frau zu.

Die Frau blickte mit hochgezogenen Augenbrauen über das Papier und gab ein sardonisches Schnauben von sich. „Das sollte besser funktionieren, Jeremy, oder mein letzter Wille und Testament besagt, dass mein ganzer Besitz dazu dient, ein Kopfgeld auf dich auszusetzen. Wo wir gerade davon sprechen—“

Der Verteidigungsprofessor griff erneut in seine Robe und warf der Hexe einen Beutel zu, der ein klimperndes Geräusch machte. Die Hexe fing ihn auf, wog ihn und machte ein zufriedenes Geräusch.

Dann stand sie auf, und die bleiche Skelettfrau schwebte neben ihr vom Boden empor. „Ich gehe zurück“, sagte die Hexe. „Ich kann meine Arbeit hier nicht beginnen.“

„Warte“, sagte der Verteidigungsprofessor und holte mit einer Geste seinen Zauberstab aus Bellatrix' Hand und Geschirr. Dann richtete seine Hand den Zauberstab auf Bellatrix und bewegte sich in einer kleinen kreisförmigen Geste, begleitet von einem leisen „\emph{Obliviate}“.

„\emph{Das reicht}“, schnappte die Hexe, „\emph{Ich bringe sie hier raus, bevor ihr noch mehr Schaden zugefügt wird}—“ Ein Arm legte sich um die knochige Form von Bellatrix Black und drückte sie an ihre Seite, und beide verschwanden mit dem lauten POP! des Apparierens.

Und es herrschte Stille an diesem klumpigen Ort, außer dem sanften Rauschen der vorbeiziehenden Wellen und einem kleinen Windhauch.

\emph{\emph{Ich glaube, die Vorstellung ist beendet}, sagte der Innere Kritiker. I\emph{ch gebe ihr zweieinhalb von fünf Sternen. Sie ist wahrscheinlich keine sehr erfahrene Schauspielerin.}}

\emph{Ich frage mich, ob ein echter Heiler eher unecht wirkt als ein Schauspieler, der einen spielen soll?} sinnierte Ravenclaw.

Es fühlte sich an wie eine Fernsehsendung, in deren Figuren man sich nicht besonders einfühlte, das war alles, was man von hinter den Glaswänden sehen und fühlen konnte.

Irgendwie schaffte es Harry, selbst seine Lippen zu bewegen, seine eigene Stimme in die stille Dämmerungsluft hinauszuschicken, und war dann überrascht, seine eigene Frage zu hören. „Wie viele verschiedene Menschen sind Sie eigentlich?“

Der bleiche Mann, der auf dem Boden lag, lachte nicht, aber aus dem Besenstiel sahen Harrys Augen, wie sich die Seiten von Professor Quirrells Lippen kräuselten, der Rand dieses vertrauten sardonischen Lächelns. „Ich kann nicht sagen, dass ich mir die Mühe gemacht hätte, mitzuzählen. Wie viele sind Sie?“

Es hätte den inneren Harry nicht so sehr erschüttern dürfen, als er diese Reaktion hörte, und doch fühlte er sich - er fühlte sich - instabil, als wäre sein eigenes Zentrum abgezogen worden—

Oh.

„Entschuldigung“, sagte Harrys Stimme. Sie klang nun so distanziert und losgelöst, wie der verblassende Harry sie fühlte. „Ich glaube, ich werde in ein paar Sekunden ohnmächtig.“

„Benutzen Sie den vierten Portschlüssel, den ich Ihnen gegeben habe, den, von dem ich sagte, er sei unsere Zufluchtsstätte“, sagte der am Boden liegende Mann ruhig, aber schnell. „Dort wird es sicherer sein. Und tragen Sie weiterhin Ihren Umhang.“

Harrys freie Hand holte einen weiteren Zweig aus seinem Beutel und zerbrach ihn.

Es gab noch einen weiteren Portschlüssel-Ruck, international lang, und dann war er irgendwo wo es schwarz war.

„\emph{Lumos}“, sagten Harrys Lippen, wobei ein Teil von ihm auf die Sicherheit des Ganzen achtete.

Er befand sich in etwas, das wie ein Muggellager aussah, ein verlassenes.

Harrys Beine kletterten vom Besenstiel herunter und legten sich auf den Boden. Seine Augen schlossen sich, und irgendein ordentlicher Teil von ihm ließ sein Licht versiegen, bevor die Dunkelheit ihn umfasste.

„Wo willst du hin?“, schrie Amelia. Sie waren fast am Rande der Schutzzauber.

„Rückwärts in der Zeit, um Harry Potter zu beschützen“, sagte der alte Zauberer, und noch bevor Amelia ihre Lippen öffnen konnte, um ihn zu fragen, ob er Hilfe wolle, fühlte sie die Grenze der Schutzzauber, als sie sie überquerten.

Es gab einen Apparationsknall, und der Zauberer und der Phönix verschwanden und ließen den geliehenen Besenstiel zurück.

