

\hypertarget{vorgeben-weise-zu-sein-teil-2}{% \section{8. Vorgeben weise zu sein, Teil 2}\label{vorgeben-weise-zu-sein-teil-2}}

—\/-\/-\/-\/- Kapitel 40: Vorgeben weise zu sein, Teil 2 -\/-\/-\/-\/-

Harry, der die Teetasse auf genau die korrekte Art und Weise hielt, die Professor Quirrell ihm dreimal hatte vorführen müssen, nahm einen kleinen, vorsichtigen Schluck. Auf der anderen Seite des langen, breiten Tisches, der das Herzstück von Marys Stube war, nahm Professor Quirrell einen Schluck aus seiner eigenen Tasse und ließ es dabei viel natürlicher und eleganter aussehen. Den Namen des Tees selbst hatte Harry nicht einmal aussprechen können oder zumindest hatte Professor Quirrell ihn jedes Mal korrigiert, wenn Harry versucht hatte, die chinesischen Worte zu wiederholen, bis er schließlich aufgegeben hatte.

Harry hatte es beim letzten Mal geschafft, einen Blick auf die Rechnung zu werfen, und Professor Quirrell hatte ihn damit davonkommen lassen.

Vorher hatte er den Drang verspürt, zuerst einen Poin-Tee zu trinken.

\emph{Selbst unter Berücksichtigung dieser Tatsache} war Harry immer noch schockiert gewesen.

Und es schmeckte immer noch wie, nunja, Tee.

Harry hatte den leisen Verdacht, dass Professor Quirrell das \emph{wusste} und absichtlich lächerlich teuren Tee kaufte, den Harry nicht wertschätzen konnte, \emph{nur um ihn zu ärgern}. Professor Quirrell \emph{selbst} mochte den Tee vielleicht nicht mal. Vielleicht mochte \emph{niemand} diesen Tee wirklich und sein einziger Zweck war, lächerlich teuer zu sein und das Opfer sich undankbar fühlen zu lassen. Es könnte vielleicht auch nur ganz gewöhnlicher Tee sein, nur dass man in einem bestimmten Code nach ihm fragte und sie einen gigantischen falschen Preis auf die Rechnung setzten…

Professor Quirrells Gesichtsausdruck war abgehärmt und nachdenklich. „Nein“, sagte Professor Quirrell, „Sie hätten dem Schulleiter \emph{nicht} von Ihrem Gespräch mit Lord Malfoy erzählen sollen. Bitte versuchen Sie das nächste Mal schneller zu denken, Mr~Potter.“

„Es tut mir leid, Professor Quirrell“, sagte Harry kleinlaut. „Ich verstehe es immer noch nicht.“ Es gab Zeiten, in denen Harry sich wie ein Betrüger fühlte, der in der Anwesenheit von Professor Quirrell nur vorgab gerissen zu sein.

„Lord Malfoy ist Albus Dumbledores Gegner“, sagte Professor Quirrell. „Zumindest momentan. Ganz Großbritannien ist ihr Schachbrett, alle Zauberer sind ihre Figuren. Bedenken Sie: Lord Malfoy drohte, alles wegzuwerfen, sein Spiel aufzugeben, um sich an dir zu rächen, wenn Mr~Malfoy verletzt würde. In diesem Fall, Mr~Potter…?“

Es dauerte noch einige lange Sekunden, bis Harry es verstand, aber es war klar, dass Professor Quirrell keine weiteren Hinweise geben würde. Nicht, dass Harry welche wollte.

Dann stellte sein Verstand schließlich die Verbindung her und er runzelte die Stirn. „Tötet Dumbledore Draco, lässt es so aussehen, als hätte \emph{ich} es getan, woraufhin Lucius sein Spiel gegen Dumbledore hinwirft, um mich zu erledigen? Das… hört sich nicht nach dem \emph{Stil} des Schulleiters an, Professor Quirrell…“ In Harrys Verstand blitzte eine ganz ähnliche Warnung von Draco auf, auf die Harry dasselbe geantwortet hatte.

Professor Quirrell zuckte mit den Achseln und trank einen Schluck seines Tees.

Harry nippte an seinen eigenen Tee und blieb schweigend sitzen. Die Tischdecke, die sich über den Tisch ausbreitete, war mit einem sehr ruhigen Muster bedeckt, so dass sie anfangs wie einfacher Stoff wirkte, aber wenn man sie lange genug anstarrte oder lange genug schwieg, begann man, fein gearbeitete Blumen zu erkennen, die darauf schimmerten; die Vorhänge des Raumes hatten ihr Muster entsprechend geändert und schienen wie durch einen lautlosen Luftzug zu schillern. Professor Quirrell war an diesem Samstag in einer kontemplativen Stimmung, genau wie Harry, und Marys Stube hatte das wie es schien nicht versäumt zu bemerken.

„Professor Quirrell“, sagte Harry plötzlich, „gibt es ein Leben nach dem Tod?“

Harry hatte die Frage sorgfältig formuliert. Nicht \emph{glauben Sie an ein Leben nach dem} \emph{Tod?}, sondern einfach \emph{gibt es ein Leben nach dem Tod?} Was die Leute \emph{fest} glaubten, kam ihnen überhaupt nicht wie \emph{Glaube} vor. Die Leute sagten nicht: „Ich glaube fest daran, dass der Himmel blau ist!“ Sie sagten einfach: „Der Himmel ist blau“. Die wahre innere Weltvorstellung fühlte sich für einen schlichtweg so an wie die Welt \emph{war} …

Der Verteidigungsprofessor hob seine Tasse wieder an seine Lippen, bevor er antwortete. Sein Gesicht wirkte nachdenklich. „Wenn ja, Mr~Potter“, sagte Professor Quirrell, „dann haben etliche Zauberer sehr viel Mühe für die Suche nach der Unsterblichkeit verschwendet.“

„Das nicht wirklich eine Antwort“, bemerkte Harry. Er hatte inzwischen gelernt, auf so etwas aufmerksam zu werden, wenn er mit Professor Quirrell sprach.

Professor Quirrell stellte seine Teetasse mit einem leisen, hohen Klickgeräusch auf seiner Untertasse ab. „Einige dieser Zauberer waren ziemlich intelligent, Mr~Potter, also können Sie davon ausgehen, dass die Existenz eines Jenseits nicht offensichtlich ist. Ich habe mich zu dem Thema selbst informiert. Viele Behauptungen dieser Art klingen so, als wären sie aus Hoffnung oder Angst entstanden. Unter den Berichten, deren Wahrheitsgehalt nicht im Zweifel steht, gibt es nichts, was nicht das Ergebnis bloßer Zauberei sein könnte. Es gibt bestimmte Apparaturen, von denen behauptet wird, dass man mit ihnen mit den Toten kommunizieren kann, aber diese projizieren, so vermute ich zumindest, nur ein Bild aus dem Kopf des Nutzers; das Ergebnis scheint sich nicht von der Erinnerung zu unterscheiden, weil es die Erinnerung \emph{ist}. Die angeblichen Geister enthüllen keine Geheimnisse, die sie im Leben kannten, noch nach dem Tod hätten lernen können, die dem Nutzer nicht bekannt sind—“

„Darum also ist der Stein der Auferstehung nicht das wertvollste magische Artefakt der Welt“, sagte Harry.

„Ganz genau“, sagte Professor Quirrell, „obwohl ich zu einer Chance ihn zu erproben nicht nein sagen würde.“ Es war ein trockenes, dünnes Lächeln auf seinen Lippen; und etwas Kälteres, entfernteres in seinen Augen. „Sie haben auch darüber mit Dumbledore gesprochen, nehme ich an.“

Harry nickte.

Die Vorhänge nahmen ein leicht blaues Muster an und eine schwache Musterung aus kunstvollen Schneeflocken schien nun auf der Tischdecke sichtbar zu werden. Professor Quirrells Stimme klang sehr ruhig. „Der Schulleiter kann sehr überzeugend sein, Mr~Potter. Ich hoffe, er hat Ihnen nichts eingeredet.“

„Nein, Verdammt“, sagte Harry. „Er hat mich keine Sekunde lang getäuscht.“

„Das will ich auch hoffen“, sagte Professor Quirrell, noch immer in diesem sehr ruhigen Ton. „Ich wäre sehr verärgert, wenn ich herausfinden würde, dass der Schulleiter Sie davon überzeugt hätte, Ihr Leben wegen eines dummen Plans wegzuwerfen, indem er Ihnen sagte, dass der Tod das nächste große Abenteuer ist.“

„Ich glaube nicht, dass der Schulleiter es selbst geglaubt hat“, sagte Harry. Er nippte wieder an seinem eigenen Tee. „Er fragte mich, was ich mit der Ewigkeit anfangen wollen könnte, gab die üblichen Kommentare darüber ab, dass es langweilig, und er schien kein Problem dazwischen und seiner eigenen Aussage, eine unsterbliche Seele zu besitzen, zu sehen. Tatsächlich hielt er mir einen ewig langen Vortrag darüber, wie schrecklich es war, Unsterblichkeit zu wollen, bevor er behauptete, eine unsterbliche Seele zu haben. Ich kann mir nicht wirklich vorstellen, was in seinem Kopf vor sich gegangen sein muss, aber ich glaube nicht, dass er \emph{tatsächlich} eine Vorstellung davon hatte, wie er im Jenseits für immer weiterexistieren würde…“

Die Temperatur des Raumes schien zu sinken.

„Sie nehmen wahr“, sagte eine Stimme wie Eis vom anderen Ende des Tisches, „dass Dumbledore nicht wirklich glaubt, während er spricht. Es ist nicht so, dass er seine Prinzipien in Frage gestellt hat. Es ist so, dass er sie von Anfang an nicht hatte. Werden Sie jetzt schon zynisch, Mr~Potter?“

Harry hatte seine Augen auf seine Teetasse gerichtet. „Ein wenig“, sagte Harry zu seinem möglicherweise ultrahochwertigen, vielleicht-lächerlich-teuren chinesischen Tee. „Ich werde sicherlich ein wenig \emph{frustriert} über… was auch immer in den Köpfen der Leute schief geht.“

„Ja“, sagte diese eisige Stimme. „Ich finde es auch frustrierend.“

„Gibt es eine Möglichkeit, die Leute dazu zu bringen, das nicht zu tun“, sagte Harry zu seiner Teetasse.

„Es gibt in der Tat einen gewissen nützlichen Zauber, der das Problem löst.“

Harry blickte daraufhin hoffnungsvoll auf und sah ein kaltes, kaltes Lächeln auf dem Gesicht des Verteidigungsprofessors.

Dann verstand Harry es. „Ich meine, \emph{außer} Avada Kedavra.“

Der Verteidigungsprofessor lachte. Harry nicht.

„Wie auch immer“, sagte Harry hastig, „Ich \emph{habe} schnell genug nachgedacht, um nicht die offensichtliche Idee über den Auferstehungsstein vor Dumbledore vorzuschlagen. Haben Sie jemals einen Stein mit einer Linie gesehen, in einem Kreis, in einem Dreieck?“

Die tödliche Kälte schien sich zurückzuziehen, sich in sich selbst zu falten, als der gewöhnliche Professor Quirrell zurückkehrte. „Nicht, dass ich mich erinnern könnte“, sagte Professor Quirrell nach einer Weile, ein nachdenkliches Stirnrunzeln auf seinem Gesicht. „Das ist der Auferstehungsstein?“

Harry stellte seine Teetasse beiseite, dann zog er auf seiner Untertasse das Symbol nach, das er auf der Innenseite seines Umhangs gesehen hatte. Und bevor Harry seinen eigenen Zauberstab herausnehmen konnte, um den Schwebezauber zu wirken, schwebte die Untertasse zuvorkommend über den Tisch zu Professor Quirrell. Harry wollte wirklich dieses stablose Zeug lernen, aber das war anscheinend weit über seinem aktuellen Lehrplan.

Professor Quirrell studierte einen Moment lang Harrys Teetasse, schüttelte dann den Kopf; und einen Moment später schwebte die Untertasse zurück zu Harry.

Harry stellte seine Teetasse wieder auf die Untertasse und bemerkte abwesend, wie er es tat, so dass das Symbol, das er gezeichnet hatte, verschwunden war. „Wenn Sie zufällig einen Stein mit diesem Symbol sehen“, sagte Harry, „und er \emph{wirklich} mit dem Jenseits spricht, lassen Sie es mich wissen. Ich habe ein paar Fragen an Merlin oder jeden, der in Atlantis war.“

„Durchaus“, sagte Professor Quirrell. Dann hob der Verteidigungsprofessor seine Teetasse wieder hoch und kippte sie zurück, als ob er das letzte was übrig war, beenden wollte. „Übrigens, Mr~Potter, ich fürchte, wir müssen den heutigen Besuch in der Winkelgasse verkürzen. Ich hatte gehofft, dass es - aber egal. Lassen wir es dabei, dass ich heute Nachmittag noch etwas anderes tun muss.“

Harry nickte und trank seinen eigenen Tee, dann stand er zur gleichen Zeit wie Professor Quirrell von seinem Platz auf.

„Eine letzte Frage“, sagte Harry, als Professor Quirrells Mantel sich von der Garderobe hob und auf den Verteidigungsprofessor zusteuerte. „Magie ist frei in der Welt, und ich vertraue meinen Vermutungen nicht mehr so sehr wie früher. Also glauben \emph{Sie} nach bestem Wissen und ohne Wunschdenken, dass es ein Leben nach dem Tod gibt?“

„Wenn ich das täte, Mr~Potter“, sagte Professor Quirrell, als er seinen Mantel anzog, „wäre ich dann immer noch \emph{hier}?“

