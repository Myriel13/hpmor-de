

\hypertarget{das-stanford-prison-experiment-teil-3}{% \section{21. Das Stanford-Prison-Experiment, Teil 3}\label{das-stanford-prison-experiment-teil-3}}

—\/-\/-\/-\/- Kapitel 53: Das Stanford-Prison-Experiment, Teil 3 -\/-\/-\/-\/-

Der Leichnam einer Frau öffnete die Augen, und die dumpfen versunkenen Kugeln blickten ins Leere.

„Verrückt“, murmelte Bellatrix mit kratziger Stimme, „es scheint, dass die kleine Bella verrückt wird…“

Professor Quirrell hatte Harry ruhig und präzise instruiert, wie er sich in Bellatrix' Gegenwart verhalten sollte; wie er die Heuchelei formen sollte, die er in seinem Kopf aufrechterhalten würde.

\emph{Sie fanden es zweckmäßig, oder vielleicht auch nur amüsant, Bellatrix dazu zu bringen, sich in Sie zu verlieben, um sie an Ihren Dienst zu binden.}

Professor Quirrell hatte gesagt, dass diese Liebe durch Askaban hindurch angehalten hätte, denn für Bellatrix wäre es kein glücklicher Gedanke gewesen.

\emph{Sie liebt Sie von ganzem Herzen, mit ihrem ganzen Wesen. Sie erwidern ihre Liebe nicht, sondern halten sie für nützlich. Das weiß sie.}

\emph{Sie war die tödlichste Waffe, die Sie besaßen, und Sie nannten sie Ihre liebe Bella.}

Harry erinnerte sich daran von der Nacht, als der Dunkle Lord seine Eltern tötete: die kalte Belustigung, das verächtliche Lachen, diese hohe Stimme voll tödlichen Hasses. Es schien überhaupt nicht schwer zu erraten, was der Dunkle Lord sagen würde.

„Ich hoffe, du bist \emph{nicht} verrückt, liebe Bella“, sagte das kalte Flüstern. „Verrückt ist nicht nützlich.“

Bellatrix' Augen flackerten, sie versuchte, sich auf die leere Luft zu focussieren.

„Mein…Herr… Ich habe auf Euch gewartet, aber Ihr seid nicht gekommen… Ich habe Euch gesucht, aber ich konnte Euch nicht finden… Ihr lebt…“ Alle ihre Worte kamen in einem leisen Murmeln heraus, ob darin Gefühle enthalten waren, konnte Harry nicht sagen.

„\emph{Zeig ihr dein Gesicht}“, zischte die Schlange zu Harrys Füßen.

Harry warf die Kapuze des Unsichtbarkeitsumhangs zurück.

Der Teil von ihm, dem Harry die Kontrolle über seinen Gesichtsausdruck übertragen hatte, sah Bella ohne die geringste Spur von Mitleid an, nur mit kühlem, ruhigem Interesse. (Während Harry in seinem Innersten dachte, \emph{Ich werde dich retten, ich werde dich retten, egal was passiert}…)

„Die Narbe…“ murmelte Bellatrix. „Dieses Kind…“

„Das denken sie alle immer noch“, sagte Harrys Stimme und kicherte ein dünnes, kleines Glucksen. „Du hast mich am falschen Ort gesucht, Bella, meine Liebe.“

(Harry hatte gefragt, warum Professor Quirrell nicht derjenige sein könne, der die Rolle des Dunklen Lords spielte, und Professor Quirrell hatte darauf hingewiesen, dass es keinen plausiblen Grund dafür gebe, dass \emph{er} vom Schatten von Ihm, dessen Name nicht genannt werden darf, besessen sein sollte).

Bellatrix' Augen blieben auf Harry gerichtet, sie sagte kein Wort.

„\emph{Ssagetwass} \emph{in} \emph{Parssel}“, zischte die Schlange.

Harrys Gesicht wandte sich der Schlange zu, um deutlich zu machen, dass er sie ansprach, und zischte: „\emph{Eins zwei drei vier fünf} \emph{ssechssssieben} \emph{acht neun zehn}.“

Es gab eine Pause.

„Diejenigen, die die Dunkelheit nicht fürchten…“, murmelte Bellatrix.

Die Schlange zischte: „\emph{Werden von ihr} \emph{verzzzehrt} \emph{werden.}“

„Werden von ihr verzehrt werden“, flüsterte die kalte Stimme. Harry wollte nicht besonders darüber nachdenken, wie Professor Quirrell an das Passwort gekommen war. Sein Gehirn, das ohnehin schon darüber nachdachte, vermutete, dass es sich wahrscheinlich um einen Todesser, einen ruhigen, isolierten Ort und irgendeine totsichere Legilimentik gehandelt hatte.

„Euer Zauberstab“, murmelte Bellatrix, „ich nahm ihn aus dem Haus der Potters und versteckte ihn, Mein Herr… unter dem Grabstein rechts vom Grab Eures Vaters… werdet Ihr mich jetzt töten, wenn das alles war, was Ihr von mir gewünscht habt… Ich glaube, ich habe mir immer gewünscht, dass Ihr derjenige seid, der mich tötet… aber ich kann mich jetzt nicht mehr erinnern, es muss ein glücklicher Gedanke gewesen sein…“

Harrys Herz zerriss in ihm, es war unerträglich, und - und er konnte nicht weinen, konnte seinen Patronus nicht verblassen lassen—

In Harrys Gesicht flackerte Verärgerung auf, und seine Stimme war scharf, als er sagte: „Genug Dummheiten. Du sollst mit mir kommen, Bella, Liebes, es sei denn, du ziehst die Gesellschaft der Dementoren vor“.

Bellatrix' Gesicht zuckte kurz verwirrt zusammen, die geschrumpften Glieder rührten sich nicht.

„\emph{Ssie} \emph{müssen} \emph{ssie} \emph{herausschweben lassen}“, zischte Harry der Schlange zu. „\emph{Ssie} \emph{kann nicht mehr an Flucht denken.}“

„\emph{Ja}“, zischte die Schlange, „\emph{aber unterschätzen} \emph{Ssiessie} \emph{nicht}, \emph{ssiewar} \emph{die tödlichste} \emph{Kriegerin}.“ Der grüne Kopf senkte sich als Warnung. „\emph{Es wäre weise, mich zu fürchten, Junge, selbst wenn ich verhungert und zu neun Zehnteln tot wäre; hüte dich vor ihr, lass keinen einzigen Fehler in deiner Heuchelei zu}“.

Die grüne Schlange glitt sanft aus der Tür.

Und kurz darauf kroch ein Mann mit fahler Haut und einem ängstlichen Ausdruck auf seinem bärtigen Gesicht mit seinem Zauberstab in der Hand in den Raum.

„Mein Herr?“, sagte der Diener zögernd.

„Tu, was dir befohlen wurde“, flüsterte der Dunkle Lord mit dieser kalten Stimme, die aus dem Körper eines Kindes noch schrecklicher klang. „Und lass deinen Patronus nicht wanken. Denke daran, wenn ich nicht zurückkehre, wird es keine Belohnung für dich geben, und es wird lange dauern, bis deine Familie sterben darf.“

Nachdem er diese schrecklichen Worte gesprochen hatte, zog der Dunkle Lord seinen Unsichtbarkeitsmantel über seinen Kopf und verschwand.

Der kriecherische Diener öffnete die Tür zu Bellatrix' Käfig und zog eine winzige Nadel aus seinem Gewand, mit der er das menschliche Skelett stieß. Der einzelne Tropfen des produzierten roten Blutes wurde bald von einer kleinen Puppe aufgenommen, die auf den Boden gelegt wurde, und der Diener begann mit flüsternden Beschwörungen.

Bald lag ein weiteres lebendes Skelett regungslos auf dem Boden. Danach schien der Diener einen Moment zu zögern, bis aus der leeren Luft ein ungeduldiger Befehl zischte. Dann richtete der Diener seinen Zauberstab auf Bellatrix und sprach ein Wort, und das lebende Skelett, das auf dem Bett lag, war nackt, und das Skelett, das auf dem Boden lag, war in ihr verblichenes Kleid gekleidet.

Der Diener riss einen kleinen Stoffstreifen aus dem Kleid, als es auf der scheinbaren Leiche lag; und aus seinem eigenen Gewand holte der ängstliche Mann dann einen leeren Glaskolben hervor, an dessen Innenseite kleine Spuren von goldener Flüssigkeit haften. Dieser Kolben wurde in einer Ecke verborgen, der Rockstreifen darüber gelegt, das ausgelaugte Tuch verschmolz fast mit der grauen Metallwand.

Eine weiteres Schwenken des Zauberstabs des Dieners ließ das auf dem Bett liegende menschliche Skelett in die Luft schweben und kleidete sie in fast derselben Bewegung in neue schwarze Gewänder. Eine ganz gewöhnliche Flasche Schokomilch wurde ihr in die Hand gedrückt, und ein leises Flüstern befahl Bellatrix, die Flasche zu ergreifen und mit dem Trinken zu beginnen, was sie tat, wobei ihr Gesicht immer noch nur verwirrt aussah.

Dann machte der Diener Bellatrix unsichtbar und machte sich selbst unsichtbar, und sie gingen. Die Tür schloss sich hinter ihnen allen und klickte, während sie schloss, und tauchte den Korridor erneut in Dunkelheit, unverändert, bis auf eine kleine Flasche, die in der Ecke einer Zelle versteckt war, und eine frische Leiche, die auf dem Boden lag.

Zuvor, in dem verlassenen Laden, hatte Professor Quirrell Harry gesagt, dass sie das perfekte Verbrechen begehen würden.

Harry hatte gedankenlos begonnen, das Standardsprichwort zu wiederholen, dass es so etwas wie ein perfektes Verbrechen nicht gäbe, bevor er tatsächlich für eine Zweidrittelsekunde darüber nachdachte, sich an ein weiseres Sprichwort erinnerte und in der Mitte des Satzes den Mund hielt.

\emph{\emph{Was glauben Sie zu wissen, und} \emph{warum} \emph{glauben Sie es zu wissen?}}

Wenn Sie das perfekte Verbrechen begangen \emph{hätten}, würde es niemand jemals herausfinden - wie könnte also jemand \emph{wissen}, dass es keine perfekten Verbrechen gab?

Und sobald man es so betrachtete, erkannte man, dass perfekte Verbrechen wahrscheinlich \emph{immer wieder} begangen wurden, und der Gerichtsmediziner bezeichnete es als Tod durch natürliche Ursachen, oder die Zeitung berichtete, dass der Laden nie sehr profitabel gewesen war und schließlich pleite gegangen war…

Als am nächsten Morgen die Leiche von Bellatrix Black tot in ihrer Zelle gefunden wurde, machte sich im Gefängnis von Askaban, aus dem (wie jeder wusste) noch nie jemand entkommen war, niemand die Mühe, eine Autopsie durchzuführen. Niemand dachte zweimal darüber nach. Sie sperrten einfach den Korridor ab und gingen, und der \emph{Tagesprophet} berichtete am nächsten Tag in den Todesanzeigen darüber…

… das war das perfekte Verbrechen, das Professor Quirrell geplant hatte.

Und es war nicht Professor Quirrell, der es vermasselte.

