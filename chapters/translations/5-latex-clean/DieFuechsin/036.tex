

\hypertarget{statusunterschiede}{% \section{4. Statusunterschiede}\label{statusunterschiede}}

—\/-\/-\/-\/- Kapitel 36: Statusunterschiede -\/-\/-\/-\/-

Qualvoll desorientierend, so fühlte es sich an den Bahnsteig 9¾ in die restliche Welt zu verlassen, die Welt von der Harry einmal gedachte hatte, es wäre die einzig reale Welt. Menschen, die in normale T-Shirts und Hosen gekleidet waren, anstatt in den würdevolleren Umhänge der Hexen und Zauberer. Verstreute Müllreste hier und da um die Bänke herum. Ein vergessener Geruch, die Abgase von verbranntem Benzin, rau und scharf in der Luft. Das Ambiente des Bahnhofes King's Cross, weniger hell und heiter als Hogwarts oder die Winkelgasse; die Menschen wirkten kleiner, ängstlicher und wären wahrscheinlich begierig darauf ihre Probleme gegen einen ~dunklen Zauberer zum Bekämpfen einzutauschen. Harry wollte \emph{Ratzeputz} auf den Dreck und \emph{Everto} auf den Müll wirken und wenn er den Spruch kennen würde den Kopfblasenzauber, mit dem er die Luft nicht mehr atmen müsste. Aber er konnte seinen Zauberstab nicht benutzen, nicht hier…

So, realisierte Harry, muss es sich anfühlen aus einem Industrieland, also einem Erste Welt Land, in ein Dritte Welt Land zu reisen.

Nur war es die Nullte Welt welche Harry verlassen hatte, die Zaubererwelt mit ihren Reinigungszaubern und Hauselfen; in der man, mit Heilerkünsten und der eigenen Magie zusammengenommen, 170 werden konnte bevor das Alter einen wirklich einholte.

Und das nicht-magische London auf der Muggel-Erde, in welches Harry zeitweise zurückgekehrt war. Dies war wo Mum und Dad den Rest ihrer Leben verbringen würden, es sei denn die Technik machte einen Sprung auf die Lebensqualität der Zauberer oder irgendwas Grundlegendes in der Welt würde sich ändern.

Ohne auch nur darüber nachzudenken drehte sich Harrys Kopf und sein Blick schnellte nach hinten um die hölzerne Truhe zu sehen, die hinter ihm her trippelte, unbemerkt von allen Muggeln, die mit Klauen besetzten Tentakeln boten eine schnelle Bestätigung, dass, ja, er sich nicht einfach alles eingebildet hatte…

Und dann war da der andere Grund für das Engegefühl in seiner Brust.

Seine Eltern wussten nichts.

Sie wussten \emph{überhaupt} nichts.

Sie wussten nichts…

„Harry?“, rief eine dünne, blonde Frau, deren perfekte glatte und makellose Haut sie ein ganzes Stück jünger aussehen ließ als 33; und Harry bemerkte mit einem Mal, dass es Magie \emph{war}, er hatte die Anzeichen vorher nicht gekannt, aber er sah sie jetzt. Und was auch immer für ein Trank so lange wirkte, er musste schrecklich gefährlich gewesen sein, weil sich die meisten Hexen dies nicht selbst antaten, sie waren nicht so verzweifelt …

Es sammelten sich Tränen in Harrys Augen.

„Harry?“, rief ein älter aussehender Mann mit einem Bauchansatz, der mit zur Schau gestellter akademischer Nachlässigkeit mit einer schwarzen Weste gekleidet war, die über ein grau-grünes T-Shirt geworfen wurde, jemand, der immer ein Professor wäre, egal wo er hinginge, der sicherlich einer der brillantesten Zauberer seiner Generation geworden wäre, wäre er mit zwei Kopien dieses Gens geboren, anstatt mit null …

Harry hob seine Hand und winkte zu ihnen herüber. Er konnte nicht sprechen. Er konnte überhaupt nichts sagen.

Sie kamen zu ihm herüber. Professor Michael Verres-Evans würde nicht rennen, sondern ging in einem stetigen, würdevollen Gang ~und Mrs~Petunia Evans-Verres würde nicht schneller gehen.

Das Lächeln auf dem Gesicht seines Vaters war nicht sehr breit, aber sein Vater lächelte nie über das ganze Gesicht; es war aber zumindest eines der breitesten, die Harry je gesehen hatte, breiter als wenn ein neuer Zuschuss hereinkam oder wenn einer seiner Studenten eine Stelle erhielt, und man konnte nicht nach einem noch breiteren Lächeln fragen.

Mum blinzelte stark und sie versuchte zu lächeln, aber bekam es nicht richtig hin.

„So!„, sagte sein Vater als er angeschritten kam. „Schon irgendwelche revolutionären Entdeckungen gemacht?“

Natürlich dachte Dad er würde scherzen.

Es hatte nicht so sehr geschmerzt, dass seine Eltern kein Vertrauen in ihn hatten, damals, als noch niemand \emph{anderes} an ihn geglaubt hatte, als Harry nicht gewusst hatte, wie es sich anfühlte von Menschen wie Schulleiter Dumbledore und Professor Quirrell ernstgenommen zu werden.

Und das war der Moment als Harry realisierte, dass der \emph{Junge der lebte} nur im magischen Britannien existierte, dass es keine solche Person im Muggel-London gab, nur einen süßen kleinen elfjährigen Jungen, der zu Weihnachten nach Hause zu seinen Eltern kam.

„Entschuldigt mich“, sagte Harry mit zitternder Stimme, „Ich werde jetzt zusammenbrechen und heulen, das heißt nicht, dass irgendetwas in der Schule schiefgegangen ist.“

Harry wollte vorwärtsgehen, hielt dann an, hin- und hergerissen ob er seinen Vater oder seine Mutter zuerst umarmen sollte, er wollte nicht, dass sich einer der beiden beleidigt fühlte oder dachte, dass Harry ihn mehr lieben würde als den anderen—

„Du“, sagte sein Vater, „bist ein sehr törichter Junge, Mr~Verres“, und er nahm Harry sanft an den Schultern und drückte ihn in die Arme seiner Mutter, die sich hingekniet hatte, während ihr bereits Tränen über die Wangen liefen.

„Hallo Mum“, sagte Harry mit schwankender Stimme, „Ich bin zurück.“ Und er umarmte sie, inmitten des Maschinenlärms und dem Gestank von verbranntem Treibstoff; und Harry fing an zu weinen, weil er wusste, dass nichts so werden \emph{konnte} wie früher, am allerwenigsten er.

Der Himmel war komplett schwarz und die Sterne waren schon zu sehen als sie den Weihnachtsverkehr zur Universitätsstadt Oxford bewältigt hatten und in der Einfahrt des kleinen, schäbig aussehenden, alten Hauses parkten, das ihre Familie nutzte um den Regen von ihren Büchern fernzuhalten.

Als sie den kurzen Streifen des Pflasterwegs entlanggingen, der zur Vordertür führte, passierten sie eine Reihe von Blumentöpfen, die matte elektrische Lampen enthielten (matt deshalb, weil sie sich während des Tages durch Solarenergie aufladen mussten) und die Lichter leuchteten genau als sie vorbeigingen auf. Der schwierige Teil war die passenden Bewegungssensoren zu finden, die wasserdicht waren und genau auf die richtigen Entfernung ausgelöst wurden…

In Hogwarts gab es echte Fackeln die so etwas konnten.

Und dann öffnete sich die Eingangstür und Harry betrat das Wohnzimmer, heftig blinzelnd.

\emph{Jeder Zoll der Wand ist von einem Bücherregal bedeckt. Jedes Bücherregal hat sechs Regalbretter und reicht fast bis an die Decke. Einige Bücherregale sind bis zum Rand mit gebundenen Büchern gefüllt: Naturwissenschaft, Mathe, Geschichte und allem anderen. Andere Regale haben zwei Schichten Science-Fiction-Taschenbücher wobei die hintere Buchreihe einen Unterbau aus alten Taschentuchschachteln oder Kanthölzern hat, so dass man die Buchrücken über denen der vorderen Bücher sehen kann. Und es ist immer noch} \emph{nicht genug. Bücher überschwemmen die Tische und Sofas und türmen sich zu kleinen Haufen unter den Fenstern …}

Der Verres-Haushalt war genau wie er ihn verlassen hatte, nur gab es mehr Bücher, was auch genau so war wie er es verlassen hatte.

Und ein Weihnachtsbaum, zwei Tage vor dem Weihnachtsabend immer noch nackt und ungeschmückt, was Harry kurz aus der Bahn brachte. Ein warmes Gefühl breitete sich in seinem Bauch aus, als er begriff dass seine Eltern natürlich \emph{gewartet} hatten.

„Wir haben das Bett aus deinem Zimmer genommen um Platz für mehr Bücherregale zu machen“, sagte sein Vater. „Du kannst in deiner Truhe schlafen, oder?“

„\emph{Du} kannst in meiner Truhe schlafen“, sagte Harry.

„Dabei fällt mir ein“, sagte sein Vater. „Was haben sie schlussendlich wegen deines Schlafrhythmusses \emph{getan}?“

"Magie„, sagte Harry und stürzte geradewegs auf seine Schlafzimmertür zu, nur für den Falle dass sein Vater \emph{nicht} scherzte …

“Das ist keine Erklärung!„ sagte Professor Verres-Evans gerade als Harry rief, \emph{“Ihr habt all den leeren Platz in meinen Bücherregalen aufgebraucht?"}

Harry hatte den 23. Dezember damit verbracht Muggel-Sachen einzukaufen, die er nicht einfach verwandeln konnte; sein Vater war beschäftigt und hatte gesagte, dass Harry laufen oder den Bus nehmen müsse, was Harry gut zupasskam. Einige der Leute im Baumarkt hatten Harry fragend angesehen, aber er hatte mit unschuldiger Stimme gesagt, dass sein Vater in der Nähe einkaufte und ihn geschickt hätte einige Dinge zu besorgen (während er eine Liste in sorgfältig erwachsen aussehender halb unleserlicher Handschrift emporhielt); und am Ende blieb Geld eben Geld.

Sie hatten den Weihnachtsbaum alle zusammen geschmückt und Harry hatte eine kleine tanzende Fee auf die Spitze gesetzt (zwei Sickel und fünf Knuts bei Gambol \& Japes).

Gringotts hatte bereitwillig Galleonen in Papiergeld gewechselt, aber sie schienen keine einfache Möglichkeit zu haben größere Mengen Gold in steuerfreies, unauffälliges Muggel-Geld auf einem Schweizer Nummernkonto einzutauschen. Das hatte Harrys Plan vorerst vereitelt das meiste seines sich selbst gestohlenen Geldes in einer vernünftigen Kombination aus 60\% internationalen Indexfonds und 40\% Berkshire Hathaway-Aktien anzulegen. Für den Moment hatte Harry seine Anlagen insofern diversifiziert, dass er spät in der Nacht herausgeschlichen war, unsichtbar und in der Zeit zurückgekehrt, und 100 goldene Galleonen im Garten vergraben hatte. Er hatte das sowieso schon ewig, ewig, \emph{ewig} machen wollen.

Ein Teil des 24. Dezembers war damit verbracht worden, dass der Professor Harrys Bücher las und Fragen stellte. Die meisten Experimente, die sein Vater vorschlug waren nicht praktikabel, zumindest momentan; von den übriggebliebenen hatte Harry viele bereits durchgeführt. („Ja, Dad, ich habe kontrolliert was passiert, wenn Hermine eine veränderte Aussprache erhält und sie nicht weiss, ob sie verändert ist, das war das allererste Experiment, dass ich gemacht habe, Dad!„)

Die letzte Frage, die Harrys Vater gestellt hatte, während er mit einem Ausdruck verwirrtem Abscheus von \emph{Zaubertränke und Zauberbräue} aufsah, war, ob es alles Sinn ergab, wenn man ein Zauberer sei; und Harry hatte mit Nein geantwortet.

Woraufhin sein Vater erklärt hatte, dass Magie unwissenschaftlich wäre.

Harry war immer noch ein wenig schockiert von der Idee auf einen Bereich der \emph{Realität} zu zeigen und ihn unwissenschaftlich zu nennen. Dad schien zu denken, dass der Konflikt zwischen seinen Intuitionen und dem Universum bedeutete, dass das Universum ein Problem hätte.

(Andererseits gab es eine Menge Physiker, die dachten Quantenmechanik sei seltsam, anstatt das Quantenmechanik normal war und sie seltsam.)

Harry hatte seiner Mutter das Heiler Set gezeigt, das er für Zuhause gekauft hatte, obwohl viele der Zaubertränke nicht bei Dad funktionieren würden. Mum hatte den Kasten auf eine Weise angestarrt, die Harry dazu veranlasste zu fragen, ob Mums Schwester jemals so etwas für Opa Edwin und Oma Elaine gekauft hatte. Und als Mum immer noch nicht geantwortet hatte, hatter Harry eilig gesagt, dass sie wohl einfach nicht daran gedacht habe. Und dann, schlussendlich, war er aus dem Zimmer geflüchtet.

Lily Evans \emph{hatte} vermutlich nicht daran gedacht, das war das Traurige an der Angelegenheit. Harry wusste, dass andere Menschen die Tendenz hatten nicht über schmerzhafte Themen nachzudenken, so wie sie die Tendenz hatten ihre Hände nicht absichtlich auf eine rotglühende Herdplatte zu legen; und Harry begann zu vermuten, dass die meisten Muggelstämmigen eine Neigung dazu entwickelten nicht über ihre Familie nachzudenken, die eh alle sterben würden, bevor sie ihr erstes Jahrhundert erreicht hätten.

Nicht das Harry die Absicht hatte \emph{das} zuzulassen, natürlich.

Und dann, im späteren Tagesverlauf des 24. Dezembers, fuhren sie zu ihrem Weihnachtsessen los.

Das Haus war groß, nicht nach Hogwarts Standards, aber mit Sicherheit nach dem Maßstab, was man bekommen konnte wenn der Vater ein angesehener Professor ist, der versucht in Oxford zu wohnen. Zwei Stockwerke aus Ziegeln, ind er untergehenden Sonne glänzend, mit Fenstern über Fenstern und einem großen Fenster, dass sich höher erstreckt als es mit Glass gehen sollte, das musste wohl ein verdammt großes Wohnzimmer sein …

Harry holte tief Luft und läutete die Türklingel.

Es war ein dumpfes “Schatz, kannst du gehen?„ hörbar, gefolgt vom Geräusch sich langsam nähernder Schritte.

Und dann öffnete sich die Tür und offenbarte einen freundlichen Mann mit runden und rosigen Wangen und sich lichtendem Haar in einem blauen, an den Nähten etwas spannenden Hemd.

“Dr~Granger?„, grüßte Harrys Vater munter, bevor Harry den Mund aufmachen konnte. “Ich bin Michael und das sind Petunia und unser Sohn Harry. Das Essen ist in der magischen Truhe„, und Dad machte eine vage Geste hinter sich—tatsächlich nicht ganz in Richtung der Truhe.

“Ja, bitte kommen Sie rein„, sagte Leo Granger. Er trat vor und nahm mit einem gemurmelten “Vielen Dank„ die Weinflasche aus der ausgestreckten Hand des Professors entgegen und trat zurück und winkte in Richtung Wohnzimmer. “Nehmen Sie doch Platz. Und„, sein Kopf wandte sich nach unten um Harry anzusprechen, “alle Spielsachen sind im Untergeschoss, ich bin sicher Herm kommt auch gleich runter, es ist die erste Tür rechts„, und zeigte auf einen Korridor.

Harry sah in einen Moment nur an, wobei ihm bewusst war, dass er seinen Eltern den Weg hinein versperrte.

“Spielsachen?„, sagte Harry in einer fröhlichen, hohen Stimme und machte große Augen. “Ich liebe Spielsachen!„

Seiner Mutter hinter ihm atmete scharf ein und Harry stiefelte ins Haus, wobei er es fertigbrachte nicht zu doll beim Gehen zu trampeln.

Das Wohnzimmer war ganz so groß wie es von außen ausgesehen hatte, mit einer riesigen gewölbten Decke, an der ein Kronleuchter baumelte, und einem Weihnachtsbaum, der mordsmäßige Schwierigkeiten gemacht haben musste ihn durch die Tür zu manövrieren. Die unteren Bereiche des Baumes waren gründlich und sorgfältig in netten Mustern in Rot und Grün und Gold dekoriert, mit einem neu entdeckten Sprenkel aus Blau und Bronze; die Höhen, die nur ein Erwachsener erreichen konnte, waren nachlässig und wahllos mit Lichterketten und Lamettakränzen behängt. Ein Hausflur erstreckte sich bis zu einer Einbauküche und eine hölzerne Treppe mit einem poliertem Metallgeländer erstreckte sich nach oben in ein zweites Stockwerk.

“Mensch!„, sagte Harry. “Ist das ein großes Haus! Ich hoffe ich verlauf mich hier drin nicht!"

Dr~Roberta Granger fühlte sich ziemlich nervös als das Abendessen näher rückte. Der Truthahn und der Braten, ihre eigenen Beiträge zu diesem gemeinsamen Projekt, garten im Ofen vor sich hin; die anderen Speisen sollten von ihren Gästen, der Verres Familie, mitgebracht werden, die einen Jungen namens Harry adoptiert hatten. Welcher in der Zaubererwelt als der Junge der lebte bekannt war. Und der außerdem der einzige Junge war, den Hermine jemals „süß„ genannt oder überhaupt je bemerkt hatte, um genau zu sein.

Die Verresens hatten gesagt, dass Hermine das einzige Kind in Harrys Altersgruppe war, dessen Existenz ihr Sohn jemals auf irgendeine Art anerkannt hatte.

Und es mochte etwas vorschnell sein; aber beide Paare hatten den heimlichen Verdacht, dass ein paar Jahre in der Zukunft Hochzeitsglocken zu hören sein könnten.

Während also der erste Weihnachtsfeiertag wie immer mit der Familie ihres Mannes verbracht werden würde, hatte sie entschieden Heiligabend damit zu verbringen die möglichen Schwiegereltern ihrer Tochter kennen zu lernen.

Es klingelte an der Tür, als sie mitten dabei war den Truthahn zu beträufeln, und sie erhob ihre Stimme und rief: “\emph{Schatz, kannst du aufmachen gehen}?„.

Es gab ein kurzes Ächzen von einem Stuhl und dem, der darauf gesessen hatte, und dann waren die schweren Tritte ihres Mannes zu hören, sowie die Tür, die aufschwang.

“Dr~Granger?„, sagte die forsche Stimme eines älteren Mannes. “Ich bin Michael und das sind Petunia und unser Sohn Harry. Das Essen ist in der magischen Truhe."

„Ja, kommen Sie rein,„ sagte ihr Mann, gefolgt von einem gemurmelten “Vielen Dank„ das darauf hindeutete, dass eine Art von Geschenk angenommen wurde und “Nehmen Sie doch Platz„. Dann veränderte sich Leos Stimme zu einem Ausdruck künstlichen Enthusiasmus und sagte „Und alle Spielsachen sind im Untergeschoss, ich bin sicher Herm kommt auch gleich runter, es ist die erste Tür rechts.“

Es gab eine kurze Pause.

Dann sagte die helle Stimme eines kleinen Jungen: „Spielzeug? Ich liebe Spielzeug!“

Es ertönte das Geräusch von Schritten, die ins Haus kamen, und dann sagte die gleiche helle Stimme: „Mensch! Ist das ein großes Haus! Ich hoffe ich verlauf mich hier drin nicht!“

Roberta schloss lächelnd den Ofen. Sie war ein wenig besorgt darüber gewesen, wie Hermine in ihren Briefe den Jungen der lebte beschrieben hatte—obwohl ihre Tochter keinesfalls etwas gesagt hatte, was darauf hindeutet, dass Harry Potter \emph{gefährlich} war; nichts was an die dunklen Hinweise aus den Büchern erinnerte, die Roberta—angeblich für Hermine—während ihres Ausflugs in die Winkelgasse gekauft hatte. Ihre Tochter hatte nicht viel erzählt, nur dass Harry sprach als käme er aus einem Buch und Hermine fleißiger lernte als je zuvor in ihrem Leben, nur um ihm im Unterricht voraus zu bleiben. Aber wie es sich anhörte, war Harry Potter ein ganz normaler ~elfjähriger Junge.

Als sie zur Haustür kam, preschte ihre Tochter gerade hektisch mit einer Geschwindigkeit, die ganz und gar nicht ungefährlich aussah, die Treppe hinunter. Hermine hatte zwar behauptet, dass Hexen weitaus robuster gegen Stürze seien, aber Roberta war sich nicht ganz sicher, ob sie das glaubte—

Roberta nahm nun Professor und Mrs~Verres zum ersten Mal in Augenschein, die beide ziemlich nervös aussahen als sich der Junge mit der legendären Narbe auf der Stirn an ihre Tochter wandte und nun mit leiser Stimme sagte: „Gehaben Sie sich an diesem schönsten aller Abende wohl, Miss~Granger.“ Er zeigte mit seiner Hand nach hinten, als ob er seine Eltern auf einem Silbertablett darbieten würde. „Darf ich Ihnen meinen Vater, Professor Michael Verres-Evans, und meine Mutter, Frau Petunia Evans-Verres, vorstellen?“

Und gerade als Robertas Mund aufklappte, drehte sich der Junge zu seinen Eltern um und sagte nun wieder mit dieser hellen Stimme: „Mum, Dad, das ist Hermine! Sie ist echt schlau!“

„\emph{Harry!} “ zischte ihre Tochter. „Hör auf damit!“

Der Junge drehte sich wieder zu Hermine um. „Ich fürchte, Miss~Granger“, sagte der Junge gewichtig, „dass Sie und ich in das labyrinthische Refugium des Kellers verbannt wurden. Überlassen wir sie also ihrer erwachsenen Konversation, die zweifellos weit über unseren eigenen kindlichen Intellekt hinausgehen würde, und setzen wir unsere laufende Diskussion über die Auswirkungen des Hume'schen Projektivismus1 auf Verwandlungen ~fort.“

„Wir gehen dann mal“, sagte ihre Tochter mit sehr fester Stimme, packte den Jungen an seinem linken Ärmel und schleppte ihn in den Flur—Roberta drehte sich hilflos um sie im Blick zu behalten und als sie an ihr vorbei kamen, winkte ihr der Junge fröhlich zu—und dann zog Hermine den Jungen durch die Kellertür und schlug sie hinter sich zu.

„Ich, ähm, ich entschuldige mich für…“, sagte Mrs~Verres mit stockender Stimme.

„Es tut mir leid“, sagte der Professor und lächelte nachsichtig, „Harry kann bei solchen Dingen etwas empfindlich sein. Aber ich nehme an, er hat Recht damit, dass wir uns nicht für ihr Gespräch interessieren.“

\emph{Ist er gefährlich?,} wollte Roberta fragen, aber sie schwieg und versuchte stattdessen, sich eine subtilere Frage zu überlegen. Ihr Mann neben ihr kicherte, als hätte er das, was sie gerade gesehen hatten, lustig und nicht beängstigend gefunden.

Der furchtbarste Dunkle Lord aller Zeiten hatte versucht, diesen Jungen zu töten, und die verbrannten Überreste seines Körpers waren neben dem Kinderbett gefunden worden.

Ihr möglicher künftiger Schwiegersohn.

Roberta war zunehmend besorgter geworden, ihre Tochter der Hexerei zu überlassen—vor allem, nachdem sie die Bücher gelesen, die Daten zusammengesetzt und erkannt hatte, dass ihre magische Mutter wahrscheinlich auf dem Höhepunkt von Grindelwalds Schreckensherrschaft getötet worden war, und \emph{nicht} bei der Geburt gestorben war, wie ihr Vater es immer behauptet hatte. Aber Professor McGonagall hatte ihnen nach ihrem ersten Ausflug noch andere Besuche gemacht, um „zu sehen, wie es Miss~Granger gehe“; und Roberta konnte nicht anders, als zu denken, dass wenn Hermine sagte, dass ihre Eltern ihrer Hexenkarriere im Weg stünden, etwas getan werden würde, um das zu \emph{beheben}…

Roberta setzte ihr bestes Lächeln auf und tat, was sie konnte, um etwas vorgesetzte Weihnachtsfreude zu verbreiten.

Der Esstisch war für sechs Personen—äh, vier Personen und zwei Kinder—eigentlich viel zu lang, aber er war trotzdem vollständig mit einer Tischdecke aus feinem weißen Leinen bezogen und das Geschirr war unnötigerweise auf ausgefallene Servierplatten gestellt worden, die zumindest aus Edelstahl und nicht aus echtem Silber waren.

Harry hatte ein wenig Mühe, sich auf den Truthahn zu konzentrieren.

Das Gespräch hatte sich natürlich zu Hogwarts gewandt; und es war für Harry offensichtlich, dass seine Eltern hofften, dass Hermine etwas mehr über Harrys Schulleben herausrutschen würde, als Harry ihnen erzählt hatte. Und entweder Hermine hatte das bemerkt oder sie hatte schon von alleine versucht das Gespräch von allen möglicherweise problematischen Dingen abzulenken.

Also war für \emph{Harry} alles in Ordnung.

Aber leider hatte Harry den Fehler gemacht, seinen Eltern alle möglichen Dinge über Hermine zu erzählen, die sie ihren \emph{eigenen} Eltern nicht gesagt hatte.

Wie zum Beispiel, dass sie der General einer Armee in ihren außerschulischen Aktivitäten war.

Hermines Mutter hatte daraufhin sehr beunruhigt ausgesehen und Harry hatte schnell unterbrochen und sein Bestes getan, zu erklären, dass alle Zaubersprüche nur betäuben würden, Professor Quirrell ständig aufpasste und viele Dinge aufgrund des Vorhandensein magischer Heilmethoden viel ungefährlicher waren, als sie klangen, woraufhin Hermine ihn unter den Tisch hart gegen das Schienbein getreten hatte. Glücklicherweise hatte Harrys Vater, der wie Harry zugeben musste in manchen Dingen einfach besser war als er, mit der Autorität des Professors verkündet, dass er sich überhaupt keine Sorgen gemacht hatte, da er sich nicht vorstellen konnte, dass Kindern solche Aktivitäten erlaubt wären, wenn sie gefährlich seien.

Das war aber nicht der Grund, warum Harry Schwierigkeiten hatte, das Abendessen zu genießen.

… das Problem mit Selbstmitleid war, dass es niemals mehr als fünf Minuten brauchte jemanden zu finden, der es schlimmer hatte.

Dr~Leo Granger hatte an einer Stelle gefragt, ob die nette Lehrerin, die Hermine zu mögen schien, Professor McGonagall, ihr eine Menge Punkte in der Schule geben würde.

Hermine hatte ja gesagt, mit einem scheinbar aufrichtigen Lächeln.

Harry hatte es mit einiger Anstrengung geschafft, sich davon abzuhalten, eisig darauf hinzuweisen, dass Professor McGonagall niemals einen Hogwarts-Schüler bevorzugt behandeln würde, und dass Hermine eine Menge Punkte erhielt, weil sie sich davon \emph{jeden einzelne verdient} hatte.

An einem anderen Punkt hatte Leo Granger dem Tisch seine Meinung geäußert, dass Hermine sehr klug sei und eine medizinische Fakultät besuchen und Zahnarzt hätte werden können, wenn nicht die ganze Hexensache dazwischen gekommen wäre.

Hermine hatte daraufhin wieder gelächelt und ein kurzer Blick von ihr hatte Harry davon abgehalten, vorzuschlagen, dass Hermine auch eine \emph{international anerkannte Wissenschaftlerin} hätte werden können, und zu fragen, ob dieser Gedanke den Grangers in den Sinn gekommen wäre, wenn sie einen Sohn anstelle einer Tochter gehabt hätten oder ob es für ihre Nachkommen in jedem Fall inakzeptabel wäre, es im Leben zu mehr zu bringen als sie.

Aber Harry begann langsam vor Wut zu schäumen.

Und er wurde \emph{viel} dankbarer für die Tatsache, dass sein eigener Vater \emph{immer} alles in seiner Macht stehende getan hatte, um Harrys Entwicklung als Wunderkind zu unterstützen, und ihn \emph{immer} ermutigt hatte, höher zu streben, und \emph{niemals} eine einzige seiner Leistungen herabsetzt hatte, selbst wenn ein Wunderkind eben doch nur ein Kind war. Hätte so seine Familie aussehen können, wenn Mum diesen Vernon Dursley geheiratet hätte?

Harry gab sich jedoch alle Mühe.

„Und sie ist wirklich in allen Fächern besser als du, ausser in den Flugstunden und in Verwandlung?“, fragte Professor Michael Verres-Evans.

„Ja“, sagte Harry mit gezwungener Ruhe, als er sich einen weiteren Bissen des Weihnachtstruthahns abschnitt. „In den meisten sogar um ein gutes Stück.“ Unter anderen Umständen hätte Harry das weniger bereitwillig zugegeben, weswegen sein Vater auch jetzt erst davon erfuhr.

„Hermine war in der Schule schon immer ziemlich gut“, sagte Dr~Leo Granger in einem zufriedenen Tonfall.

„Harry nimmt erfolgreich an nationalen Wettbewerben teil!“, sagte Professor Michael Verres-Evans.

„Schatz!“ sagte Petunia.

Hermine kicherte, und das ließ Harry sich in dieser Situation kein bisschen wohler fühlen. Dass es Hermine nicht zu stören schien, war \emph{was Harry beunruhigte}.

„Es ist mir nicht peinlich, gegen sie zu verlieren, Dad“, sagte Harry. Gerade in diesem Moment war es das auch nicht. „Habe ich eigentlich erzählt, dass sie vor dem ersten Unterrichtstag alle ihre Schulbücher auswendig gelernt hat? Und ja, ich habe das überprüft.“

„Ist das für sie, äh, \emph{üblich}?“, wandte sich Professor Verres-Evans an die Grangers.

„Oh, ja, Hermine lernt ständig irgendwelche Sachen auswendig“, sagte Dr~Roberta Granger mit einem fröhlichen Lächeln. „Sie weiß jedes Rezept aus allen meinen Kochbüchern aus dem Kopf. Sie fehlt mir jetzt immer, wenn ich Abendbrot mache.“

Nach dem Gesichtsausdruck seines Vaters zu schließen, fühlte Dad zumindest etwas von dem, was Harry fühlte.

„Keine Sorge, Dad“, sagte Harry, „jetzt bekommt sie alles an fortgeschrittenerem Lernmaterial, was sie nur möchte. Ihre Lehrer in Hogwarts wissen, dass sie klug ist, \emph{im Gegensatz zu ihren Eltern}! “

Seine Stimme hatte sich bei den letzten fünf Worten erhoben und schon, als sich alle Gesichter drehten, um ihn anzustarren, und Hermine ihn wieder trat, wusste Harry, dass er es verbockt hatte, aber es war zu viel gewesen, einfach zu viel.

„Natürlich wissen wir, dass sie klug ist“, sagte Leo Granger und begann beleidigt das Kind anzuschauen, das die Dreistigkeit besaß, seine Stimme an ihrem Esstisch zu erheben.

„Sie haben nicht die geringste Ahnung“, sagte Harry, sein Tonfall begann eisig zu werden. „Sie denken, dass sie viele Bücher liest und das süß ist, nicht wahr? Sie sehen ein perfektes Zeugnis und denken, es wäre schön, dass sie sich im Unterricht gut macht. Ihre Tochter ist die talentierteste Hexe ihrer Generation und der hellste Stern von Hogwarts und eines Tages, Dr~und Dr~Granger, wird die Tatsache, dass Sie ihre Eltern waren, der einzige Grund sein, warum sich die Geschichte an Sie erinnern wird!“

Hermine, die ruhig von ihrem Platz aufgestanden war und um den Tisch herumging, entschied sich in diesem Moment, Harrys Shirt an der Schulter zu packen und ihn aus seinem Stuhl zu ziehen. Harry ließ sich das zu, aber als Hermine ihn wegzerrte, sagte er, seine Stimme noch lauter erhoben: „Es ist durchaus möglich, dass in tausend Jahren die Tatsache, dass die Eltern von Hermine Granger Zahnärzte waren, der einzige Grund sein wird, warum sich überhaupt jemand an die Zahnmedizin erinnert!“

Roberta starrte auf die Stelle, wo ihre Tochter gerade den Jungen der lebte aus dem Raum gezogen hatte, einen geduldigen Blick auf ihrem jungen Gesicht.

„Es tut mir schrecklich leid“, sagte Professor Verres mit einem amüsierten Lächeln. „Aber bitte machen Sie sich keine Sorgen, Harry redet immer so. Sind sie nicht schon wie ein altes Ehepaar?“

Und das Erschreckende war, dass es \emph{stimmte}.

Harry hatte eine ziemlich ernsten Vortrag von Hermine erwartet.

Aber nachdem sie ihn durch die Tür zur Treppe nach unten gezogen und diese hinter ihnen geschlossen hatte, drehte sie sich um…

… und lächelte, aufrichtig, soweit Harry das beurteilen konnte.

„Bitte tu das nicht, Harry“, sagte sie mit sanfter Stimme. „Auch wenn es sehr nett von dir ist. Es ist alles in Ordnung.“

Harry sah sie nur an. „Wie kannst du das nur ertragen?“, sagte er. Er musste seine Stimme leise halten, sie wollten schließlich nicht, dass ihrer beider Eltern sie hörten, aber sie stieg in der Tonhöhe, wenn schon nicht in der Lautstärke. „\emph{Wie kannst du das aushalten?} “

Hermine zuckte mit den Achseln und sagte: „Weil Eltern so sein \emph{sollten}?“

„Nein“, sagte Harry, seine Stimme leise und angespannt, „ist es nicht, mein Vater macht mich \emph{nie} so runter—also, er \emph{tut} es schon, aber niemals so—“

Hermine hielt einen einzelnen Finger hoch und Harry wartete, sie bei der Suche nach den richtigen Worten beobachtend. Es dauerte eine Weile, bis sie schließlich sagte: „Harry… Professor McGonagall und Professor Flitwick mögen mich, weil ich die talentierteste Hexe meiner Generation und der hellste Stern von Hogwarts bin. Und Mum und Dad wissen das nicht und du wirst es ihnen niemals begreiflich machen können, aber sie lieben mich trotzdem. Was bedeutet, dass alles genau so ist, wie es sein sollte, in Hogwarts und zu Hause. Und da es \emph{meine} Eltern sind, Mr~Potter, haben \emph{Sie} sich da nicht einzumischen.“ Sie lächelte wieder einmal ihr geheimnisvolles Lächeln vom Abendessen und sah Harry sehr nachsichtig an. „\emph{Ist} das klar, Mr~Potter?“

Harry nickte bestimmt.

„Gut“, sagte Hermine, lehnte sich hinüber und küsste ihn auf die Wange.

Das Gespräch hatte gerade erst wieder begonnen, als ein entfernter, schriller Aufschrei zu ihnen durchdrang.

„\emph{Hey! Kein Küssen!} “

Die beiden Väter brachen in Gelächter aus, während die beiden Mütter mit identisch entsetzten Blicken von ihren Stühlen aufsprangen und zum Keller stürzten.

Als die Kinder zurückgebracht worden waren, sagte Hermine in einem eisigen Ton, dass sie Harry nie wieder küssen würde, und Harry sagte mit empörter Stimme, dass die Sonne zu kalter, toter Asche niederbrennen würde, bevor er sie noch einmal nahe genug heranließe, um es zu versuchen.

Was hieß, dass alles genau so war, wie es sein sollte, und sie setzten sich alle wieder hin, um ihr Weihnachtsessen zu beenden.

