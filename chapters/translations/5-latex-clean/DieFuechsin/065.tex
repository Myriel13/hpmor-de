

\hypertarget{ansteckende-luxfcgen}{% \section{33. Ansteckende Lügen}\label{ansteckende-luxfcgen}}

-\/-\/-\/-\/- Kapitel 65: Ansteckende Lügen -\/-\/-\/-\/-

Hermine Granger hatte irgendwo einmal gelesen, dass einer der Schlüssel zu einer guten Figur darin bestand, genau auf das eigene Essen zu achten. Demnach sollte man sich selbst bewusst dabei beobachten, wie und was man aß. Auf diese Weise erreichte man angeblich eine höhere Zufriedenheit mit seiner Ernährung und aß dann weniger.

Also gut: Heute Morgen hatte sie sich einen Toast gemacht und Butter auf den Toast gestrichen. Dann hatte sie Zimt auf die Butter gestreut und sich an den Tisch gesetzt. Allein diese aufwändige Vorbereitung hätte ihrer Ansicht nach wirklich reichen sollen, damit sie das köstliche Frühstück, das nun vor ihr lag, auch \emph{bewusst} wahrnahm.Doch es passierte schon wieder. Ohne den Zimt, die Butter oder überhaupt irgendetwas von ihrem Essen zu bemerken, hatte sie bereits den halben Toast verschlungen. Egal. Hastig schob sie den Gedanken beiseite und wandte sich wieder ihrem Gesprächspartner zu. „Kannst du das vielleicht nochmal erklären? Ich finde das, gelinde gesagt, immer noch ziemlich überraschend.“"Wenn man wie ein Slytherin denkt, ist es im Grunde einfach", entgegnete der Junge, den alle anderen in der Schule mittlerweile für ihre wahre Liebe hielten. Nur sie beide machten da eine Ausnahme. Harry Potter rührte mit dem Löffel geistesabwesend in seinen Frühstücksflocken herum. ~Soweit Hermine wusste, hatte er bislang nicht viel gegessen. „Zu jeder guten Sache in dieser Welt gibt es meiner Erfahrung nach einen negativen Gegenspieler. Phönixe sind da keine Ausnahme."Hermine nahm einen weiteren unbemerkten Bissen von ihrem gebutterten und zimtbestreutem Toast. „Wie kann jemand \emph{nicht verstehen}, dass Fawkes nur deshalb auf Dumbledores Schulter reitet, weil er davon überzeugt ist, dass Dumbledore ein guter Mensch ist? Bei einem dunklen Zauberer würde er das niemals machen!"Davon, wie Fawkes \emph{sie} an der Wange berührt hatte, hatte sie selbstverständlich niemandem erzählt. Sie wusste, dass das nicht richtig gewesen wäre. Man prahlte nicht damit, wenn man von einem Phönix berührt wurde. \emph{Dafür} war ein Phönix nicht da. Aber darum ging es hier auch nicht. Sie hatte nur \emph{gehofft}, dass die Gerüchte aufhören würden, nach denen Harry Potter im Begriff war, böse zu werden und sie ihm nachfolgte.Leider war das offensichtlich nicht der Fall. Und sie verstand einfach nicht, warum das so war. Harry aß noch einen Bissen von seinem Müsli. Sein Blick schweifte in die Ferne. „Stell dir einmal Folgendes vor: Du schwänzt eines Tages die Schule und erzählst deiner Lehrerin fälschlicherweise, dass du krank wärst. Daraufhin entgegnet deine Lehrerin, du müsstest ein ärztliches Attest mitbringen. Also bleibt dir nichts anderes übrig, als eines zu fälschen. Doch damit gibt sich deine Lehrerin nicht zufrieden. Sie erklärt dir ihre Absicht, deinen Arzt anzurufen, um nachzufragen, ob du wirklich krank bist. Jetzt wird die Sache für dich schon komplizierter. Um deine Geschichte aufrechterhalten zu können, musst du ihr nun eine falsche Telefonnummer geben und einen Freund dazu bringen, sich als Arzt auszugeben, wenn sie anruft."„\emph{Was} hast du gemacht?!"Harry schaute von seinem Müsli auf und blickte sie wieder direkt an. Er lächelte. „Ich habe nicht gesagt, dass ich das wirklich \emph{getan} habe, Hermine." Dann ließ er den Blick wieder auf sein Müsli wandern. „Und das habe ich auch nicht. Es war nur ein Beispiel, um dir zu verdeutlichen, dass Lügen sich fortpflanzen. Das ist es, was ich dir sagen wollte. Wenn man einmal eine Lüge erzählt, muss man in der Regel schon bald weitere hinzufügen, um die erste zu vertuschen. Schließlich muss man fortan alles, was mit der ersten Lüge in irgendeiner Form zusammenhängt, der selbst erdachten Konstruktion der Wirklichkeit anpassen. Wenn man dann \emph{immer weiter} lügt und \emph{immer weiter} versucht, die erste und die daraus resultierenden weiteren Lügen zu vertuschen, kommt man früher oder später zwangsläufig an einen Punkt, an denen man die allgemeinen Gesetze des Denkens infrage stellen muss.

Dazu möchte ich dir ein weiteres Beispiel geben. Stell dir vor, jemand möchte eine Art alternative Medizin verkaufen, die nicht funktioniert. Wenn nun jedoch eine doppelblinde, experimentelle Studie durchgeführt wird, wird sich die Wirkungslosigkeit der Medizin naturgemäß bestätigen. Um die eigene Lüge weiter verteidigen zu können, hat der Erfinder dann nur noch eine Möglichkeit: Er muss mögliche Anwender der Arznei dazu bringen, die experimentelle Methode als solche in Zweifel zu ziehen. Er könnte zum Beispiel anführen, dass die experimentelle Methode nur für die \emph{wissenschaftliche} Art von Medizin gedacht sei und die \emph{erstaunlichen Kräfte der alternativen} Medizin damit nicht zu erfassen sind. Oder der Erfinder könnte es sich noch leichter machen und einfach an den Glauben der Menschen appellieren. Ein guter und tugendhafter Mensch solle demnach prinzipiell fest an die Wirksamkeit der Medizin glauben, unabhängig davon, was Studien sagen. Nicht zuletzt könnte der Erfinder natürlich auch die Existenz einer einzigen Wahrheit als Ganzes infrage stellen und sich darauf berufen, dass es schlicht keine objektive Realität gibt. Theorien in diese Richtung existieren schließlich eine Menge. Dabei sind diese allesamt -- und das möchte ich an dieser Stelle noch einmal ausdrücklich betonen - ~nicht nur einfach \emph{falsch}, sondern \emph{systematisch} falsch. Es handelt sich um eine Anti-Epistemologie. Zu jeder Regel der Rationalität, die dir sagt wie du die Wahrheit findest, gibt es da draußen jemanden, der dich dazu bringen will das genaue Gegenteil zu glauben.

Was ich damit sagen will: Sobald du einmal lügst, ist die Wahrheit fortan für immer dein Feind und verfolgt dich. Wenn du jetzt bedenkst, dass es da draußen eine Menge Leute gibt, die Lügen erzählen -" Harry verstummte.„Was hat das dann mit Fawkes zu tun?„, fragte Hermine.Harry zog den Löffel aus seiner Müslischale und deutete damit in Richtung des Lehrertisches. „Der Schulleiter hat einen Phönix, nicht wahr? Und er ist der Oberste Hexenmeister des Zauberergamot?“ Hermine nickte.

„Also hat er zwangsläufig auch politische Gegner, so wie Lucius. Und jetzt überlege dir, wie diese Gegner wohl mit der Tatsache umgehen, dass Dumbledore einen Phönix besitzt und sie nicht. Kapitulieren sie einfach und nehmen das so hin? Oder geben sie wenigstens zu, dass Fawkes ein \emph{Beweis} dafür ist, dass Dumbledore ein guter Mensch ist? Nein, natürlich nicht. Stattdessen erfinden sie Theorien, die das Besitzen eines Phönix als \emph{unbedeutend}, wenn nicht sogar als Beleg für einen schlechten Charakter darstellen.

Zum Beispiel könnten die Gegner behaupten, dass Phönixe nur Leute begleiten, die direkt auf jeden losgehen, den sie für böse halten. Einen Phönix zu haben, würde demnach also nichts anderes bedeuten, als dass man ein Idiot oder ein gefährlicher Fanatiker ist. Eine andere Theorie wäre, dass Phönixe nur Leuten folgen, die reine Gryffindors sind, also keinerlei Tugenden der anderen Häuser besitzen. Vergleichsweise harmlos wäre schließlich noch die Theorie, dass der Phönix sich einfach dem anschließt, der ihm aus rein subjektiven Gründen sympathisch ist. Einen Politiker auf dieser Grundlage zu beurteilen, wäre dann natürlich alles andere als fair.

Kurzum: Hauptsache sie finden \emph{irgendetwas}, mit dem sie die wahre Bedeutung des Phönix verleugnen können.

Im Übrigen wette ich darauf, dass Lucius sich nicht einmal etwas Neues ausdenken musste. Ich bin sicher, dass alle diese Theorien wahrscheinlich schon vor Jahrhunderten mindestens einmal ausgesprochen und verbreitet worden sind. Seit dem Tag, an dem jemand zum ersten Mal einen Phönix auf seiner Schulter sitzen hatte und jemand anderes wollte, dass man diesem Umstand keine Bedeutung beimisst, hat es solche Theorien gegeben. Davon bin ich fest überzeugt.“

Harry machte eine kurze Pause, um Luft zu holen. Dabei bemerkte er den Löffel, den er noch immer in die Höhe hielt. Rasch ließ er ihn singen.

Dann fuhr er fort: „Zu der Zeit als Fawkes auftauchte, muss es bereits Allgemeingut gewesen sein, dass man den Vorlieben eines Phönix keine übermäßige Bedeutung beimessen sollte. ~Ob ein Phönix jemanden mag oder nicht mag als Kriterium für dessen innere Einstellung zu verwenden, muss den Menschen dementsprechend sehr \emph{seltsam} erschienen sein. Vielleicht so, als wenn eine Muggelzeitung einen politischen Kandidaten einem eigenen Test unterziehen würde, um seine wissenschaftliche Kompetenz zu bewerten.

Für jede gute Kraft, die es in diesem Universum gibt, profitiert jemand davon, dass Menschen sie als bedeutungslos einstufen oder falsch interpretieren und sie irgendwohin verlegen, wo sie ihnen nichts tun kann.“

„Aber -„, wandte Hermine ein, „Okay, ich verstehe, warum Lucius Malfoy nicht will, dass jemand anderes Fawkes wahre Bedeutung erkennt. Aber warum sollte jemand, der \emph{kein} schlechter Mensch ist, ihm \emph{glauben}?“

Harry zuckte mit den Schultern. Er ließ den Löffel zurück in sein Müsli fallen und begann gedankenverloren, darin zu rühren. „Gegenfrage: Was begeistert die Menschen am Zynismus? Ganz einfach: Sie halten es für ein Zeichen von Reife und Kultiviertheit, wenn man sich auf zynische Art und Weise über andere erhebt. Für die Leute hat es den Anschein, als ob der Zyniker bereits alles gesehen hätte und es daher besser wüsste. Vielleicht freuen sie sich auch daran, dass sie selbst größer erscheinen, weil der Zyniker etwas anderes kleinredet.

Oder -- um nun wieder konkret auf das Beispiel des Phönix zu kommen - ~ihr politischer Instinkt rät den Leuten, keine nette Dinge über diese magischen Tiere zu sagen, weil sie schließlich selbst keines haben. Als letztes fällt mir noch, dass die Leute womöglich einfach das Gefühl schätzen, eine anscheinend geheime Wahrheit zu kennen, von der das gemeine Volk keine Ahnung hat.“

Harry Potter ließ seinen Blick einen Augenblick hinüber zum Lehrertisch schweifen und fuhr dann mit leiser Stimme fort: „Vielleicht ist genau das \emph{sein} größter Fehler: Dass er alles zynisch betrachtet - außer den Zynismus selbst.“

Ohne nachzudenken guckte Hermine selbst in Richtung des Lehrertisches. Der Platz des Verteidigungsprofessors war, genau wie schon am Montag und Dienstag, leer. Das passte zu dem, was die stellvertretende Schulleiterin ihnen kurz zuvor mitgeteilt hatte. Professor Quirrells Unterricht würde demnach heute ein weiteres Mal ausfallen.

Nachdem Harry sein Müsli sowie ein paar Bissen Sirupkuchen gegessen hatte und schließlich gegangen war, sah Hermine zu Anthony und Padma hinüber. Die beiden saßen zufällig in der Nähe und hatten selbstverständlich \emph{nicht gelauscht o}der etwas in dieser Art getan. Wenn sie doch etwas mitbekommen haben sollten, war das gewiss \emph{rein zufällig} passiert.

Anthony und Padma erwiderten ihren Blick ein paar Sekunden schweigend. Dann fragte Padma zögernd: „Habe nur ich diesen Eindruck oder hat Harry Potter in den letzten Tagen wirklich angefangen, wie ein \emph{noch} \emph{komplizierteres} Buch zu sprechen? Ich meine, ich habe ihm nicht sehr lange zugehört, aber …" ~

„Diesen Eindruck hast nicht nur du", sagte Anthony.

Hermine sagte nichts, aber ihre Sorgen verstärkten sich.

Was auch immer mit Harry Potter am Tag des Phönix geschehen war, es hatte ihn verändert. Hermine war sich sicher, dass etwas Neues in ihm steckte. Er war nicht kälter geworden, aber dafür … \emph{härter}. Manchmal ertappte sie ihn dabei, wie er aus einem Fenster blicklos in die Ferne starrte. Dabei lag dann ein Ausdruck grimmiger Entschlossenheit auf seinem Gesicht, der sie beunruhigte. Außerdem erinnerte sie sich nur zu gut an einen Vorfall in der Kräuterkundestunde am Montag. Dabei war eine Venusfeuerfalle außer Kontrolle geraten, worauf Harry Terry aus dem Weg gestoßen hatte, bevor Professor Sprouts Flammenfrostzauber wirken konnte. So weit, so gut - und normal. ~Doch als Harry dann vom Boden aufgestanden und wieder an seinen Platz zurückgegangen war, hatte er den Eindruck vermittelt, als sei gerade nichts wirklich Interessantes passiert.

Und dann war da noch die Sache mit dem letzten Verwandlungstest. Dabei hatte sie ausnahmsweise einmal ein besseres Testergebnis als Harry erzielt. Was sie außer dieser Tatsache an sich irritiert hatte, war Harrys Reaktion gewesen: Er hatte ihr zugelächelt, als wolle er ihr gratulieren, anstatt wie sonst die Zähne zusammenzubeißen. Und sie musste zugeben, dass sie das doch \emph{sehr} verstört hatte.

Wie ließen sich diese Beobachtungen zusammenfassen? Nun, sie hatte irgendwie den Eindruck, dass Harry… naja, sich von ihr zurückzog.

„Er wirkt plötzlich viel \emph{älter}„, unterbrach Anthony Hermines Gedanken. „Zwar nicht wie ein Erwachsener - denn als solchen kann ich mir \emph{Harry} beim besten Willen nicht vorstellen - aber doch so, als hätte sich plötzlich in eine \emph{Viertklässlerversion} von … von \emph{was auch immer} er ist verwandelt.“

"Nun", sagte Padma. Sie tauchte zierlich einen Küchlein mit Schokoladengeschmack in etwas Zuckerguss. "Ich denke, Drachen und Sonnenschein sollten sich bei der nächsten Schlacht besser verbünden, oder Mr. Harry Potter wird uns \emph{zerfetzen}. Letztes Mal haben wir uns verbündet, und selbst da hätte das Chaos fast gewonnen -"

"Ja", sagte Anthony. "Sie haben recht, Miss Patil. Sagen Sie dem Drachengeneral, dass wir uns mit ihm treffen -"

"Nein!", sagte Hermine. "Wir sollten uns nicht gegen General Potter zusammentun \emph{müssen}, nur um eine Chance zu haben. Das macht keinen Sinn, besonders jetzt, wo niemand mehr Muggelsachen benutzen kann. Es sind immer noch vierundzwanzig Soldaten in jeder Armee."

Weder Padma noch Anthony sagten etwas dazu.

Klopf-klopf, klopf-klopf.

„Kommen Sie rein, Mr. Potter", ertönte eine Stimme aus dem Inneren des Büros.

Die Tür öffnete sich knarrend und Harry Potter schlüpfte durch die Öffnung in ihr Büro. Dann zog er die Tür mit einer Hand leise hinter sich zu und setzte sich wortlos in den gepolsterten Stuhl, der vor dem Schreibtisch stand. Sie hatte den Stuhl so oft verwandelt, dass er manchmal von selbst die Form wechselte, um ihre Stimmung widerzuspiegeln. Dafür musste sie weder ihren Zauberstab bewegen noch eine Beschwörungsformel nutzen und nicht einmal eine bewusste Absicht haben.

Harry schien das nicht zu bemerken. Er strahlte eine stille Entschlossenheit aus und hielt ihrem Blick stand. „Sie haben mich gerufen?"

„Das habe ich", bestätigte Professor McGonagall. „Denn ich habe zwei gute Nachrichten für Sie, Mr. Potter. Doch dazu muss ich ein wenig ausholen. Zunächst: Haben Sie Mr. Rubeus Hagrid, unseren Wildhüter, überhaupt schon kennengelernt? Er war ein alter Freund Ihrer Eltern."

Harry zögerte kurz. Dann sagte er: „Mr. Hagrid hat kurz nach meiner Ankunft mit mir gesprochen. Ich glaube, es war am Dienstag in meiner ersten Schulwoche. Dabei hat er aber nicht erwähnt, dass er meine Eltern kannte. Ich dachte daher, er wollte sich einfach nur dem "Jungen, der lebte" vorstellen. Meinen Sie, er hatte dabei irgendwelche Hintergedanken? Er \emph{schien} mir nicht der Typ dafür zu sein."

„Ähm…„ Es dauerte einen Moment, bis McGonagall ihre Gedanken sortiert hatte. „Die Sache mit Mr. Hagrid ist eine lange Geschichte, Mr. Potter, aber hier die Kurzfassung: Vor mittlerweile fünf Jahrzehnten wurde Hagrid fälschlicherweise beschuldigt, einen Studenten ermordet zu haben. Daraufhin wurde sein Zauberstab zerbrochen und er wurde von der Schule verwiesen. Als Professor Dumbledore etliche Jahre später Schulleiter wurde, gab er dem damals arbeitslosen Mr. Hagrid einen Job als Hüter der Ländereien und Schlüssel von Hogwarts.“

Harry beobachtete seine Lehrerin aufmerksam. „Ich erinnere mich, wie Sie mir erklärten, dass vor fünf Jahrzehnten zum letzten Mal ein Schüler in Hogwarts gestorben sei. Bei diesem Gespräch zeigten Sie sich außerdem überzeugt, dass vor besagten fünf Jahrzehnten auch zum letzten Mal jemand die geheime Botschaft des Sprechenden Hutes gehört hatte.“

McGonagall verspürte plötzlich ein leichtes Frösteln. Selbst der Schulleiter oder Severus hätten diese Verbindung nicht so schnell hergestellt, da war sie sich sicher. „Richtig, Mr. Potter. Es geht hier tatsächlich um die gleichen Vorkommnisse. Jemand hatte die Kammer des Schreckens geöffnet, was aber niemand glaubte und Mr. Hagrid wurde für den daraus resultierenden Todesfall verantwortlich gemacht.

Nun aber zu den eigentlichen Neuigkeiten: Dem Schulleiter ist es jüngst gelungen, die zusätzliche Verzauberung auf dem Sprechenden Hut ausfindig zu machen und er hat seine Erkenntnisse umgehend einer speziellen Kammer des Zauberergamots präsentiert. Infolgedessen wurde das Urteil gegen Mr. Hagrid aufgehoben und ihm wurde gestattet, sich einen neuen Zauberstab zu besorgen. Dieses Urteil ist absolut aktuell. Es wurde erst heute Morgen verkündet." McGonagall zögerte kurz. „Und … Wir haben Mr. Hagrid noch nichts davon erzählt, Mr. Potter. Wir wollten warten, bis das Urteil tatsächlich rechtskräftig ist, um ihm nach so langer Zeit keine falschen Hoffnungen zu machen. An diesem Punkt kommen Sie ins Spiel. Wir haben uns gefragt, wir Mr. Hagrid mitteilen sollten, dass Sie es waren, der ihm geholfen hat."

Die Verwandlungslehrerin betrachtete Harry aufmerksam. Sein Blick sah aus, als würde er in Gedanken gerade verschiedene Argumente gegeneinander abwägen.

„Ich kann mich erinnern, wie Mr. Hagrid Sie als Baby gehalten hat", fuhr McGonagall fort. „Und ich glaube daher, dass er sich sehr freuen würde, von diesem Umstand zu erfahren."

Sie konnte ihn in Harrys Miene sehen. Den Moment, indem er entschied, dass Rubeus ihm nichts nützen würde.

Kurz darauf schüttelte Harry wie erwartet den Kopf. „Es ist schlimm genug, dass die Vermutung aufkommen könnte, es gäbe einen Parselmund in der diesjährigen Schülergeneration", erklärte Harry. „Daher finde ich es klüger, alles so geheim wie möglich zu halten.„

McGonagall dachte einen Moment an James und Lily. Die beiden hatten nie gezögert, die Freundschaft des riesigen, ein wenig plumpen Mannes zu erwidern. Und dass, obwohl James der Erbe eines wohlhabenden Hauses war und Lily im Bereich der Zauberkünste brillierte. ~Dass Rubeus bloß ein Halbriese war, dessen Zauberstab zerbrochen wurde, hatte sie nie gekümmert.

Dann kehrte sie in die Gegenwart zurück. „Sie möchten nicht, das Mr. Hagrid es erfährt, weil Sie nicht erwarten, dass er sich für Sie als nützlich erweisen könnte“, stellte sie fest. „Ist es nicht so, Mr. Potter?"

Eine Weile herrschte Schweigen. Eigentlich hatte sie das nicht laut sagen wollen.

McGonagall sah zu, wie sich eine gewisse Traurigkeit auf Harrys Gesicht ausbreitete. „Wahrscheinlich", antwortete Harry schließlich leise. „Aber er und ich würden vermutlich auch nicht gut miteinander auskommen, oder?"

McGonagall unterdrückte den plötzlichen Drang, zu husten. Etwas schien in ihrem Hals steckengeblieben zu sein.

„Wo wir gerade von für unsere Zwecke nützlichen Menschen sprechen„, erlöste Harry sie in diesem Augenblick glücklicherweise von ihren Gedanken. „Alles deutet darauf hin, dass wir uns bald im Krieg mit einem Dunklen Lord wiederfinden. Da ich nun schonmal in Ihrem Büro bin, möchte ich die Gelegenheit nutzen und Sie bitten, meinen Schlafzyklus auf dreißig Stunden pro Tag verlängern zu lassen. Außerdem möchte ich Sie darüber informieren, dass Neville Longbottom sich dazu entschlossen hat, Unterricht im Duellieren zu nehmen. Es gibt einen älteren Hufflepuff, der ihn unterrichten möchte. ~Die beiden haben mich eingeladen, daran teilzunehmen und dieses Angebot werde ich gerne annehmen.

Darüber hinaus gibt es selbstverständlich noch viele andere Dinge, die ich ebenfalls gerne lernen möchte. Wenn Sie oder der Schulleiter meinen, ich sollte mich in nächster Zeit einem bestimmten Thema widmen, um meinem Ziel, später ein mächtiger Zauberer zu werden, näherzukommen, lassen Sie es mich bitte wissen.

Sie sehen also: Je früher Sie Madam Pomfrey anweisen, mir den entsprechenden Trank zu verabreichen oder was auch immer sie - “

\emph{„Mr. Potter!"}

Harry sah sie direkt an. „Ja Minerva? Ich weiß, es war nicht Ihre Idee, mich in dieser Weise zu nutzen. Aber ich habe die Absicht, den Gebrauch, den der Schulleiter von mir macht, zu überleben. Und ich wäre Ihnen sehr verbunden, wenn Sie dem keine Hindernisse entgegenstellen würden."

Das machte sie fertig. „Harry", flüsterte sie mit matter Stimme, „Kinder sollten nicht so \emph{denken} müssen, wie du das tust."

„Da haben Sie Recht, das sollten sie wohl nicht„, bestätigte Harry. „Aber bedenken Sie bitte, dass \emph{viele} Kinder zu früh erwachsen werden müssen. Das betrifft bei weitem nicht nur mich. Und die meisten anderen Kinder, die das betrifft, würden wahrscheinlich nur zu gerne mit mir tauschen wollen. Solange es da draußen Menschen gibt, die in echten Schwierigkeiten stecken, werde ich mich nicht selbst bemitleiden, Professor McGonagall.“

McGonagall schluckte mühsam. „Aber Mr. Potter, habe Sie bedacht, dass Sie mit dreißig Stunden pro Tag schneller \emph{altern} werden?" -- ‚So wie \emph{Albus`}, fügte sie in Gedanken hinzu.

„Natürlich. In meinem fünften Jahr werde ich physiologisch etwa so alt sein wie Hermine", entgegnete Harry. „Diese Aussicht scheint mir nicht \emph{allzu} schlimm zu sein.„ Harry warf seiner Lehrerin ein schiefes Grinsen zu. „Um ehrlich zu sein: Ich würde das wahrscheinlich auch anstreben, wenn es \emph{keinen} Dunklen Lord gäbe. Zauberer leben schließlich bereits heutzutage eine ganze Weile und im nächsten Jahrhundert wird sich diese Zeitspanne vermutlich noch weiter verlängern. Lediglich die Frage, ob Zauberer oder Muggel dafür verantwortlich sein werden, kann ich noch nicht beantworten. Meiner Meinung nach gibt es daher jedenfalls keinen Grund, \emph{nicht} so viele Stunden wie möglich in einen Tag zu packen. Ich habe einige Dinge vor und es wäre gut, wenn sie schnell erledigt würden.“

Es entstand eine lange Pause.

„In Ordnung", sagte Minerva dann leise. Sie räusperte sich und wiederholte mit erhobener Stimme: „In Ordnung, Mr. Potter, ich werde den Schulleiter fragen. Wenn er einverstanden ist, soll es so geschehen.„

Harrys Augen verengten sich für einen Moment. „Ich verstehe. Dann bitte ich Sie, den Schulleiter bei diesem Gespräch an Godric Gryffindors letzte Worte zu erinnern. Er sagte, wenn es das Richtige für ihn gewesen war, dass er dann niemandem empfehlen würde, falsch zu wählen. Und bei diesem Ratschlag schloss er ausdrücklich auch die jüngsten Schüler in Hogwarts mit ein.“

McGonagall wurde bewusst, dass sich in diesem Augenblick jede Chance, dass Albus die aktuellen Entwicklungen stoppen würde, in Nichts aufgelöst hatte. Diese Erkenntnis hinterließ einen schalen Beigeschmack. Schließlich hatte niemand anders als Albus selbst bereits mehrfach Godrics Worte zitiert. Damals, als sie eingewandt hatte, dass Cameron Edward zu jung sei, genauso wie damals, als sie das gleiche Argument bei Peter Pevensie wiederholt hatte. Daraufhin hatte sie es grundsätzlich aufgegeben, gegen Albus Entscheidungen zu protestieren.

„Wer hat Ihnen davon erzählt, Mr. Potter?" \emph{Hoffentlich war es nicht Albus gewesen … Albus würde das doch wohl niemals zu einem Schüler sagen} \emph{-}

„Ich habe in letzter Zeit sehr viel gelesen", sagte Harry. Dann befreite er sich aus dem ihn noch ~immer umhüllenden Stuhl, dann hielt er inne. "Darf ich nach der zweiten guten Nachricht fragen?"

"Oh", sagte sie. "Ah - Professor Quirrell ist aufgewacht und sagt, du darfst -"

Der Krankenflügel von Hogwarts war ein heller, offener Raum, der auf allen vier Seiten von Licht durchflutet war, obwohl er mitten im Schloss zu liegen schien. Weiße Betten in langen Reihen dehnten sich aus, nur drei von ihnen waren im Moment besetzt. Ein älterer Junge und ein älteres Mädchen auf gegenüberliegenden Seiten, beide lagen bewegungslos mit geschlossenen Augen, wahrscheinlich bewusstlos und mit einem Zauber belegt, während irgendein Heilzauber oder -trank ihre Körper auf unangenehme Weise umgestaltete; und der dritte Insasse hatte den Vorhang um sein Bett gezogen, was vermutlich gut war. Madam Pomfrey hatte ihm einen Schubs gegeben und ihm gesagt, er solle nicht gaffen, und Harry hatte sich selbst streng daran erinnern müssen, dass einige Leute immer noch nicht wussten, wer der Junge, der lebte war - entweder das, oder Madam Pomfreys Identität war mit ihrer absoluten Herrschaft über ihr eigenes Krankenhaus und so weiter, verbunden, was auch immer.

Hinter den Bettenreihen befanden sich fünf Türen, die in die Privaträume führten, in denen die Patienten untergebracht waren, die nicht Stunden, sondern Tage bleiben würden, deren Zustand aber eine Verlegung nach St. Mungos nicht rechtfertigte.

Fensterlos, unbeleuchtet bis auf eine einzige rauchlose Fackel an einer der massiven Steinwände; das war der Raum hinter der mittleren Tür. Harry hatte sich gefragt, ob Professoren Hogwarts bitten konnten, sich zu verändern; oder ob die Krankenstation immer einen solchen Raum zur Verfügung hatte, für Leute, die das Licht nicht mochten.

In der Mitte des Raumes, zwischen zwei identischen Beistelltischen, die aus demselben grauen Marmor wie die Wände geschnitzt zu sein schienen, stand ein weißes Krankenhausbett, das im Licht der nicht rauchenden Fackeln vage orangestichig aussah; und in diesem Bett, ein weißes Laken zu den Oberschenkeln hochgezogen und mit einem Krankenhauskittel bekleidet, saß Professor Quirrell, den Rücken leicht gegen das Kopfteil des Bettes gelehnt.

Es hatte etwas Beängstigendes, Professor Quirrell in einem von Madam Pomfreys Betten zu sehen, auch wenn der Verteidigungsprofessor unverletzt schien. Auch wenn er wusste, dass Professor Quirrell seine eigene scheinbare Niederlage gegen Severus absichtlich arrangiert hatte, um sich selbst einen Vorwand zu geben, um nach Askaban wieder zu Kräften zu kommen. Harry hatte noch nie \emph{selbst} jemanden in einem Krankenhausbett sterben sehen, aber er hatte zu viele Filme gesehen. Es war eine Andeutung von Sterblichkeit, und der Verteidigungsprofessor sollte \emph{nicht} sterblich sein.

Madam Pomfrey hatte Harry gesagt, dass es ihm absolut verboten sei, ihren Patienten zu nerven.

Harry hatte gesagt: "Ich verstehe", was technisch gesehen nichts über Gehorsam aussagte.

Die strenge alte Heilerin hatte sich dann umgedreht und begonnen, zu Professor Quirrell zu sagen, dass er sich auf keinen Fall überanstrengen oder… aufregen dürfe…

Madam Pomfrey war ins Stocken geraten, hatte sich eilig umgedreht und war aus dem Raum geflohen.

"Nicht schlecht", bemerkte Harry, nachdem sich die Tür hinter der flüchtenden Oberschwester geschlossen hatte. "Das muss ich auch noch lernen, irgendwann."

Professor Quirrell lächelte ein Lächeln, das absolut keinen Humor enthielt, und sagte, wobei seine Stimme ein gutes Stück trockener klang als sonst: "Danke für Ihre Kunstkritik, Mr. Potter."

Harry starrte in die blassblauen Augen und fand, dass Professor Quirrell…

… älter aus.

Es war subtil, vielleicht war es nur Harrys Einbildung, vielleicht war es die schlechte Beleuchtung. Aber das Haar über Quirinus Quirrells Stirn könnte sich ein wenig gelichtet haben, was übriggeblieben war, könnte dünner und grauer geworden sein, ein Fortschreiten der Kahlheit, die bereits an seinem Hinterkopf sichtbar war. Das Gesicht könnte ein wenig eingefallen sein.

Die blassblauen Augen waren scharf und intensiv geblieben.

"Ich bin froh", sagte Harry leise, "Sie in scheinbar bester Gesundheit zu sehen."

"Der Schein kann natürlich trügen", sagte Professor Quirrell. Er schnippte mit den Fingern, und als seine Hand die Geste beendete, hielt er seinen Zauberstab. "Würden Sie glauben, dass diese Frau denkt, sie hätte ihn mir abgenommen?"

Dann sprach der Verteidigungsprofessor sechs Beschwörungsformeln; sechs von den dreißig, die er benutzt hatte, um ihre wichtigen Gespräche in Marys Zimmer zu schützen.

Harry hob die Augenbrauen, leise fragend.

"Das ist alles, was ich im Augenblick zustande bringe", sagte der Verteidigungsprofessor. "Ich nehme an, dass es sich als ausreichend erweisen wird. Dennoch gibt es ein Sprichwort: Wenn du nicht willst, dass man etwas hört, dann sag es nicht. Betrachten Sie es in vollem Umfang als zutreffend. Wie ich höre, wollten Sie mich sprechen?"

"Ja", sagte Harry. Er hielt inne, sammelte seine Gedanken. "Hat der Schulleiter oder sonst jemand Ihnen gesagt, dass wir nicht mehr zum Mittagessen gehen können?"

"Etwas in dieser Richtung", sagte der Verteidigungsprofessor. Und ohne seinen Gesichtsausdruck zu verändern: "Natürlich tat es mir furchtbar leid, das zu hören."

"Eigentlich ist es noch viel schlimmer", sagte Harry. "Ich bin auf unbestimmte Zeit an Hogwarts und sein Gelände gebunden. Ich kann nicht ohne Wächter und einen guten Grund weggehen. Ich werde den Sommer über nicht nach Hause gehen, und vielleicht nie wieder. Ich hatte gehofft… mit Ihnen darüber zu sprechen."

Es gab eine Pause.

Der Verteidigungsprofessor stieß einen kurzen Seufzer aus und sagte: "Wir werden uns einfach auf die bekannte Tatsache verlassen müssen, dass die stellvertretende Schulleiterin jeden persönlich ermorden wird, der versucht, mich anzuzeigen. Mr. Potter, ich beabsichtige, dieses Gespräch gradlinig zu führen, so dass wir es schnell beenden können, ist das klar?"

Harry nickte und -

Im Licht der einzelnen Fackel, das zum rötlichen Ende des optischen Spektrums hin schattiert war, reflektierten die grünen Schuppen der Schlange nicht sehr stark, und die blau-weiße Bänderung war kaum zu sehen. Die Schlange erschien dunkel in diesem Licht. Die Augen, die zuvor wie graue Gruben gewirkt hatten, reflektierten nun das Fackellicht und schienen heller als der Rest der Schlange.

"\emph{Sso}", zischte die giftige Kreatur. "\emph{Was wolltesst du denn ssagen?}"

Und Harry zischte: "\emph{Sschulleiter denkt, dass der ehemalige Herr diesser Frau derjenige isst, der ssie auss dem Gefängniss gesstohlen hat.}"

Harry \emph{hatte} diesmal nachgedacht, und zwar sorgfältig, bevor er beschlossen hatte, Professor Quirrell \emph{nur} zu verraten, dass der Schulleiter das glaubte; und \emph{nichts} über die Prophezeiung zu sagen, die Voldemort auf Harrys Eltern gebracht hatte, und auch nicht, dass der Schulleiter den Orden des Phönix neu gründete … es war ein Risiko, ein erhebliches Risiko, aber Harry brauchte einen Verbündeten in dieser Sache.

"\emph{Er glaubt, dass der am Leben ist?}", sagte die Schlange schließlich. Die geteilte, zweizinkige Zunge flackerte schnell von einer Seite zur anderen, ein sardonisches Schlangenlachen. "\emph{Irgendwie bin ich nicht überrascht.}"

"\emph{Ja}", zischte Harry trocken, "\emph{ssehr amüsant, da bin ich mir ssicher. Nur dass ich jetzt für die nächsten ssechs Jahre in Hogwartss fesstsitze, zur Ssicherheit! Ich habe beschlossen, dass ich in der Tat nach Macht sstreben werde; und dafür isst die Gefangenschaft nicht hilfreich. Ich muss den Sschulleiter davon überzeugen, dass der Dunkle Lord noch nicht erwacht ist, dasss dass Entkommen das Werk einer anderen Macht war -"}

Wieder das schnelle Flackern der Schlangenzunge; das schlangenhafte Lachen war diesmal stärker, trockener. "\emph{Amateurhafte Torheit.}"

"\emph{Pardon?} ", zischte Harry.

"\emph{Du ssiehst einen Fehler, denkst an Rückgängigmachen, an das Zurücksetzen der Zeit auf den Anfang. Doch nicht einmal mit Ssanduhren kann man die Zeit rückgängig machen. Man muss stattdessen vorwärtss gehen. Du denkst daran, andere zu überzeugen, dass ssie einen Fehler gemacht haben. Es isst viel einfacher, ssie zu überzeugen, dass ssie recht haben. Also überleg mal, Junge: Welche neue Begebenheit würde Sschulleiter dazu bringen, zu entscheiden, dass du wieder ssicher bist, ssimultan deine anderen Agenden vorantreiben?}"

Harry starrte die Schlange verwirrt an. Sein Verstand versuchte, das Rätsel zu begreifen und zu entschlüsseln -

"\emph{Isst ess nicht offenssichtlich?}", zischte die Schlange. Wieder züngelte die Schlange ein sardonisches Lachen. "\emph{Um dich zu befreien, um die Macht in Britannien zu erlangen, musst du wieder dabei gessehen werden wie du den Dunklen Lord bessiegst.}"

Im rötlich-orange flackernden Fackellicht wiegte sich eine grüne Schlange über einem weißen Krankenhausbett, während der Junge in die Glut ihrer Augen starrte.

"\emph{Alsso}", sagte Harry schließlich. "\emph{Lassen Ssie mich dass noch mal zusammenfassen. Ssie schlagen vor dass wir einen Doppelgänger engagieren, der sich alss der dunkle Lord aussgibt?}"

"\emph{Ssowas in der Art. Die Frau, die wir gerettet haben, wird kooperieren, ssollte am überzeugendsten sein, wenn ssie an seiner Seite sseht.}" Noch mehr sardonisches Zungenflackern. "\emph{Du wirst aus Hogwarts an einen öffentlichen Ort entführt, viele Zeugen, Schutzzauber halten Wächter fern. Der dunkle Lord verkündet, dass er endlich seine physische Form wiedererlangt hat, nachdem er jahrelang als Geist umhergezogen ist; er sagt, dass er noch größere Macht erlangt hat, nicht einmal du könntest ihn jetzt aufhalten. Er bietet dir ein Duell an. Du zauberst einen Patronus, der Dunkle Lord lacht dich aus und sagt, er sei kein Lebensfresser. Er wirft einen Tötungsfluch auf dich, du blockst, Beobachter sehen dunklen Lord explodieren -"}

"\emph{Er zaubert einen Tötungsfluch?}", zischte Harry ungläubig. "\emph{Auf mich? Schon wieder? Zum zweiten Mal? Niemand wird glauben, dass der Dunkle Lord so dumm sein könnte -}"

"\emph{Du und ich sind die einzigen beiden Menschen im Land, die das bemerken würden}", zischte die Schlange. "\emph{Verlass dich drauf, Junge.}"

"\emph{Was ist, wenn es ein drittes Mal gibt, irgendwann?} "

Die Schlange schwankte nachdenklich. "\emph{Ich könnte ein anderes Drehbuch für die Ausführung schreiben, wenn du willst. Welches Szenario auch immer, es sollte die Möglichkeit offen lassen, dass der Dunkle Lord noch einmal zurückkehren könnte - das Volk muss denken, dass es immer noch darauf angewiesen ist, dass du es beschützt.}"

Harry starrte in die rot flackernden Gruben der Schlangenaugen.

"\emph{Nun?}", zischte die schwankende Gestalt.

Der offensichtliche Gedanke war, dass ein \emph{zweites} Mal auf die Intrigen und Täuschungen des Verteidigungsprofessors einzugehen, eine noch \emph{kompliziertere} Lüge zu spinnen, um den ersten Fehler zu vertuschen, und eine \emph{weitere} tödliche Schwachstelle zu schaffen, falls jemand jemals die Wahrheit herausfinden würde, wäre \emph{exakt} dieselbe Art von Dummheit, als wenn der vermeintliche Dunkle Lord den Tötungsfluch erneut anwenden würde. Es bedurfte nicht einmal seiner Hufflepuff-Seite, um darauf hinzuweisen, Harry dachte das mit seiner eigenen mentalen Stimme.

Aber es stellte sich auch die Frage, ob es die angemessene Moral war, die aus der letzten Erfahrung zu lernen war, immer sofort \emph{Nein} zum Verteidigungsprofessor zu sagen, oder…

"\emph{Werde darüber nachdenken}", zischte Harry. "\emph{Will nicht gleich antworten, will diesmal zuerst Rissiken und Vorteile sspezifizieren -}"

"\emph{Einverstanden}", zischte die Schlange. "\emph{Aber merke dir eins, Junge, andere Ereignisse finden ohne dich statt. Zögern ist immer leicht, selten nützlich.}"

Der Junge trat aus dem Privatzimmer in die Hauptkrankenstation und fuhr sich nervös mit den Fingern durch sein unordentliches schwarzes Haar, während er an den weißen Betten vorbeiging, den belegten und den unbelegten.

Kurz darauf verließ der Junge den Hogwarts-Krankenflügel ganz und ging mit einem zerstreuten Nicken an Madam Pomfrey vorbei nach draußen.

Der Junge ging hinaus in einen Korridor, dann in einen größeren Korridor, und blieb dann stehen und lehnte sich an die Wand.

Die Sache war die…

… er wollte wirklich \emph{nicht} die nächsten sechs Jahre in Hogwarts festsitzen; und wenn man darüber nachdachte…

… der Vorfall mit der Rettung von Bellatrix aus Askaban war nicht \emph{nur} für Harry eine Zumutung. Andere Leute würden sich Sorgen machen, in Angst vor der Rückkehr des Dunklen Lords leben, unbekannte Ressourcen aufwenden, um unbekannte Vorsichtsmaßnahmen zu treffen. Harry könnte verlangen, dass sie das Drehbuch so schreiben, dass es \emph{nicht} plausibel erschien, dass der Dunkle Lord ein drittes Mal zurückkehren würde. Und dann würden sich die Leute entspannen, es wäre alles vorbei.

Es sei denn, es gäbe \emph{tatsächlich} einen dunklen Lord, vor dem man sich fürchten müsste. Es \emph{hatte} eine Prophezeiung gegeben.

Der Junge, der an der Wand lehnte, stieß einen leisen Seufzer aus und begann weiter zu gehen.

Harry hatte es fast vergessen, aber er war dazu gekommen, Professor Quirrell das Kartenspiel zu zeigen, das er am Sonntagabend vom "Weihnachtsmann" bekommen hatte, und in dem der Herzkönig angeblich ein Portschlüssel war, der ihn zum Hexeninstitut von Salem in Amerika bringen würde. Natürlich hatte Harry Professor Quirrell weder gesagt, \emph{wer} ihm die Karte geschickt hatte, noch was sie bewirken \emph{sollte}, bevor er Professor Quirrell gefragt hatte, ob es möglich sei zu sagen, wohin der Portschlüssel ihn schicken würde.

Der Verteidigungsprofessor hatte sich in seine menschliche Gestalt zurückverwandelt und untersuchte den Herzkönig, indem er ein paar Mal mit seinem Zauberstab darauf tippte.

Und laut Professor Quirrell …

… würde der Schlüssel den Benutzer irgendwo nach London schicken, aber er konnte es nicht näher eingrenzen.

Harry hatte Professor Quirrell den Zettel gezeigt, der dem Kartenspiel beilag, und nichts von den früheren Zetteln gesagt.

Professor Quirrell hatte einen Blick drauf geworfen, ein trockenes Kichern von sich gegeben und bemerkt, dass, wenn man den Zettel \emph{genau} las, dort nicht \emph{explizit} stand, dass der Portschlüssel ihn zum Hexeninstitut von Salem bringen würde.

Man musste lernen, auf diese Art von Feinheiten zu achten, sagte Professor Quirrell, wenn man ein mächtiger Zauberer werden wollte, wenn man erwachsen wurde; oder, in der Tat, wenn man überhaupt erwachsen werden wollte.

Der Junge seufzte wieder, als er sich zum Unterricht schleppte.

Er begann sich zu fragen, ob alle anderen Zaubererschulen auch so waren, oder ob nur Hogwarts ein Problem hatte.

