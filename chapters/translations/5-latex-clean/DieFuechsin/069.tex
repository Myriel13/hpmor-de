

\hypertarget{selbstverwirklichung-teil-4}{% \section{37. Selbstverwirklichung, Teil 4}\label{selbstverwirklichung-teil-4}}

-\/-\/-\/-\/- Kapitel 69: Selbstverwirklichung, Teil 4 -\/-\/-\/-\/-

Aus dem Augenwinkel sah Hermine Granger es: eine Reflexion auf dem polierten Metall einer Statue an der Kreuzung zweier Korridore, ein Aufblitzen von Gold, ein Aufblitzen von Rot, so etwas wie ein Bild von Feuer; nur für einen Moment sah sie es, und dann war es weg.

Sie hielt inne, verwirrt, und \emph{fast} wäre sie weitergegangen, aber da war etwas Vertrautes an diesem kurzen Aufleuchten gewesen -

Hermine ging vorwärts, dorthin, wo die Statue stand, blickte den Korridor entlang, von dem die feurige Reflexion gekommen sein könnte.

Schwach, wie von einem fernen Ort, hörte sie den Schrei, den Ruf.

Hermine begann zu rennen.

Sie rannte eine Weile; jedes Mal, wenn sie an eine Kreuzung kam, hielt sie inne, holte so viel Atem, wie sie konnte, und dann sah sie einen Feuerschein, der aus der einen oder anderen Richtung reflektiert wurde, oder sie hörte diesen fernen Ruf. Hätte sie nicht an dem Armeetraining von Professor Quirrell teilgenommen, wäre sie nach dem Gerenne vor Erschöpfung umgefallen.

Sie sah den Phönix kein einziges Mal.

Und dann kam sie an eine Abzweigung, und da war \emph{nichts}, kein Zeichen, sie wartete lange Sekunden, und sie hörte keinen Schrei und sah kein Feuer, und sie begann sich gerade mit einem kranken, traurigen Gefühl zu fragen, ob sie sich das Ganze eingebildet hatte, als sie eine \emph{Person} schreien hörte.

Während ihre Füße rasend schnell um die Ecke bogen, erfasste ihr Kopf die ganze Szene mit einem Blick: drei riesige Jungen in grün-gesäumten Umhängen, die sich bereits zu ihr umdrehten, und ein kleinerer, gelb gekleideter Junge, der an einem Fuß in der Luft baumelte, von einer unsichtbaren Hand hochgehalten.

Die Sonnenschein-Generalin entwarf keinen ausgefeilten Plan. Leute, die zögerten um nachzudenken, waren nicht geeignet für einen Überraschungsangriff.

Sie hielt ihren Zauberstab in der Hand, ihre Finger machten eine Drehung und ihre Lippen formten das erste "\emph{Somnium}! " und der größte Schläger kippte um, der Hufflepuff-Junge fiel mit einem Knall aus der Luft und die anderen beiden Jungen versuchten, mit ihren Zauberstäben auf sie zu zielen, doch sie sagte erneute "\emph{Somnium}! " und ein weiterer großer Junge kippte um - sie traf denjenigen, der schneller mit seinem Zauberstab gezielt hatte.

Leider war es selbst für sie schwer, zwei Schlafzauber hintereinander zu wirken, und sie konnte keinen dritten abschießen, bevor -

Der letzte Raufbold schrie "\emph{Protego!}" und wurde von einem schimmernden blauen Leuchten umgeben.

Vor vierundzwanzig Stunden wäre Hermine bei so etwas in Panik geraten, ein \emph{echter} Schutzzauber würde den Jungen auch dann noch Zauber auf sie wirken lassen, wenn er geschützt war.

\emph{Jetzt} hatte sie -

"\emph{Stupor!}", schrie der Flegel.

Der karmesinrote Blitz schoss mit einem furchtbaren Glanz auf sie zu und leuchtete viel heller als jeder Zauber, der Harrys Zauberstab entsprungen war.

Hermine wich leicht nach links aus, und der Bolzen verfehlte sie, denn der Junge hatte nicht \emph{annähernd} so gut gezielt wie Harry; und ihr kam der Gedanke, dass es vielleicht keine Überscheidung zwischen Schlägertypen und Professor Quirrells Armeen gab.

\emph{\emph{"Stupor! "}, rief der Junge wieder. \emph{"Expelliarmus! Stupor! "}}

Wie auch immer, \emph{jetzt} hatte sie gerade eine ganze Stunde damit verbracht, an all die \emph{anderen} Zaubersprüche zu denken, die sie auf Harry und Neville hätte anwenden können -

"\emph{Jellyfy}! ", brüllte der Schläger, ein Fluch mit breitem Lichtstrahl, dem sie nicht ausweichen konnte, und ihre Knie fühlten sich plötzlich fast zu schwach an, um sie zu stützen. Und dann, mit einem wütenden Brüllen, das einen noch helleren Lichtblitz erzeugte, „\emph{Stupor!} "

Dem wich sie aus, indem sie sich absichtlich fallen ließ, und bis dahin hatte sie sich genug für den nächsten Zauber erholt, der da war -

"\emph{Glisseo}", sagte Hermine und richtete ihren Zauber auf den Boden.

"Uff", sagte der Junge, als ihm die Füße unterm Hintern wegglitten und \emph{er tatsächlich seinen Zauberstab fallen ließ}.

Der \emph{Protego} erlosch.

"\emph{Somnium}", sagte Hermine.

Sie atmete immer noch keuchend, als sie zu dem Hufflepuff-Jungen hinüberkroch, der sich aufsetzte, stöhnte und sich den Schädel rieb, da er kopfüber auf den Boden gefallen war; Gut, dass er kein Muggel gewesen war, überlegte sich Hermine, sonst hätte er sich das Genick brechen können. Daran hatte sie gar nicht gedacht.

"Äh", sagte der Junge. Sein Haar hatte eine Farbe, die man 'brünett' genannt hätte, wenn er ein Mädchen gewesen wäre, seine Augen ein undifferenziertes Braun, das irgendwie genau richtig für Hufflepuff schien, es waren keine Tränen in seinem Gesicht, aber er sah irgendwie blass aus. Sie schätzte ihn etwa auf das vierte oder dritte Schuljahr.

Dann weiteten sich die braunen Augen, als er sie anblickte. "\emph{General Sonnenschein?}"

"Ja", sagte sie. "Das bin (\emph{keuch}) ich." Wenn der Hufflepuff-Junge irgendetwas darüber sagte, dass sie die Geliebte von Harry Potter war, dann würde sie ihn umbringen.

"Wow", sagte der Hufflepuff-Junge. "Das war - du hast gerade - ich meine, ich habe dich vor Weihnachten auf den Bildschirmen gesehen, aber - wow! Ich kann nicht glauben, dass du das gerade getan hast!"

Es gab eine Pause.

\emph{Ich kann nicht glauben, dass ich das gerade getan habe}, dachte Hermine Granger, die sich auf einmal ein wenig schwach fühlte, das musste das ganze Laufen gewesen sein. "Entschuldige (keuch) mich", sagte sie, "könntest du den Zauber auf meinen Beinen entfernen?"

Der Junge nickte, drückte sich auf die Beine und griff in seinen Umhang nach seinem Zauberstab; aber Hermine musste seine Bewegungen ein paar Mal korrigieren, bevor der Gegenzauber richtig funktionierte.

"Ich bin Michael Hopkins", sagte der Junge, nachdem Hermine wieder auf ihre eigenen Füße gerollt war. Er streckte seine Hand aus. "Oder einfach nur Mike in Hufflepuff, es gibt in diesem Jahr keine anderen Mikes in ganz Hufflepuff, kannst du dir das vorstellen?"

Sie schüttelten sich die Hand, und Mike sagte: "Wie auch immer, \emph{vielen Dank}."

Hermine war nicht auf den Rausch der Euphorie vorbereitet, der sie in diesem Moment überkam, jemanden auf diese Weise zu retten, fühlte sich buchstäblich besser an als alles, was sie in ihrem \emph{gesamten} \emph{Leben} je empfunden hatte.

Sie drehte sich um und sah die schlafenden, älteren Jungen an.

Sie waren sehr groß und sahen, so schätzte sie, etwa fünfzehn Jahre alt aus, und ihr wurde plötzlich klar, wie \emph{groß} der Unterschied war, zwischen Hogwarts-Schülern, die sich für alle außerschulischen Aktivitäten von Professor Quirrell angemeldet hatten, und Schülern, die jahrelang von den schlimmsten Professoren aller Zeiten unterrichtet worden waren. Die Fähigkeit, Dinge zu \emph{treffen}, auf die man zielt, zum Beispiel; oder die Fähigkeit, mitten in einem Kampf klar genug zu denken, um zu erkennen, dass man seine gefallenen Verbündeten mir \emph{Rennervate} wieder aufwecken sollte. Und andere Dinge, die Professor Quirrell gesagt hatte, wie zum Beispiel, dass in der realen Welt fast jeder Kampf durch einen Überraschungsangriff entschieden werden würde, machten plötzlich viel mehr Sinn für sie.

Sie versuchte immer noch, Luft zu holen, und schaute wieder zu Mike.

"Kannst du dir (schnauf) vorstellen", sagte Hermine Granger, "dass ich mich vor fünf Minuten noch gefragt habe, wie man (keuch) ein Held wird?„

Hatte sie wirklich gedacht, sie bräuchte die \emph{Erlaubnis} von jemandem, oder dass Helden herumsaßen und darauf warteten, dass jemand anderes ihnen Quests gab? Es war eigentlich ganz einfach, man ging einfach dorthin, wo das Böse war, das war alles, was man brauchte, um ein Held zu sein. Sie hätte sich daran erinnern sollen, sie hätte keinen Phönix brauchen sollen, der ihr sagte, dass hier in Hogwarts manchmal schlimme Dinge passierten.

Dann blickte Hermine nervös zurück zu den drei älteren Jungen, die bewusstlos dalagen, als ihr klar wurde, dass sie sie \emph{gesehen} hatten, dass sie \emph{wissen} könnten, wer sie war, dass sie sich an sie heranschleichen und \emph{sie} überrumpeln könnten und - und dass sie ihr wirklich wehtun könnten.

Hermine blieb stehen.

Sie erinnerte sich daran, dass Harry Potter sich am ersten Schultag in die Mitte von \emph{fünf} Slytherinschlägern gestellt hatte, als er noch nicht einmal wusste, wie er seinen Zauberstab benutzte.

Sie erinnerte sich daran, dass der Schulleiter gesagt hatte, dass man erwachsen wird, wenn man in Situationen kommt, in denen man erwachsen ist, und dass die meisten Menschen in einem beengenden Kreis aus Angst leben.

Und sie erinnerte sich an Professor McGonagalls Stimme, die meinte: “Sie \emph{sind} zwölf."

Hermine holte tief Luft, einmal, zweimal und dreimal.

Sie fragte Mike, ob er in Madam Pomfreys Büro gehen müsse, was nicht der Fall war; und ließ sich von ihm die Namen der Slytherin-Jungen sagen, nur für den Fall.

Und dann schlenderte Hermine Granger von dem Haufen bewusstloser Schläger weg, wobei sie darauf achtete, ein Lächeln auf ihr Gesicht zu zaubern, während sie ging.

Sie wusste, dass sie wahrscheinlich früher oder später verletzt werden würde. Aber wenn man zu viel Angst davor hatte, verletzt zu werden, um das Richtige zu tun, dann konnte man kein Held sein, so einfach war das. Wenn man ihr in diesem Moment den Sprechenden Hut auf den Kopf gesetzt hätte, hätte er \emph{keine Sekunde} gewartet, bevor er "GRYFFINDOR!" gerufen hätte.

Sie dachte immer noch darüber nach, als sie zum Abendessen herunterkam; die Euphorie, jemanden gerettet zu haben, war immer noch nicht abgeklungen, und sie begann sich zu sorgen, dass etwas in ihrem Gehirn kaputt gegangen war.

Als sie sich dem Ravenclaw-Tisch näherte, brach eine plötzliche Flüsterepidemie aus, und Hermine fragte sich, ob der Hufflepuff-Junge schon etwas gesagt hatte, bevor ihr klar wurde, dass es sich bei dem Geflüster wahrscheinlich nicht \emph{darum} ging.

Sie setzte sich Harry Potter gegenüber, der extrem nervös aussah, wahrscheinlich weil sie immer noch lächelte.

"Äh -", sagte Harry, als sie sich frisch getoastetes Brot, Butter, Zimt, keinerlei Obst oder Gemüse und drei Portionen Schokoladenbrownies vom Buffet nahm. "Äh -" machte er wieder.

Sie ließ ihn so weiterstammeln, bis sie sich ein Glas Grapefruitsaft eingegossen hatte, und dann sagte sie: "Ich habe eine Frage an Sie, Mr~Potter. Was glauben Sie, wie die Menschen daran scheitern, sie selbst zu werden?"

"\emph{Was?}", sagte Harry.

Sie sah ihn an. "Tu so, als wäre nichts vorgefallen", sagte sie, "und sag einfach, was du gestern gesagt hättest."

"Ähm …" Harry sagte und sah sehr verwirrt und besorgt aus. "Ich denke, wir sind schon wir selbst … es ist nicht so, dass ich eine unvollkommene Kopie von jemand anderem bin. Aber ich schätze, wenn ich versuche die Prämisse der Frage anzuwenden, dann würde ich sagen, dass die Menschen nicht zu sich selbst werden, weil wir all dieses verrückte Zeug aus der Umwelt aufnehmen und dann wieder herauswürgen. Ich meine, wie viele Leute, die Quidditch spielen, würden so ein Spiel spielen, wenn sie das Spiel selbst erfunden hätten? Oder in Muggel-Britannien, wie viele Leute, die sich für Labour oder Konservative oder Liberaldemokraten halten, würden genau dieses Bündel an politischen Überzeugungen erfinden, wenn sie sich alles selbst ausdenken müssten?"

Hermine dachte darüber nach. Sie hatte sich gefragt, ob Harry etwas von Slytherin oder vielleicht sogar von Gryffindor sagen würde, aber das schien nicht in die Liste des Schulleiters zu passen; und es kam Hermine in den Sinn, dass es viel mehr Standpunkte zu diesem Thema geben könnte als nur vier.

"Okay", sagte Hermine, "andere Frage. Was macht jemanden zu einem Helden?"

"\emph{Ein Held?}", sagte Harry.

"Ja", sagte Hermine.

"Ah…" sagte Harry. Seine Gabel und sein Messer sägten nervös an einem Stück Steak und schnitten es in immer kleinere Stücke. "Ich denke, viele Leute können Dinge tun, wenn die Welt um sie herum sie dazu bringt… wenn die Leute erwarten, dass sie es tun, oder sie nur Fähigkeiten nutzen, die sie bereits haben, oder es gibt eine Autorität, die zusieht, um ihre Fehler aufzufangen und sicherzustellen, dass sie ihren Teil tun. Aber solche Probleme werden wahrscheinlich schon gelöst, verstehst du, und dann braucht man keine Helden. Ich denke also, dass die Leute, die wir als 'Helden' bezeichnen, selten sind, weil sie sich alles selbst ausdenken müssen, und die meisten Leute fühlen sich damit nicht wohl. Warum fragst du?" Harry stach mit der Gabel in drei Stücke des gründlich zerkleinerten Steaks und hob sie zu seinem Mund hoch.

"Oh, ich habe gerade drei ältere Slytherinschläger betäubt und einen Hufflepuff gerettet", sagte Hermine. "Ich werde eine Heldin sein."

Als Harry damit fertig war, sich an seinem Essen zu verschlucken (einige der anderen Ravenclaws in Hörweite husteten immer noch), sagte er: "\emph{Was?}"

Hermine erzählte die Geschichte, und noch während sie sprach, wurde sie im Flüsterton weitererzählt. (Obwohl sie den Teil über den Phönix ausließ, denn das schien eine private Sache zwischen ihnen beiden zu sein. Hermine hatte es überrascht, als sie im Nachhinein darüber nachdachte, dass ein Phönix für jemanden erscheinen würde, der ein Held sein \emph{wollte}; es schien ein bisschen egoistisch zu sein, wenn sie so darüber nachdachte; aber vielleicht war es den Phönixen egal, solange sie sahen, dass man bereit war, Menschen zu helfen.)

Als sie mit dem Reden fertig war, starrte Harry sie über den Tisch hinweg an und sagte kein einziges Wort.

"Es tut mir leid, wie ich mich vorhin verhalten habe", sagte Hermine. Sie nippte an ihrem Glas Grapefruitsaft. "Ich hätte daran denken sollen, dass es okay ist, wenn du in Verteidigung besser abschneidest, wenn ich dich im Zauberkunstunterricht immer noch fertig mache."

"\emph{Bitte} versteh das nicht falsch", sagte Harry. Er sah jetzt zu erwachsen aus, und grimmig. "Aber bist du dir sicher, dass \emph{du} das bist und nicht, um es ganz offen zu sein, ich?"

"Ich bin mir ganz sicher", sagte Hermine. "Warum, mein Name hat im Englischen praktisch dieselben Buchstaben wie Heldin: heroine, bis auf das zusätzliche 'm', das ist mir bis heute nie aufgefallen."

"Ein Held zu sein ist nicht nur Spaß und Spiel", sagte Harry. "Zumindest echtes Heldentum, wie es Erwachsene machen müssen, es wird nicht so einfach sein."

"Ich weiß", sagte Hermine.

"Es ist schwer und es ist schmerzhaft und man muss Entscheidungen treffen, auf die es keine gute Antwort gibt -"

"Ja, Harry, ich habe diese Bücher auch gelesen."

"Nein", sagte Harry, "du verstehst nicht, auch wenn die Bücher dich warnen, du kannst es nicht verstehen, bis -"

"Das hält dich nicht auf", sagte Hermine. "Es hält dich nicht einmal ein bisschen auf. Ich wette, du hast nie auch nur \emph{in Erwägung gezogen}, deswegen kein Held zu sein. Warum denkst du also, dass es mich aufhalten wird?"

Es gab eine Pause.

Ein plötzliches breites Lächeln erhellte Harrys Gesicht, ein Lächeln, das so hell und jungenhaft war, wie das Stirnrunzeln grimmig und erwachsen gewesen war, und alles war wieder in Ordnung zwischen ihnen.

"Das wird irgendwie furchtbar schief gehen", sagte Harry und lächelte immer noch gewaltig. "Das weißt du doch, oder?"

"Oh ja, das weiß ich", sagte Hermine und lächelte. Sie aß einen weiteren Bissen Toast. "Da fällt mir ein, Dumbledore hat sich geweigert, mein geheimnisvoller alter Zauberer zu sein, kann ich irgendwo hinschreiben, um einen anderen zu bekommen?"

\emph{Nachspiel:}

"… und Professor Flitwick sagt, dass ihre Entscheidung unumstößlich sei", sagte Minerva knapp und starrte den silberbärtigen, alten Zauberer an, der dafür verantwortlich war. Albus Dumbledore saß nur schweigend da und hörte ihr mit einem fernen, traurigen Blick in den Augen zu. "Miss~Granger hat nicht einmal mit der Wimper gezuckt, als Professor Flitwick ihr gedroht hat, sie nach Gryffindor versetzen zu lassen, und hat nur gesagt, dass sie, wenn sie ginge, alle Bücher mitnehmen würde. Hermine Granger hat sich entschlossen, eine Heldin zu werden, und sie akzeptiert kein Nein als Antwort. Ich bezweifle, ob du sie noch mehr dazu hättest drängen können, wenn du es \emph{versucht} hättest…"

Es dauerte ganze fünf Sekunden, bis Minervas Gehirn ihren eigenen letzten Satz verarbeitet hatte.

"\emph{ALBUS!}", kreischte sie.

"Meine Liebe", sagte der alte Zauberer, "nachdem du mit deinem dreißigsten Helden oder so zu tun hattest, wirst du feststellen, dass sie ziemlich vorhersehbar auf bestimmte Dinge reagieren; zum Beispiel, wenn man ihnen sagt, dass sie zu jung sind, oder dass sie nicht dazu bestimmt sind, Helden zu sein, oder dass es unangenehm ist, ein Held zu sein; und wenn du wirklich sicher gehen willst, solltest du ihnen alle drei Dinge sagen. Obwohl", er seufzte kurz "es nicht gut ist, \emph{zu} offensichtlich zu agieren, sonst könnte dich deine stellvertretende Schulleiterin erwischen."

"Albus", sagte Minerva, und ihre Stimme wurde noch fester, "wenn sie verletzt wird, schwöre ich, dass ich dieses Mal -"

"Sie wäre zu gegebener Zeit zur gleichen Entscheidung gekommen", sagte Albus, der ferne traurige Blick immer noch in seinen Augen. "Wenn jemand dazu bestimmt ist, ein Held zu werden, dann wird er nicht auf unsere Warnungen hören, Minerva, egal wie sehr wir uns bemühen. Und in Anbetracht dessen ist es besser für Harry, wenn Miss~Granger nicht zu weit hinter ihm zurückbleibt." Wie aus dem Nichts tauchte eine Dose in Albus Hand auf, die sich öffnete und kleine gelbe Klumpen zum Vorschein brachte. Sie hatte nie herausfinden können, wo er sie aufbewahrte, und sie war auch nicht in der Lage gewesen, die damit verbundene Magie zu bemerken. "Zitronenbonbon?"

"\emph{Sie ist ein zwölfjähriges Mädchen, Albus!}"

Nach-Nachspiel:

Hinter den Fenstern, kaum sichtbar in der abendlichen Düsternis, schwammen Fische im schwarzen Wasser; beleuchtet vom hellen Schein des Slytherin-Gemeinschaftsraums, wenn sie näher kamen oder und langsam in der Dunkelheit verschwindend, wenn sie davonschwammen.

Daphne Greengrass saß in einem bequemen schwarzen Ledersofa, den Kopf in die Hände gestützt, und glühte goldgelb, während um sie herum helle Funken aus weißem Licht aufblitzten.

Sie hatte erwartet, dass die anderen sich darüber lustig machen würden, dass sie Neville Longbottom mochte. Sie hatte erwartet, eine Menge abfälliger Bemerkungen über Hufflepuffs zu hören. Sie hatte sich auf dem Rückweg zu den Slytherin-Kerkern eine ganze \emph{Reihe} von bissigen Kontern dafür ausgedacht.

Sie hatte sich sogar darauf \emph{gefreut}, verhöhnt zu werden, weil sie Neville mochte. Mit so etwas aufgezogen zu werden, bedeutete, dass man zu einem richtigen Mädchen herangewachsen war.

Wie es sich herausstellte, hatte niemand verstanden, dass ihre Herausforderung zu einem Duell bedeutete, dass sie Neville mochte. Sie hatte gedacht, es wäre \emph{offensichtlich}, aber nein, daran hatte anscheinend noch niemand gedacht.

Es waren immer die Flüche, die man nicht sah, die einen trafen.

Sie hätte sich einfach Daphne von Sonnenschein nennen sollen, wie Neville von Chaos. Oder Sunny Daphne, wie Sunny Ron. Oder \emph{irgendetwas} anderes, außer Greengrass von Sonnenschein.

Grünes Gras von Sonnenschein.

Von da an hieß sie nur noch Lady "Sonnenbeschienenes grünes Gras unter blauem Himmel".

Dann hatte jemand schneebedeckte Berge und herumtollende Waldkreaturen hinzugefügt.

Jetzt nannte man sie die funkelnde Einhornprinzessin aus dem edlen und uralten Haus von Glitzerlicht.

Und irgendein verfluchtes Mädchen aus der sechsten Klasse hatte sie mit einem Funkelzauber belegt, und sie hatte nicht einmal gewusst, dass es so etwas wie einen Funkelzauber \emph{gab}, und \emph{Finite} \emph{Incantatem} hatte nicht funktioniert, und sie hatte ältere Mädchen gefragt, von denen sie \emph{dachte}, sie seien ihre Freundinnen (da hatte sie sich offensichtlich geirrt), und dann hatte sie dem Mädchen mit schweren politischen Schäden gedroht, die ihr Vater auf ihr Geheiß hin anrichten würde, und trotzdem saß Daphne Greengrass immer noch im Slytherin-Gemeinschaftsraum, den Kopf in den Händen, funkelte und fragte sich, wie sie zur letzten vernünftigen Person in Hogwarts geworden war.

Es war \emph{nach dem Abendessen} und sie machten \emph{immer noch} \emph{weiter}, und wenn sie bis morgen früh nicht aufhörten, würde sie nach Durmstrang wechseln und die nächste Dunkle Lady werden.

"Hey, Leute!", riefen die Carrow-Zwillinge dramatisch und wedelten mit einer Ausgabe des \emph{Tagespropheten}. "Habt ihr die Nachrichten gehört? Das Zaubergamot hat gerade entschieden, dass 'Zeig, was du drauf hast' eine offizielle Duellherausforderung ist, die so lange ausgefochten wird, bis der Herausforderer sich hinlegt und ein Nickerchen macht!"

"Wie kannst du es wagen, die Ehre der funkelnden Einhornprinzessin zu beleidigen!", rief Tracey. "Zeig, was du drauf hast!" Dann legte sich Tracey flach auf ihr Sofa und begann laut zu schnarchen.

Daphnes funkelnder, glühender Kopf sank weiter in ihre Hände. "Nachdem meine Familie die Macht übernommen hat, werde ich euch alle mit einem Anti-Apparierzauber belegen und mit Floopulver ins Meer zaubern lassen", sagte sie zu niemandem im Besonderen. "Ihr seid alle damit einverstanden, oder?"

\emph{\emph{Klopf-Klopf, Klopf-klopf-klopf, Klopf.}}

Daphne blickte überrascht auf; das war ein Sonnenschein-Code-Signal -

"\emph{Ich höre jemanden klopfen!}", brüllte Mr~Goyle. "\emph{Die} \emph{Tür} \emph{verklopfen!}"

"\emph{Zeig, was du} \emph{drauf hast, Tür!}", rief ein älterer Junge neben der Tür und riss die Tür auf.

Es gab einen Moment der völligen Überraschung.

"Ich bin gekommen, um mit Miss~Greengrass zu sprechen", sagte die Sonnenschein-Generalin und klang dabei so, als ob sie versuchte, selbstbewusst zu klingen. "Könnte bitte jemand -"

Dem Gesichtsausdruck von Hermine nach zu urteilen, hatte sie gerade bemerkt, dass Daphne funkelte.

Und \emph{das} war der Moment, in dem Millicent Bulstrode aus den unteren Schlafsälen heraufstürmte und rief: "Hey, Leute, ratet mal was gerade passiert ist, jetzt ist \emph{Granger} losgezogen und hat Derrick und das, was von seiner Crew übrig ist, zusammengeschlagen, und sein Vater hat ihm eine Eule geschickt und gesagt, dass er, wenn er nicht -"

Millicent erblickte Hermine, die in der Tür stand.

Es herrschte eine sehr \emph{laute} Stille.

"Äh", sagte Daphne. \emph{Was bei Merlin?} meinte ihr Gehirn. "Äh, was machen Sie hier, General?"

"Nun", sagte Hermione Granger mit einem seltsamen Lächeln auf dem Gesicht, "ich habe beschlossen, dass es nicht fair ist, wenn mysteriöse alte Zauberer einigen Leuten eine Chance geben, Helden zu sein, und anderen nicht, und außerdem habe ich Geschichtsbücher gelesen und es gibt nicht annähernd genug weibliche Helden darin. Also dachte ich, ich komme einfach mal vorbei und frage, ob du eine Heldin sein willst und … warum funkelst du so."

Wieder herrschte Schweigen.

"Das", sagte Daphne, "war wahrscheinlich \emph{nicht} der beste Zeitpunkt, um mir diese Frage zu stellen -"

"\emph{Ich mache mit!}", rief Tracey Davis und sprang von ihrem Sofa auf.

-\/-\/-\/-\/-\/-\/-\/-\/-\/-\/-\/-\/-\/-\/-\/-\/-\/-\/-\/-\/-\/-\/-\/-\/-\/-\/-\/-\/-\/-\/-\/-\/-\/-\/-\/-\/-\/-\/-\/-\/-\/-\/-\/-\/-\/-\/-\/-\/-\/-\/-\/-\/-\/-\/-\/-\/-\/-\/-\/-\/-\/-\/-\/-\/-\/-\/-\/-\/-\/-\/-\/-\/-\/-\/-\/-\/-\/-\/-\/-\/-\/-\/-\/-\/-\/-\/-\/-\/-\/-\/-\/-\/-\/-\/-\/-\/-\/-\/-\/-\/-\/-\/-\/-\/-\/-\/-\/-\/-\/-\/-\/-\/-\/-\/-\/-\/-\/-\/-\/-\/-\/-\/-\/-\/-\/-\/-\/-\/-\/-\/-\/-\/-\/-\/-\/-\/-\/-\/-\/-\/-\/-\/-\/-\/-\/-\/-\/-\/-\/-\/-\/-\/-\/-\/-\/-\/-\/-\/-\/-

Und so wurde das Bündnis engagierter, lengendärer, fähiger, Erstklässler-Retterinnen geboren.

(Anm. d. Übers.: engl. Society for the Promotion of Heroic Equality for Witches (S.P.H.E.W) -- siehe im Original: Hermines Bund für Elfenrechte (B.Elfe.R), der im Englischen auch S.P.E.W (Society for the Promotion of Elfish Welfare; to spew, speien) heißt.

Das Akronym heißt also in den folgenden Kapiteln \textbf{Belfer}.)

