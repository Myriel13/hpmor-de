

\hypertarget{selbstverwirklichung-teil-9-konflikte-eskalieren}{% \section{42. Selbstverwirklichung, Teil 9 -- Konflikte eskalieren}\label{selbstverwirklichung-teil-9-konflikte-eskalieren}}

-\/-\/-\/-\/- Kapitel 74: Selbstverwirklichung, Teil 9 -- Konflikte eskalieren -\/-\/-\/-\/-

Harry ging einen Schritt vorwärts, dann noch einen, bis ein Gefühl des Unbehagens ihn zu durchdringen begann, ein Unbehagen in sämtlichen seiner Nerven.

Er sagte nichts, hob keine Hand; das durchdringende Gefühl des Unbehagens würde es für ihn mitteilen.

Von hinter der geschlossenen Bürotür kam ein Flüstern, das durch die Tür drang, als ob keine Tür vorhanden wäre.

"Es ist nicht meine Bürozeit", sagte das kalte Flüstern, "und auch noch nicht die Zeit unseres Treffens. Ich ziehe Ihnen zehn Quirrell-Punkte ab, und seien Sie froh, dass es nicht mehr sind."

Harry blieb ruhig. Der Gang durch Askaban hatte seine Skala der emotionalen Störungen neu kalibriert; und der Verlust eines Hauspunktes, der früher fünf von zehn Punkten betragen hatte, lag jetzt irgendwo bei null Komma drei. Harrys Stimme war ebenfalls ruhig, als er sagte: "Sie haben eine überprüfbare Vorhersage gemacht und sie wurde falsifiziert, Professor. Ich wollte das nur anmerken."

Als Harry sich zum Gehen wandte, hörte er, wie sich die Tür hinter ihm öffnete, und er drehte sich überrascht um.

Professor Quirrell lehnte sich in seinem Stuhl zurück, sein Kopf war zurück an die Lehne gesackt, während ein Pergament vor ihm schwebte. Beide Hände des Verteidigungsprofessors ruhten schlaff auf dem Schreibtisch, als wären sie kraftlos. Er hätte eine Leiche sein können, wenn sich nicht die eisblauen Augen immer noch bewegt hätten, hin und her, hin und her.

Das Pergament verschwand und wurde durch ein anderes ersetzt, so schnell, dass es aussah, als hätte das Material nur geflackert.

Dann bewegten sich auch die Lippen. "Und daraus", flüsterten die Lippen, "schließen Sie, was, Mr~Potter?"

Harry war erschüttert von dem Anblick, aber seine Stimme blieb ruhig, als er sagte: "Dass gewöhnliche Menschen nicht immer nichts tun und dass Hermine Granger in größerer Gefahr durch Haus Slytherin ist, als Sie dachten."

Die Lippen bogen sich, ganz knapp. "Sie denken also, ich hätte in meiner Menschenkenntnis versagt. Aber das ist kaum die einzige Möglichkeit, Junge. Siehst du die andere?"

Harry runzelte die Stirn, als er den Verteidigungsprofessor anstarrte.

"Ich habe es satt", flüsterte der Verteidigungsprofessor. "Sie werden hier stehen bleiben, bis Sie es selbst herausfinden, oder Sie verschwinden." Als hätte Harry aufgehört zu existieren, blickten die Augen des Verteidigungsprofessors wieder auf das Pergament und flogen weiter hin und her.

Erst sechs Pergamente später sah Harry es und sagte laut: "Sie glauben, dass Ihre Vorhersage fehlgeschlagen ist, weil ein anderer Faktor am Werk war, der nicht in Ihrem Modell enthalten war. Irgendein Grund, warum das Haus Slytherin Hermine mehr hasst, als Sie erkannt haben. Wie damals, als die Bahnberechnungen für Uranus falsch waren und das Problem nicht in Newtons Gesetzen lag, sondern darin, dass sie nichts über Neptun wussten -"

Das Pergament verschwand, und wurde nicht ersetzt. Der Kopf erhob sich aus seiner weggesackten Position und blickte Harry direkter an, und die Stimme, die herauskam, war ruhig, aber nicht tonlos. "Ich denke, Junge", sagte Professor Quirrell leise, aber mit etwas, das seiner normalen Stimme nahekam, "wenn das ganze Haus Slytherin sie so sehr hassen würde, hätte ich es gesehen. Und doch haben drei formidable Kämpfer dieses Hauses etwas getan statt nichts, unter Risiko und auf Kosten für sie selbst. Welche Kraft könnte sie bewegt haben, oder ihre Bewegung gewollt haben?" Das eisblaue Glitzern in den Augen des Verteidigungsprofessors traf Harrys eigenen Blick. "Irgendeine einflussreiche Hand innerhalb Slytherins, vielleicht. Wie hätte diese Hand dann selbst davon profitieren können, dass dem Mädchen und ihren Anhängern Schaden zugefügt wurde?"

"Ähm …", sagte Harry. "Es müsste jemand sein, der irgendwie von Hermine bedroht wird, oder jemand, der die Lorbeeren erntet, wenn sie verletzt wird? Ich kenne niemanden, der in dieses Profil passt, aber eigentlich weiß ich auch nicht viel über jeden in Slytherin außerhalb der Erstklässler." Harry kam auch der Gedanke, dass die Schlussfolgerung eines versteckten Drahtziehers aus einem einzigen, leicht unerwarteten Angriff ein unzureichender Beweis für die vorherige Unwahrscheinlichkeit der Theorie zu sein schien; \emph{allerdings} war es Professor Quirrell, der die Schlussfolgerung machte…

Der Verteidigungsprofessor sah Harry nur an, die Augenlider leicht gesenkt, als wäre er ungeduldig.

"Und ja", sagte Harry, "ich \emph{bin} mir sicher, dass Draco Malfoy nicht dahintersteckt."

Ein Zischen, das wie ein Seufzer nach außen drang. "Er ist der Sohn von Lucius Malfoy und wurde nach strengsten Maßstäben ausgebildet. Was immer Sie von ihm gesehen haben, selbst in scheinbar unbewachten Momenten, wenn seine Maske verrutscht und Sie darauf vertrauen, dass Sie die Wahrheit darunter gesehen haben, selbst das kann alles Teil des Gesichts sein, das er wählt, um es Ihnen zu zeigen."

\emph{Nur wenn Draco erfolgreich den Patronus-Zauber wirkte, um die Maske aufrechtzuerhalten}. Aber Harry sagte das natürlich nicht; stattdessen grinste er nur leicht und sagte: "Also entweder haben Sie \emph{wirklich} noch nie Dracos Gedanken gelesen, oder das ist nur das, was Sie mich glauben machen wollen."

Es gab eine Pause. Eine der Hände drehte sich um und winkte mit einem Finger.

Harry betrat den Raum. Die Tür schloss sich hinter ihm.

"Das hätten Sie nicht laut in menschlicher Sprache sagen sollen", sagte Professor Quirrells sanfte Stimme. "Legilimentik, bei Malfoys Erbe? Wenn Lucius Malfoy davon erfahren würde, würde er mich auf der Stelle ermorden lassen."

"Er würde es \emph{versuchen}", sagte Harry. Das hätte Professor Quirrell ein Augenzwinkern entlocken müssen, aber das Gesicht des Verteidigungsprofessors war unbeweglich. "Aber okay, es tut mir leid."

Als der Verteidigungsprofessor wieder sprach, war seine Stimme wieder zu einem kalten Flüstern geworden. "Ich nehme an, dass ich Mitleid mit dem Attentäter haben müsste." Sein Kopf fiel nach hinten gegen den Stuhl, und dann zur Seite, die Augen trafen nicht mehr die von Harry. "Aber diese kleinen Spiele halten mein Interesse grade so aufrecht, so wie sie sind. Fügen Sie Legilimentik hinzu, und es hört auf, überhaupt ein Spiel zu sein."

Harry wusste nicht wirklich, was er sagen sollte. Er hatte Professor Quirrell schon ein- oder zweimal in wütender Stimmung erlebt, aber dies schien noch leerer zu sein, und Harry wusste nicht, was er dazu sagen sollte. \emph{Was bedrückt Sie, Professor Quirrell?} konnte er nicht fragen.

"Was beschäftigt Sie denn \emph{dann}?" sagte Harry ein paar Augenblicke später, nachdem er sich eine sicher erscheinende Strategie zurechtgelegt hatte, um Professor Quirrells Aufmerksamkeit auf positive Dinge zu lenken. Experimentelle Ergebnisse über das Führen eines Dankbarkeitstagebuchs als Strategie zur Verbesserung des Lebensglücks zu zitieren, würde wahrscheinlich nicht gut aufgenommen werden.

"Ich werde Ihnen sagen, was mich nicht interessiert", sagte das eisige Flüstern. "Das Benoten von Aufsätzen, die vom Ministerium vorgeschrieben sind, interessiert mich nicht, Mr~Potter. Aber ich habe das Amt des Verteidigungsprofessors in Hogwarts übernommen, und ich werde es bis zum Ende durchziehen." Ein weiteres Pergament erschien vor Professor Quirrells Kopf, und seine Augen begannen es zu überfliegen. "Reese Belka hatte vor ihrer Torheit eine hohe Position in meinen Streitkräften inne. Ich werde ihr anbieten, zu bleiben, anstatt rausgeschmissen zu werden, wenn sie mir genau sagt, welche Kräfte sie bewegt haben. Und ich werde ihr klar machen, was passiert, wenn sie lügt. Ich erlaube mir, Gesichter zu lesen."

Der Finger des Verteidigungsprofessors deutete an Harry vorbei in Richtung der Tür.

"Aber ob Sie sich nun in der menschlichen Natur geirrt haben", sagte Harry, "oder ob im Haus Slytherin eine zusätzliche Kraft am Werk ist - so oder so, Hermine Granger ist in größerer Gefahr, als Sie vorausgesagt haben. Das letzte Mal waren es drei starke Kämpfer, also was passiert danach -"

"Sie wünscht weder meine noch Ihre Hilfe", sagte eine sanfte, kalte Stimme. "Ich finde Ihre Sorgen nicht mehr so unterhaltsam wie früher, Mr~Potter. Gehen Sie."

Irgendwie war es, obwohl sie alle gleichberechtigt waren und sie definitiv nicht das Sagen hatte, immer Hermine, die in solchen Situationen als Erste sprach.

Die vier Tische von Hogwarts, die vier Häuser, die gerade frühstückten, blickten zu der Stelle hinüber, an der sie, die acht Mitglieder von B.E.L.F.E.R., sich an einer Seite versammelt hatten.

Auch Professor Flitwick starrte sie alle vom Lehrertisch aus streng an. Hermine schaute nicht dorthin, aber sie konnte Professor Flitwicks Blick in ihrem Nacken spüren. Sie spürte ihn \emph{buchstäblich}. Es war wirklich unheimlich.

"Warum haben Sie Tracey gesagt, dass Sie mit uns reden wollen, Mr~Potter?", fragte Hermine, ihr Tonfall war frostig.

"Professor Quirrell hat Reese Belka gestern Abend aus ihrer Armee ausgeschlossen", sagte Harry Potter. "Und von all ihren anderen außerschulischen Verteidigungsaktivitäten. Versteht einer von euch, was das bedeutet? Miss~Greengrass? Padma?"

Harrys Blick schweifte über sie, während Hermine einen verwirrten Blick mit Padma austauschte und Daphne den Kopf schüttelte.

"Nun", sagte Harry leise, "das würde ich eigentlich auch nicht von euch erwarten. Aber was es bedeutet, ist, dass ihr in Gefahr seid, und ich weiß nicht, wie groß die Gefahr ist." Der Junge straffte die Schultern und sah Hermine direkt in die Augen. "Ich wollte das eigentlich nicht sagen, aber… ich wollte dir nur anbieten, dich unter jeglichen Schutz zu stellen, den ich geben kann. Jedem klar machen, dass jeder, der sich mit dir anlegt, sich mit dem Jungen, der lebte anlegt."

"Harry!", sagte Hermine streng. "Du \emph{weißt}, ich will nicht -"

"Einige von ihnen sind auch \emph{meine} Freunde, Hermine." Harry wandte seinen Blick nicht von ihr ab. "Und es ist ihre Entscheidung, nicht deine. Padma? Du hast mir gesagt, dass ich dir nichts schuldig bin für das, was ich getan habe, und das ist die Art von Dingen, die ein Freund sagen würde."

Hermine löste ihren Blick von Harry, um zu sehen, wie Padma den Kopf schüttelte.

"Lavender?" sagte Harry. "Du hast in meiner Armee gut gekämpft, und ich werde für dich kämpfen, wenn du es wünschst."

"Ich danke \emph{Ihnen}, General!" Lavender sagte frostig. "Ich meine Mr~Potter. Aber nein. Ich bin eine Heldin und eine Gryffindor, und ich kann für mich selbst kämpfen."

Es gab eine Pause.

"Parvati?" sagte Harry. "Susan? Hannah? Daphne? Ich kenne keine von euch so gut, aber ich würde es jedem anbieten, der mich darum bittet, denke ich."

Eine nach der anderen schüttelten die anderen vier Mädchen den Kopf.

Hermine erkannte, was nun kam, aber sie wusste nicht, was sie dagegen tun konnte.

"Und mein treuer Soldat, die chaotische Tracey?", fragte Harry Potter.

"\emph{Wirklich?} ", keuchte Tracey, ohne die stechenden Blicke zu bemerken, die Hermine und jedes andere Mädchen auf sie richteten. Traceys Hände flogen kunstvoll zu ihren Wangen, obwohl sie es nicht wirklich schaffte zu erröten, nicht soweit Hermine es sehen konnte; und ihre braunen Augen waren, wenn nicht leuchtend, so doch zumindest sehr weit geöffnet. "Das würdest du tun? Für \emph{mich}? Ich meine - ich meine, natürlich, absolut, General Chaos -"

Und so kam es, dass Harry Potter an jenem Morgen erst zum Gryffindor-Tisch und dann zum Slytherin-Tisch ging und beiden Häusern mitteilte, dass jeder, der Tracey Davis etwas antun würde, egal was sie gerade tat, Zitat: die wahre Bedeutung von Chaos erfahren würde, Zitat Ende.

Nur mit Mühe konnte Draco Malfoy verhindern, dass er seinen Kopf wiederholt auf seinen Teller mit Toast knallte.

Sie waren nicht gerade Wissenschaftler, die Rowdies von Hogwarts.

Aber selbst \emph{sie}, das wusste Draco, würden das testen wollen.

Das Bündnis engagierter, lengendärer, fähiger, Erstklässler-Retterinnen hatte es nicht \emph{angekündigt}, es schien auch nicht sinnvoll, es \emph{anzukündigen}. Aber sie hatten alle im Stillen beschlossen (oder, im Fall von Lavender, von allen sieben anderen Mädchen angebrüllt, bis sie zustimmte), für eine Weile eine Pause vom Kampf gegen Schläger einzulegen, zumindest so lange, bis ihre Hausoberhäupter sie nicht mehr ganz so streng ansahen und ältere Schüler aufgehört hatten, Hermine gegen Wände zu stoßen.

Daphne hatte Millicent \emph{erzählt}, dass sie eine Pause machen würden.

Und so schaute Daphne ein paar Tage später mit einiger Verwunderung auf das Pergament, das ihr beim Mittagessen überreicht worden war und auf dem mit so zittriger Hand, dass es fast unleserlich war, geschrieben stand:

\emph{14 Uhr heute Nachmittag am oberen Ende der Treppe, die von der Bibliothek hinaufführt, WIRKLICH WICHTIG, jeder muss dort sein - Millicent}

Daphne sah sich um, aber sie konnte Millicent nirgendwo in der Großen Halle sehen.

"Eine Nachricht von deinem Informanten?", fragte Hermine, als Daphne es ihr sagte. "Das ist seltsam - \emph{ich} habe nicht -"

"Du hast was nicht?", fragte Daphne, nachdem das Ravenclaw-Mädchen mitten im Satz stehen geblieben war.

Die Sonnenschein-Generalin schüttelte den Kopf und sagte: "Hör zu, Daphne, ich denke, wir müssen wissen, woher diese Nachrichten kommen, bevor wir ihnen weiter nachgehen. Sieh dir an, was letztes Mal passiert ist, wie hätte irgendjemand \emph{wissen} können, wo diese drei Schläger sein würden, es sei denn, sie waren eingeweiht?"

"Das kann ich nicht sagen -" sagte Daphne. "Ich meine, ich kann nichts sagen, aber ich weiß, woher die Botschaften kommen, und ich weiß, wie jemand das wissen kann."

Hermine warf Daphne einen \emph{Blick} zu, der das Ravenclaw-Mädchen für einen Moment beängstigend wie Professor McGonagall aussehen ließ.

"Aha", sagte Hermine. "Und weißt du, wie Susan sich plötzlich in Supergirl verwandelt hat?"

Daphne schüttelte den Kopf und sagte: "Nein, aber ich denke, es könnte \emph{sehr wichtig} sein, dass, wenn wir eine Nachricht bekommen, dass wir irgendwo sein sollen, \emph{alle dort sein müssen}." Daphne hatte nicht \emph{gesehen}, was mit Susan passiert war, nachdem Daphne versucht hatte, die Prophezeiung abzuwenden, indem sie Susan fernhielt. Aber man hatte ihr danach davon \emph{erzählt}, und jetzt hatte Daphne Angst, dass…

Sie könnte möglicherweise…

Möglicherweise Etwas Zerbrochen haben könnte…

"Aha", sagte Hermine, die wieder den McGonagall-Blick machte.

Niemand schien zu wissen, wo es angefangen hatte, wer es angefangen hatte. Wenn man versucht hätte, es im Nachhinein zurückzuverfolgen, Wort für Wort und Gemurmel für Gemurmel, hätte man wahrscheinlich festgestellt, dass sich alles in einem riesigen Kreis drehte.

Peregrine Derrick wurde auf die Schulter getippt, als er an diesem Morgen Zaubertränke verließ.

Jaime Astorga hörte beim Mittagessen ein Flüstern in seinem Ohr.

Robert Jugson III entdeckte einen kleinen gefalteten Zettel unter seinem Teller.

Carl Sloper hörte zufällig, wie zwei ältere Gryffindors darüber tuschelten, und sie warfen ihm bedeutungsvolle Blicke zu, als sie vorbeigingen.

Niemand schien zu wissen, wo das Gerücht begann oder wer es zuerst ausgesprochen hatte, aber es nannte den Ort und die Zeit, und es sagte, dass die Farbe weiß sein würde.

"Jede Einzelne von euch sollte sich besser absolut im Klaren darüber sein", sagte Susan Bones. Das Hufflepuff-Mädchen, oder was auch immer für eine seltsame Kraft von ihr Besitz ergriffen hatte, \emph{tat nicht einmal mehr so}, als würde sie sich normal verhalten. Das rundliche Mädchen schritt mit festem, selbstbewusstem Gang durch die Hallen. "Wenn wir dort ankommen und es nur ein Rowdie ist, ist das in Ordnung, ihr könnt ihn oder sie auf die normale Art bekämpfen. Meine geheimnisvollen Superkräfte werden sich nicht aktivieren, wenn keine Unschuldigen in Gefahr sind. Aber wenn fünf Siebtklässler aus einem Schrank springen, wisst ihr was ihr dann tut? Richtig, ihr \emph{lauft} \emph{weg} und lasst mich gegen sie kämpfen. Einen Lehrer zu finden ist optional, das Wichtigste ist, dass ihr \emph{wegrennt}, sobald ich eine Öffnung schaffe. In einem Kampf wie diesem seid ihr eine \emph{Belastung}. Ihr seid \emph{zivile Ziele}, um deren Schutz ich mich kümmern muss. Also werdet ihr so schnell wie möglich abhauen und ihr werdet \emph{nicht versuchen}, \emph{irgendetwas} Heldenhaftes zu tun, denn in der Stunde, in der ihr aus euren Heilerbetten herauskommt, werde ich \emph{persönlich} auftauchen und euch \emph{in} \emph{den Arsch treten,} so dass ihr gleich wieder in diesen Betten landet\emph{.} Haben wir uns da alle verstanden?"

"Ja", quiekten die meisten der Mädchen, obwohl es bei Hannah als: „Ja, Lady Susan!“ herauskam.

"\emph{Nenn} mich nicht so", schnauzte Susan. "Und \emph{ich glaube, ich habe Sie nicht} \emph{gehört, Miss~Brown}! Ich warne dich, ich habe Freunde, die Theaterstücke schreiben, und wenn du irgendetwas Dummes tust, wird sich die Nachwelt an dich als Lavender, die Erstaunlich Dumme Geisel, erinnern."

(Hermine begann sich Gedanken darüber zu machen, wie viele andere Hogwartsschüler außer Harry geheimnisvolle dunkle Seiten hatten und ob es wahrscheinlich war, dass \emph{sie} auch eine entwickeln würde, wenn sie weiterhin mit ihnen herumhing.)

"In Ordnung, Captain Bones", sagte Lavender in einem ungewöhnlich respektvollen Ton, als sie auf dem kürzesten Weg zur Bibliothek um eine weitere Ecke bogen und einen ziemlich großen Korridor durchquerten, der mit sechs Doppeltüren gespickt war, drei auf jeder Seite. "Darf ich fragen, ob es eine Möglichkeit für \emph{mich} gibt, eine Doppelhexe zu werden?"

"Melde dich in deinem sechsten Jahr für das Aurorenvorbereitungsprogramm an", sagte Susan. "Das ist die nächstbeste Möglichkeit. Oh, und wenn dir ein berühmter Auror anbietet, dein Sommerpraktikum zu betreuen, ignoriere einfach jeden, der dich warnt, dass er einen schrecklichen Einfluss hat oder dass du mit ziemlicher Sicherheit sterben wirst."

Lavender nickte schnell. "Verstanden, verstanden."

(Padma, die beim letzten Mal nicht dabei gewesen war, warf Susan \emph{sehr} skeptische Blicke zu.)

Dann blieb Susan plötzlich an Ort und Stelle stehen, ihr Zauberstab schnappte hoch und sie sagte: "\emph{Protego Maximus!}"

Ein Adrenalinstoß ging durch Hermine, sie zog sofort ihren Zauberstab und wirbelte herum -

Aber sie konnte nichts sehen, durch den dichten blauen Dunst, der sie nun alle umgab.

Auch die anderen Mädchen, die sich ebenfalls in Formation gebracht hatten, schauten verwirrt.

"Entschuldigung!", sagte Susan. "Tut mir leid, Mädels. Gebt mir einen Moment, um mich hier umzusehen. Der Gedanke an eine bestimmte Person hat mich gerade daran erinnert, dass diese Halle, in der wir uns gerade befinden, mit all den Türen, ein \emph{hervorragender} Ort für einen Hinterhalt wäre."

Es gab einen Moment der Stille.

"Jetzt", sagte eine raue Männerstimme, die durch einen brummenden Unterton bis zur Unkenntlichkeit verwischt wurde.

Alle sechs Doppeltüren knallten auf.

Weiße Roben schritten schweigend vorwärts, alles verdeckende weiße Roben ohne Zeichen der Hauszugehörigkeit und weißes Tuch, das die Gesichter unter den Kapuzen verbarg. Sie marschierten und marschierten und drängten sich in dem großen Korridor in einer Anzahl, die zu groß war, um sie leicht zu zählen. Weniger als fünfzig Roben, wahrscheinlich. Sicherlich mehr als dreißig. Alle waren sie bereits von blauem Dunst umgeben.

Susan sagte ein paar Extrem Böse Worte, so furchtbar, dass es Hermine zu fast jedem anderen Zeitpunkt aufgefallen wäre.

"Diese Nachricht!" Daphne rief in plötzlichem Entsetzen. "Sie \emph{war nicht} von -"

"Millicent Bulstrode?", sagte die Stimme mit dem brummenden Unterton. "Nein, das war sie nicht. Sehen Sie, Miss~Greengrass, wenn dasselbe Mädchen jeden Tag, an dem Sie gegen einen Schläger kämpfen, eine Slytherin-Nachricht schickt, wird es bald jemand anderem auffallen. Wir werden uns mit ihr unterhalten, wenn wir mit euch fertig sind."

"Miss~Susan", sagte Hannah mit einer Stimme, die gerade zu zittern begann, "können Sie so super sein um -"

Zauberstäbe erhoben sich in vielen Händen. Es kam eine Reihe von blendenden Blitzen aus grünem Licht, eine gewaltige Salve von Schildbrechern, an deren Ende es keine schützende blaue Kuppel mehr gab, die sie umgab, und Susan war auf die Knie gefallen und umklammerte ihren Kopf.

An beiden Enden des Korridors hatten sich Barrieren aus fester Schwärze gebildet. Hinter den Doppeltüren, in die Hermine hineinsehen konnte, befanden sich nur unbenutzte Klassenzimmer, also Sackgassen.

"Nein", sagte die männliche Stimme mit dem überlagerten Brummen, "das kann sie nicht. Falls ihr es noch nicht bemerkt habt, ihr habt eine Menge Leute sehr wütend auf euch gemacht, und wir haben nicht die Absicht, dieses Mal zu verlieren. Also gut, Leute, macht euch bereit zum Feuern."

Die Zauberstäbe im Umkreis zielten wieder, niedrig genug, dass ihre Gegner sich nicht gegenseitig treffen würden, wenn sie daneben schossen.

Und dann sagte plötzlich eine andere männliche Stimme, die von einem ähnlichen Summen begleitet wurde: "\emph{Homenum Revelio!} "

Einen Augenblick später gab es eine weitere massive Salve von Schildbrechern und Flüchen, die reflexartig auf die plötzlich aufgetauchte Gestalt abgefeuert wurden und die Schilde zerschmetterte, die sich fast sofort um sie herum zu bilden begonnen hatten.

Und dann, als diese Gestalt zu Boden fiel, herrschte fassungsloses Schweigen.

"\emph{Professor Snape?}", sagte die zweite Stimme. "\emph{Er} ist derjenige, der sich eingemischt hat?"

Es war der Meister der Zaubertränke von Hogwarts, der nun bewusstlos auf dem Steinboden lag, die schmutzbesprenkelten Roben bewegten sich ein letztes Mal, bevor sie sich an ihrem Platz niederließen, die Hand des Gestürzten ausgestreckt in Richtung der Stelle, wo sein Zauberstab langsam davonrollte.

"Nein", sagte die erste männliche Stimme, die jetzt ein wenig unsicherer klang. Dann erhob sie sich: "Nein, das kann es unmöglich sein. Er hat natürlich gehört, wie wir die Kunde weitergegeben haben, und ist mitgekommen, um sicherzugehen, dass niemand es wieder vermasselt. Wir werden ihn nachher aufwecken und uns entschuldigen, und er wird die Kinder mit einem Gedächtniszauber belegen, damit sie sich nicht erinnern, er ist ja Professor, also kann er das tun. Wie auch immer, wir sollten sicherstellen, dass wir jetzt \emph{wirklich} allein sind. \emph{Veritas Oculum}! "

Es müssen danach ganze zwei Dutzend verschiedene Zauber gesprochen worden sein, aber es tauchten keine weiteren unsichtbaren Personen auf. Besonders einer davon ließ Hermines Herz sinken; sie erkannte ihn als den Zauber, der neben der Beschreibung des Wahren Unsichtbarkeitsmantels aufgeführt war und der zwar nicht den Mantel enthüllte, aber verriet, ob er oder bestimmte andere Artefakte in der Nähe waren.

"Mädels?", flüsterte Susan. Sie drückte sich langsam auf die Beine, obwohl Hermine sehen konnte, wie ihre Glieder schwankten und zitterten. "Mädels, es tut mir leid, was ich vorhin gesagt habe. Wenn ihr etwas Gescheites und Heldenhaftes vorhabt, könnt ihr es ruhig versuchen."

"Oh, ja", sagte Tracey Davis dann, ihre Stimme zitterte. "Das hätte ich fast vergessen." Das Slytherin-Mädchen erhob ihre Stimme und sprach.

"Hey, ihr alle!", rief Tracey in einem hohen, zittrigen Ton. "Hey, habt ihr auch vor, mir wehzutun?"

"Ja, eigentlich schon", sagte die brummende Stimme der Anführerin. "Das haben wir."

"Ich stehe unter dem Schutz von Harry Potter, wisst ihr! Jeder, der versucht, mir etwas anzutun, wird die wahre Bedeutung von Chaos erfahren! Also lasst ihr mich gehen?" Es hätte trotzig klingen sollen. Es klang aber eher verängstigt.

Es gab eine Pause. Einige der Kapuzen der Roben drehten sich zueinander, dann drehten sie sich wieder zu den Mädchen um.

"Hm …", sagte die brummende Männerstimme. "Hm … nein."

Tracey Davis verstaute ihren Zauberstab in ihren Roben.

Langsam und bedächtig hob sie ihre rechte Hand hoch in die Luft und presste Daumen und Zeigefinger zusammen.

"Mach schon", sagte die Stimme.

Tracey Davis schnippte mit den Fingern.

Es gab eine lange, schreckliche Pause.

Nichts geschah.

"Ja, gut", sagte die Stimme -

Tracey sagte, ihre Stimme klang noch höher und zittriger, "\emph{Acathla, mundatus sum}." Ihre Hand, die sich noch weiter nach oben streckte, schnippte ein zweites Mal mit den Fingern.

Ein namenloser Schauer lief Hermine den Rücken hinunter, ein Schauer der Angst und der Orientierungslosigkeit, als hätte sie gerade gespürt, wie sich der Boden unter ihr neigte und sie in eine darunter liegende Dunkelheit zu stürzen drohte.

"Was macht sie -", begann eine brummende Frauenstimme.

Traceys Gesicht wirkte blass, vor Angst verzerrt, aber ihre Lippen bewegten sich, spuckten in einem hohen Singsang Töne aus: "\emph{Mabra, brahoring, mabra} …"

Ein kalter Wind schien durch die Enge des Korridors zu wehen, ein dunkler Atem, der ihre Gesichter streichelte und ihre Hände mit Eis berührte.

"Feuert auf sie, auf mein Kommando!", rief die führende Stimme. "Eins, zwei, \emph{drei}! " und vielleicht vierzig Stimmen brüllten Bannsprüche und erzeugten eine riesige konzentrische Anordnung von feurigen Blitzen, die den breiten Korridor heller als die Sonne erleuchteten -

- für den kurzen Moment, bevor die Bolzen auf ein dunkelrotes Achteck trafen und verschwanden, das in der Luft um die Mädchen herum erschien und einen Moment später wieder verschwand.

Hermine sah es, sie sah es, aber sie konnte es sich immer noch nicht vorstellen; sie konnte sich keinen so mächtigen Abschirmzauber vorstellen, einen Zauber, der einer Armee standhalten würde.

Und Traceys Stimme beschwor weiter, ihre Stimme klang lauter und selbstbewusster, und ihr Gesicht verzog sich, als würde sie versuchen, sich an etwas ganz genau zu erinnern.

"\emph{Shuffle, duffle, muzzle, muff}.

\emph{Fista, wista, mista-cuff}."

Jetzt konnten es alle Anwesenden spüren, Heldinnen und Schurken gleichermaßen, das Gefühl eines dunklen Willens, der auf sie drückte, ein Kribbeln in der Luft, während sich etwas aufbaute und aufbaute und aufbaute. All die blauen Dunstschleier um die weißen Roben, all die Schutzzauber, waren erloschen, ohne dass ein sichtbarer Zauber sie berührte. Es gab weitere Lichtblitze, als weitere verzweifelte Zauber abgefeuert wurden, aber sie verpufften in der Luft wie Kerzenflammen, die Wasser berühren.

Die schwarzen Barrieren an den beiden Enden des Korridors hatten sich unter dem wachsenden Druck wie Rauch aufgelöst, aber ihr Verdunsten offenbarte, dass die Ausgänge versiegelt waren, versperrt durch geflieste Lamellen aus dunklem Metall, die aussahen, als seien sie mit Blut befleckt; und während Tracey "\emph{Lemarchand, Lament, Lemarchand}" skandierte, begann ein furchtbares blaues Licht unter den Metalllatten und zwischen ihnen hervorzuleuchten; und die sechs Paar Doppeltüren knallten alle auf einmal zu, als panische, weißgekleidete Tyrannen begannen, gegen sie zu hämmern und zu heulen.

Dann schwang Traceys Hand nach links, und sie schrie "\emph{Khornath}! ", dann zeigte ihre Hand unter sich und "\emph{Slaaneth}! ", über ihr "\emph{Nurgolth}! ", und dann, zu ihrer Rechten, "\emph{TZINTCHI}! "

Tracey hielt inne, holte tief Luft; und Hermine fand ihre Stimme und rief: "\emph{Stopp! Tracey, stopp!} "

Aber da war ein seltsames, wildes Lächeln auf Traceys Gesicht. Sie hob die Hand noch höher und schnippte ein drittes Mal mit den Fingern; und als sie wieder sprach, lag unter ihrer hohen, mädchenhaften Stimme ein Unterton, als würde ein tieferer Chor mit ihr singen.

\emph{"\emph{Dunkelheit} \emph{hinter} \emph{Dunkelheit, tiefer alspechschwarz} \emph{Nacht.}}

\emph{Begraben unter dem Fluss der Zeit…}

\emph{Von Dunkelheit zu Dunkelheit, deine Stimme hallt in der Leere wider,}

\emph{Unbekannt dem Tod, noch bekannt dem Leben}."

"\emph{Was tust du da?}", schrie Parvati, und das Gryffindor-Mädchen streckte eine Hand aus, als wolle sie die Slytherin herunterziehen, die nun begann, in die Luft zu schweben; und Daphne und Susan packten Parvatis Arm gleichzeitig, und Daphne rief: "Nicht, wir wissen nicht, was passiert, wenn das Ritual unterbrochen wird!"

"\emph{Und was passiert, wenn es} \emph{DURCHGEFÜHRT} \emph{wird?} ", schrie Hermine, so nah wie sie noch nie an einer totalen Gehirnschmelze war.

Susans Gesicht war kreidebleich, und sie flüsterte: "Es tut mir leid, Mad-Eye…"

Und Tracey sprach weiter, ihr Körper schwebte höher und höher über dem Boden, ihr schwarzes Haar peitschte wild um sie herum im kühlen Wind.

\emph{"\emph{Ihr, die ihr das Tor kennt, die ihr das Tor seid, der Schlüssel und Wächter des Tores:}}

\emph{Ich befehle euch, ihm den Weg zu öffnen und seine Macht vor mir zu offenbaren}!"

Dann wurde der Korridor in völlige Dunkelheit und Stille getaucht, so dass nur noch Tracey zu sehen und zu hören war, als gäbe es nichts mehr im Universum außer ihr und dem Licht, das sie aus irgendeiner namenlosen Quelle beleuchtete.

Das leuchtende Mädchen hob ein letztes Mal die Hand und presste mit furchtbarer Schwere Daumen und Zeigefinger zusammen.

Und in der Dunkelheit blickte Hermine in Traceys Gesicht und sah, dass die Augen des Slytherin-Mädchens jetzt genau den gleichen Grünton hatten wie die von Harry Potter.

\emph{"\emph{Harry James Potter-Evans-Verres!}}

\emph{Harry James Potter-Evans-Verres!}

\emph{HARRY JAMES POTTER-EVANS-VERRES!}"

Es gab ein Knacken wie ein Donnerschlag, und dann --

Harry hatte sich entschieden, eine eher entspannte Haltung einzunehmen, als er in einem niedrigen Stuhl vor dem mächtigen Schreibtisch des Schulleiters von Hogwarts saß: ein Bein über das andere geschlagen und die Arme lässig zu beiden Seiten ausgebreitet. Harry tat sein Bestes, um den Lärm der ihn umgebenden Geräte zu ignorieren, obwohl das Gerät direkt hinter ihm, das sich anhörte, als würde eine Eule verzweifelt schreien, während sie durch einen Holzschredder gezogen wurde, ziemlich schwer zu ignorieren war.

"Harry", sagte der alte Zauberer hinter dem Schreibtisch, die gealterte Stimme gleichmäßig, während die blauen Augen ihn unter der glänzenden Halbmondbrille anstarrten. Schulleiter Dumbledore hatte sich in Roben von mitternächtlichem Purpur gekleidet; kein echtes formelles Schwarz, aber dunkel genug um echtem tödlichen Ernst nahe zu kommen, so wie die Zaubererwelt die Bedeutung der Mode sah. "Warst du … dafür \emph{verantwortlich}?"

"Ich kann nicht leugnen, dass mein Einfluss am Werk war", sagte Harry.

Der alte Zauberer nahm seine Brille ab, beugte sich vor, um Harry direkt anzustarren, blaue Augen gegen grüne. "Ich werde dir eine Frage stellen", sagte der Schulleiter mit ruhiger Stimme. "Denkst du, dass das, was du heute getan hast, - \emph{angemessen} war? "

"Es waren Rowdies und sie kamen in diesen Flur mit der direkten Absicht, Hermine Granger und sieben anderen Erstklässlern wehzutun", sagte Harry nüchtern. "Wenn ich nicht zu jung für moralische Urteile bin, dann sind sie es auch nicht. Nein, Schulleiter, sie haben es nicht verdient zu sterben. Aber sie \emph{haben} es verdient, nackt ausgezogen und an die Decke geklebt zu werden."

Der alte Zauberer setzte seine Brille wieder auf. Zum ersten Mal, seit Harry ihn gesehen hatte, schien der Schulleiter um Worte verlegen zu sein. " Merlin selbst ist mein Zeuge ", sagte Dumbledore, "ich habe nicht den blassesten Schimmer, wie ich darauf reagieren soll."

"Das ist so ziemlich der Effekt, den ich beabsichtigt habe", sagte Harry. Er fühlte sich, als sollte er eine fröhliche Melodie pfeifen, aber leider hatte er nie gelernt, wie man zuverlässig pfiff.

"Ich brauche dich nicht zu fragen, wer \emph{direkt} dafür verantwortlich ist", sagte der Schulleiter. "Nur drei Zauberer innerhalb von Hogwarts könnten mächtig genug sein. Ich selbst habe es nicht getan. Severus hat mir versichert, dass er nicht beteiligt war. Und der dritte …" Der Schulleiter schüttelte etwas bestürzt den Kopf. "Du hast dem Verteidigungsprofessor deinen Umhang geliehen, Harry. Ich glaube nicht, dass das klug war. Denn jetzt, wo er der Entdeckung durch einfache Zauber entgangen ist, weiß er sicher, dass es sich um ein Heiligtum des Todes handelt - wenn er es nicht schon bei der ersten Berührung gewusst hat."

"Professor Quirrell hat bereits auf meinen Besitz eines Unsichtbarkeitsumhangs geschlossen", sagte Harry. "Und wie ich ihn kenne, hat er wahrscheinlich vermutet, dass es sich um ein Heiligtum des Todes handelt. Aber in \emph{diesem} Fall, Schulleiter, war es so, dass Professor Quirrell unter einem dieser gesichtsverdeckenden weißen Umhänge steckte."

Wieder gab es eine Pause.

"Wie gerissen", sagte der Schulleiter. Er lehnte sich in seinem Thron zurück und seufzte. "Ich habe mit dem Verteidigungsprofessor gesprochen. Kurz vor dir. Ich wusste nicht recht, was ich sagen sollte. Ich habe ihm gesagt, dass dies nicht die anerkannte Hogwarts-Politik für den Umgang mit Verstößen gegen die Flurdisziplin ist und dass ich es nicht für angemessen halte, dass ein Hogwarts-Professor tut, was er getan hat."

"Und was hat Professor Quirrell dazu gesagt?", fragte Harry, der von den aktuellen Richtlinien von Hogwarts zur Durchsetzung der Flurdisziplin nicht beeindruckt war.

Der Schulleiter trug einen resignierten Blick zur Schau. "Er sagte: \emph{Feuern Sie mich}."

Irgendwie schaffte Harry es, nicht laut zu jubeln.

Der Schulleiter runzelte die Stirn. "Aber \emph{warum} hat er das getan, Harry?"

"Weil Professor Quirrell keine Schultyrannen mag und ich sehr höflich gefragt habe", sagte Harry. \emph{Und er} \emph{langweilte} \emph{sich und ich dachte, das könnte ihn aufmuntern}. "Entweder das oder es ist Teil eines unglaublich tiefgründigen Plans."

Der Schulleiter erhob sich hinter dem Schreibtisch und begann, vor dem Hutständer, der den Sprechenden Hut und die roten Pantoffeln hielt, hin und her zu gehen. "Harry, hast du nicht das Gefühl, dass das alles ein bisschen …"

"Großartig?", bot Harry an.

"\emph{Völlig aus dem Ruder gelaufen} würde es besser ausdrücken", sagte Dumbledore. "Ich bin mir nicht sicher, ob es in der ganzen Geschichte dieser Schule jemals eine Zeit gegeben hat, in der die Dinge so, so… Ich habe kein Wort dafür, Harry, weil die Dinge noch nie so geworden sind, und deshalb hat es auch noch nie jemand nötig gehabt, ein Wort dafür zu erfinden."

Harry hätte versucht, Worte zu erfinden, um auszudrücken, wie sehr er das als Kompliment auffasste, wenn er nicht zu sehr gegluckst hätte, um zu sprechen.

Der Schulleiter betrachtete ihn mit zunehmender Ernsthaftigkeit. "Harry, verstehst du \emph{überhaupt}, warum ich diese Ereignisse beunruhigend finde?"

"Ehrlich gesagt?", fragte Harry. "Nein, nicht wirklich. Ich meine, natürlich hätte Professor McGonagall etwas gegen alles, was die langweilige Monotonie des Schulalltags in Hogwarts auflockert. Aber Professor McGonagall würde andererseits auch kein Huhn in Brand stecken."

Die Stirnfalten auf Dumbledores faltigem Gesicht vertieften sich. "Das, Harry, ist es nicht, was mich beunruhigt", sagte der Schulleiter leise. "In diesen Hallen wurde eine ganze Schlacht geschlagen!"

"Schulleiter", sagte Harry und bemühte sich, seine Stimme sorgfältig respektvoll zu halten, "Professor Quirrell und ich haben uns nicht dafür entschieden, dass dieser Kampf stattfindet. Das waren die Schläger. \emph{Wir} haben uns einfach dafür entschieden, dass die Seite des Lichts gewinnt. Ich weiß, dass es Zeiten gibt, in denen die Grenzen der Moral unsicher sind, aber in diesem Fall war die Linie, die die Schurken und die Heldinnen trennte, zwanzig Meter hoch und in weißem Feuer gezogen. Unser Eingreifen mag seltsam gewesen sein, aber es war sicher nicht \emph{falsch} -"

Dumbledore war zu seinem Schreibtisch zurückgekehrt, setzte sich mit einem dumpfen Aufprall auf seinen gepolsterten Thron und bedeckte nun sein Gesicht mit beiden Händen.

"Übersehe ich hier etwas?" sagte Harry. "Ich dachte, Sie wären insgeheim auf unserer Seite, Herr Direktor. Es war eine gryffindormäßige Sache. Die Weasley-Zwillinge würden es gutheißen, \emph{Fawkes} würde es gutheißen -" Harry warf einen Blick auf die goldene Sitzstange, aber sie war leer; entweder hatte der Phönix Wichtigeres zu tun, oder der Schulleiter hatte ihn nicht zu der heutigen Sitzung eingeladen.

"Das", sagte der Schulleiter mit einer alten, müden und etwas dumpfen Stimme, "ist genau das Problem, Harry. Es gibt einen Grund, warum man mutige junge Helden nicht mit der Leitung von Schulen betraut."

"In Ordnung", sagte Harry. Er konnte die Skepsis nicht ganz aus seiner Stimme heraushalten. "Was habe ich dieses Mal übersehen?"

Der alte Zauberer hob den Kopf, seine Miene bedeutungsvoll, und er wurde ruhiger. "Hör zu, Harry", sagte Dumbledore, "hör mir gut zu; denn alle, die Macht ausüben, müssen dies mit der Zeit lernen. Manche Dinge in dieser Welt sind in der Tat sehr einfach. Wenn du einen Stein aufhebst und ihn wieder fallen lässt, wird die Erde nicht schwerer dadurch, die Sterne weichen nicht von ihrer Bahn ab. Ich sage das, Harry, damit du weißt, dass ich nicht so tue, als ob ich weise wäre, wenn ich dir sage, dass, so einfach manche Dinge sind, andere komplex sind. Es gibt größere Hexereien, die Spuren in der Welt hinterlassen und bei denen, die sie anwenden, was ein einfacher Zauber \emph{nicht} tut. Diese Zaubersprüche erfordern Zögern, Abwägen der Konsequenzen, einen Moment, um die Bedeutung ihrer Zeichen abzuwägen. Und doch sind die kompliziertesten Zaubersprüche, die ich kenne, einfacher als die einfachste Seele. \emph{Menschen}, Harry, Menschen sind immer gezeichnet, durch das, was sie tun und durch das, was man ihnen antut. Verstehst du also, dass es nicht ausreicht, zu sagen: 'Hier ist die Grenze zwischen Held und Schurke!', um zu sagen, dass das, was du getan hast, richtig war?"

"Schulleiter", sagte Harry gleichmütig, "das ist keine Entscheidung, die ich willkürlich getroffen habe. Nein, ich weiß nicht genau, welche Auswirkungen das auf jeden einzelnen der anwesenden Rowdies haben wird. Aber wenn ich immer auf perfekte Informationen warten würde, bevor ich handele, würde ich nie etwas tun. Wenn es um die zukünftige psychologische Entwicklung von, sagen wir, Peregrine Derrick geht, wäre das Verprügeln von acht Erstklässlerinnen wahrscheinlich nicht gut für ihn gewesen. Und es reichte nicht aus, sie ruhig und schnell zu stoppen, denn dann würden sie es später einfach wieder versuchen; sie mussten sehen, dass es eine schützende Macht gab, die es wert war, sich zu fürchten." Harrys Stimme blieb ruhig. "Aber da ich ein guter Kerl \emph{bin}, wollte ich sie natürlich nicht dauerhaft verletzen oder ihnen auch nur Schmerzen zufügen; und doch musste die Strafe so hoch sein, dass sie jeden, der es noch einmal versuchen wollte, noch einmal darüber nachdenken ließ. Nachdem ich also die zu erwartenden Folgen so gut wie möglich mit meinem begrenzt rationalen Verstand abgewogen hatte, hielt ich es für das Klügste, die Tyrannen nackt auszuziehen und sie an die Decke zu kleben."

Der junge Held starrte direkt in den Blick des alten Zauberers, unbeirrbare grüne Augen kreuzten sich mit dem Blau hinter der Brille.

\emph{Und da ich nicht dabei war und nichts persönlich getan habe, gibt es nach den Schulregeln von Hogwarts keine legale Möglichkeit, mich zu bestrafen; der einzige, der gehandelt hat, war Professor Quirrell, und der ist} \emph{bombensicher. Und einfach die Regeln zu brechen, um mich zu erwischen, wäre nicht klug für den Helden, den man für den Kampf gegen Lord Voldemort heranzieht}… Diesmal \emph{hatte} Harry tatsächlich versucht, alle Verästelungen im Voraus zu durchdenken, bevor er Professor Quirrell den Vorschlag gemacht hatte; und ausnahmsweise hatte der Verteidigungsprofessor ihn nicht einen Narren genannt, sondern nur langsam gelächelt und dann zu lachen begonnen.

"Ich verstehe deine Absichten, Harry", sagte der alte Zauberer. "Du denkst, du hast den Schlägern von Hogwarts eine Lektion erteilt. Aber wenn Peregrine Derrick diese Lektion lernen könnte, wäre er nicht Peregrine Derrick. Er wird nur noch mehr provoziert durch das, was du tust - es ist nicht fair, es ist nicht richtig, aber so ist es nun einmal." Der alte Zauberer schloss die Augen, wie in einem kurzen Schmerz, und öffnete sie dann wieder. "Harry, die schmerzhafteste Wahrheit, die jeder Held lernen muss, ist, dass das Recht nicht jeden Kampf gewinnen kann, soll und darf. All dies begann, als Miss~Granger gegen drei ältere Feinde kämpfte und gewann. Hätte sie sich damit begnügt, wäre der Widerhall ihrer Tat mit der Zeit verklungen. Doch stattdessen verbündete sie sich mit ihren Klassenkameraden und erhob ihren Zauberstab in offener Herausforderung gegen Peregrine Derrick und alle seinesgleichen; und seinesgleichen konnte nicht anders, als ihre eigenen Zauberstäbe als Antwort zu erheben. Also ging Jaime Astorga auf die Jagd nach ihr, und im natürlichen Verlauf hätte er sie besiegt; es wäre ein trauriger Tag gewesen, aber es hätte dort geendet. Es gibt nicht genug Magie in acht Hexen des ersten Jahres zusammen, um einen solchen Feind zu besiegen. Aber das konntest du nicht akzeptieren, Harry, konntest nicht zulassen, dass Miss~Granger ihre eigenen Lektionen lernt; und so hast du den Verteidigungsprofessor geschickt, um unsichtbar über sie zu wachen und Astorgas Schilde zu durchbohren, als Daphne Greengrass ihn angriff -"

\emph{Was?} dachte Harry.

Der alte Zauberer fuhr fort zu sprechen. "Jedes Mal, als du dich eingemischt hast, Harry, eskalierte die Sache weiter und weiter. Bald stand Miss~Granger Robert Jugson selbst gegenüber, dem Sohn eines Todessers, mit zwei starken Verbündeten an seiner Seite. Es wäre in der Tat schmerzlich für sie gewesen, wenn Miss~Granger diesen Kampf verloren hätte. Und wieder einmal hat sie durch deinen Willen und Quirinus' Hand, die diesmal offener gezeigt wurde, gewonnen."

Harry kämpfte immer noch mit der Vorstellung, dass der Verteidigungsprofessor unsichtbar über B.E.L.F.E.R. wachte und die Heldinnen vor Schaden bewahrte.

"Und so", beendete der alte Zauberer, "sind wir zu dem heutigen Tag gekommen, Harry, zu vierundvierzig Schülern, die acht Erstjahreshexen angreifen. Eine regelrechte Schlacht in diesen Hallen! Ich weiß, es war nicht deine Absicht, aber du musst ein gewisses Maß an Verantwortung übernehmen. Solche Dinge sind nicht passiert, bevor du an diese Schule kamst, nicht in all meinen Jahrzehnten in Hogwarts; weder als ich Schüler war noch als ich Professor war."

"Ich danke Ihnen sehr", sagte Harry gleichmütig. "Obwohl ich denke, dass Professor Quirrell mehr Anerkennung verdient hat als ich."

Die blauen Augen weiteten sich. "Harry …"

"Diese Brutalos haben schon lange vor diesem Jahr Opfer angegriffen", sagte Harry. Trotz seiner Bemühungen wurde seine Stimme immer lauter. "Aber niemand scheint den Schülern beigebracht zu haben, dass es ihnen erlaubt ist, sich zu \emph{wehren}. Ich weiß, dass es viel schwieriger ist, einen zweiseitigen Kampf zu \emph{ignorieren} als ein paar hilflose Opfer, die verhext oder fast aus dem Fenster gestoßen werden, aber es ist nicht unbedingt \emph{schlimmer}, oder? Ich wünschte, ich hätte mehr von Godric Gryffindors Schriften gelesen, damit ich ihn zitieren könnte, da muss doch etwas darüber stehen. Offener Kampf mag \emph{lauter} sein als die im Stillen leidenden Opfer, es mag schwieriger sein, so zu tun, als ob nichts passiert, aber das Endergebnis ist besser -"

"Nein, ist es nicht", sagte Dumbledore. "Ist es nicht, Harry. Die Dunkelheit \emph{immer} zu bekämpfen, das Böse \emph{nie} unangefochten passieren zu lassen - das ist kein Heldentum, sondern einfacher Stolz. Selbst Godric Gryffindor war nicht der Meinung, dass es sich lohnt, in jedem Krieg zu kämpfen, obwohl er sein ganzes Leben lang von einer Schlacht in die nächste zog." Die Stimme des alten Zauberers wurde leiser. "In Wahrheit, Harry, sind die Worte, die du sprichst nicht böse. Nein, nicht böse, und doch haben sie mich erschreckt. Du bist einer, der eines Tages große Macht haben könnte, über die Zauberei, über deine Mitzauberer. Und wenn du dann immer noch denkst, dass das Böse niemals unangefochten bleiben darf -" Jetzt war ein Hauch von echter Sorge in die Stimme des Schulleiters gekommen. "Die Welt ist zerbrechlicher geworden seit der Zeit, in der Hogwarts erbaut wurde; ich fürchte, sie kann den Zorn eines weiteren Godric Gryffindor nicht ertragen. Und er war in seinem Zorn langsamer als du." Der alte Zauberer schüttelte den Kopf. "Du bist zu kampfbereit, Harry. Viel zu kampfbereit, und Hogwarts selbst wird um dich herum zu einem immer gewalttätigeren Ort."

"Nun", sagte Harry vorsichtig, nachdem er seine Worte abgewogen hatte. "Ich weiß nicht, ob es hilft, das zu sagen, aber ich glaube, Sie bekommen einen falschen Eindruck davon, worum es mir geht. Ich mag auch keine richtigen Kämpfe. Es ist beängstigend und gewalttätig, und jemand könnte verletzt werden. Aber ich habe heute \emph{nicht} gekämpft, Herr Direktor."

Der Schulleiter runzelte die Stirn. "Du hast den Verteidigungsprofessor an deiner Stelle geschickt -"

"Professor Quirrell hat auch nicht gekämpft", sagte Harry ruhig. "Es war niemand da, der stark genug war, um gegen ihn zu kämpfen. Was heute passiert ist, war kein Kämpfen, sondern ein Gewinnen."

Es dauerte eine Weile, bis der alte Zauberer das Wort ergriff. "Das mag sein, wie es will", sagte der Schulleiter, "aber all diese Konflikte müssen ein Ende haben. Ich kann die Anspannung in der Luft hören, und mit jedem dieser Zusammenstöße steigt sie an. All das muss ein Ende haben, und zwar ein entschiedenes und baldiges; du darfst dem Ende nicht im Wege stehen."

Der alte Zauberer gestikulierte in Richtung der großen Eichentür seines Büros, und Harry ging durch sie hinaus.

Mit einiger Überraschung trat Harry zwischen den riesigen grauen Wasserspeiern hervor, die ihm Platz gemacht hatten, und sah, dass Quirinus Quirrell immer noch gegen den Stein der Korridorwand gelehnt war, ein dicker Speichelfaden tropfte aus seinem schlaffen Mund auf seine Professorenrobe, in genau derselben Position, die er eingenommen hatte, als Harry vorhin das Büro des Schulleiters betreten hatte.

Harry wartete, aber der zusammengesackte Mann erhob sich nicht; und nach langen unbehaglichen Sekunden begann Harry, wieder den Korridor hinunterzugehen.

"Mr~Potter?", kam ein leiser Ruf, nachdem Harry um zwei Ecken gebogen war; eine leise Stimme, die unnatürlich durch die Flure trug.

Als Harry zurückkam, fand er Professor Quirrell immer noch an die Wand gelehnt, aber die blassen Augen beobachteten ihn jetzt mit scharfer Intelligenz.

\emph{Es tut mir leid, dass ich Sie ermüdet habe} -

Es war etwas, das Harry nicht aussprechen konnte. Er hatte die Korrelation zwischen der Anstrengung, die Professor Quirrell aufwandte, und der Zeit, die er mit "Ausruhen" verbringen musste, bemerkt. Aber Harry hatte sich überlegt, dass, wenn die Anstrengung zu schmerzhaft oder schädlich war, Professor Quirrell sicher einfach nein sagen würde. Jetzt fragte sich Harry, ob diese Überlegung tatsächlich richtig gewesen war, und wenn nicht, wie er sich entschuldigen sollte…

Der Verteidigungsprofessor sprach mit ruhiger Stimme, der Rest des Körpers unbewegt. "Wie verlief Ihr Treffen mit dem Schulleiter, Mr~Potter?"

"Ich bin mir nicht sicher", sagte Harry. "Nicht so, wie ich es vorausgesagt habe. Er scheint zu glauben, dass das Licht viel öfter verlieren sollte, als ich es für klug halten würde. Außerdem bin ich mir nicht sicher, ob er den Unterschied zwischen dem Versuch zu kämpfen und dem Versuch zu gewinnen versteht. Das erklärt eigentlich eine Menge …" Harry hatte nicht viel über den Zaubererkrieg gelesen, aber er hatte genug gelesen, um zu wissen, dass die Guten sich wahrscheinlich ein ziemlich genaues Bild davon gemacht \emph{hatten}, wer die meisten der schlimmsten Todesser waren, und dass sie \emph{nicht} einfach nach fünf Minuten allen eine Eule mit einer Handgranate geschickt hatten.

Ein leises, sanftes Lachen kam von den blassen Lippen. "Dumbledore begreift die Freude am Gewinnen nicht, genauso wenig wie er die Freude am Spiel begreift. Sagen Sie mir, Mr~Potter. Haben Sie diesen kleinen Plan mit der bewussten Absicht vorgeschlagen, meine Langeweile zu lindern?"

"Das war einer meiner vielen Beweggründe", sagte Harry, denn irgendein Instinkt hatte ihn gewarnt, dass er nicht einfach \emph{Ja} sagen konnte.

"Wissen Sie", sagte der Verteidigungsprofessor in sanftem, nachdenklichem Ton, "es gibt Menschen, die versucht haben, meine dunklen Stimmungen zu mildern, und solche, die tatsächlich dazu beigetragen haben, meinen Tag aufzuhellen, aber Sie sind die erste Person, der es gelungen ist, dies absichtlich zu tun." Der Verteidigungsprofessor schien sich mit einer merkwürdigen Bewegung, die sowohl Magie als auch Muskeln beinhaltet haben könnte, von der Wand aufzurichten; und begann wegzugehen, ohne einen Blick zurück in Harrys Richtung zu werfen. Nur eine kleine Geste eines Fingers zeigte an, dass Harry ihm folgen sollte.

"Die Beschwörung, die Sie für Miss~Davis komponiert haben, hat mir besonders gut gefallen", sagte Professor Quirrell, nachdem sie eine kurze Strecke gegangen waren. "Obwohl es vielleicht klüger gewesen wäre, mich vorher zu konsultieren, bevor Sie sie ihr zum Auswendiglernen gaben." Eine Hand wanderte in die Robe des Verteidigungsprofessors und zog einen Zauberstab hervor, der eine kleine Geste in der Luft nachzeichnete, woraufhin alle fernen Geräusche des Schlosses Hogwarts verstummten. "Sagen Sie mir ehrlich, Mr~Potter, haben Sie sich irgendwie mit der Theorie der dunklen Rituale vertraut gemacht? Das ist nicht dasselbe wie das Eingeständnis einer Absicht, sie zu wirken; viele Zauberer kennen die Prinzipien."

"Nein…" sagte Harry langsam. Er hatte sich schon vor einiger Zeit dagegen entschieden, zu versuchen, sich in die gesperrte Abteilung der Hogwarts-Bibliothek zu schleichen, aus demselben Grund, aus dem er sich ein Jahr zuvor \emph{dagegen} entschieden hatte, nachzuschlagen, wie man aus gewöhnlichen Haushaltsmaterialien Sprengstoff herstellen konnte. Harry war stolz darauf, \emph{etwas} \emph{mehr} gesunden Menschenverstand zu haben, als die Leute dachten.

"Oh?", sagte Professor Quirrell. Der Mann ging jetzt normaler, und die Lippen bogen sich zu einem eigentümlichen Lächeln. "Dann haben Sie wohl ein Naturtalent für dieses Gebiet."

"Ja, nun", sagte Harry müde. "Ich nehme an, Dr~Seuss hat auch ein Naturtalent für dunkle Rituale, denn der Teil mit dem \emph{Shuffle, duffle, muzzle,} \emph{muff} stammt aus einem Kinderbuch namens \emph{Bartholomäus und der Oobleck} -"

"Nein, nicht dieser Teil", sagte Professor Quirrell. Seine Stimme wurde etwas fester, nahm ihren normalen, belehrenden Tonfall an. "Ein gewöhnlicher Zauberspruch, Mr~Potter, kann nur durch das Sprechen bestimmter Worte, durch präzise Bewegungen des Zauberstabs und durch den Einsatz von etwas eigener Kraft herbeigeführt werden. Sogar mächtige Zauber können auf diese Weise beschworen werden, wenn die Magie sowohl effizient als auch wirkungsvoll ist. Aber bei den größten Zaubern reicht die Sprache allein nicht aus, um ihnen Struktur zu geben. Sie müssen bestimmte Handlungen ausführen, wichtige Entscheidungen treffen. Auch reicht der vorübergehende Einsatz der eigenen Kraft nicht aus, um sie in Gang zu setzen; ein Ritual erfordert permanente Opfer. Die Macht eines solchen größeren Zaubers kann im Vergleich zu gewöhnlichen Zaubern wie der Tag im Vergleich zur Nacht sein. Aber viele Rituale - ja, die meisten - verlangen zumindest ein Opfer, das Zimperlichkeit hervorrufen könnte. Und so wird der gesamte Bereich der rituellen Magie, der die weitesten und interessantesten Bereiche der Zauberei umfasst, weithin als Dunkel angesehen. Mit ein paar Ausnahmen, die durch die Tradition herausgearbeitet wurden, versteht sich." Professor Quirrells Stimme bekam einen sardonischen Beigeschmack. "Der Unbrechbare Schwur ist für bestimmte wohlhabende Häuser zu nützlich, als dass er gänzlich geächtet werden könnte - obwohl es in der Tat ein furchtbarer und schrecklicher Akt ist, den Willen eines Menschen für alle Tage zu binden, furchterregender als viele geringere Rituale, die Zauberer meiden. Ein Zyniker könnte zu dem Schluss kommen, dass die Frage, welche Rituale verboten sind, nicht so sehr eine Frage der Moral ist, sondern der Gewohnheit. Aber ich schweife ab…" Professor Quirrell gab ein kurzes Hustengeräusch von sich, ein Räuspern. "Der Unbrechbare Schwur erfordert drei Teilnehmer und drei Opfer. Derjenige, der den Unbrechbaren Schwur empfängt, muss jemand sein, der dem Schwörenden hätte vertrauen können, sich aber stattdessen dafür entscheidet, den Schwur von ihm zu verlangen, und er opfert diese Möglichkeit des Vertrauens. Derjenige, der das Gelübde ablegt, muss jemand sein, der sich hätte entscheiden können, das zu tun, was das Gelübde von ihm verlangt, und er opfert diese Möglichkeit der Entscheidung. Und der dritte Zauberer, der Binder, opfert permanent einen kleinen Teil seiner eigenen Magie, um das Gelübde für immer aufrechtzuerhalten."

"Ah", sagte Harry. "Ich habe mich schon \emph{gewundert}, warum dieser Zauber nicht überall eingesetzt wird, immer dann, wenn zwei Menschen Schwierigkeiten haben, einander zu vertrauen … obwohl … warum verlangen Zauberer auf dem Sterbebett nicht Geld, um unbrechbare Schwüre zu binden, und verwenden es, um ihren Kindern ein Erbe zu hinterlassen -"

"Weil sie dumm sind", sagte Professor Quirrell. "Es gibt Hunderte von nützlichen Ritualen, die durchgeführt werden könnten, wenn die Menschen mehr Verstand hätten; ich könnte zwanzig nennen, ohne Luft zu holen. Aber wie dem auch sei, Mr~Potter, die Sache mit solchen Ritualen - ob Sie sie nun als dunkel bezeichnen wollen oder nicht - ist, dass sie so gestaltet sind, dass sie magisch wirksam sind und nicht, dass sie beeindruckend wirken, wenn sie ausgeführt werden. Ich nehme an, es gibt eine gewisse Tendenz, dass die mächtigeren Rituale schrecklichere Opfer erfordern. Dennoch verlangt das schrecklichste mir bekannte Ritual nur einen Strick, an dem ein Mann gehängt wurde, und ein Schwert, mit dem eine Frau erschlagen wurde; und das für ein Ritual, das versprach, den Tod selbst herbeizurufen - obwohl ich nicht weiß, was damit wirklich gemeint ist, und ich es auch nicht herauszufinden will, da es auch hieß, dass der Gegenzauber zur Entlassung des Todes verloren gegangen ist. Der furchterregendste Zauberspruch, der mir je begegnet ist, klingt nicht einmal ein Hundertstel so furchterregend wie der, den Sie für Miss~Davis komponiert haben. Diejenigen unter den Schlägern, die flüchtig mit dunklen Ritualen vertraut waren - und ich bin mir sicher, dass es einige gab - müssen einen Schrecken empfunden haben, den man mit Worten nicht beschreiben kann. Wenn es ein echtes Ritual gäbe, das so beeindruckend aussähe, Mr~Potter, würde es die Erde zum Schmelzen bringen."

"Ähm", sagte Harry.

Professor Quirrells Lippen verzogen sich weiter. "Ah, aber das wirklich Amüsante war Folgendes. Sehen Sie, Mr~Potter, die Beschwörung eines jeden Rituals benennt das, was geopfert werden soll, und das, was gewonnen werden soll. Der Gesang, den Sie Miss~Davis gaben, sprach erstens von einer Dunkelheit jenseits der Dunkelheit, begraben unter dem Fluss der Zeit, die das Tor kennt und das Tor ist. Und das zweite, wovon Sie sprachen, Mr~Potter, war die Manifestation Ihrer eigenen Gegenwart. Und immer, in jedem Ritual, wird \emph{zuerst} das genannt, was geopfert wird, und \emph{danach} wird der Gebrauch gesagt, der ihm befohlen wird."

"Ich … verstehe", sagte Harry, während er hinter Professor Quirrell durch die Hallen von Hogwarts schritt und ihm in Richtung des Büros des Verteidigungsprofessors folgte. "Mein Spruch, so wie ich ihn geschrieben habe, impliziert also, dass der Äußere Gott, Yog-Sothoth -"

"Permanent in einem Ritual geopfert wurde, das nur kurz Ihre Anwesenheit hervorgerufen hat", sagte Professor Quirrell. "Ich nehme an, wir werden morgen herausfinden, ob das irgendjemand ernst genommen hat, wenn wir die Zeitungen lesen und sehen, ob sich alle magischen Nationen der Welt in einer verzweifelten Anstrengung zusammenschließen, um Ihren Einfall in unsere Realität abzuschotten."

Sie gingen weiter, während der Verteidigungsprofessor zu glucksen begann, seltsame kehlige Laute.

Die beiden sprachen danach nicht mehr, bis sie zum Büro des Verteidigungsprofessors kamen, und dann hielt der Mann mit der Hand an der Tür inne.

"Es ist eine sehr seltsame Sache", sagte der Verteidigungsprofessor, seine Stimme war nun wieder leise, fast unhörbar. Der Mann sah Harry nicht an, und Harry sah nur seinen Rücken. "Eine sehr seltsame Sache… Es gab eine Zeit, da hätte ich einen Finger meiner Zauberstabhand geopfert, um die Schläger von Hogwarts so zu bearbeiten, wie wir es heute getan haben. Um sie dazu zu bringen, mich so zu fürchten, wie sie Sie jetzt fürchten, um die Achtung aller Schüler und die Verehrung vieler zu haben, dafür hätte ich meinen Finger gegeben. Sie haben jetzt alles, was ich damals wollte. Alles, was ich über die menschliche Natur weiß, sagt, dass ich Sie hassen sollte. Und doch tue ich es nicht. Es ist eine sehr seltsame Sache."

Es hätte ein rührender Moment sein sollen, aber stattdessen fühlte Harry, wie ihm eine Kälte den Rücken hinunterlief, als wäre er ein kleiner Fisch im Meer, und ein riesiger weißer Hai hätte ihn gerade begutachtet und nach einem sichtbaren Zögern beschlossen, ihn nicht zu fressen.

Der Mann öffnete die Tür zum Büro des Verteidigungsprofessors, ging hinein und war verschwunden.

\emph{\emph{Nachspiel:}}

Die anderen Slytherins sahen Daphne an, als … als wüssten sie nicht im Geringsten, wie sie sie ansehen sollten.

Die Gryffindors sahen sie an, als wüssten sie nicht im Geringsten, wie sie sie ansehen sollten.

Ohne Furcht schritt Daphne Greengrass in das Klassenzimmer für Zaubertränke, eingehüllt in die gebieterische Würde eines Noblen und Uralten Hauses. Innerlich fühlte sie sich genauso wie alle anderen wahrscheinlich auch.

Es waren schon zwei Stunden vergangen, seit das \emph{Was?} passiert war, und Daphnes Gehirn arbeitete immer noch: \emph{Was? Was? Was?}

Im Klassenzimmer war es still, als sie alle auf Professor Snape warteten. Lavender und Parvati saßen in der Nähe einer Gruppe von anderen Gryffindors, umgeben von schweigenden Blicken. Die beiden kontrollierten gegenseitig ihre Hausaufgaben, bevor der Unterricht begann, und niemand sonst half ihnen oder sprach mit ihnen. Selbst Lavender, von der Daphne geschworen hätte, dass sie sich durch nichts aus der Ruhe bringen ließ, wirkte bedrückt.

Daphne setzte sich an ihr Pult, holte \emph{Magische} \emph{Elixiere} \emph{und Zaubertränke} aus ihrer Tasche und begann, ihre eigenen Hausaufgaben durchzusehen, wobei sie ihr Bestes gab, sich normal zu verhalten. Die Leute starrten sie an, sagten aber nichts -

Ein Keuchen ging durch das ganze Klassenzimmer. Mädchen und Jungen zuckten zurück und lehnten sich von der Tür weg, als wären sie Weizenhalme, die von einem Windstoß berührt wurden.

In der Tür stand Tracey Davis, eingehüllt in einen schwarzen, zerfledderten Umhang, der über ihre Hogwarts-Uniform drapiert worden war.

Tracey ging langsam in das Klassenzimmer, schwankte leicht bei jedem Schritt und sah aus, als würde sie versuchen, sich \emph{treiben} zu lassen. Sie setzte sich an ihr gewohntes Pult, das zufällig direkt neben dem von Daphne stand.

Langsam drehte sich Traceys Kopf, um Daphne anzustarren.

"Siehst du?", sagte das Slytherin-Mädchen in einem tiefen, düsteren Ton. "Ich habe dir doch gesagt, dass ich ihn vor ihr kriege."

"Was?", platzte Daphne heraus, die sich sofort wünschte, sie hätte nichts gesagt.

"Ich habe Harry Potter bekommen, bevor Granger es tat." Traceys Stimme war immer noch leise, aber ihre Augen funkelten vor Triumph. "Siehst du, Daphne, was General Potter von einem Mädchen will, ist kein hübsches Gesicht oder ein hübsches Kleid. Er will ein Mädchen, das bereit ist, seine schrecklichen Kräfte zu kanalisieren. Jetzt bin ich sein - und er ist mein!"

Diese Ankündigung erzeugte eine eisige Stille im ganzen Klassenzimmer.

"Entschuldigen Sie, Miss~Davis", sagte die kultivierte Stimme von Draco Malfoy, der scheinbar unbeteiligt seine eigenen Pergamenten für Zaubertränke durchging. Der andere Spross eines Noblen und Uralten Hauses blickte nicht einmal von seinem Schreibtisch auf, selbst als sich alle anderen zu ihm umdrehten. "Hat Harry Potter dir das wirklich \emph{gesagt}? Mit diesen Worten?"

"Nun, nein…" sagte Tracey, und dann blitzten ihre Augen wütend auf. "Aber er sollte mich \emph{besser} nehmen, jetzt, wo ich ihm meine Seele geopfert habe und so!"

"\emph{Du hast deine Seele Harry Potter geopfert?} ", keuchte Millicent. Von der anderen Seite des Raumes ertönte ein Klappern, als Ron Weasley sein Tintenfass fallen ließ.

"Nun, ich bin mir ziemlich sicher, dass ich das getan habe", sagte Tracey und klang kurz unsicher, bevor sie sich wieder aufrichtete. "Ich meine, ich habe mich in einem Spiegel betrachtet und ich sehe jetzt blasser aus, und ich kann immer spüren, dass mich Dunkelheit umgibt, und ich war ein Kanal für seine schrecklichen Kräfte und alles… Daphne, du hast auch gesehen, dass meine Augen grün geworden sind, richtig? Ich habe es nicht selbst gesehen, aber das ist es, was ich danach gehört habe."

Es gab eine Pause, die nur durch die Geräusche von Ron Weasley unterbrochen wurde, der versuchte, seinen Schreibtisch aufzuräumen.

"Daphne?", sagte Tracey.

"Ich glaube es nicht", sagte eine wütende Stimme. "Auf keinen Fall würde der nächste Dunkle Lord \emph{dich} zu seiner Braut nehmen!"

Langsam und mit erheblichem Unglauben drehten sich die Köpfe, um Pansy Parkinson anzustarren.

"Sei still, du", sagte Tracey, "oder ich werde …" Das Slytherin-Mädchen hielt inne. Dann wurde Traceys Stimme noch tiefer und sie sagte: "Still, du, oder ich werde deine Seele verschlingen."

"Das kannst du gar nicht", sagte Pansy mit dem selbstsicheren Ton einer Henne, die sich eine perfekte Hackordnung zurechtgelegt hatte, in der sie an der Spitze stand, und die nicht vorhatte, diese Überzeugung aufgrund bloßer Beweise zu aktualisieren.

Langsam, als würde sie versuchen zu schweben, stand Tracey von ihrem Schreibtisch auf. Es gab weitere Atemzüge. Daphne fühlte sich, als wäre sie auf ihrem Stuhl versteinert worden.

"Tracey?", sagte Lavender mit leiser Stimme. "Bitte mach das alles nicht noch einmal. Bitte?"

Jetzt zeigte Pansy deutliche Nervosität, als Tracey auf ihren Schreibtisch zuschwankte. "Was glaubst du, was du da tust?" sagte Pansy und schaffte es nicht ganz, entrüstet zu klingen.

"Ich habe es dir gesagt", sagte Tracey bedrohlich. "Ich werde deine Seele verschlingen."

Tracey beugte sich über Pansy, die wie erstarrt an ihrem Schreibtisch saß, und machte, während sich ihre Lippen fast berührten, ein lautes Einatmungsgeräusch.

"Da!", sagte Tracey, als sie sich aufrichtete. "Ich habe deine Seele gegessen."

"Nein, hast du nicht!", sagte Pansy.

"Hab ich doch!", sagte Tracey.

Es gab eine kleine Pause.

"Bei Merlin, sie \emph{hat} es getan! ", rief Theodore Nott. "Du siehst jetzt ganz blass aus, und deine Augen scheinen leer zu sein!"

"\emph{Was?}", kreischte Pansy und wurde blass. Das Mädchen sprang von ihrem Schreibtisch auf und begann verzweifelt in ihrer Büchertasche zu wühlen. Nachdem Pansy einen Spiegel hervorgeholt und sich selbst betrachtet hatte, wurde sie noch blasser.

Daphne gab jeden Anschein aristokratischer Haltung auf und ließ ihren Kopf mit einem dumpfen Schlag auf den Schreibtisch fallen, während sie sich fragte, ob es sich wirklich lohnte, mit all den anderen wichtigen Familien auf dieselbe Schule zu gehen wie die Chaoslegion.

"Oh, jetzt steckst du in Schwierigkeiten, Pansy", sagte Seamus Finnigan. "Ich weiß nicht genau, was passiert, wenn dich ein Dementor küsst, aber wenn dich Tracey Davis küsst, ist das wahrscheinlich noch schlimmer."

"Ich habe schon von Leuten ohne Seele gehört", sagte Dean Thomas düster. "Sie müssen sich ganz in Schwarz kleiden, und sie schreiben schreckliche Gedichte, und nichts macht sie jemals glücklich. Sie sind total \emph{emo}."

"Ich will nicht emo sein!", rief Pansy.

"Zu schade", sagte Dean Thomas. "Das musst du sein, jetzt wo deine Seele weg ist."

Pansy drehte sich um und streckte eine flehende Hand nach Draco Malfoys Schreibtisch aus. "Draco!", sagte sie flehentlich. "Mr~Malfoy! Bitte, sorgen Sie dafür, dass Tracey mir meine Seele zurückgibt!"

"Das kann ich nicht", sagte Tracey. "Ich habe sie \emph{gegessen}."

"Mach, dass sie sie ausspuckt!", schrie Pansy.

Der Erbe von Malfoy war nach vorne gesackt und stützte seinen Kopf in beide Hände, so dass niemand sein Gesicht sehen konnte. "Warum ist mein Leben nur so?", fragte Draco Malfoy.

Ein wildes Getuschel setzte ein, als Tracey zu ihrem Pult zurückkehrte und nun zufrieden lächelte, während Pansy in der Mitte des Klassenzimmers stand, die Hände rang und ihr die Tränen aus den Augen liefen.

"\emph{Seid. Ruhig.}"

Die sanfte, tödliche Stimme schien das ganze Klassenzimmer zu erfüllen, als Professor Snape durch die Tür hereinpirschte. Sein Gesicht war wütender, als Daphne es je gesehen hatte, und sandte ihr einen Schauer echter Angst über den Rücken. Hastig blickte sie auf ihre Hausaufgaben hinunter.

"Setzen Sie sich, Parkinson", zischte der Meister der Zaubertränke, "und Sie, Davis, nehmen Sie diesen lächerlichen Umhang ab -"

"\emph{Professor Snaaaaaape!}", jammerte Pansy Parkinson unter Tränen. "\emph{Tracey hat mein} \emph{Seeeehhheeele} \emph{gegessen!} "

