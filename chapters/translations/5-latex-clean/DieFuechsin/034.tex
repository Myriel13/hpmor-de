

\hypertarget{abstimmungsprobleme-teil-2}{% \section{2. Abstimmungsprobleme, Teil 2}\label{abstimmungsprobleme-teil-2}}

—\/-\/-\/-\/- Kapitel 34: Abstimmungsprobleme, Teil 2 -\/-\/-\/-\/-

Minerva und Dumbledore hatten ihre kombinierten Talente eingesetzt um die große Bühne zu beschwören auf die Quirrell jetzt langsam zu trottete. Es war im Kern stabiles Holz, aber die äußere Oberfläche schien mit dem Glanz von Marmor, eingelegt mit Platin und geschmückt mit Edelsteinen von jeder Hausfarbe. Weder sie noch der Schulleiter waren Gründer von Hogwarts, aber die Beschwörung hatte nur einige Stunden benötigt. Minerva freute sich gewöhnlich über die wenigen Anlässe, an denen sie Gelegenheit hatte sich an großen Verwandlungen zu verausgaben. Ihr hätten die vielen kleinen Gelegenheiten für Kunstfertigkeit gefallen sollen sowie die Illusion von Reichtum, aber dieses Mal hatte sie bei der Arbeit das schreckliche Gefühl ihr eigenes Grab auszuheben.

Aber Minerva ging es jetzt ein wenig besser. Es gab einen kurzen Moment, an dem die Explosion hätte kommen können, aber Dumbledore war bereits aufgestanden und applaudierte warmherzig und niemand hatte sich als töricht genug erwiesen vor dem Schulleiter zu revoltieren.

Und die explosive Stimmung verblasste rapide zu einem kollektiven Gefühl, das sich vielleicht mit der Phrase „\emph{Gib uns 'ne Pause!}“ beschreiben lässt.

Blaise Zabini hatte sich selbst im Namen von Sonnenschein erschossen und der Endpunktestand war 254 zu 254 zu 254.

\_\_\_\_\_\_\_\_\_\_\_\_\_\_\_\_\_\_\_\_\_\_\_\_\_\_\_\_\_\_\_\_\_\_\_\_\_\_\_\_

Hinter der Bühne, während sie darauf warteten heraufzusteigen, starrten sich drei Kinder mit einer Mischung aus Wut und Frustration an. Es machte es nicht besser, dass sie immer noch nass waren, weil sie grade erst aus dem See gefischt worden waren und dass die Wärmezauber nicht gegen die eiskalte Dezemberluft ankamen. Aber vielleicht war es auch nur ihre Stimmung.

„Mir \emph{reicht's},“ sagte Granger. „Ich habe genug! Keine weiteren Verräter!“

„Ich stimme Ihnen vollkommen zu, Miss~Granger,“ sagte Draco mit eisiger Stimme. „Genug ist genug.“

„Und was beabsichtigt \emph{ihr} beiden dagegen zu tun?“ blaffte Harry Potter. „Professor Quirrell“ hat bereits gesagt, dass er Spione nicht verbieten wird.

„Wir werden sie \emph{für} ihn verbieten,“ meinte Draco grimmig. Er hatte selbst nicht verstanden was er mit diesen Worten meinte als er sie sagte, aber durch den Akt des Sprechens schien sich ein Plan herauszukristallisieren—

Die Bühne war wirklich gut gemacht, zumindest für eine temporäre Struktur. Die Erschaffer waren nicht dem gewöhnlichen Trugschluss anheimgefallen von ihrer eigenen Illusion von Reichtum beeindruckt zu sein und wussten scheinbar einiges über Architektur und optischen Stil.

Dort wo Draco stand, am für ihn offensichtlichen Platz, würden ihn die zuschauenden Schüler sehen wie er vom schwachen Glitzer von Smaragden beschienen wurde; und Granger, wenn sie da stand wo er sie subtil hin gewinkt hatte würde von Ravenclaws Saphiren beschienen. Was Harry Potter betraf, Draco schaute ihn grade nicht an.

Professor Quirrell war…erwacht, oder was immer das war was er tat; und lehnte sich auf ein platinfarbenes Podium ohne jegliche Edelsteine. Mit offensichtlichem schauspielerischem Geschick mischte und stapelte der Verteidigungsprofessor die drei Briefumschläge, die die drei Pergamente enthielten, auf die die drei Generäle ihre Wünsche geschrieben hatten, während alle Hogwartsstudenten zuschauten und warteten.

Endlich sah Professor Quirrel von den Umschlägen auf. „Okay“, sagte der Verteidigungsprofessor „Das ist jetzt ungünstig.“

Ein leichtes Gekicher ging durch die Menge, mit einem scharfen Unterton.

„Ich nehme an ihr alle fragt euch was ich tun werde?“ sagte Professor Quirrel. „Es bleibt nichts anderes übrig; ich werde tun, was fair ist. Obwohl ich zuerst noch eine kleine Rede halten wollte, und davor wiederum so scheint es mir haben Mr~Malfoy und Miss~Granger etwas zu verkünden.“

Draco blinzelte und dann tauschten er und Granger schnell einige Blicke aus \emph{— soll ich?—ja, geh vor —} und Draco erhob seine Stimme.

„General Granger und ich würden beide gerne erklären“, sagte Draco mit seiner formalsten Stimme, da er wusste es wurde verstärkt und gehört, „dass wir nicht länger die Hilfe jedweder Verräter akzeptieren werden. Und wenn wir, in irgendeinem Kampf, herausfinden, dass Potter Verräter aus einer unserer Armeen akzeptiert hat, werden wir unsere Kräfte vereinen und ihn vernichten.“

Und Draco warf einen boshaften Blick zum Jungen-der-überlebte. \emph{Nimm das General Chaos!}

“Ich stimme General Malfoy vollkommen zu,“ sagte Granger neben ihm stehend, ihre Stimme klar und stark. „Keiner von uns wird Verräter benutzen und wenn General Potter es tut, werden wir ihn vom Schlachtfeld fegen.“

Es gab ein überraschtes Geflüster von den zuschauenden Schülern.

„Sehr gut,“ sagte der Verteidigungsprofessor lächelnd. „Ihr beide habt lange genug dafür gebraucht, aber ich gratuliere euch trotzdem da ihr vor allen anderen Generälen daran gedacht habt.“

Es brauchte einen Moment um einzusinken—

„In der Zukunft, Mr~Malfoy, Miss~Granger, bevor Sie mit einer Anfrage in mein Büro kommen, überlegen Sie, ob es einen Weg gibt es ohne meine Hilfe zu erreichen. Ich werde dieses Mal keine Quirrel-Punkte dafür abziehen, aber beim nächsten Mal können Sie mit den vollen fünfzig rechnen.„ Professor Quirrel setzte ein amüsiertes Grinsen auf. „Und was haben Sie dazu zu sagen Mr~Potter?“

Harry Potters Blick wanderte zu Granger, dann zu Draco. Sein Gesichtsausdruck erschien ruhig, obwohl Draco sicher war, dass \emph{kontrolliert} der bessere Ausdruck war.

Schließlich sprach Harry Potter mit leiser Stimme. „Die Chaos Legion wird weiterhin mit Freude Verräter akzeptieren. Wir sehen uns auf dem Schlachtfeld.“

Draco wusste, dass der Schock von seinem eigenen Gesicht abzulesen war. Es gab erstauntes Gemurmel von den beobachtenden Schülern, und als Draco in die erste Reihe blickte, sah er, dass sogar Harrys Chaoten verblüfft aussahen.

Grangers Gesicht war wütend und wurde immer wütender. „Mr~Potter“, sagte sie in einem scharfen Ton, als ob sie dächte, sie wäre eine Lehrerin, „\emph{versuchst} du, unausstehlich zu sein?“

„Ganz sicher nicht“, sagte Harry Potter ruhig. „Ich werde dich nicht zwingen, es jedes Mal zu tun. Wenn du mich einmal besiegst, werde ich geschlagen bleiben. Aber Drohungen sind nicht immer genug, General von Sonnenschein. Du hast mich nicht gebeten, mich dir anzuschließen, sondern einfach versucht, deinen Willen durchzusetzen; und manchmal musst du den Feind tatsächlich besiegen, um ihm deinen Willen aufzuzwingen. Weißt du, ich bin skeptisch, dass Hermine Granger, der klügste akademische Stern von Hogwarts, und Draco, Sohn von Lucius, Nachkomme des Noblen und Uralten Hauses von Malfoy, zusammenarbeiten können, um ihren gemeinsamen Feind Harry Potter zu besiegen.“ Ein amüsiertes Lächeln kreuzte Harry Potters Gesicht. „Vielleicht tue ich einfach, was Draco mit Zabini versucht hat, und schreibe einen Brief an Lucius Malfoy und schaue, was \emph{er} darüber denkt.“

„\emph{Harry}! “ keuchte Granger, die absolut entsetzt aussah, und auch im Publikum gab es ein Aufkeuchen.

Draco kontrollierte die Wut, die durch ihn hindurchfloss. Das war ein \emph{dummer} Zug von Harrys Seite gewesen, das in der Öffentlichkeit zu sagen. Wenn Harry es einfach \emph{getan} hätte, hätte es vielleicht funktioniert, Draco hatte nicht einmal darüber nachgedacht, aber wenn Vater es \emph{jetzt} tun würde, würde es so aussehen, als würde er Harry in die Hände spielen—

„Wenn du denkst, dass mein Vater, Lord Malfoy, von \emph{dir} so leicht manipuliert werden kann“, sagte Draco kalt, „hast du eine Überraschung vor dir, Harry Potter.“

Und Draco erkannte, nachdem die Worte aus seinem Mund gekommen waren, dass er gerade \emph{seinen eigenen Vater} direkt in die Ecke gedrängt hatte, mehr oder weniger ohne es überhaupt zu wollen. Vater würde das wahrscheinlich \emph{nicht} gefallen, nicht im Geringsten, aber jetzt wäre es unmöglich für ihn, das zu sagen…. Draco würde sich dafür entschuldigen, es \emph{war} wirklich ein Unfall gewesen, aber es war seltsam zu denken, dass er es überhaupt getan hatte.

„Dann mach weiter und besiege den bösen General Chaos“, sagte Harry und sah immer noch amüsiert aus. „Ich kann nicht gegen eure beiden Armeen gewinnen - nicht, wenn ihr \emph{wirklich} zusammenarbeitet. Aber ich frage mich, ob ich euch vielleicht vorher auseinanderbringen kann.“

„Das wirst du nicht, und wir werden dich \emph{vernichten}!“, sagte Draco Malfoy.

Und neben ihm nickte Hermine Granger fest.

„Nun“, sagte Professor Quirrell, nachdem sich die erstaunte Stille eine Weile hingezogen hatte. „ich habe nicht erwartet, dass dieses spezielle Gespräch so verlaufen würde.“ Der Verteidigungsprofessor hatte einen ziemlich faszinierten Gesichtsausdruck. „Ehrlich gesagt, Mr~Potter, erwartete ich von Ihnen, dass Sie sofort und mit einem Lächeln nachgeben und dann verkünden würden, dass Sie meine beabsichtigte Lektion längst begriffen hätten, aber beschlossen haben, sie nicht für andere zu ruinieren. Tatsächlich habe ich meine Rede entsprechend geplant, Mr~Potter.“

Harry Potter zuckte nur mit den Schultern. „Tut mir leid“, sagte er und sagte nichts weiter.

„Oh, keine Sorge“, sagte Professor Quirrell. „Das wird auch gehen.“

Und Professor Quirrell wandte sich von den drei Kindern ab und richtete sich auf dem Podium auf, um die ganze zuschauende Menge anzusprechen; seine übliche Aura abgeklärter Belustigung fiel von ihm ab wie eine fallende Maske, und als er wieder sprach, wurde seine Stimme lauter verstärkt als vorher.

„Wenn Harry Potter nicht gewesen wäre“, sagte Professor Quirrell, seine Stimme so eiskalt wie die Dezemberluft, „hätte Ihr-wisst-schon-wer gewonnen.“

Die Stille war sofort und total.

\_\_\_\_\_\_\_\_\_\_\_\_\_\_\_\_\_\_\_\_\_\_\_\_\_\_\_\_\_\_\_\_\_\_\_\_\_\_\_\_

„Macht keinen Fehler“, sagte Professor Quirrell. „Der Dunkle Lord \emph{war} \emph{dabei} zu gewinnen. Es gab immer weniger Auroren, die es wagten, sich ihm entgegen zu stellen, die Bürgerwehrler, die sich ihm widersetzten, wurden gejagt. Ein Dunkler Lord und vielleicht fünfzig Todesser \emph{siegten} gegen ein Land mit Tausenden. Das ist mehr als lächerlich! Es gibt keine Noten, die niedrig genug sind, um diese Inkompetenz zu markieren!“

Es gab ein Stirnrunzeln im Gesicht von Schulleiter Dumbledore; und auf den Gesichtern des Publikums, Verwirrung; und die völlige Stille hielt an.

„Versteht ihr jetzt, wie es passiert ist? Ihr habt es heute gesehen. Ich erlaubte Verräter und gab den Generälen keine Mittel, um sie aufzuhalten. Ihr habt das Ergebnis gesehen. Schlaue Handlungsstränge und raffinierter Verrat, bis der letzte Soldat auf dem Schlachtfeld sich selbst erschossen hat! Ihr könnt wohl nicht bezweifeln, dass jede dieser drei Armeen von einem äußeren Feind besiegt worden wäre, der in sich vereint war.“

Professor Quirrell lehnte sich auf dem Podium nach vorne, seine Stimme war nun mit einer grimmigen Intensität erfüllt. Seine rechte Hand streckte sich aus, die Finger geöffnet und gespreizt. „Trennung ist Schwäche“, sagte der Verteidigungsprofessor. Seine Hand schloss sich zu einer festen Faust. "Einheit ist Stärke. Der Dunkle Lord verstand das gut, was auch immer seine anderen Verrücktheiten waren; und er nutzte dieses Verständnis, um die eine einfache Erfindung zu erschaffen, die ihn zum schrecklichsten Dunkeln Lord der Geschichte machte. Eure Eltern standen einem Dunklen Lord gegenüber. Und fünfzig Todesser, die vollkommen vereint waren, wissend, dass jeder Verstoß ihrer Loyalität mit dem Tod bestraft würde, dass jede Nachlässigkeit oder Inkompetenz mit Schmerz bestraft würde. Niemand konnte dem Griff des Dunklen Lords entkommen, nachdem sie sein Zeichen angenommen hatten.

Und die Todesser stimmten zu, dieses schrecklichen Mal zu nehmen, weil sie wussten, dass sie, sobald sie es hatten, \emph{vereint} sein würden, vor einem geteilten Land. Ein Dunkler Lord und fünfzig Todesser hätten ein ganzes Land durch die Macht des Dunklen Zeichens besiegt."

Professor Quirrells Stimme war düster und hart. „Eure Eltern \emph{hätten} sich in gleicher Weise wehren können. Das haben sie nicht. Es gab einen Mann namens Yermy Wibble, der die Nation aufforderte, einen Entwurf zu erstellen, obwohl er nicht genug Weitsicht hatte, um ein Zeichen für Großbritannien vorzuschlagen. Yermy Wibble wusste, was mit ihm passieren würde; er hoffte, dass sein Tod andere inspirieren würde. Also nahm der Dunkle Lord seine Familie obendrein. Ihre leeren Häute weckten nichts als Angst, und niemand wagte es, wieder zu sprechen. Und eure Eltern hätten mit den Folgen ihrer abscheulichen Feigheit leben müssen, wenn sie nicht von einem einjährigen Jungen gerettet worden wären.“ Professor Quirrells Gesicht war voller Verachtung. „Ein Dramatiker hätte das ein \emph{Deus ex} \emph{machina*} genannt, denn sie haben nichts getan, um ihre Erlösung zu verdienen. Er-dessen-Name-nicht-genannt-werden-darf hat vielleicht nicht verdient hat zu gewinnen, aber zweifelt nicht daran, dass eure Eltern es verdient hatten zu verlieren.“

Die Stimme des Verteidigungsprofessors erklang wie Eisen. „Und merkt euch das Folgende: Eure Eltern haben nichts gelernt! Die Nation ist immer noch zerstückelt und schwach! Wie wenige Jahrzehnte vergingen zwischen Grindelwald und Ihr-wisst-schon-wem? Glaubt ihr, dass \emph{ihr} die nächste Bedrohung in eurem eigenen Leben nicht sehen werdet? Werdet \emph{ihr} dann die Dummheiten eurer Eltern wiederholen, wenn ihr die Ergebnisse gesehen hast, die heute so klar vor euch liegen? Denn ich kann euch sagen, was eure Eltern tun werden, wenn der Tag der Dunkelheit kommt! Ich kann euch sagen, welche Lektion sie gelernt haben! Sie haben gelernt, sich wie Feiglinge zu verstecken und nichts zu tun, während sie darauf warten, dass Harry Potter sie rettet!“

Ein verwunderter Ausdruck trat in die Augen von Schulleiter Dumbledore; und andere Studenten sahen ihren Verteidigungsprofessor mit Verwirrung und Wut und Ehrfurcht an.

Professor Quirrells Augen waren jetzt so kalt wie seine Stimme. "Merkt euch das und merkt es euch gut. Er-der-nicht-genannt-werden-darf wollte über dieses Land herrschen, es für immer in seinem grausamen Griff halten. Aber zumindest wollte er über ein \emph{lebendiges} Land herrschen und nicht über einen Haufen Asche! Es gab verrückte Dunkle Lords, die die Welt nur zu einem riesigen Scheiterhaufen machen wollten! Es gab Kriege, in denen ein ganzes Land gegen ein anderes marschierte! Eure Eltern verloren fast gegen ein halbes Hundert, die dachten, dieses Land lebendig übernehmen zu können! Wie schnell wären sie von einem Feind vernichtet worden, der zahlreicher war als sie, einem Feind, der sich um nichts als um ihre Zerstörung scherte?

Dieses sage ich voraus: Wenn die nächste Bedrohung sich erhebt, wird Lucius Malfoy behaupten, dass ihr ihm folgen oder untergehen müsst, dass eure einzige Hoffnung darin besteht, auf seine Grausamkeit und Stärke zu vertrauen. Und obwohl Lucius Malfoy selbst es glauben wird, wird dies eine Lüge sein. Denn als der Dunkle Lord verschwand, vereinte Lucius Malfoy die Todesser nicht, sie waren in einem Augenblick zerschmettert, sie flohen wie gepeitschte Hunde und verrieten sich gegenseitig! Lucius Malfoy ist nicht stark genug, um ein wahrer Herrscher zu sein, ob dunkel oder nicht."

Draco Malfoys Fäuste waren weiß geballt, es gab Tränen in seinen Augen, Wut und unerträgliche Scham.

„Nein“, sagte Professor Quirrell, „Ich glaube nicht, dass es Lucius Malfoy sein wird, der euch rettet. Und damit ihr nicht denkt, dass ich in meinem eigenen Namen spreche, wird die Zeit bald deutlich machen, dass dies nicht der Fall ist. Ich gebe euch keine Empfehlung, meine Schüler. Aber ich sage, wenn ein ganzes Land einen so starken Führer wie den Dunklen Lord finden würde, aber ehrenhaft und rein, und sein Zeichen nehmen würde; dann könnten sie jeden Dunklen Lord wie ein Insekt zerquetschen, und der Rest unserer gespaltenen magischen Welt könnte sie nicht bedrohen. Und wenn sich in einem Vernichtungskrieg ein noch größerer Feind gegen uns stellen würde, dann könnte nur eine vereinte magische Welt überleben.“

Es gab ein hörbares Einatmen, meist von Muggelgeborenen; die Schüler in grün besetzten Gewändern sahen nur verwirrt aus. Nun war es Harry Potter, dessen Fäuste fest und zitternd zusammengedrückt waren; und Hermine Granger neben ihm war wütend und bestürzt.

Der Schulleiter erhob sich von seinem Sitz, sein Gesicht war nun streng und obwohl er noch kein Wort sagte, war der Befehl klar.

„Ich sage nicht, \emph{welche} Bedrohung kommen wird“, sagte Professor Quirrell. „Aber ihr werdet nicht euer ganzes Leben in Frieden leben, nicht, wenn die vergangene Geschichte der Welt uns irgendeine Leitlinie für die Zukunft ist. Und wenn ihr in Zukunft tut, was ihr die drei Armeen an diesem Tag habt machen sehen, wenn ihr eure kleinlichen Streits nicht beiseiteschieben könnt und das Zeichen eines einzigen Führers nehmen könnt, dann wünscht ihr euch vielleicht, dass der Dunkle Lord gelebt hätte, um über euch zu herrschen, und bedauert den Tag, an dem Harry Potter jemals geboren wurde—“

„\emph{Genug}! “ brüllte Albus Dumbledore.

Es herrschte Stille.

Professor Quirrell drehte langsam den Kopf, um zu Albus Dumbledore zu blicken, wie er in der Wut seiner Zauberei dastand; ihre Augen trafen sich, und ein lautloser Druck presste sich wie ein Gewicht auf alle Schüler, als sie zuhörten und sich nicht zu bewegen wagten.

„Auch Sie haben dieses Land enttäuscht“, sagte Professor Quirrell. „Und Sie kennen die Gefahr genauso gut wie ich.“

„Solche Reden sind nichts für die Ohren von Studenten“, sagte Albus Dumbledore in einer gefährlich lauter werdenden Stimme. „Noch für die Münder von Professoren“!

Professor Quirrell antwortete trocken: „Es gab viele Reden, für die Ohren von Erwachsenen gemacht, als der Dunkle Lord aufstieg. Und die Erwachsenen klatschten und jubelten, und gingen nach Hause, nachdem sie die Unterhaltung des Tages genossen hatten. Aber ich werde Ihnen, Schulleiter, gehorchen, und keine weiteren Reden halten, wenn sie Ihnen nicht gefallen. Meine Lektion ist einfach. Ich werde fortfahren, nichts gegen Verräter zu machen, und wir werden sehen, was Studenten für sich selbst dagegen tun können, wenn sie nicht darauf warten, dass sie von Professoren gerettet werden.“

Und dann drehte sich Professor Quirrell zu seinen Studenten, und sein Mund zuckte zu einem ironischen Grinsen, das den furchtbaren Druck aufzulösen schien wie ein Gott, der die Wolken wegblies um sie zu zerstreuen\textless A{[}verteilen\textbar zerstreuen{]}\textgreater.

„Aber seien Sie bitte freundlich zu allen, die bis jetzt Verrätern waren“, sagte Professor Quirrell. „Sie hatten nur Spaß.“

Es gab Lachen, obwohl anfangs nervös, doch dann schien es sich zu steigern, während Professor Quirrell schief lächelnd dastand und ein Teil der Spannung verflüchtigte sich.

\_\_\_\_\_\_\_\_\_\_\_\_\_\_\_\_\_\_\_\_\_\_\_\_\_\_\_\_\_\_\_\_\_\_\_\_\_\_\_\_

Dracos Verstand wirbelte noch immer durch tausend Fragen und einen Nebel des Entsetzens, als Professor Quirrell sich vorbereitete, die Umschläge zu öffnen, in die die drei ihre Wünsche hineingeschrieben hatten.

Es war Draco noch nie in den Sinn gekommen, dass mondreisende Muggel eine größere Bedrohung darstellten als der langsame Untergang der Zauberei, oder dass sich Vater als zu schwach erwiesen hatte, um sie zu stoppen.

Und noch seltsamer war die offensichtliche Schlussfolgerung: Professor Quirrell glaubte, dass \emph{Harry} es könnte. Der Verteidigungsprofessor behauptete, keine Empfehlung abgegeben zu haben, aber er hatte Harry Potter in seiner Rede wieder und wieder erwähnt; andere würden bereits das Gleiche denken wie Draco.

Es war lächerlich. Der Junge, der einen gepolsterten Stuhl mit Glitter bedeckt und ihn einen Thron genannt hatte—

\emph{Der Junge, der Snape gegenübergestanden} \emph{und gewonnen hatte}, flüsterte eine verräterische Stimme, \emph{dieser Junge konnte zu einem} \emph{Anführer} \emph{heranwachsen, der stark genug war, um zu herrschen, stark genug, um uns alle zu retten}—

Harry war von Muggeln \emph{aufgezogen} worden! Er war selbst praktisch ein Schlammblut, er würde nicht gegen seine Adoptivfamilie kämpfen—

\emph{Er kennt ihre Künste, ihre Geheimnisse und ihre Methoden; er kann die ganze Wissenschaft der Muggel nehmen und sie gegen sie verwenden, neben unserer eigenen Macht als Zauberer.}

Aber was ist, wenn er sich weigert? Was ist, wenn er zu schwach ist?

\emph{Dann} sagte diese innere Stimme, \emph{musst} \emph{du} \emph{es} \emph{sein, nicht wahr, Draco Malfoy}?

Und dann gab es eine erneute Stille aus der Menge, als Professor Quirrell den ersten Umschlag öffnete.

„Mr~Malfoy“, sagte Professor Quirrell, „Ihr Wunsch ist, dass…Slytherin, den Hauspokal gewinnt.“

Es gab eine verwirrte Pause vom zuschauenden Publikum.

„Ja, Professor“, sagte Draco mit klarer Stimme und wusste, dass sie wieder verstärkt wurde. „Wenn Sie das nicht können, dann etwas anderes für Slytherin—“

„Ich werde Hauspunkte nicht ungerechtfertigt vergeben“, sagte Professor Quirrell. Er klopfte sich auf die Wange und sah nachdenklich aus. „Was deinen Wunsch schwierig genug macht, um interessant zu sein. Möchten Sie etwas über das Warum sagen, Mr~Malfoy?“

Draco wandte sich vom Verteidigungsprofessor ab und blickte vom Hintergrund aus Platin und Smaragden aus auf die Menge. Nicht alle von Slytherin hatten die Drachenarmee angefeuert, es gab Anti-Malfoy-Fraktionen, die diese Unzufriedenheit zum Ausdruck gebracht hatten, indem sie den Jungen-der-überlebt-hat, oder sogar Granger unterstützt hatten; und diese Fraktionen würden durch das, was Zabini getan hatte, stark ermutigt werden. Er musste sie daran erinnern, dass Slytherin Malfoy hieß und Malfoy Slytherin—

„Nein“, sagte Draco. „Das sind Slytherins, sie werden es verstehen.“

Das Publikum lachte, besonders in Slytherin, sogar von einigen Studenten, die sich einen Moment zuvor Anti-Malfoy genannt hätten.

Schmeichelei war eine schöne Sache.

Draco drehte sich um, um Professor Quirrell wieder anzusehen, und war überrascht, einen verlegenen Blick auf Grangers Gesicht zu sehen.

„Und für Miss~Granger…“ sagte Professor Quirrell. Das Geräusch eines zerreißenden Umschlags war zu hören. „Dein Wunsch ist…dass Ravenclaw den Hauspokal gewinnt?“

Das Publikum lachte viel, darunter ein Kichern von Draco. Er hatte nicht gedacht, dass Granger dieses Spiel spielte.

„Nun, ähm“, sagte Granger und klang, als stolperte sie plötzlich über eine auswendig gelernte Rede, „Ich meinte damit, das…“. Sie atmete tief durch. „In meiner Armee waren Soldaten aus allen Häusern, und ich will keinen von ihnen beleidigen. Aber auch die Häuser sollten noch für etwas zählen. Es war traurig, als Studenten vom selben Haus sich gegenseitig verfluchten, nur weil sie in verschiedenen Armeen waren. Die Menschen sollten sich auf denjenigen verlassen können, der in ihrem Haus ist. Deshalb haben Godric Gryffindor, Salazar Slytherin, Rowena Ravenclaw und Helga Hufflepuff die vier Häuser von Hogwarts überhaupt erst gegründet. Ich bin der General von Sonnenschein, aber noch davor bin ich Hermine Granger von Ravenclaw, und ich bin stolz darauf, Teil eines achthundert Jahre alten Hauses zu sein.“

„Gut gesagt, Miss~Granger!“ sagte Dumbledores dröhnende Stimme.

Harry Potter runzelte die Stirn, und etwas kitzelte am Rande von Dracos Bewusstsein.

„Ein interessanter Gedanke, Miss~Granger“, sagte Professor Quirrell. „Aber es gibt Zeiten, in denen es für einen Slytherin gut ist, Freunde in Ravenclaw zu haben, oder für einen Gryffindor, Freunde in Hufflepuff zu haben. Sicherlich wäre es das Beste, wenn Sie sich sowohl auf Ihre Freunde in Ihrem Haus als auch auf Ihre Freunde in Ihrer Armee verlassen könnten?“

Grangers Augen huschten kurz zu den beobachtenden Schülern und Lehrern, und sie sagte nichts.

Professor Quirrell nickte wie zu sich selbst, und drehte sich dann wieder zum Podium um und nahm den letzten Umschlag hoch und riss ihn auf. Neben Draco verkrampfte sich Harry Potter sichtlich, als der Verteidigungsprofessor das Pergament herauszog. „Und Mr~Potter wünscht sich—"

Es gab eine lange Pause, als Professor Quirrell das Pergament ansah.

Dann, ohne jegliche Ausdrucksänderung auf Professor Quirrells Gesicht, brach das Pergamentblatt in Flammen aus und verbrannte in einem kurzen, heftigen Feuer, das nur noch schwarzen Staub übrigließ, der aus seiner Hand rieselte.

„Bitte beschränken Sie sich auf das Mögliche, Mr~Potter“, sagte Professor Quirrell und klang sehr trocken.

Es gab eine lange Pause; Harry, der neben Draco stand, sah ziemlich erschüttert aus.

\emph{Worum in Merlins Namen hat er gebeten?}

„Ich hoffe“, sagte Professor Quirrell, „dass Sie einen weiteren Wunsch vorbereitet haben, wenn ich diesen nicht erfüllen könnte.“

Es gab noch eine weitere Pause.

Harry holte tief Luft. „Das habe ich nicht“, sagte er, „aber ich habe schon an einen anderen gedacht.“ Harry Potter drehte sich um das Publikum anzusehen, und seine Stimme wurde fester, während er sprach. „Die Menschen fürchten Verräter wegen des Schadens, den der Verräter direkt verursacht, die Soldaten, die sie erschießen, oder die Geheimnisse, die sie verraten. Aber das ist nur ein Teil der Gefahr. Was die Menschen tun, weil sie \emph{Angst} vor Verrätern haben, kostet sie auch etwas. Ich habe diese Strategie heute gegen Sonnenschein und Drachen angewendet. Ich habe meinen Verrätern nicht gesagt, dass sie so viel direkten Schaden wie möglich verursachen sollen. Ich sagte ihnen, sie sollten so handeln, wie es am meisten Misstrauen und Verwirrung stiften würde, und die Generäle dazu zu bringen die teuersten Dinge zu tun, um zu versuchen, sie daran zu hindern es wieder zu tun. Wenn es nur wenige Verräter gibt und ein ganzes Land gegen sie ist, liegt es nahe, dass das, was einige Verräter tun, weniger schädlich sein könnte als das, was ein ganzes Land tut, um sie aufzuhalten, dass die Heilung schlimmer sein könnte als die Krankheit—“

„Mr~Potter“, sagte der Verteidigungsprofessor, seine Stimme wurde plötzlich schneidend, „die Lektion der Geschichte ist, dass Sie einfach falsch liegen. Die Generation Ihrer Eltern hat zu wenig getan, um sich zu vereinen, nicht zu viel! Das ganze Land wäre fast gefallen, Mr~Potter, obwohl Sie nicht da waren, um es zu sehen. Ich schlage vor, dass Sie Ihre Zimmergenossen in Ravenclaw fragen, wie viele von ihnen Familienangehörige an den Dunklen Lord verloren haben. Oder wenn Sie klüger sind, fragen Sie \emph{nicht}! \emph{Haben} Sie einen Wunsch, Mr~Potter?“

„Wenn es Ihnen nichts ausmacht“, sagte die milde Stimme von Albus Dumbledore, „möchte ich hören, was der Junge-der-überlebt-hat, zu sagen hat. Er hat mehr Erfahrung als jeder von uns darin, Kriege zu stoppen.“

Ein paar Leute lachten, aber nicht viele.

Harry Potters Blick wanderte zu Dumbledore, und er sah einen Moment lang nachdenklich aus. „Ich sage nicht, dass Sie sich irren, Professor Quirrell. Im letzten Krieg handelten die Menschen nicht gemeinsam, und ein ganzes Land fiel fast vor ein paar Dutzend Angreifern, und ja, das war erbärmlich. Und wenn wir das nächste Mal den gleichen Fehler machen, ja, wird das noch erbärmlicher sein. Aber man führt nie zweimal denselben Krieg. Und das Problem ist, dass der Feind \emph{auch} klug sein darf. Wenn man geteilt ist, ist man auf eine Art verwundbar; aber wenn man versuchst, sich zu einigen, dann steht man vor anderen Risiken und anderen Kosten, und der Feind wird versuchen, auch diese zu seinem Vorteil zu nutzen. Man kann nicht aufhören, nur auf einer Ebene des Spiels zu denken.“

„Es spricht viel für Einfachheit, Mr~Potter“, sagte die trockene Stimme des Verteidigungsprofessors. „Ich hoffe, dass Sie heute etwas über die Gefahren von Strategien gelernt haben, die komplizierter sind als Ihr Volk zu vereinen und Ihren Feind anzugreifen. Und wenn das alles nicht irgendwie mit Ihrem Wunsch verbunden ist, werde ich ziemlich verärgert sein.“

„Ja“, sagte Harry Potter, „es war ziemlich schwierig, einen Wunsch zu finden, der die Kosten von Einheit symbolisiert. Aber das Problem des gemeinsamen Handelns betrifft nicht nur Kriege, es ist etwas, das wir unser ganzes Leben lang lösen müssen, jeden Tag. Wenn alle sich mit Hilfe derselben Regeln koordinieren und die Regeln dumm sind, und dann \emph{eine} Person beschließt, die Dinge anders zu machen, dann bricht sie die Regeln. Aber wenn \emph{jeder} beschließt, die Dinge anders zu machen, können sie es. Es ist genau das gleiche Problem, wenn alle zusammen handeln müssen. Aber für die \emph{erste} Person, die sich zu Wort meldet, scheint es, als würde sie gegen die alle anderen antreten. Und wenn man dachte, dass das einzige Wichtige ist, dass Menschen immer vereint sein sollten, dann könnte man das Spiel nie ändern, egal wie dumm die Regeln sind. Darum ist mein eigener Wunsch um zu symbolisieren was passiert, wenn sich Menschen in die falsche Richtung zusammentun, dass wir in Hogwarts Quidditch ohne den Schnatz spielen sollten.“

„\emph{WAS}?“ schrien hundert Stimmen in der Menge, als Dracos Kiefer herunterfiel.

„Der Schnatz ruiniert das gesamte Spiel“, sagte Harry Potter. „Alles, was die anderen Spieler tun, ist am Ende irrelevant. Es wäre unermesslich sinnvoller, einfach eine Uhr zu kaufen. Es ist eines dieser unglaublich dummen Dinge, die man nicht bemerkt, nur weil man damit aufgewachsen ist, die die Leute nur tun, weil alle anderen es tun—“

Aber zu diesem Zeitpunkt konnte Harry Potters Stimme nicht mehr gehört werden, denn der Aufstand hatte begonnen.

\_\_\_\_\_\_\_\_\_\_\_\_\_\_\_\_\_\_\_\_\_\_\_\_\_\_\_\_\_\_\_\_\_\_\_\_\_\_\_\_

Der Aufstand endete etwa fünfzehn Sekunden später, nachdem mit dem Geräusch von hundert Donnerschlägen ein gigantischer Feuerstrahl aus dem höchsten Turm von Hogwarts schoss. Draco hatte nicht gewusst, dass Dumbledore das \emph{konnte}.

Die Schüler setzten sich wieder sehr vorsichtig und leise hin.

Professor Quirrell lachte ohne Pause. „So sei es, Mr~Potter. Ihr Wille geschehe.“ Der Verteidigungsprofessor hielt absichtlich inne. „Natürlich habe ich nur \emph{einen} gerissenen Plan versprochen. Und das ist alles, was ihr drei bekommen werdet.“

Draco hatte die Worte schon vorher halb erwartet, aber jetzt waren sie immer noch ein Schock; Draco tauschte schnelle Blicke mit Granger aus, sie wären die offensichtlichen Verbündeten gewesen, aber ihre Wünsche waren direkt entgegengesetzt—

„Sie meinen“, sagte Harry, „wir müssen uns alle auf einen Wunsch einigen?“

„Oh, das wäre \emph{viel} zu viel verlangt“, sagte Professor Quirrell. „Ihr drei habt keinen gemeinsamen Feind, oder?“

Und für einen kurzen Moment, so schnell, dass Draco dachte, er hätte es sich eingebildet, flackerten die Augen des Verteidigungsprofessors in Richtung Dumbledore.

„Nein“, sagte Professor Quirrell, „Ich meine, dass ich drei Wünsche mit einem einzigen Plan erfüllen werde.“

Es herrschte eine verwirrte Stille.

„Das können Sie nicht tun“, sagte Harry, der neben Draco stand, schlicht. „Nicht einmal \emph{ich} kann das tun. Zwei dieser Wünsche sind miteinander unvereinbar. Es ist \emph{logisch unmöglich}—“ und dann unterbrach sich Harry selbst.

„Sie sind ein paar Jahre zu jung, um mir zu sagen, was ich nicht tun kann, Mr~Potter“, sagte Professor Quirrell mit einem kurzen, trockenen Lächeln.

Dann wandte sich der Verteidigungsprofessor wieder an die zuschauenden Studenten. „Ehrlich gesagt, habe ich kein Vertrauen in eure Fähigkeit, die Lektion des heutigen Tages zu lernen. Geht nach Hause und genießt eure Zeit mit euren Familien, oder was von ihnen übrig ist, solange sie noch leben. Meine eigene Familie ist längst tot durch die Hand des Dunklen Lords. Wir sehen uns alle zu Unterrichtsbeginn wieder.“

In der sprachlosen Stille, die entstand als der Professor sich abwandte und die Bühne verließ, hörte Draco die Stimme des Verteidigungsprofessors sagen, leise und nicht mehr verstärkt: „Aber Sie, Mr~Potter, möchte ich jetzt sprechen.“

*Deus ex machina—Technik in Film und Theater bei dem ein Problem durch göttliche Intervention oder Ähnliches beseitigt wird.

