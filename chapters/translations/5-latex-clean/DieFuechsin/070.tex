

\hypertarget{selbstverwirklichung-teil-5}{% \section{38. Selbstverwirklichung, Teil 5}\label{selbstverwirklichung-teil-5}}

-\/-\/-\/-\/- Kapitel 70: Selbstverwirklichung, Teil 5 -\/-\/-\/-\/-

Selbst nachdem man drei Jahrzehnte lang stellvertretende Schulleiterin und davor Professorin für Verwandlung gewesen war, war es selten Albus Dumbledore komplett unvorbereitet zu erwischen.

“... Susan Bones, Lavender Brown und Daphne Greengrass“, beendete Minerva ihre Aufzählung. „Ich sollte auch anmerken, Albus, dass Miss Grangers Bericht über dein scheinbar nicht sehr hilfreiches Verhalten - ich glaube, ihre Formulierung war \emph{“er sagte, ich solle froh sein, nur eine Gehilfin zu sein"} - eine Menge \emph{Interesse} bei den älteren Mädchen geweckt hat. Mehrere von ihnen kamen zu mir, um zu fragen, ob Miss Grangers Anschuldigungen stimmten, da sie gesagt hatte ich wäre dabei gewesen."

Der alte Zauberer lehnte sich in seinem riesigen Stuhl zurück und starrte sie immer noch an, wobei seine Augen unter der Halbmondbrille etwas abwesend wirkten.

„Das hat mich etwas in die Bredouille gebracht, Albus“, sagte Professor McGonagall. Ihr Gesicht blieb ganz neutral, dafür sorgte sie. „Ich weiß jetzt, dass du nicht wirklich die Absicht hattest, das Mädchen zu entmutigen. Ganz im Gegenteil. Aber du und Severus habt mir oft gesagt, dass ich, um ein Geheimnis zu wahren, kein Zeichen geben darf, das sich von der Reaktion eines wirklich Unwissenden unterscheidet. So blieb mir nichts anderes übrig, als zu bestätigen, dass Miss Grangers Schilderung korrekt war, und den angemessenen Grad von Besorgnis vorzutäuschen, mit einem leichten Unterton von Verärgerung. Hätte ich nämlich \emph{nicht} verstanden, dass du Miss Granger absichtlich manipulierst, wäre ich vielleicht ziemlich empört gewesen."

„Ich... verstehe“, sagte der alte Zauberer langsam. Seine Hände spielten abwesend mit seinem silbernen Bart, kleine schnelle Gesten.

„Zum Glück“, fuhr Professor McGonagall fort, „sind die Professoren Sinistra und Vector bisher die einzigen beiden Fakultätsmitglieder, die Miss Grangers Anstecker tragen."

„Anstecker?“, wiederholte der alte Zauberer.

Minerva zog eine kleine silberne Scheibe mit der Abkürzung BELFER hervor, legte sie auf Albus' Schreibtisch und tippte sie kurz mit dem Finger an.

Und die Stimmen von Hermine Granger, Padma Patil, Parvati Patil, Lavender Brown, Susan Bones, Hannah Abbott, Daphne Greengrass und Tracey Davis riefen unisono: „Wir bleiben nicht Zweitbeste! Es ist Zeit! Gebt Hexen eine Queste.“

„Miss Granger verkauft sie für zwei Sickel und erzählt mir, dass sie bisher fünfzig davon verkauft hat. Ich glaube, dass Nymphadora Tonks aus dem siebten Schuljahr in Hufflepuff sie für sie verzaubert. Um meinen Bericht abzuschließen“, sagte Professor McGonagall zügig, „unsere acht frisch gebackenen Heldinnen haben um Erlaubnis gebeten, vor dem Eingang Ihres Büros eine Protestaktion durchzuführen."

„Ich hoffe“, sagte Albus und runzelte die Stirn, „Du hast ihnen erklärt, dass -"

„Ich habe ihnen erklärt, dass Mittwoch um 19 Uhr in Ordnung wäre“, sagte Minerva. Sie nahm den Knopf vom Schreibtisch des Schulleiters zurück, schenkte Albus ein honigsüßes Lächeln und wandte sich der Tür zu.

„Minerva?“, fragte der alte Zauberer hinter ihr. „\emph{Minerva!} „

Die Eichentür schloss sich fest hinter ihr.

Zwischen den engen Steinmauern, die den Vorraum zum Büro des Schulleiters abgrenzten, war nicht viel Platz, und obwohl viele Leute dem Protest zuschauen wollten, hatten nur wenige einen Platz bekommen. Nur Professor Sinistra und Professor Vector, die die Anstecker trugen, und die Vertrauensschüler Penelope Clearwater und Rose Brown und Jacqueline Preece, die die Anstecker trugen. \emph{Dahinter} Professor McGonagall und Professor Sprout und Professor Flitwick, die die Anstecker nicht trugen und die ganze Angelegenheit begutachteten. Harry Potter und der Schulsprecher von Hogwarts waren da, und die Vertrauensschüler Percy Weasley und Oliver Beatson, die alle die Anstecker trugen, um Solidarität zu zeigen. Und natürlich die acht Gründungsmitglieder von BELFER, die mit ihren Schildern eine Streikpostenkette neben den Wasserspeiern bildeten. Auf Hermines eigenem Schild, das an einem massiven Holzgriff befestigt war, der in ihren Händen immer schwerer zu werden schien, je mehr Sekunden vergingen, stand NIEMANDES GEHILFIN.

Und Professor Quirrell, der mit dem Rücken an der entfernten Steinwand lehnte und mit undurchdringlicher Miene zusah. Der Verteidigungsprofessor hatte einen ihrer Anstecker bekommen, obwohl sie ihm nie einen verkauft hatte; und er trug ihn nicht, sondern warf ihn müßig mit einer Hand hin und her.

Diese ganze Idee war ihr vor vier Tagen viel besser vorgekommen, als die Feuer ihrer Empörung noch frisch und heiß gebrannt hatten und sie mit der Aussicht konfrontiert gewesen war, das alles vier Tage \emph{später} zu tun, statt \emph{jetzt}.

Aber sie musste weitermachen, denn das war es, was Helden taten, sie machten weiter, und auch, weil es unendlich peinlich gewesen wäre, es jetzt noch abzusagen. Hermine fragte sich, wie viel Heldentum nur aus solchen Gründen in die Geschichtsbücher eingegangen war. In den meisten Büchern \emph{stand} nicht: „Und dann weigerten sie sich aufzugeben, egal wie vernünftig es gewesen wäre, denn das wäre zu peinlich gewesen“; aber ein großer Teil der Weltgeschichte ergab auf diese Weise viel mehr Sinn.

Um 19:15 Uhr, hatte Professor McGonagall ihr gesagt, würde Schulleiter Dumbledore herunterkommen und ein paar Minuten mit ihnen sprechen. Professor McGonagall hatte gesagt, dass sie keine Angst haben sollten - der Schulleiter war tief im Inneren ein guter Mensch, und sie hatten die Genehmigung der Schule für den Protest ordnungsgemäß erhalten.

Aber Hermine war sich dennoch sehr bewusst, dass sie, auch wenn sie es mit unterschriebener Erlaubnis tat, immer noch die Autoritätspersonen herausforderte.

Nachdem sie beschlossen hatte, eine Heldin zu sein, hatte Hermine das Offensichtliche getan und war in die Hogwarts-Bibliothek gegangen, um sich Bücher darüber auszuleihen, wie man ein Held wird. Dann hatte sie die Bücher wieder in die Regale zurückgestellt, weil es offensichtlich war, dass keiner der Autoren selbst ein Held gewesen war. Stattdessen hatte sie einfach fünfmal hintereinander (bis sie jedes Wort auswendig kannte) die dreißig Zoll Pergamentrolle von Godric Gryffindor gelesen, die seine gesamte Autobiografie und seine Lebensratschläge waren. (Oder zumindest die englische Übersetzung; Latein konnte sie noch nicht lesen.) Godric Gryffindors Autobiografie war viel \emph{komprimierter} als die Bücher, die Hermine zu lesen gewohnt war, er benutzte \emph{einen Satz}, um Dinge zu sagen, die allein schon dreißig Zoll hätten einnehmen müssen, und dann kam noch ein \emph{weiterer} Satz danach...

Aber nach dem, was sie gelesen hatte, war klar, dass es zwar \emph{nicht} darum ging, Autoritäten \emph{aus Prinzip} zu trotzen, aber man kein Held sein konnte, wenn man zu viel Angst hatte, es zu tun. Und Hermine Granger wusste inzwischen, wie andere sie sahen, und auch, was sie \emph{ihr} nicht zutrauten.

Hermine hievte ihr Demonstrationsschild ein wenig höher und konzentrierte sich darauf, langsam und rhythmisch zu atmen, anstatt zu hyperventilieren, bis sie umkippte.

„\emph{Wirklich?}“, sagte Miss Preece in einem Ton unverhohlener Faszination. „Sie durften nicht \emph{wählen?}"

„In der Tat“, sagte Professor Sinistra. (Das Haar der Astronomieprofessorin war immer noch dunkel und ihr dunkles Gesicht nur leicht faltig; Hermine \emph{hätte} ihr Alter auf etwa siebzig geschätzt, allerdings -) „Ich kann mich noch gut an den Jubel meiner Mutter erinnern, als der Qualification of Women Act beschlossen wurde, obwohl es gar nicht für sie galt.“ (Was bedeutete, dass Professor Sinistra 1918 bei ihrer Muggelfamilie gelebt hatte.) „Und das war nicht das Schlimmste daran. Denn nur ein paar Jahrhunderte früher -"

Dreißig Sekunden später starrten alle Nicht-Muggelstämmigen, sowohl männliche als auch weibliche, Professor Sinistra mit völlig schockierten Mienen an. Hannah hatte ihr Schild fallen gelassen.

„Und \emph{das} war auch nicht das Schlimmste, bei weitem nicht“, beendete Professor Sinistra. „Aber Sie sehen, wohin so etwas potenziell führen kann."

„Merlin bewahre uns“, sagte Penelope Clearwater mit erstickter Stimme. „Sie meinen, \emph{so} würden uns Männer behandeln, wenn wir keine Zauberstäbe hätten, um uns zu verteidigen?"

„\emph{Hey!}“, sagte einer der jungen Vertrauensschüler. „\emph{Das ist} nicht -"

Ein kurzes, sardonisches Lachen ertönte aus der Richtung von Professor Quirrell. Als Hermine den Kopf drehte, um nachzusehen, sah sie, dass der Verteidigungsprofessor immer noch müßig mit dem Anstecker herumspielte. Er machte sich nicht die Mühe, zu ihnen aufzublicken, als er antwortete: „So ist die menschliche Natur, Miss Clearwater. Seien Sie versichert, dass \emph{Sie} nicht freundlicher wären, wenn Hexen Zauberstäbe hätten und Männer nicht."

„Das glaube ich kaum!“, entgegnete Professor Sinistra brüskiert.

Ein kaltes Lachen. „Ich vermute, es kommt selbst in den stolzesten Reinblüterfamilien öfter vor, als man zu glauben wagt. Irgendeine einsame Hexe erspäht einen gutaussehenden Muggel; und denkt, wie leicht es wäre, dem Mann einen Liebestrank unterzuschieben, und von ihm angebetet zu werden. Und da sie weiß, dass er ihr keinen Widerstand leisten kann, ist es nur natürlich, dass sie sich von ihm nimmt, was immer sie will -"

„\emph{Professor Quirrell!}“, sagte Professor McGonagall scharf.

„Es tut mir leid“, sagte Professor Quirrell milde, seine Augen immer noch auf den Anstecker in seiner Hand gerichtet, „tun wir alle immer noch so, als ob sowas nicht passiert? Dann entschuldige ich mich."

Professor Sinistra schnappte: „Und ich nehme an, dass Zauberer nie -"

„Es sind \emph{Kinder} anwesend, Ich bitte Sie!“ Wieder Professor McGonagall.

„Manche schon“, sagte Professor Quirrell gleichmütig, als würde er über das Wetter diskutieren. „Ich persönlich nicht"

Eine Zeit lang herrschte Schweigen. Hermine hob ihr Schild wieder hoch - es war bis auf Schulterhöhe heruntergerutscht, während sie zuhörte. Daran hatte sie nie gedacht, nicht einmal ein bisschen, und jetzt versuchte sie, \emph{nicht} daran zu denken, und ihr Magen fühlte sich etwas mulmig an. Sie schaute in Harry Potters Richtung, ohne recht zu wissen, warum sie es tat; und sie sah, dass Harrys Gesicht vollkommen ruhig war. Ein Schauer lief ihr über den Rücken, bevor sie wegschaute, nicht schnell genug um das kleine Nicken zu verpassen, das Harry ihr schenkte, als ob sie sich über etwas einig wären.

„Um fair zu sein“, sagte Professor Sinistra nach einer Weile, „seit ich meinen Hogwarts-Brief erhalten habe, kann ich mich nicht daran erinnern, auf irgendwelche Vorurteile wegen meiner Hautfarbe oder meinem Geschlecht gestoßen zu sein. Nein, hier geht es nur darum, dass ich Muggelstämmige bin. Ich glaube, Miss Granger hat gesagt, dass sie bisher \emph{nur} bei Helden ein Problem bemerkt hat.„

Hermine brauchte einen Moment, um zu erkennen, dass ihr eine Frage gestellt worden war, und dann brachte sie ein „Ja“ zu Stande, wobei ihre Stimme ein wenig quietschte. Diese ganze Sache hatte sich ein bisschen mehr aufgebläht, als sie sich zu Beginn vorgestellt hatte.

„Was genau haben Sie überprüft, Miss Granger?“, fragte Professor Vector. Sie sah älter aus als Professor Sinistra, ihr Haar begann, schon ein wenig zu ergrauen; Hermine hat mit Professor Vector noch nie ein Wort gewechselt, bis die Arithmantik-Professorin sie nach einem Anstecker gefragt hatte.

„Ähm“, sagte Hermine, ihre Stimme ein wenig hoch, „ich habe in den Geschichtsbüchern nachgeschaut und es hat genauso viele weibliche Zaubereiminister gegeben wie männliche. Dann habe ich mir die Ganz Hohen Tiere der Internationale Vereinigung von Zauberern. ~angesehen und da gab es ein paar mehr Zauberer als Hexen, aber nicht viele. Aber wenn man sich Leute wie berühmte Jäger von dunklen Zauberern ansieht, oder Leute, die Invasionen von dunklen Kreaturen aufgehalten haben, oder Leute, die böse Herrscher gestürzt haben -"

„Und natürlich die Dunklen Zauberer selbst“, sagte Professor Quirrell. \emph{Jetzt} schaut der Verteidigungsprofessor auf. „Sie können das zu Ihrer Liste hinzufügen, Miss Granger. Von allen mutmaßlichen Todessern wissen wir nur von zwei Magierinnen, Bellatrix Black und Alecto Carrow. Und ich wage zu behaupten, dass die meisten Zauberer sich schwer tun würden, eine einzige Dunkle Herrscherin außer Baba Yaga zu nennen."

Hermine starrte ihn nur an.

Das \emph{konnte} doch nicht -

„Professor Quirrell“, sagte Professor Vector, „was genau wollen Sie damit andeuten?„

Der Verteidigungsprofessor hob den Anstecker so an, dass der goldene Schriftzug BELFER zu lesen war, und sagte: „Helden“, dann drehte er den Anstecker um, sodass er seine silberne Rückseite zeigte, und sagte: „Dunkle Zauberer. Es sind ähnliche Karrierewege, die von ähnlichen Menschen beschritten werden, und man kann sich kaum fragen, warum sich junge Hexen von dem einen Weg abwenden, ohne sein Spiegelbild in Betracht zu ziehen."

„Oh, \emph{jetzt} kapier ich's!“, sagte Tracey Davis und meldete sich so plötzlich zu Wort, dass Hermine einen kleinen Schreck bekam. „Sie schließen sich unserem Protest an, weil Sie sich Sorgen machen, dass nicht genug Mädchen zu Dunklen Magierinnen werden!“ Dann kicherte Tracey, was Hermine zu diesem Zeitpunkt nicht einmal geschafft hätte, wenn man ihr eine Million englische Pfund bezahlt hätte.

Es lag ein halbes Lächeln auf Professor Quirrells Gesicht, als er antwortete: „Nicht wirklich, Miss Davis. In Wahrheit interessiert mich diese Art von Dingen nicht im Geringsten. Aber es ist müßig, die Hexen unter den Zaubereiministern und anderen gewöhnlichen Leuten zu zählen, die ein gewöhnliches Leben führen, wo doch Grindelwald und Dumbledore und Er, dessen Name nicht genannt werden darf, alles Männer waren.“ Die Finger des Verteidigungsprofessors drehten den Anstecker untätig hin und her. „Andererseits machen nur die wenigsten Leute etwas Interessantes aus ihrem Leben. Was macht es für Sie aus, ob es überwiegend Hexen oder überwiegend Zauberer sind, solange Sie nicht zu ihnen gehören? Und ich vermute, Sie werden nicht zu ihnen gehören, Miss Davis; denn obwohl Sie ehrgeizig sind, haben Sie keine Ambitionen."

„\emph{Das ist nicht wahr!}“, sagte Tracey entrüstet. „Und was soll das heißen?"

Professor Quirrell richtete sich von dort auf, wo er an der Wand gelehnt hatte. „Sie wurden nach Slytherin sortiert, Miss Davis, und ich denke, dass Sie jede Gelegenheit zum Aufstieg ergreifen werden, die Ihnen in die Hände fällt. Aber sie haben kein großes Ziel, das Sie antreibt, und Sie werden jene Gelegenheiten nicht \emph{erschaffen}. Bestenfalls werden Sie sich bis zum Zaubereiminister oder einer anderen hohen, unbedeutenden Position empor hangeln, ohne jemals die Grenzen Ihrer Existenz zu überschreiten."

Dann wandte sich Professor Quirrells Blick von Tracey ab, er sah \emph{sie} an, die blassblauen Augen starrten sie mit einer furchtbaren Intensität an - „Sagen Sie mir, Miss Granger. Haben \emph{Sie} ein Ziel?"

„Professor -“, quietschte die hohe, strenge Stimme von Professor Flitwick, und dann brach die Stimme ihres Hauslehrers ab, und aus dem Augenwinkel sah Hermine, dass Harry seine Hand auf Professor Flitwicks Schulter gelegt hatte und mit ernstem Gesichtsausdruck den Kopf schüttelte.

Hermine fühlte sich wie ein Reh, das im Scheinwerferlicht stand.

„Was hat Sie dazu getrieben, Ihre Grenzen zu überschreiten, Miss Granger?“, fragte der Verteidigungsprofessor und blickte sie immer noch direkt an. „Warum reicht es nicht mehr aus, gute Noten im Unterricht zu bekommen? Ist es wahre Größe, die Sie suchen? Sind Sie mit irgendeinem Aspekt der Welt unzufrieden, den Sie nach Ihrem Willen umgestalten müssen? Oder ist das alles nur ein Spiel für Sie? Ich wäre sehr enttäuscht, wenn es nur darum ginge, Harry Potter Konkurrenz zu machen."

„Ich -“, sagte Hermine, aber ihre Stimme war so hoch, dass sie nur eine Art Piepsen von sich gab, aber dann fiel ihr nicht ein, was sie noch sagen sollte.

„Sie können sich einen Moment Zeit zum Nachdenken nehmen, wenn Sie möchten“, sagte Professor Quirrell. „Stellen Sie sich vor, es ist ein Aufsatz, sechs Zoll, Abgabe am Donnerstag. Ich habe gehört, dass Sie darin recht eloquent sind."

Alle sahen sie an.

„Ich -“, sagte Hermine. „Ich stimme Ihnen, bei dem was Sie gerade gesagt haben, in keinster Weise zu, nirgendwo."

„Gut gesagt“, kam Professor McGonagalls klare Stimme.

Professor Quirrells Blick wankte nicht. „Das sind keine sechs Zoll, Miss Granger. \emph{Irgendetwas} treibt Sie dazu, sich dem Urteil des Schulleiters zu widersetzen und Anhänger um sich zu scharen. Vielleicht ist es etwas, das Sie lieber nicht laut aussprechen wollen?„

Hermine wusste, dass die richtige Antwort Professor Quirrell nicht beeindrucken würde, aber es war die richtige Antwort, also sagte sie sie trotzdem. „Ich glaube nicht, dass man Ehrgeiz oder ein Ziel braucht, um ein Held zu sein“, sagte Hermine. Ihre Stimme schwankte, aber sie brach nicht. „Ich glaube, man muss nur das Richtige tun. Und sie sind nicht meine Gefolgsleute, wir sind Freunde."

Professor Quirrell lehnte sich wieder an die Wand. Das angedeutete Lächeln war aus seinem Gesicht verschwunden. „Die meisten Leute reden sich ein, dass sie das Richtige tun, Miss Granger. Dabei erheben sie sich nicht über das Gewöhnliche.„

Hermine holte ein paar Mal tief Luft und versuchte, tapfer zu sein. „Es \emph{geht nicht darum}, nicht gewöhnlich zu sein“, sagte sie so tapfer, wie sie konnte. „Aber ich denke, wenn jemand einfach nur versucht, das Richtige zu tun, immer und immer wieder, und sie nicht zu faul sind, sich die ganze Arbeit zu machen, die dazu nötig ist, und sie darüber nachdenken, was sie tun, und mutig genug ist, es zu tun, auch wenn sie Angst haben -“ Hermine hielt für einen Moment inne, ihre Augen huschten zu Tracey und Daphne, „- und sie klug planen, wie sie vorgehen werden - und sie nicht nur einfach das machen, was andere Leute machen - dann denke ich, dass so jemand schon genug Ärger bekommen würde."

Einige der Mädchen und Jungen kicherten, ebenso wie Professor McGonagall, die amüsiert und stolz zugleich aussah.

„Da mögen Sie recht haben“, sagte der Verteidigungsprofessor mit halb geschlossenen Augen. Er warf Hermine den Anstecker zu, und sie fing ihn auf, ohne nachzudenken. „Meine Spende für Ihre Sache, Miss Granger. Ich glaube, dass die zwei Sickel wert sind."

Der Verteidigungsprofessor drehte sich um und ging ohne ein weiteres Wort davon.

„Ich dachte, ich werde ohnmächtig!“, keuchte Hannah, nachdem seine Schritte verklungen waren, und sie hörte, wie einige der anderen Mädchen den angehaltenen Atem ausstießen oder ihre Schilder für einen Moment ablegten.

„Ich habe \emph{auch} ein Ziel!“, sagte Tracey, die fast am Rande der Tränen zu sein schien. „Ich - ich - ich werde bis morgen herausfinden, was es ist, aber ich habe eins, da bin ich mir sicher!"

„Wenn dir wirklich nichts einfällt“, sagte Daphne und gab Tracey einen tröstenden Klaps auf die Schulter, „nimm einfach den Klassiker und versuche, die Welt zu erobern."

„Hey!“, sagte Susan scharf. „Ihr sollt doch jetzt Helden sein! Das heißt, ihr müsst \emph{gut} sein! „

„Nein, das ist schon in Ordnung“, sagte Lavender, „ich bin mir ziemlich sicher, dass General Chaos die Welt übernehmen will und \emph{er} ist schon irgendwie einer der Guten.„

Hinter den Demonstrantinnen wurde weiter diskutiert. „Meine Güte“, sagte Penelope Clearwater. „Ich glaube, das ist der \emph{am} \emph{Offensichtlichsten} böse Verteidigungsprofessor, den wir je hatten.„

Professor McGonagall hustete warnend, und der Schulsprecher sagte: „Du hast Professor Barney nicht erlebt“, was mehrere Anwesende zusammenzucken ließ.

„Professor Quirrell \emph{redet} einfach so“, sagte Harry Potter, obwohl er weniger sicher klang als zuvor. „Ich meine, überlegt doch mal, im Gegensatz zu Professor Snape \emph{handelt} er nicht so -"

„Mr. Potter“, quietschte Professor Flitwick, mit höflicher Stimme und strenger Miene, „warum haben Sie mich gebeten, nichts zu sagen?"

„Professor Quirrell hat Hermine getestet, um zu sehen, ob er ihr \emph{mysteriöser alter Zauberer} sein will“, sagte Harry. „Was auf keinen Fall geklappt hätte, aber sie musste schon für sich selbst antworten."

Hermine blinzelte.

Dann blinzelte Hermine noch einmal, als ihr klar wurde, dass es Professor Quirrell war, der Harry Potters mysteriöser alter Zauberer war, und gar nicht Dumbledore, und das \emph{war wirklich kein gutes Zeichen} -

Ein rumpelndes Geräusch erfüllte den kleinen steinernen Vorraum, und Hermine, die bereits mit den Nerven am Ende war, wirbelte schnell herum und ließ fast ihr Protestschild fallen, während ihre andere Hand zu ihrem Zauberstab wanderte.

Die Wasserspeier traten zur Seite, der Fließende Stein rumpelte wie Fels, während er sich wie ein Wesen aus Fleisch und Blut bewegte. Die riesigen hässlichen Gestalten warteten nur kurz, totengraue Augen starrten in stummer Wachsamkeit geradeaus. Dann klappten die großen Wasserspeier ihre Flügel wieder ein und traten in ihre frühere Position zurück, der Fließende Stein veränderte sein Äußeres nicht im Geringsten, als er von Flexibilität zu Bewegungslosigkeit zurückkehrte, und die kurzzeitige Lücke im Stein von Hogwarts war wieder geschlossen.

Und vor ihnen allen, in leuchtend violetten Umhängen, die wahrscheinlich nur für Muggelstämmige furchtbar aussahen, stand die hoch aufragende Gestalt von Albus Percival Wulfric Brian Dumbledore, dem Schulleiter von Hogwarts, dem Großmeister des Zauberergamot, dem Ganz hohen Tier der Internationalen Konföderation der Zauberer, dem Bezwinger des Dunklen Lords Grindelwald und Beschützer Großbritanniens, der Wiederentdecker der sagenumwobenen Zwölf Gebräuche von Drachenblut, der mächtigste lebende Zauberer; und er sah \emph{sie} an, Hermine Jean Granger, Generalin des kürzlich erweiterten Sonnenschein-Regiments, die im ersten Jahr der Hogwarts-Klassen die besten Noten bekam und sich selbst zur Heldin erklärt hatte.

Sogar ihr \emph{Name} war kürzer als seiner.

Der Schulleiter lächelte sie wohlwollend an, seine faltigen Augen funkelten fröhlich unter ihren Halbkreisen aus Glas, und sagte: „Hallo, Miss Granger.„

Das Seltsame war, dass es nicht annähernd so beängstigend war wie das Gespräch mit Professor Quirrell. „Hallo, Schulleiter Dumbledore“, sagte Hermine mit nur einem leichten Zittern in der Stimme.

„Miss Granger“, sagte Dumbledore, nun mit ernsterem Blick, „ich glaube, wir beide haben ein kleines Missverständnis gehabt. Ich wollte nicht andeuten, dass Sie keine Heldin sein können oder sollten. Ich wollte im Besonderen auch nicht andeuten, dass Hexen im Allgemeinen keine Helden sein sollten. Nur, dass Sie... ein bisschen jung bist, um an solche Dinge zu denken."

Hermine, die sich nicht darin hindern konnte, blickte zu Professor McGonagall hinüber und sah, dass Professor McGonagall ihr ein aufmunterndes Lächeln schenkte - oder sie schenkte den beiden \emph{irgendeine} Art Lächeln -, also blickte Hermine wieder zum Schulleiter und sagte, nun mit etwas mehr Zittern in ihrer Stimme: „Seit Sie vor vierzig Jahren Schulleiter wurden, gab es elf Hogwarts-Absolventen, die zu Helden wurden. Ich meine damit Leute wie Lupe Cazaril und so weiter, und \emph{zehn} davon waren Jungs. Cimorene Linderwall war die einzige Hexe."

„Hm“, sagte der Schulleiter. Es lag ein nachdenklicher Ausdruck auf seinem Gesicht; er schien zumindest darüber nachzudenken. „Miss Granger, ich war noch nie einer, der sich viel von solchen Zahlen leiten ließ. Oft ist es viel zu einfach, zu zählen, als zu verstehen. Aus Hogwarts sind viele gute Menschen hervorgegangen, sowohl Hexen als auch Zauberer; Und die, die als Helden gefeiert werden, sind nur eine bestimmte Art von guten Menschen, und vielleicht nicht die beste. Sie haben zum Beispiel Alice Longbottom oder Lily Potter nicht in Ihre Rechnung einbezogen... Aber lassen wir das beiseite. Sagen Sie mir, Miss Granger, haben Sie gezählt, wie viele Helden Hogwarts in den 40 Jahren vor mir hervorgebracht hat? Denn in dieser Zeit kann ich mich nur an drei erinnern, die heute Helden genannt werden; und unter diesen drei nicht eine Hexe."

„Ich will damit nicht sagen, dass es \emph{nur} an Ihnen liegt!“ sagte Hermine. „Ich denke nur, dass vielleicht \emph{viele} Leute, so wie die Schulleiter vor Ihnen, und vielleicht sogar eure ganze Gesellschaft und alles, Mädchen entmutigen."

Der alte Zauberer seufzte. Seine hellblauen Augen hinter den halbmondförmigen Gläsern sahen nur sie ruhig an, als wären sie die einzigen beiden anwesenden Menschen. „Miss Granger, man kann vielleicht Hexen davon abhalten, Zauberkunstlehrerin zu werden, oder Quidditch-Spielerin oder sogar Aurorin. Bei Helden ist das anders. Wenn jemand dazu bestimmt ist, ein Held zu sein, dann wird er ein Held sein. Sie werden durch Feuer gehen und durch Eis schwimmen. Dementoren werden sie nicht aufhalten, auch nicht der Tod von Freunden, und auch nicht Entmutigung."

„Nun“, sagte Hermine und hielt inne, um nach Worten zu ringen. „Nun, aber... ~aber was ist, wenn das \emph{in Wirklichkeit} gar nicht stimmt? \emph{Mir} scheint, dass wenn man will, dass mehr Hexen Helden werden, sollte man ihnen Heldentum beibringen."

„Viele Jungen und Mädchen sind Helden in ihren Träumen“, sagte Dumbledore leise. Er sah keines der anderen Mädchen an, nur sie. „In der wachen Welt sind es weniger. Viele haben sich behauptet und sich der Dunkelheit gestellt, als diese sie holen kam. Wenige gehen der Dunkelheit entgegen und zwingen sie, sich ihnen zu stellen. Es ist ein hartes Leben, manchmal einsam, oft kurz. Ich habe niemandem gesagt, er solle sich dieser Berufung verweigern, aber ich würde auch nicht wünschen, ihre Zahl zu erhöhen."

Hermine zögerte; da war etwas in dem gezeichneten Gesicht, das sie aufhielt, wie ein Hinweis auf all die Emotionen, die \emph{nicht} gezeigt wurden, über Jahrzehnte angesammelt...

\emph{\emph{Wenn es mehr Helden gäbe, wäre ihr Leben vielleicht nicht so einsam, oder so} \emph{kurz.}}

Sie brachte es nicht über sich, das zu sagen, nicht zu ihm.

„Aber dieser Punkt ist dennoch unwichtig“, sagte der alte Zauberer. Er lächelte, ein bisschen reumütig, wie sie fand. „Miss Granger, man kann Heldentum nicht lehren, wie man Zauberei lehren würde. Sie können nicht einen zwölf Zoll Aufsatz als Hausaufgabe aufgeben, wie man weitermacht, wenn alle Hoffnung verloren scheint. Man kann Schülern nicht beibringen, wann man aufsteht und dem Schulleiter sagt, dass er Unrecht getan hat. Helden werden geboren, nicht gelehrt. Und aus welchem Grund auch immer, werden mehr von ihnen als Jungen geboren als als Mädchen.“ Der Schulleiter zuckte mit den Schultern, als wollte er sagen, dass er nichts dagegen tun könne.

„Ähm“, sagte Hermine. Sie konnte nicht anders, sie blickte hinter sich.

Professor Sinistra sah ein wenig entrüstet aus. Und es \emph{stimmte nicht}, dass alle sie anstarrten, als wäre sie nur ein naives, kleines Mädchen, so wie sie es sich vorzustellen begann, während sie Dumbledore zuhörte.

Hermine wandte sich wieder Dumbledore zu, holte tief Luft und sagte: „Nun, vielleicht werden Leute, die Helden werden sollen, Helden sein, egal was passiert. Aber ich wüsste nicht, wie man das wirklich \emph{wissen} könnte, abgesehen davon, dass man es hinterher einfach sagt. Und als \emph{ich} Ihnen gesagt habe, dass ich ein Held werden will, waren Sie nicht sehr ermutigend."

„Mr. Potter“, sagte der Schulleiter milde. Seine Augen verließen ihre nicht. „Bitte schildern Sie Miss Granger Ihren Eindruck von unserer eigenen ersten Begegnung. Würden Sie sagen, dass ich ermutigend war? Sagen Sie die Wahrheit."

Es gab eine Pause.

„Mr. Potter?“, erklang Professor Vectors Stimme hinter ihr und klang verwirrt.

„Ähm“, sagte Harrys Stimme von weiter hinten und klang äußerst zögernd. „Ähm... nun, in meinem Fall hat der Schulleiter ein Huhn in Brand gesetzt."

"Er hat \emph{was}? „, platzte Hermine heraus, nur dass es mehrere andere Leute gab, die ungefähr zur gleichen Zeit etwas riefen, sodass sie nicht sicher war, dass jemand sie hörte.

Dumbledore starrte sie weiter an und sah vollkommen ernst aus.

„Ich wusste nichts von Fawkes“, sagte Harrys Stimme schnell, „also sagte er mir, dass Fawkes ein Phönix sei, während er auf ein Huhn auf Fawkes' Ständer zeigte, damit ich dachte, \emph{das} sei Fawkes, und dann zündete er das Huhn an - und außerdem gab er mir diesen Felsbrocken und sagte mir, er habe meinem Vater gehört und ich solle ihn überallhin mitnehmen -"

"Aber das ist doch \emph{verrückt}! „, platzte es Susan heraus.

Plötzlich herrschte Stille.

Der Schulleiter drehte langsam seinen Kopf und starrte Susan an.

„Ich -“, sagte Susan. „Ich meine - ich -"

Der Schulleiter beugte sich hinunter, bis er dem jungen Mädchen direkt gegenüberstand.

„Ich habe nicht -“, sagte Susan.

Dumbledore legte einen Finger auf seine Lippen und machte bl-bl-bl.

Der Schulleiter richtete sich wieder auf und sagte: „Nun, meine lieben Heldinnen, es war angenehm, mit euch zu sprechen, aber leider bleibt heute noch viel zu tun. Seid jedoch versichert, dass ich für jeden unergründlich bin, nicht nur für Hexen.„

Die Wasserspeier traten zur Seite und der Fließende Stein rumpelte wie Fels übereinander, als er sich bewegte.

Die riesigen hässlichen Gestalten warteten kurz mit toten grauen Augen, die in stummer Wachsamkeit hinausstarrten, als Albus Percival Wulfric Brian Dumbledore, so wohlwollend lächelnd wie als sein Büro verlassen hatte, zurück auf die erste Stufe der Endlosen Wendeltreppe trat.

Dann klappten die großen Wasserspeier ihre Flügel wieder ein und traten in ihre früheren Positionen zurück, wobei sie noch ein letztes kurzes „Bwa-ha-ha-ha!“ hörten, bevor sich die Lücke schloss.

Es entstand eine lange Pause.

„Er hat \emph{wirklich} ein Huhn angezündet?“, sagte Hannah.

Die acht hatten auch danach noch weiter protestiert, aber um ehrlich zu sein, waren sie nicht mehr so bei der Sache.

Nach einigen vorsichtigen Fragen von Professor Flitwick war festgestellt worden, dass Harry Potter das brennende Huhn nicht gerochen hatte. Was bedeutete, dass es sich wahrscheinlich um einen Kieselstein oder so etwas gehandelt hatte, der in ein Huhn verwandelt und dann mit einem Begrenzungszauber eingeschlossen wurde, um sicherzustellen, dass kein Rauch in die Luft entwich - sowohl Professor Flitwick als auch Professor McGonagall hatten sehr nachdrücklich darauf hingewiesen, dass niemand so etwas ohne ihre Aufsicht versuchen durfte.

Aber trotzdem...

Aber trotzdem... was bei Merlin?

Hermine wusste nicht einmal, was das bedeuten sollte.

Aber \emph{trotzdem}.

Nach vielen Blicken, die zwischen den Mädchen ausgetauscht worden waren, von denen keine die Erste sein wollte, die es sagte, hatte Hermine den Protest für beendet erklärt, und die Lehrer und Schüler waren gegangen.

„Du glaubst doch nicht, dass wir Dumbledore gegenüber unfair waren, oder?“, sagte Susan, als die Heldinnen begleitet vom Geräusch von acht Paar Füßen auf dem Steinpflaster, durch die Korridore von Hogwarts gingen. „Ich meine, wenn er allen gegenüber verrückt ist und nicht nur Hexen gegenüber, dann ist das doch keine Diskriminierung, oder?"

„Ich habe keine Lust mehr, gegen den Schulleiter zu protestieren“, sagte Hannah schwach. Das Hufflepuff-Mädchen schien ein wenig unsicher auf den Beinen zu sein. „Es ist mir egal, ob Professor McGonagall sagt, dass er es uns nicht übelnimmt, es ist einfach zu viel für meine Nerven."

Lavender schnaubte. „Ich schätze, \emph{du} wirst in nächster Zeit keine Armeen von Inferi erschlagen -"

„Hör auf damit!“ sagte Hermine scharf. „Wir müssen erst alle \emph{lernen}, Heldinnen zu sein, oder? Es ist in Ordnung, wenn es jemand nicht sofort weiß."

„Der Schulleiter glaubt nicht, dass man es lernen \emph{kann}“, sagte Padma. Das Gesicht des Ravenclaw-Mädchens war nachdenklich, ihre Schritte gemessen, als sie durch den Korridor schritt. „Der Schulleiter hält das nicht einmal für eine gute Idee.„

Daphne schritt mit geradem Rücken und kerzengeradem Kopf und sah damit in ihren Hogwarts-Umhängen mehr wie eine noble, junge Dame aus, als Hermine es in ihrem besten Festtagskleid hätte tun können. „Der Schulleiter“, sagte Daphne mit präziser Stimme, wobei ihre Schuhe harte, kurze Klack-Geräusche auf dem Stein machten, „denkt, dass wir alle ein Haufen dummer Mädchen sind, die Spielchen spielen, und dass Hermine vielleicht eines Tages eine gute Gehilfin sein wird, aber der Rest von uns hoffnungslos ist."

"Hat er \emph{recht?} „, sagte Parvati. Das Gesicht des Gryffindor-Mädchens war sehr ernst und ließ sie viel mehr wie ihren Zwilling aussehen, als sie es normalerweise tat. „Ich meine, man muss die Frage doch einmal in Raum werfen -"

„\emph{Nein!}“, stieß Tracey hervor. Das Slytherin-Mädchen stakste durch den Flur und sah aus, als wäre sie bereit, jemanden zu töten, wie ein weiblicher Miniatur-Snape. Von allen Mädchen war Tracey diejenige, die Hermine am wenigsten kannte. Hermine hatte schon einmal mit Lavender gesprochen, aber mit Tracey hatte sie noch nie etwas zu tun gehabt, außer auf der Gegenseite bei einem Kampf, bis die Slytherin vom Sofa aufgesprungen war und sich freiwillig für Belfer gemeldet hatte. „Wir werden es ihm zeigen! Wir werden es ihnen \emph{allen} zeigen! „

„Okay“, sagte Susan, „das war \emph{definitiv} böse -"

„Nein“, sagte Lavender, „das ist nur ein Motto der Chaos Legion. Nur hat sie das wahnsinnige Lachen vergessen."

„Das stimmt“, sagte Tracey, ihre Stimme tief und grimmig. „Denn diesmal lache ich nicht.“ Das Mädchen pirschte weiter durch den Korridor, als würde sie von dramatischer Musik begleitet, die nur sie hören konnte.

(Hermine begann sich Sorgen zu machen, was \emph{genau} die beeinflussbaren Jugendlichen der Chaos Legion von Harry Potter lernten.)

„Aber - ich meine -“ sagte Parvati. Sie hatte immer noch diesen nachdenklichen Gesichtsausdruck. „Ich meine, man kann doch verstehen, warum der Schulleiter uns für dumme Mädchen hält, oder? Was hat der Protest vor dem Büro des Schulleiters damit zu tun, Heldinnen zu werden?"

„Hm“, sagte Lavender, die nun selbst nachdenklich aussah. „Das ist wahr. Wir sollten etwas Heldenhaftes tun. Ich meine Heldinnenhaftes."

„Ähm -“, sagte Hannah, was sehr gut Hermines eigene Gefühle zu diesem Thema ausdrückte.

„Nun“, sagte Parvati, „hat denn schon jeder Dumbledores verbotenen Korridor im dritten Stock probiert? Ich meine, jeder aus Gryffindor war schonmal da-"

"\emph{Moment} mal! „, sagte Hermine verzweifelt. „Ich will nicht, dass wir etwas \emph{Gefährliches} machen! „

Es gab eine Pause, während alle Hermine ansahen, die viel zu spät begriff, warum Dumbledore nicht wollte, dass jemand \emph{anderes} ein Held wurde.

„Ich glaube nicht, dass man eine Heldin werden kann, wenn man nie etwas Gefährliches tut“, bemerkte Lavender vernünftig.

„Außerdem“, sagte Padma mit einem nachdenklichen Gesichtsausdruck. „Jeder weiß doch, dass in Hogwarts nie etwas \emph{wirklich} Schlimmes passiert, oder? Den Schülern, meine ich, nicht den Verteidigungsprofessoren. Wir haben all diese uralten Schutzzauber und so weiter."

„Ähm -“ Sagte Hannah wieder.

„Ja“, sagte Parvati, „das Schlimmste, was passieren kann, ist, dass wir ein paar Dutzend Hauspunkte verlieren oder so, und wir sind zwei aus jedem Haus, also wird \emph{das} alles ausgeglichen."

„Das ist ja \emph{genial}, Hermine!“, sagte Daphne in einem Ton großer Bewunderung. „So wie du es eingerichtet hast, können wir \emph{mit allem} durchkommen! Und ich habe deinen schlauen Plan bis jetzt nicht einmal bemerkt!"

„\emph{ÄHM} -“, sagten Hermine, Hannah und Susan.

„Richtig!“, sagte Parvati. „Jetzt ist es also an der Zeit, dass wir zu echten Heldinnen werden. Wir werden gegen die Dunkelheit antreten -"

„Und bringen sie dazu, sich \emph{uns} zu stellen -“ sagte Lavender.

„Und lehren sie, uns zu fürchten“, sagte Tracey Davis grimmig.

