

\hypertarget{vorabinformationen}{% \section{17. Vorabinformationen}\label{vorabinformationen}}

-\/-\/-\/-\/- Kapitel 49: Vorabinformationen -\/-\/-\/-\/-

Ein Junge wartet auf einer kleinen Lichtung am Rande des nicht verbotenen Waldes, neben einem Feldweg, der in einer Richtung zurück zu den Toren von Hogwarts und in der anderen Richtung in die Ferne führt. In der Nähe befindet sich eine Kutsche, und der Junge steht weit entfernt davon und schaut sie an, wobei sich seine Augen selten von der Kutsche lösen.

In der Ferne nähert sich eine Gestalt auf dem Feldweg: Ein Mann in Professorenrobe, der langsam mit tief heruntergezogenen Schultern heran stapft, wobei seine eleganten Schuhe beim Gehen kleine Staubwolken aufwirbeln.

Eine halbe Minute später wirft der Junge einen weiteren schnellen Blick herüber, bevor er seine Überwachung fortsetzt; und dieser Blick zeigt, dass die Schultern des Mannes gerade geworden sind, sein Gesicht gefasst und dass seine Schuhe nun leicht über den Schmutz laufen und keine Spur von Staub in der Luft hinterlassen.

„Hallo, Professor Quirrell“, sagte Harry, ohne seine Augen wieder aus der Richtung ihrer Kutsche zu bewegen.

„Grüße“, sagte die ruhige Stimme von Professor Quirrell. „Sie scheinen Abstand zu halten, Mr. Potter. Ich nehme nicht an, dass Sie an unserem Transport etwas merkwürdig finden?"

„Merkwürdig?“ echote Harry. „Aber nein, ich kann sagen, dass ich alles ordentlich finde. Es scheint von allem eine gerade Zahl zu geben. Vier Sitze, vier Räder, zwei riesige skelettartige geflügelte Pferde..."

Ein hautumhüllter Schädel drehte sich zu ihm um und bleckte seine Zähne, fest und weiß, in diesem schwarzen, höhlenartigen Maul, als ob er andeuten wollte, dass er ihm genauso zugetan war wie er ihm. Das andere schwarze lederne Pferdeskelett warf den Kopf, als ob es wieherte, aber es gab keinen Laut von sich.

„Sie werden Thestrale genannt, und sie haben schon immer die Kutsche gezogen“, sagte Professor Quirrell, der ganz ungestört auf die Vorderbank der Kutsche stieg und sich so weit wie möglich nach rechts setzte. „Sie sind nur für diejenigen sichtbar, die den Tod gesehen und begriffen haben, eine nützliche Verteidigung gegen die meisten tierischen Raubtiere. Hm. Ich nehme an, dass bei Ihrer ersten Begegnung mit dem Dementor Ihre schlimmste Erinnerung die Nacht Ihrer Begegnung mit Ihm, der nicht genannt werden darf, war?

Harry nickte grimmig. Es war die richtige Vermutung, wenn auch aus den falschen Gründen. \emph{Diejenigen, die den Tod gesehen haben...}

"Haben Sie sich dabei an etwas Interessantes erinnert?"

„Ja“, sagte Harry, „das habe ich“, nur das und nichts weiter, denn er war noch nicht bereit, Anschuldigungen zu erheben.

Der Verteidigungsprofessor lächelte eines seiner trockenen Lächeln und winkte ihn ungeduldig heran.

Harry schloss auf, kletterte in den Wagen und zuckte zusammen. Das Gefühl des Untergangs war nach dem Tag des Dementors deutlich stärker geworden, auch wenn es zuvor langsam nachgelassen hatte. Die größte Entfernung, die ihm die Kutsche von Professor Quirrell erlaubte, schien nicht mehr annähernd weit genug zu sein.

Dann trabten die Skelettpferde vorwärts, und die Kutsche setzte sich in Bewegung und brachte sie an die Außengrenzen von Hogwarts. Dabei fiel Professor Quirrell wieder in den Zombie-Modus zurück, und das Gefühl des Untergangs zog sich zurück, auch wenn es immer noch am Rande von Harrys Wahrnehmung schwebte, nicht zu ignorieren...

Der Wald rollte vorbei, als die Kutsche weiter fuhr, die Bäume bewegten sich mit einer Geschwindigkeit, die im Vergleich zu Besenstielen oder sogar Autos geradezu gletscherartig erschien. Harry fand, das es etwas seltsam Entspannendes hatte, so langsam zu reisen. Es hatte den Verteidigungsprofessor sicherlich entspannt, der mit einem kleinen Strom aus Sabber, der aus seinem schlaffen Mund kam und auf seine Gewänder lief, zusammengesunken war.

Harry hatte sich noch immer nicht entschieden, was er zum Mittagessen essen durfte.

Seine Recherchen in der Bibliothek hatten keine Anzeichen dafür ergeben, dass Zauberer mit nicht-magischen Pflanzen sprechen. Auch keine anderen nichtmagischen Tiere außer Schlangen, obwohl \emph{Spruch und Sprache} von Paul Breedlove die wahrscheinlich mystische Geschichte einer Zauberin namens „Dame der fliegenden Eichhörnchen„ erzählt hatte.

Was Harry tun \emph{wollte}, war, Professor Quirrell zu fragen. Das Problem war, dass Professor Quirrell \emph{zu klug} war. Nach dem zu urteilen, was Draco gesagt hatte, war die Sache mit dem Erbe Slytherins eine echte Bombe, und Harry war sich nicht sicher, ob er wollte, dass jemand anderes davon erfuhr. Und in dem Moment, in dem Harry über Parselmünder fragte, würde Professor Quirrell ihn mit diesen blassblauen Augen fixieren und sagen: „Ich verstehe, Herr Potter, Sie haben also Herrn Malfoy den Patronuszauber beigebracht und aus Versehen mit seiner Schlange gesprochen“.

Es wäre \emph{egal}, dass es nicht genug Beweise sein sollten, um die richtige Erklärung als Hypothese ausfindig zu machen, geschweige denn die Last der früheren Unwahrscheinlichkeit zu überwinden. Irgendwie würde der Verteidigungsprofessor sie \emph{trotzdem} herleiten. Es gab Zeiten, in denen Harry vermutete, dass Professor Quirrell viel mehr Hintergrundinformationen hatte, als er erzählte, seine A-priori-Wahrscheinlichkeiten waren einfach zu gut. Manchmal waren seine erstaunlichen Schlussfolgerungen richtig, selbst wenn seine \emph{Begründung} falsch waren. Das Problem war, dass Harry nicht erkennen konnte, wie Professor Quirrell einen zusätzlichen Hinweis über die Hälfte der Dinge, die er vermutete, einschmuggeln konnte. Nur \emph{einmal} hätte Harry gerne eine Art unglaubliche Schlussfolgerung aus dem gezogen, was Professor Quirrell sagte und damit \emph{ihn} völlig unvorbereitet getroffen.

„Ich nehme eine Schüssel grüne Linsensuppe mit Sojasauce“, sagte Professor Quirrell zur Kellnerin. „Und für Mr. Potter einen Teller mit Tenorman Familien-Chili."

Harry zögerte in plötzlicher Bestürzung. Er hatte sich entschlossen, vorerst bei vegetarischen Gerichten zu bleiben, aber er hatte bei seinen Überlegungen vergessen, dass Professor Quirrell die \emph{eigentliche Bestellung} machte - und es wäre peinlich, wenn er jetzt protestieren würde -

Die Kellnerin verbeugte sich vor ihnen und drehte sich um, um zu gehen -

"Ähm, entschuldigen Sie, ist da Fleisch von Schlangen oder fliegenden Eichhörnchen drin?"

Die Kellnerin blinzelte nicht einmal mit der Wimper, drehte sich nur zu Harry um, schüttelte den Kopf, verbeugte sich wieder höflich vor ihm und ging wieder auf die Tür zu.

(Die anderen Teile von Harry kicherten über ihn. Gryffindor machte sardonische Bemerkungen darüber, dass ein wenig soziales Unbehagen ausreiche, um ihn zum \emph{Kannibalismus!} (geschrien von Hufflepuff) zu bewegen, und Slytherin bemerkte, wie schön es sei, dass Harrys Ethik flexibel sei, wenn es um wichtige Ziele wie die Aufrechterhaltung seiner Beziehung zu Professor Quirrell gehe).

Nachdem die Kellnerin die Tür hinter sich geschlossen hatte, winkte Professor Quirrell mit der Hand, um den Riegel vorzuschieben, sprach die üblichen vier Zauber, um Privatsphäre zu gewährleisten, und sagte dann: „Eine interessante Frage, Mr. Potter. Ich frage mich, warum Sie sie gestellt haben.„

Harry hielt sein Gesicht ausdruckslos. „Ich habe vorhin ein paar Fakten über den Patronus-Zauber nachgeschlagen“, sagte er. „Laut \emph{Patronuszauber: Welche Zauberer ihn konnten und wer nicht}, stellte sich heraus, dass Godric es nicht konnte und Salazar schon. Ich war überrascht, also schlug ich den Bezug in \emph{Vier Leben der Macht} nach. Und dann entdeckte ich, dass Salazar Slytherin angeblich mit Schlangen sprechen konnte“. (Die zeitliche Abfolge war nicht dasselbe wie die Ursache, es war nicht Harrys Schuld, wenn Professor Quirrell das übersah). „Weitere Recherchen ergaben eine alte Geschichte über eine Art Muttergöttin, die mit fliegenden Eichhörnchen sprechen konnte. Ich war ein wenig besorgt über die Aussicht, etwas zu essen, das sprechen konnte.„

Und Harry nahm gelassen einen Schluck von seinem Wasser -

- just als Professor Quirrell sagte: „Mr. Potter, würde ich zu Recht annehmen, dass Sie auch ein Parselmund sind?“

Als Harry mit dem Husten fertig war, stellte er sein Glas Wasser wieder auf den Tisch, richtete seinen Blick auf Professor Quirrells Kinn, anstatt ihm in die Augen zu schauen, und sagte: „Sie sind also in der Lage, Legilimentik durch meine Okklumentik-Barrieren hindurch durchzuführen“.

Professor Quirrell grinste breit. „Ich werde das als Kompliment auffassen, Mr. Potter, aber nein."

„Das kaufe ich Ihnen nicht mehr ab“, sagte Harry. „Auf \emph{keinen} Fall sind Sie aufgrund der Beweise zu diesem Schluss gekommen."

„Natürlich nicht“, sagte Professor Quirrell gleichförmig. „Ich hatte geplant, Ihnen diese Frage heute auf jeden Fall zu stellen, und habe einfach einen günstigen Zeitpunkt gewählt. Ich habe eigentlich seit Dezember vermutet, dass -"

"\emph{Dezember?} „ sagte Harry. „Ich habe es \emph{gestern} herausgefunden! „

"Ah, Sie wussten also nicht, dass die Botschaft des Sprechenden Hutes an Sie in Parsel verfasst war?"

Der Verteidigungsprofessor hatte es auch beim zweiten Mal genau richtig getimt, gerade als Harry einen Schluck Wasser nahm, um sich vom ersten Hustenanfall zu beruhigen.

Harry \emph{hatte es nicht} bemerkt, nicht bis jetzt. Natürlich war es in dem Moment offensichtlich, als Professor Quirrell es sagte. Richtig, Professor McGonagall hatte ihm sogar \emph{gesagt}, er solle nicht mit Schlangen sprechen, wenn ihn jemand sehen könne, aber er hatte gedacht, sie meinte nicht gesehen zu werden, wie er mit Statuen oder architektonischen Elementen in Hogwarts sprach, die wie Schlangen aussahen. Doppelte Illusion von Transparenz, er dachte, er verstünde sie, sie dachte, er verstünde sie - aber \emph{wie zum Teufel} -

„Also“, sagte Harry, „haben Sie während meines ersten Verteidigungskurses Legilimentik angewandt, um herauszufinden, was mit dem Sprechenden Hut passiert ist -"

„Dann hätte ich es nicht im Dezember erfahren.“ Professor Quirell lehnte sich zurück und lächelte. „Das ist kein Rätsel, das Sie alleine lösen können, Mr. Potter, also werde ich die Antwort verraten. Während der Winterferien wurde ich darauf aufmerksam gemacht, dass der Schulleiter einen Antrag auf eine geschlossene gerichtliche Kammer gestellt hatte, um den Fall eines Mr. Rubeus Hagrid, den Sie als Hüter der Schlüssel und Ländereien von Hogwarts kennen und der des Mordes an Abigail Myrtle 1943 beschuldigt wurde, zu überprüfen.“

„Oh, natürlich“, sagte Harry, „das macht es geradezu \emph{offensichtlich}, dass ich ein Parselmund bin. Professor, \emph{was} bei allen süßen, schlängelnden Schlangen -"

„Der \emph{andere} Verdächtige für diesen Mord war Slytherins Monster, der legendäre Bewohner von Slytherins Kammer des Schreckens. Deshalb wurde ich aus bestimmten Quellen darauf aufmerksam gemacht, und darum hat es meine Aufmerksamkeit so sehr erregt, dass ich eine Menge Bestechungsgeld ausgab, um die Einzelheiten des Falls zu erfahren. Nun, Mr. Potter, Mr. Hagrid ist tatsächlich unschuldig. Lächerlich offensichtlich unschuldig. Er ist der offenkundigste unschuldige Zuschauer, der durch das magische britische Rechtssystem verurteilt wurde, seit Grindelwalds Verwirrungszauber auf Neville Chamberlain Amanda Knox angehängt wurde.

Schulleiter Dippet veranlasste einen Marionettenschüler, Mr. Hagrid anzuklagen, weil Dippet einen Sündenbock brauchte, um die Schuld für den Tod von Miss Myrtle zu übernehmen, und unser wunderbares Justizsystem stimmte zu, dass dies plausibel genug war, um Mr. Hagrids Ausschluss und das Zerbrechen seines Zauberstabs zu rechtfertigen. Unser derzeitiger Schulleiter muss lediglich einige neue Beweise vorlegen, die aussagekräftig genug sind, um den Fall wieder aufzunehmen; und da Dumbledore anstelle von Dippet Druck ausübt, ist das Ergebnis eine ausgemachte Sache. Lucius Malfoy hat keinen besonderen Grund, Mr. Hagrids Rehabilitation zu befürchten; daher wird sich Lucius Malfoy nur in dem Maße widersetzen, wie er dies kostenfrei tun kann, um Dumbledore Kosten aufzuerlegen, und Dumbledore ist eindeutig bereit, den Fall trotzdem zu verfolgen“.

Professor Quirrell nahm einen Schluck von seinem Wasser. „Aber ich schweife ab. Die neuen Beweise, die der Schulleiter zu liefern verspricht, bestehen darin, einen bisher unentdeckten Zauber auf dem Sprechenden Hut zu zeigen, der, wie der Schulleiter behauptet persönlich herausgefunden zu haben, ~nur den Slytherins, die auch Parselmünder sind, antwortet. Der Schulleiter argumentiert ferner, dass dies die Interpretation begünstigt, dass die Kammer des Schreckens tatsächlich 1943 eröffnet wurde, also ungefähr im richtigen Zeitrahmen, als Er, dessen Name nicht genannt werden darf, ein bekannter Parselmund, Hogwarts besucht hat. Es ist eine ziemlich fragwürdige Logik, aber ein richterlicher Ausschuss könnte entscheiden, dass es ausreicht, um Mr. Hagrids Schuld in Zweifel zu ziehen, wenn es ihnen gelingt, ein ehrliches Gesicht zu wahren, wenn sie das verkünden. Und nun kommen wir zur Schlüsselfrage: \emph{Wie} hat der Schulleiter diesen versteckten Zauber auf dem Sprechenden Hut entdeckt?“

Professor Quirrell lächelte nun dünn. „Nun, nehmen wir an, es gäbe einen Parselmund in der diesjährigen Schülergeneration, einen potentiellen Erben Slytherins. Sie müssen zugeben, Mr. Potter, dass Sie immer dann als Möglichkeit hervorstechen, wenn außergewöhnliche Menschen in Betracht gezogen werden. Und wenn ich mich dann noch frage, in wessen geistige Privatsphäre welchen neuen Slytherins der Schulleiter am ehesten eingedrungen wäre, speziell auf der Jagd nach den Erinnerungen an seine Einsortierung, tja, dann fallen Sie noch mehr auf“. Das Lächeln verschwand. „Sie sehen also, Mr. Potter, nicht \emph{ich} war es, der in Ihren Verstand eingedrungen ist, obwohl ich Sie nicht um eine Entschuldigung bitten werde. Es ist nicht Ihre Schuld, dass Sie Dumbledores Beteuerungen, Ihre geistige Privatsphäre zu respektieren, geglaubt haben."

„Ich bitte aufrichtig um Entschuldigung“, sagte Harry und hielt sein Gesicht ausdruckslos. Die strenge Kontrolle war ein Geständnis an sich, ebenso wie der Schweiß auf seiner Stirn; aber er glaubte nicht, dass der Verteidigungsprofessor daraus irgendwelche Beweise ziehen würde. Professor Quirrell würde einfach denken, dass Harry nervös war, weil er als Erbe Slytherins entdeckt worden war. Anstatt nervös zu sein, dass Professor Quirrell erkennen könnte, dass Harry absichtlich Slytherins Geheimnis verraten hatte... was an sich nicht mehr als ein so kluger Schachzug erschien.

"Also, Mr. Potter. Gibt es Fortschritte bei der Suche nach der Kammer des Schreckens?"

\emph{Nein}, dachte Harry. Aber um plausibel zu bleiben, brauchte man eine allgemeine Politik des Ausweichens, auch wenn man nichts zu verbergen hatten... „Bei allem Respekt, Professor Quirrell, wenn ich derartige Fortschritte gemacht hätte, ist mir nicht \emph{wirklich} klar, warum ich Ihnen davon erzählen sollte."

Professor Quirrell nippte wieder aus seinem eigenen Wasserglas. „Nun denn, Mr. Potter, ich werde Ihnen frei heraus sagen, was ich weiß oder vermute. Erstens glaube ich, dass die Kammer des Schreckens real ist, ebenso wie Slytherins Monster. Miss Myrtles Tod wurde erst Stunden nach ihrem Ableben entdeckt, obwohl die Schutzzauber den Schulleiter sofort hätten alarmieren müssen. Daher wurde ihr Mord entweder von Direktor Dippet verübt, was unwahrscheinlich ist, oder von einem Wesen, das Salazar Slytherin auf einer höheren Ebene als der des Direktors selbst in seine Schutzzauber eingeschleust hat. Zweitens vermute ich, dass der Zweck von Slytherins Monster im Gegensatz zu der populären Legende \emph{nicht} darin bestand, Hogwarts von Muggelgeborenen zu befreien. Wenn Slytherins Monster nicht mächtig genug war, den Schulleiter von Hogwarts und alle Lehrer zu besiegen, konnte es nicht mit Gewalt triumphieren. Mehrere Morde im Verborgenen würden zur Schließung der Schule führen, wie es 1943 fast geschah, oder zur Einrichtung neuer Schutzzauber. Warum also Slytherins Monster, Herr Potter? Welchen wahren Zweck erfüllt es?"

„Äh...“ Harry ließ den Blick auf sein Wasserglas fallen und versuchte nachzudenken. „Jeden zu töten, der die Kammer betrat und nicht dorthin gehörte, -"

"Ein Monster, das mächtig genug ist, um ein Team von Zauberern zu besiegen, das an den besten Schutzzaubern, die Salazar auf seine Kammer legen konnte, vorbeigekommen war? Unwahrscheinlich.„

Harry fühlte sich jetzt ein wenig unter Druck gesetzt. „Nun, sie heißt auch die Kammer der Geheimnisse, vielleicht hat das Monster also ein Geheimnis, oder \emph{ist} es ein Geheimnis?“ Und was für Geheimnisse waren überhaupt in der Kammer des Schreckens? Harry hatte nicht viel über dieses Thema recherchiert, zum Teil, weil er den Eindruck hatte, dass niemand etwas wusste -

Professor Quirrell lächelte. „Warum nicht einfach das Geheimnis aufschreiben?"

„Ahhh...“ sagte Harry. „Denn wenn das Monster Parsel spricht, würde das sicherstellen, dass nur ein wahrer Nachfahre Slytherins das Geheimnis hören könnte?"

"Es ist leicht genug, die Schutzzauber der Kammer auf einen Satz in Parselsprache zu ansprechen zu lassen. Warum sich die Mühe machen, Slytherins Monster zu erschaffen? Es kann nicht einfach gewesen sein, eine Kreatur mit einer Lebensdauer von Jahrhunderten zu erschaffen. Kommen Sie, Mr. Potter, es sollte offensichtlich sein; was sind die Geheimnisse, die von einem lebenden Verstand zum anderen weitererzählt, aber nie niedergeschrieben werden können?

Da sah Harry es, mit einem Adrenalinschub, der sein Herz zum Rasen brachte, und sein Atem wurde schneller. „\emph{Oh.}"

Salazar Slytherin war in der Tat sehr gerissen gewesen. Gerissen genug, um einen Weg zu finden, das Interdikt von Merlins zu umgehen.

Mächtige Zaubereien konnten nicht durch Bücher oder Geister übertragen werden, aber wenn man eine langlebige, ausreichend denkende Kreatur mit einem ausreichend guten Gedächtnis erschaffen könnte -

„Es erscheint mir sehr wahrscheinlich“, so Professor Quirrell, „dass Er, dessen Name nicht genannt werden darf seinen Aufstieg zur Macht mit Geheimnissen begann, die er aus Slytherins Monster gewonnen hatte. Dass Salazars verlorenes Wissen die Quelle der außerordentlich mächtigen Zauberei von Du-weißt-schon-wem ist. Daher mein Interesse an der Kammer des Schreckens und dem Fall von Mr. Hagrid."

„Ich \emph{verstehe}“, sagte Harry. Und wenn \emph{er}, Harry, Salazars Kammer des Schreckens finden könnte... dann würde all das verlorene Wissen, das Lord Voldemort erlangt hatte, auch \emph{ihm} gehören.

Ja. \emph{Genau so} sollte die Geschichte ablaufen.

Fügte man Harrys überlegene Intelligenz und einige originelle magische Forschungen und einige Muggel-Raketenwerfer hinzu, dann würde der daraus resultierende Kampf völlig einseitig sein, was genau das war, was Harry wollte.

Harry grinste jetzt, ein sehr böses Grinsen. \emph{Neue Priorität: Finde in Hogwarts alles, was im Entferntesten wie eine Schlange aussieht, und versuche, damit zu sprechen. Fange mit allem an, was du bereits versucht hast, nur dieses Mal solltest du unbedingt} \emph{Parsel} \emph{statt Englisch verwenden -- lass dich von Draco in die Schlafsäle der Slytherins einladen -}

„Seien Sie nicht zu aufgeregt, Mr. Potter“, sagte Professor Quirrell. Sein eigenes Gesicht war nun ausdruckslos geworden. „Sie müssen \emph{weiter} denken. Was waren die Abschiedsworte des Dunklen Lords an Slytherins Monster?"

„\emph{Was?}“, fragte Harry. „Wie kann das einer von uns beiden wissen?"

"Visualisieren Sie die Szene, Mr. Potter. Lassen Sie Ihre Fantasie die Details füllen. Slytherins Monster - wahrscheinlich eine große Schlange, so dass nur ein Parselmund zu ihm sprechen kann - hat sein ganzes Wissen, das es besitzt, an Ihn weitergegeben, der nicht genannt werden darf. Sie übermittelt ihm Salazars letzten Segen und warnt ihn, dass die Kammer des Schreckens nun geschlossen bleiben muss, bis der nächste Nachfahre Salazars sich als listig genug erweist, sie zu öffnen. Und derjenige, der der Dunkle Lord werden wird, nickt und sagt zu ihm -"

„Avada Kedavra“, sagte Harry, dem plötzlich schlecht wurde.

„Regel Nr. zwölf“, sagte Professor Quirrell leise. „Lass die Quelle deiner Kraft nie dort liegen, wo ein anderer sie finden kann."

Harrys Blick fiel auf die Tischdecke, die sich mit einem traurigen Muster aus schwarzen Blumen und Schatten geschmückt hatte. Irgendwie schien das... zu traurig, um es sich vorstellen zu können, denn Slytherins große Schlange hatte Lord Voldemort nur helfen wollen, und Lord Voldemort hatte einfach... es hatte etwas unerträglich Trauriges an sich, was für ein Mensch würde das einem Wesen \emph{antun}, das ihm nichts als Freundschaft angeboten hatte... „\emph{Glauben} Sie, der Dunkle Lord hätte -"

„Ja“, sagte Professor Quirrell rundheraus. „Er, dessen Name nicht genannt werden darf hinterließ eine ziemliche Spur von Leichen, Mr. Potter; ich bezweifle, dass er diese ausgelassen hätte. Wenn es dort irgendwelche Artefakte gab, die bewegt werden konnten, hätte der Dunkle Lord diese auch mitgenommen. Es könnte immer noch etwas Sehenswertes in der Kammer des Schreckens geben, und sie zu finden, würde Sie als der wahre Erbe Slytherins bestätigen. Aber machen Sie sich keine allzu großen Hoffnungen. Ich vermute, dass Sie nur die Überreste von Slytherins Monster finden werden, das still in seinem Grab ruht."

Sie saßen eine Weile in Stille.

„Ich könnte mich irren“, sagte Professor Quirrell. „Am Ende ist es nur eine Vermutung. Aber ich wollte Sie warnen, Mr. Potter, damit Sie nicht allzu sehr enttäuscht werden."

Harry nickte kurz.

„Man könnte sogar den Sieg Ihres Baby-Ichs bedauern“, sagte Professor Quirrell. Sein Lächeln wurde schief. „Würde Er, dessen Name nicht genannt werden darf leben, hätten Sie ihn vielleicht dazu überredet, Ihnen etwas von dem Wissen beizubringen, das Ihr Erbe gewesen wäre, von einem Erben Slytherins an einen anderen. Das Lächeln wurde noch schiefer, als wolle es die offensichtliche Unmöglichkeit verspotten, selbst unter dieser Prämisse.

\emph{Anmerkung an mich selbst}: dachte Harry, mit einem leichten Schauer und einem Hauch von Wut, \emph{sicherstellen, dass ich mein Erbe auf die eine oder andere Weise aus dem Geist des Dunklen Lords heraushole}.

Es gab eine weitere Stille. Professor Quirrell schaute Harry an, als warte er darauf, dass er etwas fragen würde.

„Nun“, sagte Harry, „wenn wir schon beim Thema sind, darf ich fragen, wie Sie die ganze Parselmund-Sache eigentlich finden -"

Es klopfte also an die Tür. Professor Quirrell erhob einen warnenden Finger und öffnete dann die Tür mit einem Winken. Die Kellnerin trat ein und balancierte eine riesige Platte mit ihren Mahlzeiten, als ob die ganze Ansammlung nichts wiegen würde (was wahrscheinlich auch tatsächlich der Fall war). Sie gab Professor Quirrell seine Schüssel mit grüner Suppe und ein Glas seines üblichen Chianti; und stellte vor Harry einen Teller mit kleinen Fleischstreifen in einer schwer aussehenden Soße und ein Glas seiner gewohnten Siruplimonade ab. Dann verbeugte sie sich, wobei sie es schaffte, es eher als aufrichtigen Respekt als als oberflächliche Anerkennung erscheinen zu lassen, und verschwand.

Als sie weg war, hielt Professor Quirrell wieder einen Finger hoch, als Zeichen zu schweigen, und zog seinen Zauberstab.

Und dann begann Professor Quirrell mit einer Reihe von Beschwörungsformeln, die Harry erkannte und die ihn zu einem scharfen Atemzug veranlassten. Es war die Serie und die Reihenfolge, die Mr. Bester benutzt hatte, der vollständige Satz von siebenundzwanzig Zaubersprüchen, die man vor einer Diskussion über etwas von wirklich großer Bedeutung ausführen würde.

Wenn die Diskussion über die Kammer des Schreckens nicht \emph{so} wichtig gewesen war -

Als Professor Quirrell fertig war - er hatte \emph{dreißig} Zaubersprüche ausgeführt, von denen Harry drei noch nie gehört hatte - sagte der Verteidigungsprofessor: „Jetzt werden wir eine Zeit lang nicht unterbrochen. Können Sie ein Geheimnis bewahren, Mr. Potter?"

Harry nickte.

„Ein ernstes Geheimnis, Mr. Potter“, sagte Professor Quirrell. Seine Augen waren entschlossen, sein Gesicht ernst. „Eines, das mich möglicherweise nach Askaban bringen könnte. Denken Sie darüber nach, bevor Sie antworten."

Einen Moment lang sah Harry nicht einmal, warum die Frage angesichts seiner wachsenden Sammlung von Geheimnissen so schwer sein sollte. Dann -

\emph{Wenn dieses Geheimnis Professor Quirrell nach Askaban schicken konnte, bedeutet das, dass er etwas Illegales getan hat...}

Harrys Gehirn stellte ein paar Berechnungen an. Wie auch immer das Geheimnis aussehen mochte, Professor Quirrell dachte nicht, dass seine illegale Tat in Harrys Augen einen schlechten Eindruck auf ihn machen würde. Es gab keinen Vorteil, wenn man es \emph{nicht} hörte. Und wenn es etwas enthüllte, das mit Professor Quirrell nicht in Ordnung war, dann war es für Harry sehr vorteilhaft, es zu wissen, auch wenn er versprochen hatte, es niemandem zu erzählen.

„Ich hatte nie sehr viel Respekt vor Autoritäten“, sagte Harry. „Juristische und staatliche Autorität eingeschlossen. Ich werde Ihr Geheimnis bewahren."

Harry machte sich nicht die Mühe zu fragen, ob die Enthüllung die Gefahr wert sei, die sie für Professor Quirrell darstellen würde. Der Verteidigungsprofessor war nicht dumm.

„Dann muss ich prüfen, ob Sie wirklich ein Nachfahre von Salazar sind“, sagte Professor Quirrell und stand von seinem Stuhl auf. Harry, mehr durch Reflex und Instinkt als durch Berechnung veranlasst, erhob sich ebenfalls von seinem eigenen Stuhl.

Es gab ein Verschwimmen, eine Verschiebung, eine plötzliche Bewegung.

Harry brach seinen panischen Rückwärtssprung nach der Hälfte der Zeit ab, so dass er mit seinen Armen rudern musste um nicht umzufallen, wobei ein Adrenalinschub durch ihn hindurchging.

Am anderen Ende des Raumes schwankte eine Schlange, die einen Meter hoch war, hellgrün und aufwendig in Weiß und Blau gebändert. Harry kannte nicht genug Schlangen, um sie zu erkennen, aber er wusste, dass „bunt“ „giftig“ bedeutet.

Das ständige Gefühl des Untergangs hatte ironischerweise nachgelassen, nachdem sich der Verteidigungsprofessor von Hogwarts in eine giftige Schlange verwandelt hatte.

Harry schluckte hart und sagte: „Grüße - äh, hssss, nein, äh, \emph{Grüße}“.

„\emph{Sso}“, zischte die Schlange. „\emph{Du sprichst, ich höre. Ich spreche, hörst du?} „

„\emph{Ja, ich höre}“, zischte Harry. „\emph{Sie sind ein} \emph{Animaguss?} „

„\emph{Offensichtlich}“, zischte die Schlange. „\emph{Siebenunddreißig Regeln, Nummer vierunddreißig: Werde} \emph{Animaguss. Alle vernünftigen Menschen tun das, wenn sie können. Das ist sehr selten.}“ Die Augen der Schlange waren flache Oberflächen, die in dunklen Gruben lagen, scharfe schwarze Pupillen in dunkelgrauen Feldern. „\emph{Das ist die sicherste Art zu sprechen. Versstehen Sie? Kein anderer versteht} \emph{unss.}"

"\emph{Selbst wenn es} \emph{Sschlangenanimagi} \emph{sind?} "

„\emph{Nur wenn es Erben von} \emph{Sslytherinssind.}“ Die Schlange gab eine Reihe kurzer Zischlaute von sich, die Harrys Gehirn als sardonisches Lachen übersetzte. „\emph{Sslytherin} \emph{nicht dumm.} \emph{Schlangenanimaguss} \emph{nicht dasselbe wie} \emph{Parsselmund. Das wäre ein großer Fehler im Plan.}"

Nun, \emph{das} sprach definitiv dafür, dass Parsel persönliche Magie war, und nicht das Schlangen vernunftbegabte Wesen mit einer erlernbaren Sprache waren -

„Ich bin nicht registriert“, zischte die Schlange. Die dunklen Gruben ihrer Augen starrten Harry an. „\emph{Animaguss} \emph{muss registriert sein. Die Strafe ist zwei Jahre Gefängnis. Wirst du mein Geheimnis bewahren, Junge?} „

„\emph{Ja}“, zischte Harry. „\emph{Würde nie ein Versprechen brechen.}"

Die Schlange schien wie unter Schock stillzuhalten und begann dann wieder zu schwanken. „\emph{Wir kommen in sieben Tagen hierher. Bring den Umhang zum ungesehenen Vorbeigehen, bring das Stundenglas, um durch die Zeit zu reisen -}"

"\emph{Sie wissen es?} „ zischte Harry schockiert. „\emph{Wie -}"

Wieder die Reihe von kurzen, schnellen Zischlauten, die als sardonisches Lachen übersetzt wurden. „\emph{Du kommst in meiner ersten Klasse an, während du noch in einer anderen Klasse bist, schlägst einen Feind mit Kuchen, zwei Kugeln der Erinnerung -}"

„\emph{Egal}“, zischte Harry. „\emph{Dumme Frage, ich vergaß, dass Sie schlau sind.}"

„\emph{Dumm das zu vergessen}“, sagte die Schlange, aber das Zischen schien nicht beleidigt zu sein.

„\emph{Stundenglas ist begrenzt}“, sagte Harry. „\emph{Kann nicht vor der neunten Stunde verwendet werden.}"

Die Schlange zuckte mit dem Kopf, ein schlangenartiges Nicken. „\emph{Viele Einschränkungen. Nur an den eigenen Gebrauch gebunden, kann nicht gestohlen werden. Kann keine anderen Menschen transportieren. Aber die} \emph{Sschlange, die in einem Beutel getragen wird, wird vermutlich mitgehen. Ich halte es für möglich, die} \emph{Ssanduhr} \emph{bewegungslos in der Hülle zu halten, ohne die Schutzzauber zu stören, während man die Hülle um sie} \emph{herum dreht. Wir werden} \emph{ess} \emph{in} \emph{ssieben} \emph{Tagen testen. Darüber hinaus werden wir nicht von Plänen} \emph{ssprechen.} \emph{Ssie} \emph{sagen nichts, zu niemandem. Kein} \emph{Zeichen} \emph{dass du etwas erwartest, keins. Verstanden?} „

Harry nickte.

"\emph{Antworte in der Sprache.}"

"\emph{Ja.}"

"\emph{Wirst tun, was ich gesagt} \emph{habe?} „

„\emph{Ja. Aber}“, Harry gab einen abgehackten, rauhen Ton von sich, mit dem sein Verstand ein zögerliches „Ähhh“ in Schlangisch übersetzt hatte, „\emph{ich verspreche aber nicht das zu tun,} \emph{wass} \emph{auch} \emph{immerdass} \emph{ist,} \emph{Ssie} \emph{haben nicht} \emph{gessagt} \emph{-}"

Die Schlange zitterte kurz, was Harrys Verstand als einen ernsten Blick übersetzte. „\emph{Natürlich nicht. Werde bei nächstem Treffen die Besonderheiten der Sache besprechen.}„

Das Verschwimmen und die Bewegung kehrten sich um, und Professor Quirrell stand wieder da. Einen Moment lang schien der Verteidigungsprofessor selbst zu schwanken, wie die Schlange geschwankt hatte, und seine Augen schienen kalt und flach zu sein; und dann richteten sich seine Schultern auf, und er war wieder ein Mensch.

Und die Aura des Untergangs war zurückgekehrt.

Professor Quirrells Stuhl rutschte für ihn zurück, und er setzte sich darin nieder. „Es macht keinen Sinn, das zu vergeuden“, sagte Professor Quirrell, als er den Löffel aufhob, „obwohl ich im Moment eine lebende Maus viel lieber hätte. Man kann den Geist nie ganz von dem Körper lösen, den er trägt, verstehen Sie?"

Harry nahm langsam Platz und begann zu essen.

„Die Linie von Salazar starb also doch nicht mit Du-weißt-schon-wem“, sagte Professor Quirrell nach einiger Zeit. „Es scheint, dass sich unter unserer feinen Studentenschaft bereits Gerüchte darüber verbreiten, dass Sie Böse sind; ich frage mich, was sie denken würden, wenn sie das wüssten."

„Oder wenn sie wüssten, dass ich einen Dementor zerstört habe“, sagte Harry und zuckte die Achseln. „Ich denke, dass sich die ganze Aufregung beim nächsten Mal, wenn ich etwas Interessantes mache, wieder legen wird. Hermine hat allerdings Probleme, und ich habe mich gefragt, ob Sie vielleicht Vorschläge für sie haben."

Der Verteidigungsprofessor aß schweigend mehrere Löffel Suppe, und als er wieder sprach, war seine Stimme seltsam flach. „Sie sorgen Sich wirklich um das Mädchen."

„Ja“, sagte Harry leise.

"Ich nehme an, dass sie Sie deshalb aus Ihrer Dementation herausholen konnte?"

„Mehr oder weniger“, sagte Harry. Die Aussage war in gewisser Weise wahr, nur nicht genau; es war nicht so, dass sein dementiertes Selbst sich darum gekümmert hätte, sondern dass es verwirrt gewesen war.

„Als ich jung war, hatte ich keine solchen Freunde.“ Immer noch die gleiche emotionslose Stimme. „Was wäre wohl aus Ihnen geworden, wenn Sie allein gewesen wären?"

Harry zitterte, bevor er sich stoppen konnte.

"Sie müssen ihr dankbar sein."

Harry nickte nur. Nicht ganz genau, aber wahr.

„Dann sage ich Ihnen was ich in Ihrem Alter wohl getan hätte, wenn es jemanden gegeben hätte, für den ich es hätte tun können...“

