

\hypertarget{das-stanford-prison-experiment-teil-10}{% \section{28. Das Stanford-Prison-Experiment, Teil 10}\label{das-stanford-prison-experiment-teil-10}}

-\/-\/-\/-\/-\/- Kapitel Sechzig: Das Stanford-Prison-Experiment, Teil 10 -\/-\/-\/-\/-\/-

"Aufwachen."

Harrys Augen flogen auf. Er erwachte ruckartig mit einem erstickten Keuchen. Er konnte sich an keine Träume erinnern, vielleicht war sein Gehirn zu erschöpft, um zu träumen, es schien, als hätte er seine Augen nur kurz geschlossen und dann einen Augenblick später dieses Wort gehört.

„Sie müssen aufwachen“, sagte die Stimme von Quirinus Quirrell. „Ich habe Ihnen so viel Zeit gegeben, wie ich konnte, aber es wäre klug, mindestens einen Einsatz Ihres Zeitumkehrers aufzusparen. Bald müssen wir vier Stunden zurück zu Marys Platz gehen, und in jeder Hinsicht so erscheinen, als hätten wir an diesem Tag nichts Interessantes getan. Ich wollte vorher mit Ihnen sprechen."

Harry setzte sich langsam inmitten der Dunkelheit auf. Sein Körper schmerzte, und zwar nicht nur an den Stellen, wo er auf dem harten Beton gelegen hatte. Bilder überschlugen sich in seinem Gedächtnis, sein bewusstloses Gehirn war zu müde gewesen war, um das alles in einen richtigen Alptraum zu entladen.

Zwölf schreckliche Leerstellen schwebten einen Metallkorridor hinunter und trübten das Metall um sie herum, das Licht verdunkelte sich und die Temperatur fiel, als die Leere versuchte, alles Leben aus der Welt zu saugen -

Kreideweiße Haut, knapp über den Knochen gespannt, die nach dem Verschwinden von Fett und Muskeln übriggeblieben war -

Eine Metalltür -

Die Stimme einer Frau -

\emph{Nein, ich habe es nicht so gemeint, bitte stirb nicht!} -

\emph{Ich kann mich nicht mehr an die Namen meiner Kinder erinnern} -

\emph{Geh nicht, nimm es nicht weg, nicht} \emph{nichtnicht}-

„Was war das für ein Ort?“ sagte Harry heiser, mit einer Stimme, die aus seiner Kehle drang, wie Wasser, das durch ein zu dünnes Rohr gepresst wurde, in der Dunkelheit klang sie fast so zerschlagen, wie Bellatrix Blacks Stimme gewesen war. „\emph{Was war das für ein Ort? Das war kein Gefängnis, das war die Hölle!}"

„Hölle?“, sagte die ruhige Stimme des Verteidigungsprofessors. „Sie meinen diese christliche Bestrafungsfantasie? Ich nehme an, es gibt eine Ähnlichkeit."

„Wie -“ Harrys Stimme war blockiert, es steckte etwas Riesiges in seiner Kehle. „Wie - wie konnten sie -“ \emph{Menschen} hatten diesen Ort gebaut, jemand hatte Askaban \emph{gemacht}, sie hatten es \emph{absichtlich} gemacht, sie hatten es \emph{freiwillig} getan, diese Frau, sie hatte Kinder gehabt, Kinder, an die sie sich nicht erinnern würde, irgendein Richter hatte entschieden, dass ihr das passieren sollte, jemand hatte sie in diese Zelle \emph{schleifen} und die Tür verschließen müssen, während sie schrie, jemand hatte sie jeden Tag gefüttert und ging weg, \emph{ohne sie rauszulassen} -

"\emph{WIE KONNTEN MENSCHEN DAS TUN?}"

„Warum sollten sie nicht?“, sagte der Verteidigungsprofessor. Ein blassblaues Licht erhellte das Lagerhaus, das eine hohe, höhlenartige Betondecke und einen staubigen Betonboden zeigte; und Professor Quirrell saß in einiger Entfernung von Harry und lehnte sich mit dem Rücken an eine bemalte Wand; das blassblaue Licht verwandelte die Wände in Gletscherflächen, den Staub auf dem Boden in gesprenkelten Schnee, und der Mann selbst war zu einer Eisskulptur geworden, in Dunkelheit gehüllt, wo seine schwarzen Roben über ihm lagen. „Welchen Nutzen haben die Gefangenen von Askaban für sie?„

Harrys Mund öffnete sich krächzend. Keine Worte kamen heraus.

Ein schwaches Lächeln zuckte auf den Lippen des Verteidigungsprofessors. „Wissen Sie, Mr. Potter, wenn Er, dessen Namen nicht genannt werden darf, gekommen wäre, um über das magische Großbritannien zu herrschen, und einen solchen Ort wie Askaban gebaut hätte, hätte er ihn gebaut, weil er es genoss, seine Feinde leiden zu sehen. Und wenn er stattdessen begann, ihr Leiden geschmacklos zu finden, dann würde er am nächsten Tag den Abriss Askabans anordnen. Was diejenigen betrifft, die Askaban gebaut haben, und diejenigen, die es nicht abreißen, während sie erhabene Predigten halten und sich \emph{nicht} für Schurken halten... nun, Mister Potter, ich denke, wenn ich die Wahl hätte, Tee mit ihnen zu trinken oder Tee mit Du-weißt-schon-wem zu trinken, würde ich meine Empfindungen vom Dunklen Lord weniger beleidigt finden“.

„Ich verstehe nicht“, sagte Harry, seine Stimme zitterte, er hatte von dem klassischen Experiment über die Psychologie von Gefängnissen gelesen, den gewöhnlichen College-Studenten, die sadistisch geworden waren, sobald ihnen die Rolle von Gefängniswärtern zugewiesen wurde; Erst jetzt merkte er, dass das Experiment nicht die richtige Frage untersucht hatte, die wichtigste Frage, sie hatten nicht auf die Schlüsselfiguren geschaut, nicht auf die Gefängniswärter, sondern auf \emph{alle anderen}: „Ich verstehe wirklich nicht, Professor Quirrell, wie können die Leute einfach zusehen und dies geschehen lassen, \emph{warum} \emph{tut} das Land des magischen Britannien \emph{dies} -“ Harrys Stimme verstummte.

Die Augen des Verteidigungsprofessors schienen in dem blassblauen Licht die gleiche Farbe wie immer zu haben, denn dieses Licht hatte die gleiche Farbe wie die Iris von Quirinus Quirrell, diese niemals auftauenden Eissplitter. „Willkommen, Mister Potter, zu Ihrer ersten Begegnung mit den Realitäten der Politik. Was haben die erbärmlichen Kreaturen in Askaban irgendeiner Fraktion zu bieten? Wem würde es nützen, ihnen zu helfen? Ein Politiker, der sich offen auf ihre Seite schlüge, würde assoziiert werden mit Kriminellen, mit Schwäche, mit geschmacklosen Dingen, an die die Menschen lieber nicht denken würden. Alternativ könnte der Politiker seine Macht und Grausamkeit demonstrieren, indem er längere Strafen fordert; um Stärke zu zeigen, muss man schließlich ein Opfer unter sich zerquetschen. Und die Bevölkerung applaudiert, denn es ist ihr Instinkt, den Sieger zu unterstützen“. Ein kühl amüsiertes Lachen. „Sehen Sie, Mr. Potter, niemand glaubt jemals ganz daran, \emph{selbst} nach Askaban gehen zu müssen, also sehen sie für sich selbst keinen Schaden darin. Und was sie anderen antun... Ich nehme an, man hat Ihnen einmal gesagt, dass die Menschen sich um solche Dinge kümmern? Das ist eine Lüge, Mr. Potter, die Menschen kümmern sich nicht im Geringsten darum, und wenn Sie nicht eine sehr behütete Kindheit geführt hätten, hätten Sie das schon längst gemerkt. Trösten Sie sich damit: die jetzt in Askaban Gefangenen haben für dieselben Zaubereiminister gestimmt, die versprochen hatten, ihre Zellen näher an die Dementoren zu verlegen. Ich gebe zu, Mr. Potter, dass ich wenig Hoffnung für die Demokratie als effektive Regierungsform sehe, aber ich bewundere die Poesie, wie sie ihre Opfer zu Komplizen ihrer eigenen Zerstörung macht“.

Harrys kürzlich wieder zusammenhängendes Selbst drohte nochmals in Fragmente zu zerfallen, die Worte fielen wie Hammerschläge auf sein Bewusstsein und trieben ihn Schritt für Schritt über die Klippe, wo ein gewaltiger Abgrund lauerte; und er versuchte, etwas zu finden, um sich selbst zu retten, eine kluge Erwiderung, die die Worte widerlegen würde, aber sie kam nicht.

Der Verteidigungsprofessor beobachtete Harry, der Blick reflektierte mehr Neugier als Befehl. „Es ist sehr einfach, Mr. Potter, zu verstehen, wie Askaban aufgebaut wurde und wie es weiterhin besteht. Männer sorgen sich um das, was sie selbst zu erleiden oder zu gewinnen erwarten; und solange sie nicht erwarten, dass es auf sie selbst zurückfällt, sind ihrer Grausamkeit und Nachlässigkeit keine Grenzen gesetzt. Alle anderen Zauberer dieses Landes unterscheiden sich innerlich nicht von dem, der versucht hat, über sie zu herrschen, Du-weißt-schon-wem; ihnen fehlt nur seine Macht und seine... Offenheit“.

Die Hände des Jungen waren zu Fäusten geballt, so fest, dass die Nägel in seine Handfläche schnitten. Wenn seine Finger weiß oder sein Gesicht blass war, konnte man das nicht sehen, denn das schwache blaue Licht tauchte alles in Eis oder Schatten. „Sie haben mir einmal angeboten, mich zu unterstützen, wenn mein Ehrgeiz wäre der nächste Dunkle Lord zu sein. Ist das der Grund, Professor?„

Der Verteidigungsprofessor neigte den Kopf, ein dünnes Lächeln auf den Lippen. „Lernen Sie alles, was ich Sie zu lehren habe, Mr. Potter, und Sie werden dieses Land mit der Zeit regieren. Dann können Sie das Gefängnis niederreißen, das die Demokratie geschaffen hat, wenn Sie feststellen, dass Askaban immer noch Ihre Sensibilität beleidigt. Ob es Ihnen gefällt oder nicht, Mr. Potter, Sie haben an diesem Tag gesehen, dass Ihr eigener Wille mit dem Willen der Bevölkerung dieses Landes in Konflikt steht und dass Sie nicht den Kopf beugen und sich ihrer Entscheidung unterwerfen, wenn dies geschieht. Also sind Sie für sie, ob sie es wissen oder nicht und ob Sie es anerkennen oder nicht, ihr nächster Dunkler Lord“.

In dem unbeirrbaren, monochromatischen Licht wirkten der Junge und der Verteidigungsprofessor wie unbewegliche Eisskulpturen, die Iris ihrer Augen auf ähnliche Farben reduziert so dass sie in diesem Licht sehr ähnlich aussahen.

Harry starrte direkt in diese blassen Augen. All die lange unterdrückten Fragen, von denen er sich gesagt hatte, dass er sie bis zu den Iden des Mai zurückstellen würde. Das war eine Lüge gewesen, wusste Harry jetzt, eine Selbsttäuschung, er hatte aus Angst vor dem, was er hören könnte, geschwiegen. Und nun kam alles auf einmal über seine Lippen. „An unserem ersten Unterrichtstag haben Sie versucht, meine Mitschüler davon zu überzeugen, dass ich ein Mörder bin."

„Sie sind einer.“ Amüsiert. „Aber wenn Ihre Frage lautet, warum ich ihnen das \emph{gesagt} habe, Mr. Potter, dann ist die Antwort, dass Sie auf Ihrem Weg zur Macht die Zweideutigkeit als großen Verbündeten finden werden. Geben Sie an einem Tag ein Zeichen von Slytherin, und widersprechen Sie am nächsten Tag mit einem Zeichen von Gryffindor; und die Slytherins werden in die Lage versetzt, zu glauben, was sie wollen, während auch die Gryffindors sich selbst überreden Sie zu unterstützen. Solange es Ungewissheit gibt, können die Menschen glauben, was immer zu ihrem eigenen Vorteil zu sein scheint. Und solange Sie stark erscheinen, solange Sie zu gewinnen scheinen, werden ihre Instinkte ihnen sagen, dass ihr Vorteil bei Ihnen liegt. Gehen Sie immer im Schatten, und Licht und Dunkelheit werden beide folgen.“

„Und“, sagte der Junge mit unveränderter Stimmlage, „was genau wollen \emph{Sie} mit all dem erreichen?„

Professor Quirrell hatte sich an die Wand zurückgelehnt, sein Gesicht in Schatten gehüllt, seine Augen verwandelten sich von hellem Eis in dunkle Gruben wie die seiner Schlangenform. „Ich wünsche mir, dass Großbritannien unter einem starken Führer erstarken wird; das \emph{ist} mein Wunsch. Was meine Gründe dafür betrifft“, lächelte Professor Quirrell ohne Heiterkeit, „so denke ich, dass sie meine eigenen bleiben werden.“

„Das Gefühl des Untergangs, das ich um Sie herum spüre.“ Die Worte wurden immer schwerer auszusprechen, da das Thema immer näher an etwas Schreckliches und Verbotenes heranrückte. „Sie wussten schon immer, was es bedeutet."

„Ich hatte mehrere Vermutungen“, sagte Professor Quirrell, sein Gesichtsausdruck war nicht lesbar. „Und ich werde noch nicht alles sagen, was ich erraten habe. Aber so viel will ich Ihnen sagen: Es ist \emph{Ihr} Verhängnis, das auflodert, wenn wir uns nähern, nicht meines."

Ausnahmsweise schaffte es Harrys Gehirn einmal, dies als fragwürdige Behauptung und mögliche Lüge zu markieren, anstatt alles zu glauben, was es hörte. „Warum verwandeln Sie sich manchmal in einen Zombie?"

„Persönliche Gründe“, sagte Professor Quirrell ohne jeglichen Humor in seiner Stimme.

"Was war Ihr Hintergedanke dabei, Bellatrix zu retten?"

Es gab eine kurze Stille, während der Harry sich sehr bemühte, seine Atmung zu kontrollieren, sie ruhig zu halten.

Schließlich zuckte der Verteidigungsprofessor die Achseln, als ob dies nicht von Bedeutung wäre. „Ich habe es Ihnen fast buchstabiert, Mr. Potter. Ich habe Ihnen alles gesagt, was Sie brauchten, um die Antwort abzuleiten, wenn Sie reif genug gewesen wären, diese erste offensichtliche Frage in Betracht zu ziehen. Bellatrix Black war die mächtigste Dienerin des Dunklen Lords, ihre Loyalität war am sichersten; sie war die einzige Person, der am ehesten ein Teil der verlorenen Überlieferung Slytherins anvertraut werden konnte, die Ihnen hätte gehören sollen.

Langsam kroch die Wut über Harry, langsam begann der Zorn, begann etwas Schreckliches sein Blut zum Kochen zu bringen, in nur wenigen Augenblicken würde er etwas sagen, dass er wirklich nicht sagen sollte, während sie beide allein in einem verlassenen Lagerhaus waren -

„Aber sie \emph{war} unschuldig“, sagte der Verteidigungsprofessor. Er lächelte nicht. „Und das Ausmaß, in dem ihr all ihre Wahlmöglichkeiten genommen wurden, so dass sie nie eine Chance hatte, für ihre \emph{eigenen} Fehler zu leiden... das erschien mir \emph{übertrieben}, Mr. Potter. Wenn sie Ihnen nichts Nützliches erzählt -“ Der Verteidigungsprofessor zuckte kurz mit den Schultern. „werde ich die Arbeit dieses Tages nicht als Verschwendung betrachten."

„Wie uneigennützig von Ihnen“, sagte Harry kalt. „Wenn also alle Zauberer im Innern wie Du-weißt-schon-wer sind, sind Sie da eine Ausnahme?"

Die Augen des Verteidigungsprofessors waren immer noch im Schatten, dunkle Gruben, die nicht durchdrungen werden konnten. „Nennen Sie es eine Laune, Mr. Potter. Es hat mich manchmal amüsiert, die Rolle des Helden zu spielen. Wer weiß, ob Du-weißt-schon-wer das Gleiche sagen würde."

Harry öffnete ein letztes Mal den Mund -

Und stellte fest, dass er es nicht sagen konnte, er konnte die letzte Frage, die letzte und wichtigste Frage nicht stellen, er konnte die Worte nicht herausbringen. Auch wenn eine solche Weigerung einem Rationalisten verboten war, konnte er sich nicht dazu durchringen, seine letzte Frage laut auszusprechen, obwohl er die Tarski-Litanei oder die Gendlin-Litanei rezitiert oder geschworen hatte, dass alles, was durch die Wahrheit zerstört werden könnte, in diesem einen Moment auch zerstört sein sollte. Obwohl er wusste, dass er falsch dachte, obwohl er wusste, dass er eigentlich besser sein sollte, konnte er es trotzdem nicht sagen.

„Jetzt bin ich an der Reihe, Sie zu befragen.“ Professor Quirrell setzte sich auf und lehnte sich nicht mehr gegen die Gletscherwand aus bemaltem Beton. „Ich frage mich, Mister Potter, ob Sie etwas dazu zu sagen haben, dass Sie mich beinahe umgebracht und unser gemeinsames Unterfangen ruiniert hätten. Mir wurde zu verstehen gegeben, dass eine Entschuldigung in solchen Fällen als Zeichen des Respekts betrachtet wird. Aber Sie haben mir keinen Respekt entgegengebracht. Ist es nur so, dass Sie noch nicht dazu gekommen sind, Mr. Potter?"

Der Ton war ruhig, aber schneidend. Die Schneide aber so fein und scharf, dass sie dich durchbohren würde, bevor du merktest, dass du ermordet wurdest.

Und Harry sah den Verteidigungsprofessor nur mit kühlen Augen an, die vor nichts und niemandem zurückschrecken würden, nicht einmal vor dem Tod. Er war nicht mehr in Askaban, er fürchtete sich nicht mehr vor dem Teil von sich selbst, der furchtlos war; und der massive Edelstein, der Harry war, hatte sich gedreht, um dem Druck zu begegnen, er drehte sich von einer Facette zur anderen, von Licht zu Dunkelheit, von warm zu kalt.

\emph{\emph{Ein kalkulierter Trick seinerseits, um mir ein schlechtes Gewissen zu machen, mich in eine Lage zu bringen, in der ich mich unterwerfen muss?}}

\emph{\emph{Ein echtes Gefühl seinerseits?}}

„Ich verstehe“, sagte Professor Quirrell. „Ich nehme an, das beantwortet -"

„Nein“, sagte der Junge mit kühler, gefasster Stimme, „so einfach können Sie das Gespräch nicht gestalten, Professor. Ich habe erhebliche Anstrengungen unternommen, um Sie zu schützen und sicher aus Askaban herauszubringen, \emph{nachdem} ich dachte, Sie hätten versucht, einen Polizisten zu töten. Dazu gehörte auch, zwölf Dementoren ohne Patronuszauber zu begegnen. Ich frage mich, wenn ich mich entschuldigt hätte, als Sie es verlangten, hätten Sie sich im Gegenzug dafür bedankt? Oder liege ich richtig, wenn ich denke, dass es meine Unterwerfung war, die Sie da verlangten, und nicht nur mein Respekt?

Es gab eine Pause, und dann antwortete die Stimme von Professor Quirrell, unverhüllt eisig und gefährlich. „Es scheint, Sie können sich immer noch nicht überwinden, zu verlieren, Mr. Potter."

Dunkelheit starrte aus Harrys Augen, ohne mit der Wimper zu zucken, und der Verteidigungsprofessor selbst wurde zu einem sterblichen Ding in ihnen. „Oh, und überlegen \emph{Sie} jetzt, ob \emph{sie} vorgeben sollten, gegen mich zu verlieren, und vorzugeben sich vor meinem eigenen Zorn zu demütigen, um Ihre eigenen Pläne zu schützen? Ist Ihnen der Gedanke an eine kalkulierte falsche Entschuldigung \emph{überhaupt in den Sinn gekommen}? Mir auch nicht, Professor Quirrell."

Der Verteidigungsprofessor lachte, leise und humorlos, leerer als die Leere zwischen den Sternen, gefährlich wie ein mit harter Strahlung gefülltes Vakuum. „Nein, Mr. Potter, Sie haben Ihre Lektion nicht gelernt, ganz und gar nicht."

„Ich habe oft daran gedacht, in Askaban zu verlieren“, sagte der Junge, seine Stimme blieb ruhig. „Dass ich einfach aufgeben und mich den Auroren ausliefern sollte. Zu verlieren wäre das Vernünftigste gewesen. Ich habe Ihre Stimme gehört, die mir das im Geiste sagte, und ich hätte es \emph{getan}, wenn ich allein dort gewesen wäre. Aber ich konnte es nicht über mich bringen, \emph{Sie} zu verlieren.“

Es herrschte eine Zeitlang Schweigen; als ob selbst der Verteidigungsprofessor nicht recht wüsste, was er dazu sagen sollte.

„Ich bin neugierig“, sagte Professor Quirrell schließlich. „Wofür genau sollte ich mich Ihrer Meinung nach entschuldigen? Ich habe Ihnen ausdrückliche Anweisungen für den Fall eines Kampfes gegeben. Sie sollten am Boden bleiben, aus dem Weg gehen und keine Magie wirken. Sie haben gegen diese Anweisungen verstoßen und die Mission zu Fall gebracht."

„Ich traf keine Entscheidung“, sagte der Junge, „es gab keine Wahl, nur den Wunsch, dass der Auror nicht sterben sollte, und mein Patronus war dort. Damit dieser Wunsch nie eingetreten wäre, hätten Sie mich warnen müssen, dass Sie mit einem tödlichen Fluch bluffen könnten. Standardmäßig nehme ich an, wenn Sie Ihren Zauberstab auf jemanden richten und Avada Kedavra sagen, dann nur, weil Sie seinen Tod wünschen. Sollte das nicht die erste Sicherheitsregel für Unverzeihliche Flüche sein?“

„Regeln sind für Duelle“, sagte der Verteidigungsprofessor. Etwas von der Kälte war in seine Stimme zurückgekehrt. „Und duellieren ist ein Sport, kein Zweig der Kampfmagie. In einem echten Kampf ist ein Fluch, der nicht geblockt werden kann und dem man ausweichen \emph{muss}, eine unverzichtbare Taktik. Ich habe gedacht das wäre für Sie offensichtlich, aber es scheint, dass ich Ihren Intellekt falsch eingeschätzt habe.“

„Es scheint mir auch unüberlegt gewesen zu sein“, sagte der Junge und fuhr fort, als hätte der andere nicht gesprochen, „mir nicht \emph{zu sagen}, dass jeder Zauber, den ich gegen Sie ausspreche, uns beide töten könnte. Was wäre, wenn Ihnen ein Missgeschick passiert wäre und ich es mit einem \emph{Rennervate} oder einem Schwebezauber versucht hätte? Diese Ignoranz, die Sie aus Gründen, die ich nicht erraten kann, zugelassen haben, spielte auch eine gewisse Rolle bei dieser Katastrophe.“

Es herrschte wieder einmal Schweigen. Die Augen des Verteidigungsprofessors waren zusammengekniffen, und auf seinem Gesicht war ein leicht verwirrter Blick zu sehen, als sei er in eine völlig ungewohnte Situation geraten, und doch sprach der Mann kein Wort.

„Nun“, sagte der Junge. Seine Augen waren nicht von den Augen des Verteidigungsprofessors abgewichen. „Ich bedaure es sehr, Sie verletzt zu haben, Professor. Aber ich glaube nicht, dass die Situation es erfordert, dass ich mich Ihnen unterwerfe. Ich habe das Konzept der Entschuldigung nie wirklich verstanden, noch weniger, wie es auf eine Situation wie diese zutrifft; wenn Sie mein Bedauern haben, aber nicht meine Unterwerfung, zählt das als Entschuldigung?“

Wieder dieses kalte, kalte Lachen, dunkler als die Leere zwischen den Sternen.

„Ich weiß es nicht“, sagte der Verteidigungsprofessor, „auch ich habe das Konzept der Entschuldigung nie verstanden. Es scheint, dass dieser Trick zwischen uns sinnlos wäre, da wir beide wissen, dass es eine Lüge ist. Sprechen wir also nicht mehr davon. Die Schulden zwischen uns werden mit der Zeit beglichen werden."

Eine Zeitlang herrschte Schweigen.

„Übrigens“, sagte der Junge. „Hermine Granger hätte Askaban niemals gebaut, ganz gleich, wer da reingesteckt werden sollte. Und sie würde eher sterben, bevor sie einen Unschuldigen verletzt hätte. Ich erwähne das jetzt, da Sie mal gesagt haben, dass alle Zauberer im Innern wie Du-weißt-schon-wer sind, das ist faktisch falsch. Ich hätte es früher erkannt, wenn ich nicht“, der Junge lächelte kurz und grimmig, „gestresst gewesen wäre „.

Die Augen des Verteidigungsprofessors waren halb geschlossen, sein Gesichtsausdruck weit entfernt. „Das Innere eines Menschen ist nicht immer wie sein Äußeres, Mr. Potter. Vielleicht wünscht sie sich einfach, dass andere sie für ein gutes Mädchen halten. Sie kann den Patronuszauber nicht einsetzen -"

„Hah“, sagte der Junge; sein Lächeln wirkte jetzt realer, wärmer. „Sie hat aus genau demselben Grund Schwierigkeiten wie ich. Es ist genug Licht in ihr, um Dementoren zu vernichten, da bin ich mir sicher. Sie wäre nicht in der Lage, damit \emph{aufzuhören}, Dementoren zu vernichten, selbst auf Kosten ihres eigenen Lebens...“ Die Stimme des Jungen verklang und nahm dann wieder zu. „\emph{Ich} bin vielleicht kein so guter Mensch, aber es gibt sie, und sie ist eine von ihnen.“

Trocken kam die Antwort. „Sie ist jung, und Freundlichkeit zu zeigen, kostet sie wenig."

Es gab eine Pause. Dann sagte der Junge: „Professor, ich muss Sie fragen, wenn Sie etwas Dunkles und Düsteres sehen, kommt es Ihnen dann nicht in den Sinn, es irgendwie zu \emph{verbessern}? Zum Beispiel, ja, etwas geht in den Köpfen der Menschen furchtbar schief, das sie denken lässt, es sei toll, Kriminelle zu quälen, aber das bedeutet nicht, dass sie innerlich wirklich böse sind; und wenn Sie ihnen vielleicht die richtigen Dinge beibringen, ihnen zeigen würden, was sie falsch machen, könnten Sie sie ändern -„

Professor Quirrell lachte, und nicht mit der Leere von vorhin. „Ah, Mr. Potter, manchmal vergesse ich, wie jung Sie noch sind. Eher könnten Sie die Farbe des Himmels ändern.“ Noch ein Glucksen, diesmal kälter. „Und der Grund, warum es Ihnen leichtfällt, solchen Narren zu verzeihen und gut über sie zu denken, Mr. Potter, ist, dass Sie selbst nie schwer verletzt worden sind. Sie werden weniger gern an gewöhnliche Idioten denken, nachdem ihre Torheit Sie das erste Mal etwas Liebes kostet. Wie zum Beispiel hundert Galleonen aus Ihrer eigenen Tasche, anstatt den qualvollen Tod von hundert Fremden.“ Der Verteidigungsprofessor lächelte dünn. Er nahm eine Taschenuhr aus seiner Robe und sah sie an. „Lassen Sie uns jetzt aufbrechen, wenn es zwischen uns nichts mehr zu sagen gibt."

"Sie haben keine Fragen zu den unmöglichen Dingen, die ich getan habe, um uns aus Askaban herauszuholen?"

„Nein“, sagte der Verteidigungsprofessor. „Ich glaube, die meisten davon habe ich bereits gelöst. Was den Rest betrifft, so ist es zu selten, dass ich eine Person finde, die ich nicht sofort durchschaue, sei es Freund oder Feind. Ich werde zu gegebener Zeit die Rätsel um Sie selbst lösen.“

Der Verteidigungsprofessor schob sich hoch, drückte sich mit beiden Händen von der Wand weg und erhob sich geschmeidig, wenn auch zu langsam, auf seine Füße. Der Junge, weniger anmutig, tat dasselbe.

Und der Junge platzte mit der letzten schrecklichen Frage heraus, die er zuvor nicht hatte stellen können; als ob sie durchs Aussprechen real werden würde, und als ob sie nicht schon ganz offensichtlich wäre.

"Warum bin ich nicht wie die anderen Kinder meines Alters?„

In einer menschenleeren Seitenstraße der Winkelgasse, wo man Fetzen von nicht verschwundenem Müll in den Rändern der Backsteinstraße und den leeren Backsteinwänden, die sie umgaben, zusammen mit verstreutem Schmutz und anderen Zeichen der Vernachlässigung sehen konnte, erschien ein alter Zauberer und sein Phönix.

Der Zauberer griff bereits in sein Gewand, um sein Stundenglas zu holen, als seine Augen aus Gewohnheit zu einer beliebigen Stelle zwischen der Straße und der Mauer sprangen, um sie sich einzuprägen -

Und der alte Zauberer blinzelte überrascht; an dieser Stelle war ein Stück Pergament.

Ein Stirnrunzeln zog über Albus Dumbledores Gesicht, als er einen Schritt vorwärts ging und den zerknitterten Fetzen nahm und ihn entfaltete.

Darauf stand ein einziges Wort: „NEIN“, und sonst nichts.

Langsam ließ der Zauberer es aus seinen Fingern flattern. Abwesend griff er zum Bürgersteig hinunter und hob den nächsten Pergamentfetzen auf, der dem soeben genommenen auffallend ähnlich sah; er berührte ihn mit seinem Zauberstab, und einen Augenblick später war er mit dem gleichen Wort „NEIN“ beschriftet, in der gleichen Handschrift, die seine eigene war.

Der alte Zauberer hatte geplant, zu dem Zeitpunkt zurückzugehen, an dem Harry Potter vor drei Stunden zum ersten Mal in der Winkelgasse ankam. Er hatte bereits auf seinen Instrumenten beobachtet, wie der Junge Hogwarts verließ, und das konnte nicht mehr rückgängig gemacht werden (sein einziger Versuch, seine eigenen Instrumente zu täuschen und so die Zeit zu kontrollieren, ohne das Aussehen der Zeit für ihn selbst zu verändern, endete in einer ausreichenden Katastrophe, um ihn davon zu überzeugen, nie wieder solche Tricks zu versuchen). Er hatte gehofft, den Jungen im erstmöglichen Augenblick nach seiner Ankunft zurückzuholen und ihn an einen anderen sicheren Ort, wenn nicht gar nach Hogwarts zu bringen (denn seine Instrumente hatten die Rückkehr des Jungen nicht angezeigt). Aber jetzt -

„Ein Paradoxon, wenn ich ihn sofort nach seiner Ankunft in der Winkelgasse zurückhole“, murmelte der alte Zauberer vor sich hin. „Vielleicht haben sie ihren Plan, Askaban auszurauben, erst in Gang gesetzt, nachdem sie seine Ankunft hier bestätigt hatten... oder aber... vielleicht..."

Bemalter Beton, harter Boden und weit entfernte Decken, zwei einander gegenüberstehende Figuren. Die eine Figur trug die Gestalt eines Mannes Ende dreißig, der bereits eine Glatze hatte, und die andere die eines elfjährigen Jungen mit einer Narbe auf der Stirn. Eis und Schatten, blassblaues Licht.

„Ich weiß es nicht“, sagte der Mann.

Der Junge sah ihn nur an. Und dann sagte er: „Ach, wirklich?"

„Wirklich“, sagte der Mann. „Ich weiß nichts, und von meinen Vermutungen will ich nicht sprechen. Und doch werde ich so viel sagen -“

