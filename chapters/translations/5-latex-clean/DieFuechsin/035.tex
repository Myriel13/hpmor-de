

\hypertarget{abstimmungsprobleme-teil-3}{% \section{3. Abstimmungsprobleme, Teil 3}\label{abstimmungsprobleme-teil-3}}

—\/-\/-\/-\/- Kapitel 35: Abstimmungsprobleme, Teil 3 -\/-\/-\/-\/-

Sie waren zum Büro des Verteidigungsprofessors gegangen und Professor Quirrell hatte die Tür versiegelt, ehe er sich in seinem Stuhl zurücklehnte und sprach.

Die Stimme des Verteidigungsprofessors war sehr ruhig und das verunsicherte Harry ein gutes Stück mehr, als wenn Professor Quirrell geschrien hätte.

„Ich versuche“, sagte Professor Quirrell leise, „der Tatsache Rechnung zu tragen, dass Sie jung sind. Dass ich selbst, in Ihrem Alter, ein außergewöhnlich großer Narr war. Sie sprechen auf erwachsene Art und Weise und mischen sich in die Angelegenheiten Erwachsener ein, und manchmal vergesse ich, dass Sie nur ein Wichtigtuer sind. Ich hoffe, Mr~Potter, dass Ihre kindliche Einmischung Sie nicht gerade getötet, Ihr Land ruiniert und Sie den nächsten Krieg gekostet hat.“

Es fiel Harry sehr schwer, seine Atmung zu kontrollieren. „Professor Quirrell, ich habe ein ganzes Stück weniger gesagt, als mir lieb gewesen wäre, aber ich musste etwas sagen. Ihre Vorschläge sind extrem alarmierend für jeden, der auch nur die geringste Kenntnis über die Geschichte der Muggel im letzten Jahrhundert hat. Die italienischen Faschisten, sehr scheußliche Leute, bekamen ihren Namen von den \emph{Fasces}, einem Bündel aus Ruten, welche zusammengebunden wurden, um die Idee zu symbolisieren, dass Einheit Stärke ist —“

„Dann haben die scheußlichen italienischen Faschisten also geglaubt, dass Einheit stärker als Spaltung ist“, sagte Professor Quirrell. Schärfe begann sich in seine Stimme zu schleichen. „Vielleicht haben sie auch geglaubt, dass der Himmel blau ist, und ein Politik verfochten sich keine Steine auf den Kopf zu werfen.“

\emph{Umgekehrte Dummheit ist keine Intelligenz; der dümmste Mensch der Welt kann sagen die Sonne scheint, aber das macht es draußen nicht dunkel…} „Schön, Sie haben recht, das war ein Argumentum ad hominem*, es ist nicht falsch \emph{weil} die Faschisten es gesagt haben. Aber Professor Quirrell, man kann nicht jeden in einem Land das Zeichen eines Diktators annehmen lassen! Es ist ein Single Point of Failure**! Sehen Sie, ich sage es mal so. Angenommen, wer auch immer das Mal kontrolliert wird vom Feind mit dem Imperius —“

„Mächtige Zauberer belegt man nicht so einfach mit dem Imperius“, sagte Professor Quirrell trocken. „Und wenn man keinen würdigen Anführer findet, ist man in jedem Fall verloren. Aber würdige Anführer existieren; die Frage ist, ob das Volk ihnen folgen sollte.“

Harry fuhr sich frustriert mit den Händen durch sein Haar. Er wollt um eine Auszeit bitten und Professor Quirrell dazu bringen \emph{Aufstieg und Fall des Dritten Reiches} zu lesen und die Konversation danach noch einmal zu beginnen. „Ich nehme nicht an, dass falls ich andeuten würde, Demokratie sei eine bessere Regierungsform als Diktatur —“

„Ich verstehe“, sagte Professor Quirrell. Kurz schlossen sich seine Augen, dann öffneten sie sich wieder. „Mr~Potter, die Unvernunft des Quidditch ist für Sie durchschaubar, da Sie nicht damit aufgewachsen sind, das Spiel zu verehren. Wenn Sie niemals von Wahlen gehört hätten, Mr~Potter, und Sie würden nur \emph{sehen was da ist}, es würde Ihnen nicht gefallen. Sehen Sie sich unseren gewählten Zaubereiminister an. Ist er der Weiseste, der Stärkste, der Größte aus unserer Nation? Nein; er ist Hanswurst, der sein Gehalt von Lucius Malfoy erhält. Zauberer gingen wählen und entschieden sich zwischen Cornelius Fudge und Tania Leach, die in einem großen und unterhaltsamen Wahlkampf gegeneinander angetreten sind, nachdem der Tagesprophet, den Lucius Malfoy kontrolliert, entschieden hatte, dass sie die einzigen ernstzunehmenden Kandidaten waren. Dass Cornelius Fudge ernsthaft als bester Anführer, den unserer Land zu bieten hat, ausgewählt wurde ist nichts, was irgendjemand behaupten könnte ohne eine Miene zu verziehen. In der Muggelwelt ist das nicht anders, nach allem war ich gehört oder gesehen habe; die letzte Muggelzeitung, die ich gelesen habe, erwähnte, dass der vorherige Präsident der Vereinigten Staaten ein Filmschauspieler im Ruhestand war. Wären Sie nicht mit Wahlen aufgewachsen, Mr~Potter, Sie würden Ihnen genauso offensichtlich als dumm erscheinen wie Quidditch.“

Harry saß mit offenem Mund da, um Worte ringend. „Der Punkt von Wahlen ist nicht, den einen besten Anführer zu finden, es geht darum Politiker die Wähler so fürchten zu lassen, dass sie nicht vollkommen böse werden, so wie Diktatoren es tun ~—“

„Der letzte Krieg, Mr~Potter, wurde zwischen dem Dunklen Lord und Dumbledore ausgefochten. Und obwohl Dumbledore ein unvollkommener Anführer war, der dabei war den Krieg zu verlieren, wäre es \emph{lächerlich} anzudeuten, dass \emph{irgendeiner} der in dieser Zeit gewählten Zaubereiminister Dumbledores Platz hätte einnehmen können! Stärke geht von mächtigen Zauberern und ihren Anhängern aus, nicht von Wahlen und den Narren, die sie hervorbringen. Dies ist die Lektion der jüngeren Geschichte des magischen Britanniens; und ich bezweifle, dass die Lektion des nächsten Krieges eine andere sein wird. \emph{Falls} Sie den überleben, Mr~Potter, was Sie \emph{nicht} tun werden, außer Sie geben Ihre verklärten Illusionen der Kindheit auf!“

„Falls Sie denken, dass es keine Gefahren in dem Vorgehen gibt, dass Sie befürworten“, sagte Harry und trotz allem wurde seine Stimme langsam scharf, „dann ist auch das eine kindliche Illusion.“

Harry starrte grimmig in Professor Quirrells Augen, der ohne zu blinzeln zurückstarrte.

„Solche Gefahren“, sagte Professor Quirrell kalt, „sollten in Büros wie diesem hier besprochen werden, nicht in Reden. Die Narren, die Cornelius Fudge gewählt haben, interessieren sich nicht für Komplikationen und Vorsichtsmaßnahmen. Konfrontieren Sie sie mit etwas Nuancierterem als einem frenetischen Jubel und Sie werden Ihren Krieg alleine kämpfen. \emph{Das}, Mr~Potter, war Ihr kindlicher Fehler, welchen Draco Malfoy nicht einmal im Alter von acht Jahren gemacht hätte. Es hätte selbst für \emph{Sie} offensichtlich sein sollen den Mund zu halten und \emph{mich zuerst zu konsultieren}, nicht mit ihren Bedenken vor der Menge herauszuplatzen!“

„Ich bin kein Freund von Albus Dumbledore“, sagte Harry, eine Kälte in der Stimme, ebenbürtig der Professor Quirrells. „Aber er ist kein Kind und er schien nicht zu denken meine Befürchtungen wären kindisch oder dass ich hätte warten sollen sie auszusprechen.“

„Oh“, sagte Professor Quirrell, „also orientieren Sie sich jetzt am Schulleiter, oder was?“, und stand von seinem Schreibtisch auf.

Als Blaise auf dem Weg zu eben diesem Büro um die Ecke kam, sah er, dass Professor Quirrell bereits gegen Wand gelehnt dastand.

„Blaise Zabini“, sagte der Verteidigungsprofessor, sich aufrichtend; seine Augen wirkten wie schwarze Steine in seinem Gesicht und seine Stimme jagte Blaise einen Schauder über den Rücken.

\emph{Er kann nichts gegen mich ausrichten, solange ich mir dessen bewusst bin —}

„Ich glaube“, sagte Professor Quirrell mit einer klaren, kalten Stimme, „dass ich den Namen Ihres Auftraggebers bereits erraten habe. Aber ich würde ihn lieber aus Ihrem Mund hören und nennen Sie mir auch den Preis, der Sie gekauft hat.“

Blaise wusste, dass er unter seinem Umhang schwitzte und dass die Feuchtigkeit bereits auf seiner Stirn zu sehen sein musste. „Ich bekam die Chance zu zeigen, dass ich besser als alle drei Generäle war, und ich habe sie ergriffen. Eine Menge Leute hasst mich jetzt, aber es gibt auch eine Menge Slytherins, die mich dafür lieben werden. Was lässt Sie glauben ich sei —“

„Sie haben den Plan der heutigen Schlacht nicht ausgearbeitet, Mr~Zabini. Sagen Sie mir, wer es war.“

Blaise schluckte schwer. „Nunja… Ich meine, in diesem Fall… dann wissen Sie bereits, wer es war, richtig? Die einzige Person, die verrückt genug dafür ist, ist Dumbledore. Und er wird mich beschützen, sollten Sie irgendetwas versuchen.“

„In der Tat. Nennen Sie mir den Preis.“ Der Blick des Verteidigungsprofessors war nach wie vor hart.

„Es geht um meine Cousine Kimberly“, sagte Blaise, schluckte erneut und versuchte seine Stimme unter Kontrolle zu bringen. „Es gibt sie wirklich und sie wird wirklich drangsaliert, Potter hat das überprüft, er war nicht dumm. ~Nur hat Dumbledore gesagt, dass er die Raufbolde darauf angesetzt hat, nur für diesen Plan, und wenn ich für \emph{ihn} arbeitete wäre sie hinterher in Sicherheit, aber wenn ich mich an Potter halten \emph{würde}, gäbe es noch mehr Schwierigkeiten, in die Kimberly geraten könnte!“

Professor Quirrell ~war für einen langen Moment still.

„Ich verstehe“, sagte Professor Quirrell, seine Stimme jetzt viel sanfter. „Mr~Zabini, sollte so ein Vorfall noch einmal eintreten, können Sie sich direkt an mich wenden. Ich habe meine eigenen Wege meine Freunde zu beschützen. Eine letzte Frage noch: Selbst mit all der Macht, die Sie sich aneigneten, ein Unentschieden zu erzwingen wäre schwierig geworden. Hat Dumbledore Sie instruiert, wer ansonsten gewinnen sollte?“

„Sonnenschein“, sagte Blaise.

Professor Quirrell nickte. „Wie ich es mir gedacht habe.“ Der Verteidigungsprofessor seufzte. „In Ihrer späteren Laufbahn, Mr~Zabini, rate ich Ihnen keine so komplizierten Ränke zu schmieden. Sie haben eine Tendenz fehlzuschlagen.“

„Ähm, ich habe das ~genau genommen so zum Schulleiter gesagt“, sagte Blaise, „und er sagte, dass sei der Grund, aus dem es wichtig wäre, mehr als eine Intrige zur selben Zeit am Laufen zu haben.“

Professor Quirrell fuhr sich mit einer erschöpften Handbewegung über seine Stirn. „Es ist ein Wunder, dass der Dunkle Lord nicht vom Kampf mit \emph{ihm} verrückt wurde. Sie können jetzt zu Ihrem Treffen mit dem Schulleiter ~gehen, Mr~Zabini. Ich werde nichts ~hiervon erzählen, aber falls der Schulleiter irgendwie herausfindet, dass wir miteinander geredet haben, erinnern Sie sich an mein noch bestehendes Angebot, Ihnen jedweden Schutz zu bieten. Sie können wegtreten.“

Blaise benötigte keine weitere Aufforderung, er drehte sich einfach um und floh.

Professor Quirrell wartete noch einen Moment und sagte dann: „Nur zu, Mr~Potter.“

Harry zog sich den Unsichtbarkeitsumhang vom Kopf und stopfte ihn in seinen Beutel. Er zitterte vor soviel Wut, dass er nur schwer sprechen konnte. „Er hat \emph{was}? Er hat \emph{was} getan?“

„Sie hätten von selbst darauf kommen müssen, Mr~Potter“, sagte Quirrell milde. „Sie müssen lernen Ihren Blick zu trüben, bis sie den Wald trotz der Bäume sehen können. Jeder, der die Geschichten über Sie gehört hat, und der nicht wüsste, dass Sie der Junge der lebte sind, könnte leicht auf Ihren Besitz eines Unsichtbarkeitsumhangs schließen. Man trete von den Geschehnissen einen Schritt zurück, blende die Details aus und was kann man beobachten? Es gab eine starke Rivalität zwischen den Schülern und ihr Wettbewerb endete in einem perfekten Unentschieden. Diese Art von Dingen passiert nur in Geschichten, Mr~Potter, und es gibt nur eine Person an dieser Schule, die in Geschichten denkt. Es gab eine seltsame und komplizierte Intrige, welche Sie als uncharakteristisch für den jungen Slytherin hätten erkennen müssen. Aber es gibt eine Person in dieser Schule, die solch aufwendige Intrigen spinnt und ihr Name ist nicht Zabini. Und ich habe Sie gewarnt, dass es einen Quadrupel-Agenten gibt; Sie wussten, dass Zabini zumindestens ein Tripel-Agent war und Sie hätten erahnen sollen, dass er es mit einer hohen Wahrscheinlichkeit war. Nein, ich werde die Schlacht nicht für ungültig erklären. Sie haben alle drei den Test nicht bestanden und gegen Ihren gemeinsamen Feind verloren.“

An diesem Punkt kümmerten Harry Tests nicht mehr. „Dumbledore hat Zabini \emph{erpresst}, indem er seine \emph{Cousine bedroht} hat? Nur, um unsere Schlacht in einem Unentschieden enden zu lassen? \emph{Warum}?“

Professor Quirrell gab ein freudloses Lachen von sich. „Vielleicht dachte der Schulleiter die Rivalität wäre gut für seinen persönlichen, kleinen Helden und wollte sie fortgesetzt sehen. Für das größere Wohl, natürlich. Oder vielleicht war er einfach nur verrückt. Sehen Sie, Mr~Potter, jeder weiß, dass Dumbledores Wahnsinn nur eine Maske ist, dass er vernünftig ist und dabei vorgibt, geisteskrank zu sein. Sie brüsten sich mit Ihrer schlauen Einsicht und da Sie ja die geheime Erklärung kennen, hören sie auf weiterzusuchen. Es kommt Ihnen nicht in den Sinn, dass man auch eine Maske hinter einer Maske tragen, also wahnsinnig sein kann, aber so tut, als wäre man vernünftig und dabei vorgibt, wahnsinnig zu sein. Und ich fürchte, Mr~Potter, dass ich nun dringende Geschäfte andernorts habe und abreisen muss; aber ich rate Ihnen dringend, sich nicht an Albus Dumbledore zu orientieren, wenn Sie einen Krieg führen. Bis später, Mr~Potter.“

Und der Verteidigungsprofessor neigte seinen Kopf in leichter Ironie und schritt in dieselbe Richtung, in die Zabini geflohen war, davon, während Harry immer noch geschockt mit offenem Mund dastand.

\emph{Nachspiel: Harry Potter.}

Harry schleppte sich langsam in Richtung des Schlafsaals der Ravenclaw, wobei er weder die Wände, Gemälde noch andere Schüler sah; er ging Treppen hinauf und Rampen hinunter ohne zu verlangsamen, zu beschleunigen oder wahrzunehmen, wohin er trottete.

Es hatte ihn mehr als eine Minute nach Professors Quirrells Weggang gekostet, ehe ihm klar wurde, dass seine einzigen Informationsquellen über Dumbledores Beteiligung (a) Blaise Zabini, und er müsste ein absoluter Idiot sein um dem wieder zu vertrauen, und (b) Professor Quirrell waren, der mit Leichtigkeit eine Intrige in Dumbledores Stil hätte fälschen können und der vielleicht auch denken könnte, dass ein wenig Rivalität unter den Schülern eine feine Sache sei; und der, wenn man einen Schritt zurücktrat und die Details ausblendete, gerade vorgeschlagen hatte das Land in eine magische Diktatur zu verwandeln.

Und es wäre auch möglich, dass es doch Dumbledore \emph{war}, der hinter Zabini steckte, und dass Professor Quirrell ehrlich versucht hatte das Dunkle Mal in gleicher Weise zu bekämpfen und die Wiederholung eines Verhaltens verhindern wollte, dass er als erbärmlich ansah. Hatte versucht sicherzustellen, dass Harry am Ende nicht alleine im Kampf mit dem Dunklen Lord dastand, während alle anderen sich versteckten, verängstigt, beim Versuch aus der Schusslinie zu bleiben, darauf wartend, dass Harry sie rettete.

Aber die Wahrheit war…

Nunja…

Harry fand das irgendwie in Ordnung.

Es war, das wusste er, die Art von Dingen die Helden eigentlich gereizt und verbittert machen sollten.

Zum Teufel damit. Harry war sehr dafür, dass alle anderen sich aus der Gefahr heraus hielten während der Junge der lebte den Dunklen Lord im Alleingang besiegte, plusminus eine kleine Anzahl an Gefährten. Sollte der nächste Konflikt mit dem Dunklen Lord bis zum Punkt eines Zweiten Zaubererkrieges kommen, der eine Menge Leute das Leben kostete und ein ganzes Land erfasste, würde das heißen Harry hätte \emph{schon versagt}.

Und wenn danach ein Krieg zwischen Zauberern und Muggeln ausbrach, war es egal, wer gewann, Harry wäre bereits dadurch gescheitert, dass es soweit überhaupt kam. Ganz abgesehen davon, wer sagte denn die beiden Gesellschaften konnten nicht friedvoll integriert werden, wenn die Geheimhaltung unweigerlich zusammenbrach? (Auch wenn Harry dabei die trockene Stimme Professor Quirrells in seinem Kopf hören konnte, die ihn fragte, ob er ein Narr sei, und all die anderen offensichtlichen Dinge sagte…) Und wenn Zauberer und Muggel nicht in Frieden leben konnten, dann würde Harry eben Magie und Wissenschaft kombinieren und herausfinden, wie er alle Zauberer zum Mars oder sonstwohin evakuieren könnte, anstatt einen Krieg ausbrechen zu lassen.

Denn wenn es zu einem Vernichtungskrieg kam…

Das war die Sache die Professor Quirrell nicht verstanden hatte, die eine alles entscheidende Frage die er vergessen hatte seinem jungen General zu stellen.

Der wahre Grund, warum Harry nicht die Intention hatte sich zu einem Lichten Mal überreden zu lassen, egal \emph{wie} sehr es ihm im Kampf gegen den Dunklen Lord geholfen hätte.

Ein Dunkler Lord und fünfzig markierte Anhänger waren eine Gefahr für das gesamte magische Britannien.

Sollte ganz Britannien das Mal eines starken Anführers annehmen, wären sie eine Gefahr für die restliche magische Welt.

Und wenn die gesamte magische Welt eine einziges Mal annähme, wären sie eine Gefahr für die restliche Menschheit.

Niemand wusste so genau wie viele Zauberer es auf der Welt gab. Er hatte ein paar Schätzungen mit Hermine vorgenommen und war auf eine Zahl im ungefähren Bereich von einer Million gekommen.

Aber es gab sechs Milliarden Muggel.

Und sollte es zu einem finalen Krieg kommen…

Professor Quirrell hatte vergessen zu fragen welche Seite Harry beschützen würde.

Eine wissenschaftliche Gesellschaft, nach Außen gerichtet, den Blick gen Himmel gewandt, wissend, dass es ihr Schicksal war nach den Sternen zu greifen.

Und eine magische Gesellschaft, die langsam dahin schwand, während ihr Wissen verloren ging, immer noch von einem Adel regiert der Muggel nicht ganz als Menschen ansah.

Es war ein schrecklich trauriges Gefühl, aber keines, dass auch nur eine Spur an Zweifel ließ.

\emph{Nachspiel: Blaise} \emph{Zabini}.

Blaise schlenderte mit vorsichtiger, selbst auferlegter Langsamkeit durch die Gänge, sein Herz schlug wild während er versuchte sich zu beruhigen—

„Ähem“, sagte eine trockenem, flüsternde Stimme aus einer beschatteten Nische, als er vorüberging.

Blaise schreckte zurück, aber er schrie nicht.

Langsam drehte er sich um.

In dieser schmalen, schattigen Ecke stand ein schwarzer Umhang, so weit und wogend, dass es unmöglich war herauszufinden, ob die Gestalt darunter männlich oder weiblich war, und auf dem oberen Ende des Umhangs saß ein breitkrempiger schwarzer Hut und ein schwarzer Nebel schien sich darunter zu sammeln und das Gesicht von wem oder was auch immer darunter lag zu verschleiern.

„Berichte“, flüsterte Mr~Hut-und-Umhang.

„Ich habe genau gesagt, was du mir aufgetragen hast“, sagte Blaise. Seine Stimme war ein bisschen ruhiger geworden, jetzt da er niemanden mehr anlog. „Und Professor Quirrell hat genauso reagiert wie du es erwartet hast.“

Der breite, schwarze Hut neigte sich und richtete sich wieder auf, so als ob der Kopf darunter genickt hatte. „Ausgezeichnet“, sagte das unidentifizierbare Flüstern. „Die Belohnung, die ich dir versprochen habe, ist schon per Eule auf dem Weg zu deiner Mutter.“

Blaise zögerte, aber die Neugier fraß ihn innerlich auf. „Kann ich nun erfahren, warum du Scherereien zwischen Professor Quirrell und Dumbledore erzeugen wolltest?“ Der Schulleiter hatte, soweit Blaise es wusste, nichts mit den Gryffindor-Raufbolden zu tun und hatte Blaise, abgesehen davon Kimberly zu helfen, auch angeboten ihm von Professor Binns ausgezeichnete Noten in Geschichte der Zauberei geben zu lassen, selbst im Falle, dass er leere Pergamentbögen als Hausaufgaben abgeben würde, wenngleich er immer noch den Unterricht besuchen und so tun musste, als reiche er sie ein. Genau genommen hätte Blaise alle drei Generäle umsonst verraten und seine Cousine hätte ihn dabei auch nicht gekümmert, aber er hielt es nicht für nötig das zu sagen.

Der breite schwarze Hut richtete sich zu einer Seite hin auf, wie um eine fragenden Blick zu übermitteln. „Sag mir, Freund Blaise, kam es dir je in den Sinn, dass Verräter, die so häufig einen neuen Verrat begehen, oftmals ein schlimmes Ende finden?“

„Nein“, sagte Blaise und schaute direkt in den schwarzen Nebel unter dem Hut. „Jeder weiß, dass Hogwarts-Schülern nichts \emph{wirklich} Schlimmes ~widerfährt.“

Mr~Hut-und-Umhang gab ein leises Lachen von sich. „So ist es“, sagte die Flüsterstimme. „Mit dem Mord an einer Schülerin vor fünf Jahrzehnten als Ausnahme, die die Regel bestätigt, da Salazar Slytherin wohl seinem Monster eine höhere Position in den uralten Schutzzaubern der Schule zugewiesen hat, als der des Schulleiters selbst.“

Blaise starrte den schwarzen Nebel an und sich begann nun doch etwas unwohl zu fühlen. Aber es würde einen Hogwarts-Professor brauchen um ihm etwas Ernsthaftes zu tun, ohne dabei den Alarm auszulösen. Quirrell und Snape waren die einzigen Professoren, die so etwas überhaupt machen würden, und Quirrell hatte keinen Grund \emph{sich selbst} zu täuschen und Snape würde doch wohl keinem seiner eigenen Slytherins etwas tun… oder würde er?

„Nein, Freund Blaise“, flüsterte der schwarze Nebel, „ich wollte dir nur raten niemals etwas wie das hier in deinem späteren Leben zu tun. So viel Verrat würde mit Sicherheit zu mindestens einer Vergeltung führen.“

„An meiner \emph{Mutter} hat niemals jemand Vergeltung geübt“, sagte Blaise stolz. „Obwohl sie \emph{sieben} Ehemänner hatte und jeder einzelne von ihnen unter mysteriösen Umständen starb und ihr sehr viel Geld hinterließ.“

„Wirklich?“ sagte die Flüsterstimme. „Wie hat sie nur den Siebten überzeugt sie zu heiraten, nachdem er von dem Schicksal der ersten sechs gehört hatte?“

„Ich habe Mum das gefragt“, sagte Blaise, „und sie sagte, ich würde es erst erfahren wenn ich alt genug sei, und als ich fragte wie alt alt genug wäre und sie sagte, älter als sie.“

Wieder das leise Lachen. „Nun dann, Freund Blaise, meine Glückwünsche dafür in die Fußstapfen deiner Mutter getreten zu sein. Gehe jetzt und wenn du nichts hiervon erzählst, werden wir uns nicht wiedersehen.“

Blaise wich etwas verunsichert zurück und spürte einen seltsamen Widerwillen, dem Unbekannten den Rücken zuzudrehen.

Der Hut neigte sich. „Oh, komm schon, kleiner Slytherin. Wenn du wirklich Harry Potter oder Draco Malfoy ebenbürtig wärst, hättest du längst begriffen, dass meine angedeuteten Drohungen nur dazu dienten, deine Verschwiegenheit gegenüber Albus sicherzustellen. Hätte ich dir etwas antun wollen, hätte ich nichts angedeutet; hätte ich \emph{nichts} gesagt, dann wäre Besorgnis angebracht gewesen.“

Blaise richtet sich auf, fühlte sich ein wenig beleidigt und nickte zu Mr~Hut-und-Umhang; dann drehte er sich entschieden um und schritt seinem Treffen mit dem Schulleiter entgegen.

Bis zum allerletzten Moment hatte er gehofft, dass noch jemand \emph{anderes} auftauchen würde und ihm die Chance gab, Mr~Hut-und-Umhang an ihn zu verkaufen.

Aber Mum hatte ja auch nicht sieben verschiedene Männer zur \emph{selben Zeit} verraten. Wenn man es auf \emph{diese} Weise betrachtete, machte er sich immer noch besser als sie.

Und Blaise Zabini setzte seinen Weg zum Büro des Schulleiters fort, lächelnd in dem Wissen, ein Quintupel-Agent zu sein—

Für einem Moment geriet der Junge ins Straucheln, richtete sich dann aber wieder auf, das seltsame Gefühl der Orientierungslosigkeit abschüttelnd.

Und Blaise Zabini setzte seinen Weg zum Büro des Schulleiters fort, lächelnd in dem Wissen, ein Quadrupel-Agent zu sein.

\emph{Nachspiel: Hermine Granger.}

Der Bote näherte sich ihr erst, als sie alleine war.

Als Hermine gerade die Mädchentoilette verließ, in die sie sich manchmal zurückzog um nachzudenken, sprang eine hell leuchtende Katze aus dem Nichts und sagte: „Miss~Granger?“

Sie stieß einen kleinen Schrei aus bevor sie bemerkte, dass die Katze mit Professor McGonagalls Stimme gesprochen hatte.

Auch vorher war sie nicht verängstigt, nur überrascht gewesen; die Katze war hell und leuchtend und wunderschön, in einem silber-weißen Glanz wie mondfarbenes Sonnenlicht strahlend, und Hermine konnte sich nicht einmal vorstellen vor ihr Angst zu haben.

„Was bist du?“ sagte Hermine.

„Dies ist eine Botschaft von Professor McGonagall“, sagte die Katze, immer noch mit der Stimme der Professorin. „Könnten Sie bitte in mein Büro kommen und mit niemandem darüber sprechen?“

„Ich bin sofort da“, sagte Hermine, nach wie vor überrascht, und die Katze sprang auf und verschwand; nur, dass sie nicht wirklich verschwand, sie wanderte irgendwohin davon; oder das war was ihr Kopf ihr sagte, auch wenn ihre Augen die Katze lediglich verschwinden sahen.

Als Hermine im Büro ihres Lieblingsprofessors ankam, war ihr Verstand ein Wirbel von Spekulationen. Stimmte etwas nicht mit ihren Zensuren in Verwandlung? ~Aber warum hatte Professor McGonagall ihr dann aufgetragen es keinem zu erzählen? Vermutlich ging es um Harrys Übungen in seinen partiellen Verwandlungen…

Professor McGonagalls Gesicht sah besorgt aus, nicht streng, als Hermine sich vor ihrem Tisch setzte—sie versuchte dabei ihre Blicke von den geschachtelten Regalfächer mit Professor McGonagalls Hausarbeiten fernzuhalten; sie hatte sich schon immer gefragt welche Art von Arbeit Erwachsene zu erledigen hatten um eine Schule am Laufen zu halten und ob sie wohl Hilfe von ihr gebrauchen konnten…

„Miss~Granger“, sagte Professor McGonagall, „lassen Sie mich damit anfangen, dass ich bereits über die Bitte des Schulleiters, dass Sie diesen Wunsch äußern, Bescheid weiß —“

„Er hat es Ihnen \emph{erzählt}?“, platzte Hermine überrascht heraus. Der Schulleiter hatte doch gesagt niemand würde davon erfahren!

Professor McGonagall hielt inne, sah Hermine an und gab ein schwermütiges kleines Lachen von sich. „Es ist schön zu sehen, dass Mr~Potter Sie noch nicht zu sehr korrumpiert hat. Miss~Granger, Sie sollten etwas nicht nur deshalb \emph{zugeben} weil ich behaupte es zu wissen. Wie der Zufall es will hat der Schulleiter es mir \emph{nicht} erzählt, ich kenne ihn nur zu gut.“

Hermine wurde nun puterrot.

„Es ist in Ordnung, Miss~Granger!“, sagte Professor McGonagall hastig. „Sie sind eine Ravenclaw in Ihrem ersten Jahr, niemand erwartet von Ihnen eine Slytherin zu sein.“

Das traf nun \emph{wirklich}.

„Schön“, sagte Hermine ein wenig bitter. „Ich werde also zu Harry Potter gehen und ihn nach Slytherin-Nachhilfe fragen.“

„Das \emph{war nicht} was ich damit…“, sagte Professor McGonagall und ihre Stimme versagte. „Miss~Granger, ich mache mir Sorgen gerade weil junge Ravenclaw-Mädchen normalerweise keine Slytherins sein müssten! Wenn der Schulleiter Sie in etwas hineinziehen möchte bei dem Sie sich nicht wohl fühlen, Miss~Granger, ist es wirklich in Ordnung einfach Nein zu sagen. Und falls Sie sich unter Druck gesetzt fühlen, sagen Sie dem Schulleiter bitte, dass Sie sich meine Anwesenheit wünschen oder mich zuerst sprechen möchten.“

Hermines Augen weiteten sich. „Macht der Schulleiter Dinge die falsch sind?“

Professor McGonagall sah bei diesen Worten etwas betrübt aus. „Nicht mit Absicht, Miss~Granger, aber ich denke… nunja, es \emph{ist} vermutlich wahr, dass der Schulleiter manchmal Schwierigkeiten hat sich daran zu erinnern, wie es ist ein Kind zu sein. Selbst als er noch ein Kind war, bin ich mir sicher, muss er brillant gewesen sein und einen großen Verstand ebenso wie ein großes Herz gehabt haben, mit genügend Mut für drei Gryffindors. Manchmal verlangt der Schulleiter zu viel von seinen Schülern, Miss~Granger, oder gibt nicht genügend acht um sie nicht zu verletzen. Er ist ein guter Mann, aber manchmal kann sein Ränkespiel zu weit gehen.“

„Aber es ist \emph{gut} für die Schüler stark und mutig zu sein“, sagte Hermine. „Darum haben Sie doch Gryffindor für mich vorgeschlagen, oder nicht?“

Professor McGonagall lächelte schief. „Vielleicht war ich auch nur selbstsüchtig indem ich Sie für mein Haus wollte. Hat der sprechende Hut Ihnen—nein, ich hätte nicht fragen sollen.“

„Er hat mir gesagt ich könne überall außer nach Slytherin gehen“, sagte Hermine. Sie hätte \emph{beinahe} gefragt, warum sie nicht gut genug für Slytherin sei, bevor sie sich hatte aufhalten können… „Ich \emph{habe} also Mut, Professor!“

Professor McGonagall lehnte sich über ihren Tisch. Die Sorge stand ihr nun deutlicher im Gesicht. „Miss~Granger, es geht nicht um Mut, es geht darum was gut für junge Mädchen ist! Der Schulleiter verstrickt Sie in seine Intrigen, Harry Potter lässt Sie seine Geheimnisse hüten und auf einmal schmieden Sie Bündnisse mit Draco Malfoy! und ich habe Ihrer Mutter versprochen, dass es in Hogwarts sicher sein würde!“

Hermine wusste einfach nicht was sie dazu sagen sollte. Aber es kam ihr der Gedanke, dass Professor McGonagall sie vielleicht nicht gewarnt hätte, wenn sie ein Junge aus Gryffindor und keine Mädchen aus Ravenclaw wäre und \emph{das} war, nunja… „Ich werde versuchen gut zu sein“, sagte sie, „und ich werde mir von niemandem etwas anderes einreden lassen.“

Professor McGonagall hielt sich die Hände vor das Gesicht. Als sie sie wieder senkte, wirkte ihr von Falten durchzogenes Gesicht sehr alt. „Ja“, sagte sie kaum hörbar, „ Sie hätte sich in meinem Haus gut gemacht. Bleiben Sie in Sicherheit, Miss~Granger, und seien Sie vorsichtig. Und sollten Sie sich jemals über etwas Sorgen machen oder sich unwohl bei etwas fühlen, kommen Sie bitte umgehend zu mir. Ich werde Sie nicht länger aufhalten.“

\emph{Nachspiel, Draco Malfoy:}

Keiner von ihnen wollte an diesem Samstag etwas wirklich Kompliziertes machen, nicht nachdem sie eine Schlacht ausgefochten hatten. Also saß Draco einfach in einem ungenutzten Klassenzimmer und versuchte ein Buch namens \emph{Denksport Physik} zu lesen. Es war eine der faszinierendsten Sachen die Draco je in seinem Leben gelesen hatte, zumindest die Teile die er verstehen konnte, zumindest wenn der \emph{verfluchte Idiot}, der sich weigerte seine Bücher aus seinem Sichtfeld zu lassen, es hinbekommen würde \emph{die Klappe zu halten} und Draco sich \emph{konzentrieren} zu—

„Hermine Granger ist ein Schlaaambluuut!“, sang Harry Potter von seinem nahen Tisch, an dem er ein viel weitergehenderes Buch las.

„Ich weiß was du zu erreichen versuchst“, sagte Draco ruhig ohne von den Seiten aufzuschauen. „Es wird nicht funktionieren. Wir werden uns trotzdem zusammentun und dich zerquetschen.“

„Ein \emph{Maaalfoy} verbrüdert sich mit einem \emph{Schlaaambluuut}, was werden nur all die \emph{Freuuunde} deines Vaters denken —“

„Sie werden denken Malfoys sind nicht so leicht zu manipulieren wie \emph{du} es zu glauben scheinst, \emph{Potter}!“

Der Verteidigungsprofessor war verrückter als Dumbledore, kein zukünftiger Retter der Welt könnte jemals so \emph{kindisch} und \emph{würdelos} sein, in keinem Alter.

„Hey, Draco, weißt du was wirklich zum Kotzen sein wird? \emph{Du} weißt, dass Hermine Granger zwei Kopien des magischen Allels*** hat, genau wie du und ich, aber all deine Klassenkameraden in Slytherin wissen das nicht und \emph{duuu} darfst es ihnen nicht \emph{erklääären} —“

Dracos Finger wurden weiß, wo sie das Buch festhielten. Besiegt und ausgelacht zu werden konnte doch niemals so viel Selbstkontrolle bedürfen und wenn er es Harry nicht bald heimzahlen könnte, würde er sicherlich etwas tun was belastende Tatsachen schuf—

„Also was \emph{hast} du dir das erste Mal gewünscht?“, fragte Draco.

Harry sagte darauf nichts, sodass Draco von seinem Buch aufsah und einen Anflug von Heimtücke beim Anblick von Harrys betrübten Gesicht spürte.

„Ähm“, sagte Harry. „Eine Menge Leute hat mich das gefragt, aber ich glaube nicht, dass Professor Quirrell wollen würde, dass ich darüber spreche.“

Draco setzte ein ernstes Gesicht auf. „Du kannst darüber mit \emph{mir} sprechen. Im Vergleich mit den anderen Geheimnissen, die du mir erzählt hast, ist es bestimmt nicht wichtig und wofür sind Freunde denn sonst da?“

\emph{Ganz recht, ich bin dein Freund! Fühl dich schuldig!}

„Es war gar nicht so interessant“, sagte Harry mit offensichtlich aufgesetzter Leichtigkeit. „Es war nur , \emph{ich wünschte mir Professor Quirrell würde Kampfmagie auch im nächsten Jahr unterrichten}.“

Harry seufzte und schaute wieder auf sein Buch herab.

Und sagte, nachdem wieder ein paar Sekunden vergangen waren: „Dein Vater wird diese Weihnachten vermutlich ganz schön verärgert über dein Verhalten sein, aber wenn du ihm versprichst das Schlammblut-Mädchen zu verraten und ihre Armee plattzumachen, dann wird alles schon ~wieder werden und du bekommst trotzdem deine Weihnachtsgeschenke.“

Vielleicht würde Professor Quirrell es ihm und Granger, wenn sie ihn besonders respektvoll fragten und ein paar von ihren Quirrellpunkten einsetzten, erlauben dem Chaos-General etwas interessanteres anzutun als ihn nur in Schlaf zu versetzen.

* Argumentum ad hominem:Scheinargument

** Single Point of Failure: Systemkomponenten oder Systempfade werden als Single Point of Failure (SPoF) bezeichnet, wenn durch ihren Ausfall das Gesamtsystem nicht mehr betriebsbereit ist. Das trifft immer dann zu, wenn eine Komponente eine zentrale Funktion im Gesamtsystem übernimmt und beim Ausfall die Funktionen der anderen Komponenten beeinträchtigt. (Definition vom Bundesamt für Sicherheit in der Informationstechnik)

*** Allel: Eines von zwei einander entsprechenden Genen homologer Chromosomen - siehe auch Kapitel 22: Die Wissenschaftliche Methode

