

\hypertarget{selbstverwirklichung-teil-2}{% \section{35. Selbstverwirklichung, Teil 2}\label{selbstverwirklichung-teil-2}}

-\/-\/-\/-\/- Kapitel 67: Selbstverwirklichung, Teil 2 -\/-\/-\/-\/-

In den höheren Bereiche von Hogwarts, wo sich Räume und Korridore täglich veränderten, wo das Territorium selbst unsicher war und nicht nur die Karte, wo die Stabilität des Schlosses begann, in Träume und Chaos auszufransen, ohne dass sich sein architektonischer Stil oder seine scheinbare Solidität änderten - in diesen hohen Gefilden von Hogwarts würde bald eine Schlacht geschlagen werden.

Die Anwesenheit so vieler Schüler würde die Korridore für eine Zeit lang stabilisieren, kraft der ständigen Beobachtung. Die Räume und Gänge von Hogwarts \emph{bewegten} sich manchmal sogar, wenn man sie direkt ansah, aber sie würden sich nicht \emph{verändern}. Selbst nach acht Jahrhunderten war Hogwarts immer noch ein wenig schüchtern, sich vor den Augen der Menschen „umzuziehen“.

Aber trotz dieser vergänglichen Beständigkeit (so der Verteidigungsprofessor) hatten die oberen Bereiche von Hogwarts immer noch einen militärischen Realismus: Man musste das Gelände jedes Mal neu erlernen und jedes Kämmerchen immer wieder auf geheime Gänge überprüfen.

Es war Sonntag, der erste März. Professor Quirrell hatte sich soweit erholt, dass er wieder Schlachten beaufsichtigen konnte, und sie alle holten den Rückstand auf.

Der Drachengeneral, Draco Malfoy, betrachtete zwei Kompasse, die er in beiden Händen hielt. Der eine Kompass hatte die Farbe der Sonne, der andere einen mehrfarbigen, schillernden Glanz, der Chaos anzeigen sollte. Die anderen beiden Generäle, das wusste Draco, hatten ihre eigenen Kompasse bekommen; allerdings hielten Hermine Grangers und Harry Potters Hand einen Kompass, der orange-rot war und in seinen Reflexionen wie Feuer flackerte und immer in die Richtung des größten aktiven Kontingents der Drachenarmee zeigte.

Ohne diese Kompasse hätten sie vielleicht tagelang gesucht und sich nie gefunden, was eine Gefahr bei den Kämpfen in den oberen Etagen von Hogwarts darstellte.

Draco fürchtete den Moment, in dem die Drachenarmee die Chaoslegion finden würde. Harry Potter hatte sich verändert, seit Bellatrix Black geflohen war; der Erbe von Slytherin wirkte jetzt wirklich wie ein Anführer (und woher hatte Professor Quirrell gewusst, dass das passieren würde?) Draco hätte sich viel besser gefühlt, wenn Hermine Granger mit ihren dreiundzwanzig Sonnenschein-Soldaten im Schlepptau an seiner Seite gestanden hätte, aber nein, die Sonnenschein-Generalin war dumm und stolz und weigerte sich, Hilfe gegen General Potter anzunehmen. Sie wollte Potter selbst zu Fall bringen, hatte sie ihm gesagt.

Das edle und uralte Haus Malfoy hatte seinen Einfluss über Jahrhunderte hinweg aufrechterhalten, weil es verstanden hatte, dass man nicht immer der Mächtigste sein konnte. Manchmal war ein anderer Fürst einfach stärker, und man musste sich damit begnügen, sein oberster Leutnant zu sein. Man konnte sich, über ein Dutzend Generationen als rechte Hand, auf diese Weise eine Position von \emph{immensem} Reichtum und Macht aufbauen. Man musste nur jedes Mal aufpassen, dass das eigene Haus nicht mit dem Fall des Herrn, dem man diente, mit in den Abgrund gezogen wurde. Das war die Malfoy-Tradition, die sie in Jahrhunderten an Erfahrung perfektioniert hatten.

Und so hatte Vater Draco gründlich erklärt, dass, wenn er auf jemanden traf, der offensichtlich stärker war als er, Draco ihm das \emph{nicht} übel nehmen und es \emph{nicht} leugnen und \emph{keinen} Wutanfall bekommen sollte, der seine potenzielle Position sabotieren könnte, sondern Draco sollte dafür sorgen, dass sein Platz in der Machtstruktur der nächsten Generation \emph{nicht} niedriger als der zweite war.

Granger hatte diese Belehrung offenbar nie von ihren eigenen Eltern erhalten und leugnete immer noch die offensichtliche Tatsache, dass Harry Potter stärker wurde als sie.

Also hatte sich Draco heimlich mit Captain Goldstein und Captain Bones und Captain Macmillan getroffen und sie waren übereingekommen, alle ihr Bestes zu tun, um sicherzustellen, dass die Drachen und Sonnenschein sich nicht gegenseitig angreifen würden, bevor sie die größere Bedrohung durch Chaos nicht gelöst hätten.

Es war nicht \emph{wirklich} ein Verstoß gegen die Vereinbarung gegen Verräter, denn es konnte sich ja wohl kaum um Verrat handeln, wenn man der anderen Armee ehrlich \emph{helfen} wollte.

Ein hohes Läuten schallte durch die Korridore, um den Beginn der Schlacht zu signalisieren, und einen Moment später rief Draco „\emph{Los!}“ und die Drachen begannen zu rennen. Es würde seine Soldaten ermüden, es würde sie etwas kosten, auch nachdem sie angehalten und Luft geholt hatten, aber sie \emph{mussten} Chaos direkt zwischen sich und das Sonnenschein-Regiment bringen.

Harry und Neville gingen in gemächlichem Tempo durch die Korridore, Harry beobachtete den gelb-goldenen Kompass, der den Standort des Sonnenschein-Regiments anzeigte, und Neville hielt Ausschau, nur für den Fall, dass sie jemandem begegneten.

Ihre Schritte klangen ein wenig lauter, wenn man genau hinhörte.

„Also“, sagte der chaotische Leutnant nach einer Weile. „Deshalb hast du uns das Duellieren mit all dem angeschnallten Gewicht üben lassen?"

Harry nickte und behielt den Kompass im Auge, der zu Sonnenschein führte; wenn sich die scheinbare Richtung schnell zu ändern begann, waren sie nahe dran.

„Ich wollte nichts vor den anderen sagen, aber ein paar Wochen sind nicht viel Zeit, um zusätzliche Muskeln zuzulegen“, sagte Neville. „Und das Gewicht ist anders verteilt, und ich glaube, das wiegt eigentlich mehr, und zählt das nicht als Verwandlung eines Muggelartefakts?"

„Nö“, sagte Harry. „Das habe ich im Vorfeld überprüft. Man kann sie an den Statuen von Hogwarts sehen, also \emph{haben} einige Zauberer sie getragen, auch wenn es nur eine Modeerscheinung aus dem Mittelalter war.“ Und da niemand so etwas jemals ausprobieren würde, wenn er \emph{nicht} mit schwachen Zaubern wie dem Schlafzauber gegen Erstklässler kämpfte, zählte es auch nicht als das Preisgeben von guten Ideen.

Sie kamen an eine Kreuzung, von der zwei Korridore abbogen. Lästigerweise bog keiner der beiden Korridore in die richtige Richtung, um sie auf einem direkten Abfangkurs dorthin zu bringen, wo Sonnenschein hingehen würde, wenn sie der Chaoslegion folgen würden, die der Drachenarmee folgte. Also wählte Harry die scheinbar bessere der beiden Möglichkeiten, und Neville folgte ihm.

„Wir sollten einen Schalldämpfungszauber auf das Zeug anwenden, wenn wir nah dran sind“, sagte Neville. „Es ist ziemlich laut, sie könnten uns bemerken.„

Harry nickte und sagte dann: „Gute Idee“, falls Neville ihn nicht angeschaut hatte.

Sie stapften weiter durch den steingepflasterten Korridor des oberen Bereichs von Hogwarts, der von Fenstern aus einfachem Glas oder Buntglas erhellt wurde, ab und zu vorbei an Statuen von Hexen und Drachen und sogar dem einen oder anderen Zauberer-Ritter in Plattenrüstung oder Kettenhemd.

Die Sonnenschein-Soldaten schritten mit gezückten Zauberstäben durch einen langen, breiten Korridor. Sie konnten den Prismatischen Schild nicht benutzen, während sie manövrierten, aber Parvati Patil und Jenny Rustad hielten gerade \emph{Contego-}Schilde um die Offiziersgruppe aufrecht, die die ersten Ziele eines Hinterhalts sein würden.

Ihre Taktik für das nächste Gefecht, so hatten sie und ihre Offiziere beschlossen, würde darin bestehen, sich so schnell wie möglich direkt unter die feindlichen Soldaten zu mischen - nachdem sie \emph{miteinander} geübt hatten, wie sie sich gegenseitig unterstützen, vermeiden konnten sich gegenseitig zu treffen und in Positionen zu gelangen, in denen die feindlichen Soldaten zögern würden zu schießen. Sie hatten erst vier Stunden Training hinter sich, aber sie dachte, dass ihre Truppen bereits besser in dieser Art von gemischtem Kampf sein würden als Soldaten, die überhaupt nicht geübt hatten. Es schien die Art von Taktik zu sein, die Chaos benutzen würde, aber sie hatten sie bisher noch nicht benutzt.

Es war eine gute Strategie, glaubte sie. Und trotzdem, egal wie sehr sie ihre Soldaten belehrte, flüsterten sie weiterhin ängstliche Gerüchte über das, was Harry und Neville zu tun lernten. Schließlich war sie losgezogen und hatte mit Captain Goldstein gesprochen, der sich mit Dingen wie Truppenmoral auskannte, und Anthony hatte vorgeschlagen -

„Das ist seltsam“, meldete sich Captain Macmillan plötzlich zu Wort und betrachtete stirnrunzelnd die feurigen und schillernden Kompasse, die er in beiden Händen hielt. (Ernie war, wie Harry es genannt hätte, „gut in räumlicher Visualisierung“, und so war er dazu ausersehen worden, beide Kompasse zu halten und zu versuchen, herauszufinden, was ihre Feinde taten.) „Ich glaube... die Drachen bewegen sich nicht mehr so schnell... Ich glaube, sie sind zuerst auf die andere Seite von Chaos gelangt... und es sieht so aus, als ob Chaos sich bewegt, um sie anzugreifen, anstatt zu versuchen, sich aus der Zwickmühle heraus zu manövrieren?"

Hermine runzelte die Stirn und versuchte zu verstehen, und sie sah ein ähnliches Stirnrunzeln in den Gesichtern von Anthony und Ron. Wenn Chaos und Drachen direkt aufeinander losgingen und all ihre Kräfte gegeneinander einsetzten, bedeutete das praktisch, Sonnenschein den Kampf zu überlassen...

„Potter denkt, dass wir verbündet sind, also greift er Malfoy jetzt an, bevor die Drachen sich mit uns verbünden können“, sagte Blaise Zabini aus den Reihen der gemeinen Soldaten. „Oder Potter denkt einfach, dass er beide Armeen nacheinander schlagen kann, wenn er sie getrennt angreift.“ Der Slytherin-Junge stieß einen herablassenden Seufzer aus. „Werdet ihr mich jetzt wieder zum Offizier befördern? Ohne mich seid ihr aufgeschmissen, wisst ihr."

Alle ignorierten das Gerede, das aus Zabinis Mund kam.

„Bewegen wir uns immer noch in die richtige Richtung?“, fragte Anthony.

„Ja“, sagte Ernie.

„Nähern wir uns ihnen?“, fragte Ron.

"Noch nicht -"

Das war der Moment, in dem die riesigen schwarzen Holztüren am Ende des Korridors aufflogen und gegen die Wand krachten und zwei Gestalten enthüllten, die fast vollständig in graue Umhänge gehüllt waren, grauer Stoff, der über die Gesichter unter den grauen Kapuzen gespannt war, und eine dieser Gestalten hob bereits einen Zauberstab und richtete ihn direkt auf sie.

Und dann änderte sich alles, als Harrys Stimme, hoch und vom Marsch angestrengt, das Wort schrie:

"\emph{Stupefy!}"

Der Duellier-Betäubungszauber raste auf sie zu, sie war so geschockt, dass sie sich erst bewegte, als es schon fast zu spät war, denn der rote Lichtstrahl \emph{durchschlug} den \emph{Contego}-Schild an ihrer Vorderseite und sie konnte gerade noch ausweichen, es kribbelte an ihrem Arm, als das rote Licht an ihr vorbeiflog, und aus dem Augenwinkel sah sie, wie Susan getroffen und gegen Ron geschleudert wurde -

„\emph{Somnium!}“ brüllte Anthonys Stimme, einen Moment später gefolgt von einem Dutzend Stimmen, die ebenfalls „\emph{Somnium!}“ riefen.

Hermine sprang schnell auf, und als sie sich erhob, sah sie die beiden Gestalten in den grauen Umhängen einfach nur dastehen...

Man konnte den Schlafzauber kaum \emph{sehen}, er war zu schwach -

Aber auf keinen Fall konnten sie \emph{alle} ihr Ziel verfehlt haben.

„\emph{Stupefy!}“, kreischte die Stimme von Neville Longbottom, und ein weiterer roter Strahl schoss auf sie zu, sie fiel erneut zu Boden, als sie sich verzweifelt aus dem Weg drehte, und als sie keuchend aufstand, sah sie, dass der Betäubungsblitz diesmal Ron erwischt hatte, der gerade aufstehen wollte.

„Hallo, Sonnenschein“, sagte Harrys Stimme unter seiner Kapuze.

„Wir sind die Grauen Ritter des Chaos“, sagte Nevilles Stimme.

„Wir werden in dieser Schlacht eure Gegner sein“, sagte Harrys Stimme, „während die \emph{andere} Armee des Chaos die Drachen abschlachtet."

„Und übrigens“, sagte Nevilles Stimme, „wir sind unbesiegbar."

Die beiden Jungen in ihren grauen Umhängen und Roben, graues Tuch über den Gesichtern, standen Sonnenscheins gesamter Armee gegenüber, scheinbar unbeeindruckt von einem Dutzend Schlafzaubern.

Daphne hörte ein leises Seufzen neben sich, und als sie den Kopf drehte, sah sie, dass Hannahs Lippen geöffnet waren und die Augen des Hufflepuff-Mädchens riesig waren und starr auf etwas oder jemanden gerichtet waren.

Es wäre schwer gewesen, das Wirrwarr an Gedanken zu beschreiben, das Daphne durch den Kopf schoss, als sie bemerkte, dass Hannah Neville und nicht Harry anstarrte, was wiederum einen Teil von ihr dazu zu bringen schien, zu bemerken, dass Neville in letzter Zeit, was Jungs anging, ziemlich \emph{interessant} geworden war, in der Tat wirkte der letzte Spross der Longbottoms im Moment geradezu \emph{cool}, und irgendetwas erwachte in ihr, und ihre eigenen Lippen öffneten sich, und alles, was ihre Mutter ihr jemals über sittsames Benehmen und Schmeicheleien und duftendes Shampoo beigebracht hatte, flog ihr so schnell aus dem Kopf, dass die Haare neben ihren Ohren eigentlich weggepustet werden müssten; denn sie hatte Hermine und Harry beobachtet und wusste, wie sie ihre \emph{eigene} Romanze angehen wollte.

Außerdem hatte ihre Frau Mutter sie kürzlich in ein paar Zaubersprüchen unterwiesen, die nicht zu kennen peinlich sein konnte, wenn man dem edlen und uralten Haus Greengrass angehörte.

Daphnes Zauberstab schwang nach links und sie rief „\emph{Tonare!}"

Der Stab fuhr über ihren Kopf, und sie sprach die Beschwörungsformel „\emph{RavumCalvaria!}"

Schließlich nahm sie ihren Zauberstab in beide Hände und schrie: „\emph{Lucis} \emph{Gladius!}"

Die gewaltige magische Erschöpfung zwang sie fast in die Knie, aber sie ertrug sie, und als sich die flammende Gestalt vollständig gebildet und stabilisiert hatte, war der Strom der Magie wieder erträglich.

Trotzdem hatte sie das Gefühl, dass sie besser nicht lange versuchen sollte, damit zu kämpfen.

Dass alle sie \emph{anstarrten}, verstand sich von selbst, und sie \emph{hätte} nach vorne springen sollen, um Neville mit wehendem Haar zu konfrontieren, aber alles was sie tun konnte war,

stetig vorwärts zu gehen und ihre Uralte Klinge auf Neville Longbottom zu richten. Dass alle zur Seite gingen und ihr Platz machten, verstand sich ebenfalls von selbst.

„\emph{Ich bin Daphne, aus dem edlen und uralten Haus Greengrass!}“, rief sie. „\emph{Greengrass von Sonnenschein!}“ Die Duellformulierungen waren ihr völlig entfallen, sie hatte genug Theaterstücke gesehen, um sich an Herausforderungen zu einem Duell auf Leben und Tod oder für Blutrache zu erinnern, aber sie konnte sich überhaupt nicht daran erinnern, was hier angemessen war, also richtete sie einfach das glühende Schwert auf das Objekt ihrer Begierde und rief: „\emph{Zeig, was du drauf hast,} \emph{Nevy!}„

Wieder kreischte Harrys Stimme „\emph{Stupefy!}“, und später, als sie sich daran erinnerte, konnte sie nicht so recht glauben, dass sie es geschafft hatte, aber sie holte mit ihrer Lichtklinge aus, als wäre es der Schläger eines Treibers, und schlug den roten Lichtblitz zurück auf Harry, der es gerade noch schaffte, sich aus dem Weg zu drehen.

„\emph{Tonare!}“, rief Neville, aus dem edlen und uralten Haus Longbottom. „\emph{RavumCalvaria}, \emph{Lucis} \emph{Gladius!}"

Ein paar Sekunden lang tat niemand etwas anderes, als Neville und Daphne anzustarren, während sie anfingen, aufeinander einzuschlagen. Sie bewegten sich beide langsam, und Hermine vermutete, dass der Zauber viel Kraft aus ihnen herausholte. Es war nicht sehr beeindruckend im Vergleich dazu, was sie als Muggelstämmige in bestimmten Filmen gesehen hatte.

Aber man musste ihnen trotzdem zugutehalten, dass sie überhaupt Lichtschwerter benutzten.

„Kurze Auszeit“, sagte Harrys Stimme. „Ich weiß, dass der Verteidigungsprofessor zusieht, aber ich muss trotzdem fragen: Weiß jemand, ob sie sich gegenseitig in Stücke schneiden, wenn sie tatsächlich treffen -"

„Nein“, sagte Hermine abwesend. Das hatte in einem ihrer Geschichtsbücher gestanden, obwohl sie keine Ahnung gehabt hatte, dass das magische Duellschwert \emph{so} aussah. „Sie betäuben nur, wenn sie treffen."

"Du \emph{kennst} diesen Zauber?"

"Oh, nein, das ist der Zauber der Uralten Klinge, den nur edle und uralte Häuser benutzen dürfen -"

Hermine hörte auf zu sprechen und sah Harry an, oder vielmehr Harrys graue Kapuze.

„Nun“, sagte Harrys Stimme, „ich schätze, dann kann ich den Rest des Sonnenschein-Regiments auch allein erledigen.“ Sie konnte sein Gesicht nicht sehen, aber seine Stimme klang, als würde er lächeln.

„Du bist ausgewichen, als Daphne deinen eigenen Zauber zu dir zurückgeschlagen hat“, sagte Hermine. „Was immer du also getan hast, du bist \emph{nicht} unbesiegbar. Ein \emph{Stupefy} kann dich immer noch erwischen."

„Interessante Theorie“, sagte Harrys Stimme von unter der Kapuze. „Hast du jemanden in deiner Armee, der sie testen kann?"

„Ich habe einmal etwas über den Betäubungszauber gelesen“, sagte Hermine. „Vor ein paar Monaten. Ich frage mich, ob ich mich noch richtig an die Anleitung erinnern kann.“ Ihr Zauberstab hob sich und richtete sich auf Harry.

Es gab eine kleine Pause, als in der Nähe ein Junge und ein Mädchen mit hörbarem Schnaufen langsam mit Lichtschwertern aufeinander einschlugen.

„Natürlich“, sagte Harry und richtete seinen eigenen Zauberstab auf sie, „\emph{ich} kann einfach Somnium gegen dich einsetzen. Das wird viel weniger anstrengend werden."

Neue \emph{Contego}-Schilde entstanden vor ihr, gewirkt von Jenny und Parvati, noch während Harry sprach.

Die Spitze von Hermines eigenem Zauberstab begann, kleine Bewegungen in der Luft zu machen, eine Raute in einem Kreis, eine Raute in einem Kreis, wobei sie die Geste so einstudierte, dass sie genau dem entsprach, was sie in dem Buch gesehen hatte. Es würde selbst für sie ein schwieriges Kunststück sein, aber sie \emph{musste} den Zauber beim ersten Versuch richtig ausführen, sie konnte sich keine Fehlversuche leisten, die ihre Energie aufzehren würden.

„Weißt du“, sagte Hermine Granger, „ich verstehe, dass es nicht wirklich deine Schuld ist, aber ich habe es langsam satt, dass die Leute über den Jungen, der lebte reden, als wärst du - als wärst du eine Art \emph{Gott} oder so."

„Ich muss sagen, ich auch“, sagte Harry Potter. „Es ist traurig, wie die Leute mich immer wieder unterschätzen."

Ihr Zauberstab probte immer wieder den Diamanten innerhalb des Kreises, wieder und wieder. Harry würde seine eigenen Kräfte wieder aufladen, das wusste sie, auch wenn sie vor ihrem Angriff so viel wie möglich übte. „Ich glaube langsam, du brauchst einen Dämpfer, General Chaos."

„Da könntest du recht haben“, sagte Harry gleichmütig. Seine Füße begannen durch das zu schlurfen, was sie als Tanz eines Duellanten erkannte. „Leider gibt es jetzt nichts mehr, was mich besiegen kann, außer einem anderen Harry Potter."

"Lass es mich präziser ausdrücken, Mr. Potter. \emph{Ich} bringe dich auf den Boden der Tatsachen zurück."

"Du und welche andere Armee?"

„Du hältst dich für ziemlich cool, oder?“, sagte Hermine.

„Na klar“, sagte Harry. „Manche mögen das arrogant nennen, aber soll ich die letzte Person in Hogwarts sein, die merkt, wie toll ich bin?"

Hermine hob ihre linke Hand in die Luft und machte eine Faust.

Es war ein Signal. Acht designierte Soldaten aus ihrer Armee würden ihre Zauberstäbe auf sie richten und leise \emph{WingardiumLeviosa} wirken.

Auch \emph{das} hatten sie geübt, nachdem Hermine es aufgegeben hatte, ihre Soldaten zu belehren, und auf Anthonys Vorschlag hin versucht hatte, ihnen einen Sonnenschein-General zu geben, der so \emph{aussah}, als könne er unbesiegbare Gegner besiegen.

„Du tust so, als wärst du Superman“, sagte Hermine. Sie hob ihre linke Faust hoch in die Luft, und die acht Soldaten, die sie unterstützten, ließen sie vom Boden hochschweben. „\emph{Nun, hier ist Super-Hermine!}“ Ihre Hand schob sich nach vorne, und während sie schnell durch die Luft auf Harry zuschoss und nur bedauerte, dass sie seinen Gesichtsausdruck nicht sehen konnte, formte ihr Zauberstab eine Raute in einem Kreis, und sie beschwor all ihre Magie herauf. Es kam ihr vor als legte sie ihre Hand auf einen stromdurchflossenen Draht, als der zu mächtige Zauber durch sie hindurchfloss, während die das Wort „\emph{Stupefy!} „ rief.

Der rote Blitz schoss aus ihrem Zauberstab, perfekt geformt.

Harry wich ihm aus.

Und dann, weil sie diesen Teil nicht innerhalb von engen Korridoren geübt hatten, krachte sie gegen eine Wand.

"\emph{Somnium!} „, kreischte Draco, und dann, nach zu wenigen Sekunden, um sich wieder aufzuladen, „\emph{SOMNIUM, VERFLUCHT NOCHMAL!}"

Er \emph{wusste}, dass er Theodore traf, der andere Junge versuchte nicht einmal, auszuweichen, aber der Spross von Nott grinste nur so hämisch wie sein Vater und richtete seinen Zauberstab auf ihn -

Draco schaffte es gerade noch, zur Seite zu springen, als Theodores Zauber durch die Luft zischte. Aber Dracos Kräfte schwanden, so konnte er nicht weitermachen, Theodore machte sich gar nicht erst die Mühe, auszuweichen, während Draco sich ununterborchen bewegen musste; Das war \emph{verrückt}.

Er hatte jetzt genug Kraft, um wieder zu feuern, aber -

\emph{Dummheit ist, das Gleiche zu tun und ein anderes Ergebnis zu erwarten}, hatte Harry gesagt, und das hier war irgendwie \emph{Harrys} Werk, es konnte kein Muggelartefakt mehr sein, aber Draco konnte sich nicht zusammenreimen, was es sein könnte, und er sollte sich Hypothesen ausdenken und Wege, diese zu testen, aber er war zu sehr damit beschäftigt, verzweifelt auszuweichen, während Theodore lachte und einen weiteren Schlafzauber auf ihn schoss, Diesmal spürte Draco ein leichtes Taubheitsgefühl am linken Arm, während er sich wegdrehte, das war ein sehr, sehr knapper Fehlschuss gewesen und schließlich konnte Draco es nicht mehr ertragen, er machte sich nicht die Mühe, herauszufinden, welche Theorie er testete oder warum, als er einfach -

„\emph{Luminos}! „, rief Draco, und Theodore wurde in rotes Licht gehüllt, „\emph{Dulak}! „ und es erlosch wieder (Theodore \emph{wurde} also immer noch von der Magie beeinflusst), „\emph{Expelliarmus}! „ und Theodores Zauberstab flog (das war sowieso ein guter Zauber gewesen, jetzt, wo Draco darüber nachdachte), aber Theodore stürmte mit ausgestreckten Armen auf Draco zu, um ihn zu packen, also schrie Draco „\emph{Flipendo}! „und die Füße des anderen Jungen wurden abrupt nach oben gerissen -

- und Theodores Rücken schlug mit einem überraschend lauten und metallisch klingenden Krachen auf dem Boden auf.

Dracos Sicht verschwamm, weil er vier Zauber in so schneller Folge gewirkt hatte, und Theodore rappelte sich bereits auf, so dass er nicht einmal Zeit hatte, in Worten zu denken, aber Draco schaffte es trotzdem, ein „\emph{Somnium!}“ hervorzubringen und diesmal zielte er auf Theodores Gesicht statt auf seine Brust.

Theodore wich aus (er \emph{wich aus!}) und der Junge rief: „\emph{Code sieben auf Malfoy!}"

„\emph{Prismatis!} “, rief Padmas Stimme und plötzlich stand eine schimmernde Regenbogenwand vor Draco, gerade als vier chaotische Stimmen „\emph{Somnium!}“ riefen. „

Und es gab eine Pause, als alle auf die riesige prismatische Sphäre blickten, die die Überreste der Drachenarmee schützte.

Das Wirken des fünften Zaubers hatte Draco auf Hände und Knie geschickt, aber er sah auf und schaffte es, so deutlich wie möglich zu sagen: „Wenn die Schlafzauber nicht funktionieren - zielt auf das Gesicht - ich glaube, die Leutnants tragen Metallhemden."

„Ihr habt schon zu viele Soldaten verloren“, sagte Finnigan laut von der anderen Seite der Barriere, „wir werden euch trotzdem schlagen“, und dann lachte der Gryffindor-Junge böse. Er beherrschte das böse Lachen inzwischen fast so gut wie Harry Potter, und die anderen Chaos-Legionäre fingen bald darauf an, mitzulachen.

Draco konnte aus den Augenwinkeln Gregory und Vincent bewusstlos daliegen sehen. Padma hielt immer noch die Prismatische Barriere aufrecht, die größte, die er je bei ihr gesehen hatte; aber sie atmete schwer und war immer noch sichtlich verschwitzt, weil sie alle gejoggt waren, um in Position zu kommen - das Ravenclaw-Mädchen war eine starke Hexe, aber keine \emph{athletische}.

Er hoffte wirklich, dass General Granger bald ankommen und Chaos von hinten angreifen würde. General Potter und Neville von Chaos fehlten, und Draco konnte sich denken, wo sie hingegangen waren, aber zwei Soldaten, ganz allein, konnten das ganze Sonnenschein-Regiment nicht allzu lange aufhalten oder?

Sie wusste, dass es nicht fair war, dass das andere Mädchen alles gegeben hatte, was sie konnte, aber Hermine wünschte sich trotzdem, dass Daphne länger durchgehalten hätte.

„\emph{Lagann}! „, sagte Nevilles Stimme hinter ihr, während sie flog, und es gab das Geräusch einer zerberstenden Prismenwand, Hannahs Stimme rief verzweifelt „\emph{Somnium!} „ und dann, ein paar Augenblicke später, sagte Nevilles Stimme ruhig „\emph{Somnium}“ und sie hörte den dumpfen Aufprall eines weiteren ihrer Soldaten.

Und die Kraft, die sie in der Luft hielt, ließ wieder nach, Hermine konnte spüren, wie der Griff des Schwebezaubers an ihr zerrte, aber jetzt war es einfach nicht mehr genug.

Ihr Flug stoppte und sie begann in Zeitlupe in Richtung Boden zu fallen, und sie hätte ihren Soldaten signalisieren sollen, sie einfach fallen zu lassen, aber sie war zu wütend und verwirrt und dachte nicht schnell genug und versuchte immer noch, die Kraft für einen weiteren Betäubungszauber aufzubringen, und so konnte sie nirgendwo hin, als Harry seinen Zauberstab auf sie richtete und „\emph{Somnium}“ sagte, und das war das letzte Wort, das Hermine Granger von ihrem Kampf hörte.

