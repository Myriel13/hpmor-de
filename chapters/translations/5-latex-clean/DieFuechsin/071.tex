

\hypertarget{selbstverwirklichung-teil-6}{% \section{39. Selbstverwirklichung, Teil 6}\label{selbstverwirklichung-teil-6}}

-\/-\/-\/-\/- Kapitel 71: Selbstverwirklichung, Teil 6 -\/-\/-\/-\/-

"Na ja", flüsterte Daphne, wobei sie ihre Stimme so leise wie möglich hielt, "wenigstens fühle ich mich jetzt nicht mehr wie die einzige vernünftige Person in Hogwarts."

"Weil du jetzt den Rest von uns als Freunde hast?", flüsterte Lavender Brown, die auf Zehenspitzen an ihrer linken Seite schlich.

"Ich glaube nicht, dass sie das meint", murmelte General Granger auf Lavenders linker Seite.

Sie schlichen langsam und vorsichtig durch die Korridore von Hogwarts, alle acht hielten die Ohren nach dem leisesten Geräusch von Ärger offen, als wäre es eine Schlacht und sie suchten nach feindlichen Soldaten, die sie in einen Hinterhalt locken konnten; nur in diesem Fall suchten sie nach Mobbing, das es zu besiegen galt, und nach Opfern, die es zu retten galt, in der Zeitspanne zwischen dem Ende der Frühstückszeit und dem Zeitpunkt, an dem Lavender und Parvati zu ihrem Kräuterkurs gehen mussten.

Lavender hatte argumentiert, dass, wenn eine Erstklässlerin drei ältere Schläger besiegen konnte, müssten acht Erstklässlerinnen auf Grund von einfacher Multiplikation in der Lage sein, vierundzwanzig ältere Raufbolde zu besiegen.

Nach ihrem hektischen Geplapper und Händefuchteln zu urteilen, hatte General Granger das nicht überzeugend gefunden.

Padma hatte während des darauffolgenden Streits eine Weile geschwiegen und dann nachdenklich bemerkt, dass es selbst in Hogwarts wahrscheinlich nicht gut für einen Ruf als Rowdies wäre, Erstklässlerinnen zu verprügeln.

Parvati hatte sich daraufhin aufgerichtet und ausgerufen, dass dies bedeutete, dass sie die \emph{Einzigen} waren, die etwas gegen das Mobbingproblem in Hogwarts tun konnten, was es erst \emph{wirklich heroisch} machte. Außerdem war der \emph{einzige} \emph{Grund}, warum ihre Eltern nach Großbritannien gezogen waren, der, dass die beiden die einzige magische Schule der Welt mit einer Todesrate von 0 \% besuchen konnten, und was sollte das bringen, wenn sie nicht den Vorteil nutzen und ein paar Dinge ausprobieren würden?

Worauf General Granger geantwortet hatte, dass Parvati den Sinn einer perfekten Sicherheitsbilanz \emph{überhaupt nicht} verstand -

Lavender hatte gesagt, wenn sie \emph{wirklich} alle Freunde seien und nicht Hermines Gefolgsleute, wie Professor Quirrell dachte, dann sollten sie über solche Dinge abstimmen.

Daphne hatte erwartet, dass ihre Stimme den Ausschlag geben würde, nachdem Hermine und Susan und Hannah mit Nein gestimmt hatten. Und so hatte Daphne es sich gut überlegt, nachdem ihre erste Begeisterung abgeklungen war. Sie war schließlich eine \emph{Slytherin}, und das bedeutete, dass es \emph{ihre} Aufgabe war, ein wachsames Auge auf ihre eigenen Interessen zu haben, während sie alle herumliefen und versuchten, Leuten zu helfen - ihre Aufgabe, herauszufinden, wie riskant es wirklich war und ob es sich für sie \emph{lohnen} würde, genau wie Mutter es an ihrer Stelle getan hätte. Immer auf sich selbst und ihre Freunde aufzupassen, das war es, was eine echte Slytherin ausmachte…

Hannah Abbott, das nervöse kleine Hufflepuff-Mädchen, hatte mit leiser, zitternder Stimme "Ja" gesagt.

Und nun \emph{mussten} Daphne und Susan und Hermine bei den anderen fünf bleiben, sie konnten die anderen \emph{unmöglich} allein losziehen lassen. Denn kein Gryffindor würde es je verkraften, das letzte überlebende Kind der Bones-Familie zu verletzen, und kein Slytherin würde es wagen, eine Tochter des edlen und uralten Hauses Greengrass anzugreifen. (Das \emph{hoffte} Daphne jedenfalls.) Und General Granger, die die ganze Sache angefangen hatte … da brauchte man gar nicht erst zu fragen.

Die Gänge von Hogwarts zogen an ihnen vorbei, einer nach dem anderen, die angespannten Hände nie weit von ihren Zauberstäben entfernt, als Stein und Holz und Ewigbrennende Fackeln in Sicht kamen und dann vorbeizogen. An einem Punkt hörten sie Schritte und holten tief Luft, ihre Hände hätten fast ihre Zauberstäbe fallen gelassen, aber es war nur ein einzelner älterer Ravenclaw, der sie neugierig ansah, bevor er die Stirn runzelte und sich wieder in sein Buch vertiefte, während er weiterging.

Die Heldinnen schlichen an feierlichen, mit vergoldeten Fresken verzierten Eichentafeln vorbei und kamen zu einer Sackgasse, die zu einem Badezimmer der Jungs führte, und drehten um und wanderten \emph{zurück} durch die feierlichen, mit vergoldeten Fresken verzierten Eichentafeln und bogen dann durch staubige, alte, mit abgenutztem Zement verfugte Ziegelkorridore ab, die sie im Kreis herumführten, also konsultierten sie ein Porträt und gingen dann stattdessen einen \emph{anderen} staubigen, alten Ziegelkorridor hinunter, der sie zu einer kurzen Marmortreppe führte, die sie eigentlich in den dreieinhalbten Stock hätte bringen müssen, wenn sie irgendwo anders als in Hogwarts gewesen wären, und dann ging es wieder zurück zu Steinpflaster und Oberlichtern, durch die Sonnenstrahlen fielen, obwohl sie sich nicht einmal in der Nähe des Daches befanden, und nachdem sie diesem Gang um ein paar Ecken gefolgt waren, führte er sie zu einer weiteren Jungen-Toilette, die deutlich mit einem Schild gekennzeichnet war, das die Silhouette einer in Umhänge gehüllten Gestalt zeigte, die in eine Toilette pinkelte.

Zu acht standen da und starrten mit einer gewissen Frustration auf die geschlossene Tür.

"Mir ist langweilig", sagte Lavender.

Padma machte sich ein Spaß daraus, mit einer dramatischen Geste eine Taschenuhr aus ihrem Umhang zu holen und sie zu studieren. "Sechzehn Minuten und dreißig Sekunden", sagte sie. "Ein neuer Rekord für die längste Aufmerksamkeitsspanne in Gryffindor."

"\emph{Ich} glaube auch nicht, dass das klappen wird", sagte Susan. "Und ich bin eine \emph{Hufflepuff}."

"Wisst ihr", sagte Lavender nachdenklich, "ich frage mich, ob vielleicht der springende Punkt bei einem Helden ist, dass bei ihnen \emph{tatsächlich} etwas Interessantes \emph{passiert}, wenn sie so etwas versuchen."

"Ich wette, du hast recht", sagte Tracey. "Ich wette, wenn wir \emph{Harry Potter} dabei hätten, würden wir in den ersten fünf Minuten auf drei Prügeleien und einen versteckten Raum voller Schätze stoßen. Ich wette, dass General Chaos nur auf die Toilette gehen muss, und schon findet er die Kammer des Schreckens von Slytherin oder so -"

Das konnte Daphne nicht so stehen lassen. "Meinst du, Lord Slytherin hätte den Eingang zur Kammer des Schreckens in ein \emph{Badezimmer} gelegt -"

"Was ich \emph{sagen will}", sagte Susan, als Tracey den Mund zu einer Antwort öffnete, "ist, dass wir keine Möglichkeit haben, Schläger zu \emph{finden}. Ich meine, alles, was \emph{die} tun müssen, ist, irgendwo einen Hufflepuff zu finden, aber wir müssen ihnen genau zur richtigen \emph{Zeit} über den Weg laufen, versteht ihr? Was in Wahrheit eigentlich ein \emph{Vorteil} ist, denn wenn wir sie finden \emph{würden}, würden sie uns zerquetschen wie Käfer. Können wir uns nicht einfach am verbotenen Korridor im dritten Stock versuchen, so wie es gedacht ist?"

Lavender schnaubte verächtlich. "Man wird keine \emph{echte} Heldin, wenn man nur die verbotenen Dinge tut, die einem der Schulleiter \emph{sagt}!"

(Daphnes Verstand versuchte noch, diese Aussage zu verarbeiten, während sie im Stillen dem Sprechenden Hut dafür dankte, dass er sie nicht einmal in die Nähe von Gryffindor gesteckt hatte.)

"Wenn ich so darüber nachdenke …", sagte Parvati langsam: "Ich meine, wie groß ist die Wahrscheinlichkeit, dass Harry Potter an seinem \emph{ersten Schultag} auf diese fünf Schläger traf? Er muss einen Trick benutzt haben, um sie zu finden."

Daphne stand zufällig so, dass sie Parvati und Hermine sehen konnte, und so bemerkte sie, wie sich der Gesichtsausdruck des Ravenclaw-Mädchens veränderte - und dann wurde ihr klar, dass \emph{auch} die Sonnenschein-Generalin erst vor kurzem rechtzeitig auf ein solchen Angriff gestoßen war -

"Oh!", rief Padma in dem Tonfall der plötzlichen Erkenntnis. "Natürlich! Der Geist von Salazar Slytherin hat es ihm verraten!"

"\emph{Was?}", sagte Daphne gleichzeitig mit mehreren anderen Leuten.

"Das war der Geist, der mich erschreckt hat, da bin ich mir ziemlich sicher", erklärte Padma. "Ich meine, ich habe es erst im Nachhinein herausgefunden, aber … ja. Salazar Slytherins Geist mag es nicht, wenn Slytherins Leute schikanieren, er denkt, dass es seinem Namen Schande macht, und der Geist ist immer noch in die Hogwarts-Schutzzauber eingeklinkt, also weiß er alles, was passiert, da wette ich."

Daphne blieb der Mund offenstehen, und sie sah, dass Hannah sich eine Hand an die Stirn gelegt hatte und sich gegen die Steinmauern lehnte, während Traceys Augen wie kleine braune Sterne leuchteten.

\emph{Der Geist von Salazar Slytherin?}

Hatte sich mit \emph{Harry Potter} verbündet?

Und hatte \emph{Hermine Granger} geschickt, um Derricks Gang aufzuhalten?

Sie hätte 100 Galleonen bezahlt, um dabei zu sein, wenn Draco Malfoy davon erfuhr.

Obwohl, wenn man bedachte, wie schnell sich Gerüchte in Hogwarts verbreiten, hatte Millicent, jetzt, wo Padma es ausgeplaudert hatte, es ihm wahrscheinlich schon vor dreißig Minuten gesagt…

In der Tat… jetzt, wo Daphne darüber \emph{nachdachte}…

"Also", sagte Parvati. "Wir müssen also den Jungen, der lebte fragen, wo Salazar Slytherins Geist zu finden ist? Wow, ich schätze, wenn ich so etwas laut auspreche, werde ich vielleicht tatsächlich zu einer Heldin -"

"Ja!", sagte Lavender. "Wir müssen den Jungen, der lebte fragen, wo Salazar Slytherins Geist zu finden ist!"

"Wir müssen… den Jungen, der lebte… fragen, wo der Geist von Salazar Slytherin zu finden ist…", wiederholte Hannah mit nervöser Stimme, als würde sie sich zwingen, es zu sagen.

"Und wenn \emph{das} nicht klappt", rief Tracey, "dann betäuben wir Harry Potter, fesseln ihn und nehmen ihn mit uns!"

Es sagte etwas aus, dachte Hermine Granger, und es war schon ziemlich verstörend - während die acht zurück durch das Labyrinth der verwinkelten kleinen Gänge schlenderten aus denen Hogwarts bestand (nachdem ihre Zeit vor der nächsten Klasse abgelaufen war, ohne irgendwelches Mobbing zu finden) - dass sie wirklich nicht wusste, ob Harry Potter vom Geist von Salazar Slytherin oder einem Phönix oder \emph{was auch immer} geführt worden war. Und was auch immer Harry getan hatte, sie hoffte, dass es bei ihnen \emph{nicht} funktionierte. Und vor allem hoffte sie, dass die anderen nicht für Traceys Idee stimmten, Harry Potter zu betäuben und seinen bewusstlosen Körper mit sich herumzuschleppen, um Abenteuer anzulocken. Das konnte im wirklichen Leben unmöglich funktionieren, und wenn doch, wollte sie sowieso aufgeben.

Hermine schaute von Hexe zu Hexe, Tracey plauderte mit Lavender, die anderen machten gelegentliche Bemerkungen, und dann blieb ihr Blick an dem zurückhaltenden Mädchen hängen, der einzigen Person, deren Gedanken sie im Moment überhaupt nicht erraten konnte.

"Hannah?", sagte sie zu dem Mädchen, das neben ihr lief. Hermine versuchte, ihre Stimme so sanft wie möglich klingen zu lassen. "Du musst nicht antworten, aber ist es in Ordnung, wenn ich frage, warum du für den Kampf gegen Mobbing gestimmt hast?"

Hermine hatte gedacht, sie hätte ihre Stimme sanft und unauffällig gehalten, aber alle schauten auf, und Lavender und Tracey unterbrachen ihr Gespräch und sahen sie an.

Hannahs Wangen röteten sich bereits, und gerade als Hannah den Mund öffnete…

"Es liegt daran, dass sie offensichtlich mehr Mut hat, als \emph{du} denkst", sagte Lavender.

Hannah hielt mit offenem Mund inne.

Sie schloss ihren Mund.

Sie schluckte, hart und sichtbar, während sich ihre Wangen noch mehr röteten.

Dann holte Hannah tief Luft und sagte mit leiser Stimme: "Es gibt da einen Jungen, den ich mag."

Das Hufflepuff-Mädchen zuckte zusammen, als sie das sagte, und ihr Kopf drehte sich nervös hin und her um alle anzusehen, während sich die Pause dehnte.

"Ähm, okay?" sagte Susan schließlich.

"Ich habe \emph{fünf} Jungs, die ich mag", sagte Lavender.

"Padma und ich wussten, dass wir beide die gleichen Jungs mögen", sagte Parvati, "also haben wir eine Liste gemacht und einen Knut geworfen, um zu sehen, wer zuerst wählen darf."

"Ich weiß, wen ich \emph{heiraten} werde", sagte Tracey. "Es ist mir egal, was die Welt sagt, er ist dazu bestimmt, mir zu gehören!"

Das ließ alle anderen Mädchen erwartungsvoll zu Hermine blicken, deren Gehirn Traceys letzte Aussage komplett verworfen hatte, um sich nur auf das erste, was Hannah gesagt hatte, konzentrieren zu können.

"Ähm", sagte Hermine. Sie fuhr vorsichtig mit sanfter Stimme fort: "Hannah, der Grund, warum du dem Bündnis engagierter, lengendärer, fähiger, Erstklässler-Retterinnen beigetreten bist, war, dass es einen Jungen gibt, der dich vielleicht mehr mag, wenn du eine Heldin wirst?"

Das Hufflepuff-Mädchen nickte wieder, ihre Wangen röteten sich noch mehr, während sie auf ihr eigenes Spiegelbild in ihren schwarz polierten Schuhen starrte.

"Es handelt sich dabei übrigens um Neville Longbottom", sagte Daphne. Die Slytherin stieß einen wehmütigen Seufzer aus. "Und unglücklicherweise für sie wird er eine andere heiraten. Es ist alles sehr tragisch."

Das erzeugte einen hohen, \emph{piepsigen} Laut von Hannah, die weiterhin auf ihre Füße starrte.

"Wie bitte?", fragte Lavender. "Neville wird eine andere heiraten? Woher weißt du davon? \emph{Wen} denn? "

Daphne schüttelte nur traurig und mit niedergeschlagener Miene den Kopf.

"\emph{Entschuldigung}", sagte Hermine, und dann, als die anderen sie wieder ansahen, "Ah…", während sie versuchte, ihre Gedanken zu ordnen. "Ich meine, ähm … Hannah … zu versuchen, eine Heldin zu werden, damit ein Junge dich mag, ist nicht sehr \emph{feministisch}."

"Das wird doch \emph{feminin} ausgesprochen", sagte Padma.

"Und warum nennst du Hannah unfeminin?", fragte Susan. "Es ist doch sehr feminin, einen Jungen beeindrucken zu wollen."

"Außerdem", sagte Parvati und klang verwirrt, "geht es nicht genau darum? Dass wir versuchen, Helden zu sein, obwohl das nichtfeminin ist?"

Die folgende Diskussion würde Hermine Granger nicht als einer ihrer erfolgreichsten Streifzüge in die Gefilde der politischen Bildung in Erinnerung bleiben. Sie versuchte es zu erklären, und nach der darauffolgenden Diskussion versuchte sie es noch einmal, während die anderen sieben Mädchen sie immer skeptischer ansahen. Danach erklärte Daphne im herrischen Tonfall der zukünftigen Lady Greengrass, dass, wenn diese Sache mit dem Feminismus bedeute, dass es Mädchen nicht erlaubt sei, Jungen nachzustellen, wie auch immer es ihnen gefiele, dann könne der Feminismus in den Muggelländern bleiben, wo er hingehöre. Lavender schlug vor, dass Hexismus vielleicht besagen könnte, dass Hexen alles tun dürfen, was sie wollen, was sich nach mehr Spaß anhörte als Feminismus. Und schließlich beendete Padma jede weitere Diskussion, indem sie müde feststellte, dass sie keinen Sinn darin sah, weiter zu streiten, da es bei BELFER gar nicht um \emph{irgendetwas} ging, was mit Feminismus zu tun hatte, sondern nur darum, dass mehr Mädchen zu Heldinnen wurden.

Hermine hatte an diesem Punkt aufgegeben.

Als die Zauberkunststunde an diesem Tag zu Ende war und die Ravenclaws des ersten Jahrgangs aus der Klasse schlurften, fürchte Hermine schon das darauffolgende Geschehen. Sie hatten es gerade noch vor dem Gong in die Klasse geschafft und so mussten sie sofort zu ihren Tischen rennen und sich hinsetzen, so dass \emph{noch} keine Zeit für die schreckliche Sache gewesen war; aber das bedeutete nur, dass Hermine sich auf die kommende Katastrophe während des \emph{gesamten} Unterrichts freuen konnte.

Nachdem Professor Flitwick sie entlassen hatte und sich alle von ihren Stühlen erhoben hatten, begann Harry auf sie zuzugehen; und Hermine schob ihrerseits ihr Buch in ihren Eselsfellbeutel und ging sehr schnell zur Tür hinüber, warf sie auf und hastete einen Gang hinunter, und natürlich folgte Harry ihr mit einem überraschten Blick, weil sie eine Bibliotheksstunde angesetzt hatten -

"Hermine?" sagte Harry, als er die Tür hinter sich schloss. "Was ist los?"

Die Tür flog hinter Harry auf, nicht einen Moment nachdem er sie geschlossen hatte, sodass Harry aus dem Weg springen musste, und Padma Patil trat aus dem Klassenzimmer mit einem furchtbar entschlossenen Blick auf ihrem Gesicht.

"Entschuldigen Sie, Mr. Potter", kamen die schrecklichen Worte, die hohe Stimme des jungen Mädchens hallte durch den Korridor wie die düsteren Glocken des Untergangs, "kann ich Sie etwas fragen? Ich bei etwas deine Hilfe."

Harrys hob seine Augenbrauen, und er sagte: "Du kannst natürlich \emph{fragen}."

"Kannst du uns sagen, wie wir mit dem Geist von Salazar Slytherin sprechen können? Wir wollen, dass er uns sagt, wie wir Schläger finden können, so wie er es dir sagt."

Im Korridor vor dem Klassenzimmer herrschte ein wenig Stille.

Die Tür öffnete sich wieder, und Su spähte mit einem fragenden Blick hinaus.

"Nun, wir müssen zur Bibliothek", sagte Harry ganz beiläufig, sein Gesicht sah entspannt aus, "würde es dir etwas ausmachen, uns zu folgen?" und begann in die Richtung zu gehen, die an ungeraden Tagen des Monats zur Bibliothek führte, und Su sah so aus, als wolle sie ihnen folgen, aber Harry warf ihr einen Blick zu.

Erst als Harry um eine Ecke gebogen war, zog er seinen Zauberstab, sagte mit leiser, präziser Stimme "\emph{Quietus}" und wandte sich dann an Padma und sagte: "Eine interessante Vermutung, Miss Patil."

Padma sah daraufhin ziemlich selbstgefällig aus; und sagte: „Ich hätte es eigentlich schon \emph{früher} herausfinden müssen. Da war dieses \emph{Zischen} in der Stimme des Geistes, ich hätte sofort “Parselmund" denken müssen, noch bevor er anfing, über Godric Gryffindor zu reden."

Harrys Gesicht veränderte sich nicht. "Darf ich fragen, Miss Patil, ob Sie diesen Gedanken mit -"

"Sie hat es vor allen in BELFER. gesagt", sagte Hermine.

Harrys starrte mit diesem Blick ins Leere, der bedeutete, dass er sehr schnell etwas berechnete, und dann sagte er: "Hermine, wie groß ist die Chance, dass -"

"Sie hat es vor Lavender \emph{und} Tracey gesagt."

"Ähm", sagte Padma. "Hätte ich das nicht tun sollen?"

"Warte hier", knurrte Mr. Goyle und ging um die Ecke; und dann ertönte das Klopfen an Draco Malfoys Privatraum.

Tracey hatte ein mulmiges Gefühl im Magen, und sie erinnerte sich wieder daran, dass, da Padma alles ausgeplaudert hatte, \emph{jemand} es Draco Malfoy sagen musste, und das konnte genauso gut \emph{sie} sein, und es war ja nicht so, dass sie Harry Potter etwas \emph{schuldete}, und eine Slytherin musste tun, was nötig war, um ihre Ambitionen zu erfüllen.

Seit Professor Quirrell sie derart abserviert hatte, sammelte sie Ambitionen, und bis jetzt hatte sie beschlossen, dass sie ihren eigenen Nimbus 2000-Besen besitzen, superberühmt werden, Harry Potter heiraten, jeden Tag Schokofrösche zum Frühstück essen und mindestens \emph{drei} Dunkle Lords besiegen wollte, nur um Professor Quirrell zu zeigen, wer hier gewöhnlich war.

"Mr. Malfoy wird dich empfangen", sagte die tiefe, bedrohliche Stimme von Mr. Goyle, als er zurückkam. "Und es wäre besser für dich, wenn er nicht denkt, dass du seine Zeit verschwendest." Der Junge warf ihr einen kurzen Blick zu und trat dann zur Seite.

Tracey fügte ihrer Liste der Ambitionen, eigene Diener zu haben, hinzu und trat ein.

Das private Schlafzimmer der Malfoys sah genauso aus wie das von Daphne. Insgeheim hatte sie auf diamantene Kronleuchter oder goldene Fresken an den Wänden gehofft - sie hätte es nie vor Daphne gesagt, aber das Haus Malfoy \emph{war} eine Stufe höher als das von Greengrass. Aber es war nur ein kleines Schlafzimmer wie das von Daphne, und der einzige Unterschied war, dass die Sachen von Malfoy mit silbernen Schlangen statt mit Smaragdpflanzen verziert waren.

Als sie durch die Tür trat, erhob sich Draco Malfoy - der selbst in seinem Schlafzimmer perfekt frisiert war - von seinem Schreibtischstuhl, um sie mit einer kleinen freundlichen Verbeugung zu begrüßen, wobei er ein charmantes Lächeln aufsetzte, als wäre sie jemand, der etwas \emph{bedeutete}, was Tracey so nervös machte, dass sie alles vergaß, was sie in ihrem Kopf einstudiert hatte, und einfach herausplatzte: "Ich muss dir etwas sagen!"

"Ja, das hat Gregory schon erwähnt", sagte Draco Malfoy sanft. "Bitte, Miss Davis, setzen Sie sich." Er gestikulierte zu \emph{seinem eigenen Schreibtischstuhl}, während er sich selbst auf sein Bett setzte.

Sie fühlte sich etwas benommen, als sie sich vorsichtig auf Malfoys eigenen Stuhl setzte, wobei ihre Finger gedankenlos an den Falten ihres Umhangs herumfummelten, und sie versuchte, sie so elegant und nicht zerknittert aussehen zu lassen wie die von Draco Malfoy -

"Also, Miss Davis", sagte Draco Malfoy. "Was wollten Sie mir sagen?"

Tracey zögerte, und als Malfoys sichtbar ungeduldig wurde, stammelte sie einfach alles heraus, alles, was Padma darüber gesagt hatte, dass Salazar Slytherins Geist Harry Potter geschickt hatte, um Mobbing zu verhindern, und auch, was Daphne ihr darüber erzählt hatte, dass Hermine Granger ebenfalls involviert war -

Draco Malfoys Gesichtsausdruck änderte sich überhaupt nicht, als sie sprach, nicht im Geringsten, und es dämmerte Tracey mit einem üblen Kribbeln im Magen.

\emph{"Du glaubst mir nicht!", sagte sie.}

Es entstand eine kleine Pause.

"Nun", sagte Draco Malfoy mit einem Lächeln, das nicht ganz so charmant war wie sein letztes, "ich \emph{glaube}, dass es das ist, was Padma gesagt hat und was Daphne gesagt hat, also danke trotzdem, Miss Davis." Der Junge erhob sich von dort, wo er auf dem Bett gesessen hatte, und Tracey erhob sich, ohne zu überlegen, vom Stuhl.

Als er sie zur Tür begleitete und gerade den Knauf drehen wollte, fiel Tracey noch etwas ein - "Du hast nicht gefragt, was ich für die Information wollte", sagte sie.

Draco Malfoy warf ihr eine Art Blick zu, denn sie nicht wirklich einordnen konnte, und er sagte nichts.

"Na ja, jedenfalls", sagte Tracey und änderte kurzerhand ihre bisherigen Pläne, "möchte ich \emph{nichts} für die Information, ich wollte einfach nur freundlich sein."

Ein kurzer Ausdruck der Überraschung wanderte für einen Augenblick über Draco Malfoys Gesicht, bevor sich seine Miene wieder neutralisierte. "Es ist nicht so einfach, sich mit einem Malfoy anzufreunden, Miss Davis." erwiderte er.

Tracey lächelte und meinte es auch so. "Nun, dann werde ich eben weiterhin freundlich sein", sagte sie und verließ den Raum mit leichtem Schritt und fühlte sich vielleicht zum ersten Mal in ihrem Leben wie eine echte Slytherin. Außerdem hatte sie gerade beschlossen, dass Draco Malfoy auch einer ihrer Ehemänner sein würde.

Nachdem das Mädchen gegangen war, kam Gregory herein, schloss die Tür wieder und sagte: "Geht es Ihnen gut, Mr. Malfoy?"

Draco sagte nichts zu seinem Diener und Freund. Seine Augen starrten ins Nichts, als wollte er durch die Wand seines Schlafzimmers starren, durch den Hogwarts-See, der die Slytherin-Kerker umgab, durch die Erdkruste und die Atmosphäre und den interstellaren Staub der Milchstraße, in die völlig leere und lichtlose Leere zwischen den Galaxien, die kein Zauberer und kein Wissenschaftler je gesehen hatte.

"Mr. Malfoy?" sagte Gregory und klang langsam ein wenig besorgt.

"Ich kann nicht glauben, dass ich jedes Wort davon geglaubt habe", sagte Draco.

Daphne beendete den letzten Zoll ihres Verwandlungsaufsatzes und ließ ihren Blick über den Slytherin-Gemeinschaftsraum gleiten, wo Millicent Bulstrode immer noch an ihren eigenen Hausaufgaben arbeitete. Es war an der Zeit, eine Entscheidung zu treffen.

Wenn BELFER herumging und versuchte Brutalos außer Gefecht zu setzen, würde das den Brutalos gar nicht gefallen, das war sicher. Und sie würden versuchen, etwas Unangenehmes dagegen zu tun, das war auch sicher. Andererseits, wenn die Schläger wirklich böse wurden, dann könnte Hermine Harry Potter um Hilfe bitten, oder sie könnten ihre kombinierten Quirrell-Punkte zusammenlegen und den Verteidigungsprofessor um einen Gefallen bitten… Nein, die Sache, über die sich Daphne \emph{wirklich} Sorgen machte, war, ob diese Sache ihnen Ärger mit Professor Snape einbringen würde. Man wollte \emph{niemals} auf der falschen Seite von Professor Snape landen.

Aber seit dem Tag, an dem sie Neville zu einem Uralten Duell herausgefordert hatte, hatte sie bemerkt, dass die Leute sie anders ansahen. Sogar die Slytherins, die sich über sie lustig gemacht hatten, sahen sie jetzt anders an. Es dämmerte Daphne, dass es \emph{viel} mehr Respekt einbrachte, die Tochter des noblen und uralten Hauses Greengrass zu sein, wenn man eine schöne \emph{Heldin} war, die aus einem uralten Haus stammte, und nicht nur ein hübsches adliges Mädchen. Es war der Unterschied zwischen einer Rolle, die von der Hauptdarstellerin gespielt wurde, und einer Rolle, die von einer zwei Galeonen Statistin mit einem kreischenden Lachen gespielt wurde.

Schläger zu bekämpfen war vielleicht nicht der \emph{beste} Weg, eine Heldin zu werden. Aber Vater hatte ihr einmal gesagt, dass das Problem mit dem Verpassen von Gelegenheiten darin bestand, dass es zur Gewohnheit werden konnte. Wenn man sich sagte, dass man dieses Mal noch auf eine bessere Gelegenheit wartete, würde man sich beim nächsten Mal wahrscheinlich genauso entschieden. Vater hatte gesagt, dass die meisten Menschen ihr ganzes \emph{Leben} damit verbrachten, auf eine Gelegenheit zu warten, die gut genug war, und dann verstarben. Vater hatte gesagt, dass das Ergreifen von Gelegenheiten zwar bedeuten würde, dass alle möglichen Dinge \emph{schief gehen} würden, aber es war nicht annähernd so schlimm, wie ein hoffnungsloser Fall zu sein, der nie etwas versucht hatte. Vater hatte gesagt, \emph{nachdem} sie sich angewöhnt habe Gelegenheiten zu ergreifen, \emph{dann} sei es an der Zeit wählerisch zu werden.

Andererseits hatte Mutter sie gewarnt, nicht alle Ratschläge Vaters zu befolgen, und gesagt, dass Daphne nicht nach Vaters sechstem Jahr in Hogwarts fragen dürfe, bis sie mindestens dreißig Jahre alt sei.

Aber am Ende \emph{hatte} Vater Mutter dazu gebracht, ihn zu heiraten, und sich erfolgreich einen Platz in einem der ältesten Häuser erschlichen, ganz falsch \emph{konnte} er also nicht liegen.

Millicent Bulstrode beendete ihre Hausaufgaben und begann, ihre Sachen wegzuräumen.

Daphne stand von ihrem Schreibtisch auf und ging hinüber.

Millicent schwang ihre Beine vom Tisch und stand auf, wobei sie ihre Büchertasche über eine Schulter warf, dann sah erstaunt Daphne näherkommen.

"Hey, Millicent", sagte Daphne, als sie an sie herantrat, und ihre Stimme klang leise und aufgeregt, "rate mal, was ich heute herausgefunden habe?"

"Die Sache mit dem Geist von Salazar Slytherin, der Granger hilft?", fragte Millicent. "Davon habe ich schon gehört -"

"Nein", sagte Daphne im gedämpften Flüsterton, "das ist sogar noch \emph{besser}."

"Wirklich?" Millicent sagte mit ebenso leiser, aufgeregter Stimme. "Was ist es?"

Daphne blickte sich verschwörerisch um. "Komm mit in mein Zimmer, dann erzähle ich es dir."

Sie gingen in Richtung der Treppe, die nach unten führte; die Privaträume lagen noch tiefer unter dem See als die Schlafsäle der Siebtklässler…

Schon bald saß Daphne in ihrem bequemen Schreibtischstuhl und Millicent war auf ihre Bettkante gehüpft.

"\emph{Quietus}", sagte Daphne, als sie beide saßen; und dann, anstatt ihren Zauberstab in ihren Umhängen zu verstauen, ließ Daphne ihre Hand einfach ganz natürlich an ihre Seite fallen, und behielt den Zauberstab dabei immer noch in der Hand, nur für den Fall.

"\emph{Also gut}! ", sagte Millicent. "Was ist \emph{los}?"

"Weißt du, was ich herausgefunden habe?", sagte Daphne. "Ich habe herausgefunden, dass du den Klatsch und Tratsch so schnell mitbekommst, dass du über Dinge Bescheid weißt, \emph{bevor sie tatsächlich passieren}."

Daphne hatte halb damit gerechnet, dass Millicent weiß werden und umfallen würde und das tat sie nicht wirklich. Das Mädchen zuckte ziemlich heftig zusammen, bevor sie anfing Ausflüchte zu stammeln.

"Keine Sorge", sagte Daphne mit ihrem süßesten Lächeln, "ich werde niemandem sonst erzählen, dass du eine Seherin bist. Ich meine, wir sind doch Freunde, oder?"

Rianne Felthorne, siebte Klasse in Slytherin, arbeitete fleißig an einem weiteren Aufsatz (sie belegte alle Fächer außer Wahrsagerei und Muggelkunde und ihr UTZ-Jahr schien nur aus Hausaufgaben zu bestehen), als ihr Hauslehrer auf den Tisch zuging, an dem sie gerade arbeitete, "Sie kommen mit mir, Miss Felthorne!" bellte, und davon ging, während sie noch verzweifelt begann, ihr Pergament, ihr Buch und ihren Federkiel wegzulegen.

Als sie Professor Snape einholte, wartete er vor dem Raum und starrte sie mit zusammengekniffenen Augen an, die viel zu intensiv zu sein schienen; und bevor sie fragen konnte, worum es ging, drehte er sich wortlos um und pirschte durch die Gänge, so dass sie sich anstrengen musste, um Schritt zu halten.

Ihr Weg führte sie eine Treppe hinunter und dann eine weitere, unter die unterste Ebene der Slytherin-Verließe. Und die Korridore begannen, älter auszusehen, die Architektur war um Jahrhunderte zurück und bestand aus aufgerautem Stein, der von grobschlächtigem Mörtel zusammengehalten wurde. Sie begann sich zu fragen, ob Professor Snape sie in die echten Kerker führte, von denen sie Gerüchte gehört hatte, die wahren Kerker von Hogwarts, die für alle außer dem Lehrkörper verschlossen waren; und ob Professor Snape dort unten vielleicht schreckliche Dinge mit unschuldigen, hilflosen jungen Mädchen tat; Aber das war wahrscheinlich nur Wunschdenken ihrerseits.

Sie gingen eine weitere Treppe hinunter und kamen in einen Raum, der gar kein Raum war, sondern eine leere Felsenhöhle mit einer einzigen Tür, durchbohrt von vielen dunklen Öffnungen und erleuchtet von einer einzigen Fackel im antiken Stil, die brannte, als sie eintraten.

Professor Snape zückte seinen Zauberstab und begann, einen Zauber nach dem anderen zu wirken, sie wusste nicht mehr, wie viele es waren. Als der Meister der Zaubertränke fertig war, drehte er sich wieder zu ihr um, sah sie mit seinen intensiven Augen an und sagte mit ruhiger Stimme, die so gar nicht zu seinem üblichen Tonfall passte: "Sie werden niemandem etwas von dieser Angelegenheit erzählen, Miss Felthorne, weder jetzt noch in Zukunft. Wenn das für Sie akzeptabel ist, nicken Sie. Wenn nicht, werden wir uns umdrehen und jetzt gehen."

Sie nickte, verängstigt und mit einer seltsamen Hoffnung, die in ihrem Herzen aufkeimte (na ja, nicht ganz in ihrem Herz).

"Die Aufgabe, die ich für Sie habe, ist sehr einfach, Miss Felthorne", sagte Professor Snapes tonlose Stimme, "und Ihr äußerst großzügiger Lohn von fünfzig Galleonen ist lediglich eine Entschädigung dafür, dass Sie danach mit einem Gedächtniszauber belegt werden, um alles wieder zu vergessen."

Sie holte unwillkürlich Luft. Ihre Familie mochte zwar reich sein, aber sie hatten noch andere Töchter und hielten sie an der kurzen Leine, und für \emph{sie} war das sicherlich eine Menge Geld.

Dann registrierte ihr Verstand das Wort \emph{Gedächtniszauber}, und sie war empört! Es hatte keinen Sinn, wenn sie die Erinnerungen nicht behalten konnte, für was für ein Mädchen \emph{hielt} Professor Snape sie?

"Sie wissen doch sicher", sagte Severus Snape, "von Miss Hermine Granger, dem Sonnenschein-General?"

"\emph{Was?}„, sagte Rianne Felthorne in plötzlichem Entsetzen und Abscheu. “Aber sie ist doch \emph{Erstklässlerin! Igitt!}"

