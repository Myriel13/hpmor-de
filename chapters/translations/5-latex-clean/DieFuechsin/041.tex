

\hypertarget{frontale-uxfcberbruxfcckung}{% \section{9. Frontale Überbrückung}\label{frontale-uxfcberbruxfcckung}}

-\/-\/-\/-\/- Kapitel 41: Frontale Überbrückung -\/-\/-\/-\/-

Der beißende Januarwind heulte um die riesigen, leeren Steinmauern, die die materiellen Grenzen der Burg Hogwarts absteckten, und flüsterte und pfiff in seltsamen Tonhöhen, als er an geschlossenen Fenstern und Steintürmen vorbeiflog. Der jüngste Schnee war größtenteils weggeweht, aber gelegentlich klebten noch Flecken von geschmolzenem und wieder gefrorenem Eis an der Steinwand und glänzten mit reflektiertem Sonnenlicht. Aus der Ferne muss es so ausgesehen haben, als würde Hogwarts mit Hunderten von Augen blinzeln.

Eine plötzliche Böe ließ Draco zusammenzucken und versuchen, unmöglicherweise, seinen Körper noch näher an den Stein zu drücken, der sich wie Eis anfühlte und nach Eis roch. Ein völlig sinnloser Instinkt schien davon überzeugt zu sein, dass er kurz davorstand, von der Außenwand von Hogwarts geblasen zu werden, und dass der beste Weg, dies zu verhindern, darin bestand, in hilflosem Reflex zu zucken und sich möglicherweise zu übergeben.

Draco versuchte sehr hart, \emph{nicht} an die sechs Stockwerke leere Luft unter ihm zu denken und sich stattdessen darauf zu konzentrieren, wie er Harry Potter töten würde.

"Wissen Sie, Mr. Malfoy", sagte das junge Mädchen neben ihm mit gesprächiger Stimme, "wenn mir ein Seher gesagt hätte, dass ich eines Tages mit den Fingerspitzen an der Seite eines Schlosses hängen würde, während ich versuchen würde nicht nach unten zu schauen oder daran zu denken würde, wie laut Mama schreien würde, wenn sie mich sehen würde, hätte ich \emph{keine} Ahnung gehabt, wie es passieren würde, \emph{außer} dass es Harry Potter's Schuld wäre."

\emph{Früher}:

Die beiden verbündeten Generäle traten zusammen über Longbottoms Körper, ihre Stiefel schlugen fast perfekt synchron auf den Boden.

Nur noch ein einziger Soldat stand zwischen ihnen und Harry, ein Slytherin-Junge namens Samuel Clamons, dessen Hand sich weiß um seinen Zauberstab klammerte, nach oben gehalten, um seine prismatische Wand aufrechtzuerhalten. Die Atmung des Jungen kam schnell, aber sein Gesicht zeigte die gleiche kalte Entschlossenheit, die auch die Augen seines Generals Harry Potter erhellte, der hinter der Prismenwand am Ende des Ganges neben einem offenen Fenster stand, und seine Hände mysteriös hinter seinem Rücken hielt.

Die Schlacht war lächerlich schwierig gewesen, denn der Feind war zwei zu eins in der Unterzahl. Es hätte einfach sein sollen, Drachenarmee und Sonnenscheinregiment hatten sich in Trainingssitzungen leicht vereint, sie hatten sich lange genug bekämpft, um sich gegenseitig sehr gut zu kennen. Die Moral war hoch, beide Armeen wussten, dass sie diesmal nicht nur für sich selbst, sondern auch für eine Welt ohne Verräter kämpften. Trotz der überraschten Proteste beider Generäle hatten die Soldaten der kombinierten Armee darauf bestanden, sich selbst Dramiones Sonnchen Argiment zu nennen, und Abzeichen für ihre Insignien zu produzieren mit einem lächelnden Gesicht, das in Flammen gehüllt war.

Aber Harrys Soldaten hatten alle ihre eigenen Abzeichen geschwärzt - es sah nicht aus wie Farbe, eher als hätten sie diesen Teil ihrer Uniformen \emph{verbrannt} - und sie hatten sich mit einer verzweifelten Wut durch die oberen Ebenen von Hogwarts gekämpft. Die kalte Wut, die Draco manchmal in Harry sah, schien auf seine Soldaten übertragen zu haben und sie hatten gekämpft, als wäre es kein Spiel gewesen. Und Harry hatte seine ganze Trickkiste geleert, es gab winzige Metallkugeln (Granger hatte sie als "Kugellager" identifiziert) auf Böden und Treppen, die sie unpassierbar machten, bis sie geräumt waren. Nur Harrys Armee hatte bereits koordinierte Schwebezauber geübt und sie konnten ihre eigenen Leute direkt über die Hindernisse fliegen, die sie gemacht hatten…

Man konnte keine Geräte von außen ins Spiel bringen, aber du konntest alles, was du wolltest, während des Spiels verwandeln, solange es sicher war. Und das war einfach nicht fair, wenn du gegen einen Jungen kämpfst, der von Wissenschaftlern aufgezogen wurde, die sich mit Dingen wie Kugellagern, Skateboards und Bungee-Seilen auskannten.

Und so war es dann hierzu gekommen.

Die Überlebenden der alliierten Streitkräfte hatten die letzten Überreste von Harry Potters Armee in eine Sackgasse getrieben.

Weasley und Vincent hatten sich zur gleichen Zeit auf Longbottom gestürzt und sich simultan bewegt, als ob sie es wochenlang statt stundenlang geübt hätten, und irgendwie hatte es Longbottom geschafft, sie \emph{beide} zu verzaubern, bevor er selbst fiel.

Und jetzt waren es Draco und Granger und Padma und Samuel und Harry, und nach dem Aussehen von Samuel konnte der seine prismatische Wand nicht mehr lange aufrechthalten.

Draco hatte seinen Zauberstab bereits auf Harry ausgerichtet und darauf gewartet, dass die Prismenwand von selbst fiel; es bestand keine Notwendigkeit, vorher einen Brechenden Bohrzauber zu verschwenden. Padma richtete ihren eigenen Zauberstab auf Samuel, Granger richtete ihren auf Harry …

Harry verbarg immer noch seine Hände hinter seinem Rücken, anstatt mit seinem Zauberstab zu zielen; und sah sie mit einem Gesicht an, das aus Eis gehauen hätte werden können.

Es könnte ein Bluff sein. Wahrscheinlich war es das nicht.

Es herrschte eine kurze, angespannte Stille.

Und dann sprach Harry.

"Ich bin jetzt der Bösewicht", sagte der kleine Junge kalt, "und wenn ihr denkt, dass Bösewichte so einfach zu erledigen sind, solltet ihr besser noch einmal nachdenken. Schlagt mich, wenn ich ernsthaft kämpfe, und ich werde geschlagen bleiben; aber verliert, und wir werden das das nächste Mal wieder tun."

Der Junge brachte seine Hände nach vorne, und Draco sah, dass Harry seltsame Handschuhe trug, mit einem eigentümlichen grauen Material an den Fingerspitzen und Schnallen, die die Handschuhe fest an seine Handgelenke hefteten.

Neben Draco keuchte der Sunshine General entsetzt; und Draco, ohne auch nur zu fragen, warum, feuerte einen Brechenden Bohrzauber ab.

Samuel taumelte, er ließ einen Schrei los, als er taumelte, aber er hielt die Wand; und wenn Padma oder Granger jetzt feuerten, würden sie ihre eigenen Kräfte so sehr erschöpfen, dass sie einfach verlieren könnten.

"Harry! "rief Granger. "Das kann nicht dein Ernst sein! "\\ Harry war bereits in Bewegung.\\ Und als er aus dem offenen Fenster schwang, sagte seine kalte Stimme: "Folgt mir, wenn ihr euch traut."

Der eisige Wind heulte um sie herum.\\ Dracos Arme fingen bereits an, sich müde anzufühlen.

… Es hatte sich herausgestellt, dass Harry Granger gestern sorgfältig demonstriert hatte, wie man die Handschuhe, die er gerade trug, mit so genannten "Gecko-Sets" transfigurieren konnte und wie man transfigurierte Stellen aus dem gleichen Material auf die Zehen ihrer Schuhe klebte; und Harry und Granger hatten in einem unschuldigen kindlichen Spiel versucht, ein wenig an den Wänden und der Decke herumzuklettern.

Und dass Harry Granger auch gestern noch insgesamt genau zwei Dosen Feder-Falltrank zur Verfügung gestellt hatte, die sie in ihrem Beutel "für alle Fälle" mit sich herumtragen konnte.\\ Nicht, dass Padma ihnen gefolgt wäre. Sie war nicht verrückt.

Draco löste vorsichtig seine rechte Hand, streckte sie so weit wie möglich nach vorne und schlug sie wieder auf den Stein. Neben ihm tat Granger das Gleiche.\\ Sie hatten bereits den Federfalltrank geschluckt. Es war eine Grauzone in den Spielregeln, aber der Trank wurde nicht aktiviert, solange keiner von ihnen tatsächlich fiel, und solange sie \emph{nicht} fielen, benutzten sie den Gegenstand nicht.\\ Professor Quirrell beobachtete sie.\\ Die beiden waren \emph{perfekt, komplett, absolut sicher}.

Harry Potter hingegen würde sterben.

"Ich frage mich, warum Harry das tut", sagte General Granger in einem reflektierenden Ton, als sie langsam die Fingerspitzen einer Hand mit einem ausgedehnten, klebrigen Geräusch von der Wand schälte. Ihre Hand fiel fast sofort nach dem Anheben wieder nach unten. "Ich muss ihn das fragen, nachdem ich ihn getötet habe."\\ Es war erstaunlich, wie viel sich herausstellte, dass die beiden gemeinsam hatten.\\ Draco hatte im Moment nicht wirklich Lust zu reden, aber er schaffte es, durch knirschende Zähne zu sagen: "Könnte Rache sein. Für das Date."\\ "Wirklich", sagte Granger. "Nach all der Zeit."\\ Stick. Plop.\\ "Wie süß von ihm", sagte Granger.

Stick. Plop.\\ "Ich schätze, ich werde einen wirklich romantischen Weg finden, um ihm zu danken", sagte Granger.\\ Stick. Plop.\\ "Was hat er gegen \emph{dich}? " sagte Granger.\\ Stick. Plop.\\ Der eisige Wind heulte um sie herum.

Man könnte denken, dass es sicherer wäre, wieder Boden unter den Füßen zu haben.\\ Aber wenn dieser Boden ein Schrägdach war, das mit groben Latten gedeckt war, die mit viel mehr Eis bedeckt waren als die Steinmauern, und du mit hoher Geschwindigkeit darüber läufst…\\ Dann würdest du dich \emph{leider} \emph{irren}.

"Luminos! " rief Draco.\\ "Luminos! "rief Granger.\\ "Luminos! " rief Draco.\\ "Luminos! "rief Granger.

Die entfernte Gestalt wich aus und krabbelte, während sie lief, und nicht ein einziger Schuss traf, aber sie kamen näher.\\ Bis Granger ausrutschte.\\ Im Nachhinein war es unvermeidlich, im wirklichen Leben konnte man mit hoher Geschwindigkeit nicht \emph{wirklich} über eisige Dächer laufen.\\ Und auch unvermeidlich, weil es ohne den geringsten Gedanken geschah, drehte sich Draco und griff nach Grangers rechtem Arm, und er \emph{erwischte} sie, nur war sie schon zu weit aus der Balance, sie fiel und zog Draco mit sich, alles geschah so schnell --

Es gab einen harten, schmerzhaften Aufprall, nicht nur Dracos Gewicht, das auf das Dach fiel, sondern auch ein Teil von Grangers Gewicht, und wenn sie nur ein wenig näher an den Rand gefallen wäre, hätten sie es geschafft, aber stattdessen kippte ihr Körper wieder und ihre Beine rutschten ab und ihre andere Hand packte hektisch…

Und so hielt Draco schließlich Grangers Arm so fest, dass sich seine Hand weiß verfärbte, während sich ihre andere Hand verzweifelt am Rand des Daches festklammerte und die Zehen von Dracos Schuhen gruben sich in den Rand eines Dachziegels.

"\emph{Hermine}! " Harrys Stimme kreischte entfernt.\\ "Draco", flüsterte Grangers Stimme, und Draco sah nach unten.\\ Das könnte ein Fehler gewesen sein. Es war viel Luft unter ihr, nichts als Luft, sie standen am Rande eines Daches, das aus der Hauptsteinmauer von Hogwarts herausragte.\\ "Er wird kommen und mir helfen", flüsterte das Mädchen, "aber zuerst wird er uns beiden einen \emph{Luminos} verpassen, auf keinen Fall würde er das nicht tun. Du musst mich gehen lassen."

Es hätte die einfachste Sache der Welt sein sollen.\\ Sie war nur ein Schlammblut, nur ein Schlammblut, \emph{nur ein Schlammblut}!\\ Sie wäre nicht einmal \emph{verletzt}!\\ … Dracos Gehirn hörte nichts, was Draco gerade sagte.

"Tu es", flüsterte Hermine Granger, ihre Augen loderten ohne eine Spur von Angst, "tu es, Draco, tu es, du kannst ihn allein schlagen, \emph{wir müssen gewinnen Draco}! "\\ Das Geräusch von jemandem der rannte war hörbar und es kam näher.

\emph{Oh, sei vernünftig}…

Die Stimme in Dracos Kopf klang sehr nach Harry Potter, der Unterricht gab.\\ …\emph{wirst du dein Gehirn dein Leben leiten lassen?}

\emph{Nachspiel, 1:}

Daphne Greengrass musste sich anstrengen um sich ruhig zu verhalten, als Millicent Bulstrode die Geschichte im Gemeinschaftsraum der Slytherin-Mädchen weitererzählte (ein gemütlicher, kühler Ort in den Kerkern unter dem Hogwarts-See, mit Fischen, die an jedem Fenster vorbei schwimmen, und Sofas, auf denen man sich hinlegen kann, wenn man will). Vor allem, weil es nach Ansicht von Daphne schon ohne alle \emph{Verbesserungen} von Millicent eine perfekte Geschichte war.

„Und dann was?" keuchten Flora und Hestia Carrow.

„General Granger sah zu ihm auf", sagte Millicent dramatisch, „und sie sagte: Draco! Du musst mich loslassen! Mach dir keine Sorgen um mich, Draco, ich verspreche, dass es mir gut gehen wird! Und was glaubst du, was Malfoy dann getan hat?"\\ „Er sagte: „Niemals!"“, rief Charlotte Wiland, „und hielt sie noch fester!"\\ Alle zuhörenden Mädchen außer Pansy Parkinson nickten.\\ "Nein!" sagte Millicent. "Er ließ sie los. Und dann sprang er auf und erschoss General Potter. Das Ende."

Es gab eine betäubte Pause.

"Das kannst du nicht \emph{machen}!", sagte Charlotte.\\ "Sie ist ein \emph{Schlammblut}", sagte Pansy und klang verwirrt. "\emph{Natürlich} hat er losgelassen!"

"Nun, dann hätte Malfoy sie nicht erst packen sollen!" sagte Charlotte. "Aber als er sie einmal festgehalten hatte, \emph{musste} er durchhalten! \emph{Besonders} angesichts des nahenden Untergangs!" Tracey Davis, die neben Daphne saß, nickte in Übereinstimmung.\\ "Ich sehe nicht, warum", sagte Pansy.\\ "Das liegt daran, dass du nicht den kleinsten Hauch von Romantik in dir hast", sagte Tracey. "Außerdem kannst du nicht einfach Mädchen loslassen. Ein Junge, der ein Mädchen so fallen lässt…. er würde \emph{jeden} fallen lassen. Er würde \emph{dich} fallen lassen, Pansy."

"Was meinst du damit, \emph{mich fallen lassen}? " sagte Pansy.\\ Daphne konnte nicht mehr widerstehen. "Weißt du", sagte Daphne dunkel, "du frühstückst eines Tages an unserem Tisch, und das nächste, was du weißt, Malfoy \emph{lässt dich los}, und du fällst von der Spitze von Hogwarts! Darum geht es!"

"Ja!" sagte Charlotte. "Er ist ein Hexen-fallen-lasser!"

"Weißt du, warum Atlantis fiel?", sagte Tracey. " Weil jemand wie Malfoy es \emph{fallen gelassen} hat, deshalb!"\\ Daphne senkte ihre Stimme. "Tatsächlich… was ist, wenn Malfoy derjenige war, der Hermine, ich meine General Granger, überhaupt erst hat ausrutschten lassen? Was, wenn er darauf aus ist, \emph{alle} Muggelgeborenen zum Stolpern und Fallen zu bringen?"

"Du meinst - ?" keuchte Tracey.\\ "Jawohl!" sagte Daphne dramatisch. "Was ist, wenn Malfoy \emph{der Erbe von} \emph{Slipperin*} ist? "\\ "Der nächste Drop Lord**!" sagte Tracey.

Das war ein viel zu guter Spruch, um ihn für sich zu behalten, also war er bei Einbruch der Dunkelheit überall in Hogwarts bekannt, und am nächsten Morgen war er die Schlagzeile des \emph{Quibblers}.

\emph{Nachspiel, 2:}

Hermine sorgte dafür, dass sie an diesem Abend schön früh in ihr gewohntes Klassenzimmer kam, nur damit sie allein, auf einem Stuhl, friedlich ein Buch las, als Harry dort ankam.

Wenn es irgendeine Möglichkeit gab, dass eine Tür um Verzeihung bittend knarrt, dann öffnete die Tür sich so.

"Ähm," sagte Harry Potters Stimme.\\ Hermine las weiter.\\ "Es tut mir irgendwie leid, ich wollte nicht, dass du \emph{tatsächlich} vom Dach fällst oder so…"\\ Es war in Wirklichkeit ein ziemlich unterhaltsames Erlebnis gewesen.

"Ich, ah…. Ich habe nicht viel Erfahrung mit Entschuldigungen, ich falle auf die Knie, wenn du willst, oder kaufe dir etwas Teures, \emph{Hermine ich weiß nicht, wie ich mich bei dir dafür entschuldigen soll, was kann ich tun, sag es mir einfach?} „\\ Sie las das Buch schweigend weiter.\\ Es war auch nicht so, als hätte \emph{sie} eine Ahnung, wie sich Harry entschuldigen könnte.

Im Moment spürte sie nur eine Art seltsame Neugierde, was passieren würde, wenn sie noch eine Weile ihr Buch lesen würde.

*Slipperin -- Wortspiel aus dem Hausnamen Slytherin und „slip“ = ausrutschen\\ ** Drop Lord -- Wortspiel von „Dark/Dunkler Lord“ und „Drop“ = fallen

