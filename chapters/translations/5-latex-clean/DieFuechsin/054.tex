

\hypertarget{das-stanford-prison-experiment-teil-4}{% \section{22. Das Stanford-Prison-Experiment, Teil 4}\label{das-stanford-prison-experiment-teil-4}}

-\/-\/-\/-\/- Kapitel 54: Das Stanford-Prison-Experiment, Teil 4 -\/-\/-\/-\/-

Ein schwacher grüner Funke gab das Tempo vor, und dahinter folgte eine glänzende Silberfigur, alle anderen Entitäten waren unsichtbar. Sie hatten fünf Gänge des Korridors durchquert, sich fünf Mal nach rechts gewandt und waren fünf Treppen hinaufgestiegen; und als Bellatrix ihre zweite Flasche Schokomilch ausgetrunken hatte, hatte sie feste Tafeln Schokolade zu essen bekommen.

Erst nach ihrer dritten Tafel Schokolade begannen seltsame Geräusche aus Bellatrix' Kehle zu kommen.

Es dauerte einen Moment, bis Harry die Geräusche verstanden und verarbeitet hatte. Es klang nicht wie etwas, das er je zuvor gehört hatte; der Rhythmus war zerbrochen, fast nicht wiederzuerkennen, er brauchte so lange, bis er erkannte, dass Bellatrix weinte.

Bellatrix Black weinte, die schrecklichste Waffe des Dunklen Lords weinte, sie war unsichtbar, aber man konnte sie hören, winzige erbärmliche Geräusche, die sie versuchte zu unterdrücken, sogar jetzt noch.

„Passiert das wirklich?“, sagte Bellatrix. Die Tonalität war in ihre Stimme zurückgekehrt, nicht länger ein totes Gemurmel, sie erhob sich am Ende, um die Frage zu formen. „ Passiert das wirklich?"

\emph{Ja}, dachte der Teil von Harry, der den Dunklen Lord simulierte, \emph{schweig jetzt} -

Er konnte diese Worte nicht über seine Lippen bringen, er konnte es einfach nicht.

„Ich wusste - Ihr würdet - eines Tages zu mir kommen“, Bellatrix' Stimme zitterte und brach, als sie für leise Schluchzer Luft holte, „ich wusste - Ihr lebt - dass Ihr zu mir kommen würdet - mein Herr... „ jetzt kam ein langes Einatmen wie ein gewaltiges Keuchen, „und dass selbst - wenn Ihr kommt - Ihr mich immer noch nicht lieben würdet - niemals - Ihr würdet niemals meine Liebe erwidern - deshalb - konnten sie mir meine Liebe nicht nehmen - obwohl ich mich nicht erinnern kann - mich an so viele andere Dinge nicht erinnern kann - obwohl ich nicht weiß, was ich vergessen habe - aber ich erinnere mich, wie sehr ich Euch liebe, Herr -"

Ein Messer stach Harry ins Herz, er hatte noch nie etwas so Schreckliches gehört, er wollte den Dunklen Lord jagen und ihn dafür töten...

"Habt Ihr immer noch - Verwendung für mich - mein Lord?"

„Nein“, zischte Harrys Stimme, ohne dass er auch nur daran dachte, sie schien einfach automatisch zu funktionieren, „Ich betrat Askaban aus einer Laune heraus. Natürlich habe ich Verwendung für dich! Stell keine dummen Fragen."

„Aber - ich bin schwach“, sagte Bellatrix' Stimme, und ein volles Schluchzen entging ihr, es klang viel zu laut in den Korridoren von Askaban, „Ich kann nicht für Euch töten, mein Herr, es tut mir leid, sie haben alles aufgegessen, mich aufgegessen, ich bin zu schwach zum Kämpfen, was nütze ich Euch jetzt -"

Harrys Gehirn suchte verzweifelt nach einer Möglichkeit, sie zu beruhigen, und zwar durch die Lippen eines dunklen Lords, der nie ein einziges fürsorgliches Wort sagen würde.

„Hässlich“, sagte Bellatrix. Ihre Stimme sagte dieses Wort, als wäre es der letzte Nagel zu ihrem Sarg, das letzte bisschen Verzweiflung. „Ich bin hässlich, das haben sie auch gegessen, ich bin, ich bin nicht mehr hübsch, ihr werdet mich nicht einmal mehr als Belohnung für eure Diener benutzen können, nicht einmal die Lestranges werden mir noch wehtun wollen -"

Die strahlende Silberfigur blieb stehen.

Weil Harry aufgehört hatte zu gehen.

\emph{Der Dunkle Lord, er}... Der Teil von Harrys Selbst, der weich und verletzlich war, schrie vor ungläubigem Entsetzen und versuchte, die Realität zurückzuweisen, das Verständnis zu verweigern, selbst als ein kälterer und härterer Teil das Muster vervollständigte: \emph{Sie gehorchte ihm darin, wie sie ihm in allen Dingen gehorchte.}

Der grüne Funke hüpfte eindringlich auf und ab, huschte vorwärts.

Der silberne Humanoide blieb an Ort und Stelle.

Bellatrix schluchzte immer heftiger.

"Ich bin, ich bin nicht, ich kann nicht mehr nützlich sein..."

Riesige Hände quetschten Harrys Brust, wrangen ihn wie einen Waschlappen und versuchten, sein Herz zu zerquetschen.

„Bitte“, flüsterte Bellatrix, „tötet mich einfach...“ Ihre Stimme schien sich zu beruhigen, als sie das sagte. „Bitte, Herr, tötet mich. Ich habe keinen Grund zu leben, wenn ich Euch nichts nutze... Ich will nur, dass es aufhört. Bitte, Herr, tut mir ein letztes Mal weh, tut mir weh, bis ich aufhöre... Ich liebe Euch..."

Es war das Traurigste, was Harry je gehört hatte.

Die helle silberne Form von Harrys Patronus flackerte -

Waberte -

Erhellte sich -

Die Wut, die in Harry aufkam, seine Wut gegen den Dunklen Lord, der dies getan hatte, die Wut gegen die Dementoren, gegen Askaban, gegen die Welt, die ein solches Grauen zuließ, das alles schien sich direkt durch seinen Arm und in seinen Zauberstab zu ergießen, ohne dass es irgendeine Möglichkeit gab, es zu blockieren, er versuchte es mit seinen Gedanken aufzuhalten, und nichts geschah.

„Mein Herr!“ flüsterte die verschleierte Stimme von Professor Quirrell. „Mein Zauber gerät außer Kontrolle! Helft mir, Mylord!"

Der Patronus wurde immer heller und heller und wuchs schneller als an dem Tag, an dem Harry einen Dementor vernichtet hatte.

„Mein Herr!“, sagte die Silhouette in einem entsetzten Flüstern. „Helft mir! Jeder wird es spüren, mein Herr!"

\emph{Alle werden es spüren}, dachte Harry. Seine Fantasie konnte es sich deutlich vorstellen, die Gefangenen in ihren Zellen rührten sich, als die Kälte und Dunkelheit abfiel und durch heilendes Licht ersetzt wurde.

Jede freiliegende Oberfläche brannte nun wie eine weiße Sonne in den Reflexionen, die Silhouette von Bellatrix' Skelett und der fahle Mann, der nun in der Glut deutlich sichtbar war, der Desillusionierungszauber konnte mit dem überirdischen Glanz nicht Schritt halten; nur der Mantel der Unsichtbarkeit aus den Heiligtümern des Todes hielt dem stand.

"Bitte, mein Herr! \emph{Ihr müsst das stoppen!}„

Aber Harry konnte nicht länger den Willen aufbringen es anzuhalten, er wollte nicht mehr, dass es aufhörte. Er spürte es, mehr und mehr der Funken des Lebens in Askaban wurden von seinem Patronus beschützt, während er sich wie ausgebreitete Flügel aus Sonnenlicht entfaltete, die Luft verwandelte sich in pures Silber, als er es dachte, Harry wusste, was er zu tun hatte.

\emph{\emph{„Bitte mein Herr!“}}

Die Worte blieben ungehört.

\emph{\emph{Sie waren weit von ihm entfernt, die Dementoren in ihrer Grube, aber Harry wusste, dass sie selbst aus dieser Entfernung vernichtet werden konnten, wenn das Licht hell genug aufleuchtete, er wusste, dass der Tod selbst ihn nicht aufhalten konnte, wenn er aufhörte, sich zurückzuhalten, also öffnete er alle Tore in seinem Inneren und versenkte die Quellen seines Zaubers in die tiefsten Teile seines Geistes, seines Verstandes und seines Willens, und übergab absolut alles dem Zauber -}}

Und im Inneren der Sonne bewegte sich ein nur leicht dunklerer Schatten vorwärts und streckte eine flehende Hand aus.

\emph{\emph{FALSCH}}

\emph{\emph{NICHT}}

Das plötzliche Gefühl des Untergangs kollidierte mit Harrys stählerner Entschlossenheit, seiner Furcht und Unsicherheit, die gegen das helle Ziel ankämpften, nichts anderes hätte ihn erreichen können. Die Silhouette machte einen weiteren Schritt vorwärts und einen weiteren, das Gefühl des Untergangs stieg bis zum Punkt einer schrecklichen Katastrophe an; und wie durch einen kalten Wasserguss ernüchtert, sah Harry es, erkannte er die Folgen seines Tuns, die Gefahr und die Falle.

Hätte man von außen zugesehen, hätte man gesehen, wie das Innere der Sonne heller und dunkler wurde...

Aufhellen und verdunkeln...

...und schließlich verblasste, verblasste, verblasste zu gewöhnlichem Mondlicht, das im Gegensatz dazu wie stockfinsteres Licht aussah.

In der Dunkelheit dieses Mondlichts stand ein fahler Mann, der seine Hand flehend ausgestreckt hielt, und das Skelett einer Frau, das auf dem Boden lag, mit einem verwirrten Blick auf ihrem Gesicht.

Und Harry, immer noch unsichtbar, fiel auf die Knie. Die größere Gefahr war vorüber, und jetzt versuchte Harry nur noch, nicht zusammenzubrechen, um den Zauber auf einer tieferen Ebene aufrechtzuerhalten. Er hatte etwas in den Zauber einfließen lassen, aber es hoffentlich nicht verloren - er hätte wissen müssen, hätte sich daran erinnern müssen, dass es nicht nur Magie war, die den Patronuszauber befeuerte -

„Danke, mein Herr“, flüsterte der fahle Mann.

„Narr“, sagte die harte Stimme eines Jungen, der vorgab, ein Dunkler Lord zu sein. „Habe ich dich nicht davor gewarnt, dass der Zauber sich als tödlich erweisen könnte, wenn du deine Gefühle nicht unter Kontrolle bringst?“

Professor Quirrells Augen weiteten sich natürlich nicht.

„Ja, mein Herr, ich verstehe“, sagte der Diener des Dunklen Lords mit zögernder Stimme und wandte sich an Bellatrix -

Sie stieß sich bereits vom Boden ab, langsam, wie eine alte, alte Muggel-Frau. „Wie lustig“, flüsterte Bellatrix, „du wärst beinahe von einem Patronuszauber getötet worden...“ Ein Kichern, das klang, als würde es den Staub aus ihren Kicherkanälen blasen. „Ich könnte dich vielleicht bestrafen, wenn mein Herr dich an Ort und Stelle einfrieren würde und ich Messer hätte... vielleicht kann ich doch noch nützlich sein? Oh, jetzt fühle ich mich etwas besser, wie seltsam..."

„Sei still, liebe Bella“, sagte Harry mit kühler Stimme, „bis ich dir das Wort erteile."

Es gab keine Antwort, das war Gehorsam.

Der Diener ließ das menschliche Skelett schweben und machte sie wieder unsichtbar, kurz darauf verschwand er selbst mit dem Geräusch eines weiteren zerbrechenden Eies.

Sie gingen weiter durch die Korridore von Askaban.

Und im Vorbeigehen wusste Harry, dass sich die Gefangenen in ihren Zellen rührten, als sich die Angst für einen kostbaren Moment aufhob, vielleicht sogar einen kleinen Hauch von Heilung spürten, als sein Licht an ihnen vorbeiging, und dann wieder zusammenbrachen, als die Kälte und die Dunkelheit wieder eindrangen.

Harry versuchte sehr stark, nicht daran zu denken.

Sonst würde sein Patronus so lange wachsen, bis er jeden Dementor in Askaban weggebrannt hätte und hell genug brennen, um sie selbst aus dieser Entfernung zu vernichten...

Sonst würde sein Patronus so lange wachsen, bis er jeden Dementor in Askaban weggebrannt hätte und Harrys ganzes Leben als Treibstoff genommen hätte.

Im Quartier der Auroren an der Spitze von Askaban schnarchte ein Auroren-Trio in den Baracken, ein Auroren-Trio ruhte sich im Pausenraum aus, und ein Auroren-Trio hatte Dienst im Kommandoraum und hielt Wache. Der Kommandoraum war einfach, aber groß, mit drei Stühlen hinten, auf denen drei Auroren saßen, ihre Zauberstäbe immer in der Hand, um ihre drei Patronusse zu erhalten, während die strahlend weißen Formen vor dem offenen Fenster dahinschritten und sie alle vor der Angst der Dementoren schützten.

Die drei hielten sich in der Regel hinten auf, spielten Poker und schauten nicht aus dem Fenster. Sicher, man hätte dort etwas Himmel sehen können, und es gab sogar ein oder zwei Stunden am Tag, in denen man etwas Sonne hätte sehen können, aber dieses Fenster blickte auch auf den zentralen Höllenschlund hinunter.

Nur für den Fall, dass ein Dementor hochschweben und mit ihnen sprechen wollte.

Es gab keine Möglichkeit, dass Auror Li zugestimmt hätte, hier Dienst zu tun, mit dreifacher Bezahlung oder ohne dreifache Bezahlung, wenn er keine Familie zu versorgen gehabt hätte. (Sein richtiger Name war Xiaoguang, und alle nannten ihn stattdessen Mike; er hatte seine Kinder Su und Kao genannt, was ihnen hoffentlich besser dienen würde). Neben dem Geld war sein einziger Trost, dass zumindest seine Freunde ein ausgezeichnetes Drachen-Poker-Spiel spielten. Obwohl es zu diesem Zeitpunkt schwer wäre, es \emph{nicht} zu tun.

Es war ihr 5.366. Spiel und Li hatte das wahrscheinlich beste Blatt der 5300er. Es war ein Samstag im Februar, und es gab drei Spieler, was ihm erlaubte, die Farbe einer beliebigen Hole Card zu ändern, mit Ausnahme einer Zwei, Drei oder Sieben; und das reichte aus, um mit Einhörnern, Drachen und Siebenen einen Nahkampf aufzubauen…

Auf der anderen Seite des Tisches blickte Gerard McCusker von den Tischkarten in Richtung Fenster auf und starrte.

Das Gefühl des Versinkens überkam Lis Magen mit überraschender Geschwindigkeit.

Wenn seine Herz-Sieben von einem Dementor-Modifikator getroffen wurde und sich in eine Sechs verwandelte, ging er geradewegs auf zwei Paare herunter, und McCusker könnte das schlagen -

„Mike“, sagte McCusker, „was ist mit deinem Patronus?"

Li drehte seinen Kopf und schaute.

Sein weicher Silberdachs hatte sich von seiner Wache über der Grube abgewandt und starrte nach unten auf etwas, das nur er sehen konnte.

Einen Augenblick später folgten Bahrys mondbeschienene Ente und McCuskers heller Ameisenbär, die in die gleiche Richtung starrten.

Sie alle tauschten Blicke aus und seufzten dann.

„Ich werde es ihnen sagen“, sagte Bahry. Das Protokoll sah vor, die drei Auroren, die zwar außer Dienst waren, aber nicht schliefen, zur Untersuchung von Anomalien zu schicken. „Könnte vielleicht einen von ihnen ablösen und die C-Spirale nehmen, wenn es euch beiden nichts ausmacht."

Li wechselte einen Blick mit McCusker, und beide nickten. Es war nicht allzu schwer, in Askaban einzubrechen, wenn man wohlhabend genug war, um einen mächtigen Zauberer anzuheuern, und mit Absichten, die gut genug waren, um jemanden zu rekrutieren, der den Patronuszauber wirken konnte. Menschen mit Freunden in Askaban würden das tun, einfach einbrechen, nur um jemanden einen halben Tag Patronuszeit zu schenken, eine Chance auf echte Träume anstelle von Albträumen. Ihnen einen Vorrat an Schokolade hinterlassen, den sie in ihrer Zelle verstecken können, um die Chance zu erhöhen, dass sie ihre Strafe überlebten. Und die Auroren auf Wache... nun, selbst wenn man erwischt werden sollte, könnten man die Auroren wahrscheinlich überzeugen, darüber hinwegzusehen, im Austausch gegen die richtige Bestechung.

Für Li lag das richtige Bestechungsgeld in der Regel im Bereich von zwei Knuts und einem silbernen Sickel. Er hasste diesen Ort.

Aber Bahry One-Hand hatte eine Frau, und die Frau hatte Heilerrechnungen, und wenn du es dir leisten konntest, jemanden anzuheuern, der in Askaban einbrechen konnte, dann konntest du es dir auch leisten, Bahrys verbliebene Handfläche ziemlich hart zu schmieren, wenn er derjenige war, der dich erwischte.

Durch eine unausgesprochene Übereinkunft, bei der keiner von ihnen etwas verriet, indem er es als Erster vorschlug, hatten die drei als erstes ihr Pokerspiel beendet. Li gewann, da keine Dementoren aufgetaucht waren. Und bis dahin hatten die Patronus aufgehört zu starren und waren zu ihrer normalen Patrouille zurückgekehrt, also war es wahrscheinlich nichts, aber Verfahren war Verfahren.

Nachdem Li sich den Pot geschnappt hatte, gab Bahry allen ein förmliches Nicken und stand vom Tisch auf. Die langen weißen Locken des älteren Mannes streiften gegen seine schicken roten Gewänder, seine Gewänder streiften den Metallboden des Kommandoraums, als Bahry durch die Trenntür ging, die zu den ehemals dienstfreien Auroren führte.

Li war nach Hufflepuff sortiert worden, und ihm war bei dieser Art von Geschäften manchmal etwas mulmig zumute. Aber Bahry hatte ihnen alle Bilder gezeigt, und man musste einen Mann für seine arme, kranke Frau tun lassen, was er konnte, vor allem, wenn er bis zu seiner Pensionierung nur noch sieben Monate Zeit hatte.

Der schwache grüne Funke schwebte durch die Metallkorridore, und der silberne Humanoide, der jetzt ein wenig dunkler zu sein schien, folgte ihm nach. Manchmal flammte die helle Gestalt auf, vor allem, wenn sie an einer der riesigen Metalltüren vorbeikam, aber sie dunkelte immer wieder ab.

Bloße Augen hätten die unsichtbaren anderen nicht sehen können: den elfjährigen Jungen, der lebte, und das lebende Skelett, das Bellatrix Black war, und den mit Vielsafttrank verwandelten Verteidigungsprofessor von Hogwarts, die alle zusammen durch Askaban reisten. Wenn das der Anfang eines Witzes war, kannte Harry die Pointe nicht.

Sie waren noch vier weitere Treppen hinaufgestiegen, bevor die raue Stimme des Verteidigungsprofessors einfach und ohne Betonung sagte: „Ein Auror kommt."

Es dauerte zu lange, vielleicht eine ganze Sekunde, bis Harry es verstand, bis der Adrenalinschub in sein Blut pumpte und er sich daran erinnerte, was Professor Quirrell bereits mit ihm besprochen und ihm gesagt hatte, was er in diesem Fall zu tun habe, und dann machte Harry auf dem Absatz kehrt und flog auf dem Weg zurück, auf dem sie gekommen waren.

Harry erreichte die Treppe und legte sich verzweifelt auf die dritte Stufe von oben, wobei sich das kalte Metall selbst durch seinen Mantel und seine Gewänder hindurch hart anfühlte. Der Versuch, seinen Kopf nach oben zu bewegen, über den Treppenabsatz zu schauen, zeigte, dass er Professor Quirrell nicht sehen konnte; und das bedeutete, dass Harry sich außerhalb der Schusslinie für irgendwelches Streufeuer befand.

Sein strahlender Patronus folgte ihm nach und legte sich neben ihn auf die Stufe direkt unter ihm; denn auch er durfte nicht gesehen werden.

Es gab ein schwaches Geräusch, wie von Wind oder Zischen, und dann das Geräusch von Bellatrix' unsichtbarem Körper, der auf einer Treppe weiter unten zur Ruhe kam, sie hatte darin keinen Platz, außer -

„Lieg still“, sagte das kalte hohe Flüstern, „bleiben ruhig“.

Es herrschte Ruhe und Schweigen.

Harry drückte seinen Zauberstab gegen die Seite der Metallstufe direkt über ihm. Wäre er ein anderer, hätte er einen Knut aus seiner Tasche nehmen müssen... oder ein Stück Stoff von seiner Robe abreißen müssen... oder einen seiner Nägel abbeißen müssen... oder einen Felsbrocken finden müssen, der groß genug war, dass er ihn sehen konnte, und fest genug, um an einer Stelle in einer Richtung zu bleiben, während er seinen Zauberstab berührte. Aber mit Harrys allmächtiger Kraft der partiellen Verwandlung war dies nicht notwendig; er konnte diesen speziellen Schritt der Operation überspringen und jedes Material in der Nähe verwenden.

Dreißig Sekunden später war Harry der stolze neue Besitzer eines gebogenen Spiegels, und...

„\emph{Wingardium Leviosa}“, flüsterte Harry so leise, wie er konnte.

...ließ ihn gerade über den Stufen schweben und beobachtete in dieser gekrümmten Oberfläche fast den ganzen Korridor, in dem Professor Quirrell unsichtbar wartete.

In der Ferne hörte Harry dann das Geräusch von Schritten.

Und er sah die Gestalt (etwas schwer zu erkennen im Spiegel) einer Person in roten Gewändern, die die Treppe herunterkam und den scheinbar leeren Korridor betrat; begleitet von einem kleinen Patronus-Tier, das Harry nicht ganz erkennen konnte.

Das Auror war durch einen blauen Schimmer geschützt, es war schwer, die Details zu erkennen, aber so viel konnte Harry sehen, denn das Auror hatte bereits Schilde in Position gebracht und verstärkt.

\emph{Mist}, dachte Harry. Nach Ansicht des Verteidigungsprofessors bestand die wesentliche Kunst des Duells darin, eine Verteidigung aufzubauen, die alles abwehrt, was jemand auf einen werfen könnte, und gleichzeitig zu versuchen, auf eine Art und Weise anzugreifen, die wahrscheinlich die aktuelle Verteidigung durchdringen könnte. Und der bei weitem einfachste Weg, einen echten Kampf zu gewinnen - Professor Quirrell hatte dies immer und immer wieder gesagt - bestand darin, den Feind zu erschießen, bevor er überhaupt einen Schild erhob, entweder von hinten oder aus so großer Entfernung, dass er nicht rechtzeitig ausweichen oder kontern konnte.

Obwohl Professor Quirrell immer noch in der Lage sein könnte, einen Schuss von hinten abzugeben, wenn -

Aber der Auror blieb stehen, nachdem er drei Schritte in den Korridor gegangen war.

„Schöne Desillusionierung“, sagte eine harte männliche Stimme, die Harry nicht erkannte. „Jetzt zeig dich, oder du bekommst \emph{richtig} Ärger."

Da wurde die Form des fahlen, bärtigen Mannes sichtbar.

„Und du mit dem Patronus“, sagte die harte Stimme. „Komm auch heraus. \emph{Jetzt}."

„Wäre nicht klug“, sagte die kiesige Stimme des fahlen Mannes. Es war nicht mehr die verängstigte Stimme des Dieners des Dunklen Lords; sie war plötzlich zur professionellen Einschüchterung eines kompetenten Verbrechers geworden. „Sie wollen nicht sehen, wer hinter mir steht. Vertrauen Sie mir, das wollen Sie nicht. Fünfhundert Galleonen, kaltes Bargeld im Voraus, wenn Sie sich umdrehen und weggehen. Wenn Sie das nicht tun, bedeutet das Ärger für Ihre Karriere."

Es gab eine lange Pause.

„Hören Sie, wer immer Sie sind“, sagte die harte Stimme. „Sie scheinen nicht zu wissen wie das hier läuft. Es ist mir egal, ob das Lucius Malfoy hinter Ihnen ist oder Albus verdammter Dumbledore. Ihr kommt \emph{alle} raus, ich scanne euch alle, und \emph{dann} reden wir darüber, wie viel euch das kosten wird -"

„Zweitausend Galeonen, letztes Angebot“, sagte die kiesige Stimme und nahm einen warnenden Unterton an. „Das ist zehnmal mehr als der übliche Preis und mehr als Sie in einem Jahr verdienen. Und glauben Sie mir, wenn Sie etwas sehen, das Sie nicht sehen sollten, werden Sie bereuen, dass Sie es nicht angenommen haben -"

„Halt die Klappe!“, sagte die harte Stimme. „Du hast genau fünf Sekunden, um den Zauberstab fallen zu lassen, bevor ich dich fallen lasse. Fünf, vier -„

\emph{\emph{Was tun Sie da, Professor Quirrell?} dachte Harry verzweifelt. \emph{Zuerst angreifen!} \emph{Zaubern Sie} \emph{wenigstens einen Schild!}}

“- drei, zwei, eins! \emph{Stupor!}"

Bahry starrte, ein Schauer lief ihm über den Rücken.

Der Zauberstab des Mannes hatte sich so schnell bewegt, dass es so aussah, als sei er an Ort und Stelle appariert, und Bahrys Schockzauber funkelte gerade zahm am Ende, nicht blockiert, nicht gekontert, nicht abgelenkt, \emph{gefangen} wie eine Fliege im Honig.

„Mein Angebot ist auf fünfhundert Galleonen zurückgegangen“, sagte der Mann mit einer kälteren, formelleren Stimme. Er lächelte trocken, und das Lächeln sah falsch aus auf diesem bärtigen Gesicht. „Und Sie werden einen Erinnerungszauber akzeptieren müssen."

Bahry hatte bereits die Schwingungen auf seinen Schilden vertauscht, so dass sein eigener Schockzauber nicht mehr durch sie hindurchgehen konnte, er neigte seinen Zauberstab bereits wieder in eine defensive Position, er hob bereits seine verhärtete künstliche Hand in eine Position, in der er alles blockieren konnte, was blockierbar war, und er dachte wortlose Zaubersprüche, um weitere Schichten auf seine Schilde zu legen -

Der Mann schaute Bahry nicht an. Stattdessen stöberte er neugierig in Bahrys Schockzauber, wo er immer noch am Ende seines Zauberstabs schwankte, zog rote Funken und schnippte sie mit seinen Fingern weg, wobei er den Zauber langsam wie ein Kinderstabpuzzle zerlegte.

Der Mann hatte keine eigenen Schilde aufgestellt.

„Sag mir“, sagte der Mann mit einer uninteressierten Stimme, die nicht ganz in die raue Kehle zu passen schien - Vielsaft, hätte Bahry vermutet, wenn er gedacht hätte, dass jemand aus dem Inneren des Körpers eines anderen Menschen so zart zaubern könnte - „was hast du im letzten Krieg getan? Dich in Gefahr begeben oder dich aus Schwierigkeiten heraushalten?"

„In Gefahr begeben“, sagte Bahry. Seine Stimme bewahrte die eiserne Ruhe eines Aurors, der fast hundert volle Jahre bei der Truppe war, sieben Monate vor der Zwangsversetzung in den Ruhestand, Mad-Eye Moody hätte es nicht härter sagen können.

"Gegen irgendwelche Todesser gekämpft?„

Nun zierte ein grimmiges Lächeln Bahrys eigenes Gesicht. „Zwei auf einmal.“ Zwei von Du-weißt-schon-wem's eigenen Krieger-Attentätern, persönlich ausgebildet von ihrem dunklen Meister. Zwei Todesser auf einmal gegen Bahry allein. Es war der härteste Kampf in Bahrys Leben gewesen, aber er hatte seinen Mann gestanden und ging nur mit dem Verlust seiner linken Hand davon.

„Hast du sie getötet?“ Der Mann klang gelangweilt neugierig, und er zog weiterhin Feuerfäden aus dem stark verminderten Betäubungsblitz, der immer noch am Ende seines Zauberstabs gefangen war. Seine Finger webten nun kleine Muster von Bahrys eigener Magie, bevor sie schnippten, um sie zu zerstreuen.

Schweiß brach auf Bahrys Haut unter seinen Roben aus. Seine Metallhand blitzte nach unten, riss den Spiegel von seinem Gürtel - „Bahry an Mike, ich brauche Unterstützung!

Es gab eine Pause und Stille.

"Bahry an Mike!"

Der Spiegel lag stumpf und leblos in seiner Hand. Langsam hängte Bahry ihn wieder an seinen Gürtel.

„Es ist schon eine ganze Weile her, dass ich einen ernsthaften Kampf mit einem ernsthaften Gegner hatte“, sagte der Mann und schaute immer noch nicht zu Bahry auf. „Versuchen Sie, mich nicht zu sehr zu enttäuschen. Sie können angreifen, wann immer Sie bereit sind. Oder Sie können mit fünfhundert Galeonen weggehen."

Es herrschte langes Schweigen.

Dann schrie die Luft wie Metall, das Glas schneidet, als Bahry seinen Zauberstab nach unten schwang.

Harry konnte es kaum sehen, konnte inmitten der Lichter und Blitze kaum etwas erkennen, die Krümmung seines Spiegels war perfekt (diese Taktik hatten sie schon in der Chaos-Legion geübt), aber die Szene war immer noch zu klein, und Harry hatte das Gefühl, er würde es nicht verstehen können, selbst wenn er aus einem Meter Entfernung zuschauen würde, es ging alles zu \emph{schnell}, rote Detonationen prallten von blauen Schilden ab, grüne Lichtstrahlen prallten aufeinander, schattenhafte Formen tauchten auf und verschwanden, er konnte nicht einmal sagen, wer was zauberte, außer dass der Auror Beschwörungsformel auf Beschwörungsformel schrie und verzweifelt auswich, während Professor Quirrells Vielsafttrankkörper an einer Stelle stand und seinen Zauberstab schnippte, meist stumm, aber hin und wieder Worte in nicht erkennbaren Sprachen aussprach, die den ganzen Spiegel weiß färbten und die Hälfte des Schildes des Aurors wegrissen als der zurücktaumelte.

Harry hatte Schauduelle zwischen den stärksten Siebtklässlern gesehen, und dies war so weit darüber, dass Harrys Geist sich betäubt fühlte, als er sah, wie weit er noch zu gehen hatte. Es gab keinen einzigen Siebtklässler, der eine halbe Minute gegen den Auror hätte bestehen können, alle drei Siebenklässler-Armeen zusammengenommen könnten den Verteidigungsprofessor vielleicht nicht mal ankratzen...

Der Auror war zu Boden gefallen, ein Knie und eine Hand stützten sich ab, während die andere Hand verzweifelt gestikulierte und sein Mund verzweifelte Worte rief. Die wenigen Beschwörungsformeln, die Harry erkannte, waren alles Schildzauber, als sich eine Schar von Schatten wie ein Wirbelwind von Rasiermessern um den Auror drehte.

Und Harry sah, wie Professor Quirrells Vielsafttrankform seinen Zauberstab absichtlich auf die Stelle richtete, an der der Auror kniete und die letzten Momente seines Kampfes kämpfte.

„Ergeben Sie sich“, sagte die kiesige Stimme.

Der Auror spuckte etwas Unaussprechliches aus.

„In diesem Fall“, sagte die Stimme, „\emph{Avada} -"

Die Zeit schien sehr langsam zu vergehen, so als gäbe es Zeit, die einzelnen Silben \emph{Ke}, und \emph{Da}, und \emph{Vra} zu hören, Zeit, dem Auror zuzuschauen, wie er anfing, sich verzweifelt beiseite zu werfen; und obwohl das alles so langsam geschah, gab es irgendwie keine Zeit, etwas zu \emph{tun}, keine Zeit für Harry, die Lippen zu öffnen und \emph{NEIN} zu schreien, keine Zeit, sich zu bewegen, vielleicht sogar keine Zeit zum Nachdenken.

Nur Zeit für einen verzweifelten Wunsch, dass ein unschuldiger Mann nicht sterben sollte -

Und eine glühende Silberfigur stand vor dem Auror.

Sie stand dort nur den Bruchteil einer Sekunde, bevor das grüne Licht einschlug.

Bahry drehte sich verzweifelt zur Seite, ohne zu wissen, ob er es schaffen würde -

Seine Augen waren auf den Gegner und seinen nahenden Tod gerichtet, so dass Bahry nur kurz den Umriss der brillanten Silhouette sah, den Patronus heller als alle, die er je gesehen hatte, er sah ihn gerade noch lange genug, um die unmögliche Form zu erkennen, bevor das grüne und das silberne Licht aufeinandertrafen und beide Lichter verschwanden, \emph{beide} Lichter verschwanden, \emph{der tödliche Fluch war geblockt worden}, und dann wurden Bahrys Ohren malträtiert, als er sah, wie sein schrecklicher Gegner schrie, schrie, schrie, sich an seinem Kopf festhielt und schrie, er begann zu fallen, als Bahry bereits fiel -

Bahry stürzte wegen seinem eigenen verzweifelten Sprung auf den Boden, und seine ausgerenkte linke Schulter und seine gebrochene Rippe schrien aus Protest. Bahry ignorierte den Schmerz, schaffte es, wieder auf die Knie zu kommen, holte seinen Zauberstab hoch, um seinen Gegner zu betäuben, er verstand nicht, was vor sich ging, aber er wusste, dass dies seine einzige Chance war.

"\emph{Stupor!} „

Der rote Blitz flog in Richtung des fallenden Körpers des Mannes und wurde in der Luft zerrissen und zerstreute sich - und nicht durch irgendeinen Schild. Bahry konnte es \emph{sehen}, das Wabern in der Luft, das seinen gefallenen und schreienden Gegner umgab.

Bahry konnte es wie einen tödlichen Druck auf seiner Haut spüren, den Fluss der Magie, der stärker und stärker und stärker wurde auf einen schrecklichen Ausbruch zusteuerte. Seine Instinkte schrien ihm zu, er solle rennen, bevor die Explosion kam, dies war kein Zauber, kein Fluch, dies war wildgewordene Zauberei, aber bevor Bahry überhaupt auf die Beine kommen konnte -

Der Mann warf seinen Zauberstab von sich weg (er warf seinen Zauberstab weg!), und eine Sekunde später verschwamm seine Form und verschwand völlig.

Eine grüne Schlange lag regungslos auf dem Boden, bewegungslos, noch bevor Bahrys nächster Betäubungszauber, der in schierem Reflex abgefeuert wurde, sie ohne Widerstand traf.

Als sich der schreckliche Magiefluss und Druck aufzulösen begann, als die wilde Zauberei nachließ, bemerkte Bahrys benommener Verstand, dass der Schrei weiter anhielt. Nur klang er anders, wie der Schrei eines kleinen Jungen, der von der Treppe kam, die zur nächsttieferen Ebene hinunterführte.

Auch dieser Schrei erstickte, und dann herrschte Stille, bis auf Bahrys verzweifeltes Keuchen.

Seine Gedanken waren langsam, verwirrt und durcheinander. Sein Gegner war \emph{wahnsinnig} stark gewesen, das war kein Duell gewesen, es war wie in seinem ersten Jahr als Auror-Schüler, als er versuchte, gegen Madam Tarma zu kämpfen. Die Todesser waren nicht ein Zehntel so gut gewesen, Mad-Eye Moody war nicht so gut... und wer, was, wie im Namen von Merlins Eiern hatte jemand einen \emph{tödlichen Fluch} geblockt?

Bahry schaffte es, die Energie aufzubringen, seinen Zauberstab gegen seine Rippe zu pressen, den Heilungszauber zu murmeln und ihn dann wieder an seine Schulter zu drücken. Es kostete ihn mehr, als es hätte sein sollen, nahm ihm viel zu viel, seine Magie war einem Atemzug von völliger Erschöpfung entfernt; er hatte nichts mehr übrig für seine kleinen Schnitte und Prellungen oder auch nur, um die Reste seines Schildes zu verstärken. Alles, was er tun konnte, war seinen Patronus nicht erlöschen zu lassen.

Bahry atmete tief, schwer, beruhigte seinen Atem soweit er konnte, bevor er sprach.

„Du“, sagte Bahry. „Wer auch immer du bist. komm raus."

Es herrschte Stille, und Bahry kam der Gedanke, dass derjenige, wer auch immer es war, bewusstlos sein könnte. Er verstand nicht, was gerade passiert war, aber er hatte den Schrei gehört...

Nun, es gab eine Möglichkeit, das zu testen.

„Komm raus“, sagte Bahry, und ließ seine Stimme härter klingen, „oder ich beginne mit Flächeneffekt-Flüchen.“ Er hätte wahrscheinlich nicht einen geschafft, wenn er es versucht hätte.

„Warten Sie“, sagte die Stimme eines Jungen, die Stimme eines kleinen Jungen, hoch und dünn und schwankend, als ob jemand Erschöpfung oder Tränen zurückhalten würde. Die Stimme schien nun näher zu kommen. „Bitte warten Sie. Ich - komme raus -"

„Lass die Unsichtbarkeit fallen“, knurrte Bahry. Er war zu müde, um sich mit Anti- Desillusionierungszaubern zu beschäftigen.

Einen Augenblick später tauchte das Gesicht eines kleinen Jungen aus einem sich entfaltenden Unsichtbarkeitsmantel auf, und Bahry sah das schwarze Haar, die grünen Augen, die Brille und die wütende rote Blitznarbe.

Hätte er zwanzig Jahre weniger Erfahrung gehabt, hätte er vielleicht geblinzelt. Stattdessen spuckte er einfach etwas aus, das er vor dem Jungen, der lebte wahrscheinlich nicht sagen sollte.

„Er, er“, sagte die schwankende Stimme des Jungen, sein junges Gesicht sah verängstigt und erschöpft aus, und Tränen liefen ihm noch immer über die Wangen, „er entführte mich, damit ich meinen Patronus zaubere... er sagte, er würde mich töten, wenn ich es nicht täte... aber ich konnte ich ihn Sie nicht einfach töten lassen..."

Bahrys Verstand war immer noch benommen, aber die Dinge fingen langsam an, Sinn zu ergeben.

Harry Potter, der einzige Zauberer, der je einen tödlichen Fluch überlebt hat. Bahry wäre vielleicht in der Lage gewesen, dem grünen Tod auszuweichen, er hatte es ganz klar versucht, aber wenn die Angelegenheit vor dem Zaubergamot zur Sprache käme, würden sie entscheiden, dass es sich um eine lebenslange Schuld gegenüber einem Adelshaus handelte.

„Ich verstehe“, sagte Bahry in einem viel sanfteren Knurren. Er begann, auf den Jungen zuzugehen. „Es tut mir leid, was du durchgemacht hast mein Sohn, aber du musst den Umhang fallen lassen und deinen Zauberstab ablegen."

Der Rest von Harry Potter tauchte aus der Unsichtbarkeit auf und die schweißgetränkten blauen Hogwarts-Gewänder wurden sichtbar und seine rechte Hand, die einen elf Zoll langen Stechpalmen-Zauberstab umklammerte, so fest, dass seine Knöchel weiß waren.

„Dein Zauberstab“, wiederholte Bahry.

„Entschuldigung“, flüsterte der elfjährige Junge, „hier“, und er hielt Bahry den Zauberstab entgegen.

Bahry konnte sich kaum davon abhalten, den traumatisierten Jungen, der ihm gerade das Leben gerettet hatte, anzuschnauzen. Stattdessen setzte er sich mit einem Seufzer über den Impuls hinweg und streckte nur eine Hand aus, um den Zauberstab zu nehmen. „Schau, mein Sohn, du solltest \emph{wirklich} nicht mit dem Zauberstab auf...„

Das Ende des Zauberstabs drehte sich leicht unter Bahrys Hand, gerade als der Junge „\emph{Somnium}“ flüsterte.

Harry starrte auf den zusammengefallenen Körper des Aurors, er fühlte keinen Triumph, nur ein erdrückendes Gefühl der Verzweiflung.

(Selbst da war es vielleicht noch nicht zu spät).

Harry drehte sich um, um zu der grünen Schlange, die regungslos dalag, zu schauen.

„\emph{Lehrer?} „zischte Harry. „\emph{Freund? Bitte, sind Sie am Leben?}“ Eine schreckliche Angst ergriff Harrys Herz; in diesem Moment hatte er völlig vergessen, dass er gerade gesehen hatte, wie der Verteidigungsprofessor einen Polizisten töten wollte.

Harry richtete seinen Zauberstab auf die Schlange, und seine Lippen begannen schon, das Wort „\emph{Rennervate}“ zu formen, bevor sein Gehirn ihn einholte und ihn anschrie.

Er wagte es nicht, bei Professor Quirrell Magie anzuwenden.

Harry hatte ihn gespürt, den brennenden, reißenden Schmerz in seinem Kopf, als ob sein Gehirn kurz davor wäre, sich in zwei Hälften zu teilen. Er hatte es gefühlt, seine Magie und die Magie von Professor Quirrell, zusammengeführt und anti-harmonisiert, würde das ihr Verderben sein. Das war die mysteriöse, schreckliche Sache, die passieren würde, wenn Harry und Professor Quirrell sich jemals zu nahe kommen würden, oder wenn sie jemals Magie aufeinander wirken würden, oder wenn \emph{sich ihre Zaubersprüche jemals berühren würden}, ihre Magie würde in unkontrollierbare Schwingungen versetzt -

Harry starrte die Schlange an, er konnte nicht sagen, ob sie atmete.

(Die letzten Sekunden vergingen wie im Flug).

Er drehte sich um und starrte den Auror an, der den Jungen, der lebte, gesehen hatte, der es wusste.

Das ganze Ausmaß der Katastrophe traf Harry wie tausend Hundert-Tonnen-Gewichte, es war ihm gelungen, den Auror zu betäuben, aber jetzt blieb ihm nichts mehr zu tun, keine Möglichkeit, sich zu erholen, die Mission war gescheitert, alles war gescheitert, \emph{er} hatte versagt.

Schockiert, bestürzt, verzweifelt, \emph{er hatte nicht daran gedacht}, sah nicht das Offensichtliche, erinnerte sich nicht daran, woher die hoffnungslosen Gefühle kamen, erkannte nicht, dass er den Wahren Patronuszauber erneut zaubern musste.

(Und dann war es schon zu spät.)

Auror Li und Auror McCusker hatten ihre Stühle am Tisch umgestellt, und so sahen sie beide gleichzeitig das nackte, skelettdünne Grauen, das aufstieg, um vor dem Fenster zu schweben, die Kopfschmerzen, die sie bereits von seinem Anblick bekamen.

Beide hörten die Stimme, als hätte ein längst toter Leichnam Worte gesprochen, und diese Worte selbst wären gealtert und gestorben.

Die Rede des Dementors schmerzte in ihren Ohren, als er sagte: „Bellatrix Black ist nicht mehr in ihrer Zelle“.

Einen Sekundenbruchteil lang herrschte entsetztes Schweigen, und dann riss sich Li aus seinem Stuhl und stürmte auf den Kommunikator zu, um Verstärkung aus dem Ministerium anzufordern, während McCusker nach seinem Spiegel griff und verzweifelt versuchte, die drei Auroren, die auf Patrouille gegangen waren, zu alarmieren.

