

\hypertarget{vorgeben-weise-zu-sein—teil-1}{% \section{7. Vorgeben weise zu sein - Teil 1}\label{vorgeben-weise-zu-sein—teil-1}}

—\/-\/-\/-\/- Kapitel 39: Vorgeben weise zu sein, Teil 1 -\/-\/-\/-\/-

Pfeif. Tick. Bzzzt. Ding. Glup. Pop. Platsch. Kling. Tut. Puff. Bimm. Blub. Piep. Klonk. Knister. Wusch. Zisch. Pffft. Surr.

Professor Flitwick hatte Harry an diesem Montag während der Zauberkunststunde schweigend ein gefaltetes Pergament übergeben und in der Notiz stand, dass Harry den Schulleiter wann es ihm passte aufsuchen sollte, aber so, dass niemand, vor allem nicht Draco Malfoy oder Professor Quirrell, etwas davon mitbekam. Sein Einwegpasswort für den Wasserspeier lautet `zimperlicher Bartgeier' *. Daneben war noch eine erstaunlich künstlerische Zeichnung mit Tinte von einem ihn streng anstarrenden Professor Flitwick, dessen Augen gelegentlich blinzelten; und am Ende der Notiz, dreimal unterstrichen, stand der Satz:

GERATEN SIE NICHT IN SCHWIERIGKEITEN

Und so hatte Harry die Verwandlungsstunde beendet und mit Hermine gelernt, zu Abend gegessen und mit seinen Leutnants gesprochen und schließlich, als die Uhr neun Uhr geschlagen hatte, sich unsichtbar gemacht und war auf sechs Uhr zurückgesprungen und hatte sich müde in Richtung des Wasserspeier, der Wendeltreppe, der Holztür, des Raums voller kleiner ausgetüftelter Dinge und der silberbärtigen Gestalt des Schulleiters geschleppt.

Diesmal sah Dumbledore ziemlich ernst aus, das übliche Lächeln fehlte; und er trug einen Pyjama in einem dunkleren und nüchterneren Lila als sonst.

„Danke, dass du gekommen bist, Harry“, sagte der Schulleiter. Der alte Zauberer stand von seinem Thron auf und begann langsam durch den Raum mit ~seinen seltsamen Geräte zu schreiten. „Erstens, hast du die Notizen von der gestrigen Begegnung mit Lucius Malfoy dabei?“

„Notizen?“, platzte Harry heraus.

„Sicherlich hast du es aufgeschrieben…“, sagte der alte Zauberer und seine Stimme verstummte.

Harry fühlte sich ziemlich verlegen. Ja, wenn man gerade ein mysteriöses Gespräch voller wichtiger Hinweise hinter sich hätte, die man nicht verstanden hat, war es verdammt offensichtlich alles sofort danach aufzuschreiben, bevor die Erinnerung verblasst ist, damit man später versuchen könnte, sich einen Reim darauf zu machen.

„In Ordnung“, sagte der Schulleiter, „dann also aus dem Gedächtnis.“

Harry rezitierte kleinlaut, so gut er eben konnte, und kam fast zur Hälfte des Gesprächs, bevor er bemerkte, dass es nicht klug war dem möglicherweise verrückten Schulleiter alles zu erzählen, zumindest nicht, ohne zuerst darüber nachzudenken; andererseits war Lucius definitiv einer von den bösen Jungs und Dumbledores Gegner, also war es wahrscheinlich eine gute Idee, es ihm zu sagen, und Harry hatte bereits angefangen zu reden und es war zu spät, um jetzt noch zu versuchen, die Dinge auszurechnen…

Harry beendete seine Nacherzählung ehrlich.

Dumbledores Gesicht war unnahbarer geworden als Harry weitermachte und am Ende sah er sehr alt aus und Ernst lag in der Luft.

„Nun“, sagte Dumbledore. „Ich schlage vor, dass du alles daran setzt, dass der Erbe von Malfoy nicht zu Schaden kommt. Und ich werde das Gleiche tun.“ Der Schulleiter runzelte die Stirn, seine Finger trommelten lautlos durch die tiefschwarze Oberfläche einer Platte, auf der das Wort Leliel stand. „Und ich denke, es wäre äußerst weise von dir, von nun an jede Interaktion mit Lord Malfoy zu vermeiden.“

„Haben Sie die Eulen von ihm an mich abgefangen?“ fragte Harry.

Der Schulleiter blickte Harry für einen langen Moment an und nickte dann widerstrebend.

Aus irgendeinem Grund war Harry nicht so empört, wie er es hätte sein sollen. Vielleicht war es nur so, dass Harry es gerade sehr leicht fiel, sich mit dem Standpunkt des Schulleiters zu identifizieren. Sogar Harry konnte verstehen, warum Dumbledore nicht wollte, dass er mit Lucius Malfoy interagierte; es wirkte nicht wie eine böse Tat.

Im Gegensatz dazu, dass der Schulleiter Zabini erpresste… wofür sie nur Zabinis Wort hatten und Zabini war absolut nicht vertrauenswürdig, tatsächlich gab es kaum einen sichtbaren Grund, warum Zabini nicht einfach die Geschichte erzählen würde, die ihm die meiste Sympathie von Professor Quirrell einbrachte…

„Wie wäre es, wenn ich, anstatt zu protestieren, sage, dass ich Ihren Standpunkt verstehe“, sagte Harry, „und Sie fangen weiterhin meine Eulen ab, aber Sie sagen mir, von wem sie waren?“

„Ich habe leider sehr viele Eulen für dich abgefangen“, sagte Dumbledore nüchtern. „Du bist eine Berühmtheit, Harry, und du würdest täglich Dutzende von Briefen erhalten, einige von weit außerhalb dieses Landes, wenn ich sie nicht zurückweisen würde.“

„Das“, sagte Harry und jetzt doch ein wenig empört, „scheint ein wenig zu weit zu gehen —“

„Viele dieser Briefe“, sagte der alte Zauberer leise, „werden dich um Dinge bitten, die du nicht geben kannst. Ich habe sie natürlich nicht gelesen, sondern die nicht zugestellten Briefe an ihre Absender zurückgeschickt. Aber ich weiß es, denn ich empfange sie auch. Und du bist zu jung, Harry, um dir jeden Morgen vor dem Frühstück sechs Mal das Herz brechen zu lassen.“

Harry sah auf seine Schuhe herab. Er sollte darauf bestehen, die Briefe zu lesen und selbst zu urteilen, aber… in ihm gab es eine leise Stimme des gesunden Menschenverstands und sie schrie gerade sehr laut.

„Danke“, murmelte Harry.

„Der andere Grund, warum ich dich hierher gebeten habe“, sagte der alte Zauberer, „war, dass ich deine einzigartige Begabung konsultieren wollte.“

„Verwandlung?“, fragte Harry überrascht und geschmeichelt.

„Nein, nicht diese einzigartige Begabung“, sagte Dumbledore. „Sag mir, Harry, welches Übel könntest du anstellen, wenn ein Dementor auf das Gelände von Hogwarts gelassen würde?“

Es stellte sich heraus, dass Professor Quirrell darum gebeten hatte, dass seine Schüler ihre Fähigkeiten gegen einen echten Dementor testen sollten, nachdem sie die Worte und Gesten zum Patronus-Zauber gelernt hatten.

„Professor Quirrell ist nicht in der Lage, den Patronus-Zauber selbst zu wirken“, sagte Dumbledore, als er langsam zwischen den Geräten hindurch ging. „Was nie ein gutes Anzeichen ist. Aber andererseits hat er mir diese Tatsache freiwillig erzählt, während er verlangte, dass externe Lehrpersonen hinzugezogen werden, um jedem Schüler, der ihn lernen wollte, den Patronus-Zauber beizubringen; er hat angeboten, die Kosten selbst zu tragen, wenn ich es nicht wollte. Das hat mich sehr beeindruckt. Aber jetzt besteht er darauf, einen Dementor einzubringen —“

„Schulleiter“, sagte Harry leise, „Professor Quirrell hält viel von Einsatztests unter realen Kampfbedingungen. Der Wunsch, einen echten Dementor herzuholen ist für ihn völlig charakteristisch.“

Da warf der Schulleiter Harry einen seltsamen Blick zu.

„Charakteristisch? “, fragte der alte Zauberer.

„Ich meine“, sagte Harry, „es ist völlig im Einklang mit der Art und Weise, wie Professor Quirrell normalerweise handelt…“ Harrys Stimme wurde schwächer. Warum hatte er es so ausgedrückt?

Der Schulleiter nickte. „Also hast du den gleichen Eindruck wie ich: Dass es eine Ausrede ist. Eine sehr vernünftige Ausrede, das auf jeden Fall; mehr noch, als du vielleicht denkst. Oftmals gelingt es Zauberern, die scheinbar nicht in der Lage sind, einen Patronus-Zauber zu wirken, genau das in Anwesenheit eines echten Dementors, wobei aus nicht mal einem Flimmern von Licht ein völlig gestaltlicher Patronus wird. Warum das so sein könnte, weiß niemand; aber so ist es.“

Harry runzelte die Stirn. „Dann verstehe ich wirklich nicht, warum Sie misstrauisch sind —“

Der Schulleiter breitete seine Hände wie in Hilflosigkeit aus. „Harry, der Verteidigungsprofessor hat mich gebeten, die dunkelste aller Kreaturen durch die Tore von Hogwarts zu führen. Ich muss misstrauisch sein.“ Der Schulleiter seufzte. „Und doch wird der Dementor bewacht, gesichert, in einem machtvollen Käfig, ich selbst werde da sein, um ihn jederzeit zu beobachten—ich kann mir nicht vorstellen, was für ein Übel angerichtet werden könnte. Aber vielleicht bin ich einfach nicht in der Lage es zu sehen. Und darum frage ich dich.“

Harry starrte den Schulleiter mit offenem Mund an. Er war so schockiert, dass er sich nicht einmal geschmeichelt fühlen konnte.

„Mich?“, sagte Harry.

„Ja“, sagte Dumbledore und lächelte leicht. „Ich versuche mein Bestes, um meinen Feinden zuvorzukommen, ihren verdorbenen Verstand zu begreifen und ihre bösen Gedanken vorherzusagen. Aber ich hätte mir nie vorstellen können, die Knochen eines Hufflepuffs zu Waffen zu schärfen.“

Würde jemals Gras über diese Sache wachsen?

„Schulleiter“, sagte Harry müde, „Ich weiß, dass es nicht gut klingt, aber in allem Ernst: Ich bin nicht böse, ich bin nur sehr kreativ —“

„Ich habe nicht gesagt, dass du böse bist“, sagte Dumbledore ernsthaft. „Es gibt diejenigen, die sagen, dass das Bösen zu verstehen bedeutet, böse zu werden; aber sie geben nur vor, weise zu sein. Vielmehr ist es das Böse, das die Liebe nicht kennt und es nicht wagt, sich die Liebe vorzustellen; und es kann die Liebe auch nie verstehen, ohne aufzuhören, böse zu sein. Und ich vermute, dass du dich besser in die Köpfe der Dunklen Magier hineinversetzen kannst als ich es je könnte und trotzdem Liebe verstehst. Also, Harry.“ Die Augen des Schulleiters waren entschlossen. „Wenn du in Professor Quirrells Position wärst, welche Untaten könntest du vollbringen, nachdem du mich dazu gebracht hast, einem Dementor das Betreten des Geländes von Hogwarts zu erlauben?“

„Moment mal“, sagte Harry und schleppte sich etwas benommen auf den Stuhl vor dem Schreibtisch des Schulleiters und setzte sich. Diesmal war es ein großer und bequemer Stuhl, kein Holzhocker, und Harry konnte sich geborgen fühlen, als er sich hineinsetzte.

Dumbledore bat ihn, Professor Quirrell zu überlisten.

Punkt eins: Harry mochte Professor Quirrell eher als Dumbledore.

Punkt zwei: Die Hypothese war, dass der Verteidigungsprofessor plante, etwas Böses zu tun, und in diesem konjunktiven Fall sollte Harry dem Schulleiter helfen, es zu verhindern.

Punkt drei…

„Schulleiter“, sagte Harry, „wenn Professor Quirrell dabei ist etwas zu tun, bin ich mir nicht sicher, ob ich ihn überlisten kann. Er hat viel mehr Erfahrung als ich.“

Der alte Zauberer schüttelte den Kopf und schaffte es trotz seines Lächelns irgendwie, sehr feierlich zu wirken. „Du unterschätzt dich selbst.“

Das war das erste Mal, dass jemand das zu Harry sagte.

„Ich erinnere mich“, fuhr der alte Zauberer fort, „an einen jungen Mann in diesem Büro, kalt und kontrolliert, als er dem Leiter des Hauses Slytherin gegenüberstand und seinen eigenen Schulleiter erpresste, um seine Klassenkameraden zu schützen. Und ich glaube, dass dieser junge Mann gerissener ist als Professor Quirrell, gerissener als Lucius Malfoy, dass er eines Tages Voldemort ebenbürtig sein wird. Er ist es, den ich um Rat bitten möchte.“

Harry unterdrückte das Kältegefühl, das ihn bei dem Namen durchlief, und runzelte nachdenklich die Stirn, als er den Schulleiter ansah.

Wie viel weiß er…?

Der Schulleiter hatte Harry im Griff seiner mysteriösen dunklen Seite gesehen, so tief, wie Harry jemals in sie gesunken war. Harry erinnerte sich noch daran, wie es gewesen war, zuzusehen, unsichtbar und in der Zeit zurückgekehrt, als sein vergangenes Selbst die älteren Slytherins konfrontiert hatte; der Junge mit der Narbe auf der Stirn, der sich nicht wie die anderen verhielt. Natürlich würde der Schulleiter etwas Seltsames an dem Jungen in seinem Büro bemerkt haben…

Und Dumbledore war zu dem Schluss gekommen, dass sein persönlicher, kleiner Held gerissen genug war, um seinem vorherbestimmten Gegner, dem Dunklen Lord, gleichzukommen.

Das war nicht sehr viel verlangt, wenn man bedachte, dass der Dunkle Lord ein deutlich sichtbares Dunkles Mal auf dem Arm jedes seiner Dieners gelegt hatte und dass er das gesamte Kloster abgeschlachtet hatte, das die Kampfkunst unterrichtete, die er lernen wollte.

Gerissen genug zu sein, um Professor Quirrell gleichzukommen, würde eine ganz andere Art von Problem sein.

Aber es war auch klar, dass der Schulleiter nicht zufrieden wäre, bis Harry nicht kalt und düster wurde und eine Art Antwort fand, die beeindruckend gerissen klang… die besser nicht dazu führte, dass Professor Quirrell Verteidigung tatsächlich nicht mehr unterrichten würde…

Und natürlich würde Harry zu seiner dunklen Seite übergehen und es aus dieser Richtung durchdenken, nur um ehrlich zu sein und nur für alle Fälle.

Erzählen Sie mir„, sagte Harry, “alles darüber, wie der Dementor hereingebracht und wie er bewacht werden soll."

Dumbledores Augenbrauen hoben sich für einen Moment und dann begann der alte Zauberer zu sprechen.

Der Dementor würde von einem Auroren-Trio auf das Gelände von Hogwarts transportiert werden, von dem alle drei dem Schulleiter persönlich bekannt und alle drei in der Lage waren, einen gestaltlichen Patronus zu wirken. Sie würden am Rande des Geländes von Dumbledore empfangen werden, der den Dementor durch die Schutzzauber von Hogwarts passieren lassen würde—

Harry fragte, ob die Passage dauerhaft oder nur vorübergehend möglich sei—ob jemand einfach den gleichen Dementor am nächsten Tag wieder hereinführen könne.

Die Passage war vorübergehend (antwortete der Schulleiter mit einem anerkennenden Nicken) und fuhr mit der Erklärung fort: Der Dementor befand sich in einem Käfig aus massiven Titanstäben, nicht verwandelt, sondern wirklich geschmiedet; mit der Zeit würde die Anwesenheit eines Dementors dieses Metall zu Staub zerfressen, aber nicht an einem einzigen Tag.

Schüler, die darauf warteten dranzukommen, würden sich im Hintergrund halten, hinter zwei gestaltlichen Patroni, die von zwei der drei Auroren zu jeder Zeit aufrechterhalten würden. Dumbledore würde am Käfig des Dementors mit seinem Patronus warten. Ein einzelner Schüler würde sich dem Dementor nähern; und Dumbledore würde seinen Patronus auflösen; und der Schüler würde versuchen, seinen eigenen Patronus-Zauber zu wirken; und wenn er versagte, würde Dumbledore seinen Patronus wiederherstellen, bevor der Schüler einen bleibenden Schaden erleiden könnte. Der ehemalige Meisterduellant Professor Flitwick wäre auch anwesend solange Schüler in der Nähe waren, nur um das Maß an Sicherheit zu erhöhen.

„Warum warten nur Sie beim Dementor?“, fragte Harry. „Ich meine, sollten es nicht Sie plus einem Auror —“

Der Schulleiter schüttelte den Kopf. „Sie könnten es nicht verkraften jedes Mal dem Dementor ausgesetzt zu sein, wenn ich meinen Patronus auflöste.“

Und wenn Dumbledores Patronus aus irgendeinem Grund versagte, während einer der Schüler noch in der Nähe des Dementors war, würde der dritte Auror einen anderen gestaltlichen Patronus erschaffen und ihn zum Schutz des Schülers zu schicken…

Harry hakte und fragte nach, aber er konnte keinen Fehler im Sicherheitssystem finden.

Also atmete Harry tief durch, sank weiter in den Stuhl, schloss die Augen und erinnerte sich:

„Und das macht… fünf Punkte Abzug? Nein, machen wir runde zehn Punkte Abzug von Ravenclaw daraus, fürs Widersprechen.“

Die Kälte kam jetzt langsamer, widerstrebender, Harry hatte sich in letzter Zeit nicht sonderlich auf seine dunkle Seite gestützt…

Harry musste die ganze Stunde Zaubertränke in seinem Kopf durchlaufen lassen, bevor sein Blut zur tödlich kristalliner Klarheit abkühlte.

Und dann dachte er an den Dementor.

Und es war offensichtlich.

„Der Dementor ist eine Ablenkung“, sagte Harry. Die Kälte war deutlich in seiner Stimme, denn das war es, was Dumbledore wollte und erwartete. „Eine große, auffällige Bedrohung, aber im Endeffekt einfach und leicht sich gegen sie zu verteidigen. Während sich also Ihre ganze Aufmerksamkeit auf den Dementor konzentriert, wird die eigentliche Intrige anderswo stattfinden.“

Dumbledore starrte Harry für einen Moment an und nickte dann langsam. „Ja…“ sagte der Schulleiter. „Und ich glaube, ich weiß, wovon es eine Ablenkung sein könnte, wenn Professor Quirrell böses vorhat… danke, Harry.“

Der Schulleiter starrte Harry immer noch an, ein seltsamer Blick in diesen alten Augen.

„Was?“, fragte Harry mit einem Hauch von Verärgerung, die Kälte verweilte immer noch in seinem Blut.

„Ich habe eine weitere Frage an diesen jungen Mann“, sagte der Schulleiter. „Es ist etwas, worüber ich mir schon lange Gedanken gemacht habe, das ich aber nicht verstehen konnte. Warum? “ Es klang ein Hauch Schmerz in seiner Stimme mit. „Warum sollte sich jemand selbst absichtlich zum Monster machen? Warum Böses um des Bösen willen tun? Warum Voldemort?“

Surr, Bzzzt, Tick; Ding, Puff, Platsch…

Harry starrte den Schulleiter überrascht an.

„Woher soll ich das wissen?“, sagte Harry. „Sollte ich den Dunklen Lord auf magische Weise verstehen, weil ich der Held bin, oder so?“

„Ja!“ sagte Dumbledore. „Mein eigener großer Feind war Grindelwald und ihn habe ich sehr gut verstanden. Grindelwald war mein dunkles Spiegelbild, der Mann, der ich so leicht hätte sein können, wenn ich der Versuchung nachgegeben hätte zu glauben, dass ich ein guter Mensch sei und deshalb immer im Recht. Für das Größere Wohl, das war sein Wahlspruch; und er glaubte es wirklich, während er ganz Europa zerriss wie ein verwundetes Tier. Und ihn habe ich am Ende besiegt. Aber dann kam Voldemort, um alles zu zerstören, was ich in Großbritannien beschützt hatte.“ Der Schmerz war jetzt deutlich in Dumbledores Stimme zu hören und auf seinem Gesicht zu sehen. „Er hat um Längen schlimmere Taten begangen als selbst Grindelwalds schlimmstes, hat Schrecken verbreitete, nur um des Schreckens willen. Ich habe alles geopfert, nur um ihn zurückzuhalten, und ich verstehe immer noch nicht warum! Warum, Harry? Warum hat er das getan? Er war nie der mir zugedachte Gegner, sondern deiner, also wenn du überhaupt irgendwelche Vermutungen hast, Harry, dann sag es mir bitte! Warum?“

Harry starrte auf seine Hände herab. Die Wahrheit war, dass Harry sich noch nicht über den Dunklen Lord informiert hatte und im Moment hatte er nicht die geringste Ahnung. Und irgendwie schien das nicht die Antwort zu sein, die der Schulleiter hören wollte. „Zu viele dunkle Rituale, vielleicht? Am Anfang dachte er, es würde nur ein einziges sein, aber es opferte einen Teil seiner guten Seite und das machte ihn weniger abgeneigt, andere Dunkle Rituale durchzuführen, also machte er immer mehr Rituale in einer positiven Rückkopplungsschleife, bis er schließlich zu einem enorm mächtigen Monster wurde —“

„Nein!“ Nun klang die Stimme des Schulleiters gequält. „Ich kann das nicht glauben, Harry! Es muss mehr sein als nur das!“

Warum sollte es?, dachte Harry, aber er sagte es nicht, denn es war klar, dass der Schulleiter dachte, das Universum sei eine Geschichte und hatte eine Handlung, und dass riesige Tragödien nur aus ebenso großen wie wichtigen Gründen passieren durften. „Es tut mir leid, Schulleiter. Der Dunkle Lord erscheint mir nicht gerade wie ein dunkler Spiegel, überhaupt nicht. Es gibt nichts, was ich auch nur im Geringsten darin verlockend finde, die Haut von Yermy Wibbles Familie an eine Redaktionswand zu nageln.“

„Hast du keine Weisheit zu teilen?“, fragte Dumbledore. Nun Klang Flehen in der Stimme des alten Zauberers mit, fast schon Betteln..

Böse Dinge geschehen, dachte Harry, es hat keine Bedeutung und lehrt uns nichts, außer, dass wir nicht böse sein sollen? Der Dunkle Lord war wahrscheinlich nur ein egoistisches Schwein, dem es egal war, wen er verletzte, oder ein Idiot, der vermeidbare Fehler aus Dummheit machte, die außer Kontrolle liefen. Es steckt kein Schicksal hinter den Übeln dieser Welt; wenn Hitler in die Architekturschule aufgenommen worden wäre, wie er es gewollte hatte, wäre die ganze Geschichte Europas anders gewesen; wenn wir in einer Art Universum leben würden, in dem schreckliche Dinge nur aus guten Gründen geschehen durften, würden sie überhaupt nicht passieren.

Und nichts davon war offensichtlich das, was der Schulleiter hören wollte.

Der alte Zauberer sah Harry immer noch über eines der ausgetüftelten Geräte hinweg an, wie eine gefrorene Rauchwolke, eine schmerzhafte Verzweiflung in diesen alten, erwartungsvollen Augen.

Nun, es war nicht schwer, weise zu klingen. Es war viel einfacher, als intelligent zu sein, denn man musste nichts Überraschendes sagen oder sich neue Erkenntnisse einfallen lassen. Man musste einfach nur die Mustererkennungs-Software seines Gehirns das Klischee vervollständigen lassen und dabei irgendeine bedeutungsvolle Weisheit verwenden, die man schon einmal gehört hatte.

„Schulleiter“, sagte Harry feierlich, „Ich möchte mich lieber nicht durch meine Feinde definieren lassen.“

Irgendwie, selbst inmitten all des Surrens und Tickens, herrschte eine Art Stille.

Das war etwas bedeutungsvoller herausgekommen, als Harry es sich vorgestellt hatte.

„Du bist vielleicht sehr weise, Harry…“, sagte der Schulleiter langsam. „Ich wünschte… dass ich durch meine Freunde definiert worden wäre.“ Der Schmerz in seiner Stimme war größer geworden.

Harrys Verstand versuchte schnell etwas anderes bedeutungsvoll weises zu finden, dass die unbeabsichtigte Kraft des Schlages mildern würde—

„Oder vielleicht“, sagte Harry sanfter, „ist es der Feind, der den Gryffindor formt, genau wie es der Freund ist, der den Hufflepuff prägt, und der Ehrgeiz, der den Slytherin ausmacht. Ich weiß jedenfalls, dass es immer, in jeder Generation, das Rätsel ist, das den Wissenschaftler definiert.“

„Es ist ein schreckliches Schicksal, zu dem du mein Haus verurteilst, Harry“, sagte der Schulleiter. Der Schmerz lag noch in seiner Stimme. „Denn nun, da du es gesagt hast, merke ich, dass ich sehr wohl von meinen Feinden geprägt wurde.“

Harry starrte auf seine eigenen Hände, wie sie in seinem Schoß lagen. Vielleicht sollte er einfach die Klappe halten, während er noch vorne lag.

„Aber du hast meine Frage beantwortet“, sagte Dumbledore leiser, wie zu sich selbst. „Ich hätte erkennen müssen, dass das der Schlüssel eines Slytherins sein würde. Für seinen Ehrgeiz, alles um seines Ehrgeizes willen; und das weiß ich, aber nicht warum…“ Eine Zeit lang starrte Dumbledore ins Nichts; dann richtete er sich auf, und seine Augen schienen sich wieder auf Harry zu konzentrieren.

„Und du, Harry“, sagte der Schulleiter, „du nennst dich selbst einen Wissenschaftler?“ Seine Stimme war von Überraschung und leichter Missbilligung durchdrungen.

„Sie mögen keine Wissenschaft?“, fragte Harry ein wenig erschöpft. Er hatte gehofft, Dumbledore würden Muggeldinge besser gefallen.

„Ich nehme an, es ist nützlich für diejenigen ohne Zauberstab“, sagte Dumbledore und runzelte die Stirn. „Aber es scheint eine seltsame Sache zu sein, mit der man sich selbst definieren kann. Ist Wissenschaft so wichtig wie die Liebe? Wie Güte? Wie Freundschaft? Ist es die Wissenschaft, die dich Minerva McGonagall mögen lässt? Ist es die Wissenschaft, die dich dazu bringt, dich für Hermine Granger zu interessieren? Wird es die Wissenschaft sein, die du heranziehen wirst, wenn du versuchst, Wärme in Draco Malfoys Herzen zu entfachen?“

Wissen Sie, die traurige Sache ist, Sie denken wahrscheinlich, dass Sie gerade eine Art unglaublich kluges Totschlagargument geäußert haben.

Nun denn, wie formulierte man die Erwiderung so, dass sie auch unglaublich weise klang…

„Sie sind kein Ravenclaw“, sagte Harry mit ruhiger Würde, „und darum ist es Ihnen vielleicht nicht in den Sinn gekommen, dass die Wahrheit zu respektieren und sie alle Tage deines Lebens zu suchen, auch ein Akt der Gnade sein könnte.“

Die Augenbrauen des Schulleiters hoben sich. Und dann seufzte er. „Wie bist du nur so weise geworden, in so jungen Jahren….?“ Der alte Zauberer klang traurig, als er es sagte. „Vielleicht wird es sich für dich als wertvoll erweisen.“

Nur um alte Zauberer zu beeindrucken, die übermäßig von sich selbst beeindruckt sind, dachte Harry. Er war eigentlich ein wenig enttäuscht von Dumbledores Leichtgläubigkeit; es war nicht so, dass Harry gelogen hatte, aber Dumbledore schien viel zu beeindruckt von Harrys Fähigkeit, Dinge so zu formulieren, dass sie tiefgründig klangen, anstatt sie im Klartext zu formulieren, wie Richard Feynman es mit seiner Weisheit getan hatte…

„Liebe ist wichtiger als Weisheit“, sagte Harry, nur um die Grenzen von Dumbledores Toleranz gegenüber Klischees, die selbst einer Bilder erkennen würde, zu testen, ergänzt durch schiere Mustererkennung ohne jegliche tiefergehende Analyse.

Der Schulleiter nickte ernsthaft und sagte: „In der Tat.“

Harry stand aus dem Stuhl auf und streckte seine Arme aus. Nun, dann gehe ich besser los und liebe etwas, das mir helfen wird, den Dunklen Lord zu besiegen. Und wenn Sie mich das nächste Mal um Rat fragen, werde ich Sie einfach umarmen—

„Heute hast du mir sehr geholfen, Harry“, sagte der Schulleiter. „Und da ist noch eine letzte Sache, die ich diesen jungen Mann fragen möchte.“

Na toll.

„Sag es mir, Harry“, sagte der Schulleiter (und jetzt klang seine Stimme einfach verwirrt, obwohl immer noch eine Spur von Schmerz in seinen Augen lag), „warum fürchten Dunkle Zauberer den Tod so sehr?“

„Äh,“ sagte Harry, „Entschuldigung, aber ich muss die Dunklen Zauberer in diesem Punkt unterstützen.“

Wusch, Zisch, Bimm; Glup, Pop, Blub—

„Wie bitte?“, fragte Dumbledore.

„Der Tod ist schlecht“, sagte Harry und verwarf Weisheit um der klaren Kommunikation willen. „Sehr schlecht. Extrem schlecht. Angst vor dem Tod zu haben, ist wie Angst vor einem großen Monster mit giftigen Reißzähnen zu haben. Es macht tatsächlich sehr viel Sinn und deutet nicht darauf hin, dass man ein psychologisches Problem hat.“

Der Schulleiter starrte ihn an, als ob er sich gerade in eine Katze verwandelt hätte.

„Also gut“, sagte Harry, „lassen Sie es mich so ausdrücken. Wollen Sie sterben? Denn wenn ja, dann gibt es da diese Muggelsache, die wird Hotline zur Selbstmordprävention genannt —“

„Wenn es soweit ist“, sagte der alte Zauberer leise. „Nicht früher. Ich würde nie versuchen, den Tag zu beschleunigen, noch versuchen, ihn abzulehnen, wenn er kommt.“

Harry runzelte die Stirn. „Das klingt nicht so, als hätten Sie einen sehr starken Lebenswillen, Schulleiter!“

„Harry…“ Die Stimme des alten Zauberers begann, ein wenig hilflos zu klingen; und er war zu einer Stelle gegangen, an der sein silberner Bart, unbemerkt, in ein kristallines Goldfischglas gekommen war und langsam eine grünliche Färbung annahm, die die Haare hochkroch. „Ich denke, ich habe mich vielleicht nicht klar genug ausgedrückt. Dunkle Zauberer sind nicht froh am Leben zu sein. Sie fürchten den Tod. Sie greifen nicht nach dem Licht der Sonne, sondern fliehen vor dem Einbruch der Nacht in unendlich dunkle, selbst geschaffene Höhlen, ohne Mond und Sterne. Es ist nicht das Leben, das sie sich wünschen, sondern die Unsterblichkeit; und sie sind so getrieben diese zu erreichen, dass sie ihre Seelen opfern würden! Würdest du ewig leben wollen, Harry?“

„Ja und Sie auch“, sagte Harry. „Ich will noch einen Tag leben. Morgen werde ich noch einen weiteren Tag leben wollen. Deshalb möchte ich für immer leben, ein Beweis durch vollständige Induktion auf den natürlichen Zahlen. Wenn man nicht sterben will, bedeutet das, dass man für immer leben will. Wenn man nicht für immer leben will, bedeutet das, dass man sterben will. Man kann nur das eine oder andere… Ich komme hier nicht durch, oder?“

Die beiden Parteien starrten sich über eine gewaltige Kluft der Inkommensurabilität** an.

„Ich habe hundertzehn Jahre gelebt“, sagte der alte Zauberer leise (nahm seinen Bart aus der Schale und schüttelte ihn, um die Farbe herauszubekommen). „Ich habe sehr viele Dinge gesehen und getan, zu viele, von denen ich wünschte, ich hätte sie nie gesehen oder getan. Und doch bereue ich es nicht, lebendig zu sein, denn meine Schüler heranwachsen zu sehen, ist für mich eine Freude und ich habe mich noch nicht daran sattgesehen. Aber ich möchte nicht so lange leben, bis das der Fall ist! Was würdest du tun, mit einem ewigen Leben, Harry?“

Harry atmete tief durch. „Alle interessanten Menschen auf der Welt treffen, alle guten Bücher der Welt lesen und dann etwas noch besseres schreiben, den zehnten Geburtstag meines ersten Enkelkindes auf dem Mond feiern, den hundersten Geburtstag meines ersten Ur-Ur-Ur-Enkels auf den Ringen des Saturn feiern, die tiefgreifendsten und endgültigsten Gesetze der Natur lernen, die Natur des Bewusstseins verstehen, herausfinden, warum es alles überhaupt erst gibt, andere Sterne besuchen, Außerirdische entdecken, Außerirdische erschaffen, sich mit jedem den es gibt als Gruppe auf der anderen Seite der Milchstraße treffen, sobald wir die ganze Sache erkundet haben, sich mit allen denen treffen, die auf der Alten Erde geboren wurden, um zu sehen, wie die Sonne schließlich erlischt, und ich habe mir früher Sorgen darüber gemacht, einen Weg aus diesem Universum zu entkommen zu finden, bevor ihm die Negentropie*** ausgegangen ist, aber ich bin jetzt viel hoffnungsvoller, da ich entdeckt habe, dass die sogenannten Gesetze der Physik nur bestmögliche Richtlinien sind.“

„Ich habe nicht viel davon verstanden“, sagte Dumbledore. „Aber ich muss dich fragen, ob das Dinge sind, die du wirklich so unbedingt begehrst oder ob du sie dir nur so vorstellst, um dir nicht die Müdigkeit vorstellen zu müssen, während du vom Tod davonläufst und davonläufst.“

„Das Leben ist keine endliche Liste von Dingen, die man abhakt, bevor man sterben kann“, sagte Harry fest. „Es ist das Leben, man lebt es einfach weiter. Wenn ich diese Dinge nicht tue, dann nur, weil ich etwas Besseres gefunden habe.“

Dumbledore seufzte. Seine Finger trommelten auf einer Uhr; als sie sie berührten, verwandelten sich die Zahlen in eine unentzifferbare Schrift und die Zeiger erschienen kurzzeitig in verschiedenen Positionen. „Für den unwahrscheinlichen Fall, dass ich die Einhundertfünfzig erreichen darf“, sagte der alte Zauberer, „glaube ich nicht, dass es mir etwas ausmachen würde. Aber zweihundert Jahre wären zu viel des Guten.“

„Ja, nun“, sagte Harry, seine Stimme etwas trocken, als er an seine Mutter und seinen Vater und ihre zugeteilte Spanne dachte, falls Harry nichts dagegen unternahm, „ich vermute, Schulleiter, dass, wenn man aus einer Kultur kommt, in der die Menschen daran gewöhnt sind, vierhundert Jahre zu leben, das Sterben im Alter von Zweihundert genauso tragisch verfrüht erscheinen würde wie im Alter von, sagen wir, Achtzig.“ Harrys Stimme wurde hart, mit dem letzten Wort.

„Vielleicht“, sagte der alte Zauberer versöhnlich. „Ich möchte nicht vor meinen Freunden sterben und nicht weiterleben, nachdem sie alle entschwunden sind. Die schwerste Zeit ist, wenn diejenigen, die du am meisten geliebt hast, vor dir weitergegangen sind, aber andere noch leben, um derentwillen du bleiben musst…“ Dumbledores Augen waren auf Harry gerichtet und wurden traurig. „Trauere nicht zu sehr um mich, Harry, wenn meine Zeit gekommen ist; ich werde bei denen sein, die ich lange vermisst habe, bei unserem nächsten großen Abenteuer.“

„Oh!“, sagte Harry indem er begriff. „Sie glauben an ein Leben nach dem Tod. Ich hatte den Eindruck, dass Zauberer keine Religion haben?“

Tut. Piep. Klonk.

„Wie kannst du es nicht glauben?“, fragte der Schulleiter und sah völlig geschockt aus. „Harry, du bist ein Zauberer! Du hast Geister gesehen!“

„Geister“, sagte Harry, seine Stimme matt. „Sie meinen diese Dinger, so ähnlich wie Porträts, gespeicherte Erinnerungen und Verhaltensweisen ohne Bewusstsein oder Leben, die versehentlich durch den Ausbruch von Magie, der den gewaltsamen Tod eines Zauberers begleitet, in das umgebende Material eingeprägt wurden —“

„Ich habe diese Theorie gehört“, sagte der Schulleiter und seine Stimme wurde scharf, „wiederholt von Zauberern, die Zynismus mit Weisheit verwechseln, die denken, dass auf andere herabzuschauen bedeutet, sich selbst zu erheben. Es ist eine der absurdesten Ideen, die ich in einhundert und zehn Jahren gehört habe! Ja, Geister lernen und wachsen nicht, denn hier gehören sie nicht hin! Seelen sollen weitergehen, hier gibt für sie kein Leben mehr! Und wenn nicht Geister, was ist dann mit dem Schleier? Was ist mit dem Stein der Auferstehung?“

„In Ordnung“, sagte Harry und versuchte, seine Stimme ruhig zu halten, „Ich werde mir Ihre Beweise anhören, denn das ist es, was ein Wissenschaftler tut. Aber zuerst, Schulleiter, lassen Sie mich Ihnen eine kleine Geschichte erzählen.“ Harrys Stimme zitterte. „Wissen Sie, als ich hier ankam, als ich aus dem Zug von King's Cross stieg, ich meine nicht gestern, sondern im September, als ich damals aus dem Zug stieg, Schulleiter, hatte ich noch nie einen Geist gesehen. Ich habe keine Geister erwartet. Als ich sie sah, Schulleiter, tat ich etwas wirklich Dummes. Ich zog voreilige Schlüsse. Ich, ich dachte, es gäbe ein Jenseits, ich dachte, niemand sei jemals wirklich gestorben, ich dachte, dass jeden, den die menschliche Spezies jemals verloren hatte, in Wirklichkeit doch wohlauf sei, ich dachte, dass Zauberer mit Leuten reden könnten, die gestorben waren, dass es nur den richtigen Zauber brauchte, um sie zu beschwören, dass Zauberer das tun könnten, ich dachte, ich könnte meine Eltern treffen, die für mich gestorben waren, und ihnen sagen, dass ich von ihrem Opfer gehört hatte und dass ich angefangen hatte, sie meine Mutter und meinen Vater zu nennen —“.

„Harry“, flüsterte Dumbledore. Tränen glitzerte in den Augen des alten Zauberers. Er trat einen Schritt näher durch das Büro—

„Und dann“, spuckte Harry, die Wut, die nun seine Stimme erfüllte, die kalte Wut auf das Universum, weil es so war und auf sich selbst, weil er so dumm gewesen war, „fragte ich Hermine und sie sagte, dass es nur Nachbilder waren, die durch den Tod eines Zauberers in den Stein der Burg eingebrannt wurden, wie die Silhouetten, die an den Wänden von Hiroshima hinterlassen wurden. Und ich hätte es wissen müssen! Ich hätte es wissen müssen, ohne auch nur fragen zu müssen! Ich hätte es nicht einmal für alle dreißig Sekunden glauben sollen! Denn wenn Menschen Seelen hätten, gäbe es keine Hirnschäden, wenn Ihre Seele weiter sprechen könnte, nachdem Ihr ganzes Gehirn verschwunden ist, wie könnte eine Schädigung der linken zerebralen Hemisphäre Ihre Fähigkeit zu sprechen beeinträchtigen? Und Professor McGonagall, als sie mir erzählte, wie meine Eltern gestorben waren, tat sie nicht so, als wären sie gerade auf eine lange Reise in ein anderes Land gegangen, als wären sie in den Tagen der Segelschiffe nach Australien ausgewandert, so wie die Leute handeln würden, wenn sie tatsächlich wüssten, dass der Tod nur woanders hingehen ist, wenn sie felsenfeste Beweise für ein Jenseits hätten, anstatt sich etwas auszudenken, um sich selbst zu trösten, es würde alles verändern, es wäre egal, dass jeder im Krieg jemanden verloren hat, es wäre ein wenig traurig, aber nicht schrecklich! Und ich hatte schon gesehen, dass sich die Menschen in der Zauberwelt nicht so verhalten! Also hätte ich es besser wissen sollen! Und da wusste ich, dass meine Eltern wirklich tot waren und für immer und ewig fort waren, dass es nichts mehr von ihnen gab, dass ich nie die Gelegenheit hatte, sie zu treffen und, und, und die anderen Kinder dachten, ich würde weinen, weil ich Angst vor Geistern hatte —“

Das Gesicht des alten Zauberers war entsetzt, er öffnete seinen Mund, um zu sprechen—

„Also sagen Sie es mir, Schulleiter! Erzählen Sie mir von den Beweisen! Aber wagen Sie es nicht, auch nur ein kleines bisschen zu übertreiben, denn wenn Sie mir wieder falsche Hoffnung geben und ich später herausfinde, dass Sie gelogen oder übertrieben haben, auch wenn es nur ein wenig war, werde ich Ihnen nicht vergeben. Was ist der Schleier?“

Harry hob die Hand und wischte sich die Wangen ab, während die Glasgegenstände des Büros langsam aufhörten von seinem letzten Schrei zu vibrieren.

„Der Schleier“, sagte der alte Zauberer mit nur leichtem Zittern in der Stimme, „ist ein großer steinerner Torbogen, der im Mysteriumsabteilung aufbewahrt wird; ein Tor zum Land der Toten.“

„Und woher weiß man das?“, sagte Harry. „Sagen Sie mir nicht, was Sie glauben, sondern was Sie gesehen haben! “

Die physische Manifestation der Barriere zwischen den Welten war ein großer, alter und hoher Steinbogen, der zu einer scharfen Spitze zusammen lief, mit einem zerfetzten schwarzen Schleier, gleich der Oberfläche eines Wasserbeckens, der sich zwischen den Steinen spannte; der sicher immerzu kräuselte, von der ständigen und einseitigen Passage der Seelen. Wenn man am Schleier stand, konnte man die Stimmen der Toten hören, die riefen, immer nur flüsternd riefen, genau so, dass man sie nicht verstehen konnte, und die immer lauter und zahlreicher wurden, wenn man blieb und versuchte ihnen bei ihrem Versuch zu kommunizieren zuzuhören; und wenn man zu lange zuhörte, ging man auf sie zu und in dem Moment, in dem man den Schleier berührte, würde man durchgesaugt werden und für immer verschwunden sein.

„Das klingt nicht einmal nach einem interessanten Betrug“, sagte Harry, seine Stimme beruhigte sich, jetzt, da es nichts gab, was ihm Hoffnung machen könnte oder ihn wütend macht, weil eine Hoffnung zerschlagen wurde. „Jemand baute einen steinernen Torbogen, erschafft eine kleine schwarze wellige Oberfläche dazwischen, die alles, was sie berührte, verschwinden lässt, und verzauberte sie so, dass sie den Menschen zuflüstert und sie hypnotisiert.“

„Harry…“ sagte der Schulleiter und begann, ziemlich besorgt auszusehen. „Ich kann dir die Wahrheit sagen, aber wenn du dich weigerst, sie zu hören….“

Auch nicht interessant. „Was ist der Stein der Auferstehung?“

„Ich würde es dir nicht sagen“, sagte der Schulleiter langsam, „aber ich fürchte, was dieser Unglaube dir antun könnte… also hör zu, Harry, bitte hör zu…“

Der Stein der Auferstehung war eines der drei legendären Heiligtümer des Todes, verwandt mit Harrys Umhang. Der Stein der Auferstehung konnte die Seelen von den Toten zurückrufen—sie in die Welt der Lebenden zurückbringen, wenn auch nicht so, wie sie waren. Cadmus Peverell benutzte den Stein, um seine verlorene Geliebte von den Toten zurückholen, aber ihr Herz blieb bei den Toten und nicht in der Welt der Lebenden. Und mit der Zeit machte es ihn wahnsinnig und er tötete sich selbst, um wieder wirklich bei ihr zu sein…

In aller Höflichkeit hob Harry seine Hand.

„Ja?“, sagte der Schulleiter widerwillig.

„Der offensichtliche Test, um zu sehen, ob der Stein der Auferstehung wirklich die Toten zurückruft oder einfach nur ein Bild aus dem Kopf des Nutzers projiziert, besteht darin, eine Frage zu stellen, deren Antwort man selbst nicht kennt, aber die tote Person schon, und das kann in dieser Welt definitiv bestätigt werden. Zum Beispiel könnte man —“

Dann hielt Harry inne, denn diesmal hatte er es geschafft, seiner Zunge einen Schritt voraus zu sein, schnell genug, um nicht den Vornamen und den Test zu nennen, die ihm in den Sinn gekommen waren.

„… Ihre tote Frau zurückrufen sie und fragen, wo sie ihren verlorenen Ohrring gelassen hat oder so etwas“, beendete Harry. „Hat jemand solche Tests gemacht?“

„Der Stein der Auferstehung ist seit Jahrhunderten verloren, Harry“, sagte der Schulleiter leise.

Harry zuckte mit den Schultern. „Nun, ich bin Wissenschaftler und ich bin immer bereit, überzeugt zu werden. Wenn Sie wirklich glauben, dass der Stein der Auferstehung die Toten zurückruft—dann müssen Sie glauben, dass ein solcher Test erfolgreich sein wird, oder? Wissen Sie also, wo man den Stein der Auferstehung findet? Ich habe ein Heiligtum des Todes bereits unter höchst mysteriösen Umständen bekommen und, nunja, wir beide wissen, wie der Rhythmus der Welt bei so etwas funktioniert.“

Dumbledore starrte Harry an.

Harry blickte in gleicher Art zum Schulleiter zurück.

Der alte Zauberer ließ eine Hand über seine Stirn gleiten und murmelte: „Das ist Wahnsinn.“

Irgendwie schaffte Harry es, sich vom Lachen abzuhalten.

Und Dumbledore befahl Harry, den Unsichtbarkeitsumhang aus seinem Beutel zu holen; auf Anweisung des Schulleiters starrte Harry auf die Innenseite und Rückseite der Kapuze, bis er es schließlich sah, schwach gegen das silberne Netz, in verblasstem Scharlachrot, dass wie aus getrocknetes Blut gezogen aussah, das Symbol der Heiligtümer des Todes: Ein Dreieck, in dem ein Kreis eingeschlossen ist und einer Linie, die beide in der Mitte trennt.

„Danke“, sagte Harry höflich. „Ich werde auf jeden Fall nach einem so markierten Stein Ausschau halten. Haben Sie noch andere Beweise?“

Dumbledore schien einen Kampf in sich selbst zu führen. „Harry“, sagte der alte Zauberer, seine Stimme erhebt sich, „dies ist ein gefährlicher Weg, den du gehst, ich bin mir nicht sicher, ob ich mit dem was ich nun sagen werde das Richtige tue, aber ich muss dich von diesem Weg abbringen! Harry, wie hätte Voldemort den Tod seines Körpers überleben können, wenn er keine Seele hat?“

Und da wurde Harry klar, dass es genau eine Person gab, die Professor McGonagall überhaupt erst gesagt hatte, dass der Dunkle Lord noch am Leben sei; und es war der verrückte Schulleiter ihres Irrenhauses einer Schule, der dachte, dass die Welt nach Klischees funktioniert.

„Gute Frage“, sagte Harry nach einer internen Debatte über das weitere Vorgehen. „Vielleicht hat er einen Weg gefunden, die Macht des Steins der Auferstehung zu duplizieren, nur dass er ihn im Voraus mit einer vollständigen Kopie seines Gehirnzustandes geladen hat. Oder so etwas in der Art.“ Harry war sich plötzlich nicht mehr sicher, ob er versuchte, eine Erklärung für etwas zu finden, das tatsächlich passiert war. „Können Sie mir einfach alles erzählen, was Sie zum Überleben des Dunklen Lords wissen und was es braucht, um ihn zu töten?“ Für den Fall, dass er überhaupt noch als mehr als eine Klitterer-Schlagzeile existiert.

„mich täuschst du nicht, Harry“, sagte der alte Zauberer; sein Gesicht sah jetzt alt aus, und faltiger, als nur durch das hohe Alter zu erklären war. „Ich weiß, warum du diese Frage wirklich stellst. Nein, ich lese deine Gedanken nicht, das muss ich nicht, dein Zögern verrät dich! Du suchst das Geheimnis der Unsterblichkeit des Dunklen Lords, um es für dich selbst zu nutzen!“

„Falsch! Ich will das Geheimnis der Unsterblichkeit des Dunklen Lords, um es für alle zu nutzen! “

Tick, Knister, Pfft…

Albus Percival Wulfric Brian Dumbledore stand einfach nur da und starrte mit weit offen stehendem Mund Harry an.

(Harry schnitzte sich für Montag einen Strich auf sein imaginäres Kerbholz, da er es geschafft hatte, jemandem komplett zu überwältigen, bevor der Tag vorbei war.)

„Und falls das nicht klar war“, sagte Harry, „mit allen meine ich auch alle Muggel, nicht nur alle Zauberer.“

„Nein“, sagte der alte Zauberer und schüttelte den Kopf. Seine Stimme wurde lauter. „Nein, nein, nein! Das ist Wahnsinn!“

„Wa ha ha!“ sagte Harry.

Das Gesicht des alten Zauberers hatte sich vor Wut und Sorge verzogen. „Voldemort stahl das Buch, aus dem er sein Geheimnis lernte; es war nicht da, als ich nach ihm suchte. Aber soviel weiß ich und so viel werde ich dir sagen: Seine Unsterblichkeit wurde aus einem Ritual geboren, schrecklich und dunkel, schwärzer als das tiefste Schwarz! Und es war Myrtle, die arme, süße Myrtle, die dafür gestorben ist; seine Unsterblichkeit forderte ein Opferung, sie ~verlangte Mord —“

„Nun, offensichtlich werde ich keine Methode der Unsterblichkeit popularisieren, die das Töten von Menschen erfordert! Das würde den ganzen Punkt zunichte machen! “

Es gab eine erschrockene Pause.

Langsam entspannte sich das Gesicht des alten Zauberers von seiner Wut, obwohl die Sorge noch blieb. „Du würdest kein Ritual anwenden, das Menschenopfer erfordert.“

„Ich weiß nicht, wofür Sie mich halten, Schulleiter“, sagte Harry kalt, und seine eigene Wut stieg, „aber lassen Sie uns nicht vergessen, dass ich derjenige bin, der will, dass die Menschen leben! Derjenige, der alle retten will! Sie sind derjenige, der denkt, dass der Tod fantastisch ist und jeder sterben sollte!“

„Ich bin ratlos, Harry“, sagte der alte Zauberer. Seine Füße begannen wieder einmal, ihn durch sein seltsames Büro zu schleppen. „Ich weiß nicht, was ich sagen soll.“ Er hob eine Kristallkugel auf, in der sich eine entflammte Hand zu befinden schien, und sah sie mit einem traurigen Ausdruck an. „Nur, dass ich von dir sehr missverstanden werde…. Ich will nicht, dass alle sterben, Harry!“

„Sie wollen nur nicht, dass jemand unsterblich ist“, sagte Harry sehr ironisch. Es schien, dass elementare logische Tautologien wie (Ɐx: Sterben(x)) = (∄x: Nicht Sterben(x))**** ~dem mächtigsten Zauberers der Welt zu weit gingen.

Der alte Zauberer nickte. „Ich habe weniger Angst als zuvor, aber trotzdem mache ich mir große Sorgen um dich, Harry“, sagte er leise. Seine Hand, runzlig von seinem Alter, aber immer noch stark, legte die Kristallkugel bestimmt in ihren Ständer zurück. „Denn die Angst vor dem Tod ist eine bittere Sache, eine Krankheit der Seele, durch die Menschen verdreht und verzerrt werden. Voldemort ist nicht der einzige Dunkle Zauberer, der diesen trostlosen Weg gegangen ist, obwohl ich fürchte, dass er es auf ihm weiter gebracht hat als jeder andere zuvor.“

„Und Sie denken, Sie hätten keine Angst vor dem Tod?“, fragte Harry und versuchte nicht einmal, die Ungläubigkeit in seiner Stimme zu verbergen.

Das Gesicht des alten Zauberers war friedlich. „Ich bin nicht perfekt, Harry, aber ich glaube, ich habe meinen Tod als einen Teil von mir akzeptiert.“

„Oh ho“, sagte Harry. „Sehen Sie, es gibt da diese kleine Sache, die kognitive Dissonanz genannt wird, oder, im Klartext, saure Trauben. Wenn die Leute einmal im Monat mit Knüppeln auf den Kopf geschlagen würden und niemand etwas dagegen tun könnte, gäbe es ziemlich bald alle möglichen Philosophen, die vorgäben weise zu sein, wie Sie es so schön genannt haben, die alle möglichen großartigen Vorteile daran finden würden, einmal im Monat mit einem Knüppel auf den Kopf geschlagen zu werden. Es macht härter, oder es macht glücklicher an den Tagen, an denen man nicht mit einem Knüppel geschlagen wird. Aber wenn man zu jemandem gehen würden, der nicht geschlagen wird und man ihn fragen würde, ob er damit anfangen will, im Austausch für diese großartigen Vorteile, würden er nein sagen. Und wenn Sie nicht sterben müssten, wenn Sie von einem Ort kämen, an dem noch nie jemand vom Tod gehört hat, und ich würde Ihnen vorschlagen, dass es eine großartige wunderbar tolle Idee wäre, wenn die Menschen faltig und alt werden würden und schließlich aufhören würden zu existieren, dann würden Sie mich direkt in ein Irrenhaus bringen lassen! Warum also sollte irgendjemand zu so einem dummen Gedanken kommen, dass der Tod eine gute Sache ist? Weil Sie Angst davor haben, weil Sie nicht wirklich sterben wollen und dieser Gedanke schmerzt so sehr, dass Sie ihn wegerklären müssen, etwas tun müssen, um den Schmerz zu betäuben, damit Sie nicht daran denken müssen—“

„Nein, Harry“, sagte der alte Zauberer. Sein Gesicht war sanft, während seine Hand ein beleuchtetes Wasserbecken glitt, das leise Glockentöne bei der Berührung erzeugte. „Obwohl ich verstehe, warum du das denken musst.“

„Wollen Sie die Dunklen Zauberer verstehen?“, fragte Harry, seine Stimme jetzt hart und grimmig. „Dann sehen Sie in den Teil von Ihnen, der nicht vor dem Tod, sondern vor der Angst vor dem Tod flieht, der diese Angst so unerträglich findet, dass sie den Tod als Freund akzeptiert und sich ihm anschließt, versucht, mit der Nacht eins zu werden, damit er sich als Meister des Abgrunds fühlen kann. Sie haben das schrecklichste aller Übel genommen und es gut getauft! Mit nur einem leichten Schubs würde derselbe Teil von Ihnen Unschuldige ermorden und es Freundschaft nennen. Wenn Sie den Tod besser als das Leben nennen können, dann können Sie Ihren moralischen Kompass in jedwede Richtung zeigen lassen —“

„Ich glaube“, sagte Dumbledore und schüttelte Wassertröpfchen zu leisem Glockenspiel aus seiner Hand, „dass du die Dunklen Zauberer sehr gut verstehst, ohne selbst einer zu sein.“ Er sagte es in vollkommener Ernsthaftigkeit und ohne Anklage. „Aber ein Verständnis von mir, fürchte ich, fehlt dir völlig.“ Der alte Zauberer lächelte jetzt und es war ein sanftes Lachen in seiner Stimme.

Harry versuchte, nicht noch kälter zu werden, als er es eh schon war; von irgendwo aus strömte ihm eine lodernde Wut des Grolls in den Kopf, verursacht von Dumbledores Herablassung und all dem Gelächter, das weise alte Narren jemals anstelle von Argumenten benutzt hatten. „Schon lustig, wissen Sie, ich hatte gedachte, mit Draco Malfoy zu reden würde sich als unmöglich herausstellen und stattdessen war er in seiner kindlichen Unschuld hundert Mal stärker als Sie.“

Ein verwirrter Blick huschte über das Gesicht des alten Zauberers. „Was meinst du damit?“

„Ich meine“, sagte Harry, seine Stimme beißend, „dass Draco tatsächlich seinen eigenen Glauben ernst nahm und meine Worte verarbeitet hat, anstatt sie mit einem sanften Lächeln aus dem Fenster zu schmeißen. Sie sind so alt und weise, dass Sie nicht einmal beachten können, was ich sage! Nicht verstehen, beachten!“

„Ich habe dir zugehört, Harry“, sagte Dumbledore und sah jetzt förmlicher drein, „aber zuzuhören bedeutet nicht immer zustimmen. Abgesehen von Meinungsverschiedenheiten, was ist es, von dem du denkst, dass ich es nicht verstehe?“

Dass Sie, wenn Sie wirklich an ein Leben nach dem Tod glauben würden, zum St. Mungo's gehen und Nevilles Eltern, Alice und Frank Longbottom, töten würden, damit sie zu ihrem nächsten großen Abenteuer aufbrechen könnten, anstatt sie hier in ihrem geschädigten Zustand verweilen zu lassen—

Harry hielt sich knapp, sehr knapp davon zurück es laut auszusprechen.

„In Ordnung“, sagte Harry kalt. „Ich werde also Ihre ursprüngliche Frage beantworten. Sie haben gefragt, warum Dunkle Zauberer Angst vor dem Tod haben. Tun Sie so, als würden Sie wirklich an Seelen glauben. Tun Sie so, als ob jeder die Existenz von Seelen jederzeit überprüfen könnte, tun Sie so, als ob niemand bei Beerdigungen weinte, weil sie wüssten, dass ihre Liebsten noch am Leben wären. Können Sie sich jetzt vorstellen, eine Seele zu zerstören? Sie in Stücke zu reißen, so dass nichts mehr übrig bleibt, um auf ihr nächstes großes Abenteuer zu gehen? Können Sie sich vorstellen, was für eine schreckliche Sache das wäre, das schlimmste Verbrechen, das jemals in der Geschichte des Universums begangen worden wäre, wofür Sie alles geben würden zu verhindern, dass es auch nur ein einziges mal geschehen würde? Denn das ist der Tod wirklich—die Vernichtung einer Seele!“

Der alte Zauberer starrte ihn an, ein trauriger Blick in seinen Augen. „Ich nehme an, ich verstehe es jetzt“, sagte er leise.

„Oh?“ sagte Harry. „Was verstehen Sie?“

„Voldemort“, sagte der alte Zauberer. „Ich verstehe ihn jetzt endlich. Denn um zu glauben, dass die Welt wirklich so ist, muss man glauben, dass es keine Gerechtigkeit in ihr gibt, dass sie in ihrem Kern aus Dunkelheit gesponnen ist. Ich fragte dich, warum er ein Monster wurde und du konntest keinen Grund nennen. Und wenn ich ihn fragen könnte, nehme ich an, wäre seine Antwort: Warum nicht?“

Sie standen da und starrten sich gegenseitig in die Augen, der alten Zauberer in seinen Roben und der kleine Junge mit der Blitznarbe auf der Stirn.

„Sag mir, Harry“, sagte der alte Zauberer, „wirst du ein Monster werden?“

„Nein“, sagte der Junge, eine eiserne Gewissheit in seiner Stimme.

„Warum nicht?“, fragte der alte Zauberer.

Der kleine Junge stand sehr gerade, er streckte stolz sein Kinn hoch und sagte: „Es gibt keine Gerechtigkeit in den Naturgesetzen, Schulleiter, keinen Begriff für Fairness in den Bewegungsgleichungen. Das Universum ist weder böse noch ist es gut, es ist ihm einfach egal. Den Sternen ist es egal, der Sonne und dem Himmel. Aber es ist unwichtig, ob es sie kümmert! Uns ist es nicht egal! Es gibt Licht in der Welt und wir sind es! “

„Ich frage mich, was aus dir werden wird, Harry“, sagte der alte Zauberer. Seine Stimme war leise, mit einem seltsamen Ausdruck von Verwunderung und Reue in ihr. "Es ist genug mich dazu zu bringen, leben zu wollen, nur um es zu sehen.„

Der Junge verbeugte ironisch vor ihm und ging weg; und die Eichentür schlug hinter ihm mit einem dumpfen Schlag zu.

* Zimperlicher Bartgeier: “The Magic Words are Squeamish Ossifrage„, zu Deutsch “Die Zauberworte sind zimperlicher Bartgeier", war die Lösung eines Dechiffrierungproblems, dass 1977 aufgestellt und erst in den Jahren 1993 und 1994 gelöst werden konnte. Zum Zeitpunkt dieser Geschichte, dem Jahr 1991, ist die Lösung des Problems also noch nicht bekannt.

** Inkommensurabilität - teilweise oder vollständige Unübersetzbarkeit der Begriffe einer wissenschaftlichen Theorie in die Begriffe einer anderen Theorie

*** Negentropie: Negative Entropie. Da Entropie das Maß der Unordnung beschreibt, entspricht negative Entropie einem Maß der Ordnung. Laut dem zweiten Hauptsatz der Thermodynamik nimmt die Entropie in einem geschlossenen System immer weiter zu, das heißt, die Negentropie nimmt immer weiter ab. Ist allerdings keine Ordnung mehr vorhanden, sind auch geordnete Strukturen, wie etwa organische Moleküle, nicht mehr möglich.

**** (Ɐx: Sterben(x)) = (∄x: Nicht Sterben(x)): Logisch wahre Aussage, die besagt, dass die Ausdrücke niemand ist unsterblich und alle sterben identisch sind.

