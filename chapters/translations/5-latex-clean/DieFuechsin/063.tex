

\hypertarget{das-stanford-prison-experiment-nachspiel}{% \section{31. Das Stanford-Prison-Experiment, Nachspiel}\label{das-stanford-prison-experiment-nachspiel}}

-\/-\/-\/-\/-\/- Kapitel Dreiundsechzig: Das Stanford-Prison-Experiment, Nachspiel -\/-\/-\/-\/-\/-

\emph{Nachspiel, Hermine Granger:}

Sie fing gerade an, ihre Bücher zu schließen und ihre Hausaufgaben wegzulegen, um sich bettfertig zu machen, während Padma und Mandy ihre eigenen Bücher gegenüber von ihr aufstapelten, als Harry Potter den Ravenclaw-Gemeinschaftsraum betrat; und erst da wurde ihr klar, dass sie ihn seit dem Frühstück überhaupt nicht mehr gesehen hatte.

Diese Erkenntnis wurde schnell von einer viel erschreckenderen verdrängt.

Da war ein gold-rot geflügeltes Wesen auf Harrys Schulter, ein leuchtender Feuervogel.

Und Harry sah traurig und abgekämpft und wirklich \emph{müde} aus, als wäre der Phönix das Einzige, was ihn auf den Beinen hielt, aber er hatte immer noch eine Wärme an sich, wenn sie die Augen zusammenkniff, hätte sie schwören können, den Schulleiter vor sich zu haben, das war der Eindruck, der Hermine durch den Kopf ging, auch wenn es keinen Sinn ergab.

Harry Potter stapfte durch den Ravenclaw-Gemeinschaftsraum, vorbei an Sofas voller starrender Mädchen, vorbei an Kartenspielkreisen starrender Jungen, auf sie zu.

Theoretisch redete sie noch immer nicht mit Harry Potter, seine Woche war erst morgen zu Ende, aber was auch immer vor sich ging, war eindeutig \emph{viel} wichtiger als das -

"Fawkes", sagte Harry, gerade als sie den Mund öffnete, "das Mädchen da drüben ist Hermine Granger, sie redet im Moment nicht mit mir, weil ich ein Idiot bin, aber wenn du dich an die Schulter einer guten Person hängen willst, ist sie besser als ich."

So viel Erschöpfung und Schmerz in Harry Potters Stimme -

Aber bevor sie sich überlegen konnte, was sie tun sollte, war der Phönix von Harrys Schulter weggeglitten wie ein Feuer, das im Schnelldurchlauf an einem Streichholz hochkriecht, und blitzte auf sie zu; da war ein Phönix, der vor ihr herumflog und sie mit Augen aus Licht und Flammen anstarrte.

"\emph{Krächz?}", fragte der Phönix.

Hermine starrte ihn an, sie fühlte sich, als stünde sie vor einer Frage in einem Test, für den sie vergessen hatte zu lernen, die eine wichtigste Frage, für die sie ihr ganzes Leben lang nicht gelernt hatte, sie fand nichts, was sie sagen konnte.

"Ich bin -", sagte sie. "Ich bin erst zwölf, ich habe noch nichts \emph{gemacht} -"

Der Phönix glitt einfach sanft herum, drehte sich um eine Flügelspitze da er ein Wesen aus Licht und Luft war, und schwebte zurück zu Harry Potters Schulter, wo er sich fest niederließ.

"Du dummer Junge", sagte Padma ihr gegenüber und sah aus, als wüsste sie nicht, ob sie lachen oder weinen sollte, "Phönixe sind nichts für kluge Mädchen, die ihre Hausaufgaben machen, sondern für Idioten, die sich direkt auf fünf ältere Slytherin-Schläger stürzen. Es gibt einen Grund, warum die Farben von Gryffindor rot und gold sind, weißt du."

Im Ravenclaw-Gemeinschaftsraum gab es viel freundliches Gelächter.

Hermine gehörte nicht zu den Lachenden.

Und Harry auch nicht.

Harry hatte eine Hand über sein Gesicht gelegt. "Sag Hermine, dass es mir leid tut", sagte er zu Padma, wobei seine Stimme fast zu einem Flüstern sank. "Sag ihr, dass ich vergessen habe, dass Phönixe Tiere sind, sie verstehen nichts von Zeit und Planung, sie verstehen nicht, dass es Menschen gibt, die \emph{später} gute Dinge tun werden - ich bin mir nicht sicher, ob sie wirklich verstehen, was eine Person \emph{ist}, sie sehen nur, was Menschen tun. Fawkes weiß nicht, was zwölf bedeutet. Sag Hermine, es tut mir leid - ich hätte es nicht tun sollen - es geht einfach alles schief, nicht wahr?"

Harry wandte sich zum Gehen, den Phönix immer noch auf der Schulter, und stapfte langsam auf die Treppe zu, die zu seinem Schlafsaal hinaufführte.

Und Hermine konnte es nicht dabei belassen, sie \emph{konnte} es einfach \emph{nicht} dabei belassen. Sie wusste nicht, ob es ihr Konkurrenzkampf mit Harry war oder etwas anderes. Sie konnte es einfach nicht dabei belassen, dass der Phönix sich von ihr abwandte.

Sie \emph{musste} -

Ihr Verstand durchforstete verzweifelt die Gesamtheit ihres ausgezeichneten Gedächtnisses und fand nur eine Sache -

"Ich wollte vor den Dementor rennen, um zu versuchen, Harry zu retten!", rief sie dem rotgoldenen Vogel ein wenig verzweifelt zu. "Ich meine, ich bin tatsächlich losgerannt und alles! Das war doch dumm und mutig, oder?"

Mit einem trällernden Schrei stürzte sich der Phönix wieder von Harrys Schulter, zurück zu ihr, wie eine sich ausbreitende Feuersbrunst, er umkreiste sie dreimal, als wäre sie das Zentrum eines Infernos, und für einen kurzen Moment streifte sein Flügel ihre Wange, bevor der Phönix wieder zu Harry aufstieg.

Im Ravenclaw-Gemeinschaftsraum herrschte Schweigen.

"Ich hab's ja gesagt", sagte Harry laut, und dann begann er, die Treppe zu seinem Zimmer hinaufzusteigen; er schien sehr schnell hoch zu steigen, als wäre er aus irgendeinem Grund sehr leichtfüßig, so dass er und Fawkes im Nu verschwunden waren.

Hermine hielt eine zitternde Hand an ihre Wange, wo Fawkes sie mit seinem Flügel gestreift hatte, ein Fleck Wärme verweilte dort, als ob dieser kleine Flecken Haut ganz sanft in Brand gesetzt worden wäre.

Sie hatte die Frage des Phönix beantwortet, nahm sie an, aber es fühlte sich für sie an, als wäre sie nur knapp an dem Test vorbeigeschrammt, als hätte sie eine 62 bekommen und sie hätte 104 bekommen können, wenn sie sich mehr angestrengt hätte.

Wenn sie sich \emph{überhaupt} bemüht hätte.

Sie hatte sich nicht \emph{wirklich} angestrengt, wenn sie darüber nachdachte.

Sie hatte nur ihre Hausaufgaben gemacht -

\emph{Wen hast du gerettet?}

\emph{Nachspiel, Fawkes:}

Albträume hatte der Junge erwartet, Schreie und Betteln und heulende Orkane der Leere, die Entladung der Schrecken, die in die Erinnerung gelegt werden und auf diese Weise vielleicht Teil der Vergangenheit werden.

Und der Junge wusste, dass die Albträume kommen würden.

In der nächsten Nacht würden sie kommen.

Der Junge träumte, und in seinen Träumen stand die Welt in Flammen, Hogwarts stand in Flammen, sein Zuhause stand in Flammen, die Straßen Oxfords standen in Flammen, alles brannte in goldenen Flammen, die leuchteten, aber nicht verzehrten, und alle Menschen, die durch die brennenden Straßen liefen, leuchteten in weißem Licht, heller als das Feuer, als wären sie selbst Flammen oder Sterne.

Die anderen Erstklässler gingen zu Bett und sahen es mit eigenen Augen, das Wunder, von dem sie nur Gerüchte gehört hatten, dass Harry Potter still und regungslos in seinem Bett lag, ein sanftes Lächeln auf dem Gesicht, während auf seinem Kissen ein rot-goldener Vogel über ihn wachte, dessen helle Flügel über ihn strichen wie eine Decke, die über seinen Kopf gezogen wurde.

Die Abrechnung war um eine weitere Nacht verschoben worden.

\emph{Nachspiel, Draco Malfoy:}

Draco richtete seinen Umhang und achtete darauf, dass der grüne Saum gerade war. Er strich sich mit dem Zauberstab über den Kopf und sprach einen Zauberspruch, den Vater ihm beigebracht hatte, als andere Kinder noch im Matsch spielten, einen Zauberspruch, der dafür sorgte, dass kein einziges Fussel- oder Staubkorn seine Zaubererroben beschmutzen würde.

Draco hob den geheimnisvollen Umschlag auf, den Vater ihm mit der Eule geschickt hatte, und steckte ihn in seinen Umhang. Er hatte bereits \emph{Incendio} und \emph{Everto} für den geheimnisvollen Zettel benutzt.

Und dann machte er sich auf den Weg zum Frühstück, um sich genau zum selben Tick der Uhr zu setzen, an dem das Essen erschien. Wenn er es schaffte sah es so aus, als hätten alle anderen auf sein Erscheinen gewartet, um zu essen. Denn wenn man der Spross der Malfoys war, war man in allem der Erste, auch beim Frühstück, einfach darum.

Vincent und Gregory warteten vor der Tür seines privaten Zimmers auf ihn, sie waren noch vor ihm aufgestanden - wenn auch natürlich nicht ganz so schick gekleidet.

Der Slytherin-Gemeinschaftsraum war menschenleer, jeder, der so früh aufstand, ging sowieso direkt zum Frühstück.

Die Kerkerflure waren leer und still, bis auf die eigenen Schritte und deren Echo.

In der Großen Halle herrschte trotz der relativ wenigen Ankömmlinge ein hektisches Treiben, einige jüngere Kinder weinten, Schüler rannten zwischen den Tischen hin und her oder standen in Gruppen und schrien sich gegenseitig an, ein rotgewandeter Vertrauensschüler stand vor zwei grüngekleideten Schülern und schrie sie an und Snape schritt auf das Chaos zu -

Der Lärm wurde ein wenig leiser, als die Leute Draco erblickten, einige der Gesichter drehten sich um, um ihn anzustarren, und wurden still.

Das Essen erschien auf den Tischen. Keiner sah es an.

Und Snape machte auf dem Absatz kehrt, ließ sein Ziel aus den Augen und ging direkt auf Draco zu.

Angst umschloss Dracos Herz, war Vater etwas zugestoßen - nein, sicher hätte Vater es ihm gesagt - was auch immer passiert war, warum hatte Vater es ihm nicht gesagt -.

Draco sah, dass sich um Snapes Augen tiefe Ringe abzeichneten, als der Leiter seines Hauses näherkam. Der Meister der Zaubertränke war noch nie gut gekleidet gewesen (das war eine Untertreibung), aber seine Roben waren heute Morgen noch schmutziger und unordentlicher, mit zusätzlichen Fettflecken versehen.

"Haben Sie es nicht gehört?", zischte ihr Hausoberhaupt, als er näherkam. "Um Himmels willen, Malfoy, haben Sie keine Zeitung geliefert bekommen?"

"Was gibt's, Profe-"

"Bellatrix Black wurde aus Askaban befreit!"

"\emph{Was?}", sagte Draco schockiert, als Gregory hinter ihm etwas sagte, was er wirklich nicht hätte sagen sollen, und Vincent keuchte nur.

Snape starrte ihn mit zusammengekniffenen Augen an und nickte dann abrupt. "Lucius hat Ihnen also nichts erzählt. Ich verstehe." Snape schnaubte und wandte sich ab.

"Professor!", sagte Draco. Die Auswirkungen begannen ihm gerade zu dämmern, sein Verstand drehte sich wie wild. "Professor, was soll ich tun - Vater hat mich nicht instruiert -"

"Dann \emph{schlage} ich \emph{vor}", sagte Snape spöttisch, während er wegging, "dass Sie ihnen das \emph{sagen}, Malfoy, so wie es Ihr Vater beabsichtigt hat!"

Draco warf einen Blick zurück auf Vincent und Gregory, obwohl er nicht wusste, warum er sich die Mühe machte, denn natürlich sahen sie noch verwirrter aus als er selbst.

Draco ging vorwärts zum Slytherin-Tisch und setzte sich an das hintere Ende, wo noch niemand saß.

Draco legte ein Omelett mit Würstchen auf seinen Teller und begann es mit automatischen Bewegungen zu essen.

Bellatrix Black war aus Askaban befreit worden.

Bellatrix Black war aus Askaban befreit worden…?

Draco wusste nicht, was er davon halten sollte, es war so unerwartet wie das Erlöschen der Sonne - nun, die Sonne würde erwartungsgemäß in sechs Milliarden Jahren erlöschen, aber das war so unerwartet wie das Erlöschen der Sonne \emph{morgen}. Vater hätte es nicht getan, Dumbledore hätte es nicht getan, \emph{niemand} sollte dazu \emph{in der Lage sein} - was \emph{bedeutete} es - was würde Bellatrix nach zehn Jahren in Askaban irgendjemandem \emph{nützen} - selbst wenn sie wieder stark werden würde, was nützte eine mächtige Zauberin, die vollkommen böse und wahnsinnig war und einem Dunklen Lord fanatisch ergeben, der nicht mehr da war?

"Hey", sagte Vincent von seinem Platz neben Draco, "ich verstehe das nicht, Boss, warum haben wir das getan?"

"\emph{Wir} haben es nicht getan, du Tölpel!", schnauzte Draco. "Um Himmels willen, wenn selbst \emph{du} denkst, dass wir - hat dir dein Vater nie etwas über Bellatrix Black erzählt? Sie hat Vater einmal gefoltert, sie hat \emph{deinen} Vater gefoltert, sie hat \emph{jeden} gefoltert, der Dunkle Lord hat ihr einmal befohlen, \emph{sich selbst} mit dem Cruciatus-Fluch zu foltern und sie hat es \emph{getan}! Sie hat keine verrückten Dinge getan, um Angst und Gehorsam in der Bevölkerung zu wecken, sie hat verrückte Dinge getan, weil sie verrückt ist! Sie ist eine \emph{Schlampe}, das ist sie!"

"Ach, wirklich?", sagte eine empörte Stimme hinter Draco.

Draco blickte nicht auf. Gregory und Vincent würden ihm den Rücken freihalten.

"Ich hätte gedacht, du würdest dich freuen -"

"- zu hören, dass ein Todesser befreit worden ist, Malfoy!"

Amycus Carrow war immer einer der \emph{anderen} problematischen Personen gewesen; Vater hatte Draco einmal gesagt, er solle dafür sorgen, dass er nie mit Amycus allein in einem Raum war…

Draco drehte sich um und schenkte Flora und Hestia Carrow seinen Grinser Nummer drei, den, der besagte, dass er einem edlen und sehr alten Haus angehörte und sie nicht und ja, das bedeutete etwas. Draco sagte in ihre allgemeine Richtung, ohne sich herabzulassen, \emph{sie} gezielt anzusprechen: "Es gibt Todesser und es gibt Todesser", und wandte sich dann wieder seinem Essen zu.

Es gab gleichzeitig zwei wütende Hmpfs und dann stürmten zwei Paar Schuhe in Richtung des anderen Endes des Slytherin-Tisches davon.

Ein paar Minuten später rannte Millicent Bulstrode auf sie zu, sichtlich außer Atem, und sagte: "Mr. Malfoy, haben Sie gehört?"

"Von Bellatrix Black?", fragte Draco. "Ja -"

"Nein, über Potter!"

"Was?"

"Potter ist gestern Abend mit einem Phönix auf der Schulter herumgelaufen und sah aus, als wäre er durch zehn Meilen Schlamm geschleift worden, man sagt, dass der Phönix ihn nach Askaban gebracht hat, um zu versuchen, Bellatrix aufzuhalten, und er hat sich mit ihr duelliert und sie haben die halbe Festung in die Luft gejagt!"

"\emph{Was?}", sagte Draco. "Oh, das kann doch nicht \emph{wahr} sein, dass -"

Draco hielt inne.

Das hatte er schon oft über Harry Potter gesagt und er hatte begonnen, einen Trend zu bemerken.

Millicent rannte los, um es jemand anderem zu sagen.

"Du glaubst doch nicht \emph{wirklich} -", sagte Gregory.

"Ich weiß es ehrlich gesagt nicht mehr", sagte Draco.

Ein paar Minuten später, nachdem Theodore Nott sich ihm gegenüber hingesetzt hatte und William Rosier gegangen war, um sich zu den Carrow-Zwillingen zu setzen, stupste Vincent ihn an und sagte: "Da."

Harry Potter hatte die Große Halle betreten.

Draco beobachtete ihn genau.

Auf Harrys Gesicht war keine Beunruhigung zu sehen, keine Überraschung oder Schock, er sah einfach nur…

Es war der gleiche distanzierte, in sich gekehrte Blick, den Harry trug, wenn er versuchte, die Antwort auf eine Frage herauszufinden, die Draco noch nicht verstand.

Draco schob sich hastig von der Bank des Slytherin-Tisches hoch, sagte: "Bleibt zurück", und ging mit in angemessener Geschwindigkeit auf Harry zu.

Harry schien seine Annäherung gerade zu bemerken, als der andere Junge sich zum Ravenclaw-Tisch drehte, und Draco -

- warf Harry einen kurzen Blick zu -

- und ging dann an ihm vorbei, geradewegs aus der Großen Halle hinaus.

Eine Minute später spähte Harry um die Ecke der kleinen steinernen Ecke, in der Draco gewartet hatte, es würde vielleicht nicht jeden täuschen, aber es würde eine glaubhafte Abstreitbarkeit schaffen.

"\emph{Quietus}", sagte Harry. "Draco, was -"

Draco holte den Umschlag aus seinem Umhang. "Ich habe eine Nachricht von Vater für dich."

"\emph{Hä?}", sagte Harry und nahm Draco den Umschlag ab, riss ihn auf eine eher unschöne Weise auf und zog ein Blatt Pergament heraus und entfaltete es und -

Harry atmete hörbar ein.

Dann sah Harry Draco an.

Dann schaute Harry wieder auf das Pergament hinunter.

Es gab eine Pause.

Harry sagte: "Hat Lucius dir aufgetragen, ihm meine Reaktion darauf zu berichten?"

Draco hielt einen Moment lang inne, wog ab und öffnete dann den Mund.

"Ich sehe, das hat er", sagte Harry, und Draco verfluchte sich, er hätte es besser wissen müssen, nur war es \emph{schwer} gewesen, sich zu entscheiden. "Was willst du ihm denn sagen?"

"Dass du überrascht warst", sagte Draco.

"Überrascht", sagte Harry flach. "Ja. Gut. Sag ihm das."

"Was \emph{ist}?", fragte Draco. Und dann, als er sah wie hin- und hergerissen Harry aussah: "Wenn du hinter meinem Rücken mit Vater verhandelst -"

Und Harry gab Draco ohne ein Wort das Papier.

Darauf stand:

\emph{Ich weiß, dass du es warst.}

"\emph{WAS ZUM -}"

"Das wollte ich \emph{dich} gerade fragen", sagte Harry. "Hast du \emph{irgendeine} Ahnung, was mit deinem Vater los ist?"

Draco starrte Harry an.

Dann sagte Draco: "\emph{Hast} du es getan?"

"Was?", sagte Harry. "Welchen \emph{möglichen} Grund sollte ich - \emph{wie} sollte ich -"

"Hast du es getan, Harry?"

"Nein!", sagte Harry. "Natürlich nicht!"

Draco hatte genau zugehört, aber er hatte kein Zögern oder Zittern bemerkt.

Also nickte Draco und sagte: "Ich habe keine Ahnung, was Vater denkt, aber es \emph{kann nicht}, ich meine, es \emph{kann unmöglich} gut sein. Und, ähm … die Leute sagen auch …"

"Was", sagte Harry misstrauisch, "sagen sie denn, Draco?"

"Hat dich \emph{wirklich} ein Phönix nach Askaban gebracht, um zu versuchen, Bellatrix Black an der Flucht zu hindern -"

\emph{Nachspiel: Neville Longbottom}

Harry hatte sich gerade erst zum ersten Mal an den Ravenclaw-Tisch gesetzt, in der Hoffnung, einen schnellen Happen zu essen. Er wusste, dass er gehen und über Dinge nachdenken musste, aber es gab einen winzigen Rest von Phönixfrieden (sogar nach der Begegnung mit Draco), an den er sich noch klammern wollte, ein schöner Traum, an den er sich kaum noch erinnerte; und der Teil von ihm, der sich \emph{nicht} friedlich fühlte, wartete darauf, dass all die Ambosse auf ihn herabfielen, damit er, wenn er ging, um nachzudenken und eine Weile allein zu sein, alle Katastrophen auf einmal verarbeiten konnte.

Harrys Hand griff nach einer Gabel, hob einen Bissen Kartoffelpüree in Richtung seines Mundes -

Und dann gab es einen Schrei.

Ab und zu schrie jemand, wenn er die Nachricht hörte, aber Harrys Ohren \emph{erkannten} diesen -

Harry stand blitzschnell von der Bank auf und ging auf den Hufflepuff-Tisch zu, während sich in seiner Magengrube ein schreckliches Gefühl der Übelkeit breit machte. Es war eines der Dinge, die er nicht bedacht hatte, als er sich entschlossen hatte, das Verbrechen zu begehen, denn Professor Quirrell hatte geplant, dass niemand davon erfuhr; und jetzt, im Nachhinein, hatte Harry einfach - nicht daran \emph{gedacht}…

\emph{Das}, sagte Hufflepuff mit bitterer Intensität, \emph{ist auch deine Schuld}.

Aber als Harry ankam, saß Neville schon da und aß gebratene Wurstpastetchen mit Soße aus Zwergfeigen.

Die Hände des Hufflepuff-Jungen zitterten, aber er durchteilte das Essen und aß es, ohne es fallen zu lassen.

"Hallo, General", sagte Neville, wobei seine Stimme nur leicht schwankte. "Hast du dich gestern Abend mit Bellatrix Black duelliert?"

"Nein", sagte Harry. Seine eigene Stimme schwankte aus irgendeinem Grund ebenfalls.

"Hätte ich auch nicht gedacht", sagte Neville. Es gab ein schabendes Geräusch, als sein Messer wieder die Wurst durchschnitt. "Ich werde sie jagen und töten, kann ich auf deine Hilfe zählen?"

Die Masse der Hufflepuffs, die sich um Neville versammelt hatte, schnappte erschrocken nach Luft.

"Wenn sie hinter dir her ist", sagte Harry heiser, \emph{wenn das alles ein furchtbarer Irrtum war, wenn das alles eine Lüge war}, „werde ich dich verteidigen, sogar mit meinem Leben," \emph{werde nicht zulassen, dass du verletzt wirst für das, was ich getan habe, egal wie}, "aber ich werde dir nicht helfen, sie zu jagen, Neville, Freunde helfen Freunden nicht, Selbstmord zu begehen."

Nevilles Gabel hielt auf dem Weg zu seinem Mund inne.

Dann steckte Neville den Bissen in den Mund, kaute noch einmal.

Und Neville schluckte ihn herunter.

Und Neville sagte: "Ich meinte nicht \emph{jetzt sofort}, ich meine, nachdem ich Hogwarts abgeschlossen habe."

"Neville", sagte Harry, wobei er seine Stimme sehr sorgfältig unter Kontrolle hielt, "ich glaube, dass das auch nach deinem Abschluss noch eine \emph{ziemlich dumme} \emph{Idee} sein könnte. Es gibt doch viel erfahrenere Auroren, die sie verfolgen -" \emph{oh, warte, das ist nicht gut} -

"Hör auf ihn!", sagte Ernie Macmillan, und dann sagte ein älter aussehendes Hufflepuff-Mädchen, das dicht neben Neville stand: "Nevvy, bitte, denk darüber nach, er hat recht!"

Neville stand auf.

Neville sagte: "Bitte folgt mir nicht."

Neville ging von ihnen allen weg; Harry und Ernie streckten unwillkürlich die Hand nach ihm aus, und auch einige der anderen Hufflepuffs.

Und Neville setzte sich an den Gryffindor-Tisch und aus der Ferne (obwohl sie sich anstrengen mussten, um es zu hören) hörten sie Neville sagen: "Ich werde sie nach meinem Abschluss jagen und töten, will mir jemand helfen?" und mindestens fünf Stimmen sagten "Ja" und dann sagte Ron Weasley laut: "Stellt euch in eine Reihe, ich habe heute Morgen eine Eule von Mum bekommen, sie sagt, ich soll allen sagen, dass sie sich den ersten Platz reserviert hat" und jemand sagte "\emph{Molly Weasley} gegen \emph{Bellatrix Black?} Wen will sie eigentlich verarschen?" und Ron griff nach einem Teller und wog einen Muffin in seiner Hand -

Jemand tippte Harry auf die Schulter, und er drehte sich um und sah ein ihm unbekanntes älteres Mädchen mit grünem Haar, das ihm einen Umschlag mit Pergament reichte und dann schnell davonlief.

Harry starrte einen Moment lang auf den Umschlag, dann ging er auf die nächste Wand zu. Das war nicht sehr privat, aber es sollte privat genug sein, und Harry wollte nicht den Eindruck erwecken, viel zu verbergen zu haben.

Das war eine Lieferung des Slytherin-Systems gewesen, das man benutzte, wenn man mit jemandem kommunizieren wollte, ohne dass jemand anderes wusste, dass die beiden miteinander gesprochen hatten. Der Absender gab jemandem, der den Ruf hatte, ein zuverlässiger Bote zu sein, einen Umschlag zusammen mit zehn Knuts; diese erste Person nahm fünf Knuts und gab den Umschlag zusammen mit den anderen fünf Knuts an einen anderen Boten weiter, und der zweite Bote öffnete diesen Umschlag und fand einen anderen Umschlag mit einem Namen darauf und übergab diesen Umschlag an diese Person. Auf diese Weise kannte keine der beiden Personen, die die Nachricht überbrachten, sowohl den Absender als auch den Empfänger, so dass niemand sonst wusste, dass diese beiden Parteien in Kontakt gestanden hatten…

Als Harry die Wand erreichte, steckte er den Umschlag in seinen Umhang, öffnete ihn unter den Falten des Stoffes und warf vorsichtig einen Blick auf das Pergament, das er hervorzog.

Darauf stand,

\emph{\emph{Klassenzimmer links von Verwandlung, 8 Uhr morgen} \emph{früh.}}

\emph{- LL.}

Harry starrte es an und versuchte sich zu erinnern, ob er jemanden mit den Initialen LL kannte.

Sein Verstand suchte…

suchte…

Fand --

"Das \emph{Quibbler}-Mädchen?" flüsterte Harry ungläubig und hielt sich dann den Mund zu. Sie war erst zehn Jahre alt, sie sollte überhaupt nicht in Hogwarts sein!

\emph{Nachspiel: Lesath Lestrange.}

Harry stand um 8 Uhr morgens in dem unbenutzten Klassenzimmer neben Verwandlung und wartete, er hatte es wenigstens geschafft, etwas zu essen in sich hineinzustopfen, bevor er sich der nächsten Katastrophe stellte, Luna Lovegood…

Die Tür zum Klassenzimmer öffnete sich, und Harry sah und verpasste sich einen wirklich \emph{harten} mentalen Tritt.

Noch eine Sache, an die er nicht gedacht hatte, noch eine Sache, an die er \emph{wirklich hätte denken sollen}.

Der grüne Umhang des älteren Jungen war schief, es waren rote Flecken darauf, die aussahen wie kleine Punkte frischen Blutes, und ein Mundwinkel sah aus wie eine Schnittwunde, die durch \emph{Episkey} oder einen anderen kleinen medizinischen Zauber, der den Schaden nicht ganz auslöschte, geheilt worden war.

Lesath Lestranges Gesicht war von Tränen übersät, frischen und halbgetrockneten Tränen, und in seinen Augen stand Wasser, ein Versprechen, dass noch mehr kommen würden. "\emph{Quietus}", sagte der ältere Junge, und dann "\emph{Homenum Revelio}" und einige andere Dinge, während Harry verzweifelt und ohne viel Erfolg nachdachte.

Und dann senkte Lesath seinen Zauberstab und verstaute ihn in seinem Umhang, und langsam, diesmal förmlich, sank der ältere Junge auf dem staubigen Klassenraumboden auf die Knie.

Beugte den Kopf ganz nach unten, bis auch seine Stirn den Staub berührte, und Harry hätte gesprochen, aber er hatte seine Stimme verloren.

Lesath Lestrange sagte mit brechender Stimme: "Mein Leben gehört Euch, mein Herr, und mein Tod ebenso."

"Ich", sagte Harry, ein riesiger Kloß saß ihm im Hals, und er hatte Mühe zu sprechen, "ich -" \emph{hatte nichts damit zu tun}, hätte er es sagen sollen, hätte es \emph{jetzt} sagen sollen, aber der unschuldige Harry hätte auch Mühe gehabt zu sprechen -

"Danke", flüsterte Lesath, "danke, mein Herr, oh, danke", ein ersticktes Schluchzen war von dem knienden Jungen zu hören, alles, was Harry von ihm sehen konnte, waren die Haare an seinem Hinterkopf, nichts von seinem Gesicht. "Ich bin ein Narr, mein Herr, ein undankbarer Bastard, unwürdig, Euch zu dienen, ich kann mich nicht genug erniedrigen, denn ich - ich habe Euch angeschrien, nachdem Ihr mir geholfen habt, weil ich dachte, Ihr würdet mich abweisen, und ich habe erst heute Morgen gemerkt, dass ich so ein Narr war, Euch vor Longbottom zu bitten -"

"Ich hatte nichts damit zu tun", sagte Harry.

(Es war immer noch sehr schwer, so eine glatte Lüge aufzutischen.)

Langsam hob Lesath seinen Kopf vom Boden und sah zu Harry auf.

"Ich verstehe, mein Herr", sagte der ältere Junge, seine Stimme schwankte ein wenig, "du traust mir nicht zu gerissen genug zu sein, und ich habe mich in der Tat als Narr erwiesen… Ich wollte dir nur sagen, dass ich nicht undankbar bin, dass ich weiß, dass es schwer genug gewesen sein muss, nur einen Menschen zu retten, dass sie jetzt alarmiert sind, dass du nicht - Vater holen kannst - aber ich bin nicht undankbar, ich werde nie wieder undankbar dir gegenüber sein. Wenn du jemals eine Verwendung für diesen unwürdigen Diener hast, rufe mich, wo immer ich bin, und ich werde antworten, mein Herr -"

"Ich war in keiner Weise beteiligt."

(Aber es wurde von Mal zu Mal leichter.)

Lesath blickte zu Harry auf und sagte unsicher: "Bin ich entlassen, mein Herr …?"

"Ich bin nicht dein Herr."

Lesath sagte: "Ja, mein Herr, ich verstehe", und erhob sich wieder vom Boden, stand aufrecht und verbeugte sich tief, dann wich er von Harry zurück, bis er sich umdrehte, um die Tür zum Klassenzimmer zu öffnen.

Als Lesaths Hand den Türknauf berührte, hielt er inne.

Harry konnte Lesaths Gesicht nicht sehen, als die Stimme des älteren Jungen sagte: "Hast du sie zu jemandem geschickt, der sich um sie kümmern wird? Hat sie überhaupt nach mir gefragt?"

Und Harry sagte, seine Stimme vollkommen gleichmäßig: "Bitte hör auf damit. Ich war in keiner Weise daran beteiligt."

"Ja, mein Herr, es tut mir leid, mein Herr", sagte Lesaths Stimme, und der Slytherin-Junge öffnete die Tür, ging hinaus und schloss die Tür hinter sich. Seine Füße beschleunigten sich, als er davonlief, aber nicht schnell genug, dass Harry nicht hören konnte, wie er anfing zu schluchzen.

\emph{\emph{Würde ich weinen?} fragte sich Harry. \emph{Wenn ich nichts wüsste, wenn ich unschuldig wäre, würde ich dann jetzt weinen?}}

Harry wusste es nicht, also schaute er einfach weiter auf die Tür.

Und ein unglaublich taktloser Teil von ihm dachte: \emph{Juhu, wir haben eine Quest abgeschlossen und einen Lakaien bekommen} -

\emph{Halt die Klappe. Wenn du jemals wieder über etwas abstimmen willst… Halt die Klappe.}

\emph{\emph{Nachspiel, Amelia Bones:}}

"Dann ist sein Leben nicht in Gefahr, nehme ich an", sagte Amelia.

Der Heiler, ein streng blickender alter Mann, der seine Roben weiß trug (er war ein Muggelgeborener und hielt sich an irgendeine seltsame Tradition der Muggel, nach der Amelia nie gefragt hatte, obwohl sie insgeheim dachte, dass es ihn zu sehr wie einen Geist aussehen ließ), schüttelte den Kopf und sagte: "Definitiv nicht."

Amelia betrachtete die menschliche Gestalt, die bewusstlos auf dem Bett des Heilers ruhte, das verbrannte und zerfetzte Fleisch, das dünne Laken, das ihn aus Gründen des Anstands bedeckt hatte und auf ihren Befehl hin zurückgeschlagen wurde.

Vielleicht erholte er sich vollständig.

Vielleicht aber auch nicht.

Der Heiler hatte gesagt, es sei zu früh, um das zu sagen.

Dann sah Amelia die andere Hexe im Raum an, die Detektivin.

"Und Sie sagen", sagte Amelia, "dass die brennende Materie aus \emph{Wasser} verwandelt wurde, vermutlich in Form von Eis."

Die Detektivin nickte und sagte, verwirrt klingend: "Es hätte viel schlimmer sein können, wenn nicht -"

"Wie \emph{nett} von ihnen", spuckte sie aus und presste dann eine müde Hand an ihre Stirn. Nein … nein, es \emph{war} als Nettigkeit gemeint gewesen. In der letzten Phase der Flucht würde es keinen Sinn mehr haben, jemanden zu täuschen. Wer auch immer das getan hatte, \emph{hatte} also versucht, den Schaden zu begrenzen - und sie hatten dabei an die Auroren gedacht, die den Rauch einatmeten, nicht daran, dass irgendjemand mit dem Feuer angegriffen wurde. Hätte diese Person noch die Kontrolle gehabt, kein Zweifel, sie hätte den Rocker vorsichtiger gelenkt.

Aber Bellatrix Black war allein auf dem Rocker aus Askaban geritten, darin waren sich alle beobachtenden Auroren einig, sie hatten ihre Anti-Desillusionierungszauber aktiviert und es war nur eine Frau auf dem Rocker gewesen, obwohl der Rocker zwei Paar Steigbügel gehabt hatte.

Irgendeine gute und unschuldige Person, die den Patronuszauber wirken konnte, war hereingelegt worden, um Bellatrix Black zu befreien.

Ein Unschuldiger hatte gegen Bahry Einhand gekämpft und einen erfahrenen Auror vorsichtig überwältigt, ohne ihn nennenswert zu verletzen.

Irgendein Unschuldiger hatte den Treibstoff für das Muggelartefakt, mit dem die beiden aus Askaban reiten sollten, aus gefrorenem Wasser zum Schutz ihrer Auroren verwandelt.

Und dann hatte seine oder ihre Nützlichkeit für Bellatrix Black geendet.

Man hätte erwartet, dass jeder, der fähig war, Bahry Einhand zu überwältigen, diesen Teil vorausgesehen hätte. Aber es hätte auch niemand erwartet, dass jemand, der den Patronuszauber wirken konnte, überhaupt erst versuchte, Bellatrix Black zu retten.

Amelia fuhr sich mit der Hand über die Augen und schloss sie für einen Moment in stiller Trauer. \emph{Ich frage mich, wer es war und wie Du-weißt-schon-wer ihn manipuliert hat… welche Geschichte man ihm wohl erzählt haben könnte}…

Sie merkte erst einen Moment später, dass der Gedanke bedeutete, dass sie anfing zu glauben. Vielleicht, weil es, egal wie schwierig es war, Dumbledore zu glauben, immer schwieriger wurde, die Hand dieser kalten, dunklen Intelligenz \emph{nicht} zu erkennen.

\emph{Nachspiel, Albus Dumbledore}:

Es waren vielleicht nur noch siebenundfünfzig Sekunden bis zum Ende des Frühstücks, und er hatte vielleicht vier Umdrehungen seines Zeitumkehrers gebraucht, aber am Ende hatte Albus Dumbledore es doch geschafft.

"Schulleiter?", quietschte die höfliche Stimme von Professor Filius Flitwick, als der alte Zauberer auf dem Weg zu seinem Platz an ihm vorbeiging. "Mr. Potter hat eine Nachricht für Sie hinterlassen."

Der alte Zauberer blieb stehen. Er schaute den Zaubereiprofessor fragend an.

"Mr. Potter sagte, dass ihm nach dem Aufwachen klar geworden ist, wie ungerecht die Dinge waren, die er zu Ihnen gesagt hat, nachdem Fawkes geschrien hat. Mr. Potter sagte, dass er sich nur für diesen einen Teil entschuldigen möchte und nicht für irgendetwas anderes."

Der alte Zauberer sah seinen Zaubereiprofessor weiter an und sprach immer noch nicht.

"Schulleiter?", quietschte Filius.

"Sagen Sie ihm, dass ich mich bedankt habe", sagte Albus Dumbledore, "aber dass es klüger ist, auf Phönixe zu hören als auf weise alte Zauberer", und setzte sich an seinen Platz, drei Sekunden bevor das ganze Essen verschwand.

\emph{Nachspiel, Professor Quirrell}:

"Nein", schnauzte Madam Pomfrey das Kind an, "du darfst ihn \emph{nicht} sehen! Du darfst ihn nicht \emph{belästigen}! Du darfst ihm nicht \emph{nur eine kleine Frage} stellen! Er muss \emph{im Bett} liegen und darf mindestens \emph{drei Tage lang nichts} tun! "

\emph{Nachspiel, Minerva McGonagall}:

Sie war auf dem Weg zum Krankenflügel, und Harry Potter wollte ihn gerade verlassen, als sie aneinander vorbeigingen.

Der Blick, den er ihr zuwarf, war nicht wütend.

Er war nicht traurig.

Er sagte überhaupt nicht viel aus.

Es war wie… als ob er sie gerade lange genug ansah, um deutlich zu machen, dass er es \emph{nicht} absichtlich vermied, sie anzusehen.

Und dann sah er weg, bevor sie sich überlegen konnte, welchen Blick sie ihm erwidern sollte; als ob er ihr auch das ersparen wollte.

Er sagte nichts, als er an ihr vorbeiging.

Sie auch nicht.

Was sollte es da schon zu sagen geben?

\emph{Nachspiel, Fred und George Weasley}:

Sie schrien tatsächlich laut auf, als sie um die Ecke bogen und Dumbledore sahen.

Es war nicht so, dass der Schulleiter aus dem Nichts aufgetaucht war und sie mit strengem Blick anstarrte. \emph{Das} tat Dumbledore immer.

Aber der Zauberer war in formelle schwarze Roben gekleidet und sah \emph{sehr} alt und \emph{sehr} mächtig aus, und er warf den beiden einen STRENGEN BLICK zu.

"Fred und George Weasley!", sprach Dumbledore mit einer Stimme triefend vor Macht.

"Ja, Schulleiter!", sagten sie, richteten sich auf und gaben ihm einen zackigen Militärgruß, wie sie ihn auf alten Bildern gesehen hatten.

"Hört mir gut zu! Ihr seid die Freunde von Harry Potter, richtig?"

"Ja, Schulleiter!"

"Harry Potter ist in Gefahr. Er \emph{darf} die Schutzzauber von Hogwarts \emph{nicht} verlassen. Hört mir zu, Söhne der Weasleys, ich bitte euch: Ihr wisst, dass ich genauso ein Gryffindor bin wie ihr selbst, dass auch ich weiß, dass es höhere Regeln gibt. Aber dies, Fred und George, diese eine Sache ist von furchtbarster Wichtigkeit, es darf diesmal keine Ausnahme geben, klein oder groß!

Wenn ihr Harry helft, Hogwarts zu verlassen, kann er \emph{sterben}! Schickt er euch auf eine Mission, dürft ihr gehen, bittet er euch, ihm Gegenstände zu bringen, dürft ihr helfen, aber wenn er euch bittet, seine eigene Person aus Hogwarts herauszuschmuggeln, \emph{müsst} ihr \emph{ablehnen}! Habt ihr das verstanden?"

"Ja, Schulleiter!" Sie sagten es, ohne wirklich nachzudenken, und tauschten dann unsichere Blicke miteinander aus.

Die strahlend blauen Augen des Schulleiters waren auf sie gerichtet. "Nein. Nicht ohne nachzudenken. Wenn Harry euch bittet, ihn herauszubringen, müsst ihr euch weigern, wenn er euch bittet, ihm den Weg zu zeigen, müsst ihr euch weigern. Ich werde euch nicht bitten, ihn mir zu melden, denn das würdest ihr nie tun. Aber bittet ihn in \emph{meinem} Namen, zu mir zu kommen, wenn es so wichtig ist, und \emph{ich} werde ihn auf seinem Weg bewachen. Fred, George, es tut mir leid, dass ich eure Freundschaft so strapaziere, aber es geht um sein \emph{Leben}."

Die beiden sahen sich lange an, ohne sich zu verständigen, sie dachten nur dasselbe zur selben Zeit.

Dann sahen sie wieder zu Dumbledore.

"Bellatrix Black", sagten sie, wobei ihnen ein Schauer über den Rücken lief, als sie den Namen aussprachen.

"Ihr könnt mit Sicherheit davon ausgehen", sagte der Schulleiter, "dass es mindestens so schlimm ist."

"Okay -"

"- verstanden."

\emph{Nachspiel, Alastor Moody und Severus Snape}:

Als Alastor Moody sein Auge verloren hatte, hatte er die Dienste eines äußerst gelehrten Ravenclaw, Samuel H. Lyall, in Anspruch genommen, dem Moody etwas weniger als dem Durchschnitt misstraute, weil er ihn nicht als unregistrierten Werwolf gemeldet hatte; und er hatte Lyall dafür bezahlt, eine Liste aller bekannten magischen Augen und jedes bekannten Hinweises auf ihren Standort zusammenzustellen.

Als Moody die Liste zurückbekommen hatte, hatte er sich nicht die Mühe gemacht, das meiste davon zu lesen; denn oben auf der Liste stand das Auge von Vance, das aus einer Ära vor Hogwarts stammte und sich derzeit im Besitz eines mächtigen dunklen Zauberers befand, der über ein winziges, vergessenes Höllenloch herrschte, das weder in Großbritannien noch irgendwo anders lag, wo er sich um dumme Regeln kümmern musste.

So hatte Alastor Moody seinen linken Fuß verloren und das Auge von Vance erhalten, und so waren die unterdrückten Seelen von Urulat für einen Zeitraum von etwa zwei Wochen befreit worden, bevor ein anderer Dunkler Zauberer in das Machtvakuum einzog.

Er hatte erwogen, als Nächstes den Linken Fuß von Vance zu suchen, hatte sich aber dagegen entschieden, nachdem ihm klar wurde, dass das \emph{genau das wäre, was sie erwarten würden}.

Jetzt drehte sich Mad-Eye Moody langsam, drehte sich immer weiter, und überblickte den Friedhof von Klein Hangleton. Es hätte viel düsterer sein sollen, dieser Ort, aber im hellen Tageslicht schien er nichts weiter zu sein als ein grasbewachsener Platz, der von gewöhnlichen Grabsteinen markiert war, abgegrenzt durch die angeketteten Windungen aus zerbrechlichem, leicht erklimmbarem Metall, die Muggel anstelle von Schutzzaubern verwendeten. (Moody konnte nicht nachvollziehen, was die Muggel in dieser Hinsicht dachten, ob sie nur so \emph{taten}, als hätten sie Schutzzauber oder was, und er hatte beschlossen, nicht zu fragen, ob Muggelverbrecher den Anschein respektierten.)

Moody \emph{brauchte} sich eigentlich nicht umzudrehen, um den Friedhof zu überblicken.

Das Auge von Vance sah den gesamten Globus der Welt in jeder Richtung um ihn herum, egal wohin es zeigte.

Aber es gab keinen besonderen Grund, einen ehemaligen Todesser wie Severus Snape das wissen zu lassen.

Manchmal nannten die Leute Moody 'paranoid'.

Moody sagte ihnen immer, sie sollten hundert Jahre der Jagd auf Dunkle Zauberer überleben und sich dann bei ihm melden.

Mad-Eye Moody hatte einmal ausgerechnet, wie lange er im Nachhinein betrachtet, gebraucht hatte, um das zu erreichen, was er jetzt als ein anständiges Maß an Vorsicht betrachtete - er wog ab, wie viel Erfahrung er gebraucht hatte, um \emph{gut} zu werden, anstatt \emph{Glück} zu haben - und hatte angefangen zu vermuten, dass die meisten Leute starben, bevor sie es erreichten. Moody hatte diesen Gedanken einmal Lyall gegenüber geäußert, der daraufhin ein paar Berechnungen anstellte und ihm sagte, dass ein typischer Jäger dunkler Magier im Durchschnitt achteinhalb Mal auf dem Weg zum "Paranoid werden" sterben würde. Das erklärte eine ganze Menge, vorausgesetzt, Lyall log nicht.

Gestern hatte Albus Dumbledore Mad-Eye Moody erzählt, dass der Dunkle Lord unaussprechliche dunkle Künste angewandt hatte, um den Tod seines Körpers zu überleben, und nun wach und unterwegs war, um seine Macht wiederzuerlangen und den Zaubererkrieg von neuem zu beginnen.

Jemand anders hätte vielleicht mit Ungläubigkeit reagiert.

"Ich kann nicht glauben, dass ihr mir nie etwas von dieser Auferstehungssache erzählt habt", sagte Mad-Eye Moody mit beträchtlicher Bitterkeit. "Ist euch klar, wie lange ich brauchen werde, um die Gräber aller Vorfahren aller dunklen Zauberer, die ich jemals getötet habe und die schlau genug gewesen sein könnten, einen Horkrux herzustellen, zu bearbeiten? Du machst das doch nicht \emph{erst jetzt}, oder?"

"Ich fülle diesen hier jedes Jahr wieder auf", sagte Severus Snape ruhig, öffnete das dritte Fläschchen von \emph{angeblich} siebzehn Flaschen und begann, seinen Zauberstab darüber zu schwenken. "Die anderen Ahnengräber, die wir ausfindig machen konnten, wurden nur mit den langanhaltenden Substanzen vergiftet, da einige von uns weniger Freizeit haben als du."

Moody beobachtete, wie die Flüssigkeit spiralförmig aus dem Fläschchen floss und verschwand, um in den Knochen zu erscheinen, wo einst Knochenmark gewesen war. "Aber ihr glaubt, dass es den Aufwand der Falle wert ist, anstatt die Knochen einfach verschwinden zu lassen."

"Er \emph{hat} andere Wege um ins Leben zurückzukehren, sollte er diesen als blockiert ansehen", sagte Snape trocken und öffnete eine vierte Flasche. "Und bevor du fragst: Es muss das Originalgrab sein, der Ort der ersten Bestattung, der Knochen muss während des Rituals entfernt werden und nicht vorher. Er kann es also nicht früher geholt haben; und es macht auch keinen Sinn, das Skelett eines schwächeren Vorfahren als Ersatz zu nehmen. Er würde merken, dass es jede Kraft verloren hat."

"Wer weiß noch von dieser Falle?" fragte Moody nach.

"Du. Ich. Der Schulleiter. Sonst niemand."

Moody schnaubte. "Pah. Hat Albus Amelia, Bartemius und dieser Frau McGonagall von dem Auferstehungsritual erzählt?"

"Ja -"

"Wenn Voldie herausfindet, dass Albus von dem Auferstehungsritual weiß und es \emph{ihnen} erzählt hat, wird Voldie sich denken, dass Albus es \emph{mir} erzählt hat, und Voldie \emph{weiß}, dass ich an so etwas denken würde." Moody schüttelte angewidert den Kopf. "Was sind das für andere Möglichkeiten, wie Voldie wieder ins Leben zurückkehren könnte?"

Snapes Hand hielt mit der fünften Flasche inne (es war natürlich alles desillusioniert, die ganze Operation war desillusioniert, aber das bedeutete für Moody weniger als nichts, es markierte einen in seinem Auge nur als 'versucht-sich-zu-verstecken`), und der ehemalige Todesser sagte: "Das musst du nicht wissen."

"Du lernst dazu, mein Sohn", sagte Moody mit milder Zustimmung. "Was ist in den Flaschen?"

Snape öffnete die fünfte Flasche, gestikulierte mit seinem Zauberstab, um die Substanz in Richtung Grab fließen zu lassen, und sagte: "Diese hier? Ein Muggel-Narkotikum namens LSD. Ein Gespräch gestern hat mich an Muggelsachen denken lassen, und LSD schien die interessanteste Option zu sein, also habe ich mich beeilt, etwas davon zu besorgen. Wenn es in den Auferstehungstrank eingearbeitet wird, vermute ich, dass seine Wirkung dauerhaft sein wird."

"Was bewirkt es?", fragte Moody.

"Es heißt, dass die Wirkung für jemanden, der es nicht benutzt hat, unmöglich zu beschreiben ist", sagte Snape gedehnt, "und ich habe es nicht benutzt."

Moody nickte zustimmend, als Snape das sechste Fläschchen öffnete. "Was ist mit dem hier?"

"Liebestrank."

"\emph{Liebestrank?}", sagte Moody.

"Nicht von der üblichen Sorte. Er soll eine wechselseitige Bindung mit einer unerträglich süßen Veela-Frau namens Verdandi auslösen, von der der Schulleiter hofft, dass sie sogar ihn bekehren könnte, wenn sie sich wirklich lieben."

"\emph{Gah!}", sagte Moody. "Dieser verdammte sentimentale Narr -"

"Da stimme ich dir zu", sagte Severus Snape ruhig, seine Aufmerksamkeit auf seine Arbeit gerichtet.

"Sag mir, dass du wenigstens etwas Malaclaw-Gift da drin hast."

"Zweites Fläschchen."

"Iocane-Pulver."

"Entweder die vierzehnte oder die fünfzehnte Flasche."

"Bahls Betäubung", sagte Moody und nannte ein extrem süchtig machendes Narkotikum mit interessanten Nebenwirkungen auf Menschen mit Slytherin-Tendenzen; Moody hatte einmal gesehen, wie ein süchtiger dunkler Zauberer lächerliche Anstrengungen unternahm, um ein Opfer dazu zu bringen, einen bestimmten exakten Portschlüssel in die Hand zu nehmen, anstatt der Zielperson bei ihrem nächsten Besuch in der Stadt einfach einen präparierten Knut zuzuwerfen; und nach all dieser Arbeit hatte sich der Süchtige noch \emph{zusätzlich} die Mühe gemacht, einen zweiten Portus auf \emph{denselben Portschlüssel} zu legen, der das Opfer bei einer zweiten Berührung wieder in Sicherheit brachte. Bis heute konnte sich Moody, selbst unter Berücksichtigung der Droge, nicht vorstellen, was dem Mann durch den Kopf gegangen war, als er den zweiten Portus gezaubert hatte.

"Zehnte Phiole", sagte Snape.

"Basiliskengift", bot Moody an.

"\emph{Was?}", spottete Snape. "Schlangengift ist ein positiver Bestandteil des Auferstehungstranks! Ganz zu schweigen davon, dass es die Knochen und all die anderen Substanzen auflösen würde! Und woher sollten \emph{wir} überhaupt -"

"Beruhige dich, mein Sohn, ich wollte nur sehen, ob man dir trauen kann."

Mad-Eye Moody setzte seine (insgeheim unnötige) langsame Drehung fort und begutachtete den Friedhof, und der Meister der Zaubertränke goss weiter.

"Moment mal", sagte Moody plötzlich. "Woher weißt du, dass \emph{dies} wirklich der Ort ist, an dem -"

"Weil auf dem leicht zu bewegenden Grabstein 'Tom Riddle' steht", sagte Snape trocken. "Und ich habe gerade zehn Sickel vom Schulleiter gewonnen, der gewettet hat, dass dir das vor der fünften Flasche einfallen würde. So viel zur ständigen Wachsamkeit."

Es entstand eine Pause.

"Wie lange hat Albus gebraucht, um zu begreifen -"

"Drei Jahre, nachdem wir von dem Ritual erfahren hatten", sagte Snape in einem Ton, der nicht ganz seinem üblichen sardonischen Tonfall entsprach. "Im Nachhinein betrachtet, hätten wir dich früher konsultieren sollen."

Snape öffnete die neunte Flasche.

"Wir haben auch alle anderen Gräber vergiftet, mit lang anhaltenden Substanzen", bemerkte der ehemalige Todesser. "Es \emph{ist} möglich, dass wir auf dem richtigen Friedhof sind. Vielleicht hat er nicht so weit vorausgeplant, als er seine Familie abschlachtete, und er kann das Grab selbst nicht verlegen -"

"Der richtige Ort sieht nicht mehr wie ein Friedhof aus", sagte Moody schlicht. "Er hat alle \emph{anderen} Gräber hierher verlegt und die Muggel mit einem Gedächtniszauber belegt. Nicht einmal Bellatrix Black hätte etwas davon erfahren, bis kurz vor Beginn des Rituals. \emph{Niemand} außer ihm kennt jetzt den wahren Ort."

Sie setzten ihre vergebliche Arbeit fort.

\emph{Nachspiel, Blaise Zabini}:

Der Slytherin-Gemeinschaftsraum konnte genau und präzise als eine remilitarisierte Zone beschrieben werden; in dem Moment, in dem man durch das Porträtloch trat, sah man, dass die linke Hälfte des Raumes Definitiv Nicht mit der rechten Hälfte Redete und umgekehrt. Es war ganz klar und brauchte niemandem erklärt zu werden, dass man \emph{nicht} die Möglichkeit hatte, \emph{nicht Partei zu ergreifen}.

An einem Tisch genau in der Mitte des Raumes saß Blaise Zabini ganz allein und machte grinsend seine Hausaufgaben. Er hatte jetzt einen Ruf, und er hatte vor, ihn zu behalten.

\emph{Nachspiel, Daphne Greengrass und Tracey Davis}:

"Machst du heute irgendwas Interessantes?", fragte Tracey.

"Nö", sagte Daphne.

\emph{Nachspiel, Harry Potter:}

Wenn man hoch genug in Hogwarts stieg, sah man nicht viele andere Leute um sich herum, nur Korridore und Fenster und Treppen und das gelegentliche Porträt, und ab und zu einen interessanten Anblick, wie zum Beispiel eine Bronzestatue eines pelzigen Wesens, das wie ein kleines Kind aussah und einen seltsamen flachen Speer hielt…

Wenn man hoch genug in Hogwarts stieg, sah man nicht viele andere Menschen, was Harry sehr gelegen kam.

Es gab viel schlimmere Orte, um gefangen zu sein, vermutete Harry. Tatsächlich konnte man sich wahrscheinlich keinen \emph{besseren} Ort vorstellen, um gefangen zu sein, als ein uraltes Schloss mit einer fraktalen, sich ständig verändernden Struktur, die bedeutete, dass einem nie die Orte ausgehen konnten, die man erkunden konnte, voller interessanter Leute und interessanter Bücher und unglaublich wichtigem Wissen, das der Muggelwissenschaft unbekannt war.

Hätte man Harry nicht gesagt, dass er nicht gehen \emph{durfte}, hätte er wahrscheinlich die Chance \emph{ergriffen}, mehr Zeit in Hogwarts zu verbringen, er hätte Ränke und Intrigen geschmiedet, um es zu dürfen. Hogwarts war buchstäblich \emph{optimal}, vielleicht nicht in allen möglichen Dimensionen, aber auf dem realen Planeten Erde war es sicherlich DER Ort für maximalen Spaß.

Wie konnte das Schloss und sein Gelände so viel kleiner, so viel beengender erscheinen, wie konnte der Rest der Welt so viel interessanter und wichtiger werden, in dem Moment, in dem Harry gesagt worden war, dass er nicht gehen dürfe? Er hatte \emph{Monate} hier verbracht und hatte sich \emph{damals} nicht klaustrophobisch gefühlt.

\emph{\emph{Du} kennst \emph{die} \emph{Studien} \emph{darüber}, bemerkte ein Teil von ihm, \emph{es sind nur die üblichen Knappheitseffekte}, \emph{wie damals, als, sobald ein Landkreis Phosphatwaschmittel verbot, Leute, die sich vorher nie darum gekümmert hatten, in den nächsten Landkreis fuhren, um riesige Ladungen Phosphatwaschmittel zu kaufen, und Umfragen zeigten, dass sie Phosphatwaschmittel als sanfter und effektiver und sogar leichter ausgießbar einstuften}… \emph{und wenn man Zweijährige vor die Wahl stellt zwischen einem Spielzeug im Freien und einem, das durch eine Barriere geschützt ist, um die sie herumgehen können, werden sie das Spielzeug im Freien ignorieren und sich für das hinter der Barriere entscheiden}… \emph{Verkäufer wissen, dass sie Dinge verkaufen können, indem sie dem Kunden einfach sagen, dass es vielleicht nicht verfügbar ist}… \emph{das stand alles in Cialdinis Buch} Einfluss, \emph{alles, was du gerade fühlst, das Gras ist immer grüner auf der anderen Straßenseite.}}

Hätte man Harry nicht gesagt, dass er nicht gehen durfte, hätte er wahrscheinlich die Chance \emph{ergriffen}, den Sommer über in Hogwarts zu bleiben…

… aber nicht für den Rest seines Lebens.

Das war das eigentliche Problem.

Wer wusste, ob es noch einen Dunklen Lord Voldemort \emph{gab}, den er besiegen konnte?

Wer wusste, ob Er, dessen Name nicht genannt werden darf, außerhalb der Fantasie eines möglicherweise-nicht-nur-verrückt-spielenden alten Zauberers noch existierte?

Lord Voldemorts Leichnam wurde zu Asche verbrannt aufgefunden, so etwas wie Seelen konnte es nicht wirklich geben. Wie konnte Lord Voldemort noch am Leben sein? Woher \emph{wusste} Dumbledore, dass er noch am Leben war?

Und wenn es keinen Dunklen Lord gab, konnte Harry ihn nicht besiegen und er wäre für immer in Hogwarts gefangen.

… vielleicht würde er nach seinem Abschluss im siebten Schuljahr, in sechs Jahren, vier Monaten und drei Wochen, legal fliehen dürfen. Es war nicht \emph{so} eine lange Zeitspanne, es \emph{schien} nur lang genug, um Protonen zerfallen zu lassen.

Aber es war nicht \emph{nur} das.

Es war nicht \emph{nur} Harrys Freiheit, die auf dem Spiel stand.

Der Schulleiter von Hogwarts, der Oberste Hexenmeister des Zaubergamots, der Oberste Mugwump der Internationalen Konföderation der Zauberer, schlug leise Alarm.

Einen \emph{Fehl}alarm.

Einen falschen Alarm, den \emph{Harry} ausgelöst hatte.

\emph{Weißt du}, sagte der Teil von ihm, der seine Fähigkeiten verfeinerte, \emph{du hast} \emph{doch} \emph{einmal darüber nachgedacht, wie jeder beliebige Beruf eine andere Art hat, exzellent zu sein, wie ein exzellenter Lehrer nicht gleich einem exzellenten Klempner ist; aber sie alle haben bestimmte Methoden gemeinsam, nicht dumm zu sein; und dass eine der wichtigsten dieser Techniken darin besteht, sich seinen kleinen Fehlern zu stellen, bevor sie zu GROSSEN Fehlern werden?}

… obwohl dies eigentlich schon als GROSSER Fehler zählte…

\emph{Der Punkt ist}, sagte sein innerer Beobachter, \emph{dass es buchstäblich von Minute zu Minute schlimmer wird. Die Art und Weise, wie Spione Leute umdrehen, ist, dass sie sie dazu bringen, eine kleine Sünde zu begehen, und dann benutzen sie die kleine Sünde, um sie zu einer größeren Sünde zu erpressen, und dann benutzen sie DIESE, um sie dazu zu bringen, noch größere Dinge zu tun, und dann besitzt der Erpresser ihre Seele.}

\emph{\emph{Du hast doch einmal darüber nachgedacht, wie die Person, die erpresst wird, wenn sie den ganzen Weg vorhersehen könnte, sich einfach entscheiden würde, die Bestrafung beim ersten Schritt zu ertragen, diese erste Sünde zu gestehen? Hast du nicht entschieden, dass du das tun würdest, wenn jemals jemand versuchen würde, dich zu erpressen, etwas Großes zu tun, um etwas Kleines zu verbergen? Siehst du die Ähnlichkeit hier, Harry James Potter-Evans-Verres?}}

Nur dass es keine Kleinigkeit war, es war schon jetzt keine Kleinigkeit, es gäbe eine Menge sehr mächtiger Leute, die extrem wütend auf Harry wären, nicht nur wegen des falschen Alarms, sondern weil er \emph{Bellatrix aus Askaban befreit} hatte, wenn der Dunkle Lord tatsächlich existierte und später hinter ihm her wäre, könnte dieser Krieg schon verloren sein -

\emph{\emph{Glaubst du nicht, dass sie von deiner Ehrlichkeit und Vernunft und Voraussicht beeindruckt sein werden, weil du die Sache gestoppt hast, bevor sie sich noch weiter ausbreitet?}}

Harry dachte das tatsächlich \emph{nicht}; und nach einem Moment des Nachdenkens musste der Teil von ihm, mit dem er gerade sprach, zustimmen, dass dies absurd optimistisch war.

Seine wandernden Füße brachten ihn in die Nähe eines offenen Fensters, und Harry ging hinüber, stützte sich mit den Armen auf dem Sims ab und starrte von hoch oben auf das Gelände von Hogwarts hinunter.

Braun, das waren kahle Bäume, gelb, das war totes Gras, eisfarbenes Eis, das waren zugefrorene Bäche und gefrorene Flüsse… welcher Schulbeamte ihn auch immer "Verbotener Wald" genannt hatte, er hatte wirklich nichts von Marketing verstanden, der Name machte nur noch mehr Lust, dorthin zu gehen. Die Sonne sank bereits, denn Harry dachte nun schon seit einigen Stunden nach, dachte meistens dieselben Gedanken immer und immer wieder, aber jedes Mal mit entscheidenden Unterschieden, als würden sich seine Gedanken nicht im Kreis drehen, sondern eine Spirale hinauf- oder hinabsteigen.

Er konnte immer noch nicht glauben, dass er die \emph{ganze} Sache mit Askaban durchgestanden hatte - er hatte seinen Patronus ausgeschaltet, bevor er ihm das Leben nahm, er hatte einen Auror betäubt, er hatte herausgefunden, wie er Bella vor den Dementoren verstecken konnte, er hatte sich zwölf Dementoren gestellt und sie verscheucht, er hatte den raketengestützten Besen erfunden und war darauf geritten - er hatte die \emph{ganze} Sache durchgestanden, ohne sich auch nur ein \emph{einziges} Mal mit dem Gedanken zu quälen: \emph{Ich muss das tun. … weil… ich Hermine versprochen habe, dass ich vom Mittagessen zurückkomme!} Es fühlte sich an wie eine unwiderruflich verpasste Gelegenheit; als ob er es, nachdem er es \emph{dieses} Mal falsch gemacht hatte, nie wieder \emph{richtig} hinbekommen würde, egal, welcher Art von Herausforderung er das nächste Mal gegenüberstand oder welches Versprechen er gab. Denn dann würde er es nur unbeholfen und absichtlich tun, um wiedergutzumachen, dass er es beim \emph{ersten} Mal verpasst hatte, anstatt die heldenhaften Erklärungen abzugeben, die er hätte abgeben können, wenn er sich an sein Versprechen gegenüber Hermine erinnert hätte. Als wäre diese falsche Abzweigung unwiderruflich, man hatte nur eine Chance, musste es beim ersten Versuch richtig machen…

Er hätte sich an das Versprechen an Hermine erinnern sollen, \emph{bevor} er nach Askaban ging.

Warum hatte er sich noch mal dazu entschlossen das zu tun?

\emph{Meine Arbeitshypothese ist, dass du dumm bist}, sagte Hufflepuff.

\emph{Das ist keine nützliche Fehleranalyse}, dachte Harry.

\emph{Wenn du es genauer wissen willst}, sagte Hufflepuff, \emph{dann hat der Verteidigungsprofessor von Hogwarts gesagt: "Holen wir Bellatrix Black aus Askaban raus!" und du hast gesagt: „Okay!“}

\emph{Moment mal, das ist nicht fair -}

\emph{Hey}, sagte Hufflepuff, \emph{ist dir aufgefallen, dass man, wenn man ganz oben ist und die einzelnen Bäume irgendwie verschwimmen, die Form des Waldes erkennen kann?}

Warum \emph{hatte} er das getan…?

Nicht wegen einer Kosten-Nutzen-Rechnung, das war sicher. Es war ihm zu peinlich gewesen, ein Blatt Papier zu zücken und den zu erwartenden Nutzen zu berechnen, er hatte befürchtet, dass Professor Quirrell ihn nicht mehr respektieren würde, wenn er nein sagte oder auch nur zu sehr zögerte, einer Dame in Not zu helfen.

Er hatte irgendwo tief in seinem Inneren gedacht, dass, wenn dein mysteriöser Lehrer dir die erste Mission, die erste Chance dem Ruf des Abenteuers zu folgen, anbietet und du \emph{nein} sagst, dann geht dein mysteriöser Lehrer angewidert von dir weg, und du bekommst nie wieder eine Chance, ein Held zu sein…

… ja, das war's. Im Nachhinein betrachtet, war es das. Er hatte angefangen zu glauben, sein Leben hätte eine Handlung und hier war eine Wendung, im Gegensatz zu, sagen wir, hier war ein Vorschlag, \emph{Bellatrix Black aus Askaban zu befreien}. Das war der wahre und ursprüngliche Grund für die Entscheidung in dem Sekundenbruchteil gewesen, in dem sie getroffen worden war, wobei sein Gehirn die Erzählung, in der er "Nein" gesagt hatte, als dissonant wahrgenommen hatte. Und wenn man darüber nachdachte, war das keine rationale Art, Entscheidungen zu treffen. Professor Quirrells Hintergedanke, die letzten Reste von Slytherins verlorenem Wissen zu erhalten, bevor Bellatrix starb und es unwiderruflich in Vergessenheit geriet, erschien im Vergleich dazu beeindruckend vernünftig; ein Nutzen, der dem, was damals als kleines Risiko erschienen war, angemessen war.

Es schien nicht fair, es schien nicht \emph{fair}, dass \emph{dies} geschah, wenn er nur für einen winzigen Bruchteil einer Sekunde den Halt an seiner Rationalität verlor, den winzigen Bruchteil einer Sekunde, den sein Gehirn brauchte, um zu entscheiden, dass ihm Ja-Argumente angenehmer waren als Nein-Argumente während der Diskussion, die gefolgt war.

Von hoch oben, weit genug oben, dass die einzelnen Bäume miteinander verschwammen, starrte Harry auf den Wald hinaus.

Harry \emph{wollte} nicht gestehen und seinen Ruf für immer ruinieren und alle wütend auf ihn machen und vielleicht später vom Dunklen Lord getötet werden. Lieber wäre er sechs Jahre lang in Hogwarts gefangen, als sich dem zu stellen. So fühlte er sich. Und so war es in der Tat hilfreich, eine Erleichterung, sich an einen einzigen entscheidenden Faktor klammern zu können, nämlich dass, wenn Harry gestand, Professor Quirrell nach Askaban gehen und dort sterben würde.

(Ein Stocken, eine Pause, ein Stottern in Harrys Atem.)

Wenn man es \emph{so} formulierte… konnte man sogar vorgeben, ein Held zu sein, anstatt eines Feiglings.

Harry hob seinen Blick vom Verbotenen Wald und sah hinauf in den klaren, blauen, verbotenen Himmel.

Starrte aus den Glasscheiben auf das große, helle, brennende Ding, die flauschigen Dinger, das geheimnisvolle, endlose Blau, in das sie eingebettet waren, diesen seltsamen, neuen, unbekannten Ort.

Es… half tatsächlich, es half eine ganze Menge, zu denken, dass seine eigenen Probleme nichts waren im Vergleich dazu, in Askaban zu sein. Dass es Menschen auf der Welt gab, die \emph{wirklich} in Schwierigkeiten steckten, und dass Harry Potter nicht zu ihnen gehörte.

Was sollte er gegen Askaban unternehmen?

Was sollte er gegen das magische Britannien unternehmen?

… auf welcher Seite stand er jetzt?

Im hellen Licht des Tages \emph{klang} alles, was Albus Dumbledore gesagt hatte, sicherlich viel weiser als Professor Quirrell. Besser und heller, moralischer, \emph{bequemer}, wäre es nicht schön, wenn es wahr wäre. Und die Sache, die man dabei im Kopf behalten musste war, dass Dumbledore die Dinge glaubte, \emph{weil} sie schön klangen, aber Professor Quirrell war derjenige, der \emph{zurechnungsfähig} war.

(Wieder das Stocken in seinem Atem, es passierte jedes Mal, wenn er an Professor Quirrell dachte.)

Aber nur weil sich etwas nett anhörte, war es auch nicht \emph{falsch}.

Und wenn der Verteidigungsprofessor einen Makel in seiner Zurechnungsfähigkeit \emph{hatte}, dann war es, dass er das Leben \emph{zu negativ} sah.

\emph{Wirklich?} fragte sich der Teil von Harry, der achtzehn Millionen Versuchsergebnisse darüber gelesen hatte, dass Menschen zu optimistisch und zu zuversichtlich sind. \emph{Professor Quirrell ist zu pessimistisch? So pessimistisch, dass seine Erwartungen routinemäßig hinter der Realität} zurückbleiben\emph{? Stopft ihn aus und steckt ihn in ein Museum, er ist einzigartig. Wer von euch beiden hat das perfekte Verbrechen geplant und} dann \emph{all die Spielräume für Fehler und Rückschläge eingebaut, die euch am Ende den Arsch gerettet haben,} nur für den Fall\emph{, dass das perfekte Verbrechen schief geht?} \emph{Tipp:} \emph{sein Name war nicht Harry Potter.}

Aber "pessimistisch" war nicht das richtige Wort, um Professor Quirrells Problem zu beschreiben - wenn es denn wirklich ein Problem war, und nicht die überlegene Weisheit der Erfahrung. Aber für Harry sah es so aus, als würde Professor Quirrell ständig alles im schlechtesten möglichen Licht interpretieren. Wenn man Professor Quirrell ein Glas reichte, das zu 90\% voll war, würde er einem sagen, dass der 10\% leere Teil bewies, dass sich niemand \emph{wirklich} für Wasser interessierte.

Das war eine sehr gute Analogie, jetzt, wo Harry darüber nachdachte. Nicht das ganze magische Britannien war wie Askaban, das Glas war weit mehr als halb voll…

Harry starrte hinauf in den strahlend blauen Himmel.

… obwohl, der Analogie \emph{folgend}, wenn Askaban existierte, dann \emph{bewies} es vielleicht, dass der 90\% gute Teil aus anderen Gründen dort war, Menschen, die versuchten, \emph{eine Show} \emph{mit ihrer} \emph{Freundlichkeit abzuziehen}, wie Professor Quirrell es ausgedrückt hatte. Denn wenn sie wirklich gütig wären, hätten sie Askaban nicht erschaffen, sie würden die Festung stürmen, um sie niederzureißen… oder nicht?

Harry starrte hinauf in den strahlend blauen Himmel. Wenn man ein Rationalist sein wollte, musste man furchtbar viele Abhandlungen über Fehler in der menschlichen Natur lesen, und einige dieser Fehler waren unschuldige logische Ausfälle, und einige von ihnen sahen sehr viel dunkler aus.

Harry starrte hinauf in den strahlend blauen Himmel und dachte an das Milgram-Experiment.

Stanley Milgram hatte es durchgeführt, um die Ursachen des Zweiten Weltkriegs zu erforschen, um zu verstehen, warum die Bürger Deutschlands Hitler gehorcht hatten.

Er hatte also ein Experiment entworfen, um den \emph{Gehorsam} zu untersuchen, um zu sehen, ob die Deutschen aus irgendeinem Grund eher dazu neigten, schädliche Befehle von Autoritätspersonen zu befolgen.

Zuerst hatte er eine Pilotversion seines Experiments an amerikanischen Probanden durchgeführt, als Kontrolle.

Und danach hatte er sich nicht die Mühe gemacht, es in Deutschland zu versuchen.

Versuchsapparat: Eine Reihe von 30 Schaltern, die in einer horizontalen Linie angeordnet waren, mit Beschriftungen, die bei "15 Volt" begannen und bis zu "450 Volt" reichten, mit Beschriftungen für jede Gruppe von vier Schaltern. Die erste Gruppe von vier Schaltern ist mit "Leichter Schock" beschriftet, die sechste Gruppe mit "Schock mit extremer Intensität", die siebte Gruppe mit „Gefahr: Schwerer Schock“, und die beiden letzten Schalter, die übrig blieben, waren einfach mit „XXX“ beschriftet.

Und ein Schauspieler, ein Vertrauter des Versuchsleiters, der den echten Versuchspersonen als jemand wie sie erschienen war: jemand, der auf die gleiche Anzeige für Teilnehmer an einem Lernexperiment geantwortet hatte, und der bei einer (manipulierten) Lotterie verloren hatte und zusammen mit den Elektroden auf einen Stuhl geschnallt worden war. Die echten Versuchspersonen hatten einen leichten Schock von den Elektroden bekommen, nur damit sie sehen konnten, dass es funktionierte.

Dem wahren Probanden war gesagt worden, dass das Experiment die Auswirkungen von Bestrafung auf Lernen und Gedächtnis betraf und dass ein Teil des Tests darin bestand, zu sehen, ob es einen Unterschied machte, welche Art von Person die Bestrafung verabreichte; und dass die Person, die an den Stuhl geschnallt war, versuchen würde, sich Sätze von Wortpaaren einzuprägen, und dass jedes Mal, wenn der "Lernende" eine falsche Antwort bekam, der "Lehrer" einen sukzessive stärkeren Schock verabreichen sollte.

Bei der 300-Volt-Stufe hörte der Akteur auf, Antworten zu rufen und begann, gegen die Wand zu treten, woraufhin der Versuchsleiter die Versuchspersonen anwies, Nicht-Antworten als falsche Antworten zu behandeln und weiterzumachen.

Beim 315-Volt-Pegel würde das Hämmern gegen die Wand wiederholt werden.

Danach war nichts mehr zu hören.

Wenn die Versuchsperson widersprach oder sich weigerte, einen Schalter zu drücken, sagte der Versuchsleiter, der eine teilnahmslose Haltung bewahrte und einen grauen Laborkittel trug: "Bitte fahren Sie fort", dann: "Das Experiment erfordert, dass Sie fortfahren", dann: "Es ist absolut notwendig, dass Sie fortfahren", dann: "Sie haben keine andere Wahl, Sie müssen fortfahren". Wenn die vierte Aufforderung immer noch nicht funktionierte, wurde das Experiment dort abgebrochen.

Vor der Durchführung des Experiments hatte Milgram den Versuchsaufbau beschrieben und dann vierzehn Psychologie-Studenten gefragt, wie viel Prozent der Versuchspersonen ihrer Meinung nach den ganzen Weg bis zur 450-Volt-Stufe gehen würden, wie viel Prozent der Versuchspersonen den letzten der beiden mit XXX markierten Schalter drücken würden, nachdem das Opfer aufgehört hatte zu reagieren.

Die pessimistischste Antwort war 3\%.

Die tatsächliche Zahl lag bei 26 von 40.

Die Versuchspersonen schwitzten, stöhnten, stotterten, lachten nervös, bissen sich auf die Lippen, gruben ihre Fingernägel in ihr Fleisch. Aber auf die Aufforderung des Versuchsleiters hin hatten sie, die meisten von ihnen, weitergemacht und das verabreicht, was sie für schmerzhafte, gefährliche, möglicherweise tödliche Elektroschocks hielten. Bis zum Ende.

\emph{Harry konnte Professor Quirrell im Geiste lachen hören; die Stimme des Verteidigungsprofessors sagte etwas in der Art von: \emph{Nun, Mr. Potter, selbst ich wäre} \emph{nicht so zynisch gewesen; ich wusste, dass Männer ihre wertvollsten Prinzipien für Geld und Macht verraten würden, aber mir war nicht klar, dass auch ein strenger Blick genügte.}}

Es war gefährlich, zu versuchen, evolutionäre Psychologie zu erraten, wenn man kein professioneller Evolutionspsychologe war; aber als Harry über das Milgram-Experiment gelesen hatte, war ihm der Gedanke gekommen, dass Situationen wie diese in der Umgebung der Vorfahren wahrscheinlich oft vorgekommen waren, und dass die meisten potenziellen Vorfahren, die versucht hatten, der Autorität zu widersprechen, tot waren. Oder dass es ihnen zumindest weniger gut ergangen war als den Gehorsamen. Die Menschen \emph{hielten} sich selbst für gut und moralisch, aber wenn es darauf ankam, legte sich irgendein Schalter in ihrem Gehirn um, und es war plötzlich viel schwieriger, sich der Autorität heldenhaft zu widersetzen, als sie dachten. Selbst wenn man es schaffen konnte, war es nicht einfach, es wäre keine mühelose Zurschaustellung von Heldentum. Du würdest zittern, deine Stimme würde brechen, du würdest Angst haben; würdest du selbst dann in der Lage sein, der Autorität zu trotzen?

Harry blinzelte, denn sein Gehirn hatte gerade die Verbindung zwischen Milgrams Experiment und dem hergestellt, was Hermine an ihrem ersten Tag im Verteidigungsunterricht getan hatte: Sie hatte sich geweigert, einen Mitschüler zu erschießen, selbst als die Autorität ihr gesagt hatte, dass sie es tun müsse, sie hatte gezittert und Angst gehabt, aber sie hatte sich trotzdem geweigert. Harry hatte das direkt vor seinen eigenen Augen gesehen, und er hatte die Verbindung bis jetzt noch nicht hergestellt…

Harry starrte auf den roter werdenden Horizont, die Sonne sank tiefer, der Himmel verblasste, verdunkelte sich, auch wenn der größte Teil noch blau war, würde es bald Nacht werden. Die goldenen und roten Farben der Sonne und des Sonnenuntergangs erinnerten ihn an Fawkes; und Harry fragte sich einen Moment lang, ob es traurig sein musste, ein Phönix zu sein und zu rufen und zu schreien, ohne beachtet zu werden.

Aber Fawkes würde niemals aufgeben, so oft er auch starb, er würde immer wiedergeboren werden, denn Fawkes war ein Wesen aus Licht und Feuer, und die über Askaban zu verzweifeln gehörte genauso zur Dunkelheit wie Askaban selbst.

Wenn man dir ein Glas gab, das halb leer und halb voll war, dann war das, wie die Realität war, das war die Wahrheit und es war so; aber du hattest immer noch die Wahl, wie du dich dabei \emph{fühlen} wolltest, ob du über die leere Hälfte verzweifeln oder dich über das Wasser freuen würdest, das da war.

Milgram hatte einige andere Variationen seines Tests ausprobiert.

Im achtzehnten Experiment brauchte die Versuchsperson dem auf dem Stuhl festgeschnallten Opfer nur die Testwörter zuzurufen und die Antworten zu notieren, während jemand \emph{anderes} die Schalter drückte. Es war das gleiche offensichtliche Leiden, das gleiche hektische Hämmern, gefolgt von Stille; aber es warst nicht du, der den Schalter drückte. \emph{Du} schautest nur zu und hast der gequälten Person die Fragen vorgelesen.

37 von 40 Versuchspersonen hatten ihre Teilnahme an diesem Experiment bis zum Ende fortgesetzt, dem 450-Volt-Ende, das mit "XXX" markiert war.

Und wenn du Professor Quirrell wärst, hättest du vielleicht beschlossen, das zynisch zu finden.

Aber 3 von 40 Probanden hatten sich \emph{geweigert}, bis zum Ende mitzumachen.

Die Hermines.

Es gab sie, in der Welt, die Leute, die keinen einfachen Schlagzauber auf einen Mitschüler abfeuern würden, selbst wenn der Verteidigungsprofessor es ihnen befehlen würde. Diejenigen, die während des Holocausts Zigeuner, Juden und Homosexuelle auf ihren Dachböden beherbergt hatten und dafür manchmal ihr Leben verloren.

Gehörten diese Leute zu einer anderen Spezies als die Menschheit? Hatten sie eine zusätzliche Gehirnwindung, ein zusätzliches Stück neuronaler Schaltkreise, das die weniger Sterblichen nicht besaßen? Aber das war unwahrscheinlich, angesichts der Logik der sexuellen Fortpflanzung, die besagte, dass die Gene für komplexe Maschinerie irreparabel durcheinandergebracht werden würden, wenn sie nicht universell wären.

Aus welchen Teilen Hermine auch immer gemacht war, jeder hatte dieselben Teile irgendwo in sich…

… naja, das war ein netter Gedanke, aber er stimmte \emph{strenggenommen} nicht, es gab so etwas wie buchstäbliche Hirnschäden, Menschen konnten Gene \emph{verlieren} und die komplexe Maschinerie konnte aufhören zu funktionieren, es gab Soziopathen und Psychopathen, Menschen, denen die Fähigkeit mitzufühlen fehlte. Vielleicht war Lord Voldemort so geboren worden, oder vielleicht hatte er das Gute gekannt und sich dennoch für das Böse entschieden; an diesem Punkt spielte es nicht die geringste Rolle. Aber eine \emph{übergroße Mehrheit} der Bevölkerung sollte in der Lage sein, zu lernen das zu tun, was Hermine und die Holocaust-Verweigerer taten.

Die Leute, die das Milgram-Experiment durchlaufen hatten, die gezittert und geschwitzt und nervös gelacht hatten, als sie bis zum Drücken der Schalter mit der Aufschrift "XXX" vorgedrungen waren, viele von ihnen hatten Milgram hinterher geschrieben, um sich dafür zu bedanken, was sie über sich selbst gelernt hatten. Auch das war Teil der Geschichte, der Legende dieses legendären Experiments.

Die Sonne war jetzt fast unter den Horizont gesunken, ein letzter goldener Zipfel lugte über die fernen Baumwipfel.

Harry schaute sie an, diese Sonnenspitze, seine Brille sollte gegen UV-Strahlung geschützt sein, so dass er direkt in sie schauen konnte, ohne seine Augen zu schädigen.

Harry starrte direkt darauf, auf den winzigen Bruchteil des Lichts, der nicht verdeckt und blockiert und versteckt war, auch wenn es nur 3 Teile von 40 waren, die anderen 37 Teile waren irgendwo da. Die 7,5 \% des Glases, die voll waren, was bewies, dass die Menschen sich wirklich um das Wasser kümmerten, auch wenn diese Kraft der Fürsorge in ihnen selbst zu oft besiegt wurde. Wenn die Menschen sich wirklich nicht gekümmert hätten, wäre das Glas wirklich leer gewesen. Wenn jeder innerlich wie Du-weißt-schon-wer gewesen wäre, insgeheim klug und egoistisch, dann hätte es überhaupt keinen Widerstand gegen den Holocaust gegeben.

Harry schaute in den Sonnenuntergang, am zweiten Tag des Restes seines Lebens, und wusste, dass er die Seiten gewechselt hatte.

Weil er nicht mehr daran glauben konnte, er konnte es wirklich nicht, nicht nachdem er in Askaban gewesen war. Er konnte nicht tun, wofür 37 von 40 Leuten ihn wählen würden. Jeder mochte in sich haben, was es brauchte, um Hermine zu sein, und eines Tages würden sie es vielleicht erfahren; aber \emph{eines Tages} war nicht \emph{jetzt}, nicht hier, nicht heute, nicht in der realen Welt. Wenn man auf der Seite von drei von vierzig Leuten stand, dann war man keine politische Mehrheit, und Professor Quirrell hatte Recht gehabt, Harry würde seinen Kopf nicht in Unterwerfung neigen, wenn das geschah.

Es hatte eine Art von schrecklicher Angemessenheit an sich. Man sollte nicht nach Askaban gehen und zurückkommen, ohne seine Meinung über etwas Wichtiges geändert zu haben.

\emph{\emph{Hat Professor Quirrell also recht}? fragte Slytherin. \emph{Mal abgesehen davon, ob er gut oder böse ist, hat er} Recht\emph{? Bist du für sie, ob sie es wissen oder nicht, ihr nächster Lord? Wir} \emph{lassen den dunklen Teil einfach weg, da ist er wieder zynisch. Aber ist deine Absicht jetzt zu herrschen? Ich muss sagen, das macht sogar} mich \emph{nervös.}}

\emph{\emph{Glaubst du, man kann dir Macht anvertrauen?} sagte Gryffindor. \emph{Gibt es nicht irgendeine Regel, dass Leute, die Macht wollen, sie nicht haben sollten? Vielleicht sollten wir stattdessen Hermine zur Herrscherin machen.}}

\emph{\emph{Glaubst du, du bist in der Lage, eine Gesellschaft zu leiten, ohne dass sie innerhalb von drei Wochen im totalen Chaos versinkt?} sagte Hufflepuff. \emph{Stell dir vor, wie laut Mum schreien würde, wenn sie hören würde, dass du zum Premierminister gewählt wurdest, und jetzt frag dich, ob sie damit wirklich falsch liegt?}}

\emph{Eigentlich}, sagte Ravenclaw, \emph{muss ich darauf hinweisen, dass dieser ganze politische Kram überwältigend langweilig klingt. Wie wär's, wenn wir den ganzen Wahlkampf Draco überlassen und uns an die Wissenschaft halten? Darin sind wir eigentlich gut, und es ist bekannt, dass das auch den Zustand der Menschheit verbessert, wisst ihr.}

\emph{\emph{Langsam}, dachte Harry an seinen Teile gerichtet, \emph{wir müssen nicht gleich alles entscheiden. Wir dürfen das Problem so gut wie möglich durchdenken, bevor wir zu einer Lösung kommen.}}

Der letzte Teil der Sonne sank unter den Horizont.

Es war seltsam, dieses Gefühl, nicht so recht zu wissen, wer man war, auf welcher Seite man stand, sich über etwas so Wesentliches \emph{noch nicht entschieden zu haben}, es lag ein ungewohntes Gefühl von Freiheit darin…

Und das erinnerte ihn an das, was Professor Quirrell auf seine letzte Frage geantwortet hatte, was ihn an Professor Quirrell erinnerte, was ihm wieder das Atmen schwer machte, dieses Brennen in Harrys Kehle auslöste, seine Gedanken wieder in diese Schleife der Kletterspirale schickte.

Warum war er so traurig, wann immer er an Professor Quirrell dachte? Harry war es gewohnt, sich selbst zu kennen, und er wusste nicht, warum er so traurig war…

Es fühlte sich an, als hätte er Professor Quirrell für immer verloren, ihn in Askaban verloren, so fühlte es sich an. So sicher, als wäre der Verteidigungsprofessor von Dementoren gefressen worden, verzehrt in den leeren Räumen.

\emph{Ich habe ihn verloren! Warum habe ich ihn verloren? Weil er Avada Kedavra gesagt hat und es in Wirklichkeit einen ganz guten Grund gab, auch wenn ich ihn ein paar Stunden lang nicht erkannt habe? Warum können die Dinge nicht wieder so werden, wie sie waren?}

Aber es \emph{lag nicht} am Avada Kedavra. Das mochte eine Rolle dabei gespielt haben, eine Struktur von Rationalisierungen und Zurückweichen und vorsichtigem Nicht-Denken über bestimmte Dinge irreversibel zum Einsturz zu bringen. Aber es war nicht das Avada Kedavra gewesen, das war nicht das Beunruhigende gewesen, das Harry gesehen hatte.

\emph{Was habe ich gesehen…?}

Harry blickte in den verblassenden Himmel.

Er hatte gesehen, wie Professor Quirrell sich in einen hartgesottenen Verbrecher verwandelte, während er dem Auror gegenüberstand, und der scheinbare Wechsel der Persönlichkeiten war mühelos und vollständig gewesen.

Eine andere Frau hatte den Verteidigungsprofessor als 'Jeremy Jaffe' gekannt.

\emph{Wie viele verschiedene Menschen sind Sie eigentlich?}

\emph{Ich kann nicht behaupten, dass ich mir die Mühe gemacht habe, mitzuzählen.}

Man konnte nicht umhin, sich zu fragen…

… ob "Professor Quirrell" nur ein weiterer Name auf der Liste war, nur eine weitere Person, in die er sich \emph{verwandelt} hatte, die er erfunden hatte, um irgendeinem unerklärlichen Ziel zu dienen.

Harry würde sich jetzt immer fragen, jedes Mal, wenn er mit Professor Quirrell sprach, ob es eine Maske war und welches Motiv hinter dieser Maske steckte. Mit jedem trockenen Lächeln würde Harry versuchen zu erkennen, was die Hebel der Lippen betätigte.

\emph{Werden andere Leute anfangen, so von mir zu denken, wenn ich zu sehr Slytherin werde? Wenn ich zu viele Intrigen spinne, werde ich dann nie wieder jemanden anlächeln können, ohne dass sie sich fragen, was ich wirklich damit meine?}

Vielleicht gab es eine Möglichkeit, das Vertrauen in den äußeren Schein wiederherzustellen und eine normale menschliche Beziehung wieder möglich zu machen, aber Harry fiel nicht ein, was das sein könnte.

So hatte Harry Professor Quirrell verloren, nicht die Person, aber die… Verbindung…

Warum tat das so sehr weh?

Warum fühlte er sich jetzt so einsam?

Sicherlich gab es andere Menschen, vielleicht bessere Menschen, denen er vertrauen und mit denen er sich anfreunden konnte? Professor McGonagall, Professor Flitwick, Hermine, Draco, ganz zu schweigen von Mum und Dad, Harry war ja nicht \emph{allein}…

Nur…

Ein erstickendes Gefühl stieg in Harrys Kehle auf, als er verstand.

Nur Professor McGonagall, Professor Flitwick, Hermine, Draco, sie alle wussten manchmal Dinge, die Harry nicht wusste, aber…

Sie überragten Harry nicht \emph{in seinem eigenen} Machtbereich; das Genie, das sie besaßen, war nicht wie sein Genie, und sein Genie war nicht wie ihres; er konnte sie als Gleichgestellte ansehen, aber nicht als seine \emph{Vorgesetzten}.

Keiner von ihnen war, keiner von ihnen konnte jemals…

Harrys Mentor…

Das war es, wer Professor Quirrell gewesen war.

Das war es, wen Harry verloren hatte.

Und die Art und Weise, wie er seinen ersten Mentor verloren hatte, würde es Harry vielleicht erlauben, ihn irgendwann zurückzubekommen, oder auch nicht. Vielleicht würde er eines Tages alle verborgenen Absichten von Professor Quirrell kennen und die Zweifel zwischen ihnen würden verschwinden; aber selbst wenn das möglich wäre, schien es nicht sehr wahrscheinlich zu sein.

Es gab einen Windstoß, außerhalb von Hogwarts, er beugte die leeren Bäume, kräuselte den See, dessen Herz noch nicht gefroren war, machte ein flüsterndes Geräusch, als er an dem Fenster vorbeiging, das auf die halbgefrorene Welt blickte, und Harrys Gedanken wanderten eine Zeit lang nach außen.

Dann kehrten sie wieder nach innen zurück, zur nächsten Stufe der Spirale.

\emph{Warum bin ich anders als die anderen Kinder in meinem Alter?}

Wenn Professor Quirrells Antwort darauf eine Ausflucht gewesen war, dann war es eine sehr gut kalkulierte. Tiefgründig und komplex genug, voll von Andeutungen versteckter Bedeutungen, um als Falle für einen Ravenclaw zu dienen, der sich nicht durch weniger ablenken ließ. Oder vielleicht hatte Professor Quirrell seine Antwort ehrlich gemeint. Wer wusste schon, welches Motiv den Hebel für diese Lippen betätigt haben könnte?

\emph{\emph{So viel kann ich sagen, Mr. Potter: Sie sind bereits ein Okklumentiker, und ich denke, Sie werden bald ein perfekter Okklumentiker} \emph{sein. Identität bedeutet für solche wie uns nicht das, was sie für andere Menschen bedeutet.} \emph{Wir können} \emph{jeder sein, den wir uns vorstellen können; und der wahre Unterschied zu Ihnen, Mr. Potter, ist, dass Sie eine ungewöhnlich gute Vorstellungskraft haben. Ein Dramatiker muss seine Figuren in sich aufnehmen, er muss größer sein als sie, um sie in seinem Geist darstellen zu können. Für einen Schauspieler oder Spion oder Politiker ist die seine eigene Grenze die Grenze dessen, wer er vorgeben kann zu sein, die Grenze dessen, welches Gesicht er als Maske tragen} \emph{kann.}}

\emph{\emph{Aber für solche wie Sie und} \emph{mich,} \emph{wir können jeder sein, den wir uns vorstellen können, in der Realität und nicht} \emph{nur in der Vorstellung. Während Sie sich} \emph{selbst als} \emph{Kind vorstellten, Mr. Potter,} waren \emph{Sie ein Kind. Und doch gibt es andere} \emph{Arten des Daseins, die Sie} \emph{tragen} \emph{könnten, größere} \emph{Arten des Daseins, wenn Sie es wollten. Warum} \emph{sind Sie} \emph{so frei und so groß in} \emph{Ihren Möglichkeiten, während andere Kinder} \emph{Ihres} \emph{Alters klein und beschränkt sind? Warum können Sie sich} \emph{ein erwachseneres Selbst vorstellen und} werden\emph{, als ein bloßes Kind eines Dramatikers zu} \emph{erschaffen} \emph{imstande sein sollte? Das weiß ich nicht, und ich darf nicht sagen, was ich vermute. Aber was Sie haben, Mr. Potter, ist Freiheit.}}

Wenn das Augenwischerei war, dann war es eine verdammt ablenkende.

Und der noch beunruhigendere Gedanke war, dass Professor Quirrell nicht \emph{erkannt} hatte, wie verstört Harry sein würde, wie \emph{falsch} diese Rede für ihn klingen würde, wie sehr sie sein Vertrauen in Professor Quirrell beschädigen würde.

Es sollte immer eine wirkliche Person geben, die man \emph{wirklich} war, die im Zentrum von allem stand…

Harry starrte hinaus in die hereinbrechende Nacht, in die zunehmende Dunkelheit.

… oder?

Es war schon fast Schlafenszeit, als Hermine die verstreuten Atemzüge hörte und von ihrem Exemplar von \emph{Beauxbatons: Geschichte} aufblickte, um den vermissten Jungen zu sehen, den Jungen, der an jenem Sonntag beim Mittagessen woanders war, dessen Nichterscheinen beim Abendessen von Gerüchten begleitet worden war - und sie hatte ihnen nicht geglaubt, weil sie \emph{vollkommen} \emph{lächerlich} waren, aber sie hatte ein mulmiges Gefühl in sich gespürt -, dass er aus Hogwarts ausgeschieden war, um Bellatrix Black zur Strecke zu bringen.

"\emph{Harry!}", kreischte sie und merkte gar nicht, dass sie zum ersten Mal seit einer Woche direkt mit ihm sprach, oder wie einige andere Schüler aufschreckten, weil sie quer durch den Ravenclaw-Gemeinschaftsraum brüllte.

Harrys Blick lag bereits auf ihr, er ging bereits auf sie zu, so dass sie auf halbem Weg aus ihrem Stuhl stehen blieb -

Wenige Augenblicke später saß Harry ihr gegenüber, und er steckte seinen Zauberstab weg, nachdem er eine \emph{Quietus}-Barriere um sie herum gewirkt hatte.

(Und eine ganze Menge Ravenclaws versuchten, nicht so auszusehen, als würden sie zusehen.)

"Hey", sagte Harry. Seine Stimme schwankte. "Ich habe dich vermisst. Wirst du … jetzt wieder mit mir reden?"

Hermine nickte nur, sie wusste nicht, was sie sagen sollte. Sie hatte Harry auch vermisst, aber sie stellte mit einer Art Schuldgefühl fest, dass es für ihn vielleicht noch viel schlimmer gewesen war. Sie hatte andere Freunde, Harry … es fühlte sich manchmal nicht \emph{fair} an, dass Harry nur mit ihr so redete, so dass sie mit ihm reden \emph{musste}; aber Harry sah aus, als wären auch \emph{ihm} unfaire Dinge widerfahren.

"Was war denn \emph{los}? ", sagte sie. "Es gibt alle möglichen Gerüchte. Es gab Leute, die sagten, du wärst abgehauen, um gegen Bellatrix Black zu kämpfen, es gab Leute, die sagten, du wärst abgehauen, um dich Bellatrix Black \emph{anzuschließen} -" und \emph{diese} Gerüchte hatten besagt, dass Hermine die Sache mit dem Phönix nur erfunden hatte, und sie hatte geschrien, dass der ganze Ravenclaw-Gemeinschaftsraum es gesehen hatte, und dann hatte das nächste Gerücht behauptet, sie hätte auch \emph{diesen} Teil erfunden, was eine Dummheit von so unvorstellbarem Ausmaß war, dass es sie \emph{völlig verblüffte}.

"Ich kann nicht darüber reden", sagte Harry im leisen Flüsterton. "Ich kann über vieles nicht reden. Ich wünschte, ich könnte dir alles erzählen", seine Stimme schwankte, "aber ich kann nicht … Ich schätze, wenn es hilft oder so, ich gehe nicht mehr zum Mittagessen mit Professor Quirrell…"

Harry legte seine Hände über sein Gesicht und bedeckte seine Augen.

Hermine spürte ein mulmiges Gefühl im Magen.

"Weinst du etwa?", fragte Hermine.

"Ja", sagte Harry, seine Stimme klang ein wenig belegt. "Ich will nicht, dass es jemand anderes sieht."

Es herrschte ein kurzes Schweigen. Hermine wollte helfen, aber sie wusste nicht, was sie bei einem weinenden Jungen tun sollte, und sie wusste nicht, was vor sich ging; sie hatte das Gefühl, dass um sie herum - nein, um Harry herum - große Dinge geschahen, und wenn sie wüsste, was es war, würde sie wahrscheinlich erschrecken, oder beunruhigt sein, oder so, aber sie wusste nichts.

"Hat Professor Quirrell etwas falsch gemacht?", fragte sie schließlich.

"Das ist nicht der Grund, warum ich nicht mehr mit ihm zum Mittagessen gehen kann", sagte Harry, immer noch in diesem knappen Flüsterton, die Hände über die Augen gepresst. "Das war die Entscheidung des Schulleiters. Aber ja, Professor Quirrell hat einige Dinge zu mir gesagt, die mich dazu gebracht haben, ihm weniger zu vertrauen, denke ich…" Harrys Stimme klang zittrig. "Ich fühle mich im Moment ziemlich allein."

Hermine legte ihre Hand auf die Wange, an der Fawkes sie gestern berührt hatte. Sie musste immer wieder an diese Berührung denken, vielleicht weil sie \emph{wollte}, dass sie wichtig war, dass sie ihr etwas bedeutete…

"Kann ich irgendwie helfen?", fragte sie.

"Ich möchte etwas Normales tun", sagte Harry hinter seinen Händen. "Etwas ganz Normales für Hogwartsschüler im ersten Jahr. Etwas, das Elf- und Zwölfjährige wie wir tun \emph{sollten}. Zum Beispiel eine Partie Zauberschnippschnapp spielen oder so… Ich nehme nicht an, dass du die Karten hast oder die Regeln kennst oder so was in der Art?"

"Ähm… Ich kenne die Regeln \emph{nicht}, um ehrlich zu sein…", sagte Hermine. "Ich weiß, dass sie \emph{explodieren}."

"Ich nehme an, Koboldsteine auch nicht?", sagte Harry.

"Kenne die Regeln nicht und sie \emph{spucken} dich an. Das sind \emph{Jungen}spiele, Harry!"

Es gab eine Pause. Harry rieb sich mit den Händen über das Gesicht, um es abzuwischen, und dann nahm er die Hände weg; und dann sah er sie an und wirkte ein wenig hilflos. "Nun", sagte Harry, "was \emph{machen} Zauberer und Hexen in unserem Alter, wenn sie, du weißt schon, die Art von sinnlosen, albernen Spielen spielen, die wir in diesem Alter spielen \emph{sollen}?"

"Himmel und Hölle?", fragte Hermine. "Springseil? Einhornangriff? \emph{Ich} weiß es nicht, \emph{ich} lese Bücher!"

Harry fing an zu lachen, und Hermine fing an, mit ihm zu kichern, obwohl sie nicht genau wusste, warum, aber es \emph{war} lustig.

"Ich schätze, das hat ein bisschen geholfen", sagte Harry. "Eigentlich glaube ich, dass es mehr geholfen hat, als eine Stunde lang mit Koboldsteinen zu spielen, also danke, dass du du selbst bist. Und egal was passiert, ich lasse mir \emph{nicht} alles, was ich über Infinitessimalrechnung weiß, von irgendjemandem obliviieren. Eher sterbe ich."

"\emph{Was?}", sagte Hermine. "Warum - warum solltest du \emph{das} \emph{jemals} tun wollen? "

Harry stand vom Tisch auf, und es gab einen Schwall wiederhergestellter Hintergrundgeräusche, als sein Aufstehen den Ruhezauber durchbrach. "Ich bin ein bisschen müde, also gehe ich ins Bett", sagte Harry, jetzt war seine Stimme normal und ironisch, "ich muss einiges an verlorener Zeit aufholen, aber wir sehen uns beim Frühstück und dann in Kräuterkunde, wenn das in Ordnung ist. Ganz zu schweigen davon, dass es nicht fair wäre, meine ganze Depression bei dir abzuladen. Nacht, Hermine."

"Gute Nacht, Harry", sagte sie und fühlte sich sehr verwirrt und beunruhigt. "Angenehme Träume."

Harry stolperte ein wenig, als sie das sagte, und dann ging er weiter in Richtung der Treppe, die zu den Schlafsälen der Erstklässler führte.

Harry drehte den Ruhezauber ganz nach oben, am Kopfende seines Bettes, damit er niemanden aufwecken würde, wenn er schrie.

Er stellte seinen Wecker so, dass er zum Frühstück geweckt wurde (falls er um diese Zeit nicht schon auf war, falls er überhaupt schlief).

Ging ins Bett, legte sich hin -

- fühlte die Beule unter seinem Kopfkissen.

Harry starrte hinauf zum Baldachin über seinem Bett.

Zischte leise: "Das soll ja wohl ein Scherz sein …"

Es dauerte ein paar Sekunden, bis Harry den Mut aufbrachte, sich im Bett aufzusetzen, die Decke über sich und sein Kissen zu ziehen, um die Tat vor den anderen Jungen zu verbergen, ein schwach leuchtendes \emph{Lumos} zu zaubern und zu sehen, was unter seinem Kopfkissen lag.

Es war ein Pergament und ein Stapel Spielkarten.

Auf dem Pergament stand,

\emph{Ein kleiner Vogel hat mir geflüstert, dass Dumbledore die Tür deines Käfigs geschlossen hat.}

\emph{In diesem Fall muss ich zugeben, dass Dumbledore nicht ganz Unrecht hat. Bellatrix Black ist wieder auf die Welt losgelassen. Und das ist keine gute Nachricht für einen guten Menschen. Wäre ich an Dumbledores Stelle, würde ich wohl dasselbe tun.}

\emph{\emph{Aber nur für den Fall… Das Hexeninstitut von Salem in Amerika nimmt auch Jungs auf, trotz des Namens. Sie sind gute Menschen und würden dich sogar vor Dumbledore beschützen, wenn es nötig sein sollte. In England gilt, dass} \emph{du} \emph{Dumbledores Erlaubnis brauchst, um ins magische Amerika auszuwandern, aber das magische Amerika ist anderer Meinung. In} \emph{äußerster Not} \emph{solltest du} \emph{dich außerhalb der} \emph{Schutzzauber} \emph{von Hogwarts begeben und den Herzkönig aus diesem Kartenspiel in zwei Teile reißen.}}

\emph{Dass du dazu nur im äußersten Notfall greifen solltest, versteht sich von selbst.}

\emph{Mach's gut, Harry Potter.}

\emph{- Der Weihnachtsmann}

Harry starrte auf das Kartenspiel hinunter.

Es \emph{konnte} ihn nirgendwo anders hinbringen, nicht im Moment, Portschlüssel funktionierten hier nicht.

Aber er fühlte sich immer noch entnervt von der Aussicht, es aufzuheben, selbst um es in seinem Koffer zu verstecken…

Nun, er \emph{hatte} das Pergament bereits aufgehoben, das genauso gut mit einer Falle hätte verzaubert sein können, falls es sich um eine Falle handelte.

Aber trotzdem.

"Wingardium Leviosa", flüsterte Harry und ließ das Päckchen mit den Karten neben der Stelle schweben, wo sein Wecker auf einer Ablage des Kopfteils ruhte. Er würde sich morgen darum kümmern.

Und dann legte sich Harry zurück ins Bett und schloss die Augen, um ohne Phönix zu träumen, der ihn beschützen konnte, und um seine Rechnungen zu begleichen.

Er erwachte mit einem entsetzten Keuchen, kein Schrei, er hatte in dieser Nacht noch nicht geschrien, aber seine Decke hatte sich um ihn herum verheddert, weil seine schlafende Gestalt herumzuckte, als er davon träumte zu rennen, versuchte den Lücken im Raum zu entkommen, die ihn durch einen Korridor aus Metall verfolgten, der von schwachem Gaslicht erhellt wurde, ein endlos langer Korridor aus Metall, beleuchtet von schummrigem Gaslicht, und er hatte im Traum nicht \emph{gewusst}, dass die Berührung dieser Lücken bedeutete, dass er auf schreckliche Art und Weise sterben und seinen noch atmenden Körper leer zurücklassen würde, alles was er gewusst hatte war, dass er rennen musste und rennen und rennen vor den Wunden in der Welt, die hinter ihm her glitten -

Harry fing wieder an zu weinen, nicht wegen des Schreckens der Verfolgung, sondern weil er weggelaufen war, während hinter ihm jemand um Hilfe schrie, schrie, er solle zurückkommen und sie retten, ihr helfen, sie würde gefressen werden, sie würde sterben, und in dem Traum war Harry weggelaufen, anstatt ihr zu helfen.

\emph{\emph{"GEH NICHT!" Die Stimme kam schreiend von hinter der Metalltür. "Nein, nein, nein, geh nicht, nimm es nicht weg, nimm es nicht weg, nein, nein, nein} \emph{-"}}

Warum hatte sich Fawkes jemals auf seine Schulter gesetzt? Er war weggelaufen. Fawkes sollte ihn hassen.

Fawkes sollte Dumbledore hassen. \emph{Er} war weggelaufen.

Fawkes sollte jeden hassen -

Der Junge war nicht wach, träumte nicht, seine Gedanken waren durcheinander und verwirrt in den Schattenländern, die an Schlaf und Wachen grenzten, ungeschützt durch die Sicherheitsgeländer, mit denen sein bewusster Verstand sich selbst umgab, die sorgfältigen Regeln und Zensoren. In diesem Schattenland war sein Gehirn genug aufgewacht, um zu denken, aber etwas anderes war zu schläfrig, um zu handeln; seine Gedanken liefen frei und wild, unbehindert von seinem Selbstkonzept, den Idealen seines wachen Ichs, was er nicht denken sollte. Das war die Freiheit der Träume seines Gehirns, während sein Selbstkonzept schlief. Frei, Harrys neuen schlimmsten Albtraum zu wiederholen, immer und immer wieder:

\emph{"Nein, das wollte ich nicht, bitte stirb nicht!"}

\emph{"Nein, das wollte ich nicht, bitte stirb nicht!"}

\emph{"Nein, das wollte ich nicht, bitte stirb nicht!"}

Neben dem Selbsthass wuchs in ihm eine Wut, ein schrecklich heißer Zorn / eiskalter Hass, auf die Welt, die ihr / sich selbst das angetan hatte, und in seinem halbwachen Zustand phantasierte Harry Auswege, phantasierte Wege aus dem moralischen Dilemma, er stellte sich vor, wie er über dem riesigen dreieckigen Grauen von Askaban schwebte und eine Beschwörungsformel flüsterte, die anders war als alle Silben, die jemals zuvor auf der Erde gehört worden waren, ein Flüstern, das über den ganzen Himmel hallte und auf der anderen Seite der Welt gehört wurde, und es gab eine Explosion aus silbernem Patronusfeuer wie eine Atomexplosion, die alle Dementoren in einem Augenblick auseinander riss und die Metallwände von Askaban zerriss, die langen Korridore und all die schummrigen orangenen Lichter zerschmetterte, und dann erinnerte sich sein Gehirn einen Moment später daran, dass da drin Menschen waren, und schrieb die Halbtraum-Fantasie um, um alle Gefangenen lachend zu zeigen, wie sie in Scharen aus dem brennenden Wrack von Askaban davonflogen, wobei das silberne Licht das Fleisch an ihren Gliedern wiederherstellte, während sie flogen, und Harry fing an, noch stärker in sein Kissen zu weinen, weil er es nicht tun konnte, weil er nicht Gott war -

Er hatte auf sein Leben und seine Magie und seine Kunst als Rationalist geschworen, er hatte auf alles geschworen, was ihm heilig war und auf all seine glücklichen Erinnerungen, er hatte seinen Eid geleistet, also musste er jetzt etwas tun, \emph{musste etwas tun, musste} \emph{IRGENDETWAS TUN} -

Vielleicht war es sinnlos.

Vielleicht war der Versuch, Regeln zu befolgen, sinnlos.

Vielleicht hast du aber auch Askaban niedergebrannt.

Und er hatte sich geschworen, es zu tun, also musste er es jetzt tun.

Er würde einfach alles tun, was nötig war, um Askaban loszuwerden, das war alles. Wenn das bedeutete, Britannien zu regieren, schön, wenn das bedeutete, einen Zauberspruch zu finden, den er flüstern konnte, der im ganzen Himmel widerhallte, egal, das Wichtigste war, Askaban zu zerstören.

Das war die Seite, auf der er stand, das war, wer er war, also war es erledigt.

Sein wacher Verstand hätte noch viel mehr Details verlangt, bevor er das als Antwort akzeptiert hätte, aber in seinem halb träumenden Zustand fühlte es sich wie eine ausreichende Lösung an, um seinen müden Geist wieder richtig einschlafen zu lassen und den nächsten Albtraum zu träumen.

\emph{Letztes Nachspiel}:

Sie erwachte mit einem Keuchen des Entsetzens, einer Unterbrechung ihrer Atmung, die ihr das Gefühl gab, keine Luft mehr zu bekommen, und doch bewegten sich ihre Lungen nicht, sie erwachte mit einem stimmlosen Schrei auf den Lippen und ohne Worte, keine Worte kamen hervor, denn sie konnte nicht verstehen, was sie gesehen hatte, \emph{sie konnte nicht verstehen, was sie gesehen hatte,} es war zu groß für sie, um es zu erfassen, und es nahm immer noch Gestalt an, sie konnte diese formlose Gestalt nicht in Worte fassen, und so konnte sie sie nicht entladen, konnte sie nicht entladen und wieder unschuldig und unwissend werden.

"Wie spät ist es?", flüsterte sie.

Ihr goldener, juwelenbesetzter Wecker, der schöne und magische und teure Wecker, den ihr der Schulleiter bei ihrer Einstellung in Hogwarts geschenkt hatte, flüsterte zurück: "Etwa zwei Uhr morgens. Leg dich wieder schlafen."

Ihre Laken waren schweißgetränkt, ihr Nachthemd schweißgetränkt, sie nahm ihren Zauberstab, der neben dem Kopfkissen lag und säuberte sich, bevor sie versuchte, wieder einzuschlafen, was ihr schließlich auch gelang.

Sybill Trelawney schlief wieder ein.

