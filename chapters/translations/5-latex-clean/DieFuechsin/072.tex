

\hypertarget{selbstverwirklichung-teil-7---plausible-bestreitbarkeit}{% \section{40. Selbstverwirklichung, Teil 7 - Plausible Bestreitbarkeit}\label{selbstverwirklichung-teil-7---plausible-bestreitbarkeit}}

-\/-\/-\/-\/- Kapitel 72: Selbstverwirklichung, Teil 7 - Plausible Bestreitbarkeit -\/-\/-\/-\/-

Die Wintersonne war bereits untergegangen, als das Abendessen endete, und so machte sich Hermine im friedlichen Licht der Sterne, die von der verzauberten Decke der Großen Halle herab funkelten, zusammen mit ihrem Lernpartner Harry Potter, der in letzter Zeit \emph{lächerlich} viel Zeit zum Lernen zu haben schien, auf den Weg zum Ravenclaw-Turm. Sie hatte nicht die leiseste Ahnung, wann Harry seine eigentlichen Hausaufgaben machte, außer dass sie erledigt wurden, vielleicht von Hauselfen, während er schlief.

Fast jedes einzelne Augenpaar in der ganzen Halle lag auf ihnen, als sie durch die mächtigen Türen des Speisesaals gingen, die eher den Belagerungstoren einer Burg glichen als etwas, durch das Schüler auf dem Rückweg vom Abendessen gehen sollten.

Sie gingen hinaus, ohne zu sprechen, und gingen, bis das ferne Geplapper der Schülergespräche verstummt war; und dann noch ein Stück weiter durch die steinernen Korridore, bevor Hermine endlich sprach.

"Warum hast du das getan, Harry?"

"Was getan?", sagte der Junge, der lebte in einem abstrakten Ton, als wäre er mit seinen Gedanken ganz woanders und würde über weitaus wichtigere Dinge nachdenken.

"Ich meine, warum hast du ihnen nicht einfach \emph{nein} gesagt?"

"Nun", sagte Harry, während ihre Schuhe über die Fliesen klapperten, "ich kann nicht einfach jedes Mal 'Nein' sagen, wenn mich jemand nach etwas fragt, das ich nicht getan habe. Ich meine, nehmen wir an, jemand fragt mich: 'Harry, hast du den Streich mit der unsichtbaren Farbe gemacht?' und ich sage 'Nein' und dann sagen sie: 'Harry, weißt du, wer sich am Besenstiel des Gryffindor-Suchers zu schaffen gemacht hat?' und ich sage 'Ich weigere mich, diese Frage zu beantworten.' Das würde mich verraten, oder?"

"Und \emph{deshalb}", sagte Hermine vorsichtig, "hast du allen erzählt …" Sie konzentrierte sich und erinnerte sich an die genauen Worte. "Dass, wenn es \emph{hypothetisch} eine Verschwörung gäbe, du weder bestätigen noch leugnen könntest, dass der wahre Meister der Verschwörung der Geist von Salazar Slytherin ist, und dass du nicht einmal in der Lage wärst, die Existenz der Verschwörung zuzugeben, und deshalb die Leute aufhören sollten, dir Fragen darüber zu stellen."

"Jep", sagte Harry Potter und lächelte leicht. "Das wird sie lehren, hypothetische Szenarien zu ernst zu nehmen."

"Und du hast \emph{mir} gesagt, ich solle keine Fragen beantworten -"

"Sie werden dir vielleicht nicht \emph{glauben}, wenn du es leugnest", sagte Harry. "Also ist es besser, nichts zu sagen, es sei denn, du willst, dass sie dich für eine Lügnerin halten."

"Aber -" sagte Hermine hilflos. "Aber - aber jetzt denken die Leute, ich würde \emph{Aufträge} von \emph{Salazar Slytherin} erhalten! " Die Art, wie die Gryffindors sie angeschaut hatten - die Art, wie die \emph{Slytherins} sie angeschaut hatten -

"Das gehört dazu, wenn man ein Held ist", sagte Harry. "Hast du gesehen, was der \emph{Quibbler} über \emph{mich} schreibt? "

Für eine kurze Sekunde stellte Hermine sich vor, wie ihre Eltern einen Zeitungsartikel über sie lasen, und anstatt dass es darum ging, dass sie einen landesweiten Buchstabierwettbewerb gewonnen hatte oder auf irgendeine andere Art und Weise, von der sie sich vorstellen konnte, in die Zeitung zu kommen, lautete die Schlagzeile: "HERMINE GRANGER SCHWÄNGERT DRACO MALFOY".

Das war genug, um sich die ganze Sache mit der Heldin noch mal zu überdenken.

Harrys Stimme wurde ein wenig förmlicher. "Wo wir gerade dabei sind, Miss Granger, wie läuft Ihre neuste Mission?"

"Nun", sagte Hermine, "wenn der Geist von Salazar Slytherin nicht \emph{wirklich} auftaucht und uns sagt, wo wir Mobbing finden, glaube ich nicht, dass wir viel Glück haben werden." Nicht, dass sie das bedauerte.

Sie blickte zu Harry hinüber und sah, wie der Junge sie mit einem seltsam intensiven Blick bedachte.

"Weißt du, Hermine", sagte der Junge leise, als wolle er sichergehen, dass es sonst niemand auf der Welt hörte, "ich glaube, du hast recht. Ich glaube, dass manche Leute viel mehr Hilfe bekommen als andere, um Helden zu werden. Und \emph{ich} denke auch, dass es so nicht fair ist."

Und Harry hielt sie am Ärmel ihres Umhangs fest, und drängte sie in eine Seitenabzweigung des Korridors, ihr Mund klaffte vor Überraschung auf, während Harry plötzlich seinen Zauberstab zog, sie durch eine enge Kurve geleitete, die so eng war, dass sie und Harry fast zusammengestoßen wären, und in beide Richtungen des Korridors leise "\emph{Quietus}" flüsterte.

Der Junge schaute suchend um sie herum, nicht nur zu jeder Seite, sondern sogar nach oben zur Decke und nach unten zum Boden.

Dann steckte Harry eine Hand in seinen Beutel und sagte: "Tarnumhang."

"\emph{Quiek?} ", entfuhr es Hermine.

Harry zog bereits Falten aus schimmerndem, schwarzem Stoff aus seinem Eselsfellbeutel. "Keine Sorge", sagte der Junge mit einem kleinen Grinsen, "die sind so selten, dass sich niemand die Mühe gemacht hat, eine Schulregel gegen sie aufzustellen …"

Und dann hielt Harry ihr das dunkle Samtgewebe hin und sagte mit seltsam förmlicher Stimme: "Ich gebe dir nicht, sondern leihe dir meinen Umhang, Hermine Jean Granger. Beschütze sie gut."

Sie starrte auf den schimmernden Samt des Umhangs, Stoff, der alles Licht verschluckte, das auf ihn fiel, außer seltsam glitzernden kleinen Reflexionen, Stoff, der so perfekt schwarz war, dass er eigentlich Staub oder Fussel oder \emph{irgendetwas} hätte zeigen müssen, aber das tat er nicht, je länger man hinsah, desto mehr hatte man das Gefühl, dass das, was man sah, gar nicht wirklich da war, aber dann blinzelte man wieder und es war einfach nur ein schwarzer Umhang.

"Nimm ihn, Hermine."

Ohne sich Gedanken darüber zu machen, streckte Hermine schon ihre Hand aus, um nach dem Stoff zu greifen; und dann, gerade als ihr Gehirn aufwachte und sie ihre Hand zurückziehen wollte, ließ Harry den Mantel los, und er begann zu fallen, und sie griff danach, ohne nachzudenken. Und in dem Moment, in dem ihre Finger den Mantel berührten und festhielten, fühlte sie einen nicht greifbaren Ruck durch sich hindurchgehen, als würde sie ihren Zauberstab zum ersten Mal in die Hand nehmen; und es war, als hörte sie im Hinterkopf ganz leise ein Lied, das gesungen wurde.

"Das ist einer meiner Quest-Gegenstände, Hermine", sagte Harry leise. "Es gehörte meinem Vater, und ich kann es nicht ersetzen, wenn es verloren geht. Leihe ihn niemandem, zeige ihn niemandem, sage niemandem, dass er existiert … aber wenn du ihn dir für eine Weile ausleihen willst, komm einfach zu mir und frag."

Hermine riss endlich ihre Augen von den tiefschwarzen Falten los und starrte wieder zu Harry hoch.

"Ich kann nicht -"

"Das kannst du sehr wohl", sagte Harry. "Denn es ist nicht im Geringsten fair, dass ich dieses Geschenk eines Morgens eingepackt in einer Schachtel neben meinem Bett finde und du … nicht." Harry hielt nachdenklich inne. "Es sei denn, du \emph{hast} deinen eigenen Tarnumhang bekommen, dann natürlich nicht."

Dann dämmerte ihr endlich die Tragweite des Wortes \emph{Unsichtbarkeitsumhang}, und sie zeigte mit einem schockierten Finger auf Harry, obwohl sie so nah beieinanderstanden, dass sie ihren Arm nicht richtig ausstrecken konnte, und ihre Stimme erhob sich mit beträchtlicher Empörung, als sie sagte: "\emph{So} bist du also aus dem Zaubertränkeschrank verschwunden! Und das Mal, als…", und dann brach sie ab, denn selbst \emph{mit} einem Tarnumhang konnte sie immer noch nicht verstehen, wie Harry…

Harry polierte seine Fingernägel mit kunstvoller Nonchalance an seinem Umhang und sagte: "Nun, du wusstest, \emph{dass} es einen Trick geben musste, oder? Und jetzt wird die Heldin auf geheimnisvolle Weise wissen, wo und wann sie Schläger findet - so als ob sie den Angreifern bei ihrer Planung zugehört hätte, obwohl sich \emph{niemand} in ihrem Alter unsichtbar machen könnte, um sie auszuspionieren."

Sie schwiegen eine Weile.

"Harry -", sagte sie. "Ich - ich bin mir nicht mehr sicher, ob es so eine gute Idee ist, gegen Mobbing zu kämpfen."

Harrys Augen blieben auf den ihren haften. "Weil die anderen Mädchen verletzt werden könnten?"

Sie nickte nur.

"Das ist \emph{ihre} Entscheidung, Hermine, genau wie es deine ist. \emph{Ich} habe mich entschieden, nicht die offensichtliche Dummheit zu machen, die jeder in meinen Büchern macht, nämlich zu versuchen, dich zu beschützen und dabei hilflos zu halten, bis du dadurch, wirklich wütend auf mich wirst und versuchst mich wegzustoßen, während du auf eigene Faust losziehst und in noch mehr Schwierigkeiten gerätst, und es dann heldenhaft erfolgreich durchstehst, wonach ich endlich meine Erleuchtung habe und erkenne, dass blah blah blah und so weiter. Ich weiß, wie dieser Teil meiner Lebensgeschichte abläuft, also überspringe ich ihn einfach. Wenn ich vorhersagen kann, was ich später denken werde, kann ich es genauso gut jetzt schon denken. Wie auch immer, mein Punkt ist, du solltest \emph{deine} Freunde auch nicht einschränken, um sie zu beschützen. Sag ihnen einfach im Voraus, dass es voraussichtlich schrecklich schief gehen wird, und wenn sie danach immer noch Heldinnen sein wollen, gut."

In Momenten wie diesen fragte sich Hermine, ob sie sich \emph{jemals} an die Art gewöhnen würde, wie Harry dachte. "Harry, ich will wirklich", ihre Stimme stockte für eine Sekunde, "wirklich, \emph{wirklich} nicht, dass sie verletzt werden! Besonders wegen etwas, das ich angefangen habe!"

"Hermine", sagte Harry ernst, "ich bin mir ziemlich sicher, dass du das Richtige getan hast. Ich wüsste nicht, was ihnen realistischerweise passieren könnte, was auf lange Sicht \emph{schlimmer} für sie wäre, als es \emph{nicht zu versuchen}."

"Was ist, wenn sie \emph{schwer} verletzt werden?" sagte Hermine. Sie hatte einen Kloß im Hals; und erinnerte sich an Kapitän Ernies Erzählung, wie Harry einem Schläger geradeaus in die Augen gestarrt hatte, als dieser ihm den Finger zurückbog, bevor Professor Sprout gekommen war, um ihn zu retten; und dann kam ihr ein anderes Bild in den Kopf, nämlich von Hannah und ihren zarten Händen mit den Fingernägeln, die sie jeden Morgen sorgfältig in Hufflepuff-Gelb lackierte, aber das durfte sie sich nicht vorstellen. "Und dann - werden sie nie wieder etwas Mutiges tun -"

"Ich glaube nicht, dass es so funktioniert", sagte Harry mit fester Stimme. "Selbst wenn alles wahnsinnig schief geht, glaube ich nicht, dass der menschliche Verstand so funktioniert. Das Wichtigste ist, dass man von sich selbst überzeugt ist, dass man jemand ist, der seine Grenzen überschreiten kann. Es zu versuchen und verletzt zu werden, kann unmöglich schlimmer für dich sein, als … \emph{festzustecken}."

"Und wenn du dich irrst, Harry?"

Harry hielt einen Moment inne, dann zuckte er ein wenig traurig mit den Schultern und sagte: "Was, wenn ich recht habe?"

Hermine blickte wieder auf das glänzend, schwarze Gewebe, das über ihre Hand glitt. Von innen fühlte sich der Umhang seltsam weich und dennoch widerstandsfähig an, als wollte er ihre Hand beruhigend umarmen.

Dann hob sie ihren Arm wieder hoch und hielt Harry den Mantel hin.

Harry machte keine Anstalten ihn zu nehmen.

"Ich -", sagte Hermine. "Ich meine, danke, vielen Dank, aber ich denke immer noch darüber nach, also kannst du ihn fürs Erste zurücknehmen. Und… Harry, ich glaube nicht, dass es richtig ist, Leuten \emph{nachzuspionieren} -"

"Nicht einmal um die Opfer von wohlbekannten Raufbolden zu retten?" fragte Harry. "\emph{Ich} bin noch nie gemobbt worden, aber ich habe eine realistische Simulation durchgemacht, und die war nicht sehr angenehm. Wurdest du jemals schikaniert, Hermine?"

"Nein", sagte sie mit ruhiger Stimme und hielt Harry weiterhin seinen Tarnumhang hin.

Schließlich nahm Harry den Umhang wieder an sich - sie spürte ein kleines Zucken des Verlustes, als das unhörbare Lied aus ihrem Hinterkopf verschwand - und er stopfte den schwarzen Stoff wieder in seinen Beutel.

Als der Beutel den letzten Rest des Stoffes aufgefressen hatte, wandte sich Harry von ihr ab, um die Quietus-Barriere aufzuheben.

"Und, ähm", sagte Hermine. "Das ist doch nicht \emph{der} Tarnumhang, oder? Der, von dem wir in der Bibliothek auf Seite achtzehn von Paula Vieiras Übersetzung von Gottschalks \emph{Eine} \emph{Illustrierte Schriftrolle verlorener} \emph{Geräte} gelesen haben?"

Harry drehte den Kopf zurück, grinste leicht und sagte in genau demselben Tonfall, den er zuvor beim Abendessen mit den anderen Schülern benutzt hatte: "Ich kann weder bestätigen noch leugnen, dass ich magische Artefakte von unglaublicher Macht besitze."

Als Hermine an diesem Abend ins Bett kletterte, versuchte sie immer noch, sich zu entscheiden. Ihr Leben war zur Zeit des Abendessens einfacher gewesen, damals, als es noch keine praktische Möglichkeit \emph{gegeben hatte}, Mobbing aufzuspüren; und jetzt musste sie sich wieder entscheiden; diesmal nicht für sich selbst, sondern für ihre Freunde. Vor ihrem geistigen Auge sah sie immer wieder Dumbledores faltiges Gesicht und den Schmerz, den es nicht ganz verbergen konnte, und in ihren geistigen Ohren hörte sie immer wieder Harrys Stimme, die sagte: "Das ist ihre Entscheidung, Hermine, genau wie es deine ist.

Und ihre Hand erinnerte sich immer wieder an das Gefühl des Umhangs an ihren Fingern, spielte es in Gedanken immer wieder ab. Das Gefühl hatte eine Kraft, die ihre Gedanken dazu zwang, zu ihm zurückzukehren, und zu dem Lied, das sie in einem Teil ihres Geistes und ihrer Magie gehört oder nicht gehört hatte, der jetzt wieder still war.

Harry hatte mit dem Mantel gesprochen, als wäre er eine \emph{Person}, und ihm gesagt, er solle gut auf sie aufpassen. Harry hatte gesagt, dass der Mantel seinem Vater gehört hatte, dass er ihn nicht ersetzen konnte, wenn er verloren ging…

Aber… Harry würde das nicht \emph{wirklich} tun, oder?

Ihr einfach eines der drei Heiligtümer des Todes \emph{geben}, die Jahrhunderte vor Hogwarts erschaffen wurden?

Sie könnte sagen, dass sie sich geschmeichelt fühlte, aber das ging \emph{weit} über Schmeicheleien hinaus und brachte sie dazu, sich zu fragen, was genau sie für Harry war.

Vielleicht war Harry die Art von Mensch, die herumlief und jedem, den er als Freund betrachtete, uralte, verlorene magische Artefakte lieh, aber -

Aber wenn sie darüber nachdachte, \emph{welchen} Teil seines Lebens Harry übersprungen hatte, den Teil, in dem er versuchte, sie zu beschützen und zu behüten…

Hermine starrte hinauf an die Decke des Ravenclaw-Schlafsaals. Irgendwo hinter ihrem Bett unterhielten sich Mandy und Su. Sie hatte ihren Ruhezauber so weit aufgedreht, dass sie die Worte nicht genau hören konnte, aber sie konnte immer noch ihr leises Gemurmel hören; es hatte etwas Beruhigendes, mit den anderen Mädchen in einem Schlafsaal zu schlafen. Harry hatte seinen eigenen Ruhezauber ganz aufgedreht, das wusste sie.

Sie begann sich zu fragen, ob Harry vielleicht \emph{tatsächlich}, na ja…

Du weißt schon…

sie \emph{mochte}.

Hermine Granger brauchte in dieser Nacht sehr lange, um einzuschlafen.

Und als sie am nächsten Morgen aufwachte, lugte ein kleiner Zettel unter ihrem Kopfkissen hervor, auf dem stand: „\emph{Heute, um halb elf wird ein Angriff stattfinden, im vierten Gang auf der linken Seite der Halle, wenn du das Klassenzimmer für Zaubertränke verlässt - S}.“

Als Hermine an diesem Morgen die Große Halle betrat, war ihr Magen mit fliegenden Schmetterlingen von der Größe eines Hippogreifs gefüllt; selbst als sie sich dem Ravenclaw-Frühstückstisch näherte, hatte sie sich noch nicht entschieden, was sie tun sollte.

Sie sah, dass neben Padma ein leerer Platz war. Dort würde sie sich hinsetzen, wenn sie es Padma sagen wollte und dann Padma bitten würde, es Daphne und Tracey zu sagen.

Hermine ging auf den leeren Platz neben Padma zu.

Die Worte warteten in ihrer Kehle: \emph{Padma, ich habe eine mysteriöse Nachricht bekommen} \emph{-}

Und sie spürte eine riesige Ziegelmauer in ihrer Kehle, die ihre Worte daran hinderte aus ihrem Mund zu kommen. Sie würde Hannah, Susan und Daphne \emph{in Gefahr} bringen. Sie würde sie an der Hand nehmen und sie direkt in Schwierigkeiten führen. Das war Falsch.

Oder sie könnte einfach versuchen, selbst mit dem Schlägern fertig zu werden, ohne ihren Freundinnen etwas zu sagen, und das war ganz offensichtlich auch Falsch.

Hermine wusste, dass sie sich in einem moralischen Dilemma befand, genau wie all die Zauberer und Hexen, von denen sie in Geschichten gelesen hatte. Nur dass es in den Geschichten immer eine \emph{richtige} und eine falsche Wahl gab, nicht zwei falsche, was ein bisschen unfair erschien. Aber irgendwie hatte sie das Gefühl - vielleicht kam es von der Art, wie Harry immer davon sprach, wie die Geschichtsbücher sie sehen würden -, dass sie vor einer heroischen Entscheidung stand und dass ihr ganzes Leben in die eine oder andere Richtung verlaufen könnte, je nachdem, was sie jetzt, \emph{an diesem Morgen}, wählte.

Hermine setzte sich an den Tisch, ohne einen Blick zur Seite zu werfen, starrte nur auf den Teller und das Besteck, als ob darin Antworten verborgen sein könnten, und dachte so angestrengt nach, wie sie es noch nie getan hatte, und ein paar Sekunden später hörte sie Padmas Stimme, die ihr ins Ohr flüsterte: "Daphne sagt, sie weiß, wo heute um zehn Uhr dreißig ein Angriff stattfinden wird."

Zum Scheitern verurteilt.

Susan Bones' Meinung nach waren sie alle dem Untergang geweiht.

Tantchen erzählte manchmal Geschichten, die so anfingen, Leute, die etwas taten, von dem sie \emph{wussten}, dass es dumm war, und die Geschichten endeten in der Regel damit, dass diese Leute dann auf dem ganzen Boden und an den Wänden verteilt waren und Tantchens Schuhe dreckig machten.

"Hey, Padma", murmelte Parvati, ihre Stimme war kaum zu hören über dem leisen Trippeln von acht Mädchen, die auf Zehenspitzen durch den Korridor vor dem Zaubertränkeklassenzimmer schlichen. "Weißt du, warum Hermine schon den ganzen Morgen seufzt -"

"Nicht reden!", zischte Lavender, das harsche Flüstern klang viel lauter als Parvatis Gemurmel. "Man kann nie wissen, wann das Böse zuhört!"

"\emph{Pssst!} ", sagten die drei anderen Mädchen noch lauter.

Sie waren ganz, total, komplett, extrem verdammt.

Als sie sich dem vierten Gang links vom Zaubertränkeklassenzimmer näherten, wo laut Daphnes geheimnisvollem Informanten der Überfall stattfinden sollte, bewegten sich die acht langsamer, das Geräusch ihrer Füße wurde leiser, und schließlich machte General Granger die Geste, die bedeutete \emph{Halt, ich schaue voraus}.

Lavender hob eine Hand, und als Hermine sich umdrehte, um sie anzusehen, zeigte Lavender mit verwirrtem Blick geradeaus den Korridor hinunter, zeigte auf sich selbst und versuchte dann, etwas anderes zu signalisieren, das Susan nicht verstand -

General Granger schüttelte den Kopf und machte noch einmal, diesmal mit langsameren, übertriebeneren Bewegungen, das Zeichen für \emph{Halt, ich schaue voraus}.

Lavender, die noch verwirrter aussah, zeigte den Weg zurück, den sie gekommen waren, und machte mit der anderen Hand eine hüpfende Geste.

Jetzt schauten alle anderen noch verwirrter als Lavender, und Susan dachte mit einer gewissen Bitterkeit, dass eine Stunde Training vor zwei Tagen offensichtlich nicht ausreichte, um sich einen neuen Satz Codesignale zu merken.

Hermine deutete auf Lavender, dann auf den Boden unter Lavenders Füßen, wobei ihr Gesichtsausdruck deutlich machte, dass die beabsichtigte Bedeutung war: \emph{Du. Bleib. Hier}.

Lavender nickte.

Doom doom doom, die englischen Worte für Verdammnis im Marschlied der Chaoslegion gingen Susan durch den Kopf, doom doom doom doom doom doom… {[}Anm. d. Übers.: dabei den Rhythmus des imperialen Marsches nicht vergessen {]}

Hermine griff in ihren Umhang und zog einen kleinen Stab mit einem Spiegel und einem Okular heraus. Ganz, ganz leise schlich sich das Ravenclaw-Mädchen an die Wand, direkt neben die Stelle, wo der Durchgang vom Korridor abging, und spähte nur mit der Spitze des Okulars um die Ecke.

Dann ein bisschen mehr.

Dann noch ein bisschen mehr.

Dann steckte General Granger vorsichtig den Kopf um die Seite.

General Granger drehte sich wieder zu ihnen um, nickte und machte die Handbewegung für "Folgt mir".

Susan fühlte sich ein wenig besser, als sie vorwärts schlich. Der Teil des Plans, der vorsah, dass sie dreißig Minuten vor dem Schläger eintreffen sollten, hatte anscheinend tatsächlich funktioniert. Vielleicht waren sie nur \emph{ein wenig} dem Untergang geweiht …?

Um zehn Uhr neunundzwanzig, fast pünktlich, tauchte der Rowdy auf. Wenn jemand anwesend gewesen wäre, um ihn zu hören - obwohl der Korridor anscheinend leer war -, hätte er gehört, wie seine Schuhe mit festem Schritt durch den Hauptkorridor stapften, den Gang betraten, in Richtung der ersten Ecke des Ganges gingen, um diese Ecke bogen und dann überrascht stehen blieben, als sie sahen, dass der Gang nun in einer massiven Ziegelwand endete, wo vorher keine Wand war.

Dann zuckte der Rowdy mit den Schultern und wandte sich ab, während er sich zurücklehnte, um den Hauptgang von der nächsten Ecke aus zu beobachten.

Immerhin \emph{war} es dies Schloss Hogwarts.

Hinter den hastig verwandelten dünnen Platten, die sie zum äußeren Erscheinungsbild einer Backsteinmauer zusammengesetzt hatten, warteten die Mädchen; sie sprachen nicht, bewegten sich nicht, atmeten nicht einmal, sondern beobachteten durch die Augenlöcher, die sie für sich selbst hinterlassen hatten.

Als Susans Blick den Rowdy erfasste, spürte sie, wie sich ihre Brust bis in die Zehenspitzen zusammenzog. Der Junge sah aus, als wäre er im siebten Jahr, wenn nicht sogar \emph{älter}, und seine Roben waren grün statt wie erhofft rot, und er hatte \emph{Muskeln}, und nachdem sie ihn noch etwas länger angestarrt hatte, erkannte Susan, dass seine Haltung die Balance hatte, die bedeutete, dass er sich \emph{duellierte}.

Dann hörten sie alle das Geräusch von weiteren Füßen, die sich aus dem Korridor näherten. Die Gryffindors und Slytherins aus dem vierten Jahr waren gerade aus dem Zaubertrankunterricht entlassen worden.

Die Schritte trappelten vorbei, wurden leiser und verklangen, und der Rowdy tat nichts. Einen Moment lang fühlte Susan Erleichterung.

Dann näherte sich eine weitere, kleinere Gruppe von Schritten.

Der Rowdy tat immer noch nichts, als die Schritte vorbeigingen.

Das passierte noch ein paar Mal.

Und dann, als sich ein letztes Paar Schuhe leise hörbar näherte, hörten die sieben Mädchen die Stimme des Rowdies, die klar und kalt und leise "\emph{Protego}" sagte.

Da keuchte jemand, wenn auch zum Glück sehr, sehr leise. Wenn sie nicht einmal einen einzigen Schuss abgeben konnten -

Die Schläger lernten \emph{jetzt} \emph{schon}, erkannte Susan, sie hatte nicht erwartet, dass B.E.L.F.E.R. das oft schaffen würde, bevor die Rowdies es mitbekamen - aber - Hermine hatte schon drei von ihnen besiegt - und die Schule war voller Spekulationen über den Geist von Salazar Slytherin, gestern -

\emph{Er erwartet uns!}

Susan hätte geflüstert, aufzugeben, den Plan abzubrechen, nur gab es keine Möglichkeit, eine Nachricht zu übermitteln -

"\emph{Silencio}", sagte der Rowdie mit leiser, bedächtiger Stimme, seinen Zauberstab auf den Korridor gerichtet, der blaue Schleier seines Abschirmzaubers schimmerte um ihn herum. "\emph{Accio} Opfer."

Als der Junge aus dem vierten Jahr in ihr Blickfeld kam, baumelte er kopfüber, als ob eine unsichtbare Hand ihn an einem Bein hochhielt, und sein roter Umhang begann, an den Oberschenkeln herunterzurutschen, um die darunter liegende Hose zu enthüllen. Sein Mund öffnete und schloss sich hilflos, ohne dass ein Ton herauskam.

"Ich nehme an, du fragst dich, was hier los ist", sagte der Slytherin im siebten Jahr mit ruhiger, kalter Stimme. "Mach dir keine Sorgen. Es ist so einfach, dass es sogar ein Gryffindor verstehen kann."

Damit formte die linke Hand des Slytherins eine Faust und fuhr hart in den Bauch des Gryffindors. Der Körper des Viertklässlers zuckte verzweifelt herum, aber noch immer verließ kein Wort seinen Mund.

"Du bist mein Opfer", sagte der ältere Slytherin. "Ich bin ein Schläger. Ich werde dich verprügeln. Und wir werden sehen, ob mich jemand aufhält."

In diesem Moment erkannte Susan, dass es eine Falle war.

Und fast im selben Moment ertönte die mächtige und hohe Stimme eines jungen Mädchens, das rief: "\emph{Halt, Übeltäter! Finite} \emph{Incantatem!} "

\emph{Lavender}, dachte Susan gequält. Das Gryffindor-Mädchen hatte sich freiwillig gemeldet, um als Ablenkung zu dienen, während der Rest von ihnen einen Flankenangriff von dort ausführte, wo der Tyrann sie nicht erwarten würde, das war der Plan gewesen, nur jetzt -

"Im Namen von Hogwarts", rief Lavenders Stimme, obwohl sie sie nicht sehen konnten, "und im Namen aller Heldinnen überall, befehle ich dir, diesen EEK!"

"\emph{Expelliarmus}", sagte der Rüpel. "\emph{Stupefy}. \emph{Accio} dumme Heldin."

Als Lavender in ihr Blickfeld schwebte, an einem Fuß baumelnd und bewusstlos, blinzelte Susan; das Mädchen war in einen leuchtend karmesinroten und goldenen Rock und eine Bluse gekleidet, anstelle ihrer üblichen Hogwarts-Roben.

Der Rowdie warf dem Mädchen ebenfalls einen verwunderten Blick zu, dann richtete er seinen Zauberstab auf sie und sagte "\emph{Finite} \emph{Incantatem}", aber die Kleidung blieb gleich.

Dann zuckte der Rüpel mit den Schultern und zog, immer noch in Richtung Lavender statt in Richtung des baumelnden Viertklässlers blickend, die Faust zurück -

"\emph{Lagann}! ", schrien fünf Stimmen, und fünf grüne Spiralen schossen aus fünf Zauberstäben, die durch fünf Löcher in der falschen Wand zielten, und einen Augenblick später rief Hermines Stimme "\emph{Stupefy}! "

Fünf grüne Spiralen zerschellten wirkungslos am blauen Dunst, und Hermines roter Blitz prallte vom Dunst ab und traf den Viertklässler, der zusammenzuckte und dann still war.

Und der Siebtklässler drehte sich um und lächelte grimmig, als die Erstklässlerinnen schrien und angriffen.

Susans Augen flogen auf und augenblicklich rollte sie sich von dem Ort weg, an dem sie auf dem Boden gelegen hatte. Ihre Lungen brannten noch immer und ihr ganzer Körper schmerzte von dem Schlag, der sie getroffen hatte. Es waren erst ein paar Sekunden vergangen nach dem was sie von dem Kampf sehen konnte. Hannahs Körper fiel, ihr Arm war noch immer nach Susan ausgestreckt, "\emph{Glisseo!} " schrie Hermine, aber der ältere Junge schlug einfach seinen Zauberstab nieder und hinterließ eine Spur von grünem Glühen und Hermines Zauber zerfiel sichtbar in einen Schauer von blau-weißen Funken, dann sagte der Tyrann fast in der gleichen Bewegung "Stupefy! "und Hermine wurde nach hinten geschleudert und Susan beschwor alle Magie, die sie noch hatte und schrie "\emph{Rennervate!} " auf Hermines Körper, noch während der Schläger sich ihr zuwandte, sein Zauberstab zeigte wieder in ihre Richtung und dann schrie Padma "\emph{Prismatis!} ", kurz vor dem "\emph{Impedimenta!}" des Schlägers. Die Regenbogenkugel bildete sich \emph{um den} \emph{Schläger herum} und der Siebtklässler aus Slytherin taumelte, als seine eigene Verhexung auf ihn zurückgeworfen wurde, aber einen Augenblick später schwang der Zauberstab des Rowdies zurück, um ihn selbst zu anzutippen und dann zersprang Padmas prismatische Kugel wie eine geplatzte Seifenblase, als der Zauberstab des Tyrannen sie durchtrennte und "\emph{Rennervate!} ", schrie Parvati Hannahs Körper an und Tracey und Lavender schrien gleichzeitig: "\emph{WingardiumLeviosa!} " -

Hannah Abbott streckte ihren Zauberstab mit einer Hand aus, die vor Erschöpfung zitterte, sie hatte jetzt nicht mehr genug Magie für ein einziges \emph{Rennervate}.

Der Rest des Ganges war still, verstreute Körper lagen auf dem Boden, Padma und Tracey und Lavender, Hermine und Parvati in einem Haufen an einer Wand, Susan stand in versteinerter Starre da, während ihre Augen alles hilflos verfolgten, sogar der Gryffindor-Junge lag ausgestreckt und regungslos da (Hermine hatte ihn geweckt und er hatte gekämpft, aber es war nicht genug gewesen).

Es war ein sehr kurzer Kampf gewesen.

Der Rowdie lächelte immer noch, die einzigen Anzeichen seiner Anstrengung waren ein schwankendes Kräuseln in dem blauen Schein, der ihn umgab, und ein paar Schweißperlen auf seiner Stirn.

Der Rüpel hob den Arm, wischte sich den Schweiß von der Stirn und pirschte sich wie ein lebendiger Lethifold in Menschengestalt an sie an.

Hannah drehte sich um und floh, wirbelte herum und rannte mit Schreien, die ihr in der Kehle stecken blieben, sprintete an der heruntergefallenen Vertäfelung der falschen Backsteinmauer vorbei, rannte mit aller Geschwindigkeit, die sie aufbringen konnte, den Gang hinunter und rannte im Zickzack, so gut sie konnte.

Kurz bevor Hannah die Kurve im Gang erreichte, sagte die Stimme des Tyrannen von hinten "\emph{Cluthe!} " und sie bekam furchtbare Krämpfe in den Beinen, sie fiel hin und rutschte und schlug mit dem Kopf gegen die Wand, nur den Schmerz des Aufpralls bemerkte sie gar nicht, denn sie fing an zu schreien wegen der Muskelzerrung -

Der Rowdie pirschte sich immer noch an sie heran, sah Hannah, als sie den Kopf drehte; er näherte sich ihr langsam, immer noch mit diesem furchtbaren Lächeln.

Und sie rollte sich, trotz des Schmerzes, als sich ihre Beinmuskeln um sich selbst verkrampften, sie rollte um die Ecke des Ganges und schrie: "Geh \emph{weg}! "

"Ich glaube nicht", sagte der Rowdie, seine Stimme war tief und furchteinflößend wie die eines erwachsenen Mannes und klang jetzt ganz nah.

Der Schläger ging um die Ecke, und Daphne Greengrass stach ihre Uralte Klinge direkt in seine Leiste.

Es gab einen Blitz, der den ganzen Korridor erhellte.

Mit betretenem Gesichtsausdruck verließen sieben Mädchen das Büro von Madam Pomfrey und ließen eine der ihren in einem Krankenhausbett zurück.

Hannah würde in etwa fünfunddreißig Minuten wieder in Ordnung sein, hatte die Heilerin gesagt; gerissene Muskeln waren leicht zu flicken.

Daphne hatte das Reden übernommen, und ihr zufolge hatte Hannah ein Missgeschick mit einem Straßenlaufzauber erlitten, das die Beinkrämpfe verursacht hatte. Madam Pomfrey hatte ihnen einen strengen Blick zugeworfen, aber nicht diskutiert, obwohl dieser Zauber etwa sechs Jahre über ihrem Niveau lag.

Madam Pomfrey hatte Daphne auch einen Zaubertrank gegeben, um ihren Zustand der totalen magischen Erschöpfung zu lindern, und sie davor gewarnt, in den nächsten drei Stunden keine Zauber zu sprechen. Das lag angeblich daran, dass Daphne zu viel Magie verbraucht hatte, als sie versuchte, \emph{Finite} auf Hannah zu wirken, und nicht daran, dass die Uralte Klinge all ihre Kraft aufbrachte, um den \emph{Protego} zu brechen.

Der Rest von ihnen hatte beschlossen, nichts über die blauen Flecken unter ihren Roben zu sagen, bis sie einige ältere Mädchen dazu bringen konnten, \emph{Episkey} zu zaubern. Es gab Grenzen dafür, wo sich Daphne rausreden konnte.

Die ganze Sache, dachte Susan, war zu knapp gewesen, viel zu knapp. Wenn der Schläger auch nur um die Ecke geschaut hätte - wenn er sich einen Moment Zeit genommen hätte, um seinen Schutzzauber zu erneuern -

"Wir sollten aufhören", sagte Susan, sobald die sieben aus dem Hörbereich des Heilerbüros waren. "Wir sollten aufhören, das zu tun."

Aus irgendeinem Grund drehten sich dann alle zu General Granger um, obwohl sie eigentlich über so etwas abstimmen sollten.

Die Sonnenschein-Generalin schien nicht zu bemerken, dass sie sie ansahen, sie schritt einfach weiter und starrte geradeaus.

Nach einer Weile sagte Hermine Granger mit einer Stimme, die nachdenklich und ein wenig traurig klang: "Hannah sagte, \emph{sie} wolle nicht, dass wir aufhören. Ich bin mir nicht sicher, ob es richtig ist, dass wir … weniger tapfer \emph{wegenihr} sind, als sie es ist."

Alle anderen Mädchen, außer Susan, nickten daraufhin.

"Ich denke, schlimmer kann es nicht mehr werden", sagte Parvati. "Und wir können es schaffen. Das haben wir jetzt bewiesen."

Susan fiel nichts ein, was sie darauf erwidern konnte. Sie glaubte nicht, dass es überzeugend sein würde, aus voller Kehle über eklatante Dummheit und VERDAMMNIS zu schreien. Und sie konnte die anderen Mädchen auch nicht einfach \emph{verlassen}. Reichte es nicht, mit harter Arbeit verflucht zu sein, warum mussten Hufflepuffs auch noch \emph{loyal} sein?

"Übrigens, Lavender", sagte Padma. "Was, bei Merlins Unterhosen, hattest du da vorhin \emph{an}?"

"Mein Helden-Outfit", sagte das Gryffindor-Mädchen.

Daphne klang müde als sie sprach. Sie stapfte einfach weiter durch die Halle und drehte nicht mal ihren Kopf zu ihnen. "Es ist das Kostüm des Soldaten von Gryffindor aus dem Stück \emph{Chroniken der} \emph{Lunarier}." {[}Anm.d. Übers.: wahrscheinlich die Zauberervariante von Sailor Moon{]}

"Hast du es verwandelt?", fragte Parvati und schaute verwirrt. "Aber der Schläger hat doch ein \emph{Finite} gezaubert -"

"Nö!" sagte Lavender. "Es ist \emph{echt}! Ich habe mein Helden-Outfit \emph{vorher} in ein normales Hemd und einen Rock verwandelt, also musste ich nur noch \emph{Finite} auf mich wirken, nachdem ich den Schläger gesehen hatte. Willst du dein eigenes, Parvati? Ich habe meins gestern von Katarina und Joshua im sechsten Schuljahr anfertigen lassen, für zwölf Sickel -"

"Ich glaube", sagte General Granger mit vorsichtiger Stimme, "das würde uns alle ein bisschen albern aussehen lassen."

"Nun", sagte Lavender, "wir sollten darüber abstimmen, ob wir -"

"Ich denke", sagte General Granger, "egal, wie \emph{irgendwer} abstimmt, ich werde mich \emph{nicht} \emph{mal im Traum} dabei erwischen lassen, eines dieser Kostüme zu tragen -"

Susan ignorierte die Diskussion. Sie versuchte, sich eine schlaue Strategie auszudenken, um weniger verdammt zu sein.

Die ganze Große Halle verstummte, wenn auch nur für einen Moment, als die sieben zum Mittagessen gingen.

Dann setzte der Applaus ein.

Er war verstreut, nicht der massive Applaus, bei dem alle auf einmal applaudierten. Viel davon kam vom Gryffindor-Tisch, weniger von Hufflepuff und Ravenclaw, und keiner von Slytherin.

Daphne spürte, wie sich ihr Gesicht straffte. Sie hatte \emph{gehofft} - nun ja, vielleicht würden ihre Mitschüler, nachdem sie einen Gryffindor-Schläger gestoppt und einen Slytherin gerettet hatten, erkennen das -

Sie schaute auf den Hufflepuff-Tisch.

Neville Longbottom applaudierte mit hochgehaltenen Händen über seinem Kopf, obwohl er nicht lächelte. Vielleicht hatte er von Hannah gehört, oder er fragte sich, warum Hannah nicht da war.

Dann, bevor sie sich selbst stoppen konnte, blickte sie in Richtung des Lehrertisches.

Professor Sprouts Gesicht war von Sorge gezeichnet. Sie und Professor McGonagall neigten ihre Köpfe zu Schulleiter Dumbledore, der einen erhabenen Blick aufgesetzt hatte, und alle bewegten ihre Lippen schnell. Professor Flitwick sah eher resigniert aus, und Quirrell, mit schlaffem Gesicht, stocherte mit einem Löffel in seiner zur Faust zitternd in seiner Suppe herum.

Professor Snape schaute direkt auf -

\emph{\emph{Sie?}}

Oder - auf Hermine Granger, die neben ihr stand?

Ein kleines, dünnes Lächeln ging über das Gesicht des Meisters der Zaubertränke, und er hob seine Hände, führte sie einmal in einer Bewegung zusammen, die zu langsam war, um ein echtes Klatschen zu sein; und dann wandte sich der Meister der Zaubertränke wieder seinem Teller zu und ignorierte die Gespräche um ihn herum.

Daphne spürte, wie ihr ein kleiner Schauer über den Rücken lief, und sie drehte sich hastig um, um zum Slytherin-Tisch zu gehen. Susan, Lavender und Parvati lösten sich von ihrer Gruppe und gingen in Richtung der Hufflepuff- und Gryffindor-Tische auf der anderen Seite der Großen Halle.

Es geschah, als sie an dem Teil des Slytherin-Tisches vorbeikamen, an dem das Slytherin-Quidditch-Team saß.

In diesem Moment stolperte Hermine plötzlich, stolperte heftig, als würde sie von den Füßen gerissen, und stürzte in die Lücke zwischen Marcus Flint und Lucian Bole, die dort saßen, und es gab ein trauriges, kleines Platschen, als Hermines Gesicht in Flints Teller mit Steak und Kartoffelpüree landete.

Dann schien alles zu schnell zu gehen, oder vielleicht war es nur Daphne selbst, die zu langsam dachte, denn Flint stieß einen Schrei der Empörung aus und seine Hand riss Hermine zurück und warf sie gegen den Ravenclaw-Tisch, wo sie vom Rücken eines Schülers abprallte und auf dem Boden zusammenbrach -

Die Stille breitete sich wellenförmig aus.

Hermine stieß sich auf den Händen ab, obwohl sie nicht ganz auf die Beine kam, konnte Daphne sehen, dass ihr ganzer Körper zitterte und dass ihr Gesicht noch immer mit Kartoffelbrei und verstreuten Steakstücken bedeckt war.

Einen langen Moment lang sprach niemand, niemand bewegte sich. Als könnte sich niemand in der ganzen Großen Halle vorstellen, genauso wenig wie Daphne, was als nächstes geschah.

Dann sagte Flints kräftige Stimme, die Stimme des Slytherin-Kapitäns, der auf dem Quidditchfeld Befehle brüllte, mit einem gefährlichen Grollen: "Du hast mein Essen ruiniert, Mädchen."

Ein weiterer Moment der eisigen Stille. Hermines Kopf - Daphne konnte sehen, wie er zitterte - drehte sich, um den Slytherin-Quidditch-Kapitän anzusehen.

"Entschuldige dich bei mir", sagte Flint.

Harry Potter begann, sich vom Ravenclaw-Tisch hochzudrücken, und stoppte auf halbem Weg nach oben abrupt, als ob ihm gerade etwas eingefallen wäre -

Dann standen fünf andere Leute vom Ravenclaw-Tisch auf.

Die gesamte Slytherin-Quidditch-Mannschaft stand auf, ihre Zauberstäbe sprangen in ihre Hände, und dann standen Schüler am Gryffindor-Tisch und am Hufflepuff-Tisch auf, und ohne nachzudenken, drehte sich Daphne um, um zum Lehrertisch zu sehen, und sie sah, dass der Schulleiter immer noch saß und zusah, einfach nur zusah, Dumbledore \emph{sah einfach nur zu}, und er hatte eine Hand ausgestreckt, als wolle er Professor McGonagall zurückhalten - in nur einer Sekunde würde jemand einen Zauberspruch rufen, und dann wäre es zu spät, \emph{warum tat der Schulleiter nichts} -

Und eine Stimme sagte: "Verzeihung."

Daphne drehte sich um und starrte schockiert mit offenem Mund.

" Scourgify", sagte die sanfte Stimme, und das Kartoffelpüree verschwand aus Hermines Gesicht und enthüllte den überraschten Gesichtsausdruck der Ravenclaw, als Draco Malfoy auf sie zukam, seinen Zauberstab wieder wegsteckte und dann neben ihr auf ein Knie ging und ihr die Hand reichte.

"Das tut mir leid, Miss Granger", sagte Draco Malfoys höfliche Stimme. "Da hat wohl jemand gedacht, er wäre witzig."

Hermine nahm Dracos Hand, und Daphne wurde plötzlich klar, was gleich passieren würde.

Aber Draco Malfoy hob Hermine \emph{nicht} erst halb hoch und ließ sie dann fallen.

Er zog sie einfach auf ihre Füße.

"Danke", sagte Hermine.

"Gern geschehen", sagte Draco Malfoy mit lauter Stimme und schaute nicht zur Seite, um zu sehen, wie alle vier Häuser von Hogwarts ihn völlig schockiert anstarrten. "Denk daran, dass gerissen und ehrgeizig zu sein, nicht bedeutet, dass man so sein muss."

Und dann ging Draco Malfoy zurück zu seinem Platz auf der Slytherinbank und setzte sich, als hätte er nicht - er hatte nicht grade - \emph{er hatte} \emph{grade} -

Hermine ging zum nächstbesten freien Platz auf der Ravenclaw-Bank und setzte sich.

Einige andere Leute setzten sich, eher langsam, hin.

"Daphne?", sagte Tracey. "Ist alles in Ordnung mit dir?"

Dracos Herz hämmerte so heftig in seiner Brust, dass er befürchtete, es könnte in einem Blutregen aus seiner Brust explodieren, wie der Fluch, den Amycus Carrow einmal bei einem Welpen angewendet hatte.

Dracos Gesicht blieb völlig beherrscht, denn er wusste (es war ihm immer wieder eingebläut worden), dass seine Hausmitbewohner ihn wie einen Schwarm Acromantulas zerreißen würden, wenn er auch nur das kleinste Anzeichen der Angst zeigte, die er fühlte.

Er hatte keine Zeit gehabt, sich mit Harry Potter abzusprechen, keine Zeit, einen Plan zu schmieden, keine Zeit zum Nachdenken, nur den Augenblick, in dem ihm klar wurde, dass es \emph{genau jetzt} an der Zeit war, den Ruf Slytherins zu retten.

Von allen Seiten des langen Slytherin-Tisches starrten wütende Gesichter auf Draco.

Aber sie wurden von der Anzahl der Gesichter, die einfach nur verwirrt aussahen, übertroffen.

"Also gut, ich gebe auf", sagte ein Junge aus dem sechsten Jahr, den Draco nicht erkannte und der ihm gegenüber und zwei Plätze weiter rechts saß. "Warum hast du das getan, Malfoy?"

Obwohl sein Mund sehr trocken war, schluckte Draco nicht. Das wäre ein Zeichen von Angst gewesen. Stattdessen nahm er einen Bissen Karotten, die von allem auf seinem Teller die meiste Feuchtigkeit enthielten, und kaute und schluckte, während er so schnell er konnte nachdachte.

"Weißt du", sagte Draco und machte seine Stimme so schneidend wie möglich - während sein Herz noch heftiger in seiner Brust pochte als alle um ihn herum aufhörten zu reden, um zuzuhören - "es gibt wahrscheinlich eine Möglichkeit, Slytherin noch \emph{schlimmer} aussehen zu lassen, als acht Erstklässlerinnen aus allen vier Häusern anzugreifen, die zusammenarbeiten, um Schläger zu stoppen, aber mir fällt nicht ein \emph{wie}. Auf diese Weise ziehen wir einen Vorteil aus dem, was Greengrass tut."

Die verwirrten Gesichter blieben verwirrt.

"Was?", sagte der Sechstklässler, und "Moment, \emph{welcher} Nutzen?", sagte ein Mädchen aus dem fünften Schuljahr, das rechts neben ihm saß.

"Es lässt das Haus Slytherin besser aussehen", sagte Draco.

Die Slytherins um ihn herum warfen ihm fragende Blicke zu, als hätte er gerade versucht, Algebra zu erklären.

"Besser aussehen für \emph{wen?} ", fragte der Sechstklässler.

"Aber du hast gerade einem \emph{Schlammblut} geholfen", sagte das Mädchen im fünften Jahr. "Wie soll das denn gut aussehen?"

Dracos Kehle schnürte sich zu. Sein Gehirn durchlebte eine grässliche Fehlfunktion, während der ihm nichts einfiel, was es sagen konnte, außer der Wahrheit -

Dann sagte ein Fünftklässler: "Wahrscheinlich hat Malfoy irgendeinen ungeheuer cleveren Plan ausgeheckt. Du weißt schon, wie in \emph{Die Tragödie des Lichts}, wo alles, was wie ein Rückschlag aussieht, Teil des Plans ist. Und es endet damit, dass Grangers Kopf auf einem Stock steckt und niemand ahnt, dass er es war."

"\emph{Das} macht Sinn", sagte jemand von weiter unten am Tisch, und es wurde viel genickt.

"Weißt \emph{du}, was der Boss vorhat?" murmelte Vincent leise.

Gregory Goyle antwortete nicht. In seinen Gedanken konnte er ganz deutlich die Stimme seines Meisters sagen hören, \emph{Ich kann nicht glauben, dass ich jedes Wort davon geglaubt habe}, an dem Tag, als das Gerücht aufkam, Salazar Slytherin habe Potter und Granger gezeigt, wo man Schlägereien findet.

"Mr. Goyle?", flüsterte Vincent.

Gregory Goyles Lippen formten die Worte \emph{Oh nein}, aber es kam kein Ton heraus.

Hermine hatte das Mittagessen an diesem Tag früh verlassen, aus irgendeinem Grund hatte sie keinen Hunger verspürt. Diese paar Sekunden schrecklicher Demütigung waren ihr immer wieder durch den Kopf gegangen, das Gefühl, wie ihr Gesicht in das Kartoffelpüree gequetscht und dann durch die Luft geschleudert wurde, und dann die Stimme des Slytherin-Jungen, der sagte: "Entschuldige dich bei mir"… es war vielleicht das erste Mal in ihrem ganzen Leben, dass sie Lust hatte, jemanden \emph{zu hassen}. Der Junge, der sie geworfen hatte (Marcus Flint, so hieß er angeblich) und derjenige, der den Stolperfluch auf sie geworfen hatte… sie hatte es gefühlt, einen schrecklichen Moment lang hatte sie Harry sagen wollen, dass sie sich nicht dagegen einzuwenden hätte, wenn er für sie \emph{kreativ} werden würde.

Sie war noch keine Minute aus der Großen Halle heraus, als sie hinter sich das Geräusch laufender Füße hörte und sich umdrehte, um zu sehen, wie Daphne auf sie zu rannte.

Und sie hörte sich an, was ihr Sonnenschein-Soldat zu sagen hatte…

"\emph{Verstehst} du denn nicht? " Daphnes Stimme war kaum leiser als ein Schrei. "Nur weil jemand nett zu dir ist, heißt das nicht, dass er dein Freund ist! Er ist \emph{Draco Malfoy!} Sein Vater ist ein Todesser, die Eltern all seiner Freunde sind Todesser - Nott, Goyle, Crabbe, \emph{alle um ihn herum}, verstehst du das? Sie \emph{alle} verachten Muggelgeborene, sie wollen, dass jeder wie du \emph{stirbt}, sie denken, du bist zu nichts anderem gut, als ein \emph{Opfer} in schrecklichen dunklen Ritualen zu sein! Draco ist \emph{der nächste Lord Malfoy}, er wurde von Geburt an dazu erzogen, dich zu hassen und er wurde von Geburt an dazu erzogen \emph{zu lügen!} " Daphnes graugrüne Augen starrten sie wütend an, verlangten Zustimmung und Verständnis.

"Er -" sagte Hermine zögernd. Sie erinnerte sich an das Dach, an den schrecklichen Ruck, als sie zu fallen begann, an Draco Malfoys Hand, die ihre ergriff und sie so fest hielt, dass sie danach blaue Flecken hatte. Sie hatte es ihm zweimal sagen müssen, bevor er sie endlich fallen ließ. "Vielleicht ist Draco Malfoy nicht wie sie -"

Daphnes Flüstern war fast ein Schrei. "Wenn er dir am Ende \emph{nicht} zehnmal so viel antut, wie er dir gerade geholfen hat, ist sein \emph{Leben vorbei}, verstehst du? Ich meine, Lucius Malfoy würde ihn \emph{buchstäblich} enterben! Weißt du, wie groß die Chance ist, dass er \emph{nicht} etwas im Schilde führt?"

"Winzig?", fragte Hermine mit leiser Stimme.

"\emph{Null!} ", zischte Daphne. "Ich meine \emph{keine!} Ich meine \emph{weniger} als Null! Ich meine, die Chance ist so gering, dass du ihn mit drei Lupenzaubern und einem Zeig-mir-Zauber und - und - und einer antiken Karte und einem Zentaurenpropheten nicht finden könntest! Jeder in Slytherin weiß, dass er etwas mit dir vorhat und nur nicht verdächtigt werden will. Ich habe gehört, dass jemand gesehen hat, wie er seinen Zauberstab auf dich gerichtet hat, kurz bevor du gestolpert bist - verstehst du das nicht? \emph{Das ist alles Teil von Malfoys Plan!} "

Draco saß da und aß sein Steak mit gebratenen Blumenkohlröschen und Aschenwindersoße (sie war nicht aus echten Aschenwinder Eiern gemacht, sondern schmeckte nur nach Feuer) und versuchte, nicht zu lachen und nicht zu weinen.

Er hatte von plausibler Bestreitbarkeit \emph{gehört}, aber ihm war nicht klar, wie wichtig sie war, bis er herausfand, dass die Malfoys keine hatten.

"Ihr wollt meinen Plan wissen?", sagte Draco. "\emph{Hier} ist mein Plan. Ich werde \emph{nichts} tun, und wenn die Leute das \emph{nächste} Mal denken, dass ich etwas vorhabe, können sie sich nicht sicher sein."

"Hm…", sagte der Fünftklässler. "Ich denke, dass ich dir nicht glaube, das klingt nicht gerissen genug, um es wirklich zu sein -"

"Das ist es, was er dich \emph{glauben} machen will", sagte die Fünftklässlerin.

"Albus", sagte Minerva mit einem gefährlichen Unterton in der Stimme, "hast du das alles \emph{geplant}?"

"Nun, wenn ich mit den Fingern unter dem Tisch geschnippt \emph{hätte}, würde ich dir das nicht einfach \emph{erzählen} -"

Die zitternde Hand des Verteidigungsprofessors ließ den Löffel wieder in die Suppe fallen.

"Was soll das heißen, \emph{euch} \emph{eine Falle} \emph{gestellt?} ", sagte Millicent. Die beiden saßen im Schneidersitz auf Daphnes Bett, nachdem sie nach dem Mittagessen direkt aus der Großen Halle dorthin gegangen waren. "Mit meinen Seheraugen, die durch Sie Zeit Selbst blicken, sah ich euch \emph{gewinnen}."

Daphne starrte Millicent an, ihre eigenen, bloß sterblichen Augen waren in diesem Moment ziemlich verengt. "Der Junge hat mit uns \emph{gerechnet}."

"Nun, ja!", sagte Millicent. "Jeder weiß, dass ihr Schäger jagt!"

"Hannah wurde von einem wirklich schmerzhaften Fluch getroffen", sagte Daphne. "Sie musste einen Heiler aufsuchen, Millicent! Wenn wir Freunde sind, hättest du mich \emph{vorwarnen} müssen!"

"Hör zu, Daphne, ich habe dir \emph{gesagt} -" Das Slytherin-Mädchen hielt inne, als ob sie versuchte, sich an etwas zu erinnern, und sagte dann: "Ich meine, ich habe dir gesagt, dass das, was ich Sehe, eintreten \emph{muss}. Wenn ich versuche, es zu ändern, wenn \emph{irgendjemand} versucht, es zu ändern, werden wirklich schreckliche, furchtbare, überhaupt nicht gute, extrem schlechte Dinge passieren. Und dann wird es \emph{trotzdem} eintreten. Wenn ich Sehe, dass du verprügelt wirst, \emph{kann} ich dir das nicht sagen, denn dann würdest du versuchen, \emph{nicht zu gehen}, und dann -" Millicent hielt inne.

"Und dann?" sagte Daphne skeptisch. "Ich meine, was passiert, wenn wir einfach nicht hingehen?"

"Ich \emph{weiß} es nicht! ", sagte Millicent. "Aber wahrscheinlich sieht von Lethifolds gefressen zu werden dagegen aus wie eine Teeparty!"

"Hören mal, selbst ich weiß, dass Prophezeiungen so nicht funktionieren", sagte Daphne und hielt dann inne. "Zumindest funktionieren Prophezeiungen in Theaterstücken nicht so …" Zugegeben, es gab alle möglichen Tragödien, in denen der Versuch, eine Prophezeiung zu umgehen, \emph{dazu führte}, dass sie eintrat, oder in denen andererseits der Versuch, einer Prophezeiung \emph{zu folgen}, der einzige Grund war, warum sie eintrat. Aber man \emph{konnte} Prophezeiungen auf seine eigene Weise eintreten lassen, wenn man klug genug war; oder jemand, der einen genug liebte, konnte seinen Platz einnehmen; oder mit genug Mühe war es möglich, eine Prophezeiung ganz zu brechen… Andererseits erinnerten sich die Seher in Theaterstücken auch nie daran, was sie Sahen…

Millicent musste Daphnes Zögern gesehen haben, denn das andere Mädchen begann, ein wenig zuversichtlicher zu werden. "Nun", sagte Millicent streng, "das ist kein Theaterstück! Hören mal, ich sage dir, ob ich Sehe, dass es ein harter oder ein leichter Kampf sein wird. Aber das ist \emph{alles}, was ich tun kann, verstehst du? Und wenn ich 'schwer' sage, \emph{kannst du nicht} nicht auftauchen! Oder - oder -" Millicents Augen rollten in ihrem Kopf zurück, und sie intonierte mit hohler Stimme: "\emph{Diejenigen, die versuchen, ihrem Schicksal ein Schnippchen zu schlagen, werden ein trauriges und düsteres Ende nehmen} -"

Professor Sprout schüttelte den Kopf, ihr Gesicht wirkte angespannt.

"Aber -", sagte Susan. "Aber Sie haben \emph{Harry Potter} damals geholfen -"

"Und mir wurde \emph{klar gemacht}", sagte Professor Sprout mit einer Stimme, die sich anhörte, als würde ihr jemand mit einem Schrumpfungszauber die Kehle zuschnüren, "dass es Professor Snapes Aufgabe ist und nicht meine, im Haus Slytherin für Ordnung zu sorgen - Miss Bones, \emph{bitte}, Sie \emph{müssen} das nicht tun, wenn -"

"Doch, das \emph{muss} ich", sagte Susan unglücklich. "Ich bin eine Hufflepuff, wir müssen loyal sein."

"Ein geheimnisvolles Pergament unter deinem Kopfkissen?", fragte Harry Potter und blickte von seinem Platz in der mit \emph{Quietus} belegten Ecke auf, in der sie gerade lernten. Dann verengten sich die grünen Augen des Jungen. "Es war nicht vom Weihnachtsmann, oder?"

Pause.

"Okay", sagte Hermine. "Ich werde \emph{nicht} fragen, und du wirst es mir \emph{nicht} sagen, und wir werden \emph{beide} so tun, als hättest du das nie gesagt und ich wüsste nichts davon -"

Susan näherte sich dem Tisch, sobald das ältere Mädchen allein war, und schaute sich flüchtig im Hufflepuff-Gemeinschaftsraum um, um sicherzugehen, dass niemand sie beobachtete (so wie es ihr ihre Tante beigebracht hatte, damit nicht offensichtlich war, dass sie sich umsah).

"Hey, Susie", sagte die Hufflepuff-Schülerin im siebten Jahr. "Brauchst du schon mehr -"

"Kann ich dich bitte kurz unter vier Augen sprechen?" sagte Susan.

Jaime Astorga, Siebtklässler aus Slytherin und bis vor kurzem noch ein vielversprechender Emporkömmling im Jugendduellkreis, stand kerzengerade in Professor Snapes Büro, die Zähne fest zusammengebissen und der Schweiß rann ihm den Rücken hinunter.

"Ich erinnere mich deutlich daran", sagte der Hauslehrer in einem sardonischen Tonfall, "dass ich Sie und einige andere heute Morgen gewarnt habe, dass es gewisse Erstklässlerinnen gibt, die sich als lästig erweisen können, wenn ein Kämpfer \emph{unvorsichtig} ist und sich \emph{überrumpeln} lässt."

Professor Snape pirschte in einem langsamen Kreis um ihn herum.

"Ich -", sagte Jaime, während ihm noch mehr Schweiß von der Stirn perlte. Er wusste, wie lächerlich es klang, wie erbärmlich die Ausrede war. "Sir, sie hätten nicht in der Lage sein sollen -" Eine Erstklässlerin hätte nicht in der Lage sein dürfen, sein \emph{Protego} zu brechen, egal, welchen uralten Zauber sie benutzte - Greengrass musste \emph{Hilfe} gehabt haben -.

Aber es war ganz klar, dass sein Hauslehrer das nicht glauben würde.

"Oh, da stimme ich zu", murmelte Snape in einem tiefen, drohenden Tonfall. "Sie hätten nicht. Ich beginne mich zu fragen, ob Mr. Malfoy, was auch immer er vorhat, nicht doch Recht hat, Astorga. Es kann nicht gut für den Ruf des Hauses Slytherin sein, wenn unsere Kämpfer, anstatt ihre Stärke zu demonstrieren, gegen kleine Mädchen verlieren!" Snapes Stimme erhob sich. "Es ist gut, dass Sie den guten Geschmack hatten, sich von einem kleinen Mädchen besiegen zu lassen, das ein Slytherin aus einem noblen Haus ist, Astorga, sonst würde ich selbst Ihnen Punkte abziehen!"

Jaime Astorgas Fäuste ballten sich an seiner Seite, aber ihm fiel nichts ein, was er hätte sagen können.

Es dauerte einige Zeit, bis Jaime Astorga von seinem Hauslehrers entlassen wurde.

Und danach sahen nur noch die Wände, der Boden und die Decke das Lächeln von Severus Snape.

An diesem Abend bekam Draco Besuch von der Eule seines Vaters, Tanaxu, die nicht grün war, aber nur, weil es so etwas wie grüne Eulen nicht gab. Das Beste, was Vater hatte finden können, war eine Eule mit reinen silbernen Federn, mit großen leuchtendgrünen Augen und einem Schnabel, der so scharf und grausam war wie die Reißzähne einer Schlange. Das Pergament, das um Tanaxus Bein gewickelt war, war kurz und bündig:

\emph{Was tust du, mein Sohn?}

Das Pergament, das Draco zurückschickte, war ebenso kurz und lautete,

\emph{\emph{Ich versuche, den Schaden für Slytherins Ruf} \emph{abzuwehren, Vater.}}

In der gleichen Zeit, die eine Eule brauchte, um von Hogwarts nach Malfoy Manor und wieder zurück zu fliegen, trug die Familieneule eine weitere Nachricht an Draco, und diese lautete nur:

\emph{Was machst du wirklich?}

Draco starrte auf das Pergament, das er aus dem Bein der Eule ausgewickelt hatte. Seine Hände zitterten, als er das Pergament in den Schein des Kamins hielt. Vier Worte, eingeritzt mit schwarzer Tinte, sollten nicht furchterregender sein als der Tod.

Es blieb nicht viel Zeit zum Nachdenken. Vater wusste genau, wie lange es dauerte, bis eine Nachricht von Malfoy Manor nach Hogwarts und wieder zurück gelangte; er würde es wissen, wenn Draco es hinauszögerte um eine ausgeklügelte Lüge zu verfassen.

Aber Draco wartete noch, bis seine Hand aufhörte zu zittern, bevor er seine Antwort schrieb, die einzige Antwort, die ihm einfiel und die Vater akzeptieren könnte.

\emph{Ich bereite mich auf den nächsten Krieg vor.}

Draco wickelte das Pergament um das Bein der Eule und band es fest, dann schickte er Tanaxu aus seinem Zimmer, durch die Hallen von Hogwarts, in die Nacht hinaus.

Er wartete, aber es kam keine Antwort.

