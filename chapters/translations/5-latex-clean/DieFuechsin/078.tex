

\hypertarget{kompromisse-und-tabus---vorspiel-schummeln}{% \section{46. Kompromisse und Tabus - Vorspiel: Schummeln}\label{kompromisse-und-tabus---vorspiel-schummeln}}

-\/-\/-\/-\/- Kapitel 78: Kompromisse und Tabus - Vorspiel: Schummeln -\/-\/-\/-\/-

Es war Samstag, der 4. April, im Jahr 1992.

Mr. und Mrs. Davis sahen ziemlich nervös aus, als sie in einem bestimmten Bereich der Quidditch-Tribüne von Hogwarts saßen - obwohl die gepolsterten Bänke heute nicht auf fliegende Besenstiele blickten, sondern auf ein gigantisches Quadrat aus so etwas wie Pergament; eine große weiße Leere, die bald Fenster mit Gras und Soldaten zeigen würde. Für den Moment zeigte es nur die stumpfe graue Farbe des umgebenden bedeckten Himmels. (Es sah ziemlich stürmisch aus, obwohl die Wetterzauberer versprochen hatten, dass der Regen nicht vor Einbruch der Nacht einsetzen würde.)

Normalerweise war es die uralte Tradition von Hogwarts, dass Eltern Draußen Bleiben sollten - aus demselben Grund, aus dem ungeduldige Kinder aufgefordert werden, aus der Küche zu gehen und sich nicht in die Angelegenheiten des Kochs einzumischen. Der einzige Grund für eine Eltern-Lehrer-Konferenz war, wenn ein Lehrer das Gefühl hatte, dass ein Elternteil sich nicht ordentlich benahm. Es bedurfte eines außergewöhnlichen Umstandes, damit die Hogwarts-Verwaltung das Gefühl hatte, sich vor \emph{dir} rechtfertigen zu müssen. Im Allgemeinen hatte die Hogwarts-Verwaltung bei jeder Gelegenheit den Rückhalt von achthundert Jahren angesehener Geschichte und du nicht.

Deshalb hatten Mr. und Mrs. Davis mit einigem Zögern auf einer Audienz bei der stellvertretenden Schulleiterin McGonagall bestanden. Es war schwer, ein angemessenes Gefühl der Empörung aufzubringen, wenn man derselben würdevollen Hexe gegenüberstand, die zwölf Jahre und vier Monate zuvor die beiden zwei Wochen nachsitzen ließ, nachdem sie sie auf frischer Tat bei der Zeugung von Tracey ertappt hatte.

Andererseits wurde Mr. und Mrs. Davis' Mut dadurch gestärkt, dass sie wütend mit einem Exemplar vom \emph{Quibbler} herumfuchtelten, dessen Schlagzeile in heller, fetter Schrift für alle Welt zu sehen war:

PAKTE MIT POTTER?

\hfill\break

BONES, DAVIS, GRANGER

\hfill\break

IN LIEBESVIERECK DER ANGST

Und so hatten sich Mr. und Mrs. Davis in die Fakultätsloge der Hogwarts-Quidditch-Tribüne vorgekämpft, wo sie nun mit einem ausgezeichneten Blick auf Professor Quirrells verzauberte Bildschirme saßen, so dass die beiden mit eigenen Augen sehen konnten: „Was zum Henker ist in dieser Schule eigentlich los, wenn Sie den Ausdruck verzeihen, stellvertretende Schulleiterin McGonagall!"

Links von Mr. Davis saß ein weiterer besorgter Elternteil, ein weißhaariger Mann in eleganten schwarzen Roben von unvergleichlicher Qualität, ein gewisser Lucius Malfoy, politischer Führer der stärksten Fraktion des Zaubergamot.

Links von Lord Malfoy ein spöttischer, aristokratischer Mann mit einem vernarbten Gesicht, der ihnen als Lord Jugson vorgestellt wurde.

Dann ein älterer, aber scharfäugiger Bursche namens Charles Nott, von dem man munkelt, er sei fast so wohlhabend wie Lord Malfoy, der links von Lord Jugson saß.

Zur Rechten von Mrs. Davis saßen die hübsche Lady und der noch hübschere Lord des edlen und uralten Hauses Greengrass. Sie waren jung nach der Zeitrechnung von Zauberern, und trugen graue Seidengewänder mit winzigen dunklen Smaragden, die in die Form von Grashalmen gestickt waren. Die Lady Greengrass galt als entscheidende Stimme im Zaubergamot, nachdem ihre eigene Mutter sich überraschend frühzeitig aus dem Gremium zurückgezogen hatte. Ihr charmanter Ehemann hatte, obwohl seine Familie an sich nicht adlig oder wohlhabend war, einen Sitz im Hogwarts-Schulrat eingenommen.

Zu ihrer Rechten saß eine kantige und unglaublich zäh aussehende alte Hexe, die Mr. und Mrs. Davis ohne den geringsten Anflug von Herablassung die Hand geschüttelt hatte. Das war Amelia Bones, Direktorin der Abteilung für magische Strafverfolgung.

Zu Amelias Rechten saß eine ältere Frau, die die Modeszene des magischen Britanniens auf den Kopf gestellt hatte, indem sie einen lebenden Geier in ihren Hut integriert hatte, eine Augusta Longbottom. Obwohl sie nicht mit Lady angesprochen wurde, übte Madam Longbottom die vollen Rechte der Familie Longbottom aus, solange deren letzter Spross noch nicht volljährig war, und sie galt als prominente Figur in einer Minderheitenfraktion des Zaubergamot.

An der Seite von Madam Longbottom saß kein Geringerer als der Oberste Hexenmeister, der Schulleiter Albus Percival Wulfric Brian Dumbledore, legendärer Bezwinger Grindelwalds, Beschützer Britanniens, Wiederentdecker der sagenumwobenen zwölf Verwendungen von Drachenblut, mächtigster Zauberer der Welt etc.

Und schließlich, ganz rechts, fand man den rätselhaften Verteidigungsprofessor von Hogwarts, Quirinus Quirrell, der sich auf den gepolsterten Bänken zurücklehnte, als würde er sich ausruhen; Er schien sich in der Gesellschaft eines stimmberechtigten Quorums des Hogwarts-Schulrats wohl zu fühlen, das an diesem schönen Samstag vorbeigekommen war, um zu erfahren, was in Hogwarts im Allgemeinen und bei Draco Malfoy, Theodore Nott, Daphne Greengrass, Susan Bones und Neville Longbottom im Besonderen los war. Auch über den Namen Harry Potter war viel diskutiert worden.

Oh, und natürlich durfte man Tracey Davis nicht vergessen. Direktor Bones' Augenbrauen hatten sich interessiert gehoben, als sie das junge Paar als ihre Eltern vorgestellt bekam. Lord Jugson hatte ihnen einen kurzen, ungläubigen Blick zugeworfen, bevor er sie mit einem Schnauben abtat. Lucius Malfoy hatte sie höflich begrüßt, sein Lächeln enthielt einen Hauch von grimmiger Belustigung gemischt mit Mitleid.

Mr. und Mrs. Davis, deren letzte Abstimmung über irgendetwas von Bedeutung darin bestanden hatte, mit ihren Zauberstäben den Namen von Minister Fudge zu berühren, die ganze dreihundert Galleonen in ihrem Gringotts-Tresor aufbewahrten und die jeweils als Verkäufer von Kesseln in einem Zaubertränke-Laden und als Verzauberer von Omniokularen arbeiteten, saßen eng aneinandergepresst, starr aufrecht auf ihren gepolsterten Bänken und wünschten sich verzweifelt, sie hätten schönere Roben getragen.

Der Himmel über ihnen war eine feste, in dunklere und hellere Grautöne aufgelöste Wolkenmasse, düster mit der Verheißung zukünftiger Gewitter; doch noch flackerten keine Blitze, noch hallte fernes Donnergrollen, und nur ein paar bedrohliche Tropfen waren gefallen.

Zu dem ihnen zugewiesenen Startplatz in einem bestimmten Wald marschierte das Sonnenschein-Regiment, obwohl es eigentlich eher ein langsamer Spaziergang war; man wollte sich nicht ermüden, bevor die Schlacht überhaupt begonnen hatte, und die Brise war im April ärgerlich feucht und kühl. Vor ihnen wanderte eine gelbe Flamme langsam durch die Luft und lenkte sie je nach ihrem Tempo.

Susan Bones warf dem Sonnenschein-General immer wieder besorgte Blicke zu, während sie durch den grau beleuchteten Wald marschierten. Dass Professor Snape hinter Hermine her war, schien sie wirklich erschüttert zu haben. Hermine hatte sogar ihr offizielles Planungstreffen des Sonnenschein-Regiments verpasst, was verständlich genug schien; aber als Susan ihr danach ihr Mitgefühl angeboten hatte, hatte Hermine gestammelt, dass sie die Zeit aus den Augen verloren hatte, was sie normalerweise überhaupt nicht sagte, und das Mädchen hatte erschöpft und verängstigt ausgesehen, als hätte sie gerade drei Tage mit einem Dementor in einer Toilettenkabine eingesperrt verbracht. Selbst jetzt, wo die ganze Aufmerksamkeit des Sonnenschein-Generals auf die bevorstehende Schlacht hätte gerichtet sein sollen, huschte der Blick des Ravenclaw-Mädchens ständig in alle Richtungen, als erwarte sie, dass dunkle Zauberer aus dem Gebüsch springen und sie opfern würden.

„Das Verbot von Muggelartefakten schränkt unsere Möglichkeiten sehr ein“, sagte Anthony Goldstein in dem mürrischen Ton, mit dem der Junge absichtlichen Pessimismus ausdrückte. „Ich hatte die Idee, zu versuchen, Netze zu verwandeln, um sie auf die Leute zu werfen, aber -"

„Nicht gut“, sagte Ernie Macmillan. Der Hufflepuff-Junge schüttelte den Kopf und sah noch ernster aus als Anthony. „Ich meine, das ist so, als würde man einen Fluch werfen, sie würden \emph{ausweichen}."

Anthony nickte. „Das habe ich mir auch gedacht. Hast du eine Idee, Seamus?„

Der ehemalige Chaotische Leutnant sah immer noch ein wenig nervös und deplatziert aus, als er mit seinen neuen Kameraden im Sonnenschein-Regiment mitmarschierte. „Tut mir leid“, sagte der frischgebackene Captain Finnigan. „Ich bin eher der strategische Meistertyp."

„\emph{Ich} bin der strategische Meistertyp“, sagte Ron Weasley und klang verärgert.

„Es gibt \emph{drei} Armeen“, sagte der Sonnenschein-General bissig, „das heißt, wir kämpfen gegen \emph{zwei} Armeen gleichzeitig, das heißt, wir brauchen mehr als einen Strategen, das heißt, halt die Klappe, Ron!„

Ron warf ihrem General einen überraschten und besorgten Blick zu. „Hey“, sagte der Gryffindor-Junge in einem beruhigenden Ton, „du solltest die Sache mit Snape nicht so ernst nehmen -"

„Was denkst \emph{du}, was wir tun sollten, General?“ sagte Susan sehr laut und schnell. „Ich meine, wir haben im Moment nicht wirklich einen Plan.“ Ihre offizielle Planungssitzung war \emph{erstaunlicherweise} gescheitert, weil Hermine nicht da war und sowohl Ron als auch Anthony dachten, sie hätten das Sagen.

„Brauchen wir wirklich einen Plan?“, fragte der Sonnenschein-General und klang dabei ein wenig abgelenkt. „Wir haben dich und mich und Lavender und Parvati und Hannah und Daphne und Ron und Ernie und Anthony \emph{und} Captain Finnigan."

„Das -“, begann Anthony.

„Klingt nach einer ziemlich guten Strategie“, sagte Ron mit einem zustimmenden Nicken. „Wir haben jetzt so viele starke Soldaten wie die beiden anderen Armeen zusammen. Das Chaos hat nur noch Potter und Longbottom und Nott - na ja, und Zabini auch, nehme ich an -"

„Und Tracey“, sagte Hermine.

Mehrere Leute schluckten nervös.

„Ach, hör doch auf“, sagte Susan scharf. „Sie ist einfach ein kampferprobtes Mitglied von B. E.L. F.E. R., das ist alles, was General Sunshine meint."

„Trotzdem“, sagte Ernie und drehte sich um, um Susan ernst anzuschauen, „ich denke, du solltest dich der Gruppe anschließen, die gegen Chaos kämpft, Captain Bones. Ich weiß, dass du deine doppelt magischen Kräfte nur einsetzen kannst, wenn Unschuldige in Gefahr sind, aber ich meine - nur für den Fall, dass Miss Davis, du weißt schon, \emph{außer Kontrolle} gerät und versucht, jemandes Seele zu fressen -"

„Ich werde schon mit ihr fertig“, sagte Susan, wobei ihre Stimme beruhigend klang. Zugegeben, Susan war im Moment nicht durch einen Metamorphmagus ersetzt worden, aber dann war Tracey wohl auch nicht der mit Vielsafttrank verwandelte Dumbledore oder wer auch immer.

Captain Finnigan intonierte mit tiefer, irgendwie brummiger Stimme: „Ich finde deinen Mangel an Skepsis beunruhigend.“ Er hob seine Hand, wobei sich Daumen und Zeigefinger fast berührten, und zeigte auf Ernie.

Aus irgendeinem Grund schien Anthony Goldstein einen plötzlichen Würgeanfall zu bekommen. „Was soll das denn bedeuten?“, fragte Ernie.

„Das ist nur etwas, was General Potter manchmal sagt“, sagte Captain Finnigan. „Komisch, wenn man zum ersten Mal der Chaos Legion beitritt, scheint alles verrückt zu sein, und dann nach ein paar Monaten merkt man, dass eigentlich jeder, der nicht in der Chaos Legion ist, verrückt ist -"

„Ich \emph{sagte}“, sagte Ron laut, „es klingt nach einer guten Strategie. Wir verwandeln nichts, wir ermüden uns nicht, wir werden mit allem fertig, was sie uns entgegenwerfen, und dann überrennen wir sie einfach."

„Okay“, sagte Hermine. „Dann machen wir das."

„Aber -“, sagte Anthony und warf Ron einen finsteren Blick zu. „Aber General, Harry Potter hat noch \emph{sechzehn} Leute in seiner Armee. Drachen und wir haben jeweils achtundzwanzig. Harry weiß das, er weiß, dass er sich etwas Unglaubliches einfallen lassen muss -"

„Was zum Beispiel?“, fragte Hermine und klang gestresst. „Wenn wir nicht wissen, was er plant, können wir unsere Magie genauso gut dafür aufsparen, massenhaft \emph{Finite} zu zaubern. So wie wir es letztes Mal hätten tun sollen!„

Susan berührte Hermine sanft an der Schulter. „General Granger?“, sagte Susan. „Ich denke, du solltest dir vor der Schlacht eine kleine Pause gönnen."

Sie hatte erwartet, dass Hermine widersprechen würde, aber Hermine nickte nur und ging dann etwas schneller und entfernte sich von der Offiziersgruppe des Sonnenschein-Regiments, wobei ihre Augen immer noch den Wald und manchmal den Himmel beobachteten.

Susan folgte ihr. Es würde nicht gut gehen, wenn es so aussah, als würde die Sunshine-Generalin aus ihrer eigenen Offiziersgruppe hinausgeworfen werden.

„Hermine?“ Susan sagte leise, nachdem sie ein Stück weit gegangen waren. „Du musst dich konzentrieren. Professor Quirrell hat hier das Sagen, nicht Snape, und er wird nicht zulassen, dass dir oder anderen etwas Schlimmes zustößt."

„Du bist nicht hilfreich“, sagte Hermine und klang zittrig. „Du bist überhaupt nicht hilfreich, Captain Bones."

Die beiden gingen schneller, umkreisten einige der anderen Soldaten, inspizierten die Marschrichtung und warfen einen Blick auf die umliegenden Bäume.

"Susan?"sagte Hermine mit leiser Stimme, als sie sich weiter von den anderen entfernt hatten. „Glaubst du, Daphne hat recht, dass Draco Malfoy etwas plant?"

„Ja“, sagte Susan sofort, ohne überhaupt darüber nachzudenken. „Das sieht man daran, dass in seinem Namen die Buchstaben M-A-L-F-O und Y vorkommen."

Hermine sah sich um, als wolle sie sich vergewissern, dass niemand sie beobachtete, obwohl das natürlich eine wunderbare Methode war, um andere Leute dazu zu bringen, auf einen aufmerksam zu werden. „Könnte Malfoy hinter dem stecken, was Snape getan hat?"

„Snape könnte hinter Malfoy stecken“, sagte Susan nachdenklich und erinnerte sich an Tischgespräche, die sie bei ihrer Tante gehört hatte, „oder Lucius Malfoy könnte hinter beiden stecken.“ Ein leichter Schauer lief Susan über den Rücken, als ihr dieser letzte Gedanke in den Sinn kam. Plötzlich erschien es ihr weniger vernünftig, Hermine zu sagen, sie solle sich einfach auf den kommenden Kampf konzentrieren. „Warum, hast du überhaupt einen Hinweis darauf gefunden?„

Hermine schüttelte den Kopf. „Nein“, sagte das Ravenclaw-Mädchen mit einer Stimme, die fast so klang, als würde sie gleich weinen. „Ich habe - nur selbst darüber nachgedacht - das ist alles.„

An ihrem zugewiesenen Platz in einem Wald in der Nähe von Hogwarts warteten der Drachengeneral und die Krieger der Drachenarmee, wohin ihre rote Flamme sie geführt hatte, unter einem grauen Himmel.

An Dracos rechter Seite stand Padma Patil, seine Stellvertreterin, die einst die gesamte Drachenarmee angeführt hatte, nachdem Draco betäubt worden war. In Dracos Rücken stand Vincent, der Sohn von Crabbe, einer Familie, die den Malfoys bis in die Ferne der vergessenen Erinnerung gedient hatte; der muskulöse Junge war wachsam, wie er immer wachsam war, egal ob die Schlacht für begonnen erklärt worden war oder nicht. Weiter hinten stand Gregory von den Goyles wartend neben einem der beiden Besenstiele, die die Drachenarmee erhalten hatte; wenn die Goyles den Malfoys auch nicht so lange gedient hatten wie die Crabbes, so hatten sie doch nicht weniger gut gedient.

Und an Dracos linker Seite stand nun ein gewisser Dean Thomas aus Gryffindor, ein Schlammblut oder möglicherweise Halbblut, der nichts von seinem Vater wusste.

Dean Thomas zur Drachenarmee zu schicken, war ein ganz bewusster Schachzug von Harry gewesen, da war sich Draco sicher. Drei andere ehemalige Chaoten waren ebenfalls zur Drachenarmee versetzt worden, und alle beobachteten Draco wie ein Falke, um zu sehen, ob er dem ehemaligen Leutnant auch nur die kleinste Beleidigung zufügte.

Manche hätten es vielleicht Sabotage genannt, aber Draco wusste es besser. Harry hatte auch Leutnant Finnigan zum Sonnenschein Regiment geschickt, obwohl Professor Quirrells Mandat nur verlangt hatte, dass Harry \emph{einen} Leutnant abgab. Auch das war ein bewusster Schritt gewesen, der jedem klar machte, dass Harry \emph{nicht} einfach seine schlechtesten Soldaten ausmusterte.

Auf eine Art wäre es für Draco vielleicht einfacher gewesen, die wahre Loyalität seiner neuen Soldaten zu gewinnen, wenn sie gedacht hätten, dass Harry sie nicht gewollt hatte. Auf eine andere Art... nun, das war nicht leicht in Worte zu fassen. Harry hatte ihm gute Soldaten mit intaktem Stolz gegeben, aber es war mehr als das. Harry hatte seinen Soldaten gegenüber Freundlichkeit gezeigt, aber es war \emph{mehr} als das. Es war nicht nur, dass Harry fair gespielt hatte, es war etwas, das... das man nur mit der Art und Weise vergleichen konnte, wie das Spiel im Haus Slytherin gespielt wurde.

So hatte Draco Mr. Thomas nicht im Geringsten beleidigt, sondern ihn direkt an seine Seite gestellt, ihm und Padma untergeordnet, aber niemandem sonst. Es war ein Test, hatte Draco Mr. Thomas und allen anderen gesagt, keine Beförderung. Mr. Thomas würde sich innerhalb der Drachenarmee seines Ranges würdig erweisen müssen - aber er würde eine Chance bekommen, und die Chance \emph{würde} fair sein. Mr. Thomas hatte bei der Zeremonie überrascht ausgesehen (nach dem, was Draco gehört hatte, legte die Chaoslegion keinen Wert auf Formalitäten), aber der Gryffindor-Junge hatte sich ein wenig aufrechter hingestellt und genickt.

Und dann, nachdem Mr. Thomas in einer der Trainingseinheiten der Drachenarmee gut genug abgeschnitten hatte, war er zur Strategiesitzung in das riesige Militärbüro der Drachenarmee gebracht worden. Und nach ein paar Minuten der Sitzung hatte Padma zufällig gefragt - als wäre es eine ganz normale Frage - ob Mr. Thomas irgendwelche Ideen hätte, wie man die Chaos Legion besiegen könnte.

Der Gryffindor-Junge hatte fröhlich gesagt, dass Harry vorausgesagt hatte, dass General Malfoy einen seiner Soldaten dazu bringen würde, ihn das zu fragen, und dass Harry ihm die Botschaft gegeben hatte, dass General Malfoy sich fragen sollte, wo sein relativer Vorteil lag - was Draco Malfoy tun konnte oder was die Drachenarmee tun konnte, dem die Chaoslegion nicht gewachsen war - und dann versuchen sollte, ihn nach Kräften auszunutzen. Dean Thomas konnte sich nicht vorstellen, worin dieser Vorteil bestehen könnte, aber \emph{wenn} ihm eine Idee einfiel, wie man Chaos besiegen konnte, würde er sie mitteilen. Harry hatte es ihm schließlich befohlen.

\emph{Seufz}, hatte Draco gedacht, da er nicht wirklich laut seufzen konnte. Aber es war ein guter Rat, und Draco hatte ihn befolgt und saß mit Feder und Pergament an seinem Schlafzimmertisch und listete alles auf, was einen relativen Vorteil darstellen könnte.

Und, fast zu Dracos eigener Überraschung, hatte er eine Idee gehabt, eine richtige. Genau genommen hatte er sogar \emph{zwei}.

Die hohle Glocke schallte durch den Wald und klang irgendwie bedrohlicher als je zuvor. Sofort riefen die beiden Piloten „\emph{Auf!}“ und sprangen auf ihre Besenstiele, um in den grauen Himmel zu fliegen.

Mr. und Mrs. Davis waren inzwischen leicht aneinander zusammengesackt, mehr aus schierer Muskelerschöpfung als aus einem Nachlassen der Spannung. Vor ihnen flimmerte das weite, weiße Pergament mit drei großen Fenstern, als wären Löcher in den Wald geschnitten worden, die drei Armeen auf dem Marsch zeigten. Kleinere Fenster zeigten die sechs Reiter auf ihren Besenstielen, und die Ecke des Pergaments zeigte eine Ansicht des gesamten Waldes mit leuchtenden Punkten, die Armeen und Späher anzeigten.

Das Fenster von Sonnenscheinzeigte General Granger und ihre Hauptleute, die in der Mitte des Sonnenschein Regiments marschierten, geschützt durch \emph{Contego}-Schirme, zusammen mit einer Anzahl anderer junger Hexen. Das Sonnenschein Regiment, so hatte der Verteidigungsprofessor bemerkt, wusste sehr wohl, dass es nun einen starken Überschuss an erfahrenen Soldaten erworben hatte, und es galt, diese Soldaten vor einem Überraschungsangriff zu schützen. Abgesehen davon bewegten sich die Sonnenschein Soldaten in einem gleichmäßigen Marsch vorwärts, um ihre Kräfte zu schonen.

Die Soldaten in General Malfoys Armee, zumindest die mit den besseren Zensuren in Verwandlung, sammelten Blätter ein und verwandelten sie in... nun, wenn man Padma Patil ansah, die mit ihrem fast fertig war, sah es so aus, als würde ihr Blatt zu einem linkshändigen Handschuh mit einem baumelnden Riemen werden. (Das Fenster war herangezoomt, um das zu zeigen.)

Lord Jugson betrachtete den Bildschirm mit einem unbewegten Gesichtsausdruck; seine Stimme, als er sprach, schien vor Verachtung zu triefen und zu tropfen. „Was \emph{macht} Ihr Sohn, Lucius?"

Die fremdstämmige Hexe, die an Draco Malfoys rechter Seite stand, hatte die Verwandlung ihres Handschuhs beendet und brachte ihn nun wie ein Opfer vor den Drachengeneral.

„Ich weiß es nicht“, sagte Lucius Malfoy, sein Tonfall war ruhig, wenn auch nicht weniger aristokratisch, „aber ich muss darauf vertrauen, dass er einen guten Grund dafür hat, es zu tun."

Die gesamte Drachenarmee hielt einen Moment inne, als Padma den Handschuh über ihre linke Hand schob, ihn festschnallte und ihn Draco Malfoy präsentierte; der blieb ebenfalls stehen, holte einige Male tief Luft, hob seinen Zauberstab, führte eine präzise Folge von acht Bewegungen aus und brüllte „\emph{Colloportus!} "

Die Drachenkriegerin hob ihre behandschuhte Hand, beugte sie und machte eine kleine Verbeugung vor Draco Malfoy, der sie etwas oberflächlicher erwiderte, obwohl der Drachengeneral leicht schwankte. Dann kehrte Padma auf ihren Platz an Dracos Seite zurück, und die Drachen begannen erneut zu marschieren.

„Nun“, bemerkte Augusta Longbottom. „Ich nehme nicht an, dass jemand das erklären möchte?“ Amelia Bones runzelte leicht die Stirn, als sie auf den Bildschirm starrte.

„Aus irgendeinem Grund“, sagte die amüsierte Stimme von Professor Quirrell, „scheint der Spross von Malfoy in der Lage zu sein, für einen Erstklässler erstaunlich starke Zauber zu wirken. Das liegt natürlich an der Reinheit seines Blutes. Sicherlich hätte sich der gute Lord Malfoy nicht offen über die Gesetze der Minderjährigenmagie hinweggesetzt, indem er dafür gesorgt hätte, dass sein Sohn einen Zauberstab erhielt, bevor er in Hogwarts aufgenommen wird."

„Ich schlage vor, Sie sind vorsichtig mit Ihren Andeutungen, Quirrell“, sagte Lucius Malfoy kalt.

„Oh, das bin ich“, sagte Professor Quirrell. „Ein\emph{Colloportus} kann nicht durch \emph{Finite Incantatem} aufgelöst werden; er erfordert einen \emph{Alohomora} von gleicher Stärke. Bis dahin wird ein so verzauberter Handschuh geringeren materiellen Kräften widerstehen, die Schlaf-und die Betäubungszauberabwehren. Und da weder Mr. Potter noch Miss Granger einen Gegenzauber wirken können, der stark genug ist, ist dieser Zauber auf diesem Schlachtfeld unbesiegbar. Es ist weder die ursprüngliche Absicht des Zaubers, noch die Absicht desjenigen, der Mr. Malfoy einen Notfallzauber zum Ausweichen vor seinen Feinden beigebracht hat. Aber es scheint, dass Mr. Malfoy Kreativität gelernt hat."

Lucius Malfoy hatte sich aufgerichtet, als der Verteidigungsprofessor sprach; er saß nun aufrecht auf seiner gepolsterten Bank, den Kopf merklich höher erhoben als zuvor, und als er sprach, war es mit ruhigem Stolz. „Er wird der größte Lord Malfoy sein, der je gelebt hat."

„Ein spärliches Lob“, flüsterte Augusta Longbottom ironisch; Amelia Bones kicherte, ebenso wie Mr. Davis für einen winzigen, fatalen Bruchteil einer Sekunde, bevor er mit einem erstickten Gurgeln aufhörte.

„Ich bin ganz Ihrer Meinung“, sagte Professor Quirrell, obwohl nicht klar war, zu wem er sprach. „Unglücklicherweise für Mr. Malfoy ist er noch neu in der Kunst der Kreativität, und so hat er einen klassischen Fehler von Ravenclaw begangen."

„Und was könnte das sein?“, fragte Lucius Malfoy, dessen Stimme nun wieder kühl wurde.

Professor Quirrell hatte sich in seinem Sitz zurückgelehnt, die blassblauen Augen wurden kurz unscharf, als eines der Fenster seinen Blickwinkel innerhalb des großen Bildschirms verschob und heranzoomte, um den Schweiß zu zeigen, der nun auf Draco Malfoys Stirn stand. „Es ist eine so schöne Idee, dass Mr. Malfoy ihre pragmatischen Schwierigkeiten völlig übersehen hat."

„Könnte mir das jemand erklären?“, sagte Lady Greengrass. „Nicht alle von uns Anwesenden sind Experten in solchen... Angelegenheiten."

Amelia Bones ergriff das Wort, die Stimme der alten Hexe etwas trocken. „Es wird sie dazu verleiten, zu versuchen, Flüche zu fangen, denen sie besser einfach ausweichen sollten. Umso mehr, wenn sie wenig Übung darin haben, sie zu fangen. Und das Wirken von so vielen Zaubern wird ihren stärksten Krieger ermüden."

Professor Quirrell nickte der AMS-Direktorin anerkennend zu. „Wie Sie sagen, Madam Bones. Für Mr. Malfoy ist es neu, Ideen zu haben, und wenn er eine hat, ist er stolz auf sich, weil er sie hat. Er hat noch nicht genug Ideen gehabt, um unbeirrt diejenigen zu verwerfen, die in einigen Aspekten schön und in anderen unpraktisch sind; er hat noch kein Vertrauen in seine eigene Fähigkeit erworben, sich bessere Ideen auszudenken, wenn er sie braucht. Was wir hier sehen, ist nicht Mr. Malfoys beste Idee, fürchte ich, sondern eher seine einzige Idee."

Lord Malfoy wandte sich einfach wieder den Bildschirmen zu, als hätte der Verteidigungsprofessor seine Existenzberechtigung verbraucht.

„Aber -“, sagte Lord Greengrass. „Aber was in Merlins Namen macht Harry Potter -„

Sechzehn verbliebene Soldaten der Chaoslegion - oder vielmehr fünfzehn plus Blaise Zabini - marschierten selbstbewusst durch den Wald, ihre Schuhe polterten über den noch trockenen Boden. Ihre Tarnuniformen fügten sich noch mehr als sonst in den Wald ein, alle Farben waren von den Tönen eines bedeckten Tages verwaschen.

Sechzehn Chaoslegionäre, gegen achtundzwanzig Drachenkrieger und achtundzwanzig Sonnenscheinsoldaten.

Der allgemeine Konsens war, dass es bei so schlechten Chancen praktisch unmöglich war, dass sie verlieren würden. Schließlich musste sich General Chaos bei solchen Chancen etwas wirklich \emph{Spektakuläres} einfallen lassen.

Es hatte fast etwas Alptraumhaftes an sich, dass jeder zu erwarten schien, dass Harry bei Bedarf Wunder aus dem Hut zaubern würde, wann immer es nötig war. Es bedeutete, dass du, wenn du das Unmögliche nicht schaffst, deine Freunde \emph{enttäuschst} und \emph{deinem Potenzial nicht gerecht wurdest}...

Harry hatte sich nicht die Mühe gemacht, sich bei Professor Quirrell über „zu viel Druck“ zu beschweren. Harrys mentales Modell des Verteidigungsprofessors hatte vorhergesagt dass er mit einem stark verärgerten Gesichtsausdruck etwas sagte wie: „\emph{Sie sind durchaus in der Lage, dieses Problem zu lösen, Mr. Potter; haben Sie es überhaupt versucht?}“ und dann mehrere hundert Quirrellpunkte abzog.

Von oben, wo zwei Besenstiele ihren Marsch beobachteten, rief die hohe junge Stimme von Tess Walsh „Freund!“ und nach einem weiteren Moment „Pfefferkeks!„

Ein paar Sekunden später kehrte die Soldatin, die sich den Codenamen Pfefferkeks gegeben hatte, mit einer doppelten Handvoll Eicheln zurück. Sie schwitzte leicht in der kühlen, aber feuchten Luft von dem Jogging, das sie zu der Eiche gebracht hatte, die Neville entdeckt hatte. Pfefferkeks näherte sich der Stelle, an der Shannon ein Uniformhemd in der Hand hielt, das am Hals zugebunden war, damit niemand eine Tasche verwandeln musste. Als Pfefferkeks ihre Hände nach vorne brachte, um zu versuchen, ihre Eicheln in das Halte-Shirt zu kippen, riss die chaotische Shannon kichernd das Shirt nach rechts, dann wieder nach links, als Pfefferkeks einen weiteren Versuch unternahm, die Eicheln reinzukippen, bis ein scharfes „Miss Friedman!“ von Lieutenant Nott Shannon dazu veranlasste, zu seufzen und das Shirt stillzuhalten. Pfefferkeks kippte ihre Eicheln auf die gesammelten und machte sich dann auf die Suche nach weiteren.

Irgendwo im Hintergrund sang Ellie Knight ihre ganz eigene Version des Marschlieds der Chaos Legion, und etwa die Hälfte der anderen Soldaten versuchte, mitzumarschieren, obwohl sie die Melodie nicht kannten. In der Nähe fertigte Nita Berdine, die eine hohe Punktzahl in Verwandlung hatte, eine weitere grüne Sonnenbrille an und reichte sie Adam Beringer, der die Sonnenbrille zusammenfaltete und in seine Uniformtasche steckte. Andere Soldaten trugen bereits ihre eigenen grünen Sonnenbrillen, trotz des bewölkten Tages.

Du könntest vermuten, dass es dafür eine unglaublich komplizierte und faszinierende Erklärung gab, und du hättest Recht.

Zwei Tage zuvor hatte Harry inmitten seiner Bücherregale in dem bequemen Schaukelstuhl gesessen, den er sich für das Untergeschoss seiner Truhe besorgt hatte, und in der stillen Zeit zwischen Unterricht und Abendessen still vor sich hin gegrübelt und über Macht nachgedacht.

Damit sechzehn Chaoten achtundzwanzig Sonnen und achtundzwanzig Drachen besiegen konnten, brauchten sie einen Kraftverstärker. Es gab Grenzen, was man mit Manövern machen konnte. Es \emph{musste} eine Geheimwaffe geben, und die musste unbesiegbar sein, oder zumindest einigermaßen unaufhaltsam.

Muggel-Artefakte waren jetzt in den Scheinkämpfen von Hogwarts verboten, verboten durch einen Erlass des Ministeriums. Und das Problem, einen anderen cleveren und ungewöhnlichen Zauber zu finden, war, dass eine Armee, die doppelt so groß war wie man selbst, fast alles, was man versuchte, mit der Brute-Force-Methode mittels \emph{Finite} beenden konnte. Das Sonnenscheinregiment mochte diese Taktik mit dem verwandelten Kettenhemd übersehen haben, aber jetzt, wo Professor Quirrell darauf hingewiesen hatte, würde sie niemand mehr übersehen. Und \emph{Finite Incantatem} war ein Brute-Force Gegenzauber, der mindestens so viel Magie erforderte wie der Zauber, der gebrochen werden sollte... was, wenn man zahlenmäßig stark unterlegen war, eine ganz neue Art von militärischer Herausforderung darstellte. Der Feind konnte mit \emph{Finite} alles beenden, was man versuchte, und hatte immer noch genug Magie für Schilde und Salven von Schlafzaubern übrig.

Es sei denn, man konnte irgendwie Potenzen freisetzen, die über die gewöhnliche Kraft von Hogwartsschülern im ersten Jahr hinausgingen, etwas, das für den Feind zu mächtig war, um es zu beenden.

Also hatte Harry Neville gefragt, ob er jemals von \emph{kleinen, sicheren} Opferritualen gehört hatte -

Und dann, nachdem das Geschrei und die Rufe abgeklungen waren, nachdem Harry aufgehört hatte, über Unbrechbare Schwüre zu streiten und die ganze Sache einfach als unmöglich vom Standpunkt der Öffentlichkeitsarbeit aus aufgegeben hatte, war Harry klar geworden, dass er nicht einmal dorthin hätte gehen müssen. In den normalen Hogwarts-Klassen wurde einem beigebracht, wie man Potenzen beschwört, die weit über die eigene Kraft hinausgehen.

Manchmal erkannte man, obwohl man gerade auf etwas schaute, nicht, worauf man schaute, bis man zufällig genau die richtige Frage stellte.

\emph{Verteidigung. Zauberei. Verwandlung. Zaubertränke. Geschichte der Magie. Astronomie. Besenfliegen. Kräuterkunde...}

„\emph{Feind!}“, schrie die Stimme von oben.

Gut, dass Neville Longbottom nicht die leiseste Ahnung hatte, dass seine Großmutter ihn beobachtete, sonst wäre er vielleicht zu schüchtern gewesen, wenn er aus vollem Halse furchterregende Kampfschreie gebrüllt und alle drei Sekunden \emph{Luminos} gewirkt hätte, während er durch einen dichten Wald von Bäumen raste, dicht auf den Fersen von Gregory Goyle.

(„Aber -“ sagte Augusta Longbottom, wobei ihr Gesichtsausdruck fast so viel Erstaunen wie Sorge zeigte. „Aber Neville hat doch Höhenangst!“)

(„Nicht alle Ängste sind von Dauer“, sagte Amelia Bones. Die alte Hexe musterte die große Leinwand vor ihnen mit einem messenden Blick. „Oder vielleicht hat er Mut gefunden. Am Ende ist es dasselbe.“)

Ein roter Schimmer-

Neville wich aus, fast gegen einen Baum, aber er wich aus; und dann schaffte Neville es irgendwie auch, \emph{fast} allen Ästen auszuweichen, bevor sie ihm ins Gesicht schlugen.

Jetzt zog Mr. Goyles Besenstiel immer weiter weg - obwohl sie beide auf genau dem gleichen Besenstiel saßen und Mr. Goyle mehr wog, fiel Neville irgendwie immer noch zurück. Also verlangsamte Neville, zog zurück, stieg aus dem Wald auf und begann, zurück in Richtung der Chaoslegion zu beschleunigen, die immer noch marschierte.

Zwanzig Sekunden später - es war keine lange Verfolgungsjagd gewesen, nur eine \emph{aufregende} - war Neville wieder inmitten seiner Chaoten-Kollegen und stieg von seinem Besen ab, um ein wenig auf dem Boden zu laufen.

„Neville -“, sagte General Potter. Harrys Stimme klang ein wenig abwesend, während er vorsichtig und stetig durch den Wald ging, seinen Zauberstab immer noch auf die fast fertige Form des Objekts angesetzt, das er langsam verwandelte. Neben ihm sah Blaise Zabini, der an einer kleineren Version derselben Verwandlung arbeitete, wie ein watschelnder Inferi aus, als er vorwärts stolperte. „Ich habe dir gesagt - Neville - du musst nicht -"

„Doch, muss ich“, sagte Neville. Er schaute nach unten, wo seine Finger den Besenstiel umklammerten, und sah, dass nicht nur seine Hände, sondern seine ganzen Arme zitterten. Aber wenn nicht irgend jemand anders in Chaos jeden Tag eine Stunde lang mit Mr. Diggory das Duellieren geübt hatte und danach noch eine Stunde lang privat das Zielen, war Neville wahrscheinlich der beste Schütze vom Besenstiel aus, selbst wenn man berücksichtigte, dass er kein besonders guter Flieger war.

„Gute Show, Neville“, sagte Theodore der vor ihnen allen herlief und die Chaos Legion durch den Wald anführte, während er nur sein Unterhemd trug.

(Augusta Longbottom und Charles Nott tauschten kurze erstaunte Blicke aus und rissen dann ihre Blicke wie versteinert voneinander los.)

Neville holte ein paar Mal tief Luft und versuchte, seine Hände zu beruhigen; Harry war vielleicht nicht gut für scharfsinniges strategisches Denken zu gebrauchen, während er mitten in einer ausgedehnten Verwandlung steckte. „Leutnant Nott, haben Sie eine Ahnung, warum die Drachenarmee das gerade getan hat? Sie haben einen Besen verloren -“ Die Drachen hatten den Kampf mit einer Finte begonnen, um eine Ablenkung für Mr. Goyles Annäherung durch den Wald zu schaffen; Neville hatte erst fast zu spät bemerkt, dass es \emph{zwei} Besen waren, die angriffen. Aber die Chaos Legion hatte den anderen Piloten \emph{erwischt}. Das war der Grund, warum Besen normalerweise nicht angriffen, bevor Armeen aufeinander trafen, es bedeutete, dass eine ganze Armee das Feuer auf den Besen konzentrieren würde. „Und die Drachen haben niemanden erwischt, oder?"

„Nö!“ sagte Tracey Davis stolz. Auch sie marschierte nun an der Seite von General Potter, ihren Zauberstab tief und wachsam umklammert, während ihre Augen den umliegenden Wald abtasteten. „Ich habe eine prismatische Barriere errichtet, etwa den Bruchteil einer Sekunde, bevor Mr. Goyle Zabini verflucht hat, und so wie Mr. Goyle seinen anderen Arm ausgestreckt hatte, glaube ich, dass er auch den General niederschlagen wollte.“ Die Slytherin-Hexe lächelte mit boshafter Zuversicht. „Mr. Goyle versuchte einen Brechenden Bohrzauber, musste aber zu seinem Entsetzen feststellen, dass seine schwache Magie meinen neu entdeckten dunklen Kräften nicht gewachsen war, hahahaha!"

Einige Chaoten lachten mit ihr, aber in Nevilles Magen machte sich ein mulmiges Gefühl breit, als er erkannte, wie nahe die Chaos Legion einer völligen Katastrophe gekommen war. Wenn Mr. Goyle es geschafft hatte, beide Transfigurationen zu stören--

„Bericht!“, schnappte der Drachengeneral und tat sein Bestes, um die Müdigkeit zu verbergen, die er nach dem Wirken von siebzehn Sperrzaubern verspürte, auf die noch weitere folgen sollten.

Auf Gregorys Stirn standen jetzt Schweißperlen. „Der Feind hat Dylan Vaughan erwischt“, sagte Gregory förmlich. „Harry Potter und Blaise Zabini haben jeweils etwas Dunkelgraues und Rundliches verwandelt, ich glaube nicht, dass es fertig war, aber es sah aus, als wäre es groß und hohl, eine Art Kessel. Der von Zabini war kleiner als der von Potter. Ich konnte keinen der beiden erwischen oder ihre Verwandlungen stören, Tracey Davis blockierte mich. Neville Longbottom sitzt auf einem Besenstiel und ist immer noch ein schrecklicher Flieger, aber er zielt wirklich gut."

Draco hörte zu und runzelte die Stirn, dann schaute er zu Padma und Dean Thomas, die beide den Kopf schüttelten und damit andeuteten, dass ihnen auch nicht einfiel, was groß und grau und wie ein Kessel geformt sein könnte.

„Sonst noch etwas?“, fragte Draco. Wenn das alles war, hatten sie einen Besen umsonst verloren.

„Das einzige andere Seltsame, was ich gesehen habe“, sagte Gregory und klang verwirrt, „war, dass einige Chaoten eine Art... Brille trugen?"

Draco dachte darüber nach und bemerkte nicht, dass er aufgehört hatte zu marschieren oder dass die gesamte Drachenarmee automatisch mit ihm stehen geblieben war.

„War da irgendetwas Besonderes an den Brillen?“ fragte Draco.

„Ähm...“ sagte Gregory. „Sie waren... grünlich, vielleicht?"

„Okay“, sagte Draco. Wieder ohne nachzudenken, begann er wieder zu gehen und seine Drachen folgten ihm. „Hier ist unsere neue Strategie. Wir werden nur noch elf Drachen gegen die Chaoslegion schicken, nicht vierzehn. Das sollte reichen, um sie zu besiegen, jetzt, wo wir ihren besonderen Vorteil neutralisieren können.“ Es war ein Glücksspiel, aber man musste es manchmal riskieren, wenn man in einem Dreikampf als Erster ankommen wollte.

„Sie haben den Plan von Chaos durchschaut, General Malfoy?“, fragte Mr. Thomas mit beträchtlicher Überraschung.

„Was haben sie vor?“, fragte Padma.

„Ich habe nicht die leiseste Ahnung“, sagte Draco mit einem Grinsen von feinster Selbstgefälligkeit. „Wir werden einfach das Offensichtliche tun.„

Harry, der nun seinen Kessel fertiggestellt hatte, schöpfte vorsichtig Eicheln in den Behälter, während die Späher nach einer nahegelegenen Wasserquelle suchten, die als flüssige Basis verwendet werden konnte. Sie waren im Wald schon häufig auf Senkgruben und Miniaturbäche gestoßen, es sollte also nicht lange dauern. Ein anderer Späher hatte einen geraden Stock mitgebracht, der als Rührgerät dienen würde, so dass Harry keinen verwandeln musste.

Manchmal erkannte man, obwohl man gerade auf etwas schaute, nicht, worauf man schaute, bis man zufällig genau die richtige Frage stellte...

\emph{Wie kann ich magische Kräfte beschwören, die eigentlich außerhalb der Reichweite von Erstsemestern liegen sollten?}

Es gab eine warnende Geschichte, die der Meister der Zaubertränke ihnen erzählt hatte (mit viel Spott und Gelächter, um die Dummheit als unbedeutend erscheinen zu lassen, statt als gewagt und romantisch), über eine Hexe im zweiten Jahr in Beauxbatons, die ein paar extrem limitierte und teure Zutaten gestohlen hatte und versuchte, Vielsafttrank zu brauen, um sich die Gestalt eines anderen Mädchens für Zwecke auszuleihen, die besser unerwähnt blieben. Nur hatte sie es geschafft, den Trank mit \emph{Katzenhaaren} zu verunreinigen, und anstatt sofort einen Heiler aufzusuchen, hatte sich die Hexe in einem Badezimmer versteckt, in der Hoffnung, die Wirkung würde einfach nachlassen; und als sie schließlich gefunden wurde, war es zu spät gewesen, um die Verwandlung vollständig rückgängig zu machen, was sie zu einem Leben als Anthro-Katze verdammte.

Harry hatte nicht begriffen, was das \emph{bedeutete}, bis er die richtige Frage gestellt hatte - aber das bedeutete, dass ein junger Zauberer oder eine junge Hexe mit der Herstellung von Zaubertränken Dinge tun konnte, die sie mit Zaubern nicht einmal annähernd erreichen konnten. Vielsaft war einer der mächtigsten Tränke, die man kannte... aber was Vielsafttrank zu einem Zaubertrank auf N. E.W. T.-Niveau machte, war anscheinend nicht das erforderliche Alter, bevor man genug magische Kraft hatte; es war, wie schwierig der Trank genau zu brauen war und was mit einem passierte, wenn man es vermasselte.

Niemand in der Armee hatte bis dahin versucht, einen Zaubertrank zu brauen. Aber Professor Quirrell würde einem fast alles durchgehen lassen, wenn es etwas war, was man auch in einem echten Krieg hätte tun können. \emph{Schummeln ist Technik}, hatte der Verteidigungsprofessor ihnen einmal beigebracht. \emph{Oder besser gesagt, Schummeln ist das, was die Verlierer Technik nennen, und ist bei erfolgreicher Ausführung zusätzliche Quirrell-Punkte wert}. Im Prinzip war es nicht unrealistisch, ein paar Kessel zu verwandeln und Tränke aus allem zu brauen, was gerade zur Hand war, wenn man genug Zeit hatte, bevor die Armeen aufeinander trafen.

Also hatte Harry sein Exemplar von „Zaubertränke und Zauberbräue“ hervorgeholt und sich auf die Suche nach einem sicheren, aber nützlichen Trank gemacht, den er in den Minuten vor der Schlacht brauen konnte - ein Trank, der die Schlacht zu schnell für Gegenzauber gewinnen oder Zaubereffekte erzeugen würde, die für Erstklässler zu stark waren, um sie mit \emph{Finite} zu beenden.

Manchmal, obwohl man gerade auf etwas schaute, merkte man nicht, worauf man schaute, bis man zufällig genau die richtige Frage stellte...

\emph{Welchen Zaubertrank kann ich nur mit Komponenten brauen, die ich in einem gewöhnlichen Wald gesammelt habe?}

Jedes Rezept in \emph{Zaubertränke und Zauberbräue} verwendete mindestens eine Zutat aus einer magischen Pflanze oder einem magischen Tier. Was bedauerlich war, denn alle \emph{magischen} Pflanzen und Tiere befanden sich im Verbotenen Wald, nicht in den sichereren und kleineren Wäldern, in denen die Schlachten stattfanden.

Jemand anderes hätte an dieser Stelle vielleicht aufgegeben.

Harry blätterte von einem Rezept zum nächsten, überflog es immer schneller in der dämmernden Erkenntnis, die bestätigte, was er bereits gelesen hatte und nun zum ersten Mal \emph{sah}.

Jedes einzelne Zaubertränke-Rezept schien mindestens eine magische Zutat zu verlangen, \emph{aber warum sollte das so sein?}

Zaubersprüche erforderten überhaupt keine materiellen Komponenten; man sagte nur die Worte und schwenkte den Zauberstab. Harry hatte sich die Herstellung von Zaubertränken im Wesentlichen analog vorgestellt: Anstatt dass deine gesprochenen Silben ohne nachvollziehbaren Grund einen Zaubereffekt auslösten, sammelst du einen Haufen ekliger Zutaten und rührtestviermal im Uhrzeigersinn, und das löste künstlich einen Zaubereffekt aus.

Wenn man bedenkt, dass die meisten Zaubertränke gewöhnliche Komponenten wie Stachelschweinkiele oder gedünstete Schnecken verwenden, würde man erwarten, dass es einige Tränke gibt, die \emph{nur} gewöhnliche Komponenten verwenden.

Aber stattdessen verlangte jedes einzelne Rezept in \emph{Zaubertränke und Zauberbräue} mindestens \emph{eine} Komponente aus einer magischen Pflanze oder einem magischen Tier - eine Zutat wie Seide von einer Acromantula oder Blütenblätter von einer Venusfeuerfalle.

Manchmal, obwohl man gerade auf etwas schaute, wusste man nicht, worauf man schaute, bis man zufällig genau die richtige Frage stellte...

\emph{Wenn das Herstellen eines Zaubertranks wie das Wirken eines Zaubers ist, warum falle ich dann nicht vor Erschöpfung um, nachdem ich einen so mächtigen Trank wie den Heiltrank gegen Furunkel gebraut habe?}

Am vorletzten Freitag hatte Harrys doppelte Zaubertrankklasse einen \emph{Heiltrank gegen Furunkel} gebraut... obwohl selbst die trivialsten Heilzauber, wenn man sie mit Zauberstab und Beschwörungsformel zu wirken versuchte, mindestens Viertklässlerzauber waren. Und danach hatten sie sich alle so gefühlt, wie sie sich normalerweise nach dem Zaubertrankunterricht fühlten, nämlich \emph{nicht} nennenswert magisch erschöpft.

Harry hatte sein Exemplar von Zaubertränke und Zauberbräue mit einem Schnalzen zugeklappt und war hinunter in den Ravenclaw-Gemeinschaftsraum geeilt. Harry hatte einen Ravenclaw aus dem siebten Jahr gefunden, der seine N. E.W. T.-Hausaufgaben in Zaubertränke machte, und dem älteren Jungen einen Sickel bezahlt, um sich \emph{Höchst potente Zaubertränke} für fünf Minuten auslieh; denn Harry hatte nicht den ganzen Weg zur Bibliothek laufen wollen, um eine Bestätigung zu finden.

Nachdem er fünf Rezepte in dem Buch aus dem siebten Schuljahr überflogen hatte, hatte Harry das sechste Rezept gelesen, für einen \emph{Zaubertrank für Feueratem}, für den Aschenwinder-Eier benötigt wurden... und das Buch warnte, dass das entstehende Feuer nicht heißer sein konnte als das magische Feuer, das den Aschenwinder hervorgebracht hatte, der die Eier gelegt hatte.

Harry hatte „\emph{Heureka!}“ geschrien, mitten im Ravenclaw-Gemeinschaftsraum und war von einem in der Nähe befindlichen Vertrauensschüler heftig zurechtgewiesen worden, der dachte, Mr. Potter wolle einen Zauberspruch sprechen. Niemand in der Zaubererwelt wusste oder interessierte sich für einen alten Muggel namens Archimedes, noch für die Erkenntnis des Ur-Physikers, dass das aus einer Badewanne verdrängte Wasser dem Volumen des in die Badewanne eintretenden Objekts entspricht...

Erhaltungssätze. Sie waren die entscheidende Erkenntnis bei mehr Muggel-Entdeckungen, als Harry ohne weiteres zählen konnte. In der Muggeltechnologie konnte man keine Feder einen Meter vom Boden heben, ohne dass die Kraft von \emph{irgendwoher} kam. Wenn man geschmolzene Lava betrachtete, die aus einem Vulkan sprudelte, und fragte, woher die Hitze kam, würde ein Physiker einem von radioaktiven Schwermetallen im Zentrum des geschmolzenen Erdkerns erzählen. Wenn du fragtest, woher die Energie für die Radioaktivität kam, würde der Physiker auf eine Ära vor der Entstehung der Erde verweisen und auf eine Ur-Supernova in den frühen Tagen der Galaxie, die Atomkerne schwerer als das natürliche Limit erzeugt hatte, wobei die Supernova Protonen und Neutronen zu einem engen, instabilen Paket komprimierte, das einen Teil der Energie der Supernova zurückgab, als es zerbrach. Eine Glühbirne wurde durch Elektrizität angetrieben, erzeugt durch ein Kernkraftwerk, ermöglicht durch eine Supernova... Man könnte das Spiel bis zurück zum Urknall spielen.

Magie schien nicht so zu funktionieren, um es gelinde auszudrücken. Die Haltung der Magie gegenüber Gesetzen wie dem Energieerhaltungssatz lag irgendwo zwischen einem riesigen ausgestreckten Mittelfinger und einem Achselzucken der totalen Gleichgültigkeit. \emph{Aguamenti} erschuf Wasser aus dem Nichts, soweit man wusste; es gab keinen bekannten See, dessen Wasserspiegel jedes Mal sank. Das war ein einfacher Zauber aus dem fünften Jahr, der von Zauberern nicht als beeindruckend angesehen wurde, denn ein einfaches Glas Wasser zu erschaffen, erschien ihnen nicht erstaunlich. Sie hatten nicht die verrückte Vorstellung, dass Masse konserviert werden müsste oder dass das Erzeugen von einem Gramm Masse irgendwie gleichbedeutend mit dem Erzeugen von 90.000.000.000.000 Joule Energie war. Es gab einen Oberstufenzauber, auf den Harry gestoßen war, dessen \emph{wörtliche Beschwörungsformel „Arresto Momentum!„} lautete, und als Harry gefragt hatte, ob der Impulsirgendwo anders hinginge, hatte er nur einen verwirrten Blick bekommen. Harry hielt immer verzweifelter Ausschau nach einer Art Erhaltungsprinzip in der Magie, egal wo...

...und die ganze Zeit war es in jeder Zaubertränke-Stunde direkt vor seiner Nase gewesen. Die Herstellung von Zaubertränken \emph{erschuf} keine Magie, sie \emph{bewahrte} Magie, deshalb brauchte jeder Trank mindestens eine magische Zutat. Und wenn man Anweisungen wie „viermal gegen den Uhrzeigersinn und einmal im Uhrzeigersinn rühren“ befolgte - so hatte Harry vermutet - wirkte man so etwas wie einen kleinen Zauber, der die Magie in den Zutaten umformte. (Und die physische Form auflöste, so dass sich Zutaten wie Stachelschweinkiele sanft in eine trinkbare Flüssigkeit auflösten; Harry vermutete stark, dass ein Muggel, der genau dasselbe Rezept befolgte, am Ende nichts als ein stacheliges Durcheinander erhalten würde.) Das war es, was die Herstellung von Zaubertränken \emph{wirklich} ausmachte, die Kunst, vorhandene magische Essenzen zu transformieren. Deshalb war man nach dem Zaubertrank-Unterricht ein bisschen erschöpft, aber nicht viel, denn man hat die Tränke nicht selbst hergestellt, sondern nur die Magie umgestaltet, die bereits vorhanden war. Und das war der Grund, warum eine Hexe im zweiten Jahr Vielsafttrank brauen konnte, oder zumindest nahe drankam.

Harry hatte \emph{Höchst potente Zaubertränke} weiter durchgeblättert, auf der Suche nach etwas, das seine glänzende neue Theorie widerlegen könnte. Nach fünf Minuten hatte er dem älteren Jungen eine weitere Sichel zugeworfen (trotz seiner Proteste) und weitergemacht.

\emph{Der Trank der Riesenkraft} erforderte einen Re'em, um die pürierten Sumpfkrattler zu zertreten, die man in den Trank rührte. Das war seltsam, hatte Harry nach einem Moment festgestellt, denn zerquetschte Sumpfkrattler selbst waren nicht stark, sie waren nur... sehr, sehr zerquetscht, nachdem der Re'em mit ihnen fertig war.

Ein anderes Rezept besagte, man solle es „mit geschmiedeter Bronze berühren“, d. h. einen Knut in einer Zange halten, und damit über die Oberfläche des Trankes streichen; und wenn man den Knut ganz hineinfallen ließ, warnte das Buch, würde der Trank sofort überhitzen und den Kessel überkochen.

Harry starrte auf die Rezepte und ihre Warnungen und formte eine zweite und seltsamere Hypothese. Natürlich würde es nicht so einfach sein, einen Zaubertrank zu brauen, indem man magische Potenziale in den Zutaten nutzte, so wie Muggelautos durch das Verbrennungspotenzial von Benzin angetrieben werden. Magie würde niemals so vernünftig sein...

Und dann war Harry zu Professor Flitwick gegangen - da er sich Professor Snape nicht außerhalb des Unterrichts nähern wollte - und Harry hatte Professor Flitwick erzählt, dass er einen neuen Zaubertrank erfinden wollte, und er wusste, was die Zutaten sein sollten und was der Zaubertrank bewirken sollte, aber er wusste nicht, wie er das erforderliche Rührmuster ableiten konnte -

Nachdem Professor Flitwick aufgehört hatte, entsetzt zu schreien und in kleinen Kreisen zu rennen, und Professor McGonagall in das folgende heftige Verhör gerufen worden war, um Harry zu versprechen, dass es in diesem Fall sowohl akzeptabel als auch wichtig sei, dass er seine zugrundeliegende Theorie offenlege, hatte sich herausgestellt, dass Harry keine originelle magische Entdeckung gemacht hatte, sondern ein Gesetz wiederentdeckt hatte, das so alt war, dass niemand wusste, wer es zuerst formuliert hatte:

\emph{Ein Trank gibt das aus, was in die Herstellung seiner Zutaten investiert wird.}

Die Hitze der Koboldschmieden, die den bronzenen Knut gegossen hatte, die Kraft des Re'em, das die Sumpfkrattler zermalmt hatte, das magische Feuer, das den Aschenwinder hervorgebracht hatte: All diese Potenzen konnten durch den zauberähnlichen Prozess des Umrührens der Zutaten in exakten Mustern abgerufen, freigeschaltet und umstrukturiert werden.

(Vom Muggel-Standpunkt aus gesehen war es einfach nur \emph{seltsam}, eine gestörte Version der Thermodynamik, die von jemandem erfunden wurde, der dachte, das Leben sollte \emph{fair} sein. Aus Muggelsicht war die beim Schmieden des Knuts aufgewendete Wärme nicht in die Bronzeübergegangen, die Wärme hatte sich in die Umgebung verflüchtigt und war dauerhaft weniger verfügbar. Energie war konserviert, konnte weder erzeugt noch zerstört werden; die \emph{Entropie} nahm immer zu. Aber Zauberer dachten nicht so: Aus ihrer Sicht war es logisch, dass man, wenn man eine gewisse Menge an Arbeit in die Herstellung eines Knuts gesteckt hatte, genau die gleiche Arbeit wieder herausbekommen konnte. Harry hatte versucht zu erklären, warum das etwas seltsam klang, wenn man von Muggeln erzogen worden war, und Professor McGonagall hatte amüsiert gefragt, warum die Muggelperspektive besser sei als die der Zauberer).

Das Grundprinzip der Zaubertränke-Herstellung hatte keinen Namen und keine Standardformulierung, da man sonst in Versuchung geraten könnte, es aufzuschreiben.

Und jemand, der nicht klug genug war, das Prinzip selbst herauszufinden, könnte es lesen.

Und sie würden auf alle möglichen glänzenden Ideen kommen, um neue Tränke zu erfinden.

Und dann würden sie in Katzenmädchen verwandelt werden.

Es war Harry sehr deutlich gemacht worden, dass er diese besondere Entdeckung nicht mit Neville teilen würde, und auch nicht mit Hermine nach der nächsten Schlacht der Armeen. Harry hatte versucht, etwas darüber zu sagen, dass Hermine in letzter Zeit wirklich schlecht drauf zu sein schien und dass dies genau die Art von Sache war, die sie aufmuntern könnte. Professor McGonagall hatte ganz lapidar gesagt, dass er das nicht einmal denken dürfe, und Professor Flitwick hatte seine kleinen Hände gehoben und eine Geste gemacht, als würde er einen Zauberstab in zwei Hälften zerbrechen.

Trotzdem waren die beiden Professoren so freundlich gewesen, vorzuschlagen, dass Mr. Potter, wenn er zu wissen glaubte, was die Zutaten des Zaubertranks sein sollten, in der Lage sein könnte, ein bereits existierendes Rezept zu finden, das dasselbe tat; und Professor Flitwick hatte mehrere Bände in der Hogwarts-Bibliothek erwähnt, die nützlich sein könnten...

Der riesige pergamentartige Bildschirm zeigte jetzt nur noch eine Luftansicht des Waldes, aus der man gerade noch die getarnten Formen dreier Armeen erkennen konnte, die, aufgeteilt in je zwei Gruppen, zu ihrer Dreierschlacht zusammenkamen.

Die Bänke des Quidditch-Stadions füllten sich nun schnell mit der eher leicht gelangweilten Sorte von Zuschauern, die nur bei der letzten Schlacht dabei sein und alle langweiligen Punkte auf dem Weg dorthin auslassen wollten. (Wenn etwas an Professor Quirrells Schlachten nicht stimmte, sowar man sich weitgehend einig, dann, dass seine Spektakel nicht annähernd so lange dauerten wie Quidditchspiele, wenn sie erst einmal begonnen hatten. Darauf hatte Professor Quirrell nur geantwortet: „\emph{So funktioniert Realismus}“, und das war's.)

Durch das riesige Fenster - es war jetzt ein einziges Fenster, aus dem man aus großer Höhe beobachten konnte - kamen die vagen Ansammlungen winziger, getarnter Formen immer näher.

Näher.

Man konnte sie fast berühren --

Das riesige weiße Pergamentfenster zeigte den ersten Hauch der Schlacht zwischen Sonnenschein und Chaos, eine schreiende Masse von rennenden Kindern mit Smiley-Gesichtern auf der Brust, die mit hochgehaltenen\emph{Contego}-Schilden vorwärts stürmten und andere, die „\emph{Somnium}!“ schrien-

Bis einer von ihnen mit erschrockener Stimme schrie: „\emph{Prismatis!}“, und der ganze Trupp kam vor der funkelnden Wand aus Kraft, die vor ihnen aufgetaucht war, zum Stehen.

Tracey Davis war hinter den Bäumen hervorgetreten.

„Das stimmt“, sagte Tracey, ihre Stimme tief und grimmig, während sie ihren Zauberstab auf die Barriere richtete. „Ihr solltet mich fürchten. Denn ich bin Tracey Davis, die Dunkle Lady!„

(Amelia Bones, die Leiterin der Abteilung für magische Strafverfolgung, warf einen fragenden Blick auf Mr. und Mrs. Davis, die beide aussahen, als wären sie am liebsten auf der Stelle gestorben.)

Hinter der prismatischen Barriere fand eine Art gedämpfter Streit zwischen den Sonnenschein Soldaten statt, von denen insbesondere einer von einigen der anderen gescholten zu werden schien.

Dann, einen Moment später, zuckte \emph{Tracey} zusammen.

Susan Bones hatte sich an die Spitze des Sonnenschein Kontingents gestellt.

(„Meine Güte“, sagte Augusta Longbottom. „Was \emph{glauben} Sie, was Ihre Großnichte in Hogwarts gelernt hat?“)

(„Ich weiß es nicht“, sagte Amelia Bones ruhig, „aber ich werde ihr einen Schokoladenfrosch und Anweisungen geben, mehr davon zu lernen.“)

Die prismatische Barriere verschwand.

Die Sonnenschein Soldatenstürmten wieder vorwärts.

Tracey schrie, ihre Stimme hoch vor Anstrengung, „\emph{Inflammare!}“ und der Angriff der Sonnenschein-Soldaten kam zu einem weiteren plötzlichen Halt, als eine Feuerlinie zwischen ihnen im halbtrockenen Gras aufloderte und sich ausdehnte, um dem Weg von Traceys Zauberstab zu folgen, als sie ihnschwang; einen Augenblick später rief Susan Bones „\emph{Finite Incantatem!}“ und die Flammen verdunkelten sich, hellten sich auf, verdunkelten sich im Wettstreit ihres Willens, andere Soldaten hoben ihre Stäbe, um auf Tracey zu zielen; und das war der Moment, in dem Neville Longbottom schreiend aus dem Himmel stürzte.

Einer der Drachenkrieger, Raymond Arnold, machte ein Handzeichen, das nach vorne und schräg nach links zeigte; und es gab ein plötzliches gedämpftes Flüstern unter dem Kontingent der Drachenarmee, als sie sich alle leise in Richtung des Feindes neu orientierten. Die Sonnenwussten, dass sie dort waren, natürlich wussten es beide Armeen; aber irgendwie waren sie in diesem\\ Moment alle instinktiv still geworden.

Die Drachen krochen weiter vorwärts, und dann noch weiter, die dumpfen, getarnten Formen der Sonnen begannen zwischen den entfernten Bäumen aufzutauchen, und immer noch sprach niemand, niemand brüllte den Ruf zum Angriff.

Draco stand jetzt an der Spitze seiner Soldaten, Vincent hinter ihm und Padma nur einen Schatten weiter hinten; wenn die drei den Schock von Sonnenscheins Besten aushalten konnten, hatte der Rest der Drachenarmee vielleicht eine Chance.

Dann sah Draco eine Sonne, die ihn aus der Ferne anstarrte, in der Vorhut ihrer eigenen Armee; sie starrte ihn mit einem Blick voller Wut an -

Ihre Blicke trafen sich über das Waldschlachtfeld hinweg.

Draco hatte nur den Bruchteil einer Sekunde Zeit, sich im Hinterkopf zu fragen, worüber Hermine Granger so wütend war, bevor der Ruf von beiden Armeen ertönte; und sie stürmten alle zum Angriff.

Die anderen Chaoten waren nun zwischen den Bäumen aufgetaucht, einige hatten sich aus den Bäumen \emph{fallen lassen}, und der Kampf war nun in vollem Gange, jeder feuerte in jede Richtung auf alles, was wie ein Feind aussah. Außerdem schrien einige Sonnen „\emph{Luminos!}“ auf Neville Longbottom, während der Chaos-Hufflepuff sich drehte und durch die Luft schoss, auf Bahnen, die man nur als, in der Tat, „chaotisch“ beschreiben konnte-

Und es geschah, wie es nur ein einziges Mal von zwanzig im Scheingefecht in der Luft geschah, dass Neville Longbottoms Besenstiel unter seinen geballten Händen hellrot glühte.

Das hätte bedeuten müssen, dass Longbottom aus dem Spiel war.

Dann, auf der Hogwarts-Tribüne, inmitten der zuschauenden Schülerschar, ging ein Schreilos-

\emph{Realistischer Kampf}. Das war Professor Quirrells eine Hauptregel. Man konnte mit allem durchkommen, wenn es realistisch war, und im echten Leben verschwand ein Soldat nicht einfach, wenn sein \emph{Besenstiel} von einem Fluch getroffen wurde.

Neville stürzte zu Boden und schrie: „\emph{Chaotische Landung!}“ und die Chaoten rissen ihre Aufmerksamkeit von den Kämpfen los, um den Schwebezauber zu wirken (und gleichzeitig zu rennen, damit sie keine leichte Beute waren), wobei fast alle anderen stehen blieben, um zu gaffen -

Und Neville Longbottom knallte auf den laubbedeckten Waldboden, landete auf einem Knie, einem Fuß und beiden Händen, als würde er niederknien, um zum Ritter geschlagen zu werden.

Alles blieb stehen. Selbst Tracey und Susan hielten in ihrem Zweikampf inne.

Im Stadion verschwanden alle Publikumsgeräusche.

Es herrschte eine allgemeine Stille, die sich aus Erstaunen, Besorgnis und schier sprachlosem Staunen zusammensetzte, während alle darauf warteten, was als nächstes passieren würde.

Und dann stand Neville Longbottom langsam auf und richtete seinen Zauberstab auf die Sonnenschein Soldaten.

Obwohl es niemand auf dem Schlachtfeld hörte, hatte ein großer Teil des Publikums im Stadion begonnen, in stetig ansteigenden Tönen jedes Mal, wenn das Wort ausgesprochen wurde, „DOOM DOOM DOOM DOOM“ zu skandieren, weil man das einfach nicht sehen konnte, \emph{ohne} zu denken, dass es musikalische Begleitung brauchte.

„Die Menge jubelt Ihrem Enkel zu“, sagte Amelia Bones. Die alte Hexe warf einen prüfenden Blick auf den Bildschirm.

"Das tun sie „, sagte Augusta Longbottom. „Einige, wenn ich richtig gehört habe, jubeln: \emph{Unser Blut für Neville! Unsere Seelen für Neville!}"

„Durchaus“, sagte Amelia und nahm einen Schluck aus einer Teetasse, die kurz zuvor noch nicht da gewesen war. „Das zeigt, dass der Junge Führungspotenzial hat."

„Diese Jubelrufe“, fuhr Augusta fort, wobei ihre Stimme eine noch fassungslosere Qualität annahm, „scheinen von den Hufflepuff-Bänken zu kommen."

„Es ist das Haus der Loyalen, meine Liebe“, sagte Amelia.

"Albus Percival Wulfric Brian Dumbledore! \emph{Was in Merlins Namen ist in dieser Schule geschehen?}"

Lucius Malfoy betrachtete die Bildschirme mit einem ironischen Lächeln, seine Finger klopften in einem nicht erkennbaren Muster auf seine Armlehne. „Ich weiß nicht, was beängstigender ist, der Gedanke, dass ein verborgener Plan hinter all dem steckt, oder der Gedanke, dass er keinen Plan hat."

„Sehen Sie!“, rief Lord Greengrass. Der adrette junge Mann hatte sich halb aus seinem Stuhl erhoben und deutete mit dem Finger auf den Bildschirm. „Es geht los!"

„Wir schnappen ihn und zusammen“, flüsterte Daphne. Sie wusste, dass ein paar angsterfüllte Minuten echter Kampferfahrung, ein paar Mal in der Woche, nicht ausreichen würden, um Nevilles regelmäßiges Duelltraining mit Harry und Cedric Diggory im gleichen Zeitraum zu übertreffen. „Er ist zu viel für eine von uns, aber wir beide zusammen - ich benutze meinen Zauber, du versuchst einfach, ihn zu betäuben -„

Hannah neben ihr nickte, und dann schrien sie beide aus vollem Halse und stürmten vorwärts, wobei die Schwebezauber der beiden unterstützenden Sonnenschein Soldaten sie schneller und leichtfüßiger machten, wobei Daphne bereits „\emph{Tonare!}“ rief, während Hannah ein riesiges \emph{Contego}-Schild vor ihnen in Bewegung hielt, und mit einem kurzen Extra-Schubsprangen sie über die Köpfe der vorderen Soldaten und landeten vor Neville, wobei ihre Haare hoch um sie herum wogten -

(Fotografieren war bei allen Hogwarts-Spielen strengstens verboten, aber irgendwie landete dieser Moment trotzdem auf der Titelseite des \emph{Quibblers} vom nächsten Tag.)

- und im selben Moment, weil der Kampf gegen die älteren Schläger jede Spur von Zögern weggebrannt hatte, feuerte Hannah ihren ersten Schlafzauber auf Neville ab (sie hatte die Beschwörung begonnen, während sie noch in der Luft war), während Daphne, die sich mehr auf Geschwindigkeit als auf Kraft konzentrierte, mit ihrer Uralten Klinge dort zuschlug, wo sie dachte, dass Nevilles Oberschenkel sein würden,\emph{nachdem}er ausgewichen war.

Aber Neville sprang hoch, nicht zur Seite, er sprang höher, als er hätte springen können, so dass ihr glühendes Schwert nur die Luft unter seinen Füßen durchschnitt. Irgendwie erkannte Daphne, was es bedeutete, dass Neville immer noch durch andere Chaoten mit dem Schwebezauber unterstützt wurde, und zwar rechtzeitig, um ihre Klinge über ihren Kopf zu heben, aber Neville fiel zu schnell, und als seine Klinge auf ihre prallte, war es, als würde sie von einem Klatscher getroffen. Es warf Daphne von den Füßen und schleuderte sie rückwärts auf das Gras, wo sie hart mit dem Rücken aufschlug. Dann wäre es für sie vielleicht vorbei gewesen, wenn Neville nicht selbst zu hart gelandet wäre und mit einem schmerzhaften Keuchen in die Knie gegangen wäre. Und dann, bevor Neville seine glühende Klinge zu Boden bringen konnte, schrie Hannah „\emph{Somnium!}“ und Neville taumelte hektisch nach hinten - obwohl natürlich kein Zauberspruch von Hannahs Zauberstab gekommen war, so schnell konnte das Hufflepuff-Mädchen gar nicht wieder feuern -, was Daphne eine Sekunde Zeit gab, auf die Füße zu krabbeln und wieder beide Hände um ihren Zauberstab zu bekommen--

„Lieber Merlin“, sagte Lady Greengrass. Ihre Stimme wirkte unsicher, die aristokratische Haltung war durchlöchert. „Meine Tochter kämpft mit dem Zauber der Uralten Klinge. In ihrem ersten Jahr. Ich wusste nicht, dass sie ein so außerordentliches Talent besitzt..."

„Ausgezeichnetes Blut“, sagte Charles Nott anerkennend, was Augusta zu einem Schnauben veranlasste.

„Meine werte Dame“, sagte Professor Quirrell und klang ernst. „Tun Sie Ihrer Tochter nicht so unrecht. Das ist nicht nur Talent, was Sie da sehen.“ Seine Stimme wurde ein wenig trockener. „Vielmehr ist es das, was passiert, wenn Kinder ihre wetteifernden Anstrengungen in ein Spiel stecken, bei dem es tatsächlich um Zauberei geht."

„\emph{Expelliarmus!}“ rief Draco und versuchte, seine Stimme ruhig zu halten, während er gleichzeitig dem feuerroten Blitz auswich, den Hermine Granger auf ihn abgefeuert hatte. Seine Muskeln verdrehten sich, weil er in die falsche Richtung ausweichen musste - sie hatte auf seine linke Seite gezielt, und dann mit einem überraschenden Zucken ihres Zauberstabs nach rechts geschossen -

Hermine wich dem sich schnell bewegenden Duellzauber und rief fast ohne Verzögerung: „\emph{Steleus} “, ein Weitwinkelzauber, dem Draco nicht ausweichen konnte, aber es gelang ihm, seinen Zauberstab auf sein eigenes Gesicht zu richten und „\emph{Quiescus!}“ zu rufen, bevor sich der plötzliche Drang zum Einatmen in einen Niesanfall verwandeln konnte, der die Schlacht beendet hätte.

Draco Malfoy war schon halb erschöpft von all den Sperrzaubern und Verwandlungssprüchen vorhin, aber seine Verwirrung begann dem Gefühl seines eigenen kochendes Blutes zu weichen, er wusste nicht, warum Granger ihn plötzlich so wütend angriff, aber \emph{wenn sie einen Kampf wollte, würde er ihr einen geben} -

(Die Drachen und Sonnen hielten nicht inne, um dem Duell ihrer Generäle zuzuschauen, dafür waren sie zu diszipliniert, und das bedeutete, dass auch Hermines Armee weiterkämpfen musste; aber das gespannte Publikum auf den Hogwarts Quidditch-Tribünenwurde damit sogar von Neville und Daphne abgelenkt und verlagerte seinen Blick auf das Duell zweier Generäle, als Malfoy und Granger Zauber um Zauber und Fluch um Fluch aufeinander feuerten und dabei schneller handelten, als es jeder andere Schüler in ihrem Jahrgangschaffen könnte, der trainierte Duell-Tanz des Erbens der Malfoys mit der rasenden Energie des akademischen Stars von Hogwarts. Der Kampf der beiden begann einem Erwachsenen-Duell zu ähneln, während die beiden magisch mächtigsten Erstklässler auf Zaubersprüche zurückgriffen, die exotischer waren als der übliche Schlafzauber.)

- obwohl Draco allmählich erkannte, dass er und Harry Miss Granger falsch eingeschätzt hatten, als sie sie als so gefährlich wie eine Schüssel nasser Trauben dargestellt hatten. Der Grund dafür war, dass sie sie noch nie \emph{wütend} gesehen hatten.

Daphne schlug mit ihrer Uralten Klinge zu, wobei sie wieder nicht versuchte, hart zuzuschlagen, sondern die Klinge nur so schnell wie möglich bewegte, während Hannah gleichzeitig „Somnium!“ rief und Neville sprang wieder zurück, aber es war ein weiterer Bluff gewesen und Hannah bewegte sich auf ihn zu, um einen echten Zauber fast aus nächster Nähe abzufeuern -

- und Neville Longbottom tat genau das, was - würde er nachher erklären - Cedric Diggory ihm beigebracht hatte, wenn er gegen Bellatrix Black kämpfte, nämlich sich herumzudrehen und Hannah einen \emph{wirklich kräftigen} Tritt in die Magengrube zu verpassen.

Das Hufflepuff-Mädchen gab einen traurigen Laut von sich, einen keuchenden Schmerzensschrei, als sie von den Füßen gestoßen wurde, weil der harte Schuh mit der Wucht von Nevilles ganzem Körper in ihren Unterleib eindrang.

Einen Augenblick lang stand das Schlachtfeld still, alles hielt inne, außer Hannahs fallender Gestalt.

Dann verzog sich Nevilles Gesicht zu absoluter Bestürzung und er ließ seinen Zauberstab sinken, wobei der Chaotische Leutnant instinktiv auf seine Hausgenossin zuging, während er mit der anderen Hand nach ihr griff.

Selbst als Hannah ihren Sturz in eine Rolle verwandelte und mit erhobenem Zauberstab auf ihn zustürmte.

Einen Sekundenbruchteil später versenkte Daphne, die ebenfalls nicht gezögert hatte, ihre Uralte Klinge direkt in Nevilles Rücken, wodurch die Muskeln des chaotischen Leutnants durch die betäubende Magie, die sich in ihm entlud, krampfhaft zuckten, selbst als Hannahs Schlafzauber seine Wirkung zeigte, und dann lag der letzte Spross der Longbottoms ruhig auf dem Boden, mit\\ einem Ausdruck völliger Überraschung im Gesicht.

„Heute hat Mr. Longbottom eine wertvolle Lektion über seine Gefühle von Mitleid und Reue gelernt“, sagte Professor Quirrell.

„Und über Ritterlichkeit“, sagte Amelia und nippte wieder an ihrem Tee.

„Geht es dir gut?“, flüsterte Daphne, als sie schützend über der Stelle stand, an der Hannah auf dem Boden lag und sich den Bauch hielt. Das Mädchen erwiderte nichts, außer weiteren Würgegeräuschen, die sich anhörten, als würde Hannah versuchen, sich nicht zu übergeben, während sie versuchte, nicht zu weinen.

Irgendwie, auch wenn es vielleicht keine gute Taktik war - es wäre besser gewesen, wenn Hannah direkt verhext worden wäre, als dass andere Soldaten damit beschäftigt wurden, sie zu beschützen - schienen einige Sonnen mit ihren Zauberstäben fest umklammert vor Hannah zu stehen und starrten die Chaoten wütend an. Jemand hatte eine prismatische Barriere zwischen den beiden Gruppen errichtet, Daphne konnte nicht sehen, wer.

Und aus irgendeinem Grund schienen die Chaoten den Angriff nicht fortzuführen. Sogar Tracey hatte ihren grimmigen Gesichtsausdruck völlig abgelegt und verlagerte ihr Gewicht nervös von einem Fuß auf den anderen, als hätte sie Schwierigkeiten, sich zu erinnern, auf welcher Seite sie stand.

„\emph{Halt!}“, rief eine Stimme. „\emph{Stoppt den Kampf!}"

Es war sowieso nicht viel Kampf im Gange, aber es stoppte.

General Potter, jeder Zoll der Junge, der lebte, trat aus den Bäumen hervor und hielt etwas Großes unter einer Tarnabdeckung in einem Arm.

„Atmet Miss Abbott noch normal?“ schrie General Potter.

Daphne drehte sich nicht um. Sie vertraute nicht darauf, dass es sich nicht um eine Falle handelte - es war absolut sicher, dass, wenn die Chaoten die Gelegenheit zum Angriff nutzten, Professor Quirrell es nicht nur für legal erklären, sondern ihnen hinterher auch Extrapunkte geben würde. Aber Daphne konnte die Antwort mit ihren Ohren gut genug hören, es war ja nicht so, als würde Hannah versuchen, leise zu atmen, und so sagte sie: „Irgendwie schon."

„Sie sollte hier raus und zu jemandem gehen, der Heilzauber anwenden kann“, sagte Harry. „Nur für den Fall, dass sie sich etwas gebrochen hat."

Von hinter Daphne sagte eine kleine keuchende Stimme: „Ich - kann - noch - kämpfen -"

„Miss Abbott, nicht -“ sagte Harry, gerade als hinter Daphne das Geräusch von jemandem ertönte, der auf den Rasen zurücksank, nachdem er versucht hatte, auf die Beine zu kommen, und es nicht geschafft hatte. Alle zuckten zusammen, aber Daphne drehte Harry nicht den Rücken zu.

„Warum haben die Lehrer den Kampf nicht beendet?“, fragte Susan mit wütender Stimme.

„Ich nehme an, weil Miss Abbott nicht in Gefahr ist, bleibende Schäden zu erleiden und Professor Quirrell denkt, dass wir wertvolle Lektionen lernen“, sagte Harry mit harter Stimme. „Hören Sie, Miss Abbott, wenn Sie gehen, wird sich auch Tracey aus dem Kampf zurückziehen. Sie sind bereits in der Überzahl, also ist das ein sehr guter Deal für Ihre Seite. Bitte nehmen Sie es an."

„Hannah, geh einfach!“, sagte Daphne. „Ich meine, sag einfach, dass du raus bist!"

Als Daphne einen Blick zurückwarf, sah sie, dass Hannah den Kopf schüttelte, immer noch zu einem Ball zusammengerollt im Gras lag.

„Ach, was soll's“, sagte Harry. „\emph{Chaoten! Je schneller wir sie betäuben, desto schneller ist sie hier raus! Wir werden das sehr schnell erledigen, auch wenn wir Verluste in Kauf nehmen! Waffenstillstand beenden! TUNFISCH!} "

Im Hinterkopf konnte Daphne einen Augenblick bewundern, wie Harrys wenige Worte die Chaoten politisch gerade zu den \emph{Guten} gemacht hatten, und dann tauchten die Chaoten in fast perfektem Gleichklang ihre Hände in die Taschen ihrer Uniformen und zogen grüne Sonnenbrillen in einem ungewohnten Stil hervor. Nicht wie etwas, das man am Strand tragen würde, eher wie eine Brille für fortgeschrittene Zaubertränke -

Dann erkannte Daphne, was gleich passieren würde, und hob die andere Hand, um ihre Augen zu schützen, gerade als Harry das Tuch vom Kessel riss.\\ Die Flüssigkeit, die sich ergoss, als Harry Potter den Inhalt des Kessels in die Luft schleuderte, war zu hell, um sie zu sehen, zu brillant, um sie sich vorstellen zu können, glühend wie die Sonne in dutzendfacher Vergrößerung -

(was genau das war, was es war)

(das Sonnenlicht, das investiert worden war, um die Eicheln zu erschaffen, die helle Energie, die einen Baum aus der kahlen Erde emporwachsen ließ)

(es leuchtete in einem glühenden Violett, der Farbe der gemischten blauen und roten Wellenlängen, die das Chlorophyll absorbierte)

(mit fast keinem der grünen Wellenlängen, die Chlorophyll reflektiert, um die grüne Farbe der Blätter zu erzeugen)

(was die Farbe der Sonnenbrille der Chaos-Legion war, die hergestellt wurde, um grüne Wellenlängen durchzulassen, Rot und Blau zu blockieren und selbst das glühendste violette Blenden auf etwas Erträgliches zu reduzieren)

- das violette Licht loderte weiter und weiter, Daphne versuchte, ihren Arm von den Augen zu nehmen, aber sie stellte fest, dass sie nirgends direkt hinsehen konnte, selbst das sekundäre violette Blendlicht war so hell, dass sie blinzeln musste; und sie hatte nur Zeit, ein \emph{Finite Incantatem} zu rufen, das nicht funktionierte, bevor ein Schlafzauber sie erwischte.

Was von dem Kampfübrig blieb, dauerte danach nicht mehr sehr lange.

„JETZT!“, brüllte Blaise Zabini, ehemals Sonnenschein, jetzt Kommandeur eines Trupps von Chaoslegionären. „Ich meine, TUNFISCH!“ Die Hand des Slytherin-Jungen griff nach dem Tuch, das den Kessel vor der auslösenden Berührung des Tageslichts abschirmte, und begann bereits, es zur Seite zu schieben.

„JETZT!“, brüllte Dean Thomas, ehemals Chaos, und kommandierte eine Abordnung von Drachenkriegern. „TUT, WAS SIE TUN!"

Die Chaoten von Zabinis Truppe steckten die Hände in ihre Uniformtaschen und kamen mit grünen Sonnenbrillen heraus -

- Eine Aktion, die von Dean und den Drachenkriegern fast perfekt gespiegelt wurde, die grün gefärbte Zaubertrankbrillen hervorholten und schnell die Bänder über ihre eigenen Köpfe zogen, selbst als die Chaoten ihre Sonnenbrillen aufsetzten und die violette Glut heraussprühte.

(Wie General Malfoy erklärt hatte, musste man nicht wissen, warum die Chaos-Legion grün gefärbte Zaubertrankbrillen trug, um einige Exemplare zu verwandeln, wenn Mr. Goyle davon berichtete).

„DAS IST SCHUMMELN!“, kreischte Blaise Zabini.

"DAS IST TECHNIK!"brüllte Dean zurück. „DRACHEN, ANGRIFF!„

(„Verzeihung“, sagte die Lady Greengrass. „Könnten Sie aufhören, so zu lachen, Mr. Quirrell? Es ist nervtötend.“)

„ZAUBERT FINITE AUF IHRE BRILLEN!“, schrie Blaise Zabini, als die beiden Armeen durch das allgegenwärtige, augenbetäubende violette grelle Licht kopflosaufeinander zu rannten. „WIR KÖNNEN NOCH GEWINNEN!"

„IHR HABT IHN GEHÖRT!“ brüllte Dean. „AUF IHREBRILLEN!"

Blaise Zabinis Antwort darauf war nicht gerade wortgewandt.

Dieser Kampf dauerte noch viel länger.

„\emph{Stupor!}“ schrie die Sonnenschein Generalin.

Draco wich nicht aus, er konterte nicht, er hatte für beides nicht genug Energie übrig, er konnte nur seine linke Hand in Position peitschen und hoffen -

Der rote Betäubungsbolzen löste sich wieder an Dracos mit \emph{Colloportus} verzauberten Handschuh auf, den er verwandelt und mit einem Sperrzauber an seiner Hand befestigt hatte, genau wie beim Rest der Drachenarmee. Das war alles, was ihn jetzt noch rettete, dieser Schild.

Es hätte Zeit für einen Gegenangriff sein sollen, aber Draco konnte nur noch nach Luft schnappen. Ihm gegenüber keuchte General Granger heftig, auf dem Gesicht des jungen Mädchens glitzerten Schweißperlen, das kastanienbraune Haar klebte an ihrem Gesicht. Ihre Tarnuniform zeigte Schweißflecken, ihre Schultern zitterten sichtlich vor Erschöpfung, aber ihr Zauberstab war immer noch beständig auf ihn gerichtet. Ihre Augen blitzten, ihre Wangen waren vor Wut gerötet.

\emph{Na, kleines Mädchen, warum tust du heute so, als würdest du wie eine Erwachsene kämpfen?}

Der Spruch kam ihm in den Sinn, aber er wollte Granger wirklich nicht noch wütender machen. Stattdessen sagte Draco einfach - obwohl seine Stimme krächzte - „Gibt es einen Grund, warum du wütend auf mich bist, Granger?

Das Mädchen schnappte selbst nach Luft, ihre eigene Stimme war unstetig, als sie sprach. „Ich weiß, was du vorhast“, sagte Hermine Granger und ihre Stimme hob sich. „Ich weiß, was du und Snape vorhabt, Malfoy, und ich weiß, wer dahinter steckt!"

„Hä?“ sagte Draco, ohne auch nur einen Gedanken daran zu verschwenden.

Das schien Grangers Wut nur noch zu verstärken, und ihre Finger wurden weiß um den Zauberstab, den sie auf ihn gerichtet hielt.

Und dann verstand Draco es, und sein eigenes Blut kochte in seinen Adern. Sogar \emph{sie} dachte, dass er heimlich ein Komplott gegen sie schmiedete -

„\emph{Du auch?}“ schrie Draco. „\emph{Ich habe dir geholfen, du Weibsstück, mit deiner hässlichen Visage! Du, du, du}“ - stotterte an all den Dunklen Flüchen vorbei, die ihm in den Sinn kamen, bis er etwas fand, das er tatsächlich anwenden konnte - „\emph{DENSAUGEO!}„

Aber Granger sprang aus der Flugbahn des Zahnverlängerungsfluchs, und dann hob sie ihren Zauberstab aus allernächster Nähe, selbst als Draco seinen verzauberten Handschuh zwischen ihn und sie und was immer sie schießen würde, hielt. Die Stimme der Sonnenschein Generalin hob sich zu einem Schrei über das gesamte Schlachtfeld -

„\emph{ALOHOMORA!}“

Die Zeit hätte still stehen sollen.

Tat sie aber nicht.

Stattdessen zerbarst das Schloss und der Handschuh fiel ab.

Einfach so.

Einfach so.

Die Bildschirme zeigten es dem gesamten Hogwarts Stadium sehr deutlich.

Und die Totenstille, die über jede Bank in jeder Tribüne fiel, verdeutlichte, dass jeder ganz klar verstand, was es bedeutete, wenn der Spross des Hauses Malfoy gerade von einer Muggelstämmigen bezwungen worden war.

Hermine Granger hielt nicht inne, gab kein Zeichen, dass sie überhaupt wusste, was sie getan hatte; stattdessen kickte sie mit einem Tritt Draco den Zauberstab sauber aus der Hand, während sein Verstand und sein Körper sich viel zu langsam bewegten. Draco tauchte hinter seinem Zauberstab her und krabbelte verzweifelt auf dem Boden herum, doch hinter ihm ertönte die laute Stimme eines Mädchens. „\emph{Somnium!}“, erklang hinter ihm und Draco Malfoy fiel und erhob sich nicht wieder.

Ein weiterer Moment eisiger Stille. Hermine stand etwas wackelig, als würde sie fast in Ohnmacht fallen.

Dann schrien die Drachensoldaten aus Leibeskräften und stürzten sich nach vorne, um ihren gefallenen Kommandanten zu rächen.

Mr. und Mrs. Davis zitterten, als sie von den bequemen Stühlen der Quidditch-Loge der Fakultät aufstanden; sie konnten sich beim Gehen nicht ganz festhalten, aber sie hielten sich fest an den Händen und gaben sich Mühe, unsichtbar zu sein. Wären sie Kinder gewesen, die jung genug für versehentliche Zauberei waren, hätten sie sich wahrscheinlich spontan selbst desillusioniert.

Der ältere Charles Nott sagte nichts, als er von seinem Stuhl aufstand. Der vernarbte Lord Jugson sagte nichts, als er von seinem eigenen Stuhl aufstand.

Lucius Malfoy sagte nichts, als er aufstand.

Alle drei drehten sich ohne Pause um und schritten auf die Treppe der erhöhten Tribüne zu, sie bewegten sich in unheimlichem Gleichklang wie ein Auroren-Trio -

„Lord Malfoy“, sagte der Verteidigungsprofessor in mildem Ton. Der Mann saß immer noch in seinem eigenen Stuhl und blickte auf seine pergamentartigen Bildschirme, die Arme schlaff an der Seite, als ob er aus irgendeinem Grund keine Lust hätte, sich zu bewegen.

Der weißhaarige Mann blieb kurz vor dem Ausgangtor stehen, und der ältere Mann und der vernarbte Mann blieben ebenfalls stehen und flankierten ihn. Lord Malfoys Kopf drehte sich, zu wenig, um eine Form der Zurkenntnisnahmezu sein, aber in die Richtung des Verteidigungsprofessors.

„Ihr Sohn hat sich heute außergewöhnlich gut geschlagen“, sagte Professor Quirrell. „Ich muss gestehen, dass ich ihn unterschätzt habe. Und er hat sich die Loyalität seiner Armee verdient, wie Sie gesehen haben.“ Die Stimme des Verteidigungsprofessors war immer noch sehr mild. „Als Lehrer Ihres Sohnes bin ich der Meinung, dass es ihm nicht gut tun wird, wenn Sie sich in seine -"

Lord Malfoy und seine Mitstreiter verschwanden die Treppe hinunter.

„Ein guter Versuch, Quirinus“, sagte Dumbledore leise. Auf dem Gesicht des alten Zauberers zeichneten sich kleine Sorgenfalten ab; auch er hatte sich nicht von seinem Platz erhoben und starrte auf die Pergamentschirme, als wären sie noch aktiv. „Glaubst du, er wird zuhören?"

Die Schultern des Verteidigungsprofessors zuckten leicht, die einzige Bewegung, die sie seit dem Ende des Kampfes gezeigt hatten.

„\emph{Nun}“, sagte Lady Greengrass, als sie sich erhob, mit den Fingerknöcheln knackte und sich streckte, ihr Mann schwieg neben ihr. „Ich muss sagen, das war ziemlich... interessant...„

Amelia Bones hatte sich ohne Umschweife von ihrem eigenen gepolsterten Sitz erhoben. „In der Tat interessant“, sagte Direktor Bones. „Ich muss gestehen, dass mich die Geschicklichkeit, mit der diese Kinder gegeneinander gekämpft haben, beunruhigt."

„Die Geschicklichkeit?“ sagte Lord Greengrass. „Ihre Zaubersprüche schienen mir nicht sonderlich beeindruckend zu sein. Abgesehen von dem von Daphne natürlich.„

Die alte Hexe rührte ihre Augen nicht von der Stelle, wo sie den kahlen Kopf des Verteidigungsprofessors anstarrte. „Der Betäubungsfluch ist kein Erstklässlerzauber, Lady Greengrass, aber das ist nicht die Fähigkeit, die ich im Sinn hatte. Sie haben sich mit diesen einfachen Zaubern gegenseitig unterstützt, sie haben schnell auf Überraschungen reagiert...“ Die Direktorin der AMS hielt inne, als ob er nach Worten suchte, die ein einfacher Zivilist verstehen konnte. „Mitten im Kampf“, sagte sie schließlich, „mit Zaubersprüchen, die in alle Richtungen flogen... schienen diese Kinder ganz zu Hause zu sein."

„In der Tat, Direktor Bones“, sagte der Verteidigungsprofessor. „Manche Künste werden am besten in der Jugend begonnen."

Die Augen der alten Hexe verengten sich. „Sie bereiten sie darauf vor, eine militärische Streitmacht zu werden, Professor. Zu welchem Zweck?"

„Jetzt warten Sie mal!“, warf Lord Greengrass ein. „Es gibt viele Schulen, in denen sie im ersten Jahr Duellieren lehren!"

„Duellieren?“, fragte der Verteidigungsprofessor. Von hinten war nicht zu erkennen, ob das blasse Gesicht lächelte. „Das ist nichts, Lord Greengrass, im Vergleich zu dem, was meine Schüler gelernt haben. Sie haben gelernt, im Angesicht von Hinterhalten und größeren Gegnern nicht zu zögern. Sie haben gelernt, sich anzupassen, wenn sich die Kampfbedingungen ändern und wieder ändern. Sie haben gelernt, ihre Verbündeten zu beschützen, diejenigen, die wertvoller sind, mehr zu schützen, Stücke aufzugeben, die nicht gerettet werden können. Sie haben gelernt, dass sie Befehle befolgen müssen, um zu überleben. Einige haben sogar ein wenig Kreativität gelernt. Oh nein, Lord Greengrass, diese Zauberer werden sich nicht in ihren Herrenhäusern verstecken und darauf warten, beschützt zu werden, wenn die nächste Bedrohung kommt. Sie werden wissen, dass sie wissen, wie man kämpft.„

Augusta Longbottom klatschte lautstark dreimal in die Hände.

\emph{Wir haben gewonnen.}

Es war das erste, was Draco hörte, als er auf dem Schlachtfeld aufwachte und Padma ihm erzählte, wie sich seine Soldaten nach seinem Kampf mit Granger gesammelt hatten. Wie Mr. Thomas dank der Weitsicht ihres Drachengenerals seine Truppe zum Sieg über Chaosgeführten hatte.\\ Wie General Potter den Teil des SonnenscheinRegiments besiegt hatte, der mit ihm zusammenstieß. Wie Mr. Thomas' Drachenkrieger mit ihren eigenen Schutzbrillen und den Sonnenbrillen der besiegten Chaoten wieder zum Hauptteil der Soldaten stießen. Wie nur Augenblicke später General Potters verbliebenes Kontingent die beiden anderen Armeen mit einem Trank angegriffen hatte, der sengendes violettes Licht ausstrahlte. Aber Dragon hatte sowohl gegenüber Sonnenschein als auch gegenüber Chaos den zahlenmäßigen Vorteil und genügend Sonnenbrillen für ihre Krieger gehabt; und so war es Padma gelungen, ihre geerbte Armee zum Sieg zuführen.

Padmas strahlenden Augen und ihrem arroganten Lächeln, das einem Malfoy alle Ehre gemacht hätte, nach zur urteilen, erwartete sie eine Beglückwünschung. Draco schaffte es, eine Art Lob zwischen seinen zusammengebissenen Zähnen herauszustoßen, und hätte hinterher nicht sagen können, was es war. Die ausländlische Hexe hatte anscheinend keine Ahnung, was passiert war oder was es bedeutete.

\emph{Ich} habe verloren.

Seine Armee stapfte unter dem grauen Himmel zurück nach Hogwarts, kalte Tropfen landeten schwer auf Dracos Haut, einer nach dem anderen. Während er bewusstlos war, hatte der lange versprochene Regen begonnen. Jetzt gab es für Draco nur noch eine Möglichkeit. Ein erzwungener Zug, wie Mr. Mac Nair, der Draco das Schachspiel beigebracht hatte, es genannt hätte. Harry Potter würde es wahrscheinlich nicht gefallen, zumindest wenn er wirklich in Granger verliebt war, wie alle sagten. Aber ein erzwungener Zug, wie Mr. Mac Nair ihn definiert hatte, war ein Zug, den man machen musste, wenn man die Partie überhaupt fortsetzen wollte.

Es spielte es in seinem Kopf immer wieder durch, während er wie ferngesteuert durch das massive Eingangsportal von Hogwarts ging, Vincent und Gregory mit zwei scharfen Worten wegschickte und sich in sein privates Schlafzimmer zurückzog und die Wand anstarrte. Es war, als ob ein Dementor, ihn seine Erinnerung immer wieder durchleben ließ.

Wie das Schloss an seinem Handschuh auf klickte und der Handschuh abfiel -

Draco wusste, er \emph{wusste}, was er falsch gemacht hatte. Er war so müde gewesen, nach dem er sieben und zwanzig Schlosszauber für all die anderen Drachenkrieger gezaubert hatte. Weniger als eine Minute war zu wenig, um sich nach jedem Spruch zu erholen. Also hatte er einfach \emph{nur} Colloportus auf seinen eigenen Handschuh gezaubert, \emph{nur} den Spruch gezaubert, nicht seine ganze Kraft hineingetan, um ihn stärker zu machen als Harry Potter oder Hermine Granger aufheben könnten.

Aber niemand würde das glauben, selbst wenn es wahr war. Selbst in Slytherin würde das niemand glauben. Es klang wie eine Ausrede, und eine Ausrede war alles, was sie hören würden.

\emph{Granger wirbelte herum und schrie 'ALOHOMORA!'}-

Dracos Verstand spielte es immer wieder ab, während sich der Groll aufbaute. Er hatte Granger geholfen - mit ihr bei der Verbannung von Verrätern zusammengearbeitet - ihre Hand gehalten, als sie vom Dach baumelte - einen Aufstand um sie herum in der Großen Halle verhindert - hatte sie überhaupt eine Ahnung, was er riskiert hatte, was er wahrscheinlich schon\emph{verloren}hatte, was es für den Erben des Hauses Malfoy bedeutete, das für ein \emph{Schlammblut} zu tun -

Und jetzt gab es nur noch einen Zug, und die Sache mit einem erzwungenen Zug war, dass man ihn machen\emph{musste}, auch wenn es bedeutete, Nachsitzen zu bekommen und Hauspunkte zu verlieren. Professor Snape würde es wissen und verstehen, aber es gab Grenzen (Vater hatte ihn gewarnt) indem, was der Zaubertranklehrer übersehen würde.

Er musste Granger zu einem Zaubererduell herausfordern, in offener Missachtung der Hogwarts-Vorschriften. Sie direkt angreifen, falls sie versuchen sollte, sich zu weigern. Sie in einem Zweikampf besiegen, in aller Öffentlichkeit, nicht durch geschickte Duelltechnik, sondern durch \emph{Überwältigung} ihrer Magie. Sie vollständig schlagen, sie so gründlich zerquetschen, wie der Dunkle Lord selbst seine Feinde zerquetscht hatte. Allen unmissverständlich klar machen, dass Draco einfach nur erschöpft war, weil er den Zauber so oft gewirkt hatte, und dass das Blut der Malfoys stärker war als das jedes Schlammblutes -

\emph{Nur stimmte das nicht}, flüsterte Harry Potters Stimme in Dracos Kopf. \emph{Es ist leicht zu vergessen, was wirklich wahr ist, Draco, wenn man erst einmal versucht, in der Politik zu gewinnen. Aber in Wirklichkeit gibt es nur eine Sache, die dich zu einem Zauberer macht. Erinnerst du dich?}

Draco kannte den Grund für die Beunruhigung im Hinterkopf, als er auf die leere Wand über seinem Schreibtisch starrte und über seinen erzwungenen Schritt nachdachte. Es hätte einfach sein sollen - wenn man nur einen Zug hatte, musste man ihn machen - aber -

\emph{Granger wirbelte drehte sich, schweißgetränkte Haare, die um sie herumflogen, Blitze, die von ihrem Zauberstab so schnell wie sein eigener flogen, Fluch und Gegenfluch, glühende Fledermäuse, die auf sein Gesicht zuflogen, und durch all das der wütende Blick in Grangers Augen}-

Es gab einen Teil von ihm, der das bewunderte, bevor alles schief ging, Grangers Wut und Macht bewundert hatte; ein Teil von ihm, beim ersten wirklichen Kampf, den er je geführt hatte, gegen...

...einen ebenbürtigen Gegner.

Wenn er Granger herausforderte und \emph{verlor}...

Es sollte nicht möglich sein, Draco hatte seinen Zauberstab zwei volle Jahre vor allen anderen in seinem Jahrgang bekommen.

Es gab nur einen Grund, warum sie sich normalerweise nicht die Mühe machten, Neunjährigen Zauberstäbe zu geben. Auch das Alter zählte, es ging nicht nur darum, wie lange man einen Zauberstab verwendet hatte. Grangers Geburtstag war nur wenige Tage nach Jahresbeginn gewesen, als Harry ihr diesen Beutel gekauft hatte. Das bedeutete, dass sie jetzt zwölf Jahre alt war, dass sie fast seit Beginn des Schuljahrs zwölf Jahre alt war. Und die Wahrheit war, dass Draco außerhalb des Unterrichts nicht viel geübt hatte, wahrscheinlich nicht annähernd so viel wie Hermine Granger von Ravenclaw. Draco hatte nicht gedacht, dass er mehr Übung bräuchte, um der Beste zu bleiben...

\emph{Und Granger war auch erschöpft}, flüsterte die Stimme der Gegenbeweise in ihm. Granger muss von all diesen Betäubungszaubernebenfalls erschöpft gewesen sein, und selbst in diesem Zustand war sie in der Lage gewesen, seinen Schutzzauber zu durchdringen.

Und Draco konnte es sich \emph{nicht leisten}, Granger öffentlich herauszufordern, eins-zu-eins ohne Ausreden, und zu verlieren.

Draco wusste, was man in einer solchen Situation tun sollte. Er sollte schummeln. Aber wenn jemand das entdeckte wäre es katastrophal, perfektes Erpressungsmaterial, selbst wenn es nie öffentlich bekannt würde, und alle zuschauenden Slytherins \emph{wussten} das, sie würden danach \emph{suchen} …

Und dann, konnte man beobachten, wie Draco Malfoy aufstand, zu seinem Schreibtisch ging und ein Blatt feinstes Schafshautpergament und ein perlengeschnitztes Tintenfass herausholte, gefüllt mit grünlich-silberner Tinte, die aus echtem Silber und zerstoßenen Smaragden hergestellt worden war. Aus der großen Truhe am Fuße seines Bettes zog der Slytherin ein ebenfalls in Silber und Smaragde gebundenes Buch mit dem Titel „\emph{Die Etikette der Familien von Britannien“} heraus. Und mit einem neuen, sauberen Federkiel begann Draco Malfoy zu schreiben, wobei er häufig auf das Buch schaute, das neben ihm als Referenz offen lag. Das Gesicht des Jungen zeigte ein grimmiges Lächeln, wodurch der junge Malfoy seinem Vater sehr ähnlich sah, während er jeden Buchstaben sorgfältig zeichnete, als wäre er ein separates Kunstwerk.

\emph{Von Draco, dem Sohn von Lucius, dem Sohn von Abraxis, Herren aus dem edlen undUralten Haus von Malfoy, Sohn von Narcissa, der Tochter der Druella, Herrin aus dem edlen und Uralten Haus der Blacks, Spross und Erbe des edlen und Uralten Hauses von Malfoy:}

An Hermine, die erste Granger:

(Diese Formalie hätte vor langer Zeit, als sie erfunden wurde, höflich klingen können; heutzutage, nach Jahrhunderten, in denen Schlammblüter so adressiert wurden, führte sie einen lieblichen Hauch von Missachtung mit sich).

\emph{Ich, Draco, aus Uraltem Hause, verlange Wiedergutmachung, für}

Draco hielt inne und schob den Federkiel vorsichtig zur Seite, damit er nicht das Pergament bekleckste. Er brauchte einen Vorwand, zumindest wenn er die Bedingungen des Duells definieren wollte. Die Herausgeforderten hatten die Wahl der Bedingungen,\emph{es sei denn}, sie hatten ein Adelshaus beleidigt. Er musste es so aussehen lassen, als hätte Granger ihn beleidigt...

Wieso sollte das ein Problem sein? Granger\emph{hatte}ihn beleidigt.

Draco durchblätterte das Buch bis zur Seite mit den Standardformulierungen und fand eine, die ihm angemessen erschien.

\emph{Ich, Draco, aus Uralten Hause, verlange Wiedergutmachung dafür, dass ich Ihnen dreimal aus gutem Willen geholfen habe, und im Gegenzug Sie mich\uline{fälschlicherweise}beschuldigt haben, eine Verschwörung gegen Sie zu schmieden,}

Draco musste innehalten und durchatmen, um die brodelnde Wut zu unterdrücken; jetzt begann er die Beleidigung wirklich zu fühlen, und er hatte gerade den letzten Satz geschrieben und ohne nachzudenken unterstrichen, als wäre es ein gewöhnlicher Brief. Nach einem Moment des Nachdenkens entschied er sich, ihn stehen zu lassen; es war vielleicht nicht die genaue formale Formulierung, aber er hatte einen rohen, wütenden Tonfall, der angemessen erschien.

\emph{so haben Sie mich vor den Augen Britanniens beleidigt.}

\emph{Daher fordere ich, Draco, Sie, Hermine, aufgrund der Tradition, des Gesetzes und}

„des siebzehnten Urteils des einunddreißigsten Zaubergamots“, sagte Draco laut, ohne hinzusehen, eine Zeile, die in vielen Theaterstücken vorgetragen wurde; er saß gerader, als er es sagte, und fühlte das Pumpen seines edlen Blutes in den Adern.

\emph{Daher fordere ich, Draco, Sie, Hermine, aufgrund der Tradition, des Gesetzes und des 17. Urteils des 31. Zaubergamots mir im Zaubererduell mit folgenden Bedingungen zu begegnen: Dass jeder von uns allein und schweigend kommt und weder vorher noch nachher mit jemandem spricht,}

Wenn das Duell schlecht verlief, konnte Draco einfach nichts sagen und es dabei belassen. Und wenn er Granger besiegt hätte, hätte er experimentell gelernt, dass er sie in einer öffentlichen Herausforderung wieder schlagen könnte. Es war kein Betrug, aber es war Wissenschaft; und das war fast genauso gut.

\emph{dass wir ausschließlich durch Magie, ohne Tod oder bleibende Verletzungen kämpfen},

...wo? Draco war von einem Raum in Hogwarts erzählt worden, der gut für Duelle geeignet war, in dem alles Wertvolle bereits durch Schutzzauber geschützt war, und es keine Porträts gab, die einen verpetzen konnten... welcher war es noch einmal gewesen...

\emph{im Pokalzimmer des Schlosses der Hogwarts Schule für Hexerei und Zauberei,}

Und ihr zweites und öffentliches Duell sollte besser bald stattfinden, am Besten schon morgen, denn es würde nur sehr wenig Zeit brauchen, bis sein Ruf in Slytherin unwiederbringlich im Schlamm versank. Er musste heute Abend zum ersten Mal gegen Granger kämpfen.

\emph{Um Schlag Mitternacht, der das Ende des heutigen Tages markiert}.

\emph{Draco, aus dem edlen und Uralten Haus von Malfoy.}

Draco signierte das formale Pergament und nahm dann sein gewöhnliches, kleineres Pergament sowie seine gewöhnliche Tinte für sein Postskriptum:

\emph{Wenn du nicht weißt, wie das hier funktioniert, Granger, hier die Kurzfassung. Du hast ein Uraltes Haus beleidigt, und ich habe nach dem Gesetz das Recht, dich herauszufordern. Und wenn du die Bedingungen des Duells verletzt, etwa indem du Flitwick im Pokalzimmer auftauchen lässt oder auch nur jemand anderem davon erzählst, wird mein Vater dich und deine falsche Ehre direkt vor den Zaubergamot bringen.}

\emph{Draco Malfo}

Beim letzten Wort drückte sein Federkiel so heftig auf das Pergament, dass die Feder abbrach, wodurch ein Streifen Tinte und ein kleiner Riss im Pergament entstand, und Draco entschied, dass das ebenfalls angemessen aussah.

An jenem Abend zur Essenszeit kam Susan Bones zu Harry Potter und sagte ihm, dass sie glaube, Draco Malfoy werde sein Komplott gegen Hermine sehr bald in die Tat umsetzen. Sie warnte alle Mitglieder von B. E.L. F.E. R und sie hatte Professor Sprout gewarnt, und sie hatte Professor Flitwick gewarnt, und sie wollte heute Abend einen Brief an ihre Tante schicken, und nun warnte sie auch Harry Potter. Nur konnten sie nicht mit Padma darüber sprechen - meinte Susan, die sehr ernst aussah - denn Padma fühle sich zwischen ihrer Loyalität zu Hermine und ihrer Loyalität zum General ihrer Armee hin- und hergerissen.

Harry James Potter-Evans-Verres, der sich zu diesem Zeitpunkt eher frustriert fühlte, aber keine wirklich \emph{produktive} Idee hatte, fuhr sie an, dass er ja wisse, dass etwas getan werden müsse.

Nachdem Susan Bones gegangen war, schaute Harry zum anderen Ende des Ravenclaw-Tisches hinüber, wo Hermine sich abseits von ihm, Padma, Anthony oder einem ihrer anderen Freunde hingesetzt hatte.

Aber Hermine sah nicht so aus, als ob sie es gut aufnehmen würde, wenn jemand zu ihr hinübergehen und sie belästigen würde.

Wenn Harry später zurückblickte, dachte er daran, wie die Charaktere in seinen SciFi- und Fantasy-Romanen immer ihre großen, wichtigen Entscheidungen aus großen, wichtigen Gründen trafen. Hari Seldon {[}Anm. d. Übers:\\ Foundation-Zyklus von Isaac Asimov{]} hatte seine Stiftung gegründet, um die Asche des Galaktischen Reiches wieder aufzubauen, nicht weil er wichtiger aussehen würde, wenn er seine eigene Forschungsgruppe leiten könnte. Raistlin Majere {[}Anm. d. Übers.: Drachenlanze von Margaret Weis{]} hatte die Verbindung zu seinem Bruder abgebrochen, weil er ein Gott werden wollte, und nicht, weil er inkompetent in persönlichen Beziehungen war und nicht bereit war, um Rat zu fragen. Frodo Beutlin hatte den Ring genommen, weil er ein Held war, der Mittelerde retten wollte, und nicht, weil es einfach unhöflich gewesen wäre, es nicht zu tun. Wenn irgendjemand jemals die wahre Weltgeschichte aufschreiben würde - nicht, dass irgendjemand jemals eine schreiben könnte oder würde - würden sich wahrscheinlich 97\% aller Schlüsselmomente des Schicksals als Folge von Missverständnissen, Selbstbetrug und trivialen kleinen Gedanken herausstellen, die genauso gut auch anders hätten laufen können.

Harry James Potter-Evans-Verres schaute Hermine Granger an, die sich an das andere Ende des Tisches gesetzt hatte, und wollte sie nicht stören, wenn sie schon schlecht gelaunt aussah.

Alsodachte Harry, dass es wahrscheinlich sinnvoller wäre, zuerst mit Draco Malfoy zu sprechen, nur damit er Hermine danach völlig überzeugt versichern konnte, dass Draco wirklich nicht gegen sie intrigierte.

Und später nach dem Abendessen, als Harry in die Kerker zum Gemeinschaftsraum von Slytherin herabstieg und von Vincent informiert wurde, dass \emph{der Boss nicht gestört werden dürfe}... da dachte Harry, dass er vielleicht zusehen sollte doch noch mit Hermine zu sprechen. Dass er einfach anfangen sollte, das ganze Durcheinander zu entwirren, bevor es noch weiterdurcheinandergeriet. Harry fragte sich, ob er es vielleicht nur hinauszögern wollte, weil sein Verstand gerade eine klug klingend Begründung gefunden hatte, um etwas Unerfreuliches-aber-Notwendiges hinauszuschieben.

Das dachte er tatsächlich.

Und dann beschloss Harry James Potter-Evans-Verres, dass er stattdessen einfach am nächsten Morgen, nach dem Sonntagsfrühstück, mit Draco Malfoy und \emph{dann} mit Hermine sprechen würde.

Menschen machen so etwas ständig.

Es war Sonntagmorgen, am 5. April 1992, und der simulierte Himmel über der Großen Halle von Hogwarts zeigte große Regenströme, die periodisch von Blitzen durchzuckt wurden, die die Gesichter aller Schüler erhellte, sodass sie kurzzeitig wie Geister aussahen.

Harry saß am Ravenclaw-Tisch und aß müde eine Waffel und wartete darauf, dass Draco auftauchte, damit er sich daran machen konnte, die ganze Sache in Ordnung zu bringen. Es wurde eine Ausgabe des \emph{Quibblers} herumgereicht, bei dem irgendwie Hannah Abott und Daphne Greengrass auf der Titelseite gelandet waren, aber er war noch nicht bei ihm angekommen.

Ein paar Minuten später nahm Harry gerade den letzten Bissen von seiner Waffel und sah sich dann noch einmal um, um zu sehen, ob Draco schon zum Frühstück am Slytherin-Tisch gekommen war.

Es war merkwürdig.

Draco Malfoy kam fast nie zu spät.

Da Harry in die Richtung des Slytherin-Tisches blickte, sah er Hermine Granger nicht durch die riesigen Türen der Großen Halle eintreten. Daher erschreckte er sich ziemlich, als er sich umdrehte und Hermine sich direkt neben ihm am Ravenclaw-Tisch niederließ. (als hätte sie das nicht seit mehr als einer Woche nicht mehr getan.)

„Hallo, Harry“, sagte Hermine, ihre Stimme klang fast ganz normal. Sie begann, Toast sowie eine Auswahl an gesundem Obst und Gemüse, auf ihren Teller zu laden. „Wie geht es dir?"

„Innerhalb einer Standardabweichung von meinem ganz persönlichen Durchschnitt“, antwortete Harry automatisch. „Wie geht es dir? Hast du gut geschlafen?"

Es sah die dunklen Tränensäcke unter Hermine Grangers Augen.

„Ja, mir geht es gut, warum fragst du?“, sagte Hermine Granger.

„Ähm“, sagte Harry. Er nahm sich ein Stück Kuchen (da sein Gehirn mit anderen Dingen beschäftigt war, nahm Harrys Hand einfach das leckerste Ding in Reichweite, ohne komplexe Konzepte zu bewerten, wie etwa, ob er bereit war, schon Nachtischzu essen). „Ähm, Hermine, ich muss im Laufe des Tages mit dir reden, ist das okay?„

„Sicher“ sagte Hermine. „Warum sollte es das nicht sein?"

„Weil -“ sagte Harry. „Also --du und ich -- wir haben nicht -- in den letzten Tagen -"

\emph{Klappe}, schlug ein Teil von Harrys Verstand vor, der anscheinend vor kurzem für die Regelung für alle Hermine-spezifischen Angelegenheiten allokiert worden war.

Hermine Granger sah ohnehin nicht so aus, als würde sie ihm viel Aufmerksamkeit schenken. Sie starrte nur auf ihren Teller hinunter und begann dann, nach etwa zehn Sekunden unbeholfenen Schweigens, ohne Pause ihre Tomatenscheiben zu essen.

Harry schaute sich um und begann, ein Stück Kuchen zu essen, dass sich, wie er entdeckte, irgendwie auf seinem Teller materialisiert hatte.

„Also!“ sagte Hermine Granger plötzlich, nachdem sie schweigend den größten Teil ihres Tellers leergeräumt hatte. „Was steht heute an?"

„Ähm...“ sagte Harry. Er sah sich verzweifelt nach Gesprächsstoff um.

Und so war Harry einer der ersten, der es sah und wortlos darauf zeigte, während sich plötzlich eine Welle des Flüsterns in der Große Halle ausbreite.

Der unverkennbare karminrote Farbton der Gewänder wäre überall zu erkennen gewesen, aber Harrys Gehirn brauchte noch einige Augenblicke, um die Gesichter einzuordnen. Ein asiatisch aussehender Mann, normalerweise feierlich aber heute eher grimmig dreinblickend. Ein Mann mit einem durchdringenden Blick, der den Raum scannte, und langem schwarzen Haar, das in einem Pferdeschwanz zusammengebunden war. Ein Mann, dünn und blass und unrasiert, mit einem ausdruckslosen Gesicht, dass wie aus Stein gemeißelt aussah. Harry brauchte einen Moment, um die Gesichter einzuordnen und sich an die Namen von jenem längst vergangenen Tag im Januar zu erinnern, als der Dementor nach Hogwarts gekommen war: \emph{Komodo, Butnaru, Goryanof}.

„Ein Auror-Trio?“ sagte Hermine mit einer seltsam hellen Stimme. „Ich frage mich, was sie hier wohl trieben."

Auch Dumbledore befand sich in ihrer Gesellschaft und sah so besorgt aus, wie Harry ihn noch nie gesehen hatte; und nach einer kurzen Pause, während die Augen des alten Zauberers die Große Halle absuchten und die Schüler weiter flüsterten, zeigte er -

- direkt auf Harry.

„Oh, was denn jetzt?“, flüsterte Harry. Seine inneren Gedanken waren viel panischer als das, denn er fragte sich verzweifelt, ob ihn jemand irgendwie mit dem Ausbruch aus Askaban in Verbindung gebracht hatte. Er schaute auf den Lehrertisch, versuchte, den Blick beiläufig zuhalten, und stellte fest, dass Professor Quirrell nirgends zu sehen war, heute Morgen -

Die Auroren stampften mit schnellen Schritten auf ihn zu, Auror Goryanof näherte sich von der einen Seite des Ravenclaw-Tisches und Auror Komodo und Auror Butnaru näherten sich von der anderen Seite, als wolle sie jede Fluchtmöglichkeit blockieren, der Schulleiter folgte Komodo geradewegs auf den Fersen.

Die Gespräche waren inzwischen völlig verstummt.

Die Auroren erreichten Harrys Platz am Tisch und umgaben ihn aus drei Winkeln.

„Ja?“ sagte Harry, so normal, wie er konnte. „Was ist?“, sagte Harry.

„Hermine Granger“, sagte Auror Komodo mit monotoner Stimme, „Sie sind verhaftet wegen des versuchten Mordes an Draco Malfoy.“

