

\hypertarget{belohnungsaufschub}{% \section{19. Belohnungsaufschub}\label{belohnungsaufschub}}

Draco hatte einen ernsten Gesichtsausdruck und sein grün gesäumter Umhang wirkte irgendwie sehr viel förmlicher, seriöser und maßgeschneiderter als die exakt gleichen Umhänge, die von den zwei Jungen hinter ihm getragen wurden.

„Sprich“, sagte Draco.

„Ja! Sprich!“

„Du hast gehört was der Boss gesagt hat! Sprich!“

„Ihr beiden hingegen \emph{seid still}.“

Die letzte Unterichtsstunde am Freitag würde gleich beginnen, hier im riesigen Hörsaal, in dem die Schüler aller Häuser gemeinsam Verteidigung—äh, Kampfmagie—lernten.

Die letzte Unterichtsstunde am Freitag.

Harry hoffte, dass diese Stunde nicht zu anstrengend werden würde, und dass der brillante Professor Quirrell merken würde, dass dies nicht die beste Gelegenheit war, um Harry vorzuführen. Harry hatte sich ein wenig erholt, aber …

… aber, nur für den Fall, wäre es besser, erstmal etwas Stress abzubauen.

Harry lehnte sich in seinem Stuhl zurück und betrachtete Draco und seine Lakaien mit einem feierlichen Blick.

„Ihr fragt, was ist unser Ziel?“, deklamierte Harry. „Ich kann es in einem Wort beantworten. Es ist der Sieg. Sieg um jeden Preis—Sieg, trotz aller Schrecken—Sieg, wie lang und beschwerlich der Weg dahin auch sein mag, denn ohne Sieg gibt es kein—“

„\emph{Sprich über Snape}“, zischte Draco. „\emph{Was hast du getan?}“

Harry wischte die aufgesetzte Feierlichkeit beiseite und setzte einen ernsteren Gesichtsausdruck auf.

„Du hast es gesehen“, sagte Harry. „Jeder hat es gesehen. Ich habe mit den Fingern geschnippt.“

„\emph{Harry! Spann' mich nicht auf die Folter!}“

Also war er nun \emph{Harry}. Interessant. Und Harry war sich recht sicher, dass ihm das auffallen sollte und dass er sich unwohl fühlen sollte, wenn er es nicht irgendwie erwiderte …

Harry tippte sich an die Ohren und warf den Lakaien dann einen vielsagenden Blick zu.

„Sie werden nichts ausplaudern“, sagte Draco.

„Draco“, sagte Harry, „ich will ganz ehrlich mit dir sein und muss zugeben, dass ich von Mr~Goyles Schläue gestern nicht beeindruckt war.“

Mr~Goyle zuckte zusammen.

„Ich auch nicht“, sagte Draco. „Ich habe ihm erklärt, dass ich dir deswegen nun einen großen Gefallen schuldig bin.“ (Mr~Goyle zuckte erneut zusammen.) „Doch es besteht ein großer Unterschied zwischen einem solchen Fehler und Indiskretion. Das wurde ihnen wirklich von frühster Kindheit an beigebracht.“

„Na gut“, sagte Harry. Er senkte seine Stimme, wenngleich die Hintergrundgeräusche in Dracos Gegenwart zu einem dumpfen Gemurmel geworden waren. „Ich habe eines von Severus' Geheimnissen erraten und dann etwas Erpressung betrieben.“

Dracos Gesichtsausdruck wurde härter. „Gut, und jetzt erzähle mir etwas, was du \emph{nicht} den Idioten in Gryffindor unter dem Siegel der Verschwiegenheit mitgeteilt hast; denn das ist natürlich nur die Geschichte, von der du \emph{wolltest}, dass sie sich in der ganzen Schule rumspricht.“

Harry grinste unwillkürlich und wusste, dass Draco es bemerkt hatte.

„Was hat Severus gesagt?“, sagte Harry.

„Dass ihm nicht bewusst war, wie empfindlich die Gefühlswelt junger Kinder ist“, sagte Draco. „Selbst zu den Slytherins! Selbst zu \emph{mir}!“

„Bist du dir sicher“, sagte Harry, „dass du etwas wissen möchtest, von dem dein Hauslehrer nicht will, dass du es weißt?“

„Ja“, sagte Draco ohne zu zögern.

\emph{Interessant.} „Dann wirst du deine Lakaien vorher wirklich wegschicken, denn ich bin mir nicht sicher, ob ich alles über sie glauben kannst, was du über sie glaubst.“

Draco nickte. „Okay.“

Mr~Crabbe und Mr~Goyle sahen \emph{sehr} unglücklich aus. „Boss—“, sagte Mr~Crabbe.

„Du hast Mr~Potter keinen Anlass gegeben, dir zu vertrauen“, sagte Draco. „Geh!“

Sie gingen.

„Insbesondere“, sagte Harry und senkte seine Stimme noch weiter, „bin ich mir nicht \emph{vollkommen} sicher, ob sie das, was ich erzähle, nicht an Lucius weitersagen würden.“

„Vater würde das nicht \emph{tun}!“, sagte Draco, der ernsthaft entsetzt aussah. „Sie gehören \emph{mir}!“

„Es tut mir leid, Draco“, sagte Harry. „Ich bin mir auch nicht sicher, ob ich alles glauben kann, was du über deinen Vater glaubst. Stell' dir vor, es wäre dein Geheimnis und ich würde dir sagen, dass mein Vater das nicht tun würde.“

Draco nickte langsam. „Du hast Recht. Es tut \emph{mir} Leid, Harry. Es war falsch von mir, das von dir zu verlangen.“

\emph{Wie bin ich denn} so \emph{in seiner Achtung gestiegen? Sollte er mich jetzt nicht hassen?} Harry hatte das Gefühl, dass diese Situation sich ausnutzen ließ…er wünschte bloß, sein Gehirn wäre nicht so erschöpft. Normalerweise hätte er nur zu gerne versucht, einen komplizierten Plan auszuhecken.

„Wie dem auch sei“, sagte Harry. „Ein Angebot. Ich erzähle dir einen Fakt, der nicht öffentlich bekannt ist und nicht öffentlich bekannt \emph{wird} und von dem \emph{insbesondere} dein Vater nichts erfährt, und im Gegenzug erzählst du mir, was du und die Slytherins von der ganzen Angelegenheit halten.“

„Einverstanden!“

Nun musste er das so vage wie möglich ausdrücken…so, dass kein Schaden entstünde, selbst wenn es herauskäme… „Was ich gesagt habe ist wahr. Ich habe tatsächlich eines von Severus' Geheimnissen erraten und habe dann tatsächlich etwas Erpressung betrieben. Aber Severus war nicht die einzige Person, um die es ging.“

„\emph{Ich wusste es!}“, sagte Draco triumphierend.

Harrys Magen drehte sich um. Er hatte offenbar etwas sehr Bedeutendes gesagt und er wusste nicht warum. Das war kein gutes Zeichen.

„Okay“, sagte Draco. Er grinste jetzt breit. „Also, wie war die Reaktion unter den Slytherins? Zuerst meinten alle Idioten ‚Wir hassen Harry Potter! Lasst uns ihn verprügeln!`\,“

Harry schluckte. „Was ist denn mit dem Sprechenden Hut \emph{los}? So denkt kein Slytherin, so denken die \emph{Gryffindors}—“

„Nicht alle Kinder sind so schlau wie du“, sagte Draco, doch er lächelte ihm verschwörerisch zu, als wolle er andeuten, dass er Harrys Meinung insgeheim teilte. „Und es hat ungefähr fünfzehn Sekunden gedauert, bis jemand ihnen erklärt hat, warum sie Snape damit wohl keinen Gefallen tun würden, also mach' dir keine Sorgen. Auf jeden Fall kam dann der nächste Haufen von Idioten an, die sagten, ‚Es sieht so aus, als ob Harry Potter doch nur so ein Wohltäter ist.`\,“

„Und dann?“, sagte Harry und lächelte, obwohl er keine Ahnung hatte, warum \emph{das} dumm war.

„Und dann begannen die wirklich klugen Leute zu reden. Es ist offensichtlich, dass du einen Weg gefunden hast, Snape \emph{mächtig} unter Druck zu setzen. Und auch wenn es um etwas ganz Anderes gehen könnte…der Gedanke liegt nahe, dass es etwas mit der unbekannten Sache zu tun hat, die Snape Macht über Dumbledore verleiht. Liege ich richtig?“

„Kein Kommentar“, sagte Harry. Zumindest verarbeitete sein Gehirn diesen Teil korrekt. Die Slytherins \emph{hatten} sich gefragt, warum Severus nicht rausgeschmissen wurde. Und sie hatten gefolgert, dass Severus Dumbledore erpresst. Könnte das tatsächlich stimmen …? Aber Dumbledore hatte keine Sekunde lang den Eindruck erweckt …

Draco sprach weiter. „Und als \emph{Nächstes} wiesen die klugen Leute darauf hin, dass du genug Macht über Severus hast, um ihn zu dazu zu zwingen, dass er halb Hogwarts in Ruhe lässt. Das bedeutet, dass du vermutlich auch genug Macht hast, um ihn ganz rauszuschmeißen, wenn du das wolltest. Du hast ihn gedemütigt, genau wie er versucht hat, dich zu demütigen—aber du hast uns unseren Hauslehrer gelassen.“

Harrys Lächeln wurde breiter.

„Und die \emph{wirklich} klugen Leute“, sagte Draco, nun mit ernstem Gesicht, „haben sich dann zurückgezogen und miteinander weiter diskutiert und irgendjemand wies darauf hin, dass es sehr dumm sei, einen Feind so in der Nähe zu behalten. Wenn du seine Macht über Dumbledore brechen könntest, dann wäre es der offensichtlichste Schritt, das zu tun. Dumbledore würde Snape aus Hogwarts rausschmeißen und womöglich umbringen lassen, er wäre dir \emph{sehr} dankbar und du bräuchtest dir keine Sorgen machen, dass Snape nachts mit interessanten Zaubertränken in deinen Schlafsaal schleicht.“

Harrys Gesicht war nun neutral. Er hatte das nicht bedacht, aber er hätte es wirklich, wirklich tun sollen. „Und daraus habt ihr gefolgert …?“

„Snapes Macht war irgendein Geheimnis von Dumbledore und \emph{du kennst das Geheimnis!}“ Draco sah triumphierend aus. „Es kann nicht mächtig genug sein, um Dumbledore ganz zu zerstören, sonst hätte Snape es schon benutzt. Snape weigert sich, diese Macht für irgendetwas anderes zu benutzen, als Herrscher über Slytherin zu bleiben, doch selbst da bekommt er nicht alles was er will, also muss es Grenzen haben. Aber es \emph{muss} wirklich gut sein! Vater hat \emph{seit Jahren} versucht, es von Snape zu erfahren!“

„Und“, sagte Harry, „jetzt denkt Lucius, dass \emph{ich} es ihm vielleicht verraten kann. Hast du schon eine Eule—“

„Die wird heute Nacht kommen“, sagte Draco und lachte. „Darin wird stehen“, seine Stimme nahm einen anderen, formelleren Tonfall an, „\emph{Mein geliebter Sohn: Ich habe Dir bereits von Harry Potters möglicher Bedeutung erzählt. Wie Du bereits bemerkt hast, ist seine Bedeutung nun größer und deutlicher geworden. Wenn Du irgendeinen möglichen Ansatzpunkt zu einer Freundschaft oder eine Möglichkeit, ihn unter Druck zu setzen siehst, so musst Du dies verfolgen. Sofern nötig stehen Dir hierfür sämtliche Ressourcen der Malfoys zur Verfügung.}“

Donnerwetter. „Nun“, sagte Harry, „ich will nicht darauf eingehen, ob die komplizierte Theorie, die ihr da konstruiert habt, der Wahrheit entspricht oder nicht; ich möchte nur bemerken, dass wir noch nicht \emph{so} gute Freunde sind.“

„Ich weiß“, sagte Draco. Dann wurde sein Gesichtausdruck \emph{sehr} ernst und trotz des dumpfen Gemurmels, das sie umgab, senkte er seine Stimme. „Harry, hast du daran gedacht, dass Dumbledore dich einfach töten lassen könnte, wenn du etwas weißt, was Dumbledore geheim halten will? Dann wäre der Junge der lebt kein möglicherweise konkurrierender Anführer, sondern ein sehr nützlicher Märtyrer.“

„Kein Kommentar“, sagte Harry wieder. Er hatte auch daran nicht gedacht. \emph{Schien} nicht ganz Dumbledores Stil zu sein…aber …

„Harry“, sagte Draco, „du hast offenbar \emph{unglaubliche} Talente, aber du bist unerfahren und hast keine Mentoren und du machst manchmal dumme Sachen und \emph{du brauchst dringend einen Berater, der sich damit auskennt, sonst wirst du schrecklich auf die Nase fallen}!“ Dracos Gesichtsausdruck war grimmig.

„Ach“, sagte Harry. „Einen Berater wie Lucius?“

„Wie \emph{mich}!“, sagte Draco. „Ich verspreche, dass ich deine Geheimnisse gegenüber Vater bewahre, gegenüber \emph{jedem} bewahre; ich werde dir einfach helfen herauszufinden, wie du handeln solltest!“

Wow.

Harry sah, dass Quirrell zombiegleich durch die Tür herein wankte.

„Der Unterricht fängt an“, sagte Harry. „Ich werde über das nachdenken, was du gerade gesagt hast. Ich wünsche mir tatsächlich oft, dass ich deine Erfahrung hätte; ich weiß bloß nicht, wie ich dir so schnell vertrauen könnte—“

„Das solltest du gar nicht“, sagte Draco, „dafür ist es zu früh. Siehst du? Ich gebe dir guten Rat, auch wenn es mir schadet. Aber vielleicht sollten wir uns \emph{beeilen} und engere Freunde werden.“

„Dazu bin ich bereit“, sagte Harry, der bereits darüber nachdachte, wie er das ausnutzen konnte.

„Noch ein Ratschlag“, sagte Draco eilig, während Quirrell zu seinem Tisch schlurfte. „Im Moment bist du für alle Slytherins ein Rätsel. Wenn du also die Slytherins für dich einnehmen willst, wie ich annehme, dann solltest du etwas machen, was den Slytherins Freundschaft signalisiert. Und zwar \emph{rasch}, also heute oder morgen.“

„Dass Severus weiterhin übertrieben viele Hauspunkte an Slytherin verteilen darf, reicht nicht aus?“ Es sprach ja nichts dagegen, dass Harry das auf seine Kappe nahm.

Dracos Augen leuchteten erkennend auf, dann sagte er rasch, „Das ist nicht das Gleiche, glaub mir, es muss etwas Offensichtliches sein. Schubse deine Rivalin, das Schlammblut Granger, eine Treppe runter, oder so; alle Slytherins werden das verstehen—“

„So läuft das in Ravenclaw nicht, Draco! Wenn du jemanden die Treppe runter schubsen musst, dann heißt das, dass dein Gehirn zu \emph{schwach} ist und du nicht wirklich gegen sie ankommst. \emph{Jeder} Ravenclaw weiß das—“

Der Bildschirm auf Harrys Tisch flackerte auf und erweckte in Harry eine nostalgische Sehnsucht nach Fernsehern und Computern.

Professor Quirrells Stimme räusperte sich und sprach dann aus dem Bildschirm scheinbar direkt zu Harry. „Bitte nehmt eure Plätze ein.“

\later

Und die Kinder saßen da und blickten auf die Bildschirme auf ihren Tischen oder direkt zur großen Bühne aus weißem Marmor. Professor Quirrell stand dort auf dem Podium aus dunklerem Marmor und stützte sich auf seinen Tisch.

„Heute“, sagte Professor Quirrell, „wollte ich euch eigentlich euren ersten Verteidigungszauber beibringen; ein kleiner Schild, der Vorfahre des heutigen \emph{Protego}. Aber ich habe mich dann doch entschlossen, den heutigen Unterrichtsplan aus aktuellem Anlass anzupassen.“

Professor Quirrells Blick streifte über die Reihen von Schülern. Harry zuckte auf seinem Platz in der hintersten Reihe zusammen. Er hatte das Gefühl, dass er schon wusste, wen Quirrell nun aufrufen würde.

„Draco, vom führnehmen und gar alten Haus Malfoy“, sagte Professor Quirrell.

Uff.

„Ja, Professor?“, sagte Draco. Seine Stimme wurde verstärkt und schien aus dem Bildschirm auf Harrys Tisch zu kommen, auf dem Dracos Gesicht zu sehen war, während er sprach. Dann wechselte der Bildschirm zurück zu Professor Quirrell, der sagte:

„Ist es deine Absicht, der nächste Dunkle Lord zu werden?“

„Das ist eine seltsame Frage, Professor“, sagte Draco. „Ich meine, wer wäre dumm genug, das zuzugeben?“

Einige Schüler lachten, aber nicht viele.

„In der Tat“, sagte Professor Quirrell. „Obwohl es also sinnlos wäre, jemanden von euch danach zu fragen, würde es mich kein bisschen überraschen, wenn in meinen Klassen ein oder zwei Schüler wären, die danach strebten, der nächste Dunkle Lord zu werden. Schließlich wollte \emph{ich} der nächste Dunkle Lord werden, als \emph{ich} ein junger Slytherin war.“

Dieses Mal war das Lachen viel weiter verbreitet.

„Nun, es \emph{ist} nunmal das Haus für Ehrgeizige“, sagte Professor Quirrell lächelnd. „Mir wurde erst später klar, dass ich in Wahrheit Freude an der Kampfmagie empfand und dass es mein wahres Ziel war, ein exzellenter Kampfmagier zu werden und eines Tages in Hogwarts zu lehren. Wie dem auch sei…als ich dreizehn Jahre alt war, habe ich mich durch die Geschichtsabteilung der Hogwarts-Bibliothek gelesen und die Leben und Schicksale früherer Dunkler Lords studiert und ich habe eine Liste all jener Fehler aufgestellt, die \emph{ich} nie begehen würde, wenn \emph{ich} ein Dunkler Lord wäre—“

Harry konnte sich das Kichern nicht verkneifen.

„Ja, Mr~Potter, höchst amüsant. Also, Mr~Potter, können Sie erraten, was der allererste Eintrag auf der Liste war?“

\emph{Na toll.} „Ähm…dass Sie niemals eine komplizierte Methode benutzen, um einen Feind auszuschalten, wenn ein einfacher Abrakadabra ausreicht?“

„Der \emph{Spruch}, Mr~Potter, heißt \emph{Avada Kedavra}“—Professor Quirrells Stimme klang aus irgendeinem Grund etwas scharf—„und nein, das stand \emph{nicht} auf der Liste, die ich mit dreizehn Jahren angelegt habe. Wollen Sie es nochmal versuchen?“

„Äh…dass Sie niemals vor Anderen mit ihrem bösen Masterplan prahlen?“

Professor Quirrell lachte. „Oh, \emph{das} wiederum war Nummer zwei. Meine Güte, Mr~Potter, haben wir etwa die gleichen Bücher gelesen?“

Es ertönte wieder Gelächter, diesmal mit einem nervösen Unterton. Harry presste seine Kiefer fest aufeinander und sagte nichts. Es abzustreiten wäre zwecklos.

„Aber nein. Der \emph{erste} Eintrag war ‚Ich werde keine starken, zornigen Feinde herausfordern.` Die Weltgeschichte wäre wesentlich anders verlaufen, wenn Mornelithe Falconsbane oder Hitler diesen elementaren Punkt verstanden hätten. \emph{Falls}, Mr~Potter—nur \emph{falls} Sie rein zufällig ähnliche Pläne hegen, wie ich es als junger Slytherin tat—selbst dann, hoffe ich, beabsichtigen Sie nicht, ein \emph{dummer} Dunkler Lord zu werden.“

„Professor Quirrell“, sagte Harry mit knirschenden Zähnen, „ich bin ein \emph{Ravenclaw} und ich beabsichtige nicht, dumm zu sein—Punkt. Ich weiß, dass es dumm war, was ich heute getan habe. Aber es war nicht \emph{Dunkel}! Ich habe \emph{nicht} zuerst angegriffen!“

„Sie, Mr~Potter, sind ein Idiot. Allerdings war ich das in Ihrem Alter auch. Daher habe ich Ihre Antwort erwartet und die heutige Stunde entsprechend umgeplant. Mr~Gregory Goyle, wenn Sie bitte nach vorne kommen mögen?“

Im Klassenzimmer herrschte überraschte Stille. Harry hatte das nicht erwartet.

Ebensowenig hatte das, wie es aussah, Mr~Goyle, der recht unsicher und besorgt aussah, als er auf die marmorne Bühne trat und zum Podium ging.

Professor Quirrell, der noch am Tisch gelehnt hatte, richtete sich nun auf. Er sah plötzlich stärker aus, seine Hände waren zu Fäusten geballt und er nahm eine unverkennbare Kampfstellung an.

Harrys Augen weiteten sich bei dem Anblick und ihm wurde klar, warum Mr~Goyle nach vorne gebeten wurde.

„Die meisten Zauberer“, sagte Professor Quirrell, „scheren sich nicht um das, was ein Muggel als Kampfkunst bezeichnen würde. Ist ein Zauberstab nicht stärker als eine Faust? Diese Einstellung ist dumm. Zauberstäbe werden in Fäusten gehalten. Wer ein sehr guter Kampfmagier werden will, der \emph{muss} Kampfkunst so gut beherrschen, dass es selbst einen Muggel beeindrucken würde. Ich werde nun eine gewisse außerordentlich wichtige Technik demonstrieren, die ich in einem Dojo gelernt habe—einer Kampfkunstschule der Muggel, über die ich gleich mehr erzählen werde. Doch zuerst …“ Professor Quirrell machte mehrere Schritte nach vorne, immer noch in Kampfstellung, bis er Mr~Goyle gegenüber stand. „Mr~Goyle, ich bitte Sie, mich anzugreifen.“

„Professor Quirrell“, sagte Mr~Goyle, dessen Stimme nun ebenso wie die des Professors verstärkt wurde, „darf ich fragen, welchen Grad—“

„Sechster \emph{Dan}. Sie werden nicht verletzt und ich ebenso wenig. Und falls Sie eine Lücke in meiner Deckung sehen, nutzen Sie die bitte aus.“

Mr~Goyle nickte und sah sehr erleichtert aus.

„Beachten Sie“, sagte Professor Quirrell, „dass Mr~Goyle Angst davor hatte, jemanden anzugreifen, der Kampfkunst nicht hinreichend gut beherrscht, aus Angst, dass ich—oder er—verletzt würde. Mr~Goyles Herangehensweise ist vollkommen richtig und er hat damit drei Quirrellpunkte verdient. Nun los!“

Der Junge stürzte sich mit fliegenden Fäusten voran und der Professor wehrte jeden Schlag ab, tänzelte rückwärts, Quirrell trat zu und Goyle blockte ab und drehte sich und versuchte Quirrell mit einem ausgestreckten Bein zum Stolpern zu bringen und Quirrell sprang drüber und es geschah alles zu schnell für Harry, der nicht mehr nachverfolgen konnte, was geschah, und dann lag Goyle auf dem Rücken, mit ausgestreckten Beinen, und Quirrell \emph{flog durch die Luft} und landete mit der Schulter voran auf dem Boden und rollte ab.

„Stop!“, schrie Professor Quirrell am Boden, mit leicht panischer Note. „Sie haben gewonnen!“

Mr~Goyle, der gerade auf Professor Quirrell zu stürmte, hielt so abrupt an, dass er aus dem Gleichgewicht kam und fast gestolpert und hingefallen wäre. Sein Gesicht war vom Schock verzerrt.

Professor Quirrell bog seinen Rücken durch und kam ohne die Hände zur Hilfe zu nehmen mit einer merkwürdigen, federnden Bewegung wieder auf die Beine.

Im Klassenzimmer herrschte nun Stille. Stille, die vollkommener Verwirrung entsprang.

„Mr~Goyle“, sagte Professor Quirrell, „welche äußerst wichtige Fähigkeit habe ich demonstriert?“

„Wie man richtig fällt, wenn man zu Boden geworfen wird“, sagte Mr~Goyle. „Das ist eine der ersten Sachen, die man lernt—“

„Das auch“, sagte Professor Quirrell.

Einen Moment lang war es still.

„Die äußerst wichtige Fähigkeit, die ich demonstriert habe“, sagte Professor Quirrell, „war das Verlieren. Sie dürfen gehen, Mr~Goyle, vielen Dank.“

Mr~Goyle verließ die Bühne und sah recht verwirrt aus. Harry fühlte sich genau so.

Professor Quirrell ging zurück an seinen Tisch und stützte sich wieder darauf ab. „Manchmal vergessen wir die einfachsten Dinge, weil es zu lange her ist, dass wir sie gelernt haben. Mir wurde klar, dass ich das selbe in meinem Lehrplan getan habe. Man bringt Schülern erst dann das Werfen bei, wenn sie gelernt haben, zu fallen. Und ich kann euch erst dann das Kämpfen beibringen, wenn ihr gelernt habt, zu verlieren.“

Professor Quirrells Gesicht verhärtete sich und Harry glaubte, eine Spur von Schmerz, einen Anflug der Sorge in diesen Augen zu sehen. „Ich habe gelernt zu verlieren, als ich in einem Dojo in Asien trainiert habe, wo, wie jeder Muggel weiß, alle guten Kampfkünstler leben. Dieses Dojo lehrte einen Stil, der unter Kampfmagiern den Ruf genoss, dass er sich gut für magische Duelle abwandeln lässt. Der Meister jenes Dojos—für Muggel-Verhältnisse ein alter Mann—war der beste lebende Lehrer jenes Stils. Er wusste natürlich nicht, dass Magie existiert. Ich bewarb mich, um dort zu lernen, und war einer der wenigen Schüler, die in jenem Jahr aus vielen Bewerbern ausgewählt wurden. Dabei mag ein winziges bisschen \emph{besondere Überzeugungskraft} eine Rolle gespielt haben.“

Einiges Gelächter ertönte im Klassenzimmer. Harry lachte nicht mit. Das war nicht richtig gewesen.

„Wie dem auch sei. In einem meiner ersten Kämpfe, nachdem ich auf eine besonders beschämende Weise besiegt worden war, habe ich die Kontrolle verloren und meinen Trainingspartner attackiert—“

\emph{Oje …}

„— zum Glück mit den Fäusten, nicht mit meiner Magie. Der Meister hat mich überraschenderweise nicht sofort rausgeschmissen. Aber er sagte mir, dass mein Temperament ein Problem sei. Er erklärte es mir und ich wusste, dass er Recht hatte. Und dann sagte er, dass ich lernen würde, zu verlieren.“

Professor Quirrells Gesicht war ausdruckslos.

„Auf seine strenge Anweisung hin stellten sich alle Schüler des Dojos der Reihe nach auf. Einer nach dem anderen kamen sie auf mich zu. Ich sollte mich \emph{nicht} verteidigen. Ich sollte nur um Gnade flehen. Einer nach dem anderen schlugen sie mich, traten mich, schubsten mich zu Boden. Einige spuckten mich an. Sie beleidigten mich in ihrer Sprache. Und zu jedem einzelnen musste ich sagen, ‚Ich verliere!`, und ähnliche Sachen, wie ‚Ich bitte dich, hör auf!` und ‚Ich gebe zu, du bist besser als ich!`\,“

Harry versuchte, sich das vorzustellen, und scheiterte. Es war undenkbar, dass dem würdevollen Professor Quirrell so etwas zugestoßen war.

„Bereits damals war ich ein exzellenter Kampfmagier. Mit stabloser Magie alleine hätte ich jeden einzigen Menschen in diesem Dojo töten können. Ich habe es nicht getan. Ich habe gelernt, zu verlieren. Bis zum heutigen Tag ist es für mich eine der unangenehmsten Stunden meines Lebens. Und als ich das Dojo acht Monate später verließ—was noch viel zu früh war, doch länger konnte ich nicht bleiben—sagte der Meister mir, dass er hoffte, ich würde verstehen, warum das nötig gewesen war. Und ich sagte ihm, dass es eine der wertvollsten Lektionen war, die ich je gelernt habe. Das ist und bleibt die Wahrheit.“

Professor Quirrells Gesicht nahm einen verbitterten Ausdruck an. „Ihr fragt euch nun, wo dieses wundervolle Dojo ist und ob ihr dort lernen könnt. Nicht mehr. Denn nicht viel später besuchte ein weiterer Möchtegern-Schüler jenen versteckten Ort, jenen weit entfernten Berg. Er, dessen Name nicht genannt werden darf.“

Viele Schüler schnappten gleichzeitig nach Luft. Harrys Eingeweide verknoteten sich. Er ahnte, was nun geschah.

„Der Dunkle Lord betrat die Schule ohne Verkleidung, mit glühenden roten Augen und allem Drum und Dran. Die Schüler versuchten, ihm den Weg zu versperren, und er apparierte einfach an ihnen vorbei. Es herrschte Furcht, aber Ordnung, und der Meister trat hervor. Und der Dunkle Lord forderte—er bat nicht, sondern forderte —, gelehrt zu werden.“

Professor Quirrells Gesicht war sehr hart. „Vielleicht hatte der Meister zu viele Bücher gelesen, die die Lüge verbreiteten, dass ein wahrer Meister sogar Dämonen besiegen könne. Aus irgendeinem Grund weigerte der Meister sich. Der Dunkle Lord fragte, warum er kein Schüler sein könne. Der Meister sagte ihm, dass er keine Geduld habe, und dann riss der Dunkle Lord ihm die Zunge raus.“

Die Schüler keuchten auf.

„Ihr könnt erraten, was dann geschah. Die Schüler stürzten sich auf den Dunklen Lord und wurden umgeworfen, erstarrten an Ort und Stelle. Und dann …“

Professor Quirrells Stimme setzte einen Moment lang aus, dann sprach er weiter.

„Es gibt einen unverzeihlichen Fluch, den Cruciatus-Fluch, der unerträgliche Schmerzen zufügt. Wem der Cruciatus länger als ein paar Minuten zugefügt wird, der verliert für immer den Verstand. Der Dunkle Lord hat die Schüler, einen nach dem anderen, in den Wahnsinn gefoltert und sie dann mit dem Todesfluch umgebracht, während der Meister zusehen musste. Nachdem alle seine Schüler auf diese Weise gestorben waren, erging es dem Meister ebenso. Ich erfuhr dies durch den einzigen überlebenden Schüler, den der Dunkle Lord am Leben gelassen hatte, um die Geschichte zu erzählen, und der ein Freund von mir gewesen war …“

Professor Quirrell drehte sich weg und als er sich einen Moment später wieder zurück drehte, wirkte er wieder ruhig und gefasst.

„Dunkle Zauberer haben ihr Temperament nicht unter Kontrolle“, sagte Professor Quirrell leise. „Es ist ein fast allgegenwärtiger Fehler ihrer Art und jeder, der sie häufiger bekämpft, lernt bald, sich darauf zu verlassen. Versteht, dass der Dunkle Lord an jenem Tag \emph{nicht} gewonnen hat. Sein Ziel war, Kampfkunst zu lernen, doch er ging ohne eine einzige Unterrichtsstunde. Es war töricht vom Dunklen Lord, diese Geschichte weitererzählen zu lassen. Sie demonstriert nicht seine Stärke, sondern stattdessen eine Schwäche, die sich ausnutzen lässt.“

Professor Quirrells Blick richtete sich auf ein einziges Kind im Klassenzimmer.

„Harry Potter“, sagte Professor Quirrell.

„Ja“, sagte Harry mit rauer Stimme.

„Was \emph{genau} haben Sie heute falsch gemacht, Mr~Potter?“

Harry war speiübel. „Ich hatte mich nicht unter Kontrolle.“

„Das ist \emph{nicht} genau“, sagte Professor Quirrell. „Ich werde es präziser beschreiben. Viele Tierarten führen sogenannte Dominanzkämpfe durch. Sie stürmen mit gesenkten Hörnern aufeinander zu—um einander zu Boden zu werfen, nicht aufzuschlitzen. Sie schlagen einander mit ihren Pfoten—mit verborgenen Krallen. Aber warum fahren sie die Krallen nicht aus? Sicher hätten sie höhere Chancen, wenn sie die Krallen benutzen würden? Doch dann könnte ihr Gegner ebenso die Krallen ausfahren und der Dominanzkampf würde nicht mit einem Gewinner und einem Verlierer, sondern womöglich mit schweren Verletzungen auf beiden Seiten enden.“

Professor Quirrells Blick schien sich aus dem Bildschirm heraus direkt auf Harry zu richten. „Was Sie heute demonstriert haben, Mr~Potter, ist, dass Sie—im Gegensatz zu jenen Tieren, die ihre Krallen nicht ausfahren und das Ergebnis akzeptieren—nicht wissen, wie man einen Dominanzkampf verliert. Als ein \emph{Hogwarts-Lehrer} Sie herausgefordert hat, haben Sie nicht nachgegeben. Als Sie zu verlieren drohten, haben Sie ihre Krallen ausgefahren, ungeachtet der Gefahr. Sie haben die Situation \emph{eskaliert} und \emph{nochmal} eskaliert. Es begann mit einem Angriff auf Sie durch Professor Snape, der offensichtlich dominant war. Statt zu verlieren, haben Sie zurückgeschlagen und Ravenclaw wurden zehn Punkte abgezogen. Kurz darauf haben Sie damit gedroht, Hogwarts zu verlassen. Die Tatsache, dass Sie auf irgendeine unbekannte Weise weiter eskaliert und schließlich irgendwie gewonnen haben, ändert nichts an der Tatsache, dass Sie ein Idiot sind.“

„Ich verstehe“, sagte Harry. Seine Kehle war trocken. Das \emph{war} präzise. \emph{Erschreckend} präzise. Jetzt, da Professor Quirrell es gesagt hatte, konnte Harry im Nachhinein erkennen, dass es eine \emph{vollkommen} genaue Beschreibung des Geschehens war. Wenn jemand ein so genaues Bild von dir hatte, dann musste dich das zum Nachdenken darüber bringen, ob diese Person auch in anderen Punkten Recht hatte. Etwa in Bezug auf deine Absicht, zu töten.

„Wenn Sie, Mr~Potter, das \emph{nächste} Mal einen Kampf eskalieren, statt zu verlieren, dann könnten Sie \emph{alles} verlieren. Ich weiß nicht, was ihr heutiger Einsatz war. Ich vermute, dass er viel, viel zu hoch war für den Verlust von zehn Hauspunkten.“

Das Schicksal der britischen Zaubererwelt. Das war sein Einsatz gewesen.

„Sie werden einwenden, dass Sie versuchten, ganz Hogwarts zu helfen; dass dieses viel wichtigere Ziel viel größere Risiken rechtfertigt. Das ist eine \emph{Lüge}. Falls Sie das tatsächlich—“

„— dann hätte ich den Angriff ertragen, abgewartet und den bestmöglichen Zeitpunkt für meine Reaktion gewählt“, sagte Harry mit heiserer Stimme. „Doch dann hätte ich \emph{verloren}. Hätte seine Dominanz akzeptiert. Genau das, was der Dunkle Lord gegenüber dem Meister, von dem er lernen wollte, nicht konnte.“

Professor Quirrell nickte. „Ich sehe, dass Sie es vollkommen verstanden haben. Daher, Mr~Potter, werden Sie heute lernen, zu verlieren.“

„Ich—“

„Ich will keine Widersprüche hören, Mr~Potter. Es ist offensichtlich, dass Sie dies dringend benötigen und dass Sie stark genug sind, es zu ertragen. Ich versichere Ihnen, dass ihre Erfahrungen nicht so unangenehm werden wie meine, obwohl es durchaus sein kann, dass Sie es als die schlimmsten fünfzehn Minuten ihres jungen Lebens im Gedächtnis behalten werden.“

Harry schluckte. „Professor Quirrell“, sagte er in einer leisen Stimme, „können wir das ein andermal tun?“

„Nein“, sagte Professor Quirrell schlicht. „Sie sind seit fünf Tagen auf Hogwarts und es ist bereits geschehen. Heute ist Freitag. Unsere nächste Unterrichtsstunde ist am Mittwoch. Samstag, Sonntag, Montag, Dienstag, Mittwoch…Nein, wir können \emph{nicht} so lange warten.“

Einige Schüler lachten darüber, doch nur sehr wenige.

„Bitte betrachten Sie es als Befehl Ihres Lehrers, Mr~Potter. Ich würde gerne sagen, dass ich Ihnen anderenfalls keine Angriffszauber beibringen werde, weil mir sonst zu Ohren käme, dass sie jemanden ernsthaft verletzt oder gar getötet hätten. Unglücklicherweise wurde mir berichtet, dass Ihre Finger bereits mächtige Waffen sind. Bitte schnippen Sie diese nicht während dieser Unterrichtsstunde.“

Vereinzelt ertönte nervöses Gelächter.

Harry hatte das Gefühl, dass er bald weinen müsste. „Professor Quirrell, wenn Sie etwas vorhaben wie das, worüber Sie geredet haben, dann wird es mich wütend machen und ich würde heute wirklich lieber nicht noch einmal wütend werden—“

„Es geht \emph{nicht} darum, dass Sie nicht wütend werden“, sagte Professor Quirrell mit ernstem Gesichtsausdruck. „Wut ist natürlich. Sie müssen lernen, zu verlieren, auch wenn Sie wütend sind. Oder zumindest \emph{scheinbar} zu verlieren, damit Sie Ihre Rache \emph{planen} können. Wie ich es heute mit Mr~Goyle getan habe; es sei denn, jemand hier denkt, dass er \emph{tatsächlich} besser ist—“

„Das bin ich nicht!“, rief Mr~Goyle von seinem Platz aus. Er klang leicht panisch. „Ich weiß, dass Sie nicht wirklich verloren haben! Bitte planen Sie keine Rache!“

Harrys Magen rumorte. Professor Quirrell wusste nichts von seiner mysteriösen dunklen Seite. „Professor, wir müssen dringend nach dem Unterricht darüber reden—“

„Das werden wir“, sagte Professor Quirrell und es klang wie ein Versprechen. „Nachdem Sie gelernt haben, zu verlieren.“ Sein Gesichtsausdruck war ernst. „Es versteht sich von selbst, dass ich alles untersagen werde, was Ihnen Verletzungen oder auch nur nennenswerte Schmerzen zufügen könnte. Es wird nur deshalb wehtun, weil sie verlieren werden, statt dagegen anzukämpfen und den Kampf solange zu eskalieren, bis Sie gewinnen.“

Harry atmete in kurzen, panischen Stößen. Er hatte soviel Angst, wie er seit dem Verlassen des Zaubertränke-Klassenzimmers nicht gehabt hatte. „Professor Quirrell“, stieß er aus, „ich will nicht, dass Sie deswegen gefeuert werden—“

„Das werde ich nicht“, sagte Professor Quirrell, „falls \emph{Sie} anschließend aussagen, dass es notwendig war. Und ich vertraue darauf, dass Sie das tun.“ Einen Moment lang wurde Professor Quirrells Tonfall sehr trocken. „Glauben Sie mir, die haben schon schlimmere Dinge in diesem Schloss zugelassen. Das einzig Ungewöhnliche hieran ist, dass es sich in einem Klassenzimmer abspielt.“

„Professor Quirrell“, flüsterte Harry, aber er dachte, dass seine Stimme immer noch überall wiedergegeben wurde, „glauben Sie wirklich, dass ich, falls ich das nicht mache, jemanden verletzen könnte?“

„Ja“, sagte Professor Quirrell schlicht.

„Dann“, Harry fühlte sich übel, „werde ich es tun.“

Professor Quirrell wandte sich den Slytherins zu. „Also…mit vollem Einverständnis eures Lehrers und in einem Kontext, in dem niemand Snape die Schuld für euer Handeln geben kann…wer von euch will den Jungen, der lebt von seinem hohen Ross runterholen? Ihn umherschubsen, zu Boden werfen? Ihn um Gnade betteln hören?“

Fünf Hände schossen in die Höhe.

„Wer jetzt seine Hand hebt: Ihr seid vollkommene Idioten. Welchen Teil von \emph{scheinbar} verlieren habt ihr nicht verstanden? Wenn Harry Potter der nächste Dunkle Lord wird, dann wird er euch nach Ende seiner Schulzeit jagen und töten.“

Die fünf Hände fielen schlagartig wieder auf ihre Tische zurück.

„Das werde ich nicht“, sagte Harry, dessen Stimme nun recht schwach klang. „Ich schwöre, dass ich mich niemals an denen rächen werde, die mir dabei helfen, verlieren zu lernen. Professor Quirrell…würden Sie damit \emph{bitte…aufhören}?“

Professor Quirrell seufzte. „Es tut mir leid, Mr~Potter. Mir ist bewusst, dass Sie das nervig finden, egal ob Sie planen, ein Dunkler Lord zu werden, oder nicht. Aber diese Kinder hatten \emph{auch} eine wichtige Lektion zu lernen. Wäre es in Ordnung, wenn ich Ihnen zur Entschuldigung einen Quirrell-Punkt anbiete?“

„Machen Sie zwei draus“, sagte Harry.

Überraschtes Gelächter erklang und zerstreute die Anspannung ein wenig.

„Einverstanden“, sagte Professor Quirrell.

„Und nach Ende meiner Schulzeit werde ich Sie jagen und \emph{kitzeln}.“

Wieder erklang Gelächter, doch Professor Quirrell lächelte nicht.

Harry fühlte sich, als würde er mit einer Anakonda ringen. Er versuchte, die Unterhaltung auf den schmalen Pfad zu bringen, der deutlich machte, dass er wirklich kein Dunkler Lord werden wollte…\emph{Warum} war Professor Quirrell bloß so misstrauisch?

„Professor“, sagte Dracos nicht verstärkte Stimme. „Auch ich habe nicht das Ziel, ein dummer Dunkler Lord zu werden.“

Im Klassenzimmer war es auf einen Schlag still.

\emph{Das brauchst du nicht tun!}, hätte Harry fast laut ausgerufen, hielt sich aber gerade noch zurück. Draco wollte womöglich nicht, dass bekannt wurde, dass er das aus Freundschaft zu Harry tat…oder um Freundschaft vorzutäuschen …

Bei der Unterstellung, dass \emph{das} nur \emph{vorgetäuscht} war, fühlte Harry sich kleingeistig und unfair. Wenn Draco beabsichtigt hatte, ihn zu beeindrucken, dann war es ihm vollkommen gelungen.

Professor Quirrell betrachtete Draco ernst. „\emph{Sie} machen sich Sorgen, dass Sie nicht so tun könnten, als ob Sie verlieren, Mr~Malfoy? Dass jener Fehler, den Mr~Potter hat, auch Sie betrifft? \emph{Sicher} hat Ihr Vater Ihnen das beigebracht?“

„Mit Worten, vielleicht“, sagte Draco. Nun kam seine Stimme aus dem Bildschirm auf jedem Tisch. „Aber nicht, wenn ich rumgestoßen und zu Boden geschubst werde. Ich möchte genau so stark sein wie Sie, Professor Quirrell.“

Professor Quirrells Augenbrauen zuckten empor und blieben dort. „Ich fürchte, Mr~Malfoy“, sagte er nach einer Weile, „dass die Vorbereitungen, die ich für Mr~Potter getroffen habe—einige ältere Slytherins, denen erst \emph{nachher} mitgeteilt wird, wie dumm sie waren—nicht auf Sie übertragbar sind. Doch ich bleibe bei meiner Einschätzung, dass Sie bereits sehr stark sind. Sollte mir zu Ohren kommen, dass Sie gescheitert sind, so wie Mr~Potter heute gescheitert ist, dann werde ich die notwendigen Vorbereitungen treffen und Sie, sowie jene Personen, die Sie verletzt haben, um Entschuldigung bitten.“

„Ich verstehe, Professor“, sagte Draco.

Professor Quirrell blickte umher. „Möchte sonst jemand stark werden?“

Einige Schüler blickten sich nervös um. Manche, so erschien es Harry von der hintersten Sitzreihe aus, sahen aus, als würden sie ihren Mund öffnen, ohne etwas zu sagen. Letzlich meldete sich niemand.

„Draco Malfoy wird der General einer Armee in eurer Klassenstufe sein“, sagte Professor Quirrell, „sofern er gedenkt, an jener extracurriculären Aktivität teilzunehmen. Und nun, Mr~Potter, kommen Sie bitte vor.“

\later

\emph{Ja}, hatte Professor Quirrell gesagt, \emph{es muss vor allen Schülern sein, vor deinen Freunden, denn dort hat Snape dich konfrontiert und dort musst du lernen, zu verlieren.}

Also sahen die Erstklässler zu. In magisch durchgesetzter Stille und mit der Bitte von Harry und von Professor Quirrell, nicht einzuschreiten. Hermine hatte ihr Gesicht abgewandt, doch sie hatte nichts gesagt und ihm keinen Blick zugeworfen; vielleicht deswegen, weil auch sie im Zaubertränke-Unterricht dabei gewesen war.

Harry stand auf einer weichen blauen Matte, wie man sie in einem Muggel-Dojo finden könnte. Professor Quirrell hatte sie auf den Boden gelegt, falls Harry umgeschubst würde.

Harry hatte Angst vor dem, was er tun könnte. Wenn Professor Quirrell mit seiner \emph{Absicht, zu töten,} Recht hatte …

Harrys Zauberstab lag auf Professor Quirrells Tisch. Nicht, weil er irgendwelche Zaubersprüche beherrschte, die ihm helfen könnten, sondern (so dachte Harry) weil er womöglich versucht hätte, ihn jemandem ins Auge zu rammen. Sein Beutel lag ebenfalls dort, mitsamt dem nun geschützten, aber womöglich immer noch zerbrechlichen Zeitumkehrer.

Harry hatte Professor Quirrell gebeten, ein Paar Boxhandschuhe zu verwandeln und sie ihm an den Händen festzuzaubern. Professor Quirrell hatte ihm einen stummen, verständnisvollen Blick zugeworfen und sich geweigert.

\emph{Ich habe es nicht auf die Augen abgesehen, ich habe es nicht auf die Augen abgesehen, ich habe es nicht auf die Augen abgesehen, es wäre das Ende meiner Zeit auf Hogwarts, sie würden mich einsperren}, redete Harry sich selbst ein. Er versuchte, sich den Gedanken ins Gehirn zu prägen; hoffte, dass er daran dachte, falls seine Absicht, zu töten, die Oberhand gewann.

Professor Quirrell kehrte mit dreizehn älteren Slytherins aus verschiedenen Schuljahren zurück. Harry erkannte den einen wieder, den er mit einem Kuchen beworfen hatte. Zwei andere von damals waren auch dabei. Der eine, der gesagt hatte, dass sie aufhören sollten, dass sie das wirklich nicht tun sollten, fehlte.

„Ich wiederhole“, sagte Professor Quirrell und klang sehr ernst, „Potter wird \emph{nicht} tatsächlich verletzt. Ich behandle jegliche \emph{Unfälle} so, als wäre es Absicht gewesen. Habt ihr das verstanden?“

Die älteren Slytherins nickten grinsend.

„Dann nutzt die Gelegenheit, dem Jungen, der lebt mal eins auf den Deckel zu geben“, sagte Professor Quirrell mit einem schiefen Lächeln, das nur die Erstklässler verstanden.

Durch eine Art stille Übereinkunft stand der vom Kuchen Getroffene ganz vorne in der Gruppe.

„Potter“, sagte Professor Quirrell, „das ist Mr~Peregrine Derrick. Er ist besser als du und das wird er dir jetzt zeigen.“

Derrick schritt voran und in Harrys Kopf riefen viele Stimmen durcheinander: Er durfte nicht wegrennen, er durfte sich nicht wehren—

Derrick blieb eine Armlänge vor Harry stehen.

Harry war noch nicht wütend, nur verängstigt. Und er blickte auf einen Teenager, der einen halben Meter größer war als er selbst, mit deutlich sichtbaren Muskeln, einem Bart und einem schrecklich erwartungsfrohen Grinsen.

„Bitte ihn, dir nicht wehzutun“, sagte Professor Quirrell. „Wenn du jämmerlich genug wirkst, langweilst du ihn vielleicht und er geht weg.“

Die älteren Slytherins, die zusahen, lachten.

„Bitte“, sagte Harry mit aussetzender Stimme, „tu, mir, nicht, weh …“

„Das klang nicht sehr aufrichtig“, sagte Professor Quirrell.

Derricks Lächeln wurde breiter. Der plumpe Schwachkopf blickte plötzlich sehr überlegen drein und …

… Harrys Blut wurde kälter …

„Bitte tu mir nicht weh“, versuchte Harry es erneut.

Professor Quirrell schüttelte den Kopf. „Bei Merlin, wie hast du es geschafft, dass wie eine Beleidigung klingen zu lassen, Potter? Du weißt genau, wie Mr~Derrick darauf reagieren muss.

Derrick machte einen Schritt nach vorne und stieß gegen Harry.

Harry taumelte ein paar Schritte zurück und richtete sich kühl wieder auf, ohne darüber nachzudenken.

„Falsch“, sagte Professor Quirrell, „falsch, falsch, falsch.“

„Du hast mich geschubst, Potter“, sagte Derrick. „Entschuldige dich.“

„Es tut mir leid!“

„Das \emph{klingt} nicht so, als ob es dir Leid tut“, sagte Derrick.

Harrys Augen weiteten sich vor Entrüstung, er \emph{hatte} es geschafft, flehend zu klingen—

Derrick schubste ihn kräftig und Harry fiel auf allen Vieren auf die Matte.

Der blaue Stoff schien dicht vor Harrys Augen zu verschwimmen.

Er begann, an Professor Quirrells Begründung für diese sogenannte \emph{Lektion} zu zweifeln.

Ein Fuß wurde auf Harrys Hintern gesetzt und einen Moment später wurde Harry heftig zur Seite bugsiert und landete platt auf dem Rücken.

Derrick lachte. „Das macht \emph{Spaß}“, sagte er.

Er müsste nur sagen, dass es vorbei ist. Und das alles dem Schulleiter melden. Das wäre das Ende dieses \emph{Verteidigungslehrers} und seiner kurzen Zeit als Hogwarts-Lehrer und…Professor McGonagall wäre deswegen verärgert, aber …

(Professor McGonagalls Gesicht erschien vor seinen Augen, sie sah nicht verärgert aus, nur traurig —)

„Jetzt sag ihm, dass er besser ist als du, Potter“, ertönte Professor Quirrells Stimme.

„Du, bist, besser, als, ich.“

Harry begann aufzustehen und Derrick setzte einen Fuß auf seine Brust und drückte ihn wieder runter auf die Matte.

Die Welt wurde kristallklar. Handlungsstränge und ihre Folgen breiteten sich vor ihm aus. Der Dummkopf würde nicht erwarten, dass er zurückschlägt, ein schneller Tritt in die Weichteile würde ihn lange genug außer Gefecht setzen, um—

„Versuche es nochmal“, sagte Professor Quirrell und mit einer plötzlichen, raschen Bewegung rollte Harry weg, sprang auf die Füße und stürzte auf seinen wahren Feind, den Verteidigungslehrer zu—

Professor Quirrell sagte, „Du hast keine Geduld.“

Harry stockte. Sein Pessimismus-erfahrener Geist malte sich aus, wie einem hutzeligen alten Mann Blut aus dem Mund rann, nachdem Harry ihm die Zunge rausgerissen hatte—

Einen Augenblick später schubste Derrick Harry wieder auf die Matte und setzte sich auf ihn, so dass Harry der Atem keuchend entwich.

„Hör auf!“, schrie Harry. „Hör bitte auf!“

„Besser“, sagte Professor Quirrell. „Das klang so, als ob du es ernst meinst.“

Er \emph{hatte} es ernst gemeint. Das war die fürchterliche, die schreckliche Wahrheit, er \emph{hatte} es ernst gemeint. Harry keuchte hektisch, Angst und eiskalter Zorn durchströmten ihn—

„Verliere“, sagte Professor Quirrell.

„Ich, verliere“, zwang Harry hervor.

„Das gefällt mir“, sagte Derrick von oben. „Verliere weiter.“

\later

Hände schubsten Harry, schickten ihn im Kreis der älteren Slytherins umher zu einem anderen Paar Hände, die ihn weiter schubsten. Harry hatte den Versuch, seine Tränen zurückzuhalten, längst aufgegeben und versuchte jetzt nur noch, nicht hinzufallen.

„Was bist du, Potter?“, sagte Derrick.

„Ein, V-verlierer, ich verliere, ich gebe auf, du hast gewonnen, du bist b-besser, als ich, bitte hör auf—“

Harry stolperte über einen Fuß und stürzte zu Boden, es gelang ihm nicht ganz, sich mit den Händen abzufangen. Einen Moment lang war er benommen, dann bemühte er sich, wieder auf die Beine zu kommen—

„\emph{Genug!}“, ertönte Professor Quirrells Stimme, nun scharf genug, dass sie Eisen schneiden könnte. „Lassen Sie Mr~Potter in Ruhe!“

Harry sah die überraschten Blicke auf ihren Gesichtern. Die Kälte in seinem Blut, die sich wieder und wieder aufgebäumt hatte, lächelte zufrieden.

Dann brach Harry auf der Matte zusammen.

Professor Quirrell sprach. Die älteren Slytherins keuchten auf.

„Und ich glaube, dass der Spross der Malfoys euch ebenfalls etwas erklären möchte“, schloss Professor Quirrell.

Dracos Stimme begann zu reden. Sie klang fast so scharf wie die von Professor Quirrell, nahm den gleichen Tonfall an, den Draco genutzt hatte, um seinen Vater zu imitieren, und sagte Dinge wie \emph{hättet das Haus Slytherin in die Bredouille bringen können} und \emph{wer weiß wie viele Verbündete alleine an dieser Schule} und \emph{vollkommen miserable Situationsbewertung, von Gerissenheit ganz zu schweigen} und \emph{tumbe Schlägertypen, höchstens als Lakaien zu gebrauchen} und irgendetwas in Harrys Hinterkopf stufte Harry als Verbündeten ein, trotz allem, was er sonst noch wusste.

Harry tat alles weh, er hatte vermutlich viele blaue Flecken bekommen, sein Körper war kalt, sein Kopf völlig ausgelaugt. Er versuchte, an Fawkes Gesang zu denken, doch solange der Phönix nicht da war, fiel ihm die Melodie nicht ein und als er versuchte, sie sich vorzustellen, kam ihm bloß Vogelgezwitscher in den Sinn.

Dann hörte Draco auf zu reden und Professor Quirrell teilte den älteren Slytherins mit, dass sie gehen konnten, und Harry öffnete die Augen und setzte sich mühevoll auf. „Warten Sie“, zwang Harry die Worte heraus, „es gibt noch etwas, dass ich, ihnen, sagen will—“

„Hört auf Mr~Potter“, sagte Professor Quirrell kühl zu den gehenden Slytherins.

Harry stand schwankend auf. Er achtete darauf, nicht zu seinen Mitschülern zu sehen. Er wollte nicht sehen, wie sie ihn nun anblickten. Er wollte ihr Mitleid nicht sehen.

Also sah Harry stattdessen die älteren Slytherins an, die sich immer noch in einem Schockzustand befanden. Sie starrten zurück. Furcht stand ihnen in die Gesichter geschrieben.

Seine dunkle Seite hatte sich, solange sie die Kontrolle innegehabt hatte, diesen Moment fest vorgestellt, und hatte weiter vorgetäuscht, zu verlieren.

Harry sagte, „Niemand wird—“

„Stopp“, sagte Professor Quirrell. „Wenn es das ist, was ich vermute, dann warten Sie bitte, bis die gegangen sind. Sie werden später davon hören. Wir haben alle eine Lektion zu lernen, Mr~Potter.“

„In Ordnung“, sagte Harry.

„Ihr. Geht.“

Die älteren Slytherins eilten raus und die Tür schloss sich hinter ihnen.

„Niemand soll sich an ihnen rächen“, sagte Harry heiser. „Das ist eine Bitte an alle, die sich als meine Freunde betrachten. Ich hatte eine Lektion zu lernen, sie haben mir dabei geholfen, sie hatten auch eine Lektion zu lernen, damit ist es abgeschlossen. Wenn ihr diese Geschichte weitererzählt, dann erzählt bitte auch diesen Teil weiter.“

Harry wandte sich zu Professor Quirrell.

„Du hast verloren“, sagte Professor Quirrell. Seine Stimme war zum ersten Mal sanft. Es klangt seltsam, wenn der Professor das tat, als sollte seine Stimme gar nicht dazu fähig sein.

Harry \emph{hatte} verloren. In manchen Momenten war die kalte Wut vollkommen verschwunden, von Angst verdrängt, und während jener Momente hatte er die älteren Slytherins angefleht und er hatte es ernst gemeint …

„Steht die Sonne noch am Himmel?“, sagte Professor Quirrell, immer noch so merkwürdig sanft. „Scheint sie noch? Bist Du noch am Leben?“

Harry gelang es zu nicken.

„Nicht jede Niederlage fühlt sich so an“, sagte Professor Quirrell. „Es gibt Kompromisse und ausgehandelte Kapitulationen. Es gibt andere Mittel, Angreifer zu beschwichtigen. Es ist eine eigene Kunstform, andere zu manipulieren, indem man ihnen die dominante Rolle überlässt. Aber zuerst muss das Verlieren \emph{denkbar} sein. Wirst Du Dich daran erinnern, wie Du verloren hast?“

„Ja.“

„Wirst Du verlieren können?“

„Ich…glaube schon …“

„Ich glaube auch.“ Professor Quirrell verbeugte sich so tief, dass sein dünnes Haar fast den Boden berührte. „Herzlichen Glückwunsch, Harry Potter, Du hast gewonnen.“

Niemand machte den Anfang, niemand zog die anderen mit; der Applaus brach auf einmal aus, wie ein lauter Donnerschlag.

Harry konnte den Schock nicht von seinem Gesicht verdrängen. Er wagte einen Blick zu seinen Mitschülern und sah, dass ihre Blicke nicht von Mitleid sondern von Bewunderung geprägt waren. Der Applaus kam von Ravenclaw und Gryffindor und Hufflepuff und sogar von Slytherin, vermutlich weil Draco Malfoy auch applaudierte. Einige Schüler standen von ihren Stühlen auf und halb Gryffindor stand auf den Tischen.

Also stand Harry dort, schwankte, ließ den Applaus über sich waschen, wurde stärker und vielleicht sogar ein klein wenig geheilt.

Professor Quirrell wartete, bis der Applaus nachgelassen hatte. Es dauerte eine ganze Weile.

„Überrascht, Mr~Potter?“, sagte Professor Quirrell. Seine Stimme klang amüsiert. „Sie haben soeben herausgefunden, dass es in der richtigen Welt nicht \emph{immer} so läuft wie in Ihren schlimmsten Albträumen. Ja, wenn Sie irgendein armer, unbekannter Junge wären, der gepeinigt wird, dann hätten Ihre Mitschüler Sie danach vermutlich weniger respektiert, bemitleidet sogar, und Ihnen von einem hohen Ross herab Trost zugesprochen. Das \emph{ist} menschlich, fürchte ich. Aber \emph{Sie} waren längst als mächtige Figur bekannt. Ihre Mitschüler sahen, wie Sie sich Ihrer Furcht entgegengestellt und diese weiter konfrontiert haben, obwohl Sie jederzeit einen Ausweg hatten. Hielten \emph{Sie} weniger von mir, als ich Ihnen erzählt habe, dass ich mich mit voller Absicht bespucken ließ?“

Harry spürte ein brennendes Gefühl in seiner Kehle und unterdrückte es hastig. Er traute diesem wundersamen Respekt noch nicht genug, um wieder vor allen Mitschülern zu weinen.

„Ihre \emph{außergewöhnliche} Leistung in meinem Unterricht verdient eine außergewöhnliche Belohnung, Harry Potter. Bitte nehmen Sie diese im Namen meines Hauses an und bedenken Sie ab heute, dass nicht alle Slytherins gleich sind. Es gibt Slytherins und es gibt Slytherins.“ Professor Quirrell trug ein breites Lächeln, während er das sagte. „Einundfünfzig Punkte für Ravenclaw.“

Vor Überraschung war es kurz still, dann brach ein Tumult unter den Ravenclaws aus; sie schrien und pfiffen und jubelten.

(Und im gleichen Moment hatte Harry das Gefühl, dass es \emph{falsch} war; Professor McGonagall hatte Recht gehabt, es \emph{sollte} Konsequenzen haben, er hätte einen Preis dafür zahlen müssen, er konnte die Punkte nicht einfach so wieder zurückbekommen —)

Doch Harry sah die frohen Gesichter der Ravenclaws und wusste, dass er unmöglich ablehnen konnte.

Sein Gehirn machte einen Vorschlag. Es war ein guter Vorschlag. Harry konnte einfach nicht glauben, dass sein Gehirn überhaupt noch irgendwas machte, von guten Vorschlägen ganz zu schweigen.

„Professor Quirrell“, sagte Harry, so deutlich seine brennende Kehle es zuließ. „Sie sind alles das, was ein Mitglied Ihres Hauses ausmacht und ich denke, dass Salazar Slytherin an jemanden wie Sie gedacht haben muss, als er Hogwarts mitgründete. Ich danke Ihnen und Ihrem Haus“, Draco nickt leicht und deutete ihm subtil mit der Hand an, \emph{mach weiter}, „und ich glaube, es ist Zeit für ein dreifaches Hoch auf Slytherin. Alle miteinander?“ Harry wartete kurz. „\emph{Slytherin lebe hoch!}“ Nur ein paar Schüler stimmten beim ersten Mal ein. „\emph{Hoch!}“ Dieses Mal stimmten die meisten Ravenclaws ein. „\emph{Hoch!}“ Jetzt waren es fast alle Ravenclaws, ein paar verstreute Hufflepuffs und etwa ein Viertel der Gryffindors.

Draco zeigte kurz mit dem Daumen nach oben.

Die meisten Slytherins hatten schockierte Gesichtsausdrücke. Einige starrten Professor Quirrell verwundert an. Blaise Zabini sah Harry mit einem berechnenden, faszinierten Ausdruck an.

Professor Quirrell verbeugte sich. „Ich danke \emph{Dir}, Harry Potter“, sagte er, immer noch mit jenem breiten Lächeln auf den Lippen. Er wandte sich zur Klasse. „Nun, ob ihr es glaubt oder nicht, eine halbe Stunde vom Unterricht ist noch übrig. Genug Zeit, um den Einfachen Schildzauber einzuführen. Mr~Potter wird natürlich gehen und sich eine wohlverdiente Pause gönnen.“

„Ich kann—“

„Dummkopf“, sagte Professor Quirrell sanft. Die Schüler lachten bereits. „Ihre Mitschülern können es Ihnen später beibringen, oder ich werde Ihnen Privatunterricht geben, falls das nötig sein sollte. \emph{Jetzt aber} werden Sie am Ende der Bühne durch die dritte Tür von links gehen, wo Sie ein Bett, eine Auswahl ungemein wohlschmeckender Kleinigkeiten sowie ein bisschen extrem leichte Lektüre aus der Hogwarts-Bibliothek finden werden. Sie werden nichts weiter mitnehmen, erst recht nicht ihre Schulbücher. Nun gehen Sie.“

Harry ging.

\later

\textbf{„Ihr fragt, was ist unser Ziel?“}

Harry zitiert hier den britischen Premierminister Winston Churchill, der bei seinem Amtsantritt im Frühjahr 1940 (wenige Monate nach Beginn des Zweiten Weltkriegs) \href{https://de.wikipedia.org/wiki/Blut-Schwei\%C3\%9F-und-Tr\%C3\%A4nen-Rede}{in einer berühmten Rede} schilderte, welche Opfer der Krieg kosten werde.

… und im nächsten Kapitel:

\emph{Professor Quirrell grinste. „Ich werde nun gegen Regel zwei verstoßen —~die schlicht ‚Prahle nicht` lautete~— und Ihnen von etwas erzählen, was ich getan habe. Ich denke nicht, dass dieses Wissen irgendeinen Schaden anrichten würde. Und ich vermute stark, dass Sie es ohnehin erraten hätten, sobald wir uns gut genug kennen.“}

