

\hypertarget{die-effizienz-des-marktes}{% \section{4. Die Effizienz des Marktes}\label{die-effizienz-des-marktes}}

Hier ist Kapitel vier. Übersetzt hat diesmal Alex, Martin und ich waren nur an der Überarbeitung beteiligt.

-\/-\/-\/-\/-\/-\/-\/-\/-\/-\/-\/-\/-\/-\/- ~ 4. Die Effizienz des Marktes ~ -\/-\/-\/-\/-\/-\/-\/-\/-\/-\/-\/-\/-\/-\/-

\emph{"Weltherrschaft ist so ein unschöner Ausdruck. Ich ziehe es vor, von Weltoptimierung zu sprechen."}

-\/-\/-

Haufenweise goldene Galleonen. Stapelweise silberne Sickel. Stoßweise bronzene Knuts.

Harry stand da und starrte mit offenem Mund sein Familienverlies an. Er hatte so viele Fragen --- er wusste gar nicht, wo er anfangen sollte.

Aus dem Türrahmen heraus beobachtete ihn McGonagall, die den Anschein erweckte, sich beiläufig gegen die Wand zu lehnen, aber ihn mit wachem Blick ansah. Kein Wunder. Vor einen riesigen Haufen an Goldmünzen gesetzt zu werden war so sehr Inbegriff eines Charaktertests, dass es geradezu klischeehaft war.

"Bestehen diese Münzen hier aus purem Gold?", fragte Harry schließlich.

"Was?", zischte der Kobold Griphook, der in der Nähe der Tür wartete. "Zweifeln Sie etwa an der Rechtschaffenheit der Gringotts-Bank, Mr~Potter-Evans-Verres?"

"Nein", sagte Harry gedankenverloren, "ganz und gar nicht --- es tut mir leid, wenn das falsch rüberkam, Sir. Ich habe bloß keine Ahnung, wie Ihr ganzes Finanzsystem funktioniert. Ich wollte wissen, ob Galleonen im Allgemeinen aus reinem Gold gemacht sind."

"Natürlich", sagte Griphook.

"Und kann sie jeder herstellen, oder gibt es ein Monopol darauf, durch das der Hersteller Münzprägegewinn einnimmt?"

"Was?", fragte McGonagall verständnislos.

Griphook grinste und entblößte dabei seine sehr scharfen Zähne. "Nur ein Trottel würde anderen Münzen als Koboldmünzen trauen!"

"Mit anderen Worten", sagte Harry, "sind die Münzen nicht mehr wert als das Metall, aus dem sie bestehen?"

Griphook starrte Harry an. McGonagall blickte verwirrt drein.

"Ich meine, nehmen wir mal an, ich käme hier mit einer Tonne Silber zur Tür herein. Könnte ich dafür eine Tonne Sickel bekommen?"

"Gegen eine Gebühr, Mr~Potter-Evans-Verres." Der Kobold sah ihn mit funkelnden Augen an. "Gegen eine bestimmte Gebühr. Ich frage mich, woher Sie wohl eine Tonne Silber nehmen wollen? Sicherlich würden Sie nicht… damit rechnen, einen Stein der Weisen in die Finger zu bekommen?"

"Griphook!", zischte McGonagall.

"Einen Stein der Weisen?", fragte Harry verdutzt.

"Dann wohl nicht", sagte der Kobold. Die Anspannung in seinem Körper schien ein wenig zu weichen.

"Die Frage war rein hypothetisch", sagte Harry. \emph{Für den Moment jedenfalls.} "Also… in welcher Höhe würden Sie die Gebühr ansetzen, als Bruchteil des gesamten Gewichts?"

Griphooks Blick war aufmerksam. "Ich müsste Auskunft bei meinen Vorgesetzten einholen…"

"Schätzen Sie einfach mal. Ich werde Gringotts gegenüber nicht darauf zurückkommen."

"Ein Zwanzigstel des Metalls wäre wohl ein angemessener Preis für die Münzprägung."

Harry nickte. "Vielen Dank, Mr~Griphook."

\emph{Also war die Zaubererwirtschaft nicht nur fast vollständig von der Muggelwirtschaft entkoppelt, man hatte hier nicht einmal von den Mechanismen des freien Marktes gehört.} Die Muggelwirtschaft hatte einen fluktuierenden Wechselkurs von Gold zu Silber. Jedes Mal, wenn dieser Kurs bei den Muggeln um mehr als fünf Prozent vom Gewichtsverhältnis von siebzehn Sickeln zu einer Galleone abwich, müsste der Zaubererwirtschaft entweder Gold oder Siber entzogen werden, bis es nicht mehr möglich war, den Wechselkurs beizubehalten. Man nehme eine Tonne Silber, tausche sie in Sickel um (und zahle fünf Prozent), wechsle die Sickel zu Galleonen, nehme das Gold mit in die Muggelwelt, tausche es zu mehr Silber um, als man am Anfang hatte, und wiederhole die Prozedur.

War der Gold-Silber-Wechselkurs der Muggel nicht ungefähr bei fünfzig zu eins? In jedem Fall betrug er nicht siebzehn zu eins, dachte Harry. Und die Silbermünzen schienen sogar \emph{kleiner} zu sein als die Goldmünzen.

Andererseits wiederum stand Harry in einer Bank, die ihr Geld \emph{tatsächlich} in Verliesen voller Goldmünzen aufbewahrte, bewacht von Drachen. Man musste hereingehen und Geldmünzen heraustragen, wann immer man Geld ausgeben wollte. Die feineren Gesichtspunkte des Ausnutzens von Marktineffizienzen waren ihnen wohl nicht bekannt. Harry war versucht, einen bissigen Kommentar darüber abzulassen, wie rückständig ihr Finanzsystem doch sei…

\emph{Doch traurigerweise war es vermutlich besser so.}

Andererseits könnte ein kompetenter Börsenmakler wahrscheinlich innerhalb von einer Woche die gesamte Zauberwelt kaufen. Harry nahm sich vor, dies im Hinterkopf zu behalten, falls ihm jemals das Geld ausging oder er eine Woche Zeit hatte.

In der Zwischenzeit sollten die gigantischen Goldhaufen im Potter-Verlies genügen, um seine kurzfristigen Bedürfnisse zu erfüllen.

Harry stolperte nach vorn und begann, Goldmünzen mit einer Hand aufzusammeln und sie in die andere zu legen. Als er bei zwanzig angekommen war, hüstelte McGonagall. "Ich denke, dies ist mehr als genug, um Ihre Schulsachen zu besorgen, Mr~Potter."

"Hm?", sagte Harry. "Einen Moment, ich führe gerade eine Fermi-Rechnung durch."

"Eine was?", fragte McGonagall, die etwas beunruhigt klang.

"Mathe. Benannt nach Enrico Fermi. Eine Methode, wie man sehr schnell Abschätzungen im Kopf ausrechnen kann…"

Zwanzig goldene Galleonen wogen vielleicht ein zehntel Kilo. Und Gold war, wieviel, zehntausend Britische Pfund pro Kilogramm wert? Also wäre eine Galleone ungefähr fünfzig Britische Pfund wert… Die Münzstapel sahen so aus, als seien sie ungefähr sechzig Münzen hoch, auf jeder Seite der Grundfläche zwanzig Münzen breit und pyramidenförmig, also wären sie ungefähr ein Drittel eines Quaders groß. Etwa achttausend Gallonen pro Haufen und es gab ungefähr fünf Haufen dieser Größe, also vierzigtausend Galleonen oder zwei Millionen Britische Pfund.

Nicht schlecht. Harry lächelte mit einer grimmigen Zufriedenheit. Was für ein Pech, dass er sich gerade mitten darin befand, die wunderbare neue Welt der Magie zu entdecken, und sich keine Zeit nehmen konnte, die wunderbare neue Welt des Reichseins auszukosten, die, einer schnellen Fermi-Abschätzung zufolge, ungefähr eine Milliarde mal weniger interessant war.

\emph{Trotzdem, ich werde nie wieder für ein lausiges Pfund Rasen mähen.}

Harry wandte sich von dem gigantischen Goldhaufen ab. "Entschuldigen Sie die Frage, Professor McGonagall, aber wenn ich es richtig verstanden habe, waren meine Eltern in den Zwanzigern, als sie gestorben sind. Ist dies eine normale Menge an Geld für ein junges Zaubererpaar?" Falls ja, dann kostete eine Tasse Kaffee hier wohl fünftausend Pfund. Regel Nummer eins der Wirtschaftslehre: Geld kann man nicht essen.

McGonagall schüttelte den Kopf. "Ihr Vater war der letzte Erbe einer alten Familie, Mr~Potter. Es kann auch sein…" McGonagall zögerte. "Ein Teil des Geldes könnte von Kopfgeldern stammen, die auf Ihn, dessen Name nicht genannt werden darf, ausgesetzt waren --- zu zahlen an seinen Mör-" McGonagall verschluckte das Wort. "An den, der ihn besiegen würde. Oder diese Kopfgelder wurden noch nicht eingesammelt. Ich bin mir nicht sicher."

"Interessant…", sagte Harry langsam. "Also gehört einiges davon mir. In einem gewissen Sinne. Von mir verdient, meine ich. Sozusagen. Auch wenn ich mich an das Ereignis nicht mehr erinnern kann." Harrys Finger klopften gegen sein Hosenbein. "Jetzt fühle ich mich weniger schuldig, einen \emph{sehr kleinen Bruchteil davon auszugeben! Keine Panik, Professor McGonagall!}"

"Mr~Potter! Sie sind minderjährig und als Minderjähriger dürfen Sie ausschließlich vernünftige Abhebungen von -"

"Ich bin vollkommen vernünftig! Ich stimme Ihnen in Sachen finanzpolitische Vorsicht und Impulskontrolle absolut zu! Aber ich habe auf dem Weg hierher einige Dinge gesehen, die \emph{sinnvolle, erwachsene} Einkäufe darstellen würden…"

Harry und McGonagall sahen einander in einer Art stillem Starr-Wettkampf fest in die Augen.

"Zum Beispiel?", sagte McGonagall schließlich.

„Koffer die innen größer sind als außen?“

McGonagalls Gesicht wurde ernst. "Solche Koffer sind \emph{sehr} teuer, Mr~Potter!"

"Ja, aber", wandte Harry ein, "ich bin mir sicher, dass ich so einen haben will, wenn ich erwachsen bin. Und ich kann mir einen leisten. Es ist doch dann genauso sinnvoll, jetzt einen zu kaufen statt später, und ihn unmittelbar benutzen zu können, oder? Es ist so oder so das gleiche Geld. Ich meine, ich \emph{würde} einen guten Koffer haben wollen, in den \emph{viel} hineinpasst, gut genug, damit ich später keinen besseren mehr kaufen muss…", fügte er hoffnungsvoll hinzu.

McGonagalls Blick ließ nicht nach. "Und was genau würden Sie in so einem Koffer \emph{aufbewahren}, Mr~Potter --"

"Bücher."

"Natürlich", seufzte McGonagall.

"Sie hätten mir \emph{viel früher} erzählen müssen, dass es solche Sachen gibt! Und dass ich sie mir leisten kann! Jetzt müssen mein Vater und ich die nächsten zwei Tage damit verbringen, alle Buchläden \emph{wie wahnsinnig} nach alten Lehrbüchern abzuklappern, damit ich eine anständige Sammlung über Mathematik und Naturwissenschaften mit nach Hogwarts nehmen kann --- und vielleicht eine kleine Auswahl einiger Science Fiction-Romane und Fantasy-Bücher, wenn ich etwas Vernünftiges aus den Schnäppchenkisten zusammentragen kann. Oder noch besser, ich mache das Geschäft etwas attraktiver für Sie, okay? Lassen Sie mich dazu noch folgendes kaufen --"

"Mr~Potter! Denken Sie, Sie könnten mich bestechen?"

"Was? Nein! Nicht so! Ich meine nur, Hogwarts kann einige der Bücher, die ich mitbringe, behalten, wenn Sie denken, dass welche davon gut in die dortige Bibliothek passen. Ich werde sie günstig bekommen, und ich möchte sie lediglich griffbereit haben. Es ist in Ordnung, Leute mit Büchern zu bestechen, nicht wahr? Das ist eine --"

"Familientradition."

"Ja, genau."

McGonagalls ganzer Körper schien zusammenzusacken. "Ich fürchte, ich kann die Logik Ihrer Worte nicht bestreiten, obwohl ich mir sehr wünsche, ich könnte es. Ich werde Ihnen erlauben, zusätzliche einhundert Galleonen abzuheben, Mr~Potter. Ich weiß, ich werde das bereuen, und ich mache es trotzdem."

"Das ist die richtige Einstellung! Und macht ein Beutel aus Eselsfell das, was ich denke?"

"Er kann weniger als ein Koffer", sagte McGonagall widerstrebend, "aber in einem Eselsfellbeutel, der mit einem Rückholzauber und einem Unaufspürbaren Ausdehnungszauber belegt wurde, kann so einiges verstaut werden, bis derjenige, der es in den Beutel gelegt hat, es wieder herausholt."

"Ja, ich brauche definitiv auch einen davon. Das ist ja wie die Supergürteltasche der ultimativen Großartigkeit! Batmans Werkzeuggürtel! Was soll ich mit einem Schweizer Taschenmesser, wenn ich einfach einen ganzen Werkzeugkasten mit mir herumtragen kann! Oder andere magische Gegenstände! Oder Bücher! Ich könnte die drei besten Bücher, die ich gerade lese, jederzeit bei mir haben, und einfach überall eins herausholen! Ich werde nie wieder auch nur eine Minute meines Lebens verschwenden müssen! Was sagen Sie, Professor McGonagall? Es ist für den bestmöglichen Zweck!"

"In Ordnung. Sie dürfen weitere zehn Galleonen hinzufügen."

Griphook bedachte Harry mit einem Blick, in dem Hochachtung, möglicherweise sogar unverblümte Bewunderung lag.

"Und ein wenig Taschengeld, wie Sie vorhin erwähnt haben. Ich glaube, ich habe eins oder zwei Dinge gesehen, die ich vielleicht in diesem Beutel aufbewahren möchte."

"Übertreiben Sie es nicht, Mr~Potter."

"Aber, Professor McGonagall, warum gönnen Sie mir denn nichts? Dies ist ein glücklicher Tag, an dem ich alle magischen Dinge zum ersten Mal entdecke! Warum sich wie ein mürrischer Erwachsener benehmen, wenn Sie stattdessen lächeln könnten und an Ihre eigene unschuldige Kindheit zurückdenken, während Sie die Wonne auf meinem Gesicht beobachten, wenn ich mit einem vernachlässigbaren Bruchteil des Reichtums, der mir zusteht, weil ich den fürchterlichsten Zauberer, den dieses Land je gesehen hat, besiegt habe, ein wenig Spielzeug kaufe? Nicht, dass ich Ihnen vorwerfe, undankbar zu sein oder so, aber trotzdem, was ist schon ein bisschen Spielzeug im Vergleich dazu?"

"Sie", knurrte McGonagall. Ihr Gesichtsausdruck war so furchterregend und schrecklich, dass Harry quiekte und zurücksprang, mit einem lauten Klirren über einen ganzen Stapel Goldmünzen stolperte und rückwärts in einen Geldhaufen fiel. Griphook seufzte und schlug sich die Hand vor die Stirn. "Ich würde dem magischen England einen großen Dienst erweisen, Mr~Potter, und vielleicht sogar der ganzen Welt, wenn ich Sie in dieses Verlies sperren und hier lassen würde."

Und sie gingen fort, ohne sich weiter zu streiten.

