

\hypertarget{der-satz-von-bayes}{% \section{20. Der Satz von Bayes}\label{der-satz-von-bayes}}

Normalerweise schaffe ich höchstens drei Seiten am Stück, bis mein Gehirn ausgelaugt ist und die Lust am Übersetzen verliert. Diesmal habe ich mich eines Tages hin, fing an zu übersetzen~-- und als ich viele Stunden später aufhörte, war es draußen dunkel und ich hatte 15 Seiten übersetzt. Das ist mir nicht ganz geheuer~…;)

Bitte erwartet aber nicht, dass das jetzt jedes Mal so schnell geht. Seht es stattdessen als (verfrühtes) Weihnachts-, Chanukka-, Kwanzaa- oder Was-auch-immer-ihr-sonst-so-feiert-Geschenk.\\ (Und falls ihr mir auch ein Geschenk machen wollt~-- ihr wisst ja, wie das geht~…)

Und noch etwas: Am Ende des Kapitels findet ihr wieder einen kurzen Ausschnitt aus dem nächsten Kapitel als Vorschau. Das habe ich vor ein paar Wochen bei einer anderen FF gesehen und mir gefällt es ganz gut. Euch auch?

* * *

Harry lag auf dem weichen Feldbett und blickte an die graue Decke des kleinen Zimmers. Er hatte viel von Professor Quirrells Süßigkeiten gegessen -- ausgeklügelte Kompositionen aus Schokolade und anderen Substanzen, bestreut mit glitzernden Streuseln und mit winzigen Zuckersplittern besetzt, die außerordentlich teuer aussahen und sich tatsächlich als ungemein lecker herausstellten. Harry hatte absolut kein schlechtes Gewissen dabei. \emph{Das} hatte er sich \emph{verdient}.

Er hatte nicht versucht zu schlafen. Harry hatte das Gefühl, dass ihm nicht gefallen würde, was er vor seinem inneren Auge sehen würde.

Er hatte nicht versucht zu lesen. Er könnte sich nicht darauf konzentrieren.

Seltsam, wie Harrys Gehirn immer weiter und weiter zu arbeiten schien und nie zur Ruhe kam, egal wie sehr es ermüdete. Es wurde stumpfsinnig, aber es weigerte sich \emph{herunterzufahren}.

Doch er spürte es, er fühlte es ganz genau: ein Triumphgefühl.

Ein Pluspunkt in seinem Anti-Dunkler-Lord-Harry-Programm war nicht mal \emph{annähernd} genug. Harry fragte sich, was der Sprechende Hut \emph{jetzt} wohl sagen würde, wenn er ihn noch mal aufsetzen könnte.

Kein \emph{Wunder}, dass Professor Quirrell behauptet hatte, er sei auf dem besten Weg, ein Dunkler Lord zu werden. Harry hatte viel zu lange gebraucht, um es zu erkennen; er hätte die Parallelen sofort sehen müssen --

\emph{Versteht, dass der Dunkle Lord an jenem Tag nicht gewonnen hat. Sein Ziel war, Kampfkunst zu lernen, doch er ging ohne eine einzige Unterrichtsstunde.}

Harry hatte den Kerker betreten, um Zaubertränke zu lernen. Er war ohne eine einzige Unterrichtsstunde gegangen.

Und Professor Quirrell hatte davon gehört, hatte es mit atemberaubender Präzision verstanden, hatte die Arme ausgestreckt und Harry von jenem Pfad heruntergezerrt; dem Pfad, auf dem er zu einer Kopie von Du-weißt-schon-wem geworden wäre.

Es klopfte an der Tür. „Der Unterricht ist vorbei“, sagte Professor Quirrells leise Stimme.

Harry ging zur Tür und merkte, dass er plötzlich nervös wurde. Dann ließ die Anspannung nach, als er Professor Quirrell von der Tür weggehen hörte.

\emph{Was zum Teufel ist da los? Wird er deswegen irgendwann gefeuert werden?}

Harry öffnete die Tür und sah, dass Professor Quirrell in einigen Metern Abstand wartete.

\emph{Fühlt Professor Quirrell es auch?}

Sie gingen über die nun verlassene Bühne zum Lehrertisch, auf den Professor Quirrell sich stützte. Harry blieb, wie zuvor, kurz vor dem Podium stehen.

„Nun“, sagte Professor Quirrell. Er wirkte jetzt auf irgendeine Weise freundlich, obwohl sein Gesichtsausdruck die übliche Ernsthaftigkeit beibehielt. „Worüber wollten Sie mit mir sprechen, Mr Potter?“

\emph{Ich habe eine mysteriöse dunkle Seite.} Doch Harry konnte nicht einfach so damit herausplatzen.

„Professor Quirrell“, sagte Harry, „habe ich nun den Pfad, auf dem ich ein Dunkler Lord geworden wäre, verlassen?“

Professor Quirrell blickte Harry an. „Mr Potter“, sagte er ernst, nur mit einem leichten Grinsen, „ein Ratschlag. Eine Darbietung kann auch zu perfekt sein. Ein normaler Mensch, der soeben fünfzehn Minuten lang geschlagen und gedemütigt wurde, steht nicht einfach wieder auf und vergibt seinen Feinden gütig. Das ist das Verhalten, was man an den Tag legt, wenn man versucht, die Leute zu überzeugen, dass man nicht böse ist; nicht --“

\emph{„Ich glaub' es nicht! Sie können nicht jede mögliche Beobachtung als Bestätigung Ihrer Theorie werten!“}

„Und das war ein \emph{Hauch} zu viel Empörung.“

\emph{„Was zum Teufel muss ich denn tun, um Sie zu überzeugen?“}

„Um mich zu überzeugen, dass Sie nicht beabsichtigen, ein Dunkler Lord zu werden?“, sagte Professor Quirrell und sah nun äußerst amüsiert aus. „Ich nehme an, Sie könnten einfach Ihre rechte Hand heben.“

„Was?“, sagte Harry verständnislos. „Aber ich kann meine rechte Hand heben, ganz egal, ob ich --“ Harry brach ab und fühlte sich ziemlich dumm.

„In der Tat“, sagte Professor Quirrell. „Sie können es in beiden Fällen ebenso gut tun. Sie können nichts tun, um mich zu überzeugen, da ich genau wüsste, dass Sie eben das vorhatten. Und um noch genauer zu sein: Wenngleich es wohl möglich sein mag, dass es vollkommen gute Menschen gibt, obwohl ich nie einen solchen getroffen habe, so ist es zumindest \emph{unwahrscheinlich}, dass jemand fünfzehn Minuten lang geschlagen wird und anschließend wieder aufsteht und voll gnädiger Vergebung für seine Angreifer ist. Andererseits ist es \emph{weniger} unwahrscheinlich, dass ein junges Kind sich vorstellt, dass es \emph{diese Rolle spielen} muss, um seinen Lehrer und seine Mitschüler davon zu überzeugen, dass es nicht der nächste Dunkle Lord ist. Die Bedeutung einer Handlung liegt nicht in ihrem \emph{oberflächlichen Eindruck}, Mr Potter, sondern in der Geisteshaltung, die jene Handlung wahrscheinlicher oder unwahrscheinlicher macht.“

Harry blinzelte. Ihm hatte gerade ein Zauberer den Unterschied zwischen der Repräsentativitätsheuristik und der Bayes'schen Definition von Evidenz erklärt.

„Andererseits“, sagte Professor Quirrell, „kann jeder sich Mühe geben, seine Freunde zu beeindrucken. Das muss nicht böse sein. Also, ohne dass ich daraus weitergehende Schlüsse ziehe, Mr Potter, sagen Sie mir die Wahrheit. Was ging in Ihrem Kopf in dem Moment vor, als Sie jegliche Rache ablehnten? Haben Sie da wirklich aus Vergebung gehandelt? Oder dachten Sie daran, wie Ihre Mitschüler diese Handlung interpretieren würden?“

\emph{Manchmal sind wir unser eigener Phönixgesang.}

Doch Harry sagte es nicht laut. Ihm war klar, dass Professor Quirrell ihm nicht glauben würde und ihm womöglich weniger Respekt schenkte, weil er versuchte, eine so offensichtliche Lüge zu erzählen.

Nach einigen stillen Momenten lächelte Professor Quirrell zufrieden. „Ob Sie es glauben oder nicht, Mr Potter“, sagte der Professor, „Sie brauchen sich keine Sorgen machen, weil ich Ihr Geheimnis entdeckt habe. Ich werde Ihnen \emph{nicht} sagen, dass Sie es aufgeben sollen, der nächste Dunkle Lord zu werden. Wenn ich die Zeit zurückdrehen könnte und diese Absicht irgendwie aus dem Kopf meines jüngeren Ichs entfernen könnte, würde mein heutiges Ich von dieser Änderung nicht profitieren. Denn solange ich glaubte, dass das mein Ziel war, hatte ich den Drang, zu lernen und zu üben, meine Fähigkeiten zu verfeinern und stärker zu werden. Wir werden zu dem, was uns vorherbestimmt ist, indem wir unseren Leidenschaften folgen, wohin sie uns auch führen mögen. So hat Salazar es gelehrt. Fragen Sie mich, wo die Bücher stehen, die ich als Dreizehnjähriger gelesen habe, und ich werde Sie gerne dorthin führen.“

„Verdammt noch mal“, sagte Harry und setzte sich auf den harten Marmorboden, dann legte er sich hin und starrte das hohe Deckengewölbe an. Näher konnte er einem verzweifelten Zusammenbruch kaum kommen, ohne sich dabei tatsächlich weh zu tun.

„Immer noch zu viel Empörung“, bemerkte Professor Quirrell. Harry sah nicht zu ihm, doch er konnte das unterdrückte Lachen in seiner Stimme hören.

Dann wurde es Harry klar.

„Ich glaube, ich weiß, was Sie hierdran verwirrt“, sagte Harry. „Das war es nämlich, worüber ich mit Ihnen sprechen wollte. Professor Quirrell, ich glaube, dass Sie meine mysteriöse dunkle Seite sehen.“

Einen Moment lang war es still.

„Ihre … dunkle Seite …“

Harry setzte sich auf. Professor Quirrell betrachtete ihn mit einem der seltsamsten Gesichtsausdrücke, den Harry jemals bei irgendjemandem gesehen hatte; erst recht bei jemandem, der so würdevoll wie Professor Quirrell war.

„Es passiert, wenn ich wütend werde“, erklärte Harry. „Mein Blut wird kalt, alles wird kalt, alles erscheint vollkommen klar … Im Nachhinein betrachtet hatte ich es schon seit einer Weile -- in meinem ersten Jahr auf der Muggelschule hat jemand in der Pause versucht, mir meinen Ball wegzunehmen und ich habe ihn hinter meinem Rücken versteckt und dem Jungen in den Solarplexus getreten, weil ich gelesen hatte, dass das eine empfindliche Stelle war, und danach haben die anderen Kinder mich in Ruhe gelassen. Und ich habe eine Mathelehrerin gebissen, als sie meine Überlegenheit nicht anerkannte. Aber erst in letzter Zeit stand ich unter genug Stress, um zu merken, dass es wirklich eine, sie wissen schon, mysteriöse dunkle Seite ist und nicht nur Wutanfälle, wie der Schulpsychologe meinte. Und ich habe keine magischen Superkräfte wenn das passiert, das war eine der ersten Sachen, die ich getestet habe.“

Professor Quirrell rieb sich die Nase. „Lassen Sie mich darüber nachdenken“, sagte er.

Harry wartete mehr als eine Minute lang in völliger Ruhe. Er nutzte die Zeit um aufzustehen, was ihm schwerer fiel, als er erwartet hatte.

„Nun“, sagte Professor Quirrell nach einer Weile, „offenbar gab es \emph{doch} etwas, was Sie sagen konnten, um mich zu überzeugen.“

„Ich \emph{habe} bereits erraten, dass meine dunkle Seite ein weiterer Teil von mir ist und dass die Lösung nicht ist, nie wütend zu werden, sondern dass ich lernen muss, die Kontrolle zu behalten und es zu akzeptieren. Ich bin schließlich nicht blöd und ich habe die Geschichte oft genug gesehen, um zu wissen, wie es weitergeht, aber es ist schwer und Sie machen den Eindruck, dass Sie mir helfen würden.“

„Nun … ja … sehr scharfsinnig von Ihnen, Mr Potter, das muss ich sagen … diese Seite von Ihnen ist, wie Sie offenbar schon vermutet haben, ihre Absicht, zu töten, die, wie Sie sagten, ein Teil von Ihnen ist …“

„Und trainiert werden muss“, sagte Harry, um das Muster zu vervollständigen.

„Und trainiert werden muss, ja.“ Professor Quirrell hatte immer noch diesen seltsamen Gesichtsausdruck. „Mr Potter, wenn Sie wirklich nicht der nächste Dunkle Lord werden wollen, welches Ziel war es dann, von dem der Sprechende Hut Sie abbringen wollte; wegen welches Ziels hat er Sie nach Slytherin geschickt?“

„Ich wurde nach \emph{Ravenclaw} geschickt!“

„Mr Potter“, sagte Professor Quirrell, nun mit einem sehr viel gewöhnlicher aussehenden, trockenen Lächeln, „ich weiß, Sie sind es gewohnt, nur von Narren umgeben zu sein, aber bitte verwechseln Sie mich nicht mit denen. Die Wahrscheinlichkeit, dass der Sprechende Hut zum ersten Mal seit achthundert Jahren einen Scherz macht, während er sich auf Ihrem Kopf befindet, ist so gering, dass sie es nicht wert ist, beachtet zu werden. Es ist wohl gerade noch im Rahmen des Möglichen, dass Sie mit den Fingern geschnippt und sich irgendeine einfache und schlaue Methode ausgedacht haben, jene Zauber zu umgehen, die jegliche Manipulation des Hutes verhindern sollen; wenngleich mir selbst keine solche Methode einfällt. Doch die bei weitem wahrscheinlichste Erklärung ist, dass Dumbledore mit der Wahl des Hutes für den Jungen der lebt nicht glücklich war. Das ist für jeden offensichtlich, der auch nur den kleinsten Funken gesunden Menschenverstandes besitzt, also ist Ihr Geheimnis auf Hogwarts sicher.“

Harry öffnete den Mund, schloss ihn dann wieder und fühlte sich vollkommen hilflos. Professor Quirrell irrte sich, doch er irrte auf so überzeugende Weise, dass Harry begann zu glauben, dass \emph{das} nunmal die logische Schlussfolgerung aus den Professor Quirrell zur Verfügung stehenden Fakten war. Manchmal -- man konnte nie \emph{wissen}, wann, aber manchmal geschah es eben -- bekam man unwahrscheinliche Fakten serviert und die bestmögliche Schlussfolgerung war die falsche. Wenn man einen medizinischen Test hatte, der nur in einem von tausend Fällen falsch lag, dann würde er manchmal trotzdem falsch liegen.

„Kann ich Sie bitten, nie zu wiederholen, was ich gleich sagen werde?“, sagte Harry.

„Selbstverständlich“, sagte Professor Quirrell. „Sie können davon ausgehen, dass Sie mich gebeten haben.“

Harry war auch kein Narr. „Kann ich davon ausgehen, dass Sie dieser Bitte entsprechen werden?“

„Sehr gut, Mr Potter. Sie können in der Tat davon ausgehen.“

\emph{„Professor Quirrell --“}

„Ich werde nicht wiederholen, was Sie gleich sagen werden“, sagte Professor Quirrell lächelnd.

Beide lachten, dann wurde Harry wieder ernst. „Der Sprechende Hut schien zu glauben, dass ich ein Dunkler Lord werden würde, wenn ich nicht nach Hufflepuff gehe“, sagte Harry. „Aber ich \emph{will} keiner werden.“

„Mr Potter …“, sagte Professor Quirrell. „Verstehen Sie mich nicht falsch. Ich verspreche, dass ich Ihre Antwort nicht benoten werde. Ich will bloß Ihre eigene, ehrliche Antwort hören. Warum nicht?“

Harry fühlte sich wieder so \emph{hilflos}. \emph{Du sollst kein Dunkler Lord werden} war ein so grundlegender Bestandteil seiner Moral, dass es ihm schwer fiel, das herzuleiten. „Ähm, Menschen würden darunter leiden?“

„Aber sicher wollten Sie schonmal Menschen Leid zufügen“, sagte Professor Quirrell. „Sie wollten den älteren Slytherins heute Leid zufügen. Wenn Sie ein Dunkler Lord sind, dann bedeutet das, dass die Menschen leiden, denen Sie Leid zufügen \emph{wollen}.“

Harry rang mit den Worten und entschied sich dann für die offensichtlichen. „Erst einmal: Bloß weil ich jemandem Leid zufügen will, ist das ja noch lange nicht richtig --“

„Was ist dann das Richtige, wenn nicht das, was Sie wollen?“

„Ah“, sagte Harry, „Präferenzutilitarismus.“

„Verzeihung?“, sagte Professor Quirrell.

„Das ist die ethische Theorie, wonach das gut ist, was die Präferenzen der meisten Personen --“

„Nein“, sagte Professor Quirrell. Seine Finger massierten seinen Nasensattel. „Ich glaube, das ist nicht ganz das, was ich sagen wollte. Mr Potter, letztlich machen alle Leute nur das, was sie machen wollen. Manchmal bezeichnen die Leute das, was sie tun wollen als ‚das Richtige`, aber wie könnten wir denn jemals anders handeln, als unseren eigenen Wünschen zu folgen?“

„Naja, selbstverständlich“, sagte Harry. „Ich könnte moralischen Überlegungen nicht \emph{Folge leisten}, wenn sie mich nicht überzeugen würden. Aber das bedeutet nicht, dass mein Wille, jenen Slytherins Leid zuzufügen, mich \emph{eher} überzeugt als moralische Überlegungen!“

Professor Quirrell blinzelte.

„Ganz zu schweigen davon“, sagte Harry, „dass auch viele Unbeteiligte verletzt würden, wenn ich ein Dunkler Lord werde!“

„Warum bedeutet Ihnen das etwas?“, sagte Professor Quirrell. „Was haben die für Sie getan?“

Harry lachte. „Also, \emph{das} war jetzt ungefähr so subtil wie \emph{Atlas wirft die Welt ab}.“

„Verzeihung?“, sagte Professor Quirrell wieder.

„Das ist ein Buch. Meine Eltern wollten nicht, dass ich es lese, weil sie dachten, dass es mich verderben würde, also habe ich es natürlich trotzdem gelesen und war beleidigt, dass sie dachten, dass ich auf etwas so offensichtliches reinfallen würde. Bla bla bla ich bin besser als andere, andere Leute versuchen mich kleinzuhalten, bla bla bla.“

„Sie sagen also, dass ich meine Fallen weniger offensichtlich gestalten soll?“, sagte Professor Quirrell. Er tippte sich mit einem Finger an die Wange und sah nachdenklich aus. „Daran kann ich arbeiten.“

Beide lachten.

„Aber um bei der Frage zu bleiben“, sagte Professor Quirrell, „was \emph{haben} all die anderen Leute für Sie getan?“

„Andere Leute haben \emph{jede Menge} für mich getan!“, sagte Harry. „Als meine Eltern starben, haben meine Eltern mich aufgenommen, weil sie \emph{gute Menschen} sind; und wenn ich ein Dunkler Lord werden würde, würde ich all das verraten!“

Professor Quirrell war eine Zeit lang still.

„Ich muss gestehen“, sagte Professor Quirrell leise, „dass dieser Gedanke mir in Ihrem Alter niemals kommen konnte.“

„Das tut mir Leid“, sagte Harry.

„Nicht nötig“, sagte Professor Quirrell. „Das ist lange her und ich habe meine elterlichen Probleme zu meiner vollen Zufriedenheit ausgeräumt. Also hält Sie der Gedanke zurück, dass Ihre Eltern das ablehnen würden? Heißt das, wenn Ihre Eltern in einem Unfall stürben, dass es dann nichts mehr gäbe, was Sie --“

„Nein“, sagte Harry. „Absolut nicht. Es war ihr Drang, Gutes zu tun, der mich beschützt hat. Dieser Drang wohnt nicht nur meinen Eltern inne. Und diesen Drang würde ich verraten.“

„Auf jeden Fall, Mr Potter, haben Sie meine ursprüngliche Frage nicht beantwortet“, sagte Professor Quirrell schließlich. „Was \emph{ist} Ihr Ziel?“

„Oh“, sagte Harry. „Ähm …“ Er sortierte seine Gedanken. „ Alle wichtigen Dinge zu lernen, die man über das Universum erfahren kann, dieses Wissen zu nutzen, um allmächtig zu werden, und diese Macht zu nutzen, um die Realität zu verändern, weil ich mit dem aktuellen Zustand nicht ganz einverstanden bin.“

Einen Moment lang war es still.

„Verzeihen Sie, falls das eine dumme Frage ist, Mr Potter“, sagte Professor Quirrell, „aber sind Sie sich \emph{sicher}, dass Sie nicht soeben zugegeben haben, ein Dunkler Lord werden zu wollen?“

„Das wäre man nur, wenn man diese Macht für Böses verwendet“, erklärte Harry. „Wenn man die Macht für Gutes nutzt, ist man ein Heller Lord.“

„Ich verstehe“, sagte Professor Quirrell. Er tippte sich mit einem Finger an die andere Wange. „Ich denke, damit kann ich arbeiten. Aber, Mr Potter, während die Ausmaße Ihres Ziels durchaus Salazar selbst würdig wäre -- wie genau wollen Sie das anstellen? Ist es Ihr erster Schritt, ein großer Kampfmagier zu werden oder der Leiter der Mysteriumsabteilung oder Minister für Zauberei oder --“

„Mein erster Schritt ist, ein Wissenschaftler zu werden.“

Professor Quirrell sah Harry an, als ob er sich gerade in eine Katze verwandelt hätte.

„Ein Wissenschaftler“, sagte Professor Quirrell nach einer Weile.

Harry nickte.

„Ein \emph{Wissenschaftler}?“, wiederholte Professor Quirrell.

„Ja“, sagte Harry. „Ich werde meine Ziele erreichen durch die Macht … der \emph{Wissenschaft}!“

„Ein \emph{Wissenschaftler}!“, sagte Professor Quirrell. Sein Gesichtsausdruck zeigte aufrichtige Empörung und seine Stimme hatte einen lauteren und schärferen Ton angenommen. „Sie könnten mein allerbester Schüler sein! Der größte Kampfmagier, den Hogwarts in den letzten fünf Dekaden hervorgebracht hat! Ich kann mir nicht vorstellen, wie Sie Ihre Tage damit verschwenden, in einem weißen Laborkittel nutzlose Dinge mit Ratten anzustellen!“

„Hey!“, sagte Harry. „Zu Wissenschaft gehört mehr als das! Natürlich heißt das nicht, dass irgendwas \emph{falsch} daran wäre, mit Ratten zu experimentieren. Aber Wissenschaft \emph{ist} das Mittel, mit dem man das Universum versteht und beeinflusst --“

„Narr“, sagte Professor Quirrell in einer ruhigen, verbitterten Stimme. „Du bist ein Narr, Harry Potter.“ Er fuhr sich mit der Hand über das Gesicht und sein Gesicht entspannte sich. „Oder, noch wahrscheinlicher, Sie haben Ihr wahres Ziel noch nicht entdeckt. Darf ich Ihnen nachdrücklich empfehlen, dass Sie stattdessen versuchen, ein Dunkler Lord zu werden? Ich würde zugunsten des Allgemeinwohls alles unternehmen, um Sie dabei zu unterstützen.“

„Sie mögen Wissenschaften nicht“, sagte Harry langsam. „Warum nicht?“

„Diese verdammten Muggel werden uns eines Tages alle umbringen!“ Professor Quirrells Stimme war lauter geworden. „Sie werden es vernichten! Alles vernichten!“

Harry fühlte sich etwas verloren. „Wovon sprechen wir gerade? Atombomben?“

„\emph{Ja}, Atombomben!“ Professor Quirrell schrie nun fast. „Selbst Er-dessen-Name-nicht-genannt-werden-darf hat die nie benutzt; vielleicht weil er nicht über einen Haufen Asche herrschen wollte! Sie hätten nie erfunden werden sollen! Und es wird mit der Zeit nur noch schlimmer werden!“ Professor Quirrell stand aufrecht da, statt sich auf den Tisch zu stützen. „Es gibt Tore, die man nicht öffnet; Siegel, die man nicht bricht! Jene Narren, die trotzdem damit rumspielen, werden früh von den niederen Schrecken getötet und wer überlebt, der weiß, dass es Geheimnisse gibt, \emph{die man nicht jenen verrät}, deren Intelligenz und Strebsamkeit nicht ausreichen, um die Geheimnisse selbst zu entdecken! Jeder mächtige Zauberer weiß das! Selbst die schrecklichsten Dunklen Zauberer wissen das! Und diese idiotischen Muggel scheinen es einfach nicht zu verstehen! Die eifrigen kleinen Narren, die das Geheimnis der Atombombe entdeckt haben, behielten es nicht für sich; sie haben ihren \emph{verdammten} Politikern davon erzählt und nun müssen \emph{wir} in einem Zustand leben, in dem wir ständig mit kompletter Auslöschung bedroht sind!“

Das war eine ganz andere Sicht der Dinge als die, mit der Harry aufgewachsen war. Hätten die Atomphysiker eine Verschwörung bilden sollen, um die Atombombe vor jedem geheim zu halten, der nicht selbst klug genug war, um Atomphysiker zu sein? Dieser Gedanke war ihm nie gekommen, doch er war zumindest faszinierend. Würden sie geheime Passwörter verwenden? Würden sie Masken tragen?

(Genau genommen könnte es, soweit Harry das wusste, \emph{jede Menge} unglaublich zerstörerische Geheimnisse geben, die die Physiker für sich behielten; dann wäre die Atombombe das einzige solche Geheimnis, das doch an die Öffentlichkeit gelangt war. Die Welt sähe für ihn genau gleich aus.)

„Ich muss mal darüber nachdenken“, sagte Harry zu Professor Quirrell. „Dieser Gedanke ist mir neu. Und eines der \emph{versteckten} Geheimnisse der Wissenschaft, das nur von einigen seltenen Lehrern an ihre Doktoranden weitergegeben wird, ist, wie man es vermeidet, neue Ideen sofort beiseite zu fegen, wenn man sie hört und nicht mag.“

Professor Quirrell blinzelte erneut.

„Gibt es denn irgendeine Wissenschaft, der Sie zugeneigt sind?“, sagte Harry. „Medizin vielleicht?“

„Raumfahrt“, sagte Professor Quirrell. „Doch die Muggel scheinen bei dem einzigen Projekt nicht voranzukommen, mit dem die Zaubererschaft diesem Planeten entkommen kann, bevor die ihn in die Luft sprengen.“

Harry nickte. „Ich bin selbst ein großer Anhänger des Raumfahrtprogramms. Zumindest haben wir das gemeinsam.“

Professor Quirrell blickte Harry an. Irgendetwas blitzte in den Augen des Professors auf. „Ich bitte Sie um ihr Wort, um das Versprechen und um den Schwur, dass Sie nie über das Folgende reden.“

„Ich schwöre es“, sagte Harry sofort.

„Stellen Sie sicher, dass Sie Ihren Schwur halten, sonst werden die Folgen Ihnen nicht gefallen“, sagte Professor Quirrell. „Ich werde nun einen selten und mächtigen Zauber sprechen, nicht auf Sie, sondern auf das Klassenzimmer um uns herum. Bleiben Sie stehen, damit Sie die Grenzen des Zaubers nicht berühren, wenn ich ihn einmal gesprochen habe. Sie dürfen mit der Magie, die ich aufrecht erhalte, nicht interagieren. Schauen Sie nur zu. Anderenfalls werde ich den Zauber beenden.“ Professor Quirrell wartete kurz. „Und versuchen Sie, nicht hinzufallen.“

Harry nickte verwundert und erwartungsvoll.

Professor Quirrell hob seinen Zauberstab und sagte etwas, das Harrys Ohren und Verstand nicht aufnehmen konnten; Worte, die sein Bewusstsein umschifften und dem Vergessen anheimfielen.

Ein kleiner Marmorkreis um Harrys Füße blieb erhalten. Der restliche Marmorboden verschwand, Wände und Decke verschwanden.

Harry stand auf einem kleinen Kreis aus weißem Marmor inmitten eines endlosen Himmels voller Sterne, die schrecklich hell und stetig brannten. Harry konnte keine Erde, keinen Mond, keine Sonne erkennen. Professor Quirrell stand am gleichen Ort wie zuvor und schwebte inmitten der Sterne. Die Milchstraße war als großes, ausgewaschenes Lichtband zu sehen und wurde heller, als Harrys Augen sich an die Dunkelheit gewöhnten.

Der Anblick umklammerte Harrys Herz fester als alles, was er jemals gesehen hatte.

„Sind wir … im Weltall …?“

„Nein“, sagte Professor Quirrell. Seine Stimme war traurig und ehrfürchtig. „Doch es ist ein wahres Bild.“

Harry stiegen Tränen in die Augen. Er wischte sie hektisch weg, er wollte das nicht verpassen, weil ihm irgendwelches blödes Wasser die Sicht trübte.

Die Sterne waren keine winzigen Juwelen in einer riesigen, samtenen Kuppel mehr, wie sie im Nachthimmel auf der Erde erschienen. Hier gab es keinen Himmel über ihm, keine Himmelskuppel, die ihn umgab. Nur vollkommene Lichtpunkte und vollkommene Dunkelheit, ein unendliches und leeres Nichts mit unzähligen winzigen Löchern, durch die brillante Strahlen aus irgendeinem unvorstellbaren jenseitigen Bereich durchdrangen.

Im All schienen die Sterne so schrecklich, schrecklich, schrecklich weit entfernt.

Harry wischte sich die Augen, immer und immer wieder.

„Manchmal“, sagte Professor Quirrell mit so ruhiger Stimme, dass sie fast gar nicht da war, „wenn diese makelbehaftete Welt ungewöhnlich hasserfüllt scheint, dann frage ich mich, ob es irgendeinen anderen Ort geben könnte, weit entfernt, an dem ich sein sollte. Ich kann mir nicht vorstellen, wie dieser Ort sein könnte; und wenn ich ihn mir nicht einmal vorstellen kann, wie kann ich dann daran glauben, dass er existiert? Und doch ist das Universum so unfassbar groß und vielleicht gibt es ihn ja dennoch? Aber die Sterne sind so weit, weit entfernt. Es würde eine lange, lange Zeit dauern, sie zu erreichen, selbst wenn ich den Weg wüsste. Und ich frage mich, wovon ich träumen würde, wenn ich eine lange, lange Zeit schliefe …“

Obwohl es sich wie ein Sakrileg anfühlte, gelang Harry ein Flüstern. „Bitte, lassen Sie mich eine Weile hier bleiben.“

Professor Quirrell nickte inmitten der Sterne.

Es war leicht, den kleinen Kreis aus Marmor zu vergessen, auf dem er stand, seinen Körper selbst zu vergessen, und zu einem reinen Bewusstsein zu werden, das entweder still stand oder sich bewegte. Inmitten dieser unermesslichen Entfernungen konnte er das nicht erkennen.

Eine zeitlose Zeit verging.

Und dann verschwanden die Sterne und das Klassenzimmer kehrte zurück.

„Es tut mir Leid“, sagte Professor Quirrell, „aber wir werden gleich Besuch bekommen.“

„Es ist okay“, flüsterte Harry. „Das war lang genug.“ Er würde diesen Tag niemals vergessen; und das nicht wegen der unwichtigen Dinge, die zuvor passiert waren. Er würde lernen, diesen Zauber zu sprechen, selbst wenn es das letzte war, was er jemals lernte.

Dann flogen die schweren Eichentüren des Klassenzimmers aus ihren Angeln und schlitterten mit einem schrillen Geräusch über den Marmorboden.

„QUIRINUS! WIE KANNST DU ES WAGEN!“

Wie eine riesige Gewitterwolke zog ein uralter und mächtiger Zauberer in den Raum ein, auf seinem Gesicht so glühende Wut, dass der ernste Blick, mit dem er Harry zuvor betrachtet hatte, wie nichts erschien.

In Harrys Kopf drehte sich alles, als jener Teil von ihm, der vor dem schrecklichsten Ding fliehen wollte, das er jemals gesehen hatte, sich abwandte und von dem Teil von ihm ersetzt wurde, der diesen Schock aushielt.

\emph{Keine} von Harrys Seiten war froh darüber, dass sie beim Sterne gucken unterbrochen wurde. „Schulleiter Albus Percival --“, begann Harry in eisigem Tonfall.

WUMMS. Professor Quirrells Hand schlug hart auf seinen Tisch. \emph{„Mr Potter!“}, bellte Professor Quirrell. „Das ist der \emph{Schulleiter von Hogwarts} und Sie sind ein einfacher Schüler! Sie werden ihn angemessen anreden!“

Harry sah Professor Quirrell an.

Professor Quirrell blickte Harry streng an.

Keiner von beiden lächelte.

Dumbledores lange Schritte führten ihn auf die Bühne, wo Harry vor dem Podium stand und Professor Quirrell an seinem Tisch lehnte. Der Schulleiter starrte beide schockiert an.

„Es tut mir Leid“, sagte Harry in einem sanften, höflichen Ton. „Schulleiter, danke, dass Sie mich beschützen wollen, aber Professor Quirrell hat das Richtige getan.“

Langsam veränderte sich Dumbledores Gesichtsausdruck von einem, der Stahl verdampfen konnte, in einen bloß noch wütenden. „Ich habe von Schülern gehört, dass dieser Mann dich von älteren Slytherins verprügeln ließ! Dass er dir verboten hat, dich zu verteidigen!“

Harry nickte. „Er wusste genau, welchen Fehler ich hatte, und er hat mir gezeigt, wie ich ihn beheben konnte.“

„Harry, \emph{wovon redest du}?“

„Ich habe ihm beigebracht, wie man verliert“, sagte Professor Quirrell trocken. „Das ist eine wichtige Fähigkeit im Leben.“

Es war offensichtlich, dass Dumbledore es immer noch nicht verstand, doch seine Stimme senkte sich. „Harry …“, sagte er langsam. „Wenn der Verteidigungslehrer dir mit irgendetwas gedroht hat, um zu verhindern, dass du dich beschwerst --“

\emph{Du Wahnsinniger, ausgerechnet heute denkst du wirklich, dass ich --}

„Schulleiter“, sagte Harry und versuchte, kleinlaut zu wirken, „meine Schwachstelle ist nicht, dass ich mich von Lehrern einschüchtern lasse, die ihre Position missbrauchen.“

Professor Quirrell gluckste. „Noch nicht perfekt, Mr Potter, aber gut genug für Ihren ersten Tag. Schulleiter, sind Sie lange genug geblieben, um von den einfünfzig Punkten für Ravenclaw zu hören, oder sind Sie sofort rausgestürmt, als Sie den ersten Teil gehört hatten?“

Kurz erschien Verunsicherung auf Dumbledores Gesicht, gefolgt von Überraschung. „Einundfünfzig Punkte für Ravenclaw?“

Professor Quirrell nickte. „Er hatte das nicht erwartet, aber es erschien mir angemessen. Sagen Sie Professor McGonagall, dass ich glaube, dass die Berichte, was Mr Potter durchmachen musste, um die Punkte zurückzugewinnen, ebenso genügen werden, um ihre Lektion zu vermitteln. Nein, Schulleiter, Mr Potter hat mir nichts erzählt. Es ist leicht zu erkennen, welcher Teil der heutigen Ereignisse ihre Arbeit waren, ebenso wie ich weiß, dass Sie selbst letztendlich den Kompromiss vorgeschlagen haben. Doch ich frage mich, wie um alles in der Welt Mr Potter es geschafft hat, die Oberhand über Snape und Sie zu gewinnen und wie es Professor McGonagall dann gelungen ist, die Oberhand über ihn zu gewinnen.“

Irgendwie gelang es Harry, seinen Gesichtsausdruck zu wahren. War das für einen wahren Slytherin \emph{so} offensichtlich?

Dumbledore kam Harry etwas näher und betrachtete ihn kritisch. „Du siehst etwas blass aus, Harry“, sagte der alte Zauberer. Er blickte genau in Harrys Gesicht. „Was hattest du heute zum Mittag?“

„Was?“, sagte Harry, der plötzlich völlig verwirrt war. Wieso fragte Dumbledore ihn nach frittiertem Lamm und dünnen Broccoli-Scheiben, wo das doch so ziemlich der \emph{unwahrscheinlichste} denkbare Grund für --

Der alte Zauberer trat zurück. „Nicht so wichtig. Ich denke, dir geht es gut.“

Professor Quirrell hustete laut und deutlich. Harry blickte zum Professor und sah, dass er Dumbledore mit einem scharfen Blick bedachte.

\emph{„Äh-hem!“}, sagte Professor Quirrell erneut.

Dumbledore und Professor Quirrell sahen sich in die Augen und schienen miteinander zu kommunizieren.

„Wenn Sie es ihm nicht sagen“, sagte Professor Quirrell dann, „werde ich es tun, selbst wenn Sie mich deswegen rausschmeißen.“

Dumbledore seufzte und wandte sich wieder an Harry. „Ich möchte mich dafür entschuldigen, dass ich Ihre mentale Unversehrtheit verletzt habe, Mr Potter“, sagte der Schulleiter förmlich. „Ich hatte nur die Absicht, zu ermitteln, ob Professor Quirrell dasselbe getan hatte.“

\emph{Was?}

Die Verwirrung hielt nur solange an, wie Harry brauchte, um zu verstehen, was gerade passiert war.

\emph{„Sie --!“}

„Ruhig, Mr Potter“, sagte Professor Quirrell. Sein Gesichtsausdruck war jedoch streng, als er Dumbledore anstarrte.

„Legilimentik wird manchmal mit gesundem Menschenverstand verwechselt“, sagte der Schulleiter. „Aber sie hinterlässt Spuren, die ein anderer fähiger Legilimentor erkennen kann. Nur danach habe ich Ausschau gehalten, Mr Potter, und ich habe Ihnen eine irrelevante Frage gestellt, um sicherzustellen, dass Sie an nichts Wichtiges dachten, während ich nachsah.“

\emph{„Sie hätten zuerst fragen sollen!“}

Professor Quirrell schüttelte den Kopf. „Nein, Mr Potter. Der Schulleiter hatte begründete Sorgen und hätte er um Erlaubnis gefragt, so hätten Sie an genau jene Dinge gedacht, die Sie ihm nicht zeigen wollten.“ Professor Quirrells Tonfall wurde schärfer. „Mich empört es sehr viel mehr, Schulleiter, dass Sie es nicht für nötig befunden haben, ihm danach Bescheid zu sagen!“

„Sie haben es nun schwieriger gemacht, seine mentale Unversehrtheit in Zukunft zu überprüfen“, sagte Dumbledore. Er schenkte Professor Quirrell einen kühlen Blick. „Ob das wohl Ihre Absicht war?“

Professor Quirrells Gesichtsausdruck war unnachgiebig. „Es gibt zu viele Legilimentoren an dieser Schule. Ich bestehe darauf, dass Mr Potter in Okklumentik unterrichtet wird. Werden Sie mir gestatten, sein Lehrer zu sein?“

„Natürlich nicht“, sagte Dumbledore sofort.

„Das dachte ich mir. Da \emph{Sie} ihn meiner kostenlosen Dienste beraubt haben, werden \emph{Sie} nun dafür aufkommen, dass Mr Potter von einem geprüften Okklumentik-Lehrer unterrichtet wird.“

„Solche Dienste sind nicht billig“, sagte Dumbledore und sah Professor Quirrell überrascht an. „Allerdings habe ich gewisse Beziehungen --“

Professor Quirrell schüttelte entschieden mit dem Kopf. „Nein. Mr Potter wird den Verwalter seines Vermögens bei Gringotts nach einem unabhängigen Lehrer fragen. Bei allem Respekt, Schulleiter Dumbledore, nach den Ereignissen des heutigen Morgens protestiere ich dagegen, dass Sie oder Ihre Vertrauten Zugang zu Mr Potters Geist bekommen. Ich muss zudem darauf bestehen, dass der Lehrer den unbrechbaren Schwur leistet, nichts weiterzusagen, und dass er sich damit einverstanden erklärt, dass ihm unmittelbar nach jeder Sitzung die Erinnerungen genommen werden.

Dumbledore runzelte die Stirn. „Solche Dienste sind \emph{außerordentlich} teuer, wie Sie genau wissen, und ich kann mir die Frage nicht verkneifen, warum \emph{Sie} das für nötig halten.“

„Wenn das Geld das Problem ist“, warf Harry ein, „dann habe ich ein paar Ideen, wie man schnell an große Mengen Geld kommen kann --“

„Danke, Quirinus, Deine Weisheit ist nun offensichtlich und es tut mir Leid, dass ich daran gezweifelt habe. Deine Sorge um Harry Potter ehrt Dich ebenso.“

„Nichts zu danken“, sagte Professor Quirrell. „Ich hoffe, Sie haben nichts dagegen, dass ich ihm in Zukunft meine besondere Aufmerksamkeit zukommen lasse.“ Professor Quirrells Gesichtsausdruck war nun sehr ernst und sehr ruhig.

Dumbledore blickte zu Harry.

„Ich wünsche es ebenso“, sagte Harry.

„So wird es also sein …“, sagte der alte Zauberer langsam. Ein seltsamer Ausdruck zuckte über sein Gesicht. „Harry … dir muss bewusst sein, dass du, wenn du diesen Mann als deinen Lehrer und Freund wählst, als deinen ersten Mentor, ihn auf die eine oder andere Weise verlieren wirst und je nachdem, auf welche Weise das geschieht, wirst du ihn womöglich nie wieder zurückerhalten.“

Daran hatte Harry nicht gedacht. Doch die Stelle des Verteidigungs-Lehrers \emph{war} mit einem Fluch belegt … einem, der offenbar seit Jahrzehnten ausnahmslos gewirkt hatte …

„Wahrscheinlich“, sagte Professor Quirrell ruhig, „doch bis das geschieht, wird er vollen Nutzen aus mir ziehen können.“

Dumbledore seufzte. „Zumindest ist es ökonomisch, nehme ich an, da Ihnen als Verteidigungs-Lehrer \emph{ohnehin} irgendein unbekanntes Schicksal droht.“

Harry musste sich anstrengen, um seinen Gesichtsausdruck unter Kontrolle zu behalten, als ihm klar wurde, was Dumbledore tatsächlich gemeint hatte.

„Ich werde Madam Pince mitteilen, dass es Mr Potter gestattet ist, Bücher über Okklumentik auszuleihen“, sagte Dumbledore.

„Es gibt vorbereitende Übungen, die Sie alleine durchführen müssen“, sagte Professor Quirrell zu Harry. „Und ich empfehle Ihnen, dass Sie sich damit beeilen.“

Harry nickte.

„Ich werde Sie beide dann alleine lassen“, sagte Dumbledore. Er nickte Harry und Professor Quirrell zu und ging langsam davon.

„Können Sie den Zauber noch mal sprechen?“, sagte Harry, sobald Dumbledore verschwunden war.

„Heute nicht“, sagte Professor Quirrell leise, „und morgen auch nicht, fürchte ich. Es kostet mich viel Kraft, ihn zu sprechen, aber weniger, ihn aufrecht zu erhalten, daher versuche ich normalerweise, ihn so lange wie möglich andauern zu lassen. Dieses Mal habe ich ihn aus einer Laune heraus gesprochen. Hätte ich überlegt und daran gedacht, dass wir unterbrochen werden könnten --“

Von allen Menschen auf der Welt war Dumbledore nun jener, den Harry am wenigsten leiden konnte.

Beide seufzten.

„Selbst wenn ich es nur dieses einzige Mal sehe“, sagte Harry, „werde ich nie aufhören, Ihnen dankbar zu sein.“

Professor Quirrell nickte.

„Haben Sie vom Pioneer-Programm gehört?“, sagte Harry. „Das waren Raumsonden, die an verschiedenen Planeten vorbeifliegen und Fotos aufnehmen sollten. Zwei dieser Sonden hatten Flugbahnen, die sie aus dem Sonnensystem heraus in das interstellare Medium geführt haben. Also gab es an Bord dieser Sonden eine goldene Plakette mit Bildern von einem Mann und einer Frau und einer Anleitung, wie man unsere Sonne in der Galaxie findet.“

Professor Quirrell war einen Moment lang still und lächelte dann. „Sagen Sie, Mr Potter, können Sie erahnen, was mir durch den Kopf ging, nachdem ich die Liste der siebenunddreißig Dinge zusammengestellt habe, die ich als Dunkler Lord niemals tun würde? Versetzen Sie sich in meine Situation -- was hätten Sie an meiner Stelle gedacht?“

Harry stellte sich vor, wie er auf eine Liste von siebenunddreißig Dingen blickte, die er als Dunkler Lord niemals tun würde.

„Sie haben beschlossen, dass es recht sinnlos wäre, überhaupt ein Dunkler Lord zu werden, wenn Sie dann \emph{jederzeit} die \emph{gesamte} Liste befolgen müssten“, sagte Harry.

\emph{„Exakt“}, sagte Professor Quirrell. Er grinste. „Also werde ich nun gegen Regel zwei verstoßen -- die schlicht ‚Prahle nicht` lautete -- und Ihnen von etwas erzählen, was ich getan habe. Ich denke nicht, dass dieses Wissen irgendeinen Schaden anrichten könnte. Und ich vermute stark, dass Sie es ohnehin erraten hätten, sobald wir uns gut genug kennen. Und dennoch … ich bitte Sie zu schwören, dass Sie niemals über das sprechen werden, was ich gleich erzähle.“

„Ich schwöre es!“ Harry hatte das Gefühl, dass es um etwas \emph{wirklich} tolles ging.

„Ich habe eine Muggelzeitschrift abonniert, die mich über Neuigkeiten aus der Raumfahrt auf dem Laufenden hält. Von Pioneer 10 habe ich erst erfahren, als die Sonde gestartet war. Doch als ich erfuhr, dass auch Pioneer 11 das Sonnensystem für immer verlassen würde“, sagte Professor Quirrell mit dem breitesten Grinsen, das Harry bisher auf seinem Gesicht gesehen hatte, „habe ich mich tatsächlich bei NASA eingeschlichen und ich habe einen hübschen kleinen Zauber auf diese hübsche goldene Plakette gesprochen, so dass sie sehr viel haltbarer ist, als sie es sonst wäre.“

…

…

…

„Ja“, sagte Professor Quirrell, der nun etwa zehn Mal so groß erschien, „ich dachte mir, dass Sie so reagieren würden.“

…

…

…

„Mr Potter?“

„… ich weiß nicht, was ich sagen soll.“

„‚Sie sind mir überlegen` erscheint mir angemessen“, sagte Professor Quirrell.

„Sie sind mir überlegen“, sagte Harry sofort.

„Sehen Sie?“, sagte Professor Quirrell. „Kaum vorstellbar, in was für Problemen Sie nun stecken würden, wenn Sie nicht geschafft hätten, das zu sagen.“

Beide lachten.

Harry kam ein weiterer Gedanke. „Sie haben zur Plakette doch nicht etwa weitere Informationen hinzugefügt, oder?“

„Weitere Informationen?“, sagte Professor Quirrell. Es klang so, als ob die Idee ihm nie gekommen wäre und er davon äußerst fasziniert war.

Das kam Harry sehr verdächtig vor, schließlich hatte \emph{Harry} nicht mal eine Minute gebraucht, um auf die Idee zu kommen.

„Vielleicht eine holographische Botschaft, wie in \emph{Star Wars}?“, sagte Harry. „Oder … hm. Ein Portrait scheint den gesamten Inhalt eines menschlichen Gehirns zu speichern … Sie hätten der Sonde keine zusätzliche Masse hinzufügen können, aber vielleicht hätten Sie einen der Bestandteile in ein Porträt von Ihnen selbst verwandeln können? Oder Sie haben einen Freiwilligen gefunden, der todkrank war, und den bei der NASA reingeschmuggelt und einen Zauber gesprochen um sicherzustellen, dass dessen \emph{Geist} an die Plakette gebunden wird --“

„Mr Potter“, sagte Professor Quirrell in einem plötzlich sehr scharfen Tonfall, „ein Zauber, der den Tod eines Menschen voraussetzt, würde vom Ministerium zweifelsohne als dunkle Magie eingestuft, ohne Beachtung der Umstände. Man sollte Schüler nicht von solchen Dingen reden hören.“

Und das erstaunliche an der Art, wie Professor Quirrell das gesagt hatte, war, wie er sich die Möglichkeit offen gehalten hatte, alles abzustreiten. Er hatte es genau in dem Tonfall eines Menschen gesagt, der über solche Dinge nicht sprechen wollte und der meinte, dass Schüler sich von so etwas fernhalten sollten. Harry wusste wirklich nicht, ob Professor Quirrell nur darauf wartete, dass Harry seine Gedanken schützen konnte.

„Verstanden“, sagte Harry. „Ich werde mit niemandem sonst über diese Idee reden.“

„Bitte bewahren Sie Stillschweigen über die gesamte Angelegenheit, Mr Potter“, sagte Professor Quirrell. „Ich ziehe es vor, ein Leben abseits der Öffentlichkeit zu führen. Bevor ich mich entschieden habe, Verteidigungs-Lehrer auf Hogwarts zu werden, war in den Zeitungen nichts über Quirinus Quirrell zu finden.“

Das schien etwas traurig, doch Harry verstand es. Dann begriff Harry die Implikationen. „Also … wie viele coole Sachen \emph{haben} Sie gemacht, über die sonst niemand Bescheid weiß --“

„Oh, einige“, sagte Professor Quirrell. „Aber ich denke, dass das für heute genug ist, Mr Potter. Ich muss zugeben, dass ich mich etwas ermüdet fühle --“

„Ich verstehe. Und \emph{danke}. Für \emph{alles}.“

Professor Quirrell nickte und stützte sich stärker auf seinen Tisch.

Harry verließ rasch das Zimmer.

* * *

\textbf{Atlas wirft die Welt ab:}\\ (engl. Originaltitel: „Atlas Shrugged“; auch als „Wer ist John Galt?“ oder „Der Streik“ auf Deutsch veröffentlicht) Ein Loblied auf Egoismus in Romanform. Ein hierzulande nicht besonders bekanntes Buch, das in den USA hingegen sehr einflussreich war und die neoliberale Weltanschauung geprägt hat.

\textbf{Pioneer-Raumsonden:}\\ Pioneer 10 und 11 sind 1972 und 1973 gestartet. Anfang 1990 (also 1,5 Jahre bevor Harrys erstes Schuljahr auf Hogwarts begann) flog Pioneer 11 am Neptun vorbei. Und an Bord war tatsächlich eine \href{https://de.\%20wikipedia.\%20org/wiki/Pioneer-Plakette“\%20target=}{goldene Plakette}.

Und im nächsten Kapitel:

\emph{„Doch sprechen wir nun über schönere Themen“, sagt die Figur inmitten der grünen Schatten. „Sprechen wir über Wissen und Macht. Draco Malfoy, sprechen wir über Wissenschaft.“}

„Ja“, sagte Draco. „Sprechen wir darüber.“

