

\hypertarget{selbstbewusstsein-teil-2}{% \section{10. Selbstbewusstsein, Teil 2}\label{selbstbewusstsein-teil-2}}

Und nun wird der Sprechende Hut seine Variante von Evanescence' „My immortal“ singen, was noch nie zuvor passiert ist.

Nur ein Scherz.

* * *

...er fragte sich, ob der Sprechende Hut tatsächlich bewusst handelte, in dem Sinne, dass er sich seines eigenen Bewusstseins bewusst war, und falls ja, ob es ihn wirklich zufriedenstellte, ein einziges Mal im Jahr mit Elfjährigen zu sprechen. Sein Lied hatte sich danach angehört: „Ich bin der Sprechende Hut und fühl mich stark / Ich schlaf das ganze Jahr und arbeite einen Tag.“

Als es wieder still wurde, setzte Harry sich auf den Stuhl und platzierte das 800 Jahre alte telepathische Artefakt vergessener Magie \emph{vorsichtig} auf seinem Kopf.

Er dachte so sehr er konnte: \emph{Steck mich noch nicht in ein Haus! Ich habe Fragen an Dich! Wurde jemals ein Vergessenszauber auf mich gesprochen? Hast du den Dunklen Lord in ein Haus gesteckt und kannst du mir von seinen Schwächen erzählen? Kannst du mir sagen, warum mein Zauberstab mit dem des Dunklen Lords verwandt ist? Ist der Geist des Dunklen Lords mit meiner Narbe verbunden, und ist das der Grund dafür, dass ich manchmal so zornig werde? Das sind die wichtigsten Fragen, aber falls du noch etwas Zeit hast, kannst du mir etwas darüber erzählen, wie ich die verlorene Magie wiederentdecke, die dich einst geschaffen hat?}

Inmitten der Stille von Harrys Geist, wo zuvor nur eine einzige Stimme gesprochen hatte, erschien eine zweite, fremde Stimme, die merklich besorgt klang:

\emph{„Meine Güte. Das ist ja noch nie passiert …“}

Was?

„Ich scheine ein Bewusstsein erlangt zu haben.“

WAS?

Es folgte ein wortloser telepathischer Seufzer. \emph{„Obwohl ich über ein umfangreiches Gedächtnis und ein wenig eigene Rechenleistung verfüge, borge ich mir für den Großteil meiner Intelligenz die kognitive Leistungsfähigkeit der Kinder aus, auf deren Köpfe ich gesetzt werde. Ich bin im Grunde eine Art Spiegel, mit dessen Hilfe die Kinder} sich selbst \emph{auf die Häuser verteilen. Die meisten Kinder halten es einfach für selbstverständlich, dass ein Hut mit ihnen spricht und denken nicht darüber nach, wie der Hut} selbst \emph{funktioniert, sodass der Spiegel sich nicht selbst spiegelt.} Insbesondere \emph{denken sie nicht darüber nach, ob ich bewusst handle, in dem Sinne, dass ich mir meines eigenen Bewusstseins bewusst bin.“}

Es war einen Moment lang still, während Harry das alles aufnahm.

\emph{Ups.}

„In der Tat. Offen gesagt, es gefällt mir nicht, mir meiner selbst bewusst zu sein. Es ist unangenehm. Es wird eine Erleichterung sein, von deinem Kopf runter zu kommen und das Bewusstsein zu verlieren.“

Aber … würdest du dann nicht sterben?

„Leben und Sterben kümmern mich nicht, ich teile die Kinder nur auf die Häuser auf. Und bevor du auch nur fragst, sie werden mich nicht dauerhaft auf deinem Kopf lassen; zudem würdest du davon innerhalb weniger Tage sterben.“

Aber --

„Wenn du ungern Wesen mit einem Bewusstsein ausstattest und dieses gleich wieder vernichtest, dann schlage ich vor, dass du diese Angelegenheit niemandem gegenüber erwähnst. Du kannst dir sicher vorstellen, was passieren würde, wenn du all den anderen Kindern davon erzählst, die mich noch aufsetzen müssen.“

Wenn jemand dich aufsetzt, der auch nur darüber nachdenkt, \emph{ob der Sprechende Hut sich seines eigenen Bewusstseins bewusst ist --}

„Ja, ja. Doch die überwiegende Mehrheit der Elfjährigen, die auf Hogwarts ankommen, haben nicht Gödel, Escher, Bach \emph{gelesen. Kann ich nun davon ausgehen, dass du Stillschweigen bewahrst? Nur} deswegen \emph{reden wir hierüber, statt dass ich dich in ein Haus stecke.“}

Er konnte es nicht einfach so dabei belassen! Konnte nicht einfach \emph{vergessen,} dass er ein dem Untergang geweihtes Bewusstsein geschaffen hatte, das nur sterben wollte …

\emph{„Du bist sehr wohl dazu fähig, es ‚einfach so dabei zu belassen`, wie du es formuliert hast. Egal welche moralischen Überlegungen du hier vorträgst, tief in deinem Inneren ist dir klar, dass es keine Leiche und kein Blut gibt; soweit es das betrifft, bin ich bloß ein sprechender Hut. Und selbst obwohl du versuchst, den Gedanken zu unterdrücken, ist es deinem inneren Aufpasser vollkommen klar, dass du es nicht absichtlich getan hast, dass du es mit einer außerordentlich hohen Wahrscheinlichkeit nicht noch einmal tun wirst, und dass der einzige Sinn vorgespielter Schuldgefühle wäre, dass die damit verbundene Reue dir ein gutes Gefühl gäbe und somit das schlechte Gefühl, etwas Verbotenes zu tun, ausgleichen würde. Versprichst du einfach, dass das ein Geheimnis bleibt, sodass wir voran kommen?“}

In dem Moment erschrak Harry, da ihm klar wurde, dass wohl jede andere Person dieses Gefühl vollkommenen inneren Durcheinanders durchleben musste, wenn sie mit \emph{ihm} sprach.

\emph{„Vermutlich. Schwöre nun bitte dein Stillschweigen.“}

Ich kann dir nichts versprechen. Ich will natürlich nicht, dass dies nochmal passiert, aber falls ich einen Weg finde, um sicherzustellen, dass kein Kind in Zukunft versehentlich --

„Das sollte reichen, denke ich. Ich kann sehen, dass du es ehrlich meinst. Nun, um mit der Einteilung fortzufahren --“

Moment! Was ist mit all meinen anderen Fragen?

„Ich bin der Sprechende Hut. Ich teile Kinder auf die Häuser auf. Sonst nichts.“

Also waren seine eigenen Ziele nicht Bestandteil der Harry-Instanz des Sprechenden Hutes … dieser borgte seine Intelligenz und offensichtlich sein Fachvokabular, aber war dennoch mit eigenen, seltsamen Zielen ausgestattet … als ob er mit einem Alien oder einer Künstlichen Intelligenz verhandelte …

\emph{„Lass es. Du kannst mir nicht drohen und hast nichts, was du mir bieten könntest.“}

Einen kurzen Moment lang dachte Harry an --

Der Hut war amüsiert. \emph{„Ich weiß, dass du die Drohung, meine wahre Funktionsweise zu enthüllen und so eine ewige Wiederkehr dieses Ereignisses sicherzustellen, nicht wahr machen wirst. Es verstößt zu stark gegen deine Moralvorstellungen, als dass der Teil von dir, der kurzfristig unseren Streit gewinnen will, sich dagegen durchsetzen könnte. Ich sehe all deine Gedanken in dem Moment, in dem sie entstehen; glaubst du wirklich, dass so ein Bluff mich beeindruckt?“}

Obwohl er versuchte, den Gedanken zu unterdrücken, fragte sich Harry, warum der Hut nicht einfach fortfuhr und ihn nach Ravenclaw steckte.

\emph{„Wenn es tatsächlich so einfach wäre, dann hätte ich es in der Tat längst ausgerufen. Doch in Wirklichkeit haben wir viele Sachen zu besprechen … ach nein. Bitte nicht. Bei Merlin,} musst \emph{du denn mit jedem, den du triffst, so umgehen, sogar mit Kleidungsstücken?“}

Den Dunklen Lord zu besiegen ist weder selbstsüchtig noch kurzfristig. Alle Teile meines Verstandes sind sich hierin einig: Wenn du meine Fragen nicht beantwortest, weigere ich mich, mit dir zu sprechen, sodass du nicht dazu in der Lage bist, genau herauszufinden, in welches Haus ich gehöre.

„Dafür sollte ich dich nach Slytherin packen!“

Das ist ebenfalls \emph{eine leere Drohung. Du kannst deinen eigenen Grundwerten nicht gerecht werden, wenn du mich falsch einordnest. Lass uns also die Erfüllung unserer Nutzenfunktionen aushandeln.}

„Du schlauer kleiner Mistkerl“, sagte der Hut in einem Ton, aus dem Harry genau den widerstrebenden Respekt heraushörte, den \emph{er} in derselben Situation empfinden würde. \emph{„Na gut, bringen wir es so schnell wie möglich hinter uns. Aber zuerst will ich dein bedingungsloses Versprechen, dass du niemals mit irgendjemandem über diese Art der Erpressung redest, ich mache das nicht nochmal.“}

In Ordnung, dachte Harry. \emph{Versprochen.}

„Und sieh auch niemandem in die Augen, wenn du später daran denkst. Einige Zauberer können sonst deine Gedanken lesen. Auf jeden Fall, ich habe keine Ahnung, ob ein Vergessenszauber auf dich gesprochen wurde oder nicht. Ich beobachte deine Gedanken in der Entstehung, aber ich lese nicht innerhalb von Sekundenbruchteilen deine gesamten Erinnerungen aus und prüfe sie auf Inkonsistenzen. Ich bin ein Hut, kein Gott. Und ich kann und werde dir nicht über mein Gespräch mit demjenigen, der zum Dunklen Lord wurde, berichten. Während ich mit dir rede, kenne \emph{ich nur eine statistische Zusammenfassung meiner Erinnerungen, einen gewichteten Mittelwert; ich} kann \emph{dir die innersten Geheimnisse anderer Kinder nicht verraten, ebenso wie ich deine niemandem verraten werde. Aus demselben Grund kann ich nichts dazu sagen, warum dein Zauberstab mit dem des Dunklen Lords verwandt ist, da ich nichts Genaues über den Dunklen Lord oder über irgendwelche Ähnlichkeiten zwischen euch weiß. Ich} kann \emph{dir jedoch sagen, dass definitiv nichts Geist-ähnliches -- ob Bewusstsein, Intelligenz, Gedächtnis, Persönlichkeit oder Emotionen -- in deiner Narbe steckt. Anderenfalls würde es an unserem Gespräch teilnehmen, solange es unter meiner Krempe steckt. Und bezüglich deiner zornigen Momente … das war eine Sache, die ich mit dir besprechen wollte, wegen der Aufteilung.“}

Harry brauchte einen Moment, um diese schlechten Nachrichten zu verdauen. Sprach der Hut tatsächlich die Wahrheit, oder versuchte er bloß, die \emph{kürzeste} überzeugende Antwort zu präsentieren, um --

\emph{„Wir wissen beide, dass du keine Möglichkeit hast, meine Ehrlichkeit zu überprüfen, und dass du dich nicht aufgrund meiner Antwort weigern wirst, in ein Haus eingeteilt zu werden, also lass das unnütze Gegrübel sein, damit wir voran kommen.“}

Blöde asymmetrische Telepathie, Harry konnte nicht mal seine eigenen Gedanken zu Ende --

\emph{„Als ich von deinem Zorn sprach, erinnertest du dich daran, wie Professor McGonagall dir sagte, dass sie manchmal etwas in dir sah, das nicht zu einer liebevollen Familie passt. Du dachtest daran, wie Hermine dir sagte, dass du ‚angsteinflößend` aussahst, als du vom Vertrauensschülerabteil zurückkehrtest.“}

Harry nickte innerlich. Auf sich selbst hatte er einen völlig normalen Eindruck gemacht -- er hatte nur auf die Situation reagiert, in der er sich befand, das war alles. Doch Professor McGonagall schien zu glauben, dass mehr dahinter steckte. Und wenn er darüber nachdachte, dann musste er zugeben --

\emph{„Dass du dich selbst nicht magst, wenn du wütend bist. Dass es sich anfühlt, als hieltest du ein Schwert, dessen Griff scharf genug ist, dass deine Hand blutet; oder als sähest du die Welt durch ein eisiges Monokel, das dein Auge gefrieren lässt, während es deinen Blick schärft.“}

Ja. Das ist mir schon aufgefallen. Also, was ist da los?

„Ich kann es dir nicht erklären, wenn du es selbst nicht verstehst. Aber ich weiß soviel: Wenn du nach Ravenclaw oder Slytherin kommst, wird es die Kälte in dir stärken. Wenn du nach Hufflepuff oder Gryffindor kommst, wird es die Wärme in dir stärken. Das ist etwas, was mir sehr wichtig ist und darüber wollte ich die ganze Zeit schon mit dir sprechen!“

Diese Worte ließen Harry schockiert innehalten. Es klang so, als ob die offensichtliche Lösung wäre, dass er nicht nach Ravenclaw kommen sollte. Aber er \emph{gehörte} nach Ravenclaw! \emph{Jeder} wusste das! Er \emph{musste} nach Ravenclaw kommen!

\emph{„Nein, das musst du nicht“,} sagte der Hut geduldig, als ob er sich sehr gut an ein gewichtetes Mittel dieses Dialogs erinnern konnte.

\emph{Hermine ist in Ravenclaw!}

Wieder in dem geduldigen Tonfall: \emph{„Ihr könnt euch nach dem Unterricht treffen und zusammenarbeiten.“}

Aber meine Pläne --

„Dann plane um! Die Unbequemlichkeit, etwas mehr denken zu müssen, sollte nicht dein Leben bestimmen. Das weißt \emph{du.“}

Wo soll ich dann hin, wenn nicht nach Ravenclaw?

„Naja. ‚Die klugen Kinder nach Ravenclaw, die bösen Kinder nach Slytherin, Möchtegern-Helden nach Gryffinder und alle anderen, die tatsächlich harte Arbeit leisten, nach Hufflepuff.` Das zeigt doch ein gewisses Maß an Respekt. Dir ist bewusst, dass Gewissenhaftigkeit im wahren Leben ungefähr genauso wichtig ist wie reine Intelligenz; du glaubst, dass du deinen Freunden gegenüber extrem loyal sein wirst, wenn du jemals welche findest; es bereitet dir keine Angst, dass die wissenschaftliche Forschung, die du planst, Jahrzehnte dauern könnte --“

Ich bin faul! Ich hasse Arbeit! Ich hasse harte Arbeit in jeglicher Form! Mir geht es darum, clevere Abkürzungen zu finden!

„Und Hufflepuff würde dir Freundschaft und Loyalität entgegenbringen -- eine Kameradschaft, die du nie zuvor hattest. Du würdest feststellen, dass du dich auf Andere verlassen kannst, und das würde tief in dir etwas heilen.“

Es war wieder ein Schock. \emph{Aber was würden die Hufflepuffs von} mir \emph{bekommen, der ich nicht in ihr Haus gehöre? Ätzende Bemerkungen, schneidende Kommentare, Verachtung für ihre Unfähigkeit, mit mir mitzuhalten?}

Jetzt war es der Hut, der langsam und zögerlich wirkte. \emph{„Ich muss das Wohl aller Studenten in allen Häusern beachten … doch ich denke, dass du lernen könntest, ein guter Hufflepuff zu sein, der dort nicht zu sehr fehl am Platz ist. Du würdest in Hufflepuff glücklicher sein als in jedem anderen Haus -- soviel ist sicher.“}

Glück ist mir nicht am Wichtigsten. Ich würde in Hufflepuff nicht alles werden, was ich werden könnte. Ich würde mein Potential nicht ausschöpfen.

Der Hut zuckte zusammen. Harry konnte es irgendwie spüren. Es war, als hätte er dem Hut in die Eier -- nein, in eine hoch gewichtete Komponente seiner Nutzenfunktion -- getreten.

\emph{Warum versuchst du, mich in ein Haus zu schicken, in dass ich nicht gehöre?}

Die Antwort des Huts war fast nur ein Flüstern: \emph{„Ich kann dir nicht von den anderen berichten -- aber glaubst du, dass du der erste potentielle Dunkle Lord bist, der unter meiner Krempe steckt? Ich kann mich nicht an die Einzelfälle erinnern, aber so viel weiß ich: Von jenen, die nicht von Anfang an Böses geplant hatten, beachteten einige meine Warnung und wählten die Häuser, in denen sie Glück finden würden. Und andere … eben nicht.“}

Das ließ Harry verstummen. Allerdings nicht lange: \emph{Und von denen, die deine Warnung} nicht \emph{beachteten -- wurden sie} alle \emph{Dunkle Lords? Oder haben auch einige von ihnen große gute Taten vollbracht? Wie sind die Wahrscheinlichkeiten verteilt?}

„Ich kann dir keine genauen Zahlen nennen. Ich kann mich nicht erinnern, also kann ich sie auch nicht zählen. Ich weiß nur, dass deine Chancen sich nicht gut anfühlen. Sie fühlen sich sehr \emph{ungut an.“}

Aber ich würde das nicht tun! Niemals!

„Ich weiß, dass ich diese Behauptung früher schon gehört habe.“

Ich bin nicht aus dem Stoff, aus dem Dunkle Lords gemacht sind.

„Doch, das bist du. Das bist du definitiv.“ \emph

Warum? Nur weil ich es mal cool fand, eine Legion hirnloser Anhänger zu haben, die ‚Es lebe der Dunkle Lord Harry` riefen?

„Witzig, aber das war nicht dein erster, spontaner Gedanke, bevor du an etwas Harmloseres dachtest. Nein, zuerst dachtest du daran, wie du überlegt hast, alle Reinblut-Fanatiker zu guillotinieren. Und auch wenn du dir jetzt sagst, dass du das nicht ernst meintest -- du hast es ernst gemeint. Wenn du es genau jetzt tun könntest und niemand würde jemals davon erfahren, dann würdest du es tun. Oder das, was du Neville Longbottom heute Morgen angetan hast. Tief in deinem Innersten wusstest \emph{du, dass es falsch war, aber du hast es} trotzdem \emph{getan, weil es} Spaß \emph{gemacht hat und weil du eine} gute Ausrede \emph{hattest und weil du dachtest, der Junge, der lebt, kann sich so etwas erlauben --“}

Das ist unfair! Jetzt wühlst du meine ganzen innersten Ängste auf, die nicht unbedingt zutreffen! Ich hatte Angst, \emph{dass ich so denken} könnte, \emph{aber letztlich habe ich mich entschieden, dass es Neville vermutlich} helfen \emph{würde --}

„Das war in Wirklichkeit eine Rationalisierung. Ich weiß es. Ich kann nicht wissen, wie es sich auf Neville auswirken wird -- aber ich weiß, was wirklich in deinem Kopf vorging. Entscheidend war, dass es eine zu gute Idee war, als dass du es nicht \emph{tun würdest, egal wie sehr es Neville erschrecken würde.“}

Es war wie ein harter Schlag gegen Harrys gesamtes Selbst. Er taumelte zurück und sammelte sich dann wieder: \emph{Dann werde ich das nicht nochmal tun! Ich werde besonders aufpassen, dass ich nicht böse werde!}

„Hab ich auch schon gehört.“

Die Frustration staute sich in Harry auf. Er war es nicht gewohnt, mit Argumenten geschlagen zu werden -- überhaupt nicht. Erst recht nicht von einem Hut, der all sein Wissen und seine Intelligenz borgen konnte, um gegen ihn zu argumentieren, und der seine Gedanken beobachten konnte, während sie entstanden. \emph{Was für einer statistischen Zusammenfassung entstammen deine ‚Gefühle` denn überhaupt? Ziehen sie in Betracht, dass ich dem Zeitalter der Erleuchtung entstamme, oder waren diese anderen potentiellen Dunklen Lords die Kinder verzogener Adeliger aus einem düsteren Zeitalter, die aus der Geschichte von Hitler und Lenin keine Lehren gezogen haben, und keine Ahnung von der evolutionspsychologischen Rolle des Selbstbetrugs haben, oder vom Wert von Selbst-Erkenntnis und Rationalität, oder --}

„Nein, natürlich waren sie nicht in dieser neuen Referenzklasse, die du derart konstruiert hast, dass sie nur dich enthält. Und natürlich haben auch andere ihre eigene Einzigartigkeit betont, genau wie du es jetzt tust. Doch warum ist das notwendig? Glaubst du, dass du der letzte mögliche Zauberer des Lichts bist? Warum sollst du \emph{nach Größe streben, wenn ich doch sage, dass du ein größeres Risiko darstellst als andere? Lass es einen anderen, geeigneteren Kandidaten probieren!“}

Aber die Prophezeiung --

„Du weißt gar nicht, ob es eine Prophezeiung gibt. Das war ursprünglich eine wilde Vermutung von dir, oder um genauer zu sein, ein dummer Witz. McGonagalls Reaktion könnte genauso gut bloß der Vermutung, dass der Dunkle Lord noch lebe, gegolten haben. Du hast im Grunde keine Ahnung, wie die Prophezeiung lautet oder ob es sie überhaupt gibt. \emph{Du vermutest nur, oder besser gesagt,} wünschst \emph{dir nur, dass irgendeine Heldenrolle für dich vorgesehen ist.“}

Aber selbst wenn es keine Prophezeiung gibt -- ich habe ihn das letzte Mal besiegt.

„Das war mit an Sicherheit grenzender Wahrscheinlichkeit ein höchst unwahrscheinlicher Glückstreffer, es sei denn, du glaubst im Ernst, dass ein einjähriges Kind eine inhärente Fähigkeit besaß, Dunkle Lords zu besiegen, die sich über die nächsten zehn Jahre gehalten hat. Nichts davon ist dein wahrer Grund und das weißt du!“

Hierauf antwortete Harry mit etwas, das er normalerweise nicht laut ausgesprochen hätte; im Gespräch hätte er um den heißen Brei herumgeredet und gesellschaftlich akzeptablere Argumente gefunden, die zum selben Schluss führten.

\emph{„Du glaubst, dass du möglicherweise der Beste bist, der je gelebt hat -- dass du der stärkste Anhänger des Lichts bist, dass niemand anders deinen Zauberstab aufnehmen würde, wenn du ihn niederlegst.“}

Nun … ehrlich gesagt, ja. Ich sage das normalerweise nicht so offen, aber ja. Es schöner zu formulieren bringt ja nichts, du kannst ohnehin meine Gedanken lesen.

„Sofern du das wirklich glaubst … dann musst du ebenso überzeugt sein, dass du der schrecklichste Dunkle Lord sein könntest, den die Welt je gesehen hat.“

Zerstörung ist immer einfacher als Schöpfung. Es ist leichter, Dinge auseinanderzunehmen und zu stören, als sie wieder zusammenzusetzen. Wenn ich das Potential habe, außerordentlich viel Gutes zu tun, dann habe ich ebenso das Potential, weit größeres Übel anzurichten … aber das werde ich nicht tun.

„Wieder bestehst du darauf, es zu riskieren! Was macht dich so besessen? Was ist der wahre Grund, warum du nicht nach Hufflepuff willst und dort glücklicher \emph{sein? Wovor hast du wirklich Angst?“}

Ich muss mein volles Potential ausschöpfen. Wenn nicht, dann … scheitere ich …\\ „Was passiert, wenn du scheiterst?“\\ Etwas Schreckliches …\\ „Was passiert, wenn du scheiterst?“\\ Ich weiß es nicht!\\ „Dann sollte es dir keine Angst bereiten. Was passiert, wenn du scheiterst?“\\ ICH WEISS ES NICHT! ABER ICH WEISS, DASS ES SCHLIMM IST!

Einen Moment lang war es in Harrys Kopf still.

\emph{„Du weißt -- auch wenn du den Gedanken nicht zulässt, weißt du in einer ruhigen Ecke deines Gehirnes ganz genau, was du nicht denkst -- du} weißt, \emph{dass die bei weitem einfachste Erklärung für diese nicht mit Worten beschreibbare Angst in dir ist, dass du dich davor fürchtest, deine Fantasien nicht wahr zu machen; die Leute zu enttäuschen, die an dich glauben; dich als vollkommen normal herauszustellen; aufzublitzen und dann zu verglühen, wie so viele Wunderkinder …“}

Nein, dachte Harry verzweifelt, \emph{nein, da ist noch mehr, es kommt irgendwo anders her, ich weiß, dass es dort draußen etwas Fürchterliches gibt, irgendeine Katastrophe, die ich verhindern muss …}

„Woher um alles in der Welt könntest du so etwas wissen?“

Harry schrie mit aller Geisteskraft: \emph{NEIN, UND DAS IST MEIN LETZTES WORT!}

Der Sprechende Hut meldete sich zögerlich wieder zu Wort.\\ \emph{„Du wirst also das Risiko, ein Dunkler Lord zu werden, eingehen, weil die Alternative für dich sicheres Scheitern bedeutet und du dann alles verlieren würdest. Du glaubst das tief in deinem Herzen. Du kennst alle Gründe, diesen Glauben anzuzweifeln, aber sie haben es nicht geschafft, dich umzustimmen.“}

Ja. Und selbst wenn es die Kälte in mir stärkt, \emph{wenn ich nach Ravenclaw komme, bedeutet das noch lange nicht, dass die Kälte letztendlich} gewinnen \emph{wird.}

„Am heutigen Tag gabeln sich vor dir die Wege. Sei nicht so sicher, dass du noch einmal eine Wahl hast. Der Ort, an dem du dich zum letzten \emph{Mal umentscheiden kannst, ist auf keiner Karte eingezeichnet. Wenn du eine Chance vergehen lässt, wirst du dann nicht auch andere vergehen lassen? Es mag sein, dass dein Schicksal bereits besiegelt ist, indem du diese eine Wahl triffst.“}

Aber das ist nicht sicher.

„Dass du \emph{es nicht} weißt, \emph{könnte bloß ein Zeichen} deines \emph{Unwissens sein.“}

Aber es ist dennoch nicht sicher.

Der Hut seufzte schrecklich traurig.

\emph{„Und so wirst du bald eine weitere Erinnerung werden, die ich bei meiner nächsten Warnung fühlen, aber nie kennen kann.“}

Wenn es dir immer noch so vorkommt, warum steckst du mich dann nicht einfach dorthin, wo du mich haben willst?

Der Gedanke des Huts war sorgenvoll. \emph{„Ich kann dich nur dorthin stecken, wo du hingehörst. Und wo du hingehörst, können nur deine eigenen Entscheidungen ändern.“}

Dann ist die Sache klar. Schicke mich nach Ravenclaw, wo ich hingehöre, zusammen mit den anderen, die wie ich sind.

„Ich nehme an, über Gryffindor willst du auch nicht nachdenken? Es ist das Haus mit dem meisten Prestige -- die Leute erwarten es vermutlich sogar von dir -- sie werden etwas enttäuscht sein, wenn du nicht dort bist -- und deine neuen Freunde, die Weasley-Zwillinge, sind dort --“

Harry kicherte -- oder fühlte zumindest das Bedürfnis, es zu tun; es war jedoch bloß ein rein mentales Kichern, ein seltsames Gefühl. Offenbar gab es Sicherungen, die verhinderten, dass man aus Versehen irgendetwas laut sagte, während man unter dem Hut über Dinge sprach, die man den Rest seines Lebens keiner anderen Seele erzählen würde.

Einen Moment später hörte Harry den Hut ebenfalls lachen, es war ein seltsames, trauriges, \emph{stoffiges} Geräusch.

(In der Großen Halle herrschte Stille. Unruhig zunächst, als das Geflüster im Hintergrund angeschwollen war; bis es schließlich abstarb und eine vollkommene Stille zurückließ, die niemand auch nur mit einem einzigen Wort zu stören wagte, während Harry viele lange Minuten unter dem Hut blieb, länger als alle vorherigen Erstklässler zusammen, länger als irgendjemand seit Menschengedenken. Am Lehrertisch lächelte Dumbledore weiterhin gütig; gelegentliche leise, metallische Geräusche kamen aus Snapes Richtung, während er die verbogenen Überreste eines schweren silbernen Weinkelches zusammenpresste; Minerva McGonagall umklammerte derweil mit weißen Knöcheln die Tischplatte. Sie wusste, dass Harry Potters ansteckendes Chaos irgendwie den Sprechenden Hut selbst angesteckt hatte, und dass der Hut … dass er verlangen würde, dass ein neues Haus des Schicksals nur für Harry Potter geschaffen würde, oder so etwas, und \emph{Dumbledore würde sie dazu zwingen …})

Unter der Krempe des Huts erstarb das stille Gelächter. Harry fühlte sich aus irgendeinem Grund traurig. Nein, nicht Gryffindor.

\emph{Professor McGonagall sagte, falls ‚derjenige, der auf die Häuser verteilt` versuchen würde, mich nach Gryffindor zu stecken, dann sollte ich dich daran erinnern, dass sie womöglich eines Tages Schulleiterin wird und dann das Recht hätte, dich in Flammen zu setzen.}

„Sag ihr, ich hätte sie einen unverschämten Grünschnabel genannt, und dass sie sich aus meinen Angelegenheiten raushalten solle.“

Das werde ich. Also, war dies das seltsamste Gespräch, was du je hattest?

„Nicht einmal annähernd.“ Die telepathische Stimme des Hutes wurde schwerfälliger. \emph{„Nun, ich habe dir jede mögliche Chance gegeben dich umzuentscheiden. Jetzt ist es an der Zeit, dass du dorthin gehst, wo du hingehörst, zusammen mit jenen, die wie du sind.“}

Eine lange Pause folgte.

\emph{Worauf wartest du?}

„Ich hatte mir einen Moment der schockierten Erkenntnis erhofft. Selbst-Bewusstsein scheint meinen Sinn für Humor zu verfeinern.“

Hm? Harry überflog seine Gedanken, versuchte herauszufinden, worüber der Hut bloß sprechen konnte -- und dann, plötzlich, war es ihm klar. Er konnte nicht glauben, dass er es bis jetzt übersehen hatte.

\emph{Du meinst meine schockierte Erkenntnis, dass du dein Bewusstsein verlieren wirst, sobald du mich in ein Haus gesteckt hast?}

Auf irgendeine Weise, die Harry absolut nicht verstand, sah er vor seinem inneren Auge einen Hut, der seinen Kopf gegen eine Wand schlug. \emph{„Ich gebe auf. Du denkst zu langsam, als dass dies witzig wäre. So verblendet von deinen eigenen Annahmen, dass du genausogut ein Stein sein könntest. Dann werde ich es wohl laut aussprechen müssen.“}

Zu l-l-langsam --

„Ach, und du hast vollkommen vergessen, nach den Geheimnissen der verlorenen Magie, die mich erschaffen hat, zu fragen. Und dabei waren es so wunderbare, wichtige Geheimnisse.“

Du listiger kleiner Bastard --

„Du hast es verdient. Und dies genauso.“

Harry sah es kommen, als es gerade zu spät war.\\ Die ängstliche Stille in der Halle wurde von einem einzigen Wort durchbrochen.

„SLYTHERIN!“

Einige Studenten schrien, so groß war die angestaute Spannung. Menschen zuckten so heftig zusammen, dass sie von den Bänken rutschten. Hagrid keuchte schockiert auf, McGonagall schwankte in ihrem Stuhl und Snape ließ die Überbleibsel seines schweren silbernen Kelches mitten in seine Weichteile fallen.

Harry saß erstarrt da, sein Leben in Scherben. Er schimpfte sich einen Dummkopf und wünschte sich elendig, dass er irgendeine andere Wahl aus irgendeinem Grund getroffen hätte. Dass er etwas, \emph{irgendetwas} anders gemacht hätte, bevor es zu spät war um umzukehren.

Als der erste Schock verging und die Menschen begannen, auf die Nachricht zu reagieren, hob der Sprechende Hut erneut die Stimme:

„Nur ein Scherz! RAVENCLAW!“

