

\hypertarget{fundamentaler-attributionsfehler}{% \section{5. Fundamentaler Attributionsfehler}\label{fundamentaler-attributionsfehler}}

Dieses Kapitel habe ich übersetzt; Martin, Alex und Arne haben es Korrektur gelesen und dem Text seinen Feinschliff verpasst. Falls ihr in den Reviews Lob aussprecht, richtet es bitte an uns alle!

-\/-\/-\/-\/-\/-\/-\/-\/-\/-\/-\/-\/-\/-\/- ~ 5. Fundamentaler Attributionsfehler ~ -\/-\/-\/-\/-\/-\/-\/-\/-\/-\/-\/-\/-\/-\/-

\emph{„Er ist erst elf Jahre alt, Hermine.“}

„Genau wie du.“

„Ich zähle nicht.“

-\/-\/-

Der Eselsladen war ein putziges kleines Geschäft (manche würden es gar als süß bezeichnen), das sich hinter einem Gemüsestand versteckte, welcher in einer kleinen Seitenstraße der Winkelgasse hinter einem Laden für magische Handschuhe stand. Der Inhaber war leider kein verrunzelter, mysteriöser alter Mann, sondern nur eine nervös wirkende junge Frau in einem ausgeblichenen gelben Umhang. Gerade zeigte sie einen Super-Eselsfellbeutel QX31, der mit seiner dehnbaren Öffnung und einem Unaufspürbaren Ausdehnungszauber beworben wurde: Der Inhalt war zwar begrenzt, aber man konnte dennoch recht große Gegenstände darin verstauen.

Harry hatte \emph{darauf bestanden}, zuallererst hierher zu kommen~-- er hatte so sehr darauf bestanden, wie er konnte, ohne dass McGonagall Verdacht schöpfte. Denn Harry hatte etwas, das er so schnell wie möglich in diesem Beutel verstauen wollte: Nicht den Sack Galleonen, den er aus seinem Verlies abheben durfte, sondern all die anderen Galleonen, die er heimlich in seine Taschen geschaufelt hatte, nachdem er versehentlich in einen Haufen Goldmünzen gestolpert war. Es war \emph{tatsächlich} ein Unfall gewesen, doch Harry ließ sich eine solche Gelegenheit nicht entgehen~… egal, wie plötzlich der Gedanke kam. Seitdem hatte Harry peinlich genau aufgepasst, den erlaubten Goldbeutel direkt an seiner Hosentasche zu tragen, so dass alle klimpernden Geräusche aus der richtigen Richtung kamen.

Doch es blieb immer noch die Frage, wie er die \emph{anderen} Münzen in den Beutel befördern sollte, ohne erwischt zu werden. Denn die Goldmünzen gehörten zwar ihm, aber sie waren dennoch \emph{gestohlen}~-- Selbst-gestohlen? Ego-entwendet?

Harry sah von dem Super-Eselsbeutel QX31, der vor ihm auf der Theke lag, auf. „Kann ich ihn ein wenig ausprobieren? Um sicherzustellen, dass er, ähm, zuverlässig funktioniert?“ Er machte große Augen, um eine kindliche, verspielte Unschuld auszustrahlen.

Und tatsächlich, nachdem er zehn Mal den Sack Gold in den Beutel gelegt, reingefasst, „Sack mit Gold“ geflüstert und ihn rausgenommen hatte, ging McGonagall einige Schritte beiseite um einige andere Waren im Laden zu betrachten, während der Blick der Inhaberin ihr folgte.

Harry ließ den Sack Gold mit der \emph{linken} Hand in den Beutel fallen; seine \emph{rechte} Hand wanderte mit einigen fest umklammerten Goldmünzen aus seiner Hosentasche, fasste in den Eselsfellbeutel, ließ die einzelnen Münzen los und holte (mit einem geflüsterten „Sack mit Gold“) die erlaubten Galleonen zurück. Dann schob er diesen Sack in seine \emph{linke} Hand, um ihn gleich wieder in den Beutel fallen zu lassen, während Harrys \emph{rechte} wieder in seine Hosentasche glitt~…

McGonagall sah einmal zu ihm rüber, aber Harry schaffte es, nicht zu erstarren oder auch nur verdächtig zu zucken, daher schien sie nichts bemerkt zu haben. Andererseits konnte man sich bei Erwachsenen, die einen Sinn für Humor hatten, nie ganz sicher sein.

Nach drei Wiederholungen war Harry schließlich fertig, er hatte wohl etwa dreißig Galleonen von seinem eigenen Vermögen gestohlen.

Harry erhob sich, wischte etwas Schweiß von seiner Stirn und atmete tief aus. „Ich hätte gern diesen Beutel.“

Um 15 Galleonen erleichtert, jedoch mit einem Super-Eselsbeutel QX31 bewaffnet, verließen McGonagall und Harry das Geschäft. Die Tür formte eine Hand und winkte ihnen hinterher, wobei sie ihren Arm in einem unangenehm aussehenden Winkel beugte.

Und plötzlich~…

„Bist du \emph{wirklich} Harry Potter?“, flüsterte der alte Mann, während eine Träne seine Wange herablief.

„Du würdest doch nicht lügen, nicht wahr? Es ist nur, ich habe Gerüchte gehört, dass du den Todesfluch in Wirklichkeit \emph{nicht} überlebt hättest und dass deswegen nie wieder jemand von dir gehört hat.“

…~wurde ihnen klar, dass McGonagalls Tarnzauber einen erfahrenen Zauberer nicht vollkommen täuschen konnte.

McGonagall hatte, sobald sie die Worte „Harry Potter“ gehört hatte, eine Hand auf Harrys Schulter gelegt und ihn eilig in eine abgelegene Seitenstraße gezogen. Der alte Mann war ihnen gefolgt, aber es schien zumindest so, als ob sonst niemand etwas gehört hatte.

Harry dachte über diese Frage nach. \emph{War} er tatsächlich Harry Potter? „Ich weiß nur, was andere Leute mir erzählt haben“, sagte Harry. „Schließlich erinnere ich mich nicht an meine Geburt.“ Seine Hand strich über seine Stirn. „Ich habe diese Narbe seitdem ich mich erinnern kann immer gehabt und mir wurde auch immer gesagt, dass mein Name Harry Potter wäre. Aber“, sagte Harry nachdenklich, „wenn wir schon eine Verschwörung unterstellen, dann wüsste ich nicht, was dagegen spricht, eine andere Zaubererwaise zu finden und im Glauben, dass \emph{sie} Harry Potter sei aufzuziehen~…“

McGonagall schlug verärgert die Hände über dem Kopf zusammen. „Sie sehen fast genau wie Ihr Vater James aus, als er nach Hogwarts kam; bis auf die Augen, die haben Sie von Ihrer Mutter Lily. Und ich kann Ihnen alleine auf Grundlage Ihrer \emph{Persönlichkeit} versichern, dass Sie \emph{definitiv} mit dieser Geißel des Hauses Gryffindor verwandt sind.“

„\emph{Sie} könnte auch Teil der Verschwörung sein“, bemerkte Harry.

„Nein“, versicherte der alte Mann ihm mit zitternder Stimme. „Sie hat Recht. Du hast die Augen deiner Mutter.“

„Hm“, zögerte Harry. „Ich nehme an, \emph{Sie} könnten auch Teil der Verschwörung sein~…“

„Es reicht, Mr Potter“, sagte McGonagall.

Der alte Mann erhob seine Hand, als ob er Harry berühren wollte, aber ließ sie dann wieder fallen. „Ich bin nur froh, dass du lebst“, flüsterte er. „Danke, Harry Potter. Danke, für das, was du getan hast~… ich werde dich jetzt in Ruhe lassen.“

Das Klopfen seines Gehstocks entfernte sich langsam und bog aus dem Nebenweg wieder in die Winkelgasse ein.

McGonagall sah sich mit ernstem und angespanntem Gesichtsausdruck um. Auch Harry sah sich automatisch um. Doch bis auf einige vertrocknete Blätter schien es um sie herum leer zu sein und auch in der Winkelgasse waren nur vorbeispazierende Passanten zu sehen.

Schließlich schien McGonagall sich zu entspannen. „Das war nicht gut“, sagte sie leise. „Ich weiß, dass Sie das nicht gewohnt sind, Mr Potter, aber Sie bedeuten den Menschen sehr viel. Bitte seien Sie nett zu ihnen.“

Harry sah auf seine Schuhe herab. „Das sollten sie nicht“, sagte er mit einem bitteren Unterton. „Sich Gedanken über mich machen, meine ich.“

„Sie haben die Menschen von Sie-wissen-schon-wem befreit“, sagte McGonagall. „Warum, um alles in der Welt, sollten Sie den Menschen egal sein?“

Harry sah zu McGonagall auf und seufzte. „Ich nehme an, Sie können nicht viel damit anfangen, wenn ich sage, dass es sich um einen \emph{fundamentalen Attributionsfehler} handelt?“

McGonagall schüttelte den Kopf. „Nein, aber bitte erklären Sie es mir.“

„Nun~…“, sagte Harry und überlegte, wie er diese Erkenntnis der Muggelwissenschaften beschreiben könnte. „Nehmen wir an, Sie kommen auf Arbeit an und sehen, wie ihr Kollege gegen seinen Tisch tritt. Sie denken ‚was für ein aggressiver Kerl das doch sein muss`. Ihr Kollege hingegen wurde auf dem Weg zur Arbeit von einem Fremden zur Seite geschubst und angebrüllt. \emph{Jeder} wäre in der Situation wütend, denkt er sich. Wenn wir das Verhalten anderer Leute betrachten, neigen wir dazu, alles mit Persönlichkeitsmerkmalen zu erklären; wenn wir aber unser eigenes Verhalten betrachten, dann sehen wir die Umstände, die unsere Handlung erklären. Das Verhalten eines Menschen erscheint ihm selbst vollkommen schlüssig, aber wir sehen ihre Vergangenheit nicht, wir sehen nicht, wie sie sich in anderen Situationen verhalten. Also begehen wir einen fundamentalen Attributionsfehler, indem wir Dinge mit andauernden, unveränderlichen Eigenschaften erklären, obwohl sie sich viel besser durch den Kontext erklären ließen.“

Es gab einige elegante Experimente, die das bestätigten, aber Harry hatte nicht vor, diese jetzt zu erklären.

McGonagall zog die Augenbrauen hoch. „Ich glaube, ich verstehe es~…“, sagte sie zögerlich. „Aber was hat das mit Ihnen zu tun?“

Harry trat hart gegen die Backsteinmauer, kraftvoll genug, dass sein Fuß ihm wehtat. „Die Leute denken, dass ich sie vor Sie-wissen-schon-wem gerettet habe, weil ich eine Art großer Krieger des Lichts bin.“

„Der Eine mit der Macht, den Dunklen Lord zu besiegen~…“, murmelte McGonagall mit einem ironischen Unterton, den Harry zu dem Zeitpunkt noch nicht verstand.

„Genau“, sagte Harry, während Ärger und Frustration um die Vorherrschaft in seiner Stimme kämpften, „als ob ich den Dunklen Lord zerstört hätte, weil ich so eine Art dauerhafte Dunkle-Lords-zerstör-Eigenschaft habe. Ich war 15 Monate alt! Ich \emph{weiß} nicht mal, was damals passiert ist. Wenn ich raten müsste, würde ich sagen, dass es etwas mit irgendwelchen ungewissen Umgebungseinflüssen zu tun hat, aber sicher nicht mit meiner Persönlichkeit. Den Leuten geht es nicht um \emph{mich}, sie achten nicht einmal auf \emph{mich}, sie wollen nur \emph{ohne guten Grund} meine Hand schütteln.“ Harry zögerte und sah McGonagall an. „Wissen \emph{Sie}, was wirklich passiert ist?“

„Ich \emph{habe} eine Vermutung~…“, sagte McGonagall. „Zumindest seitdem ich Sie kennengelernt habe.“

„Nämlich?“

„Sie haben den Dunklen Lord besiegt, indem Sie schrecklicher als \emph{er} waren und den Todesfluch überlebt, indem Sie grausamer als der Tod selbst waren.“

„Ha. Ha. Ha.“ Harry trat wieder gegen die Mauer.

McGonagall schmunzelte. „Lassen Sie uns als nächstes zu Madam Malkins gehen. Ich glaube, Ihre Muggelkleidung könnte die Aufmerksamkeit anziehen.“

Unterwegs begegneten ihnen zwei weitere Gratulanten.

McGonagall blieb vor der Tür von \emph{Madam Malkins Umhänge für alle Anlässe} stehen. Der Laden hatte eine äußerst langweilige Fassade, bestehend aus ziegelrotem Backstein und Glasfenstern, hinter denen normale schwarze Umhänge ausgestellt waren. Keine Umhänge, die glitzerten, ihre Form veränderten, herumwirbelten oder seltsame Strahlen aussandten, welche bis auf die Haut vordrangen und kitzelten~-- einfach nur schwarze Umhänge. Zumindest war das alles, was man durch die Schaufenster sehen konnte. Die Tür stand weit offen, als ob sie beweisen wollte, dass es hier nichts zu verbergen gab.

„Ich werde ein paar Minuten weggehen, während Sie sich Umhänge schneidern lassen“, sagte McGonagall. „Ist das in Ordnung?“

Harry nickte. Er pflegte einen leidenschaftlichen Hass gegenüber dem Kauf von Kleidung und konnte es McGonagall nicht verdenken, dass es ihr ähnlich ging.

McGonagall tippte mit ihrem Zauberstab auf seinen Kopf. „Madam Malkin sollte beim Anprobieren der Umhänge nicht getäuscht werden, daher hebe ich die Verschleierung auf.“

„Ähm~…“, sagte Harry. Das bereitete ihm einige Sorgen.

„Ich bin mit Madam Malkin nach Hogwarts gegangen“, sagte McGonagall. „Bereits damals war sie eine sehr gefasste Person. Sie würde nicht einmal blinzeln, wenn Sie-wissen-schon-wer höchstpersönlich ihr Geschäft betreten würde.“ McGonagalls Stimme klang nostalgisch und sehr angetan. „Madam Malkin wird Sie nicht belästigen und sie wird nicht zulassen, dass irgendjemand anders Sie belästigt.“

„Wohin gehen \emph{Sie}?“, erkundigte Harry sich. „Nur für den Notfall.“

McGonagall sah Harry skeptisch an. „Ich werde \emph{dort} sein“, sagte sie und deutete auf ein Gebäude auf der anderen Straßenseite, über dessen Eingangstür ein hölzernes Fässchen angebracht war, „und werde mir einen Drink genehmigen, den ich dringend nötig habe. \emph{Sie} werden Umhänge anprobieren, \emph{sonst nichts}. Ich werde in Kürze wiederkommen und erwarte, dass Madam Malkins Geschäft immer noch steht und \emph{keinesfalls} in Flammen aufgegangen ist.“

Madam Malkin war eine geschäftige alte Frau, die kein Wort sagte, als sie die Narbe auf Harrys Stirn sah und einer Gehilfin einen scharfen Blick zuwarf, als das Mädchen gerade den Mund aufmachte. Madam Malkin holte einige selbstmessende Maßbänder herbei und machte sich an die Arbeit.

Neben Harry stand ein blasser Junge mit einem schmalen Gesicht und \emph{unglaublich coolem} weißblondem Haar, der gerade die letzten Schritte dieser Prozedur durchstand. Eine der zwei Gehilfinnen inspizierte den weißhaarigen Jungen und seinen schachbrett-gemusterten Umhang sorgfältig, gelegentlich tippte sie mit ihrem Zauberstab auf eine Ecke des Umhangs und der Stoff zog sich zusammen oder weitete sich etwas.

„Hallo“, sagte der Junge. „Auch Hogwarts?“

Harry konnte sich schon denken, worauf dieses Gespräch hinauslief und beschloss in Sekundenbruchteilen, dass er jetzt genug davon hatte.

„Um Himmels Willen“, flüsterte Harry, „das kann nicht sein.“ Er machte große Augen. „Sie~… Sie heißen, Sir?“

„Draco Malfoy“, sagte Draco Malfoy leicht irritiert.

„Sie sind es! Draco Malfoy. Ich~-- ich hätte nie gedacht, dass mir jemals diese Ehre zuteil würde, Sir.“ Harry wünschte, er könnte jetzt einige Tränen aus den Augen drücken. Normalerweise fingen die Leute ungefähr in diesem Moment an zu weinen.

„Oh“, sagte Draco, der immer noch verwirrt klang. Dann breitete sich ein selbstgefälliges Lächeln auf seinem Gesicht aus. „Es tut gut, jemanden zu treffen, der weiß, wo er hingehört.“

Eine Gehilfin, die Harry bereits erkannt hatte, erstickte ihr Kichern.

Harry plapperte weiter. „Es ist so eine Freude, Sie zu sehen, Mr Malfoy. So eine unglaubliche Freude. Und im selben Jahr wie Sie in Hogwarts anzufangen! Mein Herz blüht auf!“

Ups. Der letzte Satz hatte sich etwas seltsam angehört, als ob er Draco anbaggern wollte.

„Auch mein Herz erfreut es, zu sehen, dass ich erwarten kann, mit dem der Familie Malfoy zustehenden Respekt behandelt zu werden“, erwiderte der andere Junge, begleitet von einem Lächeln, wie es der höchste aller Könige dem niedrigsten seiner Untertanen widmen würde, wenn dieser Untertan ehrenwert, aber armselig war.

Ähm~… verdammt, Harry wusste nicht, wie er fortfahren sollte. Nun, ungefähr an dieser Stelle des Gesprächs wollten die Leute üblicherweise Harry Potter die Hand schütteln, also~-- „Sobald meine Umhänge fertig sind, Sir, würden Sie mir gewähren, Ihnen die Hand zu schütteln? Nichts täte ich lieber als dies, es würde diesem Tag, nein, diesem Monat, nein, meinem Leben seinen Höhepunkt verleihen.“

Draco sah streng zurück. „Ich finde, du bittest meine Person um einen ungerechtfertigten Gefallen. Was hast du jemals für die Familie Malfoy getan, das dir das Recht geben würde, um so etwas zu bitten?“

\emph{Oh, ich muss das definitiv beim nächsten Menschen, der mir die Hand schütteln will, wiederholen.} Harry senkte seinen Kopf. „Nein, nein, Sir, ich verstehe vollkommen. Bitte verzeihen Sie die Frage. Es sollte mir vielmehr eine Ehre sein, Ihre Stiefel zu säubern.“

„In der Tat“, zischte Draco. Sein ernstes Gesicht hellte etwas auf. „Obwohl dein Wunsch natürlich leicht verständlich ist. Sag, in welches Haus wirst du kommen? Ich werde selbstverständlich im Haus Slytherin sein, wie mein Vater Lucius vor mir. Und was dich betrifft, denke ich mal: Haus Hufflepuff --- oder vielleicht Haus… elf?“

Harry grinste zurückhaltend. „Professor McGonagall sagt, ich wäre die Ravenclaw-artigste Person, die sie je gesehen oder von der sie je gehört hat --- so sehr, dass Rowena selbst mir sagen würde, ich solle mehr hinausgehen; was auch immer \emph{das} heißen soll --- und dass ich garantiert ins Haus Ravenclaw käme, wenn der Sprechende Hut nicht vor Schreck so laut aufschreit, dass wir alle taub werden. Zitat Ende.“

„Wow“, sagte Draco und klang leicht beeindruckt. Er seufzte wehmütig. „Deine Schmeichelei war aber großartig, zumindest gefiel sie mir sehr~-- du würdest auch gut nach Slytherin passen. Üblicherweise kriechen die Leute nur vor meinem Vater nieder. Aber ich \emph{hoffe}, dass die anderen Slytherins auch so schleimen, wenn ich in Hogwarts bin~… ich denke mal, das war schon ein ganz gutes Zeichen.“

Harry hustete. „Tut mir Leid, aber um ehrlich zu sein habe ich keine Ahnung, wer du bist.“

„Ach \emph{komm schon}!“, rief Draco empört aus. „Warum würdest du dann so etwas tun?“ Dracos Augen weiteten sich plötzlich, als er einen Verdacht schöpfte. „Und wie kann es sein, dass du \emph{die Malfoys} nicht kennst? Und was für \emph{Sachen} hast du denn an? Sind deine Eltern \emph{Muggel}?“

„Zwei meiner Eltern sind tot“, sagte Harry. Wenn er es so ausdrückte, versetzte es ihm einen Stich mitten ins Herz. „Meine anderen zwei Eltern sind Muggel und bei ihnen bin ich aufgewachsen.“

„\emph{Was?}“, fragte Draco. „Wer \emph{bist} du?“

„Harry Potter. Schön, dich kennenzulernen.“

„\emph{Harry Potter?}“, keuchte Draco. „\emph{Der} Harry~--“ und er brach abrupt ab.

Einen Moment lang war es still.

Dann, mit eifriger Begeisterung: „Harry Potter? \emph{Der} Harry Potter? Klasse, ich wollte dich schon immer kennenlernen!“

Die Gehilfin an Dracos Seite machte ein Geräusch, als ob sie erstickte, aber fuhr mit ihrer Arbeit fort und hob Dracos Arme um ihm vorsichtig den Umhang mit Schachbrettmuster abzunehmen.

„Sei leise“, empfahl Harry ihm.

„Kann ich ein Autogramm von dir haben? Nein, warte, ich will erstmal ein Foto mit dir!“

„Halt's\emph{Maul}Halt's\emph{Maul}Halt's\emph{Maul}.“

„Es ist einfach so eine \emph{Freude}, dich zu treffen!“

„Geh in Flammen auf und stirb.“

„Aber du bist Harry Potter, der ruhmreiche Retter der Zaubererwelt, Bezwinger des Dunklen Lords! Harry Potter, jedermanns Held! Wenn ich groß bin, wollte ich immer so werden wie du, damit ich auch Dunkle Lords besiegen~--“

Draco brach mitten im Satz ab. Sein Gesicht erstarrte schockiert.

Groß, weißhaarig, mit kühler Eleganz und in schwarzem Umhang von feinster Qualität. In einer Hand ein silbern besetzter Spazierstock, der zur todbringenden Waffe wurde, allein indem er in dieser Hand gehalten wurde. Die Augen überwachten den ganzen Raum mit dem leidenschaftslosen Ausdruck eines Scharfrichters; eines Mannes, für den Töten nicht schmerzhaft war, nicht einmal reizend verboten, sondern ebenso gewöhnlich wie das Atmen. Automatisch kam einem das Wort \emph{Perfektion} in den Sinn.

Dieser Mann war soeben durch die offene Tür eingetreten.

„Draco“, sagte der Mann in tiefer und sehr wütender Stimme, „\emph{was} sagst du da?“

Innerhalb von Sekundenbruchteilen des Mitgefühls fasste Harry einen Rettungsplan.

„Lucius Malfoy!“, keuchte Harry Potter. „\emph{Der} Lucius Malfoy?“

Eine von Madam Malkins Gehilfinnen drehte sich zur Wand.

Kalte, mörderische Augen betrachteten ihn. „Harry Potter.“

„Es ist mir solch eine Ehre, Sie zu treffen!“

Die dunklen Augen weiteten sich, die Todesdrohung wurde durch schockierte Überraschung verdrängt.

„Ihr Sohn hat mir \emph{alles} über Sie erzählt“, plapperte Harry weiter, ohne dass er wusste, was genau er da so schnell wie möglich sagen wollte. „Aber natürlich wusste ich schon vorher alles über Sie, schließlich kennt jeder Sie, den großen Lucius Malfoy! Den ehrenvollsten Absolventen des Hauses Slytherin! Ich habe darüber nachgedacht, selbst nach Slytherin zu wollen, nur weil ich gehört habe, dass Sie als Kind dort~--“

„\emph{Was sagen Sie da, Mr Potter?}“, kam fast ein Schrei von draußen und eine Sekunde später war Professor McGonagall in das Geschäft hineingeplatzt.

Ihr Gesicht war so schreckensverzerrt, dass Harry automatisch den Mund öffnete und dann mit offenem Mund dastand, weil er nicht wusste, was er sagen sollte.

„Professor McGonagall“, rief Draco aus. „Sind Sie es wirklich? Ich habe von meinem Vater so viel über Sie gehört, ich habe überlegt, ob ich versuche, nach Gryffindor zu kommen, damit ich~--“

„\emph{Was?}“, brüllten Lucius Malfoy und Professor McGonagall vollkommen gleichzeitig. Ihre Köpfe drehten sich im selben Moment zueinander und dann wichen beide voneinander zurück, als ob sie einen Synchrontanz vorführten.

Einen Moment lang herrschte Hektik, als Lucius Draco anpackte und aus dem Geschäft zog.

Dann war es still.

McGonagall sah auf das kleine Weinglas nieder, das sie in ihrer Hand gehalten hatte. Sie hatte es für einen Moment ganz vergessen und so hatte es sich zur Seite geneigt und enthielt nur noch wenige rote Tropfen.

Sie ging einige Schritte vor, bis sie Madam Malkin gegenüberstand.

„Madam Malkin“, sagte McGonagall mit ruhiger Stimme. „Was ist hier passiert?“

Madam Malkin blickte ihr vier Sekunden lang stumm in die Augen und konnte sich dann nicht mehr halten. Sie ließ sich unter schallendem Gelächter gegen die Wand fallen und steckte damit ihre Gehilfinnen an, eine von ihnen stützte sich auf Händen und Knien am Boden ab und kicherte hysterisch.

McGonagall drehte sich langsam und mit sehr kühlem Gesichtsausdruck zu Harry um. „Ich habe Sie für fünf Minuten allein gelassen. Genau fünf Minuten, Mr Potter.“

„Ich habe mir nur einen Scherz erlaubt“, wandte Harry ein, während im Hintergrund weiterhin hysterisches Lachen erklang.

„\emph{Draco Malfoy hat vor seinem Vater gesagt, dass er nach Gryffindor möchte!} Ein Scherz \emph{reicht dafür nicht aus}!“ McGonagall hielt inne und atmete schwer. „Welcher Teil von ‚lassen Sie sich Umhänge schneidern` klang für Sie nach ‚bitte sprechen Sie einen \emph{Confundus-Zauber} auf das gesamte Universum`?“

„Er befand sich in einem situationsbezogenen Kontext, in dem seine Handlungen Sinn ergaben~…“

„Nein. Erklären Sie nicht. Ich will nicht wissen, was hier passiert ist. Nie. Es gibt Dinge, die nicht dafür gemacht sind, dass ich sie verstehe. Dies hier gehört dazu. Was für eine dämonische Macht des Chaos' auch immer in Ihnen tobt, sie ist \emph{ansteckend} und ich will nicht so enden wie der arme Draco Malfoy, die arme Madam Malkin und ihre armen beiden Gehilfinnen.“

Harry seufzte. Es war offensichtlich, dass McGonagall nicht in der Stimmung war, vollkommen vernünftigen Erklärungen zuzuhören. Er sah zu Madam Malkin, die immer noch gegen die Wand gelehnt schnaufte, zu ihren Gehilfinnen, die inzwischen beide auf die Knie gefallen waren und schließlich auf seinen von Maßbändern umhüllten Körper.

„Ich bin mit den Umhängen noch nicht fertig“, sagte Harry sanft. „Wollen Sie nicht zurück gehen und noch etwas trinken?“

-\/-\/-\/-\/-\/-\/-\/-\/-\/-\/-\/-\/-\/-\/-\/-\/-\/-\/-\/-\/-\/-\/-\/-\/-\/-\/-\/-\/-\/-

Attributionsfehler:\\ (Die Erklärung befindet sich im Text.)\\ Ein ganz praktisches Beispiel in drei Schritten:

1) Wenn ihr das nächste Mal beim Einkaufen an der Kasse Schlange steht oder in den Bus einsteigt, schaut doch mal den Kassierer oder Busfahrer an und überwacht eure Gedanken. Ganz ehrlich: Ertappt ihr euch bei dem Gedanken "Das ist ja ein langweiliger Miesepeter"?\\ 2) Denkt mal nach: Ist es tatsächlich ein langweiliger Miesepeter? Oder ist es nicht wahrscheinlicher, dass es ein ganz normaler Mensch ist, der es seit Jahren gewohnt ist, von den Kunden bzw. Fahrgästen nicht beachtet oder unfreundlich behandelt zu werden?\\ 3) Lächelt ihn doch mal an! Wünscht ihm einen guten Morgen/Tag/Abend! Bedankt euch!

Wenn ihr nach Schritt 3 angelächelt und zurückgegrüßt werdet (und das geschieht meiner Erfahrung nach in den meisten Fällen!), dann wisst ihr, dass ihr am Ende von Schritt 1 einen Attributionsfehler begangen habt.\\ Ist doch ganz einfach, oder? ;)

