

\hypertarget{positive-bias}{% \section{8. Positive Bias}\label{positive-bias}}

„\emph{Erlaube mir, dich zu warnen, dass es ein gefährliches Unterfangen ist, meine Genialität in Frage zu stellen; es könnte deinen Tag um einiges surrealer machen.}“

\later

Niemand hatte um Hilfe gebeten, das war das Problem. Sie waren nur umhergegangen, hatten geschwatzt, gegessen oder in die Luft gestarrt, während ihre Eltern tratschten. Aus welchem seltsamen Grund auch immer hatte niemand herumgesessen und ein Buch gelesen, was bedeutete, dass sie sich nicht daneben setzen und ihr eigenes Buch herausholen konnte. Und selbst als sie mutig die Initiative ergriffen hatte, indem sie sich selbst hinsetzte und begann, ihre \emph{Geschichte Hogwarts'} zum dritten Mal durchzulesen, hatte sich niemand dazu veranlasst gefühlt, sich neben sie zu setzen.

Außer indem sie bei Hausaufgaben oder bei irgendetwas anderem half, wusste sie nicht, wie sie Leute kennenlernen sollte. Nicht, dass sie sich besonders schüchtern \emph{fühlte}. Sie sah sich selbst als jemand, der die Dinge in die Hand nimmt. Und doch—solange es keine Bitte der Art „Ich weiß nicht mehr, wie man schriftlich dividiert“ gab, war es einfach zu \emph{peinlich}, zu Leuten hinzugehen und…\emph{was} zu sagen? Sie hatte nie herausfinden können, was.

Und es schien kein Standardinfoblatt zu geben. Das war lächerlich. Das komplette Konzept des Leute-Kennenlernens war ihr nie besonders sinnvoll vorgekommen. Warum musste sie alles tun, wenn zwei Leute daran beteiligt waren? Warum halfen Erwachsene nie? Sie wünschte sich, ein anderes Mädchen würde zu \emph{ihr} kommen und sagen, „Hermine, die Lehrerin sagte, ich solle mich mit dir anfreunden.“

Aber es muss hier erwähnt werden, dass Hermine Granger, allein an ihrem ersten Schultag in einer der wenigen leeren Kabinen des Zuges sitzend, mit offener Tür, falls jemand aus irgendeinem Grunde mit ihr reden wollen sollte, \emph{nicht} traurig, einsam, deprimiert, verzweifelt oder von ihren Problemen besessen war. Sie war vielmehr dabei, die \emph{Geschichte Hogwarts'} zum dritten Mal zu lesen und hatte durchaus ihre Freude daran, gemischt mit nur einer Spur Genervtheit angesichts der generellen Unvernunft der Welt.

Es ertönte das Geräusch einer sich öffnenden Wagentür, gefolgt von Schritten und einem seltsamen, kriechenden Geräusch, das den Gang entlang kam. Hermine legte die \emph{Geschichte Hogwarts'} beiseite und steckte den Kopf durch die Tür—nur für den Fall, dass jemand Hilfe brauchte—und sah einen Jungen im Zaubererumhang, seiner Größe nach zu urteilen vermutlich im ersten oder zweiten Jahr, der ziemlich albern aussah, da er einen Schal um seinen Kopf gewickelt hatte. Ein kleiner Koffer stand neben ihm auf dem Boden. Als sie ihn sah, klopfte er gerade an die Tür eines anderen Abteils und sagte in einer vom Schal leicht gedämpften Stimme, „Entschuldigt mich, kann ich schnell eine Frage stellen?“

Sie hörte die Antwort nicht, die aus der Kabine kam, aber nachdem der Junge die Tür geöffnet hatte, dachte sie zu hören—es sei denn, sie hatte ihn missverstanden—„Kann mir einer von euch die Namen der sechs Quarks nennen, oder weiß jemand, wo ich ein Mädchen namens Hermine Granger finden kann?“

Nachdem der Junge die Kabinentür geschlossen hatte, sagte Hermine, „kann ich dir irgendwie helfen?“

Der schalumwickelte Kopf wandte sich ihr zu, und die Stimme sagte: „Nicht, wenn du nicht die Namen der sechs Quarks kennst oder mir sagen kannst, wo ich eine Erstklässlerin namens Hermine Granger finde.“

„Up, down, strange, charm, truth, beauty—und warum suchst du nach einer Erstklässlerin namens Hermine Granger?“

Es war auf die Distanz schwer zu erkennen, aber sie dachte, den Jungen hinter dem Schal breit grinsen zu sehen. „Ach, \emph{du} bist die Erstklässlerin namens Hermine Granger“, sagte die junge, gedämpfte Stimme. „Und sogar im Zug nach Hogwarts.“ Der Junge fing an, in ihre Richtung zu gehen, und sein Koffer kroch hinter ihm her. „Genau genommen sollte ich nur nach dir suchen, aber es scheint mir naheliegend, dass ich auch mit dir reden soll. Oder dich in meine Abenteuergruppe einladen, oder ein bedeutendes magisches Artefakt von dir erhalten oder herausfinden, dass Hogwarts auf den Ruinen eines alten Tempels erbaut wurde, oder so etwas. SC oder NSC, das ist die Frage.“

Hermine öffnete den Mund um zu antworten, aber ihr fiel keine \emph{geeignete} Antwort auf…was immer das gerade gewesen sein mochte ein, selbst als der Junge zu ihr herüberkam, in das Abteil schaute, zufrieden nickte, und sich auf der leeren Bank gegenüber von ihrem Platz niederließ, wo immer noch das Buch lag. Sein Koffer wuselte hinter ihm her, wuchs auf dreifache Größe, und schmiegte sich in einer beunruhigenden Art und Weise an ihren Koffer.

„Bitte, setz dich“, sagte der Junge, „und sei so nett, schließ doch die Tür hinter dir. Keine Angst, ich beiße niemanden, der mich nicht zuerst beißt.“ Er war bereits dabei, den Schal um seinen Kopf abzuwickeln.

Dass dieser Junge ihr unterstellte, Angst vor ihm zu haben, reichte aus, damit ihre Hand, die Tür mit unnötig viel Kraft zuwarf. Sie wirbelte herum und blickte in ein junges Gesicht mit hellen, lachenden grünen Augen, und einer zornroten Narbe auf seiner Stirn, die sie hintergründig an etwas erinnerte, aber sie hatte in diesem Moment wichtigere Dinge im Kopf. „Ich habe nicht gesagt, dass ich Hermine Granger bin!“

„\emph{Ich} habe nicht gesagt, dass du \emph{gesagt hast,} dass du Hermine Granger bist, sondern dass du Hermine Granger \emph{bist}. Falls du dich fragst, woher ich das weiß—es liegt daran, dass ich alles weiß. Guten Abend, meine Damen und Herren, mein Name ist Harry James Potter-Evans-Verres, oder kurz Harry Potter, und ich denke mal, dass \emph{dir} das ausnahmsweise mal gar nichts sagt —“

Endlich stellte Hermines Verstand den Zusammenhang her. Die Narbe auf seiner Stirn, in der Form eines Blitzes! „Harry Potter! Du stehst in \emph{Geschichte der modernen Magie} und \emph{Der Aufstieg und Untergang der Dunklen Künste} und \emph{Große Chronik der Zauberei des zwanzigsten Jahrhunderts!}“ Es war das erste Mal in ihrem Leben, dass sie tatsächlich jemandem aus einem Buch begegnete, und es war ein eher seltsames Gefühl.

Der Junge blinzelte dreimal. „Ich stehe in \emph{Büchern?} Warte, natürlich stehe ich in Büchern…— Was für ein seltsamer Gedanke.“

„Meine Güte, hast du das nicht gewusst?“, sagte Hermine. „Ich jedenfalls hätte alles über mich herausgefunden, wenn ich du gewesen wäre.“

Der Junge sprach ziemlich trocken. „Miss~Hermine Granger, es ist weniger als zweiundsiebzig Stunden her, dass ich die Winkelgasse besuchte und von meiner Berühmtheit erfuhr. Ich habe die letzten zwei Tage damit verbracht, Wissenschaftsbücher zu kaufen. \emph{Glaub mir,} ich habe vor, alles herauszufinden, was ich kann.“ Der Junge zögerte. „Was \emph{sagen} denn die Bücher über mich?“

Hermine versuchte sich zu erinnern. Sie hatte nicht damit gerechnet, über \emph{diese} Bücher ausgefragt zu werden, deswegen hatte sie sie nur einmal gelesen, aber es war erst einen Monat her, deswegen hatte sie den Inhalt noch im Kopf. „Du bist der einzige, der je den Todesfluch überlebte, deswegen nennt man dich den Jungen, der lebt. Du wurdest als Sohn von James Potter und Lily Potter, ehemals Lily Evans, am 31. Juli 1980 geboren. Am 31. Oktober 1981 griff der Dunkle Lord—Er, dessen Name nicht genannt werden darf, auch wenn ich nicht weiß, wieso nicht—euer Haus an, dessen Ort von Sirius Black verraten wurde, auch wenn da nicht drinstand, woher sie wussten, dass er es war. Du wurdest lebend mit der Narbe auf deiner Stirn in den Ruinen deines Elternhauses neben den verbrannten Überresten von ihm, dessen Name nicht genannt werden darf, gefunden. Großmeister Albus Percival Wulfric Brian Dumbledore hat dich irgendwohin gebracht, aber niemand weiß wohin. \emph{Der Aufstieg und Untergang der Dunklen Künste} behauptet, dass du wegen der Liebe deiner Mutter überlebtest, und dass deine Narbe die gesamte Macht des Dunklen Lords enthält, und dass die Zentauren dich fürchten, aber \emph{Große Chronik der Zauberei des zwanzigsten Jahrhunderts} erwähnt nichts dergleichen und \emph{Geschichte der modernen Magie} warnt, dass es jede Menge verrückter Theorien um dich gibt.“

Der Mund des Jungen hing offen. „Wurde dir gesagt, dass du im Zug nach Hogwarts auf Harry Potter warten sollst, oder irgendwas in der Art?“

„Nein. Wer hat dir von \emph{mir} erzählt?“

„Professor McGonagall, und ich glaube, ich verstehe, warum. Hast du ein eidetisches Gedächtnis, Hermine?“

Hermine schüttelte den Kopf. „Es ist nicht fotografisch. Ich habe mir immer gewünscht, dass es das wäre, aber ich musste all meine Schulbücher fünf mal lesen, um sie auswendig zu können.“

„Wirklich“, sagte der Junge mit einer leicht erstickten Stimme. „Ich hoffe, dass es dich nicht stört, wenn ich das teste—nicht, weil ich dir nicht glaube, aber wie das Sprichwort sagt: Vertrauen ist gut, Kontrolle ist besser. Es wäre sinnlos, darüber zu grübeln, wenn ich es einfach per Experiment herausfinden kann.“

Hermine grinste selbstzufrieden. Sie liebte Tests. „Leg los.“

Der Junge steckte eine Hand in seinen Beutel und sagte „Zaubertränke und Zauberbräue von Arsenius Bunsen“. Als er seine Hand wieder hervorzog, hielt er in ihr das genannte Buch.

Sofort wünschte sich Hermine so einen Beutel mehr als alles andere.

Der Junge öffnete das Buch irgendwo in der Mitte und blickte hinein. „Wenn du ein \emph{Öl der Schärfe} machen wolltest —“

„Ich kann die Seite von hier aus \emph{sehen!}“

Der Junge kippte das Buch, so dass sie nicht mehr hereinschauen konnte, und blätterte einige Seiten weiter. „Wenn du einen \emph{Trank des Spinnenlaufs} brauen wolltest, was wäre die nächste Zutat, nachdem du die Acromantula-Seide hinzufügtest?“

„Nach dem Hinzugeben der Seide, warte, bis der Trank genau die Farbe des wolkenlosen Morgenhimmels hat, 8 Grad vom Horizont und 8 Minuten bevor die Sonne zu sehen ist. Rühre achtmal gegen den Uhrzeigersinn, einmal im Uhrzeigersinn, und füge dann acht Quäntchen Einhornpopel hinzu.“

Der Junge schloss das Buch mit einem Klatschen, und tat es zurück in seinen Beutel, welcher es mit einem kleinen Rülpslaut verschlang. „Sehr sehr \emph{sehr} sehr gut. Ich möchte dir hiermit ein Angebot machen, Hermine Granger.“

„Ein Angebot?“, sagte Hermine misstrauisch. Mädchen sollten auf so etwas nicht hören.

In diesem Moment bemerkte Hermine auch die andere Sache—naja, eine andere Sache—die an dem Jungen ungewöhnlich war. Augenscheinlich war es nämlich so, dass Leute, die \emph{in} Büchern waren, \emph{wie} Bücher klangen, wenn sie sprachen. Es war eine ziemlich überraschende Feststellung.

Der Junge griff in seinen Beutel, sagte „Limonadendose“ und zog einen hellgrünen Zylinder hervor. Er hielt ihn hin und sagte, „Kann ich dir etwas zu trinken anbieten?“

Hermine nahm die Dose höflich an. In der Tat \emph{war} sie gerade etwas durstig. „Vielen Dank“, sagte Hermine, als sie den Deckel öffnete. „War das dein Angebot?“

Der Junge hustete. „Nein“, sagte er. Gerade als Hermine zu trinken begann, sagte er, „Ich möchte, dass du mir bei der Eroberung des Universums hilfst.“

Hermine trank zu Ende und senkte die Limo. „Nein danke, aber ich bin nicht böse.“

Der Junge schaute sie überrascht an, so, als hätte er eine andere Antwort erwartet. „Naja, ich habe etwas rhetorisch gesprochen“, sagte er. „Im Sinne des Baconschen Projekts, weißt du, nicht politische Macht. 'Die Bewerkstelligung aller möglichen Dinge` und so weiter. Ich möchte experimentelle Studien an Zaubersprüchen durchführen, die zugrundeliegenden Gesetze herausfinden, die Magie in den Bereich der Wissenschaft bringen, die Zauberer- und Muggelwelt zusammenführen, den Lebensstandard aller auf diesem Planeten heben, die Menschheit hunderte von Jahren voranbringen, das Geheimnis der Unsterblichkeit entdecken, das Sonnensystem bevölkern, die Galaxie erforschen, und wichtiger noch, herausfinden, was zum Teufel hier eigentlich los ist, denn das alles hier ist einfach nur unmöglich.“

Das klang etwas interessanter. „Und?“

Der Junge schaute sie ungläubig an. „\emph{Und?} Das ist \emph{nicht genug?}“

„Und was möchtest du von mir?“ sagte Hermine.

„Ich möchte natürlich, dass du mir bei meinen Forschungen hilfst. Mit deinem enzyklopädischen Gedächtnis, zusammen mit meiner Intelligenz und meiner Rationalität werden wir das Baconsche Projekt in kürzester Zeit beendet haben, wobei ich mit 'kürzester Zeit' wahrscheinlich 35 Jahre oder mehr meine.“

Hermine begann, diesen Jungen nervig zu finden. „Ich habe dich noch nichts Intelligentes machen sehen. Vielleicht erlaube ich \emph{dir}, mir bei \emph{meinen} Studien zu helfen.“

Eine bestimmte Art von Stille füllte das Abteil.

„Du möchtest also eine Demonstration meiner Intelligenz“, sagte der Junge nach einer langen Pause.

Hermine nickte.

„Erlaube mir, dich zu warnen, dass es ein gefährliches Unterfangen ist, meine Genialität in Frage zu stellen; es könnte deinen Tag um einiges surrealer machen.“

„Bisher bin ich nicht beeindruckt“, sagte Hermine. Ihre Hand mit der Limonadendose darin hob sich wieder zu ihrem Mund.

„Na, dann wird \emph{das} dich vielleicht beeindrucken“, sagte der Junge. Er lehnte sich vor und schaute sie durchdringend an. „Ich habe bereits ein wenig experimentiert, und herausgefunden, dass ich meinen Zauberstab nicht brauche. Ich kann alles, was ich will, einfach geschehen lassen, indem ich mit den Fingern schnippe.“

Das sagte er genau in dem Moment, als Hermine gerade schluckte, und sie verschluckte sich und hustete und spuckte die hellgrüne Flüssigkeit wieder aus. Auf ihren nagelneuen, noch nie getragenen Zaubererumhang, am allerersten Schultag.

Hermine schrie auf. Es war ein hoher, durchdringender Ton, der in dem geschlossenen Abteil wie ein Fliegeralarm klang. „\emph{Ihks! Meine Kleidung!}“

„Keine Panik!“, sagte der Junge. „Ich kann es wieder gut machen. Schau hin!“ Er hob seine Hand und schnippte mit den Fingern.

„Du —“ Dann schaute sie an sich runter.

Die grüne Flüssigkeit war noch da, aber sie verschwand beim Hinschauen, und innerhalb weniger Momente war es, als hätte sie nie die Limo auf dem Umhang verteilt.

Hermine starrte den Jungen an, der ein selbstgefälliges Lächeln aufgesetzt hatte.

Wortlose stablose Magie! In \emph{seinem} Alter! Wo er doch seine Bücher erst vor \emph{drei Tagen} erhalten hatte?

Dann erinnerte sie sich an das, was sie gelesen hatte, zog den Atem ein und zuckte von ihm weg. \emph{Die gesamte Macht des dunklen Lords! In seiner Narbe!}

Hastig stand sie auf. „Ich, ich, ich muss auf die Toilette, warte hier, okay? —“ Sie musste einen Erwachsenen finden, sie musste es jemandem sagen—

Das Lächeln des Jungen verschwand. „Es war nur ein Trick, Hermine. Es tut mir leid, ich hatte nicht vor, dich zu verängstigen.“

Ihre Hand hielt am Türgriff inne. „Ein \emph{Trick?}“

„Ja“, sagte der Junge. „Du hast mich gebeten, meine Intelligenz zu demonstrieren. Ich tat also etwas augenscheinlich Unmögliches—immer eine gute Art anzugeben. Ich kann nicht \emph{wirklich} alles durch ein Fingerschnippen geschehen lassen.“ Der Junge machte eine Pause. „Zumindest \emph{denke} ich nicht, dass ich das kann, ich habe es nie probiert.“ Der Junge hob seine Hand und schnippte noch einmal mit den Fingern. „Nein, keine Banane.“

Hermine war so verwirrt, wie sie es in ihrem ganzen Leben noch nicht gewesen war.

Der Junge lächelte nun wieder über den Ausdruck in ihrem Gesicht. „Ich hatte dich \emph{gewarnt}, dass es ein gefährliches Unterfangen ist, meine Genialität in Frage zu stellen. Denke daran, wenn ich dich das nächste Mal vor etwas warne.“

„Aber, aber“, stotterte Hermine, „was hast du denn \emph{dann} getan?“

Der Blick des Jungen nahm einen einschätzenden und bewertenden Ausdruck an, den sie noch nie von jemandem in ihrem Alter gesehen hatte. „Du denkst, dass du es in dir hast, eine Wissenschaftlerin zu werden, egal, ob ich dir helfe oder nicht? Dann zeig mal, wie \emph{du} ein verwirrendes Phänomen untersuchst.“

„Ich …“ Hermines Gehirn setzte für einen Moment aus. Sie liebte es, getestet zu werden, aber \emph{so} eine Aufgabe hatte sie noch nie gestellt bekommen. Hastig durchstöberte sie ihre Erinnerungen nach allen Dingen, die ein Wissenschaftler tun sollte. Ihr Gehirn schaltete sich ein, fing an, hart zu arbeiten, und spuckte schließlich die Anweisungen zur Durchführung eines „Jugend forscht“-Projekts aus:

\emph{Schritt 1: Stelle eine Hypothese auf.

Schritt 2: Führe ein Experiment durch, um deine Hypothese zu testen.

Schritt 3: Miss~das Resultat.

Schritt 4: Stelle dein Ergebnis auf einem Plakat dar.}

Der erste Schritt war, eine Hypothese aufzustellen. Das bedeutete, sich auszudenken, was gerade passiert sein \emph{könnte}. „Also gut. Meine Hypothese ist es, dass du einen Zauberspruch auf meinen Umhang gesprochen hast, der die darauf verschüttete Limonade verschwinden ließ.“

„Okay“, sagte der Junge, „ist das deine Antwort?“

Der Schock ließ nach, und Hermines Gedanken fingen an, richtig zu funktionieren. „Warte, das ist keine gute Idee. Ich habe dich weder deinen Zauberstab berühren sehen, noch irgendwelche Worte sprechen hören, wie könntest du also einen Zauberspruch gesprochen haben?“

Der Junge wartete mit neutralem Gesichtsausdruck.

„Aber nehmen wir mal an, die Umhänge werden bereits mit einem Selbstreinigungszauber verkauft, was ein sehr nützlicher Zauber für sie wäre. Das hast du vorher herausgefunden, als du \emph{selber} etwas verkleckert hast.“

Nun hoben sich die Augenbrauen des Jungen. „Ist \emph{das} deine Antwort?“

„Nein, ich habe noch nicht Schritt 2 durchgeführt, 'Führe ein Experiment durch, um deine Hypothese zu testen.“

Der Junge schloss den Mund wieder, und begann zu lächeln.

Hermine schaute die Limonadendose in ihrer Hand an, welche sie wie automatisch im Tassenhalter am Fenster abgestellt hatte. Sie betrachtete sie, und stellte fest, dass sie etwa zu einem Drittel voll war.

„Nun gut“, sagte Hermine, „das Experiment, was ich durchführen will, ist, die Limo auf meinen Umhang zu gießen und zu sehen, was passiert, und meine Vorhersage ist, dass sie verschwinden wird. Nur—wenn es \emph{nicht} klappt, beschmutze ich mir meinen Umhang, und das will ich nicht.“

„Tu es mit meinem“, sagte der Junge, „dann musst du dir keine Sorgen machen, dass deiner schmutzig wird.“

„Aber —“, sagte Hermine. Irgendwas war an dieser Art zu denken \emph{falsch}, aber sie konnte nicht genau sagen was.

„Ich habe weitere Umhänge in meinem Koffer“, sagte der Junge.

„Aber du hast hier keinen Platz, wo du dich umzuziehen kannst“, wandte Hermine ein. Dann dachte sie nochmal drüber nach. „Wobei ich denke, dass ich so lange herausgehen und die Tür schließen könnte —“

„Ich habe in dem Koffer auch Platz zum Umziehen.“

Hermine schaute seinen Koffer an, und begann zu vermuten, dass er irgendwie eine ganze Ecke spezieller als ihrer war.

„Also gut“, sagte Hermine, „wenn du das sagst“, und sie kippte zögerlich ein bisschen grüne Limonade auf eine Ecke des Umhangs des Jungen. Dann starrte sie drauf und versuchte sich zu erinnern, wie lange es bei ihr gedauert hatte, bis die Limonade verschwunden war …

Und die Limonade verschwand!

Hermine atmete auf, unter anderem, weil dies bedeutete, dass sie es nicht mit der gesamten Macht des Dunklen Lords zu tun hatte.

Nun, Schritt drei war, das Resultat zu messen, aber das bestand nur darin, zu sehen, dass die Limonade verschwand. Und sie dachte sich, dass sie Schritt 4 mit dem Plakat auch überspringen konnte. „Meine Antwort ist, dass die Umhänge verzaubert sind, um sauber zu bleiben.“

„Nicht ganz“, sagte der Junge.

Hermine fühlte einen Stich der Enttäuschung. Sie wünschte, dass sie sich nicht so fühlen würde, der Junge war zwar kein Lehrer, aber es war dennoch ein Test und sie hatte die falsche Antwort gegeben und das fühlte sich immer wie ein leichter Schlag in den Magen an.

(Eigentlich sagte es alles über Hermine, was man wissen musste; dass sie sich nämlich niemals von so etwas aufhalten ließ, geschweige denn ihre Lust, geprüft zu werden, verlor.)

„Das Traurige ist“, sagte der Junge, „dass du vermutlich alles getan hast, wie es im Buch steht. Du hast eine Vorhersage aufgestellt, die zwischen einem verzauberten und einem unverzauberten Umhang unterscheiden würde, du hast sie getestet, und die Nullhypothese aufgegeben, dass der Umhang nicht verzaubert ist. Aber solange du nicht die aller-allerbesten Bücher liest, wirst du nicht \emph{wirklich} lernen, wie man Wissenschaft betreibt. Ich meine, gut genug um \emph{wirklich} auf die Antwort zu kommen, und nicht einfach nur ein Paper nach dem anderen herauszubringen, wie die über die sich Papa immer beschwert. Also lass mich—ohne die Antwort zu verraten—erklären, was du falsch gemacht hast, und ich gebe dir eine neue Chance.“

Sie hatte begonnen, dem Jungen seinen ach-so-überlegenen Tonfall übelzunehmen, wo er doch genau wie sie erst elf Jahre alt war, doch gegenüber ihrem Drang, herauszufinden, was sie falsch gemacht hatte, war das nachrangig. „Okay, meinetwegen.“

Der Gesichtsausdruck des Jungen wurde intensiver. „Dies ist ein Spiel, das auf einem berühmten Experiment, der sogenannten 2–4–6-Aufgabe, basiert, und es geht so: Ich habe eine \emph{Regel}—die mir, aber nicht dir bekannt ist—welche auf einige Zahlentripel zutrifft, aber auf andere nicht. 2–4–6 ist ein Beispiel für ein Tripel, das der Regel entspricht. Lass mich schnell die Regel aufschreiben, damit du weißt, dass sie festgelegt ist, den Zettel zusammenfalten und dir geben. Schau bitte nicht zu, ich hab eben festgestellt, dass du auf dem Kopf lesen kannst.“

Der Junge sagte „Papier“ und „Druckbleistift“ zu seinem Beutel, und sie schloss ihre Augen fest, als er schrieb.

„So“, sagte der Junge, und er hielt ein mehrfach gefaltetes Blatt Papier in der Hand. „Steck das bitte in deine Tasche“, und sie tat genau das.

„Das Spiel funktioniert so“, sagte der Junge, „dass du mir ein Zahlentripel sagst, und ich sage dir 'Ja', wenn die drei Zahlen ein Beispiel für die Regel sind, und 'Nein', wenn nicht. Du weißt bereits, dass 2–4–6 mit 'ja' beantwortet wird. Wenn du alle Tests durchgeführt hast, die du machen willst—also nach so vielen Tripeln, wie du für nötig hältst, gefragt hast, hörst du auf und rätst die Regel, und dann kannst du das Papier auffalten und sehen, ob du Recht hattest. Verstehst du das Spiel?“

„Natürlich tue ich das“, sagte Hermine.

„Dann leg los.“

„4–6–8“, sagte Hermine.

„Ja“, sagte der Junge.

„10–12–14“, sagte Hermine.

„Ja“, sagte der Junge.

Hermine versuchte, mit ihren Gedanken etwas weiter auszuholen, da es ihr schien, dass sie bereits alle nötigen Tests durchgeführt hatte. Aber es konnte ja kaum so einfach sein, oder?

„1–3–5.“

„Ja“

„Minus 3, minus 1, plus 1.“

„Ja.“

Hermine konnte sich nichts anderes mehr vorstellen. „Die Regel ist, dass die Zahlen jedes mal um zwei hochgehen müssen.“

„Lass mich erwähnen“, sagte der Junge, „dass dieser Test schwerer ist, als er aussieht, und dass nur 20\% aller Erwachsenen ihn bestehen.“

Hermine runzelte die Stirn. Was hatte sie übersehen? Dann fiel ihr schlagartig ein Test ein, den sie noch tun musste.

„2–5–8!“, sagte sie triumphierend.

„Ja.“

„10–20–30!“

„Ja.“

„Die richtige Antwort ist, dass die Zahlen jedes Mal um den \emph{selben Betrag} steigen müssen, es muss nicht immer 2 sein.“

„Sehr gut“, sagte der Junge. „Nimm den Zettel und schau, wie du dich geschlagen hast.“

Hermine nahm den Zettel aus ihrer Hosentasche und entfaltete ihn.

\emph{Drei reelle Zahlen in steigender Ordnung, mit der niedrigsten zuerst.}

Hermines Kinn klappte herunter. Sie hatte das dumpfe Gefühl, dass ihr etwas schrecklich Unfaires angetan wurde, dass der Junge ein dreckiger Betrüger und Lügner war, aber wenn sie darüber nachdachte, konnte sie keine falschen Antworten finden, die er gegeben hatte.

„Was du gerade entdeckt hast, nennt sich in der Literatur \emph{Positive Bias}“, sagte der Junge. „Du hattest eine Regel im Kopf und du hast nur an Zahlentripel gedacht, die die Regel dazu gebracht haben, 'Ja` zu sagen. Aber du hast nicht probiert, so viele Tripel wie möglich zu finden, die die Regel 'Nein' sagen zu lassen. In der Tat hast du \emph{kein einziges} 'Nein` bekommen, deswegen hätte die Regel genauso gut 'drei beliebige Zahlen` lauten können. Es ist so ähnlich wie die Leute, die sich Experimente ausdenken, die ihre Hypothesen bestätigen könnten, anstatt solche zu finden, welche sie widerlegen könnten—nicht ganz derselbe Fehler, aber fast. Du musst lernen, die negative Seite zu sehen, in die Finsternis zu schauen. Wenn man dieses Experiment durchführt, bekommen es nur 20\% der Erwachsenen hin. Und viele der anderen denken sich faszinierend komplizierte Hypothesen aus und haben großes Vertrauen in ihre falschen Antworten, da sie so viele Experimente durchgeführt haben und alles passiert ist, wie sie es erwartet haben.“

„Also“, sagte der Junge, „möchtest du dich ein weiteres Mal an dem ursprünglichen Problem probieren?“

Seine Augen sahen sehr fokussiert aus, als wäre dies der \emph{echte} Test.

Hermine schloss ihre Augen und probierte sich zu konzentrieren. Sie schwitzte unter ihrem Umhang. Sie hatte das komische Gefühl, dass sie nie zuvor bei einem Test so intensiv nachgedacht hatte, oder vielleicht sogar zum \emph{ersten} Mal bei einem Test nachdenken musste.

Was für ein anderes Experiment konnte sie durchführen? Sie hatte einen Schokoladenfrosch, könnte sie davon etwas auf ihren Umhang reiben und schauen, ob \emph{das} verschwand? Aber das schien ihr noch nicht das verdrehte negative Denken zu sein, von dem der Junge geredet hatte. Als würde sie immer noch nach einem 'Ja` fragen, wenn der Schokoladenfleck verschwand, als nach einem 'Nein` zu fragen.

Also…ihrer Hypothese nach…wann sollte die Limonade…\emph{nicht} verschwinden?

„Ich habe mir ein Experiment überlegt“, sagte Hermine. „Ich möchte Limonade auf den Boden schütten, und schauen, ob sie \emph{nicht} verschwindet. Hast du Taschentücher in deiner Tasche, sodass ich die Limonade aufwischen kann, falls es nicht klappt?“

„Ich habe Servietten“, sagte der Junge. Sein Gesicht sah immer noch neutral aus.

Hermine nahm die Limonadendose, und schüttete etwas Limonade auf den Boden.

Ein paar Sekunden später verschwand sie.

„Heureka“, sagte Hermine leise. Es war wie ein Zwang, sie \emph{musste} es einfach sagen. In der Tat fühlte sie sich danach, es zu schreien, aber dafür war sie doch etwas zu gehemmt. Dann wurde ihr klar, was passiert war, und sie hätte sich selbst ohrfeigen können. „Natürlich! \emph{Du} hast mir die Limonade gegeben! Es war nicht der Umhang, der verzaubert war, es war von vorneherein die Limonade!“

Der Junge stand auf und verbeugte sich feierlich vor ihr. Er grinste nun sehr weit. „Also… darf ich dir bei deinen Studien behilflich sein, Hermine Granger?“

„Ich, ähm…“ Hermine spürte immer noch die Euphorie in sich wallen, aber sie wusste nicht, was sie \emph{darauf} antworten sollte.

Sie wurden von einem schwachen, zögernden, eher \emph{widerstrebenden} Klopfen an der Tür unterbrochen.

Der Junge drehte sich um und schaute aus dem Fenster, und sagte, „Ich trage meinen Schal nicht, kannst du bitte öffnen?“

In diesem Moment verstand Hermine, warum der Junge—nein, der Junge, der lebt, Harry Potter—die ganze Zeit den Schal um sein Gesicht gewickelt gehabt hatte, und fühlte sich etwas dumm, weil sie es nicht früher verstanden hatte. Es war etwas komisch, da sie gedacht hätte, Harry Potter müsse die Art von Junge sein, der sich stolz der Welt stellte; und ihr kam der Gedanke, dass er schüchterner sein könnte als er wirkte.

Als Hermine die Tür aufzog, stand ihr ein zitternder Junge gegenüber, der genauso aussah, wie er geklopft hatte.

„Entschuldigt bitte“, sagte er in einer winzigen Stimme. „Ich bin Neville Longbottom. Ich suche nach meiner Kröte, ich, ich kann sie nirgendwo in diesem Wagen finden…Habt ihr meine Kröte gesehen?“

„Nein“, sagte Hermine und dann setzte ihre Hilfsbereitschaft mit voller Kraft ein. „Hast du alle anderen Kabinen durchsucht?“

„Ja“, flüsterte der Junge.

„Dann werden wir wohl die anderen Wagen durchsuchen müssen“, sagte Hermine schnell. „Ich werde dir helfen. Mein Name ist übrigens Hermine Granger.“

Der Junge schaute sie dankbar an.

„Wartet mal“, kam die Stimme vom \emph{anderen} Jungen—Harry Potter. „Ich glaube nicht, dass das die beste Art ist, es zu tun.“

Darauf sah Neville aus, als würde er gleich weinen, und Hermine drehte sich wutentbrannt um. Wenn Harry Potter die Art von Junge war, die einen kleinen Jungen ignorieren würde, nur weil er nicht unterbrochen werden wollte… „Was? Warum \emph{nicht?}“

„Naja“, sagte Harry Potter, „es wird eine Weile dauern, den ganzen Zug per Hand zu durchsuchen, und wir könnten die Kröte trotzdem übersehen, und wenn wir sie nicht finden, bevor wir Hogwarts erreichen, könnte er Probleme bekommen. Was also mehr Sinn ergeben würde, wäre, wenn er direkt zum vorderen Wagen ginge, wo die Vertrauensschüler sind, und einen von ihnen um Hilfe bäte. Das war das erste, was ich getan habe, als ich nach dir gesucht habe, Hermine, auch wenn sie es dann nicht wussten. Aber sie könnten Zaubersprüche oder magische Gegenstände haben, die es wesentlich einfacher machen, eine Kröte zu finden. Wir sind ja nur Erstklässler.“

Das…\emph{ergab} eine Menge Sinn.

„Glaubst du, dass du alleine zum Vertrauensschülerwagen finden wirst?“, fragte Harry Potter. „Ich habe ein paar Gründe, mein Gesicht nicht allzu viel herumzuzeigen.“

Plötzlich japste Neville und ging einen Schritt zurück. „Ich erinnere mich an diese Stimme! Du bist einer der Herren des Chaos! \emph{Du warst der, der mir Süßigkeiten gegeben hat!}“

Was? Was, was, \emph{was?}

Harry Potter wandte sein Gesicht vom Fenster ab und stand dramatisch auf. „\emph{Niemals!}“, sagte er, seine Stimme voll der Entrüstung. „Sehe ich wie die Art von Bösewicht aus, der Kindern Süßigkeiten gibt?“

Nevilles Augen weiteten sich. „\emph{Du} bist Harry Potter? \emph{Der} Harry Potter? \emph{Du?}“

„Nein, nur \emph{ein} Harry Potter, es gibt drei von mir in diesem Zug —“

Neville gab einen kleinen Schrei von sich und rannte weg. Es gab ein kurzes wildes Fußgetrappel und dann das Geräusch einer sich öffnenden und wieder schließenden Wagentür.

Hermine setzte sich auf ihre Bank. Harry Potter schloss die Tür und setzte sich dann neben sie.

„Kannst du mir bitte erklären, was hier vor sich geht?“, sagte Hermine mit schwacher Stimme. Sie fragte sich, ob es immer so verwirrend wäre, Zeit mit Harry Potter zu verbringen.

„Oh, naja, was passiert ist, ist, dass Fred und George und ich diesen armen kleinen Jungen auf dem Bahnsteig sitzen gesehen haben—die Frau neben ihm war für eine Zeit weggegangen, und er sah wirklich ängstlich aus, als ob er sicher wäre, dass er gleich von Todessern angegriffen würde oder so. Nun, es gibt ein Sprichwort, dass die Angst vor etwas meist schlimmer ist als die Sache selbst. Und mir schien es, dass hier ein Junge saß, der tatsächlich davon profitieren könnte, dass seine schlimmsten Alpträume wahr würden und sie nicht so schlimm wären, wie er fürchtete —“

Hermine saß mit weit geöffnetem Mund da.

„— und Fred und George fiel dieser Zauberspruch ein, mit dem wir die Schals vor unseren Gesichtern verdunkeln und verschwimmen lassen konnten, als wäre wir untote Könige und dies wären unsere Grabtücher —“

Sie mochte die Richtung, in die diese Geschichte ging, überhaupt nicht.

„— und nachdem wir ihm all die Süßigkeiten gegeben hatten, die ich gekauft hatte, sagten wir, 'Lasst uns ihm Geld geben! Hahaha! Nimm ein paar Knuts, Junge! Nimm einen silbernen Sickel!`, und tanzten um ihn herum, böse lachend und so weiter. Ich glaube, es gab ein paar Leute im Publikum, die eingreifen wollten, aber der Zuschauereffekt hielt sie ab, zumindest lange genug, bis sie sahen, was wir taten, und dann waren sie zu verwirrt, um irgendwas zu tun. Schlussendlich sagte er 'geht weg` in einem winzigen Wispern, also schrien wir drei und rannten weg, irgendwas über das 'brennende Licht` kreischend. Hoffentlich wird er in Zukunft nicht so eine Angst davor haben gemobbt zu werden. Man nennt das übrigens Desensibilisierung.“

Okay, sie hatte \emph{nicht} richtig geraten, wohin die Geschichte ging.

Das brennende Feuer der Empörung, das einer von Hermines wichtigsten Antrieben war, sprang an, auch wenn ein Teil von ihr schon irgendwie \emph{sehen} konnte, was sie versucht hatten. „Das ist schrecklich! \emph{Du} bist schrecklich! Was du getan hast, war \emph{gemein!}“

„Ich glaube, dass das Wort, nach dem du suchst, 'unterhaltsam` ist, und in jedem Fall stellst du die falsche Frage. Die Frage ist, hat es mehr Gutes getan, als es Schaden angerichtet hat, oder andersherum? Wenn du Argumente hast, die sich um \emph{die} Frage drehen, wird es mich freuen, sie zu hören, aber ich werde keine andere Kritik akzeptieren, solange diese Frage nicht geklärt ist. Ich sehe durchaus ein, dass es schrecklich und gemein \emph{aussieht}, und als würden wir ihn mobben, da es sich um einen kleinen ängstlichen Jungen handelt, und so weiter, aber das ist kaum die Schlüsselfrage, oder? Man nennt das übrigens Konsequentialismus, es bedeutet, dass, ob eine Tat gut oder schlecht ist, nicht davon bestimmt wird, ob sie schlecht \emph{aussieht}, oder gemein oder so; die einzige Frage ist, wie es am Ende ausgeht—was die Konsequenzen sind.“

Hermine öffnete ihren Mund, um etwas absolut \emph{Vernichtendes} zu sagen, hatte aber unglücklicherweise den Moment verpasst, in dem sie sich überlegte, was sie sagen wollte, bevor sie den Mund öffnete. Alles was sie hervorbringen konnte, war, „Was, wenn er \emph{Alpträume} davon bekommt?“

„Ehrlich gesagt glaube ich nicht, dass er unsere Hilfe brauchte, um Alpträume zu haben, und wenn er jetzt \emph{hiervon} Alpträume hat, werden es Alpträume von schrecklichen Monstern sein, die ihm Schokolade geben und \emph{das} war im Prinzip der \emph{Sinn} der Sache.“

Hermines Gehirn verhaspelte sich jedes Mal, wenn sie versuchte, richtig wütend zu werden. „Ist dein Leben immer so sonderbar?“, sagte sie schließlich.

Harrys Gesicht strahlte vor Stolz. „Ich \emph{mache} es sonderbar. Du siehst das Produkt langer und harter Arbeit.“

„Also …“, sagte Hermine, und verfiel in unangenehmes Schweigen.

„Also“, sagte Harry Potter, „Wie viel Wissenschaft kennst du eigentlich? Ich kann Analysis und ich kann ein wenig Bayes'sche Wahrscheinlichkeitslehre und Entscheidungstheorie und eine Menge Cognitive Science, und ich habe Feynmans \emph{Vorlesungen über Physik (Bd. 1)} gelesen, und \emph{Judgment under Uncertainty: Heuristics and Biases} und \emph{Semantik: Sprache im Handeln und Denken} und \emph{Die Psychologie der Beeinflussung} und \emph{Rational Choice in an Uncertain World} und \emph{Gödel, Escher, Bach} und \emph{A Step Farther Out} —“

Die folgende Befragung und Gegenbefragung zog sich einige Minuten hin, bis sie von einem weiteren schüchternen Klopfen an der Tür unterbrochen wurden. „Komm rein“, sagten sie fast gleichzeitig, und die Tür glitt beiseite, um Neville Longbottom einzulassen.

Diesmal weinte Neville wirklich. „Ich bin zum vorderen Waggon gegangen und hab einen V-Vertrauensschüler gefunden, aber er hat mir g-gesagt, dass man Vertrauensschüler nicht mit so einem Kleinkram wie fehlenden K-Kröten belästigt.“

Das Gesicht des Jungen, der lebte veränderte sich. Seine Lippen wurden zu einer feinen Linie. Als er sprach, war seine Stimme kalt und grimmig. „Was waren seine Farben? Grün und Silber?“

„N-Nein, sein Abzeichen war r-rot und gold.“

„\emph{Rot und Gold!}“, brach es aus Hermine heraus. „Aber das sind \emph{Gryffindor}-Farben!“

Harry Potter \emph{zischte} daraufhin, ein beängstigendes Geräusch, das von einer echten Schlange hätte kommen können und sowohl sie als auch Neville zusammenzucken ließ.

„Ich \emph{vermute}“, spuckte Harry Potter aus, „dass die Kröte eines Erstklässlers zu finden nicht \emph{heroisch} genug für einen \emph{Gryffindor}-Vertrauensschüler ist. Komm mit, Neville, \emph{ich} komme dieses Mal mit dir, wir wollen mal sehen, ob der Junge, der lebt, mehr Aufmerksamkeit bekommt. Zuerst finden wir einen Vertrauensschüler, der einen geeigneten Zauberspruch kennen sollte, und wenn das nicht klappt, finden wir einen Vertrauensschüler, der keine Angst hat, sich die Hände schmutzig zu machen, und wenn \emph{das} nicht klappt, rekrutieren wir meine Fans, und nehmen, wenn es sein muss, den ganzen Zug Schraube für Schraube auseinander.“

Der Junge, der lebt, stand auf und griff mit seiner Hand nach Nevilles, und Hermine wurde urplötzlich klar, dass sie nahezu gleich groß waren, auch wenn ein Teil von ihr darauf bestanden hatte, dass Harry Potter mindestens dreißig Zentimeter größer, und Neville mindestens fünfzehn Zentimeter kleiner wäre.

„\emph{Bleib!}“, blaffte Harry sie an—nein, warte, er blaffte seinen \emph{Koffer} an—und er schloss die Tür fest hinter sich als er ging.

Sie hätte vermutlich mitkommen sollen, aber für einen kurzen Moment war Harry Potter so angsteinflößend geworden, dass sie eigentlich ziemlich froh war, dass sie es nicht vorgeschlagen hatte.

Hermines Gedanken waren nun so durcheinander, dass sie nicht einmal mehr dachte, dass sie „\emph{Die Geschichte Hogwarts'}“ vernünftig zu Ende lesen könnte. Sie fühlte sich, als wäre sie gerade von einer Dampfwalze überfahren worden. Sie war sich nicht sicher, was sie fühlte oder warum. Sie saß einfach am Fenster und beobachtete das vorbeiziehende Landschaftsbild.

Immerhin wusste sie, warum sie sich innerlich ein wenig traurig fühlte.

Möglicherweise war Gryffindor nicht so wunderbar, wie sie es gedacht hatte.

—\/-\/-\/-\/-\/-\/-\/-\/-\/-\/-\/-\/-\/-\/-\/-\/-\/-\/-\/-\/-\/-\/-\/-\/-\/-\/-\/-\/-\/-

\textbf{Quarks:}

Elementare Bestandteile der Materie, aus denen u.a. die Protonen und Neutronen im Atomkern zusammengesetzt sind. Es gibt sechs Arten von Quarks: `Up', `Down', `Strange', `Charm', `Top' und `Bottom'. (Die letzten beiden wurden früher—denkt dran, Harrys erstes Schuljahr beginnt im Sommer 1991, also gut 2,5 Jahre bevor das Top-Quark experimentell nachgewiesen wurde—oft auch als `Truth' und `Beauty' bezeichnet. Hermines Aufzählung ist also fehlerfrei.)

