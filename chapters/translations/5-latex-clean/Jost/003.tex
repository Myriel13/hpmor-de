

\hypertarget{die-realituxe4t-verglichen-mit-ihren-alternativen}{% \section{3. Die Realität, verglichen mit ihren Alternativen}\label{die-realituxe4t-verglichen-mit-ihren-alternativen}}

Inzwischen haben sich zwei Menschen gemeldet, die diese FF seit einiger Zeit auf Englisch lesen und beim Übersetzen helfen wollen: Vielen Dank an Alex und Martin!

Dieses Kapitel wurde von mir übersetzt, die beiden haben bei der Überarbeitung sehr geholfen.

(Falls ihr also im Review die Übersetzung loben wollt, benutzt doch bitte den Plural! ;) )

-\/-\/-\/-\/-\/-\/-\/-\/-\/-\/-\/-\/-\/-\/- ~ 3. Die Realität, verglichen mit ihren Alternativen ~ -\/-\/-\/-\/-\/-\/-\/-\/-\/-\/-\/-\/-\/-\/-

\emph{"Ich habe keine Zeit für so etwas."}

-\/-\/-

"Meine Güte", sagte der Barkeeper mit einem Blick auf Harry, "ist das --- kann das --- ?"

Harry lehnte sich so gut er konnte an die Theke des Tropfenden Kessels, die ihm allerdings bis an die Augenbrauen reichte. Eine Frage wie \emph{diese} verdiente seine beste Darbietung.

"Bin ich --- könnte ich --- vielleicht --- man weiß ja nie --- \emph{ist} es denn so --- aber es stellt sich die Frage --- \emph{warum}?"

"Mein Gott", flüsterte der alte Barkeeper, "Harry Potter… welch eine Ehre."

Harry blinzelte, fing sich jedoch schnell. "Nun, in der Tat, Sie sind sehr aufmerksam; die meisten Leute begreifen das nicht so schnell -"

"Das reicht", sagte McGonagall. Ihre Hand lag fest auf Harrys Schulter. "Lassen Sie den Jungen in Ruhe, Tom, er ist das alles nicht gewohnt."

"Aber ist er es?", trillerte eine alte Frau. "Ist es Harry Potter?" Mit einem kratzenden Geräusch erhob sie sich aus ihrem Stuhl.

"Doris…", sagte McGonagall warnend. Ihr funkelnder Blick hätte jeden einschüchtern können.

"Ich will nur seine Hand schütteln", flüsterte die Frau. Sie beugte sich zu Harry herab und streckte eine runzlige Hand aus, die Harry, der sich so verwirrt und unbehaglich fühlte wie nie zuvor in seinem Leben, zaghaft schüttelte. Aus ihren Augen fielen Tränen auf ihre fest umschlungenen Hände. "Mein Enkel war ein Auror", flüsterte sie ihm zu. "Ist neunundsiebzig gestorben. Vielen Dank, Harry Potter. Dem Himmel sei Dank."

"Keine Ursache", sagte Harry vollkommen automatisch und wandte sich dann McGonagall zu, der er einen verängstigten, hilfesuchenden Blick zuwarf.

McGonagall stampfte mit ihrem Fuß auf den Boden, gerade als die Leute Anstalten machten, auf Harry einzustürmen. Es gab ein Geräusch, bedrohlicher als Donnergrollen, und alle erstarrten in der Bewegung.

"Wir haben es eilig", sagte McGonagall in einer bemerkenswert normal klingenden Stimme.

Sie verließen den Pub ohne weitere Schwierigkeiten.

"Professor McGonagall?", sagte Harry, sobald sie den Hinterhof betraten. Er hatte fragen wollen, was genau drinnen passiert war, aber seltsamerweise fiel ihm auf, dass er eine ganz andere Frage stellte: "Wer war der blasse Mann? Der Mann im Pub, mit dem zuckenden Auge?"

"Hm", sagte McGonagall. Es klang leicht überrascht; vielleicht hatte auch sie diese Frage nicht erwartet. "Das war Professor Quirrell. Er wird dieses Jahr in Hogwarts Verteidigung gegen die Dunklen Künste unterrichten."

"Ich hatte das seltsame Gefühl, dass ich ihn kenne…" Harry massierte seine Schläfen. "Und dass ich ihm nicht die Hand schütteln sollte." Als ob er jemanden getroffen hätte, der einst ein Freund war, bevor irgendetwas schrecklich falsch gelaufen war… ~nun, nicht ganz so, aber Harry konnte es nicht in Worte fassen. "Und die anderen?"

McGonagall warf ihm einen seltsamen Blick zu. "Mr Potter… wissen Sie… \emph{wie viel} wissen Sie… darüber, wie Ihre Eltern gestorben sind?"

Harry erwiderte ihren Blick fest. "Meine Eltern leben, sind wohlauf und haben sich stets geweigert zu erzählen, wie meine \emph{biologischen} Eltern gestorben sind. Woraus ich folgere, dass es nicht schön war."

"Eine bewundernswerte Loyalität", sagte McGonagall. Ihre Stimme senkte sich. "Allerdings schmerzt es etwas, Sie so sprechen zu hören. Lily und James waren Freunde von mir."

Harry sah beschämt weg. "Es tut mir Leid", sagte er in einem leisen Ton. "Aber ich \emph{habe} eine Mutter und einen Vater. Und ich weiß, dass ich mich selbst nur unglücklich machen würde, wenn ich die Realität mit etwas… etwas Perfektem vergleichen würde, was ich mir ausmale."

"Das ist äußerst weise", sagte McGonagall sanft. "Aber Ihre \emph{biologischen} Eltern starben auf eine denkwürdige Art: Sie starben um Sie zu schützen."

\emph{Um mich zu schützen?}

Harrys Brust schnürte sich zusammen. "Was… ist passiert?"

McGonagall seufzte. Ihr Zauberstab berührte Harrys Stirn und einen Moment lang sah er seine Umgebung unscharf. "Eine Tarnung", sagte McGonagall, "damit so etwas nicht nochmal passiert. Nicht, bevor Sie dafür bereit sind." Dann erhob sie wieder ihren Zauberstab und tippte drei Mal gegen einen Ziegelstein…

… der beiseite glitt und ein kleines Loch hinterließ, welches sich ausdehnte und vergrößerte, schließlich zu einem riesigen Torbogen heranwuchs und den Blick auf eine lange Straße voller Geschäfte freigab, welche Kessel oder Drachenlebern feilboten.

Harry blinzelte nicht einmal. Schließlich verwandelte sich hier niemand in eine Katze.

Und zusammen schritten sie vorwärts, hinein in die Zaubererwelt.

Straßenhändler priesen Sprungstiefel ("Enthält echtes Flubber!") und "Messer +3! Gabeln +2! Löffel mit einem +4-Bonus!" an. Es gab Brillen, die alles, was man ansah, grün färbten, und ein Sortiment gemütlicher Clubsessel mit Schleudersitzen für den Notfall.

Harrys Kopf drehte sich in alle Richtungen, als wolle er sich selbst von den Schultern abschrauben. Es war, als ob er im \emph{Dungeons\&Dragons}-Regelbuch mitten im Kapitel über magische Gegenstände gelandet wäre. (Das Spiel hatte er zwar nie gespielt, die Regelbücher aber geradezu verschlungen.) Harry versuchte verzweifelt, sich nicht einmal die winzigste Kleinigkeit entgehen zu lassen --- schließlich könnte sich jeder noch so unscheinbare Gegenstand einmal als einer der drei Bestandteile eines unbegrenzten Wunschzaubers herausstellen.

Dann bemerkte Harry etwas, das ihn, ohne dass er überhaupt darüber nachdachte, von McGonagalls Seite wegzog. Er stürmte geradewegs auf das blaue Backsteingebäude mit bronzenen Verzierungen zu und nahm die Umgebung erst wieder wahr, als seine Lehrerin sich direkt vor ihm aufbaute.

"Mr Potter?", fragte sie.

Harry blinzelte und bemerkte dann, was er gerade getan hatte. "Entschuldigung! Ich habe einen Moment lang vergessen, dass ich mit Ihnen unterwegs bin und nicht mit meiner Familie." Harry deutete auf das Schaufenster, in dem feurige, strahlend helle Buchstaben die Worte \emph{Bigbams Brillante Bücher} bildeten. "Wenn man an einem Bücherladen vorbeikommt, den man noch nie zuvor betreten hat, dann muss man hineingehen und sich drin umsehen. Familiengesetz."

"Das ist das Ravenclaw-artigste, was ich jemals gehört habe."

"Wie bitte?"

"Nichts. Mr Potter, wir werden zuerst Gringotts, der Zaubererbank, einen Besuch abstatten. Im Verlies Ihrer \emph{biologischen} Eltern liegt all das, was Ihre \emph{biologischen} Eltern Ihnen vererbt haben; Sie werden Geld für die Schulsachen brauchen." Sie seufzte. "Und ich nehme an, es kann nicht schaden, wenn Sie außerdem ein wenig Geld für Bücher mitnehmen. Allerdings sollten Sie sich dafür etwas Zeit lassen. Hogwarts besitzt eine breit gefächerte Bibliothek zu allen magischen Themen. Und das Haus, in welches Sie höchstwahrscheinlich kommen werden, verfügt über eine noch umfangreichere Büchersammlung. Alle Bücher, die Sie jetzt kaufen würden, sind dort vermutlich längst vorhanden."

Harry nickte und sie gingen weiter.

"Verstehen Sie mich nicht falsch, es ist eine \emph{großartige} Ablenkung", sagte Harry, während sein Kopf rotierte, "vermutlich die beste Ablenkung, die ich je gefunden habe, aber glauben Sie bitte nicht, dass ich deswegen unsere ausstehende Unterhaltung vergessen habe."

McGonagall seufzte. "Es war vermutlich äußerst klug von Ihren Eltern --- oder zumindest Ihrer Mutter --- Ihnen nichts davon zu erzählen."

"Also wäre es Ihnen lieb, wenn ich in seligem Unwissen verbliebe? Dieser Plan hat eine gewisse Schwachstelle, Professor McGonagall."

"Es wäre wohl äußerst sinnlos", sagte die Hexe knapp, "wo doch jeder Mensch auf der Straße Ihnen die Geschichte erzählen kann. Nun gut."

Und sie erzählte ihm von Ihm, dessen Name nicht genannt werden darf, dem dunklen Lord, Voldemort.

"Voldemort?", flüsterte Harry. Es hätte lächerlich sein sollen, war es aber nicht. Der Name war wie ein kaltes Gefühl, schonungslos, glasklar; wie ein Hammer aus Titanstahl, der auf einen Amboss aus weichem Menschenfleisch niederschlug. Es lief Harry kalt den Rücken hinunter, als er den Namen nur aussprach, und so beschloss er an Ort und Stelle, auf harmlosere Namen wie Du-weißt-schon-wer zurückzugreifen.

Der dunkle Lord hatte in der britischen Zaubererwelt gewütet wie ein wildgewordenes Raubtier, er hatte den normalen Alltag zerrissen und zerfleischt. Andere Länder hatten händeringend zugesehen, jedoch nie eingegriffen --- ob aus trägem Eigeninteresse oder aus schierer Angst, als nächstes vom dunklen Lord angegriffen zu werden.

(\emph{Der Zuschauereffekt}, dachte Harry und erinnerte sich an das Experiment von Latane und Darley, die gezeigt hatten, dass man bei einem epileptischen Anfall eher Hilfe bekam, wenn nur eine Person in der Nähe war, als wenn es drei waren. \emph{Die Verantwortung verteilt sich auf viele Schultern, jeder hofft, dass irgendjemand anderes zuerst eingreift.})

Die Todesser folgten dem Dunklen Lord wie Aasgeier, pickten an Wunden und lähmten wie Gift. Sie waren nicht so schrecklich wie der Dunkle Lord, aber sie waren schrecklich und sie waren viele. Doch die Todesser hatten mehr als nur Zauberstäbe: In ihren maskierten Reihen verbargen sich politische Macht und schreckliche Geheimnisse, mit denen sich vieles erpressen ließ. Die Gesellschaft war wie paralysiert, strebte nur noch nach Selbsterhaltung.

Ein alter und hochgeachteter Journalist, Yermy Wibble, forderte höhere Steuern zur Finanzierung einer Armee. Er schrieb, es wäre absurd, wie so viele aus Angst vor so wenigen niederkauerten. Seine Haut --- nur seine Haut --- fand man am nächsten Morgen an der Wand seines Büros festgenagelt, neben denen seiner Frau und seiner zwei Töchter. Alle wollten etwas dagegen tun, doch niemand wagte sich, aufzustehen und es vorzuschlagen. Wer auffiel, an dem wurde ein Exempel statuiert.

Bis die Namen von James und Lily Potter ganz oben auf der Liste standen.

Auch diese zwei wären aufrecht und mit gezückten Zauberstäben gestorben und hätten es aus Heldenmut nicht bereut; doch sie hatten ein kleines Kind, ihren Sohn Harry.

In Harrys Augen traten Tränen. Er wischte sie verärgert und vielleicht auch verzweifelt weg. \emph{Ich kenne diese Menschen nicht, nicht wirklich, sie sind} jetzt \emph{nicht mehr meine Eltern, es wäre so sinnlos, deswegen traurig zu sein…}

Nachdem Harry McGonagalls Umhang vollgeschluchzt hatte, sah er auf und fühlte sich ein wenig besser, als er auch in ihren Augen Tränen sah.

"Was passierte dann?", fragte Harry mit zitternder Stimme.

"Der Dunkle Lord kam nach Godric's Hollow", flüsterte McGonagall. "Ihr wart versteckt, doch jemand hatte euch verraten. Der Dunkle Lord tötete James und er tötete Lily und er kam schließlich an dein Bett. Er sprach den Todesfluch. Und dann war es zu Ende. Der Todesfluch besteht aus purem Hass und zielt direkt auf die Seele, trennt sie vom Körper. Man kann ihn nicht abwehren. Die einzige Verteidigung ist, woanders zu sein. Aber du hast überlebt. Du bist der einzige Mensch, der je überlebt hat. Der Todesfluch wurde reflektiert und traf den Dunklen Lord, nur seine angesengte Hülle und eine Narbe auf deiner Stirn blieben zurück. Dies war das Ende des Schreckens und wir waren befreit. Deswegen, Harry Potter, möchten die Leute die Narbe auf deiner Stirn sehen und deine Hand schütteln."

Harrys Trauer hatte längst all seine Tränen aufgebraucht; er konnte nicht mehr weinen, er war nur noch erschöpft.

(Und irgendwo in den Tiefen seines Kopfes befand sich ein winziger, winziger Vermerk einer Verwirrung --- des Gefühls, dass mit dieser Geschichte irgendwas nicht stimmen konnte --- und Harry hätte eigentlich stutzig werden müssen, doch er war zu sehr abgelenkt. Es ist traurig, dass man die Kunst des rationalen Denkens gerade dann am leichtesten vergisst, wenn man sie am dringendsten bräuchte.)

Harry nahm schließlich etwas Abstand von McGonagall. "Ich… muss erst einmal darüber nachdenken", sagte er und versuchte, seine Stimme wieder unter Kontrolle zu bringen. Er starrte seine Schuhe an. "Ähm… Sie können ruhig von meinen Eltern reden, wenn Sie wollen. Sie müssen nicht 'biologische Eltern' oder so was sagen. Ich wüsste nicht, warum ich nicht zwei Mütter und zwei Väter haben kann."

McGonagall blieb stumm.

So liefen sie schweigend weiter, bis sie vor einem großen, weißen Gebäude mit weiten Bronzetüren standen.

"Gringotts", sagte McGonagall.

-\/-\/-\/-\/-\/-\/-\/-\/-\/-\/-\/-\/-\/-\/-\/-\/-\/-\/-\/-\/-\/-\/-\/-\/-\/-\/-\/-\/-\/-

Zuschauereffekt:

(Die Erklärung befindet sich ja bereits im Text.)

Auch als Genovese-Syndrom bekannt, nach \href{http://de.wikipedia.org/wiki/Kitty_Genovese}{Catherine Genovese}, die 1964 in New York ermordet wurde, ohne dass einer von 38 bekannten Teil-Augenzeugen einschritt.

\href{http://web.archive.org/web/20070706041006/http://www2.selu.edu/Academics/Faculty/scraig/gansberg.html}{Ein Artikel der New York Times} über diesen Vorfall sorgte daraufhin für viel Aufmerksamkeit in Gesellschaft und Psychologie.

-\/-\/-

Das war also Kapitel 3.

Demnächst: Harry analysiert das Finanzsystem der Zaubererwelt…

