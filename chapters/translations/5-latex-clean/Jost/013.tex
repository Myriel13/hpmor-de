

\hypertarget{die-falschen-fragen-stellen}{% \section{13. Die falschen Fragen stellen}\label{die-falschen-fragen-stellen}}

Eliezer Yudkowsky schreibt als Vorbemerkung zu diesem Kapitel:

„Keine Panik. Ich schwöre feierlich, dass es eine logische, erahnbare und mit dem Canon vereinbare Erklärung für alles gibt, was in diesem Kapitel geschieht. Es ist ein Rätsel, dessen Lösung ihr herausfinden sollt—und falls es nicht gelingt, lest einfach das nächste Kapitel.“

—\/-\/-\/-\/-\/-\/-\/-\/-\/-\/-\/-\/-\/-\/- ~ 13. Die falschen Fragen stellen ~ -\/-\/-\/-\/-\/-\/-\/-\/-\/-\/-\/-\/-\/-\/-

\later

„\emph{Das ist eines der offensichtlichsten Rätsel, von denen ich je gehört habe.}“

\later

Sobald Harry an seinem ersten Morgen auf Hogwarts im Schlafsaal der Ravenclaw-Erstklässler die Augen aufschlug, wusste er, dass etwas nicht stimmte.

Es war still.

\emph{Zu} still.

Ach, natürlich…im Kopfende seines Bettes war ein Quietus-Zauber eingebaut, der mit einem kleinen Schieber eingestellt werden konnte. Nur dadurch konnte man im Ravenclaw-Turm überhaupt einschlafen.

Harry setzte sich auf und blickte sich nach den anderen um, die auch gerade aufstehen müssten—

Der Schlafsaal war leer.

Die Betten waren zerwühlt und ungemacht.

Die Sonne schien in einem recht steilen Winkel herein.

Sein Quietus-Zauber war auf vollste Stärke eingestellt.

Und sein Wecker tickte, doch der Alarm war ausgestellt.

Er hatte offenbar bis 9:52~Uhr geschlafen. Obwohl er sich alle Mühe gegeben hatte, seinen 26-stündigen Schlafzyklus an seine Ankunft auf Hogwarts anzupassen, war er letzte Nacht erst gegen ein Uhr morgens eingeschlafen. Eigentlich wollte er wie seine Mitschüler um 7~Uhr aufstehen—etwas Schlafmangel am ersten Tag hielt er aus, solange er im Laufe des Tages irgendein magisches Hilfsmittel bekam. Doch nun hatte er das Frühstück verpasst. Und seine erste Unterrichtsstunde, Kräuterkunde, hatte vor einer Stunde und zweiundzwanzig Minuten angefangen.

Langsam, ganz langsam erwachte der Ärger in ihm. Oh, was für ein schöner kleiner Streich. Schalte seinen Wecker aus. Drehe den Quietus hoch. Soll der berühmte Mr~Harry Potter ruhig seine erste Unterrichtsstunde verpassen und Ärger bekommen, weil er verschlafen hat.

Wenn Harry rausfand, wer das getan hatte…

Nein, das konnte nur mit der Hilfe aller zwölf Jungs im Schlafsaal geschehen sein. Jeder von ihnen musste ihn schlafen gesehen haben. Jeder von ihnen hatte ihn verschlafen lassen.

Der Ärger verschwand und ließ Verwirrung und einen schrecklichen Schmerz zurück. Sie hatten ihn \emph{gemocht}. Hatte er gedacht. Letzte Nacht hatte er gedacht, dass sie ihn mochten. \emph{Warum …}

Als Harry aus dem Bett aufstand, sah er ein Blatt Papier am Kopfende seines Bettes hängen. Darauf stand:

\emph{Meine Mit-Ravenclaws,}

gestern war ein langer Tag. Bitte lasst mich noch schlafen und macht euch keine Sorgen, wenn ich das Frühstück verpasse. Ich habe die erste Unterrichtsstunde nicht vergessen.

Euer Harry Potter

Harry stand erstarrt da, Eiswasser tröpfelte durch seine Venen.

Der Zettel war in seiner eigenen Handschrift, mit seinem eigenen Drehbleistift geschrieben.

Und er konnte sich nicht daran erinnern, ihn geschrieben zu haben.

Und… Harry fixierte das Stück Papier. Und wenn er sich nicht täuschte, dann waren die Worte „nicht vergessen“ in einer anderen Art geschrieben, als ob er sich damit etwas sagen wollte …

Hatte er \emph{gewusst}, dass er einen Vergessenszauber abbekommen würde? War er lange aufgeblieben, hatte er irgendein Verbrechen oder eine geheime Tätigkeit vollbracht und dann…aber er kannte den Zauberspruch nicht…hat jemand anders…was …

Harry fiel etwas ein. Wenn er \emph{tatsächlich} gewusst hatte, dass er einen Vergessenszauber abbekommen würde …

Immer noch im Schlafanzug lief Harry um sein Bett herum zu seinem Koffer, drückte seinen Daumen gegen das Schloss, holte den Beutel raus, steckte die Hand rein und sagte „Notiz für mich selbst“.

Und ein weiteres Stück Papier erschien in seiner Hand.

Harry nahm es raus und starrte darauf. Auch dieser Zettel war in seiner eigenen Handschrift geschrieben. Darauf stand:

\emph{Liebes Ich,}

bitte spiele das Spiel. Du kannst das Spiel nur einmal in deinem Leben spielen. Dies ist eine einmalige Gelegenheit.

Erkennungszeichen 927, ich bin eine Kartoffel.

Dein Du.

Harry nickte langsam. „Erkennungszeichen 927, ich bin eine Kartoffel“ war tatsächlich die Nachricht, die er sich früher—vor einigen Jahren, während er Fernsehen schaute—ausgedacht hatte und die nur ihm bekannt war. Falls er überprüfen musste, ob ein Duplikat von ihm wirklich \emph{er} war, oder so etwas. Nur für den Fall. Sei bereit.

Harry konnte der Nachricht nicht \emph{vertrauen}, womöglich waren weitere Zaubersprüche beteiligt. Aber es schloss zumindest einen simplen Streich aus. Er hatte den Zettel definitiv geschrieben und er konnte sich definitiv nicht daran erinnern.

Während Harry den Zettel anstarrte, fiel ihm plötzlich auf, dass Tinte von der anderen Seite durchschimmerte.

Er drehte ihn um.

Auf der Rückseite stand:

\emph{SPIELANLEITUNG}

Du kennst die Regeln des Spiels nicht.

Du kennst den Spieleinsatz nicht.

Du kennst das Ziel des Spiels nicht.

Du weißt nicht, wer das Spiel steuert.

Du weißt nicht, wie das Spiel beendet wird.

Du fängst mit 100 Punkten an.

Los geht's.

Harry starrte die „Anleitung“ an. Diese Seite war nicht handgeschrieben; die Schrift war vollkommen gleichmäßig, also künstlich. Sie sah aus, als entstamme sie einer Flotte-Schreibe-Feder, wie diejenige, die er gekauft hatte, um ihr Texte zu diktieren.

Er hatte \emph{absolut keine Ahnung,} was hier los war.

Nun, Schritt eins war sich umziehen und etwas essen. Besser in umgekehrter Reihenfolge, sein Magen fühlte sich ziemlich leer an.

Das Frühstück hatte er natürlich verpasst, doch darauf war er vorbereitet, da er vorher bereits geahnt hatte, dass das passieren könnte. Harry griff in den Beutel, sagte „Müsliriegel“ und ging davon aus, dass eine Schachtel Müsliriegel in seiner Hand erscheinen würde.

Was er in der Hand hielt, fühlte sich nicht nach einer Schachtel Müsliriegel an.

Als Harry die Hand aus dem Beutel nahm, sah er zwei kleine Müsliriegel—nicht einmal annähernd genug für eine Mahlzeit—und einen Notizzettel, auf dem die gleiche Schrift zu sehen war, in der schon die Spielanleitung geschrieben war.

Darauf stand:

Versuch gescheitert: -1 Punkt

Punktzahl: 99

Physischer Zustand: weiterhin hungrig

Mentaler Zustand: verwirrt

„Aaaaaarghhh“, sagte Harrys Mund unwillkürlich.

Er stand ungefähr eine Minute lang da.

Eine Minute später ergab es \emph{immer noch} keinen Sinn und er konnte \emph{immer noch nicht} begreifen, was hier vor sich ging, als ob seine mentalen Hände in Ketten gelegt waren.

Sein Magen, der eigene Prioritäten hatte, schlug ein mögliches Experiment vor.

„Ähm …“, sagte Harry zu dem leeren Raum. „Ich nehme an, es wäre nicht zufällig möglich, dass ich einen Punkt bezahle, um die Schachtel Müsliriegel zurückzubekommen?“

Stille.

Harry griff in den Beutel und sagte „Müsliriegelschachtel.“

Eine Schachtel der richtigen Größe erschien in seiner Hand…doch sie war zu leicht und offen und leer und ein daran angebrachter Notizzettel lautete:

Punkte ausgegeben: 1

Punktzahl: 98

Du erhältst: eine Müsliriegelschachtel

„Ich würde gerne einen Punkt bezahlen und die \emph{Müsliriegel} zurückbekommen„, sagte Harry.

Wieder Stille.

Harry griff in den Beutel und sagte „Müsliriegel.“

Nichts erschien.

Verzweifelt zuckte Harry mit den Schultern und ging zum Kleiderschrank neben seinem Bett, um einen Zaubererumhang rauszuholen. Am Boden des Kleiderschranks, unter seinen Umhängen, lagen die Müsliriegel und ein Zettel:

Punkte ausgegeben: 1

Punktzahl: 97

Du erhältst: 6 Müsliriegel

Du trägst immer noch: Schlafanzug

Iss nicht, solange du deinen Schlafanzug trägst.

Du erhältst sonst eine Schlafanzug-Strafe.

\emph{Und jetzt weiß ich, dass derjenige, der dieses Spiel steuert, verrückt ist.}

„Ich vermute, dass dieses Spiel von Dumbledore gesteuert wird“, sagte Harry. Vielleicht hatte er gerade einen neuen Rekord aufgestellt, weil er es so schnell verstanden hatte.

Stille.

Doch Harry begann ein Muster zu erkennen: Der Zettel erschien immer an der Stelle, wo er als nächstes nachsah. Also sah Harry unter das Bett.

Ha! Ha ha ha ha ha!

Ha ha ha ha ha ha!

Ha! Ha! Ha! Ha! Ha! Ha!

Dumbledore steuert das Spiel nicht.

Schlechte Vermutung

Sehr schlechte Vermutung

—20 Punkte

Und du trägst immer noch einen Schlafanzug

Dies ist dein vierter Zug

Und du trägst immer noch einen Schlafanzug

Schlafanzug-Strafe: -2 Punkte

Punktzahl: 75 Punkte

Mist, jetzt war Harry ratlos. Es war erst sein erster Schultag und außer Dumbledore kannte er niemanden hier, der so verrückt war.

Fast automatisch sammelte Harry einen Umhang und Unterwäsche zusammen, öffnete den Zugang zum Kellergeschoss seines Koffers (er genierte sich etwas und es könnte ja jemand in den Schlafsaal reinplatzen), zog sich um und ging dann wieder hoch, um seinen Schlafanzug wegzulegen.

Bevor er die Schublade öffnete, in die sein Schlafanzug gehörte, hielt Harry inne. Wenn das Muster sich fortsetzte …

„Wie kann ich mehr Punkte sammeln?“, fragte Harry.

Dann öffnete er die Schublade.

Gelegenheiten, Gutes zu tun, sind überall

Doch Dunkelheit ist dort, wo das Licht gebraucht wird

Kosten der Frage: 1 Punkt

Punktzahl: 74 Punkte

Schöne Unterwäsche

Hat deine Mutter sie ausgesucht?

Harry zerknüllte den Zettel in seiner Hand mit feuerrotem Gesicht. Ihm fiel Dracos Fluch ein. \emph{Sohn eines Schlammbluts —}

Er wusste inzwischen, dass er das besser nicht laut sagen sollte. Er würde sonst vermutlich eine Schimpfwort-Strafe bekommen.

Harry rüstete sich mit dem Eselsfell-Beutel und seinem Zauberstab aus. Er riss einen Müsliriegel auf und warf die Verpackung in den Mülleimer, wo er auf einem angebissenen Schokofrosch, einem zerknüllten Briefumschlag und etwas grünem und rotem Einwickelpapier landete. Er steckte die anderen Müsliriegel in seinen Beutel.

Schließlich sah er sich ein letztes Mal um und suchte verzweifelt, letztlich aber umsonst, nach Hinweisen.

Dann verließ Harry den Schlafsaal kauend und machte sich auf die Suche nach den Slytherin-Kerkern. Zumindest \emph{vermutete} er, dass der Hinweis sich darauf bezog.

Sich in den Gängen von Hogwarts zurecht zu finden, war…nun, vermutlich \emph{nicht} so schlimm, wie in einem Escher-Gemälde herumzuwandern. Sowas sagte man nur, weil es dramatisch klang, und nicht weil es stimmte.

Kurze Zeit später war Harry zu der Feststellung gelangt, dass ein Escher-Gemälde gegenüber Hogwarts sowohl Vor- als auch Nachteile hätte. Nachteil: Kein konsistentes Oben und Unten. Vorteil: Zumindest würden die Treppen sich nicht bewegen, \emph{während man noch drauf stand.}

Harry war ursprünglich vier Treppen hochgelaufen, um in seinen Schlafsaal zu gelangen. Nachdem er nicht weniger als zwölf Treppen hinuntergelaufen war, ohne den Kerkern auch nur ein Stückchen näher zu kommen, beschloss Harry, dass erstens ein Escher-Gemälde dagegen ein \emph{Kinderspiel} war; dass er sich zweitens aus irgendeinem Grund \emph{höher} im Schloss befand als zu Anfang; und dass er drittens so vollkommen desorientiert war, dass es ihn nicht gewundert hätte, vom nächsten Fenster aus zwei Monde am Himmel zu sehen.

Plan A wäre, anzuhalten und jemanden nach der Richtung zu fragen, doch es schienen keine Menschen anwesend zu sein, die er fragen könnte—als ob sie allesamt im Unterricht waren, wo sie hingehörten, oder so etwas.

Plan B …

„Ich habe mich verlaufen“, sagte Harry. „Kann, ähm, das Bewusstsein von Schloss Hogwarts mir weiterhelfen, oder so?“

„Ich glaube nicht, dass dieses Schloss ein eigenes Bewusstsein hat“, bemerkte eine hutzelige alte Dame in einem der Gemälde an der Wand. „Es lebt vielleicht, aber ein Bewusstsein hat es nicht.“

Es war einen Moment lang still.

„Sind Sie —“, begann Harry, aber hielt dann den Mund. Wenn er genauer darüber nachdachte…nein, er würde das Gemälde nicht fragen, ob es tatsächlich bewusst handelte in dem Sinne, dass es sich seines eigenen Bewusstseins bewusst war.

„Ich bin Harry Potter“, sagte sein Mund fast automatisch. Etwa ebenso automatisch streckte Harry dem Gemälde seine Hand hin.

Die Frau im Gemälde sah auf Harrys Hand nieder und zog ihre Augenbrauen hoch.

Langsam senkte Harry die Hand.

„Tut mir leid“, sagte Harry. „Ich bin ziemlich neu hier.“

„Das nehme ich wahr, junger Ravenclaw. Wohin möchtest du gehen?“

Harry zögerte. „Ich bin mir nicht sicher“, sagte er.

„Dann bist du möglicherweise bereits da.“

„Nun, wo immer ich auch hin \emph{möchte}, ich glaube nicht, dass \emph{dies} der Ort ist …“ Harry schloss seinen Mund, ihm wurde bewusst, wie idiotisch er sich gerade anhörte. „Ich fange mal anders an: Ich spiele dieses Spiel, aber ich weiß nicht, wie die Regeln lauten —“ Das klappte auch nicht. „Okay, dritter Versuch. Ich suche nach Möglichkeiten, Gutes zu tun, damit ich Punkte sammeln kann, aber ich habe nur diesen kryptischen Hinweis, dass Dunkelheit ist, wo das Licht gebraucht wird, also habe ich versucht, runter zu gehen, aber es scheint so, als ob ich stattdessen hoch gehe …“

Die alte Dame im Gemälde beäugte ihn recht skeptisch.

Harry seufzte. „Mein Leben neigt dazu, etwas seltsam zu werden.“

„Gehe ich richtig in der Annahme, dass du nicht weißt, wohin du gehen willst, oder warum du dort überhaupt hin möchtest?“

\emph{„Vollkommen} richtig.“

Die alte Dame nickte. „Dass du dich im Schloss verirrt hast, ist vermutlich nicht dein größtes Problem, junger Mann.“

„Das stimmt, aber im Gegensatz zu den wichtigeren Problemen ist es ein Problem, dessen Lösung ich angehen kann—und \emph{wow}, diese Unterhaltung hat sich ja in eine Metapher auf die menschliche Existenz verwandelt, das ist mir bis eben gar nicht aufgefallen.“

Die Dame sah Harry abschätzend an. „Du bist also tatsächlich ein guter junger Ravenclaw. Für einen Moment war ich mir nicht sicher. Nun, die Faustregel lautet, dass du nach unten findest, wenn du immer links abbiegst.“

Das kam Harry merkwürdig bekannt vor, doch er konnte sich nicht erinnern, wo er es schonmal gehört hatte. „Ähm…Sie scheinen eine sehr intelligente Person zu sein. Oder ein Bild einer sehr intelligenten Person…auf jeden Fall, haben Sie von einem mysteriösen Spiel gehört, das man nur einmal spielen kann und dessen Regeln man nicht gesagt bekommt?“

„Das Leben“, sagte die Dame sofort. „Das ist eines der offensichtlichsten Rätsel, von denen ich je gehört habe.“

Harry blinzelte. „Nein“, sagte er langsam. „Ich meine, ich habe tatsächlich einen Zettel bekommen, worauf stand, dass ich das Spiel spielen sollte, aber dass ich die Regeln nicht erfahren würde, und irgendjemand hinterlässt mir kleine Papierschnipsel, die mir mitteilen, wie viele Punkte ich verloren habe, weil ich die Regeln verletze, zum Beispiel zwei Punkte Abzug, weil ich einen Schlafanzug getragen habe. Kennen Sie irgendjemand hier auf Hogwarts, der verrückt und mächtig genug wäre, so etwas zu tun? Ich meine, außer Dumbledore?“

Das Bild der Dame seufzte. „Ich bin nur ein Bild, junger Mann. Ich kenne das Hogwarts, das einst war—nicht das Hogwarts, das ist. Ich kann dir nur sagen, dass—wenn dies ein Rätsel wäre—die Antwort lauten würde, dass das Spiel das Leben ist, und dass wir zwar nicht alle Regeln selbst schreiben, aber wir uns immer nur selbst Punkte geben oder abziehen. Wenn es kein Rätsel, sondern die Wirklichkeit ist, dann weiß ich es nicht.“

Harry verbeugte sich tief vor dem Bild. „Ich danke Ihnen, Mylady.“

Die Dame machte einen Knicks. „Ich wünschte, ich könnte sagen, dass ich mit Freude an dich zurückdenken werde“, sagte sie, „aber ich werde mich vermutlich gar nicht an dich erinnern. Lebe wohl, Harry Potter.“

Er verbeugte sich erneut und ging dann die nächste Treppe hinunter.

Nachdem er vier Mal links abgebogen war, fand er sich in einem Gang wieder, der abrupt vor einigen großen Felsbrocken endete—fast als ob der Gang eingestürzt wäre, nur dass die umliegenden Wände und die Decke intakt waren und aus völlig normalen Mauersteinen bestanden.

„Na gut“, sagte Harry zu dem leeren Gang, „ich gebe auf. Ich bitte um einen weiteren Tipp. Wie komme ich an den Ort, wo ich hin soll?“

„Ein Tipp! Ein Tipp, sagst du?“

Die aufgeregte Stimme kam aus einem Gemälde, das nicht weit weg an der Wand hing; ein Portrait eines Zauberers von mittlerem Alter in schreiend pinkem Umhang. Der Zauberer trug einen schlaffen, alten Spitzhut mit einem Fisch darauf (nicht etwa ein Bild von einem Fisch, sondern tatsächlich einen echten Fisch).

„Ja!“, sagte Harry. „Einen Tipp! Einen Tipp, sagte ich! Aber nicht nur \emph{irgendeinen} Tipp, ich suche nach einem \emph{ganz bestimmten} Tipp, er ist für ein Spiel, das ich spiele —“

„Ja, ja! Ein Tipp für ein Spiel! Du bist Harry Potter, nicht wahr? Ich bin Cornelion Flubberwalt! Mir hat Erin die Gemahlin erzählt, der von Lord Wieselnase erzählt wurde, dem von…ich weiß es nicht mehr. Aber es war eine Nachricht, die \emph{ich} dir geben sollte! \emph{Ich!} An mich hat schon seit…ich weiß nicht wie lange, vermutlich noch nie jemand gedacht, ich hänge hier in diesem verdammten nutzlosen Korridor rum—ein Tipp! Ich habe deinen Tipp! Er wird dich nur drei Punkte kosten! Möchtest du ihn hören?“

„Ja! Ich möchte ihn hören!“ Harry überlegte sich, dass er den Sarkasmus vermutlich besser unter Kontrolle behalten sollte, aber er konnte sich einfach nicht zurückhalten.

„Die Dunkelheit findet man zwischen dem grünen Lernraum und McGonagalls Verwandlungs-Klassenzimmer! Das war der Tipp! Und jetzt beeile dich, du bist langsamer als eine Horde Schnecken! Zehn Punkte Abzug für's Bummeln! Jetzt hast du 61 Punkte! Das war der Rest der Nachricht!“

„Danke“, sagte Harry. Er geriet bei dem Spiel ganz schön in Rückstand. „Ähm…du weißt nicht zufällig, woher die Nachricht \emph{ursprünglich} kam, oder?“

„Sie wurde von einer tonlosen Stimme gesprochen, die aus einer Kluft mitten im Raum erschallte; einer Kluft, die sich über einem feurigen Abgrund auftat! So wurde es mir berichtet!“

Harry war sich inzwischen nicht mehr sicher, ob er solche Sachen anzweifeln oder sie einfach hinnehmen sollte. „Und wie finde ich den grünen Lernraum und McGonagalls Verwandlungs-Klassenzimmer?“

„Drehe dich einfach um und gehe nach links, rechts, runter, runter, rechts, links, rechts, hoch und wieder links, dann stehst du vor dem grünen Lernraum, und wenn du dann rein und auf der gegenüberliegenden Seite wieder raus gehst, stehst du in einem weiten, gewundenen Gang, der auf eine Kreuzung zuläuft, und auf der rechten Seite dieser Kreuzung befindet sich ein langer, gerader Gang, der auf die Verwandlungs-Klassenzimmer zuläuft!“ Das Bildnis des Mannes erstarrte. „So war es zumindest, als \emph{ich} auf Hogwarts war. Heute \emph{ist} ein Montag eines ungeradzahligen Jahres, oder?“

„Bleistift und Drehpapier“, sagte Harry zu seinem Beutel. „Ähm, Quatsch, Papier und Drehbleistift.“ Er sah auf. „Könntest du das wiederholen?“

Nachdem er sich zwei weitere Male verirrt hatte, bekam Harry das Gefühl, dass er die Grundregel verstanden hatte, die man im ewig wandelnden Labyrinth von Hogwarts beachten musste: \emph{Frage ein Portrait nach dem Weg.} Wenn das irgendeine unglaublich tiefsinnige Lebensweisheit widerspiegeln sollte, dann kam er nicht darauf, welche es sein konnte.

Der grüne Lernraum war ein überraschend angenehmer Ort, in den das Sonnenlicht durch grün getönte Fenster hereinfiel, die Drachen in ruhiger, ländlicher Umgebung zeigten. Im Raum standen Stühle, die extrem komfortabel aussahen, und Tische, die sehr gut dafür geeignet schienen, dort mit zwei, drei Freunden zusammen zu lernen.

Harry \emph{konnte} nicht geradewegs durch den Raum und durch die gegenüberliegende Tür hinaus gehen. An den Wänden standen \emph{Bücherschränke} und der Name „Verres“ verpflichtete ihn dazu, hinzugehen und einige Buchtitel zu lesen. Doch er erinnerte sich an den Vorwurf, dass er zu langsam sei, beeilte sich und verließ den Raum.

Er lief den „weiten, gewundenen Gang“ entlang, als er den Schrei einer hohen Jungenstimme hörte.

Das war Grund genug für Harry, unverzüglich und so schnell er nur konnte loszurennen, ohne seine Kräfte zu schonen und ohne Angst, dass er irgendwo anecken könnte; ein plötzlicher, verzweifelter Sprint, der ebenso plötzlich endete, als er fast in eine Gruppe von sechs Hufflepuff-Erstklässlern gerannt wäre …

… die sich dort verängstigt zusammendrängten und so aussahen, als wollten sie unbedingt etwas tun, jedoch nicht wussten, was—vermutlich wegen der fünf älteren Slytherins, die einen anderen Jungen eingekesselt hatten.

Harry war plötzlich sehr wütend.

„\emph{Entschuldigt mal!}“, rief Harry so laut er konnte.

Es wäre vermutlich nicht notwendig gewesen. Die Schüler hatten ihre Blicke längst auf ihn gerichtet. Doch er hatte die ganze Szene erfolgreich einfrieren lassen.

Harry ging an den Hufflepuffs vorbei auf die Slytherins zu.

Diese sahen ihn teils wütend, teils amüsiert oder sogar erfreut an.

Ein Teil von Harrys Gehirn schrie panisch, dass das viel ältere und größere Jungs seien, die ihn platt machen könnten.

Ein anderer Teil bemerkte trocken, dass jeder, der bei einem ernsthaften Angriff auf den Jungen, der lebt, beobachtet würde, \emph{massiven} Ärger bekommen würde—insbesondere dann, wenn es einige ältere Slytherins waren und sieben Hufflepuffs zusahen. Die Wahrscheinlichkeit, dass diese ihm vor so vielen Zeugen irgendwelche bleibenden Verletzungen zufügen würden, war nahezu null. Die einzige Waffe, die die älteren Jungs gegen ihn hatten, war seine eigene Furcht, so er sie denn zuließ.

Dann sah Harry, dass der Junge, den sie eingekesselt hatten, Neville Longbottom war.

Natürlich.

Damit war die Sache klar. Harry hatte beschlossen, sich demütig bei Neville zu entschuldigen, und das bedeutete, dass Neville \emph{ihm} gehörte, also wie konnten die es \emph{wagen?}

Harry holte aus, griff Neville am Handgelenk und schleuderte ihn aus dem Kreis der Slytherins raus. Der Junge stolperte schockiert heraus und Harry drängte sich fast gleichzeitig selbst durch die Lücke.

So stand Harry nun inmitten der Slytherins, wo Neville gestanden hatte, und sah zu den viel älteren, größeren und stärkeren Jungen auf.

„Hallo“, sagte Harry. „Ich bin der Junge, der lebt.“

Es folgte eine recht peinliche Pause. Niemand schien zu wissen, wohin das Gespräch nun steuern würde.

Harrys Augen senkten sich und sahen einige Bücher und Pergamente auf dem Boden verstreut. Ach, das alte Spiel, bei dem man den Jungen seine Bücher aufsammeln ließ, um sie ihm dann wieder aus der Hand zu schlagen. Harry selbst war nie das Opfer dieses Spiels gewesen, aber er hatte eine gute Vorstellungsgabe, und die machte ihn jetzt wütend. Naja, sobald die Situation geklärt war, würde Neville problemlos zurückkommen und die Bücher aufsammeln können, vorausgesetzt, dass die Slytherins sich auf ihn konzentrierten und nicht auf die Idee kamen, irgendwas mit den Büchern anzustellen.

Leider wurden seine umherwandernden Augen bemerkt. „Oh“, sagte der größte der Slytherins, „wolltest du etwa die kleinen Bücher —“

„Halt's Maul“, sagte Harry kühl. \emph{Verwirre sie. Tue nicht das, was sie erwarten. Falle nicht in ein Verhaltensmuster, was sie darin bestärkt, dich zu schikanieren.} „Ist das Teil eines unglaublich cleveren Plans, der euch irgendwelche zukünftigen Vorteile bescheren wird, oder ist das eine genau so sinnlose Schande für den Namen Salazar Slytherins, wie es —“

Der größte Junge schubste Harry Potter heftig und er fiel aus dem Kreis der Slytherins raus und der Länge nach zu Boden.

Die Slytherins lachten.

Harry stand, so hatte er das Gefühl, unerträglich langsam auf. Er wusste noch nicht, wie er seinen Zauberstab benutzte, doch angesichts der Umstände gab es keinen Grund, warum ihn das aufhalten sollte.

„Ich möchte \emph{so viele Punkte wie notwendig} bezahlen, um diese Person loszuwerden“, sagte Harry und zeigte mit seinem Finger auf den größten Slytherin.

Dann hob Harry seine andere Hand, sagte „Abrakadabra“ und schnippte mit den Fingern.

Beim Wort „Abrakadabra“ schrien zwei der Hufflepuffs auf, darunter auch Neville, drei andere Slytherins sprangen verzweifelt beiseite und der größte Slytherin stolperte mit einem schockierten Gesichtsausdruck zurück, während rote Flecken auf seinem Gesicht, seinem Hals und seinem Umhang erschienen.

\emph{Das} hatte Harry \emph{nicht} erwartet.

Langsam griff der größte Slytherin sich an den Kopf und zog die Kirschkuchenform ab, die auf seinem Kopf gelandet war. Er hielt die Form für einen Moment in der Hand, starrte sie an und ließ sie dann zu Boden fallen.

Es war vermutlich nicht der ideale Zeitpunkt um loszulachen, doch genau das tat ein Hufflepuff nun.

Dann erblickte Harry einen Zettel an der Unterseite der Kuchenform.

„Moment mal“, sagte Harry und sprang vor, um den Zettel aufzuheben. „Der ist für mich, glaube ich …“

„\emph{Du}“, knurrte der größte Slytherin, „\emph{Du. Wirst. Sowas. Von —}“

\emph{„Schau} dir das an!“, rief Harry und wedelte mit dem Zettel vor der Nase des älteren Slytherins herum. „\emph{Schau} dir das mal an! Ist das denn zu glauben, dass ich 30 Punkte für die Lieferung und Zustellung eines einzigen, lausigen Kuchens zahlen muss? 30 Punkte! Da mache ich ja einen Verlust, selbst nachdem ich einen unschuldigen Jungen aus der Bedrängnis gerettet habe! Und dazu Lagerungskosten? Transportkosten? Zustellgebühren? Warum zum Teufel denn \emph{Zustellgebühren} für einen \emph{Kuchen?}“

Wieder herrschte eine unangenehme Stille. In Gedanken stopfte Harry dem Hufflepuff, der einfach nicht aufhören konnte zu kichern, das Maul. Dieser Idiot würde damit noch Schaden anrichten.

Harry ging einen Schritt zurück und warf den Slytherins den mörderischsten Blick zu, den er hinbekam. „Nun haut ab, sonst werde ich eure Existenz immer surrealer machen, bis ihr verschwindet. Seid gewarnt…wenn ihr euch mit \emph{meinem} Leben anlegt, wird \emph{euer} Leben…\emph{eine haarige Angelegenheit} werden, wenn ihr versteht, was ich meine?“

In einer einzigen Bewegung zog der größte Slytherin seinen Zauberstab, richtete ihn auf Harry und bekam im selben Moment einen weiteren Kuchen an die andere Seite seines Kopfes geklatscht, dieses Mal mit Blaubeeren.

Die am Kuchen befestigte Notiz war diesmal groß und gut lesbar. „Du solltest dir den Zettel am Kuchen durchlesen“, bemerkte Harry. „Ich glaube, dieses Mal ist er für dich.“

Der Slytherin griff langsam hoch, nahm die Kuchenform, drehte sie um, wobei mit einem feuchten \emph{platsch} einiger Blaubeerkuchen zu Boden fiel, und las:

\emph{WARNUNG

Auf den Spieler darf KEINE Magie angewendet werden, solange das \textbf{Spiel}} andauert.

Über weitere Einmischungen wird die \textbf{Spielleitung} unterrichtet.

Die vollkommene Verwirrung auf dem Gesicht des Slytherins war ein Kunstwerk. Harry fing so langsam an, diese Spielleitung zu mögen.

„Hört mal“, sagte Harry, „wollen wir's dabei belassen? Ich glaube, die Dinge geraten hier außer Kontrolle. Wie wär's, wenn ihr zurück nach Slytherin geht, ich gehe zurück nach Ravenclaw und wir beruhigen uns alle erstmal ein bisschen, okay?“

„Ich habe eine bessere Idee“, zischte der größte Slytherin. „Wie wär's, wenn ich dir aus Versehen alle deine Finger breche?“

„Und wie, in Merlins Namen, willst du einen glaubwürdigen Unfall arrangieren, nachdem du diese Drohung vor einem Dutzend Zeugen kundgetan hast, du \emph{Idiot} …“

Der Slytherin griff langsam und absichtlich nach Harrys Händen und Harry gefror auf der Stelle, bis der Teil seines Gehirns, der das Alter und die Stärke des anderen Jungen bemerkte, sich endlich Aufmerksamkeit verschaffte und schrie, \emph{Was zum Teufel tue ich hier?}

„Warte!“, sagte einer der anderen Slytherins, dessen Stimme plötzlich panisch klang. „Stopp, du solltest das besser nicht tun!“

Der größte Slytherin ignorierte ihn, nahm Harrys rechte Hand fest in seine linke Hand und Harrys Zeigefinger in die rechte Hand.

Harry sah dem Slytherin fest in die Augen. Ein Teil von Harry schrie, das sollte nicht passieren, das \emph{durfte} nicht passieren, Erwachsene würden so etwas niemals zulassen …

Langsam bog der Slytherin seinen Zeigefinger nach hinten.

\emph{Er hat mir noch nicht den Finger gebrochen, und solange er das nicht tut, kriegt er nicht mal ein Zusammenzucken von mir zu sehen. Bis dahin ist das bloß ein weiterer Versuch, mir Angst einzujagen.}

„Stopp!“, sagte der Slytherin, der vorher schon widersprochen hatte. „Stopp, das ist eine sehr schlechte Idee!“

„Dem stimme ich zu“, sagte eine eisige Stimme. Die Stimme einer älteren Frau.

Der größte Slytherin ließ Harrys Hand los und schreckte zurück, als ob er sich verbrannt hätte.

„Professor Sprout!“, rief einer der Hufflepuffs und klang so erleichtert, wie Harry noch nie zuvor in seinem Leben jemanden gehört hatte.

Als Harry sich drehte, erschien in seinem Blickfeld eine plumpe, kleine Frau mit unordentlich gekräuseltem, grauen Haar und dreckiger Kleidung. Sie zeigte mit anklagend erhobenem Finger auf die Slytherins. „Erklären Sie sich“, sagte sie. „Was tun Sie mit meinen Hufflepuffs und …“, sie sah ihn an, „meinem guten Schüler Harry Potter.“

\emph{Oh je, das stimmt,} ihren \emph{Unterricht habe ich heute Morgen verpasst.}

„Er hat gedroht, uns zu töten“, platzte es aus dem anderen Slytherin heraus, der vorher „Stopp!“ gerufen hatte.

„Was?“, sagte Harry völlig ahnungslos. „Das habe ich \emph{nicht!} Wenn ich euch töten wollte, dann würde ich es nicht vorher in der Öffentlichkeit androhen!“

Ein dritter Slytherin lachte hilflos und hörte dann abrupt auf, als die anderen Jungs ihm böse Blicke zuwarfen.

Professor Sprouts Gesichtsausdruck wirkte eher skeptisch. „Was für eine Todesdrohung soll das gewesen sein?“

„Der Todesfluch! Er hat so getan, als ob er den Todesfluch auf uns spricht!“

Professor Sprout blickte zu Harry. „Ja, eine schreckliche Drohung, wenn das von einem Elfjährigen kommt. Allerdings sollten Sie das niemals auch nur antäuschen, Harry Potter.“

„Ich weiß nicht mal, wie der Todesfluch \emph{lautet}“, sagte Harry sofort. „Und ich hatte meinen Zauberstab niemals rausgeholt.“

Jetzt blickte Professor Sprout Harry skeptisch an. „Dann nehme ich an, der Junge hat \emph{sich selbst} einen Kuchen an den Kopf geworfen?“

„Er hat seinen Zauberstab nicht benutzt!“, platzte es aus einem der jungen Hufflepuffs heraus. „Ich weiß auch nicht, wie er es getan hat, er hat einfach mit den Fingern geschnippt und dann war da ein Kuchen!“

„Tatsächlich?“, sagte Professor Sprout nach einem Moment. Sie zog ihren eigenen Zauberstab. „Ich möchte das nicht von Ihnen fordern, da Sie hier das Opfer zu sein scheinen, aber würde es Ihnen etwas ausmachen, wenn ich Ihren Zauberstab untersuche um das zu überprüfen?“

Harry zog seinen Zauberstab. „Was soll ich —“

\emph{„Priori Incantatem“,} sagte Sprout. Sie legte die Stirn in Falten. „Wie seltsam, Ihr Zauberstab wurde offenbar noch gar nicht benutzt.“

Harry zuckte mit den Schultern. „Das wurde er auch nicht, ich hab meinen Zauberstab und meine Schulbücher erst vor wenigen Tagen bekommen.“

Sprout nickte. „Dann liegt hier ein klarer Fall von versehentlicher Zauberei vor, weil ein Junge sich bedroht fühlte. Und die Regeln sagen ausdrücklich, dass wir ihn dafür nicht verantwortlich machen. Was \emph{euch} angeht …“ Sie wandte sich den Slytherins zu. Ihre Augen senkten sich betont langsam zu Nevilles Büchern, die auf dem Boden lagen.

Eine Zeit lang herrschte Stille, während sie die fünf Slytherins musterte.

„Drei Punkte Abzug, für \emph{jeden} von Ihnen“, sagte sie schließlich. „Und sechs seinetwegen“, deutete sie auf den Jungen, der mit Kuchen bekleckert war. „Und rühren Sie \emph{nie wieder} meine Hufflepuffs oder meinen Schüler Harry Potter an. Jetzt \emph{gehen Sie.}“

Sie brauchte es kein zweites Mal sagen; die Slytherins drehten sich um und eilten davon.

Neville begann seine Bücher aufzusammeln. Er schien zu weinen, doch nur ein bisschen. Es mochte noch die Wirkung des Schocks sein oder weil die anderen Jungen ihm halfen.

\emph{„Vielen} Dank, Harry Potter“, sagte Professor Sprout zu ihm. „Sieben Punkte für Ravenclaw, einen für jeden Hufflepuff, den sie beschützt haben. Mehr werde ich dazu nicht sagen.“

Harry blinzelte. Er hatte eigentlich eine Ermahnung erwartet, dass er Ärger meiden sollte, und dazu eine gewaltige Standpauke, weil er seine allererste Unterrichtsstunde verpasst hatte.

Vielleicht \emph{hätte} er nach Hufflepuff gehen sollen. Sprout war cool.

„Scourgify“, sagte Sprout zu den Kuchenresten auf dem Boden, die sofort verschwanden. Und sie ging den Gang entlang, der zum grünen Lernraum führte.

„Wie hast du das \emph{gemacht?}“, flüsterte einer der Hufflepuffs sobald sie gegangen war.

Harry lächelte selbstgefällig. „Ich kann geschehen lassen, was immer ich will, nur indem ich mit den Fingern schnippe.“

Die Augen des Jungen wurden groß. „\emph{Wirklich?}“

„Nein“, sagte Harry. „Aber wenn du allen Leute hiervon erzählst, achte darauf, dass du auch Hermine Granger davon erzählst. Sie ist Erstklässlerin in Ravenclaw und kann eine Anekdote erzählen, die dich amüsieren könnte.“ Er hatte absolut keine Ahnung was geschah, doch er wollte die Gelegenheit nicht verpassen, seinen legendären Ruf noch zu untermauern. „Ach, und was hatte das eigentlich mit dem Todesfluch zu tun?“

Der Junge sah ihn seltsam an. „Du weißt es wirklich nicht?“

„Wenn ich es wüsste, würde ich nicht fragen.“

„Der Todesfluch lautet“—der Junge schluckte, seine Stimme wurde zu einem Flüstern und er hielt seine Arme weit von sich gestreckt, als wolle er deutlich klarstellen, dass er keinen Zauberstab hielt—„\emph{Avada Kedavra.}“

\emph{Natürlich.}

Harry setzte das auf die anwachsende Liste der Dinge, die er niemals seinem Vater, Professor Michael Verres-Evans, erzählen würde. Es war so schon schlimm genug, zu erzählen, dass man die einzige Person war, die den gefürchteten Todesfluch überlebt hat—auch wenn man nicht ergänzte, dass der Todesfluch „Abrakadabra“ lautet.

„Ich verstehe“, sagte Harry nach einer kurzen Pause. „Nun, das war wohl das letzte Mal, dass ich \emph{das} gesagt habe, bevor ich mit den Fingern schnippe.“ Obwohl der Effekt ein taktischer Vorteil sein könnte.

\emph{„Warum} hast du —“

„Ich bin bei Muggeln aufgewachsen, Muggel halten es für eine Art Witz. Wirklich, so ist das. Entschuldige, aber kannst du mir nochmal sagen, wie du heißt?“

„Ich bin Ernie Macmillan“, sagte der Hufflepuff. Er streckte seine Hand aus und Harry schüttelte sie. „Es ist eine Ehre, dich zu treffen.“

Harry verbeugte sich leicht. „Freut mich, dich zu treffen; lass das mit der Ehre.“

Dann umringten die anderen Jungen ihn und überfluteten ihn mit Vorstellungen. Als sie fertig waren, schluckte Harry. Das würde sehr schwer werden. „Ähm…wenn ihr mich alle mal entschuldigt…ich muss Neville etwas sagen.“

Alle Augen richteten sich auf Neville, der einen Schritt zurück ging und einen beklommenen Gesichtsausdruck bekam.

„Ich nehme an“, sagte Neville mit verschüchterter Stimme, „du willst sagen, ich hätte mutiger sein sollen —“

„Oh nein, nichts dergleichen!“, sagte Harry hastig. „Es hat nichts \emph{damit} zu tun. Es geht, ähm, um etwas, was der Sprechende Hut mir gesagt hat.“

Plötzlich sahen die anderen Jungen \emph{äußerst} neugierig drein, bis auf Neville, der nun noch besorgter wirkte.

Harry hatte einen Kloß in der Kehle. Er wusste, dass er es einfach sagen sollte, aber es fühlte sich an, als hätte er einen großen Ziegelstein geschluckt, der nun im Weg war.

Es war, als ob Harry seine Lippen bewusst steuern und jede Silbe einzeln herausbringen musste, doch er schaffte es schließlich. „Bitte ent-schul-di-ge“—er atmete aus und wieder tief ein—„was ich, ähm, beim letzten Mal gemacht hatte. Du brauchst nicht großherzig sein oder so, ich kann verstehen, wenn du mich deswegen hasst. Es geht hier nicht darum, dass ich mit der Entschuldigung jemanden beeindrucken will, oder dass du sie annehmen sollst. Was ich getan habe, war falsch.“

Es war einen Moment still.

Neville presste seine Bücher fester gegen die Brust. „Warum hast du das gemacht?“, fragte er in einer dünnen, zitternden Stimme. Er blinzelte, als wolle er Tränen zurückhalten. „Warum macht \emph{jeder} so etwas mit mir, sogar der Junge, der lebt?“

Harry fühlte sich plötzlich kleiner als er sich jemals zuvor in seinem Leben gefühlt hatte. „Entschuldige bitte“, sagte er nochmals, mit inzwischen heiserer Stimme. „Es ist nur…du sahst so ängstlich aus, es war, als hinge ein Schild mit der Aufschrift ‚Opfer` über deinem Kopf, und ich wollte dir zeigen, dass die Dinge \emph{nicht} immer schlimm ausgehen, dass die Monster dir manchmal Schokolade schenken…Ich dachte, wenn ich dir das zeige, dann würdest du vielleicht merken, dass es nicht viele Sachen gibt, vor denen man Angst haben muss —“

„Aber es \emph{gibt} doch welche“, flüsterte Neville. „Du hast es heute gesehen, es \emph{gibt} welche!“

„Sie hätten dir vor Zeugen nichts wirklich Schlimmes angetan. Ihre Hauptwaffe ist die Angst. Deswegen hatten sie es auf \emph{dich} abgesehen, weil sie sehen konnten, dass du Angst hast. Ich wollte dir die Angst nehmen…dir zeigen, dass die Angst schlimmer ist als die Sache selbst…so habe ich mir das zumindest eingeredet, aber der Sprechende Hut hat gesagt, dass ich mich selbst belüge und dass ich es in Wirklichkeit gemacht habe, weil es mir Spaß gemacht hat. Deswegen möchte ich mich also entschuldigen —“

„Du hast mir wehgetan“, sagte Neville. „Gerade eben. Als du mich gegriffen und von ihnen weggezerrt hast.“ Neville zeigte die Stelle auf seinem Arm, wo Harry ihn gepackt hatte. „Ich werde womöglich einen blauen Fleck bekommen, so sehr hast du gezerrt. Du hast mir sogar mehr wehgetan als die Slytherins, die mich gestoßen haben.“

\emph{„Neville!}“, zischte Ernie, „er wollte dich \emph{retten!}“

„Es tut mir leid“, flüsterte Harry. „Als ich das gesehen habe, wurde ich einfach…wirklich wütend …“

Neville sah ihn fest an. „Also hast du mich da raus gezerrt, dich selbst an die gleiche Stelle gestellt und dann gesagt ‚Hallo, ich bin der Junge, der lebt`.“

Harry nickte.

„Ich glaube, du wirst eines Tages richtig cool sein“, sagte Neville. „Aber im Moment bist du es nicht.“

Harry schluckte den Kloß in seiner Kehle runter und ging weg. Er folgte dem Korridor bis zur nächsten Kreuzung, bog dann links in einen Gang ab und lief einfach weiter, ohne aufzusehen.

Was \emph{sollte} er denn tun? Nie wütend werden? Er war sich nicht sicher, ob er irgendwas hätte tun können, wenn er nicht wütend gewesen wäre; und wer weiß, was dann mit Neville und seinen Büchern geschehen wäre. Außerdem hatte Harry genug Fantasy-Bücher gelesen, um zu wissen, wie \emph{diese} Geschichte verlief: Er würde versuchen, seine Wut zu unterdrücken, würde daran scheitern und sie würde wieder hervorbrechen. Und nach einem langen Weg bis zur Selbsterkenntnis würde er schließlich lernen, dass seine Wut ein Teil von ihm war und dass er sie nur dann nutzen konnte, wenn er das akzeptiert hatte. \emph{Star Wars} war das einzige Universum in dem es tatsächlich die richtige Antwort war, seine negativen Emotionen vollkommen loszuwerden. Und Yoda hatte irgendetwas an sich, wofür Harry den kleinen grünen Mistkerl schon immer gehasst hatte.

Der offensichtliche und zeitsparende Plan lautete also, den Weg zur Selbsterkenntnis zu überspringen und gleich zu dem Punkt zu kommen, wo er feststellte, dass er die Wut als Teil von sich selbst akzeptieren musste, um sie zu kontrollieren.

Das Problem daran war, dass er sich nicht außer Kontrolle \emph{fühlte}, wenn er wütend war. Dieser kalte Zorn sorgte dafür, dass er sich so fühlte, als hätte er alles unter Kontrolle. Erst zurückblickend stellte er fest, dass die Ereignisse offenbar…völlig außer Kontrolle geraten waren, irgendwie.

Er fragte sich, wie sehr so etwas die Spielleitung interessierte, und ob er dafür Punkte gewinnen würde, oder Punktabzug bekäme. Harry selbst hatte das Gefühl, dass er dafür einige Punkte verloren hatte, und er war sich sicher, dass die alte Dame im Bild ihm gesagt hätte, dass seine Meinung die einzige wäre, die zählte.

Außerdem fragte Harry sich, ob die Spielleitung Professor Sprout geschickt hatte. Es war eine logische Folgerung: Der Zettel hatte damit gedroht, die Spielleitung zu informieren—und plötzlich war Professor Sprout aufgetaucht. Vielleicht \emph{war} Professor Sprout die Spielleitung—die \emph{Hauslehrerin von Hufflepuff} wäre wohl die \emph{letzte} Person, die irgendwer verdächtigen würde, weshalb sie auf Harrys Liste der verdächtigen Personen weit oben stehen sollte. Er hatte schließlich den einen oder anderen Mystery-Roman gelesen.

„Wie ist mein Punktestand?“, sagte Harry laut.

Ein Stück Pergament flog über seinen Kopf, als ob es jemand von hinter ihm geworfen hätte—Harry drehte sich um, doch dort war niemand zu sehen—und als Harry sich wieder nach vorn drehte, lag das Blatt auf dem Boden.

Auf dem Blatt stand:

Punkte für Stil: 10

Punkte für umsichtiges Denken: -3.000.000

Hauspunkte-Bonus: 70

Punktzahl: -2.999.871

Verbleibende Züge: 2

„\emph{Minus drei Millionen Punkte?}“, sagte Harry empört zum leeren Korridor. „Das ist doch etwas übertrieben! Ich möchte bei der Spielleitung Einspruch erheben! Und wie soll ich drei Millionen Punkte in den nächsten zwei Zügen aufholen?“

Ein weiterer Zettel flog über seinen Kopf.

Einspruch: Abgelehnt

Die falschen Fragen gestellt: -1.000.000.000.000 Punkte

Punktzahl: -1.000.002.999.871

Verbleibende Züge: 1

Harry gab es auf. Bei einem verbleibenden Zug konnte er nur noch seine beste Vermutung äußern, auch wenn sie nicht besonders gut war. „Ich vermute, dass das Spiel das Leben repräsentiert.“

Ein letztes Blatt flog über seinen Kopf.

Versuch gescheitert

Gescheitert Gescheitert Gescheitert

Ooooooohjeeeeeeee

Punktzahl: Minus Unendlich

DU HAST DAS SPIEL VERLOREN

Letzte Anweisung:

\emph{gehe zu Professor McGonagalls Büro}

Die letzte Zeile war in seiner eigenen Handschrift geschrieben.

Harry starrte eine Weile darauf und zuckte dann mit den Schultern. Na gut. Professor McGonagalls Büro dann also. Wenn \emph{sie} die Spielleitung war …

Okay, ehrlich gesagt hatte Harry absolut keine Ahnung, wie er sich fühlen würde, wenn Professor McGonagall die Spielleitung war. Sein Gehirn lieferte ihm keine Antwort darauf. Es war wortwörtlich unvorstellbar.

Einige Portraits später—es war kein langer Weg, Professor McGonagalls Büro war nicht weit vom Verwandlungs-Klassenzimmer entfernt, zumindest nicht montags in ungeradzahligen Jahren—stand Harry vor der Tür zu ihrem Büro.

Er klopfte.

„Herein“, sagte Professor McGonagalls gedämpfte Stimme.

Er trat ein.

„Mr~Potter?“, sagte Professor McGonagall. „Ich habe Sie nicht erwartet. Worum geht es?“

