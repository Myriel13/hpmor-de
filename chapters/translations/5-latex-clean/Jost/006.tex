

\hypertarget{der-planungstrugschluss}{% \section{6. Der Planungstrugschluss}\label{der-planungstrugschluss}}

Dieses Kapitel ist etwas konfus entstanden: Eigentlich hatte Martin es schon übersetzt, dann meldete Arne sich, wollte bei der Übersetzung helfen und hat Kapitel 6 schonmal übersetzt~…

Dann hatten wir also zwei Versionen des gleichen Kapitels; ich habe daraus eine Version gebastelt, indem ich mir aus beiden Vorlagen das schönste rausgesucht habe und einige Passage selbst übersetzt habe. Überarbeitet haben wir es dann alle.

Hoffentlich merkt man beim Lesen nichts mehr von diesem Chaos?!

-\/-\/-\/-\/-\/-\/-\/-\/-\/-\/-\/-\/-\/-\/- ~ 6. Der Planungstrugschluss ~ -\/-\/-\/-\/-\/-\/-\/-\/-\/-\/-\/-\/-\/-\/-

\emph{Sie denken, Ihr Tag war surreal? Schauen Sie sich meinen an.}

\later

\emph{Einige} Leute hätten bis \emph{nach} ihrem ersten Besuch in der Winkelgasse gewartet.

"Beutel mit Element 79", sagte Harry und zog seine Hand ohne Goldbeutel aus der Eselsfelltasche.

Die \emph{meisten} Leute hätten zumindest gewartet, bis sie ihren \emph{Zauberstab} hatten.

"Beutel mit \emph{Okane}", sagte Harry. Der schwere Beutel Gold sprang ihm in die Hand. Harry entnahm den Beutel und legte ihn dann gleich wieder in den Eselsfell-Beutel zurück. Er nahm seine Hand aus der Tasche, tat sie wieder hinein und sagte "Beutel mit Einheiten des wirtschaftlichen Austauschs." Dieses Mal kam seine Hand leer zurück.

Harry Potter hatte endlich einen magischen Gegenstand in die Finger bekommen. Warum sollte er warten?

"Professor McGonagall", sagte Harry zu der amüsierten Hexe neben ihm, "können Sie mir in einer Sprache, die ich nicht kennen kann, zwei Wörter sagen, wovon eines Gold bedeutet und das andere etwas, das kein Geld ist? Aber sagen Sie mir nicht, welches Wort welches ist."

"\emph{Ahava} und \emph{zahav}," sagte McGonagall. "Das ist Hebräisch und das andere Wort bedeutet Liebe."

"Danke, Professor. Beutel mit \emph{ahava.}". Nichts.

"Beutel mit \emph{zahav}." Und er erschien sogleich in seiner Hand.

"Zahav ist Gold?", fragte Harry und McGonagall nickte.

Harry überdachte seine gesammelten experimentellen Daten. Es war nur eine überaus grobe und vorläufige Untersuchung, aber sie reichte aus, um eine Schlussfolgerung zu stützen: "\emph{Aaaaaaaarrrgh, das ergibt überhaupt keinen Sinn!}"

Die Hexe neben ihm hob eine Augenbraue. "Probleme, Mr~Potter?"

"Ich habe gerade jede einzelne Hypothese widerlegt, die ich hatte! Wie kann die Tasche wissen, dass 'Beutel mit 115 Galleonen' okay ist, aber nicht 'Beutel mit 90 plus 25 Galleonen'? Sie kann \emph{zählen} aber nicht \emph{addieren}? Sie kann Nomen verstehen, aber keine Nominalsätze gleicher Bedeutung? Derjenige, der dies gemacht hat, sprach vermutlich kein Japanisch und \emph{ich} spreche kein Hebräisch, also nutzt es weder \emph{deren} Kenntnisse noch \emph{meine} Kenntnisse ---" Harry wedelte hilflos mit der Hand. "Die Regeln scheinen \emph{irgendwie} konsistent, aber sie \emph{bedeuten} nichts. Ich werde nicht einmal danach fragen, woher eine Tasche Stimmerkennung und menschliche Sprachen beherrscht, wenn auch nach 35 Jahren harter Arbeit nicht einmal die besten Programmierer auf dem Gebiet der künstlichen Intelligenz es schaffen, die schnellsten Supercomputer dazu zu bringen", Harry rang nach Atem, "aber \emph{was geht hier vor sich}?"

"Magie" sagte Professor McGonagall. Sie zuckte die Schultern.

"Das ist doch nur ein \emph{Wort}! Selbst nachdem ich es gehört habe, kann ich keine neuen Vorhersagen treffen! Es ist, als würde man 'Phlogiston' oder 'Elan Vital' oder 'Emergenz' oder 'Komplexität' dazu sagen!"

Professor McGonagall lachte laut. "Aber es \emph{ist} Magie, Mr~Potter."

Harry ließ die Schultern sinken. "Mit Verlaub, Professor McGonagall, ich bin mir nicht sicher, ob Sie verstehen, was ich hier zu erreichen versuche."

"Mit Verlaub, Mr~Potter, ich bin mir absolut sicher, dass dem so ist. Außer --- aber das ist nur eine Vermutung --- Sie versuchen, die Weltherrschaft an sich zu reißen."

"Nein! Ich meine, ja --- also, \emph{nein!}

"Ich denke, es sollte mich beunruhigen, dass Sie Probleme haben, diese Frage zu beantworten."

Bedrückt dachte Harry an die Dartmouth-Konferenz für Künstliche Intelligenz im Jahr 1956. Es war die erste Konferenz überhaupt zu diesem Thema; jene, wo der Begriff 'künstliche Intelligenz' geprägt wurde. Man hatte wichtige Fragestellungen aufgeworfen: Wie bringt man Computern bei, Sprache zu verstehen, zu lernen und sich selbst zu verbessern? Man hatte vollen Ernstes angenommen, dass bedeutsame Fortschritte in diesen Feldern möglich wären, wenn zehn Wissenschaftler zwei Monate lang zusammen arbeiten würden.

\emph{Nein, Kopf hoch. Du} fängst gerade erst an, \emph{die Geheimnisse der Magie zu entwirren. Du} weißt \emph{nicht, ob es tatsächlich zu schwierig ist, um es in zwei Monaten zu erledigen.}

"Und Sie haben \emph{wirklich} noch nie davon gehört, dass andere Zauberer solche Fragen stellen oder solche Experimente durchführen?" Harry fragte nochmal. Es erschien ihm so \emph{offensichtlich}.

Allerdings hatte es auch \emph{nach} der Entwicklung der Grundprinzipien der wissenschaftlichen Methodik fast 200 Jahre gedauert, bis Muggelwissenschaftler systematisch erforscht hatten, was ein \emph{Vierjähriger} verstehen konnte und was nicht. Man hätte es im achtzehnten Jahrhundert herausfinden können, aber bis zum zwanzigsten Jahrhundert hatte niemand daran gedacht. Man konnte es der sehr viel kleineren Zaubererwelt also nicht vorwerfen, den Rückholzauber nicht untersucht zu haben.

McGonagall schürzte die Lippen für einen Moment und zuckte schließlich mit den Schultern. "Es ist mir noch nicht ganz klar, was Sie mit 'wissenschaftlichen Experimenten' meinen, Mr~Potter. Wie gesagt, ich habe gesehen, wie muggelstämmige Schüler versuchen, Muggelwissenschaften in Hogwarts anzuwenden; außerdem werden Jahr für Jahr neue Zaubersprüche und Tränke erfunden."

Harry schüttelte den Kopf. "Technologie ist nicht das Gleiche wie Wissenschaft. Ganz und gar nicht. Und viel auszuprobieren ist nicht das Gleiche wie zu experimentieren, um die Regeln herauszufinden." Es gab Unmengen von Leuten, die versucht hatten, Flugmaschinen zu bauen, indem sie Dinge mit Flügeln ausprobierten, aber nur die Gebrüder Wright hatten dafür einen Windkanal gebaut… "Ähm, wie viele Muggelkinder kommen pro Jahr nach Hogwarts?"

McGonagall sah einen Moment nachdenklich aus. "Um die zehn oder so."

Harry vergaß einen Schritt und stolperte beinahe über seine eigenen Füße. "\emph{Zehn?}"

Die Muggelwelt hatte sechs Milliarden Einwohner --- Tendenz steigend. Wenn du einer in einer Million wärst, gäbe es zwölf von Deiner Sorte in New York, und 1000 weitere in China. Es war unvermeidbar, dass die Muggelwelt \emph{einige} Elfjährige hervorgebracht hatte, die Differentialrechnung beherrschten --- Harry wusste, dass er nicht der einzige war. Er hatte andere \emph{Wunderkinder} bei Mathewettbewerben getroffen. Genauer gesagt hatten ihn einige Konkurrenten dort völlig demontiert, die wahrscheinlich tatsächlich \emph{den ganzen Tag} damit verbrachten, Matheaufgaben zu lösen, und die \emph{nie} ein Science Fiction-Buch gelesen hatten und die noch vor dem Ende der Pubertät ausbrennen würden und \emph{niemals irgendwas} erreichen würden, da sie nur \emph{bekannte} Techniken lernten, anstatt \emph{kreativ} zu denken. (Harry war ein ziemlich schlechter Verlierer.)

Aber… in der Zaubererwelt…

Zehn Muggelstämmige pro Jahr, die alle ihre Muggelausbildung im Alter von elf Jahren beendet hatten. Und McGonagall mochte zwar voreingenommen sein, aber sie hatte behauptet, dass Hogwarts die größte und beste Zauberschule der Welt war… doch selbst Hogwarts bildete nur bis zum Alter von siebzehn Jahren aus.

Professor McGonagall wusste zweifelsohne bis ins kleinste Detail darüber Bescheid, wie man sich in eine Katze verwandelt. Aber sie schien buchstäblich niemals etwas von wissenschaftlicher Methodik gehört zu haben. Für sie war es nur Muggelmagie und sie schien noch nicht einmal neugierig darauf zu sein, welche Geheimnisse sich dahinter verbargen, dass der Rückholzauber menschliche Sprache verstehen konnte.

Das ließ wirklich nur zwei Möglichkeiten übrig.

Möglichkeit eins: Magie war so unglaublich undurchsichtig, kompliziert und undurchdringlich, dass, obwohl Zauberer und Hexen ihr Bestes gegeben hatten, um sie zu verstehen, kein Fortschritt zu erreichen war und sie irgendwann aufgaben; in diesem Fall würde es auch Harry nicht besser ergehen.

\emph{Oder…}

Harry knackte entschlossen mit seinen Fingern, aber sie machten nur leise `Klick', statt bedeutungsvoll von den Wänden der Winkelgasse widerzuhallen.

Möglichkeit zwei: Er würde die Welt beherrschen.

Bei Gelegenheit. Zumindest nicht jetzt gleich.

Diese Art von Dingen brauchte manchmal etwas länger als zwei Monate. Die Muggelwissenschaft hatte es in der ersten Woche nach Galileo auch nicht gleich bis zum Mond geschafft.

Trotzdem konnte Harry das breite Lächeln nicht unterdrücken, welches seine Mundwinkel so weit streckte, dass sie zu schmerzen begannen.

Er hatte immer befürchtet, eines dieser Wunderkinder zu werden, die es nie zu irgendwas brachten und den Rest ihrer Tage damit verbrachten, anzugeben, wie cool sie mit zehn Jahren gewesen waren. Doch auch die meisten erwachsenen Genies brachten es zu nichts. Für jeden Einstein in der Geschichte der Menschheit gab es vermutlich tausend weitere, die ebenso intelligent waren. Aber sie hatten nie die eine Zutat gefunden, die man unbedingt braucht, um Großes zu erreichen: Sie waren nie auf ein bedeutendes Problem gestoßen.

Harry betrachtete die Wände der Winkelgasse, dachte an alle Läden und Waren, die Ladenbesitzern und Kunden, all das Land und all die Menschen des magischen Großbritanniens und der restlichen Zaubererwelt; und das ganze restliche Universum, von dem die Muggelwissenschaftler so viel weniger verstanden als sie dachten. \emph{All das ist jetzt meines. Ich, Harry James Potter-Evans-Verres, beanspruche nun dieses Territorium im Namen der Wissenschaft.}

Blitz und Donner misslang es angesichts des blauen Himmels vollkommen, die Szene mit angemessener Dramatik zu untermalen.

"Weshalb lächeln Sie?" erkundigte sich McGonagall vorsichtig und erschöpft.

"Ich frage mich, ob es einen Zauber gibt, der es im Hintergrund blitzen lässt, wann immer ich eine bedeutsame Feststellung mache", erklärte Harry. Er gab acht, sich die genaue Wortfolge seiner bedeutsamen Feststellung zu merken, so dass zukünftige Geschichtsbücher sie richtig wiedergeben würden.

"Ich habe das düstere Gefühl, dass ich etwas in dieser Sache unternehmen sollte", seufzte McGonagall.

"Ignorieren Sie es, dann geht es vorbei. Oh, da glitzert etwas!" Harry schob seine Weltherrschaftspläne vorerst beiseite und hüpfte auf ein nahegelegenes Schaufenster zu. Professor McGonagall folgte ihm.

\later

Harry hatte derweil Zaubertrankzutaten und einen Kessel gekauft und, nun ja, noch ein paar Sachen. Dinge, von denen er dachte, dass es sinnvoll wäre, sie in seiner großen Tasche (auch bekannt als: Super-Eselsfellbeutel QX31 mit Unaufspürbarem Ausdehnungszauber, Rückholzauber und dehnbarer Öffnung) mitzuführen. Kluge, praktische Anschaffungen.

Harry konnte wirklich nicht verstehen, warum McGonagall ihn so \emph{misstrauisch} beäugte.

Gerade nun stand Harry in einem Geschäft, dessen Auslagen die sich windende Winkelgasse verengten. Auf hölzernen Tischen waren verschiedenste Waren ausgebreitet, die nur von einem schwachen gräulichen Glimmen sowie einer recht jung aussehenden Verkäuferin in einem sehr kurzen Hexenumhang, der Knie und Ellbogen frei ließ, bewacht wurden.

Harry untersuchte das magische Äquivalent eines Erste-Hilfe-Kastens - den Notfall-Heilpack Plus. Da gab es zwei selbst-straffende Druckverbände. Einen Stabilisierungstrank, der Blutverlust verlangsamen und einen Schock verhindern würde. Eine Spritze enthielt etwas, das wie flüssiges Feuer aussah und dafür bestimmt war, die Blutzirkulation im behandelten Körperteil erheblich zu verlangsamen, während der Sauerstoffgehalt im Blut für bis zu drei Minuten gehalten wurde. So konnte man ein Gift daran hindern, sich im Körper auszubreiten. Weiße Tücher, die man über eine Verletzung legen konnte, um Schmerzen für eine Weile zu betäuben. Dazu noch eine Reihe anderer Gegenstände, die Harry überhaupt nicht verstand, wie das "Dementorenheilmittel", welches wie gewöhnliche Schokolade aussah und roch, oder das "Baffelschnaffler-Gegenmittel", das aussah wie ein kleines zitterndes Ei --- eine beigefügte Abbildung zeigte, wie man es jemandem in die Nase stopft.

"Nur fünf Galleonen --- das ist doch geradezu ein Schnäppchen, denken Sie nicht?", sagte Harry zu McGonagall, und die junge Verkäuferin nickte eifrig.

Harry hatte angesichts seiner Achtsamkeit und Vernunft von McGonagall eine zustimmende Bemerkung erwartet. Was er stattdessen bekam, ließ sich wohl nur als der \emph{Böse Blick} bezeichnen.

"Und warum \emph{genau}", sagte Professor McGonagall mit hörbarer Skepsis, "erwarten Sie, dass Sie ein Heiler-Set brauchen werden, junger Mann?" (Nach einem unglücklichen Vorfall im Tränkeladen vermied sie es, "Mr~Potter" zu sagen, wenn jemand in der Nähe war.)

Harrys Mund öffnete sich, und schloß sich gleich wieder. "Ich \emph{erwarte} nicht es zu brauchen. Es ist nur für den Fall!"

"Nur für den Fall, dass \emph{was} passiert?"

Harrys Augen weiteten sich. "Sie denken, dass ich etwas Gefährliches vorhabe und \emph{deswegen} diese Heilmittel kaufen will?"

Der Ausdruck düsteren Verdachts und ironischen Unglaubens auf McGonagalls Gesicht war Antwort genug.

"Großer Scott!" sagte Harry. (Er hatte diesen Ausruf vom verrückten Wissenschaftler Doc Brown aus \emph{Zurück in die Zukunft} übernommen.) "Dachten Sie das auch, als ich den Fall-wie-eine-Feder-Trank, das Dianthuskraut und das Fläschchen mit Nahrungs- und Wasserpillen gekauft habe?"

"Ja."

Harry schüttelte erstaunt seinen Kopf. "Sagen Sie, was \emph{glauben} Sie denn, was ich vorhabe?"

"Ich weiß es nicht", sagte McGonagall düster, "aber es endet wohl damit, dass Sie entweder eine Tonne Silber an Gringotts liefern, oder die Weltherrschaft übernehmen."

"Weltherrschaft ist so ein unschöner Ausdruck. Ich ziehe es vor, von Weltoptimierung zu sprechen."

Dies reichte jedoch nicht aus, um Professor McGonagall zu beruhigen, die ihm immer noch einen unheilvollen Blick zuwarf.

"Wow", sagte Harry, als ihm klar wurde, dass sie es ernst meinte. "Sie glauben das wirklich. Sie glauben wirklich, dass ich etwas Gefährliches vorhabe."

"Ja."

"Als ob das der einzige Grund wäre, warum jemand jemals ein Heiler-Set kaufen sollte. Verstehen Sie das bitte nicht falsch, Professor McGonagall, aber \emph{mit was für verrückten Kindern haben Sie es sonst für gewöhnlich zu tun}?"

"Gryffindors" --- McGonagall spuckte dieses eine Wort mit einer Bitterkeit und Verzweiflung aus, die sich wie ein ewiger Fluch auf allen jugendlichen Leichtsinn und Hochmut anhörten.

"Meine sehr verehrte Stellvertretende Schulleiterin Professor Minerva McGonagall", begann Harry und stemmte seine Hände entschlossen in die Hüften. "Ich werde nicht in Gryffindor landen ---"

In diesem Moment warf McGonagall ein, dass sie anderenfalls herausfinden würde, wie man einen Hut tötet. Eine seltsame Bemerkung, die Harry ebenso überging wie den plötzlichen Hustenanfall der Verkäuferin.

"--- sondern in Ravenclaw. Und wenn Sie wirklich denken, dass ich etwas Gefährliches plane, dann verstehen Sie mich, mit Verlaub, \emph{vollkommen falsch}. Ich \emph{mag} keine Gefahren, Gefahren sind \emph{beängstigend}. Ich bin \emph{vorsichtig}. Ich bin \emph{achtsam}. Ich plane \emph{unvorhersehbare Zwischenfälle} mit ein, genau wie meine Eltern es mir früher immer vorgesungen haben: `Sei bereit --- so geht das Pfadfinderlied. Sei bereit --- was das Leben dir auch gibt. Bei Angst und Schrecken hilft kein Selbstmitleid --- Drum sei immer bereit, sei bereit!`"

(Die anderen Textzeilen hatten seine Eltern ihm nie vorgesungen, so dass er glücklicherweise keine Ahnung vom wahren Inhalt des Liedes hatte.)

McGonagalls Haltung hatte sich etwas entspannt --- jedoch hauptsächlich in dem Moment, als Harry sie erinnerte, dass er nach Ravenclaw kommen würde. "Für \emph{was für einen Notfall} soll dieses Heiler-Set Sie wappnen, junger Mann?"

"Eine Klassenkameradin könnte von einem schrecklichen Monster gebissen werden, und während ich verzweifelt in meinem Beutel nach etwas suche, das ihr hilft, sieht sie mich traurig an und sagt mit ihrem letzten Atemzug 'Warum warst du nicht vorbereitet?' Und dann stirbt sie und während sie die Augen schließt, weiß ich, dass sie mir niemals vergeben wird ---"

Harry hörte die Verkäuferin schluchzen und als er aufsah, starrte sie ihn mit eng zusammengepressten Lippen an. Dann drehte die junge Frau sich um und flüchtete in den hinteren Teil des Ladens.

\emph{Was…?}

Professor McGonagall griff sanft aber bestimmt nach seiner Hand und zog Harry beiseite in eine kleine Einbuchtung zwischen zwei Läden, die mit schmutzigen Steinen gepflastert war und in einer dreckigen, schwarzen Wand endete.

Die große Hexe deutete mit ihrem Zauberstab auf die vorbeiführende Winkelgasse. "Quietus", sagte sie, und ein Mantel der Stille senkte sich über sie beide herab und blendete den Straßenlärm vollkommen aus.

\emph{Was habe ich bloß falsch gemacht…}

Dann drehte sich die Hexe zu Harry und sagte mit einem mächtigen, eiskalten Blick: "Ich würde es begrüßen, wenn Sie bedächten, dass es im magischen Großbritannien vor gerade einmal zehn Jahren einen \emph{Krieg} gab und dass \emph{jeder} hier jemanden verloren hat und dass es Gerede darüber, wie Freunde in den eigenen Armen sterben, \emph{einfach nicht zu geben hat}."

"Ich - ich wollte nicht ---" Plötzlich fiel es Harry wie Schuppen von den Augen. Der Krieg hatte vor zehn Jahren geendet, also war dieses Mädchen etwa acht oder neun Jahre alt gewesen, als --- als…

"Es tut mir leid, ich wollte nicht…" Harry schluckte und drehte sich weg, um McGonagalls kaltem Blick zu entkommen, sah sich jedoch einer Wand aus Lehm gegenüber, die ihm, ohne Zauberstab, jeden Ausweg blockierte. "Es tut mir leid, es tut mir leid, es tut mir so Leid."

Hinter sich hörte er einen schweren Seufzer. "Ich weiß, Mr~Potter. Ich weiß."

Harry wagte einen Blick. Der Ärger war aus Professor McGonagalls Gesicht verschwunden. "Es tut mir leid", sagte Harry nochmals und fühlte sich absolut elend. "Ich hätte das nicht sagen sollen. Ist Ihnen ---" und dann verstummte Harry und hielt sich obendrein die Hand vor den Mund. McGonagalls Gesicht wurde etwas trauriger. "Sie müssen lernen nachzudenken, bevor Sie sprechen, Mr~Potter. Anderenfalls werden Sie mit sehr wenigen Freunden durchs Leben gehen. Dies war das Schicksal von vielen Ravenclaws und ich hoffe, es wird nicht das Ihre."

Harry wollte wegrennen. Er wollte einen Zauberstab ziehen und die ganze Sache aus McGonagalls Gedächtnis löschen; wieder mit ihr vor dem Geschäft stehen; es ungeschehen machen ---

"Aber um Ihre Frage zu beantworten", sagte McGonagall, "nein, etwas Derartiges ist mir nicht passiert." Ihr Gesicht verzog sich. "Ich habe ein oder zweimal, vielleicht auch öfter, gesehen, wie ein Freund seinen letzten Atemzug nahm, aber niemand hat mich jemals im Sterben verflucht und ich dachte nie, dass derjenige mir nicht vergeben würde. \emph{Was in Merlins Namen hat Sie bewegt, so etwas zu sagen, Harry Potter?} Warum würden Sie an so etwas überhaupt \emph{denken}?"

Tränen begannen Harrys Wangen hinabzurollen. "Es tut mir leid, ich hätte nichts zu Ihnen sagen sollen. Es tut mir leid ---"

McGonagall holte tief Luft. "Ich weiß, dass es Ihnen Leid tut. Was ich nicht verstehe ist, wieso ein elfjähriger Junge an solche Sachen denkt. Wollen Sie tatsächlich dieses fünf Galleonen teure Heiler-Set kaufen und von nun an ständig in einem fünfzehn Galleonen teuren Beutel mit sich herumtragen --- und das aus der Überzeugung heraus, dass Ihre Klassenkameraden Sie anderenfalls \emph{mit dem letzten Atemzug verfluchen würden}?"

"Ich --- ich… ich…" Harry schluckte. "Es ist nur so, dass ich immer versuche, mir das Schlimmste, was möglicherweise passieren könnte, vorzustellen" --- und vielleicht hatte er auch etwas rumgewitzelt, aber er würde sich eher die Zunge abbeißen, als das jetzt zuzugeben.

"\emph{Warum}?"

"Damit ich es verhindern kann!"

"Mr~Potter…" McGonagalls Stimme verlor sich. Dann seufzte sie und beugte sich zu ihm herab. "Mr~Potter", sagte sie nun sanft, "Sie sind nicht für das Wohl der anderen Hogwartsschüler verantwortlich. Das liegt in meiner Verantwortung. Ich werde nicht zulassen, dass Ihnen oder irgendeinem anderen Schüler etwas zustößt. Hogwarts ist der sicherste Ort im ganzen magischen Großbritannien und Madam Pomfrey hat ein voll ausgerüstetes Heilerzimmer. Sie brauchen kein Heiler-Set."

"Brauche ich doch!", platzte es aus Harry heraus. "Nirgends ist es absolut sicher! Was ist, wenn meine Eltern einen Schlaganfall haben oder in einen Unfall geraten, während ich in den Weihnachtsferien daheim bin --- Madam Pomfrey wird dann nicht da sein, ich brauche mein eigenes Heiler-Set ---"

"Was in Merlins Namen…", sagte McGonagall. Sie stand auf und sah mit einem Ausdruck zwischen Besorgnis und Entnervtheit auf Harry. "Es gibt keinen Grund, derart schreckliche Dinge zu erwarten, Mr~Potter."

Harrys Tonfall wurde bitter, als er das hörte. "Doch, es gibt einen Grund. Wenn man nicht nachdenkt, schadet man nicht nur sich selbst, sondern man verletzt auch andere Menschen!"

Professor McGonagall öffnete ihren Mund und schloss ihn gleich wieder. Sie rieb sich das Nasenbein und schaute nachdenklich. "Mr~Potter… wenn ich anbiete, eine Weile still zu sein und Ihnen einfach nur zuzuhören… gibt es da etwas, worüber Sie mit mir sprechen wollen?"

"Worüber?"

"Darüber, warum Sie überzeugt sind, dass Sie immer auf der Hut vor schrecklichen Dingen sein müssen, die Ihnen widerfahren könnten."

Harry schaute sie verwundert an. Das war ein offensichtliches Axiom. "Also…", begann Harry langsam. Er versuchte seine Gedanken zu ordnen. Wie konnte er es McGonagall erklären, wo ihr doch jegliche Grundlagen fehlten? "Muggelwissenschaftler haben herausgefunden, dass Menschen immer sehr optimistisch sind. Sie sagen zum Beispiel, dass etwas zwei Tage dauern wird, aber dann werden es zehn; oder sie sagen, es dauert zwei Monate und dann dauert es fünfunddreißig Jahre. Die Wissenschaftler haben einmal Studenten gefragt, nach welcher Zeit sie ihre Hausaufgaben mit einer Wahrscheinlichkeit von 50\%, 75\% oder 99\% erledigt haben würden, aber nur 13\%, 19\% und 45\% der Studenten wurden tatsächlich innerhalb der angegebenen Zeit fertig. Es stellte sich auch heraus, dass bei den Erwartungen der Studenten statistisch kein Unterschied zwischen Bestfall und Normalfall bestand. Sehen Sie, wenn Sie jemanden fragen, was diese Person als Normalfall annimmt, dann überlegt die Person, welches bei jeder einzelnen Entscheidung der wahrscheinlichste Ausgang ist --- nämlich, dass alles nach Plan verläuft, ohne irgendwelche Probleme oder Überraschungen. Tatsächlich sieht es --- da mehr als die Hälfte der Studenten nicht in der Zeit fertig wurden, bei der sie sich zu 99\% sicher waren --- aber so aus, dass die Wirklichkeit üblicherweise etwas schlechter abläuft als das 'Worst-Case-Szenario'. Man nennt das den Planungstrugschluss und man umgeht ihn am besten, indem man nachdenkt, wie lange etwas beim letzten Mal gedauert hat. Man sagt dazu, dass man \emph{die Außenperspektive benutzt} anstelle der \emph{Innenperspektive}. Aber wenn man etwas zum ersten Mal tut, geht das nicht, also muss man wirklich, wirklich, wirklich pessimistisch sein. So pessimistisch, dass die Realität genauso oft besser ist als erwartet, wie schlechter als erwartet. Es ist außerordentlich schwierig, so pessimistisch zu sein, dass man eine gute Chance hat, das wahre Leben zu übertreffen. Beispielsweise könnte es passieren, dass ich hier einen großen Aufwand betreibe und mir vorstelle, meine Klassenkameraden werden von irgendetwas gebissen, während in Wirklichkeit stattdessen die verbliebenen Todesser die ganze Schule angreifen, um mich zu entführen. Aber positiv betrachtet ---"

"Stopp", sagte McGonagall.

Harry brach mitten im Satz ab. Er hatte gerade darauf hinweisen wollen, dass sie zumindest wussten, dass der Dunkle Lord nicht angreifen würde, da er tot war.

"Ich denke, ich habe mich nicht klar ausgedrückt", sagte McGonagall sorgsam. "Ist Ihnen persönlich etwas passiert, das Ihnen Angst bereitet?"

"Was mir selbst passiert ist, hat lediglich anekdotischen Wert", erklärte Harry ihr. "Es hat nicht dasselbe Gewicht wie ein Paper über eine replizierbare Studie mit zufälliger Zuordnung, vielen Teilnehmern, deutlich ausgeprägten Effekten und hoher statistischer Signifikanz, welches nach bestandenem Peer Review in einem anerkannten Journal veröffentlicht wurde."

McGonagall kniff sich ins Nasenbein, atmete ein und atmete wieder aus. "Ich würde es trotzdem gerne hören", sagte sie.

"Ähm…", sagte Harry. Er nahm einen tiefen Atemzug. "In unserer Nachbarschaft hatten sich ein paar Überfälle ereignet und meine Mutter bat mich, eine Pfanne zurückzubringen, die sie sich von Nachbarn geliehen hatte, welche zwei Straßen weiter wohnten. Ich sagte ihr, ich möchte das nicht tun, weil ich möglicherweise überfallen werde, und sie sagte 'Harry, sag sowas nicht!' --- als ob es passieren würde, weil ich daran dachte, aber ich sicher sein würde, solange ich nicht darüber spräche. Ich versuchte, es ihr zu erklären und sie ließ mich trotzdem die Pfanne rübertragen. Ich war zu jung um zu wissen, wie statistisch unwahrscheinlich es für einen Räuber war, mich anzugreifen, aber ich war alt genug um zu wissen, dass Nicht-Daran-Denken die Dinge nicht davon abhält zu passieren, also hatte ich wirklich Angst."

"Sonst nichts?", sagte McGonagall nach einer Pause, als klar wurde, dass Harry fertig war. "Ist Ihnen nicht noch etwas anderes passiert?"

"Ich weiß, es hört sich nicht nach viel an", verteidigte Harry sich. "Aber es war einer dieser entscheidenden Momente im Leben, wissen Sie? Ich meine, ich \emph{wusste}, dass nicht an etwas zu denken es nicht verhindert; ich \emph{wusste} das, aber ich konnte erkennen, dass meine Mutter das trotzdem dachte." Harry brach ab und rang mit dem Ärger, der sich in ihm aufbaute, wenn er daran dachte. "Sie wollte nicht zuhören. Ich versuchte, es ihr begreiflich zu machen, ich \emph{bat} sie, mich nicht hinauszuschicken, doch sie hat darüber gelacht. Alles, was ich sagte, war ein einziger großer Witz für sie…" Harry zwang den düsteren Zorn wieder beiseite. "In dem Moment wurde mir klar, dass jeder, der mich beschützen sollte, tatsächlich verrückt war, und dass sie nicht zuhören würden, egal wie sehr ich sie anflehte, und dass ich mich niemals darauf verlassen kann, dass sie das Richtige tun."

Manchmal waren gute Absichten nicht genug, manchmal musste man klar denken können…

Es folgte eine lange Stille.

Harry nutzte die Zeit, um tief durchzuatmen und sich zu beruhigen. Es war nutzlos, wütend zu werden. Es war \emph{nutzlos}, wütend zu werden. \emph{Alle} Eltern waren so; \emph{kein} Erwachsener würde sich dazu herablassen, auf gleicher Ebene mit einem Kind zu sprechen; seine biologischen Eltern würden nicht anders sein. Vernunft war ein kleines Funkeln in der Dunkelheit, eine unendlich seltene Ausnahme von der allumfassenden Herrschaft des Verrückten, also war es nutzlos, wütend zu werden.

Harry mochte sich selbst nicht, wenn er wütend war.

"Danke, dass Sie das mit mir geteilt haben, Mr~Potter", sagte McGonagall nach einer Weile. Ein abwesender Blick lag auf ihrem Gesicht. (Es war fast derselbe Blick, der beim Experimentieren mit dem Beutel auf Harrys eigenem Gesicht erschienen war --- jedoch hatte Harry es damals ohne einen Spiegel nicht bemerkt). "Ich werde darüber nachdenken müssen." Sie drehte sich zur Straße und hob den Zauberstab ---

"Ähm", sagte Harry, "können wir nun das Heiler-Set kaufen?"

McGonagall pausierte und sah ihn mit festem Blick an. "Und wenn ich sage, 'Nein, es ist zu teuer, und Sie werden es nicht brauchen', was passiert dann?"

Harrys Gesicht verzerrte sich verbittert. "Genau das, was Sie denken, Professor McGonagall. Genau das, was sie denken. Ich werde schlussfolgern, dass Sie einfach nur ein weiterer verrückter Erwachsener sind, mit dem ich nicht reden kann, und ich werde anfangen zu planen, wie ich dennoch ein Heiler-Set bekomme."

"Ich bin während dieses Ausflugs Ihre Aufsichtsperson", sagte McGonagall mit einem Anflug von Ärger. "Ich werde \emph{nicht} zulassen, dass Sie mich herumschubsen."

"Ich verstehe", sagte Harry. Er hielt die Verbitterung aus seiner Stimme heraus und verschwieg all die anderen Dinge, die ihm durch den Kopf gingen. McGonagall hatte ihm geraten zu denken, bevor er sprach. Er würde sich vermutlich morgen nicht mehr daran erinnern, aber für fünf Minuten konnte er es durchaus im Kopf behalten.

McGonagalls Zauberstab zuckte und die Geräusche der Winkelgasse kehrten zurück. "In Ordnung, junger Mann", sagte sie. "Lassen Sie uns dieses Heiler-Set holen."

Harrys Kinn klappte vor Überraschung herunter. Dann hastete er ihr, in seiner Eile fast stolpernd, nach.

\later

Der Laden sah immer noch so aus, wie sie ihn zurückgelassen hatten. Erkennbare und nicht erkennbare Gegenstände lagen auf den hölzernen Tischen aus, weiterhin bewacht vom grauen Glühen, und auch die Verkäuferin stand wieder an ihrem alten Platz. Das Mädchen schaute auf, als sie hereinkamen und ihr Gesicht zeigte Überraschung.

"Es tut mir leid", sagte sie, als sie näher kamen, und Harry sagte beinahe im selben Moment, "Ich entschuldige mich für ---"

Sie brachen ab, sahen sich an und die Verkäuferin lachte ein bisschen. "Ich wollte Dir keinen Ärger mit Professor McGonagall einbringen", sagte sie. Ihre Stimme senkte sich verschwörerisch. "Ich hoffe, sie war nicht zu schlimm zu Dir."

"Della!", sagte McGonagall empört.

"Sack voll Gold", sagte Harry zu seinem Beutel und schaute dann wieder zu der Verkäuferin, während er fünf Galleonen abzählte. "Mach Dir keine Sorgen, ich verstehe, dass sie nur deswegen so streng mit mir umgeht, weil sie mich mag."

Er gab dem Mädchen die Galleonen, während McGonagall etwas Unverständliches stammelte.

"Einen Notfall-Heilpack Plus bitte."

Es war, ehrlich gesagt, etwas beunruhigend anzusehen, wie die dehnbare Öffnung ein Heiler-Set von der Größe eines Aktenkoffers verschluckte. Harry konnte sich der Frage nicht erwehren, was wohl passieren würde, wenn er selbst versuchte, in den Beutel zu steigen --- insbesondere, da angeblich nur derjenige, der etwas hineintat, es auch wieder hervorholen konnte.

Als der Beutel seinen schwer erkämpften Einkauf … \emph{verschluckt} … hatte, hätte Harry schwören können, dass er anschließend einen Rülpser hörte. Das \emph{musste} jemand wohl absichtlich gezaubert haben --- jede andere Hypothese war zu entsetzlich, um sie sich vorzustellen… Genau genommen fiel Harry \emph{gar keine andere Hypothese} ein. Er sah zu McGonagall. "Wohin als nächstes?"

McGonagall zeigte auf einen Laden, der aussah, als ob er aus Fleisch anstelle von Steinen bestand, und mit Fell anstelle von Farbe bedeckt war. "Kleine Haustiere sind in Hogwarts erlaubt - Sie könnten zum Beispiel eine Eule kaufen, um Briefe zu senden ---"

"Kann ich für einen Knut oder sowas eine Eule mieten, wenn ich einen Brief senden möchte?"

"Ja", sagte McGonagall.

"Dann ganz sicher nicht."

McGonagall nickte, als ob sie im Kopf eine Notiz machte. "Darf ich fragen, warum nicht?"

"Ich hatte mal einen Stein als Haustier. Er starb."

"Sie denken nicht, dass Sie auf ein Haustier Acht geben könnten?"

"Ich \emph{könnte}", sagte Harry, "aber ich würde mir den ganzen Tag darüber Gedanken machen, ob ich daran gedacht habe, es zu füttern oder ob es langsam in seinem Käfig verhungert und sich fragt, wo sein Herrchen ist und warum kein Futter da ist."

"Die arme Eule", sagte McGonagall in einer sanften Stimme. "So allein gelassen. Ich frage mich, was sie wohl machen würde."

"Nun, sie würde richtig hungrig werden und versuchen, mit ihren Klauen den Käfig zu öffnen, obwohl das nicht sehr erfolgreich enden würde ---" Harry brach ab.

McGonagall setzte nach, immer noch in dieser sanften Stimme. "Und was würde danach mit ihr passieren?"

"Entschuldigen Sie", sagte Harry. Er nahm McGonagall sanft aber bestimmt an der Hand und zog sie in eine weitere Seitengasse; nachdem sie so vielen wohlmeinenden Passanten ausgewichen waren, war der Vorgang schon zur Routine geworden. "Bitte zaubern Sie diesen Quietus-Spruch."

"Quietus."

Harrys Stimme zitterte. "Diese Eule repräsentiert \emph{nicht} mich! Meine Eltern haben mich \emph{nie} in einen Schrank eingesperrt und hungern lassen, ich habe \emph{keine} Verlassensängste und \emph{ich mag die Richtung ihrer Gedanken ganz und gar nicht}, Professor McGonagall!"

Die Hexe sah auf ihn herab. "Und welche Gedanken wären das, Mr~Potter?"

"Sie denken, ich wurde", Harry hatte Schwierigkeiten es zu sagen, "ich wurde \emph{missbraucht}?"

"Wurden Sie?"

"\emph{Nein!}", schrie Harry, "Nein, das wurde ich nicht. Denken Sie, ich bin \emph{dumm}? Ich weiß über Kindesmissbrauch Bescheid, ich weiß, was unangemessene Berührungen sind und all das, und wenn irgendetwas passiert wäre, hätte ich die Polizei gerufen! Und es dem Schulleiter gemeldet! Und staatliche Stellen im Telefonbuch nachgeschlagen! Und es Oma und Opa erzählt und Mrs~Figg! Aber meine Eltern haben so etwas \emph{nie} gemacht. Nie, nicht, \emph{niemals}! Was \emph{erlauben} Sie sich, so etwas auch nur anzudeuten?"

McGonagall sah ihn ernst an. "Als Stellvertretende Schulleiterin ist es meine Pflicht, alle möglichen Anzeichen von Missbrauch bei Kindern zu verfolgen, die unter meiner Fürsorge stehen."

Harrys Ärger geriet außer Kontrolle und wurde zu einem Wirbel aus purer, schwarzer Wut. "\emph{Wagen} Sie es ja nicht, noch ein Wort von diesen, diesen \emph{Anschuldigungen} gegenüber irgendjemandem zu äußern. \emph{Niemandem}, verstehen Sie mich, McGonagall? Solche Anschuldigungen können Menschen und ganze Familien ruinieren, selbst wenn die Eltern vollkommen unschuldig sind! Ich hab darüber in der Zeitung gelesen!" Harrys Stimme steigerte sich zu einem schrillen Kreischen. "Das System kann nicht abbremsen; es glaubt weder Eltern noch Kindern, wenn sie sagen, dass nichts passiert ist! \emph{Wagen Sie es ja nicht, meine Familie damit zu bedrohen! Ich werde nicht zulassen, dass Sie meine Familie zerstören!}"

"Harry" sagte McGonagall sanft und streckte ihre Hand zu ihm aus ---

Harry machte einen Schritt zurück, seine Hand schnellte nach oben und schlug ihre weg. McGonagall erstarrte, dann zog sie ihre Hand zurück und machte einen Schritt rückwärts. "Harry, es ist in Ordnung, ich glaube Ihnen."

"\emph{Tun Sie das?}", zischte Harry. Die Wut kochte immer noch in seinem Adern. "Oder warten Sie nur darauf, dass Sie von mir wegkommen, um die Anträge auszufüllen?"

"Harry, ich habe Ihr Zuhause gesehen. Ich habe Ihre Eltern gesehen. Ihre Eltern lieben Sie, und Sie lieben sie auch. Ich glaube Ihnen, wenn Sie sagen, dass Ihre Eltern Sie nicht missbrauchen. Aber ich musste nachfragen, denn hier geht etwas sehr, sehr Merkwürdiges vor."

Harry starrte sie kühl an. "Was denn?"

McGonagall atmete tief ein. "Harry, ich habe in meiner Zeit in Hogwarts viele missbrauchte Kinder gesehen; es würde Ihnen das Herz brechen zu wissen, wie viele es waren. Und wenn Sie glücklich sind, dann benehmen Sie sich nicht wie eines von diesen Kindern. Ganz und gar nicht. Sie lächeln Fremde an, umarmen Leute, ich lege meine Hand auf Ihre Schulter und Sie zucken nicht zusammen. Aber manchmal, nur manchmal, sagen Sie etwas, das \emph{sehr} stark wirkt wie… wie jemand, der die ersten elf Jahre seines Lebens in einem Keller verbracht hat. Nicht wie die liebende Familie, die ich gesehen habe." McGonagall neigte ihren Kopf, ihr Gesichtsausdruck wirkte unsicher.

Harry nahm das auf, verarbeitete es. Die Wut begann zu verfließen, als ihm auffiel, dass ihm respektvoll zugehört wurde und dass seine Familie nicht in Gefahr war.

"Und wie erklären Sie sich ihre Beobachtungen, Professor McGonagall?"

"Ich weiß nicht", sagte sie. "Aber es ist denkbar, dass Ihnen etwas widerfahren ist, an dass Sie sich nicht erinnern."

Die Wut stieg wieder in Harry auf. Das hörte sich viel zu sehr nach etwas an, das er in den Artikeln über zerstörte Familien gelesen hatte. "Unterdrückte Erinnerungen sind ein Haufen \emph{Pseudowissenschaft}! Menschen unterdrücken traumatische Erinnerungen nicht, sie erinnern sich nur zu gut daran, für den Rest ihres Lebens!"

"Nein, Mr~Potter. Es gibt einen Vergessenszauber."

Harry erstarrte augenblicklich. "Ein Spruch, der Erinnerungen löscht?"

McGonagall nickte. "Aber nicht alle Auswirkungen der Erinnerung, wenn Sie verstehen, was ich meine, Mr~Potter."

Ein kalter Schauer lief Harrys Rücken hinunter. \emph{Diese} Hypothese ließ sich nicht so einfach zurückweisen. "Aber meine Eltern könnten das nicht tun!"

"Nein", sagte McGonagall. "Das könnte nur jemand aus der Zaubererwelt. Ich fürchte… es gibt keine Möglichkeit, das zu testen."

Harrys rationalistische Fähigkeiten sprangen langsam wieder an. "Professor McGonagall, wie sicher sind Sie sich Ihrer Beobachtungen und welche alternativen Erklärungen könnte es geben?"

McGonagall öffnete ihre Hände, als ob sie zeigen wollte, dass diese leer waren. "Sicher? Ich bin mir \emph{kein bisschen} sicher, Mr~Potter. Wenn ich Ihre gesamte Persönlichkeit betrachte, dann habe ich gewiss noch nie in meinem ganzen Leben so jemanden getroffen. Manchmal wirken Sie auf mich nicht wie ein Elfjähriger; nicht einmal wie ein \emph{Mensch}."

Harrys Augenbrauen hoben sich bis in den Himmel ---

"Es tut mir leid!", sagte McGonagall schnell. "Es tut mir sehr leid, Mr~Potter. Ich versuchte lediglich, einen rhetorischen Punkt zu machen, aber es kam, fürchte ich, etwas anders heraus als ich beabsichtigt hatte."

"Im Gegenteil, Professor McGonagall", sagte Harry und lächelte langsam. "Ich verstehe das als ein sehr großes Kompliment. Aber würde es Sie stören, wenn ich eine alternative Erklärung anböte?"

"Bitte."

"Es ist nicht üblich, dass Kinder deutlich klüger sind als ihre Eltern", sagte Harry. "Oder zumindest deutlich vernünftiger --- mein Vater könnte mich an die Wand denken, wenn er es, Sie wissen schon, tatsächlich \emph{versuchen} würde, statt seine Intelligenz hauptsächlich dafür zu verwenden, neue Gründe zu finden, um seine Meinung nicht zu ändern ---" Harry brach ab. "Ich bin zu schlau, Professor McGonagall. Normale Kinder spielen einfach nicht in meiner Liga. Erwachsene respektieren mich nicht genug, um ernsthaft mit mir zu sprechen. Und ehrlich gesagt, selbst wenn sie es täten, dann klängen sie nicht so schlau wie Richard Feynman, also kann ich stattdessen auch Bücher von Richard Feynman lesen. Ich bin \emph{isoliert}, Professor McGonagall. Ich war mein ganzes Leben lang isoliert. Vielleicht hat das ähnliche Auswirkungen wie in einem Keller eingesperrt zu sein. Und ich bin zu intelligent, um zu meinen Eltern so aufzusehen, wie es für Kinder üblich ist. Meine Eltern lieben mich, aber sie fühlen sich nicht verpflichtet, auf die Vernunft zu hören. Manchmal fühle ich mich, als wären sie Kinder --- Kinder, die nicht zuhören, aber absolute Authorität über meine Existenz haben. Ich versuche, nicht verbittert zu sein deswegen, aber ich versuche auch, ehrlich zu mir selbst zu sein --- also ja, ich bin verbittert. Und es fällt mir schwer, meinen Ärger zu beherrschen, aber daran arbeite ich. Das ist alles."

"\emph{Das ist alles}?"

Harry nickte bestimmt. "Das ist alles. Sicherlich, Professor McGonagall, lohnt es sich selbst im magischen Großbritannien, die einfachste Erklärung wenigstens zu erwägen?"

\later

Der Tag schritt voran, die Sonne sank am Sommerhimmel herab und die Menschen begannen aus den Straßen zu verschwinden. Einige Läden hatten bereits geschlossen; Harry und McGonagall hatten seine Bücher von Flourish und Blotts knapp vor dem Ladenschluss gekauft. Es gab nur einen kleinen Zwischenfall, als Harry schnurstracks zu den Arithmantik-Büchern geeilt war und feststellte, dass die Bücher für's siebte Schuljahr nichts komplizierteres als Trigonometrie enthielten.

Doch in diesem Moment dachte Harry nicht einmal an leicht erreichbare Forschungsziele.

In diesem Moment verließen Harry und McGonagall Ollivanders Geschäft und Harry starrte auf seinen Zauberstab. Er wedelte damit umher und versprühte bunte Funken, was nach all den anderen Sachen, die er gesehen hatte, wirklich keine Überraschung sein sollte; aber dennoch ---

\emph{Ich kann zaubern.}

\emph{Ich. Also, ich persönlich. Ich bin magisch; ich bin ein Zauberer.}

Er hatte die Magie seinen Arm entlang strömen gefühlt und in diesem Moment erkannt, dass er dieses Gespür schon immer gehabt hatte, dass er es schon sein ganzes Leben lang besaß; diesen sechsten Sinn, der weder Sehen noch Hören noch Riechen noch Schmecken oder Fühlen war, sondern nur Magie. Es war, als ob man Augen hatte, aber sie immer geschlossen hielt und folglich nicht einmal bemerkte, dass man Dunkelheit sieht --- und dann, eines Tages, öffnen sich die Augen und sehen die Welt. Der Schock hatte sich in ihm ausgebreitet und Teile von ihm berührt, sie erweckt und war gleich wieder verschwunden. Er hinterließ nur das Wissen, dass Harry nun ein Zauberer war und es schon immer gewesen war und es in gewisser Weise immer schon gewusst hatte.

Und-

\emph{"Es ist in der Tat sehr eigenartig, dass Sie für diesen Stab vorbestimmt sind; wo doch sein Bruder Ihnen diese Narbe gab."}

Das konnte unmöglich ein Zufall sein. Es hatte \emph{tausende} Zauberstäbe in dem Laden gegeben. Na gut, genaugenommen \emph{konnte} es ein Zufall sein; es gab sechs Milliarden Menschen auf der Welt und tausend-zu-eins-Zufälle passierten täglich. Aber der Satz von Bayes besagte nun einmal ganz unmissverständlich, dass jede vernünftige Hypothese zu bevorzugen war, der zufolge es wahrscheinlicher als ein Tausendstel war, dass er den Bruder vom Zauberstab des Dunklen Lords bekommt.

McGonagall hatte nur "wie eigenartig" gesagt, es dabei belassen und Harry somit angesichts der schieren Gleichgültigkeit der Zauberer und Hexen in einen Schockzustand versetzt. In keiner denkbaren Welt würde Harry einfach "hm" sagen und aus dem Laden gehen, ohne zumindest nach einer Hypothese zu \emph{suchen}, die erklären konnte, was hier vor sich ging.

Seine linke Hand hob sich und berührte seine Narbe.

\emph{Was… genau…}

"Sie sind nun ein richtiger Zauberer", sagte McGonagall. "Ich gratuliere."

Harry nickte.

"Und was denken Sie über die Zaubererwelt?"

"Es ist seltsam", sagte Harry. "Eigentlich sollte ich jetzt an alles denken, was ich heute über die Zauberei gelernt habe… alles, von dem ich nun weiß, dass es möglich ist, und alles, von dem ich nun weiß, dass es falsch ist; und natürlich an all die Arbeit, die nun vor mir liegt, um das alles zu verstehen. Und dennoch bin ich von vergleichsweise unbedeutenden Dingen abgelenkt, wie zum Beispiel", Harry senkte seine Stimme, "diese ganze Junge-der-überlebte-Geschichte." Es schien zwar niemand in der Nähe zu sein, doch es gab keinen Grund, das Schicksal herauszufordern.

McGonagall hüstelte. "Tatsächlich? Was Sie nicht sagen."

Harry nickte. "Ja. Es ist nur so… \emph{seltsam}. Herauszufinden, dass man ein Teil dieser großen Geschichte war, dieser Mission, den großen und furchtbaren Dunklen Lord zu vernichten; und herauszufinden, dass es schon erledigt ist. Fertig. Aus und vorbei. Als wäre man Frodo Beutlin und erfährt, dass man von den Eltern zum Schicksalsberg gebracht wurde und den Ring reinfallen ließ, als man gerade ein Jahr alt war, so dass man sich nicht mal daran erinnern kann."

McGonagalls Lächeln war etwas eingefroren.

"Wissen Sie, wenn ich irgendwer anders wäre --- egal wer --- würde ich mir vermutlich viele Gedanken darüber machen, wie ich diesem Auftakt gerecht werden kann. \emph{Mensch, Harry, was hast du gemacht, seit du den Dunklen Lord besiegt hast? Du hast einen Bücherladen eröffnet? Das ist klasse! Sag, wusstest du, dass ich mein Kind nach dir benannt habe?} Ich hoffe aber, dass \emph{das} kein Problem sein wird." Harry seufzte. "Aber dennoch… ich wäre fast froh, wenn es noch ein paar lose Enden dieses Abenteuers gäbe, nur damit ich sagen könnte, dass ich, nun ja, in gewisser Weise \emph{teilgehabt} habe."

"Oh?", sagte McGonagall in einem seltsamen Tonfall. "Was schwebt Ihnen da vor?"

"Nun, zum Beispiel haben Sie erwähnt, dass meine Eltern verraten wurden. Wer hat sie verraten?"

"Sirius Black", sagte McGonagall. Sie zischte den Namen beinahe. "Er ist in Askaban. Dem Zauberergefängnis."

"Wie wahrscheinlich ist es, dass Sirius Black aus dem Gefängnis ausbricht und ich ihn dann finden und in einem spektakulären Duell besiegen muss --- oder besser: ein großes Kopfgeld auf ihn aussetze und mich irgendwo in Australien verstecke, bis das Ganze vorbei ist?"

McGonagall blinzelte. Zweimal. "Unwahrscheinlich. Niemand ist jemals aus Askaban entkommen und ich bezweifle, dass er der Erste sein wird."

Harry traute dem "Niemand ist jemals aus Askaban entkommen"-Spruch nicht so recht. Allerdings war es mit Magie wohl möglich, ein tatsächlich fast vollkommen perfektes Gefängnis zu erschaffen, insbesondere wenn man einen Zauberstab hatte und die Insassen nicht. Der beste Weg zu entkommen wäre vermutlich, dort gar nicht erst hinzukommen.

"Also gut", sagte Harry. "Klingt so, als könnte ich das abhaken." Er seufzte und rieb sich den Kopf. "Oder vielleicht ist der Dunkle Lord in jener Nacht gar nicht richtig gestorben. Nicht vollständig. Sein Geist lauert, flüstert den Leuten in Alpträumen zu, welche in unsere alltägliche Welt hinübertropfen, und sucht nach einem Weg zurück in die Lande der Lebenden, die er zu vernichten trachtet --- und nun sind er und ich gemäß einer uralten Prophezeiung in einem tödlichen Duell aneinander gekettet, welches der Gewinner verlieren und der Verlierer gewinnen wird ---"

McGonagalls Kopf rotierte wild, ihre Augen schossen hervor und suchten die Straße nach Zuhörern ab.

"Das war ein \emph{Scherz}, Professor McGonagall", sagte Harry mit einem Anflug von Genervtheit. Verdammt nochmal, warum nahm sie immer alles so ernst ---

Ein bedrückendes Gefühl breitete sich langsam in Harrys Innerem aus.

McGonagall sah Harry mit einem ruhigen Gesichtsausdruck an. Einem sehr, \emph{sehr} ruhigen Gesichtsausdruck. Dann setzte sie ein Lächeln auf. "Natürlich war es das, Mr~Potter."

\emph{Oh, Scheiße.}

Hätte Harry seine Gedankengänge der letzten Augenblicke in Wort fassen sollen, so wäre etwa Folgendes herausgekommen: "Wenn ich die Wahrscheinlichkeit, dass McGonagalls Reaktion ein Resultat sorgfältiger Selbstkontrolle war, abschätze und mit der Wahrscheinlichkeitsverteilung all der Reaktionen abgleiche, mit denen sie \emph{normalerweise} einen schlechten Witz von mir quittieren würde, dann ist ihr Verhalten ein signifikantes Zeichen dafür, dass sie etwas vor mir versteckt."

Aber was Harry tatsächlich dachte, war: \emph{Oh, Scheiße.}

Harry drehte seinen Kopf um die Straße zu prüfen. Nein, niemand da. "Er ist \emph{nicht} tot, habe ich Recht?", seufzte Harry.

"Mr~Potter -"

"Der Dunkle Lord ist am Leben. \emph{Natürlich} ist er am Leben. Es war ein Akt von äußerstem \emph{Optimismus}, auch nur von etwas anderem \emph{geträumt} zu haben. Ich muss wohl von Sinnen gewesen sein, ich kann mir nicht vorstellen, was ich mir dabei gedacht habe. Nur weil jemand sagt, dass sein Körper zu einem Haufen Asche zerfallen gefunden wurde --- alleine deswegen habe ich schon gedacht, dass er tot wäre? \emph{Ganz offensichtlich} muss ich noch \emph{viel} über die Kunst lernen, \emph{richtig} pessimistisch zu sein."

"Mr~Potter ---"

"Sagen Sie mir zumindest, dass es nicht wirklich eine Prophezeiung gibt…" Aber McGonagall gab ihm immer noch dieses breite, festgefrorene Lächeln. "Ach, das kann doch nicht Ihr Ernst sein!"

"Mr~Potter, Sie sollten sich nicht einfach Sachen ausdenken, die Ihnen dann nur Sorgen bereiten ---"

"Wollen Sie mir jetzt wirklich \emph{so} kommen? Stellen Sie sich doch einmal meine Reaktion vor, wenn ich später herausfinde, dass es doch etwas gab, worüber man sich Sorgen machen sollte."

McGonagalls Lächeln verschwand.

Harrys Schultern erschlafften. "Ich habe eine ganze Welt von Magie zu verstehen. \emph{Ich habe keine Zeit für so etwas.}"

Beide schwiegen, als ein Mann in einem orangen Umhang auf der Straße erschien und sie langsam passierte. McGonagalls Augen verfolgten ihn unauffällig. Harrys Mund bewegte sich, während er sich hart auf die Lippe biss, und wer aufmerksam hingesehen hätte, dem wäre aufgefallen, wie ein kleiner Tropfen Blut erschien.

Als der Mann in Orange in einiger Entfernung verschwunden war, sprach Harry wieder, kaum lauter als ein Flüstern. "Werden Sie mir jetzt die Wahrheit sagen, Professor McGonagall? Und versuchen Sie nicht, es beiseite zu schieben; ich bin nicht dumm."

"Sie sind elf Jahre alt, Mr~Potter!", sagte sie in einem barschen Flüstern.

"Und daher kein vollwertiger Mensch. Entschuldigen Sie, für einen Moment hatte ich es vergessen."

"Dies sind schreckliche und bedeutende Angelegenheiten. \emph{Geheime} Angelegenheiten, Mr~Potter. Es ist eine Katastrophe, dass Sie --- noch ein Kind --- überhaupt so viel wissen. Sie dürfen es niemandem erzählen, verstehen Sie? Absolut niemandem!"

Wie es manchmal geschah, wenn Harry ausreichend wütend wurde, kühlte sein Blut ab, statt aufzuwallen, und eine erschreckende Klarheit überkam seinen Geist, berechnete alle Möglichkeiten und bewertete ihre Konsequenzen mit glasklarem Realismus.

\emph{Weise darauf hin, dass du ein Recht hast, es zu erfahren: Fehlschlag. Elfjährige Kinder haben in McGonagalls Augen kein Recht, irgendwas zu wissen.

Sag, dass du nicht mehr mit ihr befreundet sein möchtest: Fehlschlag. Sie schätzt deine Freundschaft nicht ausreichend.

Weise darauf hin, dass du in Gefahr bist, solange du nicht Bescheid weißt: Fehlschlag. Es wurden bereits Pläne geschmiedet, die auf deinem Unwissen aufbauen. Die garantierte Unannehmlichkeit, diese zu überdenken, erscheint ihnen weit unangenehmer als die lediglich unsichere Möglichkeit, dass dir etwas zustößt.

Gerechtigkeit und Vernunft werden beide scheitern. Du musst entweder etwas finden, das sie will, oder mit einem Mittel drohen, welches sie fürchtet…}

Aha.

"Also gut, Professor McGonagall", sagte Harry in einer tiefen, eisigen Stimme, "es sieht so aus, als ob ich etwas habe, das Sie wollen. Sie können mir --- wenn Sie wollen --- die Wahrheit sagen --- die ganze Wahrheit! --- und im Gegenzug werde ich Ihre Geheimnisse bewahren. Oder Sie können versuchen mich unwissend zu halten, so dass Sie mich als Spielfigur benutzen können; in diesem Falle schulde ich Ihnen gar nichts.

McGonagall blieb schlagartig mitten auf der Straße stehen. Ihre Augen loderten und ihre Stimme wurde zu einem Zischen: "Was erlauben Sie sich?"

"Was erlauben \emph{Sie} sich?", flüsterte er zurück.

"Sie wollen mich erpressen?"

Harrys Lippen verzogen sich. "Ich biete Ihnen einen \emph{Gefallen} an. Ich gebe Ihnen eine \emph{Chance}, ihre hübschen Geheimnisse zu behalten. Wenn Sie ablehnen, habe ich jeden erdenklichen Grund, mich anderswo umzuhören. Nicht um Ihnen zu schaden, sondern weil ich Bescheid wissen \emph{muss}. Überwinden Sie Ihren nutzlosen Ärger über ein \emph{Kind}, von dem Sie annehmen, dass es Ihnen gehorchen muss, und Sie werden einsehen, dass jeder vernünftige Erwachsene dasselbe machen würde. \emph{Betrachten Sie es doch mal aus meiner Sicht! Wie würden Sie sich fühlen, wenn Sie an meiner Stelle wären?}"

Harry beobachtete McGonagall, sah ihr hektisches Atmen. Es war wohl an der Zeit, den Druck etwas herauszunehmen und es für eine Weile in ihr köcheln zu lassen. "Sie müssen sich nicht gleich entscheiden", sagte Harry in einem ruhigeren Tonfall. "Ich verstehe es, wenn Sie Zeit brauchen um über mein Angebot nachzudenken… Aber ich muss Sie vor einer Sache warnen." Harrys Stimme wurde kälter. "Versuchen Sie nicht, diesen Vergessenszauber auf mich zu sprechen. Vor einiger Zeit habe ich mir ein Zeichen überlegt und ich habe es mir bereits selbst geschickt. Wenn ich das Zeichen finde und mich nicht daran erinnern kann, es gesendet zu haben…" Harry ließ seine Stimme bedeutsam verhallen.

In McGonagalls Gesicht begann es zu arbeiten und ihre Miene veränderte sich. "Ich… hatte nicht vor, den Vergessenszauber auf Sie anzuwenden, Mr~Potter… aber warum um alles in der Welt haben Sie sich so ein Signal ausgedacht, wenn Sie nicht wussten, dass ---"

"Ich hab mir das vor einer Weile ausgedacht, als ich ein Science-Fiction-Buch von Muggeln las und mir dachte 'nur für den Fall, dass…' --- und nein, ich werde Ihnen das Signal nicht verraten, ich bin nicht dumm."

"Ich hatte nicht vor zu fragen", sagte McGonagall. Sie schien in sich selbst versunken und sah auf einmal sehr alt und müde aus. "Das war ein anstrengender Tag, Mr~Potter. Können wir Ihren Koffer besorgen und Sie nach Hause bringen? Ich verlasse mich darauf, dass Sie über diese Angelegenheit mit niemandem sprechen, bis ich Zeit zum Nachdenken hatte. Beachten Sie bitte auch, dass nur zwei weitere Personen auf der ganzen Welt davon wissen, nämlich der Schulleiter Albus Dumbledore und Professor Severus Snape."

So. Neue Informationen; das war ein Friedensangebot. Harry nickte anerkennend, drehte seinen Kopf nach vorn und begann wieder zu laufen.

"Also muss ich nun einen Weg finden, um einen unsterblichen Dunklen Lord zu töten", sagte Harry und seufzte frustriert. "Ich wünschte wirklich, Sie hätten mir das gesagt, \emph{bevor} wir mit dem Einkaufen angefangen haben."

\later

Das Koffergeschäft war prunkvoller ausgestattet als jeder andere Laden, den Harry bisher besucht hatte; die schweren Vorhänge war filigran gemustert, der Fußboden und die Wände bestanden aus gebeiztem und poliertem Holz, und die Kisten und Koffer besetzten Ehrenplätze auf polierten Elfenbeinsockeln. Der Verkäufer war in Roben gekleidet, die nur einen Hauch weniger exquisit waren als jene von Lucius Malfoy und er sprach mit einer gehobenen, öligen Höflichkeit zu Harry und McGonagall.

Harry hatte einige Fragen gestellt und neigte zu einem Schrankkoffer aus schwer aussehendem Holz --- nicht poliert, aber warm und solide; mit einem geschnitzten Wächterdrachen versehen, dessen Augen jeden ansahen, der ihm nahe kam. Einige Zauber bewirkten, dass er sehr leicht war, auf Kommando schrumpfte und kleine klauenbewehrte Tentakeln aus seiner Unterseite ausfahren konnte um dem Besitzer hinterherzuwandern. Mit zwei Schubfächern auf jeder der vier Seiten, von denen jedes so groß war wie der gesamte Koffer. Ein Deckel mit vier Schlössern, von denen jedes einen anderen Raum im Inneren zugänglich machte. Und --- das war das Wichtigste --- mit einem Griff am Boden, der eine Treppe freilegte, welche nach unten in einen kleinen beleuchteten Raum führte, in dem --- so hatte Harry geschätzt --- etwa zwölf Bücherregale Platz finden würden.

Wenn es solche Koffer gab, dann verstand Harry nicht, warum sich irgendjemand die Mühe machte, ein Haus zu besitzen.

Einhundertundacht goldene Galleonen. Das war der Preis eines \emph{guten} Schrankkoffers. Kaum benutzt. Bei etwa fünfzig britischen Pfund für eine Galleone hätte das gereicht, um ein gebrauchtes Auto zu kaufen. Es war teurer als alles andere, was Harry in seinem bisherigen Leben gekauft hatte, zusammen.

Siebenundneunzig Galleonen. Soviel war noch in dem Beutel mit Gold, den Harry aus Gringotts hatte mitnehmen dürfen.

McGonagalls Gesichtsausdruck wirkte verdrießlich. Nach diesem langen Einkaufstag brauchte sie Harry, als der Verkäufer den Preis nannte, nicht fragen, wie viel Gold noch im Beutel war; offenbar konnte die Professorin ohne Stift und Papier gut Kopfrechnen. Wieder einmal erinnerte Harry sich selbst daran, dass \emph{wissenschaftlich ungebildet} und \emph{dumm} zwei verschiedene Sachen waren.

"Es tut mir leid, junger Mann", sagte McGonagall. "Das ist mein Fehler. Ich würde anbieten, Sie zurück zu Gringotts zu begleiten, aber die Bank wird nun nur noch für Notfälle geöffnet haben."

Harry atmete tief durch. Für das, was er nun vorhatte, musste er ein wenig zornig sein, denn anderenfalls würde er nicht den Mut aufbringen, es zu machen. \emph{Sie hat mir nicht zugehört}, dachte er zu sich selbst. \emph{Ich hätte mehr Gold abgehoben, aber sie wollte nicht zuhören…} Er dachte zurück an den düsteren Zorn, den er zuvor verspürt hatte, und versuchte etwas davon herbeizurufen. Er stellte sich vor, wie er jetzt auftreten musste und zog diese Persönlichkeit über sich selbst wie einen Zaubererumhang. Voll und ganz auf McGonagall und das Bedürfnis, dieses Gespräch zu seinem Nutzen umzubiegen, konzentriert sagte er: "Lassen Sie mich raten --- Sie dachten, Sie würden sich \emph{eine Menge} Spielraum lassen, und dass einhundert Galleonen \emph{mehr als genug} sein würden --- deshalb haben Sie mich nicht gewarnt, als nur noch siebenundneunzig Galleonen übrig waren."

McGonagall schloß resignierend die Augen. "Ja."

"Ich habe das erwartet. Ich habe damit gerechnet, dass so etwas passiert. Es gibt Studien, die zeigen, dass genau so etwas passiert, wenn Leute \emph{denken}, sie hätten genügend Reserven eingeplant. Wäre es \emph{meine} Entscheidung gewesen, dann hätte ich \emph{zweihundert} Galleonen mitgenommen, nur um sicher zu gehen. Es war genügend Geld im Verlies und ich hätte alles was zuviel war später wieder zurückbringen können. Aber ich \emph{wusste}, dass Sie das nicht zulassen würden. Ich wusste, es würde nichts bringen, auch nur zu fragen. Ich wusste Sie würden genervt, oder vielleicht sogar wütend reagieren, wenn ich fragen würde. Oder liege ich da falsch?"

"Nein", sagte McGonagall. "Sie haben Recht." Ihre Stimme enthielt den Anflug einer Entschuldigung, aber auch einen stolzen Unterton, als ob Harry würdigen sollte, welch eine große Ehre es war, dass Professor McGonagall sich bei ihm entschuldigte.

"Sie müssen verstehen, Professor McGonagall", Harry wählte seine Worte sehr sorgfältig, "dies ist der Grund, warum ich Erwachsenen nicht vertraue. Sie dachten, erwachsen zu sein bedeutet, dass es Ihre Rolle ist, mich davon abzuhalten, zuviel Geld aus dem Verlies zu nehmen. Nicht, dass es Ihre Rolle wäre, die Aufgabe \emph{auf jeden Fall} zu erledigen."

McGonagalls Augen weiteten sich und sie sah Harry angestrengt an.

"Nun, Professor McGonagall, wenn Sie das alles nochmal machen müssten und ich vorschlüge, hundert Galleonen extra mitzunehmen --- nur um sicher zu gehen, nur um vorbereitet zu sein --- würden Sie mich dann \emph{beachten}?"

"Ich verstehe, was Sie sagen wollen", sagten McGonagall. "Sie brauchen mir keine Belehrungen erteilen, junger Mann."

"Aber ich habe noch gar nicht gesagt, was ich sagen wollte. Kennen Sie den Unterschied zwischen einer Person, die es wert ist, mit ihr zu sprechen, und einem bloßen Hindernis, Professor McGonagall? Verstehen Sie, wie ich es sehe? Wenn es einem Erwachsenen \emph{am Wichtigsten} ist, dass er mir überlegen ist; dass ich ihm gehorche; dann ist dieser Erwachsene ein Hindernis. Ein \emph{möglicher Partner} ist jemand, für den es wichtiger ist, die Aufgabe \emph{auf jeden Fall} zu erfüllen. Erlauben Sie mir, Ihnen etwas zu zeigen, Professor McGonagall."

Der Kistenverkäufer sah beide mit unverhohlener Faszination an, als Harry seinen Eselsfellbeutel nahm und sagte: "Elf einzelne Galleonen, bitte."

Und dann war Gold in Harrys Hand.

"\emph{Woher haben Sie ---}"

"Aus meinem Verlies, Professor McGonagall. Als ich in diesen Goldhaufen fiel, habe ich etwas davon in meine Taschen geschoben und dann den Geldbeutel dagegen gehalten, so dass es nicht an der falschen Stelle klimperte. Denn --- verstehen Sie --- ich habe von Anfang an erwartet, dass so etwas passieren würde."

McGonagalls Mund stand nun weit, weit offen.

"Also lautet nun die Frage… sind Sie wütend auf mich, weil ich Ihre Autorität untergraben habe? Oder froh, dass unser Tag erfolgreich endet, statt zu scheitern? Diese Frage hat keine versteckten Implikationen. Sie hat keinen Einfluss auf unsere mögliche Kooperation in zukünftigen Fragen. Ich will nur wissen, ob Sie ein potentieller Partner, oder ein Hindernis sein werden… Minerva."

Der Verkäufer zog hörbar Luft ein.

Und die große Hexe stand dort, stumm.

"Die Disziplin in Hogwarts \emph{muss} durchgesetzt werden", sagte sie nach fast einer ganzen Minute. "Zum Wohle aller Studenten. Und das beinhaltet Höflichkeit und Gehorsam gegenüber allen Professoren."

Harry neigte den Kopf. "Ich verstehe. Professor McGonagall." Obwohl es erstaunlich war, um wie viel wichtiger das Durchsetzen der Disziplin schien, wenn man oben in der Hierarchie stand, nicht unten… aber Harry hielt es nicht für weise, diesen Punkt weiter zu diskutieren.

"Davon abgesehen… gratuliere ich Ihnen zu Ihrer Voraussicht."

Harry wollte jubeln, oder sich erbrechen, oder in Ohnmacht fallen; irgendwas. Das war das erste Mal, dass diese Ansprache jemals bei einem Erwachsenen funktioniert hatte. Das war das erste Mal, dass \emph{irgendeine} seiner Ansprachen auf \emph{irgendjemanden} gewirkt hatte. Vielleicht, weil es das erste Mal war, dass er etwas wirklich Wichtiges hatte, das ein Erwachsener von ihm wollte; aber dennoch ---

Minerva McGonagall: +1 Punkt.

Harry verbeugte sich und gab den Goldbeutel und die elf zusätzlichen Galleonen in McGonagalls Hände. "Ich überlasse es Ihnen. Denn ich muss nun auf die Toilette. Darf ich fragen, wo sie sich befindet?"

Der Verkäufer zeigte, nunmehr wieder salbungsvoll, auf eine Tür mit goldenem Knauf, die in die Wand eingelassen war. Als Harry darauf zu lief, hörte er hinter sich, wie der Verkäufer in seiner öligen Stimme fragte: "Dürfte ich mich erkundigen, wer das war, Madam McGonagall? Ein Slytherin, nehme ich an --- im dritten Jahr, womöglich? --- und aus einer altehrwürdigen Familie, obwohl ich nicht erkannt ---"

Der Schlag der Toilettentür schnitt seine Worte ab und nachdem Harry das Schloß gefunden und es einrasten lassen hatte, ließ er sich erschöpft an der Tür herabsinken. Harrys ganzer Körper war von Schweiß überzogen, der seine ganze Muggelkleidung durchnässt hatte, aber zum Glück nicht durch den Umhang sichtbar war. Er beugte sich hinunter zum goldbesetzten Rand der elfenbeinernen Toilette und würgte ein paarmal, aber glücklicherweise kam nichts hoch.

\later

Und dann standen sie wieder im Hof des Tropfenden Kessels, in diesem kleinen, blätterverkleisterten Verbindungsstück zwischen der Winkelgasse des magischen Großbritanniens und der Muggelwelt. Das war schon eine \emph{unglaublich} entkoppelte Wirtschaft… Sobald Harry wieder auf der anderen Seite war, würde er eine Telefonzelle suchen und seinen Vater anrufen. Wegen des Schrankkoffers brauchte er sich --- so schien es -- überhaupt keine Sorgen zu machen; er war ein signifikant magisches Objekt, etwas, das die meisten Muggel nicht einmal wahrnehmen würden. So etwas konnte man in der Zaubererwelt bekommen, wenn man bereit war, den Preis eines gebrauchten Autos zu bezahlen. Harry fragte sich, ob sein Vater wohl in der Lage sein würde, die Kiste zu sehen, wenn er ihn darauf hinwies.

"Hier trennen sich nun unsere Wege --- für eine Weile", sagte Professor McGonagall. Sie schüttelte verwundert ihren Kopf. "Dies war bei weitem der seltsamste Tag in meinem Leben… seit vielen Jahren. Seit dem Tag, an dem ich erfuhr, dass ein Kind Du-weiß-schon-wen besiegt hatte. Ich frage mich nun im Nachhinein, ob dies der letzte normale Tag auf dieser Welt gewesen sein mag."

Oh, als ob \emph{sie} einen Grund hätte, sich zu beschweren. \emph{Sie denken, Ihr Tag war surreal? Schauen Sie sich meinen an.}

"Ich war heute sehr von Ihnen beeindruckt", sagte Harry zu ihr. "Ich hätte daran denken sollen, es auch zu erwähnen. Ich habe Ihnen im Kopf Punkte gegeben und all sowas."

"Ich danke Ihnen, Mr~Potter", sagte McGonagall. "Wenn Sie bereits in ein Haus sortiert gewesen wären, hätte ich diesem so viele Punkte abgezogen, dass noch ihre Enkelkinder den Hauspokal verloren hätten."

"Ich danke \emph{Ihnen}, Minerva." Es war vermutlich noch nicht an der Zeit, sie Minny zu nennen.

Diese Frau mochte wohl die vernünftigste Erwachsene sein, die Harry jemals getroffen hatte, trotz ihres fehlenden wissenschaftlichen Hintergrunds. Harry überlegte sogar, ihr den Stellvertreterposten in der Organisation anzubieten, die er gründen würde, um den Dunklen Lord zu bekämpfen, aber er war nicht töricht genug, das laut auszusprechen. \emph{Was wäre wohl ein geeigneter Name dafür? Die Todesser-Esser?}

"Wir sehen uns bald wieder, wenn die Schule beginnt", sagte McGonagall. "Und, Mr~Potter, wegen Ihres Zauberstabes ---"

"Ich weiß, was Sie fragen werden", sagte Harry. Er nahm seinen kostbaren Zauberstab heraus und mit einem tiefen, stechenden Schmerz im Inneren drehte er ihn und bot ihn --- mit dem Griff nach vorne --- McGonagall an. "Nehmen Sie ihn. Ich hatte nicht vor irgendwas zu machen, nicht eine einzige Sache, aber ich möchte nicht, dass Sie Alpträume davon haben, wie ich mein Zuhause in die Luft jage."

McGonagall schüttelte schnell den Kopf. "Oh nein, Mr~Potter! So etwas machen wir nicht. Ich wollte Sie nur warnen, ihren Stab nicht zu Hause zu \emph{benutzen}, da es Mittel und Wege gibt, um Zauberei durch Minderjährige zu entdecken, und es ist verboten, das ohne Aufsicht zu tun."

"Ah", sagte Harry und lächelte. "\emph{Das} hört sich nach einer \emph{sehr vernünftigen} Regel an. Ich freue mich zu sehen, dass die Zaubererwelt solche Dinge ernst nimmt."

McGonagall blickte ihn forsch an. "Sie meinen das ernst."

"Ja", sagte Harry. "Ich verstehe es. Magie ist gefährlich und die Regeln sind aus gutem Grund da. Einige andere Sachen sind ebenfalls gefährlich, ich verstehe das auch. Denken Sie daran, dass ich nicht dumm bin."

"Ich werde es kaum jemals vergessen. Vielen Dank, Harry, das macht es mir leichter, Ihnen bestimmte Dinge anzuvertrauen. Auf Wiedersehen."

Harry wandte sich zum Gehen. Hinein in den Tropfenden Kessel und hinaus in die Muggelwelt. Als seine Hand die Türklinke umfasste, hörte er ein letztes Flüstern hinter sich.

"Hermine Granger."

"Wie bitte?", sagte Harry, die Hand noch an der Klinke.

"Halten Sie im Zug nach Hogwarts Ausschau nach einer Erstklässlerin namens Hermine Granger."

"Wer ist das?"

Es kam keine Antwort und als sich Harry umdrehte, war McGonagall entschwunden.

\later

\emph{Nachspiel:}

Professor Dumbledore lehnte sich nach vorne. Seine funkelnden Augen starrten auf McGonagall. "So, Minerva, was hältst du von Harry?"

McGonagall öffnete ihren Mund. Dann schloss sie ihn wieder. Dann öffnete Sie ihn von neuem. Keine Worte kamen hervor.

"Ich verstehe", sagte Dumbledore ernst. "Vielen Dank für Deinen Bericht, Minerva. Du kannst gehen."

-\/-\/-\/-\/-\/-\/-\/-\/-\/-\/-\/-\/-\/-\/-\/-\/-\/-\/-\/-\/-\/-\/-\/-\/-\/-\/-\/-\/-\/-

\textbf{"Sei bereit --- so geht das Pfadfinderlied. {[}…{]}"}

Im Original handelt es sich um das Lied "\href{http://www.youtube.com/watch?v=fSwjuz_-yao}{Be prepared}" von Tom Lehrer.

Die Übersetzung ist an dieser Stelle sehr frei; anders geht es bei Liedern aber wohl nicht. Wer Englisch kann und Ironie mag, dem kann ich das Original nur empfehlen!

\textbf{Ein Stein als Haustier:}

Ein Stein ist das ideale Haustier -- er macht weniger Arbeit und ist handzahm. \href{http://de.wikipedia.org/wiki/Pet_rock}{Pet Rocks} (aus dem Englischen: pet = Haustier, rock = Stein) sind eine Art analoger Vorgänger des Tamagotchi, erfunden in den 70er-Jahren in den USA.

