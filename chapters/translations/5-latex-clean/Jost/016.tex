

\hypertarget{laterales-denken}{% \section{16. Laterales Denken}\label{laterales-denken}}

\later

„\emph{Ich bin kein Psychopath, ich bin bloß sehr kreativ.}“

\later

Sobald er am Mittwoch das Verteidigungs-Klassenzimmer betreten hatte, wusste Harry, dass \emph{dieses} Fach \emph{anders} sein würde.

Es war das größte Klassenzimmer, das er bisher in Hogwarts gesehen hatte, ähnlich einem Hörsaal in einer Universität, mit Reihen von Tischen, die über eine riesige, flache Bühne aus weißem Marmor thronten. Das Klassenzimmer war hoch oben im Schloss—im fünften Stock—und Harry wusste, dass er keine bessere Erklärung dafür bekommen würde, wie so ein Raum ins Schloss passte. Ihm wurde klar, dass Hogwarts schlicht und einfach durch keine Geometrie—euklidische oder sonst eine—beschrieben werden konnte; es gab Verbindungen, aber keine Richtungen.

Im Gegensatz zu einem Hörsaal klappten die Sitze nicht hoch; stattdessen waren auf jeder Ebene in einer halbkreisförmigen Reihe die für Hogwarts völlig normalen Holztische und Holzstühle aufgebaut. Abgesehen davon, dass sich auf jedem Tisch ein flaches, weißes, rechteckiges, mysteriöses Objekt befand.

Auf einem leicht erhöhten Podium aus dunklerem Marmor in der Mitte der riesigen Bühne befand sich ein einsamer Lehrertisch. Dahinter war Quirrell in seinem Stuhl zusammengesunken, der Kopf im Nacken, mit dünnen Spuckefäden auf seinem Umhang.

\emph{Woran erinnert mich das bloß …?}

Harry war so früh zum Unterricht erschienen, dass noch keine anderen Schüler da waren. (Was die Beschreibung von Zeitreisen anging, war die englische Sprache mangelhaft; insbesondere fehlten ihr Worte, die ausdrücken konnten, wie praktisch Zeitreisen waren.) Quirrell schien im Moment nicht zu…funktionieren…und Harry verspürte ohnehin keine besondere Lust, sich Quirrell zu nähern.

Harry wählte einen Tisch, kletterte zu ihm hoch, setzte sich und holte das Verteidigungs-Lehrbuch hervor. Er hatte es zu sieben Achteln durchgelesen—eigentlich wollte er das Buch vor dieser Unterrichtsstunde durchgelesen haben, doch er war etwas im Rückstand und hatte den Zeitumkehrer heute schon zwei Mal benutzt.

Bald hörte man Lärm, während das Klassenzimmer begann, sich zu füllen. Harry ignorierte ihn.

„Potter? Was machst \emph{du} hier?“

\emph{Diese} Stimme gehörte nicht hierher. Harry sah auf. „Draco? Was machst \emph{du} in—meine Güte, du hast \emph{Lakaien.}“

Einer von den Kerlen, die hinter Draco standen, schien recht viele Muskeln für einen Elfjährigen zu haben, der andere verharrte in einer verdächtig breitbeinigen Position.

Der weißblond-haarige Junge lächelte recht selbstgefällig und deutete hinter sich. „Potter, ich möchte dir Mr~Crabbe und“, seine Hand wanderte vom Muskelprotz zum breitbeinig Dastehenden, „Mr~Goyle vorstellen. Vincent, Gregory, das ist Harry Potter.“

Mr~Goyle legte seinen Kopf schräg und warf Harry einen Blick zu, der vermutlich irgendetwas aussagen sollte, aber einfach nur verkniffen aussah. Mr~Crabbe sagte „Freut mich, Sie zu treffen“ in einer Tonlage, die klang, als versuchte er so tief zu sprechen, wie er nur konnte.

Kurz flackerte Ärger auf Dracos Gesicht auf, wurde jedoch schnell von seinem überheblichen Grinsen verdrängt.

„Du hast \emph{Lakaien!“}, wiederholte Harry. „Wo kriege \emph{ich} Lakaien her?“

Dracos Grinsen wurde breiter. „Ich fürchte, Potter, der erste Schritt ist, nach Slytherin zu kommen—“

„Was? Das ist unfair!“

„— und dann müssen eure Familien das schon vor eurer Geburt arrangiert haben.“

Harry musterte Mr~Crabbe und Mr~Goyle. Beide versuchten offenbar, so bedrohlich zu wirken wie sie nur konnten. Sie hatten sich dazu etwas nach vorne gelehnt, die Schultern hochgezogen, ihre Hälse gestreckt und starrten ihn an.

„Ähm…Moment mal“, sagte Harry. „Das wurde schon vor \emph{Jahren} eingefädelt?“

„Ganz genau, Potter. Ich fürchte, da hast du Pech.“

Mr~Goyle zog einen Zahnstocher hervor und begann, seine Zähne zu säubern, während er bedrohlich wirkte.

„Und“, sagte Harry, „Lucius hat darauf bestanden, dass du deine Bodyguards \emph{nicht} kennenlernen würdest, sondern sie erst am ersten Schultag treffen würdest.“

Das wischte das Grinsen von Dracos Gesicht. „Ja, Potter, wir wissen alle, dass du wahnsinnig klug bist, das weiß inzwischen die ganze Schule, du kannst damit aufhören, so anzugeben—“

„Also wurde ihnen \emph{ihr ganzes Leben lang} gesagt, dass sie deine Lakaien sein werden, und sie haben \emph{jahrelang} überlegt, wie Lakaien sich verhalten sollten—“

Draco schnitt eine Grimasse.

„— und noch schlimmer, \emph{sie} kennen \emph{einander} und haben \emph{geübt}—“

„Der Boss hat gesagt, du sollst's Maul halten“, grummelte Mr~Crabbe. Mr~Goyle biss auf seinen Zahnstocher, hielt ihn zwischen den Zähnen fest und benutzte eine Hand, um mit den Knöcheln der anderen zu knacksen.

„\emph{Ich habe euch doch gesagt, dass ihr das nicht vor Harry Potter tun sollt!}“

Die beiden sahen etwas belämmert aus und Mr~Goyle steckte den Zahnstocher schnell wieder in eine Tasche seines Umhangs. Doch sobald Draco sich von ihnen ab- und Harry zuwandte, versuchten sie wieder bedrohlich zu wirken.

„Ich bitte um Entschuldigung“, sagte Draco steif, „für das beleidigende Verhalten dieser \emph{Vollidioten.}“

Harry sah Mr~Crabbe und Mr~Goyle vielsagend. „Ich würde sagen, dass du etwas zu streng mit ihnen umgehst, Draco. \emph{Ich} finde, dass sie sich genau so verhalten, wie \emph{meine} Lakaien es tun sollten. Ich meine, wenn ich Lakaien hätte.“

Dracos Mund stand offen.

„Hey, Gregory, der will uns doch nich' vom Boss weglocken, oder?“

„Ich bin mir sicher, dass Mr~Potter nicht so einfältig wäre.“

„Oh, ich würde nicht im Traum daran denken“, sagte Harry ruhig. „Ihr könntet es aber im Kopf behalten, falls euer derzeitiger Arbeitgeber es an Wertschätzung vermissen lässt. Außerdem schadet es nie, andere Angebote zu haben, wenn man über Arbeitsbedingungen verhandelt, nicht wahr?“

„Was tut'n \emph{der} in Ravenclaw?“

„Ich kann es mir nicht vorstellen, Mr~Crabbe.“

„Ihr haltet jetzt beide das Maul“, knurrte Draco durch zusammengepresste Zähne. „Das ist ein \emph{Befehl.}“ Mit sichtbarer Mühe lenkte er seine Aufmerksamkeit wieder auf Harry. „Was tust du denn überhaupt im Verteidigungsunterricht der Slytherins?“

Harry runzelte die Stirn. „Warte mal.“ Seine Hand griff in den Beutel. „Stundenplan.“ Er sah auf das Pergament. „Verteidigung, 14:30~Uhr, und jetzt gerade ist es“—Harry sah auf seine Armbanduhr, die 11:23~Uhr anzeigte—„14:23, wenn ich mich nicht irre. Irre ich mich?“ Falls ja…naja, Harry wusste, wie er zu dem Unterricht kam, in dem er eigentlich sitzen sollte. Verdammt, er liebte diesen Zeitumkehrer, und irgendwann, wenn er alt genug war, würden sie heiraten.

„Nein, das stimmt schon“, sagte Draco verwirrt. Sein Blick streifte den Rest des Hörsaals, der inzwischen gefüllt war mit grün gesäumten Umhängen und …

\emph{„Gryffindors!“,} spie Draco aus. „Was tun \emph{die} hier?“

„Hm“, sagte Harry. Professor Quirrell hat gesagt…ich weiß nicht mehr genau, wie er es formuliert hat…dass er einige der konventionellen Unterrichtsmethoden missachten würde. Vielleicht hat er einfach den Unterricht aller vier Häuser zusammengelegt.“

„Hm“, sagte Draco. „Du bist der erste Ravenclaw hier.“

„Jep. Bin etwas früher hergekommen.“

„Was machst du dann hier in der hintersten Reihe?“

Harry blinzelte. „Ich weiß nicht, ist doch ein ganz guter Sitzplatz?“

Draco hustete spöttisch. „Du könntest nicht weiter vom Lehrer weg sitzen, wenn du es versuchen würdest.“ Der blondhaarige Junge rückte etwas näher. „Sag mal, stimmt es eigentlich, dass du das zu Derrick und seiner Gang gesagt hast?“

„Wer ist Derrick?“

„Hast du ihn nicht mit einem Kuchen beworfen?“

„Zwei Kuchen, genau genommen. Was soll ich zu ihm gesagt haben?“

„Dass er in Wirklichkeit gar nichts Durchtriebenes und Ehrgeiziges tut und dass er eine Schande für Salazar Slytherin ist.“ Draco sah Harry eifrig an.

„Das…trifft es ungefähr“, sagte Harry. „Ich glaube, es war eher so etwas wie ‚Ist das hier irgendein unglaublich cleverer Plan, der euch irgendwelche zukünftigen Vorteile bescheren wird, oder ist das eine so unsinnige Beleidigung des Andenkens Salazar Slytherins, wie es aussieht?`, oder so ähnlich. Ich erinnere mich nicht mehr an die genaue Wortwahl.“

„Du verwirrst alle, weißt du?“, sagte der blondhaarige Junge.

„Hä?“, sagte Harry ernsthaft verwirrt.

„Warrington sagte, dass ein langer Aufenthalt unter dem Sprechenden Hut eines der wesentlichen Kennzeichen eines großen Dunklen Zauberers ist. Alle haben darüber geredet und überlegt, ob sie sich vorsichtshalber bei dir einschleimen sollten. Dann kommst du an und beschützt einen Haufen \emph{Hufflepuffs}, um Merlins Willen. \emph{Dann} sagst du Derrick, dass er eine Schande für das Andenken Salazar Slytherins ist! Was \emph{sollen} die Leute bloß denken?“

„Dass der Sprechende Hut mich in das Haus ‚Slytherin! Nur ein Scherz! Ravenclaw!` gesteckt hat und ich mich dementsprechend verhalte.“

Mr~Crabbe und Mr~Goyle kicherten beide, woraufhin Mr~Crabbe sich sehr schnell eine Hand vor den Mund schlug.

„Wir sollten besser Platz nehmen“, sagte Draco. Er zögerte, streckte den Rücken durch und sprach in einem etwas offizielleren Tonfall: „Aber ich möchte unser letztes Gespräch fortführen und ich akzeptiere deine Bedingungen.“

Harry nickte. „Würde es dir etwas ausmachen, wenn wir bis Samstag Nachmittag abwarten? Ich bin im Moment in einer Art Wettbewerb.“

„Ein Wettbewerb?“

„Ich probiere, meine Schulbücher ebenso schnell zu lesen wie Hermine Granger.“

„Granger“, wiederholte Draco. Seine Augen verengten sich. „Das Schlammblut, das sich für Merlin hält? Wenn du vorhast, es \emph{ihr} zu zeigen, dann wünscht ganz Slytherin dir \emph{besonders} viel Glück, Potter, und ich werde dich bis Samstag nicht stören.“ Draco senkte seinen Kopf respektvoll und ging fort, gefolgt von seinen Lakaien.

\emph{Oh, ich kann mir jetzt schon vorstellen, wie toll es wird, mit den beiden zu jonglieren.}

Das Klassenzimmer füllte sich jetzt rasch mit allen vier Saumfarben: grün, rot, gelb und blau. Draco und seine beiden Freunde schienen gerade zu versuchen, drei benachbarte Sitze in der ersten Reihe zu bekommen—die natürlich schon besetzt waren. Mr~Crabbe und Mr~Goyle hatten sich äußerst bedrohlich aufgebaut, schienen aber keinen Erfolg zu haben.

Harry beugte sich über sein Verteidigungs-Lehrbuch und las weiter.

\later

Um 14:35~Uhr, als die meisten Plätze besetzt waren und niemand mehr reinkam, zuckte Professor Quirrell plötzlich zusammen, setzte sich in seinem Stuhl aufrecht hin und sein Gesicht erschien auf jedem der flachen, weißen, rechteckigen Objekte, die auf den Tischen der Schüler aufgestellt waren.

Harry war überrascht als Professor Quirrells Gesicht plötzlich in etwas auftauchte, was einem Muggel-Fernseher ähnelte. Es war ein nostalgisches, jedoch zugleich trauriges Gefühl, es fühlte sich so sehr wie ein Stück Zuhause an und war es doch nicht …

„Guten Nachmittag, meine jungen Lehrlinge“, sagte Professor Quirrell. Seine Stimme schien dem Bildschirm auf dem Tisch zu entspringen und sich direkt an Harry zu richten. „Willkommen in eurer ersten Unterrichtsstunde in Kampfmagie, wie die Gründer von Hogwarts gesagt hätten; oder, wie es Ende des zwanzigsten Jahrhunderts genannt wird, Verteidigung gegen die Dunklen Künste.“

Plötzliche Unruhe kam auf, als überraschte Schüler nach ihrem Pergament oder Schreibblock griffen.

„Nein“, sagte Professor Quirrell. „Macht euch nicht die Mühe aufzuschreiben, wie dieses Fach einst genannt wurde. Keine solch sinnlosen Fragen werden in meinem Unterricht zu eurer Note beitragen. Das verspreche ich.“

Viele Schüler setzten sich bei diesen Worten kerzengerade hin und sahen etwas schockiert aus.

Professor Quirrell lächelte dünn. „Diejenigen von euch, die ihre Zeit damit verschwendet haben, das nutzlose Verteidigungslehrbuch für Erstklässler zu lesen—“

Ein ersticktes Keuchen war zu hören. Harry fragte sich, ob es von Hermine stammte.

„— könnten den falschen Eindruck bekommen haben, dass es—obwohl das Fach ‚Verteidigung gegen die Dunklen Künste` heißt—in Wahrheit darum geht, wie man sich vor Albtraumschmetterlingen schützt, die etwas schlechtere Träume verursachen können, oder vor Säureschnecken, die ein fünf Zentimeter dickes Stück Holz zersetzen können, wenn man ihnen fast einen Tag Zeit gibt.“

Professor Quirrell schob seinen Stuhl vom Tisch zurück und stand auf. Der Bildschirm auf Harrys Tisch folgte jeder seiner Bewegungen. Professor Quirrell schritt zur Vorderseite des Raums und brüllte:

„Der Ungarische Hornschwanz ist größer als ein dutzend Männer! Sein Feueratem ist so schnell und präzise, dass er einen Schnatz im Flug schmelzen kann! Ein Todesfluch erledigt ihn!“

Viele Schüler keuchten auf.

„Der Bergtroll ist gefährlicher als der Ungarische Hornschwanz! Er ist stark genug, um durch Stahl durchzubeißen! Seine Haut ist dick genug um Schockzaubern und Schneidezaubern zu widerstehen! Sein Geruchssinn ist so genau, dass er aus der Ferne schon merkt, ob seine Beute Teil eines Rudels ist, oder alleine und schwach! Und am Fürchterlichsten, der Troll ist die einzige magische Kreatur, die ständig eine Verwandlung des eigenen Körpers aufrecht erhält—er verwandelt sich ständig in seinen eigenen Körper. Wenn ihr irgendwie schafft, ihm einen Arm abzureißen, wächst innerhalb von Sekunden ein neuer! Feuer und Säure erzeugen Narbengewebe, was die Selbstheilungskräfte des Trolls vorübergehend verwirren kann—für ein oder zwei Stunden! Sie sind klug genug, Keulen als Werkzeuge zu verwenden! Der Bergtroll ist die drittbeste Tötungsmaschine in der Natur! Ein Todesfluch erledigt ihn!“

Die Schüler sahen äußerst schockiert aus.

Professor Quirrell lächelte sehr grimmig. „Das sogenannte Verteidigungslehrbuch für Drittklässler sagt euch, dass man den Bergtroll dem Sonnenlicht aussetzen soll, woraufhin er an Ort und Stelle versteinert. Dies, meine jungen Lehrlinge, ist genau das nutzlose Wissen, welches ihr in meinem Unterricht nie zu Ohren bekommen werdet. Man trifft Bergtrolle nicht am helllichten Tag an! Die Idee, man solle Sonnenlicht benutzen, um sie aufzuhalten, entstammt dem Versuch der Lehrbuchschreiber, zulasten praktischer Belange mit ihrem Detailwissen zu prahlen. Nur weil es eine lächerlich obskure Möglichkeit gibt, sich mit Bergtrollen auseinanderzusetzen, bedeutet das nicht, dass ihr sie tatsächlich ausprobieren solltet! Der Todesfluch ist unaufhaltsam, unabwendbar und funktioniert jedes Mal bei allem, was ein Gehirn besitzt. Wenn ihr euch als erwachsene Zauberer nicht in der Lage seht, einen Todesfluch zu sprechen, dann könnt ihr einfach wegapparieren! Das Gleiche gilt, wenn ihr der zweitbesten Tötungsmaschine der Welt gegenübersteht, einem Dementor. Ihr appariert einfach weg!“

„Außer natürlich“, sagte Professor Quirrell, dessen Stimme nun tiefer und härter klang, „ihr unterliegt einem Anti-Disapparier-Fluch. Nein, es gibt nur ein einziges Monster, das euch bedrohen kann, wenn ihr erwachsen seid. Das gefährlichste Monster der Welt, so gefährlich, dass kein anderes auch nur nahe kommt. Der Dunkle Zauberer. Das ist das einzige Wesen, was euch noch bedrohlich werden kann.“

Professor Quirrells Lippen bildeten eine dünne Linie. „Ich werde euch, so sehr es mir auch widerstrebt, genug Kleinkram beibringen, damit ihr den vom Ministerium festgelegten Teil der Jahresendprüfungen bestehen könnt. Da eure Punktzahl in diesen Aufgaben keinerlei Einfluss auf euer zukünftiges Leben haben wird, ist jeder, der mehr als nur bestehen möchte, dazu eingeladen, seine Zeit mit der Lektüre dieses miserablen sogenannten Lehrbuchs zu vergeuden. Dieses Fach heißt nicht ‚Verteidigung gegen unwichtiges Ungeziefer`. Ihr seid hier, damit ihr lernt, euch gegen die Dunklen Künste zu verteidigen. Und das bedeutet, das möchte ich betonen, euch gegen Dunkle Zauberer zu verteidigen. Gegen Menschen mit Zauberstäben, die euch verletzen wollen und die höchstwahrscheinlich auch Erfolg haben werden, wenn es euch nicht gelingt, sie zuerst zu verletzen! Es gibt keine Verteidigung ohne Angriff! Es gibt keine Verteidigung ohne Kampf! Fette, überbezahlte, von Auroren bewachte Politiker, die die Lehrpläne festgelegt haben, glauben, dass diese Wahrheiten nicht für die Ohren von Elfjährigen geeignet sind. Zur Hölle mit diesen Narren! Ihr seid hier, um ein Fach zu lernen, welches in Hogwarts seit achthundert Jahren gelehrt wird! Willkommen zu eurem ersten Jahr Kampfmagie!“

Harry begann zu klatschen. Er konnte einfach nicht anders, es war zu inspirierend.

Als Harry begonnen hatte, stimmten vereinzelte Gryffindors und einige Slytherins mit ein, doch die meisten Schüler waren einfach zu geschockt um zu reagieren.

Professor Quirrell machte eine schneidende Geste und der Applaus erstarb sofort. „Vielen Dank“, sagte Professor Quirrell. „Nun zum Organisatorischen. Ich habe den Kampfunterricht aller Erstklässler zusammengelegt, so dass wir miteinander doppelt so viel Zeit im Klassenzimmer verbringen können—“

Entsetztes Aufkeuchen war zu hören.

„— ein erhöhter Arbeitsaufwand, für den ich euch entschädige, indem ich keine Hausaufgaben aufgeben werde.“

Das Entsetzen verschwand sofort.

„Ja, ihr habt mich richtig verstanden. Ich werde euch das Kämpfen lehren und nicht, wie man zwei Rollen Pergament mit einem Aufsatz über das Kämpfen vollkritzelt und den am nächsten Montag abgibt.“

Harry wünschte sich sehr, neben Hermine zu sitzen, damit er ihren Gesichtsausdruck jetzt sehen konnte, aber andererseits war er sich recht sicher, dass er ihn sich richtig vorstellte.

Außerdem hatte Harry sich verliebt. Es würde eine Dreieckshochzeit werden: er, der Zeitumkehrer und Professor Quirrell.

„Für diejenigen, die möchten, habe ich einige außerschulische Aktivitäten geplant, die ihr, denke ich, äußerst interessant und lehrreich finden werdet. Wollt ihr der Welt eure \emph{eigenen} Fähigkeiten zeigen, statt vierzehn anderen Leuten beim Quidditch zuzusehen? In einer Armee können mehr als sieben Leute kämpfen.“

\emph{Verdammt} noch mal …

„In diesen und anderen außerschulischen Aktivitäten könnt ihr außerdem Quirrellpunkte sammeln. Was sind Quirrellpunkte, fragt ihr? Das Hauspunkte-System ist für meine Zwecke nicht geeignet, weil Hauspunkte zu selten vergeben werden. Ich lasse meine Schüler lieber häufiger wissen, wie gut sie sind. Und in den seltenen Fällen, in denen ich schriftliche Aufgaben stelle, werden diese sich bewerten, während ihr sie ausfüllt. Wenn ihr zu viele Fragen zu einem Thema falsch beantwortet, werden auf dem Blatt die Namen von Schülern erscheinen, die diese Fragen richtig haben und die sich dann Quirrellpunkte verdienen können, indem sie euch helfen.“

… wow. Warum benutzen die anderen Lehrer nicht auch so ein System?

„Was bringen euch Quirrellpunkte, fragt ihr euch? Zunächst einmal werden zehn Quirrellpunkte einen Hauspunkt wert sein. Aber sie werden euch auch andere Vorteile einbringen. Würdet ihr eure Prüfung gern zu einer ungewöhnlichen Zeit schreiben? Gibt es eine bestimmte Unterrichtsstunde, die ihr lieber schwänzen würdet? Ihr werdet feststellen, dass ich zugunsten von Schülern, die genug Quirrellpunkte angehäuft haben, sehr flexibel sein kann. Quirrellpunkte werden bestimmen, wer General einer Armee wird. Und zu Weihnachten—direkt vor den Weihnachtsferien—werde ich jemandem einen Wunsch erfüllen. Irgendein schul-bezogenes Meisterwerk, das in meiner Kraft, Macht und vor allem Genialität liegt. Ja, ich war in Slytherin und ich biete an, einen ausgefeilten Plan auszutüfteln, wenn Ihr Begehren dies erfordert. Diesen Wunsch hat derjenige Schüler aus einem der sieben Schuljahre frei, der die meisten Quirrellpunkte gesammelt hat.“

Das wäre dann Harry.

„Nun lasst eure Bücher und Sachen an euren Tischen—sie sind dort sicher, die Bildschirme passen für euch darauf auf—und kommt herunter auf diese Bühne. Es ist Zeit für ein Spiel namens ‚Wer ist der gefährlichste Schüler im Klassenzimmer?`\,“

\later

Harry drehte seinen Zauberstab in seiner rechten Hand und sagte „\emph{Ma-ha-su!}“

Erneut ertönte ein hohes „bing“ von der schwebenden blauen Kugel, die Professor Quirrell Harry als Ziel zugewiesen hatte. Dieses genaue Geräusch stand für einen perfekten Treffer, den Harry jetzt in neun der letzten zehn Versuche geschafft hatte.

Irgendwo hatte Professor Quirrell einen Zauberspruch ausgegraben, der unglaublich einfach auszusprechen war \emph{und} aus einer lächerlich einfachen Zauberstabbewegung bestand \emph{und} die Angewohnheit hatte, genau dort zu treffen, wo man gerade hinsah. Professor Quirrell hatte verächtlich erklärt, dass wahre Kampfmagie wesentlich komplizierter war. Dass dieser Zauberspruch in einem echten Kampf vollkommen nutzlos war. Dass es ein fast ungeordneter Magieausbruch war, dessen einzige Schwierigkeit das Zielen war, und dass er, wenn er traf, nur einen kurzen Schmerz verursachen würde; als würde man kräftig auf die Nase geschlagen. Dass das einzige Ziel dieses Tests war, zu sehen, wer schnell lernte, da Professor Quirrell sich sicher war, dass noch niemand diesen Zauberspruch oder irgendwas Ähnliches kannte.

Harry war das egal.

„\emph{Ma-ha-su!}“

Ein \emph{roter Energieblitz} schoss aus seinem Zauberstab und traf das Ziel und die blaue Kugel macht wieder „bing“, was bedeutete, dass der Zauberspruch \emph{tatsächlich funktioniert hatte.}

Zum ersten Mal, seitdem er auf Hogwarts angekommen war, fühlte Harry sich wie ein richtiger Zauberer. Er wünschte, das Ziel würde ausweichen, so wie die kleinen Kugeln, die Ben Kenobi benutzt hatte, um Luke zu trainieren, aber aus irgendeinem Grund hatte Professor Quirrell alle Schüler und Ziele fein säuberlich in Reihen aufgestellt, sodass sie sich nicht gegenseitig trafen.

Also senkte Harry den Zauberstab, sprang nach rechts, riss seinen Zauberstab hoch, drehte ihn und rief „\emph{Ma-ha-su!}“

Ein tieferes „dong“ ertönte, was bedeutete, dass er fast richtig getroffen hatte.

Harry steckte seinen Zauberstab ein, sprang zurück nach links, zog und feuerte noch einen roten Energieblitz ab.

Das hohe „bing“ was daraufhin ertönte, war eines der zufriedenstellendsten Geräusche, die er jemals in seinem Leben gehört hatte. Harry wollte triumphierend und so laut er konnte schreien: „ICH KANN ZAUBERN! FÜRCHTET MICH, NATURGESETZE, ICH KOMME EUCH ZU VERLETZEN!“

„\emph{Ma-ha-su!}“ Harrys Stimme war laut, aber über den stetigen Singsang aus ähnlichen Schreien von überall auf der Bühne kaum zu vernehmen.

„Genug“, sagte Professor Quirrells verstärkte Stimme. (Sie klang nicht laut. Sie sprach in normaler Lautstärke und befand sich direkt hinter der linken Schulter, egal wo man relativ zu Professor Quirrell stand.) „Ich sehe, dass jeder von euch nun mindestens ein Mal erfolgreich war.“ Die Zielkugeln wurden rot und begannen zur Decke hoch zu schweben.

Professor Quirrell stand auf dem erhöhten Podium in der Mitte der Bühne und stützte sich mit einer Hand etwas auf seinem Lehrertisch ab.

„Ich hatte euch erzählt“, sagte Professor Quirrell, „dass wir ein Spiel namens ‚Wer ist der gefährlichste Schüler im Raum?` spielen werden. Es gibt einen Schüler in diesem Raum, der den Simplen Sumerischen Schlagzauber schneller gemeistert hat als irgendjemand anders—“

Ach bla bla bla.

„— und anschließend sieben anderen Schülern geholfen hat, wofür sie die ersten sieben Quirrellpunkte verdient hat, die ich in eurem Jahrgang vergebe. Komm vor, Hermine Granger, es wird Zeit für das nächste Level des Spiels.“

Hermine Granger schritt mit einer Mischung aus Triumph und Besorgnis auf dem Gesicht vor. Die Ravenclaws sahen stolz zu, die Slytherins zornig und Harry einfach nur genervt. Harry war diesmal gut gewesen. Jetzt, wo sie mit einem Zauberspruch konfrontiert wurden, den jeder andere ebensowenig kannte, und wo Harry Adalbert Schwahfels \emph{Theorie der Magie} durchgelesen hatte, gehörte er vermutlich sogar zur besseren Hälfte der Klasse. Und Hermine \emph{war immer noch besser.}

Irgendwo tief in seinem Inneren bekam er Angst, dass Hermine einfach klüger als er sein könnte.

Doch erst einmal würde Harry seine Hoffnungen darauf stützen, dass Hermine mehr als nur die vorgegebenen Lehrbücher gelesen hatte und dass Adalbert Schwahfel ein begeisterungsloser Mistkerl war, der \emph{Theorie der Magie} geschrieben hatte, um damit einen Schulbeirat zu beeindrucken, der nicht viel von Elfjährigen hielt.

Hermine erreichte das Podium und trat herauf.

„Hermine Granger hat einen vollkommen fremden Zauberspruch in zwei Minuten gemeistert, fast eine Minute schneller als der Nächstbeste.“ Professor Quirrell drehte sich langsam auf der Stelle, um alle Schüler anzusehen, die zu ihnen hoch sahen. „Könnte Miss~Grangers Intelligenz sie zur gefährlichsten Schülerin im Klassenzimmer machen? Nun? Was denkt ihr?“

Im Moment schien niemand etwas zu denken. Selbst Harry war sich nicht sicher, was er sagen sollte.

„Dann wollen wir es mal herausfinden“, sagte Professor Quirrell. Er drehte sich wieder zu Hermine um und deutete auf den Rest der Klasse. „Wählen Sie einen der Schüler aus und feuern Sie den Simplen Schlagzauber auf ihn.“

Hermine erstarrte an Ort und Stelle.

„Nun aber“, sagte Professor Quirrell ruhig. „Sie haben diesen Spruch mehr als fünfzig Mal perfekt ausgeführt. Er hinterlässt keine bleibenden Schäden und tut nicht einmal besonders weh. Er schmerzt so sehr wie ein harter Schlag und hält nur wenige Sekunden an.“ Professor Quirrells Stimme wurde härter. „Dies ist ein ausdrücklicher Befehl ihres Lehrers, Miss~Granger. Wählen Sie ein Ziel und feuern Sie den Simplen Schlagzauber ab.“

Hermines Gesicht war angstverzerrt und ihr Zauberstab zitterte in ihrer Hand. Harrys eigene Finger umklammerten seinen eigenen Zauberstab und zitterten vor Mitgefühl. Obwohl er verstand, was Professor Quirrell versuchte. Obwohl er verstand, was Professor Quirrell ihnen beibringen wollte.

„Wenn Sie ihren Zauberstab \emph{nicht} auf jemanden richten und schießen, Miss~Granger, dann verlieren Sie einen Quirrellpunkt.“

Harry starrte Hermine an und wünschte sich, dass sie zu ihm sehen würde. Seine rechte Hand tippte sanft auf seine eigene Brust. \emph{Wähle mich, ich habe keine Angst …}

Hermines Zauberstab zuckte in ihrer Hand; dann entspannte sich ihr Gesicht und sie senkte den Zauberstab.

„Nein“, sagte Hermine Granger.

Ihre Stimme war ruhig und obwohl sie nicht laut klang, hörte jeder sie in der Stille.

„Dann muss ich Ihnen einen Punkt abziehen“, sagte Professor Quirrell. „Dies ist ein Test und Sie sind durchgefallen.“

Das ging ihr nahe. Harry konnte es sehen. Aber sie blieb mit erhobenem Kopf stehen.

Professor Quirrells Stimme war verständnisvoll und schien den ganzen Raum auszufüllen. „Dinge zu wissen, reicht nicht immer, Miss~Granger. Wenn Sie nichts austeilen oder einstecken können, was etwa so sehr weh tut wie ein angestoßener Zeh, dann können Sie sich nicht verteidigen und Sie werden in Verteidigung nicht bestehen. Bitte gehen Sie zu Ihren Klassenkameraden zurück.“

Hermine ging zurück zu den beisammen stehenden Ravenclaws. Ihr Gesichtsausdruck war ruhig und aus irgendeinem seltsamen Grund wollte Harry klatschen. Obwohl Professor Quirrell \emph{Recht} gehabt hatte.

„Nun gut“, sagte Professor Quirrell. „Es ist also klar, dass Hermine Granger nicht der gefährlichste Schüler im Raum ist. Wer, glaubt ihr, könnte tatsächlich die gefährlichste Person im Raum sein? Abgesehen von mir, natürlich.“

Ohne auch nur darüber nachzudenken drehte Harry sich, um zu den Slytherins zu sehen.

„Draco, vom führnehmen und gar alten Haus Malfoy“, sagte Professor Quirrell. „Es scheint, dass viele Ihrer Mitschüler zu Ihnen schauen. Treten Sie doch bitte vor.“

Draco tat dies mit einem gewissen Stolz in seinem Gebaren. Er trat auf das Podest und sah Professor Quirrell lächelnd an.

„Mr~Malfoy“, sagte Professor Quirrell. „Feuern Sie.“

Harry hätte versucht, ihn aufzuhalten, wenn er Zeit gehabt hätte, doch in einer einzigen nahtlosen Bewegung drehte sich Draco zu den Ravenclaws, erhob seinen Zauberstab und sagte „Mahasu!“, als ob es eine einzige Silbe wäre, und Hermine sagte „Aua!“ und dann war es vorbei.

„Gut getroffen“, sagte Professor Quirrell. „Zwei Quirrellpunkte für Sie. Aber sagen Sie, warum haben Sie auf Miss~Granger gezielt?“

Es war einen Moment still.

Schließlich sagte Draco: „Weil sie am meisten herausstach.“

Professor Quirrells Lippen verzogen sich zu einem dünnen Lächeln. „Und das ist der wahre Grund, warum Draco Malfoy gefährlich ist. Hätte er jemand anders ausgesucht, würde der Schüler ihm viel eher nachtragen, dass er ihn ausgesucht hat, und Mr~Malfoy hätte vermutlich einen Feind. Und während Mr~Malfoy irgendeinen anderen Grund angeben könnte, aus dem er diesen Schüler gewählt hat, hätte ihm das keine Vorteile gebracht und einige von euch verprellt, während andere ihm ohnehin zujubeln, egal was er sagt. Das bedeutet, dass Mr~Malfoy gefährlich ist, weil er weiß, wen er angreifen kann und wen nicht; wie man Allianzen aufbaut und wie man vermeidet, sich Feinde zu machen. Zwei weitere Quirrellpunkte für Sie, Mr~Malfoy. Und da Sie einen für Slytherin beispielhaften Wert demonstriert haben, hat das Haus Slytherin, denke ich, auch einen Punkt verdient. Sie können zu Ihren Freunden zurückkehren.“

Draco verbeugte sich etwas und ging zurück zu der Gruppe von Slytherins. Einige von ihnen begannen zu klatschen, doch Professor Quirrell machte eine schneidende Geste und es wurde wieder still.

„Es mag so aussehen, als ob unser Spiel vorbei ist“, sagte Professor Quirrell. „Und doch gibt es einen einzigen Schüler in diesem Klassenzimmer, der gefährlicher ist als der Spross der Malfoys.“

Und \emph{nun} schienen viele Leute aus irgendeinem Grund ihn anzusehen.

„Harry Potter. Treten Sie bitte vor.“

Er ahnte nichts Gutes.

Harry ging widerstrebend nach vorne, Professor Quirrell entgegen, der dort erhöht auf seinem Podest stand und sich immer noch leicht am Lehrertisch anlehnte.

Die Nervosität, die er vor seinen Mitschülern empfand, schien Harrys geistige Fähigkeiten zu schärfen, während er dem Podest näher kam und die Möglichkeiten durchging, wie Professor Quirrell wohl Harrys Gefährlichkeit demonstrieren wollte. Würde er ihn bitten, einen Zauberspruch zu sprechen? Einen Dunklen Lord zu besiegen?

Seine angebliche Immunität gegenüber dem Todesfluch zu demonstrieren? Sicherlich war Professor Quirrell klug genug, \emph{das} nicht zu tun …

Harry blieb ein ganzes Stück vor dem Podest stehen und Professor Quirrell bat ihn nicht, näher heranzutreten.

„Die Ironie ist“, sagte Professor Quirrell, „dass ihr die richtige Person aus dem vollkommen falschen Grund anschaut. Ihr glaubt“, Professor Quirrells Lippen verzogen sich, „dass Harry Potter den Dunklen Lord besiegt hat und folglich sehr gefährlich sein muss. Pah. Er war ein Jahr alt. Welche Laune des Schicksals den Dunklen Lord auch getötet haben mag, sie hatte vermutlich wenig mit Mr~Potters Kampfkraft zu tun. Doch nachdem ich das Gerücht hörte, dass ein Ravenclaw sich fünf älteren Slytherins entgegengestellt hat, habe ich mehrere Augenzeugen befragt und kam zu dem Schluss, dass Harry Potter mein gefährlichster Schüler ist.“

Ein Adrenalinstoß durchfuhr Harry und er stellte sich aufrechter hin. Er wusste nicht, zu welchem Schluss Professor Quirrell gekommen war, aber es konnte kein guter gewesen sein.

„Ähm, Professor Quirrell—“, begann Harry.

Professor Quirrell sah amüsiert aus. „Sie glauben, dass ich auf eine falsche Antwort gestoßen bin, nicht wahr, Mr~Potter? Sie werden lernen, dass Sie von \emph{mir} mehr erwarten sollten.“ Professor Quirrell, der sich an den Tisch gelehnt hatte, stand auf. „Mr~Potter, alle Gegenstände haben einen gewöhnlichen Zweck. Nennen Sie mir zehn ungewöhnliche Zwecke, die Gegenstände in diesem Raum in einem Kampf erfüllen könnten!“

Einen Moment lang war Harry sprachlos—geschockt, dass jemand ihn verstanden hatte.

Dann sprudelten die Ideen aus ihm raus.

„Hier sind Tische, die schwer genug sind, dass sie jemanden umbringen können, wenn sie aus großer Höhe fallen. Hier sind Stühle mit Metallbeinen, die jemanden pfählen können, wenn sie mit genug Kraft gestoßen werden. Die Luft im Raum würde durch ihre Abwesenheit tödlich werden, da Menschen im Vakuum sterben, oder sie kann als Übertragungsweg für giftige Gase dienen.“

Harry holte kurz Luft und in diese Pause hinein sagte Professor Quirrell:

„Das sind drei. Sie brauchen zehn. Der Rest der Klasse glaubt, dass Sie schon den gesamten Inhalt des Klassenzimmers aufgebraucht haben.“

\emph{„Ha!} Der Boden kann entfernt werden, um eine stachelbesetzte Grube zu schaffen, in die Menschen reinfallen würden. Die Decke kann auf jemanden drauf fallen. Die Wände können als Rohmaterial dienen, das in eine Menge tödliche Dinge verwandelt werden kann—Messer zum Beispiel.“

„Das sind sechs. Aber so langsam sind Sie wohl am Ende Ihrer Weisheit angelangt?“

„Ich habe gerade erst angefangen! Sehen Sie nur all die Menschen an! Den Feind von einem Gryffindor angreifen lassen, ist natürlich ein \emph{gewöhnlicher} Zweck—“

„Den lasse ich nicht gelten.“

„— aber ihr Blut kann genutzt werden um jemanden zu ertränken. Ravenclaws sind für ihre Köpfe bekannt, aber ihre inneren Organe können auf dem Schwarzmarkt für genug Geld verkauft werden, um damit einen Attentäter anzuheuern. Slytherins sind nicht nur als Attentäter nützlich, sie können auch ausreichend schnell geworfen werden, um einen Feind zu zerquetschen. Und Hufflepuffs können nicht nur hart arbeiten, sondern enthalten auch Knochen, die entfernt, angespitzt und anschließend genutzt werden können um jemanden zu erstechen.“

Inzwischen starrten die anderen Schüler Harry entsetzt an. Selbst die Slytherins sahen schockiert aus.

„Das sind zehn, obwohl ich gnädig bin, wenn ich den Ravenclaw mitzähle. Als Bonus nun einen Quirrellpunkt für jede Verwendung eines Gegenstands in diesem Raum, den Sie noch nicht genannt haben.“ Professor Quirrell schenkte Harry ein kameradschaftliches Lächeln. „Die anderen Schüler denken, dass Sie jetzt in die Enge gedrängt sind, da Sie außer den Zielen alles genannt haben und Sie keine Ahnung haben, was man mit denen machen könnte.“

„Ach was! Ich habe alle Menschen aufgezählt, aber nicht meinen Umhang, der genutzt werden kann, um einen Gegner zu ersticken, wenn man ihn oft genug um dessen Kopf wickelt. Oder Hermine Grangers Umhang, der in Streifen gerissen, zu einem Seil geflochten und anschließend dazu verwendet werden kann, jemanden damit aufzuhängen. Oder Draco Malfoys Umhang, der genutzt werden kann um ein Feuer zu legen. Oder—“

„Drei Punkte“, sagte Professor Quirrell, „und keine weitere Kleidung mehr.“

„Mein Zauberstab kann durch das Auge ins Gehirn eines Gegners gestoßen werden.“ Jemand machte ein ersticktes, erschrockenes Geräusch.

„Vier Punkte, keine Zauberstäbe mehr.“

„Meine Armbanduhr könnte jemanden ersticken, wenn sie dessen Kehle runtergerammt wird—“

„Fünf Punkte und Schluss.“

„Hm“, sagte Harry. „Zehn Quirrellpunkte sind ein Hauspunkt, nicht wahr? Sie hätten mich fortfahren lassen sollen, bis ich den Hauspokal gewonnen hätte, ich habe nicht einmal damit angefangen, ungewöhnliche Verwendungszwecke für alles, was ich in meinen Taschen habe, aufzuzählen.“ Oder den Eselsfellbeutel selbst; über den Zeitumkehrer und den Tarnumhang durfte er zwar nicht sprechen, aber es musste \emph{irgendwas} geben, was er zu diesen roten Kugeln sagen könnte …

\emph{„Genug,} Mr~Potter. Nun, glaubt ihr, dass ihr wisst, was Mr~Potter zum gefährlichsten Schüler im Klassenzimmer macht?“

Ein zustimmendes Murmeln war zu hören.

„Sagt es bitte laut. Terry Boot, was macht deinen Mitschüler gefährlich?“

„Ähm…naja…er ist kreativ?“

\emph{„Falsch!“,} brüllte Professor Quirrell. Er schlug mit der Faust auf den Tisch und das magisch verstärkte Geräusch ließ alle zusammenzucken. „Mr~Potters Ideen waren allesamt vollkommen nutzlos!“

Harry starrte ihn überrascht an.

„Den Boden entfernen, um eine stachelbesetzte Grube zu schaffen? Lächerlich! Im Kampf haben Sie nicht genügend Vorbereitungszeit und selbst wenn Sie die hätten, gäbe es hundert bessere Vorschläge. Teile der Wand verwandeln? Mr~Potter beherrscht keine Verwandlung! Mr~Potter hatte genau eine Idee, die er hier und jetzt anwenden könnte, ohne aufwändige Vorbereitung oder die Mithilfe seines Gegners oder Magie, die er gar nicht kennt. Diese Idee war, seinen Zauberstab einem Gegner durchs Auge zu rammen. Was eher seinen Zauberstab beschädigen als seinen Gegner töten würde! Kurz gesagt, Mr~Potter, ich fürchte, dass ihre Vorschläge allesamt miserabel waren.“

„Was?“, sagte Harry beleidigt. „Sie haben um ungewöhnliche Ideen \emph{gebeten}, nicht um praktikable! Also habe ich unkonventionelle Lösungen gesucht! Wie würden \emph{Sie} irgendwas in diesem Klassenzimmer nutzen, um jemanden zu töten?“

Professor Quirrells Gesichtsausdruck war tadelnd, doch in seinen Augen blitzte ein Lächeln auf. „Mr~Potter, ich habe nie gesagt, dass Sie \emph{töten} sollten. In manchen Situationen ist es sinnvoll, seinen Gegner am Leben zu lassen und in einem Hogwarts-Klassenzimmer treten zumeist solche Situationen auf. Aber um Ihre Frage zu beantworten, man könnte ihn mit einer Stuhlkante in den Nacken schlagen.“

Einige Slytherins lachten, doch sie lachten mit Harry, nicht über ihn.

Alle anderen sahen recht entsetzt aus.

„Aber Mr~Potter hat nun demonstriert, warum er der gefährlichste Schüler im Klassenzimmer ist. Ich habe nach ungewöhnlichen Verwendungen von Gegenständen in diesem Raum während einem Kampf gefragt. Mr~Potter hätte vorschlagen können, mit einem Tisch einen Fluch abzublocken oder einen näherkommenden Gegner mit einem Stuhl zum Fallen zu bringen oder Stoff um seinen Arm zu wickeln, um ein Schild zu improvisieren. Stattdessen war jeder von Mr~Potters Vorschlägen offensiv statt defensiv und entweder tödlich oder möglicherweise tödlich.“

Was? Moment, das konnte nicht stimmen…Harry verspürte einen plötzlichen Schwindel, während er versuchte sich zu erinnern, was genau er vorgeschlagen hatte; sicherlich musste es ein Gegenbeispiel geben …

„Und deswegen“, sagte Professor Quirrell, „waren Mr~Potters Ideen so seltsam und nutzlos—weil er Ideen an den Haaren herbeiziehen musste, um seinen Standard \emph{den Gegner zu töten} zu erfüllen. Für ihn war jede Idee, die nicht so weit ging, keine Erwägung wert. Das deutet auf einen Wesenszug hin, den wir als \emph{Absicht zu töten} bezeichnen könnten. Ich habe ihn. Harry Potter hat ihn, daher konnte er fünf ältere Slytherins erfolgreich gegenüber treten. Draco Malfoy hat ihn nicht—noch nicht. Mr~Malfoy würde kaum davor zurückschrecken, von gewöhnlichem Mord zu sprechen, doch selbst er war schockiert—ja, das waren Sie, Mr~Malfoy, ich habe ihr Gesicht im Blick gehabt—als Mr~Potter beschrieb, wie die Körper seiner Klassenkameraden als Rohmaterial genutzt werden könnten. Ihr habt Sittenrichter in euren Köpfen, die euch vor solchen Gedanken zurückschrecken lassen. Mr~Potter denkt \emph{ausschließlich} daran, den Gegner zu töten, ihm ist jedes Mittel dafür recht, er schreckt nicht zurück, seine Sittenrichter sind still. Obwohl sein jugendliches Genie so undiszipliniert und unpraktisch denkt, dass es nutzlos ist, macht seine \emph{Absicht zu töten} Harry Potter zum gefährlichsten Schüler im Klassenzimmer. Ein letzter Punkt für ihn—nein, lieber einen Punkt für Ravenclaw—für diese unersetzliche Fähigkeit eines wahren Kampfmagiers.“

Harrys Mund stand weit offen, er bekam vor Schreck kein Wort heraus, während er hektisch überlegte, was er darauf erwidern könnte. \emph{Das widerstrebt allem, wofür ich stehe, vollkommen!}

Aber er konnte sehen, dass die anderen Schüler anfingen, es zu glauben. Harrys Gehirn ging mögliche Widersprüche durch und fand nichts, was gegen die Ehrfurcht gebietende Stimme von Professor Quirrell ankommen konnte. Das Beste, was Harry einfiel, war „Ich bin kein Psychopath, ich bin bloß sehr kreativ“, und das klang irgendwie unheilvoll. Er musste etwas Unerwartetes sagen, etwas, was die Leute zum Nachdenken bringen würde—

„Und nun“, sagte Professor Quirrell, „Mr~Potter, feuern Sie.“

Natürlich passierte nichts.

„Nun gut“, sagte Professor Quirrell. Er seufzte. „Ich sehe, wir müssen alle irgendwo anfangen. Mr~Potter, wählen Sie irgendeinen Schüler aus, auf den Sie einen Simplen Schlagzauber sprechen wollen. Sie \emph{werden} das tun, bevor ich die heutige Unterrichtsstunde beende. Wenn nicht, werde ich anfangen Hauspunkte abzuziehen und solange weitermachen, bis Sie es tun.“

Harry hob vorsichtig seinen Zauberstab. Er musste das tun, sonst hätte Professor Quirrell gleich anfangen können, ihm Hauspunkte abzuziehen.

Schleichend langsam drehte Harry sich zu den Slytherins.

Und Harrys Blick traf Dracos.

Draco Malfoy sah kein bisschen ängstlich aus. Der blondhaarige Junge gab kein sichtbares Zeichen der Zustimmung, so wie Harry es Hermine gegeben hatte, aber man konnte das wohl kaum von ihm erwarten. Die anderen Slytherins würden das sehr seltsam finden.

„Warum zögern Sie?“, sagte Professor Quirrell. „Es gibt doch nur eine offensichtliche Wahl.“

„Ja“, sagte Harry. „Nur eine \emph{offensichtliche} Wahl.“

Harry drehte den Zauberstab und sagte „\emph{Ma-ha-su!}“

Im Klassenzimmer herrschte vollkommene Stille.

Harry schüttelte seinen linken Arm, um das stechende Gefühl loszuwerden.

Immer noch herrschte Stille.

Schließlich seufzte Professor Quirrell. „Ja, sehr geschickt. Aber ich wollte eine Lektion erteilen und Sie sind dem ausgewichen. Ein Punkt Abzug für Ravenclaw, weil Sie Ihre eigene Intelligenz zu Lasten des eigentlichen Ziels zur Schau gestellt haben. Die Stunde ist beendet.“

Und bevor irgendjemand anderes etwas sagen konnte, rief Harry: „Nur ein Scherz! RAVENCLAW!“

Einen kurzen Moment lang war es still, man hörte die Schüler nachdenken, und dann begann ein Gemurmel und entwickelte sich schnell zu lauten Gesprächen.

Harry wandte sich Professor Quirrell zu, sie beide mussten miteinander reden—

Quirrell war zusammengesunken und schleppte sich zurück zu seinem Stuhl.

Nein. Inakzeptabel. Sie mussten \emph{dringend} miteinander reden. Zum Teufel mit dieser Zombie-Nummer, Professor Quirrell würde vermutlich aufmachen, wenn Harry ihn einige Male anstupste. Harry schritt vorwärts—

\emph{FALSCH

NICHT

SCHLECHTE IDEE}

Harry schwankte und blieb mit flauem Magen stehen.

Und dann stürzte sich ein Haufen Ravenclaws auf ihn und die Diskussionen begannen.

