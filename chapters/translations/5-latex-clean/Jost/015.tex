

\hypertarget{gewissenhaftigkeit}{% \section{15. Gewissenhaftigkeit}\label{gewissenhaftigkeit}}

\later

\emph{„Ich bin mir sicher, dass ich irgendwo die Zeit finden werde.“}

\later

\emph{„Frigideiro!“}

Harry tauchte einen Finger in das Wasserglas auf seinem Tisch. Es hätte kühl sein sollen. Doch lauwarm war es gewesen und lauwarm ist es geblieben. Schon wieder.

Harry fühlte sich sehr, sehr betrogen.

Im Haus der Verres fanden sich hunderte Fantasy-Romane, Harry hatte etliche davon gelesen. Und es begann so auszusehen, als ob er eine mysteriöse dunkle Seite hatte. Nachdem das Wasserglas einige Male nicht kooperiert hatte, blickte Harry sich im Zauberkunstklassenzimmer um, um sicherzugehen, dass niemand zusah. Er atmete tief durch, konzentrierte sich und machte sich wütend. Dachte an die Slytherins, die Neville drangsaliert hatten, und an das Spiel, wo jemand dir deine Bücher aus den Händen schlägt, wann immer du versuchst sie aufzuheben. Dachte daran, was Draco über das zehnjährige Lovegood-Mädchen gesagt hatte, und wie der Zaubergamot wirklich funktionierte …

Und der Zorn war ihm ins Blut übergegangen, er hatte seinen Zauberstab in der vor Hass zitternden Hand gehalten und mit kalter Stimme \emph{„Frigideiro!“} gesagt und absolut nichts war passiert.

Harry fühlte sich \emph{betrogen.} Er wollte jemandem schreiben und \emph{Erstattung} für seine dunkle Seite fordern, da sie zweifelsohne unwiderstehliche magische Kräfte haben \emph{sollte,} aber offensichtlich\emph{defekt} war.

\emph{„Frigideiro!“,} sagte Hermine am Tisch neben ihm. Ihr Wasser war nun festes Eis und am Rand des Glases bildeten sich weiße Kristalle. Sie schien ganz auf ihre eigene Arbeit konzentriert zu sein und gar nicht zu merken, wie die anderen Schüler sie mit hasserfüllten Augen anstarrten, was entweder (a) schrecklich leichtsinnig von ihr war, oder (b) eine perfekt einstudierte Darbietung, die an ein Kunstwerk grenzte.

„Oh, \emph{sehr} gut, Miss~Granger!“, quiekte Filius Flitwick, ihr Zauberkunstlehrer und Hauslehrer von Ravenclaw, ein winzig kleiner Mann, dem man nicht ansah, dass er einst ein glänzender Duellkämpfer gewesen war. „Exzellent! Erstaunlich!“

Harry hatte erwartet, dass er schlimmstenfalls nach Hermine der Zweitbeste war. Harry wäre es natürlich lieber gewesen, wenn \emph{sie ihm} auf den Fersen wäre, aber er hätte es auch andersrum akzeptiert.

Seit Montag war Harry dabei, der Schlechteste in der Klasse zu werden -- eine Position, um die er mit allen anderen bei Muggeln aufgewachsenen Schülern außer Hermine wetteiferte. Die war stattdessen ganz alleine und unbestritten an der Spitze, die Arme.

Professor Flitwick stand bei einer Muggelgeborenen am Tisch und korrigierte leise ihre Zauberstabhaltung.

Harry sah zu Hermine rüber. Er atmete tief durch. Das musste wohl ihre Rolle in dieser Geschichte sein … „Hermine?“, sagte Harry zögerlich. „Hast du irgendeine Ahnung, was ich falsch machen könnte?“

Hermines Augen strahlten voll schrecklicher Hilfsbereitschaft und etwas in Harrys Hinterkopf fing an, gedemütigt zu schreien.

Fünf Minuten später wirkte Harrys Wasser merklich kühler. Hermine hatte ihn ein paar Mal gelobt, ihm gesagt, er solle es beim nächsten Mal sorgfältiger aussprechen, und war entschwunden, um jemand anderem zu helfen.

Professor Flitwick hatte ihr einen Hauspunkt gegeben, weil sie ihm geholfen hatte.

Harry knirschte so fest mit den Zähnen, dass ihm die Kiefer wehtaten, und das tat seiner Aussprache nicht gut.

\emph{Es ist mir egal, dass es unlauterer Wettbewerb ist. Ich weiß genau, was ich mit den zwei zusätzlichen Stunden am Tag anfange. Ich werde in meinem Koffer sitzen und lernen, bis ich mit Hermine Granger mithalten kann.}

\later

„Verwandlung zählt zur kompliziertesten und gefährlichsten Magie, die Sie in Hogwarts lernen werden“, sagte Professor McGonagall. Auf dem Gesicht der strengen alten Hexe fand sich nicht die kleinste Spur eines Lächelns. „Jeder, der in meinem Unterricht Unfug baut, wird gehen und nicht zurückkehren. Sie wurden gewarnt.“

Ihr Zauberstab senkte sich und berührte den Tisch, der sich nahtlos in ein Schwein verformte. Einige Muggelgeborene stießen überraschte Rufe aus. Das Schwein sah sich um, grunzte verwirrt und wurde dann wieder zu einem Tisch.

McGonagall blickte zwischen den Schülern hin und her. Ihre Augen blieben an einer Person hängen.

„Mr~Potter“, sagte Professor McGonagall. „Sie haben Ihre Schulbücher erst vor wenigen Tagen erhalten. Haben Sie angefangen, Ihr Verwandlungs-Lehrbuch zu lesen?“

„Nein, entschuldigen Sie, Professor“, sagte Harry.

„Sie brauchen sich nicht zu entschuldigen, Mr~Potter. Wenn Sie es vorher hätten lesen sollen, hätte man es Ihnen mitgeteilt.“ McGonagalls Finger klopften auf den Tisch vor ihr. „Mr~Potter, möchten Sie raten, ob dies ein Tisch ist, den ich kurzzeitig in ein Schwein verwandelt habe, oder ob es anfangs ein Schwein war und ich die Verwandlung kurzzeitig rückgängig gemacht habe? Wenn Sie das erste Kapitel des Lehrbuchs gelesen hätten, wüssten Sie das.“

Harry zog die Augenbrauen zusammen. „Ich würde vermuten, dass es einfacher wäre, mit einem Schwein anzufangen, denn wenn es anfangs ein Tisch war, wüsste es womöglich nicht, wie man aufrecht steht.“

Professor McGonagall schüttelte den Kopf. „Ich mache Ihnen keine Vorwürfe, Mr~Potter. Aber die richtige Antwort ist, dass Sie im Verwandlungsunterricht \emph{nicht} raten möchten. Falsche Antworten führen zu einer außerordentlich strengen Bewertung, während ich bei nicht beantworteten Fragen viel Nachsicht walten lasse. Sie müssen lernen zu wissen, was Sie nicht wissen. Wenn ich Ihnen irgendeine Frage stelle, egal wie offensichtlich oder leicht sie sein mag, und Sie ‚Ich bin mir nicht sicher` antworten, werde ich es Ihnen nicht vorhalten und jeder der lacht wird Hauspunkte verlieren. Können Sie mir sagen, warum diese Regel besteht, Mr~Potter?“

\emph{Weil ein einziger Fehler bei einer Verwandlung unglaublich gefährlich sein kann.}

„Nein.“

„Richtig. Verwandlung ist sogar noch gefährlich als Apparieren, was nicht vor dem sechsten Schuljahr gelehrt wird. Leider müssen Sie Verwandlung im jungen Alter lernen und üben, um es später bestmöglich zu beherrschen. Dies ist also ein gefährliches Fach und Sie sollten Angst davor haben, irgendwelche Fehler zu machen, denn noch keiner meiner Schüler hat jemals eine bleibende Verletzung davongetragen und ich wäre \emph{außerordentlich verärgert,} wenn jemand von Ihnen mir das \emph{verdirbt.“}

Einige Schüler schluckten.

Professor McGonagall stand auf und ging zur weißen Holztafel an der Wand hinter ihrem Schreibtisch. „Es gibt viele Dinge, die Verwandlung gefährlich machen, doch eines sticht zwischen allen anderen hervor.“ Sie hatte plötzlich einen Stift in der Hand und malte rote Buchstaben, die sie dann mit dem selben Stift blau unterstrich:

\uline{VERWANDLUNG IST NICHT DAUERHAFT!}

„Verwandlung ist nicht dauerhaft!“, sagte McGonagall. „Verwandlung ist nicht dauerhaft! Verwandlung ist nicht dauerhaft! Mr~Potter, angenommen, ein Schüler verwandelt ein Stück Holz in ein Glas Wasser und Sie trinken es. Was, glauben Sie, würde mit Ihnen passieren, wenn die Verwandlung nachlässt?“ Eine kurze Pause herrschte. „Entschuldigen Sie, ich sollte das nicht von Ihnen verlangen, Mr~Potter, ich hatte vergessen, dass Sie mit einer außergewöhnlich pessimistischen Vorstellungskraft gesegnet sind --“

„Schon gut“, sagte Harry und holte tief Luft. „Also, meine erste Antwort ist, dass ich es nicht \emph{weiß“} -- McGonagall nickte -- „aber ich könnte mir \emph{vorstellen,} dass dann … Holz in meinem Magen wäre und in meinen Blutbahnen, und wenn ein Teil des Wassers in meine Zellen eingelagert wird -- würde es dann zu Zellstoff werden oder zu festem Holz oder …“ Harrys Magieverständnis verließ ihn. Er konnte nicht verstehen, wie das Holz überhaupt in Wasser umgewandelt wurde, also konnte er auch nicht verstehen, was geschehen würde, nachdem die Wassermoleküle durch die gewöhnliche Brownsche Molekularbewegung durchmischt wurden, die Magie nachließ und die Verwandlung rückgängig gemacht wurde.

McGonagalls Gesicht war regungslos. „Wie Mr~Potter richtig gefolgert hat, würde er außerordentlich krank werden und bräuchte sofort medizinische Behandlung. Bitte schlagen Sie Seite fünf im Lehrbuch auf.“

Obwohl das bewegte Bild keinen Ton von sich gab, konnte man sehen, dass die Frau mit schrecklich verfärbter Haut schrie.

„Der Verbrecher, der Gold in Wein verwandelte und es dieser Frau zu trinken gab -- ‚um seine Schulden zu begleichen`, wie er sagte -- wurde zur Strafe zehn Jahre in Askaban eingesperrt. Bitte schlagen Sie Seite sechs auf. Das ist ein Dementor. Es sind die Wächter von Askaban. Sie saugen die Magie, das Leben und alle glücklichen Gedanken aus den Insaßen heraus. Das Bild auf Seite sieben zeigt den Verbrecher zehn Jahre später, nach seiner Entlassung. Wie Sie sehen, ist er tot -- ja, Mr~Potter?“

„Professor“, sagte Harry, „im schlimmsten Fall, wenn so etwas passiert, kann man die Verwandlung irgendwie \emph{aufrechterhalten?“}

„Nein“, sagte Professor McGonagall schlicht. „Eine Verwandlung aufrechtzuerhalten zehrt ständig an Ihrer Magie, abhängig von der Größe des Zielobjekts. Und Sie müssten das Zielobjekt alle paar Stunden erneut mit dem Zauberstab antippen, was in solchen Fällen unmöglich wäre. Ein solches Desaster lässt sich \emph{nicht wiedergutmachen!“}

Professor McGonagall lehnte sich vor. Ihr Gesicht war versteinert. „Sie werden niemals und unter keinen Umständen irgendetwas in eine Flüssigkeit oder ein Gas verwandeln. Kein Wasser, keine Luft. Nichts Wasser-ähnliches, nichts Luft-ähnliches. Selbst wenn es nicht zum Trinken gedacht ist. Flüssigkeiten \emph{verdunsten,} kleine Mengen davon geraten in die Luft. Sie werden nichts in einen Brennstoff verwandeln. Dabei würde Rauch entstehen, den jemand einatmen könnte! Sie werden niemals irgendwas in eine Sache verwandeln, die auf irgendeine vorstellbare Weise in irgendeinen Körper gelangen könnte. Kein Essen. Nichts was \emph{nach Essen aussieht.} Nicht mal als lustiger kleiner Scherz, während Sie vorhaben, dem Opfer alles zu erzählen, bevor es tatsächlich reinbeißt. Sie werden das niemals tun. Punkt. Weder in diesem Klassenzimmer noch außerhalb noch \emph{irgendwo anders.} Hat das \emph{jeder einzelne Schüler} vollkommen verstanden?“

„Ja“, sagten Harry, Hermine und einige andere. Die anderen Schüler schienen sprachlos zu sein.

\emph{„Hat das jeder einzelne Schüler vollkommen verstanden?“}

„Ja“, sagten oder murmelten oder flüsterten sie.

„Wenn Sie irgendeine dieser Regeln brechen, werden Sie in Ihrer Schulzeit in Hogwarts nicht weiter Verwandlung lernen. Sprechen Sie mir nach: Ich werde niemals irgendetwas in eine Flüssigkeit oder ein Gas verwandeln.“

„Ich werde niemals irgendetwas in eine Flüssigkeit oder ein Gas verwandeln“, antworteten die Schüler durcheinander.

„Nochmal! Lauter! Ich werde niemals irgendetwas in eine Flüssigkeit oder ein Gas verwandeln.“

„Ich werde niemals irgendetwas in eine Flüssigkeit oder ein Gas verwandeln.“

„Ich werde niemals irgendetwas in eine Sache verwandeln, die nach Essen aussieht oder irgendwie sonst in einen menschlichen Körper gelangen könnte.“

„Ich werde niemals irgendetwas in einen Brennstoff verwandeln, weil dabei Rauch entstehen könnte.“

„Sie werden niemals irgendetwas in etwas verwandeln, das wie Geld aussieht, nicht einmal wie Muggelgeld“, sagte Professor McGonagall. „Die Kobolde können herausfinden, wer das getan hat. Nach geltendem Recht sind die Kobolde in einem ständigen \emph{Kriegszustand} mit allen magischen Fälschern. Sie werden keine Auroren schicken. Sie werden eine Armee schicken.“

„Ich werde niemals irgendetwas in etwas verwandeln, das wie Geld aussieht“, riefen die Schüler im Chor.

„Und \emph{vor allem“,} sagte Professor McGonagall, „werden Sie kein lebendiges Wesen verwandeln, \emph{erst recht nicht sich selbst.} Sie würden schwer erkranken und möglicherweise sterben, je nach der Verwandlung und ihrer Dauer.“ Professor McGonagall hielt kurz inne. „Mr~Potter meldet sich, weil er eine Animagus-Verwandlung gesehen hat -- nämlich einen Menschen, der sich in eine Katze und wieder zurück verwandelte. Doch eine Animagus-Verwandlung ist keine \emph{freie} Verwandlung.“

Professor McGonagall nahm ein kleines Stück Holz aus ihrer Tasche. Durch eine Berührung mit ihrem Zauberstab wurde es zu einer Glaskugel. Dann sagte sie \emph{„Crystferrium!“} und die Glaskugel wurde zu einer Stahlkugel. Sie berührte diese ein letztes Mal mit dem Zauberstab und die Stahlkugel wurde wieder zu einem Stück Holz. \emph{„Crystferrium} verwandelt einen Gegenstand aus reinem Glas in einen ähnlich geformten Gegenstand aus reinem Stahl. Umgekehrt geht es nicht und einen Tisch in ein Schwein zu verwandeln geht damit auch nicht. Mit der allgemeinsten Form der Verwandlung -- freie Verwandlung, die Sie hier lernen werden -- kann jeder Gegenstand in jede Zielform umgewandelt werden, zumindest soweit es die äußere Form betrifft. Aus diesem Grund muss freie Verwandlung wortlos geschehen; sonst bräuchte man einen eigenen Zauberspruch für jede Kombination aus Gegenstand und Zielform.“

Professor McGonagall sah ihre Schüler scharf an. \emph{„Manche} Lehrer fangen mit Verwandlungszaubern an und gehen erst später zu freier Verwandlung über. Ja, das wäre anfangs einfacher. Aber dabei könnten Sie auch schlechte Angewohnheiten annehmen, die Sie später behindern. Sie werden hier \emph{von Anfang an} freie Verwandlung lernen und das erfordert, dass Sie den Zauber wortlos durchführen, indem Sie den Gegenstand, die Zielform und die Verwandlung in Ihrem Kopf vereinen.“

„Und um Mr~Potters Frage zu beantworten“, fuhr Professor McGonagall fort, „es ist die \emph{freie} Verwandlung, die Sie niemals auf ein lebendes Wesen anwenden dürfen. Es gibt Zaubersprüche und Tränke, die Lebewesen sicher und umkehrbar, aber nur in ganz bestimmter Weise verwandeln können. Ein Animagus beispielsweise, dem eine Gliedmaße fehlt, wird diese auch nach der Verwandlung nicht haben. Freie Verwandlung ist \emph{nicht} sicher. Ihr Körper würde sich verändern, während er verwandelt ist -- Atmen beispielsweise gibt ständig etwas Materie an die Atmosphäre ab. Wenn die Verwandlung abklingt und Ihr Körper versucht, zu seiner \emph{ursprünglichen} Form zurückzugelangen, dann würde das nicht ganz gelingen. Wenn Sie mit dem Zauberstab auf Ihren Körper zeigen und sich goldenes Haar vorstellen, werden Ihnen danach die Haare ausfallen. Wenn Sie sich reinere Haut vorstellen, steht Ihnen ein langer Aufenthalt in St. Mungos bevor. Und wenn Sie sich in den Körper eines Erwachsenen verwandeln, dann werden Sie sterben, sobald die Verwandlung abklingt.“

Das erklärte, warum er dicke Jungen gesehen hatte, oder Mädchen, die nicht vollkommen perfekt aussahen. Oder auch alte Leute. Das würde nicht geschehen, wenn man sich einfach jeden Morgen verwandeln könnte … Harry hob die Hand und versuchte, Professor McGonagall mit den Augen ein Zeichen zu geben.

\emph{„Ja,} Mr~Potter?“

„Ist es möglich, ein Lebewesen in einen gleichbleibenden Gegenstand wie eine Münze -- nein, entschuldigen Sie, das tut mir furchtbar Leid -- sagen wir eine Stahlkugel zu verwandeln?“

Professor McGonagall schüttelte den Kopf. „Mr~Potter, selbst leblose Objekte verändern sich im Laufe der Zeit ein kleines bisschen. Man könnte anschließend keine Veränderung an Ihrem Körper feststellen und die erste Minute lang würden Sie nichts Schlimmes spüren. Doch nach einer Stunde wären Sie sehr krank und einen Tag später wären Sie tot.“

„Ähm, entschuldigen Sie, also, wenn ich das erste Kapitel gelesen hätte, dann hätte ich \emph{erraten} können, dass der Tisch ursprünglich ein Tisch war und kein Schwein“, sagte Harry, „aber nur wenn ich \emph{außerdem} voraussetze, dass Sie das Schwein nicht umbringen möchten, was sehr wahrscheinlich \emph{scheinen} mag, aber --“

„Ich sehe schon, dass es ein endloser Quell der Freude sein wird, Ihre Aufsätze zu benoten, Mr~Potter. Aber darf ich Sie bitten, mit den weiteren Fragen bis ans Ende der Stunde zu warten?“

„Keine weiteren Fragen, Professor.“

„Dann sprechen Sie mir alle nach“, sagte Professor McGonagall. „Ich werde nie versuchen, ein Lebewesen zu verwandeln, erst recht nicht mich selbst, es sei denn, ich werde ausdrücklich aufgefordert, dafür einen besonderen Zauberspruch oder Trank zu verwenden.“

„Falls ich mir nicht sicher bin, ob eine Verwandlung sicher ist, werde ich sie nicht ausprobieren, bevor ich Professor McGonagall oder Professor Flitwick oder Professor Snape oder Schulleiter Dumbledore gefragt habe, welche die einzigen anerkannten Experten für Verwandlung auf Hogwarts sind. Einen anderen Schüler zu fragen reicht \emph{nicht} aus, selbst wenn dieser behauptet, einem Lehrer die selbe Frage gestellt zu haben.“

„Selbst wenn der aktuelle Verteidigungslehrer mir sagt, dass eine Verwandlung sicher ist und selbst wenn ich sehe, wie der Verteidigungslehrer sie durchführt und nichts Schlimmes passiert, werde ich sie nicht selbst ausprobieren.“

„Ich habe das uneingeschränkte Recht, eine Verwandlung nicht durchzuführen, wenn ich ein winziges bisschen nervös bin. Da nicht einmal der Schulleiter von Hogwarts mir etwas anderes befehlen kann, werde ich solche Anweisungen erst recht nicht befolgen, wenn sie vom Verteidigungslehrer stammen, selbst wenn der Verteidigungslehrer droht, mir einhundert Hauspunkte abzuziehen und mich aus Hogwarts rausschmeißen zu lassen.“

„Wenn ich eine dieser Regeln breche, werde ich in meiner Zeit auf Hogwarts nicht mehr Verwandlung lernen.“

„Wir werden diese Regeln den ersten Monat lang am Anfang jeder Unterrichtsstunde wiederholen“, sagte Professor McGonagall. „Und jetzt werden wir mit Streichhölzern als Ausgangsobjekten und Nadeln als Zielobjekten anfangen … Zauberstäbe weg, wenn ich bitten darf, mit ‚anfangen` meinte ich, dass Sie anfangen, sich Notizen zu machen.“

Eine halbe Stunde vor Unterrichtsschluss teilte Professor McGonagall die Streichhölzer aus.

Zum Unterrichtsschluss hatte Hermine ein silbriges Streichholz und alle anderen Schüler, ob muggelgeboren oder nicht, hatten genau das Streichholz, mit dem sie angefangen hatten.

Professor McGonagall gab ihr noch einen Punkt für Ravenclaw.

\later

Nach Ende des Verwandlungsunterrichts kam Hermine an Harrys Tisch rüber, während Harry seine Bücher in seinen Beutel steckte.

„Weißt du“, sagte Hermine mit einem unschuldigen Gesichtsausdruck, „ich habe heute zwei Punkte für Ravenclaw verdient.“

„Das hast du“, sagte Harry kurz angebunden.

„Aber das war nicht so gut wie deine \emph{sieben} Punkte“, sagte sie. „Ich bin wohl einfach nicht so intelligent wie du.“

Harry hatte seine Schreibsachen in den Beutel gestopft und drehte sich mit verengten Augen zu Hermine. Er hatte das tatsächlich vergessen.

Sie \emph{klimperte mit den Wimpern.} „Allerdings haben wir täglich Unterricht. Ich frage mich, wie lange es wohl dauern wird, bis du die nächsten Hufflepuffs findest, die du retten kannst? Heute ist Montag. Du hast also bis Donnerstag Zeit.“

Beide sahen einander in die Augen ohne zu blinzeln.

Harry sprach zuerst. „Dir ist natürlich klar, dass das Krieg bedeutet.“

„Ich wusste nicht, dass zwischen uns Frieden herrschte.“

Alle anderen Schüler beobachteten sie jetzt fasziniert. Alle anderen Schüler und leider auch Professor McGonagall.

„Oh, Mr~Potter“, trällerte Professor McGonagall quer durchs Klassenzimmer, „ich habe gute Neuigkeiten für Sie. Madam Pomfrey war mit Ihrem Vorschlag zum Schutz der Spirndlertörchen vor Bruch einverstanden und plant, das bis Ende nächster Woche fertigzustellen. Dafür verdienen Sie … sagen wir zehn Punkte für Ravenclaw.“

Von Hermines Gesicht waren Betrug und Schock abzulesen. Harry dachte sich, dass sein eigenes Gesicht nicht sehr anders aussehen dürfte.

\emph{„Professor …“,} zischte Harry.

„Diese zehn Punkte haben Sie \emph{zweifellos} verdient, Mr~Potter. Ich würde niemals leichtfertig Hauspunkte vergeben. Für Sie mag es eine einfache Angelegenheit gewesen sein -- Sie haben etwas Zerbrechliches gesehen und einen Schutz vorgeschlagen -- aber Spirndlertörchen sind teuer und als das letzte Mal eines zerbrach, war der Schulleiter \emph{nicht} erfreut.“ McGonagall sah nachdenklich aus. „Meine Güte, ich frage mich, ob es wohl schonmal einem Schüler gelungen ist, an seinem allerersten Schultag siebzehn Hauspunkte zu verdienen. Ich werde nachsehen müssen, aber ich vermute, dass es ein neuer Rekord ist. Vielleicht sollten wir das beim Abendessen ansagen?“

„PROFESSOR!“, kreischte Harry. „Das ist \emph{unser} Krieg! Hören Sie auf sich einzumischen!“

„Jetzt haben Sie bis \emph{nächste} Woche Donnerstag Zeit, Mr~Potter. Vorausgesetzt natürlich, dass Sie nicht irgendwelchen Unfug anstellen und bis dahin Hauspunkte \emph{verlieren.} Indem Sie einen Lehrer respektlos anreden, zum Beispiel.“ Professor McGonagall legte die Hand an die Wange und sah nachdenklich aus. „Ich vermute, dass Sie es bis Freitag schaffen, die Punkte wieder zu verlieren.“

Harrys Mund flog zu. Er sah McGonagall so böse an, wie er nur konnte, doch sie schien es allenfalls amüsant zu finden.

„Ja, auf jeden Fall eine Ansage beim Abendessen“, überlegte Professor McGonagall laut. „Aber wir wollen ja nicht die Slytherins verärgern, also sollte die Ansage kurz sein. Nur die Punktzahl und die Tatsache, dass es ein neuer Rekord ist … und wenn jemand Sie um Hilfe bei den Hausaufgaben fragt und dann enttäuscht ist, weil Sie noch nicht einmal angefangen haben, Ihre Schulbücher zu lesen, könnten Sie diejenige Person immer noch an Miss~Granger verweisen.“

\emph{„Professor!“,} sagte Hermine in einer recht hohen Stimme.

Professor McGonagall ignorierte sie. „Hach, ich frage mich, wie lange es wohl dauern wird, bis Miss~Granger etwas tut, was eine Ansage beim Abendessen verdient? Ich freue mich schon darauf, was immer es auch sein mag.“

In stummer Einigkeit drehten Harry und Hermine sich um und stürmten aus dem Klassenzimmer, gefolgt von einer Gruppe vollkommen faszinierter Ravenclaws.

„Ähm“, sagte Harry, „gilt unsere Verabredung für nach dem Abendessen noch?“

„Natürlich“, sagte Hermine, „ich will doch nicht, dass du mit dem Lernen noch weiter hinter mich zurückfällst.“

„Wie lieb von dir. Und weißt du, so brilliant du ohnehin schon bist, kann ich mir die Frage doch nicht verkneifen, wie das erst sein wird, wenn du ein bisschen elementares Rationalitätstraining absolviert hast.“

„Ist das wirklich so nützlich? Es hat dir bei Zauberkunst oder Verwandlung ja anscheinend nicht viel geholfen.“

Es war einen Moment lang still.

„Na ja, ich habe meine Schulbücher erst vor vier Tagen bekommen. Deswegen musste ich diese siebzehn Hauspunkte ohne Zauberstab verdienen.“

„Vor vier Tagen? Vielleicht kannst du in vier Tagen keine acht Bücher lesen, aber du hättest zumindest \emph{eins} lesen können. Wie lange wirst du brauchen, wenn das so weiter geht? Du bist doch gut in Mathe, also sag mir, was ergibt acht mal vier geteilt durch null?“

„Ich habe jetzt Unterricht, den du nicht hattest, aber an Wochenenden habe ich frei, also … Limes Epsilon gegen Null plus von acht mal vier geteilt durch Epsilon … zehn Uhr siebenundvierzig am Sonntag.“

„Ich habe es in \emph{drei} Tagen geschafft.“

„Dann also vierzehn Uhr siebenundvierzig am Sonnabend. Ich bin mir sicher, dass ich irgendwo die Zeit finden werde.“

Und es wurde Abend, und es wurde Morgen: der erste Tag.

