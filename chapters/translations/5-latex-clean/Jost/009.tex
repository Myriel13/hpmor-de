

\hypertarget{selbstbewusstsein-teil-1}{% \section{9. Selbstbewusstsein, Teil 1}\label{selbstbewusstsein-teil-1}}

Die Übersetzung verlief mal wieder etwas chaotisch:

Das Kapitel war übersetzt und fast fertig überarbeitet, nur noch einzelne Formulierungen brauchten etwas Feinschliff~… und dann hat EY (der Originalautor) das Kapitel überarbeitet, einige Passagen gelöscht, einige neu geschrieben~… und vor uns lag wieder viel Arbeit. (Die entfernten Passagen wird es in Kapitel 11 als eine Art „Bonusmaterial“ geben.)

-\/-\/-\/-\/-\/-\/-\/-\/-\/-\/-\/-\/-\/-\/- ~ 9. Selbstbewusstsein, Teil 1 ~ -\/-\/-\/-\/-\/-\/-\/-\/-\/-\/-\/-\/-\/-\/-

\emph{Man wusste nie, welches kleine Ereignis den Ablauf des eigenen Masterplans durcheinanderbringt.}

* * *

„Abbott, Hannah!“

Pause.

„HUFFLEPUFF!“

„Bones, Susan!“

Pause.

„HUFFLEPUFF!“

„Boot, Terry!“

Pause.

„RAVENCLAW!“

Harry schaute kurz auf, um seinen neuen Hauskumpanen zu mustern und sich das Gesicht einzuprägen. Er versuchte immer noch, sich von seiner Begegnung mit den Geistern zu erholen. Die traurige, wirklich traurige, wahrhaftig traurige Sache war, dass es \emph{schien}, als bekäme er sich selbst wieder unter Kontrolle. Es kam ihm falsch vor. So, als ob es zumindest einen Tag hätte dauern sollen. Vielleicht ein ganzes Leben. Oder für immer.

„Corner, Michael!“

Lange Pause.

„RAVENCLAW!“

Am Pult vor dem großen Lehrertisch stand Professor McGonagall, die aufmerksam dreinblickte, während sie einen Namen nach dem anderen ausrief und nur Hermine und einigen anderen zugelächelt hatte. Hinter ihr, im höchsten Stuhl am Tisch -- der eher einem goldenen Thron glich -- saß ein faltiger, uralter Zauberer mit einer Brille und einem silbrig-weißen Bart, der so wirkte, als würde er fast bis auf den Boden reichen, und wachte mit einem wohlwollenden Gesichtsausdruck über die Zeremonie; so stereotypisch wie ein Weiser Alter Mann nur aussehen konnte, ohne aus Fernost zu kommen. (Allerdings traute Harry solchen Stereotypen nicht, nachdem er beim ersten Aufeinandertreffen mit Professor McGonagall erwartet hatte, dass sie hexenhaft kichern würde.) Der uralte Zauberer hatte jedem Schüler applaudiert, der in ein Haus geordnet wurde, mit einem ausdauernden Lächeln, das bei jedem Schüler auf's Neue höchst erfreut wirkte.

Zur Linken des goldenen Throns saß ein Mann mit verschlagenem Blick und mürrischem Gesicht, der niemandem applaudiert hatte und der es irgendwie schaffte, direkt zurückzublicken, wann immer Harry ihn ansah. Weiter links saß der blassgesichtige Mann, denn Harry im Tropfenden Kessel gesehen hatte, und der gelegentlich in seinem Sitz rumrutschte und zusammenzuckte; aus irgendeinem Grund wanderte Harrys Blick oft zu ihm. Links von diesem Mann saßen drei ältere Hexen, die sich nicht besonders für die Schüler zu interessieren schienen. Zur Rechten des goldenen Throns saß eine Hexe mittleren Alters mit rundem Gesicht und gelben Hut, die bis auf die Slytherins allen Schülern applaudiert hatte. Ein winziger Mann mit bauschigem weißen Bart, der jedem Schüler applaudiert, aber nur bei Ravenclaws gelächelt hatte, stand auf seinem Stuhl. Und am rechten Ende des Tisches wurden drei Plätze von dem berggleichen Wesen besetzt, dass sie empfangen hatte, nachdem sie aus dem Zug ausgestiegen waren, und sich als Hagrid, Hüter der Schlüssel und Ländereien, vorgestellt hatte.

„Ist der Mann, der auf seinem Stuhl steht, Hauslehrer von Ravenclaw?“, flüsterte Harry Hermine zu.

Ausnahmsweise antwortete Hermine nicht sofort; sie hüpfte neben ihm von einem Bein aufs andere, um den Sprechenden Hut zu sehen, und zappelte so energisch herum, dass Harry dachte, ihre Füße würden vom Fußboden abheben.

„Ja, das ist er“, sagte einer der Vertrauensschüler, der sie begleitet hatte; eine junge Frau, die Ravenclaw-blau trug. Miss Clearwater, wenn Harry sich richtig erinnerte. Ihre Stimme war ruhig, enthielt jedoch eine Spur Stolz. „Er ist der Zauberkunst-Lehrer von Hogwarts, Filius Flitwick, der kenntnisreichste Meister der Zauberkunst, der noch lebt, und ein ehemals glänzender Duellkämpfer --“

„Warum ist er so \emph{winzig?}“, zischte ein Student, an dessen Namen Harry sich nicht erinnerte. „Ist er ein \emph{Halbblut?}“

Die Vertrauensschülerin sah ihn kühl an. „Der Professor stammt tatsächlich von Kobolden ab --“

„Was?“, sagte Harry unwillkürlich, woraufhin Hermine und vier andere Schüler „Psst!“ zischten.

Nun bekam Harry einen überraschend einschüchternden Blick der Vertrauensschülerin ab.

„Ich meine --“, flüsterte Harry, „ich habe kein \emph{Problem} damit -- es ist nur -- ich meine -- wie ist das \emph{möglich?} Man kann nicht einfach so zwei verschiedene Arten mischen und lebenden Nachwuchs bekommen! Dabei sollten die genetischen Informationen über jedes Organ, worin die zwei Arten sich unterscheiden, verquirlt werden -- das wäre, als würde man --“ Zauberer hatten keine Autos, also konnte er es nicht mit einem Motor vergleichen. „-- etwas bauen wollen, was halb Kutsche, halb Boot ist, oder so …“

Die Ravenclaw-Vertrauensschülerin sah Harry immer noch streng an. „Warum sollte man so etwas \emph{nicht} bauen können?“

„Psst!“, zischte ein anderer Vertrauensschüler, obwohl die Ravenclaw-Schülerin leise gesprochen hatte.

„Ich meine --“, sagte Harry noch leiser, und überlegte, wie er danach fragen sollte, ob Kobolde von Menschen abstammten, oder ob beide einen gemeinsamen Vorfahren wie \emph{Homo erectus} hatten, oder ob Kobolde irgendwie aus Menschen \emph{gemacht} wurden -- ob sie zum Beispiel genetisch immer noch Menschen waren, aber unter dem Einfluss einer erblichen Verzauberung standen, deren Effekt verwässert wurde, wenn nur ein Elternteil ein 'Kobold' war. Das würde erklären, wieso sie sich mit Menschen fortpflanzen konnten; es würde aber auch bedeuten, dass Kobolde \emph{kein} unglaublich wertvoller zweiter Datensatz war, der Rückschlüsse auf die Entstehung von Intelligenz in anderen Arten als \emph{Homo sapiens} zuließ -- und wo Harry genauer darüber nachdachte, hatten die Kobolde in Gringotts für ihn auch \emph{nicht} wie eine völlig fremdartige, nicht-menschliche Intelligenz gewirkt, nicht wie die Dirdir oder wie Puppenspieler -- „Ich meine, woher \emph{stammen} Kobolde?“

„Litauen“, flüsterte Hermine geistesabwesend, während ihre Augen immer noch fest auf den Sprechenden Hut gerichtet waren.

Nun lächelte die Vertrauensschülerin Hermine zu.

„Schon gut“, flüsterte Harry.

Am Pult rief Professor McGonagall „Goldstein, Anthony!“

„RAVENCLAW!“

Neben Harry wippte Hermine so kräftig auf ihren Zehenspitzen, dass ihre Füße tatsächlich mit jedem Schwung vom Boden abhoben.

„Goyle, Gregory!“

Für einen langen, angespannten Moment herrschte Stille unter dem Hut. Fast eine Minute.

„SLYTHERIN!“

„Granger, Hermine!“

Hermine schoss los und rannte so schnell sie konnte zum Sprechenden Hut, hob ihn auf und rammte das fleckige alte Stück Kleidung jäh auf ihren Kopf, sodass Harry aufstöhnte. Hermine hatte \emph{ihm} vom Sprechenden Hut erzählt, aber sie \emph{behandelte} den Hut sicher nicht wie ein unersetzliches, unglaublich wichtiges, 800 Jahre altes Artefakt vergessener Magie, das kurz davor war, einen ausgeklügelten telepathischen Eingriff in ihren Geist vorzunehmen und nicht gerade in einer guten materiellen Verfassung zu sein schien.

„RAVENCLAW!“

Soviel dazu. Harry verstand überhaupt nicht, warum Hermine so nervös gewesen war. In welchem alternativen Universum könnte dieses Mädchen \emph{nicht} nach Ravenclaw sortiert werden? Wenn Hermine Granger nicht nach Ravenclaw kam, dann hatte das Haus Ravenclaw keine Existenzberechtigung.

Hermione kam am Ravenclaw-Tisch an und erntete pflichtbewussten Beifall. Harry fragte sich, ob der Applaus wohl lauter oder leiser gewesen wären, wenn sie irgendeine Ahnung gehabt hätten, was für eine Konkurrentin sie da an ihrem Tisch willkommen hießen. Harry wusste π bis 3,141592 auswendig, weil eine Genauigkeit von 1 auf 1 Million für die meisten praktischen Anwendungen ausreichte. Hermione kannte die ersten hundert Stellen von π, weil so viele Stellen auf der letzten Seite ihres Mathebuchs abgedruckt waren.

Neville Longbottom kam nach Hufflepuff, wie Harry erfreut feststellte. Wenn dieses Haus tatsächlich die Loyalität und Kameradschaft bot, für die es bekannt war, dann würden diese verlässlichen Freunde Neville sehr gut tun. Kluge Kinder nach Ravenclaw, böse Kinder nach Slytherin, Möchtegernhelden nach Gryffindor und alle anderen, die tatsächlich harte Arbeit leisten, nach Hufflepuff.

(Allerdings \emph{hatte} Harry gut daran getan, erst eine \emph{Ravenclaw}-Vertrauensschülerin zu fragen. Die junge Frau hatte nicht mal von ihrem Buch aufgesehen oder Harry erkannt, sondern nur ihren Zauberstab in Nevilles Richtung geschwungen und etwas gemurmelt. Danach hatte Neville einen glasigen Gesichtsausdruck bekommen und war in den fünften Wagen zum vierten Abteil auf der linken Seite gelaufen, wo sich in der Tat seine Kröte befand.)

„Malfoy, Draco!“ kam nach Slytherin und Harry atmete erleichtert durch. Es \emph{schien} klar wie Kloßbrühe, aber man wusste nie, welches kleine Ereignis den Ablauf des eigenen Masterplans durcheinanderbringt.

Professor McGonagall rief „Perks, Sally-Anne!“, und aus der Gruppe der Kinder trat ein blasses, verloren wirkendes Mädchen hervor, das seltsam ätherisch wirkte -- also könnte sie auf mysteriöse Weise verschwinden, sobald man den Blick von ihr abwandte, und alle Erinnerungen an sie mitnehmen.

Und dann (mit einem so fest aus ihrer Stimme und von ihrem Gesicht verbannten Zittern, dass man sie sehr gut kennen musste, um es zu bemerken) atmete Minerva McGonagall tief ein und rief „Potter, Harry!“

Plötzlich war es still in der Halle.

Alle Gespräche brachen ab.

Alle Augen richteten sich auf ihn.

Zum ersten Mal in seinem gesamten Leben hatte Harry das Gefühl, dass er nun Lampenfieber bekommen könnte.

Harry unterdrückte dieses Gefühl sofort. An ganze Räume voller Menschen, die ihn anstarrten, würde er sich gewöhnen müssen, wenn er im magischen Großbritannien leben wollte, oder wenn er überhaupt irgendwas interessantes mit seinem Leben anfangen wollte. Er befestigte ein zuversichtliches, aber falsches Lächeln auf seinem Gesicht und hob einen Fuß um nach vorne zu gehen --

„Harry Potter!“, schrie die Stimme von Fred oder George Weasley und -- „Harry Potter!“ -- die des anderen Weasley-Zwillings und einen Moment später der gesamte Gryffindor-Tisch und kurz darauf hatte ein guter Teil der Ravenclaws und Hufflepuffs mit eingestimmt.

\emph{„Harry Potter! Harry Potter! Harry Potter!“}

Und Harry ging vor. Viel zu langsam, wie er feststellte, sobald er begonnen hatte, aber dann war es bereits nicht mehr möglich, schneller zu laufen ohne dass es unbeholfen aussah.

* * *

\emph{„Harry Potter! Harry Potter! HARRY POTTER!“}

Mit einer vagen Furcht vor dem, was sie dort sehen würde, blickte Minerva McGonagall hinter sich zu beiden Seiten des Lehrertisches.

Trelawney fächelte sich wie verrückt Luft zu, Flitwick schaute neugierig drein, Hagrid klatschte mit, Sprout wirkte ernst, Vector und Sinistra irritiert, und Quirrell blickte achtlos ins Nichts. Dumbledore lächelte gutmütig und Snape hatte seinen Weinkelch so fest gekrallt, dass seine Knöchel weiß wurden und sich das Silber des Kelches langsam verformte.

Mit einem breiten Grinsen nickte Harry erst der einen und dann der anderen Seite zu, während er zwischen den vier Haustischen gemächlich nach vorn schritt, wie ein Prinz, der ein geerbtes Schloss besichtigt.

„Schütze uns vor weiteren Dunklen Lords!“, rief einer der Weasley-Zwillinge, und der andere Zwilling schrie „Besonders, wenn es Lehrer sind!“, und erntete damit allgemeines Gelächter von allen Häusern außer den Slytherins.

Minervas Lippen verzogen sich zu einer weißen Linie. Sie würde mit diesen schrecklichen Weasleys noch über die letzte Zeile reden, da die wohl dachten, dass sie keine Macht hätte, da es der erste Schultag war und Gryffindor keine Punkte hatte, die man wegnehmen könnte. Wenn die sich keine Sorgen um Nachsitzen machten, dann würde sie etwas anderes finden.

Dann, mit einem plötzlichen Schrecken, schaute sie in Snapes Richtung -- \emph{sicherlich} erkannte er, dass der Potter-Junge keine Ahnung hatte, von wem da gerade die Rede war --

-- und Snapes Gesicht hatte sich jenseits aller Rage zu einer Art angenehmer Indifferenz gewandelt. Ein kaum wahrnehmbares Lächeln spielte über seine Lippen. Er sah in die Richtung von Harry Potter, nicht zum Gryffindor-Tisch, und seine Hand komprimierte die Überreste dessen, was einmal ein Weinkelch gewesen war.

* * *

Harry Potter schritt starr lächelnd voran. Es fühlte sich gut, aber gleichzeitig doch irgendwie unangenehm an.

Sie jubelten wegen einer Sache, die er getan hatte, als er ein Jahr alt war. Eine Sache, die er nicht einmal richtig abgeschlossen hatte. Irgendwo, irgendwie war der Dunkle Lord noch am Leben. Würden sie wohl auch so laut jubeln, wenn sie das wüssten?

Aber die Macht des Dunklen Lords \emph{war} einmal gebrochen worden.

Und Harry würde sie wieder beschützen. Wenn es tatsächlich eine Prophezeiung gab, und die das vorhersagte. Nun, eigentlich unabhängig davon, was irgendeine blöde Prophezeiung besagte.

All diese Leute glaubten an ihn und jubelten ihm zu -- Harry könnte es nicht ertragen, sie zu enttäuschen. Zu leuchten und zu verblassen, wie so viele andere Wunderkinder. Eine Enttäuschung zu sein. Seiner Reputation als Symbol des Lichtes nicht gerecht zu werden, unabhängig davon, \emph{wie} er sie bekommen hatte. Er würde absolut, ganz sicher, und koste es, was es wolle -- solange es auch dauerte und was auch immer es ihm abverlangte, auch wenn es ihn umbrachte -- ihre Erwartungen erfüllen. Und diese dann noch \emph{übertreffen}, sodass sich die Leute im Nachhinein fragen würden, warum sie nur so wenig von ihm verlangt hatten.

\emph{„HARRY POTTER! HARRY POTTER! HARRY POTTER!“}

Harry machte seine letzten Schritte zum Sprechenden Hut. Er verbeugte sich vor dem Orden des Chaos am Gryffindortisch, wandte sich dann der anderen Seite der Halle zu, verbeugte sich wieder und wartete, bis der Applaus und das Kichern langsam abebbten.

(In der hintersten Ecke seines Kopfes fragte er sich, ob der Sprechende Hut tatsächlich bewusst handelte, in dem Sinne, dass er sich seines eigenen Bewusstseins bewusst war -- und falls ja, ob es ihn wirklich zufriedenstellte, ein einziges Mal im Jahr mit Elfjährigen zu sprechen. Sein Lied hatte sich danach angehört: \emph{„Oh, ich bin der Hut und mir geht`s gut / Der einmal im Jahr seine Arbeit tut …“})

Als es wieder still wurde, setzte Harry sich auf den Stuhl und platzierte das 800 Jahre alte telepathische Artefakt vergessener Magie \emph{vorsichtig} auf seinem Kopf.

Dabei dachte er, so sehr er nur konnte: \emph{Steck mich noch nicht in ein Haus! Ich habe Fragen an Dich! Wurde jemals ein Vergessenszauber auf mich gesprochen? Hast du den Dunklen Lord in ein Haus gesteckt, und kannst du mir von seinen Schwächen erzählen? Kannst du mir sagen, warum mein Zauberstab mit dem des Dunklen Lords verwandt ist? Ist der Geist des Dunklen Lords mit meiner Narbe verbunden, und ist das der Grund dafür, dass ich manchmal so zornig werde? Das sind die wichtigsten Fragen, aber falls du noch etwas Zeit hast, kannst du mir etwas darüber erzählen, wie ich die verlorene Magie wiederentdecke, die dich einst geschaffen hat?}

Inmitten der Stille von Harrys Gedanken, wo zuvor nie eine andere Stimme als seine eigene zu hören war, erklang eine fremde, zweite Stimme, die sich sehr besorgt anhörte:

\emph{„Meine Güte. Das ist ja noch nie passiert …“}

-\/-\/-\/-\/-\/-\/-\/-\/-\/-\/-\/-\/-\/-\/-\/-\/-\/-\/-\/-\/-\/-\/-\/-\/-\/-\/-\/-\/-\/-

\textbf{Dirdir}:

Alienart aus dem \href{http://en.wikipedia.org/wiki/The_Dirdir}{gleichnamigen Science-Fiction-Buch} von Jack Vance.

\textbf{„Oh ich bin der Hut und mir geht's gut“}:

Das spielt auf das \href{http://www.youtube.com/watch?v=A42Ba8naV_o}{Holzfäller-Lied} von Monty Python an.

