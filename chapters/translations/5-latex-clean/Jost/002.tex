

\hypertarget{alles-woran-ich-glaube-ist-falsch}{% \section{2. Alles, woran ich glaube, ist falsch}\label{alles-woran-ich-glaube-ist-falsch}}

Mein Dank gilt zwei bestimmten Personen, die sich die Übersetzung durchlesen und sehr viele gute Hinweise geben. Dieser Text wird dadurch wesentlich besser!

-\/-\/-\/-\/-\/-\/-\/-\/-\/-\/-\/-\/-\/-\/- ~ 2. Alles, woran ich glaube, ist falsch ~ -\/-\/-\/-\/-\/-\/-\/-\/-\/-\/-\/-\/-\/-\/-

\emph{„Natürlich war es meine Schuld. Hier gibt es sonst niemanden, der für irgendetwas verantwortlich sein könnte.“}

-\/-\/-

„Nur um sicher zu gehen, Papa“, sagte Harry, „falls~Professor McGonagall dich schweben lässt, obwohl du genau weißt, dass du nicht an irgendwelchen Seilen hängst, dann ist das ein Beweis. Du wirst es nicht abstreiten und behaupten, dass es irgendein Zaubertrick war. Das wäre unfair. Wenn du damit nicht einverstanden bist, solltest du es \emph{jetzt} sagen, dann können wir uns ein anderes Experiment ausdenken.“

Harrys Vater, Professor Michael Verres-Evans, verdrehte die Augen. „Ja, Harry.“

„Und Mama, deine Theorie besagt, dass Professor McGonagall es schafft, und wenn das nicht klappt, dann gibst du zu, dass du falsch liegst und behauptest nicht, dass Zauberei nicht funktioniert, wenn die Leute daran zweifeln oder irgendsowas.“

Die Stellvertretende Schulleiterin Professor Minerva McGonagall sah Harry irritiert an. „War das jetzt alles, Mr Potter? Darf ich jetzt beginnen?“

„\emph{Alles}? Vermutlich nicht“, sagte Harry. „Aber es sollte uns zumindest \emph{weiterbringen}. Professor McGonagall, bitte sehr.“

„\emph{Wingardium Leviosa}.“

Harry sah zu seinem Vater, der einen halben Meter über dem Boden schwebte.

„Oh“, sagte Harry.

Sein Vater sah ihn an. „Oh“, erwiderte er.

Dann wandte Professor Verres-Evans sich wieder an Professor McGonagall. „In Ordnung, Sie können mich jetzt wieder runterlassen.“

Sein Vater wurde sorgsam wieder zu Boden gelassen.

Harry fuhr sich mit der Hand durchs Haar. Vielleicht lag es ja daran, dass ein Teil von ihm schon längst überzeugt gewesen war, aber~… „Das ist ziemlich enttäuschend“, sagte Harry. „Man sollte denken, dass ein viel dramatischeres mentales Erlebnis mit der bayesischen Aktualisierung basierend auf einem Ereignis von infinitesimaler Wahrscheinlichkeit verknüpft ist,~…“ Harry brach ab. Professor McGonagall, seine Mutter und sogar sein Vater warfen ihm wieder \emph{diesen} Blick zu. „Ich meine, mit der Feststellung, dass alles, woran ich geglaubt habe, falsch ist.“

Es hätte wirklich aufregender sein sollen. Sein Gehirn hätte alle bisherigen Ansichten über das Universum herauswerfen sollen, die allesamt nicht zugelassen hätten, dass so etwas passiert. Doch stattdessen schien sein Gehirn einfach zu denken, \emph{Also gut, ich habe gesehen, wie ein Hogwarts-Lehrer mit dem Zauberstab gewedelt hat und deinen Vater in der Luft schweben ließ; was nun?}

Professor McGonagall erschien höchst amüsiert. „Wünschen Sie eine weitere Demonstration, Mr Potter?“

„Das ist nicht nötig“, sagte Harry. „Wir haben ein aussagekräftiges Experiment durchgeführt. Aber~…“ Harry zögerte. Er konnte nicht widerstehen. Und unter diesen Umständen \emph{sollte} er gar nicht widerstehen. Es war gut und richtig, neugierig zu sein. „Was können Sie noch tun?“

McGonagall verwandelte sich in eine Katze.

Harry wich erschrocken zurück, stolperte über einen Bücherstapel und landete mit einem lauten \emph{plumps} auf dem Hintern. Er versuchte sich abzufangen, doch seine Hände kamen ungünstig auf und er spürte ein unangenehmes Stechen in der Schulter, während sein Gewicht ungebremst zu Boden fiel.

Sofort verwandelte sich die kleine getigerte Katze wieder in die bemantelte Frau. „Es tut mir Leid, Mr Potter“, sagte McGonagall ernst, verzog ihre Lippen aber zu einem Lächeln. „Ich hätte Sie warnen sollen.“

Harry atmete stoßweise. „\emph{Das GEHT nicht!}“, keuchte er.

„Es ist nur eine Verwandlung“, sagte McGonagall. „Eine Animagusverwandlung, um genau zu sein.“

„Sie haben sich in eine Katze verwandelt! Eine \emph{KLEINE} Katze! Sie haben gegen die Energieerhaltung verstoßen! Das ist nicht nur irgendeine Regel, es folgt aus der Darstellung des Hamiltonoperators in der Quantenphysik! Ohne Energieerhaltung kommt es zu Inkonsistenzen und Informationsübertragung mit Überlichtgeschwindigkeit! Und Katzen sind \emph{KOMPLIZIERT}! Ein menschliches Gehirn kann sich nicht einfach die ganze Anatomie und biochemische Zusammensetzung einer Katze vorstellen, ganz zu schweigen von der \emph{Neurologie}! Wie können Sie mit einem Katzenhirn \emph{denken}?“

McGonagalls Lippen verzogen sich immer stärker. „Magie.“

„Magie reicht dafür nicht aus! Sie müssten ein Gott sein!“

McGonagall blinzelte. „Das ist das erste Mal, dass jemand \emph{so etwas} zu mir sagt.“

Harrys Blick wurde unscharf. Seit etwa dreitausend Jahren beobachteten Menschen die Welt. Es hatte mit den Alten Griechen angefangen, die vermutet hatten, dass es an verschiedenen Orten verschiedene Gesetze gäbe~-- solche, die im Himmel galten und andere, die auf der Erde galten. Über Jahrhunderte hinweg war die Menschheit von diesem Ausgangspunkt immer weiter vorangeschritten. Sie hatte die Oberfläche dieser Welt durchbrochen und in Körpern Gewebe gefunden, im Gewebe Zellen, in Zellen chemische Verbindungen und in Atomen schließlich Quarks entdeckt.

Es waren einfache, stabile und unveränderliche Dinge, die purer Kausalität folgten, sich mathematisch beschreiben ließen; hinter dem alltäglichen Sein der Welt waren sie immer in Bewegung.

Newton hatte einst seine Gravitationsgesetze formuliert, die zurückblickend das gesamte Sonnensystem von Anfang zu erklären schienen, doch selbst als die Periheldrehung des Merkur entdeckt wurde und den Newton'schen Gesetzen widersprach, hatte Einstein die neue Theorie, die neuen allgemeingültigen Regeln formuliert, die schon vom Anfang aller Dinge an gegolten hatten. Die wahren Regeln galten überall und jederzeit an jedem Ort des Universums, es gab keine Sonderfälle für bestimmte Oberflächen oder Ausnahmeregelungen, wenn es bequem erschien; das hatte die Menschheit in den letzten dreitausend Jahren gelernt. \emph{Ganz zu schweigen davon}, dass das Denken im Gehirn stattfand, das Gehirn aus Neuronen bestand und bei Gehirnverletzungen das Denken in Mitleidenschaft gezogen wurde~-- wenn der Hippocampus zerstört war, verlor eine Person die Fähigkeit, neue Erinnerungen anzulegen; das Gehirn \emph{ist} die Person~…

Und dann verwandelt sich eine Frau in eine Katze. So viel also dazu~…

Hundert Fragen kämpften um Vorherrschaft in Harrys Kopf und der Gewinner platzte schließlich aus ihm heraus: „Und~-- und was für ein Zauberspruch ist denn \emph{Wingardium Leviosa}? Wer denkt sich diese Worte aus? Vorschulkinder?“

„Das genügt, Mr Potter“, sagte McGonagall knapp, doch ihre Augen glitzerten amüsiert. „Wenn Sie ein Zauberer werden wollen, schlage ich vor, dass wir die nötigen Formalitäten erledigen, so dass Sie Hogwarts besuchen können.“

„Okay“, sagte Harry etwas benommen. Er sortierte immer noch seine Gedanken. Es war der Mut eines Rationalisten, sich dem unbekannten Drachen zu stellen und ihn zu erschlagen. Als die Alten Griechen vor zweitausendfünfhundert Jahren begannen über die Welt nachzudenken, hatten sie im Grunde \emph{nichts} gewusst; sie hatten ihre Finger angesehen und keine Ahnung gehabt, woraus diese bestanden oder warum sie diese bewegen konnten, während der Erdboden unbewegt dalag. Sie hatten nichts über das Gehirn gewusst, hätten in einen Spiegel geschaut ohne zu wissen, was sie dort sahen oder wie sie es sahen. Aber dennoch hatten sie gelebt und all das hinterfragt. Wenn alles, woran er glaubte, falsch war, dann würde Harry von vorne beginnen und wieder alles hinterfragen.

Mit diesem Entschluss stand Harry auf und verdrängte das Stechen in seiner Schulter. „Also, wie komme ich nach Hogwarts?“

McGonagall konnte ihr Gelächter nicht ganz unterdrücken. Es klang, als ob man es mit einer Pinzette aus ihr rauszog.

„Nicht so schnell, Harry“, sagte sein Vater. „Denk' dran, dass du bis jetzt nicht zur Schule gegangen bist. Was ist mit deiner Erkrankung?“

McGonagall drehte sich sofort zu Michael. „Seine Erkrankung? Was für eine?“

„Ich schlafe nicht richtig“, sagte Harry. Er zuckte hilflos mit den Schultern. „Mein Schlafzyklus dauert sechsundzwanzig Stunden, also gehe ich jeden Tag zwei Stunden später ins Bett. Vorher kann ich nicht einschlafen und am nächsten Tag gehe ich wieder zwei Stunden später ins Bett. Zehn Uhr abends, Mitternacht, zwei Uhr morgens, vier Uhr morgens und so weiter, rund um die Uhr. Selbst wenn ich versuche früher aufzustehen bin ich den ganzen Tag fix und fertig und es bringt trotzdem nichts. Deswegen habe ich bisher keine normale Schule besucht.“

„Unter anderem deswegen“, sagte seine Mutter. Harry warf ihr einen wütenden Blick zu.

McGonagall überlegte kurz. „Hm, ich kann mich nicht daran erinnern, schonmal von so etwas gehört zu haben~…“, sagte sie zögerlich. „Ich werde Madame Pomfrey fragen, ob sie ein Gegenmittel kennt.“ Dann hellte ihr Gesicht sich wieder auf. „Doch, ich bin mir sicher, dass das kein Problem wird~-- irgendwie werde ich da schon eine Lösung finden. Nun“, ihr Blick wurde wieder fragend, „welche Gründe gibt es noch?“

Harry warf seinen Eltern einen wütenden Blick zu. „Ich bin ein entschiedener Gegner der aktuellen Gesetzgebung, da ich nicht unter der vollkommenen Unfähigkeit dieses maroden Bildungssystemes leiden will, mir Lehrer oder Lehrmaterialien von auch nur annähernd erträglicher Qualität zu bieten.“

Harrys Eltern lachten laut, als ob sie das alles für einen Witz hielten. „Ach“, sagte sein Vater mit strahlenden Augen, „hast du \emph{deswegen} deine Mathelehrerin in der dritten Klasse gebissen?“

„\emph{Sie wusste nicht, was ein Logarithmus ist!}“

„Oh, selbstverständlich“, entgegnete Harrys Mutter. „Sie zu beißen war eine sehr vernünftige Antwort darauf.“

Harrys Vater nickte. „Eine angemessene Strategie zur Lösung des grundsätzlichen Problemes, dass manche Lehrer nicht wissen, was ein Logarithmus ist.“

„Ich war \emph{sieben Jahre alt!} Wie lange wollt ihr mich noch damit aufziehen?“

„Ich weiß“, sagte seine Mutter mitfühlend, „da beißt man \emph{einmal} einen Mathelehrer und wird immer wieder daran erinnert.“

Harry wandte sich an McGonagall. „Da! Sehen Sie, was ich ständig ertragen muss?“

„Entschuldigt mich“, sagte Petunia und rannte auf die Terrasse, wo sie in schallendes Gelächter ausbrach.

„Auf, ähm, auf“, McGonagall schien das Sprechen aus irgendeinem Grund schwer zu fallen, „auf Hogwarts werden keine Lehrer gebissen. Ist das klar, Mr Potter?“

Harry blickte sie finster an. „Gut, ich beiße niemanden, der mich nicht zuerst beißt.“

Bei diesen Worten musste auch Professor Michael Verres-Evans kurz den Raum verlassen.

„Nun“, seufzte McGonagall nachdem Harrys Eltern sich beruhigt hatte und zurückgekehrt waren. „Nun gut. Ich glaube, angesichts dieser Umstände sollten wir Ihre Schulsachen erst einen oder zwei Tage vor Beginn des Schuljahres einkaufen.“

„Was? Warum? Die anderen Kinder wissen doch schon so viel über Zauberei, nicht wahr? Ich muss sofort anfangen sie einzuholen.“

„Es sei Ihnen versichert, Mr Potter“, erwiderte McGonagall, „dass Hogwarts vollkommen dazu in der Lage ist, die Grundlagen zu lehren. Außerdem vermute ich, Mr Potter, dass Sie es --~sofern ich Sie zwei Monate lang mit ihren Schulbüchern alleine lasse~-- selbst ohne Zauberstab schaffen, von diesem Haus nur einen riesigen Krater zurückzulassen, aus dem purpurner Rauch aufsteigt. Wenn ich dann zurückkehre um Sie abzuholen, werde ich drumherum eine entvölkerte Stadt vorfinden und eine Horde riesiger Zebras, die ganz England in Angst und Schrecken versetzen.“

Harrys Eltern nickten in vollkommener Eintracht.

„\emph{Mama! Papa!}“

-\/-\/-\/-\/-\/-\/-\/-\/-\/-\/-\/-\/-\/-\/-\/-\/-\/-\/-\/-\/-\/-\/-\/-\/-\/-\/-\/-\/-\/-

Hamiltonoperator:\\ Der Hamiltonoperator H ist, kurz gesagt, die Summe aus kinetischer und potentieller Energie eines Systems. Mit seiner Hilfe kann man in der theoretischen Mechanik auf recht elegante Weise die Bewegungsgleichungen von Teilchen formulieren. Er wird außerdem in den meisten Formulierungen der Quantenphysik benutzt.

Überlichtgeschwindigkeit:\\ In der deutschen Wikipedia findet man hierzu nicht viel Spannendes, der Artikel begnügt sich hauptsächlich mit einigen Bemerkungen zur Relativitätstheorie, die in diversen populärwissenschaftlichen Fernsehsendungen schon zu Genüge durchgekaut wurden. Im \href{http://en.\%20wikipedia.\%20org/wiki/Faster-than-light“\%20target=}{englischen Wikipedia-Artikel} sind hingegen diverse extrem spannende Effekte aufgeführt, die man zwar nicht \emph{versteht}, bei denen man aber kaum anders kann als ehrfurchtsvoll staunen.

Periheldrehung des Merkur:\\ Der Merkur hat keine ganz elliptische Bahn um die Sonne. Daher bewegt sich der Bahnpunkt seiner geringsten Entfernung von der Sonne (Perihel) ganz langsam im Kreis. Anhand \href{http://de.\%20wikipedia.\%20org/wiki/Merkur_\%28Planet\%29\%23Periheldrehung“\%20target=}{dieser Skizze} ist das leicht verständlich.

Logarithmus:\\ Umkehroperation der Potenzierung, kommt in Mathe etwa in der neunten Klasse dran. Folgende Aussagen sind gleichbedeutend:\\ e\^x = 1 ~ ~und ~ ~x = ln(1) ~(„x ist der natürliche Logarithmus von eins.“)

-\/-\/-

Mich würde sehr interessieren, was ihr von der Geschichte, der Übersetzung und den Fußnoten haltet. Diese Erläuterungen (in diesem Kapitel z. B. zum Hamiltonoperator oder zur Periheldrehung des Merkur) sind nicht Teil der Originalgeschichte sondern von mir selbst ergänzt, weil ich mir dachte, dass einige Begriffe einer genaueren Erläuterung bedürfen.\\ Mich würde es sehr interessieren, wie diese bei euch ankommen: Lest ihr sie? Beschäftigt ihr euch nach dem Lesen des Kapitels ein bisschen mit dem Thema? Sind die Erläuterungen auf einem zu hohen oder zu niedrigen Niveau?\\ Antworten würden mir sehr helfen.

