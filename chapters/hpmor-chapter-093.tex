\partchapter{Roles}{IV}

\lettrine{H}{arry} entra dans la grande salle, ne regarda autour de lui qu'une fois, consomma suffisamment de calories pour se sustenter, sortit, remit sa Cape et trouva un petit coin choisi au hasard où manger. Voir les élèves à leurs tables…

\emph{Ressentir du dégoût en regardant d'autres humains n'est pas bon signe}, dit Poufsouffle. \emph{Ce n'est pas raisonnable de leur en vouloir de ne pas avoir eu la chance d'apprendre ce que tu as appris. L'inaction dans l'urgence n'a rien à voir avec l'égoïsme des gens. C'est le biais de normalité, comme lors de cet accident d'avion à Tener-quelque chose où quelques personnes se sont échappées en courant mais où la plupart sont restés dans leur siège, immobiles, alors que leur avion brûlait. Regarde le temps que tu as mis avant de vraiment te mettre en mouvement.}

\emph{Haïr ne sert à rien}, dit Gryffondor. \emph{Ça endommagera juste ton altruisme.}

\emph{Essaie d'inventer une méthode d'entraînement que tu pourras utiliser pour empêcher que ça se produise la prochaine fois}, dit Serdaigle.

\emph{Je vais me lancer et prendre note de cette prédiction expérimentale~:}, dit Serpentard, \emph{on observa toujours exactement ce qui aurait été prédit à partir de l'hypothèse que les gens ne peuvent être sauvés, qu'on ne peut rien leur apprendre, et qu'ils ne nous aideront jamais sur quoi que ce soit d'importance. Ah, et nous avons aussi besoin d'un moyen de garder trace de toutes les fois où j'ai raison.}

Harry ignora les voix dans sa tête et se contenta de manger ses tartines aussi vite que possible. Ce n'était pas une forme de nutrition correcte en général, mais des exceptions ne posaient pas de problème tant qu'il les rattrapait le lendemain.

À mi-bouchée, l'étincelante silhouette d'un phénix apparut de nulle part et dit de la voix d'un vieil homme fatigué~: "Enlève ta Cape Harry, s'il te plaît, j'ai une lettre à te remettre."

Harry toussota, avala une partie de la tartine qui était mal passée, se leva, ôta la Cape d'Invisibilité et dit à voix haute~: "Dis à Dumbledore que j'ai dit d'accord," puis se rassit et continua de manger sa tartine.

La tartine était finie lorsque Albus Dumbledore arriva jusqu'au recoin de Harry, des feuilles de papier pliées en main~; du vrai papier à lignes, pas du parchemin de sorcier.

"Est-ce…" dit Harry.

"De ton père et de ta mère," dit le vieux sorcier. Sans dire mot, Dumbledore tendit les feuilles pliées, et sans un mot, Harry les accepta. Le vieux sorcier hésita puis dit doucement~: "Le professeur de Défense m'a dit de me retenir de te donner des conseils et je suis arrivé à la même conclusion en prenant le temps de réfléchir. J'ai toujours mit trop longtemps à apprendre les vertus du silence. Mais si j'ai tort, il te suffit de me le dire…

--- Vous n'avez pas tort," dit Harry. Il baissa les yeux vers le papier à ligne plié et ressentit une nausée qui était la façon que son corps avait d'indiquer une prédiction très pessimiste. Ses parents ne le répudieraient pas vraiment, et ils ne pouvaient pas lui \emph{faire} grand chose (une partie de lui avait toujours viscéralement peur qu'on lui retire la permission de regarder la télévision, peu importe à quel point c'était maintenant absurde). Mais il avait quitté le rôle que les parents s'attendaient à observer chez des enfants qui, selon eux, leur étaient hiérarchiquement inférieurs. Il était stupide de s'attendre à autre chose qu'à une furie indignée, qu'à une rage totale et vertueuse, lorsqu'on se comportait ainsi envers quelqu'un qui croyait être le dominant.

"Après l'avoir lue," dit le directeur, "je crois que tu voudras venir immédiatement dans la grande salle, Harry. Il y a une annonce que tu souhaiteras entendre.

--- Les funérailles ne m'intéressent pas…

--- Non. Pas cela. S'il te plaît, Harry, viens dès que tu auras fini de lire, et viens sans ta Cape. Acceptes-tu~?

--- Oui."

Le vieux sorcier partit.

Harry se força à ouvrir la lettre. Ce qui comptait, c'était de maintenir ses amis et ses connaissances vulnérables hors de danger~; C'était peut-être un cliché, mais Harry n'aurait su trouver un défaut dans le raisonnement. Les liens endommagés pouvaient être réparés plus tard.

La première lettre disait, d'une écriture manuscrite qui exigea de Harry une grande concentration pour pouvoir la lire~:
\begin{writtenNote}
\letterAddress{Mon fils,}

Peu importe ce que tu as lu dans les livres, nous préserver du danger n'est \emph{pas} aussi important que d'avoir des adultes capables d'aider quand tu as des ennuis. Tu as décidé sans nous en dire un mot que nous t'abandonnerions à cause de ton 'coté obscur'. Le fantôme de Shakespeare sait que j'ai vu des choses cette année dont personne n'a jamais rêvé dans ma philosophie - je me demande parfois si ta mère ne fait pas que me ménager et que les autorités t'ont emmené le jour où j'ai commencé à penser que tu savais utiliser la magie - donc je ne peux pas nier qu'il est \emph{possible} que tu ai développé une sorte de… je ne sais pas bien comment l'appeler, mais 'côté obscur' semble prématuré tant qu'on ne sait pas ce qui se passe. Es-tu certain que ce n'es pas un talent télépathique bourgeonnant et que tu ne captes pas les esprits des autres sorciers autour de toi~? Leurs pensées pourraient sembler maléfiques à un enfant ayant grandi dans une civilisation plus saine d'esprit. J'admets que ce sont des spéculations sans fondement, mais tu ne devrais pas non plus tirer de conclusions hâtives.

Les deux choses les plus importantes que j'ai à te dire sont~: D'abord, mon fils, j'ai une confiance \emph{totale} en ta capacité à rester du côté clair de la Force tant que tu le choisira, et j'ai une confiance totale en ce choix. S'il y a quelque esprit maléfique qui te chuchote d'horribles conseils, ignore simplement ces conseils. Je \emph{ressens} le besoin d'insister sur le fait que tu devrais faire particulièrement attention à ignorer cet esprit maléfique même si ce qu'il suggère te semble être une idée merveilleusement créative et j'espère ne pas avoir besoin de te rappeler l'Incident du Projet Scientifique qui, je l'admet, serait beaucoup plus compréhensible si tu étais en proie à une possession démoniaque.

La seconde chose que j'ai à te dire, c'est que tu ne dois pas craindre que Maman et moi t'abandonnions à cause de ton 'côté obscur'. Nous ne nous attendions peut-être pas à ce que tu obtiennes des pouvoirs magiques ou que tu développes une affinité pour la magie noire, mais nous nous attendions à ce que tu deviennes un adolescent. Ce qui, si tu y réfléchis du point de vue de ton pauvre père, est une perspective suffisamment inquiétante quant il s'agit d'un enfant qui, à l'age de neuf ans, s'était rendu responsable de la venue d'un total de cinq camions de pompiers. Les enfants grandissent. Je ne te mentirai pas en te disant que tu te sentiras aussi proche de nous à 20 ans que tu l'es maintenant. Mais ta mère et moi nous sentirons aussi proche de toi quand nous serons vieux et gris et quand nous embêterons les robots de la maison de retraite. Les enfants grandissent toujours en s'éloignant de leurs parents, et les parents les suivent toujours, offrant des conseils utiles. Les enfants grandissent, leurs personnalités changent, ils font des choses que leurs parents auraient aimé qu'ils ne fassent pas, et ils manquent de respect envers leurs parents et les font sortir de leur école magique, et leurs parents continuent quand même de les aimer. Ainsi va la nature. Bien qu'au cas où tu n'as pas encore atteint ta puberté et que tes années adolescentes soient proportionnellement pire, nous nous réservons le droit de reconsidérer ce point de vue.

Peu importe ce qui se passe, souviens-toi que nous t'aimons et que nous t'aimerons toujours, quoi qu'il arrive. Je ne sais pas si notre amour a quelque pouvoir magique sous vos règles, mais si c'est le cas, n'hésites pas à y faire appel.

Tout cela étant dit… Harry, ce que tu as fait là n'est pas acceptable. Je pense que tu le sais. Et je sais aussi que ce n'est pas le moment de te faire la leçon à ce sujet. Mais tu dois nous écrire et nous dire ce qui se passe. Je peux très bien comprendre pourquoi tu voudrais nous faire sortir de ton école le plus rapidement possible, et je sais que nous ne pouvons te forcer à rien, mais s'il te plaît, Harry, sois raisonnable et comprends à quel point nous devons être terrifiés.

Je voudrais te dire qu'il t'est absolument interdit de jouer avec des magies que les adultes autour de toi considèrent le moins du monde dangereuses~; mais pour ce que j'en sais, les enseignants enseignent à toute l'école des cours de nécromancie avancée tous les lundis. S'il te plaît, s'il te plaît sois aussi prudent que la situation le permet, quelle que soit cette situation. Malgré ton résumé très empressé nous n'avons pas la moindre idée de ce qui se passe et j'espère que tu nous en écrira autant que tu peux. Il est clair que tu es, par certains aspects au moins, en train de grandir, et j'\emph{essaierai} de ne pas me comporter comme le père des livres pour enfants qui ne fait qu'aggraver les choses - même si j'espère que tu apprécies à quel point c'est difficile - et ta mère m'a dit un certain nombre de choses effrayantes sur la façon dont le monde sorcier reste un secret et comment je pourrais \emph{vous} causer du tort si je faisais des vagues. Je ne peux pas te dire d'éviter tout ce qui n'est pas sûr parce que ton école ne l'est pas et que ton directeur refuse de te laisser partir. Je ne peux pas te dire de n'endosser \emph{aucune} responsabilité vis à vis des événements qui t'entourent, parce que si ça se trouve il y a d'autres enfants en danger. Mais souviens toi que tu n'es moralement responsable de la protection d'aucun adulte, leur rôle est de te protéger et n'importe quel adulte bon serait d'accord. Écris, s'il te plaît, et dis-nous en plus dès que tu le pourras.

Nous souhaitons tous deux t'aider, désespérément. S'il y a quoi que ce soit que nous puissions faire, dis-le nous aussi vite que possible. Rien de pire ne pourrait nous arriver que t'apprendre que quelque chose t'est arrivé.

\letterClosing[Je t'aime,]{Papa.}
\end{writtenNote}

La dernière page disait seulement~:

\begin{writtenNote}
Tu m'as promis que tu ne laisserai pas la magie t'arracher à moi. Je ne t'ai pas appris à être un garçon qui trahi les promesses qu'il fait à sa maman. Tu dois revenir sain et sauf, parce que tu l'as promis.

\letterClosing[Je t'aime,]{Maman.}
\end{writtenNote}

Lentement, Harry abaissa les lettres et commença à marcher vers la grande salle. Ses mains tremblaient, tout son corps tremblait, et il lui semblait très difficile de ne pas pleurer~; ce que, sans avoir à se le dire, il savait devoir ne pas faire. Il n'avait pas pleuré de la journée. Et il ne pleurerait pas. Pleurer revenait à admettre sa défaite. Et ce n'était pas terminé. Il ne pleurerait donc pas.

\later

La nourriture servie dans la grande salle ce soir là fut simple, des tartines, du beurre et de la confiture, de l'eau et du jus d'orange, du porridge et autres pitances, aucun dessert. Quelques élèves avaient endossé de simples robes noires sans les couleurs de leur maison. D'autres portaient toujours les leurs. Cela aurait dû provoquer des disputes, mais au lieu de cela il avait un certain calme, le son de gens en train de manger sans parler. Il fallait deux camps pour faire un débat, et l'un des camps, cette nuit, n'avait pas vraiment goût au débat.

La directrice adjointe Minerva McGonagall s'assit à la grande table et ne mangea pas. Elle aurait dû le faire. Peut-être le ferait-elle bientôt. Mais elle ne pouvait pas se forcer à le faire maintenant.

Pour une Gryffondor, il n'y avait qu'une seule voie. Il n'avait fallu que peu de temps à Minerva pour s'en souvenir quand, après les exhortations du professeur de Défense, son esprit était demeuré vide de complot à tenter. Ce n'était pas la voie des Gryffondor~; ou peut-être aurait-elle dû dire que ce n'était pas \emph{sa} voie, Albus semblait bien s'essayer au complot… et pourtant, lorsqu'elle repensait à leur histoire, il n'y avait pas de complot lors des moments de crise, pas d'astuce ou de plan de dernier recours. Pour Albus Dumbledore, comme pour elle, la règle \emph{in extremis} était de décider de la bonne chose à faire et de la faire, peu importe le coût pour soi. Même si cela signifiait dépasser ses limites, changer de rôle, ou se défaire de son image de soi. C'était le dernier recours des Gryffondor.

Elle vit Harry se glisser discrètement à l'intérieur par une entrée latérale de la grande salle.

Il était temps.

Le professeur Minerva McGonagall se leva de sa chaise, redressa la pointe usée de son chapeau et marcha lentement jusqu'au pupitre situé devant la grande table.

Le bruit de la grande salle, déjà minime, se tut entièrement lorsque tous les élèves se tournèrent pour la regarder.

"Vous savez tous," dit-elle d'une voix pas tout à fait assurée. \emph{Que Hermione Granger est morte.} Elle ne prononça pas ces mots à voix haute, puisqu'ils le savaient tous. "Quelqu'un est parvenu à faire entrer un troll dans le château de Poudlard sans activer les systèmes de sécurité. Ce troll est parvenu à faire du mal à une élève, sans activer les systèmes de sécurité avant qu'elle ne meure. Des enquêtes visant à déterminer comment cela a pu avoir lieu sont en cours. Le conseil d'administration se réunit pour déterminer la façon dont Poudlard réagira à ceci. Justice sera rendue en temps et heure. En attendant, une autre question de justice doit être immédiatement traitée. George Weasley et Fred Weasley, venez s'il vous plaît, avancez-vous devant tout le monde."

Les jumeaux Weasley échangèrent des regards, assis à la table Gryffondor, puis ils se levèrent et marchèrent vers elle, lentement, avec réticence~; et c'est alors que Minerva se rendit compte que les jumeaux Weasley s'attendaient à ce qu'elle les exclue.

Ils pensaient sincèrement qu'elle allait les exclure.

C'était là l'œuvre de l'image du professeur McGonagall qui vivait dans sa tête.

Les jumeaux Weasley marchèrent jusqu'au pupitre en la regardant avec des visages effrayés mais résolus~; et elle sentit quelque chose dans son cœur se fissurer encore un peu.

"Je ne vais pas vous exclure," dit-elle, et elle fut plus attristée encore en voyant leur air surpris. "Fred Weasley, George Weasley, retournez-vous, faites face à vos camarades, laissez-les vous regarder."

Toujours l'air surpris, les jumeaux Weasley s'exécutèrent.

Elle s'arma de toute sa volonté et dit ce qu'il y avait à dire.

"J'ai honte," dit Minerva McGonagall, "de ce qui s'est passé aujourd'hui. J'ai honte que vous n'ayez été que deux. Honte de ce que j'ai fait à Gryffondor. De toutes les maisons, c'est Gryffondor qui aurait dû aider Hermione Granger quand elle en avait besoin, quand Harry Potter en appelait aux braves pour qu'ils l'aident. C'est vrai, un étudiant en septième année aurait pu retenir un troll des montagnes tout en cherchant Mlle Granger. Et vous auriez dû penser que la directrice de la maison Gryffondor," sa voix se brisa, "croirait en vous si vous lui désobéissiez pour faire ce qui était juste, dans des circonstances qu'elle n'avait pas prévues. Je ne croyais pas aux vertus de Gryffondor. J'ai essayé d'écraser toute tendance à la rébellion au lieu d'entraîner votre courage jusqu'à la sagesse. Quoi que le Choixpeau ait pu voir en moi pour me placer à Gryffondor, je l'ai trahi. J'ai transmis au directeur ma démission de mes postes de directrice adjointe et de directrice de la maison Gryffondor.

\later

Il y eut des cris de surprise, de désarroi, et pas seulement venus de la table Gryffondor, tandis qu'au même instant le cœur de Harry gelait dans sa poitrine. Il fallait qu'il coure, qu'il dise quelque chose, il n'avait pas voulu que \emph{ça} se…

\later

Minerva respira et poursuivit. "Cependant, le directeur n'a pas accepté ma démission," dit-elle. "Je continuerai donc mon travail et essaierai de défaire ce que j'ai façonné. Je dois parvenir à trouver un moyen d'enseigner à mes élèves comment faire ce qui est juste. Pas ce qui est sûr, pas ce qui simple, pas ce qu'on nous dit de faire. Si tout ce dont je suis capable est de vous apprendre à rendre vos devoirs à l'heure, autant ne pas avoir de maison Gryffondor. Cette voie sera plus difficile pour moi, et peut-être pour nous tous. Mais je sais à présent que jusqu'ici, je ne faisais que suivre la voie de la facilité."

Elle s'écarta du pupitre et descendit pour rejoindre les jumeaux Weasley.

"Fred Weasley, George Weasley," dit-elle. "Vous n'avez pas toujours fait ce qui est juste. Le chemin vers la sagesse n'est pas semé de défis éhontés et inutiles envers l'autorité. Et pourtant, aujourd'hui, vous avez prouvé être les derniers de notre maison à avoir survécu à mes erreurs. Parce que c'était la bonne chose à faire, vous avez défié la menace d'expulsion et avez risqué vos vies face à un troll des montagnes. Pour votre courage stupéfiant, qui fait honneur à votre maison, je vous décerne à chacun deux-cents points pour Gryffondor."

Encore cet air surpris sur leurs visages, encore la douleur, comme un lame à travers son cœur.

Elle se retourna pour faire face aux autres élèves.

"Je ne décernerai aucun point à Serdaigle," dit-elle. "Je devine que M. Potter n'en voudrait pas. Si j'ai tort, qu'il me corrige et prenne autant de points qu'il le souhaitera. Mais pour ce que ça vaut, M. Potter, je suis," elle hésita, "je suis désolée…"

\later

"\emph{Arrêtez~!}" hurla Harry, puis, encore~: "Arrêtez." Le mot râpait sa gorge. "Vous n'avez pas à faire ça, professeur." Quelque chose en lui se tordait, menaçait de s'ouvrir grand, comme des mains immenses qui l'auraient saisit pour l'ouvrir en deux. "Et, et vous ne devriez pas oublier Susan Bones et Ron Weasley… eux aussi ont aidé, ils devraient aussi recevoir des points…

--- Mlle Bones et le jeune Weasley~?" dit le professeur McGonagall. "Rubeus ne m'a rien dit de cela… qu'ont-ils fait~?

--- Mlle Bones a essayé d'étourdir M. Hagrid quand il a essayé de m'arrêter, et M. Weasley a tiré sur Neville quand Neville a essayé de m'arrêter. Ils devraient aussi avoir des points et, et Neville aussi," Harry n'avait pas pensé à y songer, à la façon dont Neville devait maintenant se sentir, mais à l'instant où il y pensa, il sut, "parce que Neville a essayé de faire quelque chose, même si ce n'était pas juste, faire ce qui est juste est la \emph{deuxième} leçon, on peut commencer à la pratiquer après avoir appris à faire quelque chose, tout court…

--- Dix points pour Poufsouffle, Mlle Bones," dit le professeur McGonagall d'une voix qui se brisa à mi-phrase. "Dix points à Gryffondor, Ron Weasley, votre famille a de quoi être extrêmement fière, aujourd'hui. Et dix points à Poufsouffle pour Neville Londubat, pour avoir fait face à M. Potter et avoir fait ce qu'il pensait être juste…

--- \emph{Vous ne devriez pas~!}" hurla une jeune voix à la table Poufsouffle, suivie d'un unique hoquet.

Harry regarda dans cette direction et revint rapidement au professeur McGonagall avant de dire d'une voix aussi assurée que possible~: "En fait Neville a raison, on ne peut pas décerner littéralement zéro points pour avoir agi correctement tout en envoyant le mauvais message, mais il était à mi-chemin alors il devrait plutôt avoir cinq points."

Le professeur McGonagall eut l'air, l'espace d'un instant, de ne pas savoir quoi dire~; puis ses yeux se dirigèrent vers Neville et elle dit~: "Comme vous le souhaitez, M. Potter. Qu'y a-t-il, Mlle Bones~?"

Harry regarda et vit que Susan Bones s'était avancé, qu'elle s'essuyait les yeux, et la Poufsouffle dit~: "En fait - professeur McGonagall - le général Potter ne l'a pas vu - mais le capitaine Weasley et moi n'avons pas été les seuls à tenter de nous interposer face à M. Hagrid après que Harry fut sorti. Avant que certains des élèves plus âgés ne nous arrêtent. Mais nous sommes parvenus à ralentir M. Hagrid pendant une minute pour que le général Potter puisse partir.

--- Vous devez leur donner des points à eux aussi," dit Ron Weasley depuis la table Gryffondor. "Ou je n'en accepterai aucun.

--- Qui d'autre~?" dit le professeur McGonagall d'une voix peu stable.

Sept autre enfants se levèrent.

\emph{Qui disait notre côté Serpentard, qu'il prédisait que rien ne fonctionnerait jamais~?} dit Poufsouffle.

Quelque chose craqua en Harry, si bien qu'il dut faire usage de toute sa force pour se retenir.

\later

Lorsque tout fut dit, lorsque tout fut fait, Minerva alla rejoindre Harry Potter. Bien que ce ne fut pas un domaine où elle excellait, elle lança une barrière afin de brouiller la vue, puis d'une autre pensée elle étouffa les sons.

"Vous, vous n'aviez pas à…" dit Harry Potter. "Vous n'auriez pas dû dire…" Il avait l'air de s'étrangler. "P-Professeur, tout ce que je vous ait dit, je cherchais à faire mal, à calmer ma haine, j'avais tort…

--- Je le savais déjà, Harry Potter," dit-elle. "Mais je souhaitais quand même m'améliorer." Il y avait une sensation de légèreté dans sa poitrine, très similaire à ce qu'on pouvait ressentir en sautant d'une falaise, quand les jambes n'avaient plus à soutenir le corps. Elle n'étais pas certaine de pouvoir faire ça, elle ne savait pas comment le faire~; et pourtant, pour la première fois, il lui sembla possible que Poudlard ne devienne pas une triste chimère de ce qu'elle était le jour où elle deviendrait sa directrice.

Harry la regarda puis émit un étrange son qui sembla avoir été forcé hors de sa gorge avant de couvrir son visage de ses mains.

Elle s'agenouilla et le prit dans ses bras. Peut-être que ça se passerait mal, mais peut-être que ça se passerait bien, et elle n'allait pas laisser cette incertitude l'arrêter~; il était temps qu'elle commence à apprendre le courage d'un Gryffondor afin qu'elle puisse ensuite l'enseigner.

"Avant, j'avais une sœur," chuchota-t-elle. Juste cela, rien de plus.

\later

\emph{Juste pour être sûr}, dit une partie de Harry, pendant que le reste sanglotait dans les bras du professeur McGonagall, \emph{ça ne veut pas dire qu'on a accepté la mort de Hermione, hein~?}

\emph{\shout{Non}} dit tout le reste de son être, chaque partie de son esprit d'un accord unanime, le chaud, le froid, ainsi qu'un lieu caché fait d'acier. \emph{Jamais, à jamais.}

\later

Et un vieux sorcier pour qui cette barrière n'était rien les regarda tous deux, la sorcière et le jeune sorcier en larmes. Albus Dumbledore souriait, une étrange tristesse dans les yeux, comme quelqu'un qui aurait fait un pas de plus vers une destination prévue.

\later

Le professeur de Défense les regardait tous deux, la femme et l'enfant qui pleurait. Ses yeux étaient très froids et très calculateurs.

Il ne pensait pas que cela suffirait.

\later

Ce n'est que le lendemain matin qu'il fut découvert que le corps de Hermione Granger avait disparu. 

%  LocalWords:  arry Tener
