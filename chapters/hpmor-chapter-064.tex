% Omake section titles

\makeatletter
\newcommand{\OmakeIVspecialsection}[2][1.5]{%
\vspace*{2\baselineskip plus 1\baselineskip minus 1\baselineskip}%
\noindent\hfill\scalebox{#1}{#2}\hfill\mbox{}%
\vskip 1\baselineskip plus 1\baselineskip%
\@afterindentfalse\@afterheading
}
\makeatother

\newcommand{\OmakeIVsection}[2][1.5]{%
  \OmakeIVspecialsection[#1]{\MakeUppercase{#2}}}

\chapter{Fichiers Omaké IV, Parallèles alternatif}

\OmakeIVspecialsection[1.6]{\fontspec[ExternalLocation]{RingBearer}
\settowidth{\versewidth}{\mbox{Le}}
Seigneur\scalebox{.40}{\parbox[b]{\versewidth}{%
        \centering de\\\nointerlineskip\vskip 4pt la}}Rationalit\raisebox{-.32ex}{É}}

Frodon scruta tous les visages mais aucun n'était tourné vers lui. Tout le Conseil s'assit, les yeux baissés, comme en pleine réflexion. Une grande terreur tomba sur lui comme s'il était sur le point de recevoir la nouvelle d'une catastrophe qu'il avait longtemps présagée et dont il avait vainement espéré qu'elle ne fut jamais mentionnée. Un désir impérieux de se retirer et de demeurer aux côtés de Bilbo à Fondcombe emplit son cœur. Il parla enfin au prix d'un effort et s'étonna d'entendre ses propres paroles comme si une autre volonté utilisait sa petite voix.

<<~Non, dit Frodon. Il ne faut pas que nous fassions cela. Ne comprenez-vous pas~? C'est exactement ce que l'Ennemi désire. Tout cela, il l'a prédit.~>>

Les visages se tournèrent vers lui, perplexes les Nains, graves les Elfes~; de la dureté dans le regard des Hommes~; et animés d'une telle ferveur ceux d'Elrond et de Gandalf, à un point tel que Frodon ne pouvait presque pas le supporter. Il fut alors très difficile de ne pas se saisir de l'anneau, encore plus difficile de ne pas le mettre et de leur faire face seulement en tant que Frodon.

<<~Ne vous interrogez-vous pas~?~>> dit Frodon d'une voix légère comme le vent et tremblante comme une brise. <<~Entre toutes choses, vous avez choisi d'envoyer l'anneau en Mordor~; ne devriez-vous pas vous interroger~? Comment en sommes-nous venu à cela~? À ce que, de tous nos choix possibles, nous faisons celui que notre Ennemi désire le plus~? Peut-être la Crevasse du Destin est-elle déjà gardée, suffisamment pour repousser Gandalf, Elrond et Glorfindel de concert~; ou peut-être le Maître de cet endroit a-t-il refroidi la lave qui s'y trouve, peut-être piégera-t-elle simplement l'anneau de façon à ce qu'il puisse simplement l'en extraire après qu'il y eut été jeté…~>> Un souvenir d'une terrible clarté revint alors à Frodon, accompagné d'un éclat de rire noir, et l'idée lui vint que c'était \emph{exactement} ce que l'Ennemi ferait. Sauf que l'idée lui était venue ainsi~: \emph{c'est ce qu'il me plairait de faire si je comptais régner…}

Des regards pleins de doute furent échangés au sein du Conseil~; Glóin, Gimli et Boromir regardaient les Elfes avec plus de scepticisme qu'avant, comme s'ils venaient de s'éveiller d'un rêve de mots.

<<~L'Ennemi est très sage, dit Gandalf, et pèse toutes choses avec soin sur la balance de sa malveillance. Mais sa seule mesure est celle du désir, du désir de pouvoir~; ainsi juge-t-il tous les cœurs. Jamais dans le sien ne germera l'idée qu'on puisse le refuser, qu'une fois en la possession de l'Anneau l'on puisse chercher à le détruire.

--- Il y \emph{pensera}~!~>> s'écria Frodon. Il cherchait ses mots, tentait de véhiculer des choses qui un moment auparavant lui avaient semblé limpides et fondaient maintenant comme de la neige. <<~Si l'Ennemi pensait que tous ses adversaires n'étaient mus que par le désir du pouvoir - alors il se tromperait, encore et encore, et alors le Faiseur de cet Anneau le \emph{verrait} et \emph{saurait} qu'il a fait une erreur quelque part~!~>> Frodon tendit ses mains d'un air plaidant.

Boromir s'agita et sa voix fut pleine de doute~: <<~Vous dites grand bien de l'Ennemi, pour un de ses adversaires.~>>

La bouche de Frodon s'ouvrit et se referma sous l'effet d'une stupéfaction teintée de désespoir~; car il savait, il savait que l'Homme était fou, mais il ne trouvait rien à répondre.

Puis Bilbo parla, et sa voix flétrie fit taire toute la pièce, même Elrond, qui avait été sur le point de parler. <<~J'ai peur que Frodon n'ait raison, murmura le vieil hobbit. Je me souviens, je me souviens de ce que c'était. De voir avec le Regard Sombre. Je m'en souviens. L'Ennemi pensera que nous ne pourrons nous faire confiance, que les plus faibles d'entre nous proposerons de détruire l'Anneau afin que les plus forts ne l'aient pas. Il sait que même une personne dont la bonté n'est pas véritable pourra quand même enjoindre de le détruire afin de faire montre de cette bonté qu'il n'a pas. Et l'Ennemi ne pensera \emph{pas} qu'il est impossible que ce Conseil prenne une telle décision car, voyez-vous, il ne s'attend pas à ce que nous fassions preuve de sagesse.~>> Un gloussement murmuré s'éleva de la gorge du vieil hobbit. <<~Et quand bien même - allons, il garderait \emph{quand même} la Crevasse du Destin. Cela lui coûterait si peu.~>>

Un sombre pressentiment était maintenant visible jusqu'aux visages des Elfes et du Sage~; Elrond faisait la grimace et les sourcils drus de Gandalf se fronçaient.

Frodon les regarda tous et sentit une démence, un désespoir le surplomber~; et une faiblesse dans son cœur laissa sa vision s'obscurcir, s'emplir de ténèbres, vaciller. Depuis ces ténèbres, Frodon vit Gandalf, et la force du sorcier fut révélée être une faiblesse, sa sagesse, une folie. Car Frodon, dont le sein était alourdi, tiraillé par l'Anneau, sut que Gandalf n'avait à aucun moment pensé à l'Histoire et aux mythes lorsqu'il avait dit que l'Ennemi ne comprendrait aucun désir autre que celui du pouvoir~; qu'il ne s'était pas souvenu de la façon dont Sauron avait rabaissé et corrompu les Hommes de Númenor au beau milieu de leurs jours de gloire. Tout comme il n'était pas venu à l'esprit de Gandalf que l'Ennemi pourrait apprendre à comprendre ses adversaires emplis de bonté en les \emph{observant}…

Le regard de Frodon passa à Elrond, mais depuis sa vision obscurcie il n'y avait là aucun espoir, au secours~; car Elrond avait laissé Isildur partir, il l'avait laissé emmener l'Anneau depuis la Crevasse du Destin, là où il aurait dû être détruit, au prix de toute cette guerre. Cela n'avait servi ni Isildur ni leur amitié car l'Anneau avait fini par tuer Isildur, et d'autres destins bien pires auraient pu lui échoir. Mais la catastrophe qui avait émané de l'acte d'Isildur avait alors semblé peu certaine à Elrond, peu certaine et lointaine~; et pourtant le coût pour Elrond d'appliquer le pommeau de son épée à l'arrière du crâne d'Isildur aurait été plus sûr, plus proche…

Comme mû par le désespoir, Frodon se tourna pour regarder Aragorn, l'homme au visage buriné qui devant ce Conseil revêtait des vêtements usés par le voyage, l'héritier de rois qui parlait aux hobbits avec gentillesse. Mais Frodon vit double, et dans la deuxième image ténébreuse il vit un Homme qui avait passé une trop grande partie de sa jeunesse entouré d'Elfes, qui avait appris à porter des vêtements humbles et tachés au beau milieu de l'or et des joyaux, sachant qu'il ne pourrait jamais égaler leur sagesse et désireux de les surpasser dans un domaine où ils ne s'aventureraient pas…

Par le regard de l'Anneau, qui n'était autre que ce lui de son Faiseur, toutes les choses nobles s'effaçaient pour révéler des stratagèmes et des mensonges, un monde de gris et de ténèbres dénué de lumière. Ils n'avaient pas fait leur choix en toute connaissance de cause, Gandalf, Elrond et Aragorn~; leurs pulsions étaient venues des parties cachées de leur être, des noires et secrètes profondeurs que l'Anneau rendait évidentes aux yeux de Frodon. Se montreraient-ils supérieurs à Sauron, eux qui ne pouvaient comprendre ni eux-mêmes ni les forces qui les faisaient agir~?

Le chuchotement acéré de Bilbo lui parvint~: <<~Frodon~!~>>, il revint à lui et interrompit sa main qui s'était élevée vers l'anneau posé contre sa poitrine, au bout d'une chaîne~; il avait l'impression qu'un immense rocher était attaché à son cou.

Qui s'était élevée pour saisir l'Anneau détenteur de toutes les réponses.

<<~Comment as-tu pu porter cette chose~?~>> murmura Frodon à Bilbo comme s'ils avaient été les deux seules âmes de la pièce, alors même que le Conseil les regardait.

<<~Pendant des années~? Je n'arrive pas à l'imaginer.

--- Je l'ai gardé dans une pièce fermée dont seul Gandalf avait la clé, dit son oncle, et quand je commençais à imaginer des moyens de l'ouvrir, je me souvenais de Gollum.~>>

Un frisson parcourut Frodon lorsqu'il se souvint des histoires. L'horreur des Monts Brumeux, à réfléchir, toujours à réfléchir dans le noir~; régner sur les gobelins depuis les ombres, à farcir les tunnels de pièges~; si Bilbo n'avait pour la première fois porté l'anneau cette fois-ci, aucun nain n'aurait survécu. Et aujourd'hui, selon l'Elfe Legolas, Gollum avait abandonné l'idée d'envoyer ses agents contre la Comté et avait fini par trouver le courage de quitter ses montagnes et de partir lui-même en quête de l'Anneau. C'était Gollum, et c'était un destin que Frodon partagerait si l'Anneau n'était pas détruit.

Sauf qu'ils n'avaient aucun moyen de le détruire.

L'Ombre avait prévu chacun des coups qui leur étaient offerts. Elle avait \emph{presque} - Frodon n'arrivait toujours pas à imaginer comment cela avait été accompli, comment l'Ombre avait manigancé un telle chose - elle avait \emph{presque} poussé le Conseil à envoyer l'Anneau droit vers le Mordor accompagné seulement d'une garde réduite, ce qu'il aurait fait si Frodon et Bilbon n'avaient été là.

Et ayant évité la plus rapide des défaites possibles, la seule question qui demeurait était de savoir combien de temps ils mettraient à perdre. Gandalf avait différé trop longtemps, bien trop longtemps avant de déclencher ce mouvement. Cela aurait été si simple si Bilbon s'y était attelé huit ans plus tôt, si seulement on lui avait dit ce que Gandalf soupçonnait alors déjà, si seulement le cœur de Gandalf n'avait pas silencieusement reculé devant la perspective d'avoir tort et d'être embarrassé…

La main de Frodon eut une spasme contre sa poitrine~; sans qu'il y pense, ses doigts recommencèrent à s'élever vers l'immense poids de la chaîne à laquelle l'Anneau était pendu.

Il n'avait qu'à le mettre.

Cela seulement, et tout deviendrait clair, la lenteur et la vase quitteraient ses pensées, tous les possibles et tous les futurs deviendraient transparents à ses yeux, son regard percevrait les plans de l'Ombre et deviserait une contre-attaque imparable -

- et il ne pourrait plus jamais l'enlever, pas cette fois, pas avec ce qu'il lui resterait de volonté. Les seuls souvenirs que Frodon avait de ces moments s'effaçaient mais il savait qu'il lui avait semblé mourir lorsqu'il avait laissé ses tours de pensées s'effondrer et qu'il était redevenu Frodon. Il lui avait semblé mourir, voilà tout ce qu'il se souvenait d'Amon Sûl. Et s'il devait à nouveau porter l'Anneau, il préférerait mourir en l'ayant au doigt, mettre un terme à sa vie alors qu'il serait encore lui-même~; car il savait qu'il ne pourrait pas supporter ces effets une seconde fois, pas après, lorsque la clarté infinie lui aurait été ravie…

Frodon observa le Conseil, les pauvres Sages perdus et sans chef, et il sut qu'ils ne pourraient pas vaincre l'Ombre par eux-mêmes.

<<~Je le porterai une dernière fois~>>, dit Frodon d'une voix brisée et mourante, comme il avait depuis le début su qu'il finirait par le dire, <<~une dernière fois pour trouver une réponse pour ce Conseil, puis il y aura d'autres hobbits.

--- \emph{Non~!}~>> s'écria la voix de Sam, et l'autre hobbit commença à se précipiter hors de sa cachette alors même que Frodon, d'un mouvement aussi rapide et précis que celui d'un Nazgûl, sortait l'Anneau de sous sa chemise~; et soudain Bilbon était déjà là et son doigt avait déjà traversé l'anneau.

Tout eut lieu avant que Gandalf ne puisse brandir son bâton, avant qu'Aragorn ne puisse lever son tronçon d'épée~; les Nains s'exclamèrent et les Elfes furent estomaqués.

<<~Bien sûr, dit la voix de Bilbon alors que Frodon commençait à pleurer, je vois, je comprends tout, enfin. Écoutez, écoutez-moi bien, voici ce que vous devez faire -~>>

% LA SORCIÈRE BLANCHE ET L'ARMOIRE MAGIQUE
\OmakeIVspecialsection[5]{\fontspec[ExternalLocation]{NarniaBLL}456}

Peter regarda d'un œil critique le camp des Centaures avec leurs arcs, des Castors avec leurs dagues et des Ours parlants vêtus de leur cotte de maille. Il commandait parce qu'il était l'un des mythiques Fils d'Adam et s'était déclaré Haut Roi de Narnia~; mais en vérité il n'y connaissait pas grand chose en camps, en armes et en patrouilles de garde. En fin de compte, tout ce qu'il pouvait voir c'était qu'ils semblaient tous fiers et confiants, et Peter devait espérer qu'ils avaient raison sur ce point~; parce que si on ne pouvait pas compter sur les siens, on ne pouvait compter sur personne.

<<~Ils \emph{me} feraient peur si je devais les combattre, dit-il enfin, mais je ne sais pas si ça suffira à la battre… \emph{elle.}

--- Tu ne penses pas que ce lion mystérieux va débarquer et nous aider~?~>> dit Lucy. Sa voix était très basse, afin qu'aucune des créatures autour d'eux ne puisse les entendre. <<~C'est juste que ça serait bien de vraiment l'avoir avec nous, tu crois pas, au lieu de juste laisser penser aux gens qu'il nous a confié le commandement~?~>>

Susan secoua la tête, agitant ainsi les flèches magiques du carquois accroché dans son dos.

<<~Si une telle personne existait vraiment, dit Susan, n'aurait-il pas empêché la Sorcière blanche d'engloutir le monde dans l'hiver pendant cent ans~?

--- J'ai eu un rêve des plus étranges, dit Lucy d'une voix encore plus basse, où on n'avait ni à unir des créatures ni à les convaincre de se battre, nous arrivions ici et le lion était déjà là, les armées étaient déjà rassemblées, et il partait, il sauvait Edmund et nous chevauchions avec lui dans cette incroyable bataille lors de laquelle il tuait la Sorcière blanche…

--- Le rêve avait-il une morale~? demanda Peter.

--- Je ne sais pas, dit Lucy en clignant des yeux et avec un air légèrement perplexe. Je ne sais pas pourquoi, mais dans le rêve, tout ça semblait futile.

--- Je pense que le monde de Narnia essayait peut-être de te dire, dit Susan, ou peut-être que c'était juste tes rêves qui essayaient de te dire que si quelqu'un comme ce lion existait, \emph{nous} n'aurions aucune utilité.~>>

\OmakeIVsection{My little pony~: L'amitié c'est la Science}

<<~Applejack, qui m'a tout de suite dit que je me trompais, représente l'esprit… \emph{d'honnêteté~!}~>> le poney au poil sombre éleva encore plus sa tête, sa crinière s'agitant comme le vent autour de la nuit noire de son long cou, ses yeux étincelant tels des étoiles. <<~Fluttershy, qui s'est approchée de la manticore pour y découvrir l'épine qu'elle avait dans sa patte, représente l'esprit… \emph{d'investigation}~! Pinkie Pie, qui s'est rendu compte que les horribles visages n'étaient que des arbres, représente l'esprit… \emph{de formulation d'hypothèses alternatives~!} Rarity, qui a résolu le problème du serpent, représente l'esprit… \emph{de créativité} \emph{!} Rainbow Dash, qui a su percer à jour la fausse promesse de ses désirs, représente l'esprit… \emph{d'analyse~!} Marie-Susan, qui nous a forcé à la convaincre de nos théories avant de financer notre expédition, représente l'esprit… \emph{d'évaluation par les pairs~!} Et lorsque ces Éléments sont allumés par l'étincelle de curiosité qui réside dans le cœur de chacun de nous, ils créent le septième élément, l'Élément de la Sci-~>>

L'explosion de pouvoir qui eut lieu fut semblable à un vent de lave lumineuse, elle saisit Marie-Susan avant que le poney n'aie pu ne serait-ce que tressaillir, arracha la chair de ses os, transforma ses os en cendres, avant même que les autres n'aient pu se cabrer de peur.

Depuis la sombre chose qui se tenait au centre de l'estrade, là où les Éléments avaient été brisés, depuis la folie cinglante et le désespoir qui entouraient la silhouette à peine reconnaissable d'un cheval noir néant vint une voix qui sembla traverser toutes les oreilles et brûler tel un feu froid, résonnant directement dans le cerveau de tous les poneys qui pouvaient l'entendre~:

\emph{Vous vous attendiez à ce que je me tienne là et que je vous laisse finir~?}

Les cris commencèrent alors, faisant écho dans cette salle du trône ancienne et abandonnée~; et Applejack tomba sur son toupet à côté des cendres encore rougeoyantes qui étaient tout ce qui restait des os de Marie-Susan. Elle était trop anéantie pour pouvoir sangloter.

Twilight Sparkle regarda l'horreur qui avait un jour été la jument Séléniaque, et son esprit fut submergé par un désespoir frénétique lorsqu'elle comprit que c'était fini, qu'elles étaient foutues, que tout espoir était perdu sans Marie-Susan, car tout le monde savait que peu importe à quel point vous étiez honnête, investigateur, sceptique, créatif, analytique ou curieux, ce qui transformait vraiment vos travaux en Science c'était de publier vos résultats dans un journal prestigieux. Tout le monde le savait…

\clearpage
\OmakeIVsection{Le village caché dans la lucidité\protect\footnotemark}
\footnotetext{This has now inspired an extended fanfiction, \emph{Lighting Up the Dark} by Velorien.}

<<~Considères la puissance de calcul nécessaire pour générer plus d'une centaine de clones de l'ombre,~>> dit le génie Uchiha d'un ton totalement objectif et froid. <<~C'est une erreur de rationalité, Sakura, que de dire “coup de chance” et de penser que tu as expliqué quoi que ce soit. “Coup de chance” est simplement le nom que l'on donne aux données que l'on ignore.

--- Mais ça \emph{doit} être un coup de chance~!~>> s'écria Sakura. Au prix d'un effort, elle parvint à calmer sa voix et à lui donner la précision méticuleuse attendue d'un ninja de la rationalité~; elle ne laisserait pas la cible de ses émois penser qu'elle était stupide. <<~Comme tu l'as dit, la puissance de calcul nécessaire pour utiliser plus d'une centaine de clones de l'ombre est tout simplement absurde. C'est du niveau d'une super-intelligence majeure. Naruto est bon dernier de notre classe. Il n'a même pas l'intelligence d'un jounin normal, encore moins d'une super-intelligence~!~>>

Les yeux du Uchiha brillèrent presque comme s'il avait activé son Malingan. <<~Naruto peut donner forme à cent clones agissant indépendamment les uns des autres. Il \emph{doit} avoir la puissance cérébrale nécessaire. Mais dans des circonstances normales, quelque chose l'empêche d'utiliser sa puissance de calcul de façon efficace… comme un esprit en guerre contre lui-même, peut-être~? Nous avons maintenant une raison de croire que Naruto est connecté à une super-intelligence, d'une façon ou d'une autre, et en tant que genin récemment diplômé, il a comme nous quinze ans. Que s'est-il passé il y a quinze ans, Sakura~?~>>

Il fallut un moment à Sakura pour comprendre, pour se souvenir, et alors elle sut.

L'attaque du démon renard à neuf cerveaux.

Juste une petite créature blanc os avec de grandes oreilles, une queue encore plus grande, et des yeux rouges perçants. Elle n'était pas plus puissante qu'un renard ordinaire, elle ne crachait pas du feu, elle n'avait pas d'yeux laser, elle n'avait ni chakra ni magie d'aucune sorte, mais son intelligence était plus de neuf-mille fois supérieure à celle d'un être humain.

Des centaines de gens avaient été tués, la moitié des immeubles avaient été ravagés et presque tout le village de Beisugakure avait été détruit.

<<~Tu penses que Kyubi se cache dans Naruto~?~>> dit Sakura. Un moment plus tard, son cerveau compléta automatiquement les conséquences évidentes de cette théorie. <<~Et le conflit logiciel entre leurs existences est la raison pour laquelle il se comporte comme un idiot les trois quarts du temps mais qu'il est capable de contrôler une centaine de clones de l'ombre. Hmm. Ça… se tient vraiment… en fait…~>>

Sasuke lui donna le bref hochement de tête méprisant de celui qui avait tout trouvé tout seul sans avoir besoin qu'on l'aide, \emph{lui}.

<<~Ano…~>> dit Sakura. Seules des années d'exercices d'équilibre mental dévièrent sa panique absolue et hurlante vers des propositions pragmatiques et utiles. <<~Ne devrions-nous pas… le \emph{dire} à quelqu'un~? Genre, avant les cinq prochaines secondes~?

--- Les adultes sont déjà au courant,~>> dit Sasuke d'une voix froide. <<~C'est l'explication évidente à la façon dont ils traitent Naruto. Non, la véritable question est de savoir comment cela s'articule avec la façon dont les Uchiha ont été dupés…

--- Je ne vois pas du tout le rapport avec - commença Sakura.

--- Ça \emph{doit} avoir un rapport~!~>> une nuance de frénésie luit dans la voix de Sasuke. <<~J'ai demandé à cet homme \emph{pourquoi} il avait fait ça, et il m'a dit que quand je connaîtrai la réponse, ça expliquerait \emph{tout}~! \emph{Ça aussi} doit certainement faire partie de ce qui sera expliqué~!~>>

Sakura soupira en son for intérieur. Son hypothèse personnelle était qu'Itachi avait juste essayé de mener son frère à la paranoïa clinique.

<<~Yo, les enfants~>>, dit la voix de leur sensei de rationalité depuis leurs écouteurs radiophoniques. <<~Il y a un village à Vague qui essaie de construire un pont et celui-ci n'arrête pas de tomber sans que personne ne comprenne pourquoi. Rendez-vous aux portes du village à midi. C'est l'heure de votre première mission d'analyse de rang C.~>>

\clearpage
\OmakeIVsection{Erdõs enchaîné}

<<~Comment as-tu pu faire ça, Anita~? dit Richard d'une voix très tendue. Comment as-tu pu coécrire un article avec Jean-Claude~? Tu \emph{étudies} les morts-vivants, tu ne collabores pas avec eux pour des articles~!

--- Et toi~? crachais-je. Tu as coécrit un papier avec Sylvie~! Alors \emph{toi} tu peux être prolifique mais pas \emph{moi}~?

--- Je suis le \emph{directeur de l'institut}~>>, gronda Richard. Je pouvais sentir les ondes de science irradier de lui~; il était en colère. <<~Je \emph{dois} travailler avec Sylvie, mais ça ne veut rien dire~! Je pensais que nos recherches étaient spéciales, Anita~!

--- Elles le \emph{sont}~>>, répondis-je, me sentant impuissante dans mon incapacité à expliquer les choses à Richard. Il ne comprenait pas à quel point cela pouvait être électrisant d'être une polymathe, de voir ces nouveaux mondes qui s'ouvraient à moi. <<~Je ne partage pas \emph{nos} recherches avec quiconque -

--- Mais tu voulais le faire~>>, dit Richard.

Je ne répondis rien car je savais que l'expression de mon visage le faisait à ma place.

<<~Dieu, Anita, tu as changé~>>, dit Richard. Il sembla s'avachir. <<~Te rends-tu compte que maintenant les monstres blaguent en parlant de nombre de Blake~? Avant, j'étais ton partenaire en tout, et maintenant - je suis juste un autre loup-garou avec un nombre de Blake de 1~>>.

\OmakeIVspecialsection[2]{\fontspec[ExternalLocation]{Thundercats}ThunderSmarts}

<<~J'en ai \emph{marre}~! s'écria Liono. J'en ai marre de faire ça \emph{chaque semaine~!} Notre espèce était capable de \emph{voyage interstellaire}, Panthro, et je \emph{connais} les quantités d'énergie que cela requiert~! Tu \emph{dois} pouvoir construire une arme nucléaire ou rediriger un astéroïde ou \emph{autre chose} et faire sauter la pyramide de cet idiot immortel~!~>>

\clearpage
\OmakeIVsection[1.2]{Musclor et les Maîtres de la Rationalité}

<<~Un savoir secret fabuleux me fut révélé le jour où j'ai tenu mon livre de magie en l'air et que j'ai dit~: \emph{Par le pouvoir du théorème de Bayes~!}~>>

\OmakeIVsection{Fate/Sain d'esprit}

\begin{emph}
Je suis le cœur de mes pensées\\
Mon corps est ce que je crois\\
Mon sang est mes choix\\
Plus de mille fois je me suis corrigé\\
Sans crainte de perte\\
Inconscient du gain\\
Par-delà la douleur\\
De la nouvelle information\\
J'attends la venue du vrai\\
C'est le seul chemin incertain\\
Toute ma vie a été…\\
L'œuvre Infinie de Bayes!
\end{emph}

\OmakeIVsection{Le nom de la Rationalité}

Le garçon de onze ans qui deviendrait un jour une légende - pourfendeur de dragons, tueur de rois - n'avait qu'une pensée à l'esprit alors qu'il approchait du Choixpeau pour entamer son étude des mystères.

\emph{N'importe où sauf Serdaigle, s'il vous plaît, n'importe où sauf Serdaigle…}

Mais à peine le rebord de l'ancien appareil feutré eut-il recouvert son front -

<<~SERDAIGLE~!~>>

Alors que la table recouverte de bleu l'applaudissait il s'approcha avec terreur de l'endroit où il passerait les sept prochaines années~; Kvothe grimaçait déjà en son for intérieur et attendait l'inévitable, et l'inévitable se produisit presque immédiatement, exactement comme il l'avait craint, avant même qu'il ait eu une chance de finir de s'asseoir.

<<~Alors~!~>> dit un garçon plus âgé avec l'expression heureuse de quelqu'un qui venait de penser à quelque chose d'incroyablement futé~: <<~Kvothe l'Aigle, hé~?~>> \footnotemark{}
\authorsnotetext{NdT: <<~Kvothe the Raven~>>, jeu de mots intraduisible en référence au poème \emph{The Raven} d'Edgar Allan Poe. Je suis à la recherche d'un jeu de mots équivalent~!}

\OmakeIVsection{Tengen Toppa Gurren Rationalité 40k}

J'ai une histoire vraiment géniale pour ce crossover que cette marge est trop étroite pour loger.

\OmakeIVspecialsection{\fontspec[ExternalLocation]{Twilight}Twilight Utilitariste\protect\footnotemark}
\footnotetext{Written after I heard Alicorn was writing a Twilight fanfic, but before I read \emph{Luminosity}. It’s obvious if you’re one of us.}

\emph{(Note~: Écrit avant que j'apprenne qu'Alicorn écrivait une fanfiction Twilight mais avant d'avoir lu {Luminosity}. C'est évident si vous êtes l'un des nôtres).}

<<~Edward~>>, dit Isabella d'une voix tendre. Elle leva la main et caressa sa joue froide et scintillante. <<~Tu n'as à me protéger de rien. J'ai fait la liste de tous les avantages et de tous les inconvénients, je leur ai assigné des poids relatifs cohérents et il était juste vraiment évident que les avantages qu'il y a à devenir un vampire l'emportent sur les inconvénients.

--- Bella,~>> dit Edward, et il déglutit avec désespoir. <<~Bella -

--- Immortalité. Santé parfaite. Pouvoirs psychiques en éveil. Plutôt simple de survivre à partir de sang animal une fois qu'on s'y met. Même la beauté, Edward, il y a des gens qui donneraient leur vie pour être beaux, et n'ai pas l'audace de dire qu'ils sont superficiels avant d'avoir essayé d'être laid. Est-ce que tu penses que j'ai peur du mot “vampire”~? J'en ai marre de tes contraintes déontologiques arbitraires, Edward. Toute l'espèce humaine devrait pouvoir s'amuser autant que vous, et alors même que tu hésites, des milliers de personnes se meurent.~>>

Le pistolet dans la main de son amant était froid contre son front. Cela ne le tuerait pas, mais cela l'handicaperait assez longtemps pour -

\OmakeIVsection{\sout{Aladdin} Jasmine}

Aladdin’s face was wistful, but determined, as the newly minted street urchin addressed the blue being of cosmic power for one last time, prepared to leave behind the wealth and hope he had so briefly tasted for the sake of his friend. “Genie, I make my third wish. I wish for you to be—”

Princess Jasmine, who had been staring at this with her mouth open, not quite believing what she was seeing, just barely managed to overcome her paralysis and yank the lamp out of the boy’s hand before he could finish the fatal sentence.

“Excuse me,” said Jasmine. “Aladdin, my darling, you’re cute but you’re an idiot, do you know that? Did you not notice how once Jafar got his hands on this lamp, he got his own three wishes—oh, never mind. Genie, I wish for everyone to always be young and healthy, I wish nobody ever had to die if they didn’t want to, and I wish for everyone’s intelligence to gradually increase at a rate of 1 IQ point per year.” She tossed the lamp back to Aladdin. “Go back to what you were doing.”

\OmakeIVsection[1.2]{Prince Hamlet and the Philosopher’s Stone\protect\footnotemark}%
\footnotetext{HonoreDB has now extended this to a complete e-book entitled
  \emph{A Will Most Incorrect to Heaven: The Tragedy of Prince Hamlet and
  the Philosopher’s Stone}, available for \$3 at \url{http://makefoil.com} (yes, really).}%

\begin{playdialog}%
HAMLET: Interloper, abandon this strange prank,\\
which makes cruel use of the blindness of my grief,\\
and the good heart of my good friend Horatio.\\
Or else, if thou hast true title to this belov’d form, tell me:\\ What drawing did I present to Hamlet King,\\ when six years old and scarce out of my sling?\\

GHOST: ’twas a unicorn clad all in mail.\\

HAMLET: What.\\

GHOST: Mark me.\\

HAMLET: Father, I will.\\

GHOST: My hour is almost come,\\
When I to sulphurous and tormenting flames\\
Must render up myself.\\

HAMLET: Thou art in torment?\\

GHOST: Ay, as are all who die unshriven.\\

HAMLET: Like every Dane this is what I’ve been taught.\\
Yet I did figure such caprice ill-suited to almighty God.\\
For all who suffer unlook’d-for deaths,\\
Unattended by God’s chosen priests,\\
to be then punish’d for the ill-ordering of the world…\\

GHOST: ’twas not the world that killed me, nor accident of any kind.\\

HAMLET: What?\\

GHOST: If thou didst ever thy dear father love,\\
Revenge his foul and most unnatural murder.\\

HAMLET: Oh God.\\

GHOST: My time grows ever shorter. Wilt thou hear the tale?\\

HAMLET: No.\\

GHOST: What?\\

HAMLET: My love for you does call me to avenge your death,\\ but greater crimes have I heard told this night.\\ If all those murdered go to Hell, and others as well,\\ who would have confess’d had they the time,\\ If people who are, in balance, good, suffer grisly\\ at the hands of God, then I defy God’s plan.\\ \\ Good Ghost, as one who dwells beyond the veil,\\ you know things that we mortals scarce conceive.\\ Tell me: is there some philter or device,\\ outside nature’s ken but not outside her means,\\ by which death itself may be escap’d?\\

GHOST: You seek to evade Hell?\\

HAMLET: I seek to deny Hell to everyone!\\
and Heaven too, for I suspect the Heaven of our mad God\\
might be a paltry thing, next to the Heaven I will make of Earth,\\ when I am its immortal king.\\

GHOST: I care not for these things.\\
Death and hell have stripp’d away all of my desires,\\
save for revenge upon my murderer.\\

HAMLET: Thou shalt not be avenged, save that thou swear:\\
an I slay thine killer, so wilt thou vouchsafe to me the means\\
by which I might slay death.\\
\\
He who killed you will join you in the Pit,\\
and then that’s it. No further swelling of Hell’s ranks will I permit.\\

GHOST: Done.\\
When my brother is slain, he who poured the poison in my ear,\\
then will I pour in yours the precious truth:\\
the making of the Philosopher’s Stone.\\
With this Stone, thou may’st procure\\
a philter to render any man immune to death, and more transmute\\ base metal to gold, to fund the gift of this philter to all mankind.\\

HAMLET: Truly there is nothing beyond the dreaming of philosophy.\\ Wait.\\ The man whom I must kill—my uncle the king?\\

GHOST: Ay, that incestuous, that adulterate beast,\\
With witchcraft of his wit, with traitorous gifts—\\

HAMLET: Indeed, he has such gifts I near despair,\\
of killing him and yet succeeding to his throne.\\
’twill be an awesome fight for awesome stakes.\\
Hast thou advice?\\

\emph{A cock crows. Exit Ghost.}
\end{playdialog}

\OmakeIVsection{Moby Dick et les Méthodes de la Rationalité}

<<~Vengeance~? dit l'homme à la jambe de bois. Contre une \emph{baleine}~? Non, j'ai décidé de passer à autre chose.~>>

\OmakeIVsection[0.9]{Alice in the Land Where Things Are Even Crazier Than Here}

Alice was sitting by her sister on the bank, reading a book. She had several friends who were older, and if she just asked nicely, they were often happy to lend her books without \emph{quite} so many pictures and conversations as was thought appropriate for a girl her age.

Hot days often made her feel sleepy and stupid, so Alice had thoughtfully wet a handkerchief and placed it at the back of her neck. Still her mind had gone off wandering (just as if it was some little kitten whose owner had taken their eyes off it for just a moment), and she had just decided that the pleasure of making a daisy-chain would be worth around 4/3 of the trouble of getting up and picking the daisies, which was nonetheless not equal to the opportunity cost of putting down her book, when suddenly a White Rabbit with pink eyes ran close by her.

There was nothing so \emph{very} remarkable in that; nor, in fact, did Alice think it so \emph{very} much out of the way to hear the Rabbit say to itself, “Oh dear! Oh dear! I shall be late!” But when the Rabbit actually \emph{took a watch out of its waistcoat pocket,} and looked at it, and then hurried on, Alice froze in sudden clarity and fear, for she had never before seen a rabbit with either a waistcoat pocket, or a watch to take out of it. “Oh bother,” she said to herself (though not aloud; she had long since cured herself of that habit, as it made people take her even less seriously than they already did). “If I did not immediately recognize how much curiouser that was than the average rabbit, then something is interfering with my curiosity, and that is most curious of all.” So, burning with questions, she ran across the field after it, and was just in time to see it pop down a large rabbit-hole under the hedge.

\OmakeIVsection{Bienvenue dans le monde réel}

\begin{playdialog}
MORPHEUS~: Pendant longtemps j'ai refusé d'y croire. Mais alors j'ai vu les champs de mes yeux, je les ai vus liquéfier les morts pour en nourrir les vivants -\\

NEO (\emph{poliment})~: Euh, excuse-moi.\\

MORPHEUS~: Oui, Néo~?\\

NEO~: J'ai réussi à ne rien dire pendant longtemps, mais là je ressens un certain besoin de parler. Le corps humain est la source d'énergie la moins efficace qu'on puisse imaginer. L'efficacité d'une centrale électrique dans sa capacité à convertir l'énergie thermique en électricité \emph{diminue} à mesure qu'on abaisse la température des turbines. Si tu as n'importe quel nourriture comestible par des humains, il serait plus efficace de la brûler dans un fourneau que de la leur donner à manger. Et maintenant tu me dis que la nourriture c'est \emph{les corps des morts, donnés à manger aux vivants~?} Est-ce que tu as déjà entendu parler des lois de la thermodynamique~?\\

MORPHEUS~: Où as-\emph{tu} entendu les lois de la thermodynamique, Néo~?\\

NEO~: Toute personne ayant été à un seul cours de science au lycée se doit de connaître les lois de la thermodynamique~!\\

MORPHEUS~: Et où as-tu été au lycée, Néo~?\\

(Silence.)\\

NEO~: … dans la Matrice.\\

MORPHEUS~: Les machines profèrent d'intelligents mensonges.\\

(Silence.)\\

NEO (\emph{d'une petite voix})~: S'il te plaît, est-ce que je pourrais avoir un vrai manuel de physique~?\\

MORPHEUS~: Ça n'existe pas, Néo. L'univers n'est pas basé sur les maths.\\ \end{playdialog}

%  LocalWords:  Ratîonalît Applejack Fluttershy Sci everypony Velorien Kage
%  LocalWords:  Uchiha Sakura Bunshin superintelligence Naruto’s jōnin 40k
%  LocalWords:  Uchiha’s Smartingan Naruto genin Beisugakure Kyubey Sasuke
%  LocalWords:  Anōōō emotionlessly Sasuke’s Itachi ThunderSmarts Liono Ay
%  LocalWords:  Panthro Kvothe Tengen Toppa Gurren Jafar HonoreDB
%  LocalWords:  belov’d twas unshriven unlook’d punish’d confess’d philter
%  LocalWords:  escap’d stripp’d may’st Neo math
