\partchapter{L'Expérience de Prison de Stanford}{IV}

\lettrine{U}{ne} faible étincelle verte s'avança et instaura un rythme. Une silhouette brillante et argentée la suivait, mais toutes les autres entités présentes étaient invisibles. Ils avaient traversé cinq sections de couloir, tourné cinq fois à droite et monté quatre volées de marches~; et lorsque Bellatrix avait fini sa deuxième bouteille de lait au chocolat, ils lui avaient donné des barres de chocolat solide.

C'est après sa troisième barre de chocolat que d'étranges sont commencèrent à sortir de sa gorge.

Harry eut du mal, à déchiffrer les sons car ils ne ressemblaient à rien de ce qu'il avait entendu auparavant~; le rythme était brisé, presque impossible à reconnaître, et il mit bien longtemps à se rendre compte que Bellatrix pleurait.

Bellatrix Black pleurait, l'arme la plus terrible du Seigneur des Ténèbres pleurait, elle était invisible, mais on pouvait les entendre, les petits sons pathétiques qu'elle essayait encore de contenir.

«Est-ce réel~?» dit-elle. Sa voix était redevenue tonale, elle n'était plus un marmonnement mort et s'était élevée en fin de phrase afin de dénoter une question. «Est-ce réel~?»

\emph{Oui}, pensa la partie de Harry qui simulait le Seigneur des Ténèbres, \emph{tais-toi maintenant -}

Il n'arrivait pas à laisser ces mots passer ses lèvres, cela lui était impossible.

«Je savais - que vous - viendriez pour moi - un jour», lorsqu'elle inspira l'air nécessaire à quelques sanglots silencieux, la voix de Bellatrix chevrota et se fractura. «Je savais - que vous étiez en vie - que vous viendriez - pour moi - seigneur…» il y eut une longue inhalation, comme un immense halètement, «et que même - lorsque vous viendriez - vous ne m'aimeriez toujours pas - jamais - vous ne m'aimeriez jamais en retour - c'est pour ça - qu'ils n'ont pas pu me prendre - mon amour - même si je ne me souviens pas - me souviens pas de tant d'autres choses - même si je ne sais pas ce que j'ai oublié - mais je sais à quel point je vous aime, seigneur -»

Un couteau fut planté en travers du cœur de Harry, il n'avait jamais rien entendu d'aussi horrible, il voulait pourchasser le Seigneur des Ténèbres et le tuer, juste pour cette…

«Ai-je encore - une utilité pour vous - seigneur~?

--- Non», siffla la voix de Harry, sans même qu'il doive penser, elle semblait être en pilote automatique «j'ai pénétré Azkaban par caprice. Bien sûr que vous m'êtes utile~! Ne posez pas de questions stupides.

--- Mais - je suis faible», dit la voix de Bellatrix, et un sanglot complet lui échappa, bien trop fort dans les couloirs d'Azkaban, «je ne peux pas tuer pour vous, seigneur, je suis désolée, ils ont tout mangé, ils m'ont dévorée, je suis trop faible pour me battre, quel intérêt ai-je pour vous maintenant -»

Le cerveau de Harry chercha désespérément une façon de la rassurer à travers les lèvres d'un Seigneur des Ténèbres qui ne prononcerait jamais un seul mot affectueux.

«Laide», dit Bellatrix. Sa voix prononça ce mot comme si ça avait été l'ultime clou dans son cercueil, son désespoir ultime. «Je suis laide, ils ont mangé ça aussi, je suis, je ne suis plus jolie, vous ne pourrez même, pas, m'utiliser, comme récompense, pour vos serviteurs - même les Lestranges, ne voudront plus, me faire mal, plus maintenant -»

La silhouette lumineuse cessa de marcher.

Parce que Harry avait cessé de marcher.

\emph{Le Seigneur des Ténèbres, il…} la partie de Harry qui était douce et vulnérable hurlait d'une horreur incrédule, essayait de rejeter la réalité, de refuser de comprendre, alors même que la partie plus froide et plus dure complétait le motif~: \emph{Elle lui obéissait en cela comme elle lui obéissait en toutes choses.}

L'étincelle verte sautilla instamment, bondit vers l'avant.

L'humanoïde d'argent resta où il était.

Bellatrix sanglota plus fort.

«Je ne, je ne suis pas, je ne peux plus, vous être utile, plus maintenant…»

Des mains géantes écrasaient la poitrine de Harry, le tordaient comme un chiffon, essayaient de broyer son cœur.

«S'il vous plaît, dit Bellatrix, tuez-moi…» sa voix sembla se calmer après qu'elle eut prononcé ces mots. «Seigneur, s'il vous plaît, tuez-moi, je n'ai aucune raison de vivre si je ne peux vous être utile… je veux seulement que cela s'arrête… faites-moi mal une dernière fois, seigneur, faites-moi mal jusqu'à ce que je m'arrête… je vous aime…»

C'était la chose la plus triste que Harry avait jamais entendue.

La claire silhouette d'argent du Patronus de Harry vacilla -

Ondula -

S'intensifia -

La furie qui montait en Harry, sa rage contre le Seigneur des Ténèbres, qui avait accompli cela, sa rage contre les Détraqueurs, contre Azkaban, contre le monde qui autorisait de telles horreurs, elle semblait se déverser directement depuis son bras dans sa baguette, sans qu'il y ait moyen de la bloquer, il essayait de lui intimer de s'arrêter et rien ne se passait.

«Seigneur~! chuchota la voix déguisée du professeur Quirrell. Je perds le contrôle de mon sortilège~! Aidez-moi, seigneur~!»

Plus lumineux le Patronus, de plus en plus lumineux, il croissait plus vite que le jour où Harry avait détruit un Détraqueur.

«Seigneur~! dit la silhouette d'un murmure terrifié. Aidez-moi~! Tout le monde va le sentir, seigneur~!»

\emph{Tout le monde va le sentir}, pensa Harry. Son imagination pouvait clairement les lui représenter, les prisonniers dans leur cellule s'éveillant à une lumière guérissante qui remplaçait le froid et les ténèbres écrasantes.

Les reflets de chaque surface présente brûlaient à présent avec la force d'un soleil blanc, et la silhouette du squelette de Bellatrix et l'homme au teint cireux étaient clairement visibles à travers l'éclat car les sortilèges de désillusion ne parvenaient pas à soutenir la force de la lumière surnaturelle~; seule la Cape d'Invisibilité, la relique de la Mort, pouvait y résister.

«Seigneur~! \emph{Vous devez l'arrêter~!}»

Mais Harry ne pouvait plus lui imposer sa volonté, il ne voulait plus que cela s'arrête. Il pouvait les sentir, les étincelles de vie d'Azkaban, toujours plus nombreuses à être protégées par son Patronus \emph{alors même que celui-ci déployait, tel des ailes de lumière solaire, les airs transformés en argent par sa pensée, et Harry sut ce qu'il devait faire.}

«\emph{S'il vous plaît, Seigneur~!»}

Les mots ne furent pas entendus.

\emph{Ils étaient loin de lui, les Détraqueurs dans leur fosse, mais Harry savait que si l'éclat de la lumière devenait assez puissant ils pourraient être détruits, même à cette distance, il savait que la Mort elle-même ne pourrait plus s'opposer à lui s'il arrêtait de se retenir, alors il descella toutes les portes de son être, creusa le puits de son sortilège jusqu'au ultimes tréfonds de son âme, de son esprit, de sa volonté, donna absolument tout ce qu'il possédait au sort -}

Et à l'intérieur du soleil, une ombre à peine moins lumineuse s'avança, tendit une main implorante.

\emph{NON}

\emph{ARRÊTE}

Le sensation funeste entra en collision avec la détermination d'acier de Harry, la terreur et l'incertitude luttant contre le noble but~; rien d'autre que cela n'aurait pu l'atteindre. La silhouette fit un autre pas en avant, puis un autre, et la sensation funeste s'éleva jusqu'à devenir le sentiment d'une catastrophe terrible, imminente~; et trempé d'une eau glacée, Harry comprit, il vit les conséquences de ce qu'il était en train de faire, il vit le danger et le piège.

Si vous aviez observé de l'extérieur vous auriez vu le cœur d'un soleil s'intensifier et se ternir…

S'intensifier et se ternir…

… et enfin s'estomper, s'estomper, s'estomper jusqu'à devenir un éclat lunaire ordinaire, qui par contraste rappelait les ténèbres les plus complètes.

Dans les ténèbres de cette éclat lunaire se tenait un homme au teint cireux, sa main implorante tendue en avant, et le squelette d'une femme, étendu au sol, un air perplexe sur le visage.

Et Harry, toujours invisible, tomba à genoux. Le grand danger était passer, et Harry essayait seulement de ne pas s'effondrer, de maintenir le sortilège à un niveau plus bas. Il avait puisé dans quelque chose et espérait n'en avoir rien perdu - il aurait dû le savoir, il aurait dû se souvenir que la simple magie n'était pas la seule chose qui alimentait le sortilège du Patronus -

«Merci, Seigneur, chuchota l'homme au teint cireux.

--- Idiot, dit la dure voix d'un garçon qui prétendait être le Seigneur des Ténèbres. Ne t'ai-je pas prévenu que le sortilège pouvait s'avérer fatal si tu échouais à contrôler tes émotions~?»

Les yeux du professeur Quirrell ne s'écarquillèrent bien sûr pas.

«Oui, Seigneur, je comprends», dit le serviteur du Seigneur des Ténèbres d'une voix chancelante, et il se tourna alors vers Bellatrix -

Elle se relevait déjà, lentement, comme une vieille femme Moldue

 «Que c'est amusant, chuchota Bellatrix, tu as failli être tué par un Patronus…». Un gloussement, qui sembla souffler de la poussière hors de ses tuyaux à gloussement. «Je pourrais peut-être te punir, si mon Seigneur te figeait sur place et que j'avais des couteaux… peut-être que je peux être utile, après tout~? Oh, je me sens déjà un peu mieux, comme c'est étrange…

--- Sois silencieuse, chère Bella, dit Harry d'une voix froide, jusqu'à ce que je t'octroie le droit de parler.»

Il n'y eut pas de réponse, ce qui constituait de l'obéissance.

Le serviteur fit léviter le squelette humain et le rendit de nouveau invisible peu avant de disparaître lui même au son d'un autre œuf brisé.

Ils continuèrent dans les couloirs d'Azkaban.

Et Harry sut, alors qu'ils avançaient, que les prisonniers s'éveillaient dans leur cellules alors que, le temps d'un moment précieux, la peur disparaissait, qu'ils ressentaient peut-être même la caresse salutaire de sa lumière, avant de s'effondrer à nouveau, avant que le froid et les ténèbres ne les recouvrent de nouveau.

Harry essayait très fort de ne pas y penser.

Sans quoi son Patronus croîtrait jusqu'à avoir brûlé chaque Détraqueur d'Azkaban, brillant assez fort pour les détruire, même à cette distance…

Sans quoi son Patronus croîtrait jusqu'à avoir brûlé chaque Détraqueur d'Azkaban, consumant toute la vie de Harry en retour.

\later

Au sommet d'Azkaban, dans le quartier général Auror, un trio ronflait dans la caserne, un autre se reposait dans la salle du personnel, et un autre montait la garde dans la salle de commandement. La salle de commandement était simple mais spacieuse, avec trois chaises à l'arrière, où les Aurors s'asseyaient, leur baguette toujours en main afin de maintenir leur trois Patronus, et les formes lumineuses et blanches faisaient les cent pas devant la fenêtre ouverte, les protégeant tous de la peur des Détraqueurs.

Ils restaient généralement tous les trois à l'arrière et jouaient au poker sans regarder par la fenêtre. Vous auriez pu y voir un peu de ciel, pour sûr, et il y avait même une heure ou deux par jour où vous auriez pu voir un peu de soleil, mais cette fenêtre donnait aussi sur la fosse centrale de l'enfer.

Juste au cas où un Détraqueur voudrait venir y flotter pour vous parler.

Jamais l'Auror Li n'aurait accepté d'y servir, triple paie ou pas, s'il n'avait pas eu une famille à nourrir. (Son vrai nom était Xiaoguang, et tout le monde préférait l'appeler Mike~; il avait nommé ses enfants Su et Kao en espérant que cela leur porterait meilleure chance). Sa seule consolation, mis à part l'argent, était qu'au moins ses amis jouaient excellemment bien au Poker Dragon. Même si à ce stade, il aurait été dur de \emph{mal} jouer.

C'était leur 5366\textsuperscript{ème} partie et Li avait ce qui serait probablement sa meilleure main des parties 5300-5400. On était un samedi de février et il y avait trois autres joueurs, ce qui le laissait modifier la couleur de n'importe quelle carte à un trou, sauf les deux, les trois ou les sept~; et c'était suffisant pour lui permettre de construire un Corps-à-Corps avec des Licornes, des Dragons et des sept…

De l'autre côté de la table, Gerard McCusker releva les yeux et regarda en direction de la fenêtre.

La nausée monta dans l'estomac de Li avec une rapidité surprenante.

Si son sept de cœur se faisait avoir par un Modificateur Détraqueur et se faisait transformer en six, il descendait direct à deux paires et McCusker pourrait battre ça -

«Mike, dit McCusker, qu'est-ce qu'il a, ton Patronus~?»

Li tourna la tête et regarda.

Son doux blaireau d'argent s'était détourné de sa garde au-dessus de la fosse et regardait en bas, vers quelque chose que lui seul pouvait voir.

Un instant plus tard, le canard couleur lune de Bahry et l'étincelant fourmilier de McCusker l'imitèrent, regardant tous dans la même direction.

Ils échangèrent des regards puis soupirèrent.

«Je vais leur dire», dit Bahry. Le protocole requérait que l'on envoie les trois Auror qui étaient de repos mais qui ne dormaient pas afin qu'ils inspecte toute anomalie. «Épargne peut-être l'un deux et prends la spirale C, si ça ne te dérange pas.»

Li échangea un regard avec McCusker et ils hochèrent tous deux la tête. Il n'était pas trop difficile de pénétrer par effraction dans Azkaban si l'on était assez riche pour s'offrir un sorcier puissant et que l'on avait des intentions assez pures pour recruter quelqu'un capable de lancer le Patronus. Des gens dont les amis étaient à Azkaban faisaient souvent cela, ils entraient par effraction uniquement pour donner une demi-journée de Patronus à quelqu'un, une chance d'avoir de véritables rêve, et pas seulement des cauchemars. En leur donnant une réserve de chocolat à cacher dans leur cellule, afin d'augmenter leurs chances de survivre à leur peine. Et les Aurors de garde… eh bien, si vous vous faisiez prendre, vous pourriez probablement convaincre les Aurors de regarder ailleurs en échange du bon pot-de-vin.

Pour Li, le bon pot-de-vin était quelque part en deux Noises et une Mornille d'argent. Il haïssait cet endroit.

Mais Bahry Une-Main avait une femme, et cette femme avait des frais de Guérisseur, et, quand c'était lui qui les attrapait, ceux qui pouvaient se payer une effraction dans Azkaban pouvaient bien se payer un graissage en règle de la dernière patte de Bahry.

Par un accord tacite, sans qu'aucun d'eux ne révèle quoi que ce soit en le proposant, ils commencèrent par terminer leur partie de poker. Li gagna car aucun Détraqueur n'était apparu. Les Patronus avaient alors déjà cessé de regarder et étaient retournés à leur patrouilles habituelles. Ce n'était probablement donc rien, mais la règle était la règle.

Après que Li eut ratissé les sous, Bahry leur donna à tous des hochements de tête officiels et se leva de la table. Les longues mèches blanches de l'homme âgé tombèrent le long de ses robes rouges et élégantes, ses robes tombèrent sur le sol de métal de la salle de commande, et il passa la porte qui les séparait des Aurors précédemment en repos.

Li avait été Trié à Poufsouffle et ce genre d'histoires le mettait parfois un peu mal à l'aise. Mais Bahry lui avait montré les images, et il fallait bien laisser ce type faire ce qu'il pouvait pour sa pauvre femme malade, d'autant plus qu'il ne lui restait plus que sept mois avant la retraite.

\later

La faible étincelle verte flottait entre les couloirs de métal, et l'humanoïde d'argent la suivait, maintenant un peu plus terne. Parfois la silhouette étincelait, surtout lorsqu'elle passait devant les immenses portes de métal, mais elle redevenait toujours terne.

Des yeux non entraînés auraient été incapables de voir les autres, ceux qui étaient invisibles~: le Survivant de onze ans, le squelette vivant nommé Bellatrix Black et le professeur de Poudlard polynectaré qui parcouraient Azkaban. Si c'était le début d'une histoire drôle, Harry ne connaissait pas la chute.

Ils avaient monté quatre autres volées de marches lorsque la voix rauque du professeur de Défense dit d'un ton égal et sans emphase~: «Un Auror approche».

Harry mit trop longtemps, peut-être une seconde entière, avant de comprendre, avant que l'adrénaline n'atteigne son sang, avant qu'il ne se rappelle ce que le professeur Quirrell lui avait dit de faire dans ce cas, et il pivota sur ses talons et repartit en courant en sens inverse.

Il atteignit les escaliers et se jeta frénétiquement derrière la troisième marche. Il sentit le métal froid, même à travers ses robes et sa Cape. Essayer de relever la tête et de jeter un coup d'œil au-dessus de la marche supérieure lui montra qu'il ne pouvait pas voir le professeur Quirrell~; et cela signifiait que Harry était hors de ligne de mire de tout tir perdu.

Son Patronus étincelant le suivit et s'allongea à ses côtés, une marche plus bas~; car lui aussi ne devait pas être vu.

Il y eut un léger son, comme un souffle ou une bourrasque, puis le corps invisible de Bellatrix qui venait se blottir quelques marches plus bas, elle n'aurait rien à faire hormis -

«Ne bouge pas, dit le chuchotement froid et aigu, tais-toi.»

Puis il y eut silence et immobilité.

Harry appuya sa baguette contre la marche de métal située juste au-dessus de lui. S'il avait été quelqu'un d'autre, il aurait eu besoin de sortir une Noise de sa poche… ou de déchirer un peu de tissu de ses robes… ou de se ronger un bout d'ongle… ou de trouver un caillou assez grand pour être visible et assez lourd pour rester immobile au contact de sa baguette. Mais grâce au tout-puissant pouvoir de Métamorphose partielle de Harry, cela n'était pas nécessaire~; il pouvait sauter cette étape et utiliser n'importe quel matériau.

Trente secondes plus tard, Harry était le fier propriétaire d'un miroir courbe, et …

«\emph{Wingardium Leviosa}», chuchota-t-il aussi doucement qu'il le pouvait.

… il le faisait léviter juste au-dessus des marches, observant dans cette surface courbe la quasi-totalité du couloir où le professeur Quirrell attendait, invisible.

Harry les entendit alors au loin, les bruits de pas.

Et il vit la silhouette (un peu difficile à deviner dans le miroir) d'une personne en robes rouges qui descendait les marches et entrait dans le couloir accompagné d'un petit Patronus animal que Harry ne pouvait pas bien voir.

L'Auror était protégé d'un chatoiement bleuté, les détails étaient peu clairs mais cela Harry pouvait le voir~: L'Auror avait ses boucliers levés et renforcés.

\emph{Mince}, pensa Harry. Selon le professeur de Défense, l'essence du duel consistait à essayer de lever des défenses qui bloqueraient tout ce qu'on risquait de vous envoyer tout en essayant de percer celles de l'autre. Et le moyen de loin le plus simple de gagner n'importe quel combat réel - le professeur Quirrell l'avait répété encore et encore - était d'abattre l'ennemi avant qu'il n'ait levé un seul bouclier, soit par derrière, soit d'assez près pour qu'il ne puisse ni éviter ni contrer assez vite.

Mais le professeur Quirrell pourrait peut-être placer un tir dans le dos, si -

Mais l'Auror s'arrêta après avoir fait trois pas dans le couloir.

«Jolie désillusion, dit une dure voix masculine que Harry ne reconnue pas. Maintenant, montrez-vous ou vous aurez de \emph{vrais} ennuis.»

L'homme barbu au teint cireux devint alors visible.

«Et vous, avec le Patronus, dit la voix dure. Sortez. \emph{Maintenant.}

--- Ça ne serait pas malin», dit la voix râpeuse de l'homme au teint cireux. Ce n'était plus la voix terrifiée du serviteur du Seigneur des Ténèbres~; elle était soudain devenue l'intimidation professionnelle d'un criminel compétent. «Vous ne voulez pas voir la personne derrière moi. Croyez-moi, vous n'en avez pas la moindre envie. Cinq-cents Gallions, en liquide, d'avance, si vous faites demi-tour et que vous partez. De gros problèmes pour votre carrière sinon.»

Il y eut une longue pause.

«Écoutez, qui que vous soyez, dit la voix dure. Vous semblez mal comprendre comment les choses fonctionnent. Je me fiche que ce soit Lucius Malfoy derrière vous, ou Albus Sacré Dumbledore. Vous sortez \emph{tous}, je vous inspecte, et \emph{alors} on discute de combien ça va vous coûter -

--- Deux-mille Gallions, dernière offre, dit la voix râpeuse avec un ton d'avertissement. C'est dix fois le prix habituel et plus que ce que vous gagnez en un an. Et croyez-moi, si vous voyez quelque chose que vous n'auriez pas dû voir, vous allez regretter de n'avoir pas accepté cette -

--- La ferme~! dit la voix dure. Vous avez exactement cinq secondes pour vous débarrasser de cette baguette avant que je ne me débarrasse de vous. Cinq, quatre -»

\emph{Professeur Quirrell, qu'est-ce que vous faites~?} pensa Harry avec angoisse. \emph{Attaquez~! Ou levez un bouclier, au moins~!}

« - trois, deux, un~! \emph{Stupéfix~!}»

\later

Bahry observait alors qu'un frisson descendait le long de sa colonne vertébrale.

La baguette de l'homme avait bougé si vite que c'était comme si la baguette avait transplané dans sa main, et le sortilège d'étourdissement de Bahry étincelait docilement au bout de celle-ci, pas bloqué, pas contré, pas dévié, \emph{saisi} comme une mouche dans du miel.

«Mon offre est redescendue à cinq-cents Gallions», dit l'homme d'une voix plus froide, plus formelle. Il avait un sourire sec, et ce sourire ne collait pas au visage barbu. «Et vous devrez accepter un sortilège d'Oubliettes.»

Bahry avait déjà modifié les harmoniques de ses boucliers afin que son sortilège d'étourdissement ne puisse pas l'atteindre, il avait déjà relevé sa baguette en position défensive, il avait déjà levé sa main artificielle renforcée dans une position qui lui permettrait de bloquer tout ce qui était bloquable, et il pensait déjà des sortilèges muets afin d'ajouter plus de couches sur ses boucliers -

L'homme ne regardait pas Bahry. Il s'intéressait au sortilège d'étourdissement de ce dernier, qui oscillait toujours au bout de la baguette de l'homme, et il en extrayait des étincelles avant de les faire s'envoler d'une pichenette, déconstruisant lentement le sort comme s'il s'était agi d'un casse-tête pour enfant.

Il n'avait levé aucun bouclier.

«Dites-moi», dit-il d'une voix désintéressée qui n'allait pas tout à fait à la gorge râpeuse - Polynectar, se serait dit Bahry s'il avait pensé que quiconque pouvait pratiquer la magie à un degré si délicat depuis l'intérieur du corps d'un autre - «qu'avez-vous fait pendant la dernière guerre~? Fait face au danger, ou planqué~?

--- Face au danger», dit Bahry. Sa voix garda le calme d'acier d'un Auror en service depuis presque cent ans et à sept mois de la retraite. Maugrey Fol-Œil ne l'aurait pas dit avec plus de conviction.

«Combattu des Mangemorts~?»

Un sourire sinistre éclot sur le visage de Bahry. «Deux à la fois.» Deux des combattants assassins de Vous-Savez-Qui, personnellement entraînés par leur maître. Bahry seul contre deux Mangemorts à la fois. Ça avait été le combat le plus difficile de sa vie, mais il avait tenu bon et en était ressorti sans rien d'autre qu'une main gauche perdue.

«Vous les avez tués~?» L'homme semblait animée d'une curiosité née de l'ennui il continuait de tirer des filaments de feu de l'étincelle maintenant bien diminuée, toujours captive au bout de sa baguette, et ses doigts tissaient maintenant des motifs avec la magie même de Bahry avant de les disperser d'un geste.

Bahry se mit à suer. Sa main de métal descendit d'un coup sec, arracha le miroir de sa ceinture - «Bahry à Mike, j'ai besoin de renforts~!»

Il y eut une pause, et du silence.

«Bahry à Mike~!»

Le miroir était terne et sans vie dans la main de Bahry. Il le remit lentement à sa ceinture.

«Voilà longtemps que je n'ai pas eu un combat sérieux face à un opposant sérieux,» dit l'homme, ne regardant toujours pas Bahry. «Essayez de ne pas trop me décevoir. Vous pouvez attaquer quand vous voulez. Ou vous pouvez repartir avec cinq-cents Gallions.»

Il y eut un long silence.

Puis le hurlement du métal traversant le verre parcourut les airs alors que Bahry abattait sa baguette.

\later

Harry pouvait à peine voir, à peine comprendre quelque chose au milieu des lumières et des éclairs, la courbe de son miroir était parfaite (il avait pratiqué cette tactique auparavant avec la Légion du Chaos) mais la scène était quand même trop petite et Harry avait l'impression qu'il n'aurait pas pu comprendre même s'il avait observé à un mètre, tout se passait trop \emph{vite}, des tirs rouges déviés par des boucliers bleus, des blocs de lumière verte qui s'affrontaient, des formes sombres qui apparaissaient et disparaissaient, il ne savait même pas qui lançait quoi, seulement que l'Auror criait incantation après incantation et esquivait avec frénésie tandis que la forme polynectarée du professeur Quirrell se tenait immobile et bougeait imperceptiblement sa baguette, généralement silencieusement, mais à l'occasion il prononçait dans mots dans des langues impossibles à reconnaître, et tout le miroir était alors envahi d'une lumière blanche avant de révéler la moitié du bouclier de l'Auror arraché et ce dernier faisant plusieurs pas en arrière.

Harry avait vu des duels de démonstration entre les plus puissants des élèves de septième année, et ce qu'il observait était si loin au-dessus de cela que l'esprit de Harry se sentait anesthésié par l'immensité du chemin qui lui restait à parcourir. Il n'y avait pas un seul élève de septième année qui aurait tenu une demi-minute contre l'Auror, et les trois armées de septième années mises ensemble n'auraient peut-être même pas infligé une égratignure au professeur Quirrell.

L'Auror était tombé au sol, un genou et une main pour le soutenir tandis que l'autre main faisait des gestes frénétique et que sa bouche hurlait des mots désespérés, les quelques incantations que Harry reconnaissait étaient toutes des boucliers, et une nuée d'ombres tournoyait autour de l'Auror comme un tourbillon de rasoirs.

Et Harry vit que la forme polynectarée du professeur Quirrell pointait délibérément sa baguette vers l'Auror agenouillé, qui vivait les derniers instants de son combat.

«Rendez-vous,» dit la voix râpeuse.

L'Auror cracha des mots indicibles.

«Dans ce cas, dit la voix, \emph{Avada -}»

Le temps sembla avancer très lentement, comme s'il y avait le temps d'entendre les syllabes individuelles, \emph{Ke}, et \emph{Da}, et \emph{Vra}, le temps de voir l'Auror commencer à se jeter de côté avec désespoir~; et bien que cela ait lieu si lentement, il n'y avait malgré tout pas le temps de \emph{faire} quoi que ce soit, pas le temps que Harry ouvre ses lèvres et crie \scream{non}, pas le temps de bouger, peut-être même pas le temps de penser.

Seulement le temps pour le vœu désespéré qu'un homme innocent ne meure pas -

Et une silhouette d'argent étincelante se tint devant l'Auror.

Se tint là juste une fraction de seconde avant que la lumière verte ne frappe.

\later

Bahry se propulsait de côté d'un mouvement frénétique, ne sachant pas s'il allait s'en sortir -

Ses yeux avaient mis au point sur son adversaire et sa mort imminente, et il ne vit donc que brièvement le contour de la silhouette lumineuse, du Patronus plus lumineux qu'il n'en avait jamais vu, il le vit juste assez longtemps pour reconnaître la forme impossible, avant que les lumières verte et argentées n'entrent en collision et que les deux ne s'évanouissent, les \emph{deux} s'évanouissent, \emph{le Sortilège de la Mort avait été bloqué}, et les oreilles de Bahry furent alors percées, il vit son terrible opposant hurler, hurler, hurler en se tenant la tête, en commençant à tomber, alors que Bahry lui-même s'effondrait -

Bahry heurta le sol, propulsé par son bond frénétique, et son épaule gauche disloquée et sa hanche brisées hurlèrent en signe de protestation. Bahry ignora la douleur, parvint à se remettre à genoux, sortit sa baguette pour étourdir son adversaire, il ne comprenait pas ce qui se passait mais il savait que c'était sa seule chance.

«\emph{Stupéfix~!}»

La lumière rouge fonça vers le corps tombant de l'homme et fut déchirée à mi-parcours, dissipée - pas par un bouclier. Bahry pouvait la \emph{voir}, l'ondulation dans l'air qui entourait son opposant criant et tombant.

Il pouvait le sentir comme une pression mortelle sur sa peau, le flux de magie qui montait et montait et montait vers un terrible point de rupture. Son instinct lui hurlait de courir avant que l'explosion ne se produise, ce n'était pas un sortilège, pas une malédiction, c'était de la sorcellerie devenue folle, mais avant que Bahry ne puisse même finir de se remettre sur pied -

L'homme rejeta sa baguette loin de lui (il jeta sa baguette~!) et une seconde plus tard, sa forme devint floue et disparut entièrement.

Un serpent vert se tenait sur le sol, déjà immobile avant même que le prochain sort d'étourdissement de Bahry, lancé par pur réflexe, ne l'atteigne sans rencontrer de résistance.

Alors que l'épouvantable pression commençait à se dissiper, alors que la sorcellerie folle commençait à s'estomper, l'esprit abasourdi de Bahry remarqua que le cri continuait. Mais il était différent, semblable à celui d'un jeune garçon, et il venait des escaliers qui menaient au niveau inférieur.

Le cri s'étrangla lui aussi, et il n'y eut que du silence hormis les halètements frénétiques de Bahry.

Ses pensées étaient lentes, brouillées, désordonnées. La puissance de son adversaire avait été \emph{insensée}, ça n'avait pas été un duel, ça avait été comme sa première année d'entraînement Auror face à Madame Tarma. Les Mangemorts n'avaient pas eu un dixième de cette puissance, Maugrey Fol-Œil n'était pas aussi bon… et qui, quoi, comment, par les couilles de Merlin, avait-on bloqué un \emph{Sortilège de la Mort}~?

Bahry parvint à trouver l'énergie d'appuyer sa baguette contre ses côtes, de marmonner le sort de soin puis de l'appuyer contre son épaule. Cela lui demanda plus d'effort que cela n'aurait dû, cela le draina beaucoup trop, sa magie était à un souffle de l'épuisement total~; il n'avait plus rien pour s'occuper de ses coupures mineures et de ses ecchymoses, ni même pour renforcer ce qui restait de son bouclier. C'était tout ce qu'il pouvait faire sans laisser son Patronus se dissiper.

Bahry inspira profondément, lourdement, stabilisa sa respiration autant qu'il le pouvait avant de parler.

«Vous, dit Bahry. Qui que vous soyez. Sortez.»

Il y eut un silence, et l'idée vint à Bahry que cette personne était peut-être inconsciente. Il ne comprenait pas ce qui venait de se passer, mais il avait entendu le cri…

Eh bien, il y avait un moyen de tester ça.

«Sortez, dit Bahry d'une voix plus dure, ou j'utiliserai des sortilèges à large zone d'effet.» Il n'y serait probablement pas parvenu.

«Attendez», dit la voix d'un garçon, d'un \emph{jeune} garçon, haute, fluette, fluctuante, comme si elle retenait de l'épuisement ou des larmes. La voix semblait maintenant venir d'un peu plus près. «Attendez s'il vous plaît. Je - sors -

--- Laissez tomber l'invisibilité», gronda Bahry. Il était trop fatigué pour s'embêter avec des sortilèges anti-Désillusion.

Un instant plus tard, le visage d'un jeune garçon émergea d'une cape d'invisibilité, et Bahry vit les cheveux noirs, les yeux verts, les lunettes et la cicatrice rageuse, rouge, en forme d'éclair.

S'il avait eu vingt ans d'expérience de moins il aurait peut-être cillé. Au lieu de cela, il cracha quelque chose qu'il n'aurait probablement pas dû dire face au Survivant.

«Il, il», dit la voix vacillante du garçon, son jeune visage exprimait la frayeur, la fatigue, et des larmes coulaient encore le long de ses joues, «il m'a kidnappé, pour me faire lancer mon Patronus… il a dit qu'il me tuerait sinon… mais je ne pouvais pas le laisser vous tuer…»

L'esprit de Bahry était encore embrumé mais les choses commençaient à lentement s'emboîter.

Harry Potter, le seul sorcier à avoir jamais survécu à un Sortilège de la Mort. Bahry aurait peut-être pu éviter la mort verte, il avait certainement essayé, mais si l'affaire se retrouvait face au Magenmagot, ils décréteraient une dette de vie envers une maison Noble.

«Je vois», dit Bahry d'un grondement bien plus amical. Il commença à marcher vers le garçon. «Fiston, je suis désolé que tout ça te soit arrivé, mais j'ai besoin que tu lâches ta cape et ta baguette.»

Le reste de Harry Potter émergea de l'invisibilité, révélant les robes à revers bleu de Poudlard, imbibées de sueur, et sa main droite serrée si fort autour d'une baguette de houx de trente centimètres que ses jointures en étaient blanches.

«Ta baguette, répéta Bahry.

--- Pardon, chuchota le garçon de onze ans, voilà, et il tendit la baguette vers Bahry.»

Bahry parvint à peine à s'empêcher de gronder contre le garçon traumatisé qui venait de lui sauver la vie. Il écrasa la pulsion d'un soupir et se contenta de tendre sa main pour prendre la baguette. «Écoutes, fiston, tu n'es \emph{vraiment} pas censé pointer ta baguette vers -»

Le pointe de celle-ci fit un léger arc de cercle sous la main de Bahry alors même que le garçon chuchotait~: «\emph{Somnium}.»

\later

Harry fixa le corps ratatiné de l'Auror. Il n'y avait pas de sentiment de triomphe, seulement un désespoir écrasant.

(Même alors, il n'aurait peut-être pas été trop tard).

Harry se tourna pour regarder l'endroit où le serpent vert se trouvait, toujours immobile.

«\parsel{Professeur~?}», siffla Harry. «\emph{Ami~? Ss'il vous plaît, êtess-vouss en vie~?}» Une horrible peur se saisissait du cœur de Harry~; en cet instant, il avait entièrement oublié qu'il venait de voir le professeur de Défense essayer de tuer un officier de police.

Harry pointa sa baguette vers le serpent et ses lèvres commencèrent même à former le mot \emph{Innerver} avant que son cerveau ne le rattrape et ne lui hurle dessus.

Il n'osait pas utiliser de magie sur le professeur Quirrell.

Harry l'avait sentie, la brûlure, la douleur déchirante dans sa tête, comme si son cerveau avait été sur le point de se diviser en deux. Il les avait senties, sa magie et celle du professeur Quirrell, adaptées l'une à l'autre, dans une harmonie inverse porteuse de catastrophe. C'était la chose mystérieuse et horrible qui se produirait si jamais Harry et le professeur Quirrell se rapprochaient trop ou s'ils se jetaient jamais un sortilège ou \emph{si jamais leurs sortilèges entraient en contact}, leur magie résonnerait jusqu'à échapper à tout contrôle -

Harry regarda le serpent, il ne savait pas si ce dernier respirait.

(Les dernières secondes s'écoulèrent).

Il se tourna pour regarda l'Auror, qui avait vu le Survivant, qui savait.

L'amplitude du désastre écrasa Harry comme mille poids de cent tonnes, il était parvenu à étourdir l'Auror, mails il ne restait maintenant plus rien à faire, aucun moyen de récupérer, la mission avait échouée, tout avait échoué, \emph{il} avait échoué.

En état de choc, désarçonné, entièrement découragé~: \emph{il n'y pensa pas}, il ne vit pas l'évidence, il ne se souvint pas de l'origine de ses sentiments vides d'espoir, il ne se rendit pas compte qu'il lui fallait encore relancer le Véritable Patronus.

(Et alors il était déjà trop tard).

\later

Les Aurors Li et McCusker avaient réarrangés leurs chaises autour de la table, si bien qu'ils la virent tous deux en même temps, la fine horreur nue et squelettique qui s'était élevée pour flotter devant la fenêtre, et la migraine les atteignit dès qu'elle fut dans leur champ de vision.

Ils entendirent tous deux la voix, comme si un corps mort depuis longtemps avait prononcé des mots et que ces mots eux-mêmes avaient vieilli et étaient ensuite morts.

Les paroles du Détraqueur leur firent mal aux oreilles~: «Bellatrix Black est hors de sa cellule.»

Il y eut une demi-seconde de silence horrifié, puis Li s'arracha à sa chaise, se dirigea vers le communicateur pour appeler des renforts au Ministère, alors même que McCusker saisissait son miroir et commençait frénétiquement à essayer de joindre les trois Aurors qui étaient partis en patrouille. 

%  LocalWords:  Xiaoguang Kao 300s McCusker Li’s Bahry’s McCusker’s Bahry
%  LocalWords:  Ke Vra Tarma Pleasse
