\chapter[Quelque chose à protéger~: Drago Malfoy]{Quelque chose à protéger~: Drago Malfoy\protect\footnotemark}
\authorsnotetext{Adieu, Terry Pratchett, 1948-2015. Vos personnages m'ont inspirés et je comprends maintenant tout ce qu'ils m'ont appris sur l'Intelligence des Personnages de niveau 1 et 3~: que la conscience de soi se manifeste souvent par l'humour ou la connaissance de son genre littéraire~; qu'un étincelle d'optimisation peut briller aussi vivement chez les personnages que l'on dit (mais qui ne se montrent pas) être humbles et stupides~; que les personnages intelligents peuvent être accompagnés d'une étincelle de bonté à travers leur histoire, plutôt que par du cynisme. J'aurais aimé vous rencontrer et vous parler de vos méthodes. Nombreux sont ceux qui vous aimaient, et au moins l'un d'eux aurait été jusqu'à disloquer les fondements de la réalité pour vous sauver~; mais votre cerveau est maintenant mort et chaud, et votre histoire se conclut donc.}
    \begin{verse}
      Even if the stars should die in heaven, \\
      Our sins can never be undone. \\
      No single death will be forgiven \\
      When fades at last the last lit sun. \\
      Then in the cold and silent black \\
      As light and matter end, \\
      We’ll have ourselves a last look back \\
      And toast and absent friend. \\
    \end{verse}

\lettrine{L}{e} garçon était assis dans un bureau non loin de l'ancien bureau de la directrice adjointe. Ses larmes s'étaient asséchées des heures plus tôt. Il ne lui restait que l'attente de son devenir, celui d'un orphelin de Poudlard, dont la vie et la joie résidaient entre les mains des ennemis de sa famille. On l'avait fait venir ici, et il avait obtempéré par manque d'alternatives. Vincent et Gregory l'avaient quitté, rappelés par leurs mères pour les funérailles hâtives de leurs pères. Peut-être le garçon aurait-il dû les accompagner, mais il n'avait pas réussi à le faire. Il n'aurait pas su jouer le rôle d'un Malfoy. Le sentiment de vide qui l'emplissait était si profond qu'il ne restait plus de place pour de la fausse courtoisie.

Ils étaient tous morts.

Son père était mort, son parrain MacNair aussi, tout comme son parrain de secours, M. Avery. Même Sirius Black, le cousin de sa mère, avait réussi à y passer, et tout ce qui restait de la maison Black éprouvait peu d'amitié pour les Malfoys.

Ils étaient tous morts.

On frappa à la porte du bureau~; et quand le garçon ne répondit pas, la porte s'ouvrit pour révéler…

«Vas-t'en», dit Drago Malfoy au Survivant. Il ne parvint pas à prononcer les mots avec force.

«Je partirai bientôt, dit Harry Potter en entrant. Mais une décision doit être prise, et toi seul en es capable.»

Drago se tourna vers le mur, parce que la simple vue de Harry Potter lui demandait plus d'efforts qu'il ne pouvait en fournir.

«Tu dois décider, dit Harry, ce qui va arriver à Drago Malfoy. Je ne veux pas être menaçant~; quoi qu'il arrive, tu deviendras le riche héritier d'une Noble et Très Ancienne Maison. Le truc,» dit Harry d'une voix à présent vacillante, «le truc, c'est qu'il y a quelque chose d'horrible que tu ignores, et je n'arrête pas de penser que si tu le savais, tu me dirais que tu ne veux plus être mon ami. Et je ne veux pas arrêter d'être ton ami. Mais de juste… ne jamais te le dire… et toujours maintenir ce mensonge, pour pouvoir continuer d'être ton ami… je ne peux pas. Et ce serait mal. Je ne… je ne veux plus faire ça, je ne veux plus te \emph{manipuler}. Je t'ai fait trop de mal.»

\emph{Alors arrête d'essayer d'être mon ami, tu t'y prends mal de toute façon.} Les mots surgirent dans la conscience de Drago et ses lèvres les rejetèrent. Il avait déjà le sentiment d'avoir presque entièrement perdu Harry, à cause de la façon dont ce dernier avait joué avec leur amitié, à cause de ses mensonges, de ses manipulations~; et pourtant l'idée de retourner seul à Serpentard, peut-être sans Vincent et Gregory si leurs mères mettaient fin à l'accord… il ne souhaitait pas cela, il ne souhaitait pas retourner à Serpentard et vivre uniquement entourés de gens qui avaient acceptés d'y être Répartis. Il avait encore la présence d'esprit de comprendre combien de ses vrais amis étaient aussi amis avec Harry, que Padma était à Serpentard, et que même Théodore était un lieutenant du Chaos. Tout ce qui restait de la maison Malfoy était une tradition~; et cette tradition disait qu'il était idiot de dire au gagnant de la guerre de partir, de rejeter son amitié.

--- Très bien, dit Drago d'une voix vide. Dis-moi.

--- C'est ce que je vais faire, dit Harry. Et ensuite la directrice va venir après mon départ et sceller ta dernière demi-heure de souvenirs. Mais avant, en connaissant toute la vérité, tu vas décider si tu veux toujours passer du temps avec moi.» Sa voix tremblait. «Hmm. Selon les archives que je lisais avant de venir ici, l'histoire a vraiment commencé en 1926 avec la naissance de Tom Erfin Jedusor. Sa mère est morte en couches et il a grandi dans un orphelinat Moldu jusqu'à ce que sa lettre de Poudlard lui soit apportée par Dumbledore…»

Le Survivant continua de parler et les mots s'écrasèrent sur l'esprit de Drago comme une boule de démolition.

\emph{Le Seigneur des Ténèbres était un Sang-Mêlé. Il n'a jamais cru à la pureté du sang.}

\emph{Pour Tom Jedusor, Lord Voldemort était une mauvaise blague.}

\emph{Les Mangemorts avaient été censés perdre face à David Monroe, pour qu'il obtienne le pouvoir.}

\emph{Après avoir abandonné ce projet, Tom Jedusor avait continué de jouer le rôle de Voldemort au lieu d'essayer de gagner parce qu'il avait aimé mener les Mangemort à la baguette.}

\emph{Voldemort m'a utilisé pour faire accuser mon Père de tentative de meurtre contre moi et autre une fois pour obtenir la Pierre Philosophale.} Il ne s'en souvenait pas, mais on lui avait déjà dit qu'il avait été un pion, comme le professeur Chourave, et qu'il ne serait accusé de rien.

Et la dernière horreur.

«Tu… murmura Drago Malfoy. Tu…

--- C'est moi qui ai tué ton père et tous les autres Mangemorts, cette nuit-là. Ils avaient reçu l'ordre de me tirer dessus à la seconde où je ferai quelque chose, et je devais donc les tuer pour avoir une chance d'avoir Voldemort~; il présentait un danger pour la terre entière.» La voix de Harry Potter était rauque. «Je n'ai pas pensé à toi, à Théodore, à Vincent ou à Gregory, mais si j'y avais pensé, je l'aurais fait quand même. Mon esprit a réussi à ne se rendre compte qu'après que M. Blanc était Lucius, mais même si je m'en étais rendu compte, je n'aurais toujours pas pris le risque de le laisser en vie, au cas où il aurait su pratiquer la magie muette. L'idée m'est venue il y a longtemps qu'il serait pratique, d'un point de vue politique, de tuer d'un coup tous les Mangemorts. J'ai toujours pensé que les Mangemorts étaient des gens horribles, bien plus que ce que je ne te l'ai laissé entendre, et ce depuis le jour où je t'ai rencontré. Mais si ton père n'avait pas été là, et si j'avais eu un bouton pour le tuer à distance, je n'aurais pas appuyé sur le bouton pour des raisons purement politique. Ce que je ressens quand je pense à ce que j'ai fait, et le remords que je peux ressentir… eh bien, une partie de moi hurle d'horreur à l'idée d'avoir tué quelqu'un. Et une autre partie dit que d'un point de vue moral, les Mangemorts ont accepté leur sort le jour où ils ont accepté de travailler pour Voldemort. Ils ont braqué leurs baguettes sur moi les premiers, bla bla bla etc. Mais là, je me sens juste malade de t'avoir infligé tout ça. Une fois de plus. Je sens que,» sa voix vacilla un peu, «tout ce que je fais te fait du mal, en dépit de mes \emph{bonnes intentions}, qu'être à côté de moi ne t'a occasionné que des pertes. Alors, si tu me dis de ne plus fréquenter Drago Malfoy, c'est ce que je ferai. Et si tu veux que j'essaie d'être vraiment ton ami, sans jamais réessayer de te manipuler, sans jamais t'utiliser ou prendre le risque de te faire du mal, alors je le ferai, je te le jure.»

Le prochain Lord Malfoy pleurait ouvertement face à son ennemi. Le décor et les apparences avaient été abandonnés, parce qu'il n'y avait plus personne à qui faire croire quoi que ce soit.

Un mensonge.

Un mensonge.

Tout avait été un mensonge, c'étaient des mensonges empilés sur d'autres, mensonges mensonges mensonges…

«\emph{Tu} devrais mourir, se força-t-il à dire. Tu devrais mourir pour avoir tué Père.» Les mots ne firent que le remplir de vide, mais il fallait qu'ils fussent dits.

Harry Potter secoua simplement la tête.

«Et si ce n'est pas possible~?

--- Tu devrais \emph{souffrir}.»

À nouveau, Harry Potter secoua simplement la tête.

Le Survivant insista, pressa Lord Malfoy de prendre une décision.

Lord Malfoy s'y refusa. Il ne parvenait pas à prononcer une réponse, quelle qu'elle soit. Il ne voulait pas laisser le gagnant et leurs amis communs l'abandonner, et il n'allait pas non plus donner à Harry l'absolution qu'il désirait.

Drago Malfoy se refusa donc à répondre, et les souvenirs de celui-ci disparurent.

\later

Le garçon était assis dans un bureau non loin de l'ancien bureau de la directrice adjointe. Ses larmes s'étaient asséchées des heures plus tôt. Il ne lui restait que l'attente de son devenir, celui d'un orphelin de Poudlard, dont la vie et la joie résidant entre les mains des ennemis de sa famille. On l'avait fait venir ici, et il avait obtempéré par manque d'alternatives. Vincent et Gregory l'avaient quittés, rappelés par leurs mères pour les funérailles hâtives de leurs pères. Peut-être le garçon aurait-il dû les accompagner, mais il n'avait pas réussi à le faire. Il n'aurait pas su jouer le rôle d'un Malfoy. Le sentiment de vide qui l'emplissait était si profond qu'il ne restait plus de place pour de la fausse courtoisie.

Ils étaient tous morts.

Ils étaient tous morts, et tout avait été futile depuis le début.

On frappa à la porte du bureau, puis, après un moment de flottement poli, elle s'ouvrit pour révéler la directrice McGonagall, habillée comme à son habitude. «M. Malfoy, dit la famille de son ennemi victorieux. Venez avec moi, s'il vous plaît.»

Drago Malfoy se leva avec indifférence et la suivit hors du bureau. Voir Harry Potter debout devant la porte l'interloqua, mais son esprit écarta simplement la pensée.

«Une dernière chose dit Harry Potter. Je l'ai trouvé dans un parchemin roulé dont l'extérieur disait que c'était la dernière arme à utiliser contre la Maison Malfoy et de ne pas continuer à lire à moins que l'issue de toute la guerre ne soit en jeu. Je ne voulais pas te le dire avant parce que je pensais que ça t'influencerait injustement. Si tu étais quelqu'un de bien, que tu n'avais jamais tué ni menti, mais que tu devais faire l'un ou l'autre, qu'est-ce qui serait pire~?»

Drago l'ignora, continua après la directrice McGonagall, et laissa un Harry à l'air chagriné derrière lui.

Ils se rendirent au vieux bureau de la directrice où cette dernière alluma la cheminée d'un mouvement de baguette et dit à la flamme verte~: «Bureau des voyages de Gringotts» avant de s'y plonger, un regard ferme posé sur lui.

N'ayant pas d'autre choix, Drago Malfoy la suivit.

\later

Elle était au lit, plus indifférente qu'à son habitude, ce matin-là, réveillée trop tôt par un soleil en train de se lever - même si sa lumière directe était bloquée par les grattes-ciels qui faisaient de l'ombre à sa maison. Une légère gueule de bois rongeait ses tempes et asséchait sa bouche~; elle avait essayé de ne pas trop boire (même si elle se demandait bien pourquoi elle se fatiguait) mais hier elle s'était sentie… plus déprimée qu'à son habitude, comme si, sans savoir quoi, elle avait perdu quelque chose. Pas pour la première fois ni pour la centième, elle songea à déménager - vers Adelaide, vers Perth, peut-être même Perth Amboy s'il le fallait. Elle avait toujours eu le sentiment de devoir être ailleurs~; mais, même si elle pouvait vivre confortablement grâce au revenu mensuel fourni par la compagnie d'assurance, elle ne pouvait se permettre d'être dépensière. Elle ne pouvait pas aller vadrouiller de par le monde à la recherche d'un lieu qui comblerait son sentiment de manque indistinct. Elle avait assez regardé la télévision et loué assez de documentaires de voyages pour savoir qu'aucun endroit montré par son magnétoscope ne lui conviendrait mieux que Sydney.

Depuis l'accident de voiture qui lui avait ravi ses souvenirs, elle s'était sentie figée, arrêtée dans le temps - pas seulement les souvenirs d'une famille morte qui ne signifiait plus rien pour elle, mais aussi la façon de faire marcher un four. Elle soupçonnait, non, elle \emph{savait} que la clé qu'attendait son cœur, qu'attendait sa vie pour le mouvement reprenne avait été perdue dans ce monospace incontrôlable. Elle y pensait presque tous les matins et essayait de deviner ce qui lui manquait, manquait, manquait dans sa vie et son esprit.

Quelqu'un sonna à la porte.

Elle grogna et tourna la tête pour lire l'heure sur réveil-matin électronique. 6h31 du matin. \emph{Sérieux~?} Eh bien, cet idiot pouvait attendre qu'elle se lève à son rythme.

Elle se leva lentement, bien à son rythme, et ignora l'alarme qui rentit de nouveau quand elle entra dans la salle de bains et commença à s'habiller.

Elle descendit les escaliers sans conviction et ignora l'agaçant sentiment que quelqu'un d'autre aurait dû ouvrir la porte pour elle. «Qui est là~?» dit-elle à la porte fermée~; elle était munie d'un judas, mais il était obscurci.

«Êtes-vous Nancy Manson~? dit une voix féminine dotée d'un accent Écossais marqué.

--- Oui, dit-elle d'un ton appréhensif.

--- \emph{Eunoe}», dit la voix Écossaise, et Nancy fit un bond en arrière lorsque l'éclat lumineux traversa la porte et la \emph{frappa} et…

Nancy faillit tomber, mit une main contre son front. Des éclats de lumières qui traversaient les portes et frappaient les gens, c'était… c'était… ça n'était pas très surprenant…

«Pourriez-vous ouvrir la porte~? dit la femme Écossaise. La guerre est finie et vos souvenirs devraient bientôt vous revenir. Il y a quelqu'un qui aimerait vous voir.»

\emph{Mes souvenirs…}

Elle avait l'impression d'avoir l'esprit congestionné, comme si elle avait arraché quelque chose à son cerveau, mais elle parvint à tendre la main et à ouvrir la porte.

Face à elle se trouvait une femme habillée comme une sorcière (\emph{parfaitement normale}), vêtue de robes noires, d'un grand chapeau pointu…

… et à côté d'elle un garçon aux courts cheveux blonds-blancs et vêtu de robes noires à liserets verts (\emph{parfaitement normales})~; il la regardait bouche bée, ses yeux écarquillés, des larmes naissantes.

\emph{Des robes à liserets verts, des cheveux blonds-blancs…}

Quelque chose de chaud s'agita dans ses souvenirs. Elle sentit son cœur grandir, l'étouffer presque, lorsqu'elle se rendit compte que ce qu'elle avait cherché pendant ces dix dernières années était peut-être juste en face d'elle. En un lieu enfoui en elle, de la glace se brisait autour de son cœur~; la partie arrêtée depuis si longtemps se préparait à bouger à nouveau.

Le garçon la regardait, ses lèvres bougeaient, mais aucun son ne s'échappait.

Un nom mystérieux lui vint à l'esprit, monta jusqu'à ses lèvres.

«Lucius~?» murmura-t-elle.

%  LocalWords:  Morfin Amboy Eunoe
