\partchapter{Humanisme}{I}

\lettrine{L}{e} doux soleil de janvier brillait sur froides étendues herbeuses aux alentours de Poudlard.

Pour certains élèves, c'était l'heure de l'étude~; d'autres avaient déjà fini les cours. Les première année qui s'étaient inscrits à la pratique d'un sort très particulier, un sort qui s'apprenait mieux en extérieur, sous un ciel bleu limpide et un soleil radieux plutôt que confiné dans une salle de classe. Les cookies et la limonade aidaient aussi.

Les premiers mouvements du sortilège étaient complexes et précis~; vous tordiez votre baguette une fois, deux fois, trois fois et quatre fois, de petites inclinaisons à des angles précis, puis vous décaliez votre index et votre pouce de la distance exacte…

Selon le ministère, cela impliquait qu'il était futile d'essayer d'enseigner ce sort à qui que ce soit avant sa cinquième année. Il y avait quelques cas connus d'enfants plus jeunes qui s'étaient révélés capables de l'apprendre, mais ils avaient été rejetés comme autant d'exceptions dues au "génie".

Ça n'avait peut-être pas été une façon très polie de le dire, mais Harry commençait à voir pourquoi le professeur Quirrell avait déclaré que le comité du ministère en charge du curriculum des élèves aurait été plus bénéfique pour les sorciers s'il avait été utilisé comme dépotoir.

Eh bien oui, les gestes étaient complexes et délicats. Cela n'empêchait personne de l'apprendre à onze ans. Cela voulait dire qu'il fallait être particulièrement attentif et qu'il fallait en pratiquer chaque partie plus longtemps que d'habitude, voilà tout.

La plupart des sortilèges pouvaient seulement être enseignés aux élèves plus âgés parce qu'ils nécessitaient une force magique au-delà de ce qu'un élève jeune aurait pu rassembler. Mais le Patronus ne fonctionnait \emph{pas} comme ça, il n'était pas difficile parce qu'il nécessitait trop de magie mais parce qu'il nécessitait \emph{plus} que de la simple magie.

Il nécessitait les sentiments heureux de ceux qui peuplaient votre cœur, les souvenirs aimants, un type de force qui était différente et qui n'était pas nécessaire aux sorts ordinaires.

Harry tordit sa main une fois, deux fois, trois fois et quatre fois, il décala ses doigts de la distance exacte…

"\emph{Bonne chance à l'école, Harry. Penses-tu que je t'ai acheté assez de livres~?"}

"\emph{On ne peut jamais acheter assez de livres… mais tu as certainement essayé, c'était un très, très, très bon essai…"}

La première fois que Harry s'en était souvenu et qu'il avait essayé de l'utiliser pour le sort, cela lui avait fait monter les larmes aux yeux.

Harry leva sa baguette et lui fit décrire un arc de cercle, puis il la brandit d'un geste qui n'avait pas à être précis, seulement hardi et plein de défiance.

"\emph{Expecto Patronum~!}" s'écria-t-il.

Rien ne se produisit.

Pas une seule étincelle de lumière.

Lorsqu'il releva les yeux, Remus Lupin examinait encore la baguette, un air assez inquiet sur son visage légèrement balafré.

Il finit par secouer la tête. "Je suis navré Harry," dit doucement l'homme. "Ton mouvement était parfait."

Et il n'y avait pas une seule étincelle de lumière où que ce soit, parce que tous les autres première année censés pratiquer leur Patronus avaient préféré regarder Harry du coin de l'œil.

Les larmes menaçaient de revenir dans les yeux de Harry, et ce n'étaient pas des larmes de joie. Entre toutes choses, entre toutes choses, Harry ne se serait jamais attendu à cela.

Qu'est-ce qu'Anthony Goldstein avait que Harry n'avait pas, et qui faisait briller sa baguette de cette vive lumière~?

Est-ce qu'il aimait plus son père~?

"Quelle pensée as-tu utilisée pour le lancer~?" dit Remus.

"Mon père," dit Harry d'une voix tremblante. "Je lui ai demandé de m'acheter quelques livres avant que je n'aille à Poudlard, et il l'a fait, et ils coûtaient cher, et puis il m'a demandé s'il y en avait assez -"

Il n'essaya pas d'expliquer la devise de la famille Verres.

"Reposes-toi avant d'essayer une autre pensée, Harry," dit Remus. Il fit un geste en direction d'un autre élève qui était assis au sol, et le visage de ce dernier avait une expression qui aurait pu signifier la déception, la gêne ou le regret. "Tu n'arriveras pas à lancer un Patronus si tu te sens honteux à l'idée de ne pas témoigner d'assez de gratitude." Il y avait une douce compassion dans la voix de M. Lupin, et l'espace d'un instant, Harry eut envie de frapper sur quelque chose.

Au lieu de cela il se détourna et darda jusqu'à l'endroit où les autres perdants étaient assis. Ces autres élèves dont les mouvements avaient été déclarés parfaits et qui étaient maintenant censés chercher des pensées plus heureuses~; à en voir leur tête, ils n'avançaient pas beaucoup. De nombreuses robes étaient bordées de bleu sombre, ainsi que quelques rouges, et une fille Poufsouffle esseulée pleurait encore. Les Serpentard ne s'étaient même pas fatigués à venir, mis à part Daphné Greengrass et Tracey Davis qui s'essayaient encore aux mouvements.

Harry se laissa chuter sur l'herbe froide et morte de l'hiver, à côté de l'élève dont l'échec l'avait le plus surpris.

"Alors tu n'as pas pu le faire non plus," dit Hermione. Elle avait d'abord fuit le lieu d'entraînement, mais elle avait fini par revenir, et il fallait observer ses yeux rougis de près pour voir qu'elle avait pleuré.

"Je," dit Harry, "je, je me sentirais probablement bien pire si tu n'avais pas échoué, tu es la plus gentille des personnes que je connais, que j'ai jamais rencontré, Hermione, et si \emph{tu} ne peux pas le faire non plus, ça veut dire que je suis peut-être quelqu'un de, quelqu'un de bien…"

"J'aurais dû aller à Gryffondor," chuchota Hermione. Elle cligna des yeux plusieurs fois mais elle ne les essuya pas.

\later

Le garçon et la fille marchaient ensemble, certainement pas en se tenait la main, mais chacun tirant une sorte de force de la présence de l'autre, quelque chose qui les laissait ignorer les murmures de leurs camarades tandis qu'il traversaient les couloirs approchants les grandes portes de Poudlard.

Harry n'était pas parvenu à lancer le Patronus, et ce quelle que soit la pensée heureuse qu'il ait essayée. Cela n'avait surpris personne, ce qui était encore pire. Hermione n'y était pas parvenue non plus. Cela avait \emph{beaucoup} surpris, et Harry l'avait vue commencer à recevoir le même genre de regard en biais que ceux qui lui étaient adressés. Les autres Serdaigle qui avaient échoué ne recevaient pas ces regards. Mais Hermione était le général Soleil et ses fans traitaient cela comme un manquement à ses engagements, comme si elle avait trahi une promesse qu'elle n'avait jamais faite.

Ils s'étaient tous deux rendus à la bibliothèque pour faire des recherches sur le Patronus, car c'était la façon dont Hermione réagissait en situation de détresse, tout comme Harry le faisait parfois. Étudier, apprendre, essayer de comprendre \emph{pourquoi}…

Les livres avaient confirmé ce que le directeur avait dit à Harry~; les sorciers incapables de jeter un Patronus parvenaient souvent à le faire en présence d'un vrai Détraqueur, passant de l'échec complet à un Patronus corporel. Cela défiait toute logique, l'aura de peur du Détraqueur aurait dû rendre l'apparition d'une pensée heureuse \emph{plus difficile}~; mais c'était ainsi.

Alors ils allaient tous les deux essayer une dernière fois. Il était impensable qu'ils en fassent autrement.

C'était le jour où le Détraqueur venait à Poudlard.

Plus tôt, Harry avait démétamorphosé le rocher de son père et l'avait enlevé de son emplacement habituel, à savoir l'anneau qu'il portait à son petit doigt, où le rocher avait été serti sous la forme d'un minuscule diamant, et il avait remis l'énorme rocher gris dans sa bourse. Juste au cas où sa magie lui ferait entièrement défaut lorsqu'il serait confronté à la plus sombre de toutes les créatures.

Harry commençait déjà à se sentir pessimiste, et il ne faisait même pas encore face à un Détraqueur.

"Je parie que tu peux le faire et que je ne peux pas," dit Harry dans un souffle. "Je parie que c'est ce qui va se passer."

"Je ne n'ai rien ressenti," dit Hermione, sa voix encore moins audible que la sienne. "J'ai essayé ce matin et je m'en suis rendue compte. Quand j'ai brandi ma main à la fin, avant même de dire les mots, je n'ai rien ressenti."

Harry ne répondit pas. Il avait eu la même impression depuis le début, même s'il lui avait fallu cinq autres essais et cinq autres pensées heureuses avant de pouvoir se l'avouer. Il s'était senti creux à chaque fois qu'il avait essayé de brandir sa baguette~; le sort qu'il essayait d'apprendre ne lui correspondait pas.

"Ça ne veut pas dire qu'on va être des mages noirs," dit Harry. "Il y a plein de gens qui ne peuvent pas lancer de Patronus et qui ne sont pas des mages noirs. Godric Gryffondor n'était pas un mage noir…"

Godric avait vaincu des Seigneurs des Ténèbres, il s'était battu pour protéger le peuple des aristocrates et les Moldus des sorciers. Il avait eux de nombreux vrais amis et n'en avait pas perdu plus de la moitié pour une cause ou pour une autre. Il avait prêté l'oreille aux cris des blessés dans les armées qu'il avait levées pour défendre l'innocent~; de jeunes sorciers courageux s'étaient ralliés à ses appels, et il les avait ensuite enterrés. Jusqu'à ce qu'enfin, arrivé à un âge avancé, alors que sa magie venait de commencer à le quitter, il ne rassemble les trois autres sorciers les plus puissants de son temps pour faire émerger Poudlard du sol~; le seul grand accomplissement de Godric qui n'ait pas trait à la guerre, à aucune guerre, aussi juste soit-elle. C'était Salazar, et non Godric, qui avait enseigné le premier cours de magie de Bataille de Poudlard. Godric avait enseigné le premier cours de Botanique de Poudlard, la magie de la vie verte.

Il était demeuré incapable de lancer un Patronus jusqu'à la fin de ses jours.

Godric Gryffondor avait été un homme bon, pas un homme heureux.

Harry ne croyait pas à l'angoisse existentielle, il ne supportait pas les histoires de héros geignards, il savait qu'un milliard de personnes auraient tout donné pour pouvoir échanger leur place avec lui, et…

Et sur son lit de mort, Godric avait dit à Helga (car Salazar l'avait abandonné et Rowena était partie avant) qu'il ne regrettait rien, et qu'il ne mettait \emph{pas} en garde ses étudiants contre le projet de suivre ses traces, que personne ne devrait \emph{jamais} dire qu'il avait dit à quiconque de ne pas suivre ses traces. Si ça avait été la bonne voie pour \emph{lui}, alors il ne dirait jamais à personne que c'était un mauvais choix, pas même au plus jeune des élèves de Poudlard. Et pourtant, pour ceux qui \emph{suivraient} ses traces, il espérait qu'ils se rappelleraient qu'il avait dit à sa Maison qu'ils avaient le droit d'être plus heureux qu'il ne l'avait été. Qu'à partir de ce jour, le rouge et or seraient des couleurs chaudes.

Et entre ses larmes, Helga lui avait promis qu'elle s'en assurerait une fois devenue directrice.

Ce sur quoi Godric était mort sans laisser de fantôme derrière lui~; et Harry avait vivement remis le livre dans les mains de Hermione et s'était un peu éloigné pour qu'elle ne le voit pas pleurer.

Vous n'auriez pas cru qu'un livre avec un titre innocent comme "Le Patronus~: Ceux qui en étaient capables et les autres" serait le livre le plus triste que Harry avait jamais lu.

Harry…

Harry n'en avait aucune envie.

Finir dans ce livre.

Pas la moindre envie.

Le reste de l'école semblait croire que \emph{Pas de Patronus} voulait dire \emph{Méchant}, point à la ligne. Étrangement, le fait que Godric Gryffondor n'avait pas non plus été capable de lancer un Patronus semblait n'être jamais répété. Peut-être que les gens n'en parlaient pas pour respecter son dernier vœu, Fred et George n'étaient probablement pas au courant et Harry n'allait certainement pas le leur dire. Ou peut-être que tous les autres, ceux qui avaient eux aussi raté, ne le mentionnaient pas parce que c'était moins honteux, parce que l'amour-propre et le statut social étaient moins mis à mal par la réputation d'être méchant que par celle d'être malheureux.

Harry vit que Hermione clignait des yeux~; et il se demanda si elle pensait à Rowena Serdaigle, qui avait elle aussi beaucoup aimé les livres.

"Bon", chuchota Harry. "Un peu de bonne humeur. Si tu obtiens un Patronus corporel complet, que penses-tu que ton animal sera~?"

"Un loutre," répondit immédiatement Hermione.

"Un \emph{loutre}~?" chuchota Harry d'un ton incrédule.

"Oui, une loutre," dit Hermione. "Et le tien~?"

"Un faucon pèlerin," dit Harry sans hésiter. "Il peut plonger à plus de trois-cents kilomètres par heure, c'est l'animal le plus rapide qui soit." Le faucon pèlerin avait été l'animal préféré de Harry depuis toujours. Harry était fermement décidé à devenir un Animagus un jour, juste pour revêtir cette forme et voler de ses propres ailes et voir le monde en contrebas avec des yeux plus perçants… "Mais pourquoi une \emph{loutre}~?"

Hermione sourit mais ne répondit pas.

Et les immenses portes de Poudlard s'ouvrirent grand.

Les deux enfants marchèrent un moment le long d'un chemin qui menait à la forêt dés-interdite, et ils continuèrent à travers celle-ci. Le soleil descendait et s'approchait de l'horizon, les ombres s'étiraient, la lumière était filtrée par les branches nues des arbres d'hiver~; car on était en janvier et que les première année étaient les derniers de la journée à venir apprendre.

Puis le chemin fit un écart et prit une nouvelle direction, et ils la virent tous au loin, la clairière dans la forêt, le dur sol hivernal, l'herbe jaune séchée et blanchie par quelques restes de neige.

Les silhouettes humaines étaient encore petites à cette distance. Deux points de lumière blanche tamisée venaient des Patronus des Aurors, et le point plus vif de lumière argentée venait de celui du directeur, à côté de quelque chose…

Harry plissa les yeux.

Quelque chose…

Ça avait dut être son imagination, car il n'aurait dû être impossible qu'un Détraqueur atteigne quelque chose situé par-delà trois Patronus corporels, mais il pensa avoir ressenti la caresse du vide dans son esprit, effleurant la partie la plus profonde de celui-ci sans le moindre respect pour les barrières Occlumantiques.

Lorsqu'il rejoint les autres élèves qui grouillaient sur l'herbe flétrie et tachetée de neige, Seamus Finnigan avait un teint cendré et tremblait. Son Patronus avait réussi, mais il restait toujours cet intervalle entre le moment où le directeur dissipait son Patronus et celui où l'on était censé lancer le sien, cet intervalle pendant lequel vous faisiez face au Détraqueur sans la moindre protection.

Un maximum de vingt secondes d'exposition à vingt pas était certainement sans danger, même pour un sorcier de onze ans doté d'une faible résistance et d'un cerveau encore en maturation. La variance de la force avec laquelle un Détraqueur atteignait les gens était très grande, et c'était un autre de ces phénomènes encore mal compris~; mais vingt secondes étaient certainement sans danger.

Quarante secondes d'exposition à cinq pas d'un Détraqueur aurait \emph{peut-être} pu suffire à causer des dommages permanents, mais seulement chez les sujets les plus sensibles.

C'était un entraînement rude, même par rapport aux normes de Poudlard, où on apprenait à voler à dos d'hippogriffe en se faisant jeter sur l'un d'eux et en se faisant ordonner d'y aller. Harry n'était pas féru des tendances surprotectrices, et comparer la maturité d'un élève de Poudlard en quatrième année à celle d'un Moldu de quatorze montrait clairement que les Moldus étouffaient leurs enfants… mais même Harry commençait à se demander si ce n'était pas exagéré. Toutes les blessures ne pouvaient pas être soignées.

Mais si vous ne pouviez pas jeter le sort dans de telles circonstances, cela voulait dire que vous ne pouviez pas compter sur la protection du Patronus~; et l'excès de confiance en soi était encore plus dangereux pour un sorcier que pour un Moldu. Les Détraqueurs pouvaient non seulement vous vider de vos pensées heureuses mais aussi de votre magie et de votre énergie physique, ce qui signifiait que vous ne seriez peut-être \emph{pas} capable de transplaner si vous attendiez trop longtemps ou si vous ne pouviez par reconnaître la peur qui s'approchait avant que le Détraqueur ne soit suffisamment proche pour pouvoir lancer son attaque (lors de ses lectures, Harry avait découvert avec une horreur considérable qu'à en croire certains livres le baiser du Détraqueur \emph{mangeait votre âme} et que c'était la raison à l'origine du coma végétatif dans lequel il plongeait ses victimes. Et que des sorciers qui \emph{croyaient à cela} avaient délibérément utilisé le baiser du Détraqueur pour exécuter \emph{des criminels}. Il était certain que certains prétendus criminels étaient innocents, et même s'ils ne l'était pas… \emph{détruire leur âme~?} Si Harry avait cru aux âmes, il aurait… aucune idée, il était incapable de trouver une réponse qui aurait été appropriée).

Le directeur prenait la sécurité au sérieux, et il y avait donc trois Aurors montant la garde. Leur chef était un homme aux traits vaguement asiatiques, solennel sans être sinistre. C'était l'Auror Komodo, dont la baguette ne quittait jamais la main. Son Patronus était un orang-outan de lumière lunaire et celui-ci faisait des allers et retours entre le Détraqueur et les première année qui attendaient leur tour~; à côté de l'orang-outan avançait la panthère d'un blanc éclatant qui appartenait à l'Auror Butnaru. C'était un homme au regard perçant qui portait de longs cheveux noirs regroupés en un catogan et une longue barbiche tressée. Les deux Aurors et leur deux Patronus regardaient le Détraqueur. Du côté des élèves, l'Auror Goryanof se reposait. C'était un homme grand et pince, pâle et mal rasé, assis sur une chaise qu'il avait invoquée sans mot ni baguette, et il arborait maintenant une expression à la fois impénétrable et rêveuse tout en balayant son environnement immédiat du regard. Le professeur Quirrell était arrivé peu après que les essais de première année aient commencés, et ses yeux ne s'éloignaient jamais beaucoup de Harry. Le petit professeur Flitwick, qui avait été un champion de duel, jouait avec sa baguette d'un air absent~; et \emph{ses} yeux, depuis l'énorme barbe qui lui servait de visage, restaient braqués sur le professeur Quirrell.

Et ça devait être l'imagination de Harry, mais le professeur Quirrell semblait légèrement tressaillir à chaque fois que le Patronus du directeur disparaissait pour tester le prochain élève. Peut-être que le professeur Quirrell imaginait le même effet placebo que Harry, la vague de vide caressant son esprit.

"Anthony Goldstein," dit la voix du directeur.

Harry marcha en silence vers Seamus tandis qu'Anthony commençait à s'approcher du phénix d'argent, et de… cette chose qui était sous la cape en lambeaux.

"Qu'est-ce que tu as vu~?" demanda Harry à Seamus d'une voix basse.

De nombreux élèves n'avaient pas répondu à Harry lorsqu'il avait essayé d'obtenir des informations~; mais Seamus était Finnigan du Chaos, l'un des lieutenants de Harry. Peut-être que ce n'était pas juste mais…

"Mort," dit Seamus dans un souffle, "gris et gluant… mort et laissé dans l'eau un moment…"

Harry hocha la tête. "C'est ce que beaucoup de gens voient," dit Harry. Il exhibait de la confiance en lui, même si elle était fausse, parce que Seamus en avait besoin. "Vas manger du chocolat, tu te sentiras mieux."

Seamus hocha à son tour la tête et alla jusqu'à aux douceurs soignantes en chancelant.

"\emph{Expecto Patronum~!}" s'écria la voix d'un jeune garçon.

Il y eut des hoquets audibles, même venant des Aurors.

Harry pivota pour regarder-

Un brillant oiseau d'argent se tenait entre Anthony Goldstein et la cage. L'oiseau leva la tête et laissa échapper un cri, et le cri était lui aussi d'argent, et il avait la force et la beauté du métal.

Et quelque chose au fond de l'esprit de Harry dit~: \emph{si c'est un faucon pèlerin, je vais l'étrangler dans son sommeil.}

\emph{Tais-toi}, dit Harry à la pensée, \emph{tu veux qu'on devienne un Seigneur des Ténèbres~?}

\emph{À quoi bon~? C'est comme ça que tu vas finir de toute façon…}

Ce… ce n'était pas quelque chose que Harry se serait dit en temps normal.

\emph{C'est un effet placebo}, se répéta-t-il. \emph{Le Détraqueur ne peut pas vraiment m'atteindre à travers trois Patronus corporels, je m'imagine juste ce que c'est de le ressentir. Quand je ferai vraiment face au Détraqueur, j'aurai une sensation totalement différente et alors je saurai que je me comportais comme un idiot.}

Un léger frisson descendit le long de sa colonne vertébrale car l'idée lui était venu que oui, ce \emph{serait} complètement différent, et pas dans le bon sens du terme.

Le phénix d'argent éclatant revint à la vie par la baguette du directeur, et l'oiseau moins imposant disparut~; et Anthony Goldstein commença à retourner d'où il était venu.

Le directeur venait avec Anthony au lieu d'appeler le prochain nom, et le Patronus attendait derrière, gardant le Détraqueur.

Harry jeta un coup d'œil vers l'emplacement où Hermione se tenait, juste derrière la panthère lumineuse. Son tour aurait été le suivant, mais apparemment, il allait falloir attendre.

Elle avait l'air stressée.

Plus tôt, elle avait poliment demandé à Harry de bien vouloir arrêter d'essayer de la détendre.

Dumbledore avait un léger sourire en raccompagnant Anthony jusqu'aux autres~; léger seulement parce qu'il avait l'air très, très fatigué.

"Incroyable," dit Dumbledore d'une voix qui semblait bien faible comparée à son coffre habituel. "Un Patronus corporel, sa première année. Et un nombre incroyable de succès chez les autres jeunes élèves. Quirinus, je dois reconnaître que tu avais raison."

Le professeur Quirrell inclina la tête. "Simple à deviner, il me semble. Un Détraqueur attaque par la peur, et les enfants ont moins peur."

"\emph{Moins} peur~?" dit l'Auror Goryanof depuis son siège.

"C'est ce que j'ai dit moi aussi," dit Dumbledore. "Et le professeur Quirrell a fait remarquer que les adultes ont plus de courage, pas moins de peur~; je dois avouer que cela ne m'avait jamais traversé l'esprit."

"Ce n'était pas ma formulation \emph{exacte}," dit le professeur Quirrell d'une voix sèche, "mais cela suffira. Et pour le reste de notre accord, directeur~?"

"Comme vous le voudrez," dit Dumbledore avec réticence. "J'admets que je ne m'attendais pas à perdre ce pari, Quirinus, mais tu as prouvé ta sagesse."

Tous les élèves les regardaient, perplexes~; mis à part Hermione, qui regardait vers la cage et vers les grandes robes en déliquescence~; et à part Harry, qui regardait tout le monde puisqu'il s'imaginait être devenu paranoïaque.

Le professeur Quirrell dit d'un ton qui n'invitait à aucune réponse, "Je suis autorisé à enseigner le sortilège de la Mort à ceux qui voudront l'apprendre. Ce qui augmentera de façon conséquente leur protection contre les mages noirs et autres pestes, et il serait naïf de croire qu'ils n'apprendrait de toute façon aucune autre forme de magie mortelle." Le professeur Quirrell s'interrompit et ses yeux se plissèrent. "Directeur, je remarque avec respect que vous ne semblez pas aller bien. Je suggère que vous partiez pour laisser le reste du travail au professeur Flitwick."

Dumbledore secoua la tête. "Nous en avons presque fini, Quirinus. Je tiendrai le coup."

Hermione s'approcha d'Antony. "Capitaine Goldstein," dit-elle, et sa voix ne tremblait qu'un peu, "pourriez-vous me donner des conseils~?"

"N'aies pas peur," dit Anthony d'une voix ferme. "Ne penses à aucune des choses auxquelles il essaiera de te faire penser. Tu ne fais pas que tenir ta baguette devant toi comme un bouclier contre la peur, tu la \emph{brandis} pour faire partir la peur, c'est comme ça que tu transformes une pensée heureuse en quelque chose de solide…" Anthony haussa les épaules d'un air impuissant. "Je veux dire, j'avais \emph{entendu} tout ça avant, mais…"

D'autres élèves commençaient à se rassembler autour d'Anthony, armés de leurs questions.

"Mademoiselle Granger~?" dit le directeur. Sa voix avait été aimable, ou peut-être seulement faible.

Hermione redressa les épaules et le suivit.

"Qu'as-tu vu sous la cape~?" dit Harry à Anthony.

Anthony regarda Harry avec surprise et répondit~: "Un grand homme mort, je veux dire, comme mort et d'une couleur morte… ça faisait mal de le regarder et je savais que c'était comme ça que le Détraqueur essayait de m'atteindre."

Harry jeta un regard vers l'endroit où Hermione faisait face à la cage et à la cape.

Elle mit sa baguette en position, prête à exécuter les premiers gestes.

Le phénix du directeur disparut dans un éclair.

Et Hermione laissa échapper un petit cri pathétique, elle flancha -

- fit un pas en arrière, Harry pouvait voir sa baguette bouger, puis elle la brandit et dit "Expecto Patronum~!"

Rien ne se passa.

Elle fit demi-tour et courut.

"\emph{Expecto Patronum~!}" dit la voix plus grave du directeur, et le phénix d'argent revint à la vie dans un autre éclat de lumière.

La jeune fille chancela mais elle continua de courir. D'étranges sons s'échappaient de sa gorge.

"\emph{Hermione}" hurla Susan, tout comme Hannah, Daphné et Ernie, et ils commencèrent tous à courir vers elle tandis que Harry, qui avait toujours un temps d'avance, pivotait sur ses talons et courait vers la table sur laquelle se trouvait le chocolat.

Même après qu'il lui ait fourré le chocolat dans la bouche, qu'elle ait mâché et avalé, elle continua de respirer à grandes goulées et de pleurer, et ses yeux ne semblaient plus mettre au point.

\emph{Elle ne peut pas avoir été détraquée de façon permanente} songea Harry avec désespoir à l'attention de la confusion qui régnait à l'intérieur de lui, de l'horrible peur et de la furie mortelle qui commençaient à s'enrouler l'une autour de l'autre, \emph{ce n'est pas possible, elle n'a pas été exposée pendant plus de dix secondes, certainement pas quarante…}

Mais elle avait pu être \emph{temporairement} détraquée car, Harry s'en rendit compte à l'instant, il n'y avait aucune règle qui interdisait que l'on soit \emph{temporairement} blessé par un Détraqueur en seulement dix secondes, si l'on était assez sensible.

Ses yeux semblèrent alors mettre au point et darder autour d'elle, puis il s'arrêtèrent sur Harry.

"Harry," hoqueta-t-elle, et les autres élèves étaient silencieux. "Harry, non. \emph{Non~!}"

Harry eut soudain peur de demander à quoi elle faisait référence, était-\emph{il} dans ses pires souvenirs, ou dans un cauchemar qu'elle revivait éveillée~?

"\emph{Ne t'en approches pas~!}" dit Hermione. Sa main se tendit, elle l'attrapa par le revers de ses robes. "Tu ne dois pas t'en approcher Harry~! \emph{Il m'a parlé, il te connaît, il sait que tu es ici~!}"

"Qu'est-ce qui est -" dit Harry, puis il jura intérieurement d'avoir posé cette question.

"\emph{Le Détraqueur~!}" dit Hermione. Sa voix devint un cri perçant. "\emph{Le professeur Quirrell veut que le Détraqueur te mange~!}"

Le professeur Quirrell s'avança de quelques pas avec un empressement soudain~; mais il ne vint ensuite pas plus près (Harry était là, après tout). "Mademoiselle Granger," dit-il, la voix grave, "je pense que vous devriez prendre plus de chocolat."

"\emph{Professeur Flitwick, ne laissez pas Harry essayer, renvoyez-le~!}"

Le directeur était alors arrivé, et lui et le professeur Flitwick échangeaient des regards inquiets.

"Je n'ai pas entendu le Détraqueur parler," dit le directeur. "Mais quand-même…"

"Posez simplement la question," dit le professeur Quirrell d'un ton un peu las.

"Le Détraqueur a-t-il dit \emph{comment} il atteindrait Harry~?" dit le directeur.

"Ses parties les plus goûteuses en premier," dit Hermione, "il man- il mangerait…"

Hermione cligna des yeux. De la folie sembla revenir dans ses yeux.

Puis elle se mit à pleurer.

"Vous avez été trop courageuse, Mlle Granger," dit le directeur. Sa voix était aimable et clairement audible. "Bien plus brave que ce à quoi je m'attendais. Vous auriez dû vous détourner et courir, ne pas endurer cela et ne pas essayer d'achever votre sortilège. Lorsque vous serez plus âgée et plus forte, Mlle Granger, je sais que vous essaierez de nouveau et que vous réussirez."

"Je suis désolée," dit Hermione entre deux hoquets, "je suis désolée, je suis désolée, je suis désolée, … je suis désolée Harry, je ne peux pas te dire ce que j'ai vu, je n'ai pas regardé, je n'ai pas osé, je savais que c'était trop horrible pour être jamais vu…"

Ça aurait dû être Harry, mais il avait hésité parce que ses mains étaient pleines de chocolat~; et Ernie et Susan furent là, aidant Hermione à se relever de là où elle s'était écroulée, la menant vers la nourriture disposée sur la table.

Cinq barres de chocolat plus tard, Hermione semblait de nouveau aller mieux, et elle alla voir le professeur Quirrell et lui présenta ses excuses, mais elle regardait toujours Harry, à chaque fois qu'il regardait dans sa direction. Il avait fait un pas vers elle, un fois seulement, et s'était arrêté quand elle avait fait un pas en arrière. Ses yeux lui avait silencieusement demandé pardon et lui avaient demandé de bien vouloir la laisser en paix.

\later

Neville Londubat avait vu quelque chose de mort à moitié dissout qui courait en suintant et dont le visage ressemblait à une éponge écrasée.

C'était la pire chose que quiconque ai dit avoir vu jusqu'à présent. Neville avait pu créer une petite étincelle de lumière mais il avait fait preuve d'une grande présence d'esprit et d'intelligence en se détournant et en fuyant au lieu d'essayer de lancer son Patronus.

(Le directeur n'avait rien dit aux autres élèves, il n'avait dit à personne d'être moins courageux~; mais le professeur Quirrell avait calmement remarqué que c'était quand on commettait une erreur \emph{après} avoir été prévenu que l'ignorance devenait stupidité).

"Professeur Quirrell~?" dit Harry d'une voix basse, s'étant autant approché du professeur Quirrell qu'il l'osait. "Que voyez-\emph{vous} quand vous -"

"Ne me posez pas cette question." La voix était très neutre.

Harry hocha respectueusement la tête. "Si je puis me permettre, quelle était votre tournure de phrase \emph{initiale} lorsque vous avez parlé avec le directeur~?"

Sèchement. "Nos pires souvenirs ne peuvent qu'empirer avec l'âge."

"Ah," dit Harry. "Logique."

Quelque chose d'étrange brilla dans les yeux du professeur Quirrell, puis, regardant Harry~: "Espérons," dit le professeur Quirrell, "que vous réussirez cet essai, M. Potter. Car si vous le faites, le directeur vous enseignera peut-être sa technique consistant à utiliser un Patronus pour envoyer des messages impossibles à falsifier et à intercepter, et son importance en situation militaire ne saurait être trop soulignée. Ce serait un formidable avantage pour la Légion du Chaos, et un jour, je le soupçonne, pour tout ce pays. Mais si vous ne réussissez \emph{pas}, M. Potter… eh bien, \emph{je} comprendrai."

\later

Morag MacDougal avait dit "Ouille" d'une voix vacillante et le directeur avait immédiatement relancé son Patronus.

Parvati Patil avait créé un Patronus corporel en forme de tigre, plus grand que le phénix de Dumbledore mais pas tout à fait aussi éclatant. Il y avait eu une grande salve d'applaudissement venue du public mais pas autant de surprise que lorsque Anthony y était parvenu.

Puis ce fut le tour de Harry.

Le directeur appela son nom, et Harry prit peur.

Il savait, il savait qu'il allait échouer, et il savait que cela allait faire mal.

Mais il fallait quand même qu'il essaie~; parce que parfois, en présence d'un Détraqueur, un sorcier créait un Patronus corporel complet là où il n'avait jamais su produire une seule étincelle de lumière, et personne ne comprenait pourquoi.

Et parce que si Harry ne \emph{pouvait pas} se défendre contre les Détraqueurs, il lui faudrait apprendre à détecter leur approche, à reconnaître la sensation qu'ils provoqueraient dans son esprit et à courir avant qu'il ne soit trop tard.

\emph{Quel est mon pire souvenir…~?}

Harry s'était attendu à ce que le directeur le regarde avec inquiétude, ou avec espoir, ou qu'il lui donne un conseil profondément sage~; mais au lieu de cela Albus Dumbledore se contentait de le regarder d'un air calme et tranquille.

\emph{Il pense que je vais échouer mais il ne va pas me saper mes forces en me le disant}, pensa Harry, \emph{s'il avait quelque chose de vraiment encourageant à me dire il me l'aurait dit…}

La cage approcha. Elle était déjà ternie mais pas rouillée jusqu'à la moelle, pas encore…

La cape approcha. Elle était défaite et percée de trous jamais rapiécés~; selon l'Auror Goryanof, elle avait été neuve ce matin.

"Professeur~?" dit Harry. "Que voyez-vous~?"

La voix du directeur était calme elle aussi. "Les Détraqueurs sont faits de peur, et à mesure que ta peur du Détraqueur diminue, l'horreur de sa forme fait de même. Je vois un homme grand, mince et nu. Il ne pourrit pas. C'est juste qu'il est légèrement douloureux de le regarder. C'est tout. Que vois-tu, Harry~?"

… Harry ne pouvait pas voir sous la cape.

Pas vraiment, c'était que son esprit \emph{refusait} de voir ce qui se trouvait sous la cape.

Non, son esprit essayait de voir \emph{autre chose} sous la cape, Harry pouvait sentir ses yeux essayer de forcer une erreur. Mais il s'était entraîné du mieux qu'il pouvait à remarquer ce petit sentiment de confusion, à automatiquement s'empêcher d'inventer des explications~; et à chaque fois que son esprit essayait de commencer à inventer un mensonge au sujet de ce qui se trouvait sous la cape, ce réflexe était assez rapide pour le bloquer.

Harry regarda sous la cape et vit…

Une question posée. Il refusait de laisser son esprit voir quelque chose de faux, et il ne voyait donc rien, comme si la partie de son cortex visuel qui recevait ce signal avait juste cessé d'exister. Il y avait un angle mort sous la cape. Harry ne pouvait pas savoir ce qui s'y trouvait.

Sauf que c'était bien pire que n'importe quelle momie pourrissante.

L'horreur impossible à voir était toute proche maintenant, mais l'éclatant oiseau lunaire et le blanc phénix se tenaient encore entre eux.

Harry voulait s'enfuir comme les autres élèves l'avaient fait. La moitié de ceux qui n'avaient pas réussi leur Patronus ne s'était tout simplement pas présentée aujourd'hui. De ceux qui restaient, la moitié avait fuit avant que le directeur n'ait dissipé son Patronus, et personne n'avait rien dit. Il y avait eu un petit rire quand Terry s'était détourné et était rentré avant même que ce soit son tour~; et Susan et Hannah, qui étaient déjà passées, avaient crié sur tous les autres et leur avaient ordonnés de se taire.

Mais Harry était le Survivant et il perdrait beaucoup de respect si on le voyait abandonner sans même essayer.

L'orgueil et les rôles semblaient s'amoindrir et tomber face à cette chose inconnue qui se trouvait sous la cape.

\emph{Pourquoi suis-je encore ici~?}

Ce ne fut pas la honte à l'idée que les autres le croient lâche qui maintint ses pieds où ils étaient.

Ce ne fut pas le désir de réparer sa réputation qui lui fit lever sa baguette.

Ce ne fut pas l'envie de réussir un Patronus qui plaça ses doigts dans la position initiale.

C'était autre chose, il \emph{fallait} que quelqu'un s'oppose à ce qui se trouvait sous la cape, c'étaient là les véritables ténèbres et Harry devait découvrir s'il abritait en lui-même le pouvoir de les repousser.

Il avait prévu d'essayer une dernière fois de penser à sa journée d'emplettes littéraires avec son père, mais au lieu de cela, au dernier moment, face au Détraqueur, un autre souvenir occupa son esprit, quelque chose qu'il n'avait encore jamais essayé~; une pensée qui n'était pas chaleureuse et plaisante dans le sens ordinaire mais qui semblait pourtant plus juste.

Et Harry se souvint des étoiles, il se souvint d'elles brûlant d'une terrible lueur, inébranlables dans la Nuit Silencieuse~; il laissa cette image croître en lui, croître comme une barrière Occlumantique d'un bout à l'autre de son esprit, il devint à nouveau la conscience corporelle du vide.

Le clair phénix d'argent disparut.

Et le Détraqueur s'écrasa dans son esprit, tel le poing de Dieu lui-même.

\textbf{PEUR / FROID / TÉNÈBRES}

Pendant un instant, les deux forces se heurtèrent de front et le paisible souvenir constellé d'étoiles tint bon face à la peur, alors que les doigts de Harry commençaient les mouvements de baguette pratiqués jusqu'à être devenus automatiques. Ils n'étaient pas chaleureux et plaisants, ces points de lumières éclatants sur fond de noir absolu~; mais c'était une image que le Détraqueur ne pouvait pas facilement percer. Car les étoiles silencieuses et brûlantes étaient vastes, elles ne connaissaient pas la peur, et briller au milieu du froid et des ténèbres constituait leur état naturel.

Mais il y eut un défaut, une fêlure, une ligne de fracture dans l'objet immuable qui tentait de résister à l'irrésistible force. Harry ressentit une pointe de colère contre le Détraqueur qui osait essayer de se nourrir de lui, et ce fut comme de glisser sur de la glace mouillée. L'esprit de Harry commença à s'écarter vers l'amertume, vers la furie noire, vers la haine mortelle -

La main de Harry s'était levée dans le mouvement de brandissement final.

Ça n'allait pas.

"Expecto Patronum," dit sa voix, les mots creux et vides de sens.

Et Harry tomba dans son côté obscur, tomba dans son côté obscur, plus loin et plus vite et plus profond que jamais, plus bas plus bas plus bas et la chute accéléra alors que le Détraqueur s'accrochait aux parties exposées et vulnérables et qu'il s'en nourrissait, mangeant la lumière. Un réflexe faiblissant fouilla à la recherche de chaleur, mais même lorsqu'une image de Hermione lui venait, ou une image de Maman et Papa, le Détraqueur la tordait, lui montrait Hermione allongée par terre, morte, les corps de son père et de sa mère, puis même cela fut absorbé.

Du vide émergea le souvenir, le pire de tous, quelque chose d'oublié il y a si longtemps que la structure neuronale n'aurait plus dû exister.
\begin{em}
"Lily, prends Harry et pars~! C'est lui~!" cria la voix d'un homme. "Cours~! Allez~! Je le retiendrai~!"

Et Harry ne pouvait s'empêcher de penser, depuis les profondeurs vides de son côté obscur, à quel point l'excès de confiance en lui de James avait été ridicule. Retenir Lord Voldemort~? Avec quoi~?

Puis l'autre voix parla, haut percée comme le sifflement d'une théière, et ce fut comme de la glace séchée qu'on aurait répandu sur chaque nerf de Harry, comme un tison de métal refroidi jusqu'à avoir atteint la température de l'hélium liquide qu'on aurait fait passer à la surface de tout son corps. Et la voix dit~:

"Avadakedavra."
\end{em}

(La baguette s'envola des doigts flasques du garçon tandis que son corps commençait à convulser et à tomber, les yeux du directeur maintenant alarmé s'écarquillant tandis qu'il commençait à lancer son propre Patronus).
\begin{em}

"Pas Harry, pas Harry, s'il vous plaît pas Harry~!" hurla la voix de la femme.

Le peu qui restait de Harry écoutait cela alors que toute lumière avait été extirpée de lui, dans le vide mort de son corps, et il se demanda si elle pensait que Lord Voldemort s'arrêterait parce qu'elle avait demandé poliment.

"Écarte-toi, femme~!" dit la voix stridente d'un froid brûlant. "Ce n'est pas pour toi que je suis venu, mais pour le garçon."

"Pas Harry~! S'il vous plaît… ayez pitié… ayez pitié…"

Harry songea que Lily Potter ne semblait pas comprendre quel genre de personne devenait Seigneur des Ténèbres en premier lieu~; et si c'était là la meilleure stratégie qu'elle pouvait concevoir pour sauver la vie de son fils, alors c'était aussi son échec final en tant que mère.

"Je te donne la rare chance de t'échapper," dit la voix stridente. "Mais je ne ferai pas l'effort de te maîtriser, et ta mort, ici, ne sauvera pas ton enfant. Écarte-toi, femme imbécile, si tu as le moindre bon sens~!"

"Pas Harry, s'il vous plaît, non, prenez moi, tuez moi à la place~!"

La chose vide qu'était devenu Harry se demanda si Lily Potter imaginait sérieusement que Lord Voldemort dirait oui, qu'il la tuerait et qu'il laisserait son fils sain et sauf.

"Très bien," dit la voix de la mort d'un ton à présent froidement amusé. "J'accepte le marché. Tu mourras, et l'enfant vivra. Maintenant abaisses ta baguette que je puisse te tuer."

Il y eut un silence hideux.

Lord Voldemort commença à rire, horrible rire méprisant.

Puis, enfin, la voix de Lily cria avec une haine désespérée~: "Avada ke -"

La voix mortelle finit la première, le sort rapide et précis.

"Avadakedavra."

Un éclat de vert aveuglant marqua la fin de Lily Potter.

Et le garçon dans son berceau les vit, ces yeux, ces deux yeux pourpres, qui semblaient briller d'un rouge vif, flamber comme deux soleils miniatures, emplissant le champ de vision de Harry alors qu'ils se braquaient sur ses yeux à lui -

\end{em}

\later

Les autres enfants virent Harry tomber, ils l'entendirent crier, un fin cri haut perché qui sembla percer leurs oreilles, tel un couteau.

Il y eut l'éclat d'argent et la voix du directeur qui mugissait "\emph{Expecto Patronum~!}", et le phénix embrasé revint à la vie.

Mais l'horrible cri de Harry continua encore et encore, alors que le directeur prenait le garçon dans ses bras et qu'il le portait loin du Détraqueur, alors que Neville Londubat et le professeur Flitwick couraient tous deux vers le chocolat au même moment et -

Hermione l'avait su, elle l'avait su lorsqu'elle l'avait vu, elle avait su que ses cauchemars avaient été réels, que ça devenait vrai, d'une façon ou d'une autre quelque chose était en train de devenir réalité.

"Donnez-lui du chocolat~!" exigea la voix du professeur Quirrell, bien inutilement car la forme menue du professeur Flitwick fonçait déjà vers l'endroit vers lequel le directeur accourait, non loin des élèves.

Hermione avançait elle aussi, même si elle ne savait pas ce qu'elle était censée faire -

"\emph{Lancez des Patronus~!}" hurla le directeur alors qu'il plaçait Harry derrière les Aurors. "\emph{Tous ceux qui le peuvent~! Mettez-les entre Harry et le Détraqueur~! Il se nourrit encore de lui~!}"

Il y eut un instant d'horreur glacée.

"\emph{Expecto Patronum}~!" crièrent le professeur Flitwick et l'Auror Goryanof, puis Anthony Goldstein, mais il échoua la première fois, puis Parvati Patil, qui réussit, puis Anthony Goldstein essaya à nouveau et son oiseau d'argent étendit ses ailes et hurla en direction du Détraqueur, et Dean Thomas rugit les mots comme s'ils avaient été écrits de lettres de feu et sa baguette donna naissance à un immense ours blancs~; il y avait huit Patronus qui brillaient le long d'une ligne séparant Harry du Détraqueur, et Harry continuait de crier et de crier alors que le directeur l'allongeait sur l'herbe sèche.

Hermione ne pouvait lancer de Patronus, alors elle courut vers l'endroit où Harry gisait. Dans son esprit, quelque chose essayait de deviner combien de temps s'était écoulé. Vingt secondes~? Plus~?

Il y avait une effroyable expression d'agonie et de perplexité sur le visage d'Albus Dumbledore. Sa longue baguette noire était dans sa main mais il ne prononçait aucun sort, il regardait seulement seulement le corps agité de convulsions de Harry avec horreur -

Hermione ne savait pas quoi faire, elle ne savait pas quoi faire, elle ne comprenait pas ce qui se passait, et le sorcier le plus puissant du monde semblait tout autant perdu qu'elle.

"\emph{Utilisez votre phénix~!}" mugit le professeur Quirrell. "\emph{Éloignez-le autant que possible de ce Détraqueur~!"}

Sans un mot le directeur prit Harry dans ses bras et disparut dans un éclat de feu au côté de Fumsec, qui venait subitement d'apparaître~; et le Patronus du directeur s'effaça instantanément, laissant un espace vide là où il avait gardé le Détraqueur.

Horreur et confusion et bavardages soudains.

"M. Potter devrait récupérer," dit le professeur Quirrell, élevant sa voix, mais son ton était à nouveau calme, "je pense que c'était juste un peu plus de vingt secondes."

Puis le phénix blanc embrasé apparut de nouveau, comme s'il était arrivé depuis les airs, et la créature de lumière lunaire alla vers Hermione Granger, et il cria de la voix d'Albus Dumbledore~:

"\emph{Il se nourrit encore de lui, même ici~! Comment~? Si tu le sais, Hermione Granger, tu dois me le dire~! Dis-le moi~!"}

L'Auror le plus âgé pivota pour la regarder, et de nombreux élèves firent de même. Le professeur Flitwick ne pivota pas, il tenait à présent sa baguette braquée sur le professeur Quirrell, qui avait mis ses mains vides en évidence.

D'infinies secondes s'écoulèrent.

Elle ne pouvait pas se souvenir, elle n'arrivait pas à se souvenir du cauchemar avec assez de précision, elle ne pouvait pas se souvenir de la raison pour laquelle elle avait pensé que ce serait possible, pourquoi elle avait eu peur -

Hermione se rendit compte de ce qu'elle devait faire, et c'était la décision la plus difficile de sa vie.

Et si ce qui était arrivé à Harry lui arrivait à elle aussi~?

Tous ses membres étaient aussi froids que la mort, son champ de vision s'assombrit, la peur écrasa tout~; elle avait vu Harry mourant, Maman et Papa mourant, tous ses amis mourants, tout le monde mourant, pour qu'à la fin, quand elle mourrait, ce soit seule. C'était le cauchemar secret dont elle n'avait jamais parlé à personne, le cauchemar qui avait permis au Détraqueur de la dominer, la chose pire entre toutes~: mourir seule.

Elle ne voulait pas y retourner, elle, elle ne voulait pas, elle ne voulait pas y rester pour toujours -

\emph{Tu as assez de courage pour Gryffondor}, dit la voix calme du Choixpeau, venue de ses souvenirs, \emph{mais tu feras le bien quelle que soit la maison que je te donne. Tu apprendras, tu soutiendras tes amis, quelle que soit la maison que tu choisis. Alors n'aie pas peur, Hermione Granger, choisis juste l'endroit qui te correspond…}

Il n'y avait pas assez de temps pour choisir, Harry était mourant.

"Je ne peux pas m'en souvenir pour l'instant," dit Hermione d'une voix qui se brisait, "mais attendez juste, je vais retourner voir le Détraqueur…"

Elle commença à courir vers le Détraqueur.

"Mademoiselle Granger~!" couina le professeur Flitwick, mais il ne fit rien pour l'arrêter, il garda juste sa baguette pointée vers le professeur Quirrell.

"\emph{Tout le monde~!}" hurla l'Auror Komodo de la voix d'un commandant militaire. "\emph{Mettez vos Patronus hors de sa route~!}"

"\shout{Flitwick~!}" rugit le professeur Quirrell. "\shout{Appelez la baguette de potter~!}"

Et, alors que Hermione comprenait, le professeur Flitwick criait déjà "\emph{Accio~!}", et elle vit le bout de bois grandir en s'approchant depuis l'endroit où il était tombé, où il avait presque touché la cage du Détraqueur.

\later

Les yeux s'ouvrirent, morts et vacants.

"\emph{Harry~!}" s'étrangla une voix venue du monde sans couleurs. "\emph{Harry~! Parle moi~!}"

Le visage de Albus Dumbledore se pencha jusqu'à entrer dans le champ de vision qui avait été précédemment occupé par un lointain toit de marbre.

"Tu es énervant," dit la voix vide. "Tu devrais mourir."  it felt wrong.”

Harry didn’t say anything. He’d felt the same thing, right from the start, though it had taken another five attempts using five other happy thoughts before he’d been able to acknowledge it to himself. Every time he tried to brandish his wand, it had felt hollow; the spell he was trying to learn didn’t fit him.

“It doesn’t mean we’re going to be Dark Wizards,” said Harry. “Lots of people who can’t cast the Patronus Charm aren’t Dark Wizards. Godric Gryffindor wasn’t a Dark Wizard…”

Godric had defeated Dark Lords, fought to protect commoners from Noble Houses and Muggles from wizards. He’d had many fine friends and true, and lost no fewer than half of them in one good cause or another. He’d listened to the screams of the wounded, in the armies he’d raised to defend the innocent; young wizards of courage had rallied to his calls, and he’d buried them afterwards. Until finally, when his wizardry had only just begun to fail him in his old age, he’d brought together the three other most powerful wizards of his era to raise Hogwarts from the bare ground; the one great accomplishment to Godric’s name that wasn’t about war, any kind of war, no matter how just. It was Salazar, and not Godric, who’d taught the first Hogwarts class in Battle Magic. Godric had taught the first Hogwarts class in Herbology, the magics of green growing life.

To his last day he’d never been able to cast the Patronus Charm.

Godric Gryffindor had been a good man, not a happy one.

Harry didn’t believe in angst, he couldn’t stand reading about whiny heroes, he knew a billion other people in the world would have given anything to trade places with him, and…

And on his deathbed, Godric had told Helga (for Salazar had abandoned him, and Rowena passed before) that he didn’t regret any of it, and he was \emph{not} warning his students not to follow in his footsteps, no-one was \emph{ever} to say he’d told anyone not to follow in his footsteps. If it had been the right thing for \emph{him} to do, then he wouldn’t tell anyone else to choose wrongly, not even the youngest student in Hogwarts. And yet for those who \emph{did} follow in his footsteps, he hoped they would remember that Gryffindor had told his House that it was all right for them to be happier than him. That red and gold would be bright warm colours, from now on.

And Helga had promised him, weeping, that when she was Headmistress she would make sure of it.

Whereupon Godric had died, and left no ghost behind him; and Harry had shoved the book back to Hermione and walked away a little, so she wouldn’t see him crying.

You wouldn’t think that a book with an innocent title like “The Patronus Charm: Wizards Who Could and Couldn’t” would be the saddest book Harry had ever read.

Harry…

Harry didn’t want that.

To be in that book.

Harry didn’t want that.

The rest of the school just seemed to think that \emph{No Patronus} meant \emph{Bad Person,} plain and simple. Somehow the fact that Godric Gryffindor also hadn’t been able to cast the Patronus Charm seemed not to get repeated. Maybe people didn’t talk about it to respect his last wish, Fred and George probably didn’t know and Harry certainly wasn’t about to tell them. Or maybe the other failures didn’t mention it because it was less shameful, the smaller loss of pride and status, to be thought Dark rather than unhappy.

Harry saw that Hermione, beside him, was blinking hard; and he wondered if she was thinking of Rowena Ravenclaw, who’d also loved books.

“Okay,” Harry whispered. “Happier thoughts. If you do go to a full corporeal Patronus, what do you think your animal will be?”

“An otter,” Hermione said at once.

“An \emph{otter?}” Harry whispered incredulously.

“Yes, an otter,” said Hermione. “What about yours?”

“Peregrine falcon,” Harry said without hesitation. “It can dive faster than three hundred kilometres per hour, it’s the fastest living creature there is.” The peregrine falcon had been Harry’s favourite animal since forever. Harry was determined to become an Animagus some day, just to get that as his form, and fly by the strength of his own wings, and see the land below with sharper eyes…”But why an \emph{otter?}”

Hermione smiled, but didn’t say anything.

And the vast doors of Hogwarts swung open.

They walked for a time, the children, over a pathway that led toward the un-forbidden forest, and continued through the forest itself. The Sun was lowering to near the horizon, the shadows long, the sunlight filtered through the bare branches of the winter trees; for it was January, and the first-years the last to learn, that day.

Then the path swerved and took a new direction, and they all saw it in the distance, the clearing in the forest, and the sere winter grounds, yellowing dried grass whitened by a few small remnants of snow.

The human figures still small at that range. The two spots of dim white light from the Aurors’ Patronuses, and the brighter spot of silver light from the Headmaster’s, next to something…

Harry squinted.

Something…

It must have been purely Harry’s imagination, because there shouldn’t have been any way for a Dementor to reach past three corporeal Patronuses, but he thought he could feel a touch of emptiness brushing at his mind, brushing straight at the soft inner centre of himself without any respect for Occlumency barriers.

\later

Seamus Finnigan was ashen and trembling as he rejoined the students milling about on the withered and snow-spotted grass. Seamus’s Patronus Charm had been successful, but there was still that interval between when the Headmaster dispelled his own Patronus and when you were supposed to cast your own, when you faced the Dementor’s fear unshielded.

Up to twenty seconds of exposure at five paces was certainly safe, even for an eleven-year-old wizard with weak resistance and a still-maturing brain. There was a lot of variance in how hard the Dementor’s power hit people, which was another thing not quite understood; but twenty seconds was definitely safe.

Forty seconds of Dementor exposure at five paces might \emph{possibly} have been enough to cause permanent damage, though only to the most sensitive subjects.

It was harsh training even by the standards of Hogwarts, where the way you learned to fly on a hippogriff was by being tossed on one and told to get going. Harry was no fan of over-protectiveness, and if you looked at the difference in maturity between a fourth-year in Hogwarts and a fourteen-year-old Muggle, it was clear that Muggles were smothering their children…but even Harry had started to wonder if this was pushing it. Not every hurt could be healed afterwards.

But if you couldn’t cast the spell under those conditions, it meant you couldn’t rely on using the Patronus Charm to defend yourself; overconfidence was even more dangerous to wizards than to Muggles. Dementors could drain your magic and your physical vitality, not just your happy thoughts, which meant you might \emph{not} be able to Apparate away if you waited too long, or if you didn’t recognize the approaching fear until the Dementor was within range for its attack. (During his reading, Harry had discovered with considerable horror that some books claimed the Dementor’s Kiss would \emph{eat your soul} and that this was the reason for the permanent mindless coma into which it put the victims. And that wizards who \emph{believed this} had deliberately used the Dementor’s Kiss to \emph{execute criminals.} It was a certainty that some called criminals were innocent, and even if they weren’t, \emph{destroying their souls?} If Harry had believed in souls, he would have…drawn a blank, he just couldn’t think of an appropriate response to that.)

The Headmaster was taking security seriously, and so were the three Aurors standing guard. Their leader was an Asian-looking man, solemn without being grim, Auror Komodo, whose wand never left his hand. His Patronus, an orang-utan of solid moonlight, paced back and forth between the Dementor and the first-years awaiting their turn; beside the orang-utan moved the bright white panther of Auror Butnaru, a man with a piercing gaze, long black hair in a ponytail, and a long braided goatee. Those two Aurors, and their two Patronuses, were all watching the Dementor. On the opposite side of the students was the resting Auror Goryanof, tall and thin and pale and unshaven, sitting back on a chair he’d conjured without word or wand, and maintaining an absent-minded poker face as he scanned the entire scene. Professor Quirrell had shown up not long after the first-years began their attempts, and his eyes never strayed far from Harry. The tiny Professor Flitwick, who had been a champion duellist, was fiddling absently with his wand; and \emph{his} eyes, peering out from within the huge puffy beard that served as his face, stayed focused on Professor Quirrell.

And it must have been Harry’s imagination, but Professor Quirrell seemed to wince slightly each time the Headmaster’s Patronus winked out to test the next student. Maybe Professor Quirrell was imagining the same placebo effect as Harry, that backwash of emptiness caressing at his mind.

“Anthony Goldstein,” called the voice of the Headmaster.

Harry quietly walked toward Seamus, even as Anthony began to approach the shining silver phœnix, and…whatever it was beneath the tattered cloak.

“What did you see?” Harry asked Seamus in a low voice.

A lot of students hadn’t answered Harry, when he’d tried to gather the data; but Seamus was Finnigan of Chaos, one of Harry’s lieutenants. Maybe that wasn’t fair, but…

“Dead,” said Seamus in a whisper, “greyish and slimy…dead and left in water for a while…”

Harry nodded. “That’s what a lot of people see,” Harry said. He projected confidence, even though it was fake, because Seamus needed it. “Go eat some chocolate, you’ll feel better.”

Seamus nodded and stumbled off toward the table of healing sweets.

“\emph{Expecto Patronum!}” cried a young boy’s voice.

Then there were gasps of shock, even from the Aurors.

Harry spun around to look—

There was a brilliant silver bird standing between Anthony Goldstein and the cage. The bird reared its head and let out a cry, and the cry was also silver, as bright and hard and beautiful as metal.

And something in the back of Harry’s mind said, \emph{if that’s a peregrine falcon, I’m going to strangle him in his sleep.}

\emph{Shut up,} Harry said to the thought, \emph{do you want us to be a Dark Wizard?}

\emph{What’s the point? You’re going to end up as one eventually.}

That…wasn’t something Harry would usually have thought…

\emph{It’s a placebo effect,} Harry told himself again. \emph{The Dementor can’t actually get to me through three corporeal Patronuses, I’m just imagining what I think it’s like. When I actually face the Dementor, it’ll feel completely different, and then I’ll know I was just being silly before.}

A slight chill went down Harry’s spine then, because he had a feeling that yes, it \emph{would} be completely different, and not in a positive direction.

The blazing silver phœnix sprang back into existence from the Headmaster’s wand, the lesser bird vanished; and Anthony Goldstein began to walk back.

The Headmaster was coming with Anthony instead of calling out the next name, the Patronus waiting behind to guard the Dementor.

Harry glanced over to where Hermione was standing, just behind the glowing panther. Hermione’s turn would have come next, but had apparently just been delayed.

She looked stressed.

Earlier, she’d politely asked Harry to please stop trying to de-stress her.

Dumbledore was smiling slightly as he escorted Anthony back toward the others; smiling only slightly, because the Headmaster looked very, very tired.

“Unbelievable,” said Dumbledore in a voice that sounded much weaker than his accustomed boom. “A corporeal Patronus, in his first year. And an astounding number of successes among the other young students. Quirinus, I must acknowledge that you have proved your point.”

Professor Quirrell inclined his head. “A simple enough guess, I should think. A Dementor attacks through fear, and children are less afraid.”

“\emph{Less} afraid?” said Auror Goryanof from where he was sitting.

“So I said as well,” said Dumbledore. “And Professor Quirrell pointed out that adults had more courage, not less to fear; which thought, I confess, had never occurred to me before.”

“That was not my \emph{precise} phrasing,” Professor Quirrell said dryly, “but it will do. And the rest of our agreement, Headmaster?”

“As you say,” Dumbledore said reluctantly. “I admit I was not expecting to lose that wager, Quirinus, but you have proven your wisdom.”

All the students were looking at them, puzzled; except Hermione, who was staring in the direction of the cage and the tall decaying robes; and Harry, who was watching everyone, since he was imagining himself feeling paranoid.

Professor Quirrell said, in tones that did not invite further comments, “I am allowed to teach the Killing Curse to students who wish to learn it. Which will render them considerably safer from Dark Wizards and other pests, and it is foolish to think they will otherwise know no deadly magics.” Professor Quirrell paused, his eyes narrowing. “Headmaster, I respectfully observe that you are not looking well. I suggest leaving the remainder of the day’s task to Professor Flitwick.”

Dumbledore shook his head. “We are almost done for the day, Quirinus. I will last.”

Hermione had approached Anthony. “Captain Goldstein,” she said, and her voice trembled only a little, “can you give me any advice?”

“Don’t be afraid,” Anthony said firmly. “Don’t think about anything it tries to make you think about. You’re not just holding up the wand in front of you as a shield against the fear, you’re \emph{brandishing} your wand to drive the fear away, that’s how you make a happy thought into something solid…” Anthony shrugged helplessly. “I mean, I \emph{heard} all that before, but…”

Other students were starting to congregate around Anthony, with their own questions.

“Miss~Granger?” the Headmaster said. His voice might have been gentle, or just weakened.

Hermione straightened her shoulders, and followed him.

“What did you see under the cloak?” Harry said to Anthony.

Anthony looked at Harry, surprised, and then answered, “A very tall man who was dead, I mean, sort of dead-shaped and dead-coloured…it hurt to see him and I knew that was the Dementor trying to get at me.”

Harry looked back out at where Hermione was confronting the cage and the cloak.

Hermione raised her wand into position for the first gestures.

The Headmaster’s phœnix winked out of existence.

And Hermione gave a tiny, pathetic shriek, flinched—

—took a step back, Harry could see her wand moving, and then she brandished it and said “Expecto Patronum!”

Nothing happened.

Hermione turned and ran.

“\emph{Expecto Patronum!}” said the Headmaster’s deeper voice, and the silver phœnix blazed back to life.

The young girl stumbled, and kept running, strange sounds beginning to come from her throat.

“\emph{Hermione!}” Susan yelled it, and Hannah, and Daphne, and Ernie, and they all started to run toward her; even as Harry, who was always thinking one step ahead, spun on his own heel and ran for the table with the chocolate.

Even after Harry had shoved the chocolate into Hermione’s mouth and she’d chewed and swallowed, she was still breathing in great gasps and crying, her eyes still seemed unfocused.

\emph{She can’t have been permanently Demented,} Harry thought desperately at the confusion inside him, the horrible fear and deathly fury beginning to twist around each other, \emph{she can’t have been, she wasn’t exposed for even ten seconds let alone forty—}

But she could be \emph{temporarily} Demented, as Harry realized in that moment, there wasn’t any rule that you couldn’t be \emph{temporarily} injured by a Dementor in just ten seconds if you were sensitive enough.

Then Hermione’s eyes seemed to focus, and dart around, and settle on him.

“Harry,” she gasped, and the other students went silent. “Harry, don’t. \emph{Don’t!}”

Harry was suddenly afraid to ask what he shouldn’t do, was \emph{he} in her worst memories, or some sleep’s nightmare that she was now reliving in waking life?

“\emph{Don’t go near it!}” said Hermione. Her hand reached out, grabbed him by the lapel of his robes. “You mustn’t go near it, Harry! \emph{It spoke to me, Harry, it knows you, it knows you’re here!}”

“What—” Harry said, and then cursed himself for asking.

“\emph{The Dementor!}” said Hermione. Her voice rose to a shriek. “\emph{Professor Quirrell wants it to eat you!}”

In the sudden hush, Professor Quirrell came forward a few steps; but he didn’t approach any closer (Harry was there, after all). “Miss~Granger,” he said, and his voice was grave, “I think you should have some more chocolate.”

“\emph{Professor Flitwick, don’t let Harry try, send him back!}”

The Headmaster had arrived by then, and he and Professor Flitwick were exchanging worried looks.

“I did not hear the Dementor speak,” the Headmaster said. “Still…”

“Just ask,” said Professor Quirrell, sounding a little weary.

“Did the Dementor say \emph{how} it would get to Harry?” said the Headmaster.

“All his tastiest parts first,” said Hermione, “it would—it would eat—”

Hermione blinked. Some sanity seemed to come back into her eyes.

Then she started crying.

“You were too brave, Hermione Granger,” the Headmaster said. His voice was gentle, and clearly audible. “Too much braver than I comprehended. You should have turned and run, not endured and tried to complete your Charm. When you are older and stronger, Miss~Granger, I know that you will try again, and I know that you will succeed.”

“I’m sorry,” Hermione said in gasps, “I’m sorry, I’m sorry, I’m sorry…I’m sorry, Harry, I can’t tell you what I saw, I didn’t look at it, I didn’t dare look at it, I knew it was too horrible to ever be seen…”

It should have been Harry, but he’d hesitated, because his hands were all chocolatey; and then Ernie and Susan were there, helping Hermione from where she’d fallen on the grass, leading her toward the snacks table.

Five bars of chocolate later, Hermione seemed to be all right again, and she went over and apologized to Professor Quirrell; but she was always watching Harry, every time that he glanced in her direction. He’d stepped toward her only once, and stopped when she’d stepped away. Her eyes had silently apologized, and silently pleaded for him to leave her be.

\later

Neville Longbottom had seen something dead and half-dissolved, oozing and running with a face like a squashed sponge.

It was the worst thing anyone had yet described seeing. Neville had been able to produce a small flicker of light from his wand before, but he had, intelligently and with great presence of mind, turned and run away instead of trying to cast his own Patronus Charm.

(The Headmaster had said nothing to the other students, told no-one else to be less brave; but Professor Quirrell had calmly observed that if you made the mistake \emph{after} being warned, that was when ignorance became stupidity.)

“Professor Quirrell?” Harry said in a low voice, having come as close to the Defence Professor as he dared. “What do \emph{you} see when you—”

“Don’t ask.” The voice was very flat.

Harry nodded respectfully. “What was your \emph{original} phrasing to the Headmaster, if I can ask?”

Dryly. “Our worst memories can only grow worse as we grow older.”

“Ah,” Harry said. “Logical.”

Something strange flickered in Professor Quirrell’s eyes, then, as he looked at Harry. “Let us hope,” Professor Quirrell said, “that you succeed upon this try, Mr~Potter. For if you do, the Headmaster may teach you his trick of using a Patronus to send messages that cannot be forged or intercepted, and the military importance of that is impossible to overstate. It would be a tremendous advantage to the Chaos Legion, and some day, I suspect, this entire country. But if you do \emph{not} succeed, Mr~Potter…well, \emph{I} shall understand.”

\later

Morag MacDougal had said, in a wavering voice, “Ouch”, and Dumbledore had recast his Patronus right away.

Parvati Patil had produced a corporeal Patronus in the form of a tiger, larger than Dumbledore’s phœnix, though not nearly as bright. There had been a great burst of applause from all the watchers, though not the same shock as when Anthony had done it.

And then it was Harry’s turn.

The Headmaster called the name of Harry Potter, and Harry was afraid.

Harry knew, he knew that he was going to fail, and he knew that it was going to hurt.

But he still had to try; because sometimes, in the presence of a Dementor, a wizard went from not a flicker of light to a full corporeal Patronus, and no-one understood why.

And because if Harry \emph{couldn’t} defend himself from Dementors, he had to be able to recognize their approach, recognize the feeling of them in his mind, and run before it was too late.

\emph{What is my worst memory…?}

Harry had expected the Headmaster to give him a worried look, or a hopeful look, or deeply wise advice; but instead Albus Dumbledore only watched him with quiet calm.

\emph{He thinks I’m going to fail, but he won’t sabotage me by telling me so,} thought Harry, \emph{if he had true words of encouragement to speak, he would speak them…}

The cage came closer. It was already tarnished, but not rusted away to nothing, not yet.

The cloak came closer. It was unravelling and shot through with unpatched holes; it had been new that morning, Auror Goryanof had said.

“Headmaster?” Harry said. “What do you see?”

The Headmaster’s voice was also calm. “The Dementors are creatures of fear, and as your fear of the Dementor diminishes, so does the fearsomeness of its form. I see a tall, thin, naked man. He is not decaying. He is only slightly painful to look upon. That is all. What do you see, Harry?”

…Harry couldn’t see under the cloak.

Or that wasn’t right, it was that his mind was \emph{refusing} to see what was under the cloak…

No, his mind was trying to see the \emph{wrong} thing under the cloak, Harry could feel it, his eyes trying to force a mistake. But Harry had done his best to train himself to notice that tiny feeling of confusion, to automatically flinch away from making stuff up; and every time his mind tried to start inventing a lie about what was under the cloak, that reflex was fast enough to shut it down.

Harry looked under the cloak and saw…

An open question. Harry wouldn’t let his mind see something false, and so he didn’t see anything, like the part of his visual cortex getting that signal was just ceasing to exist. There was a blind spot under the cloak. Harry couldn’t know what was under there.

Just that it was far worse than any decaying mummy.

The unseeable horror beneath the cloak was very close, now, but the blazing bird of moonlight, the white phœnix, yet lay between them.

Harry wanted to run away like some of the other students had. Half the ones who’d had no luck with their Patronus Charms just hadn’t shown up today in the first place. Of those remaining, half had fled before the Headmaster had even dispelled his own Patronus, and no-one had said a word. There’d been a little laughter when Terry had turned and walked back before his own try; and Susan and Hannah, who’d gone before, had yelled at everyone to shut up.

But Harry was the Boy-Who-Lived, and he would lose much respect if he was seen to give up without even trying…

Pride and roles seemed to diminish and fall away, in the presence of whatever lay beneath the cloak.

\emph{Why am I still here?}

It wasn’t the shame of others thinking him cowardly, that kept Harry’s feet in place.

It wasn’t the hope of repairing his reputation that brought up his wand.

It wasn’t the desire to master the Patronus Charm as magic, that moved his fingers into the initial position.

It was something else, something that \emph{had} to oppose whatever lay beneath the cloak, this was the true darkness and Harry had to find out whether it lay within him, the power to drive it back.

Harry had planned to try one final time to think of his book-shopping spree with his father, but instead, at the last minute, facing the Dementor, a different memory occurred to him, something he hadn’t tried before; a thought that wasn’t warm and happy in the ordinary way, but felt righter, somehow.

And Harry remembered the stars, remembered them burning terribly bright and unwavering in the Silent Night; he let that image fill him, fill all of him like an Occlumency barrier across his entire mind, became once again the bodiless awareness of the void.

The bright silver shining phœnix vanished.

And the Dementor smashed into his mind like the fist of God.

\textbf{FEAR / COLD / DARKNESS}

There was an instant when the two forces clashed head-on, when the peaceful starlit memory held its own against the fear, even as Harry’s fingers began the wand motions, practised until they had become automatic. They weren’t warm and happy, those blazing points of light in perfect blackness; but it was an image the Dementor could not easily pierce. For the silent burning stars were vast and unafraid, and to shine in the cold and darkness was their natural state.

But there was a flaw, a crack, a fault-line in the immovable object trying to resist that irresistible force. Harry felt a twinge of anger at the Dementor for trying to feed on him, and it was like slipping on wet ice. Harry’s mind began to slide sideways, into bitterness, black fury, deathly hatred—

Harry’s wand came up in the final brandish.

It felt wrong.

“Expecto Patronum,” his voice spoke, the words hollow and pointless.

And Harry fell into his dark side, fell down into his dark side, further and faster and deeper than ever before, down down down as the slide accelerated, as the Dementor latched onto the exposed and vulnerable parts and fed on them, eating away the light. A fading reflex scrabbled for warmth, but even as an image of Hermione came to him, or an image of Mum and Dad, the Dementor twisted it, showed him Hermione lying dead on the ground, the corpses of his mother and father, and then even that was sucked away.

Into the vacuum rose the memory, the worst memory, something forgotten so long ago that the neural patterns shouldn’t have still existed.

\begin{em}
“Lily, take Harry and go! It’s him!” shouted a man’s voice. “Go! Run! I’ll hold him off!”

And Harry couldn’t help but think, in the empty depths of his dark side, how ridiculously overconfident James Potter had been. Hold off Lord Voldemort? With what?

Then the other voice spoke, high-pitched like the hiss of a kettle, and it was like dry ice laid on Harry’s every nerve, like a brand of metal cooled to liquid helium temperatures and laid on every part of him. And the voice said:

“Avada Kedavra.”
\end{em}

(The wand flew from the boy’s nerveless fingers as his body began to convulse and fall, the Headmaster’s eyes widening in alarm as he began his own Patronus Charm.)

\begin{em}
“Not Harry, not Harry, please not Harry!” screamed the woman’s voice.

Whatever was left of Harry listened with all the light drained out of him, in the dead void of his heart, and wondered if she thought that Lord Voldemort would stop because she asked politely.

“Step aside, woman!” said the shrill voice of burning cold. “For you I am not come, only the boy.”

“Not Harry! Please…have mercy…have mercy…”

Lily Potter, Harry thought, seemed not to understand what type of people became Dark Lords in the first place; and if this was the best strategy she could conceive to save her child’s life, that was her final failure as a mother.

“I give you this rare chance to flee,” said the shrill voice. “But I will not trouble myself to subdue you, and your death here will not save your child. Step aside, foolish woman, if you have any sense in you at all!”

“Not Harry, please no, take me, kill me instead!”

The empty thing that was Harry wondered if Lily Potter seriously imagined that Lord Voldemort would say yes, kill her, and then depart leaving her son unharmed.

“Very well,” said the voice of death, now sounding coldly amused, “I accept the bargain. Yourself to die, and the child to live. Now drop your wand so that I can murder you.”

There was a hideous silence.

Lord Voldemort began to laugh, horrible contemptuous laughter.

And then, at last, Lily Potter’s voice shrieked in desperate hate, “Avada Ke—”

The lethal voice finished first, the curse rapid and precise.

“Avada Kedavra.”

A blinding flare of green marked the end of Lily Potter.

And the boy in the crib saw it, the eyes, those two crimson eyes, seeming to glow bright red, to blaze like miniature suns, filling Harry’s whole vision as they locked to his own— \end{em}

\later

The other children saw Harry Potter fall, they heard Harry Potter scream, a thin high-pitched scream that seemed to pierce their ears like knives.

There was a brilliant silver flash as the Headmaster bellowed “\emph{Expecto Patronum!}” and the blazing phœnix returned to being.

But Harry Potter’s horrible scream went on and on and on, even as the Headmaster scooped up the boy in his arms and bore him away from the Dementor, even as Neville Longbottom and Professor Flitwick both went for the chocolate at the same time and—

Hermione knew it, she knew it as she saw it, she knew that her nightmare had been real, it was coming true, somehow it was coming true.

“Get him chocolate!” demanded the voice of Professor Quirrell, pointlessly, because Professor Flitwick’s tiny form was already cannonballing toward where the Headmaster was racing toward the students.

Hermione was moving forward herself, though she didn’t know what else she meant to do—

“\emph{Cast Patronuses!}” shouted the Headmaster, as he brought Harry behind the Aurors. “\emph{Everyone who can! Get them between Harry and the Dementor! It’s still feeding on him!}”

There was a moment of frozen horror.

“\emph{Expecto Patronum!}” shouted Professor Flitwick and Auror Goryanof, and then Anthony Goldstein, but he failed the first time, and then Parvati Patil, who succeeded, and then Anthony tried again and his silver bird spread its wings and screamed at the Dementor, and Dean Thomas roared the words like they had been written in letters of fire and his wand gave birth to a towering white bear; there were eight blazing Patronuses all in a line between Harry and the Dementor, and Harry went on screaming and screaming as the Headmaster laid him on the dried grass.

Hermione couldn’t cast a Patronus Charm, so she ran toward where Harry lay. In her mind, something tried to guess how long it had been already. Was it twenty seconds? More?

There was a dreadful agony and bewilderment on the face of Albus Dumbledore. His long black wand was in his hand, but he spoke no spells, only looked down at Harry’s convulsing body in horror—

Hermione didn’t know what to do, she didn’t know what to do, she didn’t understand what was happening, and the most powerful wizard in the world seemed equally at a loss.

“\emph{Use your phœnix!}” bellowed Professor Quirrell. “\emph{Take him far away from that Dementor!}”

Without a single word the Headmaster scooped up Harry in his arms and vanished in a crack of fire along with the suddenly appearing Fawkes; and the Headmaster’s Patronus winked out, where it had guarded the Dementor.

Horror and confusion and sudden babble.

“Mr~Potter should recover,” Professor Quirrell said, raising his voice, but his tone now calm once again, “I think it was just over twenty seconds.”

Then the blazing white phœnix appeared again, like it was flying before them from elsewhere, to Hermione Granger came the creature of moonlight, and it cried to her in Albus Dumbledore’s voice:

“\emph{It still feeds on him, even here! How? If you know, Hermione Granger, you must tell me! Tell me!}”

The senior Auror turned to stare at her, and so did many students. Professor Flitwick didn’t turn, he was now levelling his wand on Professor Quirrell, who was holding out clearly empty hands.

Seconds ticked past, uncounted.

She couldn’t remember it, she couldn’t remember the nightmare clearly, she couldn’t remember why she had thought it was possible, why she had been afraid—

Hermione realized then what she ought to do, and it was the hardest decision of her life.

What if whatever had happened to Harry, happened to her too?

All her limbs cold as death, her vision gone dark, fear overwhelming everything; she’d seen Harry dying, Mum and Dad dying, all her friends dying, everyone dying, so that in the end, when she died, she would be alone. That was her secret nightmare she’d never talked about with anyone, that had given the Dementor its power over her, the loneliest thing was to die alone.

She didn’t want to go to that place again, she, she didn’t, she didn’t want to stay there forever—

\emph{You have courage enough for Gryffindor,} said the calm voice of the Sorting Hat in her memory, \emph{but you will do what is right in any House I give you. You will learn, you will stand by your friends, in any House you choose. So don’t be afraid, Hermione Granger, just decide where you belong…}

There was no time for deciding, Harry was dying.

“I can’t remember now,” said Hermione, her voice cracking, “but just hold on, I’ll go in front of the Dementor again…”

She started to run toward the Dementor.

“Miss~Granger!” squeaked Professor Flitwick, but he made no move to stop her, only kept holding his wand on Professor Quirrell.

“\emph{Everyone!}” shouted Auror Komodo in a voice of military command. “\emph{Get your Patronuses out of her way!}”

“\shout{Flitwick!}” roared Professor Quirrell. “\shout{Summon Potter’s wand!}”

Even as Hermione understood, Professor Flitwick was already crying “\emph{Accio!}”, and she saw the stick of wood zooming up from where it had lain almost touching the Dementor’s cage.

\later

The eyes opened, dead and vacant.

“\emph{Harry!}” gasped a voice in the colourless world. “\emph{Harry! Speak to me!}”

The face of Albus Dumbledore leaned over into the field of vision, which had been occupied by a distant marble ceiling.

“You’re annoying,” said the empty voice. “You should die.”

%  LocalWords:  un Ke
