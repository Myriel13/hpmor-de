\partchapter{Humanisme}{I}

\lettrine{L}{e} doux soleil de janvier brillait sur froides étendues herbeuses aux alentours de Poudlard.

Pour certains élèves, c'était l'heure de l'étude~; d'autres avaient déjà fini les cours. Les première année qui s'étaient inscrits à la pratique d'un sort très particulier, un sort qui s'apprenait mieux en extérieur, sous un ciel bleu limpide et un soleil radieux plutôt que confiné dans une salle de classe. Les cookies et la limonade aidaient aussi.

Les premiers mouvements du sortilège étaient complexes et précis~; vous tordiez votre baguette une fois, deux fois, trois fois et quatre fois, de petites inclinaisons à des angles précis, puis vous décaliez votre index et votre pouce de la distance exacte…

Selon le ministère, cela impliquait qu'il était futile d'essayer d'enseigner ce sort à qui que ce soit avant sa cinquième année. Il y avait quelques cas connus d'enfants plus jeunes qui s'étaient révélés capables de l'apprendre, mais ils avaient été rejetés comme autant d'exceptions dues au "génie".

Ça n'avait peut-être pas été une façon très polie de le dire, mais Harry commençait à voir pourquoi le professeur Quirrell avait déclaré que le comité du ministère en charge du curriculum des élèves aurait été plus bénéfique pour les sorciers s'il avait été utilisé comme dépotoir.

Eh bien oui, les gestes étaient complexes et délicats. Cela n'empêchait personne de l'apprendre à onze ans. Cela voulait dire qu'il fallait être particulièrement attentif et qu'il fallait en pratiquer chaque partie plus longtemps que d'habitude, voilà tout.

La plupart des sortilèges pouvaient seulement être enseignés aux élèves plus âgés parce qu'ils nécessitaient une force magique au-delà de ce qu'un élève jeune aurait pu rassembler. Mais le Patronus ne fonctionnait \emph{pas} comme ça, il n'était pas difficile parce qu'il nécessitait trop de magie mais parce qu'il nécessitait \emph{plus} que de la simple magie.

Il nécessitait les sentiments heureux de ceux qui peuplaient votre cœur, les souvenirs aimants, un type de force qui était différente et qui n'était pas nécessaire aux sorts ordinaires.

Harry tordit sa main une fois, deux fois, trois fois et quatre fois, il décala ses doigts de la distance exacte…

"\emph{Bonne chance à l'école, Harry. Penses-tu que je t'ai acheté assez de livres~?"}

"\emph{On ne peut jamais acheter assez de livres… mais tu as certainement essayé, c'était un très, très, très bon essai…"}

La première fois que Harry s'en était souvenu et qu'il avait essayé de l'utiliser pour le sort, cela lui avait fait monter les larmes aux yeux.

Harry leva sa baguette et lui fit décrire un arc de cercle, puis il la brandit d'un geste qui n'avait pas à être précis, seulement hardi et plein de défiance.

"\emph{Expecto Patronum~!}" s'écria-t-il.

Rien ne se produisit.

Pas une seule étincelle de lumière.

Lorsqu'il releva les yeux, Remus Lupin examinait encore la baguette, un air assez inquiet sur son visage légèrement balafré.

Il finit par secouer la tête. "Je suis navré Harry," dit doucement l'homme. "Ton mouvement était parfait."

Et il n'y avait pas une seule étincelle de lumière où que ce soit, parce que tous les autres première année censés pratiquer leur Patronus avaient préféré regarder Harry du coin de l'œil.

Les larmes menaçaient de revenir dans les yeux de Harry, et ce n'étaient pas des larmes de joie. Entre toutes choses, entre toutes choses, Harry ne se serait jamais attendu à cela.

Qu'est-ce qu'Anthony Goldstein avait que Harry n'avait pas, et qui faisait briller sa baguette de cette vive lumière~?

Est-ce qu'il aimait plus son père~?

"Quelle pensée as-tu utilisée pour le lancer~?" dit Remus.

"Mon père," dit Harry d'une voix tremblante. "Je lui ai demandé de m'acheter quelques livres avant que je n'aille à Poudlard, et il l'a fait, et ils coûtaient cher, et puis il m'a demandé s'il y en avait assez -"

Il n'essaya pas d'expliquer la devise de la famille Verres.

"Reposes-toi avant d'essayer une autre pensée, Harry," dit Remus. Il fit un geste en direction d'un autre élève qui était assis au sol, et le visage de ce dernier avait une expression qui aurait pu signifier la déception, la gêne ou le regret. "Tu n'arriveras pas à lancer un Patronus si tu te sens honteux à l'idée de ne pas témoigner d'assez de gratitude." Il y avait une douce compassion dans la voix de M. Lupin, et l'espace d'un instant, Harry eut envie de frapper sur quelque chose.

Au lieu de cela il se détourna et darda jusqu'à l'endroit où les autres perdants étaient assis. Ces autres élèves dont les mouvements avaient été déclarés parfaits et qui étaient maintenant censés chercher des pensées plus heureuses~; à en voir leur tête, ils n'avançaient pas beaucoup. De nombreuses robes étaient bordées de bleu sombre, ainsi que quelques rouges, et une fille Poufsouffle esseulée pleurait encore. Les Serpentard ne s'étaient même pas fatigués à venir, mis à part Daphné Greengrass et Tracey Davis qui s'essayaient encore aux mouvements.

Harry se laissa chuter sur l'herbe froide et morte de l'hiver, à côté de l'élève dont l'échec l'avait le plus surpris.

"Alors tu n'as pas pu le faire non plus," dit Hermione. Elle avait d'abord fuit le lieu d'entraînement, mais elle avait fini par revenir, et il fallait observer ses yeux rougis de près pour voir qu'elle avait pleuré.

"Je," dit Harry, "je, je me sentirais probablement bien pire si tu n'avais pas échoué, tu es la plus gentille des personnes que je connais, que j'ai jamais rencontré, Hermione, et si \emph{tu} ne peux pas le faire non plus, ça veut dire que je suis peut-être quelqu'un de, quelqu'un de bien…"

"J'aurais dû aller à Gryffondor," chuchota Hermione. Elle cligna des yeux plusieurs fois mais elle ne les essuya pas.

\later

Le garçon et la fille marchaient ensemble, certainement pas en se tenait la main, mais chacun tirant une sorte de force de la présence de l'autre, quelque chose qui les laissait ignorer les murmures de leurs camarades tandis qu'il traversaient les couloirs approchants les grandes portes de Poudlard.

Harry n'était pas parvenu à lancer le Patronus, et ce quelle que soit la pensée heureuse qu'il ait essayée. Cela n'avait surpris personne, ce qui était encore pire. Hermione n'y était pas parvenue non plus. Cela avait \emph{beaucoup} surpris, et Harry l'avait vue commencer à recevoir le même genre de regard en biais que ceux qui lui étaient adressés. Les autres Serdaigle qui avaient échoué ne recevaient pas ces regards. Mais Hermione était le général Soleil et ses fans traitaient cela comme un manquement à ses engagements, comme si elle avait trahi une promesse qu'elle n'avait jamais faite.

Ils s'étaient tous deux rendus à la bibliothèque pour faire des recherches sur le Patronus, car c'était la façon dont Hermione réagissait en situation de détresse, tout comme Harry le faisait parfois. Étudier, apprendre, essayer de comprendre \emph{pourquoi}…

Les livres avaient confirmé ce que le directeur avait dit à Harry~; les sorciers incapables de jeter un Patronus parvenaient souvent à le faire en présence d'un vrai Détraqueur, passant de l'échec complet à un Patronus corporel. Cela défiait toute logique, l'aura de peur du Détraqueur aurait dû rendre l'apparition d'une pensée heureuse \emph{plus difficile}~; mais c'était ainsi.

Alors ils allaient tous les deux essayer une dernière fois. Il était impensable qu'ils en fassent autrement.

C'était le jour où le Détraqueur venait à Poudlard.

Plus tôt, Harry avait démétamorphosé le rocher de son père et l'avait enlevé de son emplacement habituel, à savoir l'anneau qu'il portait à son petit doigt, où le rocher avait été serti sous la forme d'un minuscule diamant, et il avait remis l'énorme rocher gris dans sa bourse. Juste au cas où sa magie lui ferait entièrement défaut lorsqu'il serait confronté à la plus sombre de toutes les créatures.

Harry commençait déjà à se sentir pessimiste, et il ne faisait même pas encore face à un Détraqueur.

"Je parie que tu peux le faire et que je ne peux pas," dit Harry dans un souffle. "Je parie que c'est ce qui va se passer."

"Je ne n'ai rien ressenti," dit Hermione, sa voix encore moins audible que la sienne. "J'ai essayé ce matin et je m'en suis rendue compte. Quand j'ai brandi ma main à la fin, avant même de dire les mots, je n'ai rien ressenti."

Harry ne répondit pas. Il avait eu la même impression depuis le début, même s'il lui avait fallu cinq autres essais et cinq autres pensées heureuses avant de pouvoir se l'avouer. Il s'était senti creux à chaque fois qu'il avait essayé de brandir sa baguette~; le sort qu'il essayait d'apprendre ne lui correspondait pas.

"Ça ne veut pas dire qu'on va être des mages noirs," dit Harry. "Il y a plein de gens qui ne peuvent pas lancer de Patronus et qui ne sont pas des mages noirs. Godric Gryffondor n'était pas un mage noir…"

Godric avait vaincu des Seigneurs des Ténèbres, il s'était battu pour protéger le peuple des aristocrates et les Moldus des sorciers. Il avait eux de nombreux vrais amis et n'en avait pas perdu plus de la moitié pour une cause ou pour une autre. Il avait prêté l'oreille aux cris des blessés dans les armées qu'il avait levées pour défendre l'innocent~; de jeunes sorciers courageux s'étaient ralliés à ses appels, et il les avait ensuite enterrés. Jusqu'à ce qu'enfin, arrivé à un âge avancé, alors que sa magie venait de commencer à le quitter, il ne rassemble les trois autres sorciers les plus puissants de son temps pour faire émerger Poudlard du sol~; le seul grand accomplissement de Godric qui n'ait pas trait à la guerre, à aucune guerre, aussi juste soit-elle. C'était Salazar, et non Godric, qui avait enseigné le premier cours de magie de Bataille de Poudlard. Godric avait enseigné le premier cours de Botanique de Poudlard, la magie de la vie verte.

Il était demeuré incapable de lancer un Patronus jusqu'à la fin de ses jours.

Godric Gryffondor avait été un homme bon, pas un homme heureux.

Harry ne croyait pas à l'angoisse existentielle, il ne supportait pas les histoires de héros geignards, il savait qu'un milliard de personnes auraient tout donné pour pouvoir échanger leur place avec lui, et…

Et sur son lit de mort, Godric avait dit à Helga (car Salazar l'avait abandonné et Rowena était partie avant) qu'il ne regrettait rien, et qu'il ne mettait \emph{pas} en garde ses étudiants contre le projet de suivre ses traces, que personne ne devrait \emph{jamais} dire qu'il avait dit à quiconque de ne pas suivre ses traces. Si ça avait été la bonne voie pour \emph{lui}, alors il ne dirait jamais à personne que c'était un mauvais choix, pas même au plus jeune des élèves de Poudlard. Et pourtant, pour ceux qui \emph{suivraient} ses traces, il espérait qu'ils se rappelleraient qu'il avait dit à sa Maison qu'ils avaient le droit d'être plus heureux qu'il ne l'avait été. Qu'à partir de ce jour, le rouge et or seraient des couleurs chaudes.

Et entre ses larmes, Helga lui avait promis qu'elle s'en assurerait une fois devenue directrice.

Ce sur quoi Godric était mort sans laisser de fantôme derrière lui~; et Harry avait vivement remis le livre dans les mains de Hermione et s'était un peu éloigné pour qu'elle ne le voit pas pleurer.

Vous n'auriez pas cru qu'un livre avec un titre innocent comme "Le Patronus~: Ceux qui en étaient capables et les autres" serait le livre le plus triste que Harry avait jamais lu.

Harry…

Harry n'en avait aucune envie.

Finir dans ce livre.

Pas la moindre envie.

Le reste de l'école semblait croire que \emph{Pas de Patronus} voulait dire \emph{Méchant}, point à la ligne. Étrangement, le fait que Godric Gryffondor n'avait pas non plus été capable de lancer un Patronus semblait n'être jamais répété. Peut-être que les gens n'en parlaient pas pour respecter son dernier vœu, Fred et George n'étaient probablement pas au courant et Harry n'allait certainement pas le leur dire. Ou peut-être que tous les autres, ceux qui avaient eux aussi raté, ne le mentionnaient pas parce que c'était moins honteux, parce que l'amour-propre et le statut social étaient moins mis à mal par la réputation d'être méchant que par celle d'être malheureux.

Harry vit que Hermione clignait des yeux~; et il se demanda si elle pensait à Rowena Serdaigle, qui avait elle aussi beaucoup aimé les livres.

"Bon", chuchota Harry. "Un peu de bonne humeur. Si tu obtiens un Patronus corporel complet, que penses-tu que ton animal sera~?"

"Un loutre," répondit immédiatement Hermione.

"Un \emph{loutre}~?" chuchota Harry d'un ton incrédule.

"Oui, une loutre," dit Hermione. "Et le tien~?"

"Un faucon pèlerin," dit Harry sans hésiter. "Il peut plonger à plus de trois-cents kilomètres par heure, c'est l'animal le plus rapide qui soit." Le faucon pèlerin avait été l'animal préféré de Harry depuis toujours. Harry était fermement décidé à devenir un Animagus un jour, juste pour revêtir cette forme et voler de ses propres ailes et voir le monde en contrebas avec des yeux plus perçants… "Mais pourquoi une \emph{loutre}~?"

Hermione sourit mais ne répondit pas.

Et les immenses portes de Poudlard s'ouvrirent grand.

Les deux enfants marchèrent un moment le long d'un chemin qui menait à la forêt dés-interdite, et ils continuèrent à travers celle-ci. Le soleil descendait et s'approchait de l'horizon, les ombres s'étiraient, la lumière était filtrée par les branches nues des arbres d'hiver~; car on était en janvier et que les première année étaient les derniers de la journée à venir apprendre.

Puis le chemin fit un écart et prit une nouvelle direction, et ils la virent tous au loin, la clairière dans la forêt, le dur sol hivernal, l'herbe jaune séchée et blanchie par quelques restes de neige.

Les silhouettes humaines étaient encore petites à cette distance. Deux points de lumière blanche tamisée venaient des Patronus des Aurors, et le point plus vif de lumière argentée venait de celui du directeur, à côté de quelque chose…

Harry plissa les yeux.

Quelque chose…

Ça avait dut être son imagination, car il n'aurait dû être impossible qu'un Détraqueur atteigne quelque chose situé par-delà trois Patronus corporels, mais il pensa avoir ressenti la caresse du vide dans son esprit, effleurant la partie la plus profonde de celui-ci sans le moindre respect pour les barrières Occlumantiques.

Lorsqu'il rejoint les autres élèves qui grouillaient sur l'herbe flétrie et tachetée de neige, Seamus Finnigan avait un teint cendré et tremblait. Son Patronus avait réussi, mais il restait toujours cet intervalle entre le moment où le directeur dissipait son Patronus et celui où l'on était censé lancer le sien, cet intervalle pendant lequel vous faisiez face au Détraqueur sans la moindre protection.

Un maximum de vingt secondes d'exposition à vingt pas était certainement sans danger, même pour un sorcier de onze ans doté d'une faible résistance et d'un cerveau encore en maturation. La variance de la force avec laquelle un Détraqueur atteignait les gens était très grande, et c'était un autre de ces phénomènes encore mal compris~; mais vingt secondes étaient certainement sans danger.

Quarante secondes d'exposition à cinq pas d'un Détraqueur aurait \emph{peut-être} pu suffire à causer des dommages permanents, mais seulement chez les sujets les plus sensibles.

C'était un entraînement rude, même par rapport aux normes de Poudlard, où on apprenait à voler à dos d'hippogriffe en se faisant jeter sur l'un d'eux et en se faisant ordonner d'y aller. Harry n'était pas féru des tendances surprotectrices, et comparer la maturité d'un élève de Poudlard en quatrième année à celle d'un Moldu de quatorze montrait clairement que les Moldus étouffaient leurs enfants… mais même Harry commençait à se demander si ce n'était pas exagéré. Toutes les blessures ne pouvaient pas être soignées.

Mais si vous ne pouviez pas jeter le sort dans de telles circonstances, cela voulait dire que vous ne pouviez pas compter sur la protection du Patronus~; et l'excès de confiance en soi était encore plus dangereux pour un sorcier que pour un Moldu. Les Détraqueurs pouvaient non seulement vous vider de vos pensées heureuses mais aussi de votre magie et de votre énergie physique, ce qui signifiait que vous ne seriez peut-être \emph{pas} capable de transplaner si vous attendiez trop longtemps ou si vous ne pouviez par reconnaître la peur qui s'approchait avant que le Détraqueur ne soit suffisamment proche pour pouvoir lancer son attaque (lors de ses lectures, Harry avait découvert avec une horreur considérable qu'à en croire certains livres le baiser du Détraqueur \emph{mangeait votre âme} et que c'était la raison à l'origine du coma végétatif dans lequel il plongeait ses victimes. Et que des sorciers qui \emph{croyaient à cela} avaient délibérément utilisé le baiser du Détraqueur pour exécuter \emph{des criminels}. Il était certain que certains prétendus criminels étaient innocents, et même s'ils ne l'était pas… \emph{détruire leur âme~?} Si Harry avait cru aux âmes, il aurait… aucune idée, il était incapable de trouver une réponse qui aurait été appropriée).

Le directeur prenait la sécurité au sérieux, et il y avait donc trois Aurors montant la garde. Leur chef était un homme aux traits vaguement asiatiques, solennel sans être sinistre. C'était l'Auror Komodo, dont la baguette ne quittait jamais la main. Son Patronus était un orang-outan de lumière lunaire et celui-ci faisait des allers et retours entre le Détraqueur et les première année qui attendaient leur tour~; à côté de l'orang-outan avançait la panthère d'un blanc éclatant qui appartenait à l'Auror Butnaru. C'était un homme au regard perçant qui portait de longs cheveux noirs regroupés en un catogan et une longue barbiche tressée. Les deux Aurors et leur deux Patronus regardaient le Détraqueur. Du côté des élèves, l'Auror Goryanof se reposait. C'était un homme grand et pince, pâle et mal rasé, assis sur une chaise qu'il avait invoquée sans mot ni baguette, et il arborait maintenant une expression à la fois impénétrable et rêveuse tout en balayant son environnement immédiat du regard. Le professeur Quirrell était arrivé peu après que les essais de première année aient commencés, et ses yeux ne s'éloignaient jamais beaucoup de Harry. Le petit professeur Flitwick, qui avait été un champion de duel, jouait avec sa baguette d'un air absent~; et \emph{ses} yeux, depuis l'énorme barbe qui lui servait de visage, restaient braqués sur le professeur Quirrell.

Et ça devait être l'imagination de Harry, mais le professeur Quirrell semblait légèrement tressaillir à chaque fois que le Patronus du directeur disparaissait pour tester le prochain élève. Peut-être que le professeur Quirrell imaginait le même effet placebo que Harry, la vague de vide caressant son esprit.

"Anthony Goldstein," dit la voix du directeur.

Harry marcha en silence vers Seamus tandis qu'Anthony commençait à s'approcher du phénix d'argent, et de… cette chose qui était sous la cape en lambeaux.

"Qu'est-ce que tu as vu~?" demanda Harry à Seamus d'une voix basse.

De nombreux élèves n'avaient pas répondu à Harry lorsqu'il avait essayé d'obtenir des informations~; mais Seamus était Finnigan du Chaos, l'un des lieutenants de Harry. Peut-être que ce n'était pas juste mais…

"Mort," dit Seamus dans un souffle, "gris et gluant… mort et laissé dans l'eau un moment…"

Harry hocha la tête. "C'est ce que beaucoup de gens voient," dit Harry. Il exhibait de la confiance en lui, même si elle était fausse, parce que Seamus en avait besoin. "Vas manger du chocolat, tu te sentiras mieux."

Seamus hocha à son tour la tête et alla jusqu'à aux douceurs soignantes en chancelant.

"\emph{Expecto Patronum~!}" s'écria la voix d'un jeune garçon.

Il y eut des hoquets audibles, même venant des Aurors.

Harry pivota pour regarder-

Un brillant oiseau d'argent se tenait entre Anthony Goldstein et la cage. L'oiseau leva la tête et laissa échapper un cri, et le cri était lui aussi d'argent, et il avait la force et la beauté du métal.

Et quelque chose au fond de l'esprit de Harry dit~: \emph{si c'est un faucon pèlerin, je vais l'étrangler dans son sommeil.}

\emph{Tais-toi}, dit Harry à la pensée, \emph{tu veux qu'on devienne un Seigneur des Ténèbres~?}

\emph{À quoi bon~? C'est comme ça que tu vas finir de toute façon…}

Ce… ce n'était pas quelque chose que Harry se serait dit en temps normal.

\emph{C'est un effet placebo}, se répéta-t-il. \emph{Le Détraqueur ne peut pas vraiment m'atteindre à travers trois Patronus corporels, je m'imagine juste ce que c'est de le ressentir. Quand je ferai vraiment face au Détraqueur, j'aurai une sensation totalement différente et alors je saurai que je me comportais comme un idiot.}

Un léger frisson descendit le long de sa colonne vertébrale car l'idée lui était venu que oui, ce \emph{serait} complètement différent, et pas dans le bon sens du terme.

Le phénix d'argent éclatant revint à la vie par la baguette du directeur, et l'oiseau moins imposant disparut~; et Anthony Goldstein commença à retourner d'où il était venu.

Le directeur venait avec Anthony au lieu d'appeler le prochain nom, et le Patronus attendait derrière, gardant le Détraqueur.

Harry jeta un coup d'œil vers l'emplacement où Hermione se tenait, juste derrière la panthère lumineuse. Son tour aurait été le suivant, mais apparemment, il allait falloir attendre.

Elle avait l'air stressée.

Plus tôt, elle avait poliment demandé à Harry de bien vouloir arrêter d'essayer de la détendre.

Dumbledore avait un léger sourire en raccompagnant Anthony jusqu'aux autres~; léger seulement parce qu'il avait l'air très, très fatigué.

"Incroyable," dit Dumbledore d'une voix qui semblait bien faible comparée à son coffre habituel. "Un Patronus corporel, sa première année. Et un nombre incroyable de succès chez les autres jeunes élèves. Quirinus, je dois reconnaître que tu avais raison."

Le professeur Quirrell inclina la tête. "Simple à deviner, il me semble. Un Détraqueur attaque par la peur, et les enfants ont moins peur."

"\emph{Moins} peur~?" dit l'Auror Goryanof depuis son siège.

"C'est ce que j'ai dit moi aussi," dit Dumbledore. "Et le professeur Quirrell a fait remarquer que les adultes ont plus de courage, pas moins de peur~; je dois avouer que cela ne m'avait jamais traversé l'esprit."

"Ce n'était pas ma formulation \emph{exacte}," dit le professeur Quirrell d'une voix sèche, "mais cela suffira. Et pour le reste de notre accord, directeur~?"

"Comme vous le voudrez," dit Dumbledore avec réticence. "J'admets que je ne m'attendais pas à perdre ce pari, Quirinus, mais tu as prouvé ta sagesse."

Tous les élèves les regardaient, perplexes~; mis à part Hermione, qui regardait vers la cage et vers les grandes robes en déliquescence~; et à part Harry, qui regardait tout le monde puisqu'il s'imaginait être devenu paranoïaque.

Le professeur Quirrell dit d'un ton qui n'invitait à aucune réponse, "Je suis autorisé à enseigner le sortilège de la Mort à ceux qui voudront l'apprendre. Ce qui augmentera de façon conséquente leur protection contre les mages noirs et autres pestes, et il serait naïf de croire qu'ils n'apprendrait de toute façon aucune autre forme de magie mortelle." Le professeur Quirrell s'interrompit et ses yeux se plissèrent. "Directeur, je remarque avec respect que vous ne semblez pas aller bien. Je suggère que vous partiez pour laisser le reste du travail au professeur Flitwick."

Dumbledore secoua la tête. "Nous en avons presque fini, Quirinus. Je tiendrai le coup."

Hermione s'approcha d'Antony. "Capitaine Goldstein," dit-elle, et sa voix ne tremblait qu'un peu, "pourriez-vous me donner des conseils~?"

"N'aies pas peur," dit Anthony d'une voix ferme. "Ne penses à aucune des choses auxquelles il essaiera de te faire penser. Tu ne fais pas que tenir ta baguette devant toi comme un bouclier contre la peur, tu la \emph{brandis} pour faire partir la peur, c'est comme ça que tu transformes une pensée heureuse en quelque chose de solide…" Anthony haussa les épaules d'un air impuissant. "Je veux dire, j'avais \emph{entendu} tout ça avant, mais…"

D'autres élèves commençaient à se rassembler autour d'Anthony, armés de leurs questions.

"Mademoiselle Granger~?" dit le directeur. Sa voix avait été aimable, ou peut-être seulement faible.

Hermione redressa les épaules et le suivit.

"Qu'as-tu vu sous la cape~?" dit Harry à Anthony.

Anthony regarda Harry avec surprise et répondit~: "Un grand homme mort, je veux dire, comme mort et d'une couleur morte… ça faisait mal de le regarder et je savais que c'était comme ça que le Détraqueur essayait de m'atteindre."

Harry jeta un regard vers l'endroit où Hermione faisait face à la cage et à la cape.

Elle mit sa baguette en position, prête à exécuter les premiers gestes.

Le phénix du directeur disparut dans un éclair.

Et Hermione laissa échapper un petit cri pathétique, elle flancha -

- fit un pas en arrière, Harry pouvait voir sa baguette bouger, puis elle la brandit et dit "Expecto Patronum~!"

Rien ne se passa.

Elle fit demi-tour et courut.

"\emph{Expecto Patronum~!}" dit la voix plus grave du directeur, et le phénix d'argent revint à la vie dans un autre éclat de lumière.

La jeune fille chancela mais elle continua de courir. D'étranges sons s'échappaient de sa gorge.

"\emph{Hermione}" hurla Susan, tout comme Hannah, Daphné et Ernie, et ils commencèrent tous à courir vers elle tandis que Harry, qui avait toujours un temps d'avance, pivotait sur ses talons et courait vers la table sur laquelle se trouvait le chocolat.

Même après qu'il lui ait fourré le chocolat dans la bouche, qu'elle ait mâché et avalé, elle continua de respirer à grandes goulées et de pleurer, et ses yeux ne semblaient plus mettre au point.

\emph{Elle ne peut pas avoir été détraquée de façon permanente} songea Harry avec désespoir à l'attention de la confusion qui régnait à l'intérieur de lui, de l'horrible peur et de la furie mortelle qui commençaient à s'enrouler l'une autour de l'autre, \emph{ce n'est pas possible, elle n'a pas été exposée pendant plus de dix secondes, certainement pas quarante…}

Mais elle avait pu être \emph{temporairement} détraquée car, Harry s'en rendit compte à l'instant, il n'y avait aucune règle qui interdisait que l'on soit \emph{temporairement} blessé par un Détraqueur en seulement dix secondes, si l'on était assez sensible.

Ses yeux semblèrent alors mettre au point et darder autour d'elle, puis il s'arrêtèrent sur Harry.

"Harry," hoqueta-t-elle, et les autres élèves étaient silencieux. "Harry, non. \emph{Non~!}"

Harry eut soudain peur de demander à quoi elle faisait référence, était-\emph{il} dans ses pires souvenirs, ou dans un cauchemar qu'elle revivait éveillée~?

"\emph{Ne t'en approches pas~!}" dit Hermione. Sa main se tendit, elle l'attrapa par le revers de ses robes. "Tu ne dois pas t'en approcher Harry~! \emph{Il m'a parlé, il te connaît, il sait que tu es ici~!}"

"Qu'est-ce qui est -" dit Harry, puis il jura intérieurement d'avoir posé cette question.

"\emph{Le Détraqueur~!}" dit Hermione. Sa voix devint un cri perçant. "\emph{Le professeur Quirrell veut que le Détraqueur te mange~!}"

Le professeur Quirrell s'avança de quelques pas avec un empressement soudain~; mais il ne vint ensuite pas plus près (Harry était là, après tout). "Mademoiselle Granger," dit-il, la voix grave, "je pense que vous devriez prendre plus de chocolat."

"\emph{Professeur Flitwick, ne laissez pas Harry essayer, renvoyez-le~!}"

Le directeur était alors arrivé, et lui et le professeur Flitwick échangeaient des regards inquiets.

"Je n'ai pas entendu le Détraqueur parler," dit le directeur. "Mais quand-même…"

"Posez simplement la question," dit le professeur Quirrell d'un ton un peu las.

"Le Détraqueur a-t-il dit \emph{comment} il atteindrait Harry~?" dit le directeur.

"Ses parties les plus goûteuses en premier," dit Hermione, "il man- il mangerait…"

Hermione cligna des yeux. De la folie sembla revenir dans ses yeux.

Puis elle se mit à pleurer.

"Vous avez été trop courageuse, Mlle Granger," dit le directeur. Sa voix était aimable et clairement audible. "Bien plus brave que ce à quoi je m'attendais. Vous auriez dû vous détourner et courir, ne pas endurer cela et ne pas essayer d'achever votre sortilège. Lorsque vous serez plus âgée et plus forte, Mlle Granger, je sais que vous essaierez de nouveau et que vous réussirez."

"Je suis désolée," dit Hermione entre deux hoquets, "je suis désolée, je suis désolée, je suis désolée, … je suis désolée Harry, je ne peux pas te dire ce que j'ai vu, je n'ai pas regardé, je n'ai pas osé, je savais que c'était trop horrible pour être jamais vu…"

Ça aurait dû être Harry, mais il avait hésité parce que ses mains étaient pleines de chocolat~; et Ernie et Susan furent là, aidant Hermione à se relever de là où elle s'était écroulée, la menant vers la nourriture disposée sur la table.

Cinq barres de chocolat plus tard, Hermione semblait de nouveau aller mieux, et elle alla voir le professeur Quirrell et lui présenta ses excuses, mais elle regardait toujours Harry, à chaque fois qu'il regardait dans sa direction. Il avait fait un pas vers elle, un fois seulement, et s'était arrêté quand elle avait fait un pas en arrière. Ses yeux lui avait silencieusement demandé pardon et lui avaient demandé de bien vouloir la laisser en paix.

\later

Neville Londubat avait vu quelque chose de mort à moitié dissout qui courait en suintant et dont le visage ressemblait à une éponge écrasée.

C'était la pire chose que quiconque ai dit avoir vu jusqu'à présent. Neville avait pu créer une petite étincelle de lumière mais il avait fait preuve d'une grande présence d'esprit et d'intelligence en se détournant et en fuyant au lieu d'essayer de lancer son Patronus.

(Le directeur n'avait rien dit aux autres élèves, il n'avait dit à personne d'être moins courageux~; mais le professeur Quirrell avait calmement remarqué que c'était quand on commettait une erreur \emph{après} avoir été prévenu que l'ignorance devenait stupidité).

"Professeur Quirrell~?" dit Harry d'une voix basse, s'étant autant approché du professeur Quirrell qu'il l'osait. "Que voyez-\emph{vous} quand vous -"

"Ne me posez pas cette question." La voix était très neutre.

Harry hocha respectueusement la tête. "Si je puis me permettre, quelle était votre tournure de phrase \emph{initiale} lorsque vous avez parlé avec le directeur~?"

Sèchement. "Nos pires souvenirs ne peuvent qu'empirer avec l'âge."

"Ah," dit Harry. "Logique."

Quelque chose d'étrange brilla dans les yeux du professeur Quirrell, puis, regardant Harry~: "Espérons," dit le professeur Quirrell, "que vous réussirez cet essai, M. Potter. Car si vous le faites, le directeur vous enseignera peut-être sa technique consistant à utiliser un Patronus pour envoyer des messages impossibles à falsifier et à intercepter, et son importance en situation militaire ne saurait être trop soulignée. Ce serait un formidable avantage pour la Légion du Chaos, et un jour, je le soupçonne, pour tout ce pays. Mais si vous ne réussissez \emph{pas}, M. Potter… eh bien, \emph{je} comprendrai."

\later

Morag MacDougal avait dit "Ouille" d'une voix vacillante et le directeur avait immédiatement relancé son Patronus.

Parvati Patil avait créé un Patronus corporel en forme de tigre, plus grand que le phénix de Dumbledore mais pas tout à fait aussi éclatant. Il y avait eu une grande salve d'applaudissement venue du public mais pas autant de surprise que lorsque Anthony y était parvenu.

Puis ce fut le tour de Harry.

Le directeur appela son nom, et Harry prit peur.

Il savait, il savait qu'il allait échouer, et il savait que cela allait faire mal.

Mais il fallait quand même qu'il essaie~; parce que parfois, en présence d'un Détraqueur, un sorcier créait un Patronus corporel complet là où il n'avait jamais su produire une seule étincelle de lumière, et personne ne comprenait pourquoi.

Et parce que si Harry ne \emph{pouvait pas} se défendre contre les Détraqueurs, il lui faudrait apprendre à détecter leur approche, à reconnaître la sensation qu'ils provoqueraient dans son esprit et à courir avant qu'il ne soit trop tard.

\emph{Quel est mon pire souvenir…~?}

Harry s'était attendu à ce que le directeur le regarde avec inquiétude, ou avec espoir, ou qu'il lui donne un conseil profondément sage~; mais au lieu de cela Albus Dumbledore se contentait de le regarder d'un air calme et tranquille.

\emph{Il pense que je vais échouer mais il ne va pas me saper mes forces en me le disant}, pensa Harry, \emph{s'il avait quelque chose de vraiment encourageant à me dire il me l'aurait dit…}

La cage approcha. Elle était déjà ternie mais pas rouillée jusqu'à la moelle, pas encore…

La cape approcha. Elle était défaite et percée de trous jamais rapiécés~; selon l'Auror Goryanof, elle avait été neuve ce matin.

"Professeur~?" dit Harry. "Que voyez-vous~?"

La voix du directeur était calme elle aussi. "Les Détraqueurs sont faits de peur, et à mesure que ta peur du Détraqueur diminue, l'horreur de sa forme fait de même. Je vois un homme grand, mince et nu. Il ne pourrit pas. C'est juste qu'il est légèrement douloureux de le regarder. C'est tout. Que vois-tu, Harry~?"

… Harry ne pouvait pas voir sous la cape.

Pas vraiment, c'était que son esprit \emph{refusait} de voir ce qui se trouvait sous la cape.

Non, son esprit essayait de voir \emph{autre chose} sous la cape, Harry pouvait sentir ses yeux essayer de forcer une erreur. Mais il s'était entraîné du mieux qu'il pouvait à remarquer ce petit sentiment de confusion, à automatiquement s'empêcher d'inventer des explications~; et à chaque fois que son esprit essayait de commencer à inventer un mensonge au sujet de ce qui se trouvait sous la cape, ce réflexe était assez rapide pour le bloquer.

Harry regarda sous la cape et vit…

Une question posée. Il refusait de laisser son esprit voir quelque chose de faux, et il ne voyait donc rien, comme si la partie de son cortex visuel qui recevait ce signal avait juste cessé d'exister. Il y avait un angle mort sous la cape. Harry ne pouvait pas savoir ce qui s'y trouvait.

Sauf que c'était bien pire que n'importe quelle momie pourrissante.

L'horreur impossible à voir était toute proche maintenant, mais l'éclatant oiseau lunaire et le blanc phénix se tenaient encore entre eux.

Harry voulait s'enfuir comme les autres élèves l'avaient fait. La moitié de ceux qui n'avaient pas réussi leur Patronus ne s'était tout simplement pas présentée aujourd'hui. De ceux qui restaient, la moitié avait fuit avant que le directeur n'ait dissipé son Patronus, et personne n'avait rien dit. Il y avait eu un petit rire quand Terry s'était détourné et était rentré avant même que ce soit son tour~; et Susan et Hannah, qui étaient déjà passées, avaient crié sur tous les autres et leur avaient ordonnés de se taire.

Mais Harry était le Survivant et il perdrait beaucoup de respect si on le voyait abandonner sans même essayer.

L'orgueil et les rôles semblaient s'amoindrir et tomber face à cette chose inconnue qui se trouvait sous la cape.

\emph{Pourquoi suis-je encore ici~?}

Ce ne fut pas la honte à l'idée que les autres le croient lâche qui maintint ses pieds où ils étaient.

Ce ne fut pas le désir de réparer sa réputation qui lui fit lever sa baguette.

Ce ne fut pas l'envie de réussir un Patronus qui plaça ses doigts dans la position initiale.

C'était autre chose, il \emph{fallait} que quelqu'un s'oppose à ce qui se trouvait sous la cape, c'étaient là les véritables ténèbres et Harry devait découvrir s'il abritait en lui-même le pouvoir de les repousser.

Il avait prévu d'essayer une dernière fois de penser à sa journée d'emplettes littéraires avec son père, mais au lieu de cela, au dernier moment, face au Détraqueur, un autre souvenir occupa son esprit, quelque chose qu'il n'avait encore jamais essayé~; une pensée qui n'était pas chaleureuse et plaisante dans le sens ordinaire mais qui semblait pourtant plus juste.

Et Harry se souvint des étoiles, il se souvint d'elles brûlant d'une terrible lueur, inébranlables dans la Nuit Silencieuse~; il laissa cette image croître en lui, croître comme une barrière Occlumantique d'un bout à l'autre de son esprit, il devint à nouveau la conscience corporelle du vide.

Le clair phénix d'argent disparut.

Et le Détraqueur s'écrasa dans son esprit, tel le poing de Dieu lui-même.

\textbf{PEUR / FROID / TÉNÈBRES}

Pendant un instant, les deux forces se heurtèrent de front et le paisible souvenir constellé d'étoiles tint bon face à la peur, alors que les doigts de Harry commençaient les mouvements de baguette pratiqués jusqu'à être devenus automatiques. Ils n'étaient pas chaleureux et plaisants, ces points de lumières éclatants sur fond de noir absolu~; mais c'était une image que le Détraqueur ne pouvait pas facilement percer. Car les étoiles silencieuses et brûlantes étaient vastes, elles ne connaissaient pas la peur, et briller au milieu du froid et des ténèbres constituait leur état naturel.

Mais il y eut un défaut, une fêlure, une ligne de fracture dans l'objet immuable qui tentait de résister à l'irrésistible force. Harry ressentit une pointe de colère contre le Détraqueur qui osait essayer de se nourrir de lui, et ce fut comme de glisser sur de la glace mouillée. L'esprit de Harry commença à s'écarter vers l'amertume, vers la furie noire, vers la haine mortelle -

La main de Harry s'était levée dans le mouvement de brandissement final.

Ça n'allait pas.

"Expecto Patronum," dit sa voix, les mots creux et vides de sens.

Et Harry tomba dans son côté obscur, tomba dans son côté obscur, plus loin et plus vite et plus profond que jamais, plus bas plus bas plus bas et la chute accéléra alors que le Détraqueur s'accrochait aux parties exposées et vulnérables et qu'il s'en nourrissait, mangeant la lumière. Un réflexe faiblissant fouilla à la recherche de chaleur, mais même lorsqu'une image de Hermione lui venait, ou une image de Maman et Papa, le Détraqueur la tordait, lui montrait Hermione allongée par terre, morte, les corps de son père et de sa mère, puis même cela fut absorbé.

Du vide émergea le souvenir, le pire de tous, quelque chose d'oublié il y a si longtemps que la structure neuronale n'aurait plus dû exister.
\begin{em}
"Lily, prends Harry et pars~! C'est lui~!" cria la voix d'un homme. "Cours~! Allez~! Je le retiendrai~!"

Et Harry ne pouvait s'empêcher de penser, depuis les profondeurs vides de son côté obscur, à quel point l'excès de confiance en lui de James avait été ridicule. Retenir Lord Voldemort~? Avec quoi~?

Puis l'autre voix parla, haut percée comme le sifflement d'une théière, et ce fut comme de la glace séchée qu'on aurait répandu sur chaque nerf de Harry, comme un tison de métal refroidi jusqu'à avoir atteint la température de l'hélium liquide qu'on aurait fait passer à la surface de tout son corps. Et la voix dit~:

"Avadakedavra."
\end{em}

(La baguette s'envola des doigts flasques du garçon tandis que son corps commençait à convulser et à tomber, les yeux du directeur maintenant alarmé s'écarquillant tandis qu'il commençait à lancer son propre Patronus).
\begin{em}

"Pas Harry, pas Harry, s'il vous plaît pas Harry~!" hurla la voix de la femme.

Le peu qui restait de Harry écoutait cela alors que toute lumière avait été extirpée de lui, dans le vide mort de son corps, et il se demanda si elle pensait que Lord Voldemort s'arrêterait parce qu'elle avait demandé poliment.

"Écarte-toi, femme~!" dit la voix stridente d'un froid brûlant. "Ce n'est pas pour toi que je suis venu, mais pour le garçon."

"Pas Harry~! S'il vous plaît… ayez pitié… ayez pitié…"

Harry songea que Lily Potter ne semblait pas comprendre quel genre de personne devenait Seigneur des Ténèbres en premier lieu~; et si c'était là la meilleure stratégie qu'elle pouvait concevoir pour sauver la vie de son fils, alors c'était aussi son échec final en tant que mère.

"Je te donne la rare chance de t'échapper," dit la voix stridente. "Mais je ne ferai pas l'effort de te maîtriser, et ta mort, ici, ne sauvera pas ton enfant. Écarte-toi, femme imbécile, si tu as le moindre bon sens~!"

"Pas Harry, s'il vous plaît, non, prenez moi, tuez moi à la place~!"

La chose vide qu'était devenu Harry se demanda si Lily Potter imaginait sérieusement que Lord Voldemort dirait oui, qu'il la tuerait et qu'il laisserait son fils sain et sauf.

"Très bien," dit la voix de la mort d'un ton à présent froidement amusé. "J'accepte le marché. Tu mourras, et l'enfant vivra. Maintenant abaisses ta baguette que je puisse te tuer."

Il y eut un silence hideux.

Lord Voldemort commença à rire, horrible rire méprisant.

Puis, enfin, la voix de Lily cria avec une haine désespérée~: "Avada ke -"

La voix mortelle finit la première, le sort rapide et précis.

"Avadakedavra."

Un éclat de vert aveuglant marqua la fin de Lily Potter.

Et le garçon dans son berceau les vit, ces yeux, ces deux yeux pourpres, qui semblaient briller d'un rouge vif, flamber comme deux soleils miniatures, emplissant le champ de vision de Harry alors qu'ils se braquaient sur ses yeux à lui -

\end{em}

\later

Les autres enfants virent Harry tomber, ils l'entendirent crier, un fin cri haut perché qui sembla percer leurs oreilles, tel un couteau.

Il y eut l'éclat d'argent et la voix du directeur qui mugissait "\emph{Expecto Patronum~!}", et le phénix embrasé revint à la vie.

Mais l'horrible cri de Harry continua encore et encore, alors que le directeur prenait le garçon dans ses bras et qu'il le portait loin du Détraqueur, alors que Neville Londubat et le professeur Flitwick couraient tous deux vers le chocolat au même moment et -

Hermione l'avait su, elle l'avait su lorsqu'elle l'avait vu, elle avait su que ses cauchemars avaient été réels, que ça devenait vrai, d'une façon ou d'une autre quelque chose était en train de devenir réalité.

"Donnez-lui du chocolat~!" exigea la voix du professeur Quirrell, bien inutilement car la forme menue du professeur Flitwick fonçait déjà vers l'endroit vers lequel le directeur accourait, non loin des élèves.

Hermione avançait elle aussi, même si elle ne savait pas ce qu'elle était censée faire -

"\emph{Lancez des Patronus~!}" hurla le directeur alors qu'il plaçait Harry derrière les Aurors. "\emph{Tous ceux qui le peuvent~! Mettez-les entre Harry et le Détraqueur~! Il se nourrit encore de lui~!}"

Il y eut un instant d'horreur glacée.

"\emph{Expecto Patronum}~!" crièrent le professeur Flitwick et l'Auror Goryanof, puis Anthony Goldstein, mais il échoua la première fois, puis Parvati Patil, qui réussit, puis Anthony Goldstein essaya à nouveau et son oiseau d'argent étendit ses ailes et hurla en direction du Détraqueur, et Dean Thomas rugit les mots comme s'ils avaient été écrits de lettres de feu et sa baguette donna naissance à un immense ours blancs~; il y avait huit Patronus qui brillaient le long d'une ligne séparant Harry du Détraqueur, et Harry continuait de crier et de crier alors que le directeur l'allongeait sur l'herbe sèche.

Hermione ne pouvait lancer de Patronus, alors elle courut vers l'endroit où Harry gisait. Dans son esprit, quelque chose essayait de deviner combien de temps s'était écoulé. Vingt secondes~? Plus~?

Il y avait une effroyable expression d'agonie et de perplexité sur le visage d'Albus Dumbledore. Sa longue baguette noire était dans sa main mais il ne prononçait aucun sort, il regardait seulement seulement le corps agité de convulsions de Harry avec horreur -

Hermione ne savait pas quoi faire, elle ne savait pas quoi faire, elle ne comprenait pas ce qui se passait, et le sorcier le plus puissant du monde semblait tout autant perdu qu'elle.

"\emph{Utilisez votre phénix~!}" mugit le professeur Quirrell. "\emph{Éloignez-le autant que possible de ce Détraqueur~!"}

Sans un mot le directeur prit Harry dans ses bras et disparut dans un éclat de feu au côté de Fumsec, qui venait subitement d'apparaître~; et le Patronus du directeur s'effaça instantanément, laissant un espace vide là où il avait gardé le Détraqueur.

Horreur et confusion et bavardages soudains.

"M. Potter devrait récupérer," dit le professeur Quirrell, élevant sa voix, mais son ton était à nouveau calme, "je pense que c'était juste un peu plus de vingt secondes."

Puis le phénix blanc embrasé apparut de nouveau, comme s'il était arrivé depuis les airs, et la créature de lumière lunaire alla vers Hermione Granger, et il cria de la voix d'Albus Dumbledore~:

"\emph{Il se nourrit encore de lui, même ici~! Comment~? Si tu le sais, Hermione Granger, tu dois me le dire~! Dis-le moi~!"}

L'Auror le plus âgé pivota pour la regarder, et de nombreux élèves firent de même. Le professeur Flitwick ne pivota pas, il tenait à présent sa baguette braquée sur le professeur Quirrell, qui avait mis ses mains vides en évidence.

D'infinies secondes s'écoulèrent.

Elle ne pouvait pas se souvenir, elle n'arrivait pas à se souvenir du cauchemar avec assez de précision, elle ne pouvait pas se souvenir de la raison pour laquelle elle avait pensé que ce serait possible, pourquoi elle avait eu peur -

Hermione se rendit compte de ce qu'elle devait faire, et c'était la décision la plus difficile de sa vie.

Et si ce qui était arrivé à Harry lui arrivait à elle aussi~?

Tous ses membres étaient aussi froids que la mort, son champ de vision s'assombrit, la peur écrasa tout~; elle avait vu Harry mourant, Maman et Papa mourant, tous ses amis mourants, tout le monde mourant, pour qu'à la fin, quand elle mourrait, ce soit seule. C'était le cauchemar secret dont elle n'avait jamais parlé à personne, le cauchemar qui avait permis au Détraqueur de la dominer, la chose pire entre toutes~: mourir seule.

Elle ne voulait pas y retourner, elle, elle ne voulait pas, elle ne voulait pas y rester pour toujours -

\emph{Tu as assez de courage pour Gryffondor}, dit la voix calme du Choixpeau, venue de ses souvenirs, \emph{mais tu feras le bien quelle que soit la maison que je te donne. Tu apprendras, tu soutiendras tes amis, quelle que soit la maison que tu choisis. Alors n'aie pas peur, Hermione Granger, choisis juste l'endroit qui te correspond…}

Il n'y avait pas assez de temps pour choisir, Harry était mourant.

"Je ne peux pas m'en souvenir pour l'instant," dit Hermione d'une voix qui se brisait, "mais attendez juste, je vais retourner voir le Détraqueur…"

Elle commença à courir vers le Détraqueur.

"Mademoiselle Granger~!" couina le professeur Flitwick, mais il ne fit rien pour l'arrêter, il garda juste sa baguette pointée vers le professeur Quirrell.

"\emph{Tout le monde~!}" hurla l'Auror Komodo de la voix d'un commandant militaire. "\emph{Mettez vos Patronus hors de sa route~!}"

"\shout{Flitwick~!}" rugit le professeur Quirrell. "\shout{Appelez la baguette de potter~!}"

Et, alors que Hermione comprenait, le professeur Flitwick criait déjà "\emph{Accio~!}", et elle vit le bout de bois grandir en s'approchant depuis l'endroit où il était tombé, où il avait presque touché la cage du Détraqueur.

\later

Les yeux s'ouvrirent, morts et vacants.

"\emph{Harry~!}" s'étrangla une voix venue du monde sans couleurs. "\emph{Harry~! Parle moi~!}"

Le visage de Albus Dumbledore se pencha jusqu'à entrer dans le champ de vision qui avait été précédemment occupé par un lointain toit de marbre.

"Tu es énervant," dit la voix vide. "Tu devrais mourir."

%  LocalWords:  un Ke
