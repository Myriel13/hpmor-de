\partchapter{Mesures de précaution}{I}

\section{Le 13 mai 1992.}

\lettrine{S}{ous} la lumière de la lampe à huile qu'il brandissait, le visage d'Argus Rusard semblait déformé par les ombres qui dansaient à sa surface. Les portes de Poudlard s'estompèrent derrière eux et le sol sombre se rapprocha. La piste qu'ils suivaient à présent était boueuse, indistincte.

Les arbres dont les branches avaient été dénudées par l'hiver n'étaient pas encore entièrement habillée du printemps~: elles s'étiraient vers le ciel comme des doigts fins, des squelettes visibles au milieu du mince feuillage. La lune était claire mais les nuages qui filaient devant elle les plongeaient souvent dans les ténèbres, exception faite des flammes de la lampe de Rusard.

Drago tenait fermement sa baguette.

<<~Où nous emmenez-vous~?~>> dit Tracey Davis. Elle avait été surprise avec Drago par Rusard alors qu'ils se rendaient à un meeting des Serpentard Scintillants programmé pour après le couvre-feu et, comme lui, avait écopé d'une retenue.

<<~Contentez-vous de me suivre,~>> dit Argus Rusard.

Drago se sentait passablement agacé par ce contretemps. Les Serpentard Scintillants auraient dû être une institution reconnue par l'école. Il n'y avait aucune raison d'interdire à une conspiration secrète de se réunir après le couvre-feu si c'était pour le plus grand bien de Poudlard. Encore un coup comme celui-ci et il irait parler à Daphné Greengrass, qui irait parler à son père, et Rusard apprendrait alors comme il était sage de regarder ailleurs lorsqu'il s'agissait des Malfoy.

Lorsque Rusard parla de nouveau, les lumières du château étaient plus lointaines, plus réduites. <<~Hé, je parie que vous y réfléchirez à deux fois avant d'enfreindre une des règles de l'école~!~>> Il se détourna de la lampe afin de pouvoir lorgner sur les quatre élèves qui le suivaient. <<~Oh oui… Si vous voulez mon avis, le dur labeur et la douleur sont les meilleurs professeurs… C'est vraiment dommage qu'ils aient laissé disparaître les vieilles méthodes punitives… vous pendre par les poignets au plafond pendant quelques jours, j'ai encore les chaînes dans mon bureau, j'les garde bien huilées au cas où on en aurait besoin…

--- Hé~! dit Tracey avec un soupçon d'indignation. Je suis trop jeune pour entendre parler de… de ce genre de… vous savez quoi~! Surtout si les chaînes sont bien huilées~!~>>

Drago ne lui prêta pas attention. Rusard ne jouait tout simplement pas dans la même cour qu'Amycus Carrow.

Derrière eux, l'une des deux Serpentard les plus âgés ricana, mais elle ne dit rien. À côté d'elle se trouvait l'autre, un grand garçon aux traits slaves et qui s'exprimait encore avec un accent. Ils avaient été surpris pour une autre raison, en rapport le genre de chose dont Tracey n'arrêtait pas de parler, et semblaient être en troisième ou quatrième année. <<~Bah, dit le garçon plus grand. À Durmstrang ils vous pendent par les orteils. Par un orteil si vous êtes insolent. Poudlard a toujours été trop clémente.~>>

Argus Rusard demeura silencieux pendant environ trente seconde, comme s'il essayait de trouver une bonne répartie, puis il eut un gloussement.

<<~On verra ce que vous direz… quand vous apprendrez ce que vous allez faire ce soir~! Ha~!

--- J'ai \emph{dit} que j'étais trop jeune pour ce genre de choses~! dit Tracey Davis. Je dois attendre d'être plus âgée~!~>>

Devant eux se trouvait une chaumière aux fenêtres éclairées et dont les proportions semblaient comme erronées.

Rusard siffla, un son aigu et sec, et un chien commença à aboyer.

De la chaumière s'avança une silhouette autour de laquelle les arbres semblaient petits. Elle était suivie d'un chien qui, par comparaison, ressemblait à un chiot, jusqu'à ce que vous l'observiez séparément de la silhouette et compreniez qu'il était immense, proche d'un loup.

Les yeux de Drago se plissèrent avant qu'il ne puisse s'en empêcher. En tant que Serpentard Scintillant, il n'était pas censé avoir de préjugé envers un être sentient, en particulier pas quand d'autres risquaient de le voir.

<<~Qu'est-ce que c'est que ça~?~>> dit la silhouette de la voix puissante et bourrue des demi-géants. Son parapluie projetait une lueur blanche, plus forte que la lampe de Rusard. Dans son autre main se trouvait une arbalète~; un carquois de flèches était lacé à son bras.

<<~Des élèves en retenue, dit Rusard avec force. Ils sont ici pour vous aider à chercher dans la forêt ce qui… ce qui les a mangées.

--- La \emph{Forêt}~? s'étrangla Tracey. On ne peut pas y aller la nuit~!

--- Absolument~>>, dit Rusard en se détournant de Hagrid pour les regarder avec colère. <<~C'est dans la Forêt que vous allez, et je serais bien détrompé si vous ressortiez tous en un seul morceau.

--- Mais… dit Tracey. Il y a des loups-garous, on m'a dit, \emph{et} des vampires, et tout le monde sait ce qui se passe quand un loup-garou, une fille et un vampire se retrouvent au même endroit~!~>>

L'immense demi-géant fronçait les sourcils. <<~Argus, j'pensais toi et p'têt queq' septième année. Y'a pas grand sens à les ramener pour donner un coup d'main si j'dois les surveiller tou'c'temps.~>>

Une satisfaction cruelle illumina le visage d'Argus. <<~Sont à l'affût, hein~? Vous auriez dû y penser, à ces loups-garous, avant de vous attirer des ennuis. Envoie-les seuls. Faudrait pas être trop bon avec eux, Hagrid. Après tout, ils sont ici pour être punis.~>>

Le demi-géant laissa échapper un immense soupir (on aurait dit qu'un homme normal venait de voir tout l'air expulsé de ses poumons sous le coup d'un sortilège matraqueur).

<<~T'as fait ton boulot. Je m'occupe de la suite.

--- Je reviendrai à l'aube, dit Rusard, pour ce qui reste d'eux~>>, ajouta-t-il avec méchanceté avant de se détourner et de repartir vers le château dans la lumière de sa lampe, branlante au milieu des ténèbres.

<<~Très bien, dit Hagrid, maintenant, écoutez-moi bien, pac'que c'est dangereux c'qu'on va faire ce soir et j'y veux pas voir quelqu'un prendre des risques. Suivez-moi par ici.~>>

Il les mena à la lisière de la Forêt interdite. En brandissant sa lampe il désigna un sentier étroit et sinueux qui disparaissait entre d'épais arbres noirs. Une légère brise souffla par-dessus la tête de Drago lorsqu'il regarda dans la Forêt.

<<~Y'a un truc là-dedans qu'y a mangé des licornes~>>, dit l'immense homme.

Drago hocha la tête~; il se souvenait plus ou moins avoir entendu quelque chose à ce sujet deux semaines auparavant, vers la fin d'avril.

<<~Est-ce que vous nous avez fait venir pour remonter une piste de sang argenté jusqu'à une licorne blessée~? dit Tracey avec excitation.

--- Non~>>, dit Drago, quoi qu'il parvint à interrompre son ricanement réflexe. <<~Rusard nous a informé de notre retenue aujourd'hui à midi. M. Hagrid n'attendrait pas aussi longtemps pour aller chercher une licorne blessée, et si nous en cherchions une, nous le ferions en pleine journée, dans la lumière. Donc,~>> Drago leva un doigt comme il avait vu l'inspecteur León le faire dans des pièces, <<~j'en déduis que nous cherchons quelque chose qui ne sort que la nuit.

--- Yep, dit le demi-géant d'un ton pensif. T'es pas comme j'pensais, Drago Malfoy. Pas du tout comme j'pensais. Et toi tu s'rais Tracey Davis alors. J'ai entendu parler d'toi. Une des copines de la pauvre Mlle Granger.~>> Rubeus Hagrid observa les Serpentard plus âgés sous l'éclairage de son parapluie lumineux. <<~Et tu s'rais qui, toi~? J'me souviens pas t'avoir beaucoup vu, garçon.

--- Cornelia Walt, dit la sorcière, et c'est Yuri Yuliy~>>, dit-elle en montrant le garçon à l'air slave qui avait parlé de Durmstrang. <<~Sa famille nous rend visite des terres Ukrainiennes, alors il est à Poudlard pour cette année seulement.~>> Le garçon plus âgé hocha la tête, un air légèrement dédaigneux sur le visage.

<<~C'est Croc~>>, dit Hagrid en montrant le chien.

Ils partirent tous les cinq dans les bois.

<<~Qu'est-ce qui pourrait tuer des licornes~?~>> dit Drago après quelques minutes de marche. Il avait quelques connaissances sur les créatures des ténèbres mais il n'arrivait pas à se rappeler d'une connue pour chasser la licorne. <<~Est-ce que quelqu'un saurait quel genre de créature fait ça~?

--- Des loups-garous~! dit Tracey.

--- Mlle Davis~?~>> dit Drago, et lorsqu'elle le regarda, il dirigea silencieusement un doigt vers la lune. Elle était gibbeuse croissante mais pas encore pleine.

<<~Ah, ouais, dit Tracey.

--- Pas de garous dans la forêt, dit Hagrid. C'est juste des sorciers en général, savez. Pourrait pas être des loups non plus, pas assez rapides pour avoir une licorne. C'est des créature puissantes, les licornes, j'en avais jamais vu une s'faire faire mal avant.~>>

Drago écouta cela et songea au puzzle presque malgré lui.

<<~Alors qu'\emph{est-ce} qui est assez rapide pour attraper une licorne~?

--- Ça s'rait pas une question d'vitesse, dit Hagrid en donnant à Drago un regard indéchiffrable. Les créatures, elles chassent de mille façons. Poison, ténèbres, pièges. Des diablotins qu'on peut ni voir ni entendre ni se rappeler alors même qu'y vous mangent le visage. Toujours queq'chose de nouveau à apprendre.~>>

Un nuage passa devant la lune et plongea la forêt dans une ombre éclairée seulement par le parapluie de Hagrid.

<<~Moi-même, continua Hagrid, j'pense qu'on pourrait bien avoir une hydre Parisienne su' les bras. É' sont pas dangereuses pour un sorcier, y'a qu'à les maintenir assez longtemps et c'est pas possib' de perdre. J'veux dire littéralement y'a pas moyen de perdre tant qu'tu t'bas. L'problème c'est que contre une hydre Parisienne la plupart des créatures, elles abandonnent longtemps avant ça. Pacque tu vois, ça prend un moment d'couper toutes les têtes.

--- Bah, dit le garçon étranger. À Durmstrang nous apprenons à combattre hydre de Bucholz. Impossible d'imaginer plus dur à combattre~! Je veux dire littéralement, peux pas imaginer. Première année nous croient pas quand on leur dit que victoire est possible~! Instructeur doit donner second ordre, itérer jusqu'à compréhension.~>>

Ils marchèrent pendant presque une demi heure, de plus en plus profond dans la Forêt, jusqu'à ce que le chemin devienne presque impossible à suivre à cause de l'épaisseur des arbres.

Puis Drago les vit, les épaisses éclaboussures à la racine des arbres, scintillantes d'une couleur claire sous la lumière de la lune.

<<~Est-ce que c'est…

--- Du sang de licorne~>>, dit Hagrid. L'immense homme avait une voix triste.

Dans une clairière devant, visible à travers les branches entremêlées d'un grand chêne, ils virent la créature tombée, magnifiquement et tristement étalée au sol, avec autour d'elle une terre luisante d'un sang argent lunaire. Elle n'était pas blanche mais bleu pâle, du moins il le semblait, sous la lune, sous le ciel nocturne. Manifestement cassées, ses pattes élancées formaient d'étranges angles, et sa crinière, vert-noir mais au lustre de perle, se répandait sur les sombres feuilles. Sur son flanc, une petite forme blanche, comme un jet d'étoiles, un centre entouré de huit rayons droits. La moitié de son flanc avait été arraché, les bords en lambeaux semblaient révéler des marques de dents, les os et les organes internes étaient visibles.

Un étrange sensation d'étranglement monta dans la poitrine de Drago.

<<~C'telle~>>, dit Hagrid, son triste murmure aussi fort qu'une voix d'homme normal. <<~Juste où j'lai trouvé c'matin, morte comme une poignée de porte morte. C'est… c'était… la première licorne qu'jai jamais rencontré dans ces bois. J'l'appelais Alicorn, mais j'pense qu'elle s'en fiche bien, maintenant.

--- Vous avez appelé une licorne Alicorn, dit la fille plus âgée d'une voix sèche.

--- Mais elle n'a pas d'ailes, dit Tracey.

--- Une alicorne c't'une corne de licorne, dit Hagrid d'une voix maintenant plus forte. J'sais pas c'qui vous a tous pris d'croire que ça voulait dire une licorne à ailes, jamais entendu parler d'une chose pareille. C'est juste comme d'appeler un chien Croc,~>> dit-il en montrant son immense chien à l'apparence de loup qui atteignait à peine ses genoux. <<~Comment vous l'auriez appelé, hein~? Anna ou queq' chose comme ça~? J'lui ai donné un nom qu'aurait voulu dire queq'chose pour \emph{elle}. D'la courtoisie, voilà c'que c'est.~>>

Personne ne répondit et, après un moment, l'immense homme hocha brutalement le menton. <<~On commencera à chercher d'ici, l'dernier endroit où l'a frappé. On va faire deux groupes 'pi suivre la piste dans des directions différentes. Vous deux, Walt et Yuliy, z'irez par là, et prenez Croc. Y'a rien dans la Forêt qui pourrait vous faire du mal si vous êtes avec lui. Envoyez des étincelles vertes si vous voyez queq'chose d'intéressant, et des rouges si y'a des ennuis. Davis, Malfoy, avec moi.~>>

La Forêt était noire et silencieuse. Rubeus Hagrid avait diminué la lumière de son parapluie après qu'ils furent partis, si bien que Drago et Tracey devaient se guider par celle de la lune, non sans quelque trébuchement et chute occasionnels. Ils dépassèrent une souche d'arbre recouverte de mousses~; le son de l'eau parlait d'un ruisseau non loin. De temps à autres un rayon de lune qui traversait les branches éclairait une tache de sang argent et bleu sur les feuilles mortes~; ils suivaient la trace du sang, vers l'endroit où la créature devait avoir commencé à attaquer la licorne.

<<~Y'a des rumeurs sur toi, dit Hagrid d'une voix basse après qu'ils eurent marché un moment.

--- Eh bien, elles sont toutes vraies, dit Tracey. \emph{Toutes}.

--- Pas toi, dit Hagrid. T'as vraiment témoigné sous Veritaserum que t'as essayé d'aider Mlle Granger, trois fois, c'est ça~?~>>

Drago soupesa ses mots pendant quelques instants puis dit enfin~: <<~Oui~>>. Il n'aurait pas été judicieux de sembler empressé de s'attribuer ce crédit.

L'immense homme secoua la tête, et ses grands pieds piétinaient toujours les bois, silencieusement.

<<~Ça m'étonne, pour être honnête. Et toi aussi, Davis, remettre d'l'ordre dans les couloirs. Z'êtes sûrs que le Choixpeau vous a bien mis où y faut~? Y'a pas un seul sorcier ou sorcière mal tourné qu'était pas à Serpentard, c'est ce qu'on a toujours dit.

--- Ce n'est pas vrai, dit Tracey. Et Xiaonan Tong le Corbeau Noir, Spencer Hill, et Mister Kayvon~?

--- Qui~? dit Hagrid.

--- Juste certains des meilleurs mages noirs des deux siècles précédents, dit Tracey. Ils sont probablement \emph{les} meilleurs de Poudlard à ne pas avoir été à Serpentard.~>> Sa voix chuta, perdit de son enthousiasme. <<~Mlle Granger me disait toujours que je devrais m'informer sur tout ce qui…

--- \emph{Quoi qu'il en soit}, dit rapidement Drago, ça n'est pas vraiment pertinent, M. Hagrid. Même si…~>> Drago travailla la phrase mentalement, essaya de traduire la différence entre \emph{probabilité de Serpentard sachant mage noir} et \emph{probabilité de mage noir sachant Serpentard} dans un langage non scientifique. <<~Même si la plupart des mages noirs sont de Serpentard, très peu de Serpentard sont des mages noirs. Il n'y a pas tant de mages noirs que ça, donc tous les Serpentard ne peuvent pas en être un.~>> Ou comme père avait dit, même si un Malfoy se devait certainement de connaître nombre des secrets de la tradition, il valait mieux laisser les rituels plus… \emph{coûteux} aux mains d'idiots utiles tels qu'Amycus Carrow.

<<~Donc vous dites, dit Hagrid, que la plupart des mages noirs sont Serpentard… mais…

--- Mais la plupart des Serpentard ne sont pas des mages noirs,~>> dit Drago. Il avait l'épuisante sensation que ça allait prendre un moment, mais comme face à une hydre, le plus important était de ne pas abandonner.

<<~Je n'y avait jamais pensé comme ça, dit l'immense homme d'un ton abasourdi. Enfin, si vous n'êtes pas qu'une maison de serpents, alors pourquoi… \emph{planquez-vous derrière cet arbre~!}~>>

Hagrid saisit Drago et Tracey, il les souleva, loin du chemin, derrière un haut chêne. Il se saisit d'une flèche et la plaça sur son arbalète avant de la lever, prêt à tirer. Ils tendirent l'oreille. Quelque chose glissait sur les feuilles mortes, non loin~: on aurait dit le son d'une cape qui traînait contre le sol. Hagrid plissa les yeux en direction du sombre chemin, mais après quelques secondes, le bruit s'estompa.

<<~Je le savais, murmura Hagrid. Y'a queq'chose ici qui d'vrait pas y être.~>>

Ils continuèrent après l'origine du bruissement, Hagrid en tête, Tracey et Drago tous deux main fermement serrée sur leur baguette, prêts, mais ils ne trouvèrent rien malgré leur parcours en spirale, oreilles tendues à l'affût du moindre bruit.

Ils continuèrent à travers les arbres noirs et denses. Drago continua de regarder par-dessus son épaule avec le sentiment qu'on les regardait. Ils venaient de dépasser un tournant quand Tracey hurla et pointa du doigt.

Au loin, une pluie d'étincelles rouges éclairait le ciel.

<<~Attendez ici, vous deux~! cria Hagrid. Restez ou vous êtes, je reviens vous chercher~!~>>

Avant que Drago ne puisse parler, Hagrid pivota et fonça dans les broussailles.

Drago et Tracey se regardèrent jusqu'à ce qu'ils n'entendent plus que le bruissement des feuilles autour d'eux. Tracey semblait être effrayée et désireuse de le cacher. Drago était plus agacé qu'autre chose. On avait clairement l'impression que Rubeus Hagrid, lorsqu'il avait devisé son plan de recherche, n'avait même pas passé cinq secondes à visualiser les conséquences qu'aurait un contretemps.

<<~Et maintenant~?~>> dit Tracey d'une voix un peu aiguë.

<<~On attend que M. Hagrid revienne.~>>

Les minutes s'étirèrent. Les oreilles de Drago semblaient plus perçantes qu'à l'habitude~; elles repéraient le moindre soupir du vent, le moindre craquement de brindille. Tracey continuait de regarder la lune, comme pour se rassurer quant au fait qu'elle n'était pas encore pleine.

<<~Je… chuchota Tracey. Je deviens un peu nerveuse, M. Malfoy.~>>

Drago y songea un moment. Pour être honnête, il y \emph{avait} là quelque chose… eh bien, ce n'était pas de la lâcheté, ni même de la peur. Mais il y avait eu un meurtre à Poudlard, et s'il s'était observé dans une pièce de théâtre, fraîchement abandonné dans la Forêt interdite par un demi-géant, il aurait eu envie de hurler au garçon sur scène qu'il devait…

Drago plongea la main dans ses robes et sortit un miroir. En toucher la surface révéla un homme vêtu de robes rouges qui fronça presque immédiatement les sourcils.

<<~Capitaine Auror Eneasz Brodski~>>, dit clairement l'homme, d'un ton fort qui fit sursauter Tracey dans la forêt silencieuse. <<~Qu'est-ce qu'il y a, Drago Malfoy~?

--- Vérifiez où j'en suis dans 10 minutes~>>, dit Drago. Il avait décidé de ne pas directement se plaindre de sa retenue. Il ne voulait \emph{pas} avoir l'air d'un môme pourrit gâté. <<~Si je ne répond pas, venez me chercher. Je suis dans la Forêt interdite.~>>

Dans le miroir, les sourcils de l'Auror s'arquèrent.

<<~Qu'est-ce que vous faites dans la Forêt interdite, M. Malfoy~?

--- Je cherche le mangeur de licorne avec M. Hagrid~>>, dit Drago, puis il éteint le miroir d'un touché et le remit dans ses robes avant que l'Auror ne puisse poser de question au sujet de la retenue ou de dire quoi que ce soit sur le fait que Drago avait obtempéré sans se plaindre.

Tracey tourna la tête vers lui, mais il faisait un peu trop sombre pour qu'il puisse déchiffrer l'expression sur son visage. <<~Euh, merci~>>, murmura-t-elle.

Les rares feuilles qui avaient émergé des branches bruissèrent lorsqu'une autre brise, plus froide, traversa la forêt.

La voix de Tracey était un peu plus forte lorsqu'elle parla de nouveau. <<~Tu n'avais pas à…~>> dit-elle, comme un peu timide.

<<~Pas la peine d'en parler, Mlle Davis.~>>

La sombre silhouette de Tracey passa une main sur sa joue, comme pour masquer un rougissement qui n'était de toute façon pas visible.

<<~Je veux dire, tu n'avais pas à faire ça pour \emph{moi}…

--- Non, vraiment, dit Drago. Vraiment pas la peine. Du tout.~>> Il aurait bien menacé de sortir le miroir et d'ordonner au capitaine Brodski de ne pas la sauver \emph{elle}, mais il avait peur qu'elle interprète cela comme du flirt.

La silhouette de la tête de Tracey se détourna de lui et regarda au loin. Elle dit enfin, d'une voix plus petite~: <<~C'est trop tôt, n'est-ce pas…~>>

Un cri perché fit écho à travers la forêt, un son pas tout à fait humain, le cri d'une créature proche du cheval~; et Tracey glapit et courut.

<<~\emph{Non, espèce de crétine~!}~>> s'écria Drago en plongeant après elle. Le son avait été si étrange qu'il n'était pas tout à fait certain de son origine… mais il pensait que Tracey Davis était peut-être en train de courir justement vers la source de cet étrange cri.

Des ronces giflèrent les yeux de Drago et il dut garder une main devant lui pour s'en protéger tout en essayant de ne pas perdre Tracey, car il semblait évident que, si cela avait été une pièce, et qu'ils avaient été séparés, \emph{l'un} d'eux allait mourir. Drago songea au miroir en sécurité dans ses robes mais il lui sembla savoir que si jamais il essayait de l'extirper d'une seule main tout en courant, le miroir chuterait inévitablement et qu'il serait perdu…

Devant eux, Tracey s'était arrêtée, et Drago se sentit soulagé l'espace d'un instant, avant de voir.

Une autre licorne était tombée au sol, entourée d'une mare de sang argenté de plus en plus grande~; le bord de la mare avançait au sol exactement comme du mercure qu'on aurait renversé. Elle avait une robe violette, comme un ciel nocturne, sa corne avait exactement la même couleur crépusculaire que sa peau, son flanc visible était marqué d'une trace d'étoile rose tachetée de blanc. La vue déchira le cœur de Drago, encore plus que la licorne précédente, car les yeux vitreux de celle-ci le regardaient, et parce qu'il y avait une…

… forme floue, déformée…

… qui se nourrissait à même une plaie béante sur le flanc de la licorne, comme si elle buvait…

… Drago ne pouvait pas comprendre, ne pouvait pas tout à fait reconnaître ce qu'il voyait…

…\emph{ça les regardait.}

Le flou, le grouillement, la noirceur méconnaissable sembla se retourner pour les observer. Un sifflement s'en échappa, comme venu du serpent le plus mortel à avoir jamais existé, bien plus dangereux encore qu'un bungarus candidus.

Puis il se pencha de nouveau au-dessus de la licorne blessée et continua de boire.

Le miroir était dans la main de Drago. Il demeurait sans vie, tandis que ses doigts touchaient sa surface, encore et encore, d'un geste mécanique.

Tracey tenait à présent sa baguette et prononçait des mots comme <<~Prismatis~>> et <<~Stupéfix~>>, mais rien ne se produisait.

Puis la silhouette grouillante se leva, comme un homme qui se serait remit sur pieds, mais non~: il semblait se précipiter en avant par un étrange bond au-dessus des jambes de la licorne mourante~; il s'approchait d'eux.

Tracey tira sa manche et fit volte-face, prête à fuir, à fuir la chose capable de rattraper des licornes. Avant qu'elle ne puisse faire trois pas vint un autre sifflement terrible qui brûla ses oreilles, puis Tracey tomba au sol et ne bougea plus.

Quelque part, dans un recoin de pensée, Drago sut qu'il était sur le point de mourir. Même si l'Auror vérifiait son miroir à cet instant, il était impossible que qui que ce soit arrive à temps. Il n'y avait pas assez de \emph{temps}.

Courir avait échoué.

La magie avait échoué.

La silhouette grouillante s'approcha, alors que Drago tentait, dans ses derniers instant, de résoudre cette énigme.

Puis une étincelante sphère d'argent plongea du ciel nocturne et se suspendit dans les airs, éclaira la forêt comme en plein jour, et la silhouette grouillante fit un bond en arrière, comme horrifiée par la lumière.

Quatre balais plongèrent du ciel, trois Aurors entourés de boucliers aux vives couleurs, et Harry Potter tenait sa baguette, assis derrière le professeur McGonagall, derrière un bouclier plus large.

<<~Partez d'ici~!~>> rugit le professeur McGonagall…

\begin{em}
Un instant avant que la chose grouillante ne laisse échapper un autre sifflement terrible, que tous les sortilèges de boucliers ne disparaissent soudain. Les trois Aurors et le professeur McGonagall tombèrent de leur balai, chutèrent lourdement sur le sol forestier, demeurèrent immobiles.

Drago ne pouvait pas respirer, la peur la plus intense qu'il ait jamais ressentie comprimait toute sa poitrine, enserrait son cœur comme une ronce.

Harry Potter, qui était demeuré indemne, dirigea silencieusement son balai vers le sol…

… puis bondit pour se tenir entre Drago et la silhouette grouillante, pour s'interposer, comme un bouclier vivant.

<<~Cours~!~>> dit Harry Potter, tournant à moitié sa tête vers Drago. La lumière argentée de la lune se réfléchit sur son visage. <<~Cours, Drago~! Je le retiendrai~!

--- Tu ne peux pas combattre cette chose seul~!~>> s'écria Drago. Une nausée, un poids dans son estomac qui, rétrospectivement, lui sembla avoir été à la fois proche et éloigné d'un sentiment de culpabilité, comme s'il n'y avait eu que les sensations mais pas tout à fait les émotions.

<<~Il le faut, dit sombrement Harry Potter. Vas-t'en~!

--- Harry, je… je suis désolé, pour tout, je…~>> Même si plus tard, Drago ne pourrait pas tout à fait se souvenir de ce pour quoi il avait voulu s'excuser, peut-être que cela avait été son plan de renverser la conspiration de Harry, il y a bien longtemps.

La silhouette grouillante, à présent comme plus sombre, plus terrible, s'éleva dans les airs, flotta au-dessus du sol.

<<~\shout{Vas-t'en}~!~>> hurla Harry.

Drago se retourna, fuit droit dans les bois, des branches fouettèrent son visage. Derrière lui, Drago entendit un autre sifflement terrible et la voix de Harry qui s'élevait, qui criait une chose que Drago, à cette distance, ne put comprendre~; Drago tourna la tête un instant pour regarder, et il buta alors contre quelque chose, se cogna \emph{fort} la tête et s'évanouit.
\end{em}

\later

Harry tenait fermement sa baguette. Une sphère prismatique brillait autour de lui. Il regarda avec assurance la forme grouillante et floue face à lui et dit~: <<~Bon sang, mais qu'est-ce que vous faites~?~>>

Le flou grouillants se concentra, se reforma, et se mut en une forme encapuchonnée. Quelle qu'ait été la dissimulation à l'œuvre - un artefact plutôt qu'un charme, avait deviné Harry, car la magie avait été capable de l'atteindre lui aussi - elle avait empêché son esprit de reconnaître la forme, et même de reconnaître que la forme avait été humaine. Mais elle n'avait pas empêché Harry reconnaître la forte sensation funeste.

Le professeur Quirrell se tenait droit, du sang argenté à l'avant de sa grande cape noire, et il laissa échapper un soupir en regardant les trois Aurors, Tracey Davis, Drago Malfoy et le professeur McGonagall. <<~Je pensais sincèrement, murmura le professeur Quirrell, avoir brouillé le miroir sans provoquer d'alarme. Qu'est-ce que deux Serpentard de première année faisaient dans la forêt~? M. Malfoy a plus de bon sens que ça… Quel fiasco.~>>

Harry ne répondit pas. La sensation funeste n'avait jamais été plus forte, la sensation d'un pouvoir environnant si grand qu'il en était presque tangible. Une partie de lui était encore viscéralement en état de choc face à la vitesse à laquelle les boucliers autour des Aurors avaient été mis en lambeaux. Il n'avait presque pas été capable de \emph{voir} les traits de couleur qui avaient déchiré les boucliers comme des mouchoirs. Cela laissait à penser que le duel entre le professeur Quirrell et l'Auror d'Azkaban avait été une blague, un jeu d'enfant… Même si le professeur Quirrell avait alors prétendu que, s'il s'était battu sérieusement, l'Auror serait mort en quelques secondes~; et Harry sut alors qu'il avait dit vrai.

Jusqu'où montait l'échelle du pouvoir~?

<<~Je suppose, dit Harry en parvenant à garder une voix stable, que le fait que vous mangiez des licornes a un lien avec ce qui vous fera être renvoyé du poste de professeur de Défense. J'imagine que vous ne voudriez pas m'expliquer, avec beaucoup de détails~?~>>

Le professeur Quirrell le regarda. La sensation de pouvoir quasiment tangible avait diminué, était revenue dans le professeur Quirrell. <<~Je vais en effet m'expliquer, dit le professeur de Défense. Je dois d'abord lancer quelques sortilèges de faux souvenirs, et nous pourrons ensuite partir et discuter, car il ne serait pas sage que je demeure. Je sais que vous reviendrez en cet instant, plus tard.~>>

Harry se fit voir à travers la Cape dont il était maître, et il sut qu'un autre Harry se tenait à côté de lui, caché par sa propre Relique de la Mort. Harry dit alors à sa Cape de se cacher à nouveau de lui, et elle le fit~; si l'on pouvait percevoir son soi futur, il fallait être capable de correspondre au souvenir, plus tard.

La voix de Harry dit alors, étrange aux oreilles du Harry actuel~: <<~Il a une explication étonnamment bonne.~>>

Le Harry actuel fit de son mieux pour se souvenir des mots. Ils ne se dirent rien de plus.

Le professeur Quirrell marcha jusqu'à Drago et entonna le sortilège de faux souvenirs. Il se tint là pendant peut-être une minute, comme perdu.

Harry avait étudié les sortilèges d'Oubliette les deux semaines précédentes - quoi qu'il aurait été incapable de lancer le sortilège, à moins d'être prêt à s'épuiser complètement et que, pour une étrange raison, l'on ait besoin qu'un Auror perde tous ses souvenirs associés à la couleur bleu. Mais Harry avait à présent quelque idée de la concentration que le bien plus difficile sortilège de faux souvenirs exigeait. Il fallait essayer de vivre la vie entière de l'autre dans sa propre tête, du moins si l'on souhaitait créer des faux souvenirs à une vitesse moins de seize fois inférieure à celle de l'écoulement réel du temps en ayant à construire séparément seize pistes mémorielles majeures. C'était peut-être silencieux, il n'y avait peut-être pas de signe extérieur, mais Harry savait à présent quelque choses des difficultés, et il savait qu'il y avait lieu d'être impressionné.

Le professeur Quirrell finit et passa à Tracey Davis, puis aux trois Aurors, puis enfin au professeur McGonagall. Harry attendit, mais le Harry futur ne protesta pas. Il était possible que même éveillée, le professeur McGonagall n'aurait pas protesté. Nous n'en étions pas encore aux Ides de Mai, et apparemment l'explication serait suffisamment bonne.

D'un geste, le corps étourdi de Drago fut soulevé et envoyé non loin, dans les bois, avant d'être précautionneusement déposé au sol. Puis un geste final du professeur Quirrell arracha un énorme morceau du flanc de la licorne et laissa derrière lui des bords en lambeaux~; la chair crue flotta dans les airs puis ondula avant de s'estomper sous l'effet d'un sortilège de disparition.

<<~Voilà, dit le professeur Quirrell. Je dois maintenant quitter ce lieu, M. Potter. Venez avec moi et restez ici.~>>

Le professeur Quirrell s'éloigna à grands pas, Harry le suivit et demeura.

Ils traversèrent les bois en silence pendant un moment avant que Harry n'entende des filets de voix, au loin. Le prochain groupe d'Aurors, probablement, après que le contact avec le premier ait été coupé. Ce que son lui futur disait, Harry l'ignorait.

<<~Ils ne nous détecteront pas et ne nous entendront pas parler~>>, dit le professeur Quirrell. La sensation funeste, la présence d'un pouvoir, étaient toujours fortes autour du professeur Quirrell. L'homme s'assit sur une souche, où la lumière de la lune presque pleine tomba presque entièrement sur lui. <<~Je dois d'abord vous dire que, quand vous parlerez aux Aurors, plus tard, vous devrez leur dire que vous avez effrayé la chose noire grouillante comme vous l'avez fait avec ce Détraqueur. C'est ce que M. Malfoy se souviendra avoir vu.~>> Le professeur Quirrell laissa échapper un petit soupir. <<~Cela causera peut-être quelque alarme s'ils arrivent à la conclusion qu'une horreur semblable aux Détraqueurs et assez puissante pour briser des boucliers d'Auror déambule dans la Forêt interdite. Mais je n'ai pas su quoi faire d'autre. Si la forêt est mieux gardée après cela… mais avec de la chance, j'ai déjà consommé ce dont j'ai besoin. Voudriez-vous bien m'expliquer comment vous êtes arrivé si vite~? Comment saviez-vous que M. Malfoy avait des ennuis~?~>>

Après que la capitaine Brodski ait entendu que Drago Malfoy était dans la Forêt interdite, visiblement en compagnie de Rubeus Hagrid, il avait commencé à poser des questions destinées à découvrir qui avait autorisé cela et avait toujours été bredouille lorsque Drago Malfoy ne l'avait pas à nouveau contacté, dix minutes plus tard. En dépit des protestations de Harry, le capitaine Auror, qui était habilité en matière de Retourneurs de Temps, avait refusé d'autoriser un déploiement avant l'instant où Drago aurait dû les contacter~; c'était la procédure standard lorsqu'on manipulait le Temps. Mais Brodski avait donné à Harry un ordre écrit qui l'autorisait à déployer un trio d'Aurors une seconde avant cet instant. Un Patronus avait permit de localiser Drago - Harry était parvenu à lui donner l'apparence d'une sphère de lumière argentée - et le convoi d'Aurors était arrivé à la seconde près.

<<~J'ai peur de ne pouvoir vous le dire~>>, dit Harry d'une voix assurée. Le professeur Quirrell était encore un des principaux suspects et il valait mieux pour lui qu'il ignore les détails. <<~Maintenant, pourquoi mangez-vous des licornes~?

--- Ah, dit le professeur Quirrell. Quant à cela…~>> l'homme hésita. <<~Je buvais leur sang, je ne les mangeais pas. La chair manquante, les lambeaux - c'était pour obscurcir l'affaire, pour laisser croire que c'était un autre prédateur. L'usage de sang de licorne est trop connu.

--- Je l'ignore, dit Harry.

--- Je sais que vous l'ignorez, dit sèchement le professeur de Défense. Sans quoi vous ne m'embêteriez pas avec ça. Le pouvoir du sang de licorne est de préserver la vie pendant quelques temps, même lorsque l'on est à l'article de la mort.~>>

Il y eut un moment pendant lequel le cerveau de Harry refusa de comprendre les mots, ce qui était bien sûr un mensonge, car on ne pouvait connaître le sens de ce que l'on refusait de comprendre sans l'avoir déjà compris.

Un étrange sentiment de vide s'empara de Harry, une absence de réaction. Peut-être était-ce ce que les autres ressentaient quand quelqu'un s'éloignait du script établi et qu'ils ne trouvaient pas quoi dire, pas quoi faire.

Bien sûr que le professeur Quirrell était mourant et pas seulement occasionnellement malade.

Le professeur Quirrell avait su qu'il se mourrait. Après tout, il s'était porté volontaire pour le poste de professeur de Défense à Poudlard.

Bien sûr que son cas s'était aggravé au fil de l'année. Bien sûr que les maladies qui empiraient avaient une destination prévisible.

Le cerveau de Harry l'avait sûrement déjà su, quelque part dans un recoin sûr de son esprit, là où il pouvait refuser de comprendre des choses qu'il avait déjà comprises.

Bien sûr que c'était pour ça que le professeur Quirrell ne pourrait pas enseigner la magie de combat l'année prochaine. Le professeur McGonagall n'aurait même pas à le renvoyer. Il serait juste…

… mort.

<<~Non, dit Harry d'une voix un peu secouée. Il doit y avoir un moyen…

--- Je ne suis ni stupide ni particulièrement impatient de mourir. J'ai déjà cherché. J'ai dû en arriver là uniquement pour finir mes cours, car je dispose de moins de temps que ce que je pensais, et…~>> la tête de la silhouette éclairée par la lune se détourna. <<~Je ne pense pas que vous souhaitiez entendre ça, M. Potter.~>>

La respiration de Harry eut un à-coup. Trop d'émotions surgissaient en lui en même temps. Après le déni vint la colère, selon un rituel que quelqu'un avait un jour inventé. Et pourtant il semblait étonnamment convenir.

<<~Et pourquoi…~>> la respiration de Harry eut un autre à-coup. <<~Alors pourquoi le sang de licorne ne fait-il pas partie des kits de soin standards, alors~? Pour garder une personne en vie, même si elle est sur le point de mourir parce que ses jambes ont été dévorées~?

--- Parce qu'il y a des effets secondaires permanents, dit doucement le professeur Quirrell.

--- Des effets secondaires~? \emph{Des effets secondaires~?} Quel genre d'effet secondaire serait pire que la \shout{mort~?}~>> La voix de Harry s'éleva le long de ce dernier mot, jusqu'à ce qu'il le crie.

<<~Tout le monde ne pense pas comme vous, M. Potter. Et pour être honnête, le sang doit venir d'une licorne vivante, et l'acte de boire doit tuer la licorne. Sinon, serais-je ici~?~>>

Harry se retourna et regarda la arbres autour de lui.

<<~Gardez un troupeau de licornes devant Sainte. Mangouste. Envoyez les patients là-bas par cheminette, ou utilisez des Portoloins.

--- Oui, cela fonctionnerait.~>>

Le visage de Harry se contracta, et le seul signe extérieur, mis à part ses mains tremblantes, de tout ce qui jaillissait en lui. Il fallait qu'il crie, que quelque chose sorte, il avait besoin de \emph{quelque chose} qu'il ne savait nommer, et enfin Harry leva la main vers un arbre et s'écria~: <<~\emph{Diffindo~!}~>>

Il y eut le son sec d'un arrachement, et une coupure apparut dans le bois.

<<~\emph{Diffindo~!}~>>

Un autre coupure. Harry n'avait appris le sortilège que dix jours plus tôt, après s'être mis à l'autodéfense sérieusement. En théorie, c'était un sortilège de deuxième année, mais la colère qui l'envahissait semblait n'avoir aucune limite~; il en savait assez pour ne pas risquer l'épuisement et disposait encore d'assez de pouvoir.

<<~\emph{Diffindo~!}~>> Harry visa une branche, cette fois, et elle tomba au sol dans un bruit de brindilles et de feuilles.

Il ne semblait pas y avoir de larmes en lui, seulement une pression sans issue.

<<~Je vais vous laisser~>>, dit doucement le professeur Quirrell. Le professeur de Défense se leva de sa souche, la sang de licorne éclairé par la lune toujours sur sa cape, et il rabattit la capuche sur sa tête.
%  LocalWords:  rgus Pfeh ere’s Yeh’ve yer takin eatin León yeh Yuliy weres
%  LocalWords:  o ter Meself ands yeh’ve holdin long’s fightin Buchholz teh
%  LocalWords:  mornin diff’rent yeh’ll nothin interestin tryin Xiaonan
%  LocalWords:  Kayvon yeh’re shouldn Eneasz
