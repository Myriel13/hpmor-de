\partchapter{Compromis Tabous, Fin}{Fin}

\lettrine{L}{e} voyage par phénix provoquait une sensation complètement différente de celle provoquée par le Transplanage ou les Portoloins. Vous preniez feu -- vous vous sentiez clairement prendre feu, mais sans ressentir de douleur -- et au lieu de tomber en cendres, vous sentiez le feu vous brûler de part en part puis vous \emph{transformer} en feu à votre tour, et vous vous éteigniez alors, pour aller vous embraser ailleurs. Cela ne donnait pas la nausée comme les Portoloins et le Transplanage mais c'était néanmoins une expérience assez déroutante. Si la vérité sous-jacente au voyage par phénix était que l'on devenait réellement une instance spécifique d'un Feu général, cela laissait entendre qu'on pouvait potentiellement brûler \emph{n'importe où} -- même dans un passé lointain, dans un autre univers ou dans deux endroits à la fois. Que l'on pouvait s'éteindre quelque part et s'embraser dans cent lieux à la fois, et que celui qui arriverait à Poudlard pourrait n'en avoir jamais conscience. Mais Harry avait lu autant qu'il le pouvait au sujet des phénix dans le but d'obtenir le sien et il n'avait pas vu d'indice sur quoi que ce soit qui ressemble de près ou de loin à cette capacité.

Harry prit feu, disparut, s'embrasa de nouveau et d'un coup lui, le directeur et le corps inconscient de Hermione Granger, tenu par les bras du directeur, se retrouvèrent ailleurs avec Fumseck au-dessus d'eux trois. Une pièce calme et chaude avec des colonnes de pierres illuminées par le ciel par ses quatre murs, peuplée de lits blancs disposés en longues rangées, dont quatre entourés de rideaux et les autres vides.

Dans un coin de son champ de vision, Harry put voir une Mme Pomfresh a l'air surprise se tourner vers eux. Dumbledore sembla ne prêter aucune attention à la guérisseuse en chef alors qu'il allongeait précautionneusement Hermione sur un lit blanc vide.

Dans un coin lointain il y eut un flash vert et d'un feu émergea le professeur McGonagall qui s'époussetait un peu des cendres de Cheminette.

Le vieux sorcier se détourna du lit, tendit un de ses bras pour entourer de nouveau Harry et le Survivant et son sorcier disparurent dans un autre embrasement.

\later

Lorsque Harry se fut entièrement rallumé, il se tenait dans le bureau du directeur, entre les bruits d'une dizaine de dizaines de gadgets inexplicables.

Le jeune garçon fit un pas pour s'écarter du vieux sorcier puis se retourna vers lui, croisant les yeux de saphir de son regard d'émeraude.

Ils ne parlèrent pas pendant un moment et se regardèrent, comme si tout ce qu'ils avaient à se dire pouvait seulement être communiqué par des regards et d'aucune autre façon.

Le garçon finit par énoncer ces mots, lentement et précisément~:

<<~Je ne peux pas croire qu'un phénix se trouve encore sur votre épaule.

--- Les phénix ne choisissent qu'une fois, dit le vieux sorcier. Ils peuvent peut-être quitter un maître qui choisit le mal plutôt que le bien~; mais ils ne choisiront pas un maître forcé de choisir entre un bien et un autre. Les phénix ne sont pas arrogants. Ils connaissent les limites de leur propre sagesse.~>> Son regard ancien était des plus sévères <<~Contrairement à toi, Harry.

--- Choisir entre un bien et un autre, dit Harry en écho, sans timbre. Comme la vie de Hermione Granger contre cent-mille Gallions.~>> La rage et l'indignation que Harry avait voulu mettre dans sa voix n'était pas tout à fait présente, peut-être parce que…

<<~Tu es loin d'être bien placé pour me parler de cela, Harry Potter.~>> La voix du directeur était trompeusement douce. <<~Ou qu'était cet air récalcitrant que j'ai vu sur ton visage, là-bas, dans la Très Ancienne Chambre~?~>>

La sensation qu'il avait d'être creux empira. <<~Je cherchais des alternatives, mordit-il. Un moyen de la sauver qui ne me fasse \emph{pas} perdre d'argent.~>>

\emph{Waouh}, dit Serdaigle. \emph{Tu viens de proférer un mensonge pur et simple. Non seulement ça mais en plus je crois que tu y a vraiment} cru \emph{pendant les secondes que tu as mises à le prononcer. Ça fait plutôt peur.}

<<~\emph{Est-ce} ce à quoi tu pensais, Harry~?~>> Les yeux bleus étaient comme acérés et Harry vécut un instant terrifiant pendant lequel il se demanda si le sorcier le plus puissant du monde pouvait parfaitement voir à travers ses barrières occlumantiques.

<<~\emph{Oui}, dit Harry, j'ai reculé face à la douleur de me défaire de tout l'argent de ma chambre forte. Mais je l'ai quand même \emph{fait}~! C'est \emph{ça} qui compte~! Et \emph{vous…}~>> l'indignation qui avait quitté la voix de Harry revint. <<~Vous avez \emph{vraiment} mis un prix sur la vie de Hermione Granger et vous l'avez mis en dessous de cent-mille Gallions~!

--- Oh~? dit doucement le vieux sorcier. Et quel prix mettrais-tu sur sa vie, alors~? Un million de Gallions~?

--- Êtes-vous familier du concept économique de 'coût de remplacement'~?~>> Les mots jaillissaient des lèvres de Harry trop vite pour qu'il puisse les examiner. <<~Le coût de remplacement de Hermione est \emph{infini~!} Je ne peux m'en racheter une autre nulle part~!~>>

\emph{Maintenant tu profères juste des non-sens mathématiques}, dit Serpentard. \emph{Serdaigle, tu peux me soutenir sur ce coup~?}

<<~La vie de Minerva a-t-elle aussi une valeur infinie~? dit le vieux sorcier d'un ton brusque. Sacrifierais-tu Minerva pour sauver Hermione~?

--- Oui et oui, lâcha Harry. Ça fait partie du travail du professeur McGonagall et elle le sait.

--- Alors la valeur de Minerva n'est pas infinie, dit le vieux sorcier, malgré tout l'amour qu'on peut lui porter. Il ne peut y avoir qu'un seul roi sur l'échiquier, Harry Potter, une seule pièce pour laquelle tu sacrifierais n'importe quelle autre pièce afin de la sauver. Et Hermione Granger n'est pas cette pièce. Ne t'y trompes pas, Harry Potter, aujourd'hui est peut-être le jour où tu as perdu notre guerre.~>>

Et si les mots du vieux sorcier ne l'avaient pas touché si durement, n'avaient pas été aussi proches de la vérité, Harry n'aurait peut-être pas dit ce qu'il dit alors.

<<~Lucius avait raison, grinça Harry. Vous n'avez jamais eu de femme, vous n'avez jamais eu de fille, vous n'avez jamais eu que la guerre…~>>

La main gauche du vieux sorcier se referma avec force autour du poignet de Harry, les doigts osseux s'enfoncèrent dans les muscles encore en développement du bras de ce dernier, et l'espace d'un instant Harry fut paralysée par le choc~; il avait oublié les conséquences que pouvait avoir la force supérieure des adultes.

Albus Dumbledore ne sembla pas le remarquer. Il se contenta de pivoter en traînant Harry derrière lui et s'avança d'un pas dur vers le mur de la pièce.

<<~\emph{Prix du phénix}.~>>

Harry fut tiré le long des escaliers noirs.

<<~\emph{Destin du phénix.}~>>

La pièce aux piédestaux noirs et à la lumière d'argent tombant sur des baguettes fracassées.

<<~Vous pensez, cria Harry après que ses lèvres se furent débloquées, que vous pouvez gagner n'importe quel débat juste en allant ici~?~>>

Le vieux sorcier l'ignora et traîna Harry à travers la pièce. Sa main droite ne tenait plus sa baguette et saisit une fiole d'un fluide argenté…

Harry cligna des yeux, stupéfié~: la fiole au fluide argenté avait été placée à côté d'une image de \emph{Dumbledore}, ou du moins c'est ainsi que Harry l'avait vu pendant le bref moment où il s'était fait traîner devant l'image.

Derrière tous les piédestaux, dans la partie la plus éloignée de la pièce, s'élevait un grand bassin de pierre gravé de runes que Harry ne reconnut pas. Son centre était peu profond et rempli d'un liquide transparent. Le vieux sorcier y déversa le fluide argenté qui commença immédiatement à se répandre, à tourbillonner, et à illuminer le bassin entier d'un blanc surnaturel.

La main du vieux sorcier relâcha le bras de Harry et fit un geste en direction du bassin lumineux, ordonnant d'un ton brusque~: <<~Regarde~!~>>

Comme on le lui avait demandé, Harry regarda l'eau lumineuse.

<<~Mets ta tête dans la Pensine, Harry Potter.~>> La voix du vieux sorcier était sévère.

Harry avait déjà entendu ce mot mais il n'arrivait plus à se souvenir du contexte.

<<~Qu'est-ce… que ça fait…

--- Souvenirs, dit le vieux sorcier. Tu verras mon souvenir. C'est sans danger, je t'en fais le serment. Maintenant regarde dans la Pensine, Serdaigle, si tu te soucies encore de ta précieuse vérité~!~>>

C'était une demande que Harry ne pouvait pas rejeter, aussi, il s'avança et plongea sa tête dans l'eau lumineuse.
\later

\begin{em}
Harry était assis derrière le bureau du directeur de Poudlard et les mains ridées agrippées à sa tête étaient marquées par l'âge et parsemées de poils blancs.

<<~Il est tout ce que j'ai~!~>> pleura une voix, l'étrange voix de Dumbledore telle que lui s'en souvenait~; de l'intérieur elle semblait bien moins sévère et bien moins sage. <<~Le dernier membre de ma famille~! Tout ce qui me reste~!~>>

On n'avait laissé aucun émotion traverser la Pensine, seulement la sensation physique de sembler prononcer ces mots. Harry entendit la profonde désolation dans les mots de Dumbledore, dans les sons qui semblaient venir de la gorge de Harry lui-même, mais il ne ressentit rien d'autre que ce son.

<<~Tu n'as pas le choix~>>, dit une voix dure.

Les yeux se déplacèrent et le champ de vision sauta jusqu'à un homme que Harry ne reconnut pas, porteur d'une veste teintée du cramoisi Auror mais faite d'un cuir solide et dotée de nombreuses poches.

Son œil droit était surdimensionné et avait une pupille d'un bleu électrique qui se déplaçait en permanence.

<<~Tu ne peux pas me demander ça, Alastor~! la voix de Dumbledore devenait folle. Pas ça~! Tout sauf ça~!

--- Je ne te le demande pas, gronda l'homme. C'est Voldy qui le demande, et tu vas lui dire non.

--- Pour de l'argent, Alastor~? la voix de Dumbledore était suppliante. Seulement pour de l'argent~?

--- Tu paies la rançon d'Aberforth, tu perds la guerre, dit sèchement l'homme. C'est aussi simple que ça. Cent-mille Gallions, presque tout ce que nous avons dans notre trésor de guerre, et si tu l'utilises comme ça, on n'ira pas le remplir à nouveau. Qu'est-ce que tu vas faire, essayer de convaincre les Potter de vider leur coffre comme les Londubat l'ont déjà fait~? Voldy va juste kidnapper quelqu'un d'autre et exiger autre chose. Alice, Minerva, tous ceux auxquels tu tiens, ils seront tous des cibles si tu paies les Mangemorts. Ce n'est pas la leçon que tu devrais essayer de leur faire apprendre.

--- Si je fais ça, je n'aurais plus personne. Plus personne.~>> La voix de Dumbledore se brisa, le monde s'inclina, la tête d'où émanait le champ de vision tomba entre les vieilles mains et de terribles sons émergèrent de la gorge, qui n'était pas celle de Harry, lorsque Dumbledore commença à sangloter comme un enfant.

<<~Devrais-je dire non au messager de Voldy~?~>> dit la voix d'Alastor, cette fois avec une étrange douceur. <<~Tu n'as pas à le faire toi-même, mon vieil ami.

--- Non… Je le dirai moi-même… Je dois…~>>
\end{em}

\later

Le souvenir prit fin d'un coup brusque et Harry arracha sa tête de l'eau lumineuse en haletant comme s'il avait été privé d'air.

La transition entre les scènes, entre la réalité vieille de dix ans et l'instant présent, fut une secousse pour l'esprit de Harry~; d'une certaine façon, son immersion dans le passé l'avait décoincé~: le vieil homme brisé qui pleurnichait dans son bureau avait été quelqu'un d'autre, d'un autre temps~; cela, Harry l'avait compris~; quelqu'un de plus doux…

Avant que tout ne disparaisse comme de la fumée qui se dissipe, que tout ne revienne à \emph{maintenant}, à aujourd'hui.

Terrible et sévère, le vieux sorcier, comme creusé dans la pierre~; sa barbe tressée de fils qu'on aurait cru faits d'acier, ses lunettes en demi-lune, comme des miroirs, et les pupilles derrière, aussi acérées et inflexibles que du diamant noir.

<<~Souhaites-tu aussi voir mon frère lorsqu'il a péri par Doloris~? demanda Albus Dumbledore. Voldemort m'a aussi envoyé ce souvenir~!

--- Et c'est…~>> Harry avait du mal à donner de la voix à cause de la nausée qui montait dans sa poitrine. <<~C'est \emph{là} que…~>> les mots semblaient brûler sa gorge à mesure que le terrible savoir apparaissait en lui, la compréhension atroce. <<~C'est là que vous avez brûlée vive Narcissa Malfoy dans sa propre chambre.~>>

Le regard d'Albus Dumbledore fut froid lorsqu'il répondit.

<<~À cette question seul un idiot répondrait pas l'affirmative ou la négative. Ce qui compte, c'est que les Mangemorts croient que je l'ai tuée, et cette croyance a maintenu les familles de tous ceux qui servaient l'Ordre du Phénix en vie -- jusqu'à ce jour. Comprends-tu maintenant ce que tu as fait~? Ce que tu as fait à tes \emph{amis}, Harry Potter, et à tous ceux qui se tiennent à tes côtés~?~>> Le vieux sorcier sembla devenir encore plus grand, encore plus terrible, et sa voix devint plus forte. <<~Tu as fait d'eux des cibles, et ils le resteront~! Jusqu'à ce que tu prouves, de la seule façon qu'on puisse le prouver, que tu ne désires plus payer de tels prix~!

--- Et est-ce vrai~?~>> dit Harry. Il était saisi d'une sorte de bourdonnement, comme si son corps s'éloignait de plus en plus. <<~Ce que Drago a dit, que Narcissa Malfoy ne s'est jamais salie les mains, qu'elle n'était que la femme de Lucius~? C'était une acolyte, je comprends ça, mais je ne peux pas soutenir l'idée que cela mérite d'être \emph{brûlé vif}.

--- Rien de moindre ne les aurait convaincus que j'en avais fini avec l'hésitation.~>> La voix du vieux sorcier ne tolérait aucune question, aucun refus. <<~J'avais toujours été trop réticent à faire ce que je devais et les autres avaient toujours payé le prix de ma pitié. C'est ce qu'Alastor m'avait dit depuis le début mais je ne l'avais pas écouté. Toi, je m'attends à ce que tu t'avères meilleur que moi pour ce genre de décisions.

--- Je suis surpris,~>> dit Harry, émerveillé que sa voix soit aussi stable. <<~Je me serais attendu à ce qu'à moins de tous les avoir du premier coup, les Mangemorts s'en prennent à une autre famille du camp des gentils et commencent un cycle de représailles toujours plus intenses.

--- Si mon adversaire avait été Lucius, peut-être.~>> Les yeux de Dumbledore ressemblaient à des pierres. <<~J'ai entendu dire que Voldemort avait rit à l'annonce de la nouvelle et qu'il avait proclamé à ses Mangemorts que j'avais enfin grandi, que j'étais enfin un adversaire valable. Peut-être avait-il raison. Suite au jour où j'ai condamné mon frère à sa mort, j'ai commencé à soupeser ceux qui me suivaient, à comparer le poids de l'un à celui de l'autre, à me demander qui je risquerais, qui je sacrifierais et dans quel but. Il fut étrange de constater à quel point j'ai perdu moins de pièces une fois que j'ai su ce qu'elles valaient.~>>

La mâchoire de Harry semblait bloquée, comme s'il avait besoin d'un effort colossal pour faire bouger ses lèvres. <<~Mais là ce n'est pas comme si Lucius avait délibérément rançonné Hermione, dit-il d'une voix faible. Du point de vue de Lucius, quelqu'un d'autre a brisé la trêve. En gardant cela à l'esprit, combien de Gallions Hermione vaut-elle exactement~? En laissant la question du Danegeld de côté, si c'était juste une menace de mort ordinaire, combien devrais-je avoir payé pour la sauver~? Dix-mille Gallions~? Cinq mille~?~>>

Le vieux sorcier ne répondit pas.

<<~C'est drôle, dit Harry d'une voix vacillante comme l'image d'un objet immergé. Vous savez, le jour où j'ai fait face au Détraqueur, vous savez ce qu'a été mon pire souvenir~? C'était la mort de mes parents. J'entendais leurs voix et tout.~>>

Les yeux du vieux sorcier s'écarquillèrent derrière les lunettes en demi-lune.

<<~Et voilà le truc, dit Harry, le truc auquel j'ai pensé et repensé. Le Seigneur des Ténèbres a donné à Lily Potter une chance de partir. Il lui a dit qu'elle pouvait fuir. Il lui a \emph{dit} que mourir devant le berceau ne sauverait pas son bébé~: “Écarte-toi, femme imbécile, si tu as le moindre bon sens…”~>> Un terrible frisson parcourut Harry lorsqu'il prononça ces mots de ses propres lèvres, mais il s'en débarrassa et continua. <<~Et après j'ai continué de me dire, je ne pouvais pas m'empêcher de me dire~: est-ce que le Seigneur des Ténèbres n'avait pas \emph{raison}~? Si seulement Mère s'était écartée. Elle a essayé de lancer un sortilège au Seigneur des Ténèbres, mais c'était du suicide, et elle devait le \emph{savoir}. Son choix n'était pas entre sa vie et la mienne, son choix était entre sa vie ou aucune des deux vies~! Si seulement elle avait agit de façon logique et qu'elle s'était écartée, je veux dire, j'aime aussi Maman, mais Lily Potter serait en vie et elle serait ma mère~!~>> Des larmes brouillaient les yeux de Harry. <<~Ce n'est que maintenant que je comprends, que je sais ce que Mère doit avoir ressentit. Elle ne \emph{pouvait pas} s'écarter du berceau. Elle ne pouvait pas~! L'amour ne s'écarte pas~!~>>

C'était comme si le vieux sorcier avait reçu un coup, un coup de burin qui l'avait brisé de bout en bout.

<<~Qu'ais-je dit~? chuchota le vieux sorcier. Que t'ais-je dit~?

--- Je ne sais pas~! cria Harry. Je n'écoutais pas non plus~!

--- Je… je suis désolé, Harry… Je…~>> le vieux sorcier appuya ses mains contre son visage et Harry vit qu'Albus Dumbledore sanglotait. <<~Je n'aurais pas dû dire -- te dire de telles choses… je n'aurais pas dû -- en vouloir -- à ton innocence…~>>

Harry regarda le sorcier une seconde de plus puis se retourna et sortit de la pièce noire, descendit les escaliers, traversa le bureau…

<<~Je ne sais vraiment pas pourquoi tu es toujours sur son épaule~>>, dit-il à Fumseck.

… par la porte et le long de la spirale tournoyante infinie.

\later

Harry était arrivé en cours de Métamorphose avant tout le monde, même avant le professeur McGonagall. Il y avait eu un cours de sortilèges plus tôt mais il n'avait même pas essayé de se rendre à celui-ci. Il ignorait si le professeur McGonagall serait présente au cours d'aujourd'hui. Il y avait quelque chose de menaçant dans les bureaux vides autour de lui, dans l'absence sur l'estrade. Comme s'il se tenait seul dans Poudlard, tous ses amis partis.

Selon le programme du cours, la leçon d'aujourd'hui porterait sur les métamorphoses maintenues, sur toutes les règles que Harry avait apprises par cœur lorsqu'il métamorphosait un énorme rocher en un petit diamant porté sur son petit doigt. Ce serait un sujet théorique plutôt que pratique pour le reste de la classe, ce qui était dommage, car il aurait fort apprécié une transe de métamorphose.

Harry remarqua avec distance que sa main tremblait à un point tel qu'il avait du mal à défaire les ficelles de sa bourse afin d'en sortir son manuel de Métamorphose.

\emph{Tu as été monstrueusement injuste envers Dumbledore}, dit la voix que Harry avait surnommée Serpentard, sauf que maintenant elle semblait aussi être la Voix de la Raison Économique et peut-être aussi celle de la Conscience.

Les yeux de Harry descendirent jusqu'à son manuel mais cette partie était tellement familière qu'il aurait aussi bien pu s'agir d'un parchemin vierge.

\emph{Dumbledore a mené une guerre contre un Seigneur des Ténèbres qui s'était délibérément donné le but de le briser de la façon la plus cruelle possible. Il a dû choisir entre perdre sa guerre et perdre son frère. Albus Dumbledore sait, et il l'a appris de la pire des façons possibles, qu'il y a des limites à la valeur d'une vie~; et l'admettre a failli le rendre fou. Mais toi, Harry Potter --} tu \emph{le sais déjà.}

<<~Tais-toi~>>, chuchota le garçon à une salle de Métamorphose vide, alors qu'il n'y avait personne pour l'entendre.

\emph{Tu as déjà lu les expériences de Philip Tetlock où l'on demande à des gens d'échanger une valeur sacrée contre une valeur profane, comme un administrateur qui devait choisir entre dépenser un million de dollars pour sauver un enfant de cinq ans ou dépenser le million de dollars pour acheter plus d'équipement hospitalier ou payer le salaire de médecins. Et les sujets de l'expérience s'indignaient et voulaient punir l'administrateur parce qu'il avait ne serait-ce que réfléchi à ce choix. Te souviens-tu avoir lu ça, Harry Potter~? Te souviens-tu t'être dit que c'était incroyablement stupide puisque si l'équipement hospitalier et les salaires des docteurs ne sauvaient pas eux aussi des vies, avoir des hôpitaux et des docteurs n'aurait aucun sens~? L'administrateur de l'hôpital aurait-il dû payer un milliard de livres sterling pour ce foie même si l'hôpital allait faire faillite le lendemain~?}

<<~Tais-toi~!~>> chuchota le garçon.

\emph{Chaque fois que tu dépenses de l'argent pour avoir une probabilité donnée de sauver une vie, tu établis une borne inférieure sur la valeur monétaire de cette vie. Chaque fois que tu refuses de payer pour obtenir une probabilité de sauver une vie, tu établis une borne supérieure sur la valeur monétaire de cette vie. Si tes bornes supérieures et inférieures sont incohérentes, cela veut dire que tu pourrais déplacer de l'argent d'un point à un autre et sauver plus de vies pour le même prix. Donc si tu veux utiliser une quantité bornée d'argent pour sauver autant de vies que possible, tes choix doivent être en adéquation avec} une \emph{valeur monétaire assignée à la vie humaine~; sinon, alors tu pourrais redistribuer le même argent et obtenir un meilleur résultat. Qu'elle est triste, qu'elle est creuse l'indignation de ceux qui refusent de dire que l'argent et la vie ne peuvent jamais être comparés alors que tout ce qu'il font c'est d'interdire la stratégie qui sauve le plus de gens au nom d'une prétention morale grandiloquente.}

\emph{Tu} savais \emph{cela et tu as quand même dit ce que tu as dit à Dumbledore.}

\emph{Tu as délibérément} tenté \emph{de faire de la peine à Dumbledore.}

Il \emph{n'a jamais essayé de te faire de la peine, Harry, pas un seule fois.}

La tête de Harry tomba entre ses mains.

Pourquoi avait-il dit ce qu'il avait dit à un vieux sorcier triste qui s'était battu de toutes ses forces et qui avait enduré plus que quiconque n'aurait jamais dû avoir à endurer~? Même si le vieux sorcier avait tort, méritait-il de souffrir pour cela, après tout ce qui lui était arrivé~? Pourquoi y avait-il une partie de Harry qui semblait se mettre en colère contre le vieux sorcier au-delà de toute raison, qui semblait se déchaîner contre lui avec plus de force que Harry n'avait jamais frappé personne, sans la moindre tentative de modération une fois montée la rage qui s'effaçait à l'instant où Harry quittait sa présence~?

\emph{Est-ce parce que tu sais que Dumbledore ne se défendra pas~? Que quoi que tu lui dises, aussi injuste que ce soit, il n'utilisera jamais son pouvoir contre toi, il ne te traitera jamais comme tu le traites~? Est-ce ainsi que tu traites les gens quand tu sais qu'ils ne te frapperont pas s'il le peuvent~? Les gènes de brutes de James Potter qui se manifestent enfin~?}

Harry ferma les yeux.

Comme si le Choixpeau parlait dans sa tête…

\emph{Quelle est la véritable raison de ta colère~?}

\emph{De quoi as-tu peur~?}

Un tourbillon d'images sembla alors illuminer l'esprit de Harry, le Dumbledore passé qui sanglotait entre ses mains~; l'apparence actuelle du vieux sorcier, grand et terrible~; une vision de Hermione hurlant entre ses chaînes dans la chaise de métal alors que Harry l'abandonnait aux Détraqueurs, et l'image construite d'un femme aux longs cheveux blancs (avait-elle ressemblé à son mari?) qui tombait entre les flammes de sa chambre alors qu'une baguette était braquée vers elle et qu'une lumière orange se reflétait dans des lunettes en demi-lune.

Albus Dumbledore semblait avoir pensé que, face à ce genre de problèmes, Harry serait meilleur que lui.

Et Harry savait qu'il le serait probablement. Après tout, il en connaissait les équations.

Mais il était admis, sans que l'on sache comment, il était admis que les éthiciens utilitaristes ne cambriolaient pas \emph{vraiment} de banques pour pouvoir donner de l'argent aux pauvres. Le résultat final d'un rejet complet de toutes les contraintes éthiques n'était pas \emph{vraiment} du soleil, des roses et du bonheur pour tous. La prescription du conséquentialisme était de choisir l'action qui menait aux meilleures conséquences absolues, pas les actions qui avaient une conséquence positive et qui dévastaient tout le reste. Les maximiseurs d'utilité espérée avaient le droit de prendre le bon sens en compte lorsqu'ils calculaient leur espérance.

Harry était parvenu à comprendre cela, et ce avant même que quiconque ne le mette en garde. Avant d'avoir lu l'histoire de Vladimir Lenin ou l'histoire de la révolution française, il l'avait su. Cela avait peut-être été grâce à ses livres de science-fiction précocement lus, ses livres qui l'avaient mis en garde contre les gens pleins de bonnes intentions~; ou peut-être Harry avait-il vu la logique de lui-même. Il l'avait su depuis le début, que s'il s'écartait de son éthique \emph{à chaque fois qu'il avait une raison de le faire}, le résultat ne serait pas bon.

Une dernière image lui vint alors~: Lily Potter, debout devant le berceau de son bébé, en train de mesurer l'intervalle entre les résultats possibles~: le résultat final si elle restait et essayait de combattre l'ennemi (Lily morte, Harry mort) et le résultat final si elle partait (Lily vivante, Harry mort)~; en train de mesurer l'utilité attendue et de faire le seul choix raisonnable.

Si elle l'avait fait, elle aurait été la mère de Harry.

<<~Mais les êtres humains ne peuvent pas vivre comme ça, chuchotèrent les lèvres du garçon à l'adresse de la salle vide. Les humains ne peuvent pas vivre comme ça.~>> 

%  LocalWords:  hœnix
