\partchapter{Accomplissement de soi}{V}

\lettrine{I}{l} demeurait rare, même après avoir été directrice adjointe pendant trente ans et professeur de Métamorphose avant ça, de voir Albus Dumbledore pris complètement au dépourvu.

<<~… Susan Bones, Lavande Brown et Daphné Greengrass, conclut Minerva. Je dois aussi remarquer, Albus, que le récit de Mlle Granger sur votre attitude indifférente - je crois que sa phrase était 'il a dit que je devrai me contenter de n'être qu'un faire-valoir' - a généré beaucoup \emph{d'intérêt} chez les autres filles. Plusieurs d'entre elles sont venues me demander si les accusations de Mlle Granger étaient vraie, car cette dernière avait mentionné ma présence.~>>

Le vieux sorcier s'enfonça dans son immense fauteuil, les yeux toujours braqués sur elle, un air plutôt distrait derrière ses lunettes en demi-lune.

<<~Cela m'a mis dans un sacré dilemme, Albus~>>, dit le professeur McGonagall. Elle s'assura de maintenir une expression neutre. <<~Je sais que vous ne souhaitiez pas vraiment décourager la fille. Bien au contraire, en fait. Mais vous et Severus m'avez souvent dit que pour garder un secret je ne dois faire montre d'aucun signe qui différerait de la réaction d'une personne qui serait réellement dans l'ignorance. Je n'avais donc pas d'autre choix que de confirmer la véracité du témoignage de Mlle Granger et de feindre le degré d'inquiétude adapté, accompagnée d'un ton légèrement offensé. Après tout, si je n'avais \emph{pas} su que vous essayiez délibérément de manipuler Mlle Granger, j'aurais été passablement énervée.

--- Je… vois~>>, dit lentement le vieux sorcier. Sa main joua distraitement avec sa barbe d'argent par petits gestes rapides.

<<~Heureusement, continua le professeur McGonagall, pour l'instant, les seuls membres du personnel à porter les badges de Mlle Granger sont les professeurs Sinistra et Vector.

--- Badges~?~>> répéta le vieux sorcier.

Minerva exhiba un petit disque d'argent qui portait les initiales SPEHS, le posa sur le bureau d'Albus et le frappa brièvement du doigt.

Et les voix de Hermione Granger, Padma Patil, Parvati Patil, Lavande Brown, Susan Bones, Hannah Abbott, Daphné Greengrass et Tracey Davis s'écrièrent à l'unisson~:

<<~\emph{Les sorcières sont pas en reste, c'est l'heure de nous donner une quête~!}

--- Mlle Granger les vend pour deux Mornilles et m'a fait savoir qu'elle en a pour le moment écoulées cinquante. Je crois que Nymphadora Tonks, une Poufsouffle en septième année, les enchante pour elle. Enfin, dit le professeur McGonagall d'une voix brusque, nos huit nouvelles héroïnes ont demandé l'autorisation de manifester devant l'entrée de votre bureau.

--- J'espère, dit Albus en fronçant les sourcils, que vous leur avez expliqué que -

--- Je leur ai expliqué que mercredi à 19h conviendrait très bien~>>, dit Minerva. Elle reprit le badge du bureau du directeur, offrit à Albus un sourire mielleux et se tourna vers la porte.

<<~Minerva~? dit le vieux sorcier derrière elle. \emph{Minerva~!}~>>

La porte de chêne se referma solidement dans son sillage.

\later

Il n'y avait pas beaucoup de place entre les étroits murs de pierre qui délimitaient l'entrée du bureau du directeur, si bien que même si beaucoup de gens avaient voulu regarder la manifestation, peu avaient reçu l'autorisation de venir. Seuls étaient présents les professeur Sinistra et Vector, qui portaient des badges, et les préfets Pénélope Deauclaire, Rose Brown et Jacqueline Preece, qui les portaient aussi. Derrière \emph{elles}, les professeurs McGonagall, Chourave et Flitwick, qui ne portaient pas de badges, surveillaient la situation. Harry Potter et le président des élèves de Poudlard étaient présents, ainsi que les préfets mâles Percy Weasley et Oliver Beatons, tous porteurs de badges en signe de Solidarité. Et bien sûr, les huit fondatrices de SPEHS, qui formaient un piquet à côté de la gargouille, armées de pancartes. Celle de Hermione, attachée à un bâton de bois qui lui semblait de plus en plus lourd à mesure que les secondes s'écoulaient portait l'inscription~: FAIRE-VALOIR DE PERSONNE.

Et le professeur Quirrell, le dos appuyé contre le mur de pierre, les observait d'un regard impénétrable. Il avait obtenu un badge alors que Hermione ne lui en avait jamais vendu un, et au lieu de le porter il le faisait rebondir dans une de ses mains.

Tout cela avait semblé être une bien meilleure idée quatre jours plus tôt, lorsque le feu de l'indignation avait été encore brûlant et qu'elle avait fait face à la perspective de manifester quatre jours \emph{plus tard} et non pas \emph{tout de suite}.

Mais elle devait continuer, parce que c'était ce que les héros faisaient, ils continuaient, et aussi parce que cela lui avait semblé infiniment trop horrible de dire à tout le monde qu'elle annulait. Elle se demanda combien d'actions héroïques avaient continué pour ce genre de raisons. La plupart des livres ne \emph{disaient} pas~: <<~Et ils refusèrent d'abandonner, peu importe à quel point cela aurait été sensé, car cela aurait été trop embarrassant~>>, mais une grande partie de l'Histoire s'expliquait beaucoup mieux vu sous cet angle.

À 19h15, selon les dires du professeur McGonagall, Dumbledore descendrait et leur parlerait pendant quelques minutes. Le professeur McGonagall avait dit de ne pas avoir peur - le directeur était quelqu'un de gentil, au fond, et elles avaient dûment obtenu l'autorisation de l'école avant de manifester.

Mais Hermione était parfaitement consciente du fait que même si elle le faisait avec une autorisation signée, elle Défiait quand même l'Autorité.

Après avoir décidé d'être une héroïne, Hermione avait fait la seule chose qu'il y avait à faire~: aller à la bibliothèque de Poudlard et prendre des livres sur comment être une héroïne. Puis elle avait rendu ces livres à leurs étagères, car il avait été manifestement évident qu'aucun des auteurs n'avaient eux-mêmes été des héros. Au lieu de cela, elle avait relu cinq fois, jusqu'à connaître chaque mot par cœur, les trois quarts de mètre de parchemin qui constituaient l'entière autobiographie de Godric Gryffondor accompagnée de ses conseils (la traduction anglaise, du moins~; elle ne savait pas encore lire le Latin). L'autobiographie de Godric Gryffondor avait été bien plus \emph{compacte} que les autres livres que Hermione avait l'habitude de lire, il utilisait \emph{une phrase} pour dire des choses qui auraient dû prendre près d'un mètre à elles toutes seules, et ensuite, il y avait une \emph{autre} phrase…

Mais il était clair, d'après ses lectures, que même si Défier l'Autorité n'était pas le \emph{but} d'un héros, on ne pouvait pas en être un si on avait trop peur de celle-ci. Et Hermione Granger savait alors déjà ce que les autres voyaient quand ils la regardaient et ce dont ils la croyaient incapable.

Elle leva sa pancarte un peu plus haut et se concentra pour respirer de façon lente et rythmée au lieu d'hyperventiler jusqu'à l'écroulement.

<<~\emph{Vraiment~?} dit Mlle Preece avec une fascination non dissimulée. Elles ne pouvaient pas \emph{voter}~?

--- En effet~>>, dit le professeur Sinistra. (Les cheveux du professeur d'Astronomie étaient encore noirs et sa peau noire n'était que légèrement ridée~; Hermione aurait \emph{deviné} qu'elle avait environ soixante-dix ans, sauf que…) <<~Je me souviens bien que ma mère s'était réjouie quand ils ont annoncé l'acte de qualification des femmes, même si elle ne correspondait pas aux critères de qualification.~>> (Ce qui voulait dire que le professeur Sinistra avait été avec sa famille moldue en 1918). <<~Et ce n'était pas le pire. Allons, seulement quelques siècles plus tôt -~>>

Trente secondes plus tôt tous les non-nés-Moldus, mâles et femelles, regardaient le professeur Sinistra avait un air profondément choqué. Hannah fit tomber sa pancarte.

<<~Et \emph{ça} n'était pas le pire non plus, loin de là, conclut le professeur Sinistra. Mais vous voyez où ce genre de chose peut potentiellement mener.

--- Merlin nous protège, dit Pénélope Deauclaire d'une voix étranglée. Vous voulez dire que c'est comme \emph{ça} que les hommes nous traiteraient si nous n'avions pas de baguettes pour nous défendre~?

--- \emph{Hé~!} dit l'un des préfets mâles. \emph{Ça} n'est pas -~>>

Il y eut un court rire sardonique venu du professeur Quirrell. Lorsque Hermione tourna sa tête elle vit que ce dernier jouait toujours avec le badge et, sans même se fatiguer à lever les yeux vers eux, il dit~:

<<~Telle est la nature humaine, Mlle Deauclaire. Soyez assurée que \emph{vous} ne seriez en rien plus douce si les sorcières avaient des baguettes et que les hommes n'en avaient pas.

--- J'en doute fort~!~>> lâcha le professeur Sinistra.

Un gloussement froid.

<<~Je soupçonne que cela se produise plus souvent que personne n'ose le laisser entendre dans les plus fières des familles Sang Pur. Une sorcière esseulée repère un beau Moldu et songe à quel point il serait simple de lui glisser un philtre d'amour, d'être adorée par lui, exclusivement et absolument. Et puisqu'elle sait qu'il ne peut faire montre d'aucune résistance, allons, il est tout naturel qu'elle prenne de lui ce qu'elle désire -

--- \emph{Professeur Quirrell~!} dit le professeur McGonagall d'une voix coupante.

--- Pardon~>>, dit le professeur Quirrell d'une voix douce, les yeux toujours baissés vers le badge dans sa main, <<~continuons-nous tous à prétendre que cela n'arrive jamais~? Dans ce cas je vous présente mes excuses.~>>

Le professeur Sinistra rétorqua~:

<<~Et j'imagine que les sorciers ne -

--- Des \emph{enfants} sont présents, professeurs~! Encore le professeur McGonagall.

--- Certains le font~>>, dit le professeur d'une voix calme comme pour discuter du temps qu'il faisait. <<~Quoique personnellement, j'évite.~>>

Il y eut un assez long silence. Hermione releva sa pancarte - elle avait glissé jusqu'à son épaule pendant qu'elle écoutait. Elle n'y avait jamais songé, même pas un petit peu, maintenant elle essayait de ne \emph{pas} y penser et son estomac était un peu barbouillé. Elle regarda en direction Harry Potter sans bien savoir pourquoi elle le faisait et elle vit que le visage de ce dernier était parfaitement immobile. Un frisson parcourut son échine avant qu'elle ne détourne le regard car elle ne fut pas tout à fait assez rapide pour manquer le petit hochement de tête que Harry fit à son intention, comme s'ils venaient de tomber d'accord sur quelque chose.

<<~Pour être honnête, dit le professeur Sinistra après un moment, je ne peux pas me souvenir avoir fait face au moindre préjugé dû à mon sexe ou à ma couleur de peau depuis que j'ai reçu ma lettre de Poudlard. Non, aujourd'hui c'est toujours parce que je suis une née-Moldue. Je crois que Mlle Granger a dit que pour le moment le problème était lié \emph{uniquement} aux héroïnes~?~>>

Hermione mit un moment à se rendre compte qu'on lui avait posé la question et elle répondit alors <<~oui~>> en couinant légèrement. Toute cette histoire avait pris des proportions bien plus grandes que ce qu'elle avait imaginé au moment de commencer.

<<~Où exactement avez-vous pris vos renseignements, Mlle Granger~?~>> dit le professeur Vector. Elle avait l'air plus vieille que le professeur Sinistra et ses cheveux commençaient à grisonner un peu~; Hermione ne s'était jamais retrouvée à proximité du professeur d'Arithmancie avant que celle-ci ne vienne lui demander un badge.

<<~Euh, dit Hermione d'une voix légèrement aiguë, j'ai été voir dans les livres d'Histoire et il y a eu autant de femmes que d'hommes ministres de la Magie. Puis j'ai été voir les Manitous Suprêmes et il y a eu un peu plus de sorciers que de sorcières mais pas beaucoup. Mais si vous cherchez des chasseurs de Seigneurs des Ténèbres connus ou des gens qui ont empêché des invasions de créatures maléfiques ou des gens qui ont renversé des Seigneurs des Ténèbres -

--- Et des Seigneurs des Ténèbres eux-mêmes, bien sûr,~>> dit le professeur Quirrell. \emph{Cette fois} il avait levé les yeux. <<~Vous pouvez les ajouter à votre liste, Mlle Granger. Parmi tous les Mangemorts présumés on ne peut trouver que deux sorcières, Bellatrix Black et Alecto Carrow. Et je crois que la plupart des sorciers auraient bien du mal à mentionner une seule Dame des Ténèbres à l'exception de Baba Yaga.~>>

Hermione se contenta de le fixer du regard.

Il ne pouvait pas \emph{vraiment} vouloir -

<<~Professeur Quirrell, dit le professeur Vector, qu'insinuez-vous exactement~?~>>

Le professeur de Défense éleva le badge afin que les lettres SPEHS soient face à elles et dit~: <<~Héros~>>, puis il retourna le badge pour montrer son fond argenté et dit~:

<<~Mages Noirs. Ce sont des carrières similaires choisies par des gens similaires, et on peut difficilement demander pourquoi les jeunes sorcières se détournent d'une de ces voies sans envisager son envers.

--- Oh, je comprends \emph{maintenant}~! dit Tracey Davis si soudainement que Hermione eut un petit sursaut. Vous vous joignez à notre manifestation parce que vous avez peur que pas assez de filles ne deviennent des mages noires~!~>> Puis Tracey gloussa, ce que Hermione aurait été incapable de faire à un moment pareil même si vous l'aviez payée un million de livres sterling.

Lorsque le professeur répondit, on put voir un demi-sourire sur son visage, <<~pas vraiment, Mlle Davis. À vrai dire, ce genre de choses m'est profondément égal. Mais il est futile de compter les sorcières parmi les ministres de la Magie et autres gens ordinaires vivant des existences ordinaires alors que Grindelwald, Dumbledore et Celui-Dont-Il-Ne-Faut-Pas-Prononcer-Le-Nom étaient tous des hommes.~>> Les doigts du professeur de Défense retournaient distraitement le badge, encore et encore. <<~Mais enfin, seules quelques rares personnes font jamais quoi que ce soit d'intéressant de leurs vies. Quelle importance pour vous si \emph{ils} sont en majorité sorciers ou sorcières tant que \emph{vous} n'en faites pas partie~? Et je soupçonne que ce ne sera pas le cas, Mlle Davis~; car même si vous êtes ambitieuse, vous n'avez aucune ambition.

--- \emph{Ce n'est pas vrai~!}  dit Tracey, indignée. Et puis qu'est-ce que ça veut dire~?~>>

Le professeur Quirrell se redressa et s'écarta du mur contre lequel il s'était appuyé. <<~Vous avez été répartie à Serpentard, Mlle Davis, et je m'attends à ce que vous saisissiez la moindre opportunité d'être promue qui tombera entre vos mains. Mais vous n'êtes mue par la volonté d'accomplir aucune grande ambition en particulier et vous ne \emph{créerez pas} vos opportunités. Au mieux, vous vous hisserez jusqu'aux hauteurs du ministère de la Magie ou à une autre position élevée et dénuée d'importance, sans jamais dépasser les limites de votre existence.~>>

Le regard du professeur Quirrell s'écarta alors de Tracey et la regarda \emph{elle}, ses yeux bleus pâles braqués sur elle, exerçant une terrible force -

<<~Dites-moi, Mlle Granger. Avez-\emph{vous} une ambition~?

--- Professeur -~>> couina la voix perchée du professeur Flitwick, puis le directeur de sa maison se tut et du coin de l'œil elle put voir que Harry avait posé sa main sur l'épaule du professeur Flitwick et secouait la tête, une expression très adulte sur le visage.

Hermione eut l'impression d'être un cerf pris dans les phares d'une voiture.

<<~Qu'est-ce qui vous a poussé à dépasser vos limites, Mlle Granger~? dit le professeur de Défense en la fixant toujours. Pourquoi cela ne vous suffit-il plus d'avoir de bonnes notes~? Est-ce la véritable grandeur que vous recherchez~? Un aspect du monde vous rend-il suffisamment insatisfaite pour que vous dussiez le remodeler conformément à votre volonté~? Ou tout cela n'est-il pour vous qu'un jeu puéril~? Je serais fort déçu si cela s'avérait n'être qu'une conséquence de votre rivalité avec Harry Potter.

--- Je -~>> dit Hermione, sa voix si aiguë qu'elle en devint presque un pépiement, mais elle ne sut alors quoi dire.

<<~Vous pouvez prendre un moment pour réfléchir si vous le souhaitez, dit le professeur Quirrell. Dites-vous que c'est un devoir à rendre pour jeudi. J'ai entendu dire que vous étiez assez éloquente à l'écrit.~>>

Tout le monde la regardait.

<<~Je… dit Hermione. Je suis en désaccord avec absolument tout ce que vous avez dit.

--- Bien dit~>>, lui parvint la voix brusque du professeur McGonagall.

Le regard du professeur Quirrell ne vacilla pas. <<~Cela n'est pas assez développé pour un devoir à rendre, Mlle Granger. \emph{Quelque chose} vous pousse à défier le verdict du directeur et à rassembler des partisans. Peut-être s'agit-il de quelque chose que vous préféreriez garder sous silence~?~>>

Hermione savait que la bonne réponse n'impressionnerait pas le professeur Quirrell mais c'était quand même la bonne réponse et elle la dit~: <<~je ne pense pas qu'il y ait besoin d'être ambitieux pour devenir un héros.~>> Sa voix vacilla mais ne se brisa pas. <<~Je pense qu'il suffit de faire ce qui est juste. Et ce ne sont pas mes partisans, ce sont mes amis.~>>

Le professeur Quirrell se rappuya contre le mur. Le demi-sourire avait quitté son visage. <<~La plupart des gens se disent qu'ils font ce qui est juste, Mlle Granger. Ils ne sortent pas pour autant de l'ordinaire.~>>

Hermione prit deux profondes inspirations et essaya d'être courageuse. <<~Il ne \emph{s'agit pas} de sortir de l'ordinaire, dit-elle aussi vaillamment qu'elle en était capable. Mais si quelqu'un essaie seulement de faire ce qui est juste, encore et encore, et qu'il n'est pas trop paresseux pour faire le travail requis, qu'il réfléchit à ce qui est juste et ce qui ne l'est pas, qu'il est assez courageux pour le faire même quand il a peur -~>> Hermione s'interrompit un instant et ses yeux passèrent sur Tracey et Daphné <<~- et qu'il est assez intelligents pour trouver le moyen d'accomplir ses buts - et qu'il ne fait pas seulement ce que les autres lui disent de faire - alors je pense que quelqu'un comme ça rencontrerait déjà bien assez d'ennuis.~>>

Quelques-uns des garçons et des filles gloussèrent, y compris le professeur McGonagall, qui avait l'air à la fois narquoise et fière.

<<~Vous avez peut-être raison~>>, dit le professeur de Défense, ses yeux mis-clos. Il jeta le badge à Hermione et elle l'attrapa sans y penser. <<~Mon don à votre cause, Mlle Granger. J'ai cru comprendre qu'il valait deux Mornilles.~>>

Le professeur de Défense fit demi-tour et s'en fut sans un mot de plus.

<<~J'ai cru que j'allais m'évanouir~!~>> haleta Hannah après que les bruits de pas du professeur eurent disparus, et Hermione entendit d'autres filles enfin se permettre d'expirer ou de poser leur pancarte pendant un moment.

<<~Moi \emph{aussi} j'ai une ambition~!~>> dit Tracey, qui semblait être au bord des larmes. <<~Je - je - demain j'aurais trouvé ce que c'est, mais j'en ai une, j'en suis sûre~!
--- Si tu ne peux vraiment rien trouver, dit Daphné en donnant une tape affective sur l'épaule de Tracey, prends la bonne vieille ambition de conquérir le monde.

--- Hé~! dit Susan d'un ton sec. Vous êtes censées être des héroïnes maintenant~! Ça veut dire que vous devez être \emph{bonnes~!}

--- Pas de problème, dit Lavande, je suis quasiment sûre que le général Chaos veut conquérir le monde et c'est plutôt un chic type.~>>

D'autres conversations s'engagèrent parmi les spectateurs. <<~Eh bien, dit Pénélope Deauclaire. Je pense que c'est le professeur de Défense le plus \emph{ouvertement} maléfique qu'on ait jamais eu.~>>

Le professeur McGonagall toussa en signe d'avertissement, le président des élèves dit~: <<~Tu n'étais pas là à l'époque du professeur Barney~>>, et plusieurs personnes tiquèrent.

<<~Le professeur Quirrell \emph{parle} juste comme ça~>>, dit Harry Potter, même s'il avait l'air moins sûr de lui qu'avant. <<~Enfin, réfléchissez, il ne \emph{fait} rien qui se rapproche de ce que Rogue fait -

--- M. Potter~>>, couina le professeur Flitwick, sa voix polie et son visage sévère, <<~pourquoi m'avez-vous demandé de rester silencieux~?

--- Le professeur Quirrell testait Hermione pour voir s'il voulait être son vieux sorcier mystérieux, dit Harry. Ce qui n'aurait jamais, jamais fonctionné, mais elle devait répondre elle-même.~>>

Hermione cligna des yeux.

Et cligna de nouveau en se rendant compte que le vieux sorcier mystérieux de Harry était le professeur Quirrell, que ce n'était pas Dumbledore du tout, et que ce n'était \emph{vraiment pas bon signe} -

Un grondement emplit le petit vestibule de pierre, et Hermione, qui était déjà sur les nerfs, pivota rapidement et fit presque tomber sa pancarte lorsque sa main fonça vers sa baguette.

Les gargouilles s'écartaient, la Pierre Fluide grondant comme de la pierre au rythme des déplacements de sa chair. Les immenses et horribles créatures n'attendirent qu'un bref instant, leurs yeux morts et gris braqués droit devant elles tels des vigies silencieuses. Puis les grandes gargouilles replièrent leurs ailes et reprirent leur position initiale sans que la Pierre Fluide ne change en aucune façon d'apparence lors du passage de la souplesse à l'immobilité, et le trou momentané dans la pierre de Poudlard fut de nouveau remplit.

Et devant eux, vêtus de robes violet vif qui n'étaient probablement hideuses qu'aux yeux d'un né-Moldu, se tenait l'immense silhouette d'Albus Percival Wulfric Brian Dumbledore, directeur de Poudlard, sorcier en chef du Magenmagot, Manitou suprême de la confédération internationale des sorciers, vainqueur du Seigneur des Ténèbres Grindelwald, protecteur de l'Angleterre, redécouvreur des légendaires douze usages du sang de dragon et plus puissant sorcier en vie~; et il la regardait \emph{elle}, Hermione Jean Granger, général du récemment agrandi régiment Soleil, qui avait les meilleures notes de première année de tout Poudlard et qui s'était récemment déclarée héroïne.

Même son \emph{nom} était plus court que le sien.

Le directeur lui sourit avec bienveillance, ses yeux ridés pétillants de joie derrières ses lunettes en demi-lune et dit~: <<~Bonjour, Mlle Granger.~>>

Ce qui était bizarre, c'était que c'était loin d'être aussi effrayant que de parler au professeur Quirrell. <<~Bonjour, monsieur le directeur,~>> dit Hermione sans autre chose qu'un léger chevrotement dans la voix.

<<~Mlle Granger,~>> dit Dumbledore, et il avait l'air maintenant plus sérieux, <<~je pense que vous et moi nous sommes assez mal compris. Je ne souhaitais pas sous-entendre que vous ne pouviez pas ni ne deviez pas être une héroïne. Je n'ai certainement pas insinué que les sorcières en général ne devraient pas être des héroïnes. Seulement que vous étiez… un peu jeune pour penser à ce genre de choses.~>>

Hermione, incapable de s'en empêcher, regarda le professeur McGonagall et vit que celle-ci lui donnait un sourire encourageant - ou en tout cas qu'elle leur \emph{souriait} à eux deux - et Hermione revint donc au directeur et dit, le petit chevrotement cette fois plus important~:

<<~Depuis que vous êtes devenu directeur voilà quarante ans, onze personnes passées par Poudlard sont devenues des héros, je parle de gens comme Lupe Cazaril et d'autres, et \emph{dix} d'entre elles étaient des garçons. Cimorene Linderwall était la seule sorcière.

--- Hmm~>>, dit le directeur. Son visage était pensif~; il avait au moins \emph{l'air} d'y réfléchir. <<~Mlle Granger, ça n'a jamais été mon genre de comparer ce genre de chiffres. Il est souvent bien plus simple de compter que de comprendre. Nombre de gens bons sont sortis de Poudlard, sorcières et sorciers~; et ceux célébrés pour leur héroïsmes ne sont qu'un type de bonne personne, et peut-être pas du type le plus élevé qui soit. Vous n'avez inclus ni Alice Londubat ni Lily Potter dans votre estimation… mais laissons cela. Dites-moi, Mlle Granger, avez-vous compté le nombre de héros à sortir de Poudlard pendant les quarante années qui m'ont précédées~? Car de cette époque je ne puis compter que seules trois personnes aujourd'hui considérées comme des héros~; et parmi ces trois, aucune sorcière.

--- Je n'essaie pas de dire que c'est \emph{seulement} vous~! dit Hermione. Seulement, je pense que \emph{beaucoup} de gens, comme les directeurs avant vous, peut-être même toute votre société et tout ça, découragent peut-être les filles.~>>

Le vieux sorcier soupira. Ses yeux en demi-lunettes ne regardaient qu'elle, comme s'ils étaient les deux seules personnes présentes.

<<~Mlle Granger, il est peut-être possible de dissuader des sorcières de devenir professeurs, joueuses de Quidditch ou même Aurors. Mais pas héroïnes. Si quelqu'un est censé devenir un héros ou une héroïne, alors il ou elle le sera. Ces personnes seraient prêtes à traverser des incendies et à nager dans de la glace. Ni les Détraqueur ni la mort de leurs amis ne les arrêterait, pas plus que la dissuasion.

--- Eh bien~>>, dit Hermione, et elle s'interrompit, luttant pour trouver ses mots. <<~Eh bien, enfin… et si ce n'est pas \emph{vraiment} vrai~? Enfin, de \emph{mon point de vue} il semble que si on veut que plus de sorcières deviennent des héroïnes, on devrait mieux leur apprendre à se héroïfier.

--- Beaucoup de garçons et de filles rêvent d'être des héros~>>, dit Dumbledore à voix basse. Il ne regardait qu'elle et aucune des autres filles. <<~Ceux qui continuent une fois éveillés sont moins nombreux. Beaucoup ont tenu bon et se sont battus lorsque le Mal est venu les chercher. Ceux qui sont allés au devant du Mal et l'ont forcé à se défendre sont moins nombreux. C'est une vie difficile, parfois solitaire et souvent courte. Je n'ai dit à personne d'ignorer cet appel, mais je ne souhaiterais pas non plus que leur nombre augmente.~>>

Hermione hésita~; quelque chose sur ce visage ridé l'arrêtait, comme une allusion à toute l'émotion qui était demeurée masquée, à toutes les années…

\emph{Peut-être que s'il y avait plus de héros, leur vie ne serait pas si solitaire ni si courte.}

Mais elle ne parvint pas à le dire~; pas à lui.

<<~Enfin, ce débat est stérile~>>, dit le vieux sorcier. Son sourire parut un peu triste à Hermione. <<~Mlle Granger, on ne peut enseigner l'héroïsme comme on enseignerait les Charmes. On ne peut demander un devoir sur comment continuer quand tout espoir semble perdu. On ne peut inculquer par routine quand le moment est venu de se lever et de dire au directeur qu'il a mal agi. Les héros sont nés, pas éduqués. Et pour une raison ou une autre, ce sont plus souvent des garçons que des filles.~>> Le directeur haussa les épaules comme pour dire qu'\emph{il} ne pouvait rien y faire.

<<~Euh~>>, dit Hermione. Elle ne put s'empêcher ne regarder derrière elle.

Le professeur Sinistra avait l'air un peu indignée. Et \emph{non}, tout le monde ne la regardait pas comme si elle s'était comportée comme une idiote, ce qu'elle avait commencé à s'imaginer en écoutant Dumbledore lui parler.

Hermione se tourna de nouveau pour faire face à Dumbledore, prit une profonde inspiration et dit~:

<<~Eh bien, peut-être que les gens qui sont destinés à devenir des héros le deviendront quoi qu'il arrive. Mais je ne vois pas comment quiconque pourrait vraiment \emph{savoir} ça plutôt que de juste l'affirmer après-coup. Et quand \emph{je} vous ai dit que je voulais être une héroïne, vous n'avez pas été très encourageant.

--- M. Potter~>>, dit le directeur avec douceur. Ses yeux ne quittèrent pas ceux de Hermione. <<~Dites s'il vous plaît à Mlle Granger votre impression sur notre première rencontre. Diriez-vous que j'étais encourageant~? Répondez sincèrement.~>>

Il y eut un silence.

<<~M. Potter~? dit la voix perplexe du professeur Vector derrière Hermione.

--- Euh~>>, dit la voix de Harry, venue d'un peu plus loin, extrêmement réticente. <<~Euh, enfin, à vrai dire dans mon cas le directeur a mis le feu à un poulet.

--- Il a \emph{quoi}~?~>> lâcha Hermione, sauf que plusieurs autres personnes s'étaient exclamées à peu près au même moment si bien qu'elle n'était pas certaine qu'on l'ait entendue.

Dumbledore la regardait, l'air parfaitement sérieux.

<<~Je n'étais pas au courant pour Fumseck, dit rapidement la voix de Harry, alors il m'a dit que Fumseck était un phénix tout en me montrant un poulet sur le perchoir de Fumseck pour que je pense que ce poulet \emph{était} Fumseck et ensuite il a mis le feu au poulet - et il m'a aussi donné ce gros rocher et il m'a dit qu'il appartenait à mon père et que je devrai le transporter partout où j'allais -

--- Mais c'est de la \emph{folie}~!~>> lâcha Susan.

Tout le monde se tut immédiatement.

Le directeur pivota lentement la tête pour regarder Susan.

<<~Je… dit Susan. Je veux dire - je -~>>

Le directeur se pencha jusqu'à se retrouver à quelques centimètres du visage de la jeune fille.

<<~Je n'ai pas…~>> dit-elle.

Dumbledore mit un doigt entre ses lèvres et les remua de haut en bas, \emph{blebleblebleblebleble}.

Puis il se redressa et dit~: <<~Eh bien, mes chères héroïnes, il m'a été agréable de vous parler, mais malheureusement il me reste beaucoup à faire aujourd'hui. Cela dit, soyez assurées que je suis aussi énigmatique pour le reste du monde que pour les sorcières.~>>

La gargouille fit un pas de côté, la Pierre Fluide grondant comme de la pierre au rythme du mouvement de sa chair.

Les immenses et horribles silhouettes attendirent brièvement, leurs yeux gris et morts braqués droit devant elles tels des vigies silencieuses, tandis qu'Albus Percival Wulfric Brian Dumbledore, souriant d'un air aussi bienveillant que lorsqu'il avait émergé de son bureau, remontait dans l'Enchantement des Escaliers Infinis.

Puis les grandes gargouilles refermèrent leurs ailes, reprirent leur position initiale et seul un dernier bref <<~Bah-ha-ha~!~>> fit écho avant que l'ouverture ne se referme.

Il y eut un long silence.

<<~Il a \emph{vraiment} mis le feu à un poulet~?~>> dit Hannah.

\later

Elles avaient ensuite continué leur manifestation, mais le cœur n'y était plus.

Il \emph{avait} été établi, suite à de minutieuses questions du professeur Flitwick, que Harry Potter n'avait pas senti le poulet brûler. Ce qui voulait dire que ça avait probablement été un galet ou quelque chose d'approchant, métamorphosé en poulet puis enfermé dans un sortilège d'enclos afin qu'aucune fumée ne s'échappe - mais le professeur Flitwick et le professeur McGonagall avaient catégoriquement exigés que personne ne s'y essaie sans leur supervision.

Mais quand même…

Mais quand même… quoi~?

Hermione ne \emph{savait} même pas quoi.

Mais \emph{quand même}.

Après que de nombreux regards eurent été échangés entre des filles dont aucune ne voulaient être la première à le dire, Hermione avait déclaré que la manifestation était finie et les adultes et les garçons étaient partis.

<<~Tu ne penses quand même pas qu'on a été injustes envers Dumbledore~?~>> dit Susan alors que les héroïnes s'éloignaient au bruit de huit paires de pieds foulant la pierre des couloirs de Poudlard. <<~Enfin, s'il \emph{est} fou avec tout le monde et pas seulement avec les sorcières, ce n'est pas de la discrimination, si~?

--- Je ne veux plus manifester contre le directeur~>>, dit faiblement Hannah. La Poufsouffle avait l'air un peu instable. <<~Je me fiche que le professeur McGonagall dise qu'il ne nous en veut pas, mes nerfs ne peuvent pas le supporter.~>>

Lavande pouffa.

<<~J'imagine que \emph{tu} ne vas pas pourfendre des armées d'Inferi de sitôt -

--- Arrêtez~! dit Hermione d'un ton sévère. Écoutez, on doit toutes \emph{apprendre} à être des héroïnes, d'accord~? Ce n'est pas grave si quelqu'un ne sait pas tout de suite comment l'être.

--- Le directeur ne pense pas que ça \emph{puisse} être appris~>>, dit Padma. Le visage de la Serdaigle était pensif et ses pas sur le couloir étaient mesurés. <<~Il ne pense même pas que ce soit une bonne idée.~>>

Daphné marchait à grandes enjambées, le dos droit et la tête levée, et elle ressemblait plus à une Jeune Fille Convenable dans ses robes de Poudlard que Hermione n'aurait jamais pu l'être dans sa meilleure robe de soirée.

<<~Le directeur~>>, dit Daphné d'une voix précise alors que ses chaussures frappaient la pierre avec force, <<~pense que nous ne sommes qu'un ramassis de petites idiotes qui s'amusent, qu'un jour Hermione pourra être un bon faire-valoir mais que nous sommes sans espoir.

--- Est-ce qu'il a \emph{raison}~?~>> dit Parvati. Le visage de la Gryffondor était très sérieux, ce qui la faisait plus ressembler à sa jumelle que d'habitude. <<~Je pense qu'il faut que la question soit posée -

--- \emph{Non~!}~>> cracha Tracey. La Serpentard traversait le couloir avec l'air d'être prête à \emph{tuer} quelqu'un comme une Rogue femelle miniature. De toutes les filles, Tracey était celle que Hermione connaissait le moins. Hermione avait parlé à Lavande une fois auparavant mais elle n'avait jamais vraiment \emph{vu} Tracey, sauf au bout de sa baguette pendant les batailles, jusqu'au moment où la Serpentard avait bondi de son canapé pour se porter volontaire. <<~On leur fera voir~! On leur fera voir à \emph{tous}~!

--- OK, dit Susan, ça c'était \emph{clairement} maléfique -

--- Non, dit Lavande, c'est une devise de la légion du Chaos. Sauf qu'elle n'a pas fait le rire dément.

--- C'est vrai, dit Tracey d'une voix basse et lugubre. Cette fois je ne rigole pas.~>> La fille continua de rôder dans le couloir comme si une musique dramatique qu'elle seule pouvait entendre l'accompagnait.

(Hermione commençait à se demander ce que Harry Potter enseignait \emph{exactement} aux jeunes et influençables membres de la légion du Chaos).

<<~Mais - enfin -~>> dit Parvati. Elle avait toujours cet air contemplatif. <<~Ce que je veux dire c'est qu'on peut comprendre \emph{pourquoi} le directeur penserait qu'on n'est qu'un ramassis d'idiotes. Quel est le rapport entre être une héroïne et manifester devant le bureau du directeur~?

--- Ha~>>, dit Lavande, et elle eut l'air pensive à son tour. <<~C'est vrai. Nous devrions faire quelque chose d'héroïque. Je veux dire d'héroïnique.

--- Euh -~>> dit Hannah, ce qui exprimait très bien les pensées de Hermione sur le sujet.

<<~Eh bien, dit Parvati, est-ce que tout le monde ici a déjà traversé le couloir du troisième étage de Dumbledore~? Parce que tous les Gryffondor l'ont déjà fait -

--- \emph{Attendez~!} dit Hermione avec désespoir. Je ne veux pas que vous fassiez quoi que ce soit de \emph{dangereux}~!~>>

Il y eut un silence pendant lequel tout le monde regarda Hermione, qui se rendait compte, bien trop tard, pourquoi Dumbledore n'avait pas voulu que qui que ce soit \emph{d'autre} devienne un héros.

<<~Je ne pense pas qu'on puisse devenir une héroïne si on ne fait jamais rien de dangereux~>>, dit Lavande, ce qui était raisonnable.

<<~Et puis, dit Padma avec l'air de penser à voix haute, tout le monde dit que rien de vraiment grave n'arrive jamais à Poudlard. Aux élèves je veux dire, pas aux professeur de Défense. On a toutes ces anciennes protections.

--- Euh - répéta Hannah;

--- Ouais, dit Parvati, le pire qui puisse arriver c'est qu'on perde une petite douzaine de points, et comme nous sommes deux de chaque maison \emph{ça} sera équitable.

--- Mais c'est \emph{génial}, Hermione~! dit Daphné avec le plus grand émerveillement. Tu l'as conçu de façon à ce qu'on puisse faire \emph{ce qu'on veut} en toute impunité~! Et je n'avais même pas remarqué ton plan retors jusqu'à maintenant~!

--- \emph{EUH -}~>> dirent Hermione, Hannah et Susan.

<<~C'est ça~! dit Parvati. Donc maintenant il est temps pour nous de devenir de véritables héroïnes. Nous irons chercher le Mal -

--- Et \emph{le} forcerons à \emph{nous} faire face - dit Lavande.

--- Et lui apprendrons à avoir peur~>>, dit Tracey Davis d'un ton lugubre
%  LocalWords:  ven Preece Cazaril Cimorene Linderwall Bwa heroinic
