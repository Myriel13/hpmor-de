\chapter{Un jour à très faible probabilité}

\lettrine{C}{haque} centimètre carré d'espace mural est caché derrière un pan de bibliothèque. Chaque bibliothèque a six étagères et atteint presque le
plafond. Certaines étagères sont pleines à ras bord de manuels~: science, mathématiques, histoire et ainsi de suite. D'autres ont deux rangées chacune de livres de science-fiction, et les rangées du fond sont juchées sur de vieilles boîtes de mouchoirs\footnotemark{} ou de petites planches de bois pour que l'on puisse les voir. Mais cela ne suffit pas. D'autres livres débordent des tables et des canapés~; d'autres encore forment de petits tas sous les fenêtres.
\authorsnotetext{Je fais cela chez moi.}

C'est le salon de la maison où vivent l'éminent professeur Michael Verres-Evans, sa femme Mme Pétunia Evans-Verres, et leur fils adoptif~: Harry James Potter-Evans-Verres.

Il y a une lettre sur la table du salon, et une enveloppe sans timbre faite d'un parchemin jaunâtre, adressées à un M. H. Potter d'une encre vert-émeraude.

Le professeur et sa femme discutent sèchement, mais ils ne crient pas. Le professeur trouverait cela fort peu civilisé.

<<~Tu plaisantes,~>> dit Michael à Pétunia. Son ton indiquait qu'il avait bien peur que non.

<<~Ma s\oe{}ur était une sorcière,~>> répéta-t-elle. Elle semblait effrayée, mais elle tenait bon. <<~Son mari était un sorcier.~>>

<<~C'est absurde~!~>> répondit sèchement Michael. <<~Ils étaient à notre mariage\ldots{} Ils nous avait rendu visite à Noël\ldots~>>

<<~Je leur ai dit que tu ne devais pas être mis au courant,~>> murmura Pétunia. <<~Mais c'est vrai. J'ai vu des choses\ldots~>>

Le professeur leva les yeux au ciel. <<~Ma chérie, il me semble que tu ne connais pas bien la littérature sceptique. Tu ne comprends probablement pas à quel point un magicien bien formé peut facilement avoir l'air d'accomplir l'impossible. Est-ce que tu te souviens de la fois où j'ai appris à Harry comment tordre des petites cuillères~? Et s'ils avaient l'air de toujours savoir ce que tu avais à l'esprit, c'était ce qu'on appelle la lecture à froid\ldots~>>

<<~Ils ne tordaient pas de cuillères\ldots

--- Ils faisaient quoi, alors~?~>>

Pétunia se mordit la lèvre. <<~Je ne peux pas te le dire. Tu penseras que je suis\ldots~>> Elle déglutit. <<~Écoute. Michael. Je n'ai pas toujours été\ldots{} comme ça\ldots~>> Elle se désigna elle-même, comme pour faire référence à sa silhouette bien proportionnée. <<~C'est grâce à Lily. Parce que je\ldots{} parce que je l'ai \emph{suppliée}. Pendant des années. Elle avait \emph{toujours} été plus belle que moi, et ça m'avait rendue\ldots{} méchante envers elle. Et puis elle a eu des \emph{pouvoirs}. Est-ce que tu peux imaginer ce que j'ai ressenti~? Alors je l'ai \emph{suppliée} d'utiliser sa magie pour que je puisse être belle moi aussi. Au moins, même si je n'avais pas sa magie, je pourrais être belle.~>>

Des larmes s'étaient accumulées dans les yeux de Pétunia.

<<~Et elle refusait, elle inventait les excuses les plus fantasques. Comme quoi ce serait la fin du monde si elle acceptait d'aider sa s\oe{}ur, ou qu'un centaure l'avait mise en garde\ldots{} je la haïssais de me donner ces excuses minables. Et puis, juste après l'université, je sortais avec ce garçon, Vernon Dursley. Il était gros et c'était le seul garçon qui voulait bien me parler. Et il m'a dit qu'il voulait avoir des enfants, que son premier fils s'appellerait Dudley. Et je me demandais \emph{quel genre de parent appellerait son enfant Dudley Dursley}. C'était comme si j'avais vu ma vie future défiler devant moi, et je n'ai pas pu le supporter. Alors j'ai écrit à ma s\oe{}ur, et je lui ai dit que si elle ne m'aidait pas, alors autant\ldots~>>

Elle se tut.

<<~Enfin,~>> continua-t-elle d'une petite voix, <<~elle a cédé. Elle m'a dit que c'était dangereux, je lui ai dit que je m'en fichais, j'ai bu sa potion et j'ai été malade pendant des semaines. Mais une fois remise, ma peau avait perdu ses boutons, j'avais enfin des formes et\ldots{} j'étais belle. Les gens étaient \emph{gentils} avec moi.~>> Sa voix se brisa. <<~Après, je n'ai plus pu haïr ma s\oe{}ur. Surtout quand j'ai appris ce que sa magie lui avait fait\ldots~>>

<<~Ma chérie,~>> dit Michael avec gentillesse, <<~tu es tombée malade, tu as pris du poids pendant ta convalescence, et ta peau s'est arrangée toute seule. Ou alors la maladie t'a fait changer de régime alimentaire\ldots~>>

<<~C'était une sorcière~>>, répéta Pétunia. <<~Je l'ai vu.~>>

<<~Pétunia,~>> dit Michael. L'exaspération devenait audible. <<~Tu \emph{sais} que c'est impossible. Dois-je vraiment t'expliquer pourquoi~?~>>

Pétunia se tordit les mains. Elle semblait être sur le point de pleurer. <<~Mon amour, je sais que je ne peux pas remporter ce genre de débat avec toi, mais s'il te plaît, tu dois me faire confiance là\ldots~>>

<<~\emph{Papa~! Maman~!}~>>

Ils se turent et regardèrent Harry comme s'ils avaient oublié qu'une troisième personne se trouvait là, avec eux.

Harry inspira lentement. <<~Maman, \emph{tes} parents avaient-ils des pouvoirs magiques~?~>>

<<~Non~>>, dit Pétunia d'un ton interloqué.

<<~Alors personne chez toi ne connaissait la magie quand Lily a reçu sa lettre. Comment est-ce qu'ils \emph{vous} ont convaincus~?~>>

<<~Ah\ldots~>> dit Pétunia. <<~Ils n'ont pas fait qu'envoyer une lettre. Ils ont envoyé un professeur de Poudlard. Il\ldots~>> elle regarda brièvement Michael. <<~Il nous a fait une démonstration de sa magie.~>>

<<~Alors pas la peine de vous battre,~>> dit Harry d'un ton ferme. Il espérait contre toute attente que cette fois, cette fois seulement, ils l'écouteraient. <<~Si c'est vrai, on n'a qu'à faire venir un professeur de Poudlard et voir cette magie nous-mêmes. Papa devra admettre que c'est bien réel. Et sinon, alors maman devra admettre que ça n'existe pas. C'est à ça que sert la méthode expérimentale. Pour qu'on n'ait pas à tout résoudre à force de débats.~>>

Le professeur se retourna et baissa les yeux vers son fils avec son air dédaigneux habituel. <<~Allons Harry. Vraiment, de la \emph{magie}~? Je pensais que toi au moins, tu aurais l'intelligence de ne pas prendre ça au sérieux, même si tu n'as que dix ans. Il n'y a rien de moins scientifique que la magie~!~>>

Harry serra les lèvres avec amertume. Son père l'avait bien traité. Probablement mieux que la plupart des pères biologiques ne traitaient leur enfant. Il avait été aux meilleures écoles primaires, et quand les choses s'étaient mal passées, on l'avait confié à des précepteurs venus de ce puits sans fond que sont les doctorants affamés. On l'avait toujours encouragé à étudier ce qui l'intéressait le plus, on lui avait acheté tous les livres qu'il désirait, on l'avait soutenu à tous les concours mathématiques et scientifiques auxquels il avait participé. On lui donnait tout ce qu'il pouvait raisonnablement désirer, sauf peut-être la moindre trace de respect. On pouvait difficilement s'attendre à ce qu'un professeur de biochimie employé à Oxford écoute les conseils d'un petit garçon. Il fallait, bien sûr, toujours montrer son intérêt~: voilà ce que faisaient les bons parents, et donc ce que faisaient ceux qui se voyaient comme de bons parents. Mais prendre un enfant de dix ans \emph{au sérieux}~? Certainement pas.

Harry avait parfois envie de hurler sur son père.

<<~Maman,~>> dit Harry. <<~Si tu veux remporter ce débat contre papa, regarde le chapitre deux du premier manuel de physique de Feynman. Il parle de ce que racontent les philosophes sur la méthode scientifique, et il explique qu'ils ont tous tort, que la seule règle en science, c'est que l'observation est maîtresse. Qu'il suffit d'observer le monde et de rendre compte de ce que l'on a vu. Et\ldots{} je n'arrive pas à me souvenir de tête d'une source qui explique que l'un des idéaux de la science est de régler les désaccords par l'expérimentation plutôt que par le débat\ldots~>>

Sa mère baissa les yeux et lui sourit. <<~Merci Harry. Mais\ldots~>> elle releva les yeux et regarda son mari. <<~Je ne veux pas remporter un débat. Je veux que mon mari\ldots{} je veux qu'il écoute sa femme qui l'aime et qu'il lui fasse confiance au moins pour cette fois.~>>

Harry ferma brièvement les yeux. \emph{C'était sans espoir}. Ses deux parents étaient des causes perdues.

Voilà qu'ils reprenaient une de \emph{ces} disputes. Où sa mère essayait de faire culpabiliser son père, et où ce dernier essayait de la faire se
sentir stupide.

<<~Je vais dans ma chambre~>> annonça Harry. Sa voix tremblait un peu. <<~Essayez de ne pas trop vous battre, papa et maman. On aura la réponse bien assez vite, non~?~>>

<<~Bien sûr Harry,~>> dit son père, sa mère lui envoya un baiser rassurant et ils continuèrent de se disputer pendant que Harry grimpait les escaliers jusqu'à sa chambre.

Il ferma la porte et essaya de réfléchir.

Il remarqua avec amusement qu'il aurait \emph{dû} être du côté de son père. Personne n'avait jamais vu de magie, alors que maman disait qu'un monde magique entier existait. Comment pouvait-on garder ce genre de choses secret~? Avec plus de magie~? C'était plutôt suspect, comme excuse. 

Ça aurait dû être une alternative simple entre~: maman faisait une blague, maman mentait, maman était folle -- par ordre d'horreur croissant. Si elle avait envoyé la lettre elle-même, cela expliquait son arrivée dans la boîte aux lettres malgré l'absence de timbre. Un peu de folie était bien moins incroyable qu'un univers rempli de magie. 

Sauf que quelque chose en Harry était foncièrement convaincu que la magie était bien réelle, depuis l'instant où il avait vu la lettre soi-disant venue de l'École de Sorcellerie de Poudlard. 

Il se frotta le front en grimaçant. \emph{Ne crois pas tout ce que tu penses}, disait l'un de ses livres. 

Mais cette étrange certitude\ldots{} il \emph{s'attendait} tout simplement à ce qu'un professeur de Poudlard apparaisse, agite une baguette, et fasse de la magie. Cette étrange certitude ne faisait aucun effort pour prévenir de possibles réfutations~; elle n'était pas accompagnée d'excuses préalables au cas où aucun professeur ne viendrait, ou au cas où le professeur ne ferait rien d'autre que tordre de petites cuillères. 

\emph{D'où viens-tu, étrange petite prédiction~?} dit-il à la pensée nichée dans son cerveau. \emph{Pourquoi est-ce que je crois ce que je crois~?} 

Habituellement, Harry pouvait fort bien répondre à cette question. Mais dans ce cas, il ignorait \emph{totalement} où son cerveau pouvait bien avoir été pêcher ça. 

Il eut un haussement d'épaules mental. Tout comme une plaque de métal sur une porte était faite pour être poussée, tout comme une poignée était faite pour qu'on la tire, lorsque l'on était face à une hypothèse, il fallait la tester. 

Il se saisit d'une feuille de papier et commença à écrire. 

\emph{Chère directrice adjointe,} 

Harry s'interrompit et réfléchit. Puis il jeta la feuille au profit d'une autre et fit sortir un autre millimètre de graphite de son critérium. Cela méritait un effort calligraphique plus méticuleux. 

\emph{Chère directrice adjointe, Minerva McGonagall,} 

\emph{Ou à toute personne concernée~:} 

\emph{J'ai récemment reçu votre lettre d'admission à Poudlard, adressée à M. H. Potter. Peut-être ignorez-vous que mes parents biologiques, James Potter et Lily Potter (anciennement Lily Evans) sont morts. J'ai été adopté par Pétunia Evans-Verres, la sœur de Lily, et par son mari, Michael Verres-Evans.} 

\emph{Je désire ardemment me rendre à Poudlard, si tant est qu'un tel lieu existe effectivement. Seule ma mère Pétunia dit connaître l'existence de la magie, et elle est elle-même incapable de la pratiquer. Mon père est très sceptique. Je suis moi-même hésitant. J'ignore aussi comment me procurer les livres et l'équipement que vous mentionnez dans la lettre d'admission.} 

\emph{Ma mère a dit que vous aviez envoyé un représentant de Poudlard chez Lily Potter (Lily Evans à cette époque) afin de démontrer la réalité de la magie à sa famille et, je suppose, d'aider Lily à se procurer ses affaires de classe. Il serait fort utile que vous fassiez de même avec ma famille.} 

\emph{Bien à vous,} 

\emph{Harry James Potter-Evans-Verres.} 

Harry ajouta son adresse puis il plia la lettre, la mit dans une enveloppe, et l'adressa à Poudlard. Quelques instants de réflexion le poussèrent à se procurer une bougie et à faire tomber de la cire sur le rabat de l'enveloppe dans laquelle il grava H.J.P.E.V à l'aide d'un canif. Quitte à devenir fou, autant que ce soit avec classe. 

Il ouvrit enfin la porte de sa chambre et descendit les escaliers. Son père était dans le salon et lisait un livre de mathématiques avancées pour montrer à quel point il était intelligent~; sa mère était dans la cuisine et préparait l'un des plats favoris de son mari pour montrer à quel point elle était aimante. La discussion semblait être close. Aussi effrayantes que les disputes puissent être, leur \emph{absence} pouvait être bien pire. 

«~Maman,~» dit Harry au milieu de ce silence troublant, «~je vais tester mon hypothèse. Que dit ta théorie sur la meilleure façon d'envoyer une chouette à Poudlard~?~» 

Sa mère se détourna de l'évier et le regarda d'un air surpris. «~Je\ldots{} je ne sais pas, je crois qu'il faut avoir une chouette magique.~» 

Cela aurait dû être fort suspect. \emph{Oh, alors on ne peut pas tester ta théorie~?} Mais l'étrange certitude en Harry était prête à aller plus loin. 

«~Bon, la lettre est bien arrivée jusqu'ici,~» dit-il, «~alors je vais l'agiter dehors en disant "Lettre pour Poudlard~!" pour voir si une chouette vient la ramasser. Papa, tu veux venir regarder~?~» 

Son père secoua lentement la tête et continua sa lecture. \emph{Bien sûr}, songea Harry. La magie était une de ces croyances honteuses pour nigauds~; si son père allait jusqu'à \emph{tester} l'hypothèse, s'il allait jusqu'à \emph{observer} le test, ce serait comme de \emph{s'associer} à celle-ci\ldots{} 

Ce n'est qu'en sortant par la porte de derrière et vers le jardin qu'il comprit que, si jamais une chouette venait vraiment récupérer la lettre, il aurait bien du mal à l'expliquer à son père. 

\emph{Mais\ldots{} c'est impossible, non~? Même si mon cerveau a l'air d'y croire. Si une chouette vient vraiment prendre cette enveloppe, j'aurai des soucis beaucoup plus sérieux que l'opinion de papa.} 

Il inspira profondément et leva l'enveloppe en l'air. 

Il déglutit. 

S'écrier \emph{Lettre pour Poudlard~!,} du fond de son jardin, avec une enveloppe au bout de son bras dressé\ldots{} Réflexion faite, c'était vraiment ridicule. 

\emph{Non. Je vaux mieux que papa. J'utiliserai la méthode scientifique, même si je me sens stupide.} 

«~Lettre\ldots~» dit Harry, mais ce fut plutôt une espèce de coassement chuchoté. 

Il rassembla son courage et cria vers le ciel vide~: «~\emph{Lettre pour Poudlard~! Est-ce que je peux avoir une chouette~?}~» 

«~Harry~?~» dit une voix de femme déconcertée. C'était l'une des voisines. 

Il baissa sa main comme si elle avait pris feu et cacha l'enveloppe dans son dos comme si c'était de l'argent sale. La honte lui brûlait les joues. 

Le visage d'une vieille femme émergea par-dessus une clôture. Des cheveux grisonnants s'échappaient de la résille qu'elle portait sur la tête. C'était Mlle Figg, qui lui servait parfois de baby-sitter. «~Harry, qu'est-ce que tu fais~?~» 

«~Rien,~» dit-il d'une voix étranglée. «~Je\ldots{} teste juste une théorie idiote\ldots~» 

«~Est-ce que Poudlard t'a envoyé ta lettre d'admission~?~» 

Il se figea. 

«~Oui,~» répondirent ses lèvres quelques instants plus tard. «~J'ai reçu une lettre de Poudlard. Ils disent qu'ils veulent avoir reçu ma chouette avant le 31 juillet, mais\ldots~» 

«~Mais tu n'as \emph{pas} de chouette. Pauvre petit~! Comment ont-ils pu être assez \emph{bêtes} pour t'envoyer la lettre habituelle~?~» 

Un bras ridé passa par-dessus la clôture et ouvrit une main. À peine capable de réfléchir, Harry lui donna l'enveloppe. 

«~Laisse-la-moi, mon petit,~» dit Mlle Figg, «~et je ferai venir quelqu'un en un rien de temps.~» 

Puis son visage disparut derrière la clôture. 

Il y eut un long silence. 

Et la voix d'un petit garçon dit calmement et doucement~: «~Quoi~?~»
