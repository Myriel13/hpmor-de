\chapter{Préoccupation}

\section{Le 2 juillet 1992.}

\lettrinepara{L}{e} professeur Quirrell était très malade.

\hplettrineextrapara
Il avait semblé aller mieux pendant un moment, après avoir bu le sang de licorne en mai, mais l'intense aura de pouvoir qui l'avait ensuite entouré n'avait même pas tenu un jour. Dès les Ides de mai, les mains du professeur Quirrell tremblaient de nouveau, quoi que de façon subtile. Il semblait que le régime médicinal du professeur de Défense avait été interrompu trop tôt.

Six jours plus tôt, le professeur de Défense s'était effondré en plein dîner.

Madame Pomfresh avait tenté d'interdire au professeur Quirrell de continuer ses enseignements et le professeur Quirrell lui avait crié dessus devant tout le monde. Le professeur de Défense avait crié qu'il était de toute façon mourant et qu'il utiliserait le temps qui lui restait comme il l'entendait.

Madame Pomfresh avait donc, les lèvres pincées, interdit au professeur de Défense de faire quoi que ce soit \emph{hormis} enseigner. Elle avait demandé un volontaire prêt à aider le professeur Quirrell à se rendre à l'infirmerie de Poudlard. Plus de cent étudiants s'étaient levés, et seule la moitié avait porté du vert.

Le professeur de Défense ne s'asseyait plus à la grande table pendant les repas. Il ne lançait plus de sortilèges pendant ses leçons. Les élèves seniors avec le plus de points Quirrell l'aidaient à enseigner, ces septième année qui avaient déjà obtenu leur ASPIC de Défense en mai. Ils se relayaient pour le faire léviter de l'infirmerie à ses cours et lui apportaient de la nourriture pendant les repas. Le professeur Quirrell surveillait ses cours de magie de bataille depuis une chaise, assis.

Voir Hermione mourir avait été plus douloureux, mais au moins cela avait été plus rapide.

\emph{C'est le véritable Ennemi.}

Harry avait déjà pensé cela après la mort de Hermione. Être forcé à voir le professeur Quirrell mourir jour après jour, semaine après semaine, ne l'avait en rien conduit à changer d'avis.

\emph{C'est le véritable ennemi auquel je dois faire face}, songea Harry pendant le cours de Défense de mercredi, lorsque professeur Quirrell pencha dangereusement d'un côté de sa chaise avant qu'un assistant de septième année ne le rattrape. \emph{Tout le reste n'est qu'ombres, distractions.}

Harry avait ruminé la prophétie de Trelawney en se demandant si le véritable Seigneur des Ténèbres n'était peut-être pas Lord Voldemort du tout. \emph{Né de ceux qui l'ont trois fois défié} semblait fortement faire référence aux frères Peverell et aux trois Reliques de la Mort -- bien que Harry ne voyait pas tout à fait comment la Mort aurait pu le marquer comme son égal, puisque cela semblait impliquer une sorte d'action délibérée de la part de la Mort.

\emph{Cela seul est le véritable ennemi}, songea Harry. \emph{Après ça viendront le professeur McGonagall, Maman et papa, et même Neville, à moins que les blessures du monde ne soient guéries avant. La Mort seule est mon dernier Ennemi~; et c'est ce qui m'a été dit sur la tombe de mes parents.}

Harry ne pouvait rien faire. Madame Pomfresh faisait déjà pour le professeur Quirrell ce dont la magie était capable, et lorsqu'il était question de santé, la magie semblait strictement supérieure aux techniques Moldues.

Harry ne pouvait rien faire.

Il ne pouvait rien faire.

Rien.

Rien du tout.

\later

Harry leva sa main et frappa à la porte au cas où la personne qui se trouvait là ne pouvait plus le détecter.

<<~Qu'y a-t-il~? dit une voix accablée depuis l'infirmerie.

--- C'est moi.~>>

Il y eut un long silence. <<~Entrez~>>, dit la voix.

Harry se glissa à l'intérieur, ferma la porte derrière lui et lança le sortilège de Mutisme. Il se tenait aussi loin que possible du professeur Quirrell, juste au cas où sa propre magie mettrait le professeur Quirrell dans quelque inconfort.

Mais la sensation funeste s'estompait, jour après jour.

Le professeur Quirrell était allongé dans son lit d'infirmerie, seule sa tête maintenue par un oreiller. Un couvre-lit d'un matériau cotonneux, rouge à coutures noires, le recouvrait jusqu'au torse. Un livre flottait devant ses yeux, une page entourée d'une pâle lueur qui enrobait aussi un cube noir posé à côté du lit. Ce n'était donc pas la magie du professeur de Défense mais un appareil quelconque.

Le livre était <<~Penser la physique~>>, d'Einstein, le même livre que Harry avait prêté à Drago quelques mois auparavant. Harry avait cessé de se tracasser quant aux mauvais usages que l'on pourrait en faire plusieurs semaines plus tôt.

<<~C'est…~>>, dit le professeur Quirrell, et il eut une toux étrange. <<~C'est un livre fascinant… si jamais je m'étais rendu compte…~>> Un rire, mêlé à une autre toux. <<~Pourquoi ai-je présumé que les arts Moldus… ne devaient être miens~? Qu'ils ne me seraient… d'aucun usage~? Pourquoi n'ai-je jamais pris la peine d'essayer… de vérifier expérimentalement… comme vous diriez~? Au cas où… ma présomption… aurait été fausse~? Je me sens comme l'idiot le plus profond… rétrospectivement…~>>

Harry avait beaucoup plus de mal à parler que le professeur Quirrell. Sans un mot, Harry mit la main dans sa bourse et posa un mouchoir sur le sol qu'il déplia pour révéler un petit caillou blanc, doux et rond.

<<~Qu'est-ce que cela~? dit le professeur de Défense.

--- C'est une… c'est une… une licorne métamorphosée.~>>

Harry avait consulté les livres et avait appris que puisqu'il était trop jeune pour avoir des pensées sexuelles, il pourrait approcher une licorne sans peur. Les mêmes livres n'avaient rien dit de l'intelligence des licornes. Harry avait déjà remarqué que toutes les espèces magiques intelligentes étaient au moins partiellement humanoïdes, et on savait de nombre d'entre elles qu'elles pouvaient se reproduire avec des humains. Harry était déjà arrivé à la conclusion que la magie ne créait pas de nouvelles intelligences mais modifiait juste la forme d'humains génétiques. Les licornes étaient des équidés, pas même partiellement humanoïde, ne parlaient pas, n'utilisaient pas d'outils, et ils n'étaient presque certainement rien d'autre que des chevaux magiques. S'il était juste de manger une vache pour se nourrir pendant une journée, il se \emph{devait} d'être juste de boire le sang d'une licorne pour repousser la mort de quelques semaines. On ne pouvait faire deux poids, deux mesures.

Harry s'était donc rendu dans la Forêt Interdite, muni de sa Cape. Il avait cherché le Bosquet des licornes jusqu'à la voir, une fière créature à la robe d'un blanc pur et à la crinière violette, avec trois taches bleues sur le flanc. Harry s'était approchée d'elle et les yeux de saphir l'avaient observé avec curiosité. Harry avait tapé sur le sol, 1,2, 3, avec ses pieds. La licorne n'avait donné aucune réponse similaire. Harry s'était approché, avait pris sa corne dans ses mains, et avait tapé la même séquence sur la corne de la licorne. Celle-ci n'avait fait que le regarder avec curiosité.

Et pourtant, donner à la licorne les carrés de sucre imbibés de potion de sommeil lui avait donné l'impression de commettre un meurtre.

\emph{Cette magie donne à leur existence un poids qu'aucun simple animal ne pourrait posséder… tuer une chose innocente pour se sauver soi~: voilà un grave péché.} Ces deux phrases, du professeur McGonagall, du centaure, s'étaient répétées dans l'esprit de Harry, encore et encore, lorsque la licorne blanche avait baillé, s'était allongée et avait fermé ses yeux pour ce qui serait la dernière fois. La métamorphose avait duré une heure et des larmes avaient perlé aux yeux de Harry plusieurs fois pendant qu'il travaillait. La mort de la licorne n'avait peut-être pas déjà eu lieu, mais elle viendrait bien assez tôt, et l'idée de rejeter une responsabilité était étrangère à Harry. Il n'avait plus qu'à espérer qu'on pourrait le pardonner d'avoir tué la licorne, non pas pour se sauver, mais pour aider un ami.

Les yeux du professeur Quirrell s'étaient élevés. Sa voix fut moins douce, reprit une partie de son acerbité usuelle, lorsqu'il dit~:

<<~Je vous interdis de recommencer.

--- Je me demandais si vous diriez cela~>>, dit Harry. Il déglutit de nouveau. <<~Mais cette licorne est déjà, déjà perdue, alors autant que vous la preniez, professeur…

--- Pourquoi avez-vous fait ça~?~>>

Si le professeur de Défense n'était pas capable de comprendre cela, c'était la personne la plus lente à la détente que Harry avait jamais rencontrée. <<~Je n'arrêtais pas de me dire que je ne pouvais rien faire, dit Harry. Et j'en ai eu assez de penser ça.~>>

Le professeur Quirrell ferma les yeux. Il reposa sa tête sur l'oreiller. <<~Vous avez eu de la chance, dit le professeur de Défense d'une voix douce, qu'une licorne métamorphosée… ne déclenche pas le système de protection de Poudlard contre les créatures étranges… je devrai… sortir de ces limites pour pouvoir en faire usage… mais des moyens existent. Je leur dirai que je souhaite observer le lac… je vous demanderai de faire perdurer la métamorphose avant de partir, afin qu'elle dure assez longtemps… et avec mes dernières forces, je dissiperai toute alarme placée sur le troupeau en cas de mort de l'un de ses membres… une alarme que vous n'avez pas encore déclenchée puisque la licorne n'est que métamorphosée, et non tuée… vous avez eu beaucoup de chance, M. Potter.~>>

Harry hocha la tête. Il commença à parler. Une fois de plus, les mots semblèrent rester bloqués dans sa gorge.

\emph{Tu as déjà calculé l'utilité espérée, si ça marche, si ça échoue. Tu as assigné des probabilités, tu as multiplié, puis tu as jeté le résultat et as suivi ton nouvel instinct, qui disait la même chose. Alors vas-y.}

<<~Connaissez-vous, dit Harry d'une voix chancelante, un moyen quelconque de sauver votre vie~?~>>

Les yeux du professeur s'ouvrirent.

<<~Pourquoi… me demandes-tu cela, jeune homme~?

--- Il y a… un sortilège dont j'ai entendu parler, un rituel…

--- Silence~>>, dit le professeur de Défense.

Un instant plus tard, un serpent se trouvait dans le lit.

Même les yeux du serpent étaient mornes.

Il ne se dressa pas.

\parsel{<<~Reprendss…~>>}, siffla le serpent. Seule sa langue bougeait.

<<~Il y a… \parsel{il y a un rituel dont le directeur m'a parlé, par le moyen duquel il pensse que le Sseigneur des Ténèbres pourrait avoir continué de vivre. Il ss'appelle…}~>> et Harry se tut lorsqu'il se rendit compte qu'il connaissait le mot en Fourchelangue. <<~\parsel{Horcruxe. J'ai entendu dire qu'il néçcesssite une mort. Mais ssi vous êtes de toute façon mourant, vous pourriez esssayer d'adapter le rituel, même ssi le nouveau ssortilège est plus rissqué, afin qu'il puissse être jeté par autre sacrifice. Si vous réusssissiez, le monde entier serait transsformé~; je ne connais rien du ssortilège, mais le directeur m'a dit qu'il arrachait un fragment de l'âme, même ssi je ne vois pas comment cela pourrait être vrai…}~>>

Le serpent siffla un rire, un étrange rire sec, presque hystérique. <<~\parsel{Tu me parles de ce ssort~? À moi~? Tu dois apprendre la prudence à l'avenir, garçon. Mais cela n'a pas d'importance. J'ai entendu parler du ssortilège de Horcruxe il y a longtemps. Il est abssurde.}

--- Absurde~? dit Harry à voix haute, surpris.

--- \parsel{Serait inssenssé, même ssi âme exisstait. Arracher fragment d'âme~? C'est un menssonge. Disstraction pour cacher vrai ssecret. Sseul celui qui ne croit pas aux ssimples menssonges pousssera raisonnement, verra par-delà obsscurcisssement, comprendra comment lancer ssort. Meurtre requis n'est pas rituel ssacrificiel. Mort soudaine produit parfois fantôme, si magie jaillisssante ss'imprime sur objet proche. Ssortilège Horcruxe envoie jaillisssement de mort de victime à travers lançceur, créé propre fantôme au lieu de çcelui de victime, imprime fantôme dans objet sspécial. Sseconde victime ramassse objet Horcruxe, objet imprime mémoire dans elle. Mais seulement mémoire au moment de construction d'objet Horcruxe. Vois-tu défaut~?}~>>

La sensation de brûlure était revenue dans la gorge de Harry.

<<~\parsel{Pas de continuité de…}~>> mais les serpents n'avaient pas de mot pour désigner la conscience, <<~\parsel{ssoi, on continue de pensser après avoir fait Horcruxe, puis ssoi avec nouvelle mémoire meurt et n'est pas resstauré…}

--- \parsel{Oui, tu vois. Ausssi Interdit de Merlin empêche ssortilège puisssant de passser par tel objet, puissque pas vraiment en vie. Mages Noirs qui veulent revenir ssont plus faibles, façciles à éliminer. Aucun n'a tenu longtemps par tel moyen. Perssonnalité change, mélange avec çcelle de victime. Mort pas vraiment vaincue. Vrai ssoi mort, comme tu dis. Pas à mon goût actuel. Admet que je l'ai conssidéré, il y a longtemps.}~>>

Un homme était à nouveau allongé dans le lit de l'infirmerie. Le professeur de Défense inspira puis fit un pitoyable bruit de toux.

<<~Pourriez-vous me donner la recette complète du sortilège~? dit Harry après avoir délibéré un instant. Il pourrait y avoir un moyen de corriger les défauts avec assez de recherche. Un moyen de rester éthique et que cela fonctionne.~>> Comme de faire le transfert dans un corps cloné avec un cerveau vide au lieu d'une victime innocente, ce qui augmenterait aussi la fidélité du transfert… même si demeuraient encore les autres problèmes.

Le professeur Quirrell émit un son court qui aurait pu être un rire. <<~Vous savez, garçon, murmura le professeur Quirrell, j'avais pensé tout vous apprendre… les graines de tous les secrets que je connais… d'un esprit vivant à l'autre… afin que plus tard, en ouvrant les bons livres, vous compreniez… je vous aurai transmis mon savoir, mon héritier… nous aurions commencé dès que vous auriez demandé… mais vous ne l'avez jamais fait.~>>

Même le chagrin qui submergeait Harry s'écarta pour laisser place à l'immensité de l'opportunité loupée. <<~Étais-je censé…~? Je ne savais pas que j'étais censé…~!~>>

Un autre ricanement toussoteux.

<<~Ah oui… le né-Moldu ignorant… par héritage sinon par sang… c'est vous. Mais je me… suis ravisé… vous ne devriez pas suivre ma voie… elle s'est avérée… être mauvaise.

--- Il n'est pas trop tard, professeur~!~>> dit Harry. Une partie de lui s'écria qu'il se comportait en égoïste, puis une autre fit taire la première~: il y aurait d'autres gens à aider.

<<~Si, il est trop tard… et vous ne me… persuaderez pas du contraire… je me suis… ravisé… comme j'ai dit… j'ai trop… de secrets qui devraient le rester… \emph{regardez-moi.}~>>

Harry regarda, presque malgré lui.

Il vit un visage encore sans ride à l'air vieux et endolori, sous une tête dont les cheveux tombaient rapidement, dont même les tempes s'effilochaient. Harry vit un visage qu'il avait toujours pensé être sec, maintenant révélé comme \emph{émacié}, à mesure que le muscle et le gras disparaissaient du visage comme des bras au-dessous, telle la forme squelettique de Bellatrix Black qu'il avait vue à Azkaban…

Sans même réfléchir, Harry détourna violemment son visage.

<<~Vous voyez, murmura le professeur. Je n'aime pas énoncer des clichés… M. Potter… mais la vérité est que… les Arts que l'ont dit sombres… ne sont en définitive… pas bons pour ceux qui les pratiquent.~>>

Le professeur Quirrell inspira puis expira. L'infirmerie fut tranquille un moment. Ils ne regardaient que la pierre richement décorée des murs.

<<~Reste-t-il quoi que ce soit… que nous ne nous soyons dit~? continua le professeur Quirrell. Je ne meurs pas aujourd'hui… au fait… pas tout de suite… mais je ne sais pas combien de temps… je pourrai continuer de discuter.

--- Il y a… dit Harry déglutissant à nouveau. Il y a beaucoup de choses… beaucoup trop… mais… peut-être est-ce une mauvaise question mais je ne veux pas… que celle-ci demeure sans réponse… serpent~?~>>

Un serpent se trouvait sur le lit.

<<~\parsel{J'ai appris comment fonctionne le ssortilège de la Mort. Néçcesssite une véritable haine pour être lancé, pas beaucoup, mais on doit vouloir la mort, c'est ce qu'on dit. Dans la prison avec manges-vie, vous avez lancé ssortilège de la Mort contre garde -- avez dit que vous ne le vouliez pas mort -- était-ce menssonge~? Ici, maintenant, à cette disstance, vous pouvez dire vérité, même ssi vous avez peur de mauvaise apparensce. Çcela ne devrait pas compter maintenant, professseur. Je ssouhaite ssavoir. Dois ssavoir. Ne vous abandonnerai pas, qu'importe réponsse.}~>>

Un homme était étendu sur le lit.

<<~Écoutez-moi attentivement, dit le professeur Quirrell. Je vais vous donner une énigme… une devinette pour un sortilège dangereux… quand vous connaîtrez la réponse à ce puzzle… vous connaîtrez aussi… la réponse à votre question… est-ce que vous écoutez~?~>>

Harry hocha la tête.

<<~Il y a une limite… au sortilège de la Mort. Pour le lancer une fois… en combat… il faut haïr assez… pour vouloir la mort de l'autre. Pour lancer Avada… Kedavra deux fois… on doit haïr assez… pour tuer deux fois… pour leur trancher la gorge de ses propres mains… pour les voir mourir… et recommencer. Très peu… peuvent haïr assez… pour tuer quelqu'un… cinq fois… ils s'ennuieraient.~>> Le professeur de Défense prit plusieurs inspirations avant de poursuivre. <<~Mais si tu observes l'Histoire… tu verras certains mages noirs… qui pouvaient lancer le sortilège de la Mort… encore et encore. Une sorcière du dix-neuvième siècle… elle se faisait appeler Dark Evangel… les Aurors l'appelaient A.K. McDowell. Elle pouvait lancer le sortilège de la Mort… une douzaine de fois… dans un seul combat. Demandez-vous… comme je me le suis demandé… quel secret… connaissait-elle~? Qu'est-ce qui est plus mortel que la haine… et ne connaît aucune limite~?~>>

\emph{Un second niveau à Avada Kedavra, exactement comme avec le Patronus…}

<<~Ça ne m'intéresse pas vraiment~>>, répondit Harry.

Le professeur de Défense eut un gloussement enroué. <<~Bien. Vous… apprenez. Alors vous voyez…~>> Une pause, pour se transformer. <<~\parsel{Je ne ssouhaitais pas voir le garde mourir, après tout. J'ai lanscé le sortilège de la Mort, mais pas avec de la haine}~>>. De nouveau, un homme.

Harry déglutit. C'était à la fois mieux et pire que ce qu'il avait soupçonné~; et tout à fait caractéristique du professeur Quirrell. Une âme fracturée, certainement~; mais le professeur Quirrell n'avait jamais clamé être encore entier.

<<~Autre chose… à dire~? continua l'homme alité.

--- Êtes-vous absolument certain, dit Harry, que vous n'avez jamais entendu parler de quelque chose qui pourrait vous sauver, professeur~? Parmi toutes vos connaissances~? Trouver et réunir les trois Reliques de la Mort, un vieil artefact que Merlin aurait scellé derrière une énigme que personne n'a jamais pu résoudre~? Vous avez vu ce dont je suis capable. Mon talent pour les énigmes. Vous savez que je comprends parfois des choses que les autres sorciers ne peuvent appréhender. Je…~>> la voix de Harry se brisa, <<~je préfère fortement vous voir vivre que vous voir mort, professeur Quirrell.~>>

Il y eut une longue pause.

<<~Une chose, dit le professeur Quirrell. Une chose… pourrait m'aider… ou peut-être pas… mais l'obtenir… dépasse vos forces et les miennes…~>>

\emph{Oh, c'est exactement le discours d'introduction d'une quête annexe}, dit le Critique Interne de Harry.

Tout le reste de son être demanda à cette partie de lui-même de la fermer. La vie ne fonctionnait pas comme ça. D'anciens artefacts pouvaient être découverts, mais pas en un mois, pas quand on ne pouvait pas quitter Poudlard et qu'on était encore en première année.

Le professeur Quirrell prit une profonde inspiration. Exhala. <<~Je suis navré… c'était… trop théâtral. Ne vous… faites pas d'idées… M. Potter. Vous avez demandé… n'importe quoi… aussi improbable… que ce soit. Il existe… un certain objet… appelé…~>>

Un serpent se trouvait sur le lit.

<<~\parsel{La Pierre Philosophale,}~>> siffla-t-il.

S'il y avait eu un moyen de fabriquer de l'immortalité à la chaîne pendant tout ce temps et que personne ne s'était fatigué à l'utiliser, Harry allait craquer et assassiner tout le monde.

<<~\parsel{J'ai lu à sson sujet}, siffla Harry. \parsel{Conclut que c'était évidemment un mythe. Aucune bonne raisson que le même objet fournissse à la fois l'immortalité et une quantité infinie d'or. Pas à moins que quelqu'un ne ssoit jusste en train d'inventer des hisstoires réconfortantes. Ssans parler du fait que toute perssonne ssenssée aurait cherché le moyen d'en fabriquer d'autres ou de kidnapper sson créateur pour qu'il en fassse. J'ai penssé à vous en particulier, professseur.}~>>

Un sifflement hilare, froid.

<<~\parsel{Raisonnement ssage, mais pas asssez ssage. Comme avec ssortilège horcruxse, abssurdité cache vrai ssecret. Véritable Pierre pas ce que légende dit. Véritable pouvoir pas ce qu'histoires racontent. Créateur ssuppossé de Pierre pas celui qui l'a faite. Celui qui la détient maintenant, pas né ssous nom utilisé aujourd'hui. Pourtant, Pierre vrai artefact capable de guérir. En as-tu entendu parler~?}

--- \parsel{Sseulement dans le livre.}

--- \parsel{Çcelui qui détient Pierre posssède très grande connaisssance. A ensseigné nombreux ssecrets à directeur d'école. Directeur d'école n'a rien dit de posssessseur de Pierre, rien de Pierre~? Pas d'indiçce~?}

--- \parsel{Rien qui me revienne facilement}, répondit sincèrement Harry.

--- \parsel{Ah}, siffla le serpent. \parsel{Ah, bon.}

--- \parsel{Pourrais demander à directeur…}

--- \parsel{Non~! Ne lui demande pas, garçon. Il ne prendrait pas bien la question.}

--- \parsel{Mais ssi la Pierre ne fait que guérir…}

--- \parsel{Directeur ne croit pas çça, ne croirait pas çça. Trop ont recherché la Pierre ou désiré connaisssansce du détenteur. Ne demande pas. Ne dois pas demander. N'esssaie pas d'obtenir Pierre toi-même. J'interdis.}~>>

Un homme se trouvait à nouveau dans le lit. <<~Je suis… à la limite…, dit le professeur Quirrell. Je dois récupérer… des forces… avant d'aller… dans la forêt… avec votre cadeau. Partez maintenant… mais maintenez métamorphose… avant de partir.~>>

Harry tendit la main, toucha le caillou blanc sur le mouchoir et renouvela la métamorphose. <<~Elle devrait durer une heure et cinquante-trois minutes, maintenant~>>, dit Harry.

<<~Vos études… avancent bien.~>>

C'était bien plus longtemps que ce que Harry pouvait produire au début de l'année. Les sortilèges de seconde année lui semblaient simples à présent, ne lui demandaient pas d'effort~; ce qui n'était pas surprenant puisqu'il aurait douze ans dans moins de deux mois. Il aurait même pu lancer un sortilège d'Oubliettes s'il s'était avéré bon que quelqu'un oublie tout ce qui concernait son bras gauche. Il grimpait l'échelle du pouvoir, lentement, depuis le barreau le plus bas.

La pensée lui vint accompagné d'une potentielle tristesse, l'idée d'une porte qui s'ouvrait tandis qu'une autre se refermait~; ce qu'il rejeta aussi.

\later

La porte de l'infirmerie se referma derrière Harry, et le Survivant avança rapidement, vers un but, tout en jetant sa Cape d'Invisibilité sur ses épaules. Bientôt, le professeur de Défense appellerait sûrement pour que l'on vienne l'aider~; et un trio d'élèves plus âgés le guideraient jusqu'à un endroit tranquille, peut-être la forêt, avec une excuse, comme regarder le lac par exemple. Quelque part, le professeur de Défense mangerait une licorne, inaperçu, après que la métamorphose de Harry se fut achevée.

Puis le professeur Quirrell irait mieux, pendant un moment. Ses pouvoirs lui reviendraient, aussi fort que jamais, mais pour une durée bien plus courte.

Cela ne durerait pas.

Les poings de Harry se refermèrent, de la tension irradia des muscles de ses bras. Si le régime thérapeutique du professeur de Défense n'avait pas été interrompu par Harry et les Aurors qu'\emph{il} avait amené à Poudlard…

Il était idiot de s'en vouloir. Harry le savait, et pourtant, son cerveau le blâmait quand même. Comme si celui-ci cherchait avec attention quelque moyen de dire que c'était sa faute, au diable la crédibilité de l'explication.

Comme si être fautif était la seule façon qu'il avait de porter le deuil.

Un trio de Serpentard de septième année dépassèrent la forme invisible de Harry dans les couloirs, en direction de l'infirmerie, où attendait le professeur. Ils avaient l'air fort sérieux, fort soucieux. Était-ce ainsi que les autres portaient le deuil~?

Où est-ce qu'ils s'en \emph{fichaient}, quelque part au fond d'eux-mêmes, comme le professeur Quirrell le pensait~?

\emph{Il y a un second niveau au sortilège de la Mort.}

Le cerveau de Harry avait instantanément résolu l'énigme à la seconde où il l'avait entendue~; comme si ce savoir avait toujours été en lui et avait attendu de pouvoir se révéler.

Harry avait un jour lu, quelque part, que l'opposé de la joie n'était pas la tristesse mais l'ennui~; et l'auteur avait ensuite dit que pour trouver le bonheur, il ne fallait pas chercher ce qui nous rendait heureux mais ce qui nous enthousiasmait. Et, par le même raisonnement, la haine n'était pas le véritable opposé de l'amour. Si on se souciait assez d'une chose pour désirer la voir morte plutôt que vivante, c'était que l'on pensait à elle.

L'idée était apparue bien plus tôt, pendant le procès, lors d'une conversation avec Hermione~; quand elle avait dit que l'Angleterre magique était pleine de préjugés, avec amples justifications. Et Harry avait songé -- mais pas dit -- qu'au moins on l'avait laissée entrer à Poudlard pour pouvoir lui cracher dessus.

Contrairement à certaines personnes dans certains pays, qui étaient \emph{dites} aussi humaines que les autres~; que l'on \emph{disait} être des êtres sentients, plus importants qu'une simple licorne. Mais que l'on ne laissait pourtant pas vivre en Angleterre moldue. Sur ce point au moins, aucun Moldu n'avait le droit de toiser un sorcier. L'Angleterre magique était peut-être pleine de discriminations anti-nés-Moldus, mais au moins elle les laissait entrer pour pouvoir leur cracher dessus en personne.

\emph{Qu'est-ce qui est plus mortel que la haine et ne connaît aucune limite~?}

<<~L'indifférence~>>, murmura Harry, le secret d'un sortilège qu'il ne pourrait jamais lancer~; et il continua d'avancer vers la bibliothèque pour lire tout ce qu'il pourrait trouver, absolument tout, au sujet de la Pierre Philosophale.

%  LocalWords:  rofessor Evangetl
