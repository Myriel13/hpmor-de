\chapter{Réflexions}

\lettrine{M}{\emph{ême}} \emph{le plus puissant des artefacts peut être vaincu par un contre-artefact plus faible mais spécialisé.}

C'était ce que le professeur de Défense avait dit à Harry après avoir laissé tomber les replis crépusculaires de la véritable Cape d'Invisibilité juste aux pieds de Harry.

\emph{Le Miroir de Parfaite Réflexion a emprise sur ce qu'il reflète, et on dit que ce pouvoir est insurmontable. Mais puisque la véritable Cape d'Invisibilité produit une parfaite absence d'image, elle devrait se soustraire à ce principe plutôt que de s'y opposer.}

Il y avait ensuite eu une série de questions en Fourchelangue visant à établir que Harry n'avait pas l'intention de faire l'idiot ou de s'enfuir~; il y avait aussi eu des rappels de la capacité du professeur de Défense à sentir sa présence, de sortilèges capables de détecter la Cape, et des centaines de vies qu'il tenait en otage en plus de celle de Hermione.

Puis Harry reçut l'ordre de porter la cape et d'ouvrir la porte située derrière le feu tarit avant de la traverser jusqu'à la salle finale~; le professeur Quirrell se tenait loin en arrière, sur un côté de la porte.

La dernière chambre était illuminée d'or doux et les murs de pierre étaient d'un blanc apaisant, plaqués de marbre.

Au centre de la pièce, un simple cadre sans ornements~; dans le cadre, un portail vers une autre pièce illuminée d'or munie d'une porte qui donnait sur une autre salle des potions~: c'était ce que le cerveau de Harry lui disait. La transformation de la lumière opérée par le Miroir était si parfaite qu'un effort conscient était nécessaire pour comprendre que la pièce dans le cadre n'était qu'un reflet, pas un portail (même s'il lui aurait peut-être été plus simple de s'en rendre compte s'il n'avait pas été invisible).

Le Miroir ne touchait pas le sol~; le cadre d'or n'avait pas de pieds. Il n'avait pas l'air de flotter mais plutôt d'être figé, plus solide et immobile que les murs eux-mêmes, comme s'il avait été cloué au référentiel du mouvement terrestre.

<<~Est-ce que le Miroir est là~? Est-ce qu'il bouge~?~>> La voix autoritaire du professeur Quirrell lui parvint depuis la salle des potions.

<<~\parsel{Est là}, répondit Harry dans un sifflement. \parsel{Ne bouge pas.}~>>

Un ordre toujours impérieux lui parvint. <<~Fais le tour du Miroir.~>>

De derrière, le cadre ne présentait aucun reflet, il était d'un or massif, et Harry le dit en Fourchelangue.

<<~Maintenant, enlève ta Cape~>>, ordonna la voix du professeur Quirrell, toujours depuis la salle des potions. <<~Dis-moi immédiatement si le Miroir se retourne vers toi.~>>

Harry enleva la Cape.

Le Miroir demeura cloué au référentiel du mouvement terrestre~; et Harry le dit.

Peu après, on entendit des sifflements et des crépitements, un phénix enflammé traversa un mur de marbre en train de fondre situé derrière Harry, et la salle prit une teinte rouge. Le professeur Quirrell entra à sa suite dans le nouveau corridor qui venait d'être creusé. Ses souliers noirs avançaient sans dommage sur le sol rouge, lumineux et encore mou. <<~Eh bien, dit le professeur Quirrell, voilà un piège possible d'évité. Et maintenant…~>> il souffla longuement. <<~Maintenant, nous réfléchissons à de stratégies d'extraction de la Pierre du Miroir, et tu les essaies~; car je préfère ne pas laisser mon image apparaître dans ce Miroir. Sois prévenu~: ça pourrait être ennuyeux.

--- Je suppose que ce n'est pas un problème que le Feudeymon pourrait résoudre~?

--- Ha~>>, dit le professeur Quirrell, et il fit un geste.

Le phénix de enflammé avança vivement, dans un élan de terreur cramoisie, et la lumière rouge projeta des ombres tordues sur les murs de marbre encore debout. Harry bondit en arrière par pur réflexe.

L'éclat terrible de noir et rouge passa à côté du professeur Quirrell, fonça dans le dos d'or du Miroir et disparut aussi vite qu'il toucha l'or.

Puis le feu fut parti, et la pièce perdit sa teinte rougeâtre.

Il n'y avait pas de marque sur la surface d'or, pas d'émanation révélant une absorption de chaleur. Le Miroir était simplement resté là, intact.

Des frissons traversèrent Harry. S'il avait été en train de jouer à Donjons \& Dragons et que le maître de jeu lui avait décrit ça, Harry se serait cru victime d'une illusion mentale et aurait jeté les dés pour tenter de recouvrer la vue.

Au centre de l'envers d'or, une séquence de runes était apparue. D'un alphabet inconnu, elles n'étaient que de noires absences faites de fines lignes et de courbes disposées en rang. L'idée vint à Harry qu'une illusion mineure de dissimulation venait d'être consumée par le Feudeymon~; un enchantement bien plus faible, ajouté pour empêcher les enfants de lire ces lettres…

<<~Quel âge a ce Miroir~? dit-il presque dans un souffle.

--- Nul ne le sait.~>> Le professeur de Défense tendit les doigts vers les runes, un air de profond respect sur le visage~; mais ses doigts ne touchèrent pas l'or. <<~Enfin, à ce sujet, mon avis vaut bien le tien. Certaines légendes dont on ignore l'authenticité racontent que ce Miroir se reflète parfaitement \emph{lui-même}, et que son existence est donc absolument stable. Si stable qu'il a survécu alors que tout ce qui est sorti d'Atlantis a été défait, ses conséquences amputées du Temps. Tu comprends pourquoi ta suggestion d'utiliser le Feudeymon m'a amusé.~>> Le professeur de Défense laissa retomber son bras.

Même dans cette situation, Harry ressentait un certain émerveillement - si tout cela était vrai. Le cadre doré n'avait pas été rendu plus étincelant par la révélation, mais on pouvait l'imaginer exister loin, loin en arrière, jusqu'à une civilisation qui n'aurait jamais dû être…

<<~Qu'est-ce que… qu'est-ce qu'il \emph{fait}, exactement~?

--- Une excellente question, dit le professeur Quirrell. La réponse est inscrite sur les runes écrite derrière lui. Lis-les pour moi.

--- Je ne reconnais pas son alphabet. On dirait des gribouillis d'elfes de Tolkien orientés au hasard.

--- Lis-les quand même. \parsel{Ne préssente pas de danger.}

--- Les runes disent~: \emph{etneré hoceèlo partxe ètnol ovat si amegasiv notsa pert nome nej…}~>> il s'interrompit. Des frissons étaient revenus parcourir son échine.

Harry savait ce que la rune pour etneré \emph{signifiait}. Ça signifiait etneré. Et la rune suivante disait de hoceèlo le etneré jusqu'à ce qu'il atteigne partxe, puis de garder la partie à la fois étnol et ovat. Cette idée ressemblait à du savoir~; et il aurait pu répondre 'Oui' avec assurance si on lui avait demandé si le pert nome était notsa ou amegasiv. C'était juste que rien ne lui venait lorsqu'il essayait de lier ces concepts à d'autres.

<<~\parsel{Comprends-tu le ssens de ces mots, petit~?}

--- \parsel{Je ne pensse pas.}~>>

Le professeur Quirrell poussa un léger soupir. Ses yeux ne quittèrent pas le cadre d'or.

<<~Je m'étais demandé si les Mots de Fausse Compréhension auraient pu être clairs pour un étudiant de la science Moldue. Apparemment pas.

--- Peut-être…~>> commença Harry.

\emph{Vraiment, Serdaigle~?} dit Serpentard. \emph{Tu nous fais ça MAINTENANT~?}

<<~Peut-être que je pourrais réessayer si j'en savais plus sur le Miroir~?~>> dit la partie Serdaigle de Harry, qui venait de prendre les commandes.

Les lèvres du professeur Quirrell se retroussèrent. <<~Comme pour toute chose ancienne, les érudits ont écrit tant de mensonges à son sujet qu'il est difficile d'être sûr de quoi que ce soit. Il est certain que le Miroir est au moins aussi vieux que Merlin, puisque l'on sait que Merlin s'en servait comme d'un outil. On sait aussi qu'après sa mort, Merlin laissa des instructions écrites disant que le Miroir n'avait pas besoin d'être mis sous scellé, et ce même s'il renfermait des pouvoirs qui pourraient normalement causer quelque inquiétude. Il écrivit que, compte tenu du mal que les créateurs du Miroir s'étaient donné pour qu'il ne détruise pas le monde, on aurait plus de chance de parvenir à le détruire avec un bout de fromage.~>>

Harry ne trouva pas ceci tout à fait rassurant.

<<~Certains autres faits concernant le Miroir sont confirmés par de célèbres sorciers raisonnablement sceptiques et dont la parole s'est par ailleurs avérée fiable. Le pouvoir le plus caractéristique du Miroir est qu'il peut créer des mondes alternatifs entiers, même si ceux-ci ne dépassent pas la taille de ce qui peut être observé dans le Miroir~; on sait aussi que des gens et des objets peuvent y être entreposés. De nombreuses autorités prétendent que, de toutes les choses magiques, seul le Miroir possède un véritable sens moral, même si je ne suis pas certain du sens concret de cette phrase. Je m'attends à ce que les moralistes disent qu'Endoloris est 'mauvais' et que le Patronus est 'bon'~; j'ignore ce qu'un moraliste pourrait trouver \emph{plus} moral que ça. Mais on dit par exemple que les phénix sont entrés dans notre monde depuis un autre conçu dans ce Miroir.~>>

Alors qu'il fixait l'envers d'or du Miroir, des mots comme \emph{Bordel}, et d'autres qui auraient déplu à ses parents, traversaient l'esprit de Harry, sans grande cohérence.

<<~J'ai parcouru le monde et écouté de nombreuses histoires que l'on entend rarement, dit le professeur Quirrell. La plupart m'ont semblé fausses, mais certaines avaient un ton plus historique que narratif. Sur un mur de métal, en un lieu vide depuis des siècles, j'ai trouvé écrit que quelques Atlantes avaient prévu la fin de leur monde et désiré forger un appareil dont l'immense pouvoir empêcherait l'inévitable catastrophe. S'il avait été achevé, disait l'histoire, il aurait existé d'une façon parfaitement stable et aurait été capable de supporter un apport illimité de magie pour ensuite exaucer des vœux. L'appareil aurait aussi pu - et il était dit que c'était la tâche de loin la plus difficile - parvenir à empêcher les inévitables catastrophes que tout individu censé s'attendrait à voir naître d'un tel outil. Ce qui m'a le plus intéressé, c'est que, selon ces écrits, le reste d'Atlantis ignora le projet et continua sa vie. On louait parfois l'entreprise pour son caractère bienfaisant et altruiste, mais presque tous les Atlantes avaient mieux à faire que d'aider. Même les nobles ignoraient la perspective de voir un autre qu'eux obtenir un pouvoir sans conteste~; ce qui surprendra peut-être un cynique peu expérimenté. Avec relativement peu de soutien, la poignée d'individus désireux de construire l'appareil travaillèrent dans des conditions pas tant terriblement difficiles qu'inutilement agaçantes. Ils finirent par manquer de temps, et Atlantis fut détruite, l'appareil loin d'être achevé. J'ai reconnu dans cette histoire des échos de mes propres expériences que je ne m'attends pas à trouver dans de simples contes.~>> Un tiraillement agita le sourire sec. <<~Mais peut-être n'est-ce que le résultat de ma préférence d'une histoire parmi cent autres légendes. Tu auras néanmoins remarqué un écho de Merlin affirmant que le Miroir avait été créé pour ne pas détruire le monde. Plus important en ce qui nous concerne, cela pourrait expliquer la capacité jusqu'alors inconnue du Miroir évoquée par Dumbledore ou Perenelle~: celle de montrer à tous ceux qui se tiennent face à lui l'image d'un monde dans lequel l'un de leurs désirs est devenu réalité. C'est le genre de précaution raisonnable que l'on pourrait s'attendre à trouver dans une création capable d'exaucer des vœux et dont le créateur ne souhaite pas que les choses tournent atrocement mal.

--- Waouh~>>, murmura Harry, et c'était sincère. C'était de la Magie avec un grand M, le genre de Magie qu'on pouvait trouver dans \emph{Alors, on veut être un sorcier}, pas un simple assortiment arbitraire de violations de la physique réalisables avec une baguette.

Le professeur Quirrell fit un geste vers l'envers d'or.

<<~La dernière propriété sur laquelle les histoires s'accordent, c'est que, quelle que soit la méthode permettant le contrôle du Miroir - et sur cette Clé, aucun témoignage crédible n'existe - les instructions ne peuvent faire mention d'une personne en particulier. Donc Perenelle ne pourrait pas dire~: 'ne donne la Pierre qu'à Perenelle'. Dumbledore ne pourrait pas dire~:“Ne donne la Pierre qu'à celui qui souhaite la donner à Nicholas Flamel.” Il y a dans ce miroir un aveuglement semblable à celui que les philosophe attribuent à l'idéal de la justice~; il doit traiter ceux qui se présentent à lui de la même façon, quelle que soit la règle en place. Il doit donc exister une règle qui permet d'accéder à la cachette de la Pierre Philosophale, une règle telle que n'importe qui peut en faire usage. Et tu vois maintenant pourquoi c'est toi, le \emph{Survivant}, qui dois implémenter les stratégies que nous concocterons tous deux. Car il est dit que cette chose a un sens moral, et peut-être que les ordres qu'elle a reçu en font usage. Je sais parfaitement qu'en termes conventionnels, on te dit Bon et qu'on me dit Mauvais.~>> Le professeur Quirrell eut un sourire plutôt sombre. <<~Notre première tentative - mais pas la dernière, sois-en assuré - sera de voir comment le Miroir réagit alors que tu tentes d'obtenir la Pierre pour sauver la vie de Hermione Granger et de centaines de tes camarades.

--- Et dans la \emph{première} version de ce plan~>>, dit Harry, qui commençait enfin à comprendre, <<~celui que vous avez inventé le premier vendredi de cette année, c'est l'enfant prodige de Dumbledore, le Survivant, qui récupérait la Pierre, faisant une tentative altruiste et noble pour sauver la vie de son professeur de Défense mourant.

--- Bien sûr~>>, dit ce dernier.

Harry songea que c'était une ruse presque poétique, mais les circonstances l'empêchèrent de pleinement en apprécier l'élégance.

Puis une autre idée lui vint.

<<~Hmm, dit Harry. Vous pensez que le Miroir est un piège qui vous est tendu…

--- Il est totalement impensable que ce ne soit pas le cas.

--- C'est-à-dire que c'est un piège contre Lord Voldemort. Sauf que ça ne peut pas être un piège contre lui en particulier. Il doit y avoir une règle générale qui le soutient, une sorte de caractéristique généralisable de Lord Voldemort qui le déclenche.~>> Sans s'en rendre compte, Harry s'était mis à durement froncer les sourcils.

<<~Comme tu le dis~>>, dit le professeur Quirrell, qui, ayant vu Harry, commençait à lui aussi froncer les sourcils.

<<~Eh bien, le premier jeudi de l'année, le directeur fou, Dumbledore, que je venais de voir incinérer un poulet sous mes yeux, m'a dit que je n'avais aucune chance de jamais entrer dans son couloir interdit puisque je ne connaissais pas le sortilège \emph{Alohomora}.

--- Je \emph{vois}, dit le professeur Quirrell. Oh. Si seulement tu avais pensé à m'en faire part beaucoup plus tôt.~>>

Aucun d'eux n'eut besoin de dire l'évidence~: ce petit tour de psychologie inversée avec réussi à assurer que Harry reste le plus loin possible du couloir interdit de Dumbledore.

Harry réfléchissait toujours. <<~Pensez-vous que Dumbledore soupçonne que je suis, dans ses termes, un Horcruxe de Lord Voldemort, ou plus généralement, que certains aspects de ma personnalité viennent de lui~?~>> Au moment même de la dire, Harry se rendit compte de la stupidité de sa question et de la montagne d'indices en ce sens qu'il avait déjà observés…

<<~Il est \emph{impossible} que Dumbledore n'ait pas remarqué ça, dit le professeur Quirrell. Ce n'est pas vraiment subtil. Qu'est-ce que Dumbledore aurait dû croire~? Que tu es un acteur dans une pièce dont l'auteur imbécile n'a jamais rencontré un véritable enfant de onze ans~? Seul un bègue attardé pourrait croire… Bah, laisse tomber.~>>

Ils regardèrent le Miroir en silence.

Le professeur Quirrell dit enfin~: <<~J'ai peur d'avoir été trop malin pour mon bien. Ni toi ni moi ne souhaitons être réfléchi par ce Miroir. Je vais devoir ordonner au professeur Chourave de défaire mes Oubliettes lancés à M. Nott et Mlle Greengrass… Vois-tu, l'autre difficulté majeure posée par le Miroir, c'est qu'il ignore les forces extérieures qui pourraient affecter celui qui le regarde, comme le sortilège de faux souvenirs ou celui de Confusion. Le Miroir ne reflète que les forces nées de la personne elle-même, que l'état d'esprit qui résulte de ses choix propres - c'est du moins ce qui est souvent écrit. C'est pour cela que je me suis arrangé pour que M. Nott et Mlle Greengrass pensent tous deux que la Pierre doit être récupérée, mais pour des raisons différentes.~>> Le professeur Quirrell se frotta l'arête du nez. <<~J'ai soufflé d'autres raisons à d'autres élèves, qui viendront à mon signal… mais à mesure que ce jour approchait, je me suis senti de plus en plus pessimiste. Néanmoins, et n'ayant pas de meilleure idée, les raisons de Nott et Greengrass méritent d'être essayées. Je me demande quand même si Dumbledore a essayé de construire ce puzzle pour qu'il résiste précisément aux ruses de Voldemort. Je me demande s'il n'a pas réussi. Si tu as une autre idée et que je t'autorise à l'essayer, \parsel{je promets de ne jamais faire de mal aux pions que j’utiliserais, ni aujourd'hui, ni jamais~; et je ne m'attends pas à rompre cette promessse.} Et je te rappelle que les vies de Mlle Granger et de tous les autres dépendent de ma réussite.~>>

Ils regardèrent à nouveau dans le Miroir, le vieux et le jeune Tom Jedusor.

<<~Je pense, professeur, dit Harry au bout d'un moment, que toutes votre hypothèses sur le besoin de vouloir la Pierre pour de bonnes raisons sont fausses. Le directeur n'aurait pas choisi une règle de ce genre.

--- Pourquoi~?

--- Parce que Dumbledore sait à quel point il est simple de croire qu'on fait le bien alors que ce n'est pas le cas. Ce serait la première chose qui lui viendrait à l'esprit.

--- \parsel{Esst-ce que tu dis vrai, ou est-ce que tu esssaies de m'avoir~?}

--- \parsel{Je ssuis honnête}.~>> répondit Harry.

Le professeur Quirrell hocha la tête.

<<~Dans ce cas, d'accord.

--- Je ne sais pas pourquoi vous pensez qu'on peut résoudre cette énigme, dit Harry. Il suffirait d'avoir une règle comme~: on doit avoir une petite pyramide bleue et deux grandes pyramides rouges dans la main gauche, et on doit verser de la mayonnaise sur un hamster de la main droite…

--- Non, dit le professeur Quirrell. Non, je ne pense pas. Les légendes n'explicitent pas quelles règles sont autorisées, mais je pense que cela doit avoir un rapport avec l'usage premier du Miroir… un rapport avec les désirs et les souhaits profonds d'une personne. Pour la plupart des gens, verser de la mayonnaise sur un hamster n'entre pas dans cette catégorie.

--- Hmm, dit Harry. Peut-être que la règle dit que la personne ne doit pas vouloir utiliser la Pierre… non, c'est trop facile, l'histoire que vous avez raconté à M. Nott résout ce problème.

--- En un sens, tu comprends peut-être Dumbledore mieux que moi, dit le professeur Quirrell. Donc réfléchis à cela~: comment Dumbledore utiliserait-il son acceptation de la mort pour m'empêcher d'obtenir la Pierre~? C'est ce qu'il me croit le moins capable d'appréhender, et il n'a pas tout à fait tort.~>>

Harry y réfléchit un moment, eut quelques idées et les rejeta. Puis il pensa à quelque chose et préféra se taire… jusqu'à ce qu'il prévoie l'inévitable moment où le professeur Quirrell lui demanderait en Fourchelangue s'il avait eu une idée.

Harry parla avec réticence.

<<~Est-ce que Dumbledore pourrait penser que ce Miroir peut atteindre l'au-delà~? Est-ce qu'il pourrait mettre la Pierre dans ce qu'il \emph{croirait} être l'au-delà, si bien que seuls ceux qui croient à l'au-delà pourraient la voir~?

--- Hmm… dit le professeur Quirrell. C'est possible… oui, cela semble plausible. En utilisant la capacité du Miroir à montrer leurs désirs profonds aux autres… Albus Dumbledore se verrait réunit avec sa famille. Il se verrait uni avec eux \emph{dans la mort}, il voudrait mourir plutôt que de les voir revenir à la vie. Son frère Aberforth, sa sœur Ariana, ses parents Kendra et Percival… et je pense que Dumbledore aurait donné la pierre à Aberforth. Le Miroir saurait-il que c'est Aberforth en particulier qui a reçu la Pierre~? Ou n'importe quel parent mort ferait-il l'affaire, si celui qui fait face au miroir croit que l'esprit d'un membre de leur famille va leur donner la Pierre~?~>> Le professeur Quirrell faisait les cent pas le long d'un petit cercle, loin de Harry et du Miroir. <<~Mais ce n'est là qu'une idée. Trouvons-en une autre.~>>

Harry commença à se tapoter la joue, puis s'arrêta à la seconde où il comprit d'où lui venait ce geste.

<<~Et si c'est Perenelle qui a caché la Pierre~? Peut-être qu'elle a dit au Miroir de ne donner la Pierre qu'à la personne qui l'a placée dedans.

--- Perenelle a vécu assez longtemps pour connaître ses limites, dit le professeur Quirrell. Elle ne surestime pas son intelligence, elle n'est pas orgueilleuse. Sans quoi elle aurait perdu la Pierre il y a longtemps. Elle n'essaiera pas de trouver une bonne règle pour le Miroir elle-même, pas si Maître Flamel peut laisser cela au sage Dumbledore… mais la règle qui dit de ne rendre la Pierre qu'à celui qui l'a mise dans le Miroir fonctionne aussi si c'est Dumbledore qui l'y a mise. La règle serait difficile à tromper, je ne pourrai pas simplement faire croire à quelqu'un qu'il a placé la Pierre… je devrai créer une fausse pierre, un faux Miroir, rejouer la scène…~>> Le professeur Quirrell fronça les sourcils. <<~Mais Dumbledore pourrait toujours croire qu'avec assez de temps, Voldemort serait capable d'organiser cela. Dans le mesure du possible, Dumbledore voudra que le Miroir s'ouvre face à un état d'esprit que je ne \emph{pourrais pas} susciter chez un pion - ou que je serais incapable de concevoir, comme le fait d'accepter sa propre mort. C'est pour cela que ta première idée m'a semblé plausible.~>>

Puis Harry eut une idée.

Il n'était pas sûr qu'elle soit bonne.

… mais ce n'était pas comme s'il avait vraiment le choix.

<<~Nous ne savons peut-être pas ce qui est nécessaire à l'obtention de la Pierre. Mais une condition \emph{suffisante} verrait Albus Dumbledore ou peut-être quelqu'un d'autre, persuadé que le Seigneur des Ténèbres a été vaincu, que la menace est passée, et qu'il est temps de prendre la Pierre, de la rendre à Nicholas Flamel. Nous ne savons pas ce qui, dans cet état esprit, sera important, sera, selon Dumbledore, inaccessible à Voldemort~; mais sous ces conditions, l'état d'esprit global de Dumbledore devrait \emph{suffire}.

--- Raisonnable, dit le professeur Quirrell. Et alors~?

--- La stratégie qui en découle, dit Harry avec prudence, est d'imiter l'état d'esprit de Dumbledore sous ces conditions, de façon aussi détaillée que possible, debout face au Miroir. Cet état d'esprit devra avoir été produit par des forces internes, pas des forces externes.

--- Mais comment en arriver là sans Légilimancie ou sortilège de Confusion, qui nécessitent tous deux des forces… ah. Je \emph{vois}.~>> Les yeux de glace du professeur Quirrell devinrent soudain perçants. <<~Tu suggères que je me lance le sortilège de confusion \emph{moi-même}, comme tu l'as fait sur toi au premier jour de magie de bataille. Pour que ce soit une force interne, un état d'esprit issu de mes choix. Dis-moi si tu as suggéré cela afin de me piéger, petit. Dis-le en Fourchelangue.

--- \parsel{Vous m'avez demandé de trouver une stratégie, mais peut-être ais-je été influencé par des buts annexes - qui sait~? Je ssavais que vous me ssoupçonneriez, diriez ce que vous venez de dire. C'est votre choix, professseur. Ssur le rissque d'un échec, je n'en ssais pas plus que vous. Ne me traitez pas de traître si vous choisisssez d'esssayer et que cela échoue.}~>> Harry eut soudain très envie de sourire, mais il se retint.

<<~Merveilleux~>>, dit le professeur Quirrell qui, lui, souriait bien. <<~J'imagine que certaines menaces venues d'esprits inventifs ne peuvent même pas être neutralisées par un interrogatoire en Fourchelangue.~>>

\later

Harry mit la Cape d'Invisibilité sur ordre du professeur Quirrell afin d'\parsel{empêcher l'homme qui sse croira être le directeur de te voir}, comme l'avait dit le professeur Quirrell en Fourchelangue.

<<~Avec ou sans Cape, tu te tiendras à proximité du Miroir~>>, avait aussi dit le professeur Quirrell. <<~Si un jet de lave jaillit, tu brûleras aussi. Cette symétrie me semble juste.~>>

Le professeur Quirrell désigna un endroit à droite de la porte par laquelle ils étaient entrés, hors de la zone de réflexion du Miroir. Sous la Cape, Harry s'y rendit sans discuter. Il était de moins en moins sûr que la mort des deux Jedusor soit une mauvaise chose, même avec la vie de centaines d'autres élèves en jeu. En dépit de toutes ses bonnes intentions, Harry s'était surtout comporté en idiot, et le retour de Lord Voldemort menaçait la Terre entière.

(De toute façon, Harry ne voyait pas Dumbledore s'amuser avec de la lave. Il était probablement assez remonté contre Voldemort pour faire fi de sa modération habituelle, mais de la lave n'arrêterait pas définitivement une entité incorporelle).

Puis le professeur Quirrell dirigea sa baguette vers Harry, et un cercle scintillant apparut autour de lui, à même le sol. Le professeur Quirrell dit que ce cercle deviendrait bientôt un Cercle de Dissimulation majeur, et que rien à l'intérieur du cercle ne pourrait être vu ou entendu de l'extérieur. Harry ne pourrait pas se faire remarquer par le faux Dumbledore, ni en enlevant sa Cape, ni en criant.

<<~Tu ne traverseras \emph{pas} le cercle après son activation, dit le professeur Quirrell. Cela interagirait avec ma magie, et le sortilège de Confusion m'empêchera peut-être de savoir comment mettre fin à la résonance qui nous détruirait tous deux. De plus, puisque je ne veux pas que tu jettes tes chaussures…~>> le professeur Quirrell fit un autre geste, et à l'intérieur du Cercle de Dissimulation majeur surgit un autre léger scintillement, une distorsion sphérique. <<~\parsel{Cette barrière explosera si quoi que ce soit la touche.} La résonance m'atteindra peut-être ensuite, mais tu seras mort aussi. Maintenant, dis-moi en Fourchelangue que tu ne comptes pas traverser ce cercle, enlever ta Cape ou faire \emph{quoi que ce soit} d'impulsif ou de stupide. Dis-moi que tu attendras tranquillement ici, sous la Cape, jusqu'à ce que ce soit fini.~>>

Ce que Harry fit.

Puis les robes du professeur Quirrell devinrent noires, teintées d'or, telles que Dumbledore les porterait lors d'une occasion solennelle, et il dirigea sa baguette vers sa tête.

Il demeura longtemps immobile, sa baguette toujours sur sa tête. Ses yeux étaient fermés~; il se concentrait.

Puis le professeur Quirrell dit~: <<~\emph{Confundus}.~>>

Le visage de l'homme changea aussitôt~; il cligna plusieurs fois des yeux, comme hébété, et abaissa sa baguette.

Une profonde lassitude apparut sur le visage qu'avait porté le professeur Quirrell~; sans changement visible, il parut plus vieux, et les quelques rides de son visage attiraient plus l'attention.

Ses lèvres dessinaient un sourire triste.

Sans se presser, l'homme marcha tranquillement jusqu'au Miroir. Comme s'il avait tout son temps.

Il arriva dans la zone réfléchie par le Miroir, rien ne se passa, et il regarda à la surface de celui-ci.

Ce que l'homme pouvait y voir, Harry l'ignorait. Il lui sembla que la surface plate et parfaite reflétait toujours la pièce qui lui faisait face, comme un portail vers un autre lieu.

<<~Ariana, souffla l'homme. Mère, père. Et toi, mon frère, c'est fait.~>>

L'homme se tint immobile. Il semblait écouter.

<<~Oui, c'est fait, dit l'homme. Voldemort est venu jusqu'au miroir, et a été piégé par la méthode de Merlin. Il n'est maintenant qu'une horreur de plus sous scellé.~>>

À nouveau, l'immobilité.

<<~J'aimerais pouvoir t'obéir, mon frère, mais cela vaut mieux ainsi.~>> L'homme s'inclina. <<~La mort lui est pour toujours refusée~; cette vengeance est suffisamment terrible.~>>

Harry sentit un tiraillement, la sensation que ce n'était \emph{pas} ce que Dumbledore aurait dit. On aurait cru à un automate, une imitation superficielle… mais ce n'était pas non plus le véritable esprit d'Aberforth, c'était le professeur Quirrell imaginant Dumbledore imaginant Aberforth, et cette image doublement réfléchie du frère ne remarquerait rien…

<<~Il est temps de rendre la Pierre Philosophale, dit l'homme qui se prenait pour Dumbledore. Elle doit revenir à la garde de Maître Flamel, maintenant.~>>

Immobilité. Écoute.

<<~Non, dit l'homme, Maître Flamel l'a gardée toutes ces années à l'abri de ceux qui recherche l'immortalité, et je pense que c'est entre ses mains qu'elle sera le plus en sûreté… non, Aberforth, je pense qu'il a de bonnes intentions.~>>

Harry n'arrivait pas à contrôler la tension qui l'avait saisie, comme de l'électricité pure~; il avait du mal à respirer. Imparfait, le sortilège de Confusion du professeur Quirrell avait été imparfait. La personnalité sous-jacente du professeur suintait, remarquait l'évidente question~: si l'immortalité était horrible, pourquoi Nicholas Flamel pourrait-il avoir la Pierre~? Même si le professeur Quirrell avait imaginé que Dumbledore était aveugle à cette question, il n'avait pas précisé dans le sortilège que \emph{l'image d'Aberforth de Dumbledore} n'y penserait pas~; et tout cela n'était au fond qu'un reflet de l'esprit du professeur Quirrell, une image venue de l'intelligence de Tom Jedusor…

<<~La détruire~? dit l'homme. Peut-être. Je ne suis pas sûr que ce soit \emph{possible}, sans quoi Maître Flamel l'aurait fait il y a longtemps. Je pense qu'il a souvent regretté de l'avoir créée… Aberforth, je lui ai promis, et nous n'avons ni son ancienneté ni sa sagesse. La Pierre Philosophale doit revenir à la garde de celui qui l'a faite.~>>

Et Harry cessa de respirer.

L'homme tenait un morceau irrégulier de verre écarlate, peut-être aussi grand que la phalange supérieure du pouce de Harry. La surface patinée du verre écarlate lui donnait un air mouillé~; c'était l'apparence du sang, suspendu dans le temps et devenu une surface brute.

<<~Merci, mon frère~>>, dit doucement l'homme.

\emph{Est-ce à ça que la Pierre devrait ressembler~? Est-ce que le professeur Quirrell sait à quoi la vraie Pierre devrait ressembler~? Est-ce que le Miroir doit rendre la Pierre dans de telles conditions, ou seulement fabriquer une imitation et la donner à la place~?}

Et alors…

<<~Non, Ariana, dit l'homme avec un doux sourire, j'ai peur de devoir partir, à présent. Sois patiente, ma très chère, je vous rejoindrai vraiment bientôt… pourquoi~? Eh bien, je ne sais pas vraiment pourquoi je dois partir… maintenant que j'ai la Pierre, je dois m'écarter du Miroir et attendre que Maître Flamel me contacte, mais je ne sais pas pourquoi je devrais m'écarter du Miroir pour faire ça…~>> L'homme soupira. <<~Ah, je me fais vieux. Heureusement que cette horrible guerre a pris fin, maintenant. Je ne vois pas pourquoi je ne pourrais pas te parler un moment, ma très chère, si c'est ce que tu souhaites.~>>

Harry commençait à avoir mal au crâne~; une partie de lui essayait de lui signaler qu'il n'avait pas respiré depuis longtemps, mais personne ne l'écoutait. \emph{Imparfait}, le sortilège de Confusion du professeur Quirrell avait été imparfait, l'image d'Ariana de Dumbledore de Quirrell voulait parler à Dumbledore, et refusait d'attendre, peut-être parce que, quelque part, le professeur Quirrell savait qu'il n'y avait pas vraiment d'au-delà, et que la pulsion préparée disant de partir une fois la Pierre obtenue \emph{ne résistait pas aux arguments de Jedusor-Ariana…}

Puis il se sentit devenir très calme. Il se remit à respirer.

Dans tous les cas, Harry n'y pouvait pas grand-chose. Le professeur Quirrell avait empêché Harry d'intervenir~; eh bien, qu'il récolte les conséquences de cette décision. Tant pis si Harry tombait avec lui.

L'homme qui se croyait être Dumbledore hochait patiemment la tête et répondait parfois à sa très chère sœur. Il jetait parfois un regard agité d'un côté~; comme s'il ressentait une forte envie de partir, mais il la réprimait avec la grande patience, politesse et préoccupation pour sa sœur, comme le professeur Quirrell avait imaginé que Dumbledore ferait.

Harry le vit à l'instant où le sortilège de Confusion se dissipa~: le visage de l'homme changea, redevint celui du professeur Quirrell.

Et au même instant, le Miroir changea. Il ne montra plus à Harry le reflet de la pièce mais le véritable Albus Dumbledore, exactement comme s'il s'était tenu juste derrière le Miroir et que l'on avait pu voir au travers.

Le visage du vrai Dumbledore était dur et sévère.

<<~Bonjour, Tom~>>, dit Albus Dumbledore.
%  LocalWords:  ven balefire noitilov detalo partxe tnere ruoy becafruoy wo
%  LocalWords:  hsi
