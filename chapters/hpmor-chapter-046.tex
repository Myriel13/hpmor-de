\partchapter{Humanisme}{IV}

\lettrine{L}{e} dernier morceau de soleil s'enfonçait sous l'horizon, la lumière rouge quittait les toits, seul le ciel bleu éclairait les six personnes debout sur l'herbe séchée par l'hiver, constellée de neige, près d'une cage vacante au sol de laquelle se trouvait une cape vide, en lambeaux.

Harry se sentait… eh bien, de nouveau \emph{normal}. Plus ou moins sain d'esprit. Le sort n'avait pas annulé cette journée et les dégâts qu'elle avait causés, il n'allait pas faire comme si les blessures n'avaient jamais eu lieu, mais ses douleurs avaient été… soignées, arrangées~? C'était difficile à décrire.

Dumbledore aussi semblait en meilleure forme, bien que pas totalement rétabli. La tête du vieux sorcier se détourna un instant et ses yeux se plantèrent dans ceux du professeur Quirrell avant de revenir à Harry.

<<~Harry, dit-il, es-tu sur le point de t'écrouler de fatigue, peut-être même de mourir~?

--- Assez étrangement, non, dit Harry. Ça m'a arraché quelque chose, mais bien moins que ce que je pensais.~>> \emph{Ou peut-être que ça m'a donné une chose en retour, tout en m'en prenant une autre…} <<~Honnêtement, je m'attendais à ce que mon corps s'effondre lourdement à peu près maintenant.~>>

Il y eut le son distinctif d'un corps-qui-s'effondrait-assez-lourdement.

<<~Merci de vous être occupé de cela, Quirinus,~>> dit Dumbledore au professeur Quirrell, qui se tenait maintenant derrière les formes inconscientes des trois Aurors et les surplombait. <<~J'avoue être encore un peu patraque. Je m'occuperai néanmoins moi-même des sorts d'amnésie.~>>

Le professeur Quirrell inclina la tête puis regarda Harry.

<<~Je vais omettre une bonne dose d'incrédulité inutile, dit le professeur Quirrell, des remarques quant au fait que Merlin lui-même n'a jamais pu accomplir cela, etc. Allons droit à la question importante. Par tous les suaves serpents susurrants, qu'est que c'était que \emph{ça}~?

--- Le Patronus, dit Harry. Version 2.0.

--- Je me réjouis de voir que tu es de nouveau toi-même, dit Dumbledore. Mais vous n'irez \emph{nulle part}, jeune Serdaigle, avant de me dire en quoi cette pensée heureuse et réconfortante consistait exactement.

--- Hmm…~>> dit Harry. Il se tapota la joue d'un index contemplatif. <<~Je me demande si je devrais~?~>>

Le professeur Quirrell sourit soudain.

<<~S'il te plaît~? dit le directeur. Un joli s'il te plaît avec du sucre par-dessus~?~>>

Harry ressentit une impulsion soudaine, et il décida de lui obéir. C'était dangereux, mais une meilleure opportunité ne se produirait peut-être pas avant la fin des temps.

<<~Trois sodas~>>, dit Harry à sa bourse, puis il releva les yeux vers le professeur de Défense et le directeur de Poudlard. <<~Messieurs, dit Harry, j'ai acheté ces sodas lors de mon premier passage sur la plate-forme neuf trois-quart, le jour où je suis arrivé à Poudlard. Je les ai réservés pour une occasion spéciale~; un enchantement mineur garantit qu'ils seront bus au bon moment. C'est tout ce qui me reste, mais je pense que plus belle occasion se présenta jamais. Alors~?~>>

Dumbledore prit une canette des mains de Harry, et ce dernier en jeta un autre au professeur Quirrell. Les deux hommes plus âgés marmonnèrent des charmes identiques sur la canette et froncèrent ensuite brièvement les sourcils. Harry, lui, ouvrit simplement la canette et but.

Le professeur Quirrell et le directeur de Poudlard l'imitèrent poliment.

Harry dit~: <<~Mon rejet absolu de la mort m'a semblé être la réponse évidente.~>>

Ce n'était peut-être pas le genre de sentiment heureux nécessaire au Patronus, mais ça allait tout de même dans le top 10 de Harry.

Les regards qu'il reçut du professeur de Défense et du directeur le rendirent nerveux l'espace d'un court instant, alors que l'Hilari-Thé disparaissait~; mais les deux hommes se jetèrent un bref coup d'œil et semblèrent décider tous deux qu'ils ne pouvaient pas se permettre de faire quoi que ce soit de vraiment horrible à Harry en présence de l'autre.

<<~M. Potter, dit le professeur Quirrell, même \emph{moi}, je sais que ce n'est pas comme ça que les choses sont censées fonctionner.

--- En effet, dit Dumbledore. Explique.~>>

Harry ouvrit la bouche, puis, alors que la compréhension le frappait soudain, la referma aussi sec. Godric Gryffondor ne l'avait dit à personne, pas plus que Rowena, si encore elle l'avait su~; un nombre inconnu de mages l'avaient peut-être découvert et étaient restés muets. On ne pouvait pas l'oublier si on \emph{savait} que c'était ce qu'on essayait de faire~; quand on comprenait \emph{comment} ça fonctionnait, alors la forme animale de votre Patronus ne pourrait plus jamais fonctionner -- et la plupart des sorciers n'avaient pas reçu une éducation leur permettant de faire face aux Détraqueurs et de les détruire -

<<~Euh, désolé, dit Harry. Mais je viens de me rendre compte à l'instant que ce serait une \emph{incroyable mauvaise idée} que de vous l'expliquer avant que vous ne compreniez certaines choses par vous-même.

--- Harry, est-ce vrai~? dit lentement Dumbledore. Ou fais-tu seulement semblant d'être sage -

--- \emph{Directeur~!}~>> dit le professeur Quirrell, et il semblait sincèrement choqué. <<~M. Potter vous a dit qu'on ne saurait parler de ce sort avec ceux qui ne peuvent le lancer~! On ne sonde pas plus avant un sorcier sur de tels sujets~!

--- Si je vous disais… commença Harry.

--- Non, dit le professeur Quirrell d'un ton plutôt sévère. Vous ne nous dites pas \emph{pourquoi}, M. Potter, vous nous dites simplement que nous ne pouvons pas savoir. Si vous souhaitez formuler un indice, vous le faites avec précaution, en prenant votre temps, pas au milieu d'une conversation.~>>

Harry hocha la tête.

<<~Mais, dit le directeur. Mais, mais qu'est-ce que je vais dire au Ministère~? On ne peut pas juste \emph{perdre} un Détraqueur~!

--- Dites-leur que je l'ai mangé~>>, dit le professeur Quirrell, et Harry s'étouffa sur la canette qu'il avait portée à ses lèvres sans réfléchir. <<~Ça ne me dérange pas. Et si on rentrait, M. Potter~?~>>

Ils commencèrent tous deux à marcher le long du chemin de terre qui menait à Poudlard, laissant derrière eux Albus Dumbledore, qui regardait d'un air désespéré la cage vide et les trois Aurors endormis qui attendaient leur sortilège d'amnésie.

\latersection{Après-coup, Harry Potter et le professeur Quirrell~:}

Ils marchèrent un moment avant que le professeur Quirrell ne parle, et tous les bruits alentours se turent alors.

<<~Vous êtes exceptionnellement doué lorsqu'il s'agit de tuer quelque chose, cher élève, dit le professeur Quirrell.

--- Merci, répondit sincèrement Harry.

--- Je ne cherche pas à vous forcer la main, dit le professeur Quirrell, mais si par miracle le directeur était le \emph{seul} auquel vous ne souhaitiez pas confier le secret…~?~>>

Harry considéra cela. Le professeur Quirrell n'était déjà pas capable de lancer un Patronus.

Mais on ne pouvait pas reprendre un secret une fois qu'il avait été dit, et Harry apprenait assez vite pour se rendre compte qu'il devrait au moins \emph{réfléchir} un moment avant de lâcher ce secret dans la nature.

Il secoua la tête, et le professeur Quirrell hocha la sienne en signe d'acceptation.

<<~Par pure curiosité, professeur Quirrell, dit Harry, si le fait d'amener le Détraqueur à Poudlard avait fait partie d'un plan maléfique, quel aurait été son but~?

--- Assassiner Dumbledore lorsqu'il était affaibli, dit le professeur Quirrell sans même hésiter. Hmm. Le directeur vous a dit qu'il me soupçonnait de quelque chose~?~>>

Harry ne dit rien pendant une seconde, il essayait de trouver une réponse, puis il abandonna lorsqu'il se rendit compte qu'il avait déjà répondu.

<<~Intéressant… dit le professeur Quirrell. M. Potter, il n'est pas hors de question qu'un complot ait \emph{bien} eu lieu aujourd'hui. Le fait que votre baguette s'égare non loin de la cage du Détraqueur \emph{pourrait} avoir été un accident. Ou l'un des Aurors pourrait avoir été victime de l'Imperius, du sortilège de Confusion ou d'une Legilimancie permettant d'exercer une influence sur lui. Flitwick et même moi-même ne devrions pas être exclus des suspects dans vos calculs. On peut remarquer que le professeur Rogue a annulé tous ses cours aujourd'hui, et je soupçonne qu'il est assez puissant pour se Désillusionner lui-même~; les Aurors ont jeté des sortilèges de détection au début, mais ils ne les ont pas réitérés juste avant votre tour. Mais plus simple que tout cela, M. Potter, le méfait aurait pu avoir été commis par Dumbledore lui-même~; et si c'\emph{est} le cas, eh bien, il a peut-être aussi pris des mesures pour détourner vos soupçons.~>>

Ils marchèrent de quelques pas.

<<~Mais pourquoi \emph{ferait-il} cela~?~>>, dit Harry.

Le professeur de Défense resta silencieux un moment, puis il dit~:

<<~M. Potter, quelles mesures avez-vous prises pour vous renseigner sur la personnalité du directeur~?

--- Pas beaucoup~>>, dit Harry. Il ne s'était rendu compte que récemment… <<~vraiment pas assez.

--- Alors j'observerai, dit le professeur Quirrell, qu'on ne découvre pas tout ce qu'il y a à savoir au sujet d'un homme en interrogeant uniquement ses amis.~>>

Ce fut au tour de Harry de faire quelques pas en silence sur le chemin de terre légèrement battue qui menait à Poudlard. Il était vraiment censé être plus malin que ça, à présent. Le terme technique était biais de confirmation~; cela voulait dire, entre autres choses, que quand on choisit ses sources d'information, il existe une tendance notable à choisir des sources qui sont en accord avec ses opinions actuelles.

<<~Merci, dit Harry. En fait… je ne vous l'ai pas dit plus tôt, n'est-ce pas~? Merci pour \emph{tout}. Si un autre Détraqueur vous menace jamais, ou tant qu'on y est, s'il vous agace un peu, faites-le-moi savoir et je lui présenterai Monsieur Lumière. Je n'aime pas quand les Détraqueurs agacent un peu mes amis.~>>

Cela lui valut un regard indéchiffrable de la part du professeur Quirrell.

<<~Vous avez détruit le Détraqueur parce qu'il me menaçait~?

--- Euh, dit Harry, j'avais plus ou moins déjà décidé de le faire avant cela, mais oui, ça aurait été une raison suffisante en soi.

--- Je vois, dit le professeur Quirrell. Et qu'auriez-vous fait quant à cette menace si votre sortilège \emph{n'avait} \emph{pas} fonctionné et que le Détraqueur n'avait pas été détruit~?

--- Plan B, dit Harry. Prendre le Détraqueur dans un métal dense doté d'un point de fusion élevé, probablement du tungstène, puis le jeter dans un volcan actif et espérer qu'il atterrisse au centre du manteau Terrestre. Ah, toute la planète est remplie de lave fondue en dessous de la surface -

--- Oui, dit le professeur Quirrell, je sais~>>. Le professeur de Défense avait un sourire très étrange. <<~Je devrais vraiment y avoir pensé moi-même, tout bien considéré. Dites-moi, M. Potter, si vous vouliez perdre quelque chose là où personne ne le trouverait jamais, ou le mettriez-vous~?~>>

Harry considéra la question.

<<~J'imagine que je ne devrais pas vous demander \emph{ce que} vous avez trouvé et qui a besoin d'être perdu -

--- C'est exact,~>> dit le professeur Quirrell, ce à quoi Harry s'était attendu~; puis~: <<~Peut-être l'apprendrez-vous quand vous serez plus âgé~>>, ce à quoi Harry ne s'était pas attendu.

<<~Eh bien, dit Harry, à part essayer de lui faire atteindre le cœur en fusion de la planète, vous pourriez l'enterrer dans de la roche solide un kilomètre sous terre dans un endroit choisi aléatoirement -- peut-être l'y téléporter, s'il existe une façon de le faire à l'aveugle, ou creuser un trou pour le réparer ensuite~; l'important est de ne laisser aucune trace qui puisse mener là, que ce ne soit qu'un mètre cube anonyme quelque part sous la croûte terrestre. Vous pourriez le laisser tomber dans la fosse Marianne, c'est la fosse océanique la plus profonde de la planète -- ou juste choisir une autre fosse océanique au hasard, pour que ce soit moins évident. Si vous pouviez le rendre flottant et invisible, vous pourriez le jeter dans la stratosphère. Ou idéalement vous le lanceriez dans l'espace, avec une protection contre la détection et un facteur d'accélération à fluctuation variable qui le mènerait hors du système solaire. Et bien sûr, vous vous jetteriez un sort d'Oubliettes ensuite, pour que même vous ne connaissiez pas son emplacement exact.~>>

Le professeur de Défense riait, et ce son était encore plus étrange que son sourire.

<<~Professeur Quirrell~? dit Harry.

--- Toutes d'excellentes suggestions, dit le professeur Quirrell. Mais dites-moi, M. Potter, pourquoi ces cinq-là en particulier~?

--- Hein~? dit Harry. C'étaient juste celles qui m'ont semblé évidentes.

--- Oh~? dit le professeur Quirrell. Mais elles présentent un motif intéressant, voyez-vous. On pourrait même y voir les fruits de jeux du sort. Je dois admettre, M. Potter, que même si cette journée a eu ses hauts et bas, elle fut étonnamment bonne dans l'ensemble.~>>

Et ils continuèrent de marcher le long du chemin qui menait aux portes de Poudlard, à une bonne distance l'un de l'autre~; car Harry, sans même y penser, restait automatiquement loin du professeur de Défense, assez pour ne pas déclencher cette sensation funeste qui, pour une raison mystérieuse, semblait étrangement puissante à ce moment précis.

\latersection{Après-coup, Daphné Greengrass~:}

Hermione avait refusé de répondre à la moindre question, et dès qu'elles eurent dépassé la bifurcation qui menait aux donjons de Serpentard, Daphné et Tracey avaient immédiatement pris la poudre d'escampette, marchant aussi vite qu'elles le pouvaient. Les rumeurs voyageaient vite à Poudlard, et il leur fallait donc se rendre aux donjons sur le champ si elles voulaient être les premières à raconter l'histoire à tout le monde.

<<~Maintenant rappelle-toi, dit Daphné, ne balance pas le coup du baiser dès qu'on entre, d'accord~? Ça marche mieux si on raconte toute l'histoire dans l'ordre.~>>

Tracey hocha la tête avec excitation.

Et dès qu'elles entrèrent dans la salle commune de Serpentard, Tracey Davis prit une grande inspiration et hurla~: <<~\emph{Tout le monde~! Harry Potter ne pouvait pas lancer de Patronus et le Détraqueur l'a presque mangé et le professeur Quirrell l'a sauvé mais alors Potter est devenu méchant jusqu'à ce que Granger le ramène avec un baiser~! C'est le véritable amour, c'est sûr~!}~>>

C'était une narration ordonnée, en un sens, songea Daphné.

La nouvelle échoua à produire la réaction attendue. La plupart des filles lui jetèrent un coup d'œil puis restèrent sur leurs sofas, et les garçons continuèrent simplement de lire dans leurs chaises.

<<~Oui~>>, dit Pansy avec aigreur, de là où elle était assise, avec les pieds de Gregory sur ses genoux, affalée, lisant ce qui semblait être un livre de coloriage, <<~Millicent nous l'a déjà dit.~>>

\emph{Comment -}

<<~Pourquoi est-ce que \emph{tu} ne l'as pas embrassé d'abord, Tracey~? dirent Flora et Hestia Carrow depuis leur chaise. Maintenant Potter va épouser une Sang-de-Bourbe~! \emph{Tu} aurais pu être son véritable amour et aller dans une riche maison noble et tout si seulement tu l'avais embrassée en premier~!~>>

Le visage de Tracey était l'image même de la compréhension soudaine.

<<~\emph{Quoi} \emph{?} piailla Daphné. L'amour ne fonctionne pas comme ça~!

--- Bien sûr que si,~>> dit Millicent depuis l'endroit où elle pratiquait une espèce de sortilège tout en regardant les eaux tourbillonnantes du lac de Poudlard à travers une fenêtre. <<~Le premier baiser remporte le prince.~>>

\emph{<<~Ce n'était pas leur premier baiser~!} hurla Daphné. Hermione était \emph{déjà} son véritable amour~! C'est pour ça qu'\emph{elle} a pu le ramener~!~>> Puis Daphné se rendit compte de ce qu'elle venait de dire et grimaça intérieurement, mais après tout, il fallait bien s'adapter à son public.

<<~Whoa, whoa, whoa, quoi~?~>> dit Gregory, bondissant sur ses pieds, loin des genoux de Pansy. <<~Qu'est-ce que c'est que ça~? Mlle Bulstrode n'a pas raconté cette partie de l'histoire.~>>

Tout le monde regardait Daphné à présent.

<<~Ah, ouais, dit Daphné, Harry l'a repoussée et il a crié “\emph{Je t'ai dit pas de bisou~!}” Puis il a hurlé comme s'il était mourant et Fumseck a commencé à chanter pour lui -- je ne suis plus sûre de ce qui s'est produit en premier, en fait -

--- Eh bien \emph{moi}, je ne trouve pas que ça ressemble à du véritable amour, dirent les jumelles Carrow. On dirait que c'est la \emph{mauvaise personne} qui l'a embrassé.

--- C'était censé être \emph{moi}~>>, murmura Tracey. Son visage exprimait encore la surprise. <<~\emph{J'}étais censée être son véritable amour. Harry Potter était \emph{mon} général. J'aurais dû, j'aurais dû me battre contre Granger, pour lui -~>>

Daphné pivota vers elle, scandalisée~:

<<~\emph{Toi~?} Prendre Harry des bras de Hermione~?

--- Ouais~! dit Tracey. Moi~!

--- Tu es dingue, énonça Daphné d'un ton convaincu. \emph{Même} si tu l'avais embrassé en premier, tu sais ce que ça aurait fait de toi~? La pauvre petite fille amoureuse qui meurt à la fin de l'Acte 2.

--- \emph{Retires ça~!}~>> hurla Tracey.

Pendant ce temps, Gregory avait traversé la pièce jusqu'à l'endroit où Vincent faisait ses devoirs. <<~M. Crabbe, dit Gregory d'une voix basse, je pense que M. Malfoy devrait être mis au courant.~>>

\latersection{Après-coup, Hermione Granger~:}

Hermione regardait le papier scellé à la cire à la surface duquel était simplement écrit le nombre \emph{42}.

\emph{J'ai trouvé pourquoi on ne pouvait pas lancer le Patronus, Hermione, ça n'a rien à voir avec la possibilité qu'on ne soit pas assez heureux. Mais je ne peux pas te le dire. Je ne pourrais même pas le dire au directeur. Ça doit être encore plus secret que la métamorphose partielle, pour l'instant en tout cas. Mais si tu dois jamais combattre un Détraqueur, le secret est là, crypté, afin que quelqu'un qui ne sait pas que c'est au sujet des Détraqueurs et du Patronus ne sache pas ce que cela veut dire…}

Elle avait dit à Harry qu'elle l'avait vu mourir, qu'elle avait vu ses parents mourir, tous ses amis mourir, tout le monde mourir. Elle ne lui avait pas parlé de sa terreur de mourir seule car, sans qu'elle sache pourquoi, celle-ci était encore trop douloureuse pour être prononcée.

Harry lui avait dit qu'il s'était souvenu de la mort de ses parents, et qu'il avait trouvé ça drôle.

\emph{Il n'y a pas de lumière là où le Détraqueur t'emmène, Hermione. Pas de chaleur. Pas d'empathie. C'est un lieu où tu ne peux même pas comprendre le bonheur. Il y a de la douleur, et de la peur, et elles peuvent toujours te pousser à agir. Tu peux haïr et prendre du plaisir à détruire ce que tu hais. Tu peux rire en regardant d'autres gens souffrir. Mais tu ne peux jamais être heureux, tu ne peux même pas te souvenir de ce qui n'est plus là… je ne pense pas pouvoir jamais t'expliquer ce dont tu m'as sauvé. D'habitude, j'ai honte de causer des tracas aux autres, je ne peux pas supporter qu'on fasse des sacrifices pour moi, mais pour cette unique fois je dirai que peu importe ce que ce ça te coûtera de m'avoir embrassé, ne doute jamais une seconde que tu as bien agi.}

Hermione ne s'était pas rendu compte à quel point le Détraqueur l'avait \emph{peu} touchée, à quel point les ténèbres où il l'avait emmenée avaient été réduites, superficielles.

Elle avait vu tout le monde mourir, et ça avait réussi à lui faire mal.

Hermione remit le papier dans sa bourse, comme une bonne fille se devait de le faire.

Mais elle avait vraiment eu envie de le lire.

Elle avait peur des Détraqueurs.

\latersection{Après-coup, Minerva McGonagall~:}

Elle se sentait glacée~; elle n'aurait pas dû être si frappée, elle n'aurait pas dû avoir autant de mal à faire face à Harry, mais après ce qu'il avait traversé… elle avait examiné le jeune garçon qui lui faisait face, à la recherche du moindre signe de détraquage, et elle avait échoué à en trouver. Mais quelque chose dans le calme avec lequel il posait une question porteuse de si sombres présages lui causait une grande inquiétante. <<~M. Potter, il est impensable que je discute de tels sujets sans la permission du directeur~!~>>

Le garçon dans son bureau reçut cette information sans changer d'expression. <<~Je préférerais ne pas déranger le directeur avec cette affaire,~>> dit calmement Harry Potter. <<~J'\emph{insiste} pour qu'il ne soit pas dérangé, en fait, et vous avez promis que notre conversation resterait privée. Alors laissez-moi le formuler ainsi. Je sais qu'il existe bel et bien une prophétie. Je sais que vous êtes celle qui l'a initialement entendue de la bouche du professeur Trelawney. Je sais que la prophétie identifie l'enfant de James et Lily comme un danger pour le Seigneur des Ténèbres. Et je sais qui je suis, en fait tout le monde sait maintenant qui je suis, alors vous ne révélerez rien de dangereux ou de nouveau si vous me dites simplement ceci~: Quelle était la \emph{formulation exacte} qui \emph{m'a} identifiée, moi, l'enfant de James et Lily~?~>>

La voix creuse de Trelawney fit écho dans son esprit…
\prophesy{Il naîtra de ceux qui l'ont trois fois défié,\\
il sera né lorsque mourra le septième mois…}

<<~Harry, dit le professeur McGonagall, je ne peux pas te dire ça~!~>> Que Harry en sache autant la glaçait déjà jusqu'aux os, elle n'arrivait pas à imaginer comment Harry avait pu apprendre -

Le garçon la regarda avec des yeux étranges, emplis de peine.

<<~Ne pouvez-vous pas éternuer sans la permission du directeur, professeur McGonagall~? Car je vous promets que j'ai une bonne raison de vous poser cette question, et une bonne raison de garder cette question privée.

--- S'il te plaît Harry, ne fais pas ça, murmura-t-elle.

--- Très bien, dit Harry. Une simple question. S'il vous plaît. Le nom de famille Potter a-t-il été \emph{nommément} mentionné~? La prophétie a-t-elle littéralement dit “Potter”~?~>>

Elle regarda Harry pendant un bon moment. Elle n'aurait su dire ni pourquoi ni d'où lui venait ce sentiment que c'était un moment critique, qu'elle ne pouvait ni rejeter ni accepter cette requête avec légèreté -

<<~Non, dit-elle enfin. S'il te plaît, Harry, ne m'interroge plus.~>>

Le garçon sourit, et lui sembla que c'était avec une légère tristesse, et il dit~: <<~Merci, Minerva. Vous êtes quelqu'un de bon et d'honnête.~>>

Et, alors qu'elle était encore bouche bée sous l'effet de la surprise, Harry Potter se leva et quitta son bureau~; et ce n'est qu'alors qu'elle comprit que Harry avait pris son refus pour une réponse, et la bonne réponse qui plus est -

Harry referma la porte derrière lui.

La logique s'était présentée à lui avec l'étrange clarté du diamant. Harry n'aurait su dire si cela lui était venu pendant le chant de Fumseck, ou peut-être même avant.

Lord Voldemort avait tué James Potter. Il avait préféré épargner la vie de Lily Potter. Il avait donc continué son attaque dans le seul but de tuer leur jeune enfant.

Les Seigneurs des Ténèbres n'étaient généralement pas effrayés par les bébés.

Donc il y avait une prophétie disant que Harry Potter était un danger pour Lord Voldemort, et Lord Voldemort avait été au courant de cette prophétie.

<<~\emph{Je te donne la rare chance de t'échapper. Mais je ne ferai pas l'effort de te maîtriser, et ta mort, ici, ne sauvera pas ton enfant. Écarte-toi, femme imbécile, si tu as le moindre bon sens~!}~>>

Cela avait-il été un caprice, lui donner cette chance~? Mais alors il n'aurait pas essayé de la persuader. La prophétie l'avait-elle mis en garde contre toute tentative de tuer Lily Potter~? Mais alors il \emph{aurait} fait l'effort de la maîtriser. Lord Voldemort avait été \emph{moyennement} enclin à ne pas tuer Lily Potter. La préférence avait été plus forte qu'un caprice, mais pas aussi forte qu'un avertissement.

Supposons alors que quelqu'un que Lord Voldemort considérait comme un allié moindre ou un serviteur, quelqu'un d'utile sans être indispensable, ait supplié le Seigneur des Ténèbres d'épargner la vie de Lily. Celle de Lily, mais pas celle de James.

Cette personne aurait su que Lord Voldemort attaquerait la maison des Potter. Elle aurait été à la fois au courant de la prophétie et du fait que le Seigneur des Ténèbres était au courant. Sans quoi elle n'aurait pas supplié pour que Lily ait la vie sauve.

Selon le professeur McGonagall, excepté elle, les deux autres personnes au courant de la prophétie étaient Albus Dumbledore et Severus Rogue.

Severus Rogue, qui avait aimé Lily avant qu'elle ne soit Lily Potter, et qui avait haï James.

Rogue, donc, avait entendu la prophétie, et il l'avait répétée au Seigneur des Ténèbres. Ce qu'il avait fait parce que la prophétie n'avait pas nommément mentionné les Potter. Ça avait été une énigme, et Severus ne l'avait résolue que trop tard.

Mais s'il avait été le \emph{premier} à entendre la prophétie et qu'il avait été prêt à la transmettre au Seigneur des Ténèbres, alors pourquoi l'aurait-il aussi dite à Dumbledore, ou au professeur McGonagall~?

Donc c'était soit Dumbledore soit le professeur McGonagall qui l'avaient entendue en premier.

Le directeur de Poudlard n'avait aucune raison évidente de faire part d'une prophétie extrêmement cruciale et délicate au professeur de Métamorphose. Mais le professeur de Métamorphose avait toutes les raisons du monde de le dire au directeur.

Il semblait donc probable que ce soit le professeur McGonagall qui ait été la première à l'entendre.

La probabilité à priori disait que ça avait été de la bouche professeur Trelawney, le professeur de Divination à demeure de Poudlard. Les voyantes étaient rares, donc si vous comptiez la plupart des secondes que le professeur McGonagall avait passé en présence d'une voyante au cours de sa vie, la plupart de ces secondes-voyantes seraient des secondes-Trelawney.

Le professeur McGonagall l'avait dit à Dumbledore, et elle ne l'aurait dite à personne d'autre sans en avoir reçu la permission.

C'était donc Albus Dumbledore qui s'était arrangé pour que Severus Rogue apprenne la prophétie, d'une façon ou d'une autre. Et Dumbledore lui-même avait résolu l'énigme avec succès, sans quoi il n'aurait pas choisi \emph{Severus}, qui avait jadis aimé Lily, comme intermédiaire.

Dumbledore s'était délibérément arrangé pour que Lord Voldemort entende la prophétie, dans l'espoir de l'attirer vers sa fin. Peut-être que Dumbledore s'était arrangé pour que Severus n'entende qu'une \emph{partie} de la prophétie, ou peut-être existait-il d'autres prophéties dont Severus était resté ignorant… par un moyen ou un autre, Dumbledore avait su qu'une attaque \emph{immédiate} sur les Potter mènerait forcément à la défaite \emph{immédiate} de Lord Voldemort, même si Lord Voldemort lui-même n'avait pas cru cela. Ou peut-être que cela avait juste été un coup de chance dans la folie de Dumbledore, dans son goût pour les complots étranges…

Severus avait fini par se retrouver au service de Dumbledore~; peut-être les Mangemorts ne l'auraient-ils pas regardé d'un œil tendre s'ils avaient appris le rôle qu'il avait joué dans leur défaite.

Dumbledore avait essayé de s'arranger pour que la mère de Harry soit épargnée. Mais cette partie de son plan avait échoué. Et il avait sciemment condamné James Potter à mourir.

Dumbledore était responsable de la mort des parents de Harry. \emph{Si} tout le raisonnement logique était correct. Harry ne pouvait honnêtement pas dire que mettre fin à la guerre des sorciers ne comptait pas comme circonstance atténuante. Mais cela… \emph{l'embêtait quand même beaucoup}.

Et l'heure était venue, et même depuis longtemps passée, de demander à Drago Malfoy ce que l'\emph{autre} camp de cette guerre avait à dire sur la personnalité d'Albus Percival Wulfric Brian Dumbledore.

%  LocalWords:  thuddish Erm untell
