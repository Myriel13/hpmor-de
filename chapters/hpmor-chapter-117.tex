\chapter{Après-coup, quelque chose à protéger~: Minerva McGonagall}

\lettrine{N}{ous} étions le lendemain matin. Tous les élèves s'étaient assemblés silencieusement autour des quatre tables de Poudlard, Harry James Potter-Evans-Verres parmi eux. Il s'était effondré d'épuisement la nuit précédente et s'était réveillé au matin, toujours dans les vapes, avec la Pierre Philosophale dans sa chaussette gauche.

On aurait pu croire qu'une épidémie de peste avait ravagé la Grande Table.

Le trône de Dumbledore avait disparu, et rien n'était venu le remplacer. Le centre de la Grande Table était vide.

Severus Rogue occupait un siège flottant, l'équivalent magique d'un fauteuil roulant.

Le professeur Chourave manquait à l'appel. La veille, on avait dit à Harry qu'une assemblée de Legilimens l'examinerait à la recherche d'ordres laissée dans son esprit mais qu'elle ne serait probablement pas inquiétée. Le professeur Chourave n'était probablement qu'une victime, et Harry avait fait tout son possible pour le souligner. Le Survivant avait déclaré n'avoir vu dans l'esprit de Voldemort aucune preuve que Chourave avait agi intentionnellement.

Le professeur Flitwick manquait toujours à l'appel, probablement pour rester au chevet de Hermione.

Le professeur Sinistra manquait à l'appel et Harry ne savait ni où elle était, ni ce qu'elle faisait.

Comme l'aurait fait une couverture de survie, l'engourdissement qui enveloppait son esprit lui apportait plus de protection que de réconfort. Des scènes où des robes noires tombaient et du sang se répandait s'y jouaient l'espace d'un instant avant d'être repoussées. Il y ferait face plus tard, pas maintenant. Plus tard serait mieux~: pour ce qui était de les surmonter, le futur Harry détiendrait un avantage comparatif.

Quelque part en lui se cachait la peur qu'il ne souffrirait \emph{pas}, qu'il n'y aurait aucun prix à payer. Mais cette peur pouvait aussi être remise à plus tard.

Le petit déjeuner n'était pas apparu sur les tables. Les élèves assis à côté de Harry attendaient, en silence, effrayés. Interdiction avait été faite aux chouettes d'entrer ou de sortir de Poudlard depuis la veille au soir.

Les portes de la Grande Salle s'ouvrirent à nouveau et la directrice adjointe Minerva McGonagall s'avança. Ses robes étaient noires, son crâne dépourvu de son chapeau de sorcière habituel. Ses cheveux gris-chatain-blonds étaient tressés en une natte, comme dans l'attente d'un chapeau~; mais pour l'instant, Harry n'avait d'yeux que pour cette tête qu'il voyait découverte pour la première fois.

Minerva McGonagall s'avança vers le pupitre situé devant la Grande Table.

Tous les regards étaient braqués sur elle.

«J'ai peur d'avoir de nombreuses nouvelles à vous annoncer», dit Minerva. À son accent Écossais sec se mêlait de la tristesse. «Et la plupart sont terribles. D'abord. Si c'est moi qui m'adresse à vous, c'est parce que le directeur de Poudlard, Albus,» elle buta sur le mot, «Percival Wulfric Brian Dumbledore nous a été pris. Vous-Savez-Qui l'a enfermé hors du Temps et nous ignorons si nous pourrons un jour le ramener. Nous… nous avons perdu… celui qui a peut-être été le meilleur directeur… que Poudlard ait jamais eu.»

Des susurrements d'horreur s'élevèrent des tables, pas des hoquets ou des plaintes, seulement de fortes inspirations~; la plupart venues de Gryffondor, et certaines de Poufsouffle et Serdaigle. La mauvaise nouvelle avait déjà circulée, mais l'autorité venait de la confirmer.

«Ensuite. Vous-Savez-Qui est brièvement revenu, mais il est à nouveau mort. Il ne restait que ses mains, serrées autour du cou de Mlle Granger. Nous pensons qu'il n'est plus une menace.» Minerva McGonagall inspira à nouveau. «Troisièmement. Le professeur Quirrell est mort baguette en main, face à Vous-Savez-Qui. Nous l'avons retrouvé non loin de là où Vous-Savez-Qui est mort, victime de son sortilège de la Mort.» Une autre susurration d'horreur confirmée, cette fois venue des quatre tables.

Minerva inspira encore. «La nuit dernière, nous avons aussi perdu celui a peut-être été le meilleur professeur de Défense de l'histoire de Poudlard. Ses qualités pédagogiques seules… Notre professeur de Défense a porté de nombreux noms, mais son vrai nom était David Monroe. Étant le dernier représentant de la Noble et Très Ancienne Maison Monroe, ses funérailles - ses secondes funérailles, les vraies - se dérouleront dans deux jours, dans la Chambre Très Ancienne du Magenmagot. Mais nous aurons aussi une veillée pour le professeur de Défense de Poudlard, pour notre professeur Quirrell, ici, dans ce château. Cet homme est aussi mort en portant les robes de professeur de Poudlard, avec autant d'honneur qu'il soit possible.»

Harry écoutait en silence en résistant aux larmes qui remontaient dans ses yeux. Ce n'était même pas \emph{vrai}, et encore moins surprenant. Et pourtant, c'était douloureux à entendre. Assis juste à côté, Anthony Goldstein posa une main réconfortante sur celle de Harry, et Harry ne bougea pas.

«Quatrièmement. Une nouvelle extrêmement heureuse et inattendue. Hermione Granger est en vie ainsi qu'en bonne santé mentale et physique. Mlle Granger est sous observation à l'hôpital de Sainte Mangouste pour voir si ce qui lui est arrivé - quoi que ce puisse être - lui a laissé des effets secondaires. Mais compte tenu de sa situation précédente, son état de santé est stupéfiant.»

Si la nouvelle était tombée par surprise ou que sa préface avait été différente, de folles exclamations de joies auraient jailli de Serdaigle et Gryffondor. Ici, Harry ne vit que quelques brefs sourires. Peut-être avaient-ils sauté de joie plus tôt, mais il n'y avait pour l'instant que du silence. Harry comprenait. Il ne poussait pas de cris de joies non plus. Pas pour le moment.

«Enfin…» la voix de Minerva McGonagall faiblit, puis elle se reprit. «J'ai peur d'avoir la plus grave des nouvelles à annoncer à certains de nos élèves. Il semblerait que Vous-Savez-Qui a convoqué ceux qui étaient jadis ses partisans~; et nombre d'entre eux ont obéi, peut-être en vertu d'une loyauté profondément malavisée, peut-être par peur pour leur famille en cas de refus. Il semble qu'un sacrifice était requis pour achever la résurrection de Vous-Savez-Qui~; ou peut-être Vous-Savez-Qui les a-t-il jugés responsables de sa défaite. On a retrouvé trente-sept corps~; on ignorait qu'il avait autant de partisans à l'extérieur d'Azkaban. J'ai peur…» la voix de Minerva McGonagall faiblit une fois de plus. «J'ai peur que de les parents de nombre de nos élèves soient à compter parmi les victimes…»

non non non non non non NON NON NON NON

Comme attirés par un horrible aimant, les yeux de Harry se fixèrent sur l'image d'horreur absolue qu'était le visage de Drago Malfoy alors même que le coton réconfortant qui enveloppait les pensées de Harry était arraché comme du tissu.

Comment avait-il pu ne pas penser, ne pas se rendre compte…

Quelque part, quelqu'un criait déjà~; et pourtant la pièce semblait très silencieuse.

«Sheila, Flora, et Hestia Carrow. Ont perdu leur deux parents hier. Élèves ayant perdu leurs pères~: Robert Jugson. Ethan Jugson. Sara Jugson. Michael MacNair. Riley et Randy Rookwood. Lily Lu. Sasha Sproch. Daniel Gibson. Jason Gross. Elsie Ambrose…»

\emph{Peut-être que Lucius a compris, peut-être qu'il a été assez malin pour ne pas venir, peut-être qu'il a compris que c'était Voldemort qui s'en était pris à Drago…}

«… Theodore Nott. Vincent Crabbe. Gregory Goyle. Drago Malfoy. C'est tout.»

Un élève de Gryffondor poussa un cri de joie et fut immédiatement giflé par un autre si fort qu'un Moldu en aurait perdu une dent.

«Trente points de moins pour Gryffondor, et en retenue tout le premier mois de l'année prochaine», dit le professeur McGonagall d'une voix qui aurait fendu des pierres.

«\emph{Mensonges~!} glapit un grand Serpentard qui s'était levé. \emph{Mensonges~! Mensonges~! Le Seigneur des Ténèbres reviendra et il… il vous apprendra à tous le sens de…}

--- M. Jugson», dit la voix de Severus Rogue. Même si la voix était vacillante, même si elle ne ressemblait pas à celle du maître des potions, même si elle était faible, le Serpentard se tut. «Robert. Le Seigneur des Ténèbres a tué votre père.»

Robert Jugson poussa un cri de furie terrifiée et se rua hors de la salle~; Drago se replia sur lui-même comme une maison qui s'écroule et émit des sons que personne n'entendit parce les bavardages avaient déjà commencé.

Harry se leva à moitié et s'arrêta

qu'est-ce que tu dirais à Drago il n'y a rien à lui dire tu ne peux pas aller le voir faire semblant d'être son ami

tu veux arranger les choses améliorer les choses mais tu ne peux pas c'est impossible qu'est-ce que tu lui as fait qu'est-ce que tu as fait à Vincent à Gregory qu'est-ce que tu as fait à Theodore

Le monde devant Harry se brouilla, il vit à peine Padma Patil se lever, s'avancer vers la table Serpentard et vers Drago, Seamus s'avancer vers Theodore.

Et parce qu'il avait lu les livres de science-fiction et de fantasy et de son père, parce qu'il avait déjà lu cette scène dix fois quand elle était arrivée à d'autres protagonistes, une image de Maugrey Fol-Œil apparut dans son esprit, une image de l'homme balafré appelé Alastor. Et son image lui dit, du même ton qu'il aurait eu avec Albus Dumbledore, que les Mangemorts avaient tous pointé leur baguette vers Harry, qu'ils avaient déjà choisi de porter la Marque, qu'ils avaient été coupables de crimes impensables peut-être même pour Harry, qu'ils avaient renoncé aux protections offertes aux bons, qu'ils s'étaient désignés comme cible potentielle le jour où un sacrifice devrait être fait. Que vouloir protéger les parents innocents de Harry de tortures et d'un voyage à Azkaban l'avait rendu nécessaire, que vouloir protéger le monde de Voldemort l'avait rendu nécessaire. Que tuer des Mangemorts prêts à vous abattre était moralement bien plus simple que les actes quotidiens des Aurors ordinaires et des juges, qui rendaient la justice dans des contextes bien moins clairs mais néanmoins nécessaires à la société. Que si les actes de Harry n'avaient pas été justes, s'il n'était pas juste d'agir de façon \emph{beaucoup} plus moralement ambigüe, alors la société des hommes ne pouvait pas exister. Que si on était donné d'une once de sens commun, il était impossible de juger Harry. Neville ne l'aurait pas jugé, McGonagall non plus, Dumbledore non plus, et même Hermione, quand elle saurait, lui dirait que c'était la bonne chose à faire.

Et tout cela était vrai.

Tout comme il était vrai qu'une partie de l'esprit de Harry avait calculé que l'élimination de l'élite politique des Puristes du Sang faciliterait grandement son travail de reconstruction de l'Angleterre magique. Cela n'avait pas été important, mais cela avait été pris en compte dans son calcul, dans ces instants de réflexion hâtive, lors d'une rapide vérification des conséquences à long terme de son acte, au cas où elles auraient été catastrophiques, et lors de la découverte que ces conséquences seraient plutôt bienvenues. Et cette vérification avait oublié que les Mangemorts avaient des enfants à Poudlard et que l'un d'eux ressemblait beaucoup au père de Drago. Cela n'aurait rien changé. Cela n'aurait rien changé du tout. Mais, avec seulement quelques secondes pour agir, c'était là tout ce que l'esprit de Harry avait calculé.

Si jamais les proches de Mangemort avaient le moindre souci financier, Harry pourrait au moins faire quelque chose. Métamorphoser de l'or et utiliser la Pierre pour rendre la métamorphose permanente - à moins que fabriquer de telles quantités d'or ne pose un problème à l'économie magique ou ne dérange les gobelins qui ne comprenaient pas les économies de marché monétaristes… enfin, ce n'était pas comme si Harry n'avait pas de compétences utiles à monnayer.

D'autres morceaux de coton étaient maintenant arrachés de son esprit.

«Il semble probable», dit Minerva d'un ton toujours bas, mais néanmoins plus distinct que tous les autres sons, «que certains de nos élèves auront perdu leurs gardiens légaux. Si vous devenez pupille de Poudlard, sachez que j'honorerai mes responsabilités avec un sérieux absolu. Vous serez parfaitement traité. Votre chambre forte familiale sera gérée de façon compétente. Je ferai de mon mieux pour traiter chacun d'entre vous comme mes enfants… et je vous protégerai comme j'aurais protégé les miens, ni plus, ni moins. J'espère que c'est clair pour \emph{TOUT LE MONDE À POUDLARD.}»

Les élèves hochèrent vivement la tête.

«Bien», dit-elle. Sa voix se radoucit. «Alors il reste une seule chose à faire.»

Avec un air triste et solennel, le professeur Sinistra émergea d'une porte latérale. Elle portait des robes blanches plutôt que son marron habituel, et avait préféré un chapeau carré à pompons presque gris à son chapeau de sorcière habituel.

Dans les mains du professeur Sinistra se trouvait le Choixpeau magique.

Avec l'air de suivre un rituel immuable depuis des siècles, Aurora Sinistra posa un genou au sol devant Minerva McGonagall et lui présenta le Choixpeau magique des deux mains.

Minerva McGonagall prit le Choixpeau magique des mains du professeur Sinistra et le posa sur sa tête.

Il y eut un long silence.

«DIRECTRICE~!

--- Comme Albus Dumbledore n'est pas mort», dit Minerva d'une voix si basse que les élèves eurent du mal à l'entendre, «mais qu'il nous a été pris, j'accepte ce poste uniquement par intérim… jusqu'au retour de Dumbledore.»

Un cri perçant traversa la Grande Salle et Fumseck fut là, au-dessus des quatre tables. Il décrivit un lent arc en spirale, survola chacune des tables, et fredonna du bec un chant de loyauté absolue, un chant qui survivrait à tous les feux du monde physique. \emph{Attendez}, semblait dire le chant. \emph{Attendez son retour, et faites-lui honneur.}

Fumseck décrivit trois cercles au-dessus d'elle, des plumes l'effleurèrent, des larmes commencèrent à couler sur ses joues~; puis l'oiseau s'élança vers une lucarne au-dessus de la Grande Salle et s'en fut. 

%  LocalWords:  Rookwood Sproch
