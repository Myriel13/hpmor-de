\chapter{Information à priori}

\lettrine{U}{n} garçon attend dans une petite clairière, à la lisière de la forêt non-interdite, près d'un chemin de terre qui se prolonge vers les portes de Poudlard d'un côté et vers l'horizon de l'autre. Une charrette se trouve non loin, et le garçon se tient à bonne distance de celle-ci. Il la regarde fixement.

Au loin, une silhouette s'approche le long du chemin de terre~: Un homme en robe de fonction se traînant lentement, ses épaules affaissées, ses chaussures élégantes soulevant de petits nuages de poussière le long de ses pas.

Une demi-minute plus tard, le garçon lui jette un autre coup d'œil rapide avant de revenir à sa surveillance~; et cet aperçu lui montre que les épaules de l'homme se sont redressées, que son visage s'est affermi, et que ses chaussures marchent maintenant avec légèreté au-dessus de la terre sans laisser de trace de poussière derrière elles.

<<~Bonjour, professeur Quirrell, dit Harry sans laisser ses yeux se détourner à nouveau de la charrette.

--- Salutations, dit la voix calme du professeur Quirrell. Vous semblez garder vos distances, M. Potter. J'imagine que vous ne remarquez rien d'étrange quant à notre moyen de transport~?

--- Étrange~? répondit Harry en écho. Ma foi non, je ne peux pas dire que j'y vois quoi que ce soit d'étrange. Tout semble y être dans les proportions habituelles~: quatre sièges, quatre roues, deux immenses chevaux squelettes ailés…~>>

Un crâne entouré de peau se détourna et lui montra ses dents, solides et blanches dans leur bouche caverneuse et noire, comme pour indiquer que l'affection était réciproque. L'autre cheval squelette, noir et tanné, fit un geste de la tête, comme pour piaffer, mais aucun son ne fut produit.

<<~Ce sont des Sombrals, et ils ont toujours tiré la charrette~>>, dit le professeur Quirrell d'un ton assez paisible, alors qu'il montait sur le banc avant de la charrette et s'asseyait le plus à droite possible. <<~Ils ne sont visibles qu'à ceux qui ont vu la mort et l'ont comprise, une défense utile contre la plupart des prédateurs. Hmm. J'imagine que la première fois que vous avez fait face au Détraqueur, votre pire souvenir s'est avéré être la nuit de votre rencontre avec Celui-Dont-On-Ne-Doit-Pas-Prononcer-Le-Nom~?~>>

Harry hocha la tête d'un air lugubre. Le professeur Quirrell avait deviné juste, mais pour les mauvaises raisons. \emph{Ceux qui ont vu la mort…}

<<~De ce fait, vous êtes-vous souvenu de quoi que ce soit d'intéressant~?

--- Oui, dit Harry, en effet~>>, cela seulement, et rien de plus, car il n'était pas encore prêt à émettre des accusations.

Le professeur de Défense eut un de ses sourires secs et lui fit signe de venir d'un doigt impatient.

Harry franchit la distance qui les séparait et monta dans la charrette en grimaçant. La sensation funeste était devenue sensiblement plus forte après le Détraqueur, même si elle s'était lentement affaiblie auparavant. La distance maximale entre lui et le professeur Quirrell autorisée par la charrette ne semblait plus suffire, loin de là.

Les chevaux squelette se mirent alors au trot et la charrette se mit en mouvement, les menant vers les frontières externes de Poudlard. Le professeur Quirrell repassa en mode zombie, et même si la sensation funeste se rétracta, elle planait toujours à la limite de la perception de Harry, impossible à ignorer…

La forêt défila, alors que la charrette les emportait, les arbres passant à une vitesse qui semblait tout à fait dérisoire comparée à celle d'un balai ou même d'une voiture. Harry songea qu'il y avait quelque chose d'étrangement relaxant à voyager si lentement. Cela avait certainement détendu le professeur de Défense, qui était affaissé, un petit filet de bave sortant de sa bouche molle et faisant une flaque sur sa robe.

Harry n'avait toujours pas décidé de ce qu'il pourrait avoir à déjeuner.

Ses recherches en bibliothèque ne lui avaient pas permis de déceler le moindre signe d'un sorcier parlant à des plantes non magiques. Ou à aucun autre animal non magique, hormis les serpents, même si \emph{Sort et Parole} de Paul Breedlove avait fait le récit de la fable probablement mythique d'une sorcière nommée la Dame des Écureuils Volants.

Ce que Harry \emph{voulait} faire, c'était en parler au professeur Quirrell. Le problème était que le professeur Quirrell était \emph{trop malin}. À en juger par la réaction de Drago, cette histoire d'Héritier de Serpentard était une bombe, et Harry n'était pas certain de vouloir mettre qui que ce soit d'autre au courant. Et au moment où Harry parlerait de Fourchelangue, le professeur Quirrell le fixerait de ses yeux bleu pâle et dirait~: 'Je vois, M. Potter, vous avez donc enseigné le Patronus à M. Malfoy et avez accidentellement parlé à son serpent.'

Le fait qu'il n'y aurait pas dû y avoir de preuves assez fortes pour localiser la bonne explication parmi toutes les hypothèses n'avait \emph{aucune importance}, et c'était sans parler de la façon dont il aurait pu dépasser la force de l'improbabilité à priori. D'une façon où d'une autre, le professeur de Défense l'aurait \emph{quand même} déduit. Harry soupçonnait parfois le professeur Quirrell d'avoir bien plus d'informations qu'il ne le laissait savoir, car ses à priori étaient tout simplement trop bons. Parfois, ses déductions incroyables étaient justes même quand ses \emph{raisons} étaient mauvaises. Le problème était que Harry n'arrivait pas à voir comment le professeur Quirrell aurait pu se procurer un seul indice de plus au sujet de la moitié des choses qu'il devinait. Juste \emph{une fois}, Harry aurait aimé faire une espèce de déduction incroyable à partir d'une parole du professeur Quirrell, une déduction qui \emph{le} prendrait totalement au dépourvu.

\later

<<~Je prendrai un bol de soupe aux lentilles avec de la sauce soja, dit le professeur Quirrell à la serveuse. Et pour M. Potter, un plat de Chili Tenorman maison.~>>

Harry hésita, en plein désarroi. Il avait pris la résolution de s'en tenir aux plats végétariens pour le moment, mais il avait oublié au cours de ses délibérations que c'était le professeur Quirrell qui \emph{commandait les plats} - et il serait maladroit de protester maintenant -

La serveuse s'inclina, et se tourna pour partir -

<<~Euh, excusez-moi, ça ne contient pas de viande de serpent ou d'écureuil volant~?~>>

La serveuse ne battit pas un cil et se tourna simplement vers Harry, puis elle secoua sa tête, s'inclina de nouveau poliment et reprit son trajet vers la porte.

(Les autres parties de Harry se moquaient de lui. Gryffondor faisait des commentaires sardoniques quant au fait qu'un peu de désagrément social suffisait à le pousser au \emph{Cannibalisme~!} (crié par Poufsouffle), et Serpentard remarquait qu'il était bon que l'éthique de Harry soit flexible lorsqu'il s'agissait de buts importants tels que le maintien de bonnes relations avec le professeur Quirrell).

Après que la serveuse eut fermé la porte derrière elle, le professeur Quirrell agita une main pour fermer le loquet, prononça les quatre sortilèges d'intimité habituels, et dit~: <<~Une question intéressante, M. Potter. Je me demande pourquoi vous l'avez posée.~>>

Harry maintint une expression neutre. <<~Je m'informais sur le Patronus un peu plus tôt, dit-il. À en croire \emph{“Le Patronus~: ceux qui en étaient capables et les autres”}, il s'avère que Godric n'en était pas capable, mais que Salazar, si. J'ai été surpris, alors j'ai suivi la référence, dans \emph{Quatre vies de Pouvoir}. Et j'ai \emph{ensuite} découvert que Salazar Serpentard était censé pouvoir parler aux serpents.~>> (La séquentialité et la causalité étaient deux choses différentes, mais ce n'était pas la faute de Harry si le professeur Quirrell était passé à côté). <<~D'autres recherches révélèrent une vieille histoire au sujet d'une déesse mère capable de parler aux écureuils volants. J'étais un peu inquiet à l'idée de manger une chose douée de parole.~>>

Et Harry prit une gorgée d'eau d'un air décontracté -

- juste quand le professeur Quirrell répondit~: <<~M. Potter, aurais-je raison de deviner que vous aussi êtes Fourchelangue~?~>>

Quand Harry eut fini de tousser, il reposa son verre sur la table et fixa son regard sur le menton du professeur Quirrell plutôt que sur ses yeux, puis il dit~: <<~Vous êtes donc capable d'opérer une Légilimancie à travers mes barrières Occlumantiques.~>>

Le professeur Quirrell avait un large sourire.

<<~Je prendrai cela comme un compliment, M. Potter, mais non.

--- Je ne marche plus, dit Harry. Il est \emph{impossible} que vous en soyez arrivé à cette conclusion en vous basant sur ces informations.

--- Bien sûr que non, dit calmement le professeur Quirrell. Je comptais vous poser cette question aujourd'hui de toute façon, et j'ai simplement choisi un moment opportun. Je le soupçonne depuis décembre, à vrai dire -

--- \emph{Décembre~?} dit Harry. Je l'ai découvert \emph{hier}~!

--- Ah, alors vous ne vous étiez pas rendu compte que le message du Choixpeau à votre intention avait été en Fourchelangue~?~>>

Le professeur Quirrell avait parfaitement choisi son moment, une fois de plus, juste quand Harry avait pris un peu d'eau pour s'éclaircir la gorge après la première quinte de toux.

Harry ne s'en était \emph{pas} rendu compte, pas avant cet instant. C'était bien sûr devenu évident dès que le professeur Quirrell l'avait dit. Mais oui, le professeur McGonagall lui avait même \emph{dit} de ne pas parler aux serpents quand il pouvait être vu, mais il avait pensé qu'elle avait fait référence à des statues ou à des monuments architecturaux de Poudlard ressemblant à des serpents. Double illusion de transparence, il avait pensé qu'elle l'avait compris, elle avait pensé qu'il l'avait comprise - mais \emph{comment} -

<<~Donc, dit Harry, vous avez opéré une Légilimancie sur moi pendant mon premier cours de Défense afin de découvrir ce qui s'était passé avec le Choixpeau -

--- Alors je ne l'aurais pas découvert en décembre.~>> Le professeur Quirrell s'inclina contre le dossier de sa chaise en souriant. <<~Ce n'est pas une énigme que vous pouvez résoudre seul, M. Potter, et je vais donc vous révéler la réponse. Pendant les vacances d'hiver, j'ai été informé du fait que le directeur avait soumis une requête pour qu'un jury à huis clos réexamine le cas d'un certain M. Rubeus Hagrid, que vous savez être le Gardien des Clés de Poudlard, et qui a été accusé du meurtre de Mimi Geignarde en 1943.

--- Oh, bien sûr, dit Harry, voilà qui rend parfaitement \emph{évident} le fait que je suis un Fourchelangue. Professeur, par tous les suaves serpents susurrants -

--- L'\emph{autre} suspect du meurtre était le monstre de Serpentard, le légendaire habitant de la Chambre des Secrets. C'est la raison pour laquelle mes sources m'ont alerté de ce fait, et pourquoi il a suffisamment attiré mon attention pour que je dépense une bonne quantité de pots-de-vins afin d'apprendre les détails de l'affaire. Maintenant, M. Potter, il se trouve que M. Hagrid est innocent. Tellement évidemment innocent que c'en est ridicule. Il est le témoin à l'innocence la plus flagrante à avoir été condamné par le système judiciaire d'Angleterre magique depuis que le sort de Confusion opéré par Grindelwald sur Neville Chamberlain a été mis sur le dos d'Amanda Knox. Le directeur Dippet a usé de l'un des élèves comme d'une marionnette afin d'accuser M. Hagrid parce qu'il avait besoin d'un bouc-émissaire auquel reprocher la mort de Mlle Geignarde, et notre merveilleux système judiciaire a agréé que l'histoire était suffisamment plausible pour justifier l'expulsion de M. Hagrid et la confiscation de sa baguette. Notre directeur actuel a simplement besoin d'apporter quelques éléments de preuve supplémentaires suffisamment importants pour que le jury se réunisse de nouveau~; et puisque ce n'est plus Dippet qui fait pression mais Dumbledore, le résultat est couru d'avance. Lucius Malfoy n'a pas de raison particulière de craindre le jour où M. Hagrid sera innocenté~; il ne résistera donc à cet appel que dans la mesure où cela ne lui coûtera rien tout en imposant des frais à Dumbledore, et Dumbledore est clairement prêt à poursuivre en dépit de cela.~>>

Le professeur Quirrell but une gorgée d'eau. <<~Mais je digresse. Le nouvel élément de preuve promit par le directeur est un sortilège lié au Choixpeau, jusqu'alors jamais détecté, et qui, selon le directeur, répond uniquement aux Fourchelangues de Serpentard. Le directeur soutient que cela favorise l'interprétation selon laquelle la Chambre des Secrets a bien été ouverte en 1943, soit approximativement à l'époque où Celui-Dont-On-Ne-Doit-Pas-Prononcer-Le-Nom, qu'on sait être Fourchelangue, étudiait à Poudlard. C'est un raisonnement assez discutable, mais un jury pourrait décider que cela fait suffisamment pencher la balance pour mettre en doute la culpabilité de M. Hagrid, du moins si le jury parvient à garder un air sérieux en disant cela. Et nous en arrivons maintenant à la question cruciale~: \emph{comment} le directeur a-t-il découvert ce sort caché dans le Choixpeau~?~>>

Le professeur Quirrell avait maintenant un fin sourire.

<<~Eh bien, en supposant maintenant qu'un Fourchelangue se soit trouvé dans la fournée d'élèves annuelle, ce serait un Héritier de Serpentard potentiel. Vous devez admettre, M. Potter, que vous vous démarquez à chaque fois qu'il s'agit de trouver une personne extraordinaire. Et si je me demande alors quel nouveau Serpentard courait le plus le risques de voir son intimité mentale envahie par le directeur dans le but spécifique de trouver des souvenirs du Triage, eh bien, vous vous démarquez encore plus.~>> Le sourire disparut. <<~Vous voyez donc, M. Potter, que ce n'est pas \emph{moi} qui ai envahi votre esprit, même si je ne vous demanderai pas de présenter vos excuses. Ce n'est pas votre faute si vous avez cru aux dénégations de Dumbledore quant à son intrusion dans votre intimité mentale.

--- Mes excuses les plus sincères~>>, dit Harry en gardant une expression neutre. Le strict contrôle qu'il exerçait sur lui-même constituait une confession à lui seul, tout comme la sueur qui maculait son front~; mais il ne pensait pas que le professeur Quirrell en tirerait quelque information que ce soit. Il penserait seulement que Harry était nerveux à l'idée d'avoir été révélé être l'Héritier de Serpentard plutôt qu'à l'idée qu'il puisse se rendre compte que Harry avait délibérément choisi de trahir le secret de Serpentard… choix qui n'avait maintenant l'air plus si malin que ça.

--- Donc, M. Potter. Quelque progrès dans votre quête de la Chambre des Secrets~?~>>

\emph{Non}, pensa Harry. Mais pour maintenir un déni plausible, il fallait avoir l'habitude d'esquiver certaines questions, même lorsque l'on n'avait rien à cacher… <<~Avec tout votre respect, professeur Quirrell, si j'avais fait de tels progrès, il ne m'apparaît pas comme \emph{totalement} évident que je devrais vous en faire part.~>>

Le professeur Quirrell sirota de nouveau son verre d'eau.

<<~Dans ce cas, M. Potter, je vous dirai librement ce que je sais ou soupçonne. D'abord, je crois que la Chambre des Secrets est réelle, tout comme le monstre de Serpentard. La mort de Mlle Geignarde n'a été découverte que plusieurs heures après s'être produite, alors que l'école aurait dû instantanément alerter le directeur. Son meurtre a donc été commis soit par le directeur Dieppe, ce qui est peu probable, soit par une entité à laquelle Salazar Serpentard a donné un niveau d'accès aux donjons supérieur à celui du directeur lui-même. Ensuite, je soupçonne que, contrairement à la légende populaire, la fonction du monstre de Serpentard ne soit \emph{pas} de débarrasser Poudlard de tous les nés-Moldus. À moins que le monstre ne soit assez puissant pour vaincre le directeur et tous les enseignants, il ne pourrait triompher par la force. Plusieurs meurtres commis dans le secret feraient fermer l'école, comme cela a failli arriver en 1943, ou provoqueraient l'installation de nouveaux locaux. Alors, pourquoi le monstre de Serpentard, M. Potter~? Quel est sa véritable fonction~?

--- Ah…~>> Harry baissa les yeux jusqu'à son verre d'eau et essaya de réfléchir. <<~Pour tuer quiconque entre dans la Chambre et n'y a pas sa place -

--- Un monstre assez puissant pour vaincre une équipe de sorciers ayant brisé les meilleures barrières dont Salazar aurait pu équiper sa Chambre~? Peu probable.~>>

Harry se sentait un peu sous pression. <<~Eh bien, on l'appelle la Chambre des Secrets, alors peut-être que le monstre a un secret, ou qu'il \emph{est} le secret~?~>> D'ailleurs, quel genre de secrets la Chambre des Secrets abritait-elle~? Harry n'avait pas fait beaucoup de recherches sur le sujet, en partie parce qu'il avait eu l'impression que personne ne savait rien -

Le professeur Quirrell souriait.

<<~Pourquoi ne pas simplement écrire le secret~?

--- Ahhh… dit Harry. Parce que si le monstre parle Fourchelangue, cela assure que seul un véritable descendant de Serpentard pourrait l'entendre~?

--- Il serait assez simple de lier les barrières de la Chambre à une phrase prononcée en Fourchelangue. Pourquoi prendre la peine de créer le monstre~? Ça ne peut pas avoir été facile de créer une créature capable de vivre pendant des siècles. Allons, M. Potter, cela devrait être évident~; quels sont les secrets qui peuvent être transmis d'un être vivant à un autre, mais jamais écrits~?~>>

Harry le vit alors, avec un choc d'adrénaline qui fit bondir son cœur, et sa respiration s'accéléra. <<~\emph{Oh.}~>>

Serpentard avait en effet été fort rusé. Assez pour trouver un moyen d'outrepasser l'Interdit de Merlin.

Les sortilèges puissants ne pouvaient être transmis par des fantômes ou par des livres, mais si vous pouviez faire une créature sentiente capable de vivre assez longtemps et avec une assez bonne mémoire -

<<~Il me semble très probable, dit le professeur Quirrell, que Celui-Dont-On-Ne-Doit-Pas-Prononcer-Le-Nom a commencé son ascension au pouvoir grâce à des secrets obtenus de la bouche du monstre de Serpentard. Que le savoir perdu de Salazar est la source de l'extraordinaire puissance de Vous-Savez-Qui. D'où mon intérêt pour la Chambre des Secrets et le cas de M. Hagrid.

--- Je \emph{vois}~>>, dit Harry. Et si \emph{lui}, Harry, pouvait trouver la Chambre des Secrets de Salazar… alors tout le savoir perdu que Lord Voldemort avait obtenu serait aussi \emph{sien}.

Oui. C'était \emph{exactement} comme ça que l'histoire devrait se dérouler.

Ajoutez l'intelligence supérieure de Harry, quelques découvertes magiques originales et quelques lance-roquettes Moldus, et le combat serait totalement déséquilibré, ce qui était exactement ce que Harry souhaitait.

Harry souriait à présent, un sourire très maléfique. \emph{Nouvelle priorité~: Trouver tout ce qui à Poudlard ressemble de près ou de loin à un serpent et essayer de lui parler. En commençant avec tout ce que tu as déjà essayé, mais en parlant en Fourchelangue et pas en anglais - arranges-toi pour que Drago te laisse entrer dans les dortoirs Serpentard -}

<<~Ne vous réjouissez pas trop, M. Potter~>>, dit le professeur Quirrell. Le visage de ce dernier s'était vidé de toute expression. <<~Vous devez \emph{continuer} de réfléchir. Quels furent les derniers mots du Seigneur des Ténèbres à l'intention du monstre~?

--- \emph{Quoi~?} dit Harry. Comment l'un de nous deux pourrait-il le savoir~?

--- Visualisez la scène, M. Potter. Laissez votre imagination se charger des détails. Le monstre de Serpentard - probablement une sorte de grand serpent, afin que seul un Fourchelangue puisse lui parler - a fini de faire part de tout le savoir qu'il possède à Celui-Dont-On-Ne-Doit-Pas-Prononcer-Le-Nom. Il lui transmet l'ultime bénédiction de Salazar et le prévient que la Chambre des Secrets doit maintenant rester fermée jusqu'à ce que le prochain descendant de Salazar se révèle être assez rusé pour l'ouvrir. Et celui destiné à devenir le Seigneur des Ténèbres hoche la tête et lui dit -

--- Avada Kedavra, dit Harry se sentant soudain malade.

--- Règle numéro douze, dit doucement le professeur Quirrell. Ne laisse jamais la source de ton pouvoir traîner là où quelqu'un pourrait la trouver.~>>

Le regard de Harry descendit jusqu'à la nappe, qui s'était décorée d'un motif endeuillé fait de fleurs et d'ombres noires. Cela semblait… trop triste pour être imaginé, le grand serpent de Serpentard avait seulement voulu aider Lord Voldemort, et ce dernier avait juste… il y avait quelque chose d'insupportablement douloureux dans cette image, quel genre de personne \emph{ferait} ça à un être qui ne lui aurait offert rien d'autre que de l'amitié…

<<~\emph{Pensez-vous} que le Seigneur des Ténèbres aurait -

--- Oui, dit le professeur Quirrell d'un ton catégorique. Celui-Dont-On-Ne-Doit-Pas-Prononcer-Le-Nom a laissé un sacré sillage de corps derrière lui, M. Potter~; je doute qu'il ai omis celui-ci. S'il y avait d'autres artefacts mobiles dans la Chambre, il les aura aussi pris avec lui. Il pourrait toujours y avoir quelque chose qui mérite d'être vu dans la Chambre des Secrets, et en la trouvant vous prouveriez être le véritable Héritier de Serpentard. Mais n'y comptez pas trop. J'ai idée que tout ce que vous y trouverez seront les restes du monstre de Serpentard reposant paisiblement dans sa tombe.~>>

Ils restèrent silencieux pendant quelques instants.

<<~Je pourrais avoir tort, dit le professeur Quirrell. Ce n'est en fin de compte qu'une conjecture. Mais je souhaitais vous prévenir, M. Potter, afin que la déception ne soit pas trop douloureuse.~>>

Harry hocha légèrement la tête.

<<~On pourrait presque en venir à regretter votre victoire d'enfant~>>, dit le professeur Quirrell. Son sourire se tordit. <<~Si seulement Vous-Savez-Qui avait vécu, vous auriez pu le persuader de vous enseigner une partie du savoir qui aurait été votre héritage, transmit d'un Héritier à l'autre.~>> Le sourire se tordit encore plus, comme pour se moquer de l'impossibilité, évidente même au vu de cette supposition.

\emph{Note à moi-même}, pensa Harry, avec un léger frisson et une pointe de colère, \emph{m'assurer d'extraire mon héritage de l'esprit du Seigneur des Ténèbres, d'une façon ou d'une autre.}

Il y eut un autre silence. Le professeur Quirrell regardait Harry comme s'il attendait à ce que ce dernier lui demande quelque chose.

<<~Eh bien, dit Harry, puisqu'on en parle, pourrais-je vous demander comment toute cette histoire de Fourchelangue a -~>>

On frappa alors à la porte. Le professeur Quirrell leva un doigt, comme pour le mettre en garde, puis il l'ouvrit d'un geste. La serveuse entra, tenant en équilibre un immense plateau, sur lequel se trouvaient leurs plats, comme si l'ensemble n'avait rien pesé (de fait, c'était probablement le cas). Elle donna son bol de soupe verte et un verre de son Chianti habituel au professeur Quirrell~; et elle plaça devant Harry une assiette de petits morceaux de viande trempés dans une sauce qui semblait épaisse, plus un verre de son soda à la mélasse habituel. Puis elle s'inclina, parvenant à donner au salut une apparence de respect sincère plutôt que de politesse formelle, et elle s'en fut.

Le professeur Quirrell leva de nouveau un doigt, demandant le silence, et il sortit sa baguette.

Il commença alors un certain enchaînement d'incantations que Harry reconnut, et cela le fit inspirer brutalement. C'était la série et l'ordre que M. Bester avait utilisé, l'ensemble complet de vingt-sept sorts que vous utiliseriez avant de parler de quoi que ce soit de vraiment important.

Si la discussion sur la Chambre des Secrets \emph{n'avait pas} été importante -

Quand le professeur Quirrell en eut fini - il avait réalisé \emph{trente} sorts, dont trois que Harry n'avait jamais entendus - le professeur de Défense dit~: <<~Nous ne serons pas interrompus avant un moment. Sauriez-vous garder un secret, M. Potter~?~>>

Harry hocha la tête.

<<~Un secret sérieux, M. Potter~>>, dit le professeur Quirrell. Ses yeux étaient attentifs, son visage solennel. <<~Un secret qui pourrait m'envoyer à Azkaban. Réfléchissez avant de me répondre.~>>

Pendant un moment, Harry ne vit même pas pourquoi il pourrait être difficile de répondre à la question étant donné sa collection de secrets en constante augmentation. Puis -

\emph{Si ce secret peut envoyer le professeur Quirrell à Azkaban, ça veut dire qu'il a fait quelque chose d'illégal…}

Le cerveau de Harry fit quelques calculs. Quel que soit le secret, le professeur Quirrell ne pensait pas que son acte illégal aurait un impact négatif sur son image aux yeux de Harry. Il n'avait aucun avantage à ne \emph{pas} connaître le secret. Et si cela révélait quelque chose de mauvais au sujet du professeur Quirrell, il serait alors tout à l'avantage de Harry de le savoir, même s'il avait promis de ne le dire à personne.

<<~Je n'ai jamais eu grand respect pour l'autorité, dit Harry. Y compris les autorités légales et gouvernementales. Je garderai votre secret.~>>

Harry ne prit pas la peine de demander si la révélation valait le danger qu'elle poserait au professeur Quirrell. Le professeur de Défense n'était pas stupide.

<<~Alors je dois vérifier que vous êtes vraiment un descendant de Salazar~>>, dit le professeur Quirrell, et il se leva de sa chaise. Harry, mû plus par un réflexe et par son instinct que par un calcul, se propulsa hors de sa propre chaise.

Il y eut un flou, un passage, un mouvement soudain.

Harry interrompit son bond en arrière paniqué et se retrouva à mouliner des bras et à essayer de ne pas tomber tandis qu'un flot d'adrénaline s'écoulait frénétiquement dans ses veines.

À l'autre bout de la pièce ondulait un serpent d'un mètre de haut, aux yeux vert vif et au corps tortueusement couvert de bandes bleues et blanches. Harry ne s'y connaissait pas assez en serpents pour reconnaître l'espèce, mais il savait que 'couleurs vives' signifiait 'venimeux'.

La sensation funeste avait ironiquement diminué après que le professeur de Défense de Poudlard se fut transformé en un serpent venimeux.

Harry déglutit et dit~:

<<~Salutations - ah, hssss, non, ah, \parsel{ssalutationss.}

--- \parsel{Alors}, siffla le serpent. \parsel{Vouss parlez, j'entends. Je parle, vous entendez~?}

--- \parsel{Oui, je comprends,} siffla Harry. \parsel{Vous êtess un Animaguss~?}

--- \parsel{Bien ssûr}, siffla le serpent. \parsel{Trente-ssept règles, numéro trente-quatre~: Devenir un Animaguss. Touss les gens ssensés le font s'ils en ssont capables. Très rare, donc.}~>> Les yeux du serpent étaient des surfaces plates logées dans des abysses noires, des pupilles noires acérées au milieu d'espaces gris. <<~\parsel{C'est le moyen de sse parler le plus ssûr. Voyez-vous pourquoi~? Perssonne d'autre ne nouss comprend.}

--- \parsel{Même ss'ils ssont des sserpents Animaguss~?}

--- \parsel{Pas à moins que l'héritier de Sserpentard ne le veuille.}~>> Le serpent émit une série de courts sifflements que le cerveau de Harry traduit en un rire sardonique. <<~\parsel{Sserpentard pas sstupide. Sserpent Animaguss pas pareils que Fourchelangue. Sserait énorme défaut dans le plan.}~>>

Eh bien \parsel{voilà} qui soutenait certainement que le Fourchelangue était une magie personnelle, pas le résultat d'une population de serpents sentiente dotée d'un langage qu'un humain pouvait apprendre -

<<~\parsel{Je ne ssuis pas enregisstré}~>>, siffla le serpent. Les abysses noires fixaient Harry. <<~\parsel{Les Animaguss doivent être enregisstrés. La peine est deux ans d'emprisonnement. Garderez-vous mon ssecret, enfant~?}

--- \parsel{Oui,} siffla Harry. \parsel{Ne briserai jamais promessse.}~>>

Le serpent sembla rester immobile, comme sous l'effet d'un choc, puis il se remit à onduler. <<~\parsel{Nouss reviendronss dans ssept jourss. Amenez cape pour passser ssans être vu, et ssablier pour déplacement dans le temps.}

--- \parsel{Vous ssavez~?} siffla Harry éberlué. \parsel{Comment -}~>>

Encore la série de courts sifflements rapides qui se traduisaient par un rire sardonique.

<<~\parsel{Vous arrivez à mon premier cours alors que vous êtes encore à un autre, abattez vos ennemis avec tarte, deux Rapeltouts -}

--- \parsel{Laisssez tomber}, siffla Harry. \parsel{Quesstion sstupide, oublié que vous étiez malin.}

--- \parsel{Sserait idiot de l'oublier,}~>> dit le serpent, mais le sifflement ne révélait pas d'offense.

<<~\parsel{Ssablier resstreint,} dit Harry. \parsel{Ne peux pas utilisser avant neuvième heure.}~>>

Le serpent se tordit le cou, un hochement de tête reptilien. <<~\parsel{Nombreuses resstrictions. Utilisable sseulement par vous, ne peut être volé. Ne peut transsporter d'autres humains. Mais sserpent porté dans boursse viendra avec, je pensse. Pensse possible de tenir ssablier immobile dans coquille ssans sse faire repérer pendant que vous tournerez la coquille. Tessteronss dans ssept jours. Ne parlerons pas de sce qui vient après. Vous ne dites rien, à perssonne. Ne donnez aucun ssigne d'attente, aucun. Compris~?~>>}

Harry hocha la tête.

<<~\parsel{Répondez en parole.}

--- \parsel{Oui.}

--- \parsel{Ferez comme j'ai dit~?}

--- \parsel{Oui. Mais,}~>> Harry émit un grincement vacillant que son esprit traduit par un <<~Euuuuh~>> hésitant, <<~\parsel{Je ne promet pass de faire, quoi que ce ssoit, que vous n'avez pass dit -}~>>

Le serpent eut un frisson que l'esprit de Harry traduit par un regard sévère. <<~\parsel{Bien ssûr que non. En disscuterons à prochaine rencontre.}~>>

Le flou et le mouvement s'inversèrent, et le professeur Quirrell était de nouveau là. Pendant un moment, le professeur lui-même sembla onduler comme l'avait fait le serpent, et ses yeux semblèrent froids et plats~; puis ses épaules se redressèrent et il fut de nouveau humain.

Et l'aura de sensation funeste était revenue.

La chaise du professeur Quirrell fila vers lui, et il s'y assit. <<~Il serait absurde de gâcher ceci~>>, dit le professeur Quirrell en se saisissant de sa cuillère, <<~même si pour l'instant je préférerais une souris vivante. On ne peut jamais tout à fait délier un esprit du corps qu'il porte, voyez-vous…~>>

Harry reprit lentement son siège et commença à manger.

\later

<<~Alors finalement la lignée de Salazar n'est pas morte avec Vous-Savez-Qui, dit le professeur Quirrell après un moment. Il semblerait que des rumeurs disant que vous êtes maléfique ont déjà commencé à se répandre parmi notre cher corps étudiant~; je me demande ce qu'ils penseraient de cela s'ils le savaient.

--- Ou s'ils savaient que j'ai détruit un Détraqueur~>>, dit Harry, et il haussa les épaules. <<~J'imagine que toute cette histoire s'apaisera la prochaine fois que je ferai quelque chose d'intéressant. Hermione a du mal, cela dit, et je me demandais si vous auriez des suggestions pour elle.~>>

Le professeur de Défense but quelques cuillerées de sa soupe en demeurant silencieux, puis~; et lorsqu'il parla, sa voix était étrangement neutre~:

<<~Vous tenez vraiment à cette fille.

--- Oui, dit doucement Harry.

--- J'imagine que c'est pourquoi elle a été capable de vous faire sortir de votre détraquage~?

--- Plus ou moins~>>, dit Harry. La phrase était vraie en un sens, mais pas précise~; ce n'était pas que sa personnalité détraquée s'était souciée de Hermione mais qu'elle avait été désorientée.

<<~Je n'avais pas de tels amis quand j'étais jeune.~>> Toujours la même voix vide d'émotion. <<~Je me demande ce qui serait advenu de vous si vous aviez été seul~?~>>

Harry frissonna avant de pouvoir s'en empêcher.

<<~Vous devez éprouver de la gratitude pour elle.~>>

Harry hocha simplement la tête. Pas tout à fait exact, mais vrai.

<<~Alors voilà ce que j'aurais pu faire à votre âge, s'il y avait eu quelqu'un pour qui le faire -~>> 

%  LocalWords:  Breedlove Tenorman’s Erm Dippet Ahhh
