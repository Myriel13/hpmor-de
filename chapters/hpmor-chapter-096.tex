\partchapter{Rôles}{VII}

\section[La quatrième rencontre~:\\
(Le 17 avril 1992 à 16h38)]{La quatrième rencontre~:\\
(Le 17 avril 1992 à 16h38)\protect\footnotemark}
\authorsnotetext{Pour ceux qui n'ont pas lu l'histoire originale~: la pancarte a quelque peu changé, mais l'inscription est la même que dans la version de J.K. Rowling.}

\lettrine{L}{'homme} qui portait la manteau usé et chaud ainsi que trois légères cicatrices pour toujours gravées sur sa joue observait Harry d'aussi près que possible tandis que le garçon déambulait poliment entre les rangées de maisonnettes. Pour quelqu'un dont la meilleure amie était morte la veille, Harry Potter semblait étrangement tranquille, sans que cela ne laisse toutefois penser qu'il était insensible ou qu'il trouve cela normal. \emph{Je ne souhaite pas en parler}, avait dit le garçon, \emph{avec vous ou qui que ce soit d'autre}. Il avait dit 'souhaite' et non pas 'veux', comme pour insister sur le fait qu'il était capable d'utiliser des mots d'adulte et de prendre des décisions d'adulte. Remus n'avait pu songer qu'à une chose susceptible d'aider après avoir reçu les chouettes du professeur McGonagall et de cette étrange homme nommé Quirinus Quirrell.

"Il y a beaucoup de maisons vides," dit Harry, après avoir jeté un nouveau coup d'œil.

Godric's Hollow avait changé pendant la décennie depuis que Remus Lupin avait cessé de s'y rendre fréquemment. Nombre des vieilles maisonnettes pointues semblaient désertes, et de grandes vignes feuillues grimpaient sur les fenêtres et les portes. La Grande Bretagne s'était réduite de façon notable à la suite de la guerre des sorciers car elle n'avait pas perdu que les morts mais aussi les fuyards. Godric's Hollow avait été particulièrement touché. Plus tard, encore d'autres familles s'étaient déplacées, vers Pré-au-Lard ou la Londres magique, car les maisons désertées étaient devenues un rappel trop dérangeant.

D'autres étaient restés. Godric's Hollow était plus ancien que Poudlard, plus ancien que Godric Gryffondor, qui lui avait donné son nom, et il y avait là des familles qui résideraient ici jusqu'à la fin du monde et de sa magie.

Les Potter avaient été l'une de ces familles et le seraient à nouveau si le dernier d'entre eux le décidait.

Remus Lupin essaya d'expliquer tout cela, simplifiant de son mieux pour le jeune garçon. Le Serdaigle hochait pensivement la tête et ne parlait pas, comme s'il comprenait tout sans avoir besoin de poser de questions. Peut-être était-ce le cas~; l'enfant de James Potter et Lily Evans, préfet et préfète en chef de Poudlard, ne risquait pas d'être stupide. L'enfant lui avait certainement semblé très intelligent lors de leur courte conversation de janvier, bien qu'alors ce fut surtout Remus qui avait parlé.

(Il y avait aussi cette histoire avec le Magenmagot dont Remus avait entendu des rumeurs, mais il n'en croyait pas un mot, pas plus qu'il ne croyait que James avait promis son fils en mariage à la plus jeune fille de Molly).

"Voilà le monument," dit Remus en tendant la main vers l'avant.

\later

Harry marcha à côté de M. Lupin vers l'obélisque de marbre noir tout en réfléchissant en silence. Il lui semblait que cette aventure était profondément malavisée~; il n'avait besoin d'aucun accompagnement dans le deuil, n'ayant pas choisi cette voie. En ce qui le concernait, les cinq étapes du deuil étaient Rage, Remords, Résolution, Recherche et Résurrection (non pas que les 'cinq étapes du deuil' aient jamais bénéficié de la moindre confirmation expérimentale dont Harry ait entendu parler). Mais M. Lupin était apparu trop sincère pour que Harry puisse refuser~; et il avait semblé à Harry qu'il se devait de ne pas refuser une visite chez James et Lily. Il marchait donc, empli d'un sentiment d'étrange détachement~; il marchait silencieusement comme au milieu d'une pièce dont la lecture du texte ne l'aurait pas intéressé.

On avait dit à Harry de ne pas porter la Cape d'Invisibilité pendant son voyage afin que M. Lupin puisse le garder à l'œil.

Harry était certain que Dumbledore, et peut-être aussi Maugrey Fol-Œil, attendaient tous les deux, invisibles, pour voir si quelqu'un mordrait à l'hameçon. Il était impensable qu'on ait laissé Harry sortir de Poudlard avec Remus Lupin pour seule protection. Harry ne s'attendait pas à ce que quoi que ce soit se produise. Il n'avait rien vu venir contredire l'hypothèse selon laquelle tout le danger était concentré à Poudlard et à Poudlard seulement.

Alors qu'ils s'approchaient du centre de la ville, l'obélisque de marbre se transforma en…

Harry inspira brusquement sous le coup de la surprise. Il s'était attendu à voir un James Potter dans une pose héroïque, baguette brandie face à Lord Voldemort, et Lily Potter bras tendus devant le berceau.

Au lieu de cela on pouvait voir un homme aux cheveux et aux lunettes sales, une femme aux cheveux détachés, un bébé dans ses bras, et c'était tout.

"Ça m'a l'air très… normal," dit Harry, et il sentit que sa voix lui échappait.

"Madame Londubat et le professeur Dumbledore y ont particulièrement tenu," dit M. Lupin, qui regardait plus Harry que le monument. "Ils ont dit que nous devions nous souvenir des Potter comme ils avaient vécu, pas comme ils étaient morts."

Harry regarda la statue et réfléchit. Très étrange, de se voir sous la forme d'un bébé de pierre, sans cicatrice sur le front. C'était un aperçu d'un monde parallèle, d'un monde où Harry James Potter (sans Evans-Verres dans son nom) devenait un érudit intelligent mais ordinaire du monde des sorciers après avoir peut-être été Réparti à Gryffondor. Un Harry Potter qui grandissait pour devenir un bon petit sorcier, sans grand savoir scientifique, même avec une mère née-Moldue. Qui faisait au final… bien peu. James et Lily n'auraient pas éduqué leur fils pour qu'il ait ce que le professeur Quirrell aurait appelé \emph{l'ambition} et ce que le professeur Verres-Evans aurait appelé \emph{la poursuite de l'effort commun}. Ses parents biologiques l'auraient beaucoup aimé, et ça n'aurait pas aidé grand-monde sauf Harry. Si quelqu'un avait défait leur mort…

"Vous étiez leur ami," dit Harry en se retournant vers Lupin. "Depuis longtemps, depuis votre enfance."

M. Lupin acquiesça silencieusement.

La voix du professeur Quirrell résonna dans le souvenir approximatif de Harry~: \emph{La différence la plus probable n'est pas que vous y accordez plus d'importance. C'est plutôt que, étant une créature plus logique qu'eux, seul vous êtes conscient du fait que le rôle d'Ami exige cet acte de votre part…}

"Quand Lily et James sont morts," dit Harry, "vous êtes-vous le moins du monde demandé s'il pourrait y avoir quelque objet magique capable de les ramener~? Comme Orphée et Eurydice~? Ou comme les, comment s'appelaient-ils déjà, les frères Elrin~?"

"Aucune magie ne peut défaire la mort," dit doucement M. Lupin. "Il y a certains mystère que les sorciers ne peuvent atteindre."

"Avez-vous effectué une vérification mentale de ce que vous saviez, de pourquoi vous le saviez, et de la probabilité de cette conclusion~?"

"Quoi~?" dit M. Lupin. "Pourrais-tu répéter ça, Harry~?"

"Je vous demande si vous y avez quand même réfléchi."

M. Lupin secoua la tête.

"Pourquoi pas~?"

"Parce que c'était fait, c'était accompli," dit Remus Lupin avec gentillesse. "Parce que, où que soient James et Lily à présent, ils voudraient que j'agisse pour le bien des vivants, pas pour celui des morts."

Harry hocha silencieusement la tête. Il avait été assez certain de ce que serait la réponse avant de poser la question. Il connaissait déjà ce script. Mais il avait quand même posé la question, juste au cas où M. Lupin aurait vécu une semaine obsédé par cette question, car Harry aurait pu se tromper.

La douce voix du professeur de Défense sembla parler dans l'esprit de Harry. \emph{Si Lupin se souciait vraiment d'eux, il n'aurait certainement pas eu besoin d'une aide particulière pour faire une chose aussi simple que d'y réfléchir cinq minutes avant d'abandonner…}

\emph{Si}, répondit Harry à la voix mentale. \emph{Un humain n'obtiendrait pas cette capacité simplement parce qu'il est préoccupé. Je l'ai acquise parce que j'ai lu certains livres à la bibliothèque, produits par un immense édifice scientifique…}

Et l'autre partie de Harry dit de cette même voix douce~: \emph{Mais il y a une autre hypothèse, M. Potter, et elle cadre beaucoup plus simplement avec les observations.}

\emph{Pas du tout~! Comment les gens sauraient-ils quoi prétendre si personne ne s'était jamais soucié de son prochain~?}

\emph{Ils ne le savent pas. C'est ce que vous observez.}

Ils continuèrent d'avancer vers une certaine maison derrière une longue rangée de maisonnettes habitées et d'autres recouvertes de vignes.

Ils arrivèrent enfin à la maison dont la partie supérieure avait volée en éclat et où des feuilles vertes s'avançaient vers l'intérieur, située derrière une haie longeant l'allée devenue sauvage et qui montait à hauteur d'épaule, ainsi que derrière un étroit portail en métal (Hagrid l'avait probablement enjambé car il aurait été trop gros pour passer). Le trou dans le plafond donnait l'impression qu'une bouche géante avait prit une bouchée circulaire de la maison, laissant dépasser des tiges de bois qui avaient probablement été des poutres de soutien. À droite, une cheminée se dressait encore, épargnée par la bouchée géante, mais dangereusement penchée maintenant qu'elle était privée de son ancien tuteur. Les vitres étaient fracassées. Là où l'ont aurait dû trouver une porte d'entrée on ne trouvait plus que des échardes de bois.

Ici, Lord Voldemort était venu, \emph{en silence, il faisait moins de bruit que les feuilles mortes qui glissaient le long des pavés…}

Remus Lupin plaça une main sur l'épaule de Harry. "Touche la porte," dit M. Lupin.

Harry tendit une main et s'exécuta.

Un panneau jaillit comme une fleur à l'éclosion soudaine entre les mauvaises herbes enchevêtrées derrière le portail, un panneau de bois aux lettres d'or, et qui disait~:
\begin{center}
Ici, la nuit du 32 Octobre 1981,\\
Lily et James Potter ont perdu la vie.

Ne leur survit que leur fils, Harry Potter,\\
le seul sorcier à avoir jamais résisté au sortilège de la Mort,\\
le Survivant, qui brisa le pouvoir de Vous-Savez-Qui.

Cette maison a été laissée dans son état de ruine,\\
en monument aux Potter,\\
en rappel de leur sacrifice.
\end{center}

Dans l'espace vide sous les lettres d'or étaient inscrits d'autres messages, des dizaines, d'une encre magique qui montait à la surface et luisait assez longtemps pour être lue avant de s'estomper et de laisser place à d'autres messages~:
\begin{center}

Mon Gideon est enfin vengé

Merci, Harry Potter. Sois bénis, où que tu sois.

Nous serons à jamais débiteurs de Harry Potter

James, Lily, je suis navré.

J'espère que vous êtes en vie, Harry Potter

Il y a toujours un prix

J'aurais aimé que nos derniers mots soient plus tendres, James. Je suis navré.

Après la nuit vient toujours l'aube.

Repose sereine, Lily

Sois béni, Survivant. Tu fus notre miracle.
\end{center}

"J'imagine…" dit Harry. "J'imagine que c'est ce que les gens font… au lieu d'essayer d'arranger les choses…" Harry se tut. Cette pensée ne lui sembla pas digne de cet endroit. Il leva les yeux et vit Remus Lupin l'observer avec tant de gentillesse que Harry s'arracha à cette vue et s'intéressa au toit détruit, explosé.

\emph{Tu fus notre miracle}. Harry avait toujours entendu utiliser le mot 'miracle' dans des phrases destinées à indiquer qu'ils n'existaient pas. Et pourtant, en regardant cette maison détruite, il sut soudain exactement ce que le mot signifiait, ce soupçon de grâce inexpliquée. Le Seigneur des Ténèbres avait presque gagné, puis en une nuit toute la terreur, toutes les ténèbres s'étaient achevées, un salut injustifié, une aube soudaine jaillie des ténèbres sans que personne ne sache même \emph{pourquoi}…

Si Lily Potter avait survécu à sa confrontation avec Lord Voldemort, c'est ce qu'elle aurait ressenti en voyant que son bébé avait survécu.

"Allons-y," murmura le bébé, dix ans plus tard.

Ils partirent.

L'entrée du cimetière était protégée par un portail sans cadenas, de ceux qui maintenaient les animaux à l'écart, avec un endroit où se tenir pendant que l'on déplaçait la porte. Remus sortit sa baguette (Harry tenait déjà la sienne) et sa vue se troubla brièvement lorsqu'ils entrèrent.

Certaines des pierres plantées dans le sol avaient l'air d'être aussi vieilles que ce mur à Oxford que son père avait dit être vieux d'environ mille ans.

\emph{Hallie Fleming}, pouvait-on lire sur la première pierre que vit Harry, les mots gravés rendus presque invisibles par l'érosion du temps. \emph{Vienna Wood}, pouvait-on lire sur une autre.

Il y avait longtemps que Harry avait visité un cimetière. La dernière fois, son esprit avait encore été enfantin, longtemps avant qu'il ne plonge son regard dans l'ombre de la Mort. Venir ici, maintenant, était… étrange, triste, déroutant, et \emph{cela se produit depuis si longtemps, pourquoi les sorciers n'ont ils pas essayé de l'empêcher, pourquoi n'investissent-ils pas toutes leurs forces dans ce projet comme les Moldus dans la recherche médicale, mais avec encore plus d'intensité, les sorciers ont de meilleures raisons d'espérer…}

"Les Dumbledore vivaient aussi à Godric's Hollow~?" dit Harry lorsqu'ils dépassèrent deux pierres relativement neuves marquées \emph{Kendra Dumbledore} et \emph{Ariana Dumbledore}.

"Depuis très, très longtemps," dit M. Lupin.

Ils s'avancèrent dans le cimetière, en atteignirent presque le bout, derrière tous ces morts, tous pleurés.

Puis M. Lupin indiqua deux pierres tombales jumelles, reliées, d'un marbre encore blanc et neuf.

"Y aura-t-il des messages ici aussi~?" dit Harry. Il ne voulait plus avoir à se confronter à la façon dont les autres se confrontaient à la mort.

M. Lupin secoua la tête.

Ils s'avancèrent jusqu'aux pierres blanches reliées.

Et se tint devant…

"Qu'est-ce c'est~?" chuchota Harry. "Qui… \emph{qui a écrit ça~?}"
\begin{center}
JAMES POTTER\\
NÉ LE 28 MARS 1960\\
MORT LE 31 OCTOBRE 1981
\end{center}

"Écrit quoi~?" dit M. Lupin, interloqué.

\begin{center}
LILY POTTER\\
NÉE LE 30 JANVIER 1960\\
MORTE LE 31 OCTOBRE 1981
\end{center}

"\emph{Ça~!}" s'écria Harry. "L'\emph{inscription~!}" Des larmes s'amoncelaient dans ses yeux, face à la lueur étrangère et inexpliquée, face au soupçon de grâce là où il n'y aurait pas dû y en avoir, face à la mystérieuse bénédiction, ses larmes s'amoncelaient face à
\begin{center}
LE DERNIER ENNEMI QUI SERA DÉTRUIT EST LA MORT
\end{center}

"Ça~?" dit M. Lupin. "C'est la… devise, j'imagine qu'on pourrait dire ça, c'est la devise des Potter. Même si je pense que ça n'a jamais été aussi formel que ça. C'est juste un dicton transmis depuis très, très longtemps…"

"C'est… ce…" Harry s'agenouilla tant bien que mal face à la tombe, toucha l'inscription d'une main tremblante. "\emph{Comment~?} Ce genre de chose ne peut pas être, être \emph{génétique}…"

Puis Harry vit ce que ses larmes avaient brouillées, la légère gravure d'une ligne dans un cercle dans un triangle.

Le symbole des Reliques de la Mort.

Et Harry comprit.

"Ils ont essayé," chuchota Harry.

\emph{Les trois frères Peverell.}

\emph{Avaient-ils perdu quelqu'un qui leur était précieux, est-ce ainsi que cela avait commencé~?}

"Ils y ont consacré leur vie, ils ont essayé, ils ont progressé…"

\emph{La Cape d'Invisibilité, capable d'échapper au regard des Détraqueurs.}

"… mais ils n'ont pas terminé leurs recherches…"

\emph{Se cacher de l'ombre de la Mort ne revient pas à vaincre la Mort elle-même. La Pierre de Résurrection ne peut pas réellement faire revenir qui que ce soit. La Baguette de Sureau ne protège pas de la vieillesse.}

"… ils ont alors transmis la mission à leurs enfants et aux enfants de leurs enfants."

\emph{Génération après génération.}

\emph{Jusqu'à en venir à moi.}

Le Temps pouvait-il produire de tels échos, rimer entre un futur et un passé si lointains~? Cela ne \emph{pouvait pas} être une coïncidence. Pas ce message, pas ici.

\emph{Ma famille.}

\emph{Vous étiez véritablement mon père et ma mère.}

"Ça ne parle pas de ressusciter les morts, Harry," dit M. Lupin. "Ça parle d'accepter la mort, de pouvoir la dépasser, la maîtriser."

"Est-ce que James vous a dit ça~?" dit Harry d'une voix étrange.

"Non," dit M. Lupin, "mais…"

"Bien."

Harry se releva lentement de sa position agenouillée avec la sensation de pousser un soleil sur ses épaules, de faire se lever l'aube au-dessus de l'horizon.

\emph{Bien} sûr \emph{que d'autres sorciers ont essayé. Je ne suis pas unique. Je n'ai jamais été seul. Ces sentiments que je ressens en moi, ils ne sont pas particulièrement spéciaux, pas plus dans le monde des sorciers que dans le monde Moldu.}

"Harry, ta baguette~?" La voix de M. Lupin s'était soudain agitée, et lorsque Harry leva sa baguette pour la regarder de plus près, il vit qu'elle luisait, très légèrement, d'une lueur argentée qui suintait du bois.

"Lance le Patronus~!" le pressa M. Lupin. "Réessaye de le lancer, Harry~!"

\emph{Ah oui, pour ce qu'il en sait, je ne peux pas…}

Harry sourit, rit même un peu. "Il ne vaut mieux pas," dit Harry. "Si j'essayais de lancer le sortilège dans mon état d'esprit actuel, ça me tuerait probablement."

"\emph{Quoi~?}" dit M. Lupin. "Le Patronus ne fait pas ça~!"

Harry James Potter-Evans-Verres leva sa main gauche, toujours en riant, et essuya quelques-unes des larmes.

"Vous savez, M. Lupin," dit Harry, "il faut vraiment être capable d'interpréter les choses de façon \emph{baroque} pour pouvoir penser que quelqu'un, songeant que la mort est une chose que nous devrions tous accepter, communiquerait son état d'esprit en disant~: 'Le dernier ennemi qui sera détruit est la mort'. Peut-être que quelqu'un a trouvé que c'était poétique, qu'il a récupéré la phrase et a tenté de l'interpréter différemment, mais celui ou celle qui l'a dit en premier n'appréciait pas beaucoup la mort." Harry trouvait parfois mystérieux de voir comment les gens étaient capables de ne même pas \emph{remarquer} qu'ils tordaient une phrase jusqu'à lui faire effectuer un revirement complet, à l'opposé de sa signification première évidente. Ça ne pouvait pas être une question d'intelligence brute, les gens pouvaient comprendre le sens évident de la plupart des phrases en anglais. "D'autre part, 'sera détruit' fait référence à une situation future, cela ne peut donc pas parler de l'état actuel des choses."

Remus Lupin le regardait avec des yeux écarquillés. "Tu es certainement l'enfant de James et Lily," dit l'homme d'un air plutôt stupéfait.

"Oui, effectivement," dit Harry. Mais ça ne suffisait pas, il devait faire plus, alors il leva sa baguette et dit d'une voix aussi assurée que possible~: "Je suis Harry James Potter-Evans-Verres, le fils de Lily et James Potter de la maison Potter, et j'accepte la quête de ma famille. La Mort est mon ennemi et je le vaincrai."

\emph{Thrayen beyn Peverlas soona ahnd thrih heera toal thissoom Dath bey yewoonen.}

"Quoi~?" dit Harry. Les mots étaient apparus dans sa conscience comme s'ils avaient été l'une de ses pensées, sans autre explication.

"Qu'est-ce que c'était~?" dit Remus Lupin au même instant.

Harry se retourna, parcourut le cimetière du regard, mais il ne vit rien. À côté de lui, M. Lupin faisait de même.

Aucun d'eux ne remarqua la grande pierre, apparemment vieille de mille ans, sur laquelle une ligne dans un cercle dans un triangle luisait, très légèrement, d'une couleur argent, semblable à la lumière qui avait émané de la baguette de Harry, invisible à cette distance, sous le soleil encore radieux.

\later

\emph{Quelques temps plus tard~:}

"Merci encore, M. Lupin," dit Harry au grand homme légèrement balafré qui se préparait à repartir. "Même si j'aurais vraiment aimé que vous ne…"

"Le professeur Dumbledore a dit que je devais te ramener par Portoloin à Poudlard si quoi que ce soit d'inhabituel arrivait, que cela ressemble à une attaque ou non," dit M. Lupin avec conviction. "Ce qui est fort raisonnable."

Harry hocha la tête. Puis, ayant choisi avec soin de prononcer cette question en dernier~: "Avez-vous la moindre idée de ce que ces mots signifiaient~?"

"Si je le savais, je ne te le dirais pas," dit M. Lupin d'un air plutôt sévère. "Certainement pas sans la permission du professeur Dumbledore. Je peux comprendre ton impatience, mais tu ne devrais essayer de découvrir aucun secret ancestral des Potter avant d'être un adulte. C'est-à-dire après avoir passé des ASPICs, Harry, ou au moins tes BUSEs. Et je pense toujours que tu as très mal compris le sens premier de la devise de ta famille~!"

Harry hocha la tête, soupira intérieurement, et fit ses adieux à M. Lupin.

\later

Harry retraversa Poudlard en direction de la tour Serdaigle avec une sensation étrange, comme s'il était devenu plus fort. Jamais il ne se serait attendu à aucun de ces événements, mais ils avaient tous été pour le mieux.

Il traversait la salle commune Serdaigle, en chemin vers son dortoir.

C'est alors que la créature lumineuse vint vers lui, d'un doux blanc qui étincelait sous les torches de la salle commune Serdaigle, se glissant de nulle part, le serpent blanc.

\later
\begin{center}
\emph{Þregen béon Pefearles suna and þrie hira tól þissum Déað béo gewunen.}

Trois seront les fils Peverell et trois seront leurs instruments au moyen desquels la Mort sera vaincue.

- Dit en présence des trois frères Peverell\\
dans une petite taverne en périphérie de ce que l'on appellerait plus tard Godric's Hollow. \end{center}

%  LocalWords:  Elrin Hallie Thrayen beyn Peverlas soona ahnd thrih heera
%  LocalWords:  toal thissoom Dath bey yewoonen Þregen béon Pefearles suna
%  LocalWords:  þrie hira tól þissum Déað béo gewunen
