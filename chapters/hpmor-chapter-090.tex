\partchapter{Rôles}{I}

\lettrine{H}{arry} se tenait encore devant le corps de Hermione, il n'avait pas bougé et réfléchissait aussi vite que possible dans un sentiment de dissociation, de fragmentation temporelle. Devait-il faire quelque chose \emph{maintenant}, une opportunité d'agir était-elle en train de lui échapper irrévocablement~? De réduire le niveau d'omnipotence magique dont il aurait besoin plus tard~? Un effet de balise temporelle pour marquer cet instant, dans l'idée d'un voyage temporel futur, s'il trouvait un jour un moyen de retourner plus de six heures dans le passé~? Il existait des théories sur le voyage temporel dans le cadre de la relativité générale (qui lui avaient semblé beaucoup moins plausibles avant qu'il ne rencontre les Retourneurs de Temps) et ces théories disaient que l'on ne pouvait revenir à avant la construction de la machine~: une machine à remonter dans le temps fondée sur des principes relativistes maintenait un passage continu à travers le temps mais ne téléportait rien. Mais Harry ne voyait rien d'utile qu'il puisse accomplir avec les sortilèges dont il disposait, Dumbledore n'était pas très coopératif, et quoi qu'il en soit on était plusieurs minutes dans le Temps après l'instant crucial.

«Harry», chuchota le directeur en posant sa main sur l'épaule de Harry. Il avait disparu de là où il s'était tenu, au-dessus des jumeaux Weasley et était apparu à côté de Harry~; George Weasley s'était comme téléporté de sa position assise à une position agenouillée au côté de son frère, et Fred était à présent allongé bien droit, ses yeux grands ouverts, une grimace sur le visage à chaque respiration. «Harry, tu dois quitter cet endroit.

--- Attendez, dit la voix de Harry. J'essaie de voir si je peux faire autre chose.»

La voix du vieux sorcier attestait de son impuissance.

«Harry - je sais que tu ne crois pas aux âmes - mais que Hermione te regarde maintenant ou pas, je ne pense pas qu'elle souhaiterait te voir ainsi.»

… non, c'était évident.

Harry dirigea sa baguette sur le corps de Hermione…

«Harry~! Qu'est-ce que tu…»

… et déversa \emph{tout} ce qu'il avait le long de son bras, le long de sa main…

«\emph{Frigideiro~!}

--- … fais~?

--- Hypothermie», dit Harry d'une voix distante tout en chancelant. Cela avait été l'un des sortilèges avec lesquels, lors d'une vie antérieure, lui et Hermione avaient fait des expériences. Il était donc capable de le contrôler avec précision mais il lui avait fallu beaucoup de puissance pour affecter une telle masse. La corps de Hermione devait à présent être à exactement cinq degrés Celsius. «Des gens ont été réanimés après avoir été dans l'eau froide sans respirer pendant plus de trente minutes. Vous voyez, le froid vous protège des dommages cérébraux, il ralentit tout. Les docteurs moldus ont un dicton~: on n'est pas mort tant qu'on n'est pas chaud et mort - je pense qu'ils refroidissent même le patient lors de certaines opérations, quand ils ont besoin d'arrêter le cœur de quelqu'un pendant un moment.»

Fred et George commencèrent à sangloter.

Le visage de Dumbledore était déjà strié de larmes. «Je suis désolé, chuchota-t-il. Harry, je suis tellement navré, mais tu dois arrêter de faire ça.» Le directeur prit Harry par les épaules et le tira en arrière.

Harry se laissa être détourné du corps de Hermione, marcha en suivant les poussées du directeur qui l'éloignaient du sang. Le sortilège de refroidissement lui donnerait du temps. Des heures au moins, peut-être des jours s'il parvenait à continuer de lancer le sortilège sur Hermione ou s'ils entreposaient son corps dans un endroit froid.

Maintenant, il avait le temps de réfléchir.

\later

Minerva avait vu le visage d'Albus et avait su que quelque chose n'allait pas~; elle avait eu le temps de se demander ce qui s'était passé, et même qui était mort~: son esprit lui avait fait voir Alastor, Augusta, Arthur et Molly, les cibles les plus probables lors d'un retour de Voldemort. Elle avait pensé avoir trempé sa volonté dans l'acier, elle s'était crue prête au pire.

Puis Albus parla et tout l'acier la quitta.

\emph{Pas Hermione… pas…}

Albus lui donna un bref moment pour pleurer puis lui dit que Harry Potter, qui avait vu Mlle Granger mourir, s'était assis devant le débarras de l'infirmerie où les restes de Mlle Granger étaient conservés, qu'il refusait de bouger, et qu'il disait à tous ceux qui lui adressaient la parole de partir afin qu'il puisse réfléchir.

La seule chose qui avait provoqué quelque réaction que ce soit chez le garçon avait été quand Fumseck avait essayé de chanter pour lui~; Harry Potter avait glapit au phénix de ne pas faire ça, que ses émotions étaient réelles et qu'il ne voulait pas que la magie essaie de les \emph{soigner} comme si elles étaient une maladie. Après cela, Fumseck avait refusé de chanter.

Albus pensait qu'elle était peut⁻être celle qui avait le plus de chance de pouvoir atteindre Harry Potter.

Alors elle dut se reprendre, nettoyer son visage~; elle aurait le temps plus tard de vivre son deuil, en privé, quand ceux de ses enfants qui avaient survécu n'auraient plus besoin d'elle.

Minerva McGonagall récupéra les parties disloquées de son être, essuya ses yeux une dernière fois et mit la main sur la poignée de la section de l'infirmerie dont on utilisait maintenant le débarras - pour la seconde fois du siècle et pour la cinquième fois depuis la fondation de Poudlard - comme lieu de repos pour un jeune élève prometteur.

Elle ouvrit la porte.

Les yeux de Harry Potter la regardèrent. Le garçon était assis au sol devant la porte du débarras et avait sa baguette en main. Si son regard étaient endeuillé, s'il était vide, si même il était brisé, on ne pouvait le voir en regardant le visage du garçon. Il n'y avait pas de larmes séchées sur ces joues.

«Pourquoi êtes-vous là, professeur McGonagall~? dit Harry Potter. J'ai dit au directeur que je voulais être laissé seul un moment.»

Elle ne sut pas quoi répondre. \emph{Pour vous aider - vous n'allez pas bien} - mais elle ne savait pas quoi dire~: il n'y avait rien dont elle puisse imaginer que le dire arrangerait les choses. N'ayant pas été au mieux de sa forme, elle n'avait rien prévu à l'avance avant d'entrer dans la pièce.

«À quoi réfléchissez-vous~?» dit-elle. Ce fut la seule phrase qui lui vint à l'esprit. Albus lui avait dit que Hary avait dit, encore et encore, qu'il réfléchissait~; et il fallait qu'elle parvienne à faire parler Harry d'une façon ou d'une autre.

Harry regarda, à moitié vers elle, à moitié à travers elle, et elle retint sa respiration lorsqu'une tension apparut sur le visage de ce dernier.

Harry mit un moment à parler.

«J'essaie de trouver s'il y a quoi que ce soit que je devrais faire maintenant, dit-il. Mais c'est difficile. Mon esprit n'arrête pas d'imaginer comment le passé aurait pu se dérouler différemment, si j'avais réfléchi plus vite, et je ne peux pas éliminer la possibilité qu'une réponse se cache là.

--- M. Potter… dit-elle d'une voix hésitante. Harry, je ne pense pas que ce soit sain que vous… pensiez comme ça…

--- Je ne suis pas d'accord. C'est quand on ne pense pas que les gens meurent.» Les mots avaient été prononcés d'un ton monotone, comme s'il avait récité un texte appris dans un livre.

«Harry», dit-elle, réfléchissant à peine en prononçant ces paroles, «il n'y a rien que vous auriez pu faire…»

Quelque chose grésilla dans l'expression de Harry. Ses yeux semblèrent se poser sur elle pour la première fois.

«Rien que j'aurais pu \emph{faire}~?» la voix de Harry s'éleva sur ce dernier mot. «\emph{Rien que j'aurais pu FAIRE~? J'ai perdu compte du nombre de façons dont j'aurais pu la sauver~! Si j'avais demandé qu'on ait tous des miroirs communicants~! Si j'avais insisté pour que Hermione soit évacuée de Poudlard et mise dans une école qui n'est pas dérangée~! Si j'étais sorti immédiatement de la grande salle au lieu d'essayer de discuter avec des gens normaux~! Si je m'étais souvenu du Patronus plus tôt~! Même à la dernière minute ça aurait pu ne pas être trop tard~! J'ai tué le troll, je me suis tourné vers elle et elle était encore EN VIE et je me suis juste agenouillé pour écouter ses derniers mots comme un IDIOT au lieu de relancer le Patronus et de dire à Dumbledore d'envoyer Fumseck~!} Ou si j'avais juste vu tout le problème sous un autre angle - si j'avais cherché un élève avec un Retourneur de Temps pour envoyer un message dans le passé \emph{avant} que je découvre si quelque chose lui était arrivé au lieu de me retrouver avec un résultat inaltérable - j'ai \emph{demandé} au directeur de revenir dans le passé, de sauver Hermione et de tout simuler, de mettre un faux corps, de modifier les souvenirs de tout le monde, mais Dumbledore a dit qu'il avait déjà essayé ça une fois, que ça n'avait pas fonctionné et qu'au lieu de ça il avait perdu un autre ami. Ou si j'avais… si j'avais suivi… si, cette nuit là…»

Harry appuya ses mains sur son visage et lorsqu'il les retira, il était à nouveau calme et mesuré.

«Quoi qu'il en soit, dit Harry Potter d'une voix à nouveau monotone, je ne veux pas réitérer cette erreur, donc je vais réfléchir jusqu'à l'heure du dîner pour voir s'il y a quelque chose que je devrais faire. Si je n'ai rien trouvé à l'heure du dîner, alors j'irai manger. Maintenant partez, s'il vous plaît.»

Elle se rendait compte que des larmes coulaient maintenant le long de ses joues à elle.

«Harry… Harry, vous devez croire que ce n'est pas de votre faute~!

--- Bien sûr que c'est ma faute. Il n'y a personne d'autre ici qui pourrait être responsable de quoi que ce soit.

--- Non~! Vous-Savez-Qui a tué Hermione~!» Elle se rendait à peine compte de ce qu'elle disait, elle se rendait à peine compte qu'elle n'avait pas protégé la pièce contre d'éventuelles oreilles indiscrètes. «Pas vous~! Quoi que vous auriez pu faire, ce n'est pas vous qui l'avez tuée, c'est Voldemort~! Vous ne pouvez pas croire ça, sinon vous deviendrez fou, Harry~!

--- Ce n'est pas comme ça que la responsabilité fonctionne, professeur.» La voix de Harry était patiente, comme s'il expliquait quelque chose à un enfant qui allait certainement ne pas comprendre. Il ne la regardait plus, il regardait juste le mur sur sa droite. «Lorsque l'on procède à analyse des défaillances, il est inutile de porter le blâme sur une partie du système que l'on ne pourra pas changer~: c'est comme de se laisser tomber d'une colline et de blâmer la gravité. La gravité n'aura pas changé la prochaine fois. Il est inutile d'essayer de tenir pour responsables des gens qui ne changeront pas leurs actes. Une fois qu'on regarde les choses sous cet angle, on se rend compte que blâmer ne sert jamais à rien sauf quand on se blâme soi-même, parce que l'on est le seul dont les actions peuvent changer suite à un blâme. C'est pour ça que Dumbledore a cette pièce pleine de baguettes brisées. Au moins, il comprend ça.»

Une lointaine partie de l'esprit de Minerva nota d'attendre beaucoup plus tard pour aller parler très durement au directeur au sujet de ce qu'il montrait à de jeunes enfants impressionnables. Elle lui crierait peut-être même dessus, cette fois. Elle avait de toute façon songé à lui crier dessus, à cause de Hermione Granger…

«Vous n'êtes \emph{pas} responsable, dit-elle même si sa voix tremblait. Ce sont les professeurs - ce sont nous qui sommes responsable de la sécurité des élèves, pas vous.»

Les yeux de Harry revinrent à elle. «\emph{Vous} êtes responsable~?» Sa gorge semblait serrée. «Vous voulez que je vous tienne pour responsable, professeur McGonagall~?»

Elle leva le menton et hocha la tête. Ce serait bien, bien mieux que de voir Harry se blâmer lui-même.

Le garçon se releva en appuyant sur le sol et fit un pas en avant. «Alors très bien, dit-il d'une voix monotone. J'ai essayé d'être raisonnable quand j'ai vu que Hermione manquait à l'appel et qu'aucun des professeurs n'étaient au courant. J'ai demandé à des élèves de septième année de venir avec moi sur un balai volant et de me protéger pendant que l'on cherchait Hermione. J'ai demandé qu'on m'aide. J'ai supplié qu'on m'aide. Et personne ne m'a aidé. Parce que vous aviez donné à tout le monde l'ordre absolu de rester quelque part sous peine d'être exclu, sans excuses possibles. Peu importe les erreurs de Dumbledore, au moins il voit ses élèves comme des gens, pas comme des animaux qui doivent être menés dans un enclos et tenus hors de couloirs~; vous saviez que certains élèves étaient tactiquement et stratégiquement meilleurs que vous et vous nous avez quand même coincés dans une pièce sans rien laisser à notre discrétion. Alors, quand quelque chose que vous n'aviez pas prévu s'est produit et qu'il aurait été parfaitement sensé d'envoyer un élève de septième année sur un balai rapide à la recherche de Hermione Granger, les élèves ont su que vous ne comprendriez pas, que vous ne pardonneriez pas. Ils n'avaient pas peur du troll, ils avaient peur de vous. La discipline, la conformité, la \emph{lâcheté} que vous avez insufflée en eux m'a retardé juste assez pour que Hermione meure. Non pas que j'aurais dû demander de l'aide à des gens normaux, bien sûr, et je vais changer, et je serai moins stupide la prochaine fois. Mais si j'étais assez stupide pour tenir pour responsable quelqu'un d'autre que moi, c'est ce que je dirais.»

Des larmes coulaient sur les joues de Minerva.

«Voilà ce que je vous dirais si je pensais que vous pouviez être responsable de quoi que ce soit. Mais les gens normaux ne choisissent pas en fonction des conséquences, ils ne font que jouer des rôles. Il y a dans votre tête l'image d'une sévère adepte de la discipline et vous faites tout ce que cette image ferait, que ça ait un sens ou pas. Une adepte de la discipline ordonnerait à ses élèves de retourner dans leurs chambres, même avec un troll rôdant dans les couloirs. Une adepte de la discipline ordonnerait à ses élèves de ne pas quitter la grande salle sous peine d'expulsion. Et la petite image du professeur McGonagall que vous avez dans votre tête ne peut pas apprendre de ses expériences ou se modifier elle-même, donc cette conversation est inutile. Les gens comme vous ne sont pas responsables, ce sont les gens comme moi qui le sont, et quand nous échouons il n'y a personne d'autre à blâmer.»

Le garçon avança pour se tenir directement face à elle. Sa main plongea dans ses robes et en ressortit la sphère dorée qu'était la coque protectrice fournie par le ministère de son Retourneur de Temps. Il parla d'une voix morne, neutre, sans aucune emphase. «Ceci aurait pu sauver Hermione si j'avais été capable de l'utiliser. Mais vous avez pensé que c'était votre rôle de me le refuser, de me barrer la route. Personne n'est mort à Poudlard depuis cinquante ans, vous avez dit cela quand vous l'avez verrouillé, vous vous souvenez~? J'aurais dû vous le demander à nouveau quand Bellatrix Black s'est échappée d'Azkaban, ou après que Hermione eut été accusée de tentative de meurtre. Mais j'ai oublié, j'ai été stupide. S'il vous plaît, débloquez-le maintenant avant qu'un autre de mes amis ne meure.»

Incapable de parler, elle sortit sa baguette et s'exécuta, défaisant l'enchantement minuté qu'elle avait mêlé au loquet de la coque.

Harry Potter ouvrit la coque dorée, regarda le petite sablier coincé entre ses cercles concentriques, hocha la tête et referma la coque. «Merci. Maintenant, partez.» La voix du garçon se brisa à nouveau. «Je dois réfléchir.»

\later

Elle ferma la porte derrière elle, et un son terrible et pourtant presque entièrement étouffé s'échappa de sa gorge…

Albus apparut dans une ondulation à côté d'elle, prenant brièvement un teint criard le temps que le sortilège de Désillusion s'estompe.

Elle ne bondit pas tout à fait de surprise. «Je vous ai dit d'arrêter de faire ça», dit Minerva. Sa voix semblait morne à ses propres oreilles. «C'était une conversation privée.»

Albus agita ses doigts vers la porte derrière elle.

«J'avais peur que M. Potter ne te fasse du mal.» Le directeur s'interrompit puis dit doucement~: «Je suis très surpris que tu aies encaissé sans bouger.

--- Tout ce que j'avais à faire, c'était de dire~: 'M. Potter', et il se serait arrêté.» Sa voix était descendue au niveau d'un murmure. «Cela seulement, et il se serait arrêté. Et alors il n'aurait eu personne à qui dire ces terribles choses, personne.

--- J'ai trouvé que les remarques de M. Potter étaient entièrement injuste et imméritées, dit Albus.

--- Si cela avait été vous, Albus, vous n'auriez pas menacé d'expulser quiconque quitterait la pièce. Pouvez-vous honnêtement me dire le contraire~?»

Albus leva un sourcil. «Ton rôle dans ce désastre a été mince, tes décisions étaient raisonnables au moment où tu les a prises, et ce n'est que du point de vue privilégié de celui qui regarde le passé que Harry Potter peut croire qu'il en est autrement. Tu es certainement trop sage pour te blâmer pour ceci, Minerva.»

Elle savait parfaitement que Albus placerait une image de Hermione dans sa terrible pièce, qu'elle y occuperait une place de choix. Elle était certaine que, même s'il n'avait même pas été présent à Poudlard à ce moment là, Albus se tiendrait \emph{lui} pour responsable. Mais pas elle.

\emph{Alors vous non plus, vous ne pensez pas que je mérite qu'on me tienne pour responsable…}

Elle s'appuya contre le mur le plus proche et essaya de ne pas laisser les larmes émerger de nouveau~; elle n'avait jamais vu Albus pleurer, à trois exceptions près.

«Vous avez toujours cru en vos élèves, ce que je n'ai jamais fait. Ils n'auraient pas eu peur de vous. Ils auraient su que vous comprendriez.

--- Minerva…

--- Je ne suis pas apte à vous succéder à la direction. Nous le savons tous les deux.

--- Tu as tort, dit doucement Albus. Quand le jour viendra, tu seras la quarante-cinquième directrice de Poudlard et tu feras un excellent travail.»

Elle secoua la tête. «Et maintenant, Albus~? S'il ne m'écoutera pas moi, qui d'autre~?»

\later

Nous étions peut-être une heure plus tard. Le garçon gardait encore la porte où le corps de sa meilleure amie reposait, il la veillait, assis. Il regardait vers le bas, vers sa baguette, entre ses mains. Parfois son visage se contractait au milieu d'une pensée, à d'autres moments il se détendait.

Même si la porte ne s'ouvrit pas et qu'il n'y eut aucun son, le garçon leva les yeux. Il composa son expression. Sa voix, lorsqu'il parla, fut morne. «Je ne veux pas de compagnie.»

La porte s'ouvrit.

Le professeur de Défense de Poudlard entra dans la pièce, ferma la porte derrière lui, et prit place avec précaution dans un coin entre deux murs, aussi loin du garçon qu'il lui fut possible. Une sensation aiguë de catastrophe avait emplit l'espace qui les séparait et se tenait là, immuable.

«Pourquoi êtes-vous là~?» dit le garçon.

L'homme inclina légèrement la tête. Des yeux pâles examinèrent le garçon comme s'il était un spécimen d'une vie venue d'une lointaine planète, avec le danger potentiel que cela impliquait.

«Je suis venu vous présenter mes excuses, M. Potter, dit doucement l'homme.

--- Vous excuser pour quoi~? dit le garçon. Pourquoi, qu'auriez-\emph{vous} pu faire pour empêcher la mort de Hermione~?

--- J'aurais dû penser à vérifier votre présence, celle de M. Londubat et celle de Mlle Granger, car vous étiez tous des cibles évidentes, dit le professeur de Défense sans hésiter. M. Hagrid n'était pas mentalement apte à commander un contingent d'élèves. J'aurais dû ignorer la demande de silence que m'avait faite la directrice adjointe et lui dire de laisser le professeur Flitwick en arrière, car il aurait mieux été capable de défendre les élèves contre n'importe quelle menace tout en maintenant la communication via Patronus.

--- Correct.» La voix du garçon avait le tranchant d'un fil de rasoir. «J'avais oublié qu'il y avait à Poudlard quelqu'un d'autre capable d'être responsable. Alors pourquoi n'y avez-vous pas pensé, Professeur~? Parce que je ne crois pas que \emph{vous} ayez été stupide.»

Il y eut un silence et les doigts du garçon blanchirent autour de sa baguette.

«Vous n'y avez pas pensé non plus sur le moment, M. Potter.» La voix du professeur de Défense révélait une certaine fatigue. «Je suis plus intelligent que vous. Je pense plus vite que vous. J'ai plus d'expérience que vous. Mais l'écart entre nous n'est pas le même que celui entre eux et nous. Si vous pouvez rater quelque chose, alors moi aussi.» Les lèvres de l'homme se tordirent. «Vous voyez, j'ai immédiatement compris que le troll n'était qu'une distraction d'autre chose et n'avait pas en lui-même grande importance. Tant que personne n'envoyait les élèves se promener inutilement à travers les couloirs ou ne dépêchait inconsciemment les jeunes Serpentard dans ces mêmes donjons où le troll avait été vu.»

Le garçon ne sembla pas se détendre.

«J'imagine que c'est plausible.

--- Quoi qu'il en soit, dit l'homme, s'il y a ici quiconque qui peut être déclaré responsable de la mort de Mlle Granger, c'est moi-même, pas vous. C'est moi, et non pas vous, qui aurais dû…

--- Je perçois que vous avez parlé au professeur McGonagall et qu'elle vous a donné un texte à réciter.» Le garçon ne se fatigua pas à dissimuler son amertume. «Si vous avez quelque chose à dire, professeur, dites-le sans masque.»

Il y eut un silence.

«Comme vous le souhaitez», dit le professeur de Défense d'une voix vide d'émotions. Les yeux pâles demeuraient vifs, aux aguets. «Je regrette effectivement que la fille soit morte. Elle était une bonne élève en cours de Défense et aurait pu être votre alliée plus tard. J'aimerais vous consoler de votre perte mais j'ignore entièrement comment procéder. Naturellement, si je découvre les responsables, je les tuerai. Vous serez invité à participer si les circonstances le permettent.

--- Comme c'est touchant, dit le garçon d'une voix froide. Vous ne prétendez donc pas avoir aimé Hermione~?

--- Je crois avoir été insensible à ce qui faisait son charme. Je ne tisse plus de tels liens avec facilité.»

Le garçon hocha la tête. «Merci pour votre honnêteté. Est-ce tout, professeur~?»

Il y eut un silence.

«Le château est meurtri, à présent, dit l'homme debout dans un coin.

--- Quoi~?

--- Lorsqu'un certain artefact ancien m'a informé que Mlle Granger était sur le point de mourir, j'ai lancé ce sortilège de feu maudit dont je vous ai un jour parlé. J'ai traversé certains murs et certains étages en les brûlant afin que mon balai volant puisse emprunter une voie plus directe.» L'homme parlait d'une voix sans timbre. «Poudlard ne guérira pas facilement de ces blessures, si jamais elle en guérit. J'imagine qu'il sera nécessaire de colmater les trous au moyens de conjurations plus faibles. Je le regrette maintenant, puisqu'il s'est avéré que j'arrivais trop tard.

--- Ah,» dit le garçon. Il ferma brièvement les yeux. «Vous vouliez la sauver. Vous le vouliez tellement que vous avez fait un vrai effort. J'imagine que votre esprit, contrairement au leur, est capable de cela.»

Un bref sourire sec venu de l'homme.

«Merci, professeur. Mais je voudrais être laissé seul jusqu'au dîner. Vous entre tous saurez comprendre. Est-ce tout~?

--- Pas tout à fait», dit l'homme. Une nuance de sécheresse sardonique était à présent revenue dans sa voix. «Voyez-vous, étant donné mes expériences récentes, je suis inquiet à l'idée que ayez maintenant pour projet de faire quelque chose d'extrêmement stupide.

--- Comme quoi~? dit le garçon.

--- Je n'en suis pas tout à fait certain. Peut-être avez-vous décidé qu'un univers sans Mlle Granger n'a aucune valeur et devrait être détruit à cause des insultes qu'il vous a faites.»

Le garçon sourit sans gaîté.

«Vos propres problèmes se révèlent, professeur. Ce n'est pas vraiment mon genre. Cela a-t-il été le vôtre, un jour~?

--- Pas particulièrement. Je n'ai pas un grand amour pour l'univers, mais c'est là que je vis.»

Il y eut un silence.

«Que préparez-vous, M. Potter~? dit l'homme dans un coin. Vous avez formé quelque importante résolution, bien que vous essayiez de me le cacher. Que comptez-vous faire à présent~?»

Le garçon secoua la tête.

«Je réfléchis encore, et je voudrais être laissé seul pour le faire.

--- Je me souviens d'une offre que vous m'avez faite, il y a quelques mois, dit le professeur de Défense. Souhaitez-vous avoir quelqu'un d'intelligent à qui parler~? Je comprendrais que vous ne soyez pas agréable à côtoyer.»

Le garçon secoua de nouveau la tête.

«Non, merci.

--- Eh bien dans ce cas, dit le professeur de Défense. Que diriez-vous de quelqu'un de puissant et pas particulièrement entravé par de naïfs scrupules~?»

Il y eut une hésitation, puis le garçon secoua encore la tête.

«Quelqu'un qui a connaissance de nombreux secrets et magies que certains pourraient juger être contre nature~?»

Les yeux du garçon se plissèrent légèrement, d'une façon si imperceptible que quelqu'un d'autre aurait pu ne pas…

«Je vois, dit le professeur de Défense. Allez-y, interrogez-moi, alors. Je vous donne ma parole de ne rien en répéter aux autres.»

Le garçon mit un moment à parler, et lorsqu'il parla ce fut d'une voix brisée.

«Je compte ramener Hermione. Parce qu'il n'y a pas d'au-delà, et que je ne vais pas juste la laisser… juste \emph{ne pas être}…»

Le garçon appuya ses mains contre son visage, et lorsqu'il les enleva, il semblait à nouveau aussi vide d'émotions que l'homme debout dans le coin.

Les yeux du professeur de Défense étaient pensifs et légèrement perplexes.

«Comment~? dit enfin l'homme.

--- Comme ce sera nécessaire.»

Il y eut un autre silence.

«Peu importe les risques, dit l'homme dans un coin. Peu importe à quel point la magie nécessaire à accomplir ceci sera dangereuse.

--- Oui.»

Les yeux du professeur étaient pensifs.

«Mais quelle approche générale avez-vous à l'esprit~? Je suppose que transformer son corps en Inferi n'est pas ce que vous…

--- Serait-il capable de penser~? dit le garçon. Son corps se décomposerait-il toujours~?

--- Non, et oui.

--- Alors non.

--- Et la Pierre de Résurrection de Cadmus Peverell, si quelqu'un pouvait l'obtenir pour vous~?»

Le garçon secoua la tête. «Je ne veux pas d'une illusion de Hermione tirée de mes souvenirs. Je veux qu'elle soit capable de \emph{vivre} sa \emph{vie}…» la voix du garçon se brisa. «Je n'ai pas encore choisi un angle d'attaque concret. Si je dois attaquer le problème par force brute, acquérir assez de puissance et de savoir pour juste \emph{faire que ça ait lieu}, c'est ce que je ferai.»

Un autre silence.

«Et pour faire \emph{ça}, dit l'homme dans le coin, vous utiliserez votre outil favori, la science.

--- Bien sûr.»

Le professeur de Défense expira, presque un soupir.

«Voilà qui donne un sens à la chose, j'imagine.

--- Êtes-vous prêt à m'aider ou pas~? dit le garçon.

--- Quelle aide recherchez-vous~?

--- La magie. D'où vient-elle~?

--- Je l'ignore, dit l'homme.

--- Et tout le monde est dans le même cas~?

--- Oh, la situation est bien pire, M. Potter. Il existe à peine un érudit des savoirs ésotériques qui n'a pas déchiffré la nature de la magie, et chacun d'entre eux a une explication différente.

--- D'où viennent les nouveaux sortilèges~? Je n'arrête pas de lire que quelqu'un a inventé un sortilège pour faire une chose ou une autre mais il n'y a aucune mention du \emph{comment}.»

Un haussement d'épaules sous des robes.

«D'où viennent les nouveaux livres, M. Potter~? Ceux qui en lisent beaucoup deviennent parfois capable d'en écrire à leur tour. Comment~? Personne ne le sait.

--- Il y a des livres sur comment écrire…

--- Les lire ne fera pas de vous un dramaturge célèbre. Une fois tous ces conseils lus et appris, ce qui demeure est un mystère. L'invention de nouveaux sortilèges est un mystère similaire, d'une forme plus pure.» La tête de l'homme se pencha. «De telles entreprises sont dangereuses. On dit qu'il faut choisir entre ne pas avoir d'enfant ou attendre qu'il ait déjà grandi. Si tant d'innovateurs semblent répondre de Gryffondor, plutôt que de Serdaigle comme on aurait pu s'y attendre, c'est qu'il y a une bonne raison.

--- Et les magies d'un genre plus puissant~? dit le garçon.

--- Un sorcier légendaire inventera peut-être un rituel sacrificiel dans sa vie et transmettra le savoir à ses héritiers. Tenter d'en inventer cinq serait du suicide. C'est pour cela que les sorciers réellement puissants sont ceux qui ont appris des savoirs ancestraux.»

Le garçon hocha la tête d'un air distant.

«Tant pis pour la solution directe, alors. Il aurait été agréable de juste inventer le sortilège 'Relever les morts', 'Devenir Dieu', ou 'Invoquer un Terminal'. Savez-vous quoi que ce soit au sujet d'Atlantis~?

--- Seulement ce que tout érudit en sait, dit sèchement l'homme. Si vous souhaitez entendre les dix-huit meilleures théories standard - ne me regardez pas comme ça, M. Potter. Si c'était si facile, je l'aurais fait il y a de nombreuses années.

--- Je comprends. Pardon.»

Il y eut un autre temps de silence. Le regard du professeur de Défense reposait sur le garçon qui semblait regarder dans le vide.

«Il y a des magies que je compte apprendre. Des sortilèges que j'aurais pu utiliser plus tôt aujourd'hui, si j'avais songé à les apprendre à l'avance.» La voix du garçon était froide. «Des sortilèges dont j'aurai besoin, si ce genre de chose continue de se produire. Je m'attends à pouvoir légitimement apprendre la plupart d'entre eux. Mais pas tous.»

La professeur de Défense inclina la tête. «Je vous enseignerai presque n'importe quelle magie que vous désirerez apprendre, M. Potter. J'ai certaines limites, mais vous pourrez toujours demander. Que recherchez-vous exactement~? Vous n'avez pas la puissance brute nécessaire au sortilège de la Mort et à la plupart des autres sortilèges considérés comme interdits…

--- Ce sortilège de feu maudit. J'imagine que ce n'est pas un rituel sacrificiel que même un enfant pourrait utiliser s'il l'osait~?»

Les lèvres du professeur de Défense tressaillirent. «Il requiert le sacrifice permanent d'une goutte de sang~; votre corps serait allégé du poids de cette goutte à partir de ce jour. Pas le genre de chose que l'on souhaiterait faire souvent, M. Potter. Une force de volonté est requise pour empêcher le feu maudit de se retourner contre vous et de vous consumer~; l'entraînement habituel consiste à d'abord se rôder sur des épreuves plus simples. Et bien qu'il ne s'agisse pas d'un des ingrédients principaux du rituel, j'ai peur qu'il ne nécessite plus de magie que vous n'en posséderez pendant encore quelques années.

--- Dommage, dit le garçon. Il aurait été agréable de voir la tête de l'ennemi la prochaine fois qu'il aurait essayé d'utiliser un troll.»

Le professeur de Défense inclina la tête, ses lèvres tressaillirent de nouveau.

«Et les sortilèges liés à la mémoire~? Les jumeaux Weasley se comportaient étrangement et le directeur a dit qu'il pense qu'ils ont été victime d'un sortilège d'Oubliettes. Cela semble être l'une des techniques préférées de l'ennemi.

--- Règle numéro huit, dit le professeur de Défense. Toute technique assez bonne pour m'avoir une fois est assez bonne pour mériter que je l'apprenne.»

Le garçon eut un sourire sans gaîté. «Et j'ai entendu dire qu'une adulte avait lancé Oubliettes alors qu'elle était presque entièrement vidée de sa magie. Cela ne doit donc pas en nécessiter beaucoup. Il n'est même pas considéré comme Impardonnable, bien que je ne puis comprendre pourquoi. Si j'avais pu faire en sorte que M. Hagrid se souvienne d'autres ordres…

--- Ce n'est pas aussi simple, dit le professeur de Défense. Vous n'êtes pas assez puissant pour utiliser le sortilège de faux souvenirs, et même un simple Oubliettes serait à la limite de votre endurance actuelle. C'est un art dangereux, dont la pratique est illégale sans autorisation ministérielle, et je vous recommanderais de ne pas l'utiliser lorsque les circonstances rendraient gênantes l'effacement accidentel de dix ans de la vie de quelqu'un. J'aimerais pouvoir vous promettre que j'obtiendrai l'un de ces tomes hautement gardés du département des mystères et que je vous le ferai livrer sous quelque déguisement. Mais ce que je dois en fait vous dire, c'est que vous trouverez le manuel d'instruction standard sur le sujet dans les piles nord-nord-ouest de la bibliothèque principale de Poudlard, à la lettre M.

--- Sérieusement, dit le garçon d'un ton neutre.

--- En effet.

--- Merci pour votre aide, professeur.

--- Votre créativité est devenue bien plus pragmatique depuis que je vous ai rencontré, M. Potter.

--- Merci pour le compliment.» Le garçon ne releva pas les yeux de la baguette qu'il regardait, entre ses mains. «Je voudrais me remettre à réfléchir à présent. S'il vous plaît, expliquez-leur pour moi ce qui se passera si je suis dérangé.»

\later

La porte du débarras s'ouvrit et le professeur Quirrell en sortit. Son visage avait un air mort, vide d'émotions~; elle aurait dit que cela lui rappelait Severus, mais Severus n'avait jamais tout à fait ressemblé à cela.

Au moment où la porte se referma, Minerva lança muettement une barrière anti-bruit. Les mots s'écoulèrent d'elle rapidement~: «Comment ça s'est passé - vous étiez là un bon moment - est-ce que Harry parle, maintenant~?»

Le professeur Quirrell avança rapidement dans la pièce jusqu'au mur proche de l'entrée et la regarda. Le visage sans émotion glissa, comme s'il enlevait un masque, et révéla derrière lui quelque chose de très sinistre.

«J'ai parlé à M. Potter comme il s'attendait à ce que je lui parle et j'ai évité de dire des choses qui auraient pu l'agacer. Je ne pense pas l'avoir consolé. Je ne pense pas avoir de talent pour ça.

--- Merci… c'est déjà bien qu'il ait parlé…» Elle hésita. «Qu'a dit M. Potter~?

--- J'ai peur de lui avoir promis de ne pas en parler. Et maintenant… je pense que je dois faire une visite à la bibliothèque de Poudlard.

--- La \emph{bibliothèque}~?

--- Oui», dit le professeur Quirrell. Une tension inhabituelle apparut dans sa voix. «Je compte renforcer la sécurité de la section interdite avec certaines précautions de ma propre fabrication. Les protections actuelles sont une plaisanterie. Et M. Potter doit être maintenu hors de la section interdite \emph{à tout prix}.»

Elle regarda le professeur de Défense, et son cœur se mit à battre la chamade.

Le professeur Quirrell continua de parler~: «Vous ne direz \emph{pas} au garçon que je vous ai dit cela. Vous confirmerez auprès de Flitwick et Vector que le garçon devra être distrait par les évasions habituelles s'il pose des questions trop précoces sur la création de sortilèges. Et bien qu'il ne s'agisse pas d'un domaine où je suis expert, madame la directrice adjointe, si vous pouvez imaginer quelque moyen que ce soit pour convaincre le garçon de ne pas s'enfoncer plus avant dans son chagrin et sa folie - n'importe quel moyen de défaire les résolutions qu'il est en train de prendre - alors je vous suggère de l'appliquer \emph{immédiatement}.» 
