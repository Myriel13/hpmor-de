\namedpartchapter{Accomplissement de soi}{SA}{X}{Responsabilité}

\lettrine{C}{'était} un passage noueux et plein de méandres situé au cœur de Poudlard qui serpentait comme une mèche de cheveux si mal coiffés qu'il se croisait parfois lui-même mais dont il était impossible d'atteindre le bout si l'on se laissait aller à la tentation offerte par ces faux raccourcis.

Au bout de ce fouillis, six élèves appuyés sur des pierres brutes, leur robe noire en contraste avec le mur gris, bordée de vert, aux aguets, les yeux de chacun passant de l'un à l'autre, et des torches, qui brûlaient dans des chandeliers à nu et dont la lumière repoussait les ténèbres et le froid des donjons Serpentard.

<<~J'en suis \emph{certaine}, dit Reese Belka avec hargne, absolument \emph{certaine}, ce n'était pas un vrai rituel. Les petites sorcières de première année ne peuvent pas faire ce genre de magie, et même si elles le pouvaient, qui a jamais entendu parler d'un rituel noir qui \emph{sacrifierait} une horreur scellée contre -- \emph{ça}~?

--- Est-ce que tu étais -- dit Lucian Bole. Je veux dire -- après que cette fille a claqué des doigts…~>>

Le regard de Belka aurait dû le faire frire sur place.

<<~Non, cracha-t-elle, certainement \emph{pas}.

--- C'est-à-dire qu'elle n'était pas nue~>>, dit Marcus Flint d'une voix traînante, ses larges épaules appuyée contre le mur de pierre de façon à laisser penser qu'il était détendu. <<~Couverte de glaçage au chocolat, oui, mais pas nue.

--- Potter a aujourd'hui fait grande insulte à nos maisons, dit la voix lugubre de Jaime Astorga.

--- Oui, enfin, désolé d'être aussi direct~>>, dit Randolph Lee d'une voix égale. Le duettiste de septième année se frotta le menton où l'on avait laissé poussé un léger début de barbe. <<~Mais quand quelqu'un vous colle au plafond, c'est un message, Astorga. C'est un message qui dit~: Je suis un mage noir incroyablement puissant qui aurait pu vous faire ce qui m'aurait chanté, et je me fiche bien que vos maisons en soient offensées.~>>

Robert Jugson III eut un rire doux et grave, un gloussement qui envoya des frissons descendre le long de plusieurs échines.

<<~C'est à s'en demander si on n'a pas choisi le mauvais camp, vous ne trouvez pas~? J'ai entendu des histoires au sujet de \emph{messages} de ce genre, envoyés sur ordre de l'ancien Seigneur des Ténèbres…

--- Je ne suis pas tout à fait prêt à m'agenouiller devant Potter, dit Astorga en regardant Jugson droit dans les yeux.

--- Moi non plus~>>, dit Belka.

Jugson tenait sa baguette, et il la faisait pivoter avec nonchalance dans un sens puis dans l'autre entre ses doigt, la faisant ainsi pointer successivement vers le plafond puis vers le sol. <<~Es-tu Gryffondor ou Serpentard~? dit Jugson. Tout le monde a un prix. Tous les gens malins.~>>

Cette affirmation produisit un moment de silence.

<<~Malfoy ne devrait-il pas être ici~?~>> dit timidement Bole.

Flint fit un geste dédaigneux de la main.

<<~Quoi que Malfoy manigance, il veut avoir l'air innocent. Il ne peut pas se permettre de manquer à l'appel en même temps que nous.

--- Mais tout le monde \emph{sait} déjà ça, dit Bole. Même dans les autres maisons.

--- Oui, très maladroit~>>, dit Belka. Elle grogna. <<~Malfoy ou pas, c'est qu'un petit de première année et on a pas besoin de lui ici.

--- J'enverrai une chouette à mon père, dit Jugson d'une voix basse, et \emph{lui} parlera à Lord Malfoy en personne…~>> puis il s'arrêta de parler abruptement.

<<~Je ne sais pas pour \emph{vous}, mes chéris, dit Belka d'une voix faussement suave, mais \emph{je} ne compte pas être épouvantée par un faux rituel et \emph{je} n'en ai pas fini avec Potter et sa Sang-de-Bourbe de compagnie.~>>

Personne ne répondit. Tous regardaient derrière elle.

Belka se retourna lentement afin de voir ce que les autres regardaient fixement.

<<~Vous ne ferez \emph{rien}~>>, siffla leur directeur de maison. Le visage de Severus Rogue était empli de rage et de petits points de bave s'envolaient de ses lèvres à chacun de ses mots, tachant d'autant plus sa robe déjà salie. <<~Vous en avez \emph{assez} fait, idiots~! Vous avez suffisamment embarrassé ma maison -- \emph{perdu} contre des première année -- vous parlez maintenant de mêler de nobles Lords du Magenmagot dans vos chamailleries puériles et \emph{pathétiques}~? \emph{Je} m'occuperai de cette affaire. \emph{Vous} n'embarrasserez pas de nouveau cette maison et vous ne prendrez même pas le \emph{risque} de le faire~! C'en est \emph{fini} de vos combats contre ces sorcières, et s'il en est autrement…~>>

\later

Si vous aviez cru qu'ils seraient assis l'un à côté de l'autre au dîner, vous auriez eu sacrément tort.

<<~Qu'est-ce qu'elle \emph{veut}~?~>> dit le cri plaintif d'un garçon qui, en dépit de sa connaissance considérable de la science actuelle, était encore naïf sur certains sujets. <<~Est-ce qu'elle \emph{voulait} se faire battre~?~>>

Les Serdaigle plus âgés qui s'étaient assis à côté de lui échangèrent de rapides regards jusqu'à ce que, selon un protocole tacite, le plus expérimenté d'entre eux parle~:

<<~Écoute~>>, dit Arty Grey, qui était en septième année et devançait les autres à cette compétition de trois sorcières et d'un professeur de Défense, <<~le truc à comprendre, c'est que ce n'est pas parce qu'elle est en \emph{colère} que tu as perdu des points à ce jeu. Mlle Granger est en colère parce qu'elle a été tout épouvantée et qu'elle \emph{peut te blâmer}, tu comprends~? Mais en même temps, même si elle ne l'admettra pas, elle sera touchée que son petit copain en soit allé à des extrémités aussi ridicules et franchement folles pour la protéger.

--- Ce n'est pas une question de \emph{points},~>> grinça Harry. Les mots glissaient entre ses dents serrées. Son dîner attendait devant lui, sur la table, ignoré. <<~C'est une question de \emph{justice}. Et \emph{Je. Ne. Suis. Pas. Son. Petit. Copain~!}~>>

Tous ceux présents lui répondirent par une bonne dose de rire sarcastique.

<<~Oui, enfin, dit un Serdaigle en sixième année, je pense qu'après qu'elle t'a embrassé pour te faire sortir de ton détraquage et que tu as collé quarante-quatre brutes au plafond pour elle, on est bien au-delà de “non, vraiment, c'est pas ma copine”, et plutôt à se demander à quoi vos enfants ressembleront. Waouh, ça fait peur à imaginer…~>> Le Serdaigle laissa sa phrase en suspens puis dit d'une voix plus faible~: <<~S'il te plaît, ne me regarde pas comme ça.

--- Écoute, dit Arty Grey, désolé d'être direct, mais tu peux avoir la justice, tu peux avoir les filles, mais pas les deux à la fois.~>> Il claqua d'une main amicale sur l'épaule de Harry. <<~Tu as du potentiel, petit, plus qu'aucun sorcier que j'ai jamais vu, mais tu dois apprendre à \emph{l'utiliser}, tu vois~? Sois un peu plus doux avec elles, apprends quelques sortilèges pour arranger ce souk que tu appelles des cheveux. Et surtout, tu dois mieux cacher que tu es maléfique -- pas \emph{trop} bien, mais mieux. Les beaux garçons bien habillés sont des tombeurs, les mages noirs sont des tombeurs, mais les beaux garçons bien habillés soupçonnés d'être \emph{secrètement} des mages noirs tombent plus de filles que tu ne pourrais l'imaginer…

--- Pas intéressé, dit catégoriquement Harry tout en ôtant la main de l'autre garçon de son épaule et en la faisant tomber sans cérémonies.

--- Mais tu le seras, dit Arty Gray d'une voix basse et prophétesse. Ah, tu le seras~!~>>

Ailleurs à la même table…

<<~\emph{Romantique~?}~>> glapit Hermione Granger si fort que certaines des filles proches d'elle firent la grimace. <<~\emph{Qu'est-ce qui était romantique là-dedans~?} Il n'a rien \emph{demandé}~? Il ne \emph{demande} jamais~! Il envoie juste des fantôme pourchasser des gens et il les colle au plafond et il fait ce qu'il veut de \emph{ma} vie~!

--- Mais tu ne vois pas~? dit une sorcière en quatrième année. Ça veut dire que même s'il est maléfique, il \emph{t'aime}~!

--- Tu ne fais pas avancer les choses~>>, dit Pénélope Deauclaire un peu plus loin à la table, mais on l'ignora. Quelques sorcières plus âgées s'étaient approchées de Hermione après qu'elles s'être assise à l'extrémité opposée à celle où se trouvait Harry Potter, mais un nuage de filles plus jeunes et plus rapide avait alors entouré celle-ci d'une barrière impénétrable.

<<~Les garçons, dit Hermione Granger, ne devraient pas avoir le droit d'aimer les filles sans leur demander d'abord~! C'est vrai à plusieurs égards, et en particulier quand il s'agit de coller des gens au plafond~!~>>

On ignora aussi cela. <<~C'est comme dans une pièce de théâtre~!~>> soupira une fille en troisième année.

<<~Une pièce~? dit Hermione. J'aimerais voir une pièce où \emph{ce} genre de choses se produit~!

--- Oh, dit la troisième année. Je pensais à cette pièce vraiment \emph{romantique} où il y a ce garçon tout mignon et tout gentil qui fait un appel de cheminette, sauf qu'il prononce mal le nom de sa destination et qu'il tombe dans cette pièce pleine de mages noires qui sont en train d'accomplir un rituel interdit qui aurait dû sombrer dans l'abîme du temps et ils sont en train de sacrifier sept victimes pour desceller une ancienne horreur censée octroyer un souhait lors de sa libération, donc bien sûr la présence du garçon interrompt le rituel, et pendant que l'horreur est en train de manger tous les mages noirs et que tout le monde meure, la dernière pensée du garçon est qu'il aurait bien aimé avoir une copine, et l'instant d'après il est assis sur les genoux d'une femme magnifique dont les yeux brûlent d'une lumière terrifiante, sauf qu'elle ne comprend pas ce que c'est que d'être humaine alors le garçon doit toujours l'empêcher de manger des gens. C'est exactement comme dans cette pièce, sauf que tu es le garçon et que Harry Potter est la fille~!

--- Ça… dit Hermione passablement surprise. En fait ça ressemble \emph{bien} à quelque chose comme…

--- \emph{Vraiment}~?~>> laissa échapper une deuxième année assise face à Hermione en se penchant en avant, visiblement autant horrifiée que fascinée.

<<~Non~! dit Hermione. Je veux dire -- \emph{ce n'est pas mon petit copain~!}~>>

Deux secondes plus tard, ses oreilles eurent vent de ce que ses lèvres venaient de dire.

La quatrième année mit une main sur l'épaule de Hermione et la serra d'un geste réconfortant. <<~Mlle Granger, dit-elle d'une voix apaisante, je pense que si tu es vraiment honnête avec toi-même, tu admettras que la véritable raison pour laquelle tu es en colère contre ton sombre maître, c'est parce qu'il a canalisé ses pouvoirs innommables au travers de Tracey Davis plutôt qu'au travers de toi.~>>

La bouche de Hermione s'ouvrit mais sa gorge se bloqua avant que les mots ne puissent sortir, ce qui fut probablement préférable, car si elle avait vraiment hurlé aussi fort qu'elle en avait eu l'intention, elle se serait probablement cassé quelque chose.

<<~D'ailleurs, comment c'est possible~? dit la troisième année. Je veux dire que Harry Potter puisse œuvrer par une autre fille alors qu'il s'est lié à toi~? Est-ce que vous avez un, tu sais, un de ces arrangements à trois~?

--- \emph{Aaaaaargh}~>>, dit Hermione Granger, sa gorge toujours bloquée, son cerveau sur arrêt, ses cordes vocales imitant spontanément le bruit d'une gorge qui régurgitait un yak.

\latersection{(Plus tard)}

<<~Je ne comprends pas pourquoi tu es si \emph{déraisonnable}~>>, dit une autre sorcière de deuxième année qui avait remplacé la troisième année après que Hermione eut menacé de demander à Tracey de manger l'âme de celle-ci. <<~Enfin franchement, si quelqu'un comme Harry Potter \emph{me} sauvait, je lui… Je lui enverrais des cartes de remerciements, et je lui ferais des câlins, et~>>, le visage de la fille rougit quelque peu, <<~enfin, je l'embrasserais, j'espère.

--- Ouais~! dit une autre deuxième année. Je n'ai jamais compris pourquoi les filles dans les pièce se mettent en \emph{colère} quand le personnage principal se met en quatre pour leur faire plaisir. \emph{Je} ne me comporterais pas comme ça si le héros m'aimait \emph{moi}.~>>

Hermione Granger avait laissé tomber sa tête sur la table et ses mains tiraient lentement ses cheveux.

<<~Vous ne comprenez juste pas la psychologie masculine, dit la quatrième année d'un ton d'experte. Granger doit lui donner \emph{l'impression} d'être mystérieusement capable de pouvoir résister à son charme.~>>

\latersection{(Encore plus tard)}

Et Hermione Granger se tourna donc rapidement vers la seule personne restante à laquelle elle pouvait parler, la seule personne dont elle savait que cette dernière comprendrait son point de vue…

<<~Ils sont tous fous~>>, dit Hermione Granger en marchant à grandes enjambées vers la tour Serdaigle après être partie du dîner bien avant qu'il ne se termine. <<~Tout le monde sauf toi et moi, Harry, et je veux dire \emph{tout le monde} à Poudlard sauf nous, ils sont tous complètement fous. Et les filles de Serdaigle sont les \emph{pires}, je ne sais ce que les Serdaigle plus âgées se mettent à lire quand elles grandissent mais je suis certaine que ça n'est pas bon pour elles. Une sorcière nous a demandé si on avait lié nos âmes, et je vais faire des recherches à la bibliothèque cette nuit mais je suis relativement certaine que ça ne s'est jamais produit…

--- Je n'ai même pas de \emph{mot} pour ce genre de raisonnement fallacieux,~>> dit Harry Potter. Le garçon marchait d'un pas normal, si bien qu'il devait parfois sautiller pendant quelques pas pour rester à hauteur de la vitesse indignée de Hermione. <<~Je pense sérieusement que si ça ne dépendait que \emph{d'eux}, ils nous traîneraient par les pieds pour faire changer nos noms et qu'on s'appelle Potter-Evans-Verres-Granger… berk, en le disant à voix haute je me rends compte à quel point ça sonne mal.

--- Tu veux dire que \emph{ton} nom serait Potter-Evans-Verres-Granger et que le \emph{mien} serait Granger-Potter-Evans-Verres, dit Hermione. C'est trop atroce pour que j'arrive à le concevoir.

--- Non, répondit le garçon, la maison Potter est une maison noble, donc je pense que ce nom reste devant…

--- \emph{Quoi~?} dit-elle indignée. Qui a dit qu'on \emph{devait}…~>>

Il y eut un silence soudain et horrible, brisé seulement par le bruit sourd de leurs semelles.

<<~\emph{Enfin bref}, continua Hermione avec hâte, quelques-unes des absurdités qu'ils ont dit au dîner m'ont fait réfléchir, et donc je veux juste te dire, Harry, que je te suis vraiment reconnaissante de nous avoir sauvés d'une raclée, moi et tout le monde, et même si certains aspects de ce que tu as fait cet après-midi m'énervent je suis certaine qu'on peut juste en discuter calmement.

--- Ah…~>> dit Harry, avec un léger sourire timide. Ses yeux révélaient un mélange d'appréhension et d'incompréhension, <<~j'imagine que c'est… bien~?~>>

Pour être précis, il y avait eu cette sorcière en quatrième année qui avait expliqué que puisque Harry était le méchant sorcier amoureux de Hermione et que Hermione était la fille pure et innocente qui le rachèterait ou qui serait séduite par les forces du mal, il s'ensuivait que Hermione \emph{devait} être perpétuellement indignée par tout ce que Harry faisait, même s'il s'agissait de la sauver d'une fin certaine, juste pour que leur liaison n'atteigne pas son dénouement avant la fin de l'acte IV. Et \emph{là}, Pénélope Deauclaire, que Hermione avait cru être plus intelligente que ça, avait remarqué d'une voix forte que pour la même raison il était \emph{impossible} que Hermione se contente d'aller juste parler à Harry comme quelqu'un de raisonnable quant à ce qui l'avait blessée puisque de toute façon ce qui attirait les mages noirs chez les femmes, c'était leur résistance passionnée, pas leur esprit logique. C'est à ce moment que Hermione s'était levée brusquement de son banc, avait furieusement martelé le sol jusqu'à l'endroit où Harry avait été assis et qu'elle lui avait demandé d'un ton raisonnable s'ils pourraient aller marcher un peu et régler cette histoire.

<<~Donc en d'autres termes~>>, dit Hermione de la voix la plus calme qu'elle avait jamais eu, <<~on n'est pas en froid, je te parle encore, on est encore amis et on étudie toujours ensemble. On n'est \emph{pas} train de se disputer. D'accord~?~>>

Cela ne sembla qu'augmenter l'appréhension de Harry Potter.

<<~D'accord, dit le Survivant.

--- Génial~! dit Hermione. Donc, \emph{avez-vous} trouvé pourquoi je suis en colère, M. Potter~?~>>

Il y eut un silence. <<~Tu voulais que je ne m'occupe pas de tes affaires~? dit prudemment Harry. Enfin -- je sais que tu voulais gérer les choses toi-même. Et je \emph{restais} hors de ton chemin jusqu'à ce que j'entende que tu t'étais fait prendre en embuscade par trois Mangemorts juniors et franchement, je ne m'attendais pas à ça. Le \emph{professeur Quirrell} ne s'attendait pas à ça. J'ai commencé à m'inquiéter que la situation commence à te dépasser, sans vouloir te vexer Hermione, quarante-quatre brutes amassées dans une embuscade sont \emph{loin} au-delà de ce que n'importe qui pourrait gérer sans aide. C'est pour ça que j'ai pensé que tu aurais vraiment besoin d'aide juste cette fois…

--- Non, ça, ça ne me dérange pas, dit Hermione. On \emph{était} dépassé par les événements. Essayez de deviner encore, M. Potter.

--- Euh, dit Harry. Ce que Tracey a fait t'a… surpris~?

--- Surpris, M. Potter~?~>> Peut-être y eut-il une touche d'acidité dans sa voix. <<~Non, M. Potter, j'ai eu \emph{peur}. J'ai été \emph{effrayée}. Je ne voudrais pas admettre que j'ai peur de simples \emph{dragons} ou de quelque chose comme ça, les gens pourraient penser que je suis une \emph{pleutre}, mais quand on entend des voix lointaines crier “Tekeli-li~! Tekeli-li~!” et qu'il y a des mares de sang qui coulent de sous les portes, alors c'est normal d'avoir peur.

--- Je \emph{suis} désolé, dit Harry d'un ton qui semblait exprimer un regret sincère. Je pensais que tu comprendrais que c'était moi.

--- Et la \emph{raison} pour laquelle on a toutes eu peur comme ça, M. Potter, c'est que vous n'avez pas commencé par nous \emph{demander la permission}~! Hermione découvrit que sa voix commençait à monter en dépit de ses bonnes intentions. Tu aurais dû me \emph{demander} avant de faire quelque chose comme ça, Harry~! Tu aurais dû très précisément demander~: “Hermione, est-ce que je peux faire couler du sang de sous les portes~?” C'est important d'être précis quand on demande la permission de faire ce genre de choses~!~>>

Le garçon continua de marcher et se frotta la nuque.

<<~Je… franchement, j'ai juste pensé que tu aurais \emph{dû} dire non.

--- Oui, M. Potter, \emph{j'aurais pu dire non}. C'est \emph{exactement pour ça que vous auriez dû me demander d'abord}~!

--- Non, je veux dire que tu aurais \emph{dû} dire non, que ce soit ce que tu voulais \emph{vraiment} ou pas. Et alors vous auriez toutes reçu une raclée et ça aurait été \emph{ma} faute parce que je vous aurais demandé la permission.~>>

Les sourcils de Hermione s'élevèrent sous le coup de la surprise et elle fit quelques pas de plus en essayant de comprendre.

<<~Quoi~? dit-elle.

--- Eh bien… dit le garçon avec une certaine lenteur. Je veux dire… tu es le général Soleil, non~? Tu ne \emph{pourrais pas} me donner la permission de faire peur aux gens, pas même aux brutes, pas même pour empêcher tes amies de se prendre une raclée. Tu aurais \emph{dû} dire non et alors tu aurais souffert. Mais maintenant, tu peux honnêtement dire aux gens que tu n'avais pas la moindre idée de ce qui allait se passer et que ce n'est pas ta faute. C'est pour ça que je ne t'ai pas prévenue.~>>

Hermione s'arrêta de marcher et plutôt que de simplement tourner la tête, elle se retourna pour faire entièrement face à Harry. Sa voix fut précautionneusement neutre lorsqu'elle dit~: <<~Harry, tu \emph{dois} arrêter d'inventer des raisons alambiquées de faire des choses stupides.~>>

Les sourcils de Harry s'élevèrent. Après un moment, il dit.

<<~Écoute… je comprends ce que tu dis, bien sûr, mais la question reste en suspens~: est-ce que c'était \emph{vraiment} une bonne idée ou est-ce que c'était juste un plan astucieux…

--- Je comprends pourquoi tu as fait ce que tu as fait aujourd'hui, continua Hermione. Mais je veux que tu me promettes qu'à partir de maintenant, tu commenceras par me demander, toujours, même si tu trouves une raison pour laquelle tu ne devrais pas le faire.~>>

Il y eut un silence qui s'étira et Hermione put sentir son cœur se serrer.

<<~Hermione -- commença Harry.

--- \emph{Pourquoi~?}~>> la frustration éclata dans la voix Hermione. <<~\emph{Pourquoi est-ce si terrible~? Tout ce que tu as à faire, c'est de me demander~!}~>>

Harry avait un air très sérieux.

<<~Qui à la SPEHS essayes-tu de défendre le plus, Hermione~? Pour qui as-tu le plus peur quand tu te bats~?

--- Hannah Abbott~>>, dit Hermione sans avoir à y penser, et elle se sentit alors un peu coupable parce que Hannah \emph{essayait} de toutes ses forces et qu'elle \emph{s'était} beaucoup améliorée…

<<~Est-ce que tu te permettrais de compter sur quelqu'un d'autre, sur Tracey par exemple, pour être l'\emph{ultime} responsable du bien-être de Hannah~? Si tu savais que Hannah allait tomber dans une embuscade et que tu trouvais un plan pour la protéger, te sentirais-tu bien à l'idée de permettre à Tracey de te dire si tu as ou non le droit de le mettre à exécution~?

--- Euh… non~?~>> dit Hermione, perplexe.

Les yeux verts du Survivant étaient braqués sur elle.

<<~Ferais-tu confiance à \emph{Hannah} pour prendre la décision finale sur son besoin d'être protégée ou non~?

--- Je…~>> dit Hermione, et elle s'interrompit. C'était étrange, elle connaissait la bonne réponse mais elle savait aussi que la bonne réponse n'était pas la vraie réponse. Hannah essayait tellement de prouver qu'elle n'avait pas peur, alors que \emph{c'était le cas}, et c'était facile de voir comment la Poufsouffle pourrait en faire \emph{trop}…

Puis Hermione comprit ce que cela impliquait.

<<~Tu penses que je suis comme \emph{Hannah}~?

--- Pas… exactement…~>> Harry passa sa main dans le fatras de ses cheveux. <<~Écoute, Hermione, qu'est-ce que \emph{tu} m'aurais suggéré de faire si je t'avais prévenu qu'il y aurait une embuscade tendue par quarante-quatre brutes~?

--- J'aurais réagi de façon \emph{responsable}, je l'aurais dit au \emph{professeur McGonagall} et je l'aurais laissée s'en occuper, répondit Hermione immédiatement. Et \emph{alors} il n'y aurait eu ni ténèbres ni hurlements ni terrible lumière bleue…~>>

Mais Harry se contenta de secouer la tête.

<<~Ce n'est \emph{pas} une réaction responsable. C'est ce que quelqu'un jouant le \emph{rôle} d'une fille responsable ferait. \emph{Oui}, j'ai pensé à aller voir le professeur McGonagall. Mais elle aurait empêché le désastre \emph{une fois}. Probablement avant le début des troubles, par exemple en disant aux brutes qu'elle était au courant de tout. Si les brutes ne s'étaient fait punir que pour avoir comploté, ça se serait traduit par une perte de points, ou au pire par un jour de retenue, mais pas par quelque chose capable de vraiment leur faire peur. Et alors les brutes auraient \emph{réessayé}. Moins nombreuses et dotées d'une meilleure sécurité opérationnelle pour que je n'en entende pas parler. Elles auraient probablement tendu une embuscade à \emph{l'une} d'entre vous, seule. Le professeur McGonagall n'est pas \emph{autorisée} à faire quelque chose de suffisamment effrayant pour pouvoir te protéger -- et \emph{elle} n'aurait pas outrepassé son autorité parce qu'elle n'est pas vraiment responsable.

--- Le \emph{professeur McGonagall} n'est pas responsable~?~>> dit Hermione d'un ton incrédule. Elle plaqua ses mains contre ses hanches et le regarda d'un air ouvertement furieux. <<~Est-ce que tu es \emph{dingue}~?~>>

Le garçon ne cilla pas.

<<~Tu pourrais peut-être appeler ça la responsabilité héroïque, dit Harry Potter. Pas le genre habituel. Ça veut dire que quoi qu'il arrive, quelles que soient les circonstances, c'est \emph{toujours} de ta faute. Même si tu en parles au professeur McGonagall, ce n'est pas elle qui est responsable de ce qui arrive, c'est \emph{toi}. Suivre le règlement de l'école n'est pas une excuse, être sous l'autorité de quelqu'un d'autre n'est pas une excuse, même faire de ton mieux n'est pas une excuse. Il n'y a aucune excuse, tu dois \emph{faire le travail, quoi qu'il arrive}.~>> Le visage de Harry s'était pincé. <<~C'est pour ça que je dis que tu n'as pas une attitude responsable, Hermione. Penser que ton travail est fini quand tu en as parlé au professeur McGonagall -- ce n'est pas penser comme une héroïne. Comme si ça devenait alors \emph{acceptable} que Hannah se fasse frapper sous prétexte que ce ne serait plus \emph{ta faute}. Être une héroïne, ça veut dire que ton travail n'est terminé que quand tu as fait \emph{le nécessaire} pour protéger les autres filles de façon \emph{permanente}.~>> Un fragment de la dureté qu'il avait acquise depuis le jour où Fumseck s'était posé sur son épaule s'était glissé dans sa voix. <<~Tu ne peux pas t'imaginer que tu as accompli ton devoir en suivant les règles.

--- Je pense, dit Hermione d'une voix neutre, que vous et moi sommes peut-être en désaccord sur certaines choses, M. Potter. Comme de savoir qui est le plus \emph{responsable}, de vous et du professeur McGonagall, ou si un comportement \emph{responsable} se termine souvent par des gens criant et courant dans tous les sens, ou si c'est une bonne idée de respecter le règlement de l'école. Et ce n'est pas parce que nous ne sommes pas d'accord, M. Potter, que \emph{vous} devez avoir le dernier mot.

--- Eh bien, dit Harry, tu m'as demandé ce qu'il y avait de \emph{si} terrible à te demander la permission d'abord, et c'était une question étonnamment bonne, donc j'ai examiné mon esprit et c'est ça que j'ai trouvé. Je pense que ma véritable peur, c'est que si Hannah a des ennuis et que je trouve un moyen de la sauver qui semble étrange, sombre ou quelque chose comme ça, tu ne prendras peut-être pas en compte les conséquences que cela aura pour Hannah. Tu n'accepteras peut-être pas la responsabilité héroïque de trouver \emph{un} moyen de la sauver, quoi qu'il en coûte. Au lieu de ça, tu risques de juste jouer le \emph{rôle} de Hermione Granger, la Serdaigle raisonnable~; et le \emph{rôle} de Hermione Granger dit non automatiquement, qu'elle ait un meilleur plan en tête ou pas. Et alors quarante-quatre brutes feront la queue pour tabasser Hannah Abbott et ce sera ma faute parce que je \emph{l'aurais su}, même si je ne voulais pas que les choses soient ainsi, j'aurais su que c'est comme ça qu'elles se dérouleraient. Je suis quasiment certain que c'était là ma peur secrète, indicible et muette.~>>

La frustration s'accumulait de nouveau en Hermione. <<~C'est \emph{ma} vie~!~>> éclata-t-elle. Elle pouvait imaginer comment ce serait si Harry se mêlait tout le temps de sa vie, inventait en permanence des justifications qui lui permettraient de ne pas lui demander la permission, de ne pas écouter ses objections. Elle n'aurait pas dû devoir \emph{gagner un débat} juste pour -- <<~Il y aura \emph{toujours} une raison, tu pourras \emph{toujours} dire que je n'ai pas les idées claires~! Je veux avoir ma \emph{propre vie}~! Sinon je m'éloignerai, vraiment Harry, je suis sérieuse.~>>

Harry soupira. <<~C'est exactement là où je ne voulais pas que nous en arrivions, et nous y voilà. Tu as peur exactement de la même chose que moi, hein~? Peur que si \emph{tu} lâches le gouvernail, on se fracassera.~>> Le coin de ses lèvres se tordit mais ça ne ressembla pas à un vrai sourire.

<<~Je peux comprendre ça.

--- Je ne pense pas que tu comprennes \emph{du tout}~! dit Hermione d'un ton brusque. Tu as dit qu'on serait \emph{partenaires}, Harry~!~>>

Elle vit que ces mots l'avaient arrêté.

<<~Qu'est-ce que tu penses de ça~? finit-il par dire. Je promets de te demander avant de faire quoi que ce soit qui pourrait être considéré comme une intrusion dans tes affaires. Sauf que \emph{tu} dois \emph{me} promettre d'être raisonnable, Hermione. Je veux dire \emph{vraiment}, sincèrement, d'y réfléchir pendant vingt secondes d'abord, de l'étudier comme un véritable choix. D'être assez raisonnable pour te rendre compte que je t'offre un moyen de protéger les autres filles et que si tu dis automatiquement \emph{non} sans l'envisager sérieusement, il y a cette \emph{conséquence réelle} où Hannah Abbott se retrouve à l'hôpital.~>>

Hermione continua de regarder fixement Harry lorsque la récitation de ce dernier s'acheva.

<<~Eh bien~? dit Harry.

--- Je ne devrais pas avoir à faire de promesses, dit-elle, juste pour être \emph{consultée} au sujet de ma \emph{propre vie}.~>> Elle se détourna de Harry et commença à marcher vers la tour Serdaigle sans le regarder. <<~Mais j'y réfléchirai quand même.~>>

Elle entendit Harry soupirer, puis il marchèrent en silence pendant un moment, passèrent sous une arche faite d'un métal rougeâtre, semblable au cuivre, puis entrèrent dans un couloir parfaitement semblable à celui qu'ils venaient de quitter hormis son dallage composé de pentagones et non plus de carrés.

<<~Hermione, dit Harry. Depuis le jour où tu as dit que tu voulais être une héroïne, je t'ai observé et j'ai réfléchi. Tu \emph{as} le courage. Tu te battras pour ce qui est juste, même face à des ennemis qui en effraieraient d'autres. Tu as certainement l'intelligence brute nécessaire, et tu es probablement quelqu'un de mieux de moi. Mais même ainsi… eh bien, pour être honnête, Hermione… je ne te vois pas vraiment dans rôle de Dumbledore, à mener l'Angleterre magique contre Tu-Sais-Qui. Pas encore, du moins.~>>

Hermione avait tourné la tête pour regarder Harry, mais ce dernier avait juste continué de marcher, comme perdu dans ses pensées. Jouer \emph{ce} rôle~? Elle n'avait jamais essayé de s'imaginer ainsi. Elle n'avait jamais \emph{imaginé} s'imaginer ainsi;

<<~Et peut-être que j'ai tort, dit Harry tandis qu'ils marchaient. Peut-être que j'ai lu trop d'histoires où les héros ne sont jamais raisonnables, ne suivent jamais les règles et n'en parlent jamais au professeur McGonagall, si bien que mon cerveau ne pense pas que tu es une bonne héroïne de fiction. Peut-être que c'est toi qui es saine d'esprit, Hermione, et moi qui suis un idiot. mais à chaque fois que tu parles de suivre les règles ou de te reposer sur les professeurs, j'ai la même sensation, que c'est lié à cette dernière chose qui te bloque, cette dernière chose qui endort ton toi PJ et te transforme à nouveau en PNJ…~>> Harry laissa échapper un soupir. <<~Peut-être que c'est pour ça que Dumbledore a dit que j'aurais dû avoir des parents malfaisants.

--- Il a dit \emph{quoi}~?~>>

Harry hocha la tête. <<~Je ne sais toujours pas s'il blaguait ou… le truc, c'est qu'en un sens il avait \emph{raison}. J'ai \emph{eu} des parents aimants mais je n'ai jamais eu l'impression de pouvoir compter sur leurs décisions, ils n'étaient pas assez \emph{sains d'esprits}. J'ai toujours su que si je ne réfléchissais pas aux choses moi-même, je risquerais d'avoir mal. Le professeur McGonagall mettra tout en œuvre pour faire ce qu'il faut faire \emph{si} je suis là pour harceler mais elle ne brise pas les règles seule sans être supervisée par un héros. Le professeur Quirrell \emph{est} vraiment quelqu'un qui fait ce qu'il faut faire quoi qu'il en coûte, et c'est la seule autre personne que je connaisse qui remarque certaines choses, comme par exemple que le Vif gâche le Quidditch. Mais \emph{lui}, je ne sais pas si c'est quelqu'un de \emph{bien}. Même si c'est triste, je pense que ça fait partie de l'environnement qui crée ce que Dumbledore appelle un héros -- quelqu'un qui n'a personne d'autre à qui refiler la responsabilité et qui forme donc l'habitude mentale de tout vérifier lui-même.~>>

Hermione ne répondit rien mais elle repensa à ce que Godric Gryffondor avait écrit presque à la fin de sa très courte autobiographie. Brièvement et sans explication, parce que le parchemin avait été destiné à être copié à la main, des siècles avant que la presse typographique moldue n'inspire les sorciers à inventer la Plume à Lire-Écrire.

\emph{Aucun sauveur n'a de sauveur}, avait écrit Godric Gryffondor. \emph{Aucun seigneur n'a de champion, de mère ou de père, seul le vide au-dessus.}

Si c'était \emph{ça} le coût qu'il y avait à être un héros, Hermione n'était pas certaine de vouloir le payer. Ou peut-être -- mais ce n'était pas le genre de chose qu'elle aurait pensé avant qu'elle ne commence à passer du temps avec Harry -- peut-être Godric Gryffondor avait-il eu \emph{tort}.

<<~Est-ce que tu fais confiance à \emph{Dumbledore}~? demanda Hermione. Je veux dire, il vit dans cette école et c'est le héros le plus légendaire du monde…

--- C'\emph{était} le héros le plus légendaire du monde, dit Harry. Maintenant il met le feu à des poulets. Franchement, est-ce que Dumbledore t'a l'air fiable à \emph{toi}~?~>>

Hermione ne répondit pas.

Côte à côte, ils commencèrent à grimper l'énorme spirale d'escaliers dont les marches alternaient entre le bronze et la pierre bleue suivi de l'approche finale jusqu'au portrait Serdaigle qui protégeait la porte par des énigmes stupides.

<<~Oh, et je viens de penser à quelque chose que je dois te dire, continua Harry lorsqu'ils eurent parcouru environ la moitié de l'escalier. Puisque ça affecte ta vie et tout ça. Vois ça comme une avance…

--- Qu'est-ce que c'est~? dit Hermione.

--- Je prédis que la SPEHS va prendre sa retraite.

--- \emph{Sa retraite}~? dit Hermione en trébuchant presque sur une des marches.

--- Ouais, dit Harry. Enfin, je pourrais me tromper, mais j'ai dans l'idée que les professeurs vont sévèrement réprimer les combats dans les couloirs.~>> Harry souriait tout en parlant et une lueur dans ses yeux couverts par ses lunettes laissait entendre quelque savoir secret. <<~Lancer des nouveaux sortilèges pour détecter les maléfices offensifs ou commencer à vérifier les rapports de brutalisation sous Veritaserum -- je peux imaginer plusieurs méthodes de répression. Mais si j'ai raison, il faut que vous le fêtiez, toi et toutes les autres. Vous avez suffisamment chahuté pour les pousser à \emph{faire} quelque chose au sujet des brutalisations. De \emph{toutes} les brutalisations.~>>

Et alors, lentement, un sourire naquit sur les lèvres de Hermione, et en atteignant la dernière marche des escaliers, en s'avançant vers le portrait de Serdaigle pour entendre son énigme, Hermione se sentit assez légère, une merveilleuse sensation de soulèvement, comme si elle s'était fait injecter de l'hélium.

Étrangement, en dépit de tous les efforts qu'elle et les sept autres y avaient investi, elle ne s'était pas attendue à \emph{tant}, elle ne s'était pas attendu à ce que ça \emph{fonctionne} vraiment.

Elles avaient \emph{changé les choses}…

\later

Le matin suivant, à la fin du petit déjeuner.

Les élèves de chaque année étaient assis sur leurs bancs, immobiles, toutes les têtes tournées dans la même direction~: vers la table d'honneur, face à une fille de première année, debout seule, figée, le menton relevé pour regarder le directeur de la maison Serpentard.

Le visage du professeur Rogue était déformé par la furie et le triomphe, aussi vindicatif que le visage d'un mage noir~; et derrière lui les autres professeurs assis à la table d'honneur le regardaient, leurs visages aussi rigides que s'ils avaient été sculptés dans la pierre.

<<~… définitivement dissolue~>>, cracha le professeur de potions. <<~Votre Société autoproclamée est dorénavant \emph{illégale} au sein de Poudlard, par mon ordre~! Si votre Société ou un membre de celle-ci est à nouveau découvert se battant dans les couloirs, Granger, vous serez \emph{personnellement} tenue pour responsable et exclue, par moi, de l'école de Sorcellerie de Poudlard~!~>>

La fille de première année se tenait là, face à la table d'honneur où elle n'avait auparavant été conviée que pour recevoir des félicitations et des sourires~; elle se tenait là, son dos droit et haut, courbé comme un arc de centaure, sans rien céder à l'ennemi.

La fille de première année se tenait là, ses larmes et sa colère contenue, son visage immobile, sans aucun changement d'apparence extérieure, tandis qu'elle sentait que quelque chose se cassait lentement à l'intérieur d'elle.

Cette chose se cassa encore plus lorsque le professeur lui donna deux semaines de retenue pour violence au sein de l'école avec l'air narquois qu'il leur avait donné à voir à tous lors de leur premier jour de potions et un petit rictus en coin qui indiquait qu'il savait très bien à quel point il était injuste.

Cette chose en elle, quoi qu'elle fut, se déchira d'un bout à l'autre, de haut en bas, lorsque le professeur Rogue retira cent points à Serdaigle.

Ce fut alors fini, et Rogue lui dit qu'elle pouvait partir.

Elle se retourna et vit qu'à la table Serdaigle, Harry Potter était toujours assis à sa place. Elle ne pouvait pas bien voir son visage d'ici, ses poings étaient sur la table mais elle ne pouvait pas voir si, comme les siens, ils étaient serrés à en être devenus blancs. Lorsque le professeur Rogue l'avait appelée, elle lui avait chuchoté qu'il ne devrait rien faire sans lui demander avant.

Hermione fit demi-tour pour regarder la table d'honneur au moment même où Rogue se détournait pour revenir à sa place.

<<~J'ai dit que vous pouviez partir, jeune fille~>>, dit la voix narquoise, mais il avait un sourire satisfait, comme s'il attendait qu'elle fasse quelque chose…

Hermione fit cinq pas de plus jusqu'à la table d'honneur et dit d'une voix brisée~: <<~M. le directeur~?~>>

Un silence absolu tomba sur la grande salle.

Dumbledore ne dit rien, ne bougea pas. C'était comme s'il avait lui aussi été sculpté dans la pierre.

Hermione détourna les yeux pour regarder le professeur Flitwick dont la tête, à peine visible au-dessus de la table, semblait penchée vers ses genoux. À côté de lui, le visage du professeur Chourave était très pincé, elle semblait se forcer à observer la scène, ses lèvres tremblèrent, mais elle ne dit rien.

La chaise du professeur McGonagall était vide. La directrice adjointe n'était pas venue ce matin.

<<~Pourquoi est-ce qu'aucun d'entre vous ne dit rien~?~>> demanda Hermione. Sa voix tremblait, chargée de son dernier espoir, de l'ultime tentative de cette chose en elle pour rechercher de l'aide. <<~Vous \emph{savez} que ce que vous faites est mal~!

--- Deux semaines de retenue supplémentaire pour insolence~>>, dit Rogue d'une voix soyeuse.

La chose se fracassa.

Elle regarda la table d'honneur pendant quelques secondes de plus, vers le professeur Flitwick, vers le professeur Chourave, vers la chaise vide où le professeur McGonagall aurait dû se trouver. Puis Hermione Granger se retourna et commença à marcher vers la table Serdaigle.

Un babillage s'éleva et les élèves cessèrent d'être pétrifiés.

Et alors, au moment où elle atteignait presque la table Serdaigle…

La voix sèche du professeur Quirrell trancha tout le reste, et cette voix dit~: <<~Cent points pour Serdaigle pour avoir fait ce qui était juste.~>>

Hermione tomba presque par terre, puis elle continua, alors même que Rogue criait quelque chose avec rage, alors même que le professeur Quirrell s'enfonçait dans sa chaise et commençait à rire, alors que la voix de Dumbledore disait quelque chose qu'elle ne comprit pas, puis elle fut de nouveau assise à la table Serdaigle, à côté de Harry Potter.

À côté d'elle, Harry était figé, comme s'il avait trop peur pour pouvoir bouger.

<<~Tout va bien~>>, lui dit sa voix, automatiquement, sans qu'elle l'ait choisi ni qu'elle y ait pensé et même si tout n'allait pas bien du tout. <<~Mais est-ce que tu pourrais essayer de me sortir de la retenue de Rogue comme tu l'as fait pour toi la dernière fois~?~>>

Harry Potter eut un hochement de tête saccadé. <<~Je… dit Harry. Je… Je suis désolé, c'est… c'est entièrement ma faute…

--- Ne sois pas \emph{absurde}, Harry.~>> C'était étrange d'entendre sa voix, parfaitement normale, sans qu'elle ait à réfléchir à ce qu'elle voulait dire. Elle se pencha sur son assiette mais manger semblait clairement hors de question. Il y avait des roulements et des tourbillons dans son estomac qui suggéraient qu'elle était sur le point de vomir, ce qui était étrange car elle aurait pu en même temps jurer que tout son corps était engourdi, comme si elle n'avait rien pu sentir.

<<~Et, dit sa voix, si tu veux enfreindre le règlement ou quelque chose comme ça, tu peux me demander, je te promets que je ne te répondrai pas juste~: “non”~>>.
\later

\begin{center}
Non est salvatori salvator,\\
neque defensori dominus,\\
nec pater nec mater,\\
nihil supernum.

… Godric Gryffindor,\\
1202 ap. JC.
\end{center}

%  LocalWords:  firstie Gaaaaack Tekeli li McGonagalls NPC salvatori neque
%  LocalWords:  salvator defensori dominus nec
