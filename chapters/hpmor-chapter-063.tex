\chapter{L'Expérience de Prison de Stanford, Après-coup}

\section{Après-coup, Hermione Granger~:}

\lettrine{I}{mitée} par Padma et Mandy qui rangeaient les leurs à l'autre bout de la table, Hermione commençait tout juste à refermer ses livres et à ranger ses devoirs avant de se préparer à aller dormir quand Harry Potter entra dans la salle commune de Serdaigle~; ce n'est qu'alors qu'elle se rendit compte qu'elle ne l'avait pas vu depuis le petit déjeuner.

Ce fait fut rapidement écrasé par un autre, bien plus saisissant.

Sur l'épaule de Harry se trouvait une créature d'or et de rouge, un oiseau de feu étincelant.

Et Harry avait l'air triste, usé, vraiment \emph{fatigué}, comme si le phénix était la seule chose qui le faisait tenir sur pied, mais une chaleur émanait toujours de lui et si vous aviez croisé son regard vous auriez pu croire que vous regardiez le directeur~; voilà ce qui passa par la tête de Hermione même si cela n'avait aucun sens.

Harry Potter traversa péniblement la salle commune et se dirigea vers Hermione, passant devant des sofas emplis de filles et des cercles de joueurs de cartes qui le fixaient du regard.

Elle ne reparlait théoriquement pas encore à Harry car la semaine ne s'achèverait que le lendemain, mais il se passait clairement quelque chose de \emph{beaucoup} plus important que cela -

<<~Fumseck, dit Harry alors même qu'elle ouvrait la bouche, cette fille s'appelle Hermione Granger, et elle ne me parle pas en ce moment parce que je suis un idiot, mais si tu veux être sur l'épaule de quelqu'un de bien~: elle vaut mieux que moi.~>>

Tant d'épuisement et de douleur dans la voix de Harry -

Mais avant qu'elle n'ait le temps de savoir comment réagir le phénix avait déjà glissé le long de l'épaule de Harry, comme la flamme d'une allumette qui se serait avancée en accéléré, droit vers elle~; et voilà qu'un phénix volait devant elle et la regardait de ses yeux de feu et de lumière.

<<~\emph{Croa~?}~>> demanda le phénix.

Hermione le regarda en ayant l'impression qu'elle faisait face à une question d'un contrôle qu'elle aurait oublié de réviser, comme si ça avait été la plus importante de toutes les questions et qu'elle avait vécu sa vie entière sans s'y préparer~; et elle ne savait pas quoi répondre.

<<~Je… dit-elle. J'ai seulement douze ans, je n'ai encore rien \emph{fait} -~>>

Le phénix pivota d'un mouvement souple autour de l'extrémité d'une de ses ailes, tel la créature aérienne et lumineuse qu'il était, avant de s'élever jusqu'à l'épaule de Harry où il revint fermement se poser.

<<~Espèce d'idiot~>>, dit Padma depuis l'autre côté de la table avec l'air de décider si elle devait rire ou faire la grimace, <<~les phénix, ce n'est pas pour les filles intelligentes qui font leurs devoirs, c'est pour les imbéciles qui foncent droit sur cinq brutes Serpentard plus vieilles que lui. Tu sais, si les couleurs de Gryffondor sont l'or et le rouge, c'est pour une bonne raison.~>>

De nombreux rires amicaux s'élevèrent dans la salle commune de Serdaigle.

Hermione ne faisait pas partie des rieurs.

Harry non plus.

Il avait placé une main sur son visage. <<~Dis à Hermione que je suis désolé~>>, dit-il à Padma, sa voix presque au niveau d'un murmure. <<~Dis-lui que j'ai oublié que les phénix sont des animaux, qu'ils ne comprennent pas ce qu'est le temps ni l'idée de se préparer, qu'ils ne comprennent pas les gens qui \emph{feront} le bien un jour — je ne suis pas certain qu'ils comprennent vraiment l'idée qu'une personne puisse \emph{être} quelque chose, tout ce qu'ils voient c'est ce que les gens \emph{font}. Fumseck ne sait pas ce que “douze” veut dire. Dis à Hermione que je suis désolé — je n'aurais pas — tout tourne toujours mal, pas vrai~?~>>

Puis il se détourna pour partir, le phénix toujours sur son épaule, se traînant lentement vers la cage d'escalier qui menait à son dortoir.

Mais Hermione ne pouvait pas s'en tenir là, ce n'était pas \emph{possible}. Elle ne savait pas si c'était dû à l'esprit de compétition qu'elle entretenait avec Harry ou à autre chose. Mais elle ne pouvait simplement pas laisser le phénix se détourner d'elle.

Il \emph{fallait} qu'elle -

Son esprit propagea une question désespérée à l'intégralité de son excellente mémoire, trouva une seule chose -

<<~J'allais courir devant le Détraqueur pour essayer de sauver Harry~! cria-t-elle d'un ton un peu désespéré à l'intention de l'oiseau rouge et or. Enfin j'ai vraiment commencé à courir et tout~! C'était stupide et courageux, non~?~>>

Avec un cri strident, le phénix se propulsa de nouveau depuis l'épaule de Harry et revint vers elle comme une flamme qui aurait jailli, puis il fit trois cercles autour d'elle, si bien qu'elle se retrouva au cœur d'un brasier, et les ailes du phénix effleurèrent sa joue l'espace d'un instant avant que ce dernier ne s'élève de nouveau vers Harry.

On put entendre des chuchotements dans la salle commune de Serdaigle.

<<~Je te l'avais dit~>>, dit Harry à voix haute, puis il commença à gravir les escaliers qui menaient à sa chambre~; il sembla les monter très vite, comme s'il avait été particulièrement léger, si bien qu'un moment plus tard lui et Fumseck étaient partis.

Hermione leva une main tremblante vers la joue contre laquelle Fumseck avait fait courir son aile, une zone de chaleur qui demeurait comme si ce petit morceau de peau avait été immolé avec beaucoup de gentillesse.

Elle pensait avoir répondu à la question du phénix mais il lui sembla qu'elle avait à peine réussi, comme si elle avait eu un 11 et qu'elle aurait pu avoir un 21 si elle avait fait un petit effort.

Si elle avait seulement \emph{essayé}.

À y réfléchir, elle n'avait pas \emph{vraiment} essayé.

Elle avait seulement fait ses devoirs -

\emph{Qui as-tu sauvé~?}

\latersection{Après-coup, Fumseck~:}

Le garçon s'était attendu à des cauchemars, à des cris, à des supplications, à des ouragans de vide hurlants, à ce que la décharge d'horreurs se déverse dans sa mémoire et commence peut-être ainsi à appartenir au passé.

Et le garçon savait que les cauchemars viendraient.

La nuit suivante.

Il rêva, et dans son rêve le monde était en feu, Poudlard était en feu, sa maison était en feu, les rues d'Oxford étaient en feu, inondées de flammes qui brillaient mais ne consumaient rien, et tous ceux qui marchaient à travers les rues étincelantes brillaient eux-même d'une lumière blanche plus vive que celle du feu, comme s'ils étaient eux-mêmes des flammes ou des étoiles.

Les autres garçons de première année allèrent se coucher et virent de leurs yeux la merveille qui était déjà parvenue à leurs oreilles~: que Harry Potter était allongé dans son lit, silencieux et immobile, tandis que perché sur son oreiller un oiseau rouge et or le surveillait et que ses ailes lumineuses étaient étendues au-dessus de lui comme une couverture tirée sur son visage.

Ses dettes attendraient un jour de plus.

\latersection{Après-coup, Drago Malfoy~:}

Drago remit sa robe en place et s'assura que la bordure verte était bien droite. Il agita sa baguette au-dessus de sa tête et prononça un sortilège que Père lui avait enseigné alors que les autres enfants jouaient encore dans la boue, un sortilège qui garantissait que pas une peluche, pas une poussière ne viendrait salir sa robe de sorcier.

Il se saisit de l'enveloppe mystérieuse que Père lui avait envoyée par l'entremise d'une chouette et la glissa dans sa robe. Il avait déjà utilisé \emph{Incendio} et \emph{Everto} sur le message mystérieux.

Puis il se rendit au petit déjeuner avec l'intention de s'asseoir s'il y parvenait à la seconde exacte où la nourriture apparaîtrait afin de donner l'impression que les autres avaient attendu qu'il apparaisse avant de manger. Pourquoi~? Parce que si vous étiez le descendant Malfoy vous étiez premier en tout, même au petit déjeuner, voilà pourquoi.

Vincent et Grégory l'attendaient à l'extérieur de sa chambre privée, debout avant lui — même s'ils n'étaient bien sûr pas aussi élégamment vêtus.

Le salle commune de Serpentard était déserte, mais n'importe qui debout à cette heure se serait de toute façon immédiatement dirigé vers le petit déjeuner.

Exception faite de leurs propres bruits de pas, les pièces du donjon étaient silencieuses, vides et pleines d'échos.

En dépit du faible nombre de personnes présentes, la Grande Salle était un brouhaha d'inquiétude. Quelques enfants plus jeunes pleuraient, d'autres couraient en tous sens entre les tables ou se tenaient face à face et se criaient dessus. Un préfet en robe rouge faisait face à deux élèves aux robes bordées de vert et leur hurlait dessus alors que Rogue se dirigeait vers le chaos -

Le bruit diminua un peu à mesure que l'on remarquait la présence de Drago, que certains visages se tournaient vers lui et se taisaient.

La nourriture apparut sur la table. Personne ne la regarda.

Et Rogue pivota sur ses talons, abandonnant ainsi sa cible, et se dirigea droit vers Drago.

Le cœur de Drago se serra sous l'effet de la peur, quelque chose était-il arrivé à Père — non, Père le lui aurait sûrement dit — quoi qu'il se soit produit, pourquoi Père ne lui avait-il pas dit -

Lorsque Rogue s'approcha, Drago vit que l'on pouvait voir des cernes sous ses yeux~; le maître des potions n'avait jamais été des plus chics (et c'était un euphémisme) mais ce matin là, sa robe était encore plus sale et désordonnée que jamais, couverte de nouvelles taches de graisse.

<<~Vous ne savez pas~?~>> lui siffla son directeur de maison lorsqu'il fut assez près. <<~De grâce, ne vous faites-vous pas livrer le journal~?

--- Qu'y a-t-il, profe-

--- On a fait sortir Bellatrix Black d'Azkaban~!

--- \emph{Quoi~?}~>> dit Drago sous l'effet de la surprise, alors que Grégory derrière lui dit quelque chose qu'il n'aurait vraiment pas dû dire et Vincent s'étrangla.

Rogue le regarda avec des yeux étroits puis hocha abruptement la tête. <<~Lucius ne vous a donc rien dit. Je vois.~>> Rogue eut un reniflement et se détourna -

<<~Professeur~!~>> dit Drago. Il commençait tout juste à comprendre ce que cela impliquait et son esprit tournait à plein régime. <<~Professeur, que devrais-je faire — Père ne m'a pas donné d'instructions -

--- Alors je \emph{suggère}, dit Rogue avec dédain tout en continuant de s'éloigner d'un pas vif, que vous \emph{disiez} cela, Malfoy, comme votre père s'y attend~!~>>

Drago se retourna et jeta un coup d'œil à Vincent et Grégory tout en se demandant à quoi bon~: bien sûr, ils avaient l'air encore plus perplexes que lui.

Puis il s'avança jusqu'à la table de Serpentard et s'assit à une extrémité qui était encore vide.

Drago plaça une omelette à la saucisse dans son assiette et commença à manger par gestes automatiques.

On a fait sortir Bellatrix Black d'Azkaban.

On a fait sortir Bellatrix Black d'Azkaban…~?

Drago ne savait pas quoi faire de cela, c'était aussi inattendu que l'extinction du Soleil — enfin, le Soleil s'éteindrait comme prévu dans six-milliards d'années, mais c'était aussi inattendu que l'extinction du soleil \emph{demain}. Père n'aurait pas fait ça, Dumbledore n'aurait pas fait ça, \emph{personne} n'aurait dû être \emph{capable} de le faire — qu'est-ce que ça voulait \emph{dire} — quelle \emph{utilité} Bellatrix pourrait-elle avoir pour quiconque après dix ans passés à Azkaban — même si elle redevenait forte, que pouvait-on bien faire d'une sorcière à la fois puissante, totalement maléfique, folle et fanatiquement dévouée à un Seigneur des Ténèbres qui n'était plus là~?

<<~Hé, dit Vincent depuis sa chaise, j'comprends pas, patron, pourquoi on a fait ça~?

--- \emph{On} ne l'a pas fait, abruti~! rétorqua Drago. Oh, par Merlin, si même \emph{toi} tu penses qu'on — ton père ne t'a jamais raconté d'histoires sur Bellatrix Black~? Elle a déjà torturé Père, elle a déjà torturé \emph{ton} père, elle a torturé tout le monde~; un jour le Seigneur des Ténèbres lui a dit de se lancer Doloris sur \emph{elle-même} et elle l'a \emph{fait}~! Elle n'a pas fait des trucs dingues pour inspirer la peur et l'obéissance dans la populace, elle a fait des trucs dingues parce qu'elle est dingue~! C'est une \emph{garce}, voilà ce qu'elle est~!

--- Ah, vraiment~?~>> dit une voix outrée venue de derrière lui.

Il ne leva pas les yeux. Gregory et Vincent couvriraient ses arrières.

<<~Je t'aurais cru heureux…

--- … d'entendre qu'un Mangemort a été libéré, Malfoy~!~>>

Amycus Carrow avait toujours été une de ces \emph{autres} personnes à problèmes, et Père avait un jour dit à Drago de se débrouiller pour ne jamais être seul dans la même pièce que lui…

Drago se retourna et envoya à Flora et à Hestia son Rictus Méprisant Numéro Trois, celui qui disait~: <<~J'appartiens à une maison Noble, pas vous, et oui, c'est important.~>> Drago dit, vaguement dans leur direction, ne daignant certainement pas s'adresser à \emph{eux} en particulier~: <<~Il y a Mangemort et Mangemort~>>, puis il revint à son assiette.

On put entendre deux personnes prendre la mouche et deux paires de chaussures déguerpir avec rage jusqu'à l'autre bout de la table Serpentard.

Quelques minutes plus tard, Millicent Bulstrode courut jusqu'à eux et dit, visiblement essoufflé~:

<<~M. Malfoy, vous avez entendu~?

--- Au sujet de Bellatrix Black~? dit Drago. Ouais…

--- Non, au sujet de Potter~!

--- Quoi~?

--- Potter s'est promené avec un \emph{phénix} sur l'épaule la nuit dernière et il avait l'air d'avoir été traîné à travers cinq kilomètres de boue. On dit que le phénix l'a emmené à Azkaban pour essayer d'arrêter Bellatrix, qu'il se sont battus en duel et qu'ils ont fait sauter la moitié de la forteresse~!

--- \emph{Quoi~?} dit Drago. Il est tout simplement \emph{impossible} que…~>>

Drago s'interrompit.

Il avait dit cela au sujet de Harry Potter un certain nombre de fois et il avait commencé à remarquer la tendance.

Millicent s'en fut raconter l'histoire à quelqu'un d'autre.

<<~Vous ne pensez pas \emph{vraiment} que — dit Gregory.

--- Franchement, je ne sais plus quoi penser~>>, dit Drago.

Quelques minutes plus tard, après que Theodore Nott se fut assis en face de lui et que William Rosier soit allé s'asseoir avec les jumeaux Carrow, Vincent lui donna un coup de coude et dit~: <<~Là.~>>

Harry Potter était entré dans la Grande Salle.

Drago ne le quitta pas des yeux.

Il ne pouvait voir ni inquiétude ni surprise ni indice de traumatisme sur le visage de Harry, il avait juste l'air…

C'était le même air distant et introspectif que Harry revêtait lorsqu'il essayait de trouver la réponse à une question que Drago ne pouvait même pas encore comprendre.

Drago se leva avec hâte du banc Serpentard, dit <<~restez en arrière~>> et s'avança à une vitesse honorable en direction de Harry.

Harry sembla remarquer son approche au moment où Drago frôla la table Serdaigle, et ce dernier -

- jeta un rapide regard à Harry -

- avant de continuer tout droit vers la sortie de la Grande Salle.

Une minute plus tard, Harry apparut au coin du petit renfoncement de pierre où Drago l'avait attendu~; même si cela n'avait peut-être pas trompé tout le monde, le déni serait plausible.

<<~\emph{Sourdinam}, dit Harry. Drago, qu'est-ce qui…~>>

Drago sortit l'enveloppe de sa robe.

<<~J'ai un message de Père pour toi.

--- \emph{Hein~?}~>> dit Harry. Il prit l'enveloppe des mains de Drago, la déchira d'une façon bien peu soignée, en sortit une feuille de parchemin qu'il déplia et -

Harry inspira subitement.

Puis il regarda Drago.

Puis il rabaissa les yeux vers le parchemin.

Il y eut un silence.

Harry dit~: <<~Lucius t'a-t-il demandé de lui rapporter ma réaction~?~>>

Drago ne répondit rien pendant un moment, soupesant les possibilités, puis il ouvrit la bouche -

<<~Je vois donc qu'il te l'a demandé~>>, continua Harry, et Drago se maudit. Il aurait dû le savoir, mais ça \emph{avait} été difficile de décider. <<~Que vas-tu lui dire~?

--- Que tu étais surpris, dit Drago.

--- Surpris, dit Harry d'un ton catégorique. Ouais. Bien. Dis-lui ça.

--- Qu'est-ce que c'\emph{est}~?~>> dit Drago. Puis, remarquant que Harry semblait hésiter~: <<~Si tu mijotes avec Père derrière mon dos -~>>

Et Harry, sans dire un mot, donna la feuille à Drago.

Elle disait~:

\emph{Je sais que c'était vous.}

<<~\emph{\shout{C'est quoi ce -}}

--- J'allais \emph{te} le demander, dit Harry. As-tu la \emph{moindre} idée de ce qui a pris à ton père~?~>>

Drago fixa Harry.

Puis il dit~:

<<~\emph{Est-ce que} tu l'as fait~?

--- Quoi~? dit Harry. Pour \emph{quelle} raison est-ce que je pourrais vouloir — \emph{comment} est-ce que je pourrais -

--- Harry, est-ce que tu l'as fait~?

--- Non~! dit Harry. Bien sûr que non~!~>>

Drago avait écouté avec attention et n'avait détecté ni hésitation ni vacillement.

Il hocha donc la tête et dit~:

<<~Je n'ai pas la moindre idée de ce que Père pense mais il est \emph{impossible}, il n'y a pas la \emph{moindre chance} que ce soit bon signe. Et, euh… les gens disent aussi…

--- Quoi, dit Harry d'une voix usée, que disent les gens, Drago~?

--- Est-ce qu'un phénix t'a \emph{vraiment} emmené à Azkaban pour que tu essaies d'empêcher Bellatrix Black de s'échapper -~>>

\latersection{Après-coup, Neville Londubat~:}

Harry venait juste de s'asseoir à la table Serdaigle pour la première fois de la journée avec l'espoir de manger un morceau en vitesse. Il savait qu'il devait s'isoler et réfléchir à certaines choses, mais il restait un minuscule fragment de la paix du phénix (même après sa rencontre avec Drago) à laquelle il voulait encore s'accrocher, comme un magnifique rêve dont il n'aurait pu rien se remémorer mis à part sa beauté~; et la partie de lui qui ne ressentait \emph{pas} cette paix attendait que toutes les enclumes aient fini de lui tomber de la tête dans l'espoir de pouvoir gérer tous les désastres d'un coup une fois qu'il serait seul et prêt à réfléchir.

La main de Harry saisit une fourchette, souleva une bouchée de purée, l'approcha de sa bouche -

Et on entendit un cri.

On entendait parfois des cris lorsque les gens apprenaient la nouvelle, mais les oreilles de Harry avait \emph{reconnu} celui là -

Il fut debout en un éclair et se dirigea vers la table Poufsouffle, une nausée naissante dans l'estomac. C'était l'un des paramètres auxquels il n'avait pas pensé lorsqu'il avait décidé de commettre le crime car le professeur Quirrell n'avait pas prévu que quiconque serait au courant~; et maintenant, plus tard, Harry n'avait simplement — il n'y avait pas \emph{pensé} -

\emph{Ça}, dit Poufsouffle avec amertume, \emph{ça aussi, c'est de ta faute.}

Mais lorsque Harry parvint à la table, Neville était déjà assis et mangeait des saucisses grillées à la sauce de Figuevive.

Même si les mains du garçon de Poufsouffle tremblaient, il coupait sa nourriture et la mangeait sans rien faire tomber.

<<~Bonjour, général~>>, dit Neville d'une voix qui ne vacillait que légèrement. <<~T'es-tu battu en duel contre Bellatrix Black la nuit dernière~?

--- Non~>>, dit Harry. Quelque chose faisait vaciller sa voix à lui aussi.

<<~Ça m'aurait étonné~>>, dit Neville. On entendit le bruit d'un couteau qui traversait une saucisse et raclait contre une assiette. <<~Je vais la pourchasser et la tuer. Je peux compter sur ton aide~?~>>

On put entendre des hoquets d'inquiétude venus de la masse de Poufsouffle qui s'était assemblée autour de Neville.

<<~Si elle s'en prend à toi,~>> dit Harry d'une voix rauque, \emph{s'il s'avère que j'ai commis une terrible erreur, que tout n'était qu'un mensonge}, <<~je te défendrai de ma vie,~>> \emph{je ne laisserai pas ce que j'ai fait te mettre en danger, quoi qu'il en coûte}, <<~mais je ne t'aiderai pas à la pourchasser. On n'aide pas ses amis à se suicider, Neville.~>>

La fourchette de Neville s'interrompit sur son trajet vers sa bouche.

Puis il mit le morceau de nourriture dans sa bouche, mâcha de nouveau.

Avala.

Et dit~:

<<~Je ne voulais pas dire \emph{tout de suite}, je voulais dire après Poudlard.

--- Neville~>>, dit Harry en contrôlant sa voix avec une grande précaution, <<~je pense que, même après Poudlard, il se peut que ça soit toujours une \emph{idée stupide}. Il y a sûrement des Aurors avec bien plus d'expérience qui sont sur sa trace -~>> \emph{oh, attends, ça ne va pas -}

<<~Écoute-le~!~>> dit Ernie Macmillan, puis une Poufsouffle plus âgée qui se tenait non loin de Neville renchérit~: <<~Nevy, s'il te plaît, réfléchis-y, il a raison~!~>>

Neville se leva.

Et dit~: <<~Ne me suivez pas, s'il vous plaît.~>>

Puis il s'éloigna d'eux~; Harry et Ernie tendirent involontairement la main vers lui et d'autres Poufsouffle firent de même.

Neville s'assit à la table Gryffondor et ils purent entendre (en faisant un effort) sa voix lointaine qui disait~: <<~Je vais la pourchasser et la tuer après Poudlard, quelqu'un veut m'aider~?~>> et au moins trois voix dirent <<~Oui~>> puis Ron Weasley dit d'une voix forte~: <<~Faites la queue, vous autres, j'ai eu une chouette de Maman qui m'a dit d'annoncer à tout le monde qu'elle est prems~>>, et quelqu'un dit~: <<~\emph{Molly Weasley} contre \emph{Bellatrix Black}~? Mais elle, à qui veut-elle faire croire ça -~>> et Ron tendit la main vers une assiette, saisit un muffin -

Quelqu'un toucha l'épaule de Harry et il se retourna pour découvrir une fille au visage peu familier, à la robe bordée de vert et plus âgée que lui qui lui tendit une enveloppe en parchemin avant de s'éloigner à grands pas.

Il fixa l'enveloppe pendant un moment puis commença à se diriger vers le mur le plus proche. Ce n'était pas très intime mais ça suffirait, et il ne voulait pas donner l'impression qu'il avait tant que ça à cacher.

Ça avait été une livraison du Système Serpentard, qui était utilisé par ceux qui ne voulaient pas que quiconque sache qu'ils avaient communiqué. L'expéditeur donnait une enveloppe et une dizaine de Noises à quelqu'un qui avait la réputation d'être un messager respectable~; cette personne récupérait cinq Noises, passait l'enveloppe à un autre messager avec les cinq Noises restantes, et ce second messager ouvrait l'enveloppe pour y trouver une autre enveloppe sur laquelle était écrit le nom de la personne à qui elle devait être délivrée. Ainsi, aucune des deux personnes impliquées dans la transmission du message ne connaissaient à la fois l'expéditeur \emph{et} le destinataire, si bien que personne d'autre ne savait que les deux camps avaient été en contact…

Lorsqu'il eut atteint le mur, il mit l'enveloppe dans sa robe, l'ouvrit à l'abri de celle-ci et jeta un coup d'œil discret au parchemin qu'il en sortit.

Le parchemin disait~:

\emph{Salle de classe à gauche de Métamorphose, 8 heures du matin.}

\emph{- LL.}

Harry le regarda, essayant de se souvenir s'il connaissait quelqu'un ayant pour initiales LL.

Son esprit chercha…

Chercha…

Trouva -

<<~La fille du \emph{Chicaneur}~?~>> chuchota Harry d'un ton incrédule, puis il ferma la bouche. Elle n'avait que dix ans, elle n'aurait même pas dû être à Poudlard~!

\latersection{Après-coup, Lesath Lestrange~:}

À huit heures du matin, Harry attendait dans la salle de classe vide voisine de celle de Métamorphose après être parvenu à ingérer un peu de nourriture en préparation du prochain désastre~: Luna Lovegood…

La porte de la salle de classe s'ouvrit, Harry le vit, et il se donna un \emph{énorme} coup de pied mental.

Encore une de ces choses auxquelles il n'avait pas pensées, une de ces choses auxquelles il \emph{aurait vraiment dû penser}.

La robe bordée de vert du garçon plus âgé était de travers et maculée de taches rouges qui ressemblaient vraiment à de petits points de sang frais tandis qu'un coin de sa bouche avait l'air d'avoir été coupé puis soigné par un \emph{Episkey} ou un autre de ces charmes médicaux mineur qui ne réparaient pas entièrement les dommages.

Le visage de Lesath Lestrange était couvert de larmes, certaines fraîches, d'autres à moitié séchées, et ses yeux humides en annonçaient d'autres à venir. <<~\emph{Sourdinam}~>> dit le garçon plus âgé, puis <<~\emph{Hominum Revelio}~>>, puis d'autres, tandis que Harry réfléchissait aussi vite qu'il le pouvait sans rien trouver.

Puis Lesath abaissa sa baguette, la rangea dans sa robe, et, lentement cette fois, cérémonieusement, le garçon plus âgé que Harry tomba à genoux sur le sol poussiéreux de la salle de classe.

Il abaissa sa tête jusqu'à ce que son front touche la poussière~; Harry aurait bien voulu parler, mais il était sans voix.

Lesath Lestrange dit, d'une voix brisée~:

<<~Ma vie comme ma mort vous appartiennent, seigneur.

--- Je~>>, dit Harry, il avait une énorme boule dans la gorge, il n'arrivait pas à parler, <<~Je…~>> \emph{n'ai rien à voir là-dedans}, il aurait dû le dire, il aurait dû le dire \emph{tout de suite}, mais après tout le Harry innocent aurait lui aussi eu du mal à parler -

<<~Merci, chuchota Lesath, merci, seigneur, oh, merci~>>, le bruit d'un sanglot étranglé émana du garçon agenouillé dont Harry ne pouvait voir rien d'autre que les cheveux. <<~Je suis un imbécile, seigneur, un bâtard ingrat qui ne mérite pas de vous servir, je ne saurais m'abaisser assez devant vous car je — je vous ai crié dessus après que vous m'eûtes aidé, parce que je pensais que vous me rejetiez, et ce n'est que ce matin que j'ai compris à quel point j'avais été un imbécile de vous poser la question face à Londubat -

--- Je n'ai rien à voir là-dedans,~>> dit Harry.

(C'était encore très difficile de mentir directement comme cela)

Lesath releva lentement la tête et regarda Harry.

<<~Je comprends, seigneur,~>> dit le garçon plus âgé d'une voix qui vacillait un peu, <<~vous n'avez pas confiance en moi, et il est vrai que je me suis montré idiot… je voulais seulement vous dire que je ne suis pas un ingrat, que je sais que ça a dû être difficile de ne sauver qu'une seule personne, qu'ils sont au courant maintenant, que vous ne pouvez pas — sauver Père — mais je ne suis pas ingrat, je ne serai plus jamais ingrat. Si vous pensez un jour pouvoir faire usage du serviteur indigne que je suis, appelez-moi, où que je sois, et je répondrai, seigneur -

--- Je n'ai été impliqué en aucune façon~>>

(Mais ça devenait chaque fois plus facile)

Lesath regarda de nouveau Harry et dit, incertain~:

<<~Suis-je excusé de votre présence, seigneur…~?

--- Je ne suis pas ton seigneur.~>>

Lesath dit~: <<~Oui, seigneur, je comprend~>>, et il se releva, se tint bien droit, s'inclina bien bas, puis s'éloigna de Harry jusqu'à atteindre la porte de la salle.

Lorsque sa main toucha la poignée, il s'arrêta.

Harry ne put voir le visage qui posait la question~: <<~L'avez-vous envoyé à quelqu'un qui va prendre soin d'elle~?~>>

Et Harry dit, d'une voix parfaitement maîtrisée~:

<<~Arrête, s'il te plaît. Je n'ai rien à voir là-dedans.

--- Oui seigneur, pardonnez-moi seigneur~>>, dit la voix de Lesath~; et le Serpentard ouvrit la porte, sortit et la referma derrière lui. Son pas accéléra à mesure qu'il s'éloignait, mais il ne fut pas assez rapide pour empêcher Harry de l'entendre commencer à sangloter.

\emph{Est-ce que je pleurerais~?} se demanda Harry. \emph{Si je ne savais rien, si j'étais innocent, est-ce que je serais en train de pleurer~?}

Il l'ignorait et se contenta donc de continuer à fixer la porte du regard.

Et une partie incroyablement grossière de sa personne pensa~: \emph{Youpi, on a bouclé une quête et on a gagné un sous-fifre -}

\emph{Tais-toi. Si tu veux jamais avoir voix au chapitre… tais-toi.}

\latersection{Après-coup, Amelia Bones~:}

<<~J'en conclus que sa vie n'est pas en danger~>>, dit Amelia.

Le guérisseur, un vieil homme au regard sévère vêtu d'une robe blanche (il était né-Moldu et rendait ainsi honneur à d'étranges traditions Moldues au sujet desquelles Amelia ne l'avait jamais interrogées même si en son for intérieur elle trouvait que cela lui donnait une apparence bien trop fantomatique) secoua la tête et dit <<~Certainement pas.~>>

Amelia regarda la forme humaine inconsciente qui reposait sur le lit du guérisseur, la chair brûlée et flétrie, la fine couverture qui par souci de pudeur avait recouvert celui-ci ôtée sur son ordre.

Il retrouverait peut-être toutes ses fonctions.

Peut-être pas.

Le guérisseur avait dit qu'il était trop tôt pour se prononcer.

Puis Amelia regarda l'autre sorcière dans la pièce, la détective.

<<~Et vous dites, dit Amelia, que la matière inflammable avait été métamorphosée à partir d'\emph{eau}, vraisemblablement sous forme de glace.~>>

La détective hocha la tête et dit d'un ton perplexe~:

<<~Ça aurait pu être bien pire s'il avaient -

--- Que c'est \emph{gentil} de leur part~>>, cracha-t-elle, puis elle appuya une main usée sur son front. Non… non, ça \emph{avait} été une marque de bonté. À cette étape de l'évasion, il était inutile d'essayer de tromper qui que ce soit. L'individu qui avait fait ça \emph{avait} essayé de mitiger les dommages — et il avait pensé au risque que les Aurors respirent la fumée, pas à l'avantage qu'il aurait à les brûler avec le feu. Si cette personne avait encore été là, la fouliée aurait sans aucun doute été maniée avec plus de clémence.

Mais Bellatrix Black avait chevauché la fouliée seule, tous les Aurors présents étaient d'accord là-dessus, leurs sortilèges de Désillusion étaient actifs et on n'avait vu qu'une femme juchée sur cette fouliée même si cette dernière avait été équipée de deux paires d'étriers.

Une personne innocente et pleine de bonté capable de lancer le Patronus avait été dupée et convaincue de sauver Bellatrix Black.

Une personne innocente s'était battue contre Bahry Une-Main et avait précautionneusement maîtrisé un Auror sans lui infliger de dommages notables.

Une personne innocente avait métamorphosé le carburant d'un engin Moldu sur lequel elle et Bellatrix Black devaient monter afin de s'échapper d'Azkaban en utilisant de l'eau gelée pour le bien des Aurors d'Amelia.

Puis Bellatrix Black avait considéré que cette personne n'était plus utile.

On se serait attendu à ce que toute personne capable de maîtriser Bahry Une-Main soit capable de prévoir ça. Mais après tout, personne ne se serait attendu à voir quelqu'un capable de lancer le Patronus essayer de sauver Bellatrix Black en premier lieu.

Amelia se passa une main sur les yeux et les ferma dans un moment de deuil silencieux. \emph{Je me demande qui c'était et comment Vous-Savez-Qui l'a manipulé… quelle histoire on a} bien pu \emph{lui raconter…}

Ce n'est que quelques instants plus tard qu'elle se rendit compte que cette pensée signifiait qu'elle commençait vraiment à y croire. Peut-être parce que, aussi difficile que ce soit de croire Dumbledore, il devenait encore plus difficile de ne \emph{pas} reconnaître là la marque de cette intelligence froide et sombre.

\latersection{Après-coup, Albus Dumbledore~:}

Il n'arriva peut-être que cinquante-sept secondes avant la fin et il lui fallut peut-être quatre tours de son Retourneur de Temps pour y parvenir, mais Albus Dumbledore finit par prendre son petit déjeuner.

<<~Directeur~?~>> couina la voix polie du professeur Filius Flitwick tandis que le vieux sorcier le dépassait en direction de sa chaise. <<~M. Potter a laissé un message pour vous.~>>

Le vieux sorcier s'arrêta et regarda le professeur de sortilèges d'un regard interrogateur.

<<~M. Potter a dit qu'après son réveil, il s'est rendu compte à quel point ce qu'il vous a dit après que le phénix eut crié était injuste. Il a ajouté qu'il ne parlait de rien d'autre, qu'il s'excusait seulement pour cette partie précise.~>>

Le vieux sorcier continua de regarder son professeur de sortilèges et se tint coi.

<<~Directeur~? couina Filius.

--- Dis-lui que je le remercie, dit Albus Dumbledore, mais qu'il est plus sage d'écouter les phénix que d'écouter les sages vieux sorciers~>>, et il s'assit trois secondes avant que toute la nourriture ne disparaisse.

\latersection{Après-coup, Professeur Quirrell~:}

<<~Non, dit madame Pomfresh à l'enfant d'un ton vif, vous ne pourrez \emph{pas} le voir~! Vous ne pouvez \emph{pas} le tourmenter~! Vous ne pourrez pas lui poser \emph{une seule petite question~!} Il va rester \emph{au lit} et ne \emph{rien faire} pendant au moins \emph{trois jours}~!~>>

\latersection{Après-coup, Minerva McGonagall~:}

Elle se dirigeait vers l'infirmerie lorsqu'elle croisa Harry qui en sortait.

Il la regarda sans colère.

Son regard était triste.

Il n'y avait pas grand chose à y voir.

C'était comme… comme s'il la regardait juste assez longtemps pour rendre clair le fait qu'il ne l'évitait \emph{pas} exprès.

Puis il détourna les yeux avant qu'elle ne trouve un regard à lui rendre~; comme s'il avait voulu lui épargner cela aussi.

Il ne dit rien, alors qu'il la dépassait.

Elle non plus.

Qu'auraient-il bien pu avoir à se dire~?

\latersection{Après-coup, Fred et George Weasley~:}

Croyez-le ou non mais ils glapirent lorsqu'ils tombèrent sur Dumbledore au détour d'un couloir.

Ce n'était pas parce qu'il était apparu de nulle part et qu'il les regardait d'un œil sévère. Il faisait toujours \emph{ça}.

Mais le sorcier était habillé d’une robe noire formelle qui lui donnait l'air \emph{très} ancien et \emph{très} puissant et son regard était PERÇANT.

<<~Fred et George Weasley~! dit Dumbledore d'une Voix Impérieuse.

--- Oui, monsieur le directeur~!~>> répondirent-ils en claquant les talons et en lui offrant un salut militaire soigné tel qu'ils en avaient vu sur de vieilles images.

<<~Écoutez-moi bien~! Vous êtes des amis de Harry Potter, n'est-ce pas~?

--- Oui, monsieur le directeur~!

--- Harry Potter est en danger. Il ne \emph{doit pas} sortir de l'enceinte de Poudlard. Écoutez-moi, fils de Weasley, je vous en prie~: vous savez que je suis aussi Gryffondor que vous, que je sais moi aussi qu'il existe des règles au-dessus les règles. Mais ceci, Fred et George, ceci est de l'importance la plus terrible, il ne doit pas y avoir d'exception cette fois-ci, grande ou petite~! Si vous aidez Harry à quitter Poudlard, il \emph{mourra} peut-être~! S'il vous envoie en mission, allez-y, s'il vous demande de lui rapporter quelque chose, aidez-le, mais s'il vous demande de lui faire clandestinement quitter Poudlard, vous \emph{devez refuser~!} Comprenez-vous~?

--- Oui, monsieur le directeur~!~>> Dirent-ils sans vraiment réfléchir, puis ils échangèrent un regard incertain -

Les yeux bleu clair du directeur étaient résolument braqués sur eux. <<~Non. Pas sans réfléchir. Si Harry vous demande de le faire sortir, vous devez refuser, s'il vous demande de lui indiquer le chemin, vous devez refuser. Je ne vous demanderai pas de m'en faire part, car je sais que vous ne le ferez jamais. Mais si son projet est d'une importance telle qu'il doit sortir, alors priez-le de ma part de venir \emph{me} voir et \emph{je} garderai ses pas. Fred, George, je suis navré de porter ainsi atteinte à votre amitié, mais il s'agit de sa \emph{vie}.~>>

Les deux jumeaux se regardèrent pendant un moment, sans communiquer mais en ayant les mêmes pensées au même moment.

Ils se retournèrent vers Dumbledore.

Et dirent, alors qu'un frisson les traversait~:

<<~Bellatrix Black.

--- Vous pouvez présumer sans risque de vous tromper, dit le directeur, que c'est au moins aussi grave que ça.

--- D'accord -

--- — compris.~>>

\latersection{Après-coup, Alastor Maugrey et Severus Rogue~:}

Lorsqu'Alastor Maugrey avait perdu son œil, il avait commandité les services d'un Serdaigle des plus érudits, Samuel H. Lyall, dont Maugrey se méfiait moins que la moyenne car il avait décidé de ne pas le dénoncer comme loup-garou non déclaré~; et il avait engagé Lyall pour que ce dernier compile une liste de tous les yeux magiques connus et de tout indice existant quant à leur emplacement.

Lorsqu'il eut la liste entre les mains, il ne s'était pas fatigué à en lire la majeure partie parce qu'au sommet de celle-ci s'était trouvé l'Œil de Vance, venu d'une ère antérieure à Poudlard et alors en la possession d'un puissant mage noir à la tête d'un petit trou à rat oublié qui n'était ni en Angleterre ni dans un pays où il y aurait à se soucier de règles idiotes.

C'est ainsi qu'Alastor Maugrey perdit son pied gauche, acquit l'Œil de Vance, et que les âmes opprimées d'Urulat furent libérées pendant une période de deux semaines au terme de laquelle un autre mage noir combla le vide de pouvoir qui avait ait formé.

Il songea ensuite à aller s'emparer du Pied Gauche de Vance, mais décida de ne pas le faire lorsqu'il comprit que c'était \emph{exactement ce à quoi les autres s'attendaient.}

Maugrey Fol-Œil pivotait lentement et sans cesse, surveillant le cimetière de Little Hangleton. Le lieu aurait dû être plus morbide, mais sous la lumière du jour il n'avait l'air d'être qu'un parc herbeux jonché de tombes ordinaires encerclé de chaînes d'un métal torsadé, fragile et facile à grimper que les Moldus utilisaient en lieu et place d'une enceinte (Maugrey ne comprenait pas ce que les Moldus pensaient au sujet de ces chaînes~: \emph{prétendaient-ils} qu'elles étaient des murailles~? Il avait décidé de ne pas demander si les criminels Moldus prétendaient eux aussi qu'il y avait une enceinte).

Maugrey n'avait pas \emph{vraiment} besoin de pivoter pour surveiller le cimetière.

L'Œil de Vance voyait tout le globe terrestre dans toutes les directions, et ce quelle que soit son orientation.

Mais Maugrey n'avait aucune raison particulière de laisser un ancien Mangemort tel que Severus Rogue obtenir une telle information.

Parfois, les gens disaient que Maugrey était “paranoïaque”.

Maugrey leur avait toujours répondu de survivre cent ans à combattre des mages noirs et de revenir lui en parler après.

Maugrey Fol-Œil avait un jour calculé combien de temps il lui avait fallu pour obtenir un niveau de prudence qu'il considérait aujourd'hui comme acceptable~; il avait soupesé l'expérience qui lui avait été nécessaire pour être \emph{bon} plutôt que \emph{chanceux} — et il avait commencé à soupçonner que la plupart des gens mouraient avant d'en arriver là. Il avait un jour fait part de cette pensée à Lyall qui avait fait quelques diagrammes et quelques calculs avant de répondre que le chasseur de mages noirs typique mourrait en moyenne huit fois et demie avant de devenir “paranoïaque”. Voilà qui expliquait beaucoup, si Lyall ne mentait pas.

Hier, Albus Dumbledore avait dit à Maugrey Fol-Œil que le Seigneur des Ténèbres avait fait usage de ses arts noirs indicibles pour survivre à la mort de son corps et qu'il attendait maintenant, éveillé et en retrait, cherchant à retrouver ses pouvoirs et à recommencer la guerre des sorciers.

Quelqu'un d'autre aurait pu être incrédule.

<<~Je n'arrive pas à croire que tu ne m'aies jamais parlé de cette histoire de résurrection, dit Maugrey Fol-Œil d'un ton particulièrement acerbe. S'que tu t'rends compte du temps que ça va me prendre de visiter les tombes de tous les ancêtres de tous les mages noirs que j'ai tué qui auraient pu être assez malins pour faire un horcruxe~? Ne me dis pas que c'est \emph{aujourd'hui} que tu t'occupes de celle-ci~?

--- Je lui ré-administre une dose chaque année~>>, répondit Severus Rogue d'un ton calme en débouchant la troisième fiole d'une série dont l'homme avait \emph{prétendu} qu'elle comprendrait dix-sept bouteilles et en commençant à agiter sa baguette au-dessus de celle-ci. <<~Les autres tombes ancestrales que nous avons pu localiser ont été empoisonnées uniquement avec des substances durables car certains d'entre nous ont moins de temps libre que vous.~>>

Maugrey regarda le fluide qui s'échappait de la fiole le long d'une spirale et qui disparaissait dans la terre pour se rendre là où la moelle des os s'était un jour trouvée.

<<~Mais tu penses que ça mérite l'effort de tendre ce piège au lieu de simplement détruire les os.

--- S'il considérait que celui-ci était bloqué, il \emph{aurait} d'autres moyens de revenir à la vie, dit Rogue d'un ton sec en débouchonnant une quatrième bouteille. Et avant que vous ne me le demandiez, ça doit être la tombe originelle, celle du premier enterrement, dont les os doivent être enlevés pendant le rituel mais pas avant. Il ne peut donc pas les avoir récupérés plus tôt~; et il n'y a pas non plus d'intérêt à substituer à ce squelette celui d'un ancêtre plus faible. Il remarquerait que les os ont perdu tout leur pouvoir.

--- Qui d'autre est au courant de ce piège~? demanda Maugrey.

--- Vous. Moi. Le directeur. Personne d'autre.~>>

Maugrey renifla.

<<~Bah. Albus a-t-il parlé du rituel de résurrection à Amelia, à Bartemius et à cette femme, McGonagall~?

--- Oui -

--- Si Voldy découvre que Albus est au courant du rituel de résurrection et qu'il le \emph{leur} a dit, il comprendra qu'il me l'a dit à \emph{moi}, et Voldy \emph{sait} que je penserai à ça.~>> Il secoua sa tête d'un air dégoûté. <<~Quels autres moyens Voldy a-t-il de revenir à la vie~?~>>

La main de Rogue s'arrêta sur la cinquième bouteille (elle était évidemment sous l'effet d'un sortilège de Désillusion, comme tout le reste de la procédure, mais ce genre de choses n'avait strictement aucune importance pour Maugrey et ne faisait que vous marquer dans le champ de vision de son Œil comme quelqu'un qui essayait de se cacher), et l'ancien Mangemort répondit~:

<<~Vous n'avez pas besoin de le savoir.

--- Tu apprends, fiston, dit Maugrey d'un ton moyennement approbateur. Qu'est-ce qu'il y a dans les bouteilles~?~>>

Rogue ouvrit la cinquième bouteille puis fit un geste de baguette afin que la substance commence à couler vers la tombe avant de dire~:

<<~Celle-ci~? Un narcotique Moldu appelé LSD. Une conversation que j'ai eue hier m'a mit le monde Moldu en tête et le LSD m'a semblé être l'option la plus intéressante, si bien que je me suis dépêché d'en obtenir. Si cette substance est incorporée à la potion de résurrection, je soupçonne que ses effets puissent être permanents.

--- Qu'est-ce que ça fait~? dit Maugrey.

--- On dit que ses effets sont impossibles à décrire à quelqu'un qui ne l'a pas utilisé, répondit Rogue d'une voix traînante, et je ne m'y suis pas essayé.~>>

Maugrey approuva d'un hochement de tête alors que Rogue ouvrait la sixième fiole.

<<~Et celle-ci~?

--- Philtre d'amour.

--- \emph{Philtre d'amour~?} dit Maugrey.

--- Pas du genre habituel. Elle est censée déclenché un lien mutuel avec une femelle Veela atrocement gentille du nom de Verdandi dont le directeur espère que, s'ils s'aimaient vraiment l'un l'autre, elle pourrait bien le racheter, même lui.

--- \emph{Arr~!}  dit Maugrey. Ce satané idiot, quel sentimental -

--- Tout à fait~>>, dit Severus Rogue d'un ton calme, concentré sur son travail.

<<~Dis moi au moins que tu as un peu de venin de Malagrif là-dedans.

--- Seconde fiole.

--- Poudre d'iocane.

--- Soit la quatorzième soit la quinzième.

--- Stupéfaction de Bahl~>>, dit Maugrey, nommant ainsi un narcotique particulièrement addictif doté d'effets secondaires intéressants sur les personnes à tendance Serpentard~; Maugrey avait un jour vu un mage noir extrêmement dépendant déployer des efforts grotesques pour voir une victime poser ses mains sur un Portoloin très précis au lieu de juste faire en sorte que quelqu'un lui jette une Noise piégée lors de sa prochaine visite en ville~; et après avoir fait tout ce travail, la personne intoxiquée avait fait l'effort \emph{supplémentaire} de mettre en place un \emph{second Portus} sur le \emph{même Portoloin} qui avait, lors d'un deuxième contact, ramené la victime en sécurité. Encore aujourd'hui, même en prenant la drogue en compte, Maugrey n'arrivait pas à imaginer ce qui avait pu traverser l'esprit de l'homme au moment où il avait lancé le second Portus.

<<~Dixième fiole, dit Rogue.

--- Venin de Basilic, proposa Maugrey.

--- \emph{Quoi~?} cracha Rogue. Le venin de serpent est un composant actif de la potion de résurrection~! Sans parler du fait qu'il dissoudrait les os et toutes les autres substances~! Et où est-ce qu'\emph{on} mettrait la main sur -

--- Calme-toi fiston, je vérifiais juste si on pouvait te faire confiance.~>>

Maugrey Fol-Œil continua sa rotation (secrètement inutile) et surveilla le cimetière tandis que le maîtres des potions continuait de verser.

<<~Attends, dit soudain Maugrey. Comment est-ce que tu sais que \emph{c'est} vraiment là que -

--- Parce qu'il y a marqué “Tom Jedusor” sur la pierre tombale facile à déplacer, dit sèchement Rogue. Et je viens de gagner dix Mornilles du directeur qui a parié que vous y penseriez avant la cinquième bouteille. Au temps pour votre vigilance constante.~>>

Il y eut un silence.

<<~Combien de temps Albus a-t-il mit à comp-

--- Trois ans après que nous ayons appris l'existence du rituel, répondit Rogue d'un ton qui n'était pas tout à fait de son sardonique habituel. Rétrospectivement, nous aurions dû vous consulter plus tôt.~>>

Rogue déboucha la neuvième bouteille.

<<~Nous avons aussi empoisonné toutes les autres tombes, avec des substances durables, remarqua l'ancien Mangemort. Il \emph{est} possible que nous soyons dans le bon cimetière. Il n'avait peut-être pas prévu les choses autant à l'avance lorsqu'il était en train de massacrer sa famille et il pourrait ne pas avoir déplacé la tombe -

--- Son véritable emplacement ne ressemble plus à un cimetière, dit catégoriquement Maugrey. Il a déplacé toutes les \emph{autres} tombes et il a lancé \emph{Oubliettes} aux Moldus. Même Bellatrix Black n'en saurait rien jusqu'à quelques instants avant le début du rituel. À part lui, \emph{personne} ne connaît son véritable emplacement.~>>

Ils poursuivirent leur futile ouvrage.

\latersection{Après-coup, Blaise Zabini~:}

La salle commune de Serpentard aurait pu être très précisément décrite comme une zone remilitarisée~; au moment où vous auriez franchi le trou du portrait vous auriez pu voir que la moitié gauche de la pièce n'Adressait Certainement Plus La Parole à la moitié droite, et vice versa. Il était éminemment clair et il n'y avait nul besoin d'expliquer qu'on n'avait \emph{pas} la possibilité de ne \emph{pas choisir son camp.}

À une table au centre de la pièce, Blaise Zabini était assis, seul, un sourire en coin, penché sur ses devoirs. Il avait une réputation à présent, et il comptait bien la garder.



\latersection{Après-coup, Daphné Greengrass et Tracey Davis~:}

<<~T'as fait quelque chose d'intéressant aujourd'hui~? dit Tracey.

--- Nan, répondit Daphné.~>>

\latersection{Après-coup: Harry Potter~:}

Si vous montiez assez haut dans Poudlard, vous finissiez par ne plus voir grand monde, seulement des couloirs, des fenêtres, des escaliers, un portrait ici où là, parfois une curiosité comme une statue en bronze ou une petite créature au poil touffu semblable à un petit enfant et munie d'une étrange lance à pointe plate…

Si vous montiez assez haut dans Poudlard, vous finissiez par ne plus voir grand monde, ce qui convenait très bien à Harry.

Il y avait probablement des endroits bien pires où se retrouver emprisonné. En fait, il aurait probablement été difficile d'imaginer une \emph{meilleure} prison qu'un ancien château dont la structure fractale et éternellement changeante signifiait qu'on ne pouvait jamais être à court de lieux à explorer, un château rempli de gens et de livres intéressants, rempli de connaissances incroyablement importantes et ignorées de la science Moldue.

Si on n'avait pas dit à Harry qu'il ne \emph{pouvait pas} partir il aurait probablement \emph{sauté} sur une opportunité de passer plus de temps à Poudlard~; il aurait intrigué, il aurait rusé pour y parvenir. Poudlard était littéralement \emph{optimal}, peut-être pas entre toutes les possibilités concevables, mais au moins sur Terre c'était L'Endroit le Plus Amusant.

Comment le château et son enceinte avaient-ils pu paraître si rétrécis, si confinant, comment le reste du monde avait-il pu sembler tellement plus intéressant, tellement plus important à l'instant où on avait dit à Harry qu'il n'avait pas le droit de partir~? Il y était depuis des \emph{mois} et il ne s'était \emph{jamais} senti claustrophobe jusqu'à maintenant.

\emph{Tu} connais \emph{les recherches qui ont été faites sur le sujet}, remarqua une partie de lui, \emph{c'est juste un effet de rareté standard, comme dans cette affaire où dès qu'un département a rendu illégal les détergents au phosphate, des gens qui n'en avaient jamais rien eu à faire ont voyagé jusqu'au département voisin afin d'en acheter des quantités énormes~; et des sondages ont montré que ces gens jugeaient ce type de détergent plus doux, plus efficace, et même plus facile à verser… et si on donne le choix à un enfant de deux ans entre un jouet accessible et un autre protégé par une barrière qu'il peut contourner, il ignorera le jouet directement accessible et prendra celui derrière la barrière… les vendeurs savent qu'ils peuvent vendre quelque chose juste en disant au client que ça n'est peut-être pas disponible… tout ça c'est dans le livre Cialdini}, Influence et Manipulation\emph{, tout ce que tu ressens maintenant, l'impression que l'herbe est toujours plus verte du côté défendu.}

Si on n'avait pas dit à Harry qu'il ne pouvait pas partir, il aurait probablement \emph{bondi} sur l'opportunité de rester à Poudlard pendant l'été…

… mais pas pendant toute sa vie.

En fait, c'était ça le problème.

Savait-on si s'il y avait \emph{encore} un Seigneur des Ténèbres à vaincre~?

Savait-on si Celui-Dont-On-Ne-Doit-Pas-Prononcer-Le-Nom existait toujours, hors de l'imagination d'un vieux sorcier qui-ne-faisait-pas-forcément-semblant-d'être-fou~?

Le corps de Voldemort avait été découvert brûlé jusqu'à la moelle et les âmes ne pouvaient pas \emph{vraiment} exister. Comment Voldemort aurait-il pu être encore en vie~? Comment Dumbledore \emph{savait}-il qu'il était encore en vie~?

Mais s'il n'y avait pas de Seigneur des Ténèbres Harry ne pourrait pas le vaincre et il serait coincé à Poudlard pour toujours.

… peut-être aurait-il le droit de s'enfuir après avoir fini sa septième année, c'est-à-dire dans six ans, quatre mois et trois semaines. Ce n'était pas \emph{si} long que ça à l'échelle humaine mais ça \emph{semblait} être juste assez long pour que des protons aient le temps de se désintégrer.

Sauf que ça n'était pas le \emph{seul} problème.

Ce n'était pas \emph{seulement} que la liberté de Harry était en jeu.

Le directeur de Poudlard, le Manitou Suprême de la Confédération Internationale des Sorciers, le président sorcier du Magenmagot sonnait tranquillement l'alarme.

Une \emph{fausse} alarme.

Une fausse alarme que \emph{Harry} avait déclenché.

\emph{Dis-moi}, dit la partie de lui qui affûtait ses compétences, \emph{est-ce que tu n'as jamais remarqué comment toutes les professions ont chacune leur voie vers l'excellence, comment un excellent professeur n'est pas la même chose qu'un excellent plombier mais qu'ils ont tous deux en commun certaines méthodes leur permettant de ne pas être stupide, et que l'une des plus importantes de ces techniques est de faire face à ses petites erreurs avant qu'elles ne deviennent de GROSSES erreurs~?}

… même si ça semblait déjà mériter le qualificatif de GROSSE erreur, pour être franc…

\emph{L'idée étant}, dit son surveillant interne, \emph{que ça devient pire de minute en minute, au sens propre. Les espions transforment les gens en traîtres en leur faisant commettre un petit péché, puis ils utilisent ce petit péché pour les faire chanter et leur faire commettre un péché plus gros, puis ils utilisent CE péché pour leur faire faire des choses encore pires, et à ce stade il les tient par leur âme.}

\emph{N'as tu jamais pensé au fait que la personne que l'on fait chanter, si elle pouvait voir la voie qui l'attendait, déciderait simplement de recevoir le premier coup, de révéler son premier péché~? N'as-tu jamais décidé que c'est ce que tu ferais si quelqu'un essayait un jour de te faire chanter et d'obtenir de toi que tu fasses quelque chose de grave pour masquer quelque chose de bénin~? Vois-tu la similarité qui se présence ici, Harry James Potter-Evans-Verres~?}

Sauf que ce n'était pas bénin, ça ne l'était déjà plus, il y aurait beaucoup de gens très puissants qui seraient très en colère contre lui, pas seulement pour la fausse alarme mais pour avoir \emph{libéré Bellatrix Black d'Azkaban}, et si le Seigneur des Ténèbres existait \emph{vraiment} et s'en prenait à lui plus tard, alors la guerre était peut-être déjà perdue -

\emph{Tu ne penses pas qu'ils seront impressionné par ton honnêteté, par ta rationalité, par ta grande capacité de prévoyance démontrée par le fait que tu arrêtes tout ceci avant que cela n'aille plus loin~?}

À vrai dire, non, et après un moment de réflexion la partie de lui à laquelle il parlait dut admettre que c'était exagérément optimiste.

Ses jambes le menèrent jusqu'à une fenêtre ouverte. Il s'y pencha, appuya ses bras contre la rambarde et regarda les pelouses de Poudlard depuis son perchoir.

Le marron des arbres désolés, le jaune de l'herbe morte, la glace couleur de glace qu'étaient les ruisseaux et les torrents gelés… le fonctionnaire de l'école qui l'avait appelée “La Forêt Interdite” n'avait vraiment rien compris au marketing~: le nom ne faisait que donner encore plus envie d'y aller. Le soleil sombrait dans le ciel car cela faisait maintenant quelques heures que Harry réfléchissait, principalement les mêmes pensées répétées en boucle, mais chaque fois dotées de différences cruciales, comme si au lieu de parcourir un cercle, ses pensées montaient ou descendaient une spirale.

Il n'arrivait toujours pas à croire qu'il avait traversé \emph{toutes} ces épreuves à Azkaban — il avait éteint son Patronus avant que celui-ci ne le draine de toute sa vie, il avait assommé un Auror, il avait trouvé comment masquer Bella de la vue des Détraqueurs, il avait fait face à douze de ces créatures et les avait effrayées, il avait inventé le balai-fusée, il l'avait piloté — il avait traversé \emph{tout ça} sans \emph{une seule fois} se ressaisir en pensant~: \emph{Je dois le faire… parce que… j'ai promis à Hermione que je rentrerai du déjeuner~!} Ça lui semblait être une opportunité irrémédiablement perdue~; comme si, en ayant manqué l'occasion \emph{cette} fois-ci, il s'était condamné à ne jamais \emph{réussir}, quel que soit son prochain défi, quelle que soit la promesse qu'il aurait faite cette fois là. Parce qu'alors il le ferait avec maladresse, délibérément, pour réparer son erreur de la \emph{première} fois, au lieu de prononcer les déclarations héroïques qu'il aurait pu faire s'il s'était souvenu de sa promesse à Hermione. Comme si ce mauvais tournant était irrévocable, comme si on n'avait qu'une seule chance et qu'il fallait réussir du premier coup…

Il aurait dû se souvenir de cette promesse à Hermione \emph{avant} d'aller à Azkaban.

Pourquoi est-ce qu'il avait décidé de faire ça, déjà~?

\emph{Mon hypothèse de travail est que tu es stupide}, dit Poufsouffle.

\emph{Ce n'est pas utile, comme analyse de défaillances}, pensa Harry.

\emph{Si tu veux plus de détails}, dit Poufsouffle, \emph{le professeur de Défense de Poudlard était là~: “Faisons sortir Bellatrix Black d'Azkaban~!” et tu étais là~: “Ça marche~!”}

\emph{Attends, ÇA n'est pas juste -}

\emph{Hé}, dit Poufsouffle\emph{, t'as vu comment, maintenant que t'es là haut, tous les arbres se fondent les uns dans les autres et comment la forme de la forêt devient visible~?}

Pourquoi \emph{avait}-il fait ça…~?

Pas suite à un calcul des coûts et des bénéfices, ça, c'était certain. Il avait été trop gêné pour sortir une feuille de papier et commencer à calculer les utilités attendues, il avait eu peur que le professeur Quirrell cesse de le respecter s'il refusait ou même s'il hésitait trop à secourir une demoiselle en détresse.

Quelque part au fond de lui, il s'était dit que si son mystérieux professeur lui offrait sa première mission, sa première chance, un l'appel vers l'aventure, et qu'il disait \emph{non}, alors son mystérieux professeur le quitterait, dégoûté, et qu'il n'aurait plus jamais une chance d'être un héros…

… ouais, c'était ça. Rétrospectivement, c'était ça l'explication. Il s'était mis à penser que sa vie avait une intrigue et qu'il s'agissait plus là d'un rebondissement que, oh, disons, que d'une proposition de \emph{faire sortir Bellatrix Black d'Azkaban.} Ça avait été la véritable raison derrière sa décision, pendant la fraction de seconde où elle avait été prise, quand son cerveau avait reconnu la perception d'un fil narratif où une réponse négative aurait été dissonante. Et quand on y réfléchissait, ce n'était pas une façon rationnelle de prendre des décisions. Le but réel du professeur Quirrell avait été d'obtenir les derniers restes du savoir perdu de Serpentard avant que Bellatrix ne meure et qu'il ne soit irrévocablement perdu~; cela semblait incroyablement sain d'esprit par comparaison~; c'était un bénéfice proportionné avec ce qui n'avait alors semblé être qu'un faible risque.

Ça ne semblait pas juste, vraiment pas \emph{juste} que ce soit \emph{ça} qui se produise quand il perdait sa mainmise sur la rationalité pendant juste une petite fraction de seconde, la petite fraction de seconde dont son cerveau avait eu besoin pour décider qu'il trouverait les arguments pour “oui” plus agréables que les arguments pour “non” pendant la discussion qui avait suivie.

De là haut, loin au-dessus des arbres fondus les uns parmi les autres, Harry regarda la forêt.

Il ne \emph{voulait pas} se confesser, ternir sa réputation à tout jamais et mettre tout le monde en colère contre lui, peut-être même finir tué par un Seigneur des Ténèbres. Il aurait préféré être enfermé à Poudlard plutôt que de subir ça. Tel était son sentiment. Et il était donc agréable, soulageant même, de pouvoir s'accrocher à un seul facteur décisif, qui était que si Harry avouait, le professeur Quirrell irait à Azkaban et y mourrait.

(Une pause, un temps d'arrêt, un bégaiement dans le souffle de Harry)

Si on le formulait \emph{comme ça}… eh bien, on pouvait même se croire un héros plutôt qu'un lâche.

Harry éleva ses yeux au-dessus de la Forêt Interdite et regarda le ciel, bleu et interdit.

À travers les vitres, il regarda la grande chose brûlante et lumineuse, les choses duveteuses, le bleu infini et mystérieux où elles étaient encastrés, le nouvel endroit inconnu et étrange.

Ça… l'aidait vraiment, en fait, de penser au fait que ses ennuis n'étaient rien comparés au fait d'être à Azkaban. Au fait que certaines personnes avaient de \emph{vrais} ennuis et que Harry Potter n'était pas l'une de ces personnes.

Qu'allait-il faire pour Azkaban~?

Qu'allait-il faire pour l'Angleterre magique~?

… et maintenant, dans quel camp était-il~?

Sous la radieuse lueur du jour, toutes les paroles d'Albus Dumbledore avaient certainement \emph{semblé} bien plus sages que celles du professeur Quirrell. Supérieures, plus intelligentes, plus morales, plus \emph{commodes}~: ne serait-ce pas bien mieux si elles étaient vraies~? Et il lui fallait se souvenir que c'était Dumbledore qui croyait à quelque chose \emph{parce que} ça sonnait bien mais que c'était le professeur Quirrell qui était sain d'esprit.

(Encore une interruption dans son souffle, comme à chaque fois qu'il pensait au professeur Quirrell.)

Mais ce n'était pas non plus parce que quelque chose sonnait bien que c'était \emph{faux}.

Et si le professeur Quirrell \emph{avait} un déséquilibre mental, c'était qu'il avait un point de vue \emph{trop négatif} sur les choses.

\emph{Vraiment~?} demanda la partie de Harry qui avait lu dix-huit-millions de résultats expérimentaux sur l'excès de confiance et d'optimisme dont faisaient preuve les gens. \emph{Le professeur Quirrell est trop pessimiste~? Si pessimiste que ses attentes sont régulièrement} en-deçà \emph{de la réalité~? Empaille-le et mets-le dans un musée alors, parce que c'est le seul au monde. Lequel de vous deux a prévu le crime parfait} avant \emph{d'y incorporer toute la marge d'erreur et tous les plans de secours qui ont fini par te sauver la peau} juste au cas où \emph{le crime parfait tournerait mal~? Indice~: il ne s'appelle pas Harry Potter.}

Mais <<~pessimiste~>> n'était pas une description correcte du problème dont souffrait le professeur Quirrell — s'il avait vraiment un problème, et non pas juste une plus grande sagesse née de son expérience. Mais aux yeux de Harry, il semblait que le professeur Quirrell interprétait toujours tout de la pire des façons possibles. Si vous donniez au professeur Quirrell un verre à 90~\% plein, il vous dirait que la partie vide à 10~\% prouvait que personne n'accordait \emph{vraiment} d'importance à l'eau.

C'était une très bonne analogie, maintenant que Harry y pensait. Toute l'Angleterre magique n'était pas comme Azkaban, le verre était bien plus qu'à moitié rempli…

Il regarda le ciel bleu et limpide

… même si, en \emph{suivant} cette analogie, l'existence d'Azkaban \emph{prouvait} peut-être que les bons 90~\% étaient là pour d'autres raisons, pour \emph{faire montre de bonté} comme l'avait dit le professeur Quirrell. Car s'ils étaient vraiment bons, auraient-ils jamais construit Azkaban~? N'auraient-ils pas pris d'assaut la forteresse afin de la détruire~?

Il regarda le ciel bleu et limpide. Si vous vouliez être un rationaliste, il fallait lire un sacré nombre d'études sur les failles de la nature humaine, et si certaines de ces failles n'étaient que d'innocentes erreurs de logique, d'autres étaient bien plus sombres.

Il regarda le ciel bleu et limpide et pensa à l'expérience de Milgram.

Stanley Milgram avait inventé cette expérience pour étudier les causes de la seconde guerre mondiale, pour essayer de comprendre pourquoi les citoyens Allemands avaient obéi à Hitler.

Alors il avait devisé une expérience pour étudier \emph{l'obéissance}, pour voir si les Allemands, pour une raison une autre, étaient plus prompt à obéir à des ordres dictant de faire souffrir autrui.

Il avait commencé par une version pilote de son expérience sur des sujets Américains afin d'avoir un groupe témoin.

Après ça, il ne s'était pas embêté à essayer en Allemagne.

Dispositif expérimental~: une série de 30 boutons alignés horizontalement avec des étiquettes allant de “15 volts” à “450 volts”, avec une étiquette supplémentaire par groupe de quatre boutons. Le premier groupe était marqué “Choc léger”, le sixième “Choc d'une extrême intensité”, le septième “Danger~: Choc sévère” et les deux derniers boutons étaient seulement marqués d'un “XXX”.

Un acteur, complice de l'expérimentateur, était présenté au véritable sujet comme étant quelqu'un comme lui~: quelqu'un qui avait répondu à une annonce cherchant des participants pour une expérience sur l'apprentissage et qui avait perdu à une loterie (truquée), si bien qu'il serait attaché à une chaise et équipé d'électrodes. Le véritable sujet de l'expérience recevait un léger choc au moyen des électrodes, juste pour qu'il voit comment ça fonctionne.

On disait ensuite au véritable sujet que l'expérience portait sur les effets de la punition sur l'apprentissage et la mémoire et que cette partie du test cherchait à savoir si les résultats changeaient en fonction de la personne qui infligeait la punition~; la personne attachée à la chaise devrait s'efforcer de mémoriser des ensembles de paires de mots et à chaque fois que “l'apprenant” ferait une erreur, “l'enseignant” devrait administrer un choc de plus en plus fort.

À 300 volts, l'acteur cessait d'essayer de répondre et commençait à frapper le mur, ce sur quoi l'expérimentateur disait au sujet de considérer une absence de réponse comme une mauvaise réponse et de continuer.

À 315 volts, l'acteur frappait de nouveau sur le mur.

Puis plus un son jusqu'à la fin de l'expérience.

Si le sujet soulevait des objections ou refusait d'appuyer sur un bouton, l'expérimentateur, tout en maintenant une attitude passive, habillé d'une blouse grise de laboratoire, disait “Poursuivez, s'il vous plaît”, puis “L'expérience requiert que vous poursuiviez”, puis “Il est absolument essentiel que vous poursuiviez”, puis “Vous n'avez pas le choix, vous \emph{devez} continuer”. Si la quatrième incitation échouait, l'expérience s'arrêtait.

Avant de procéder à son expérience, Milgram avait décrit le dispositif expérimental et avait demandé à quatorze doctorants en psychologie quel pourcentage de sujets irait, selon \emph{eux}, jusqu'à 450 volts et jusqu'aux deux derniers boutons marqués XXX, après que la victime aura cessé de réagir.

La réponse la plus pessimiste avait été 3~\%.

Le véritable résultat avait été 26 sur 40.

Les sujets avaient sué, grogné, bégayé, rit nerveusement, mordu leurs lèvres, enfoncés leurs ongles dans leur chair. Mais sur indication de l'expérimentateur, ils avaient pour la plupart continué à administrer ce qu'ils croyaient être des chocs électriques douloureux, dangereux et potentiellement mortels. Jusqu'à la fin.

Harry pouvait entendre le professeur Quirrell rire dans son esprit~; sa voix disant quelque chose comme~: \emph{Eh bien, M. Potter, même moi je n'avais pas été aussi cynique~; je savais que les hommes trahissaient les principes qui leur étaient les plus chers pour l'argent et le pouvoir, mais je ne m'étais pas rendu compte qu'il suffisait d'un regard sévère.}

Il était dangereux d'essayer de faire des conjectures en psychologie évolutionniste si on n'était pas un psychologue évolutionniste professionnel~; mais quand Harry avait apprit l'existence de l'expérience de Milgram, l'idée lui était venue que de telles situations s'étaient probablement produites de nombreuses fois dans l'environnement ancestral et que la plupart des ancêtres potentiels qui avaient essayé de désobéir à l'Autorité étaient morts. Ou du moins qu'ils ne s'en étaient pas aussi bien sortis que les plus serviles. Les gens \emph{se croyaient} bons et moraux, mais quand on les poussait un peu, un interrupteur s'allumait dans leur cerveau et il était soudain beaucoup plus difficile de défier l'Autorité qu'ils ne l'avaient cru. Même si vous y parveniez, ce ne serait pas simple, ce ne serait pas juste une démonstration d'héroïsme sans effort. Vous trembleriez, votre voix se briserait, vous auriez peur~; dans ces conditions, seriez-vous toujours capable de défier l'Autorité~?

Harry cligna alors des yeux car son cerveau venait d'établir un lien entre l'expérience de Milgram et ce que Hermione avait fait lors de son premier cours de Défense, lorsqu'elle avait refusé de tirer sur un de ses camarades même lorsque l'Autorité lui avait dit qu'elle devait le faire~; elle avait tremblé, elle avait eu peur, mais elle avait quand même refusé. Harry avait vu cet événement se produire sous ses yeux et il n'avait pas établi de lien avant cet instant…

Il baissa les yeux vers l'horizon rougissant. Le soleil sombrait de plus en plus bas, le ciel s'assombrissait, se grisait, et même s'il était encore bleu il deviendrait bientôt noir. Les couleurs rouge et or du soleil et de son coucher lui rappelaient Fumseck~; et il se demanda, l'espace d'un instant, si c'était triste d'être un phénix, d'appeler, de crier sans jamais recevoir de réponse.

Mais Fumseck n'abandonnerait jamais, il renaîtrait autant de fois qu'il mourrait, car il était un être de lumière et de feu, et désespérer au sujet d'Azkaban appartenait autant aux ténèbres qu'Azkaban elle-même.

Si on vous donnait un verre à moitié vide et à moitié plein, telle était la réalité, telle était la vérité des choses~; mais il restait la possibilité de choisir ce que vous \emph{ressentiez}, la possibilité de désespérer de sa moitié vide ou de se réjouir de l'eau présente.

Milgram avait essayé certaines variantes de son test.

Lors de la dix-huitième expérience, le sujet n'avait eu qu'à énoncer les questions à la victime attachée à sa chaise et à enregistrer les réponse pendant que quelqu'un d'\emph{autre} appuyait sur les boutons. Vous voyiez la même souffrance, les mêmes coups frénétiques suivis d'un silence~; mais ce n'était pas \emph{vous} qui appuyiez sur les boutons. \emph{Vous} ne faisiez que regarder et poser les questions à la personne qui se faisait torturer.

37 des 40 sujets de cette expérience avaient continué de participer jusqu'à la fin, jusqu'au bouton de 450 volts marqué “XXX”.

Et si vous aviez été le professeur Quirrell, vous auriez pu décider d'en tirer un certain cynisme.

Mais 3 des 40 sujets avaient \emph{refusé} de participer jusqu'à la fin.

Les Hermione.

Elles existaient vraiment, ces personnes qui ne lanceraient pas un sort d'attaque simple sur un de leurs camarades même si le professeur de Défense le leur avait ordonné. Celles qui avaient abrité les Gitans, les Juifs et les homosexuels dans leurs greniers pendant l'Holocauste et avaient parfois perdu leur vie en retour.

Ces gens appartenaient-ils à une espèce séparée du reste de l'humanité~? Possédaient-ils un appareillage cérébral différent, un ensemble de circuits neuronaux supplémentaire que les mortels moindres n'avaient pas~? Mais c'était peu probable, compte tenu de la logique de la reproduction sexuée qui impliquait que les gènes de machineries complexes se désordonnaient au-delà de tout espoir de réparation lorsqu'ils n'étaient pas universels.

Quels que soient les composants qui formaient Hermione, tout le monde avait les mêmes quelque part à l'intérieur…

… enfin, c'était une pensée agréable mais elle n'était pas \emph{strictement} vraie, les lésions cérébrales étaient une réalité, les gens pouvaient \emph{perdre} des gènes, la machinerie complexe pouvait cesser de fonctionner, les sociopathes et les psychopathes existaient, ces gens qui ne possédaient pas les composants qui leur permettaient de se soucier des autres. Peut-être Voldemort était-il né ainsi, ou peut-être avait-il su ce qu'était le bien mais avait-il quand même choisi le mal~; à ce stade ça n'avait plus la moindre importance. Mais une \emph{supermajorité} de la population devait être capable d'apprendre à faire ce que Hermione et ceux qui avaient résisté à l'Holocauste avaient fait.

Ces gens qui avaient vécu l'expérience de Milgram, ceux qui avaient tremblé, sué et rit nerveusement tout en appuyant sur tous les boutons jusqu'à ceux marqués d'un “XXX”, nombre de ces gens avaient ensuite écrit à Milgram pour le remercier de ce qu'ils avaient découverts sur eux-mêmes. Cela aussi faisait partie de l'histoire, de la légende de l'expérience légendaire.

Le soleil était maintenant presque englouti et un dernier fragment surplombait les sommets des arbres lointains.

Harry regarda ce fragment~; ses lunettes étaient censées protéger des ultraviolets, il pouvait donc le regarder directement sans endommager ses yeux.

Il la regarda, cette petite fraction de Lumière qui n'était ni obscurcie ni bloquée ni cachée, même si ce n'étaient que trois quarantièmes, même si les 37 autres étaient là, quelque part. Les 7,5~\% du verre qui étaient pleins, qui prouvaient que les gens accordaient de l'importance à l'eau, même si cette préoccupation intérieure était trop souvent vaincue. Si les gens n'en avaient vraiment rien eu à faire, le verre aurait été entièrement vide. Si tout le monde avait vraiment été comme Vous-Savez-Qui en son for intérieur, il n'y aurait eu aucun résistant contre l'Holocauste.

Harry regarda le crépuscule du deuxième jour du reste de sa vie et sut qu'il avait changé de camp.

Parce qu'il ne pouvait plus vraiment y croire, pas vraiment, pas après s'être rendu à Azkaban. Il ne pouvait plus faire ce que 37 personnes sur 40 attendraient de lui en l'élisant. Tout le monde avait peut-être en lui ce qu'il fallait pour être une Hermione, et peut-être qu'un jour tout le monde le découvrirait~; mais \emph{un jour}, ce n'était pas \emph{maintenant}, pas ici, pas aujourd'hui, pas dans la réalité. Quand on était du côté des 3 sur 40 on ne constituait pas une majorité politique, et le professeur Quirrell avait eu raison~: Harry ne s'y soumettrait pas.

Il y avait là une sorte de terrible justesse. On n'aurait pas dû pouvoir aller à Azkaban et revenir sans avoir changé d'avis sur quoi que ce soit d'importance.

\emph{Alors le professeur Quirrell a-t-il raison~?} demanda Serpentard. \emph{Indépendamment du fait qu'il soit bon ou mauvais, est-ce qu'il a} raison \emph{? Es-tu leur prochain Seigneur, qu'ils le sachent ou non~? Oublions le “des Ténèbres”, c'est juste son cynisme habituel. Mais ton intention est-elle maintenant de diriger~? Je dois avouer que même} moi \emph{ça me rend nerveux.}

\emph{Penses-tu qu'on puisse te remettre le pouvoir en toute confiance~?} dit Gryffondor. \emph{N'y a-t-il pas une règle qui dit que ceux qui désirent le pouvoir ne devraient pas l'obtenir~? Peut-être devrions-nous plutôt donner le pouvoir à Hermione.}

\emph{Te crois-tu capable de diriger une société sans la voir s'effondrer dans un chaos total en moins de trois semaines~?} dit Poufsouffle. \emph{Imagine le hurlement que Maman pousserait si elle apprenait que tu as été élu Premier Ministre et demande-toi si tu es sûr qu'elle aurait tort~?}

\emph{À vrai dire}, continua Serdaigle, \emph{je dois remarquer que toutes ces histoires politiques ont l'air extraordinairement ennuyeuses. Et si on laissait l'électoralisme à Drago et qu'on s'en tenait à la science~? C'est là qu'on est vraiment doué, et on a déjà pu voir la science améliorer la condition humaine, tu sais.}

\emph{Moins vite}, dit Harry à ses composants, \emph{on ne doit pas décider de tout tout de suite. On a le droit de réfléchir au problème autant que possible avant d'arriver à une conclusion.}

La dernière fraction du Soleil fut engloutie sous l'horizon.

Cette sensation de ne pas tout à fait savoir qui on était, de ne pas savoir de quel côté on se trouvait, de ne pas avoir \emph{déjà un avis arrêté} sur un sujet d'une telle importance, voilà qui était étrange, qui recelait une sensation de liberté peu familière…

Et cela lui rappela ce que le professeur Quirrell avait répondu à sa dernière question, ce qui lui remit le professeur Quirrell en tête, rendit de nouveau sa respiration difficile, déclencha une sensation de brûlure dans sa gorge, renvoya ses pensées le long de cette boucle ou de cette spirale ascendante.

Pourquoi était-il maintenant si triste à chaque fois qu'il pensait au professeur Quirrell~? Harry était habitué à bien se comprendre et il ne savait pas pourquoi il se sentait si triste…

Il avait l'impression d'avoir perdu le professeur à Azkaban, voilà ce qu'il ressentait. Aussi sûrement que s'il avait été mangé par des Détraqueurs, consumé par des néants creux.

\emph{Perdu~? Pourquoi l'ai-je perdu~? Parce qu'il a dit Avada Kedavra et qu'il y avait de fait une excellente raison à cela, même si je ne l'ai pas perçue pendant deux heures~? Pourquoi les choses ne peuvent-elles pas redevenir comme avant~?}

Mais ça n'était \emph{pas} le Avada Kedavra. Ça avait peut-être aidé à faire s'effondrer de façon irréversible la structure de rationalisations, d'hésitations, de pensées précautionneusement évitées. Mais ça n'avait pas été l'Avada Kedavra, ça n'avait pas été ça l'élément troublant qu'il avait vu.

\emph{Qu'est-ce que j'ai vu…~?}

Harry regarda le ciel qui s'assombrissait.

Il avait vu le professeur Quirrell se transformer en un criminel endurci face à l'Auror et le changement apparent de personnalité avait été absolu et sans effort.

Une autre femme l'avait connu sous le nom de “Jeremy Jaffe”.

\emph{Et sinon, vous êtes combien de personnes différentes~?}

\emph{Je ne peux pas prétendre m'être embêté à en garder le compte.}

On ne pouvait faire autrement que de se demander…

… si le “professeur Quirrell” n'était qu'un nom de plus sur la liste, une personne de plus à avoir été \emph{faite}, construite au service d'un dessein impossible à deviner.

Dorénavant, il se poserait toujours la question, à chaque fois qu'il lui parlerait~: si ce n'était qu'un masque et quels pouvaient être les motifs derrière ce masque. À chaque sourire sec, Harry essaierait de voir ce qui tirait les leviers de ces lèvres.

\emph{Est-ce ainsi que les autres me verront si je deviens trop Serpentard~? Si je réussis trop de complots, ne pourrais-je plus jamais sourire à quelqu'un sans qu'il se demande ce que je veux vraiment dire~?}

Peut-être existait-il un moyen de restaurer une confiance dans les apparences extérieures et de rendre possible une relation humaine normale, mais Harry n'en voyait aucun.

C'est ainsi que Harry avait perdu le professeur Quirrell, pas la personne mais la… connexion…

Pourquoi était-ce si douloureux~?

Pourquoi se sentait-il si seul à présent~?

Il y avait certainement d'autres personnes, peut-être même meilleures, à qui donner son amitié et sa confiance~? Le professeur McGonagall, le professeur Flitwick, Hermione, Drago, sans parler de Maman et de Papa, ce n'était pas comme si Harry était \emph{seul}…

Sauf que…

La gorge de Harry se serra lorsqu'il comprit.

Sauf que le professeur McGonagall, le professeur Flitwick, Hermione et Drago, même s'ils savaient parfois des choses que Harry ignorait…

Ils ne le dépassaient pas en excellence \emph{dans} son domaine~; tout génie qu'ils possédaient n'était pas semblable au sien, et le sien n'était pas semblable au leur~; peut-être les voyait-il comme des pairs mais ils ne les respectait pas comme des \emph{supérieurs}.

Aucun d'eux n'avait été, aucun d'eux ne pourrait être…

Le mentor de Harry…

Voilà ce que le professeur Quirrell avait été.

Voilà ce que Harry avait perdu.

Et la manière dont il l'avait perdu ne lui permettrait peut-être jamais de le récupérer. Peut-être un jour connaîtrait-il tous les buts secrets du professeur Quirrell, peut-être les doutes qui les séparaient s'évanouiraient~; mais même si cela semblait possible, cela ne semblait pas très probable.

Il y eut une rafale de vent à l'extérieur, elle fit pencher les arbres vides, créa des rides à la surface du lac dont le centre était encore liquide, émit un soupir en passant devant la fenêtre qui surplombait le monde à moitié crépusculaire, et les pensées de Harry vagabondèrent hors de lui un moment.

Puis elles revinrent à l'intérieur, vers la prochaine étape de la spirale.

\emph{Pourquoi ne suis-je pas comme les autres enfants de mon âge~?}

Si sa réponse avait été une esquive, elle avait été très bien calculée. Suffisamment profonde et complexe, suffisamment riche en indications de sens caché, pour servir de piège à un Serdaigle que rien moins n'aurait su distraire. Ou peut-être le professeur Quirrell avait-il répondu avec sincérité. Qui pouvait savoir le dessein qui avait manié les leviers derrière ces lèvres~?

\emph{Je dirais ceci, M. Potter~: Vous êtes déjà un Occlumens et je pense que vous deviendrez un Occlumens parfait sous peu. Pour les gens tels que nous, l'identité n'a pas le même sens que pour les autres. Tout ce que nous pouvons imaginer, nous pouvons le devenir~; et ce qui vous rend vraiment différent, M. Potter, c'est que vous avez une imagination exceptionnellement bonne. Un dramaturge doit contenir ses personnages, il doit être plus grand qu'eux afin de les faire jouer dans son esprit. Pour un acteur, un espion ou un politicien, la limite de son ampleur est la limite de ce qu'il peut prétendre être, la limite du visage qu'il sait porter comme un masque. Mais pour ceux semblables à vous et moi, tout ce que nous pouvons imaginer, nous pouvons l'}être\emph{, réellement, sans faire semblant. Lorsque vous vous imaginiez un enfant, M. Potter, vous} étiez \emph{un enfant. Et pourtant vous pourriez si vous le souhaitiez vivre d'autres existences, plus grandes. Pourquoi êtes-vous si libre, pourquoi votre circonférence est-elle si grande, tandis que les autres enfants de votre âge sont si petits et si restreints~? Pourquoi pouvez-vous imaginer et} devenir \emph{des} Je \emph{plus adultes que ceux qu'un jeune dramaturge devrait être capable de composer~? Cela, je l'ignore, et je ne dois pas révéler ce que je devine. Mais ce que vous possédez, M. Potter, c'est la liberté.}

Si c'était du baratin, il était sacrément efficace pour détourner l'attention.

Et la pensée encore plus angoissante, c'était que le professeur Quirrell ne s'était pas \emph{rendu compte} du trouble dans lequel sa réponse allait jeter Harry, du point auquel ce discours allait lui sembler \emph{faux}, des dommages qu'il causerait sur sa confiance envers le professeur Quirrell.

Il fallait bien qu'il y ait un \emph{vrai} soi, au centre de tout…

Harry regarda la nuit qui tombait, les ténèbres qui s'amoncelaient.

… non~?

\later

Il était presque l'heure du coucher lorsque Hermione entendit les inspirations coupées et leva les yeux de son exemplaire de \emph{l'Histoire de Beauxbâtons} pour voir le garçon disparu, celui qu'on avait perdu depuis dimanche midi, dont l'absence au dîner avait été accompagnée de rumeurs, qu'elle n'avait pas crues parce qu'elles étaient \emph{complètement ridicules} mais qui l'avaient quand même mise mal à l'aise et selon lesquelles il avait quitté Poudlard afin de pourchasser Bellatrix Black.

<<~\emph{Harry~!}~>> glapit-elle sans même se rendre compte qu'elle s'adressait directement à lui pour la première fois depuis une semaine ni remarquer la réaction de quelques autres élèves après que son eut ait traversé toute la salle commune de Serdaigle.

Les yeux de Harry s'étaient déjà braqués vers elle, il marchait déjà dans sa direction, si bien qu'elle s'interrompit, à moitié levée de sa chaise -

Quelques instants plus tard Harry était assis face à elle et il reposait sa baguette après avoir lancé un sortilège de Silence autour d'eux.

(Un sacré paquet de Serdaigle essayaient de ne pas avoir l'air de regarder).

<<~Hé~>>, dit Harry. Sa voix vacilla. <<~Tu m'as manqué. Tu… vas me reparler maintenant~?~>>

Hermione hocha la tête, rien de plus, car elle ne savait pas quoi répondre. Il lui avait manqué à elle aussi mais elle commençait à se rendre compte, avec une certaine culpabilité, que ça avait peut-être été bien pire pour lui. Elle avait d'autres amis, tandis que Harry… parfois elle trouvait que ça n'était pas \emph{juste} qu'il ne parle qu'à elle, car cela voulait dire qu'elle \emph{devait} lui parler~; mais Harry avait l'air d'avoir \emph{lui aussi} subi plusieurs injustices.

<<~Qu'est-ce qui se \emph{passe}~? dit-elle~? Il y a plein de rumeurs. Des gens disent que tu es allé te battre contre Bellatrix Black, d'autres que tu es parti la \emph{rejoindre} -~>> et \emph{ces} rumeurs disaient que Hermione avait tout inventé au sujet du phénix alors Hermione avait hurlé que toute la salle commune de Serdaigle l'avait vu, si bien que la rumeur suivante avait prétendu que Hermione avait aussi inventé \emph{ça}, ce qui était d'un niveau de stupidité tellement inconcevable que ça l'avait laissée \emph{complètement pantoise}.

<<~Je ne peux pas en parler, dit Harry dans un souffle. Je ne peux presque rien en dire. J'aimerais pouvoir tout te raconter~>>, sa voix trembla, <<~mais je ne peux pas… je ne sais pas mais en tout cas si ça peut t'aider~: je n'irai plus déjeuner avec le professeur Quirrell…~>>

Il mit alors ses mains sur son visage et masqua ses yeux.

Hermione sentit le malaise qui se répandait dans son estomac.

<<~Est-ce que tu pleures~? dit Hermione.

--- Ouais, dit Harry d'une voix qui se rapprochait du chuchotement. Je ne veux pas que quelqu'un d'autre me voie.~>>

Il y eut un court silence. Elle voulait l'aider mais ignorait quoi faire pour aider les garçons qui pleuraient, elle ne savait pas ce qui se passait~; elle sentait que des événements colossaux se produisaient autour d'elle — non, autour de Harry — et que si elle avait su ce qui se passait elle aurait probablement été effrayée, alarmée, ou quelque chose dans le genre, mais elle ignorait tout.

<<~Le professeur Quirrell a-t-il fait quelque chose de mal~? dit-elle enfin~?

--- Ce n'est pas pour ça que je n'irais plus déjeuner avec lui~>>, dit Harry, toujours dans un souffle, ses mains toujours contre ses yeux. <<~C'est le directeur qui en a décidé ainsi. Mais ouais, le professeur Quirrell m'a dit certaines choses et je crois qu'elles ont diminué ma confiance en lui…~>> la voix de Harry vacillait beaucoup. <<~Je me sens assez seul en ce moment.~>>

Hermione mit sa main sur sa propre joue, là où Fumseck l'avait effleurée la veille. Elle avait continué à penser à ce contact, encore et encore, peut-être parce qu'elle \emph{voulait} que ce soit important, que ça ait un sens…

<<~Est-ce que je peux faire quoi que ce soit pour t'aider~? demanda-t-elle.

--- Je veux faire quelque chose de normal, dit-il de derrière ses mains. Quelque chose de très normal pour des élèves de Poudlard en première année. Quelque chose que les enfants de onze et douze ans comme nous sont \emph{censés} faire. Comme de jouer à une partie de bataille explosive par exemple… est-ce que tu aurais les cartes et que tu connaîtrais les règles par hasard~?

--- Euh… à vrai dire non, je ne \emph{connais} pas les règles… dit-elle. Je sais qu'elles \emph{explosent}.

--- Et des Bavboules~? dit Harry.

--- Connais pas les règles, et elles te \emph{crachent} dessus. Harry, ce sont des jeux de garçon~!~>>

Il y eut un silence. Harry appuya ses mains contre son visage pour l'essuyer puis il les retira~; et il la regarda alors avec un air sans défense.

<<~Eh bien, dit Harry, que les sorciers et les sorcières de notre âge \emph{font-ils} quand ils jouent~? Tu sais, le genre de jeu idiot et futile auquel ont est \emph{censé} jouer à cet âge~?

--- La marelle~? dit Hermione. Le saut à la corde~? L'attaque des licornes~? \emph{Je} ne sais pas, \emph{je} lis des livres~!~>>

Harry commença à rire, et Hermione commença à glousser avec lui, \emph{c'était} drôle sans vraiment qu'elle sache pourquoi.

<<~J'imagine que ça aide un peu, dit Harry. En fait, je pense que ça m'a plus aidé que jouer aux Bavboules pendant une heure n'aurait jamais pu le faire, donc merci d'être qui tu es. Et quoi qu'il arrive, je ne demanderai \emph{jamais} à personne d'Oublietter tout ce que je sais sur le calcul différentiel. Je préférerais mourir.

--- \emph{Quoi~?} dit Hermione. Pourquoi — pourquoi \emph{voudrais-tu} faire une chose \emph{pareille}~?~>>

Il se leva de la table et le son ambiant revint dans un souffle à l'instant où son changement de position brisa le sortilège de silence. <<~Je suis un peu fatigué, je vais me coucher, dit Harry d'une voix maintenant normale et narquoise, j'ai perdu un peu de temps et je dois le rattraper, mais je te verrai au petit déjeuner puis en Botanique, si ça ne te dérange pas. Sans parler du fait que ça ne serait pas juste de te faire porter tout le poids de ma dépression. 'Nuit, Hermione.

--- Bonne nuit, Harry, dit-elle tout en se sentant très perplexe et inquiète. Beaux rêves.~>>

Harry trébucha un peu lorsqu'elle dit cela, puis il continua vers les escaliers qui menaient au dortoir des garçons de première année.

\later

Harry monta le sortilège de silence de sa tête de lit au maximum afin de ne réveiller personne si jamais il criait.

Il régla son alarme pour qu'elle le réveille pour le petit déjeuner (s'il n'était pas déjà réveillé à cette heure, voir même s'il avait dormi).

Entra dans son lit, s'allongea -

- sentit la bosse sous son oreiller.

Il leva les yeux vers le baldaquin au-dessus de sa tête.

Siffla dans sa barbe~: <<~Oh, c'est pas vrai…~>>

Il lui fallut plusieurs secondes avant de pouvoir trouver le courage de s'asseoir dans son lit, de mettre sa couverture sur sa tête et sur son oreiller afin de se masquer au regard des autres garçons, de lancer un \emph{Lumos} de faible intensité et de regarder ce qui se trouvait sous l'oreiller.

C'était un parchemin et un jeu de cartes à jouer.

Le parchemin disait~:

\begin{writtenNote}
Un petit oiseau m'a dit que Dumbledore a fermé la porte de ta cage.

Je dois admettre que dans le cas présent, Dumbledore n'a peut-être pas tort. Bellatrix Black parcourt de nouveau le monde, et ce n'est une bonne nouvelle pour personne. Si j'étais à la place de Dumbledore, je ferais peut-être pareil.

Mais juste au cas où… L'institut des Sorcières de Salem aux États-Unis accepte aussi les garçons, en dépit de son nom. Ce sont des gens bien et ils te protégeraient même de Dumbledore si tu en avais besoin. L'Angleterre considère que tu as besoin de l'autorisation de Dumbledore pour émigrer en Amérique magique, mais l'Amérique magique n'est pas d'accord. Donc en dernier recours, sors de l'enceinte de Poudlard et déchire le roi de cœur de ce jeu de cartes.

Il va sans dire que tu devrais en faire usage en dernier recours uniquement.

Porte-toi bien, Harry Potter.

- Le Père Noël
\end{writtenNote}

Harry regarda le paquet de cartes.

Il ne \emph{pourrait pas} l'emmener ailleurs, pas maintenant, les Portoloins ne fonctionnaient pas ici.

Mais il était quand même nerveux à l'idée de le prendre et de le cacher dans sa malle…

Eh bien, il avait \emph{déjà} touché le parchemin qui aurait aussi bien pu être muni d'un piège magique~; si piège il y avait.

Mais quand même.

<<~Wingardium Leviosa~>>, chuchota Harry, et il fit léviter le paquet de cartes à côté de son alarme, dans une poche de sa tête de lit. Il s'en occuperai demain.

Il se rallongea dans son lit, ferma les yeux, prêt à rêver sans phénix pour le protéger, prêt à payer ses dettes.

\later

Il se leva dans un sursaut d'horreur, pas dans un cri, il n'avait pas crié de la nuit, mais sa couverture était enroulée autour de lui suite aux mouvements brusques qu'avait eu sa forme endormie lorsqu'il avait rêvé qu'il courait, qu'il essayait d'échapper aux trous dans l'espace qui le poursuivaient à travers un couloir de métal éclairé par de faibles lampes à gaz, un couloir de métal infiniment long éclairé par de faibles lampes à gaz, et il n'avait pas \emph{su}, dans le rêve, que toucher ces vides l'aurait tué de façon horrible et aurait laissé un corps vide qui respirait toujours derrière lui, tout ce qu'il avait su c'était qu'il fallait qu'il coure, qu'il coure, qu'il coure loin des blessures dans le monde qui glissaient derrière lui -

Il recommença à pleurer, non pas à cause de l'horreur de la poursuite mais parce qu'il s'était enfui alors que quelqu'un derrière lui appelait à l'aide, lui criait de revenir et de la sauver, de l'aider, qu'on la mangeait, qu'elle allait mourir, et dans le rêve Harry s'était enfui au lieu de la sauver.

<<~\scream{Ne pars pas~!}~>> De derrière la porte de métal, la voix lui parvint dans un cri. <<~\emph{Non, non, non, ne pars pas, ne l'emmène pas, non non non -}~>>

Pourquoi Fumseck s'était-il jamais posé sur son épaule. Harry était parti. Fumseck aurait dû le haïr.

Fumseck aurait dû haïr Dumbledore. \emph{Il} était parti.

Fumseck aurait dû haïr tout le monde -

Le garçon n'était pas éveillé, il ne rêvait pas non plus, ses pensées étaient mêlées, confuses, dans les ombres à la frontière du sommeil et de l'éveil, sans la protection des barrières de sécurité que son esprit conscient s'imposait, sans les règles et les censeurs. Dans ces ombres, son cerveau était assez réveillé pour penser mais quelque chose d'autre était trop ensommeillé pour agir~; ses pensées étaient sauvages, libres des contraintes du concept de soi, des idéaux de son soi éveillé quant à ce qu'il ne devait pas penser. C'était la liberté des rêves de son cerveau pendant le sommeil de son image de lui. Libre de répéter, encore et encore, le nouveau pire cauchemar de Harry~:

<<~\emph{Non, je ne voulais pas, ne meurs pas s'il te plaît~!}

--- \emph{Non, je ne voulais pas, ne meurs pas s'il te plaît~!}

--- \emph{Non, je ne voulais pas, ne meurs pas s'il te plaît~!}~>>

Une rage grandit le long du dégoût de soi, un courroux terrible et brûlant, une haine de glace pour le monde qui avait infligé ça à la femme, pour lui-même, et dans cet état de demi-sommeil Harry fantasma d'évasions, de façons d'échapper au dilemme moral, il s'imagina flottant au-dessus de la vaste horreur triangulaire d'Azkaban, chuchotant une incantation dans un langage qui ne ressemblait à rien qui ait jamais été entendu sur Terre, proférant des murmures qui faisaient écho à travers le ciel et qu'on pouvait entendre à l'autre bout du monde, puis l'explosion d'un feu d'argent de Patronus, semblable à une explosion nucléaire, qui déchiqueta tous les Détraqueurs en l'espace d'un instant avant de déchirer les murs de métal d'Azkaban, de broyer les longs couloirs et les faibles lumières oranges, puis son cerveau se souvint que des gens s'y trouvaient et réécrit le fantasme à moitié rêvé pour révéler tous les prisonniers qui riaient en s'envolant par nuées de la carcasse flambante d'Azkaban, alors que la lumière d'argent restaurant la chair de leurs membres en plein vol, et Harry commença à pleurer de plus en plus en fort dans son oreiller, parce qu'il en était incapable, parce qu'il n'était pas Dieu -

Il avait juré sur sa vie, sur sa magie, sur son art de rationaliste, il avait juré sur tout ce qu'il tenait pour sacré et tous ses souvenirs heureux, il avait fait le serment, alors il fallait qu'il fasse quelque chose, \emph{il fallait qu'il fasse quelque chose, il fallait qu'il FASSE QUELQUE CHOSE -}

Peut-être que c'était futile.

Peut-être que suivre les règles était futile.

Peut-être qu'il fallait juste brûler Azkaban.

Et de fait, il avait juré qu'il le ferait, alors maintenant il fallait qu'il le fasse.

Il ferait juste ce qui était nécessaire pour se débarrasser d'Azkaban, c'est tout. Si ça voulait dire qu'il devrait diriger l'Angleterre, très bien, si ça voulait dire qu'il devrait trouver un sortilège à murmurer dont l'écho se répandrait dans le ciel, pourquoi pas, ce qui était important c'était de détruire Azkaban.

C'était le camp qu'il avait choisi, c'était qui il était, alors voilà, ce serait fait.

Son esprit éveillé aurait demandé beaucoup plus de détails avant d'accepter cette réponse, mais dans cet état de demi-sommeil ça semblait être une résolution assez sérieuse pour laisser son esprit retomber dans un véritable sommeil et vivre son prochain cauchemar.

\latersection{Dernier après-coup~:}

Elle se leva dans un sursaut d'horreur, une interruption de son souffle qui la laissa privée d'air, mais ses poumons ne bougèrent pas, elle s'éveilla, un cri silencieux sur les lèvres mais aucun mot, aucun mot ne sortit, car elle ne comprenait pas ce qu'elle avait vu, \emph{elle ne comprenait pas ce qu'elle avait vu}, c'était trop immense pour être saisi, ça prenait encore forme, elle ne pouvait ni nommer la chose sans forme ni s'en décharger, s'en décharger et redevenir innocente et ignorante.

<<~Quelle heure est-il~?~>> murmura-t-elle.

Son réveil-matin en or et serti de joyaux, le magnifique réveil-matin magique que le directeur lui avait offert le jour où elle avait été embauchée à Poudlard lui murmura en retour~: <<~Environ deux heures du matin. Retourne te coucher.~>>

Ses draps et sa chemise de nuit étaient trempés de sueur, elle prit sa baguette à côté de l'oreiller et se sécha avant d'essayer de se rendormir, elle essaya et finit par y parvenir.

Sybill Trelawney se rendormit.
%  LocalWords:  Profe Snippyfig Nevvy Episkey Bahry Lyall Urulat
%  LocalWords:  Hangleton Pfah Voldie’ll Verdandi Gah Malaclaw reali
%  LocalWords:  Iocane Bahl’s Portus XXX y’know G’night
