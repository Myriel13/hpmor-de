\chapter{Forçage frontal}

\lettrine{L}{e} vent mordant de janvier hurlait autour des vastes murs de pierre nue qui délimitaient les frontières matérielles du château de Poudlard, chuchotant et sifflant d'étranges tons lorsqu'il passait à proximité des fenêtres fermées et des tourelles de pierre. La dernière neige avait été en grande partie emportée, mais des plaques occasionnelles de glace fondue puis à nouveau gelée s'accrochaient encore au visage de pierre et réfléchissaient l'éclatante lumière du soleil. De loin, on aurait sûrement eu l'impression que Poudlard clignait de mille yeux.

Une rafale soudaine fit tressaillir Drago et le poussa à essayer, bien inutilement, d'approcher son corps de la pierre encore plus qu'il ne l'était déjà, une pierre qui ressemblait à de la glace et qui avait une odeur de glace. Un instinct profondément inutile le persuadait qu'il était sur le point d'être projeté loin des murs extérieurs de Poudlard et que la meilleure façon d'empêcher que cela arrive était de s'agiter par pur réflexe moteur, et peut-être même de vomir.

Drago essayait très fort de ne \emph{pas} penser aux six étages de vide qui se trouvaient en dessous de lui, et de se concentrer plutôt sur la façon dont il allait tuer Harry Potter.

«Vous savez, M. Malfoy», dit la jeune fille à côté de lui, sur le ton de la conversation, «si une voyante m'avait un jour dit que je me tiendrais à la paroi d'un château, retenue par le bout de mes doigts, en essayant de ne pas regarder en bas ni de penser à la force avec laquelle maman hurlerait si elle me voyait, je n'aurais \emph{aucune} idée de la façon dont cela allait se produire, \emph{mis à part} que ce serait par la faute de Harry Potter.»

\latersection{Plus tôt~:}

Les deux généraux alliés enjambèrent le corps de Londubat, les bottes frappant le sol avec une synchronisation quasi parfaite.

Un seul soldat se tenait maintenant entre eux et Harry, un garçon Serpentard du nom de Samuel Clamons, dont la main était blanche à force d'être crispée autour de sa baguette, tenue à la verticale afin de maintenir un mur prismatique. La respiration du garçon était rapide, mais son visage exprimait la même détermination froide que celle qui brillait dans les yeux de son général, Harry Potter, qui se tenait lui-même derrière le mur prismatique au fond du cul-de-sac du couloir, à côté d'une fenêtre ouverte, ses mains mystérieusement tenues derrière son dos.

La bataille avait été si difficile face à un ennemi surpassé à deux contre un que c'en était devenu ridicule. Cela aurait dû être facile, l'armée Dragon et le régiment Soleil avaient combiné leurs forces sans effort pendant les sessions d'entraînement, car ils s'étaient battus l'un contre l'autre assez longtemps pour commencer à bien se connaître. Le moral avait été élevé, car les deux armées savaient que cette fois elles ne se battaient pas seulement pour gagner mais aussi pour que le monde soit libéré des traîtres. En dépit des protestations surprises des deux généraux, les soldats de l'armée combinée avaient insisté pour s'appeler l'“Argiment Solgon de Dramione”, et ils avaient fait apparaître des blasons portant le signe d'un visage souriant enveloppé de flammes.

Mais les soldats de Harry avaient noirci le leur -- il n'avait pas l'air recouvert de peinture mais plutôt d'avoir été \emph{brûlé} -- et ils s'étaient battus à travers les niveaux supérieurs de Poudlard avec une furie désespérée. La rage froide que Drago voyait parfois en Harry semblait s'être infiltrée dans ses soldats, qui s'étaient battus comme si ce n'était pas un jeu. Et Harry avait vidé tout son sac de techniques, il y avait eu de petites billes de métal (Granger les avait appelées «roulements à billes») sur le sol et les escaliers, les rendant infranchissables tant qu'ils n'avaient pas été déblayés, sauf que l'armée de Harry avait déjà pratiqué des sortilèges de lévitation coordonnés et ils pouvaient faire voler les leurs \emph{juste au-dessus} des obstacles qu'ils avaient construits…

On ne pouvait pas utiliser des appareillages venus de l'extérieur lors de la partie, mais on pouvait métamorphoser tout ce qu'on voulait \emph{pendant} le jeu, du moment que c'était sûr. Et ce n'était tout simplement pas juste quand on se battait contre un garçon élevé par des scientifiques, qui connaissait des choses comme les roulements à bille et les skateboards et les cordes élastiques.

Et on en était donc arrivé là.

Les survivants des forces alliées avaient acculé les derniers restes de l'armée de Harry Potter dans un couloir en cul-de-sac.

Weasley et Vincent avaient sauté sur Londubat en même temps, bougeant ensemble comme ils l'avaient pratiqué, pas pendant des heures mais pendant des semaines, et pourtant Londubat était mystérieusement parvenu à les atteindre \emph{tous les deux} avant de s'effondrer.

Et maintenant il y avait Drago et Granger et Padma et Samuel et Harry, et vu la tête que faisait Samuel, son mur prismatique n'allait pas tenir encore bien longtemps.

Drago avait déjà mis sa baguette au niveau de Harry, attendant que le mur prismatique tombe de lui-même~; inutile de gâcher un sortilège de bris de bouclier avant que cela ne se produise. Padma avait mis sa propre baguette à hauteur de Samuel, Granger la sienne à hauteur de Harry…

Qui cachait toujours ses mains derrière son dos au lieu de viser avec sa baguette~; et il les regardait avec une expression qui aurait pu avoir été sculptée dans de la glace.

Ça pouvait être un bluff. Ce n'en était probablement pas un.

Il y eut un bref silence tendu.

Puis Harry parla.

«Je suis le méchant à présent, dit le jeune garçon avec froideur, et si vous pensez que les méchants sont si faciles que ça à achever, vous feriez mieux de reconsidérer cette idée. Battez-moi lorsque je combats sérieusement et je resterai battu~; mais perdez et la prochaine fois sera une répétition de la bataille d'aujourd'hui.»

Le garçon avança ses mains et Drago vit que Harry portait d'étranges gants, avec un matériau grisâtre très particulier sur le bout de ses doigts et des boucles qui accrochaient fermement les gants à ses poignets.

À côté de Drago, le général Soleil eut un hoquet d'horreur~; et sans même prendre le temps de l'interroger, Drago jeta un sortilège de bris de bouclier.

Samuel flancha en laissant échapper un cri, mais il maintint le mur~; et si Padma ou Granger tiraient maintenant, elles épuiseraient leurs propres forces à un point tel que cela pourrait les faire perdre.

«\emph{Harry~!} hurla Granger. \emph{Tu n'es pas sérieux~!}»

Harry était déjà en mouvement.

Et, alors qu'il ouvrait grand la fenêtre, sa voix froide dit~: «Suivez-moi si vous l'osez.»

\later

Le vent gelé hurlait tout autour d'eux.

Les bras de Drago commençaient déjà à être fatigués.

… il s'était avéré que, hier, Harry avait minutieusement démontré à Granger comment métamorphoser les gants qu'il portait en ce moment même, qui utilisaient quelque chose nommé “setæ de Gecko”~; et comment coller des morceaux métamorphosés de ce même matériau au bout de leurs chaussures~; et Harry et Granger avaient très innocemment essayé de grimper un peu aux murs et au plafond.

Et, hier aussi, Harry avait fourni à Granger un total d'exactement deux doses de potion chute plumée à transporter dans sa bourse, «juste au cas où».

Non pas que Padma les aurait suivis de toute façon. \emph{Elle} n'était pas folle.

Drago détacha sa main droite avec précaution, l'avançant autant qu'il le pouvait, puis il la plaqua de nouveau contre la pierre. À côté de lui, Granger fit de même.

Ils avaient déjà avalé la potion de chute plumée. C'était jouer à la limites des règles, mais la potion ne s'activerait pas avant que l'un d'eux ne tombe vraiment, et ils n'utilisaient donc pas l'objet tant qu'ils ne tombaient pas.

Le professeur Quirrell les regardait.

Ils étaient tous deux \emph{parfaitement, complètement, absolument en sûreté.}

Harry Potter, par contre, allait mourir.

«Je me demande pourquoi Harry fait cela,» dit le général Granger d'un ton pensif, détachant lentement le bout des doigts d'une de ses mains du mur au son d'un “shlick” prolongée. Sa main se rabattit contre le mur presque immédiatement après avoir été soulevée. «Il faudra que je le lui demande, après l'avoir tué.»

C'était incroyable ce qu'ils avaient en commun tous les deux.

Drago n'avait pas vraiment envie de parler pour le moment, mais il parvint à dire, à travers des dents serrées~:

«Ça pourrait être une vengeance. Pour le rendez-vous galant.

--- Vraiment, dit Granger. Après tout ce temps.»

Shlick. Plop.

«Que c'est gentil de sa part», dit Granger.

Shlick. Plop.

«J'imagine que je trouverai une façon vraiment romantique de le remercier», dit Granger.

Stick. Plop.

«Qu'est-ce qu'il a contre \emph{toi}~?» dit Granger.

Stick. Plop.

Le vent gelé hurlait tout autour d'eux.

\later

Vous auriez pu croire qu'avoir de nouveau un sol sous les pieds aurait donné l'impression d'être plus en sûreté.

Mais si ce sol avait été un toit incliné fait de tuiles rugueuses sur lequel se serait trouvé bien plus de neige que sur les murs de pierre et que vous l'auriez traversé en courant à grande vitesse…

Alors vous auriez eu \emph{tristement tort}.

«\emph{Luminos~!} cria Drago.

--- \emph{Luminos~!} cria Granger.

---\emph{Luminos~!} cria Drago.

--- \emph{Luminos~!}» cria Granger.

La silhouette lointaine évitait et grimpait tout en courant, et pas un seul tir ne l'atteint, mais ils gagnaient du terrain.

Jusqu'à ce que Granger ne glisse.

Rétrospectivement, c'était inévitable. Dans la vraie vie vous ne pouviez pas \emph{vraiment} courir à grande vitesse sur un toit verglacé.

Inévitablement aussi, car cela arriva sans la moindre pensée consciente, Drago pivota et saisit le bras droit de Granger et il la \emph{rattrapa}, seulement elle était déjà trop déséquilibrée, elle tombait et entraînait Drago avec elle, tout se passait si vite…

Il y eut un impact dur et douloureux, pas seulement du poids de Drago contre le toit mais aussi un peu de celui de Granger, et si elle l'avait heurté un tout petit peu plus près du bord ils auraient pu s'en sortir, mais au lieu de cela son corps bascula de nouveau et ses jambes glissèrent et son autre main chercha frénétiquement une prise…

Et c'est ainsi que Drago se retrouva à retenir le bras de Granger d'une poigne blanche tandis que son autre main serrait frénétiquement le rebord du toit et que les bouts de ses chaussures s'enfonçaient sur la tranche d'une tuile.

«\emph{Hermione}~! glapit la voix distante de Harry.

--- Drago», chuchota Granger, et Drago baissa les yeux.

Cela avait peut-être été une erreur. Il y avait beaucoup de vide en dessous d'eux, rien que du vide, ils étaient au bord d'un toit qui surplombait les murs de Poudlard.

«Il va venir m'aider, chuchota la fille, mais d'abord il va nous lancer un \emph{Luminos} à tous les deux, impossible qu'il fasse autrement. Tu dois me laisser tomber.»

Cela aurait dû être la chose la plus facile du monde.

Elle n'était qu'une sang-de-bourbe, juste une sang-de-bourbe, \emph{juste une sang-de-bourbe~!}

Elle n'aurait même pas \emph{mal} \emph{!}

… le cerveau de Drago refusait d'écouter ce que Drago avait à lui dire.

«Fais-le», chuchota Hermione, ses yeux étincelants, dénués de la moindre trace de peur, «fais-le, Drago, fais-le, tu peux le battre seul, Drago, \emph{il faut qu'on gagne~!}»

Il y eut le son de quelqu'un qui courait, et le son se rapprochait.

\emph{Oh, sois rationnel…}

La voix dans la tête de Drago ressemblait horriblement à celle de Harry Potter lorsque celui-ci lui prodiguait des leçons.

\emph{… vas-tu laisser ton cerveau diriger ta vie~?}

\latersection{Après-coup, 1~:}

Daphné Greengrass avec besoin de faire de sacrés efforts pour rester muette tandis que Millicent Bulstrode racontait à nouveau l'histoire dans la salle commune des filles Serpentard (un lieu frais et confortable des donjons, situé sous le lac de Poudlard, avec des poissons qui nageaient à chaque fenêtre et des canapés où l'on pouvait s'étendre si on le désirait). Surtout parce que, du point de vue de Daphné, c'était une histoire déjà parfaite sans avoir besoin des \emph{améliorations} de Millicent.

«Et alors~? glapirent Flora et Hestia Carrow.

---Le général Granger a levé les yeux vers lui, dit Millicent d'une voix théâtrale, et elle a dit~:'Drago~! Tu dois me lâcher~! Ne t'en fais pas pour moi, je te promets que tout ira bien~! Et que pensez-vous que Malfoy a alors fait~?

--- Il a dit~: “Jamais~!” cria Charlotte Wiland, et il l'a tenue encore plus fort~!»

Toutes les filles qui écoutaient hochèrent la tête, à part Pansy Parkinson.

«Nan~! dit Millicent. Il l'a lâchée. Et alors il s'est redressé et il a abattu le général Potter. Fin.»

Il y eut une pause abasourdie.

«Ça ne se fait \emph{pas}~! dit Charlotte.

--- C'est une \emph{sang-de-bourbe}, dit Pansy l'air confuse. Bien \emph{sûr} qu'il l'a lâchée~!

--- Eh bien, Malfoy n'aurait pas dû la rattraper en premier lieu alors~! dit Charlotte. Mais une fois qu'il l'avait attrapée, il \emph{devait} la tenir~! \emph{Surtout} face à une fin certaine qui s'approchait~!» Tracey Davis, assise à côté de Daphné, approuvait énergiquement de la tête.

«Je ne vois pas pourquoi, dit Pansy.

--- C'est parce que tu n'as pas une once de romantisme en toi, dit Tracey. Et puis, on ne peut pas juste laisser tomber les filles comme ça. Un garçon qui laisserait tomber une fille comme ça… il laisserait tomber \emph{n'importe qui}. Il te laisserait tomber \emph{toi}, Pansy.

--- Qu'es'tu veux dire, \emph{il me laisserait tomber}~?» dit Pansy.

Daphné ne pouvait plus se retenir.

«Tu sais, dit Daphné d'un air sombre, tu prends le petit déjeuner un jour à notre table, et voilà-t-y pas que Malfoy \emph{te laisse tomber}, et tu tombes du sommet de Poudlard~! Voilà ce qu'elle veut dire~!

--- Ouais~! dit Charlotte. C'est un tombeur de sorcière~!

--- Vous savez pourquoi Atlantis est tombée~? dit Tracey. Pasque quelqu'un comme Malfoy l'a \emph{laissée tomber}, voilà pourquoi~!»

Daphné baissa la voix.

«En fait… et si Malfoy était celui qui avait fait tomber Hermione, je veux dire le général Granger~? Et s'il s'est donné pour mission de faire tomber \emph{tous} les nés-Moldus~?

--- Tu veux dire -~? s'étrangla Tracey.

--- C'est ça~! dit Daphné d'un ton théâtral. Et si Malfoy était -- \emph{l'héritier de Glisserpentard~?}

--- Le prochain Tombeur des Ténèbres~!» dit Tracey.

Et c'était une expression bien trop géniale pour qu'elles la gardent pour elle, et à la tombée de la nuit elle avait fait le tour de Serpentard, et le matin suivant c'était le gros titre du \emph{Chicaneur}.

\latersection{Après-coup, 2~:}

Ce soir-là, Hermione s'assura d'arriver à leur salle de classe habituelle en avance, juste pour que lorsque Harry arrive, elle soit là, seule, dans une chaise, lisant un livre paisiblement.

S'il existait une façon de s'excuser en ouvrant une porte grinçante, c'était comme ça que Harry venait de le faire.

«Euh», dit la voix de Harry Potter.

Hermione continua de lire.

«Je suis, euh, assez désolé, je ne comptais pas \emph{vraiment} te faire tomber du toit ou quoi que ce soit…»

À vrai dire, cela avait été une expérience assez divertissante.

«Je, ah… je ne suis pas très expérimenté en excuses, je tomberai à genoux si tu veux, ou je t'achèterai quelque chose de cher, \emph{Hermione je ne sais pas comment te demander pardon pour ce que j'ai fait est-ce que tu pourrais juste me dire quoi faire~?}»

Elle continua de lire son livre en silence.

Ce n'était pas comme si \emph{elle} avait la moindre idée de la façon dont Harry aurait pu s'excuser.

Pour le moment, elle ressentait seulement une étrange curiosité quant à ce qui se passerait si elle continuait de lire son livre un moment.
%  LocalWords:  Clamons Dramione’s Sungon Argiment Wiland Slipperin
