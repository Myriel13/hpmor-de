\partchapter{Faire semblant d'être sage}{I}

\lettrine{S}{iffle}.  Tick. Bzzzt. Ding. Glorp. Pop. Sploutch. Cling. Pouet. Pouf. Vring. Bloup. Bip. Crac. Woosh. Shhh. Pffft. Vrrr.

Ce lundi, le professeur Flitwick avait sans mot dire passé un parchemin à Harry pendant le cours de sortilèges et enchantements, et la note disait que Harry devait rendre visite au directeur quand cela lui conviendrait mais en faisant en sorte que personne ne le remarque, en particulier Drago Malfoy et le professeur Quirrell. Son mot de passe à usage unique pour la gargouille serait “ossifrage chochotte”. La note avait été accompagnée d'un dessin à l'encre remarquablement bien réalisé qui représentait le professeur Flitwick le regardant d'un air sévère avec des yeux qui clignaient parfois~; et en bas de la note, soulignée trois fois, se trouvait la phrase~: NE T'ATTIRE PAS D'ENNUIS

Et Harry avait donc fini son cours de Métamorphose, puis il avait étudié avec Hermione, dîné, parlé à ses lieutenants, et enfin, quand l'horloge avait sonné neuf heures, il s'était rendu invisible, était revenu à 18h et s'était péniblement traîné jusqu'à la gargouille, jusqu'aux escaliers en spirale, jusqu'à la porte en bois, jusqu'à la pièce pleine de petites choses délicates, jusqu'à la silhouette à barbe d'argent du directeur.

Cette fois, Dumbledore semblait assez sérieux, son sourire habituel était absent~; et il était habillé d'un pyjama d'un violet plus sombre et plus sobre qu'à l'ordinaire.

«Merci d'être venu, Harry», dit le directeur. Le vieux sorcier se leva de son trône et commença à lentement marcher à travers la pièce et entre les étranges appareils. «Tout d'abord, as-tu les notes de ta rencontre d'hier avec Lucius Malfoy~?

--- Notes~? laissa échapper Harry.

--- Tu en as \emph{certainement} pris note…» dit le vieux sorcier, et il laissa sa voix en suspens.

Harry se sentit plutôt gêné. Oui, si vous passiez tant bien que mal au travers d'une conversation mystérieuse pleine de sous-entendus significatifs que vous ne compreniez pas, il était \emph{bougrement évident} qu'il fallait immédiatement l'écrire, avant que le souvenir ne s'efface, afin de pouvoir essayer de la comprendre plus tard.

«Très bien, dit le directeur, de mémoire, alors.»

Honteux, Harry récita du mieux qu'il pouvait, et il en était presque arrivé à la moitié lorsqu'il se rendit compte que ce n'était pas intelligent d'aller tout raconter à un directeur qui était peut-être fou, du moins pas sans y \emph{penser} avant, mais après tout Lucius était \emph{clairement} un méchant et l'ennemi de Dumbledore donc c'\emph{était} probablement une bonne idée de tout lui dire, et Harry avait déjà commencé à parler et maintenant il était trop tard pour essayer de faire des calculs…

Harry conclut son compte-rendu en toute franchise.

Le visage de Dumbledore était devenu de plus en plus lointain au fur et à mesure du discours de Harry, et à la fin on pouvait voir quelque chose d'antique sur son visage, comme une gravité dans l'air.

«Bien, dit Dumbledore. Je suggère alors que tu prennes grand soin de t'assurer qu'il n'arrive aucun malheur à l'héritier Malfoy. Et j'en ferai de même.» Le directeur fronçait les sourcils, ses doigts battaient silencieusement sur la surface d'une plaque noire comme de l'encre où était inscrit le mot \emph{Leliel}. «Et je pense qu'il serait extrêmement sage que tu évites dorénavant \emph{toute} interaction avec Lord Malfoy.

--- \emph{Avez-vous} intercepté ses chouettes qui m'étaient adressées~?» dit Harry.

Le directeur regarda Harry pendant un long moment, puis il hocha la tête avec réticence.

Pour une raison qu'il ignorait, Harry ne se sentit pas aussi outragé qu'il aurait dû l'être. Peut-être était-ce simplement qu'il lui était très facile de comprendre le directeur à cet instant précis. Même Harry pouvait comprendre pourquoi Dumbledore ne voudrait pas le voir interagir avec Lucius Malfoy~; ça ne semblait pas être un acte \emph{mauvais}.

Contrairement à son chantage contre Zabini… pour lequel il n'y avait que la parole de ce dernier, et il était extraordinairement indigne de confiance. En fait, il était difficile de voir pourquoi Zabini ne se serait \emph{pas} contenté de raconter l'histoire qui lui obtiendrait le plus de compassion de la part du professeur Quirrell…

«Et si, au lieu de protester, je disais que je comprends votre point de vue, dit Harry, et que vous continuiez à intercepter mes chouettes, mais que vous me disiez qui les envoie~?

--- J'ai peur d'avoir intercepté de nombreuses chouettes qui t'étaient destinées, dit Dumbledore d'un ton neutre. Tu es une célébrité, Harry, et si je ne les renvoyais pas, tu recevrais des dizaines de lettres par jour, venues parfois de pays lointains.

--- \emph{Ceci,}» dit Harry, commençant maintenant à ressentir un peu d'indignation, «semble aller un peu trop loin…

--- Nombre de ces lettres, dit le vieux sorcier avec douceur, te demanderont des choses que tu ne peux pas donner. Je ne les ai pas lues, bien sûr, seulement renvoyées à leur émetteur. Mais je le sais, car je les reçois moi aussi. Et, Harry, tu es trop jeune pour avoir le cœur brisé six fois chaque matin avant le petit déjeuner.»

Harry regarda ses chaussures. Il \emph{aurait dû} insister pour lire les lettres et juger par lui-même mais… il y avait une petite voix de bon sens à l'intérieur de lui, et elle était en train de hurler.

«Merci, marmonna Harry.

--- L'autre raison pour laquelle je t'ai demandé de venir, dit le vieux sorcier, est que je souhaite consulter ton talent génial et unique.

--- La Métamorphose~? dit Harry surpris et flatté.

--- Non, pas \emph{ce} talent génial unique, dit Dumbledore. Dis-moi, Harry, quel mal pourrais-tu accomplir si un Détraqueur était autorisé à venir dans l'enceinte de Poudlard~?»

\later

Il fut révélé que le professeur Quirrell avait demandé, ou plutôt exigé, que ses élèves mettent à l'épreuve leurs capacités contre un véritable Détraqueur, après avoir appris les mots et les gestes du Patronus.

«Le professeur Quirrell est lui-même incapable de créer un Patronus,» dit Dumbledore, déambulant lentement entre les appareils. «Ce qui n'est jamais bon signe. Mais après tout, il m'a \emph{volontairement} révélé ce fait en me demandant que des professeurs extérieurs soient amenés pour enseigner le sortilège à chaque élève désireux de l'apprendre~; il a offert de payer les frais lui-même si jamais je refusais. Cela m'a grandement impressionné. Mais maintenant il insiste pour amener un Détraqueur…

--- Professeur, dit doucement Harry, le professeur Quirrell accorde une immense importance aux tests sous feu réel, dans des conditions de combat réalistes. Vouloir amener un véritable Détraqueur correspond \emph{complètement} à son personnage.»

Le directeur jetait maintenant un étrange regard à Harry.

«\emph{Correspond à son personnage~?} dit le vieux sorcier.

--- Je veux dire, dit Harry, c'est parfaitement cohérent avec la façon dont il agit d'habitude…» Harry laissa sa phrase en suspens. Pourquoi \emph{l'avait-il} formulé ainsi~?

Le directeur hocha la tête. «Alors tu as la même impression que moi~; que c'est une excuse. Une excuse très \emph{raisonnable}, c'est certain~; plus encore que tu ne t'en rends peut-être compte. Souvent, des sorciers incapables de créer un Patronus réussiront soudain en présence d'un vrai Détraqueur, passant de rien, même pas une étincelle, à un Patronus totalement formé. Pourquoi, personne ne le sait~; mais c'est ainsi.»

Harry fronça les sourcils. «Alors je ne vois vraiment pas pourquoi vous soupçonnez…»

Le directeur ouvrit les mains comme pour exprimer son impuissance. «Harry, le professeur de Défense m'a demandé de faire passer la plus ténébreuse de toutes les créatures entre les portes de Poudlard. Je \emph{dois} être suspicieux.» Il soupira. «Et pourtant le Détraqueur sera gardé, prisonnier d'une énorme cage, je serai moi-même présent pour l'observer à tout moment -- je ne peux pas imaginer ce qu'on \emph{pourrait} faire de mal. Mais peut-être suis-je incapable de le voir. Et c'est pourquoi je te le demande.»

Harry fixa le directeur, la bouche ouverte. Il était tellement surpris qu'il ne pouvait même pas se sentir flatté.

«\emph{Moi} \emph{?} dit Harry.

--- Oui, dit Dumbledore souriant légèrement. Je fais de mon mieux pour anticiper sur les actes de mes ennemis, pour comprendre leur esprit tordu et prédire leurs maléfiques pensées. Mais \emph{je} n'aurais jamais pensé à aiguiser les os d'une Poufsouffle pour en faire des armes.»

Harry allait-il \emph{jamais} réussir à faire oublier ça~?

«Directeur, dit Harry d'un ton usé, je sais que ça n'est pas rassurant à entendre, mais très sérieusement~: je ne suis pas maléfique, je suis juste très créatif…

--- Je n'ai pas dit que tu étais maléfique, dit Dumbledore avec sérieux. Il y a ceux qui disent que comprendre le mal est devenir le mal~; mais ils font seulement semblant d'être sage. Ce qui est maléfique, c'est plutôt ce qui n'aime pas, ce qui n'ose pas imaginer l'amour et qui ne pourra jamais comprendre l'amour sans cesser d'être maléfique. Et je te soupçonne de pouvoir t'imaginer dans l'esprit de sorciers ténébreux mieux que je ne l'ai jamais fait, et ce tout en continuant de savoir ce qu'est l'amour. Donc, Harry.» Le regard du directeur s'était intensifié. «Si tu étais à la place du professeur Quirrell, quels méfaits pourrais-tu accomplir, après m'avoir roulé pour que j'autorise la présence d'un Détraqueur dans l'enceinte de Poudlard~?

--- \emph{Attendez} un instant», dit Harry, et, vaguement étourdi, il parvint jusqu'à la chaise située face au bureau du directeur et s'assit. Cette fois, c'était une chaise large et confortable, pas un tabouret de bois, et Harry pouvait se sentir être enveloppé à mesure qu'il s'y enfonçait.

Dumbledore lui demandait de déjouer les manigances du professeur Quirrell.

Remarque une~: Harry préférait grandement le professeur Quirrell à Dumbledore.

Remarque deux~: L'hypothèse était que le professeur de Défense projetait de faire quelque chose de mal, auquel cas Harry se \emph{devait} d'aider le directeur à l'empêcher.

Remarque trois…

«Professeur, dit Harry, si le professeur Quirrell se prépare \emph{en effet} à faire quelque chose, je ne suis pas certain de \emph{pouvoir} déjouer ses manigances. Il a beaucoup plus d'expérience que moi.»

Le vieux sorcier secoua la tête, parvenant à avoir un air solennel en dépit de son sourire. «Tu te sous-estimes.»

C'était la première fois que quiconque avait dit cela à Harry.

«Je me souviens, continua le vieux sorcier, d'un jeune homme, dans ce bureau, froid et avec le plein contrôle de lui-même, alors qu'il faisait face au directeur de la maison Serpentard, faisant chanter son directeur pour protéger ses camarades de classe. Et je crois que ce jeune homme est plus rusé que le professeur Quirrell, plus rusé que Lucius Malfoy, qu'il deviendra l'égal de Voldemort lui-même. C'est lui que je souhaite consulter.»

Harry réprima le frisson qui avait parcouru son échine à l'écoute du nom et fronça les sourcils d'un air pensif, à l'attention du directeur.

\emph{Que sait-il exactement…~?}

Le directeur avait vu Harry aux prises avec son mystérieux côté obscur, lorsqu'il y avait été plus profondément enfoui que jamais. Harry se souvenait encore de ce que ça avait été de regarder, invisible et remonté dans le temps, son lui passé faire face aux Serpentard plus âgés~: le garçon avec une cicatrice sur le front qui ne se comportait pas comme les autres. \emph{Bien sûr} que le directeur aurait remarqué qu'il y avait quelque chose d'étrange chez le garçon dans son bureau…

Et Dumbledore en avait conclu que son héros de compagnie avait la ruse nécessaire pour rivaliser avec l'ennemi auquel il était prédestiné~: le Seigneur des Ténèbres.

Ce qui n'était pas en demander beaucoup étant donné que le Seigneur des Ténèbres avait mis une Marque des Ténèbres clairement visible sur le bras gauche de tous ses serviteurs et qu'il avait massacré la totalité du monastère où l'on enseignait l'art martial qu'il avait souhaité apprendre.

Assez rusé pour rivaliser avec le \emph{professeur Quirrell} serait un problème d'un \emph{tout} \emph{autre} ordre.

Mais il était aussi clair que le directeur ne serait pas satisfait avant que Harry ne soit devenu tout froid et tout sombre et qu'il ait trouvé une réponse qui ait l'air d'être la marque d'une impressionnante rouerie… et qui ferait mieux de ne pas \emph{vraiment} entraver le professeur Quirrell dans son projet d'enseigner la Défense…

Et bien sûr que Harry \emph{allait} passer de son côté obscur et se pencher sur le sujet depuis cette perspective, juste par honnêteté, et juste au cas où.

«Dites-moi, dit Harry, tout ce qui concerne la façon dont le Détraqueur doit être amené, et comment il sera gardé.»

Les sourcils de Dumbledore s'élevèrent l'espace d'un instant, puis le vieux sorcier commença à parler.

Le Détraqueur serait transporté dans l'enceinte de Poudlard par un trio d'Aurors, tous trois personnellement connus du directeur, et tous trois capables de créer un Patronus corporel. Ils rencontreraient Dumbledore à la limite des murs de Poudlard, et ce dernier les franchirait en sens inverse avec le Détraqueur.

Harry demanda si le laissez-passer était permanent ou temporaire -- si quelqu'un pourrait simplement amener le même Détraqueur le lendemain.

Le laissez-passer était temporaire (répondit Dumbledore avec un hochement de tête approbateur), et l'explication continua~: Le Détraqueur serait dans une cage aux barreaux faits de titane, pas métamorphosés mais réellement forgés~; avec assez de temps, la présence du Détraqueur corroderait le métal jusqu'à ce qu'il ne reste que de la poussière, mais cela prendrait plus d'une journée.

Les élèves attendant leur tour resteraient à bonne distance du Détraqueur, derrière deux Patronus corporels maintenus à chaque moment par deux des trois Aurors. Dumbledore attendrait à côté de la cage du Détraqueur avec son Patronus. Un élève seul s'approcherait du Détraqueur~; Dumbledore dissiperait alors son Patronus~; et l'élève tenterait de créer son propre Patronus. S'il échouait, Dumbledore restaurerait alors son Patronus avant que l'élève n'ait pu subir de dommages permanents. Le professeur Flitwick, ancien champion de duel, serait lui aussi présent lorsque les élèves s'approcheraient, juste pour ajouter une marge de sécurité.

«Pourquoi serez-vous \emph{seul} à attendre à côté du Détraqueur~? dit Harry. Je veux dire, cela ne devrait-il pas être vous, accompagné d'un Auror…»

Le directeur secoua sa tête. «Ils ne pourraient pas supporter l'exposition répétée au Détraqueur à chaque fois que je dissiperais mon Patronus.»

Et si le Patronus de Dumbledore échouait pour une raison ou une autre alors qu'un des élèves était encore proche du Détraqueur, le troisième Auror créerait un autre Patronus corporel et l'enverrait voler à la défense de l'élève…

Harry fouina et farfouilla, mais il ne put voir aucun défaut dans la sécurité.

Alors il prit une profonde inspiration, s'enfonça davantage dans la chaise, ferma les yeux et se souvint~:

«\emph{Et ce sera… cinq points~? Non, disons dix points retirés à Serdaigle pour impertinence.}»

Le froid vint plus lentement cette fois, avec plus de réticence, Harry n'avait pas beaucoup fait appel à son côté obscur ces derniers temps…

Il dut se passer l'intégralité du cours de Potions dans son esprit avant que son sang ne se pétrifie en quelque chose approchant une clarté cristalline et mortelle.

Et il pensa alors au Détraqueur.

Et ce fut évident.

«Le Détraqueur est une distraction», dit Harry. La froideur était claire dans sa voix, puisque c'était ce que Dumbledore voulait et ce à quoi il s'attendait. «Une menace remarquable, immense, mais en fin de compte directe et simple à contrer. Alors, tandis que toute votre attention sera concentrée sur le Détraqueur, le véritable plan se déroulera ailleurs.»

Dumbledore fixa Harry un moment, puis il hocha lentement la tête. «Oui… dit le directeur. Et je crois savoir de \emph{quoi} cela pourrait essayer de me distraire, si le professeur Quirrell a de mauvaises intentions… merci, Harry.»

Le directeur fixait encore Harry, un regard étrange dans ces yeux anciens.

«\emph{Quoi~?}» dit Harry avec une nuance d'exaspération, le froid toujours dans son sang.

«J'ai une autre question pour ce jeune homme, dit le directeur. C'est quelque chose que je me suis longtemps demandé et que j'ai pourtant été incapable de comprendre. \emph{Pourquoi~?}» Il y avait une nuance de douleur dans sa voix. «Pourquoi quiconque deviendrait-il un monstre de façon délibérée~? Pourquoi faire le mal pour faire le mal~? Pourquoi Voldemort~?»

\later

\emph{Siffle. Tick. Bzzzt. Ding. Glorp. Pop…}

Harry regarda le directeur avec surprise.

«Comment le saurais-\emph{je}~? dit Harry. Suis-je censé magiquement comprendre le Seigneur des Ténèbres parce que je suis le héros, ou quelque chose comme ça~?

--- \emph{Oui~!} dit Dumbledore. Mon grand ennemi était Grindelwald, et \emph{lui}, je le comprenais parfaitement. Il était mon miroir obscur, l'homme que j'aurais si facilement pu être, aurais-je cédé à la tentation de croire que j'étais quelqu'un de bien et que j'avais par conséquent toujours raison. \emph{Pour le plus grand bien}, c'était son slogan~; et lui-même le croyait vraiment, même après avoir traversé l'Europe en la déchirant, tel un animal blessé. Et lui, j'ai fini par le vaincre. Mais après lui vint Voldemort, destiné à détruire tout ce que j'avais protégé en Angleterre.» La douleur était maintenant claire dans la voix de Dumbledore, révélée sur son visage. «Il a commit des actes bien pires que les pires de Grindelwald, l'horreur pour l'horreur. J'ai tout sacrifié seulement pour le retenir, et je ne comprends toujours pas \emph{pourquoi}~! \emph{Pourquoi}, Harry~? Pourquoi l'a-t-il fait~? Il n'était jamais destiné à être mon ennemi, mais le tien, alors si tu as la moindre idée, s'il te plaît, dis-le-moi~! \emph{Pourquoi~?}»

Harry regarda ses mains. La vérité, c'était que Harry ne s'était pas encore documenté sur le Seigneur des Ténèbres, et pour le moment il n'avait pas la moindre idée. Et étrangement, cela ne semblait pas être ce que le directeur voulait entendre.

«Trop de rituels obscurs peut-être~? Au début il pensait qu'il en ferait un seul, mais il a sacrifié une partie de son bon côté, et cela l'a rendu moins réticent à pratiquer d'autres rituels obscurs, alors il en a fait encore et encore dans un cycle de rétroaction positif jusqu'à ce qu'il se retrouve être un monstre effroyablement puissant…

--- Non~!» La voix du directeur était maintenant angoissée. «Je ne peux croire à cela, Harry~! Il doit y avoir plus que ça~!»

\emph{Pourquoi devrait-il y avoir autre chose~?} songea-t-il, mais il ne le dit pas, car il était clair que le directeur pensait que l'univers était une histoire, qu'il avait une intrigue, que d'immenses tragédies n'avaient pas le droit d'avoir lieu en l'absence de raisons également immenses et significatives.

«Je suis désolé, professeur. Le Seigneur des Ténèbres n'a pas tellement l'air d'être mon miroir obscur. Il n'y a rien dans l'idée de clouer les peaux de la famille de Yermy Wibble au mur d'une rédaction que je trouve le \emph{moins du monde} attrayante.

--- N'as-tu \emph{aucune} sagesse à partager~?» dit Dumbledore. Sa voix était plaidante, presque suppliante.

\emph{Le mal se produit}, pensa Harry, \emph{cela ne veut rien dire et ne nous enseigne rien, à part de ne pas être mauvais~? Le Seigneur des Ténèbres n'était probablement qu'un bâtard égoïste qui se fichait de savoir qui souffrait, ou un idiot qui a fait des erreurs bêtement évitables et qui ont fait un effet boule de neige. Il n'y a pas de destin derrière les malheurs du monde~; si Hitler avait eu le droit d'aller en école d'architecture comme il l'avait voulu, toute l'Histoire de l'Europe aurait été différente~; si nous vivions dans le genre d'univers où les choses horribles n'avaient le droit de se produire que pour de bonnes raisons, elles ne se produiraient tout simplement pas.}

Et évidemment, rien de cela ne constituait ce que le directeur voulait entendre.

Le vieux sorcier regardait encore Harry, au-dessus d'une petite chose délicate qui ressemblait à une volute de fumée figée, un désespoir douloureux dans ces yeux anciens qui semblaient attendre quelque chose.

Eh bien, avoir l'air sage n'était pas difficile. C'était beaucoup plus facile que d'être intelligent, à vrai dire, puisque vous n'aviez pas à dire quoi que ce soit de surprenant ni à trouver de nouvelles idées perspicaces. Vous laissiez juste le logiciel de reconnaissance de forme de votre cerveau compléter le cliché en utilisant n'importe quelle Sagesse Profonde enregistrée auparavant.

«Directeur, dit Harry d'un ton solennel, je préférerais ne pas me définir par rapport à mes ennemis.»

Bizarrement, au milieu de tous ces roulements et ces déclics, il y eut une sorte de silence.

Ça avait sonné un peu plus Profondément Sage que Harry ne l'avait voulu.

«Peut-être es-tu très sage, Harry, dit lentement le directeur. J'aurais en effet souhaité… avoir pu être défini par mes amis.» La douleur dans sa voix était devenue plus profonde.

L'esprit de Harry chercha frénétiquement quelque autre Sagesse Profonde à dire qui adoucirait la force inattendue de ce coup…

«Ou peut-être, dit Harry plus doucement, est-ce l'ennemi qui fait le Gryffondor, comme c'est l'ami qui fait le Poufsouffle, et l'ambition qui fait le Serpentard. Je sais que c'est toujours, à chaque génération, le puzzle qui fait le scientifique.

--- C'est un terrible destin auquel tu condamnes ma Maison, Harry,» dit le directeur. La douleur était encore présente dans sa voix. «Car maintenant que tu le mentionnes, je pense que j'ai en grande partie été fait par mes ennemis.»

Harry regarda ses propres mains, posées sur ses genoux. Peut-être qu'il devrait juste se taire tant qu'il était en tête de course.

«Mais tu \emph{as} répondu à ma question,» dit Dumbledore plus doucement, comme à lui-même. «J'aurais dû me rendre compte que la clé viendrait de Serpentard. Pour son ambition, tout pour son ambition~; et \emph{cela} je le sais, mais je ne sais pas \emph{pourquoi}…» Pendant un moment Dumbledore regarda dans le vide~; puis il se raidit et ses yeux semblèrent de nouveau mettre au point sur Harry.

«Et toi, Harry, dit le directeur, tu te dis un \emph{scientifique}~?» Sa voix était entrelacée de surprise et d'une légère désapprobation.

«Vous n'aimez pas la science~?» dit Harry d'un ton un peu usé. Il avait espéré que Dumbledore ait plus d'affection pour les choses Moldues.

«Je suppose que c'est utile pour ceux qui n'ont pas de baguette, dit Dumbledore en fronçant les sourcils. Mais me semble bien étrange de se définir par cela. La science est-elle aussi importante que l'amour~? Que la gentillesse~? Que l'amitié~? Est-ce la science qui te donne ton affection pour Minerva McGonagall~? Est-ce la science qui fait que tu te soucies de Hermione Granger~? Est-ce vers la science que tu te tournes lorsque tu essaies de faire naître de la chaleur dans le cœur de Drago Malfoy~?»

\emph{Vous savez, ce qui est triste, c'est que vous pensez probablement avoir à l'instant proféré une sorte d'argument final incroyablement sage.}

Maintenant, comme formuler la réplique de façon à ce qu'elle sonne elle aussi comme quelque chose d'incroyablement sage…

«Vous n'êtes pas Serdaigle, dit Harry avec une calme dignité, et il ne vous est donc peut-être pas venu à l'esprit que respecter la vérité et la rechercher chaque jour de sa vie peut aussi constituer un acte de grâce.»

Les sourcils du directeur s'élevèrent. Puis il soupira. «Comment es-tu devenu si sage, si jeune…~?» Le vieux sorcier semblait triste prononçant ces mots. «Peut-être cela aura-t-il de la valeur pour toi.»

\emph{Seulement pour impressionner les vieux sorciers qui sont déjà trop impressionnés par eux-mêmes}, pensa Harry. Il était à vrai dire un peu déçu par la crédulité de Dumbledore~; ce n'était pas que Harry avait \emph{menti}, mais Dumbledore semblait bien trop impressionné par la capacité de Harry à formuler les choses pour qu'elles aient l'air profondes, au lieu de les dire dans un français simple, comme Richard Feynman l'avait fait avec \emph{sa} sagesse…

«L'amour est plus important que la sagesse», dit Harry, juste pour tester les limites de la tolérance de Dumbledore pour les clichés aveuglément évidents complétés par simple reconnaissance de forme sans aucune forme d'analyse détaillée.

Le directeur hocha gravement la tête et dit «En effet.»

Harry se leva de sa chaise et étira ses bras. \emph{Eh bien, il vaut mieux que je me dépêche d'aller aimer quelque chose alors, ça va sûrement m'aider à vaincre le Seigneur des Ténèbres. Et la prochaine fois que vous me demanderez un conseil, je vous donnerai juste un câlin…}

«Tu m'as grandement aidé aujourd'hui, Harry, dit le directeur. Et il y a donc une dernière chose que je voudrais demander à ce jeune homme.»

\emph{Génial.}

«Dis-moi, Harry», dit le directeur (et maintenant sa voix semblait simplement perplexe, bien qu'il y ait toujours un soupçon de douleur dans ses yeux), «pourquoi les mages noirs ont-ils tant peur de la mort~?

--- Euh, dit Harry, désolé, mais sur ce coup je suis avec les mages noirs.»

Woosh. Shhh. Ding~; glop, plop, blup…

«\emph{Quoi~?} dit Dumbledore.

--- La mort est quelque chose de mauvais», dit Harry, rejetant la sagesse au bénéfice d'une communication claire. «Très mauvais. Extrêmement mauvais. Avoir peur de la peur, c'est comme d'avoir peur d'un grand monstre avec des crocs empoisonnés. C'est sensé, et en réalité, n'indique pas qu'on a un problème psychologique.»

Le directeur le regardait comme s'il venait de se transformer en chat.

«D'accord, dit Harry, laissez-moi le formuler ainsi. \emph{Souhaitez}-vous mourir~? Parce que si oui, il y a ce truc Moldu appelé le SOS suicide…

--- Lorsque le temps sera venu, dit le vieux mage avec douceur. Pas avant. Je ne chercherai jamais à en accélérer la venue, pas plus que je ne la refuserai lorsqu'elle sera là.»

Harry fronçait gravement les sourcils.

«On ne dirait pas que vous avez une très grande volonté de vivre, directeur~!

--- Harry…» La voix du vieux sorcier commençait à laisser entendre une légère impuissance~; et il avait déambulé jusqu'à un endroit où sa barbe avec subrepticement glissé dans le bocal cristallin d'un poisson rouge, et elle adoptait lentement une teinte verdâtre qui remontait le long des poils. «Je pense ne pas m'être bien fait comprendre. Les mages noirs ne sont pas empressés de vivre. Ils \emph{craignent la mort}. Ils ne s'élèvent pas vers la lumière du soleil mais fuient la nuit qui approche vers des cavernes de leur fabrication, infiniment plus sombres, sans lune ni étoiles. Ce n'est pas la vie qu'ils désirent mais \emph{l'immortalité}~; et ils sont tellement désireux de l'obtenir qu'ils sacrifieraient jusqu'à leur âme même~! Veux-tu vivre pour \emph{toujours}, Harry~?

--- Oui, et vous aussi, dit Harry. Je veux vivre un jour de plus. Demain, je voudrai encore en vivre un de plus. Par conséquent je veux vivre pour toujours, preuve par induction sur les entiers positifs. Si vous ne voulez pas mourir, cela veut dire que vous voulez vivre pour toujours. Si vous ne voulez pas vivre pour toujours, cela veut dire que vous voulez mourir. Vous devez faire l'un ou l'autre… je ne me fais pas comprendre, n'est-ce pas~?»

Les deux cultures s'observèrent d'un bord à l'autre d'un vaste espace d'incommensurabilité.

«J'ai vécu cent-dix ans», dit le vieux sorcier avec douceur (sortant sa barbe du bol et la secouant pour faire partir la couleur). «Et j'ai vu et fait un grand nombre de choses, que pour la plupart je souhaiterais n'avoir jamais vues ni faites. Et pourtant je ne regrette pas d'être en vie, car regarder mes étudiants grandir est une joie qui n'a pas commencé à m'être pesante. Mais je ne souhaite pas vivre assez pour que cela le devienne~! Que \emph{ferais}-tu avec l'éternité, Harry~?»

Harry prit une profonde inspiration.

«Rencontrer toutes les personnes intéressantes de la Terre, lire tous les livres et ensuite écrire quelque chose d'encore meilleur, célébrer le dixième anniversaire de mon premier petit-enfant sur la Lune, célébrer le centième anniversaire de mon premier petit-petit-petit-enfant vers les anneaux de Saturne, apprendre les règles profondes et finales de la Nature, comprendre la nature de la conscience, découvrir pourquoi la réalité existe en premier lieu, visiter d'autres étoiles, découvrir des aliens, créer des aliens, avoir un rendez-vous avec tout le monde pour une fête de l'autre côté de la Voie Lactée une fois que nous l'aurons entièrement explorée, retrouver tous ceux nés sur l'Ancienne Terre pour voir le Soleil finalement s'éteindre, et je m'inquiétais jusque-là de trouver un moyen d'échapper à cet univers avant qu'il ne se vide de néguentropie, mais j'ai beaucoup plus d'espoir maintenant que j'ai découvert que les prétendues lois de la physique ne sont que des indications optionnelles.

--- Je n'ai pas compris grand-chose de tout cela, dit Dumbledore. Mais il me faut demander si ce sont des choses que tu désires désespérément ou si tu ne les imagines que pour t'imaginer ne pas être un jour fatigué, dans ta fuite permanente de la mort.

--- La vie n'est pas une liste finie de choses que l'on coche avant d'avoir le droit de mourir, dit Harry avec fermeté. C'est la vie, on continue juste de la vivre. Si je ne fais pas ces choses, ce sera parce que j'aurais trouvé quelque chose de mieux.»

Dumbledore soupira. Ses doigts martelèrent une horloge~; alors qu'ils la touchaient, les chiffres se changèrent en une écriture indéchiffrable, et les aiguilles apparurent brièvement à d'autres emplacements.

«Au cas peu probable où je serais autorisé à tarder jusqu'à cent-cinquante ans, dit le vieux sorcier, je ne pense pas que cela me gênerait. Mais deux-cents ans seraient vraiment abuser des bonnes choses.

--- Oui, eh bien», dit Harry, sa voix un peu asséchée lorsqu'il songea à son père et à sa mère et à \emph{leur} espérance de vie si Harry n'y faisait rien, «je soupçonne, professeur, que si vous étiez venu d'une culture où les gens étaient habitués à vivre jusqu'à quatre-cents ans, alors mourir à deux-cents vous semblerait tout autant tragiquement prématuré que de mourir à, disons, \emph{quatre-vingt}.» La voix de Harry se durcit sur ce dernier mot.

«Peut-être, dit paisiblement le vieux sorcier. Je ne souhaiterais pas mourir avant mes amis, ni vivre après qu'ils seront tous partis. Les temps les plus difficiles sont quand ceux qu'on aimait le plus sont déjà partis, et que d'autres vivent pourtant, pour le bien desquels il faut rester…» Les yeux de Dumbledore étaient braqués sur Harry et s'emplissaient de tristesse.

«Ne me pleure pas trop, Harry, lorsque mon heure viendra~; je serai avec ceux qui m'ont longtemps manqué, en direction de notre grande aventure à venir.

--- Oh~!» dit Harry, frappé d'un éclair de compréhension soudaine. «Vous croyez à un \emph{au-delà}. J'avais l'impression que les sorciers n'avaient pas de religion~?»

\later

Pouet. Pouf. Clong.

«\emph{Comment peux-tu ne pas y croire~?}» dit le directeur, l'air complètement sidéré. «\emph{Harry, tu es un sorcier~! Tu as vu des fantômes~!}

--- Des fantômes, dit Harry d'une voix sans timbre. Vous voulez dire ces choses comme les portraits, des mémoires enregistrés et des comportements sans conscience ni vie, accidentellement imprimées dans le matériel environnant par les éclats de magie qui accompagnent la mort violente d'un sorcier…

--- J'ai entendu cette théorie», dit le directeur, sa voix devenant coupante, «répétée par des sorciers qui confondent le cynisme et la sagesse, qui pensent que rabaisser les autres permet de s'élever soi-même. C'est l'une des idées les plus sottes que j'ai entendu en cent-dix ans~! \emph{Oui}, les fantômes n'apprennent pas et n'évoluent pas, parce que ce n'est \emph{pas à ce monde qu'ils appartiennent~!} Les âmes sont censées continuer, il n'y plus de vie pour eux \emph{ici}~! Et hormis les fantômes, qu'en est-il du Voile~? Et de la Pierre de Résurrection~?

--- Très bien», dit Harry, essayant de maintenir une voix calme, «j'écouterai toutes les preuves parce que \emph{c'est ce que les scientifiques font}. Mais d'abord, professeur, laissez-moi vous raconter une petite histoire.» La voix de Harry tremblait. «Vous savez, lorsque je suis venu ici, lorsque je suis descendu du train arrivé de King's Cross, je ne parle pas d'hier mais de septembre dernier, lorsque je suis descendu du train à ce moment-là, professeur, je n'avais jamais vu un fantôme. Je ne \emph{m'attendais pas} à voir des fantômes. Alors lorsque je les ai vus, professeur, j'ai fait quelque chose de vraiment bête. J'ai \emph{tiré des conclusions hâtives}. Je, j'ai pensé qu'il y \emph{avait} un au-delà, j'ai pensé que personne n'était jamais vraiment mort, j'ai pensé que tous ceux que l'espèce humaine avait jamais perdus étaient en fait en bonne santé, j'ai pensé que les mages pouvaient parler à ceux qui avaient disparu, qu'il suffisait du bon sort pour les invoquer, que les sorciers pouvaient \emph{faire} ça, j'ai pensé que je pourrais rencontrer mes parents qui étaient morts pour moi et leur dire que j'étais au courant de leur sacrifice et que je commencerais à les appeler maman et papa…

--- Harry», murmura Dumbledore. De l'eau étincela dans les yeux du vieux sorcier. Il s'approcha d'un pas…

«Et \emph{alors}», cracha Harry, la furie entrant pleinement dans sa voix, la rage froide envers l'univers parce qu'il était ainsi et envers lui-même parce qu'il avait été aussi stupide, «j'ai demandé à Hermione et elle m'a dit qu'ils n'étaient que des \emph{images rémanentes}, brûlées dans la pierre du château par la mort d'un sorcier, comme les silhouettes laissées sur les murs de Hiroshima. Et j'aurais dû le savoir~! J'aurais dû le savoir sans même demander~! Je n'aurais même pas dû y croire pendant trente secondes~! Parce que si les gens avaient une âme, alors les choses comme les lésions cérébrales n'existeraient pas, si votre âme pouvait continuer de parler après que votre cerveau eut disparu, comment une lésion à l'hémisphère gauche pourrait-elle vous enlever votre capacité à parler~? Et le professeur McGonagall, lorsqu'elle m'a raconté comment mes parents étaient morts, elle ne s'était pas comportée comme s'ils étaient juste partis pour un long voyage vers un autre pays, comme s'ils avaient émigré en Australie à l'époque des bateaux à voiles, ce qui est exactement la façon dont les gens se comporteraient s'ils \emph{savaient vraiment} que la mort consistait juste à aller ailleurs, s'ils avaient les preuves concrètes de l'existence d'un au-delà, au lieu de s'inventer des choses pour se consoler, ça changerait \emph{tout}, ça n'aurait pas \emph{d'importance} que tout le monde ait perdu quelqu'un pendant la guerre, ce serait un peu triste mais pas \emph{horrible}~! Et j'avais déjà vu que les gens dans le monde magique ne se comportaient pas comme ça~! Alors j'aurais dû le savoir~! Et c'est à ce moment que j'ai su que mes parents étaient vraiment morts et partis pour toujours et toujours, et qu'il ne restait rien d'eux, et que je n'aurais jamais la possibilité de les rencontrer et, et, et les autres enfants pensaient que je pleurais parce que j'avais \emph{peur des fantômes…}»

Le visage du vieux sorcier était horrifié, il ouvrit la bouche pour parler…

«Alors dites-moi, professeur~! Dites-moi quelles sont les preuves~! Mais \emph{ne vous avisez pas} d'exagérer d'un iota, parce que si vous me donnez de faux espoirs à nouveau, et que je découvre plus tard que vous avez menti et que vous avez déformé un tout petit peu la réalité, je ne vous pardonnerai jamais~! \emph{Qu'est-ce que le voile}~?»

Harry éleva sa main et essuya ses joues alors que les objets de verre du bureau cessaient de vibrer sous l'effet de son dernier cri.

«Le Voile, dit le vieux sorcier d'une voix qui ne tremblait que légèrement, est une grande arcade de pierre, gardée au Département des Mystères~; un passage vers le monde des morts.

--- Et comment quiconque sait-il cela~? dit Harry. Ne me dites pas ce que vous croyez, dites-moi ce que vous avez \emph{vu}~!»

La manifestation physique de la barrière entre les mondes était une grande arcade de pierre, vieille et haute et se finissant en une pointe, avec un voile noir en lambeaux semblables à la surface d'une étendue d'eau étiré entre les pierres~; en permanence parcouru d'ondulations par le passage à sens unique des âmes. Si vous vous teniez face au Voile vous pouviez entendre les voix des morts qui appelaient, appelant toujours de souffles à la limite de la compréhension, devenant plus puissants et plus nombreux si vous restiez et essayez d'entendre et qu'ils essayaient de communiquer~; et si vous écoutiez trop longtemps, vous iriez les rencontrer, et au moment où vous toucheriez le Voile, vous seriez absorbé à l'intérieur et plus personne n'entendrait plus jamais parler de vous.

«Ça n'a même pas l'air d'être une fumisterie \emph{intéressante}», dit Harry, sa voix plus calme maintenant qu'il n'y avait plus rien qui puisse le faire espérer ou l'énerver d'avoir eu ses espoirs brisés. «Quelqu'un a construit une arcade de pierre, a fait une petite surface noire ondulante au milieu qui jette un sortilège de Disparition sur tout ce qui la touche, et l'a ensorcelée pour murmurer aux gens et pour les hypnotiser.

--- Harry…» dit le directeur, commençant à avoir l'air assez inquiet. «Je peux te dire la vérité, mais si tu refuses de l'entendre…»

\emph{Pas intéressant non plus.}

«Qu'est-ce que la Pierre de Résurrection~?

--- Je ne te le dirais pas, dit lentement le directeur, si je ne craignais ce que cette incroyance pourrait te faire… alors écoute, Harry, s'il te plaît, écoute…»

La Pierre de Résurrection était l'une des trois légendaires Reliques de la Mort, comme la Cape de Harry. La Pierre de Résurrection pouvait faire revenir les âmes d'entre les morts -- les ramener dans le monde des vivants, mais pas tels qu'ils étaient. Cadmus Peverell avait utilisé la pierre pour rappeler sa bien-aimée d'entre les morts, mais son cœur était resté parmi eux, et pas dans le monde des vivants. Et cela avait fini par le rendre fou, alors il s'était tué pour être de nouveau à ses côtés…

Harry leva la main avec politesse.

«Oui~? dit le directeur avec réticence.

--- Le test évident pour voir si la Pierre de Résurrection rappelle \emph{vraiment} les morts ou ne projette qu'une image venue de l'esprit de l'utilisateur est de poser une question dont \emph{vous} ne connaissez pas la réponse mais que la personne morte \emph{connaîtrait} et qui pourrait être vérifiée avec certitude dans notre monde. Par exemple, rappeler…»

Puis Harry s'interrompit, parce que \emph{cette fois} il était parvenu à penser avec un temps d'avance sur sa langue, assez vite pour ne \emph{pas} dire le premier nom et le premier test qui lui soit venu à l'esprit.

«… votre femme décédée, et lui demander où elle a laissé sa boucle d'oreille perdue, ou quelque chose comme ça, conclut Harry. Quelqu'un a-t-il fait le moindre test qui ressemble à ceci~?

--- La Pierre de Résurrection est perdue depuis des siècles, Harry», dit doucement le directeur.

Harry haussa les épaules. «Eh bien, je suis un scientifique, et je suis toujours prêt à être convaincu. Si vous croyez \emph{vraiment} que la Pierre de Résurrection rappelle les morts -- alors vous devez croire qu'un test comme celui-ci réussirait, non~? Donc savez-vous quoi que ce soit au sujet de son emplacement~? J'ai déjà reçu \emph{une} Relique de la Mort en des circonstances hautement mystérieuses, et, eh bien, nous savons tous deux comment le rythme du monde fonctionne en ce qui concerne ce genre de choses.»

Dumbledore fixa Harry.

Harry le regarda calmement en retour.

Le vieux sorcier se passa une main sur le front et marmonna~: «C'est de la folie.»

(Sans savoir comment, Harry parvint à s'empêcher de rire).

Et Dumbledore dit à Harry de sortir la Cape d'Invisibilité de sa bourse~; sous les ordres du directeur, Harry regarda la face intérieure de la capuche jusqu'à ce qu'il le voie, faiblement dessiné sur les mailles argentées, d'un cramoisi passé semblable à du sang séché, le symbole des Reliques de la Mort~: un triangle avec un cercle dessiné à l'intérieur et une ligne les divisant tous deux.

«Merci, dit poliment Harry. Je m'assurerai de garder l'œil ouvert pour une pierre ainsi marquée. Avez-vous d'autres preuves~?»

Dumbledore semblait être en proie à une lutte intérieure.« Harry», dit le vieux sorcier, et sa voix s'éleva, «c'est une route dangereuse que tu empruntes, et je ne suis pas certain qu'il soit bon que je dise ce que je m'apprête à dire, mais je \emph{dois} te tirer de ce chemin~! Harry, \emph{comment Voldemort aurait-il pu survivre à la mort de son corps s'il n'avait pas eu une âme~?}»

Et c'est \emph{là} que Harry se rendit compte qu'il y avait exactement une personne qui avait \emph{initialement} dit au professeur McGonagall que le Seigneur des Ténèbres était encore en vie~; et c'était le directeur fou de l'asile qu'on appelait leur école, le directeur fou qui pensait que le monde fonctionnait à base de clichés.

«Bonne question», dit Harry après un peu de débat interne sur la façon de procéder. «Peut-être a-t-il trouvé une façon de dupliquer le pouvoir de la Pierre de Résurrection, seulement il l'a chargée à l'avance avec une copie \emph{complète} de l'état de son cerveau. Ou quelque chose comme ça.» Harry était soudain bien peu sûr d'être en train d'essayer de trouver une explication à quelque chose qui avait \emph{réellement eu lieu}. «En fait, pourriez-vous juste me dire tout ce que vous savez sur la façon dont le Seigneur des Ténèbres a survécu et sur ce qu'il faudrait faire pour le tuer~?» \emph{Si encore il est plus réel que des gros titre du Chicaneur}.

«Tu ne me trompes pas, Harry», dit le vieux sorcier~; son visage avait maintenant l'air ancien et ridé par autre chose que des années. «Je sais la vraie raison pour laquelle tu poses cette question. Non, je ne lis pas dans ton esprit, je n'ai pas à le faire, ton hésitation te trahit~! Tu recherches le secret de l'immortalité du Seigneur des Ténèbres afin de l'utiliser pour toi-même~!

--- Faux~! Je veux le secret de l'immortalité du Seigneur des Ténèbres afin de l'utiliser pour \emph{tout le monde}~!»

\later

Tick, crac, fzzt…

Albus Percival Wulfric Brian Dumbledore était immobile et fixait Harry, la bouche grande ouverte, l'air hébété.

(Harry se décerna un point pour lundi puisqu'il était parvenu à complètement déboussoler quelqu'un avant la fin de la journée).

«Et au cas où ça n'aurait pas été clair, dit Harry, quand je dis \emph{tout le monde} je veux dire tous les Moldus aussi, pas seulement les sorciers.

--- Non,» dit le vieux sorcier, secouant sa tête. Sa voix s'éleva. «Non, non, non~! \emph{C'est de la folie~!}

--- Bwah ha ha~!» dit Harry.

Le visage du vieux sorcier était contracté sous l'effet de l'inquiétude et de la colère.

«Voldemort a volé le livre d'où il a glané son secret~; il n'était pas là lorsque je suis allé le chercher. Mais ce que j'en sais, et ce que je vais t'en dire, c'est que son immortalité est née d'un rituel noir et terrible, plus noir que le noir absolu~! Et c'est Mimi Geignarde, pauvre douce Mimi, qui est morte pour cela~; son immortalité a demandé un sacrifice, elle a demandé un \emph{meurtre…}

--- Eh bien \emph{évidemment} que je ne vais pas populariser une méthode pour atteindre l'immortalité qui requiert que l'on tue des gens~! Ça serait totalement contre-productif~!»

Il y eut une pause étonnée.

Lentement, le visage du vieux sorcier sortit de sa colère et se détendit, mais l'inquiétude était toujours présente.

«Tu n'utiliserais aucun rituel qui requière un sacrifice humain.

--- Je ne sais pas pour qui vous me prenez, \emph{professeur},» dit froidement Harry, sa propre colère montante, «mais n'oublions que c'est \emph{moi} qui souhaite que les gens \emph{vivent}~! Celui qui veut \emph{sauver} tout le monde~! C'est \emph{vous} qui pensez que la mort est géniale et que tout le monde devrait mourir~!

--- Je suis perdu, Harry,» dit le vieux sorcier. Ses pieds commencèrent de nouveau à parcourir l'étrange bureau. «Je ne sais que dire.» Il ramassa une sphère de cristal qui semblait contenir une main entourée de flammes et l'observa d'un air triste. «Seulement que je suis bien mécompris… je ne \emph{veux} pas que tout le monde meure, Harry~!

--- Vous ne voulez juste pas que tout le monde soit immortel», dit Harry avec une grande ironie dans la voix. Il semblait que les tautologies logiques élémentaires telles que \emph{Pour tout x: Meurt(x) = N'existe pas x: Ne meure pas(x)} étaient au-delà des capacités de raisonnement du sorcier le plus puissant du monde.

Le vieux sorcier hocha la tête. «J'ai moins peur qu'avant, mais je suis toujours grandement inquiet pour toi, Harry,» dit-il doucement. Sa main, légèrement flétrie par le temps mais toujours forte, replaça fermement la sphère de cristal sur son socle. «Car la peur de la mort est une chose amère, une maladie de l'âme qui tord et déforme les gens. Voldemort n'est pas le seul mage noir à avoir parcouru cette route désolée, même si je crains qu'il ne soit allé plus loin que quiconque avant lui.

--- Et vous pensez que \emph{vous} n'avez pas peur de la mort~?» dit Harry, n'essayant même pas de masquer l'incrédulité présente dans sa voix.

Le visage du vieux sorcier était paisible. «Je ne suis pas parfait, Harry, mais je pense avoir accepté ma mort comme une partie de mon être.

--- Oh oh, dit Harry. Vous voyez, il y a cette petite chose nommée \emph{dissonance cognitive}, ou en français plus simple, le \emph{dépit}. Si on frappait les gens sur la tête avec des matraques une fois par mois, et que personne ne pouvait rien y faire, il y aurait bientôt toutes sortes de philosophes, \emph{prétendant être sages}, comme vous le dites, qui trouveraient toutes sortes \emph{d'incroyables avantages} à être frappé sur la tête par une matraque une fois par mois. Par exemple, ça vous rend plus fort, ou ça vous rend plus heureux les jours où on ne vous frappe \emph{pas} avec une matraque. Mais si vous alliez voir quelqu'un qui ne se faisait \emph{pas} frapper et que vous lui demandiez s'il voulait \emph{commencer}, en échange de ces \emph{incroyables avantages}, il dirait non. Et si vous ne \emph{deviez pas} mourir, si vous veniez d'un endroit où personne n'avait jamais entendu \emph{parler} de la mort, et si je vous suggérais que ce serait une \emph{idée incroyablement fantastique} que les gens deviennent ridés et vieux et finissent par ne plus exister, allons, vous me feriez tout de suite expédier à l'asile de fous~! Alors pourquoi quiconque pourrait-il penser quelque chose d'aussi stupide que “la mort est quelque chose de \emph{bien}”~? Parce que vous en avez peur, parce que vous ne voulez pas \emph{vraiment} mourir, et que cette pensée vous fait si mal que vous devez la rationaliser, que vous devez faire quelque chose pour atténuer la douleur, pour ne plus avoir à y penser…

--- Non, Harry», dit le vieux sorcier. Son visage était paisible, sa main parcourait un bac d'eau éclairé qui émettait de petits carillons au passage de ses doigts. «Même si je peux comprendre pourquoi tu le penses.

--- Vous voulez comprendre le Seigneur des Ténèbres~?» dit Harry, sa voix maintenant dure et sinistre. «Alors regardez la partie de vous qui fuit non pas la mort mais la \emph{peur} de la mort, la partie qui trouve cette peur si insupportable qu'elle étreindra la Mort comme une amie et se rapprochera d'elle, qu'elle essaiera de ne faire qu'une avec la nuit afin de pouvoir se dire maîtresse de l'abysse. Vous avez pris l'un des maux les plus terribles et avez dit que c'était un \emph{bien}~! Avec seulement une légère torsion de cette partie de vous-même vous tueriez des innocents et appelleriez cela de l'amitié. Si vous pouvez dire que la mort est meilleure que la vie alors vous pouvez tordre votre compas moral pour qu'il pointe \emph{n'importe où}…

--- Je pense,» dit Dumbledore, éjectant quelques gouttes d'eau de sa main en la secouant, au son de petites clochettes tintantes, «que tu comprends \emph{très} bien les mages noirs, sans en être encore un toi-même.» C'était dit avec un parfait sérieux et sans accusation. «Mais ta compréhension de \emph{moi} est cruellement imparfaite, j'en ai peur.» Le vieux sorcier souriait à présent, et il y avait un rire sympathique dans sa voix.

Harry essayait de ne pas devenir plus froid qu'il ne l'était déjà~; d'une source inconnue jaillissait une étincelante furie de ressentiment qui se déversait dans son esprit, contre la condescendance de Dumbledore et contre tout le rire que les sages vieux idiots avaient toujours utilisé en lieu et place d'un argument. «C'est drôle, vous savez, je pensais que ce serait à ce point qu'il serait impossible de dialoguer avec Drago Malfoy, et au lieu de cela, dans son innocence enfantine, il a été cent fois plus fort que vous.»

Un air d'incompréhension passa sur le visage du vieux sorcier.

«Que veux-tu dire~?

--- Je veux dire», dit Harry, la voix mordante, «que Drago \emph{prenait ses propres croyances au sérieux} et \emph{considérait} mes mots au lieu de simplement les \emph{jeter par la fenêtre} en souriant avec une aimable supériorité. Vous êtes si vieux et sage que vous ne pouvez même pas \emph{remarquer} ce que je dis~! Pas comprendre, \emph{remarquer}~!

--- Je t'\emph{ai} écouté, Harry», dit Dumbledore, l'air maintenant plus solennel, «mais écouter n'est pas toujours être d'accord. Désaccords mis à part, que crois-tu que je ne comprends pas~?»

\emph{Que si vous croyiez vraiment à un au-delà, vous iriez à Sainte Mangouste et tueriez les parents de Neville, Alice et Frank Londubat, pour qu'ils puissent passer à leur} grande aventure à venir \emph{au lieu de les laisser s'éterniser ici, endommagés…}

Harry parvint à peine, \emph{à peine}, à s'empêcher de le dire à voix haute.

«Très bien, dit froidement Harry. Je répondrai alors à votre question initiale. Vous avez demandé pourquoi les mages noirs ont peur de la mort. Prétendez, professeur, que vous croyez \emph{vraiment} aux âmes. Prétendez que chacun soit capable de vérifier l'existence des âmes à n'importe quel moment, prétendez que personne ne pleure aux funérailles, parce qu'ils \emph{sauraient} que ceux qu'ils aiment sont toujours en vie. Maintenant, pouvez-vous imaginer la \emph{destruction} d'une âme~? Détruite en morceaux pour que rien n'en reste pour vivre sa grande aventure à venir~? Pouvez-vous imaginer la chose horrible que ce serait, le pire crime jamais commis dans l'Histoire de l'univers, contre laquelle vous seriez prêt à tout, pour empêcher que cela ne se produise ne serait-ce qu'une seule fois~? Parce que \emph{c'est cela}, la Mort -- l'annihilation d'une âme~!»

Le vieux sorcier le regardait avec tristesse.

«Je suppose que je \emph{comprends} à présent, dit-il avec douceur.

--- Oh~? dit Harry. Vous comprenez quoi~?

--- Voldemort, dit le vieux sorcier. Je le comprends enfin. Parce pour croire que le monde est réellement ainsi, il faut croire qu'il est dénué de justice, qu'il est parcouru de ténèbres jusqu'à son cœur. Je t'ai demandé pourquoi il était devenu un monstre, et tu n'as pu me donner une raison. Et si je pouvais \emph{lui} demander, je présume que sa réponse serait~: Pourquoi pas~?»

\later

Ils se tinrent là, se fixant l'un-l'autre, le vieux sorcier dans sa robe, et le jeune garçon avec la cicatrice en forme d'éclair sur son front.

«Dis-moi, Harry, dit le vieux sorcier, vas-\emph{tu} devenir un monstre~?

--- Non», dit le garçon, une certitude d'acier dans la voix.

«Pourquoi pas~?» dit le vieux sorcier.

Le jeune garçon se redressa, son menton fièrement levé, et il dit~:

«Il n'y a pas de justice dans les lois de la Nature, professeur, pas de terme pour l'équité dans les équations du mouvement. Mais l'univers n'est ni mauvais ni bon, il ne s'en soucie tout simplement pas. Les étoiles ne s'en soucient pas, ni le soleil, ni le ciel. Mais ils n'ont pas à le faire~! \emph{Nous} nous en soucions~! Il y \emph{a} de la lumière dans le monde, et c'est \emph{nous}~!

--- Je me demande ce qu'il adviendra de toi, Harry,» dit le vieux sorcier. Sa voix était douce, mêlée d'un étrange émerveillement et d'un étrange regret. «Assez pour me faire souhaiter de vivre, juste pour le voir.»

Avec une lourde ironie, le garçon s'inclina devant lui et s'en fut~; et la porte en chêne se referma brutalement derrière lui d'un bruit sourd.
%  LocalWords:  histle Bzzzt Glorp Pffft Leliel bzzzt glorp fzzzt Bwa
%  LocalWords:  pitchest
