\chapter{L'illusion de la planification}

\lettrine{C}{\emph{ertains}} enfants auraient attendu d'avoir \emph{terminer} leur premier voyage au Chemin de Traverse.

\emph{Certains} enfants auraient attendu d'avoir \emph{terminer} leur premier voyage au Chemin de Traverse.

<<~Sac de l'élément 79,~>> dit Harry, et il retira sa main vide de la bourse en peau de Moke.

La plupart des enfants auraient au moins attendu d'avoir leur \emph{baguette magique}.

<<~Sac d'\emph{okane},~>> dit Harry. Le lourd sac d'or apparut dans sa main.

Harry sortit le sac, puis le plongea à nouveau dans la bourse. Il sortit sa main, la remit à l'intérieur, et dit~: <<~Sac de gages d'échange économique.~>> Cette fois-ci sa main ressortit vide.

Harry Potter avait mis la main sur au moins un objet magique. Pourquoi attendre~?

<<~Professeur McGonagall, dit Harry à la sorcière perplexe qui marchait à ses côtés, pourriez-vous me donner deux mots, un qui signifie or, et un autre signifiant autre chose n'étant pas de l'argent, le tout dans une langue que je connais pas~? Mais ne me dites pas lequel est lequel.

---\emph{Ahava} et \emph{zahav}, dit McGonagall. C'est de l'Hébreu, et l'autre mot veut dire amour.

--- Merci, Professeur. Sac d'\emph{ahava}.~>> Vide.

<<~Sac de \emph{zahav}.~>> Et il apparut dans sa main.

<<~Zahav veut dire or~?~>> s'enquit Harry, et McGonagall hocha la tête.

Harry contempla les données expérimentales qu'il avait recueillies. C'était un effort des plus bruts et des plus préliminaires, mais c'était suffisant pour soutenir au moins une conclusion~:

<<~\emph{Aaaaaarrrgh ça n'a aucun sens~!}~>>

La sorcière à ses côtés souleva un noble sourcil. <<~Des problèmes, M. Potter~?

--- Je viens de falsifier chacune des hypothèses que j'avais~! Comment la bourse peut-elle savoir que “sac de 115 Gallions” est valide, mais pas “sac de 90 plus 25 Gallions”~? Elle peut \emph{compter}, mais elle ne peut pas \emph{additionner}~? Elle peut comprendre les noms, mais pas les syntagmes nominaux de même sens~? La personne qui l'a créée ne parlait probablement pas Japonais et \emph{je} ne parle pas Hébreux, donc ça n'utilise pas \emph{son} savoir ni \emph{mon} savoir~-~>> Harry agita une main avec impuissance. <<~Les règles paraissent \emph{en gros} cohérentes, mais elles ne \emph{veulent rien dire}~! Et je ne vais même pas commencer à m'interroger sur la façon dont une \emph{bourse} peut être équipée d'une reconnaissance vocale et d'une compréhension du langage naturel, alors qu'après trente-cinq ans de dur labeur les meilleurs programmeurs en Intelligence Artificielle ne peuvent faire réaliser cette prouesse aux superordinateurs les plus rapides,~>> Harry haleta à la recherche d'oxygène, <<~mais \emph{qu'}est-ce qui se \emph{passe}~?

--- Magie~>>, dit le professeur McGonagall. Elle haussa les épaules.

<<~C'est juste un \emph{mot}~! Même après m'avoir dit ça, je ne peux pas faire de nouvelles prédictions. C'est exactement comme de dire “phlogiston” ou “élan vital” ou “émergence” ou “complexité”~!~>>

Le Professeur McGonagall rit à haute voix. <<~Mais c'\emph{est} de la magie, M. Potter.~>>

Harry s'effondra un peu. <<~Avec tout mon respect, Professeur McGonagall, je ne suis pas tout à fait sûr que vous compreniez que ce j'essaie de faire ici.

--- Avec tout mon respect, M. Potter, je suis tout à fait sûre de ne pas le comprendre. À moins que~- c'est juste une supposition, dites-vous bien~- vous ne soyez en train d'essayer de conquérir le monde~?

--- Non~! Je veux dire oui~- enfin, \emph{non}~!

--- Je pense que je devrais probablement être alarmée par le fait que vous avez quelque difficulté à répondre à cette question.~>>

Harry se remémora sombrement la Conférence de Dartmouth sur l'Intelligence Artificielle de 1956. Ça avait été la première conférence jamais organisée sur ce sujet, celle qui avait créé l'expression “Intelligence Artificielle”. Ils avaient identifié les problèmes clés, tels que faire en sorte que les ordinateurs comprennent le langage, apprennent, et s'améliorent eux-mêmes. Ils avaient suggéré, avec un parfait sérieux, que des progrès significatifs pourraient être accomplis par dix scientifiques travaillant ensemble pendant deux mois sur ces problèmes.

\emph{Non. Relève la tête. Tu} commences \emph{juste à démêler les secrets de la magie. Tu ne} sais \emph{pas vraiment si ça va être difficile à faire en deux mois.}

<<~Et vous n'avez \emph{vraiment} pas entendu parler d'autres sorciers posant ce genre de questions ou faisant ce genre d'expériences scientifiques~?~>> demanda à nouveau Harry. Ça lui semblait tellement \emph{évident}.

Mais après tout, il avait fallu attendre plus de deux cents ans \emph{après} l'invention de la méthode scientifique pour qu'un scientifique Moldu pense à étudier de façon systématique ce qu'un \emph{humain de quatre ans} pouvait et ne pouvait pas comprendre. Ils auraient pu découvrir ça au dix-huitième siècle, mais personne n'avait jamais pensé à regarder avant le vingtième. Donc vous ne pouviez pas vraiment blâmer le monde magique, qui était bien plus petit, s'ils n'avaient pas encore étudié le sort de Récupération.

McGonagall, après avoir pincé ses lèvres pendant un moment, haussa les épaules. <<~Je ne suis toujours pas certaine de ce que vous voulez dire par “expérience scientifique”, M. Potter. Comme je l'ai dit, j'ai vu des étudiants nés-Moldus essayer de faire fonctionner la science Moldue à Poudlard, et les gens inventent de nouveaux Charmes et de nouvelles Potions chaque année.~>>

Harry secoua la tête. <<~La technologie et la science ne sont pas du tout la même chose. Et essayer de faire quelque chose de plein de façons différentes n'est pas du tout semblable à expérimenter pour comprendre les règles.~>> Il y avait beaucoup de gens qui avaient essayé d'inventer des machines volantes en essayant plein de choses-à-ailes, mais seuls les frères Wright avaient construit un tunnel à vent… <<~Hmm, combien d'enfants éduqués-Moldus \emph{acceptez-vous} à Poudlard chaque année~?~>>

McGonagall eut l'air pensive pour un moment. <<~Environ dix~?~>>

Harry fit un faux pas et failli se faire un croc-en-jambe. <<~\emph{Dix}~?~>>

Le monde Moldu avait une population de six milliards en augmentation. Si on vous choisissait parmi un million de personnes, alors il y en avait douze comme vous à New York et mille de plus en Chine. Il était inévitable que le monde Moldu produise \emph{quelques} enfants de onze ans capables de résoudre des équations~- Harry savait qu'il n'était pas le seul. Il avait rencontré d'autres prodiges aux compétitions de mathématiques. À vrai dire il avait été complètement écrasé par des concurrents qui avaient probablement passé littéralement \emph{toutes leurs journées} à pratiquer des problèmes de mathématiques et qui n'avaient \emph{jamais} lu de livre de science-fiction et qui allaient \emph{complètement} craquer avant leur \emph{puberté} et ne feraient \emph{jamais} rien de leur vie future parce qu'ils avaient simplement utilisé des techniques \emph{connues} au lieu d'apprendre à penser de façon \emph{créative}. (Harry était du genre mauvais perdant)

Mais… dans le monde magique…

Dix enfants éduqués-Moldus par an qui finissaient leur éducation Moldue à l'âge de onze ans~? Et McGonagall n'était peut-être pas objective, mais elle avait prétendu que Poudlard était la plus grande et la plus éminente des écoles de magie du monde… et son cursus n'allait que jusqu'à l'âge de dix-sept ans.

Le Professeur McGonagall connaissait sans aucun doute les plus petits détails de la façon dont les gens se transformaient en chat. Mais semblait n'avoir littéralement \emph{jamais} entendu parler de la méthode scientifique. Pour elle, c'était juste de la magie Moldue. Et elle ne semblait même pas \emph{curieuse} des secrets qui pouvaient se cacher derrière la compréhension du langage naturel que possédait le sort de Récupération.

Ce qui n'offrait que deux possibilités, vraiment.

Possibilité une~: La magie était si incroyablement opaque, convolutée et impénétrable que même si les sorciers et sorcières avaient fait de leur mieux pour la comprendre, ils n'avaient fait aucun progrès et avaient fini par laisser tomber~; et Harry ne ferait pas mieux.

\emph{Ou…}

Harry se craqua les doigts avec détermination, mais ils ne firent qu'un petit craquement discret et pas un écho menaçant qui aurait rebondi sur les murs du Chemin de Traverse.

Possibilité deux~: il allait conquérir le monde.

À la longue. Peut-être pas tout de suite.

Ce genre de choses prenait \emph{vraiment} plus de deux mois. La science Moldue ne s'était pas rendue sur la lune une semaine après Galilée.

Mais Harry ne pouvait arrêter l'immense sourire qui s'était tant étendu sur ses joues qu'elles commençaient à lui faire mal.

Il avait toujours eu peur de finir comme un de ces enfants prodiges qui ne faisaient rien de leur vie et passaient le reste de celles-ci à raconter à quel point ils étaient cools à dix ans. Mais cela dit la plupart des génies adultes ne faisaient rien de leur vie. Il y avait probablement mille personnes aussi intelligentes qu'Einstein pour chaque Einstein de l'Histoire. Mais ils n'avaient pas mis la main sur la seule chose dont vous aviez absolument besoin pour parvenir à la grandeur. Ils n'avaient jamais trouvé un problème d'importance.

\emph{Tu es à moi à présent}, pensa Harry à l'intention des murs du Chemin de Traverse, et de tous les magasins et leurs objets, et de tous les tenanciers et des clients, et de toutes les terres et de tous les habitants de l'Angleterre magique, et du vaste monde magique, et de l'univers entier dont les scientifiques Moldus comprenaient bien moins que ce qu'ils croyaient. \emph{Moi, Harry James Potter-Evans-Verres, revendique ce territoire au nom de la Science}.

Les éclairs et le tonnerre ne brillèrent et ne grondèrent pas dans le ciel sans nuage.

<<~Pourquoi souriez-vous~?, s'enquit McGonagall avec prudence et lassitude.

--- Je me demande s'il existe un sort permettant de faire jaillir des éclairs en arrière-plan à chaque fois que je prends une résolution de mauvaise augure,~>> expliqua Harry. Il était en train de soigneusement mémoriser les mots exacts de sa résolution de mauvaise augure afin que les livres d'histoire du futur ne se trompent pas.

<<~J'ai comme un lointain sentiment me disant que je devrais faire quelque chose à ce sujet, soupira McGonagall.

--- Ignorez-le, et ça partira. Ooh, joli~!~>> Harry mit ses pensées de conquête mondiale en attente et alla droit jusqu'à un magasin à la devanture ouverte, et le professeur McGonagall suivit.

\later

Harry avait maintenant acheté ses ingrédients de potions, un chaudron, et, oh, quelques petites choses supplémentaires. Des objets qu'il semblait intelligent de transporter dans le Sac Conteneur de Harry (aussi connu sous le nom de Super Bourse en Peau de Moke QX31 avec Charme d'Extension Indétectable, Charme de Récupération, et Ouverture Élargissante). Des achats intelligents et raisonnables.

Harry ne comprenait honnêtement pas pourquoi McGonagall avait l'air si \emph{méfiante}.

Harry était pour le moment dans une boutique suffisamment luxueuse pour être exposée dans la rue principale et sinueuse du Chemin de Traverse. La boutique avait une devanture ouverte où la marchandise était disposée sur des étals de bois inclinés, gardés seulement par de légères lueurs grisâtres et une jeune vendeuse vêtue d’une version fortement raccourcie de robe de sorcière normale, révélant ses genoux et ses coudes.

Harry examina l'équivalent magique d'un kit de premier soins, le Pack de Soins d'Urgence Plus. Il avait~: deux garrots auto-serrants. Une Potion de Stabilisation qui ralentissait la perte de sang et empêchait les chocs. Une seringue de quelque chose qui ressemblait à du feu liquide et était supposé considérablement réduire la circulation sanguine dans la zone traitée tout en continuant à oxygéner le sang pendant trente minutes maximum, si jamais vous aviez besoin d'empêcher un poison de se répandre dans le corps. Un tissu blanc qui pouvait être enroulé autour d'une partie du corps pour temporairement diminuer la douleur. Plus une quantité d'autres objets que Harry ne comprenait absolument pas, comme le “Traitement pour l'Exposition aux Détraqueurs”, qui ressemblait à et sentait comme du chocolat ordinaire. Ou le “contre-BaffleSnaffle”, qui ressemblait à un petit oeuf frémissant et portait une affichette montrant comment l'enfoncer dans la narine de quelqu'un.

<<~Un achat obligatoire pour cinq Gallions, qu'en dites-vous~?~>> dit Harry à McGonagall, et la jeune vendeuse, non loin, hocha la tête avec enthousiasme.

Harry s'était attendu à ce que McGonagall fasse une remarque approbatrice sur sa prudence et préparation.

Ce qu'il reçut à la place ne pouvait être décrit que comme le mauvais œil.

<<~Et \emph{pourquoi} donc, dit le professeur McGonagall avec une lourde note de scepticisme dans la voix, vous attendriez-vous à avoir \emph{besoin} d'un kit de soin, jeune homme~?~>> (Après le malheureux incident au magasin de potions, McGonagall essayait d'éviter de dire <<~M. Potter~>> lorsque quelqu'un se trouvait non loin.)

La bouche de Harry s'ouvrit puis se ferma. 

<<~Je ne m'\emph{attends} pas à en avoir besoin~! C'est juste au cas où~!

--- Juste au cas où \emph{quoi}~?~>>

Les yeux de Harry s'agrandirent. <<~Vous pensez que je \emph{prépare} quelque chose de dangereux, que c'est pour \emph{ça} que je veux un kit médical~?~>>

L'air de soupçon sinistre et d'incrédulité ironique qu'arborait le visage de McGonagall suffit à lui répondre.

<<~Grand Scott~!~>> dit Harry (c'était une expression qu'il avait apprise grâce au scientifique fou Doc Brown de \emph{Retour vers le Futur}.) <<~Pensiez-vous à ça quand j'ai acheté la Potion de Chute-sur-Plumes, la Branchiflore, et la bouteille de pilules de Nourriture et d'Eau~?

---Oui.~>>

Harry secoua la tête avec stupéfaction.

<<~Et quelle sorte de plan pensez-vous que j'ai \emph{mis} \emph{en route}~?

--- Je ne sais pas, dit tristement McGonagall, mais ça se termine soit avec vous délivrant une tonne d'argent à Gringotts, soit en domination mondiale.

--- La domination mondiale est une expression si laide. Je préfère l'appeler optimisation mondiale.~>>

Cela échoua à rassurer le professeur McGonagall, qui lui donnait toujours le Regard de la Mort.

<<~Wow, dit Harry se rendant compte qu'elle était sérieuse. Vous le croyez vraiment. Vous pensez vraiment que je prévois de faire quelque chose de dangereux.

---Oui.

--- Comme si c'était la \emph{seule} raison pour laquelle qui que ce soit achèterait jamais un kit de premiers soins~? Ne le prenez pas mal, Professeur McGonagall, mais \emph{de quelle sorte d'enfants fous avez-vous l'habitude de vous occuper}~?

--- Des Gryffondors,~>> cracha le Professeur McGonagall, le mot transportait un convoi d'amertume et de désespoir tel qu'on aurait dit une malédiction éternelle jetée sur tout l'héroïsme et la vivacité de la jeunesse.

<<~Directrice Adjointe Professeur McGonagall~>> dit Harry, posant fermement ses mains sur ses hanches. <<~Je ne vais pas aller à Gryffondor~-~>>

À cet instant, McGonagall glissa quelque chose au sujet du fait que s'il y \emph{allait}, elle découvrirait comment on s'y prend pour tuer des chapeaux, une remarque étrange que Harry laissa passer sans commentaire, bien que la vendeuse sembla être prise d'une soudaine crise de toux.

<<~ Je vais aller à Serdaigle. Et si vous pensez vraiment que je prévois de faire quelque chose de dangereux, alors, avec tout mon respect, vous ne me comprenez pas \emph{du tout}. Je n'\emph{aime} pas le danger, ça me fait \emph{peur}. Je suis \emph{prudent}. Je suis \emph{précautionneux}. Je me prépare à des \emph{contingences imprévues}. Comme mes parents me chantonnaient~: \emph{Soyez prêts~! C'est la chanson de marche des Scouts, soyez prêts~! Comme on marche dans la vie~! Pas nerveux, pas énervé, pas effrayé, soyez prêt~!}~>>

(Les parents de Harry n'avaient de fait chanté que \emph{ces vers-là} de la chanson de Tom Lehrer, et Harry vivait dans l'heureuse ignorance du reste de cette chanson.)

La posture de McGonagall s'était légèrement adoucie~- surtout quand Harry lui avait rappelé qu'il se rendait à Serdaigle.

<<~À quelle sorte de \emph{contingence} imaginez-vous que ce kit vous prépare, \emph{jeune homme}~?

--- L'un de mes camarades de classe se fait mordre par un horrible monstre, et alors que je fouille frénétiquement dans ma bourse en peau de Moke à la recherche de quelque chose qui pourrait l'aider, elle me regarde avec tristesse, et dans son dernier souffle me dit~: “\emph{Pourquoi n'étais-tu pas prêt~?}” Puis elle meurt, et je sais, alors que ses yeux se ferment, qu'elle ne me pardonnera jamais~-~>>

Harry entendit la vendeuse manquer d'air~; il leva les yeux et la vit, les yeux braqués sur lui, ses lèvres fermement serrées. Puis la jeune femme fit un demi-tour et fuit vers le fond du magasin.

\emph{Quoi…?}

Le Professeur McGonagall prit les mains de Harry dans les siennes, gentiment, mais très fermement, et tira Harry jusqu'à la rue principale du Chemin de Traverse, le menant dans une ruelle située entre deux magasins qui était pavée de briques sales et se terminait par un cul-de-sac de terre noire et compacte.

La grande sorcière pointa sa baguette en direction de la rue principale et dit~: <<~\emph{Sourdinam}~>>, et un écran de silence s'abattit autour d'eux, bloquant les bruits de la rue.

\emph{Qu'est-ce que j'ai fait de mal…}

Puis la sorcière se tourna et envoya à Harry un regard glacé pleine puissance.

<<~Je vous serai reconnaissante de vous rappeler, M. Potter, qu'il y avait une \emph{guerre} dans l'Angleterre magique il n'y a pas dix ans et que \emph{tout le monde} ici a perdu quelqu'un et que parler d'amis mourants dans vos bras \emph{n'est pas, quelque chose, qui se fait~!}

--- Je, je ne voulais pas~-~>> La conclusion tomba comme une pierre dans l'imagination exceptionnellement visuelle de Harry. La guerre avait pris fin dix ans auparavant, donc cette fille avait eu huit ou neuf ans tout au plus quand, quand, <<~Je suis désolé, je ne voulais pas…~>> Harry s'étrangla, et se détourna du regard froid de McGonagall, mais il y avait un mur de terre en travers de son chemin et il n'avait pas encore sa baguette magique. <<~Je suis désolé, je suis désolé, je suis \emph{désolé}~!~>>

Un lourd soupir s'éleva de derrière lui. <<~Je sais que vous l'êtes, M. Potter.~>>

Harry osa jeter un coup d'œil derrière lui. La colère avait quitté le visage du Professeur McGonagall. <<~Je suis désolé,~>> dit à nouveau Harry, se sentant l'être le plus misérable du monde. <<~Je n'aurais pas dû dire ça. Est-ce que quelque chose v-~>> et Harry ferma ses lèvres et se plaqua la main sur la bouche pour faire bonne mesure.

Le visage de McGonagall devint un peu plus triste. <<~Vous \emph{devez} apprendre à penser avant de parler, M. Potter. Sinon vous traverserez l'existence avec bien peu d'amis. Ça a été le sort de bien des Serdaigles, et j'espère que ce ne sera pas le vôtre.~>>

Harry voulait juste s'enfuir en courant. Il voulait faire jaillir une baguette et effacer toute l'histoire de la mémoire de McGonagall, être à nouveau avec elle devant le magasin, \emph{faire que ça n'ait pas eu lieu}.

<<~Mais pour répondre à votre question, dit McGonagall, non, rien de \emph{tel} ne m'est jamais arrivé.~>> Son visage prit une étrange expression. <<~J'ai certainement vu un ami exhaler son dernier souffre, une fois ou deux, voir plus. Mais aucun d'entre ne m'a jamais maudit alors qu'il trépassait, et je n'ai jamais pensé qu'ils ne me pardonneraient pas. \emph{Par Merlin, Harry Potter, qu'est-ce qui a pu vous posséder pour vous pousser à dire une chose pareille~?} Pourquoi même y \emph{penseriez}-vous~?~>>

Des larmes coulaient le long des joues de Harry. <<~Je suis désolé, je n'aurais jamais rien dû dire, je suis désolé~-~>>

McGonagall prit une courte inspiration. <<~Je \emph{sais} que vous êtes désolé. Ce que je ne comprends pas, c'est pourquoi un enfant de onze ans \emph{pense} à ces choses-là. Avez-vous vraiment décidé d'acheter un kit de soin à cinq Gallions pour le transporter dans une bourse à quinze Gallions parce que vous êtes convaincu qu'autrement vos camarades de classes vont vous \emph{maudire en mourant}~?

--- Je, je,  je, Harry avala sa salive. C'est juste que j'essaie toujours d'imaginer la pire chose qui puisse arriver,~>> et peut-être qu'il avait aussi voulu blaguer un peu, mais il aurait plutôt mordu sa langue que de dire ça maintenant.

<<~\emph{Pourquoi}~?

--- Pour que je puisse empêcher que ça ait lieu~!

---M. Potter…~>> la voix de McGonagall s'effaça. Puis elle soupira, et s'accroupit à côté de lui. <<~M. Potter, dit-elle gentiment cette fois, ce n'est pas votre responsabilité que de prendre soin des étudiants de Poudlard. C'est la mienne. Je ne laisserai rien vous arriver, ni à qui que ce soit d'autre. Poudlard est l'endroit le plus sûr de toute l'Angleterre magique, et madame Pomfrey a un cabinet de guérisseur complet. Vous n'avez pas besoin d'un kit de soin.

--- Mais \emph{si}~! éclata Harry. \emph{Aucun endroit} n'est parfaitement sûr~! Et si mes parents avaient une crise cardiaque ou un accident quand je rentrais à Noël~- Madame Pomfrey ne serait pas là, j'aurais besoin d'avoir mon propre kit de soin~-

--- Par Merlin, \emph{qu'est-ce qui}…~>> dit McGonagall. Elle se leva, et regarda Harry avec une expression divisée entre la préoccupation et l'irritation. <<~Il n'y a aucun besoin de penser à des choses aussi terribles M. Potter~!~>>

Lorsqu'il entendit ça, l'expression de Harry devint amère. <<~Si, il y en \emph{a}~! Si vous n'y pensez pas, vous vous faites mal, ou vous faites mal aux autres~!~>>

Le Professeur McGonagall ouvrit sa bouche, puis la ferma. Elle frotta l'arête de son nez avec un air pensif. 

<<~M. Potter… si je vous offrais de rester silencieuse et de vous écouter un moment… y a-t-il quelque chose dont vous voudriez me parler~?

---À propos de quoi~?

--- À propos de la raison pour laquelle vous êtes convaincu que vous devez toujours être sur vos gardes contre les terribles choses qui pourraient vous arriver.~>>

Harry la fixa, perplexe. C'était un axiome qui allait de soi. <<~Eh bien…~>> dit lentement Harry. Il essaya d'organiser ses pensées. Comment \emph{pouvait}-il l'expliquer de lui-même à McGonagall, si elle ne connaissait même pas les bases~?

<<~Les chercheurs Moldus ont découvert que les gens sont toujours très optimistes, par exemple ils disent que quelque chose va prendre deux jours et ça en prend dix, ou ils disent que ça va prendre deux mois et ça prend trente-cinq ans. Par exemple, ils ont demandé à des étudiants les durées maximums avant lesquelles ils étaient sûrs à 50~\%, 75~\% et 99~\% qu'ils auraient terminé leurs devoirs, et seuls 13~\%, 19~\% et 45~\% des étudiants ont terminé dans les temps qu'ils avaient donnés. Et les chercheurs se sont rendu compte que c'était parce que lorsqu'on demande aux gens leur estimation dans le meilleur des cas possible, si tout allait le mieux possible, et leur estimation dans le cas moyen, si tout se passe normalement, on reçoit des réponses qui sont statistiquement indistinguables. Vous voyez, si vous demandez à quelqu'un ce à quoi il s'attend dans le cas \emph{normal}, il visualise ce qui semble être le plus probable à chaque étape du parcours~- c'est-à-dire, que tout se déroule parfaitement, sans erreurs ou surprises. Mais en réalité, puisque plus de la moitié des étudiants n'ont pas fini dans le temps où ils étaient certains à 99~\% d'avoir fini, ça veut dire que la réalité est généralement légèrement pire que le “pire des cas possibles”. C'est ce qu'on appelle l'\emph{illusion de la planification}, et la meilleure façon de la dissiper est de vous demander combien de temps vous avez mis à faire quelque chose la dernière fois que vous l'avez essayée. C'est ce qu'on appelle utiliser le point de vue extérieur au lieu du point de vue intérieur. Mais quand vous faites quelque chose de nouveau, et que vous ne pouvez pas utiliser cette méthode, vous devez juste être vraiment, vraiment, vraiment pessimiste. En gros, tellement pessimiste que la réalité finit par être \emph{meilleure} que ce à quoi vous vous attendiez environ aussi souvent qu'elle finit par être pire. C'est vraiment \emph{très dur} d'être \emph{tellement} pessimiste qu'on se retrouve avec de bonnes chances de \emph{sous-estimer} la réalité. Par exemple si je faisais un gros effort pour être morbide et que j'imaginais qu'un de mes camarades se faisait mordre, mais que ce qui se passait réellement c'était que les Mangemorts survivants attaquaient l'école entière pour m'avoir. Mais le bon côté des choses c'est que~-

--- Arrêtez~>> dit McGonagall.

Harry s'arrêta. Il avait été sur le point de remarquer qu'au moins ils savaient que le Seigneur des Ténèbres n'attaquerait pas puisqu'il était mort.

<<~Je pense ne pas avoir été claire, dit McGonagall avec précaution. Y a-t-il quoi que ce soit qui vous soit arrivé \emph{à vous} et qui vous fasse peur~?

--- Ce qui m'est arrivé ne constitue que des éléments anecdotiques, lui expliqua Harry. Ça n'a pas le même poids qu'un article de journal scientifique, répliqué, évalué par des pairs, au sujet d'une étude contrôlée et répartie au hasard avec beaucoup de sujets, une grande amplitude d'effet et statistiquement significative.~>>

McGonagall pinça l'arête de son nez, inhala et exhala. 

<<~Je voudrais tout de même que vous m'en parliez, dit-elle.

--- Euh…~>> dit Harry. Il prit une profonde inspiration. <<~Il y a eu quelques vols dans mon voisinage, et ma mère m'a demandé de ramener une poêle qu'elle avait empruntée aux voisins deux pâtés de maisons plus loin, et j'ai dit que je ne voulais pas y aller parce que je risquais de me faire voler, et elle a dit “Harry, ne dis pas des choses pareilles~!” comme si y penser allait \emph{faire} que ça ait lieu, comme si, en n'en parlant pas, je serais en sécurité. J'ai essayé de lui expliquer ça et elle m'a fait rapporter la poêle quand même. J'étais trop jeune pour savoir à quel point il était statistiquement improbable qu'un voleur me prenne pour cible, mais j'étais assez vieux pour savoir que ne pas penser à quelque chose ne l'empêchait pas d'avoir lieu, donc j'étais vraiment effrayé.

--- Rien d'autre~?~>> dit McGonagall après une pause, lorsqu'il devint clair que Harry avait terminé. <<~Il n'y a rien d'\emph{autre} qui vous soit arrivé~?

--- Je sais que ça n'a pas \emph{l'air} d'être grand-chose, se défendit Harry. Mais c'était un de ces moments cruciaux d'une vie, vous voyez~? Je veux dire que je \emph{savais} que ne pas penser à quelque chose ne l'empêchait pas d'avoir lieu, je le \emph{savais}, mais je pouvais voir que Maman ne pensait vraiment pas comme ça.~>> Harry s'arrêta, luttant contre la colère qui commençait à monter à chaque fois qu'il y pensait. <<~Elle ne \emph{voulait pas écouter}. J'ai essayé de lui dire, je l'ai \emph{suppliée} de ne pas m'envoyer dehors, et elle \emph{en a rit}. Tout ce que je disais, elle le traitait comme une sorte de blague…~>> Harry força la rage noire à redescendre. <<~C'est là que je me suis rendu compte que tous ceux qui étaient censés me protéger étaient en réalité fous, et qu'ils ne m'écouteraient pas, peu importe que je les supplie, et que je ne pourrai jamais vraiment compter sur eux pour ne pas se tromper.~>> Parfois les bonnes intentions ne suffisaient pas, parfois il fallait être sain d'esprit…

Il y eut un long silence.

Harry prit le temps de respirer profondément et de se calmer. Il n'y avait aucun sens à se mettre en colère. Il n'y avait aucun sens à se mettre en colère. \emph{Tous} les parents étaient comme ça, \emph{aucun} adulte n'était prêt à renoncer à assez de son statut pour se mettre au même niveau qu'un enfant, ses parents génétiques n'auraient pas été différents. La santé mentale était une petite étincelle dans la nuit, une exception infinitésimale à la règle et à la domination de la folie, il était donc futile de se mettre en colère.

Harry ne s'aimait pas quand il était en colère.

<<~Merci de m'avoir fait part de cela, M. Potter~>>, dit McGonagall après un moment. Elle avait une expression distraite, (presque exactement la même que celle qui était apparue sur le visage de Harry alors qu'il faisait des expériences avec la bourse en peau de Moke, si seulement Harry s'était vu dans un miroir et s'en était rendu compte). <<~Je vais devoir y réfléchir.~>> Elle se tourna vers la bouche de la ruelle et leva sa baguette~-

<<~Euh, dit Harry, peut-on aller prendre le kit de soin maintenant~?~>>

McGonagall s'interrompit, et le regarda à nouveau, fermement. <<~Et si je dis non, c'est trop cher et vous n'en aurez pas besoin, qu'est-ce qui se passe~?~>>

Le visage de Harry se tordit d'amertume. 

<<~Exactement ce que vous pensez, Professeur McGonagall. \emph{Exactement} ce que vous pensez. J'en conclus que vous êtes un autre adulte fou auquel je ne peux pas parler, et je commence à élaborer une façon de mettre la main sur un kit de soin.

--- Je suis votre gardien pour cette sortie, dit McGonagall avec une nuance de danger dans la voix. Je ne \emph{vais pas} vous permettre de me bousculer.

--- Je comprends~>>, dit Harry. Il garda la rancœur hors de sa voix, et ne dit aucune des autres choses qui lui venaient à l'esprit. McGonagall lui avait dit de penser avant de parler. Il ne s'en souviendrait probablement pas demain, mais il pouvait au moins s'en souvenir pendant cinq minutes.

La baguette de McGonagall eut un mouvement sec, et les bruits du Chemin de Traverse revinrent. <<~Très bien, jeune homme, dit-elle. Allons acheter ce kit de soin.~>>

La mâchoire de Harry tomba de surprise. Puis il se dépêcha à sa suite, trébuchant presque dans sa précipitation.

\later

Le magasin était tel qu'ils l'avaient laissé, avec des objets reconnaissables et d'autres incompréhensibles, disposés sur l'étal de bois incliné, la lueur grise les protégeant toujours et la vendeuse de retour à sa position originale. Elle les regarda alors qu'ils s'approchaient, son visage exprimant de la surprise.

<<~Je suis désolée~>>, dit-elle quand ils s'approchèrent, et Harry dit presque au même instant~: <<~Je vous demande pardon pour~-~>>

Ils s'interrompirent et se regardèrent, puis la vendeuse eut un petit rire. 

<<~Je ne voulais pas vous causer d'ennuis avec le Professeur McGonagall~>>, dit-elle. Sa voix baissa et prit un ton de conspiratrice. <<~J'espère qu'elle n'a pas été \emph{trop} terrible avec vous.

--- \emph{Della~!} dit McGonagall scandalisée.

--- Sac d'or~>>, dit Harry à sa bourse, et il s'adressa à nouveau à la vendeuse pendant qu'il comptait ses cinq Gallions. <<~Ne vous en faites pas, je comprends bien que si elle est aussi terrible avec moi c'est seulement parce qu'elle m'aime.~>>

Il donna les Gallions à la vendeuse pendant que McGonagall pulvérisait un objet sans importance. <<~Un Pack de Soins d'Urgence Plus, s'il vous plaît.~>>

C'était assez inquiétant, en fait, de voir l'Ouverture Élargissante avaler le kit médical qui avait la taille d'une mallette. Harry ne pouvait pas s'empêcher de se demander ce qui se passerait s'il essayait de grimper dans la bourse lui-même, étant donné que seule la personne qui y avait mit quelque chose était censée pouvoir le récupérer.

Lorsque la bourse eut fini de… manger… son achat durement gagné, Harry jura avoir entendu un petit rot. Ça \emph{devait} avoir été ensorcelé ainsi à dessein. L'hypothèse alternative était trop horrifiante pour être contemplée… en fait Harry ne pouvait même pas \emph{imaginer} une hypothèse alternative. Harry regarda McGonagall à nouveau. <<~Où allons-nous ensuite~?~>>

McGonagall pointa du doigt un magasin qui semblait être fait de chair plutôt que de briques et couvert de fourrure plutôt que de peinture. 

<<~Les petits animaux sont autorisés à Poudlard~- vous pourriez avoir une chouette pour envoyer des lettres, par exemple~-

--- Pourrais-je payer une noise ou quelque chose et \emph{louer} une chouette quand j'aurai besoin d'envoyer du courrier~?

--- Oui, dit McGonagall.

--- Alors absolument \emph{pas}.~>>

McGonagall hocha la tête, comme si elle cochait une case. 

<<~Pourrais-je vous demander pourquoi~?

--- J'avais un rocher de compagnie. Il est mort.

--- Vous ne pensez pas pouvoir prendre soin d'un animal domestique~?

--- Je \emph{pourrais}, dit Harry, mais je me vois déjà, obsédé à longueur de journée, me demandant si je me suis bien souvenu de le nourrir ou s'il meurt lentement de faim dans sa cage, ne sachant ni où est son maître ni pourquoi il n'y a pas de nourriture.

--- Pauvre chouette, dit McGonagall d'une voix douce. Abandonnée comme ça. Je me demande ce qu'elle ferait.

--- Eh bien, elle commencerait à avoir vraiment faim et à essayer de becqueter une ouverture hors de sa boîte ou de sa cage, mais ça ne fonctionnerait probablement pas~-~>> Harry s'arrêta net.

McGonagall continua, toujours de cette voix douce~: <<~Et que se passerait-il ensuite~?~>>

<<~Excusez-moi~>>, dit Harry, et il prit McGonagall par la main, gentiment, mais fermement, et la tira vers une autre ruelle~; après avoir évité tant de sympathisants le procédé était, presque imperceptiblement, devenu une routine.

<<~Jetez le truc de Sourdinam s'il vous plaît.

--- \emph{Sourdinam}

La voix de Harry tremblait. <<~Cette chouette ne me représente \emph{pas}, mes parents ne m'ont \emph{jamais} enfermé dans un placard ni laissé affamé, je n'ai \emph{pas} de peurs d'abandon et \emph{je n'aime pas votre fil de pensée, Professeur McGonagall~!}~>>

La sorcière le regarda.

<<~Et quelles seraient ces pensées, M. Potter~?

--- Vous pensez que j'ai subi, Harry avait du mal à le dire, que j'ai subi des \emph{abus}~?

--- En avez-vous subi~?

"--- \emph{Non~!} cria Harry. Non, jamais~! Pensez-vous que je suis \emph{stupide}~? Je \emph{connais} le concept d'abus infantile, je \emph{sais} ce que sont des attouchements inappropriés et si quoi que ce soit de ce style arrivait j'appellerais la police~! Et j'en parlerais au principal de l'école~! Et je chercherais le numéro des bureaux gouvernementaux dans l'annuaire~! Et j'en parlerais à grand-mère et à grand-père et à Mme Figg~! Mais mes parents n'ont \emph{jamais} fait quoi que ce soit de ce genre, jamais jamais \emph{jamais}~! Comment \emph{osez-vous} suggérer une chose pareille~!~>>

McGonagall le fixait d'un œil solide. <<~Il est de mon devoir en tant que Directrice Adjointe d'enquêter sur tout signe d'abus possible chez les enfants dont je prends soin.~>>

La colère de Harry tournoyait hors de contrôle et devenait une furie noire et pure.

<<~N'ayez jamais \emph{l'audace} de souffler un mot de ces, de ces \emph{insinuations} à qui que ce soit~! \emph{Personne}, vous m'entendez, McGonagall~? Une accusation comme celle-là peut briser des gens et détruire des familles même lorsque les parents sont totalement innocents~! J'ai lu des choses à ce sujet dans les journaux~!~>> La voix de Harry montait et devenait un cri aigu. <<~Le \emph{système} ne sait pas s'\emph{arrêter}, il ne croit pas les parents \emph{ni} les enfants lorsqu'ils disent que rien ne s'est passé~! \emph{Ne vous avisez pas de menacer ma famille avec ça~! Je ne vous laisserai pas détruire mon foyer~!}

--- Harry,~>> dit doucement McGonagall, et elle tendit sa main vers lui~-

Harry fit un rapide pas en arrière, sa main jaillit et il repoussa la sienne~-

McGonagall se figea, puis retira sa main, et fit elle aussi un pas en arrière.

<<~Harry, tout va bien, dit-elle. Je vous crois.

--- \emph{Vous me croyez}~>>, siffla Harry. La furie grondait toujours dans ses veines. <<~Ou vous attendez juste de vous être éloignée de moi pour aller remplir des formulaires~?

--- Harry, j'ai vu votre maison. J'ai vu vous parents. Ils vous aiment. Vous les aimez. Je vous crois lorsque vous dites que vous parents n'ont pas abusé de vous. Il \emph{fallait} que je pose la question, car il y a quelque chose de très étrange à l'oeuvre.~>>

Harry la fixa froidement. <<~Comme quoi~?~>>

McGonagall prit une profonde inspiration. <<~Harry, j'ai vu de nombreux enfants victimes d'abus durant mon temps à Poudlard, ça vous briserait le cœur de savoir combien. Et quand vous êtes joyeux, vous ne vous comportez pas comme l'un de ces enfants, pas du \emph{tout}. Vous souriez aux étrangers, vous faites des câlins aux gens, j'ai mis ma main sur votre épaule et vous n'avez pas bronché. Mais parfois, seulement parfois, vous dites quelque chose qui vous fait \emph{fort} ressembler à… quelqu'un qui aurait passé les premières onze années de sa vie enfermé dans une cave. Pas dans la famille aimante que j'ai vue.~>> McGonagall inclina sa tête, son expression devenant à nouveau perplexe.

Harry absorba tout cela, traitant les informations. La rage noire se vida, et il réalisa qu'on l'écoutait avec respect, et que sa famille n'était pas en danger.

<<~Et comment expliquez-vous vos observations, Professeur McGonagall~?

--- Je ne sais pas, dit-elle. Mais il est possible que quelque chose vous soit arrivé, quelque chose dont vous ne vous souvenez pas.

La furie monta à nouveau en Harry. Ça ressemblait beaucoup trop aux histoires de familles brisées qu'il avait lues dans les journaux.

<<~Les souvenirs refoulés sont de la \emph{pseudoscience}~! Les gens ne répriment \emph{pas} leurs souvenirs traumatiques, ils ne s'en souviennent que \emph{trop bien} pour le restant de leurs vies~!

--- Non, M. Potter. Il existe un charme nommé Oubliettes.~>>

Harry se figea. <<~Un sort qui efface les mémoires~?~>>

McGonagall acquiesça. <<~Mais pas les effets du souvenir, si vous voyez ce que je veux dire, M. Potter.~>>

Un frisson parcouru la colonne vertébrale de Harry. \emph{Cette} hypothèse… n'était \emph{pas} simple à réfuter.

<<~Mais mes parents ne pourraient pas faire ça~!

--- Non, dit McGonagall. Il faudrait quelqu'un venu du monde magique. Il n'y a… aucun moyen de le savoir, j'en ai peur~- pas que je sache.~>>

Les talents de rationaliste de Harry se remirent en route. <<~Professeur McGonagall, à quel point êtes-vous certaine de vos observations, et quelles explications alternatives pourrait-il y avoir~?~>>

McGonagall ouvrit ses mains comme pour montrer qu'elles étaient vides. <<~Certaine~? Je ne suis certaine de \emph{rien}, M. Potter. Si je considère votre individu dans son entier, alors je n'ai jamais rencontré une personne pareille de toute ma vie. Parfois vous ne paraissez tout simplement pas avoir onze ans ni même être vraiment \emph{humain}.~>>

Les sourcils de Harry s'élevèrent vers le ciel~-

<<~Pardon~! dit vivement McGonagall. Je suis vraiment désolée, M. Potter. J'essayais de démontrer un détail par de la rhétorique et j'ai peur que ça ait sonné différemment de la façon dont je l'avais à l'esprit~-

--- Au contraire, Professeur McGonagall, dit Harry et il sourit lentement. Je prendrai cette remarque comme un très grand compliment. Mais objecteriez-vous à ce que je propose une explication alternative~?

--- Allez-y, je vous en prie.

--- Les enfants ne sont pas censés être beaucoup plus intelligents que leurs parents, dit Harry. Ou peut-être beaucoup plus sains d'esprit~- mon père pourrait probablement se montrer plus malin que moi s'il, vous savez, \emph{essayait} vraiment, au lieu d'utiliser son intelligence d'adulte pour trouver de nouvelles raisons de ne pas changer d'avis~- Harry s'interrompit. Je suis trop intelligent, McGonagall. Les enfants normaux ne sont tout simplement pas dans la même catégorie que moi. Les adultes ne me respectent pas assez pour me parler. Et franchement, même s'ils le faisaient, ils ne diraient pas des choses aussi intelligentes que Richard Feynman, donc il vaut mieux que je lise quelque chose écrit par Richard Feynman. Je suis \emph{isolé}, Professeur McGonagall. J'ai été isolé toute ma vie. Peut-être que ça produit quelques-uns des effets qu'on ressent quand on est enfermé dans une cave. Je suis trop intelligent pour admirer mes parents de la façon dont les enfants sont censés le faire. Mes parents m'aiment, mais ils ne se sentent pas obligés de répondre à la raison, et parfois j'ai la sensation que ce sont eux les enfants~- des enfants qui \emph{n'écoutent pas}, et qui ont une autorité absolue sur toute mon existence. J'essaie de ne pas être trop amer à ce sujet, mais j'essaie aussi d'être \emph{honnête} avec moi-même, et donc, oui, je suis amer. J'ai aussi un problème de contrôle de ma colère, mais j'y travaille. C'est tout.

--- \emph{C'est tout~?}~>>

Harry acquiesça avec ferveur. <<~C'est tout. Professeur McGonagall, l'explication normale mérite d'être \emph{prise en considération}, même dans l'Angleterre magique, non~?~>>

\later

Plus tard dans la journée, le soleil descendait sur un ciel d'été et les acheteurs commençaient à disparaître des rues. Certains magasins avaient déjà fermé~; Harry et McGonagall avaient acheté ses manuels chez Fleury et Bott juste avant la fermeture. Il y avait seulement eu une légère explosion quand Harry avait foncé droit vers le mot-clé “Arithmancie” et avait découvert que les livres de septième année ne contenaient rien de plus mathématiquement avancé que la trigonométrie.

Mais pour le moment, les rêves d'opportunités faciles étaient très loin de l'esprit de Harry.

Pour le moment, Harry et McGonagall sortaient de chez Ollivander's, et Harry fixait sa baguette. Il l'agita et produit des étincelles multicolores, ce qui n'aurait vraiment pas dû le choquer particulièrement après tout ce qu'il avait déjà vu, mais malgré tout~-

\emph{Je peux faire de la magie.}

\emph{Moi. Comme dans “Moi, personnellement.” Je suis magique~; je suis un sorcier.}

Il avait \emph{sentit} la magie affluer dans son bras, et à cet instant il avait réalisé qu'il avait toujours eu ce sens, qu'il l'avait possédé toute sa vie, le sens qui n'était ni la vue ni le son ni l'odeur ni le goût ni le toucher, mais seulement la magie. Comme d'avoir des yeux, mais de les avoir toujours gardés fermés, et que vous ne vous rendiez pas compte que vous voyez du noir~; et le jour où vous les ouvriez, vous découvriez le monde. Le choc s'était déversé en lui, touchant plusieurs parties de son être, les réveillant, et disparaissant ensuite en quelques secondes~; ne laissant que la certitude qu'il était maintenant un sorcier, l'avait toujours été, et d'une certaine façon, qu'il l'avait toujours su.

Et~-

<<~\emph{Il est en effet très curieux que vous soyez destiné à cette baguette, sachant que sa sœur, eh bien, sa sœur vous a donné cette cicatrice.}~>>

Ça ne \emph{pouvait} pas être une coïncidence. Il y avait des \emph{milliers} de baguettes dans ce magasin. Bon, d'accord, ça \emph{pouvait} être une coïncidence, il y avait six milliards de personnes sur Terre, des coïncidences à une chance sur mille avaient lieu tous les jours. Mais, Théorème de Bayes 101~: toute hypothèse raisonnable impliquant qu'il avait \emph{plus} d'une chance sur mille que Harry se retrouve avec la baguette sœur de celle du Seigneur des Ténèbres avait un avantage.

McGonagall avait simplement dit \emph{comme c'est curieux} et en était restée là, ce qui avait mit Harry en état de choc face à la pure, à l'écrasante \emph{inconscience} des sorciers et sorcières. Harry n'aurait pu, dans aucun monde \emph{imaginable}, simplement faire <<~Hmm~>> et sortir du magasin sans même \emph{essayer} de trouver une hypothèse expliquant ce qui s'était passé.

Sa main gauche s'éleva et toucha sa cicatrice.

Qu'est-ce qui… \emph{exactement}…

<<~Vous êtes un sorcier complet à présent, dit McGonagall. Félicitations.~>>

Harry hocha la tête.

<<~Et que pensez-vous du monde magique~?

--- C'est étrange, dit Harry. Je devrais être en train de penser à tout ce que j'ai vu de la magie… tout ce que je sais maintenant être possible, et tout ce que je sais maintenant être un mensonge, et tout le travail qui me reste à accomplir avant de vraiment comprendre. Et pourtant je me trouve distrait par de relatives trivialités telles que,~>> Harry baissa la voix, <<~toute cette histoire de Survivant.~>> Il ne semblait y avoir personne aux alentours, mais autant ne pas tenter le sort.

McGonagall \emph{ahema}. <<~Vraiment~? Sans blague.~>>

Harry hocha la tête. <<~Oui. C'est juste… \emph{curieux}. De se rendre compte que vous faites partie de cette grande histoire, la quête pour vaincre le grand et terrible Seigneur des Ténèbres, et c'est déjà \emph{fini}. Terminé. Complètement réglé. Comme si vous étiez Frodon Sacquet, que vous appreniez que vos parents vous avaient emmené à la Montagne du Destin quand vous aviez un an, qu'ils vous avaient fait jeter l'anneau et que vous ne vous en souveniez même pas.~>>

Le sourire de McGonagall s'était plus ou moins figé.

<<~Vous savez, si j'étais qui que ce soit d'autre, vraiment n'importe qui d'autre, je serais plutôt anxieux à l'idée de vivre à la hauteur de ce démarrage. \emph{Grand dieu Harry, qu'avez-vous fait depuis que vous avez vaincu le Seigneur des Ténèbres~? Votre propre librairie~? C'est super~! Dites-moi, saviez-vous que j'ai donné votre nom à mon enfant~?} Mais j'ai bon espoir que cela ne soit pas un problème.~>> Harry soupira. <<~Tout de même… c'est presque assez pour me faire espérer qu'il y ait \emph{quelques} détails de cette quête à finir, juste pour que je puisse dire que j'ai vraiment, vous savez, \emph{participé} d'une façon quelconque.

--- Oh~? dit McGonagall sur un ton étrange. Qu'aviez-vous à l'esprit~?

--- Eh bien par exemple, vous avez mentionné que mes parents ont été trahis. Qui les a trahis~?

--- Sirius Black~>>, dit McGonagall. Elle siffla son nom plus qu'elle ne le prononça. <<~Il est à Azkaban. Prison des sorciers.

--- Quelle est la probabilité que Sirius Black s'échappe de prison et que je doive le traquer et le vaincre dans un duel spectaculaire, ou encore mieux, mettre une large prime sur sa tête et me cacher en Australie pendant que j'attends le résultat~?~>>

McGonagall cligna des yeux. Deux fois. <<~Peu probable. Personne ne s'est jamais échappé d'Azkaban, et je doute qu'\emph{il} soit le premier.~>>

Harry était un peu sceptique de ce “\emph{personne} ne s'est \emph{jamais} échappé d'Azkaban”. Mais bon, peut-être qu'avec la magie vous pouviez faire approcher votre prison de 100~\% de perfection, et encore plus si vous aviez une baguette et pas l'autre. La meilleure façon de sortir serait de ne jamais y être entré.

<<~Très bien, dit Harry. Ça m'a l'air bien ficelé.~>> Il soupira, et gratta sa paume contre sa tête. <<~Ou peut-être que le Seigneur des Ténèbres n'est pas \emph{vraiment} mort cette nuit-là. Pas complètement. Son esprit erre, chuchotant aux gens dans leurs cauchemars, qui se répandent dans le monde éveillé, et il cherche à revenir sur les terres des vivants, qu'il a promi de détruire, et maintenant, en accord avec l'ancienne prophétie, lui et moi sommes coincés dans un duel à mort où le gagnant perdra et le perdant gagnera~-~>>

La tête de McGonagall pivota, et ses yeux dardèrent aux alentours, à la recherche de personnes prêtant l'oreille.

<<~Je \emph{plaisante}, Professeur McGonagall~>>, dit Harry, un peu contrarié. Bon sang, pourquoi devait-elle toujours tout prendre si sérieusement~-

Une lente sensation coula doucement jusqu'au fond de l'estomac de Harry.

McGonagall regarda Harry avec un air calme. Un air très, \emph{très} calme. Puis un sourire fut ajouté. <<~Bien sûr que vous plaisantez, M. Potter.~>>

\emph{Oh crotte.}

Si Harry avait eu besoin de rationaliser l'inférence muette qui venait de flasher dans son esprit, ça aurait été quelque chose comme~: <<~Si j'estime la probabilité que McGonagall a fait ce que je viens de voir parce qu'elle s'est contrôlée avec soin, contre la distribution de probabilités pour toutes les choses qu'elle ferait \emph{naturellement} si j'avais fait une mauvaise blague, alors ce comportement est un élément de preuve significatif pointant vers le fait qu'elle cache quelque chose.~>>

Mais ce que Harry pensa fut~: \emph{Oh crotte}.

Harry pivota sa propre tête pour scanner la rue. Non, personne dans le coin.

<<~Il n'est \emph{pas} mort, c'est ça~? soupira Harry.

---M. Potter~-

--- Le Seigneur des Ténèbres est vivant. \emph{Bien sûr} qu'il est vivant. C'était un \emph{acte} de pur et simple \emph{optimiste} que de seulement \emph{rêver} qu'il en soit autrement. J'ai \emph{dû} perdre la raison, je ne peux pas \emph{imaginer} à quoi je \emph{pensais}. Juste parce que \emph{quelqu'un} a dit que son corps avait été retrouvé \emph{calciné}, je ne peux pas imaginer pourquoi j'ai pu penser qu'il était \emph{mort}. J'ai \emph{clairement beaucoup} à apprendre sur l'art correct du \emph{pessimisme}.

--- M. Potter~-

--- Dites-moi au moins qu'il n'y a pas vraiment de prophétie…~>> Mais McGonagall lui donnait ce sourire intense et figé. <<~Oh, bon sang, mais c'est une \emph{blague}.

---M. Potter, vous ne devriez pas inventer des choses comme ça.

--- C'est \emph{vraiment} \emph{ça} que vous voulez me dire~? Imaginez ma réaction plus tard, quand j'apprendrai qu'il y avait quelque chose dont j'aurais dû me soucier après tout.~>>

Le sourire de McGonagall se flétrit.

Les épaules de Harry s'affaissèrent. <<~J'ai un monde entier de magie à analyser. Je n'ai \emph{pas} de temps à consacrer à ça.~>>

Puis les deux se turent, et un homme en robe orange et flottante apparut dans la rue et les dépassa lentement. Les yeux de McGonagall le suivirent discrètement. La bouche de Harry bougeait, car il mâchait sa lèvre inférieure, et quelqu'un observant de près aurait remarqué un léger point de sang apparaître.

Lorsque l'homme en robe orange fut loin, Harry parla à nouveau, d'un bas murmure.

<<~Allez vous me dire la vérité à présent, Professeur McGonagall~? Et n'essayez pas de prétendre qu'il n'y a rien, je ne suis pas stupide.

--- Vous avez \emph{onze ans}, M. Potter~! dit-elle dans un murmure cassant.

--- Et par conséquent sous-humain. Pardon… pour un moment j'avais \emph{oublié}.

--- Ce sont des affaires importantes et terribles~! Ce sont des \emph{secrets}, M. Potter~! C'est une \emph{catastrophe} que vous, encore un enfant, en sachiez autant~! Vous ne devez le dire à \emph{personne}, vous comprenez~? Absolument personne~!~>>

Et, comme cela arrivait parfois quand Harry se mettait \emph{suffisamment} en colère, son sang devint froid au lieu de chaud, et une terrible clarté obscure s'abattit sur son esprit, décrivant toutes les tactiques possibles et jugeant les conséquences avec un réalisme d'acier.

\emph{Fais remarquer que tu as le droit de savoir~: Échec. Les enfants de onze ans n'ont le droit de savoir rien du tout, aux yeux de McGonagall.}

\emph{Dis que vous ne serez plus amis~: Échec. Elle n'accorde pas assez de valeur à ton amitié.}

\emph{Fais remarquer que tu seras en danger si tu ne sais pas~: Échec. Des plans ont déjà été pensés, basés sur ton ignorance. Le déplaisir} certain \emph{de repenser le plan leur semblera bien plus désagréable que la perspective} incertaine \emph{de te voir blessé.}

\emph{La justice et la raison échoueront. Tu dois soit trouver quelque chose que tu as et qu'elle veut, soit quelque chose que tu peux faire et qu'elle craint…}

Ah.

<<~Très bien, dans ce cas, Professeur McGonagall, dit Harry d'un ton bas et glacé, on dirait que j'ai quelque chose que vous désirez. Vous pouvez, si vous le souhaitez, me dire la vérité, \emph{toute} la vérité, et en retour je garderai vos secrets. Ou vous pouvez essayer de me garder dans l'ignorance et m'utiliser comme un pion, auquel cas je ne vous devrai rien.~>>

McGonagall s'arrêta net au milieu de la rue. Ses yeux flamboyèrent et sa voix se transforma en un sifflement.

<<~Comment osez-vous~!

--- \emph{Comment osez-vous~!} chuchota-t-il en retour.

--- Vous me faites \emph{chanter}~?~>>

Les lèvres de Harry se tordirent. <<~Je vous \emph{offre} une \emph{faveur}. Je vous \emph{donne} une chance de garder \emph{notre} précieux secret. Si vous refusez, j'aurais \emph{tous} les motifs du monde pour aller poser des questions ailleurs, non par rancune envers vous, mais parce que \emph{j'ai besoin de savoir}~! Dépassez votre colère futile envers un \emph{enfant} qui, vous le croyez, se doit de vous obéir, et vous comprendrez que tout adulte sain d'esprit ferait de même~! \emph{Regardez les choses de mon point de vue~! Comment vous sentiriez-vous si c'était VOUS~?}~>>

Harry regarda McGonagall, observa sa respiration saccadée. Il se rendit compte qu'il était temps d'adoucir la pression, de la laisser pondérer un moment. <<~Vous n'avez pas à décider tout de suite, dit Harry sur un ton plus normal. Je comprendrais si vous vouliez plus de temps pour réfléchir à mon \emph{offre}… mais je vous préviens d'une chose~>>, dit Harry sa voix devenant plus froide. <<~N'essayez pas ce Charme d'Oubliettes sur moi. Il y a quelque temps, j'ai conçu un signal, et je me le suis déjà envoyé à moi-même. Si je trouve ce signal et que je ne me \emph{souviens} pas l'avoir envoyé…~>> Harry laissa sa voix traîner d'une façon lourde de sens.

Le visage de McGonagall travaillait sous le coup de divers changements d'expression.

<<~Je… je ne pensais pas à vous lancer Oubliettes, M. Potter… mais pourquoi auriez-vous \emph{inventé} un signal si vous ne connaissiez pas l'existence de~-

--- J'y ai pensé en lisant un livre de science-fiction Moldu, et je me suis dit, \emph{bon, juste au cas où}… Et non, je ne vous dirai pas le signal, je ne suis pas stupide.

--- Je ne comptais pas vous le demander,~>> dit McGonagall. Elle parut se replier sur elle-même, et eut l'air soudain très vieille et très fatiguée. <<~Ça a été une journée épuisante, M. Potter. Pourrions-nous prendre votre malle et vous envoyer chez vous~? Je vous fais confiance pour ne pas parler de cette affaire avant que j'aie eu le temps d'y réfléchir. Gardez à l'esprit qu'il n'y a que deux autres personnes au monde qui soient au courant de cette affaire, et ce sont le Directeur Albus Dumbledore et le Professeur Severus Rogue.~>>

Donc. De nouvelles informations~; c'était une offre de paix. Harry acquiesça, tourna la tête vers l'avant et commença à marcher à nouveau.

<<~Donc maintenant je dois trouver un moyen de tuer un Seigneur des Ténèbres immortel~>>, dit Harry, et il soupira de frustration. <<~J'aurais vraiment aimé que vous me disiez ça \emph{avant} qu'on commence à faire du shopping.~>>

\later

Le magasin de malles était plus richement décoré que tout autre magasin que Harry ait visité auparavant~; les rideaux étaient luxueux et ornés de motifs délicats, le sol et les murs étaient faits de bois teint et poli, et les malles occupaient des places d'honneur sur des plate-formes en ivoire poli. Le vendeur était habillé en robe d'une qualité seulement un cran en dessous de celles de Lucius Malfoy, et il parlait avec une politesse huileuse et exquise tant à Harry qu'à McGonagall.

Harry avait posé ses questions, et avait gravité vers une malle de bois lourd, pas polie, mais chaude et solide, gravée avec le motif d'un dragon gardien dont les yeux se déplaçaient pour regarder toute personne s'approchant. Une malle charmée pour être légère, réduire de taille sur commande, et faire pousser des petits tentacules griffus de sa base et se tortiller derrière son maître. Une malle avec deux tiroirs sur chacun de ses quatre côtés qui glissaient pour révéler des compartiments aussi profonds que la malle entière. Un couvercle équipé de quatre cadenas, et chacun d'entre eux révélait un espace intérieur différent. Et~- et c'était la partie importante~- une poignée sur le fond qui glissait et révélait un cadre contenant des marches menant vers une petite pièce éclairée qui, estima Harry, pouvait contenir environ douze étagères.

s'ils faisaient des malles comme cella-là, Harry ne savait pas pourquoi qui que ce soit s'embêtait à posséder une maison.

Cent huit Gallions. C'était le prix d'une \emph{bonne} malle, légèrement usée. À cinquante livres le Gallion, c'était assez pour s'offrir une voiture usagée. C'était plus cher que la somme de tout ce que Harry avait acheté de sa vie.

Quatre-vingt-dix-sept Gallions. C'était ce qui restait dans le sac d'or que Harry avait été autorisé à retirer de chez Gringotts.

McGonagall avait un air chagriné. Après une longue journée de shopping elle n'avait pas eu besoin de demander à Harry combien d'or il restait dans le sac après que le vendeur eut donné son prix, ce qui voulait dire que le Professeur pouvait faire du calcul mental sans crayon ni papier. À nouveau, Harry se rappela à lui-même que \emph{scientifiquement illettré} n'était pas la même chose que \emph{stupide}.

<<~Je suis désolée, jeune homme, dit McGonagall. C'est entièrement de ma faute. Je vous proposerais bien de vous ramener à Gringotts, mais la banque est à présent fermée hormis pour ses services d'urgence.~>>

Harry prit une profonde inspiration. Il devait devenir un peu en colère pour ce qu'il voulait maintenant essayer, autrement il n'aurait sûrement pas le courage de le faire. Il se dit~: \emph{Elle ne m'a pas écouté, j'aurais pris plus d'or, mais elle ne voulait pas écouter}… Il repensa à la rage noire, plus tôt, et essaya d'en faire revenir un peu. Il visualisa \emph{la personne qu'il avait besoin d'être}, se revêtit de cette personnalité comme d'une robe de sorcier. Concentrant son univers entier sur McGonagall et le besoin qu'il avait de tordre cette conversation à ses fins, il parla.

<<~Laissez-moi deviner, dit Harry. Vous pensiez que vous vous donniez une \emph{grande} marge d'erreur, que cent Gallions seraient \emph{plus} que suffisants, et c'est pourquoi vous n'avez pas pris la peine de me prévenir quand nous sommes descendus à quatre-vingt-dix-sept.~>>

McGonagall ferma les yeux avec résignation.

<<~Oui.

--- J'ai anticipé cela, Professeur McGonagall. J'ai anticipé que cela arriverait. Il y a des études montrant que c'est ce qui se passe quand les gens pensent qu'ils \emph{se donnent une grande marge d'erreur}. Si c'était \emph{moi}, j'aurai pris \emph{deux cents} Gallions, juste pour être sûr~; il y avait plein d'argent dans cette chambre forte, et j'aurai pu y remettre la monnaie plus tard. Mais je \emph{savais} que vous ne me laisseriez pas. Je savais qu'il était futile de demander. Je savais que vous seriez agacée et peut-être même \emph{énervée} si je vous demandais. Ai-je tort~?

--- Non, dit McGonagall, vous avez raison.~>> Sa voix avait une note d'excuse, mais aussi une note d'orgueil personnel, comme si Harry était censé remarquer le grand, l'immense honneur que c'était de voir le \emph{Professeur McGonagall} s'excuser auprès de lui.

--- Vous devriez comprendre, Professeur McGonagall, Harry prononça ces mots avec soin, que c'est pour ça que je ne fais pas confiance aux adultes. Vous pensiez qu'être adulte voulait dire que c'était votre rôle de m'empêcher de prendre trop d'argent dans ma chambre forte. Pas que c'était votre rôle de \emph{vous assurer que le travail soit fait quoi qu'il arrive}.~>>

Les yeux de McGonagall s'ouvrirent grand, et elle jeta un regard dur à Harry.

<<~Eh bien, Professeur McGonagall, si tout était à refaire, et que je suggérais de prendre cent Gallions de plus \emph{juste pour être sûr}, sans justification autre que celle d'être \emph{prêt}, m'écouteriez vous \emph{cette fois}~?

--- J'accepte votre argument, dit McGonagall, Vous n'avez pas besoin de \emph{me} sermonner, jeune homme~!

--- Ah, mais je n'en suis pas encore \emph{arrivé} à mon argument. Connaissez-vous la différence entre quelqu'un qui mérite qu'on lui parle et un simple obstacle, Professeur McGonagall~? De mon point de vue~? Si un adulte pense que m'être supérieur, qu'être au-dessus de moi, qu'obtenir mon obéissance, sont les choses les \emph{plus importantes} pour lui, alors il sera un obstacle. Un \emph{collaborateur potentiel} est quelqu'un qui pense que \emph{faire le travail} est plus important que de s'assurer que je reste à ma place. Laissez-moi vous montrer quelque chose, Professeur McGonagall.~>>

Le vendeur de malle les observait avec une fascination non dissimulée, et Harry sortit sa bourse en peau de Moke et dit <<~Onze Gallions en vrac, s'il vous plaît.~>>

Et il y avait de l'or dans la main de Harry.

<<~\emph{Où avez-vous obtenu cet~-}

--- Dans ma chambre forte, Professeur McGonagall, quand je suis tombé dans ce tas d'or. J'ai fourré de l'argent dans ma poche et j'ai ensuite tenu le sac d'or contre ma poche, pour que les tintements semblent venir de là où il fallait. Car, vous comprenez, je m'attendais depuis le début à ce que cela ait lieu.~>>

La bouche de McGonagall était grande, grande ouverte.

<<~La question est maintenant… êtes-vous en colère parce que j'ai défié votre autorité~? Ou contente que notre journée se termine par un succès au lieu d'un échec~? Je ne vous demande rien \emph{d'autre} en vous posant cette question. Je ne vous promets ni ne vous demande une coopération dans nos affaires futures. Je veux seulement savoir si vous êtes une \emph{collaboratrice potentielle} ou un obstacle… Minerva.~>>

Le vendeur s'étrangla bruyamment.

Et la puissante sorcière resta silencieuse.

<<~La discipline \emph{doit} être appliquée à Poudlard, dit-elle après qu'une minute entière se soit écoulée. Pour le bien de \emph{tous} les étudiants. Et cela \emph{doit} inclure la courtoisie et l'obéissance à \emph{tous} vos professeurs.~>>

Harry inclina sa tête. <<~Je comprends, Professeur McGonagall.~>> Mais il était tout de même incroyable que, bizarrement, il semble \emph{beaucoup plus} important d'appliquer la discipline quand \emph{vous} étiez en \emph{haut} de la pile que quand vous étiez en bas… mais Harry ne jugea pas sage d'appuyer sur ce point.

<<~Dans ce cas… je vous félicite pour votre grande préparation.~>>

Harry voulait applaudir, ou vomir, ou s'évanouir, ou quelque chose. C'était la première fois que ce discours avait jamais fonctionné sur un adulte. C'était la première fois qu'\emph{aucun} de ses discours avait jamais fonctionné sur \emph{qui que ce soit}. Peut-être aussi parce que c'était la première fois qu'il avait quelque chose dont un adulte avait sérieusement besoin, mais tout de même~-

Minerva McGonagall, +1 point.

Harry s'inclina, et donna le sac d'or et les onze Gallions supplémentaires aux mains de McGonagall. <<~Je vous le laisse, madame. Pour ma part, je dois utiliser les toilettes. Puis-je demander où~-~>>

Le vendeur, onctueux à nouveau, pointa du doigt en direction d'une porte incrustée dans le mur et munie d'une poignée d'or. Alors que Harry s'éloignait, il entendit le vendeur derrière lui dire de sa voix huileuse~: <<~Puis-je m'informer de l'identité de cette personne, Madame McGonagall~? J'imagine qu'il est Serpentard~- troisième année peut-être~?~- et d'une importante famille, mais je n'ai pas reconnu~-~>>

Le claquement de la porte de la salle de bain coupa ses mots, et après que Harry eut identifié le loquet et l'ait mis en place, il s'effondra contre la porte. Son corps entier était baigné d'une sueur qui avait traversé ses vêtements Moldus, mais au moins ça ne se voyait pas sur sa robe. Il se pencha au-dessus de la cuvette or-ivoire, eut quelques haut-le-cœur, mais heureusement rien ne vint.

\later

Ils se tenaient à nouveau dans le jardin du Chaudron Baveur, sur la petite interface couverte de feuilles entre le Chemin de Traverse de l'Angleterre magique et le monde Moldu. C'était une économie \emph{horriblement} découplée… Harry devait aller à une cabine téléphonique et téléphoner à son père une fois de l'autre côté. Il ne devait pas, apparemment, s'inquiéter de voir son bagage volé~; il avait le statut d'objet magique majeur, un type d'objet que les Moldus ne remarqueraient pas. C'était une partie de ce que vous pouviez obtenir dans le monde magique, si vous étiez prêt à payer le prix d'une voiture de seconde main. Harry se demanda si son père serait capable de voir la malle après que Harry la lui a explicitement montrée.

<<~C'est ici que nos chemins se séparent, pour un temps~>>, dit le Professeur McGonagall. Elle secoua sa tête avec émerveillement. <<~Ça a été le jour le plus étrange de ma vie depuis… depuis bien des années. Depuis le jour où j'ai appris qu'un enfant avait vaincu Vous-Savez-Qui. Je me demande maintenant, rétrospectivement, si c'était le dernier jour sensé de ce monde.~>>

Oh, comme si \emph{elle} avait à se plaindre de quoi que ce soit. \emph{Vous pensez que votre journée était surréaliste~? Essayez la mienne pour voir}.

<<~Vous m'avez grandement impressionné aujourd'hui, lui dit Harry. J'aurais dû penser à vous complimenter à voix haute, je vous donnais des points dans ma tête et tout.

--- Merci, M. Potter, dit McGonagall. Si vous aviez déjà été trié dans une Maison je vous aurais déduit tant de points que vos petits-enfants perdraient encore la Coupe des Maisons.

--- Merci à \emph{vous}, Minerva.~>> Il était probablement encore trop tôt pour l'appeler Minny.

Cette femme était peut-être l'adulte le plus sain d'esprit que Harry ait jamais rencontré, en dépit de son manque de savoir scientifique. Harry envisageait même de lui offrir la position de numéro deux dans le groupe qu'il formerait pour combattre le Seigneur des Ténèbres, mais il n'était pas assez idiot pour dire ça à voix haute. \emph{Et quel serait un bon nom pour ce groupe…? Les Mangemangemorts~?}

<<~Je vous verrai très bientôt, quand l'école commencera, dit McGonagall. Et, M. Potter, à propos de votre baguette~-

--- Je sais ce que vous allez me demander~>>, dit Harry. Il sortit sa précieuse baguette et, avec un immense pincement de douleur intérieure, la retourna dans sa main. La poignée vers l'extérieur, il la présenta à McGonagall. <<~Prenez-la. Je ne comptais pas faire quoi que ce soit, pas une seule petite chose, mais je ne veux pas que vous ayez des cauchemars où je fais exploser ma maison.~>>

McGonagall secoua vivement la tête.

<<~Oh, non, M. Potter~! On ne fait pas ce genre de choses. Je voulais juste vous prévenir de ne pas \emph{utiliser} votre baguette chez vous, car il y a des moyens de détecter l'usage de la magie chez les mineurs et c'est interdit sans supervision.

--- Ah, dit Harry et il sourit. \emph{Cela} me semble être une règle \emph{très} sensée. Je suis heureux de voir que le monde magique prend ce genre de choses sérieusement.~>>

McGonagall le regarda intensément.

<<~Vous le pensez vraiment.

--- Oui, dit Harry. Je comprends. La magie est dangereuse et les règles sont là pour une bonne raison. Certaines affaires sont elles aussi dangereuses. Je le comprends. Souvenez-vous que je ne suis pas stupide.

--- J'ai bien peu de chances de l'oublier. Merci, Harry Potter, cela m'aide à me sentir mieux concernant certaines choses au sujet desquelles je vais devoir vous faire confiance. Au revoir pour l'instant.~>>

Harry se détourna pour partir, vers le Chaudron Baveur et jusqu'au monde Moldu.

Et alors que sa main touchait la poignée de la porte, il entendit un dernier murmure derrière lui.

<<~Hermione Granger.

--- Quoi~? dit Harry, sa main toujours sur la porte.

--- Cherchez une fille de première année nommée Hermione Granger sur le train vers Poudlard.

--- Qui est-elle~?~>>

Il n'y eut pas de réponse, et quand Harry se retourna, McGonagall était partie.

\latersection{Après coup~:}

Le Directeur Dumbledore se pencha par-dessus son bureau. Ses yeux pétillants dévisagèrent McGonagall. <<~Alors Minerva, qu'avez-vous pensé de Harry~?~>>

McGonagall ouvrit sa bouche. Puis elle ferma sa bouche. Puis elle ouvrit à nouveau sa bouche. Aucun mot ne sortit.

<<~Je vois, dit Dumbledore avec gravité. Merci pour votre rapport, Minerva. Vous pouvez y aller.~>>

%  LocalWords:  ome zahav ahava Aaaaaaarrrgh QX31 ahemmed Sheesh
%  LocalWords:  Aw
