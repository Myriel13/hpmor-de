\namedpartchapter{La Vérité}{The Truth}{I}{Jeux du sort et réponses}

\section{Le 13 juillet 1992.}

\lettrine{C}{'était} la dernière semaine de cours et le professeur Quirrell était toujours en vie. À peine. Il se trouvait dans un lit d'hôpital, comme il l'avait été pendant la majeure partie de la semaine.

Selon la tradition de Poudlard, les examens avaient lieu la première semaine de juin, étaient rendus la deuxième, et un grand festin de départ avait lieu le dernier dimanche avant que le Poudlard Express n'emmène les élèves à Londres, le dernier lundi.

Il y a bien longtemps, lorsqu'il avait entendu parler de cette tradition, Harry s'était demandé ce que les élèves pouvaient bien faire pendant la deuxième semaine de juin. <<~Attendre les résultats des examens~>> ne ressemblait pas à une activité. La réponse l'avait surpris.

Mais la deuxième semaine était à présent terminée~; nous étions dimanche, et les seuls événements notables à venir étaient le festin de départ du 14 et le Poudlard Express du 15.

Aucune question n'avait trouvé de réponse.

Aucun problème n'avait été résolu.

L'assassin de Hermione était dans la nature.

Une partie de Harry avait cru que la vérité éclaterait certainement avant la fin de l'année, comme s'il s'agissait d'un roman d'intrigues dont les clés lui auraient été promises. Il fallait au moins qu'elle surgisse avant la… mort… du professeur Quirrell~; il ne pouvait quand même pas \emph{mourir} sans réponses, sans que tout ait été réglé. Une histoire devait se conclure sur une réponse, pas sur des notes d'examen, et certainement pas sur une mort…

Mais à moins de croire à la dernière théorie de Drago Malfoy, selon qui le professeur Chourave avait mis plus de temps à rendre leurs copies aux élèves peu de temps avant que Hermione ne soit accusée de tentative de meurtre et avait donc forcément passé son temps à mettre en place le coup monté, la vérité demeurait cachée.

Au lieu de ça, comme si les priorités du monde étaient celles des autres, l'année allait se conclure par un grand match de Quidditch final.

\later

Au-dessus du stade, de lointaines silhouettes juchées sur des balais pirouettaient et voltigeaient les unes autour des autres. Le tétraèdre tronqué rouge écarlate qu'était le Souafle se faisait saisir, balancer, bloquer et parfois envoyer à travers des anneaux volants, accompagné par des éclats de triomphe ou de consternation qui faisaient vibrer le stade. Des robes à liserés bleus, verts, jaunes et rouges vociféraient l'enthousiasme si facilement ressenti par ceux qui savent que rien ne pèse sur leurs épaules.

C'était le premier match de Quidditch de Poudlard auquel Harry se rendait, et il avait déjà décidé que ce serait le dernier.

<<~Davies a le Souafle~! s'écria la voix amplifiée de Lee Jordan. Dix points de plus pour Serdaigle dans sept… six… cinq… sapristi~! C'est déjà fait~! En plein milieu de l'anneau central~! Je n'ai jamais vu une telle série de coups d'éclat - je parie ce que vous voulez que Davies sera capitaine l'année prochaine après Bortan…~>>

La voix de Lee fut abruptement coupée et la voix du professeur McGonagall, elle aussi amplifiée, dit~:

<<~Ça regarde l'équipe Serdaigle, M. Jordan. Tenez-vous-en au match, s'il vous plaît.

--- Et les Serpentard prennent possession… Flint envoie le Souafle à la ravissante…

--- M. Jordan~!

--- À la potable Sharon Vizcaino dont les cheveux flamboient comme une comète alors qu'elle fonce vers la défense Serdaigle - maintenant avec deux Cognards juste après elle~! Pucey la suit de près - qu'est-ce que tu fais, Inglebee~? - et elle pivote à pleine vitesse pour éviter - \shout{Est-ce que c'est le vif d'or~? aller Cho Chang, aller, Higgs est déjà - qu'est-ce que vous faites}~?

--- Calmez-vous, M. Jordan~!

--- \shout{Comment je peux calmer c'était le pire foirage que j'ai jamais vu}~! Et le Vif est parti, peut-être pour de bon vu la façon grotesque dont il a été raté - Pucey fonce maintenant vers les buts, Inglebee est très loin de lui…~>>

Dans un autre Âge de l'Histoire, peut-être même dans un autre monde, le professeur Quirrell avait œuvré pour que la coupe des Maisons soit décernée à Serdaigle ou Serpentard. Ou peut-être aux deux~; car il avait promis que les trois vœux seraient exaucés. Pour l'instant, il semblait qu'on en était à deux sur trois.

Si on s'en tenait au score actuel, Poufsouffle avait une avance dans la course de près de cinq-cents points, grâce aux étudiants de Poufsouffle, qui faisaient leurs devoirs et \emph{évitaient les ennuis}. Il semblait que le professeur Rogue avait fort stratégiquement fait perdre des points aux Poufsouffle pendant les sept dernières années. La maison Serpentard, championne en titre depuis sept ans, était avantagée par une certaine \emph{générosité} de son directeur de maison~; et cela suffisait à mettre les Serpentard au coude à coude avec la maison Serdaigle, terre d'accueil des élèves brillants. Gryffondor était bonne dernière, là où la maison des anticonformistes se devait d'être~; Gryffondor était aussi dissipée et douée en cours que Serpentard, mais n'avait pas l'équivalent d'un professeur Rogue. Même Fred et George avaient presque fini dans le rouge.

Serdaigle et Serpentard devaient toutes les deux trouver des points \emph{quelque part} si elles voulaient rattraper Poufsouffle en moins de deux derniers jours.

Et à ce qu'on en savait, le professeur Quirrell n'avait rien fait pour obtenir le résultat attendu. Les choses avançaient toutes seules, maintenant qu'un professeur de Poudlard avait enseigné comment résoudre ses problèmes de façon créative.

Pour le dernier match de l'année, Serdaigle affrontait Serpentard. Plus tôt, l'avance de Gryffondor avait disparu après que leur nouvel Attrapeur, Emmette Shear, tomba d'un balai peut-être défaillant lors de son second match. Cela avait aussi conduit à une réorganisation hâtive des matchs suivants.

Ce dernier match ne prendrait fin que lorsque quelqu'un attraperait le Vif d'or.

Les points de Quidditch s'ajoutaient directement au total de chaque maison.

Et, oh surprise, il semblait bien qu'aujourd'hui, les Attrapeurs de Serdaigle et de Serpentard étaient… résolument… incapables… d'attraper… le Vif.

<<~\shout{Il était quasiment sur ta tête, sombre crétin aveugle}~!

--- Exprimez-vous correctement, M. Jordan, ou je vous fais expulser de ce stade~! Bien que j'admette que c'était effectivement très mal joué.~>>

Harry devait lui-même admettre que Lee Jordan et le professeur McGonagall formaient un merveilleux duo comique, avec Jordan en clown rouge et le professeur McGonagall en clown blanc. Il regrettait un peu de n'avoir pas été aux matchs précédents. C'était un côté du professeur McGonagall qu'il n'avait pas vu auparavant.

Quelques sièges devant Harry, dans la section Poufsouffle des gradins de Quidditch, on pouvait voir l'immense silhouette de Cédric Diggory. Le Super Poufsouffle avait observé la collision évitée de peu entre Cho Chang et Terence Higgs avec l'œil acéré d'un Attrapeur et d'un capitaine d'équipe à part entière.

<<~L'Attrapeur Serdaigle est nouveau, dit-il. Mais Higgs est en septième année. J'ai joué contre lui. Il vaut mieux que ça.

--- Tu penses que c'est une stratégie~? demanda l'un des Poufsouffle assis à côté de Cédric.

--- Ça aurait un sens si Serpentard avait besoin de quelques points supplémentaires pour remporter la coupe de Quidditch, répondit Cédric. Mais ils l'ont déjà dans la poche. Qu'est-ce qui leur passe par la tête~? Ils auraient pu gagner~!~>>

Le match avait commencé à six heures du soir. La partie se terminait d'habitude aux environs de sept heures, suite à quoi on allait dîner. Les juins écossais offraient de longues journées~; le crépuscule n'arriverait pas avant dix heures.

La montre de Harry affichait huit heures six minutes lorsque Cédric bondit hors de son siège et s'écria, juste après un but Serpentard faisant passer le score à 170-140.

<<~\emph{Ces bâtards~!}

--- Ouais~!~>> répondit un jeune garçon à côté de lui, bondissant lui aussi. <<~Pour qui ils se prennent, à marquer des points comme ça~?

--- C'est pas ça, s'écria Cédric. Ils… ils essaient de nous \emph{voler la coupe~!}

--- Mais on a déjà perdu la…

--- \emph{Pas la coupe de Quidditch~! La coupe des Maisons~!}~>>

La rumeur se répandit et déclencha des éclats d'indignation.

Harry se dit que le moment était venu.

Il demanda poliment à un Poufsouffle assis à côté de lui et à une autre situé un gradin au-dessus de bien vouloir se décaler. Puis il extirpa un immense parchemin de sa bourse et le déroula en une bannière de deux mètres de haut qui se mit à léviter. L'enchantement venait d'un Serdaigle en sixième année connu pour être encore plus ignorant du Quidditch que ne l'était Harry.

Sur d'immenses lettres violettes et lumineuses, on pouvait lire~:

\begin{center}
ACHETEZ-VOUS UN CHRONO

2h06mn47s
\end{center}

En dessous se trouvait un Vif d'or. Un grand X rouge clignotait par-dessus.

\later

Chaque seconde, le chrono avançait.

À mesure qu'il avançait, des Poufsouffle toujours plus nombreux désiraient soudain venir s'asseoir sous la bannière de Harry.

Lorsque le match s'étira passé neuf heures, de nombreux Gryffondor les avaient rejoints.

Lorsque le soleil se coucha et que Harry lança un Lumos pour lire ses livres - ils avaient arrêté de suivre le match il y a longtemps - une quantité respectable de Serdaigle les avaient rejoints, trahissant le patriotisme au profit du bon sens.

Et le professeur Sinistra.

Et le professeur Vector.

Et lorsque les étoiles se firent visible, le professeur Flitwick aussi.

Le grand match final de Quidditch… s'éternisait.

\later

Lorsqu'il avait préparé tout cela, Harry n'avait pas prévu qu'il serait encore là - il jeta un coup d'œil à sa montre - à onze heures quatre. Il lisait à présent un livre de Métamorphose de sixième année~; ou plutôt, il avait ouvert le livre, l'avait illuminé avec un bâtonnet luminescent Moldu, et faisait l'un des exercices. La semaine dernière, Harry avait surpris une conversation entre des Serdaigle de dernière année au sujet de leurs notes d'ASPIC et il avait entendu que les exercices pratiques consistaient parfois à s'entraîner à “former” ses Métamorphoses, ce qui exigeait plus de contrôle et de précision de pensée que de puissance brute~; Harry avait promptement décidé de s'y mettre, juste après s'être donné un bon coup sur le front pour n'avoir pas essayé plus tôt de lire \emph{tous} les manuels des années suivantes. Le professeur McGonagall avait autorisé Harry à pratiquer un exercice de forme destiné à contrôler la façon dont un objet en pleine Métamorphose atteignait sa forme finale - par exemple, à métamorphoser une plume afin que la tige pousse en premier et que les barbes apparaissent ensuite. Harry faisait un exercice similaire sur des crayons~; il faisait d'abord pousser la mine, puis le bois, et enfin la gomme. Comme il s'en était douté, la concentration sur une partie précise de la transformation s'était avérée similaire à la discipline mentale qu'il utilisait lors de Métamorphoses partielles. La Métamorphose partielle aurait d'ailleurs pu produire le même effet, en métamorphosant d'abord les couches externes de l'objet. Mais cette façon de faire s'avérait plus simple.

Harry finit de métamorphoser un crayon et releva les yeux vers le match de Quidditch. Oui, toujours ennuyeux à mourir. Lee Jordan poursuivait son commentaire d'un ton de morne dégoût~: <<~Dix points de plus, youhou, ouais, et maintenant quelqu'un reprend le Souafle, je sais pas qui et je m'en fiche.~>>

Presque aucun des spectateurs encore présents ne faisaient plus attention au match puisqu'ils avaient presque tous découvert un sport bien plus intéressant~: débattre de la meilleure façon de modifier les règles de la coupe des Maisons et/ou du Quidditch. Les échanges étaient devenus si virulents que tous les professeurs présents suffisaient à peine à maintenir l'ordre et que l'on était à un cran d'une bataille rangée. Malheureusement, ce débat se jouait entre bien plus de deux camps. De sales gêneurs proposaient des alternatives raisonnables à l'élimination pure et simple du Vif d'or, ce qui menaçait de diviser les votes et de freiner l'élan réformiste.

Rétrospectivement, songea Harry, il aurait été bon de voir Drago dérouler sa bannière dans le camp d'en face, disant “LES VIFS D'OR ON ADORE”, afin d'établir les polarités du débat. Plus tôt, Harry avait scruté les gradins Serpentard, mais il n'avait vu Drago nulle part. Severus Rogue, qui aurait peut-être eu la gentillesse de bien vouloir jouer le rôle de l'infâme opposant, manquait aussi à l'appel.

<<~M. Potter~?~>> dit une voix à côté de lui.

Sur le siège suivant se trouvait un garçon de Poufsouffle, petit, mais plus âgé, quelqu'un que Harry n'avait jamais vu auparavant, avec dans les mains une enveloppe de parchemin vierge scellée par un sceau de cire. La cire était vierge elle aussi, sans sceau identificateur.

<<~De quoi s'agit-il~? demanda Harry.

--- C'est \emph{moi}, dit le garçon. Avec l'enveloppe que vous m'avez remise. Je sais que vous avez dit que je ne devais pas vous parler, mais…

--- Alors ne me parle pas~>>, répondit Harry.

Le garçon jeta l'enveloppe à Harry et s'en fut d'un air offensé. Harry grimaça un peu, mais était donné les problèmes temporels en jeu, ça n'avait probablement pas été la \emph{mauvaise} décision…

Il brisa ensuite le sceau de cire et sortit son contenu de l'enveloppe. À la place du papier qu'il s'attendait à trouver, il découvrit que c'était une feuille de parchemin, mais l'écriture était la sienne, faite à la plume plutôt qu'au stylo. Le parchemin disait~:

\begin{writtenNote}
Prends garde à la constellation,\\
et aide celui qui contemple les étoiles
\\
Passe inaperçu des complices des mange-vie,\\
et des sages et des bienveillants.
\\
À six et sept dans un carré,\\
dans le lieu interdit et vraiment débile.
\end{writtenNote}

Harry lut les vers en un clin d'œil, replia la feuille et la plaça dans sa bourse avec un soupir. “Prends garde à la constellation”, vraiment~? Il se serait attendu à ce qu'une énigme laissée par lui-même à sa propre intention soit plus facile à interpréter… même si elle était évidente par endroits. Le futur Harry avait clairement eu peur que le parchemin soit intercepté, et même si le Harry du présent n'aurait pas naturellement vu les Aurors comme des “complices des Détraqueurs d'Azkaban”, peut-être que c'était la meilleure façon de les décrire sans alerter un lecteur inopportun désireux de décrypter le message. Si l'on traduisait l'idiome Fourchelangue qu'il avait utilisé lors de l'incident à Azkaban… peut-être bien que ça collait.

Le message indiquait que le professeur Quirrell avait besoin d'aide et que tout devait se faire à l'insu des Aurors, de Dumbledore, de McGonagall et de Flitwick. Puisque des voyages temporels étaient en jeu, la solution évidente était de partir aux toilettes, remonter le temps, et revenir au match juste après son départ.

Il commença à se lever de son siège, puis hésita. Son côté Poufsouffle lui fit remarquer qu'il laissait son escorte d'Aurors derrière lui et qu'il n'alertait pas le professeur McGonagall. Il se demanda si son lui futur n'agissait pas comme un \emph{idiot}.

Harry déroula le parchemin une fois de plus et le lut à nouveau.

À y regarder de plus près… l'énigmatique parchemin ne disait pas que Harry ne pouvait emmener \emph{personne} avec lui. Drago Malfoy… est-ce qu'il était absent parce que le futur Harry, plusieurs heures dans le passé, avait emmené Drago avec lui pour le soutenir~? Mais ça n'avait aucun sens, emmener un autre élève de première année n'augmentait la sécurité que de façon marginale…

… Drago Malfoy qui aurait certainement dû être présent ici, quoi qu'il pense du Quidditch, pour regarder Serpentard se saisir de la coupe des Maisons. Est-ce que quelque chose lui était arrivé~?

Harry se sentit soudain beaucoup moins fatigué.

Des gouttes d'adrénaline se répandaient en lui, mais non, ce ne serait pas comme avec le troll. Le message avait \emph{dit} à Harry quand il devait arriver. Il ne serait pas en retard, pas cette fois.

Harry regarda Cédric Diggory, dont la tête oscillait de gauche à droite, visiblement déchiré entre des Serdaigle affirmant que le Vif devait être maintenu parce que c'était la tradition et que les règles étaient ce qu'elles étaient~; et des Poufsouffle affirmant que ce n'était pas juste que l'Attrapeur soit plus important que les autres.

Cédric Diggory avait été un excellent professeur de duel et il semblait à Harry qu'ils entretenaient de bons rapports. Mais plus important, un élève qui suivait littéralement tous les cours facultatifs avait probablement son propre Retourneur de Temps. Peut-être pouvait-il essayer d'emmener Cédric dans le passé avec lui~? Le Super Poufsouffle serait probablement une bonne baguette à avoir sous la main en situation difficile…

\later

\emph{Plus tard, et plus tôt~:}

La montre de Harry indiquait maintenant 11:45, c'est-à-dire 6h45 après être revenu 5 heures dans le passé.

<<~Il est temps~>>, murmura Harry au couloir vide, et il commença à s'avancer au-dessus du grand escalier du troisième étage, à droite.

<<~Le lieu interdit~>>, signifiait généralement la Forêt Interdite~; c'était probablement ce que penserait quelqu'un qui intercepterait le message. Mais elle était immense et contenait plus d'un lieu notable. Pas de point Schelling où se retrouver ni d'événement autour duquel se rassembler.

Mais lorsqu'on ajoutait <<~et vraiment débile~>>, un seul endroit interdit de Poudlard demeurait plausible.

Et Harry s'avança donc sur ce chemin interdit où, si la rumeur était vraie, tous les Gryffondor de première année s'étaient déjà rendus. Le couloir du troisième étage, à droite. Une porte mystérieuse menait à une série de pièces remplies de pièges dangereux, potentiellement mortels, que personne ne pouvait espérer traverser, encore moins si on était en première année.

Harry lui-même ignorait le genre de piège qui l'attendait. Ce qui, réflexion faite, signifiait que les élèves qui s'y étaient rendus avaient particulièrement pris soin de ne pas gâcher l'énigme pour les autres. Peut-être qu'il y avait là-bas un panneau disant \emph{Faites-moi une faveur, ne dites rien aux autres. Bien à vous, Albus Dumbledore}. Tout ce que Harry savait, c'était que la porte extérieure serait ouverte par un \emph{Alohomora} et que la dernière pièce contenait un miroir magique qui vous présentait votre reflet dans une situation particulièrement désirable, et c'était apparemment là toute la récompense.

Le couloir du troisième étage était éclairé d'une lumière bleue tamisée apparemment sans point d'origine. Ses arches étaient recouvertes de toilettes d'araignées, comme si cela faisait des siècles, et non des heures, que personne ne l'avait utilisé.

La bourse de Harry était remplie d'objets Moldus et sorciers utiles, ainsi que de tout ce qui pouvait ressembler à un objet de quête (Harry avait demandé au professeur McGonagall de lui recommander un sorcier capable d'augmenter la capacité de sa bourse et elle l'avait fait elle-même). Harry avait lancé le charme appris en lors de batailles qui maintenait ses lunettes collées à son visage. Il avait rechargé les Métamorphoses qu'il maintenait, le petit joyau sur son anneau et l'autre, juste au cas où il s'évanouirait. Il n'était pas prêt à tout, pas au sens propre, mais il était aussi prêt qu'il pensait pouvoir l'être.

Les carreaux hexagonaux du sol craquaient sous ses pieds et disparaissaient derrière lui comme le futur devenant passé. Il était presque 6h49 - \emph{six, et sept dans un carré}. Évident si on pensait en termes mathématiques Moldus, beaucoup moins sinon.

Alors que Harry allait prendre un virage, quelque chose lui picota l'esprit, et il entendu une voix faible.

<<~Personne raisonnable… attendre plus tard… après qu'un certain membre du personnel soit parti…~>>

Harry s'arrêta puis s'avança aussi discrètement que possible. Il ne franchit pas l'angle et essaya de mieux entendre la voix du professeur Quirrell.

Puis vint une toux, plus forte, et la voix parla de nouveau. <<~Mais s'ils partaient… eux aussi… au même moment… murmura la voix, ils pourraient penser… que le match final… serait la meilleure des distractions… restantes de l'année… une distraction prévisible. Alors j'ai cherché… quelles personnes importantes… étaient absentes du match… et j'ai vu que le directeur manquait… mais si j'écoutais ma magie… il pourrait aussi bien… être dans un autre monde… et j'ai aussi remarqué… votre absence… et j'ai décidé de me rendre… où vous étiez. C'est ce que je fais ici… maintenant… que faites \emph{vous} ici~?~>>

Harry eut soudain le souffle court. Il continua d'écouter.

<<~Et comment saviez-vous que j'étais ici~?~>> répondit la voix traînante de Severus Rogue, tellement plus forte que Harry sursauta.

Un petit rire dans une quinte de toux. <<~Vérifiez votre baguette… la Trace.~>>

Severus prononça quelques mots en pseudo-Latin magique, puis~:

<<~Vous avez osé trafiquer ma baguette~? Vous avez \emph{osé}~?

--- Vous êtes un suspect… tout comme moi… aussi votre fausse indignation… est en vain… si finement jouée soit-elle… maintenant, dites-moi… que faites-vous~?

--- Je garde cette porte, dit le professeur Rogue. Et je vous demanderai de vous en éloigner~!

--- À quel titre… me donnez-vous des ordres… cher collègue~?~>>

Il y eut un silence. <<~Eh bien, sur ordre du directeur, répondit Severus Rogue d'une voix mielleuse. Il m'a ordonné de garder cette porte pendant le match de Quidditch, et en tant que professeur, je me dois d'obéir à ses lubies. J'en parlerai plus tard au conseil d'administration, mais pour le moment, je dois obéir. Maintenant, partez, ainsi que le désire le directeur.

--- Comment~? Vous voulez dire que je dois croire… que vous avez abandonné vos Serpentard… pendant leur match le plus… important de l'année… et avez bondi comme un chien… au premier ordre de Dumbledore~? Eh bien… je dois dire… que c'est tout à fait plausible. Malgré tout… je pense qu'il serait sage… que je vous garde moi-même… pendant que vous gardez cette magnifique porte.~>> Il y eut un bruit de tissu et un coup léger, comme si quelqu'un s'était brutalement assis ou était tombé par terre.

<<~Oh, pour l'amour de Merlin…~>> la voix de Severus Rogue était à présent pleine de colère. <<~Levez-vous~!

--- Ba-blu-bu-blah… répondit un professeur de Défense en mode zombie.

--- Levez-vous~!~>> dit Severus Rogue, et il y eut un coup étouffé.

\emph{Aide celui qui contemple les étoiles…}

Harry passa l'angle. Peut-être l'aurait-il fait même sans message intertemporel. Le professeur Rogue venait-il de donner un coup de pied au professeur Quirrell~? Même si ce dernier avait été mort et \emph{enterré}, ç'aurait été fort imprudent.

Une porte au sommet arrondi, faite de bois sombre, était encadrée d'une arche de pierre et engoncée dans les briques de marbres poussiéreuses de Poudlard. Là où un Moldu aurait placé une poignée, on ne trouvait qu'une barre de métal poli~; il n'y avait aucun loquet, aucune serrure visible. De chaque côté, accrochées au mur, deux torches brûlaient d'une inquiétante lueur orange. Devant la porte se trouvait le maître des potions et ses robes tachées comme à l'habitude. À côté, à gauche et sous une torche, la silhouette affaissée du professeur Quirrell, dos contre le mur, visage levé. Les yeux semblèrent briller, comme pris entre la conscience et le néant.

<<~Qu'est-ce que vous pouvez bien~>>, dit la haute silhouette du maître des potions, <<~faire ici, \emph{Potter}~?~>>

À en juger son visage et son ton, le maître des potions était plutôt en colère contre Harry et n'était certainement pas un de ses co-conspirateurs dans une entreprise à laquelle le professeur de Défense n'aurait pas été invité.

<<~Je n'en suis pas sûr~>>, répondit Harry. Il n'était pas certain du rôle qu'il était censé joué, et donc, par désespoir, se rabattait sur l'honnêteté. <<~Je crois que je suis peut-être censé garder un œil sur le professeur de Défense.~>>

Le maître des potions le regarda froidement. <<~Où est votre \emph{escorte}, Potter~? Les élèves ne doivent pas se promener seuls dans ces couloirs~!~>>

Harry ne sut sincèrement pas quoi répondre. La partie avait commencé et il ne connaissait même pas les règles. <<~Je ne sais pas bien comment répondre à ça…~>>

Les traits froids du professeur Rogue vacillèrent.

<<~Peut-être que je devrais appeler les Aurors, dit-il.

--- Attendez~!~>> dit hâtivement Harry.

Le maître des potions avait approché une main de ses robes.

<<~Et pourquoi~? répondit-il.

--- Je… je pense que vous ne devriez pas les appeler…~>>

La baguette du maître des potions était soudain dans sa main. <<~\emph{Nullus confundio~!}~>>. Un trait noir jaillit et toucha Harry là où il venait de bondir afin d'esquiver. Quatre autres sortilèges suivirent le premier, composés de mots comme \emph{Polyfluis} et \emph{Metamorphus}~; pour ceux-là, Harry se tint poliment immobile.

Lorsqu'il fut clair qu'aucun de ces sortilèges n'avait produit de résultat, Severus Rogue fixa Harry d'un regard sombre, apparemment sincère cette fois.

<<~Je vous suggère de vous expliquer, M. Potter, dit-il doucement.

--- Je ne peux pas m'expliquer, dit Harry. Je n'en ai pas encore le Temps.~>>

Il regarda droit dans les yeux du maître des potions en prononçant les mots \emph{je} et \emph{temps} tout en écarquillant les siens dans l'espoir de lui transmettre l'essentiel. Le maître des potions hésita.

Harry tentait désespérément de comprendre qui prétendait être qui. Puisque le professeur Quirrell n'était pas convié aux plans secrets de Dumbledore, Severus prétendait être le maléfique maître des potions de Poudlard envoyé ici par le directeur… qui l'avait peut-être bien envoyé ici… mais le professeur Quirrell pensait ou prétendait croire que le professeur Rogue devait être surveillé… et Harry avait été envoyé ici par le Harry du futur, bien qu'il ignore pourquoi… et d'abord, pourquoi se tenaient-ils tous devant la porte interdite du directeur~?

Et alors…

Derrière Harry…

On entendit le bruit de pas qui approchaient~; rapides et nombreux.

Le professeur Rogue agita sa baguette une fois et le professeur de Défense, toujours étendu, fut entouré de ténèbres.

<<~\emph{Muffiato}, siffla le maître des potions. M. Potter, si vous souhaitez rester, cachez-vous~! Mettez votre cape d'invisibilité~! Mon devoir est de garder cette porte, au cas où \emph{il} surgirait. Il y a eu… une \emph{perturbation} destinée à faire partir le directeur… du moins c'est ce qu'il pense…

--- Qui…~>>

Severus fit un grand pas en avant et frappa la tempe de Harry du bout de sa baguette. Harry eut la sensation qu'on lui avait brisé un œuf sur le crâne~: c'était un sortilège de Désillusion. Les mains de Harry s'estompèrent, et le reste suivit bientôt.

Les ténèbres qui avaient englouti tout un pan du mur se dissipèrent comme une lourde brume et la silhouette recroquevillée et silencieuse du professeur Quirrell lui fut de nouveau visible.

Harry s'éloigna sur la pointe des pieds aussi silencieusement que possible, puis il se retourna pour observer.

Les pas qui approchaient dépassèrent le coin du couloir…

<<~Qu'est-ce que vous faites là~?~>> dirent plusieurs voix à l'unisson.

Liserés de trois verts Serpentard et d'un jaune Poufsouffle, c'étaient Theodore Nott, Daphné Greengrass, Susan Bones et Tracey Davis.

<<~\emph{Où donc}~>>, dit le professeur Rogue, dont la colère semblait en pleine ascension, <<~sont vos \emph{escortes}, jeunes gens~? Les première année doivent être en permanence accompagnés d'un étudiant de sixième ou septième année~! Vous en particulier~!~>>

Theodore Nott leva une main. <<~Nous, euh, répondit-il. C'est un exercice de la Légion du Chaos, on travaille notre esprit d'équipe… et vous voyez, comme on s'est rendu compte qu'aucun de nous n'avait encore essayé d'entrer la pièce interdite du directeur et que l'année se termine bientôt… et Harry Potter nous l'a permis, professeur, il a spécifiquement dit que vous deviez nous laisser faire.~>>

Severus Rogue leva les yeux vers l'endroit où Harry s'était caché~; un orage semblait se préparer en ses sourcils, une rage sombre naître dans ses yeux.

\emph{Je… peut-être~?} il avait encore une heure devant lui, ce n'était donc pas impossible.

<<~Harry Potter n'a aucune autorité en la matière, répondit le maître des potions d'un ton doucereux. Maintenant, expliquez-vous.

--- Sérieusement~? répondit Susan Bones. Vous êtes sérieusement en train de dire au professeur Rogue que Harry Potter a autorisé cette mission~? C'est votre idée d'un bluff~?~>> La jeune Poufsouffle se retourna vers le professeur Rogue. Sa voix était étrangement ferme~: <<~Monsieur, Drago Malfoy a disparu et nous pensons qu'il est venu ici. C'est la vérité, et le temps presse…

--- Si M. Malfoy a disparu, dit le professeur Rogue, \emph{pourquoi les Aurors n'en ont-ils pas été informés~?}

--- Pour certaines… pour certaines \emph{raisons}~! s'écria Daphné Greengrass. On n'a pas le temps, vous devez nous laisser passer~!~>>

Le ton du professeur Rogue était plus sardonique que jamais. <<~Vous seriez-vous mis en tête que vous êtes en pleine aventure, bande d'idiots~? Détrompez-vous. Je vous assure que M. Malfoy n'a pas franchi cette porte.

--- Nous pensons que M. Malfoy a une cape d'invisibilité, répondit vivement Susan Bones. Vous rappelez-vous avoir vu la porte s'ouvrir, sans raison apparente~?

--- Non, répondit que maître des potions. Maintenant, partez. Ce couloir est interdit d'accès pour le reste de la journée.

--- C'est le couloir interdit de \emph{Dumbledore}, dit Tracey. Il a dit lui-même que personne ne devait venir ici. Vous vous prenez pour qui, à l'interdire aussi~?

--- Mlle Davis, dit le maître des potions, je vous suggère d'éviter la compagnie des Gryffondor, en particulier de ceux qui s'appellent Lavande Brown. Et si vous êtes encore ici dans une minute, je soumettrai une requête de transfert qui vous enverra chez eux.

--- \emph{Vous n'oseriez pas~!}, s'écria-t-elle.

--- Euh~>>, dit Susan Bones, les traits tirés par la concentration. <<~Professeur Rogue, vous arrive-t-il d'ouvrir la porte pour jeter un coup d'œil à ce qu'il y a derrière~?~>>

Le professeur Rogue se figea. Puis il pivota et plaça sa main sur le heurtoir en métal…

Harry regardait la main placée sur le heurtoir, et il ne vit donc pas ce que Rogue faisait de sa main gauche avant d'entendre de soudains éclats de voix.

<<~À vrai dire non~>>, dit le professeur Rogue, sa main gauche maintenant serrée autour du col de Drago Malfoy. Hormis sa tête, le reste du corps de ce dernier était toujours masqué par une cape d'invisibilité. <<~Bien essayé, toutefois.

--- \emph{Quoi~?}~>> s'écrièrent Tracey et Daphné.

Susan Bones se frappa le front.

<<~J'arrive pas à croire que je me suis fait avoir par ça…

--- Donc, M. Malfoy~>>, continua le professeur Rogue. Sa voix avait baissé d'un ton. <<~Vous envoyez vos amis ici pour me rouler… dans le seul espoir de franchir cette porte~? Pourquoi feriez-vous une chose pareille~?

--- Je pense qu'on peut lui faire confiance… dit Theodore Nott. M. Malfoy, on \emph{doit} lui faire confiance, c'est le seul professeur qui prendra notre parti~!

--- Non~!~>> s'écria la tête en lévitation de Drago Malfoy, une main toujours accrochée à son col. <<~Ne dites rien~! Taisez-vous~!

--- On doit tenter le coup~! cria Theodore. Professeur Rogue, M. Malfoy a fini par comprendre ce qui s'est passé cette année, et pourquoi… pourquoi Dumbledore essaie d'empêcher Nicholas Flamel de remettre les mains sur la Pierre Philosophale~! C'est parce qu'il pense que personne ne devrait être immortel~! Alors il a essayé de convaincre Flamel que le Seigneur des Ténèbres était revenu et qu'il avait besoin de la pierre pour continuer à vivre, et il a demandé à Flamel de la lui donner à lui, mais Flamel n'a pas voulu et a mis la pierre dans le miroir magique qui est derrière cette porte et Dumbledore est en train de découvrir comment l'atteindre et ensuite il ira la chercher et on doit la prendre en premier~! Dumbledore sera vraiment tout-puissant s'il met la main sur la Pierre Philosophale~!

--- \emph{Quoi~?} dit Tracey. C'est pas ce que t'avais dit plus tôt~!

--- Ça…~>> répondit Daphné. Elle avait l'air effrayée, mais résolue. <<~Ça n'a pas d'importance… Professeur Rogue, s'il vous plaît, vous devez me faire confiance. J'ai consulté les livres que Hermione avait empruntés à la bibliothèque. Elle faisait des recherches sur la Pierre Philosophale juste avant d'être tuée. Ses notes disaient que quelque chose de dangereux peut se produire quand on laisse la Pierre Philosophale dans le miroir pendant trop longtemps. On doit la faire sortir du château tout de suite.~>>

Les mains de Susan Bones recouvraient à présent le visage de cette dernière. <<~Je ne suis pas avec eux. Je les ai juste accompagnés pour leur éviter de faire quelque chose d'encore plus stupide.~>>

Severus Rogue regardait Theodore Nott et les autres. Puis il se retourna vers Drago Malfoy.

<<~M. Malfoy, dit-il d'une voix traînante. Comment avez-vous découvert le plan de Dumbledore~?

--- Je l'ai déduit à partir d'indices~!~>> dit la tête de Drago Malfoy.

Celle du professeur Rogue pivota vivement vers Theodore Nott.

<<~Comment comptiez-vous récupérer la pierre cachée dans un miroir censé mettre Dumbledore lui-même en échec~? Répondez, maintenant~!

--- On allait prendre le miroir et le renvoyer à Flamel, dit Theodore Nott. Ce n'est pas comme si on voulait la pierre, on veut juste empêcher Dumbledore de la voler.~>>

Le professeur Rogue hocha la tête comme s'il venait d'obtenir une confirmation et dévisagea les autres élèves. <<~Dites-moi, avez-vous observé l'un d'entre vous se comporter de façon étrange~? Plus particulièrement, cette personne aurait-elle un objet étrange en sa possession, ou connaîtrait-elle des sortilèges qu'un élève de première année ne devrait pas connaître~?~>> La main droite du professeur Rogue dirigea sa baguette vers Susan Bones. <<~Je constate que Mlle Greengrass et Mlle Davis essaient de ne pas vous regarder, Mlle Bones. S'il y a une explication logique à ça, je vous suggère de m'en faire part \emph{immédiatement}.~>>

Les cheveux de Susan Bones devinrent rouge vif, mais son visage ne changea pas.

<<~J'imagine que ce n'est plus la peine de me cacher, dit-elle, puisque je serai diplômée dans deux jours.

--- Les doubles sorcières reçoivent leur diplôme \emph{six ans} plus tôt~? dit Tracey Davis. C'est injuste~!

--- \emph{Bones est une double sorcière~?} s'écria Theodore.

--- Non, c'est Nymphadora Tonks, une métamorphomage, dit le professeur Rogue. Comme vous le savez, le règlement interdit strictement de se faire passer pour un autre élève. Il n'est pas trop tard pour que vous soyez renvoyée de Poudlard, ce qui serait tragique… de votre point de vue. Du mien, ce serait hilarant. Maintenant, dites-moi exactement ce que vous faites ici.

--- Je comprends mieux, dit Daphné Greengrass. Euh, est-ce qu'il y a une \emph{vraie} Susan Bones, ou est-ce que notre Maison se meurt tellement qu'ils vous ont fait…~>>

La personne qui ressemblait à Susan Bones semblait avoir envie de se taper la tête contre les murs.

<<~Oui, Mlle Greengrass, il y a une véritable Susan Bones. Elle ne m'envoie que quand \emph{vous} êtes sur le point de vous mettre dans de beaux draps. Professeur Rogue, je suis ici parce que Drago Malfoy avait disparu, et que ceux-là ont \emph{soutenu} qu'il fallait aller le chercher plutôt que d'alerter les Aurors. Mlle Bones m'a dit qu'elle n'avait pas le temps de m'expliquer ses raisons, et je me rends compte maintenant que c'était stupide. Mais de jeunes élèves ne doivent pas s'aventurer seuls dans les couloirs et doivent être en permanence accompagnés d'un étudiant en sixième ou septième année. Et maintenant, on a Drago Malfoy et on peut tous rentrer. S'il vous plaît~? Avant que ça ne devienne encore plus ridicule~?

---\emph{Par Merlin, qu'est-ce qui se passe ici~?}

--- Ah~>>, dit le professeur Rogue, qui pointait toujours sa baguette vers une Susan Bones aux cheveux rouges et dont l'autre main tenait toujours le col de la tête sans corps de Drago Malfoy, et dont les pieds étaient à quelques centimètres du corps affalé du professeur de Défense. <<~Professeur Chourave, n'est-ce pas.

--- Ce n'est pas ce que vous croyez~>>, s'expliqua Tracey Davis.

La silhouette courtaude du professeur de Botanique s'avança à grands pas. Elle avait déjà sorti sa baguette, mais ne la dirigeait vers personne en particulier. <<~\emph{Je ne sais même pas quoi en penser~!} Abaissez tous vos baguettes \emph{immédiatement}~! Vous aussi, professeur~!~>>

\emph{Distraction}. L'idée vint à Harry avec une soudaine clarté. Ce qu'il regardait maintenant, invisible, retiré de l'action, ce n'était pas la scène importante~; ce n'était pas le fil narratif principal~; ça avait été \emph{planifié}. L'arrivée du professeur Chourave avait mis fin à la suspension de l'incrédulité de Harry. Ce genre de chose ne se produisait pas par pur souci de produire un comique de situation. Quelqu'un avait délibérément déclenché tout ce chaos, mais dans quel but~?

Harry espérait vraiment ne pas être remonté dans le temps pour faire tout ça… parce que c'était exactement le genre de chose qu'il aurait pu faire.

Severus Rogue abaissa sa baguette. Son autre main relâcha Drago Malfoy.

<<~Professeur Chourave, dit le maître des potions, je suis ici sur ordre du directeur afin de garder cette porte. Toutes les autres personnes présentes ici ne devraient \emph{pas} l'être, et je vous prie de les faire partir.

--- Une histoire plausible,~>> dit le professeur Chourave. <<~Pourquoi Dumbledore choisirait vous plutôt que n'importe qui d'autre pour garder la porte de son terrain de jeu~? Ce n'est pas comme s'il voulait empêcher les élèves d'entrer, oh que non, il faut qu'ils rentrent et qu'ils se prennent dans \emph{mon} filet du diable~! Chère Susan, vous avez un miroir de communication~? Utilisez-le pour prévenir les Aurors.~>>

Harry hocha la tête pour son propre compte. C'était \emph{ça}, le but. Les Aurors embarqueraient tous ceux qui avaient participé à cette situation absurde, n'écouteraient aucune explication, et il n'y aurait plus personne pour surveiller la porte.

Mais Harry était-il censé se rendre dans le couloir interdit~? Ou observer, voir qui arriverait enfin, après le départ des autres~?

Une violente quinte de toux dirigea tous les regards vers l'endroit où le professeur de Défense reposait.

<<~Rogue… écoutez-moi… dit le professeur de Défense entre deux quintes de toux. Si… Chourave… est ici…~>>

Le maître des potions baissa les yeux.

<<~Sortilège d'Oubliettes… nécessite… un professeur…~>> le professeur de Défense se remit à tousser.

<<~\emph{Hein~?}~>>

Et la logique se déroula dans l'esprit de Harry, le désarroi cristallin, chaque étape déjà soupçonnée, la terrible compréhension répétée, plus sûre à chaque fois.

Quelqu'un avait modifié les souvenirs de Hermione pour qu'elle croie avoir essayé de tuer Drago.

Seul un professeur de Poudlard aurait pu le faire sans déclencher d'alarme.

Le véritable cerveau de l'affaire n'avait donc eu besoin que de lancer Legilimens ou Imperius sur un professeur de Poudlard.

Et la dernière personne que l'on soupçonnerait était la directrice de Poufsouffle.

Rogue pivota brusquement, le professeur Chourave leva sa baguette, et le maître des potions parvint à ériger silencieusement un mur translucide entre eux. Mais le tir qui émana de la baguette de Chourave fut d'un marron noirâtre qui produisit une soudaine anxiété dans l'esprit de Harry~; le tir fit disparaître le bouclier un instant avant de le toucher et entailla le bras du maître des potions en pleine esquive. Il émit un cri étouffé, sa main fut saisie d'un spasme, et il laissa tomber sa baguette.

Le tir suivant fut rouge vif, de la couleur d'un sortilège d'étourdissement, et sembla augmenter en intensité et en vitesse dès qu'il s'éloigna de la baguette, provoquant une nouvelle montée d'anxiété~; le tir projeta le maître des potions contre la porte et le laissa immobile au sol.

Une Susan Bones aux cheveux roses était déjà enveloppée d'un halo bleu à facettes, et elle tira sort après sort vers le professeur Chourave. Celle-ci les ignora et invoqua des pousses qui enserrèrent les plus jeunes élèves qui essayaient de s'enfuir~; sauf Drago Malfoy, qui avait une fois de plus disparu sous sa cape d'invisibilité.

Celle qui n'était pas Susan Bones cessa d'envoyer des sortilèges. Elle tint sa baguette en position horizontale, prit une profonde inspiration et cria une incantation qui envoya des vers de lumière grignoter le bouclier qui entourait le professeur Chourave. Le professeur de Botanique fit à nouveau face à pas-Susan, le regard vide, de nouveaux tentacules dressés derrière elle. Ces tiges semblaient être d'un vert plus sombre et disposer de leurs propres boucliers.

Harry Potter murmura, apparemment dans le vide~:

<<~Attaque Chourave. Aide Bones. Attaques non mortelles seulement.

--- Oui, Seigneur,~>> murmura un Lesath Lestrange caché sous la Cape d'Invisibilité de Harry~; et le Serpentard en cinquième année s'avança vers la bataille, toujours dissimulé.

Harry baissa les yeux vers ses mains et découvrit avec stupeur et déplaisir que son sortilège de Désillusion n'était pas aussi total qu'auparavant. L'air semblait se tordre à chaque fois qu'il bougeait…

Il fit quelques pas lents en arrière jusqu'à passer un angle, et il s'accroupit. Puis il sortit son miroir de communication… qui était brouillé. Bien sûr. Il fit léviter le miroir afin de pouvoir observer la fin de la… distraction~? \emph{Qu'est-ce qui se passait~? Et pourquoi~?}

Sous des volées de lumière et de feuilles, le duel entre le professeur Chourave et pas-Susan Bones se poursuivait. Le vert vif d'un sortilège de bris de bouclier supérieur surgit de nulle part et dévora la moitié des boucliers du professeur Chourave. Le professeur de Botanique pivota et envoya une grande vague jaune vers l'origine du sortilège de bris, mais la vague sembla passer sans rien toucher.

Des éclairs jaunes, des facettes bleues, des pousses vert sombre et des pétales mauves tourbillonnants…

C'est quand le professeur Chourave commença à faire jaillir des arcs rouge cramoisi en tous sens que l'une des lames rouges se planta dans quelque chose. La Cape d'Invisibilité ne cacha pas l'absorption et la disparition de l'arc~; la présence invisible de Lesath tomba au sol.

Ce qui donna à pas-Susan Bones le temps de reprendre son souffle et de hurler quelque chose qui inspira à Harry un nouveau sentiment d'appréhension~; l'éclair blanc traversa les boucliers mâchonnés du professeur Chourave, son armure de plante, et l'abattit.

Celle qui n'était pas Susan Bones tomba à genoux, haletante, ses robes trempées de sueur.

Elle regarda autour d'elle, les corps stupéfixés ou entourés de plantes.

<<~Quoi, dit pas-Susan. Quoi. Quoi. \emph{Quoi.}~>>

Il n'y eut pas de réponse. Les victimes prises dans les tiges du professeur Chourave semblaient immobiles, mais elles paraissaient respirer.

<<~Malfoy…~>> dit une fausse Susan aux cheveux roses, toujours haletante. <<~Drago Malfoy, où es-tu~? Est-ce que tu es là~? Appelle vite les Aurors~! Bordel de Merlin - \emph{Hominum Revelio~!}~>>

Et Harry fut de nouveau visible, et vit dans son miroir un Drago Malfoy à moitié visible derrière une cape miroitante, debout derrière pas-Susan, sa baguette pointée droit vers un trou dans le halo bleu de cette dernière.

L'esprit de Harry se déplaça par bonds de compréhension soudaine, trop lentement, et pourtant trop vite~; il inhala, bouche béante, se prépara à crier.

\emph{prends garde à la constellation}\\ il y avait une constellation du Dragon\\ si on pouvait contrôler un professeur, on pouvait contrôler un élève

<<~\emph{Baisse-toi~!}~>> hurla Harry, mais c'était trop tard. Tiré à bout portant, une salve rouge frappa la tête de pas-Susan et la fracassa au sol.

Harry franchit l'angle du couloir et dit <<~\emph{Somnium Somnium Somnium Somnium Somnium Somnium.}~>>

La silhouette miroitante de Drago s'effondra.

Harry prit un moment pour reprendre son souffle. Puis il dit <<~\emph{Stupéfix~!}~>> et vérifia que le sortilège d'étourdissement avait bien touché le corps de Drago.

(On pouvait croire à tort qu'un Somnium avait touché sa cible. Harry avait vu assez de films d'horreur, sans parler de cette histoire avec le Régiment Soleil, pour ne jamais refaire \emph{cette} erreur.)

Après avoir un peu plus réfléchi, Harry lança un autre sortilège d'étourdissement vers le corps prostré du professeur Chourave.

Puis il serra sa baguette et observa la scène en respirant lourdement, épuisé. Il n'avait plus assez de magie pour envoyer un Patronus prévenir Dumbledore. Cette fois, il aurait \emph{vraiment, vraiment} dû penser à cette possibilité immédiatement. Il commença à revenir vers son miroir, pour voir s'il n'était plus brouillé.

C'est là qu'il hésita.

La note qu'il s'était laissée à lui-même disait d'éviter de prévenir les Aurors, et Harry ne savait \emph{toujours pas} ce qui se passait.

Le corps étendu du professeur Quirrell eut une autre quinte de toux, leva une main vers le mur situé derrière lui et se remit lentement sur pieds.

<<~Harry, croassa le professeur Quirrell. Harry. Tu es là~?~>>

C'était la première fois que le professeur Quirrell l'appelait par son prénom.

<<~Je suis là~>>, dit Harry. Sans même qu'il réfléchisse, ses pieds avançaient.

<<~S'il te plaît, dit le professeur Quirrell. S'il te plaît, je n'ai… plus beaucoup de temps. S'il te plaît… emmène-moi au miroir… aide-moi… à trouver la pierre.

--- La pierre \emph{philosophale}~?~>> dit Harry. Il regarda les corps éparpillés autour de lui, mais Drago n'était plus visible, le sortilège de révélation s'était dissipé. <<~Vous pensez que M. Nott avait \emph{raison}~? Je ne pense pas que Dumbledore…

--- Pas… Dumbledore, s'étrangla le professeur Quirrell. Parce que… Chourave…

--- Je comprends~>>, dit Harry. Si Dumbledore avait été derrière tout cela, il n'aurait pas eu besoin de contrôler l'esprit d'un professeur pour utiliser des sortilèges d'Oubliettes.

<<~Miroir… vieille relique… pourrait dissimuler… n'importe quoi… pierre devrait y être… nombreux autres la désirent… l'un d'eux a… envoyé Chourave…~>>

Harry répéta rapidement~:

<<~Le miroir là-bas est une ancienne relique qui permet de cacher des choses et la Pierre Philosophale s'y trouve peut-être. Si la Pierre Philosophale est dans le miroir, plein de gens peuvent vouloir s'en emparer. L'un d'eux contrôle Chourave, et ça expliquerait son véritable but… seulement… ça n'explique pas pourquoi celui qui contrôle Chourave s'en prendrait à Hermione…

--- Harry, s'il te plaît,~>> dit le professeur Quirrell. Sa respiration était plus laborieuse, sa voix d'une lenteur insoutenable. <<~C'est la seule chose… qui peut me sauver… et je me rends compte… que je ne veux pas mourir… aide-moi, s'il te plaît…~>>

Et le voile se déchira.

Étrangement, c'en fut trop.

Le sentiment de détachement qui s'était emparé de Harry après l'arrivée de Chourave, l'incrédulité retrouvée~: ils étaient à nouveau là. Son Critique Interne soupesait tout d'un seul bloc. Le timing, les probabilités, la présence de tant de gens face à cette porte, le désespoir du professeur de Défense… rien ici ne semblait réel. Mais peut-être qu'il pouvait trouver une \emph{réponse} si seulement il prenait le temps de réfléchir avant au lieu de bondir dès que l'aventure appelait. Toute l'expérience qu'il avait accumulée pendant l'année s'était enfin cristallisée et l'avait un peu aguerri. Un instinct né dans des désastres passés lui soufflait que s'il fonçait droit devant, il aurait une triste conversation où il comprendrait que son comportement avait été stupide. \emph{Une fois de plus}.

<<~Laissez-moi réfléchir, dit-il. Laissez-moi réfléchir une minute avant qu'on y aille.~>> Il se détourna du professeur de Défense et regarda les corps inconscients au sol. Il avait déjà obtenu tellement de pièces du puzzle, peut-être que tout se mettrait en place s'il en trouvait une de plus…

<<~Harry… dit le professeur de Défense d'une voix chancelante, Harry, je suis mourant…~>>

\emph{Une minute de plus ou de moins ne peut pas avoir tant d'importance que ça, il a été malade TOUTE L'ANNÉE, il est PEU PROBABLE que sa vie se joue précisément à cette minute, indépendamment de ce qui est arrivé à Hermione…}

<<~Je \emph{sais}~! répondit Harry. Je vais réfléchir \emph{vite~!}~>>

Harry regarda les corps et essaya de réfléchir. Il n'avait pas le temps de douter, de discuter, de se freiner ou de faire marche arrière, il fallait juste saisir ses pensées au vol et \emph{les suivre}…

En arrière-plan, des fragments de pensée abstraite voletaient, des heuristiques de pensée pratique qu'il n'avait pas le temps de verbaliser. Des flashs muets les firent apparaître, situer le problème principal.

\emph{… je remarque que je suis confus par quoi…}

\emph{… le premier aspect où chercher l'erreur est l'aspect le plus improbable d'une situation…}

\emph{… les explications simples sont les plus probables, il faut éliminer les improbabilités qui n'ont pas de cause propre…}

Le professeur Rogue avait déjà été là, puis le professeur Quirrell était arrivé, puis Harry (via Retourneur de Temps), puis le groupe d'aventuriers, Drago avait été révélé (et faisait partie du groupe), puis le professeur Chourave s'était montrée.

Trop de personnes étaient arrivées \emph{en même temps}, la coïncidence était trop grande, il était \emph{peu probable} que tant de groupes se retrouvent au même endroit sur une durée de cinq minutes, il \emph{devait} y avoir des liens cachés.

Appelons celui qui a ordonné à Chourave de manipuler les souvenirs de Hermione le cerveau. Le cerveau avait envoyé Chourave.

Selon le professeur Rogue, le directeur l'avait envoyé garder la porte suite à une \emph{perturbation}~; si le cerveau avait aussi provoqué cette perturbation, cela expliquait la présence de Severus.

Harry n'était plus sûr que Drago ait été contrôlé par le cerveau. Cette hypothèse lui était venue sur le moment, mais Drago aurait pu essayer d'abattre pas-Susan pour pouvoir entrer dans le couloir sans être gêné…

Non c'était la mauvaise façon d'attaquer le problème, changer de point de vue, essayer d'\emph{expliquer} la présence de Drago et de son groupe à ce moment, pas le temps de douter, \emph{saisis l'hypothèse et fonce}, donc suppose que le cerveau derrière Chourave a envoyé Drago ou s'est arrangé pour qu'il vienne.

Cela expliquait trois arrivées.

Harry était venu parce que sa note à lui-même avait dit de le faire. On pouvait attribuer ça au voyage dans le temps.

Il restait le professeur de Défense, qui disait avoir suivi Rogue, sauf que ça ne ressemblait pas à une bonne explication à la présence du professeur Quirrell, et Harry ne se sentait pas moins confus, alors peut-être que le cerveau avait aussi orchestré le moment de la venue du professeur Quirrell et s'était aussi arrangé pour que Harry lui-même entre dans une boucle temporelle.

L'esprit de Harry rencontra une embûche. Il ne voyait pas comment poursuivre ce raisonnement.

Il n'avait pas le temps de contempler l'embûche.

Sans temps mort, sans ralentir, l'esprit de Harry attaqua le problème d'un autre angle.

Le professeur Quirrell avait déduit qu'un professeur de Poudlard était contrôlé du besoin de manipuler les souvenirs de Hermione, donc le cerveau derrière Chourave avait piégé puis assassiné Hermione donc le cerveau derrière Chourave avait accès à des informations précises sur la vie à l'intérieur de Poudlard et peut-être même un intérêt particulier pour le Survivant et ses amis.

L'esprit de Harry lui balança enfin le souvenir adéquat, Dumbledore disant que la meilleure chance de survie de Voldemort était ici, à Poudlard, \emph{continue sur cette hypothèse, ne t'arrête pas}, donc cette possibilité de résurrection était la Pierre Philosophale, cachée dans le miroir \emph{pourquoi est-ce que Dumbledore l'a mise dans un couloir pour élèves en première année non ignore cette question ça n'a pas d'importance pour l'instant} et le professeur Quirrell avait dit que le Pierre Philosophale avait de grands pouvoirs de guérison, ce qui collait au reste.

Mais si c'était la Pierre Philosophale qu'on avait cachée dans le miroir pour la tenir à l'écart du Seigneur des Ténèbres, ça voulait dire que le miroir contenait aussi la seule chose au monde capable de sauver la vie du professeur de Défense…

L'esprit de Harry tenta d'hésiter, de se dérober~; il ressentit une soudaine appréhension quant à l'issue de cette pensée.

Mais aucun temps n'avait été alloué à l'hésitation.

… et c'était une coïncidence bien trop grande, une improbabilité trop importante si on ne laissait pas son esprit l'ignorer, la prendre comme un incroyable rebondissement, se croire dans une fiction.

Le supposé Seigneur des Ténèbres aurait-il pu aussi manipuler le professeur Quirrell afin que ce dernier découvre comment survivre exactement au bon moment, afin que Harry et le professeur aillent chercher l'outil de résurrection dans le miroir, qui n'était peut-être même pas la Pierre Philosophale, et alors l'avatar du Seigneur des Ténèbres ou un de ses serviteurs apparaîtrait, s'en emparerait, ça expliquerait \emph{toutes} les coïncidences.

Ou le professeur Quirrell avait-il su depuis le début que son seul salut était dans ce miroir et était-ce pour cela qu'il avait accepté d'enseigner le cours de Défense à Poudlard et maintenant il essayait enfin de mettre la main dessus, mais alors pourquoi attendre d'être malade pour essayer et pourquoi Chourave était-elle apparue en même temps que le professeur Quirrell…

L'esprit de Harry chancela encore.

Son œil intérieur pointait vers une direction où il n'osait pas regarder.

\emph{Le message que je me suis envoyé disait d'aider celui qui observe les étoiles. Je ne m'enverrai pas un message de ce genre si je n'avais pas déjà déduit plus tard que c'était la bonne chose à faire… peut-être que le message me dit juste de ne pas hésiter…}

Un soupçon de doute fut propulsé jusqu'à sa conscience.

Le message codé sur le parchemin… une ligne ou deux n'avaient pas été dans le ton, n'avait pas ressemblé à un code que Harry s'attendrait à se voir utiliser…

<<~Harry~>>, murmura la voix mourante du professeur Quirrell, derrière lui. <<~Harry, s'il te plaît,

--- J'ai presque fini de réfléchir~>>, dit la voix de Harry, et il se rendit compte que c'était vrai.

Sous l'angle inverse.

Regarde les choses du point de vue de l'Ennemi, du point de vue depuis lequel il planifie tout, hors de ta vue.

Il y a des Aurors à Poudlard et ta cible, Harry Potter, est à présent sur ses gardes. Il appellera les Aurors au premier signe d'un problème, ou il enverra un Patronus à Albus Dumbledore. Vu comme un puzzle, une solution inventive est de…

… falsifier un message prétendument venu du futur Harry lui disant de ne \emph{pas} appeler à l'aide, lui disant de se rendre au lieu qui convient à l'heure où tu souhaites le voir venir. Tu laisses la cible déjouer toutes les protections qu'il a mises en place. Tu déjoues même son scepticisme grâce à l'autorité suprême qu'est le jugement de son lui futur.

Ce n'est pas même difficile à faire. Tu peux manipuler les souvenirs de n'importe quel élève pour qu'il se souvienne avoir vu Harry Potter lui donner une enveloppe à lui remettre plus tard.

Tu peux faire ça à l'élève parce que tu es un professeur de Poudlard.

Tu ne te fatigues pas à voler un crayon et une feuille de papier dans la bourse de Harry. Tu préfères falsifier l'écriture de Harry sur un parchemin de sorcier. Tu peux la falsifier parce que tu l'as vue sur des examens du ministère que tu as toi-même corrigés.

Tu appelles Drago Malfoy “la constellation” parce que tu sais que Harry Potter s'intéresse à l'astronomie, que tu es un sorcier, que tu as suivi des cours d'astronomie et que tu as mémorisé les noms des constellations. Mais ce n'est pas le code que Harry Potter aurait instinctivement utilisé pour parler de Drago Malfoy à lui-même. Il aurait dit <<~l'apprenti~>>.

Tu appelles le professeur Quirrell “celui qui contemple les étoiles” et tu dis à Harry Potter de l'aider.

Tu sais que mange-mort est le nom d'un Détraqueur en Fourchelangue et tu comptes sur Harry pour voir les Aurors comme leurs complices.

Tu encodes 6h49 en écrivant “six, et sept dans un carré” parce que tu as lu un livre de physique Moldu que Harry Potter t'a donné.

Dans ce cas, qui es-tu~?

Harry remarqua que sa respiration s'était accélérée et, au prix d'un battement cardiaque plus rapide, il la fit de nouveau baisser. Le professeur Quirrell \emph{l'observait}.

Si, hypothétiquement, le professeur Quirrell était le cerveau, et avait falsifié le message de Harry, ça expliquait l'arrivée comique et simultanée des cinq groupes et alors le professeur Chourave avait été manipulé seulement pour donner un déni plausible à Quirrell pour que le sortilège d'Oubliettes soit attribué à un autre une fois les choses calmées, mais,

Mais pourquoi le professeur Quirrell risquerait-il la fragile alliance entre Harry et Drago via la tentative de fausse accusation de meurtre

(que le professeur Quirrell avait “détectée” et “empêchée” via un mouchard placé sur Drago)

Pourquoi le professeur Quirrell tuerait-il Hermione

(si sa première tentative pour la faire partir avait échoué)

Si le professeur Quirrell était le méchant alors peut-être avait il entièrement menti en matière de horcruxes et peut-être que ce n'était pas du tout une coïncidence que la seule chose capable de le sauver est celle capable de ressusciter le Seigneur des Ténèbres et si le Seigneur des Ténèbres avant organisé ça aussi

(un jour, David Monroe avait mystérieusement disparu, présumé assassiné par le Seigneur des Ténèbres)

Une atroce intuition avait saisi Harry, quelque chose séparé de tout son raisonnement jusqu'à ce point, une intuition qu'il ne pouvait formuler, mais lui et le professeur de Défense étaient semblables par bien des aspects, et falsifier un message venu d'un soi futur était exactement le genre de solution inventive que Harry aurait utilisée pour déjouer les protections d'une cible…

Et c'est alors que Harry comprit enfin ce qui aurait dû être évident depuis le tout début.

\later

Le professeur Quirrell était intelligent.

Le professeur Quirrell était intelligent à la façon de Harry.

Le professeur Quirrell était intelligent exactement comme l'était le mystérieux côté obscur de Harry.

S'il avait fallu deviner le jour où le Survivant avait acquis son mystérieux côté obscur, la réponse évidente aurait été~: la nuit du 31 octobre 1981.

\later

Et Et Et le professeur Quirrell avait été en possession d'un mot de passe qui leur avait permis de se faire passer pour le Seigneur des Ténèbres auprès de Bellatrix Black et sa présence procurait une sensation funeste au Survivant et l'interaction de leurs magies était destructrice et son sortilège préféré était Avada Kedavra et et et…

Ce fut comme si un immense barrage se brisait, un torrent de compréhension le traversa, engloutit son esprit dans un courant irrésistible qui balaya tout sur son passage.

Derrière toutes ces observations, une seule réalité.

Si différentes observations semblent aller dans des directions différentes, c'est que la bonne hypothèse n'a pas encore été formulée.

Et dans ces cas-là, lorsque la bonne hypothèse est enfin pensée, tout s'aligne à sa suite, par-delà le déni ou l'horreur, et arrache les doutes et les émotions qui auraient pu s'interposer.

… alors “David Monroe” et “Lord Voldemort” n'avaient été qu'une seule personne jouant sur les deux tableaux pendant la guerre des sorciers, et c'était pour ça que la famille Monroe avait été tuée avant de pouvoir rencontrer “David Monroe”, comme Maugrey s'en était douté…

La réalité se figea en un point stable, une situation cohérente capable de générer élégamment l'ensemble des observations.

Harry ne sursauta pas, il ne haleta pas, il essaya de ne montrer aucun signe de l'agonie et de l'horreur qui inondaient son esprit.

L'Ennemi était derrière lui. L'observait.

<<~Très bien~>>, dit Harry à voix haute dès qu'il fit à nouveau confiance à sa voix. Il continua de regarder les corps, loin du professeur Quirrell, parce qu'il ne faisait pas confiance à son visage. Il leva une manche pour essuyer la sueur sur son front, essaya de rendre le geste nonchalant~; il ne pouvait contrôler ni la sueur ni les rapides battements de son cœur. <<~Allons chercher cette Pierre Philosophale.~>>

Tout ce dont Harry avait besoin, c'était d'un seul moment de distraction pour pouvoir utiliser son Retourneur de Temps.

Aucune réponse ne vint.

Le silence s'étira.

Lentement, Harry se retourna.

Le professeur Quirrell se tenait droit. Il souriait.

Dans la main du professeur Quirrell, un objet en métal était pointé vers le bras armé de Harry, tenu avec l'assurance de celui qui sait parfaitement de servir d'un pistolet semi-automatique.

Le bouche de Harry était sèche~; même ses lèvres tremblaient, emplies d'adrénaline, mais il parvint à parler. <<~Bonjour, Lord Voldemort.~>>

Le professeur Quirrell inclina la tête en retour, et dit~: <<~Bonjour, Tom Jedusor.~>>
%  LocalWords:  Bortan Vizcaino Pucey’s Inglebee Higgs Inglebee’s Ba blu bu
%  LocalWords:  bluh Metamorphus Muffliato revealment
