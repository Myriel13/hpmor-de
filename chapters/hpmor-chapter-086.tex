% FIXME: Find a better way to format the hypotheses, and maybe the intro.
\chapter{Inférences multiples}

\begin{headlines}

\header{(Gros titres internationaux du 7 avril 1992)}

\label{Tribune Magique de Toronto~:}

\headline{Magenmagot britannique\\
rapporte avoir vu 'survivant'\\
faire peur à un détraqueur}

\headline{Expert en créatures magiques~:\\
«il faudrait vraiment arrêter de mentir»}

\headline{france et allemagne accusent angleterre\\
d'avoir tout inventé}

\label{Revue diurne de l'enchanteur néo-zélandais~:}

\headline{Qui a rendu folle la législature anglaise~?\
Notre gouvernement prochain en lice~?}

\headline{Expert liste les 28 meilleures raisons\\
de croire que c'est déjà le cas}

\label{Le Mage Américain~:}

\headline{Clan de loups-garous deviennent\\
premiers habitants du Wyoming}

\label{Le Chicaneur~:}

\headline{Malfoy fuit poudlard\\
à l'éveil de ses pouvoirs vélane}

\label{Gazette du Sorcier~:}

\headline{Faille juridique libère\\
«moldue cinglée»\\
et Potter menace ministère\\
d'attaquer Azkaban}
\end{headlines}

\section{Hypothèse~: Voldemort\\
(8 avril 1992, 19h22)}

\lettrine{I}{ls} s'étaient rassemblés autour de l'antique bureau du directeur de Poudlard, de ses tiroirs dans des tiroirs dans des tiroirs où toute la paperasserie passée de Poudlard était entreposée~; la légende racontait que la directrice Shehla s'était un jour perdue dans ce bureau, qu'elle s'y trouvait même toujours, et qu'elle ne pourrait en sortir avant d'avoir fait de l'ordre dans ses affaires. Minerva n'attendait pas avec une impatience particulière le jour où elle hériterait de ces tiroirs -- si aucun d'eux vivait pour le voir.

Albus Dumbledore était assis derrière son bureau, grave, mesuré.

Severus Rogue se tenait à côté de la cheminée et de ses cendres, inquiétant comme le vampire que les élèves l'accusaient parfois de prétendre qu'il était.

Maugrey Fol Œil devait les rejoindre mais n'était pas encore arrivé.

Et Harry…

Un petite silhouette enfantine, perchée sur un accoudoir, comme si les énergies qui le traversaient étaient trop fortes pour lui permettre de s'asseoir normalement. Visage figé, cheveux en sueur, yeux verts résolus, et au centre de tout cela, l'éclair dentelé de sa cicatrice jamais estompée. Il semblait à présent plus sombre, même comparé à la semaine précédente.

L'espace d'un instant, Minerva se souvint de leur passage par le Chemin de Traverse, qui semblait avoir eu lieu il y a des années. Déjà alors, ce sombre garçon \emph{en} Harry avait été là. Ce n'était pas entièrement sa faute à elle ni celle d'Albus. Mais il y avait pourtant quelque chose de si triste que c'en était quasiment insupportable dans le contraste entre le jeune garçon qu'elle avait rencontré pour la première fois et ce que l'Angleterre magique avait fait de lui. Elle avait compris que Harry n'avait jamais eu une enfance tout à fait ordinaire~; ses parents adoptifs lui avaient dit qu'il parlait peu et jouait encore moins avec les enfants moldus. Il était douloureux de songer que Harry n'aurait eu que quelques mois pour s'amuser avec les autres enfants de Poudlard avant que les exigences de la guerre ne lui arrachent tout. Peut-être y avait-il un autre visage que Harry montrait aux enfants de son âge lorsqu'il n'était pas en train de toiser le Magenmagot. Mais elle ne pouvait s'empêcher de se représenter l'enfance de Harry comme un tas de petit bois d'où elle et Albus tiraient une à une les branches pour les jeter dans les flammes.

«Les prophéties sont fort étranges», dit Albus Dumbledore. Les yeux du vieux sorcier étaient mis-clos, comme las. «Vagues, floues, d'un sens qui vous échappe comme de l'eau s'écoulerait entre vos doigts écartés. Elles sont toujours un fardeau car elles ne comportent aucune réponse~; seulement des questions.»

Harry Potter était assis, tendu. «Monsieur le directeur, dit le garçon avec une douce précision, mes amis sont pris pour cible. Hermione Granger a failli aller à Azkaban. Comme vous l'avez dit, la guerre a commencé. La prophétie du professeur Trelawney constitue une information clé pour mon estimation de l'équilibre entre mes différentes hypothèses au sujet de ce qui se trame. Sans même parler de la bêtise -- et du \emph{danger} -- qu'il y a à me laisser \emph{dans l'ignorance} quand le Seigneur des Ténèbres \emph{la connaît, lui}.»

D'un regard, Albus posa une sombre question à Minerva et elle secoua la tête en réponse~; quel que soit l'inimaginable moyen par lequel Harry avait découvert que Trelawney était à l'origine de la prophétie et que le Seigneur des Ténèbres la connaissait, il ne l'avait pas appris d'elle.

«Voldemort, en cherchant à empêcher cette prophétie, courut à sa défaite par tes mains, dit alors le vieux sorcier. La connaître n'a fait que lui nuire. Réfléchis-y bien, Harry Potter.

--- Oui, monsieur le Directeur, je comprends cela. Ma culture natale a aussi une tradition littéraire de prophéties auto-réalisatrices et mal interprétées. J'interpréterai avec précaution, soyez-en assuré. Mais j'en ai déjà deviné une grande partie. Est-il plus sûr de me laisser travailler sur la base d'une version partielle~?»

Un certain temps s'écoula.

«Minerva, dit Albus. Si tu veux bien.

--- Celui qui a le pouvoir de vaincre le Seigneur des Ténèbres approche, il naîtra de ceux qui l'ont par trois fois défié, il sera né lorsque mourra le septième mois…

--- \emph{Et le Seigneur des Ténèbres le marquera comme son égal,}» dit la voix de Severus, la faisant sursauter. La silhouette du Maître des Potions s'éleva devant la cheminée. «Mais il aura un pouvoir que le Seigneur des Ténèbres ignore et l'un devra détruire l'autre, n'en laissant qu'un vestige, car ces deux différents esprits ne peuvent exister dans le même monde.»

Cette dernière phrase fut prononcée par Severus d'un ton si lourd de présages que Minerva en eut les os gelés~; on aurait presque cru entendre Sibylle Trelawney.

Harry écoutait en faisant la moue.

«Vous pourriez répéter~? demanda-t-il.

--- \emph{Celui qui a le pouvoir de vaincre le Seigneur des Ténèbres approche, il naîtra de ceux qui l'ont par trois fois défié, il sera né lorsque mourra le septième mois…}

--- En fait, attendez, vous pourriez l'écrire~? Je dois l'analyser avec \emph{attention}.»

Ce fut fait, avec Albus et Severus penchés au-dessus du parchemin tels des faucons, comme pour s'assurer qu'aucune main invisible ne s'approche et ne dérobe la précieuse information.

«Voyons… dit Harry. Je suis mâle et né un 31 juillet, check. J'ai effectivement vaincu le Seigneur des Ténèbres, check. Pronom ambigu ligne deux… mais je n'étais pas encore né donc il est difficile de voir comment mes parents auraient pu \emph{me} défier trois fois. La cicatrice pourrait évidemment être la marque…» Harry toucha son front. «Puis il y a le pouvoir que le Seigneur des Ténèbres ignore, qui est probablement une référence à mon éducation scientifique…

--- Non», dit Severus.

Harry regarda le maître des potions avec surprise.

Les yeux de Severus étaient fermés, son visage contrit par la concentration.

«Le Seigneur des Ténèbres pourrait connaître ce pouvoir en étudiant les mêmes livres que vous, Potter. Mais la prophétie n'a pas dit \emph{un pouvoir que le Seigneur des Ténèbres n'a pas}. Ni même \emph{un pouvoir que le Seigneur des Ténèbres ne peut avoir}. Elle a parlé d'un \emph{pouvoir que le Seigneur des Ténèbres ignore}… ce sera quelque chose qui lui sera plus étranger que des artefacts moldus. Peut-être quelque chose qu'il ne peut pas du tout comprendre, même après l'avoir vu…

--- La science n'est pas un ramassis d'astuces technologiques, dit Harry. Ce n'est pas juste la version moldue d'une baguette. Ce n'est même pas de la connaissance, comme d'apprendre la table périodique. C'est une autre façon de \emph{penser}.

--- Peut-être…» murmura le maître des potions, mais son ton était sceptique.

«Il est périlleux, dit Albus, de voir tant dans une prophétie, même après l'avoir entendue soi-même. Elles sont des plus frustrantes.

--- Je vois ça», dit Harry. Il leva la main, frotta la cicatrice sur son front. «Mais… d'accord, si \emph{c'est} vraiment tout ce qu'on sait… écoutez, je vais être franc. Comment \emph{savez-vous} que le Seigneur des Ténèbres a vraiment survécu~?

--- \emph{Quoi~?}» s'écria-t-elle. Albus ne fit que soupirer et s'enfonça dans son vaste fauteuil de directeur.

«Enfin, dit Harry, imaginez à quoi cette prophétie ressemblait quand elle a été dite. Vous-Savez-Qui l'entend et j'ai l'air d'être destiné à grandir puis à le renverser. On dirait que nous sommes censés avoir une bataille finale où l'un d'entre nous ne devra laisser presque rien de l'autre. Donc Vous-Savez-Qui attaque Godric's Hollow et est \emph{immédiatement} vaincu en laissant derrière lui une \emph{sorte} de reste qui pourrait être ou ne pas être son âme désincarnée. Peut-être que les Mangemorts sont les restes, ou que c'est la Marque des Ténèbres. Ce que je veux dire, c'est que la prophétie pourrait déjà s'être réalisée. Comprenez-moi bien -- je me rends compte que mon interprétation a l'air tirée par les cheveux. La formulation de Trelawney n'est pas naturelle si elle fait \emph{seulement} référence aux événements historiques du 31 octobre 1981. Attaquer un bébé et voir le sortilège rebondir n'est pas quelque chose que l'on appellerait normalement “le pouvoir de vaincre”. Mais si on considère que la prophétie concernait \emph{plusieurs} futurs possibles, dont \emph{un} seulement a vraiment eu lieu lors d'Halloween, alors la prophétie pourrait être déjà réalisée.

---Mais… bredouilla Minerva. Mais le raid sur Azkaban…

--- \emph{Si} le Seigneur des Ténèbres a survécu, alors oui, il est le suspect le plus probable pour l'évasion d'Azkaban, dit Harry d'un ton raisonnable. On pourrait même dit que l'évasion d'Azkaban est un indice, au sens Bayésien du terme, en faveur de la survie du Seigneur des Ténèbres, parce qu'une évasion d'Azkaban a plus de chances de se produire dans les mondes où il vit que dans ceux où il est mort. Mais ce n'est pas un indice Bayésien \emph{fort}. Ce n'est pas quelque chose qui ne \emph{peut pas} se produire à moins que le Seigneur des Ténèbres soit en vie. Le professeur Quirrell, qui n'est \emph{pas} parti du principe que Vous-Savez-Qui est toujours dans le coin, n'a eu aucun problème à trouver sa propre explication. Pour lui, il est évident que quelque sorcier puissant pourrait désirer Bellatrix Black parce qu'elle connaîtrait un secret du Seigneur des Ténèbres, par exemple un savoir magique qu'il n'aurait révélé qu'à elle. Les à priori concernant la survie à une mort physique sont très bas, même si c'est magiquement possible. La \emph{plupart} du temps ça ne se produit pas. Donc s'il y a \emph{seulement} l'évasion d'Azkaban… Je dois dire que formellement cela ne constitue qu'un indice assez fort, au sens bayésien. L'improbabilité des observations dans un monde où l'hypothèse est fausse n'est pas à la hauteur de l'improbabilité à priori de l'hypothèse elle-même.

--- Non, dit Severus d'un ton catégorique. La prophétie ne s'est pas réalisée. Je le saurais, sinon.

--- En êtes-vous \emph{certain}~?

--- Oui, Potter. Si la prophétie s'était déjà réalisée, je la \emph{comprendrais}~! J'ai entendu les paroles de Trelawney, je me souviens de sa voix, et si je connaissais les événements correspondants, je les \emph{reconnaîtrais}. Ce qui s'est déjà produit… ne convient \emph{pas}.» Le maître des potions parlait avec certitude.

«Je ne sais pas vraiment quoi faire de ça», dit Harry. Il leva la main et frotta distraitement son front. «Peut-être que c'est seulement ce que vous \emph{croyez} s'être produit qui ne convient pas et que la véritable histoire est différente…

--- Voldemort \emph{est} en vie, dit Albus. Il y a d'autres indications.

--- Telles que~?» répondit instantanément Harry.

Albus s'interrompit. «Il existe de terribles rituels au moyen desquels les sorciers sont revenus d'entre les morts, dit lentement Albus. Cela, chacun peut le discerner en étudiant l'Histoire et les légendes. Et pourtant ces livres sont manquants, je ne parviens pas à les trouver~; c'est Voldemort qui les a pris, j'en suis sûr…

--- Donc vous ne \emph{pouvez pas} trouver les livres sur l'immortalité et cela prouve que c'est Vous-Savez-Qui qui les a~?

--- En effet, dit Albus. Il existe un certain livre -- je ne le nommerai pas à voix haute -- qui a disparu de la section interdite de la bibliothèque de Poudlard. Ainsi qu'un ancien parchemin qui aurait dû se trouver chez Barjow et Beurk mais où il n'y a plus qu'un emplacement vide permettant seulement de constater où il se trouvait…» le vieux sorcier s'interrompit. «Mais je suppose», continua-t-il, comme à lui-même, «que tu vas dire que même si Voldemort a essayé de se rendre immortel, cela ne prouve pas qu'il a réussi…»

Harry soupira.

«Des preuves, monsieur le directeur~? Il n'y a jamais que des probabilités. Si l'on connaît des livres sur les rituels d'immortalité et qu'ils manquent à l'appel, cela augmente la probabilité que quelqu'un s'en est emparé. Ce qui augmente donc la probabilité à priori que le Seigneur des Ténèbres ait survécu à sa mort. Je concède cela, et je vous remercie d'avoir signalé ce fait. La question est de savoir si la probabilité augmente \emph{assez}.

--- Certainement, dit doucement Albus, si tu concèdes ne serait-ce qu'une \emph{chance} que Voldemort a survécu, cela mérite de s'en protéger~?»

Harry inclina la tête.

«Comme vous le dites, monsieur le directeur. Quoi qu'une fois qu'une probabilité chute suffisamment, c'est aussi une erreur de continuer à être obsédé par elle… Étant donné que des livres sur l'immortalité ont disparu et que cette prophétie serait \emph{un peu} plus naturelle si elle faisait référence à une bataille future entre le Seigneur des Ténèbres et moi, je concède qu'il est plus probable que possible que le Seigneur des Ténèbres est en vie. Mais d'autres probabilités doivent \emph{aussi} être prises en compte -- et dans les mondes probables où Vous-Savez-Qui n'est \emph{pas} en vie, quelqu'un d'autre a piégé Hermione.

--- Bêtise, dit doucement Severus. Profonde bêtise. La Marque des Ténèbres ne s'est pas estompée, pas plus que son maître.

--- Voilà, c'est de \emph{ça} que je parle quand je parle de preuves Bayésiennes formellement insuffisantes. Bien sûr tout ça est glauque, de fort mauvais présage et tout ça, mais est-ce \emph{si} improbable que ça de voir une marque magique rester après la mort de son créateur~? Imaginez qu'on soit certain que la Marque des Ténèbres persistera tant que la conscience du Seigneur des Ténèbres existera, mais \emph{qu'à priori} on aurait estimé à seulement vingt pour cent les chances qu'elle demeure après sa mort. Alors l'observation 'La Marque des Ténèbres ne s'est pas estompée' a cinq fois plus de chances d'avoir lieu dans un monde où le Seigneur des Ténèbres est en vie que dans un monde où il est mort. Est-ce vraiment comparable à l'improbabilité à priori de son immortalité~? Mettons que les chances à priori soit de une sur cent pour la survie du Seigneur des Ténèbres. Si l'hypothèse a cent fois plus de chances d'être fausse que d'être vraie et qu'on observe une information cinq fois plus probable quand l'hypothèse est vraie que quand elle est fausse, on se doit de croire que l'hypothèse a vingt fois plus de chances d'être fausse que d'être vraie. Une chance sur cent, fois un rapport de vraisemblance de cinq contre un, égal vingt fois plus de chances que le Seigneur des Ténèbres soit mort…

--- \emph{Où} allez-vous chercher tous vos nombres, Potter~?

--- C'est \emph{là} la faiblesse reconnue de la méthode, répondit prestement Harry. Mais ce que je veux montrer, \emph{qualitativement}, c'est la raison pour laquelle l'observation 'La Marque des Ténèbres ne s'est pas estompée ' ne soutient pas suffisamment l'hypothèse 'Le Seigneur des Ténèbres est immortel'. L'observation n'est pas aussi extraordinaire que l'affirmation.» Harry marqua une pause. «Sans parler du fait que même si le Seigneur des Ténèbres est en vie, ce n'est pas \emph{forcément lui} qui a fait accuser Hermione. Comme un homme rusé l'a dit un jour, il peut y avoir plus d'un comploteur et plus d'un complot.

--- Comme le professeur de Défense, dit Severus avec un fin sourire. J'imagine que je dois agréer qu'il est suspect. C'était le professeur de Défense l'année dernière, après tout~; et l'année d'avant, et celle \emph{encore avant}.»

Les yeux de Harry retombèrent sur le parchemin posé sur ses genoux. «Avançons. Sommes-nous \emph{certains} que cette prophétie est correcte~? Personne n'a joué avec la mémoire du professeur McGonagall, n'a peut-être modifié ou enlevé une ligne~?»

Albus marqua une pause puis parla lentement.

«Il existe un grand sortilège lancé sur l'Angleterre Magique et qui enregistre toute prophétie prononcée au sein de nos frontières. Bien en dessous de la plus ancienne salle du Magenmagot, au département des Mystères, les prophéties sont enregistrées.

--- La salle des prophéties» murmura Minerva. Elle avait lu des choses au sujet de cet endroit que l'on disait être une grande pièce aux étagères emplies d'orbes lumineuses qui apparaissent au fil des années, les unes à la suite des autres. Les gens racontaient que Merlin lui-même l'avait bâtie~; que c'était le plus grand soufflet du sorcier à la face du Destin. Toutes les prophéties ne menaient pas au bien, et Merlin avait souhaité qu'au moins ceux mentionnés dans une prophétie sachent ce qui avait été dit \emph{d'eux}. C'était le respect que Merlin avait donné à leur libre arbitre~: que le Destin ne puisse les contrôler de l'extérieur, à leur insu. Ceux mentionnés dans une prophétie verraient un orbe lumineux flotter vers leur main puis entendraient la prophétie de la véritable voix du prophète. Il était dit que les autres qui tenteraient de toucher l'orbe en deviendraient fous -- ou peut-être juste que leur tête exploserait, les légendes étaient peu claires sur ce point. Quelles qu'aient été les intentions initiales de Merlin, les Langues-de-Plomb n'avaient laissé personne y entrer pendant des siècles, à ce qu'elle en savait. \emph{Travaux des sorciers ancestraux} avait indiqué que, plus tard, les Langues-de-Plomb s'étaient rendu compte qu'informer les sujets des prophéties pouvait bloquer la capacité des voyants à relâcher les pressions temporelles (quelles qu'elles soient) et les héritiers de Merlin avaient donc scellé cette salle. Il vint à l'esprit de Minerva de se demander (maintenant qu'elle avait passé quelques mois à proximité de M. Potter) comment quiconque pourrait \emph{savoir} cela, mais elle sut bien se garder de poser la question à Albus, ne prenant ainsi pas le risque qu'il essaie de lui répondre. Minerva croyait fermement qu'on ne devait se soucier du Temps que si l'on était une horloge.

«La salle des prophéties, confirma Albus d'une voix basse. Ceux qu'une prophétie mentionne peuvent aller l'y écouter. Comprends-tu ce que cela implique, Harry~?»

Harry fronça les sourcils.

«Eh bien… je pourrais l'écouter, le seigneur des Ténèbres aussi… oh, mes \emph{parents}. Ceux qui l'ont par trois fois défié. Ils étaient aussi mentionnés dans la prophétie donc ils ont pu entendre l'enregistrement~?

--- Si James et Lily ont entendu autre chose que ce qu'a rapporté Minerva, dit Albus d'un ton neutre, ils ne m'en ont pas fait part.

--- Vous avez amené James et Lily \emph{là-bas}~? dit Minerva.

--- Fumseck peut se rendre en de nombreux lieux, dit Albus. Ne mentionne pas ce fait.»

Harry regardait Albus droit dans les yeux. «Puis-\emph{je} aller à ce département des mystères et entendre la prophétie enregistrée~? Le ton de voix original peut aider, d'après ce qu'on m'a dit.»

Un reflet de lumière passa sur les lunettes en demi-lune d'Albus lorsque le vieux sorcier secoua la tête.

«Je pense que ce serait fort peu sage, dit Albus. Pour d'autres raisons que celles qui sont déjà évidentes. Ce lieu que Merlin a bâti est dangereux, plus dangereux pour certains que pour d'autres.

--- Je vois», dit Harry d'une voix sans timbre, et il rabaissa les yeux vers le parchemin. «Je considérerai la prophétie comme exacte pour le moment. La partie suivante dit que le Seigneur des Ténèbres m'a marqué comme son égal. Des idées sur ce que cela veut dire~?

--- Certainement pas, dit Albus, que tu devrais l'imiter de quelque manière que ce soit.

--- Je ne suis pas \emph{stupide}, monsieur le directeur. Les moldus ont compris une chose ou deux au sujet des paradoxes temporels, même si tout cela est théorique pour eux. Je ne vais pas jeter mon éthique en l'air juste parce qu'un signal venu du futur prétend que ça va se produire, parce qu'alors ça deviendrait la seule raison pour laquelle ça ce serait produit. Mais à nouveau~: qu'est ce que ça veut \emph{dire~?}

--- Je l'ignore, dit Severus.

--- Tout comme moi», dit-elle.

Harry saisit sa baguette et la fit tourner entre ses main en observant le bois d'un air méditatif. «Trente centimètre, en houx, avec un cœur en plume de phénix, dit Harry. Et ce phénix dont la plume est dans cette baguette n'en a jamais donné qu'une autre que M… comment s'appelle-t-il, Olive-quelque chose… a mis dans la baguette du Seigneur des Ténèbres. \emph{Et} je suis Fourchelangue. Ça faisait déjà beaucoup de coïncidences. Et maintenant j'apprends qu'il y a une prophétie qui annonce que je serai l'égal du Seigneur des Ténèbres.»

Les yeux de Severus étaient pensifs, ceux du directeur, indéchiffrables.

«Se pourrait-il, dit une Minerva hésitante, que Vous-Savez-Qui -- que Voldemort -- a transféré certains de ses pouvoirs à M. Potter, la nuit où il a reçu cette cicatrice~? Certainement pas quelque chose qu'il aurait souhaité mais quand même… je ne vois pas comment M. Potter pourrait être son \emph{égal} puisqu'il possède moins de magie que le Seigneur des Ténèbres lui-même…

--- Mouais», dit Harry, toujours méditatif devant sa baguette. «Je me battrais contre le Seigneur des Ténèbres sans aucune magie si besoin était. \emph{Homo sapiens} n'est pas devenu l'espèce dominante de cette planète en ayant les griffes les plus acérées ou l'armure la plus dure -- bien que je suppose que ce point échappe aux sorciers. Néanmoins, il serait indigne de moi, un humain, d'avoir peur d'une chose qui n'est pas plus intelligente que moi~; et à ce que j'ai entendu, le Seigneur des Ténèbres n'est pas très effrayant dans ce domaine.»

Le maître des potions parla et sa voix avait reprit une partie de son mépris traînant.

«Vous vous imaginez plus intelligent que le Seigneur des Ténèbres, Potter~?

--- Effectivement», dit Harry en remontant sa manche gauche puis en roulant la manche de chemise située en dessous, exposant son coude nu. «Oh, ça me fait penser~! Assurons-nous que personne ici n'a la marque clairement visible placée à l'endroit habituel et facile à regarder qui les révélerait être un espion ennemi.»

Albus fit un geste de silence qui interrompit le maître des potions avant que ce dernier ne puisse répliquer d'un ton acerbe.

«Dis-moi, Harry, dit Albus, comment aurais-\emph{tu} conçu la Marque des Ténèbres~?

--- Emplacements inhabituels, dit prestement Harry, difficiles à trouver sans gêne ni protestations, même si bien sûr toute personne prudente vérifierait quand même. La rendre plus petite, si possible. Ajouter quelque tatouage non-magique pour masquer la forme exacte -- encore mieux, la recouvrir d'une fausse couche de peau…

--- Rusé, en effet, dit Albus. Mais dis-moi, imagine que tu puisses choisir donner n'importe quelle caractéristique à la Marque, que tu puisses la faire s'effacer ou apparaître à ton goût. Que ferais-tu alors~?

--- Je la rendrais entièrement invisible, tout le temps, dit Harry en ayant l'air d'énoncer l'évidence. On ne voudrait pas qu'il y ait des différences détectables entre un espion et un non-espion.

--- Imagine que tu es encore plus rusé, dit Albus. Tu es un maître de la tromperie, de la supercherie, et tu utilises tes capacités au maximum.

--- Eh bien… le garçon s'interrompit et fronça les sourcils. Cela me semble inutilement compliqué, plus proche d'une tactique qu'un méchant pourrait utiliser dans un jeu de rôle que d'une que l'on essaierait dans une vraie guerre. Mais j'imagine qu'on pourrait placer une fausse Marque des Ténèbres sur de faux Mangemorts et garder les Marques des Ténèbres des vrais Mangemorts invisibles. Mais alors il y aurait la question de savoir pourquoi les gens auraient commencé à croire que la marque identifie un Mangemort… je devrais y réfléchir au moins cinq minutes si je voulais prendre le problème au sérieux.

--- Je te demande cela, dit Albus d'un ton doux, parce que j'ai effectivement, au début de la guerre, opéré des tests semblables à ceux que tu as suggéré. L'Ordre n'a survécu à ma folie que parce qu'Alastor ne faisait pas confiance aux bras nus qu'il voyait. J'ai ensuite pensé que les porteurs de la Marque pouvaient la cacher ou la montrer à leur convenance. Et pourtant lorsque nous avons présenté Igor Karkaroff au Magenmagot, la Marque était clairement visible sur son bras, en dépit de toutes ses protestations d'innocence. Quelle règle secrète gouverne la Marque des Ténèbres~? Je l'ignore. Même Severus est toujours contraint par sa Marque de ne révéler aucun de ses secrets à qui ne les connaît déjà.

--- Oh, eh bien alors c'est \emph{évident}, dit immédiatement Harry. Attendez, un instant… vous étiez un \emph{Mangemort}~?» Harry transféra son regard vers Severus.

Severus lui renvoya un fin sourire.

«Je le suis toujours, pour autant qu'ils le savent.

--- Harry», dit Albus, les yeux braqués sur le garçon. «Que veux-tu dire par 'c'est évident'~?

--- Théorie de l'information, première leçon, dit le garçon d'un ton professoral. Observer la variable X apporte une information au sujet de la variable Y si et seulement si les valeurs possibles de X ont des probabilités différentes selon les différents états de Y. À l'instant où vous entendez parler de quoi que ce soit qui varie entre un espion et un non-espion, vous devriez immédiatement penser à l'exploiter afin de distinguer les uns des autres. De même, pour distinguer la réalité d'un mensonge, vous avez besoin d'un processus qui se comporte différemment face à la vérité et face au mensonge -- c'est pour cela que la 'foi' ne fonctionne pas comme discriminant alors 'faire des prédictions expérimentales et les vérifier' fonctionne. Vous dites que tout porteur de la Marque ne peut révéler ses secrets à qui ne les connaît pas déjà. Donc pour découvrir le fonctionnement de la Marque des Ténèbres, notez tous les modes d'opération qui vous viennent à l'esprit, et observez le professeur Rogue pendant qu'il essaye de les dire à un témoin test -- peut-être à quelqu'un qui ignore l'objet de l'expérience -- j'expliquerai la recherche binaire plus tard pour qu'on puisse jouer au portrait afin d'aller plus vite. Ainsi, il sera incapable de lire la vérité à haute voix. Comme vous le comprenez, son silence sera observable en présence d'affirmations vraies, mais pas en présence de mensonges.

Minerva se rendit compte qu'elle avait la bouche grande ouverte et la referma aussi sec. Même Albus semblait surpris.

«Et après ça, comme j'ai dit, \emph{toute} différence comportementale entre espions et non-espions peut être utilisée pour identifier les espions. Une fois que vous avez identifié au moins un secret magiquement censuré par la Marque des Ténèbres, vous pouvez vérifier si quelqu'un l'a en regardant s'il peut révéler ce secret à quelqu'un qui ne le connaît pas déjà…

--- \emph{Merci, M. Potter.}»

Tout le monde regarda Severus. Le maître des potions se redressait, ses dents se révélaient derrière un sourire de triomphe rageur.

«M. le directeur, je puis à présent librement parler de la Marque. Si nous savons avoir été reconnus pour ce que nous sommes, des Mangemorts, face à d'autres n'ayant pas encore vu nos bras nus, nos Marques se révèlent que nous le voulions ou pas. Mais si nos bras ont déjà été vus, elle ne se révèle pas~; pas plus que si l'on ne fait que nous tester en raison d'un soupçon. Ainsi, la Marque des Ténèbres donne l'apparence d'identifier les Mangemorts -- mais seulement ceux qui ont déjà été repérés, comme vous le comprenez.

--- Ah… dit Albus. Merci, Severus.» Il ferma brièvement les yeux. «Cela expliquerait donc pourquoi Black a échappé jusqu'à la surveillance de Peter… ah, enfin. Et le test proposé par Harry~?»

Le maître des potions secoua la tête. «Le Seigneur des Ténèbres n'était pas un idiot, en dépit des illusions de Potter. À l'instant où un tel test est suggéré, la Marque cesse de lier nos langues. Mais je ne pouvais guider vers cette possibilité, seulement attendre que quelqu'un la déduise.» Un autre fin sourire. «Je vous décernerais bien de nombreux points, M. Potter, si cela n'allait pas compromettre ma couverture. Mais comme vous pouvez le constater, le Seigneur des Ténèbres était assez fourbe.» Son regard devint plus distant. «Oh, souffla Severus, il était \emph{particulièrement} fourbe…»

Harry Potter resta assis un long moment.

Puis…

«Non», dit Harry. Le garçon secoua la tête. «Non, ça ne peut pas être \emph{vraiment} vrai. D'abord, il s'agit du genre de puzzle logique qu'on trouverait au \emph{premier} chapitre d'un livre de Raymond Smullyan, bien \emph{loin} du niveau auquel les scientifiques moldus travaillent au quotidien. Et ensuite, pour ce que j'en sais, le Seigneur des Ténèbres a pu mettre cinq mois de réflexion à trouver ce puzzle que je viens de résoudre en cinq secondes…

--- Vous est-il \emph{si} inconcevable que cela, Potter, que quiconque puisse être aussi intelligent que vous~?» la voix du maître des potions contenait plus de curiosité que de dédain.

«C'est ce qu'on appelle une probabilité à priori, professeur Rogue. Les observations sont tout autant compatibles avec un Seigneur des Ténèbres inventant le puzzle sur une période de cinq mois qu'en moins de cinq secondes, mais dans toute population il y aura bien plus de gens capables de le faire en cinq mois qu'en cinq secondes…» Harry se plaqua une main sur le front. «Bon sang, comment j'explique ça~? J'imagine que de votre point de vue, le Seigneur des Ténèbres a inventé un puzzle astucieusement, que je l'ai astucieusement résolu et que ça nous donne l'air d'être \emph{égaux}.

--- Je me souviens de votre premier jour en cours de potions, dit sèchement le maître des potions. Je pense que vous avez encore du chemin à parcourir.

--- Paix, Severus, dit Albus. Harry a déjà accompli plus que tu ne sais. Pourtant, Harry, dis-moi… \emph{pourquoi} penses-tu que le Seigneur des Ténèbres t'est inférieur~? Il est certainement une âme brisée de bien des façons. Mais ruse contre ruse… je dirais que tu n'es pas encore prêt à lui faire face~; et je connais tous tes accomplissements.»

\later

Ce qui était frustrant, dans cette conversation, c'était que Harry ne \emph{pouvait pas dire les véritables raisons pour lesquelles il n'était pas d'accord}, ce qui violait plusieurs principes de base du dialogue coopératif.

Il ne pouvait pas expliquer comment Bellatrix Black avait vraiment été tirée d'Azkaban -- pas par Vous-Savez-Qui sous quelque déguisement mais par l'intelligence combinée de Harry et du professeur Quirrell.

Harry ne voulait pas dire face au professeur McGonagall que l'existence des dommages cérébraux signifiait que les âmes n'existaient pas. Ce qui rendait la réussite d'un rituel d'immortalité… eh bien, pas \emph{impossible}, Harry comptait certainement tracer \emph{un jour} une route vers l'immortalité magique, mais ce serait \emph{beaucoup plus difficile} et demanderait \emph{beaucoup plus d'ingéniosité} que de juste lier une âme déjà existante à un phylactère de liche. Ce qu'aucun sorcier intelligent et ayant connaissance de l'immortalité des âmes ne ferait en premier lieu.

Et la véritable, l'honnête raison pour laquelle Harry savait que le Seigneur des Ténèbres ne pouvait avoir été si malin… eh bien… il n'y avait aucune façon polie de le dire mais…

Harry avait \emph{été} au Magenmagot. Il avait \emph{vu} les risibles 'précautions de sécurité', si on pouvait les appeler ainsi, qui gardaient les plus profonds niveau du ministère de la Magie. Ils n'avaient même pas la cascade des voleurs utilisée à Gringotts pour effacer le Polynectar et les Imperiums lancés sur les visiteurs. La méthode de prise de pouvoir évidente serait de lancer un Imperium sur le ministre de la magie ainsi que quelques chefs de départements et d'envoyer une grenade par chouette à ceux trop puissants pour tomber sous le coup d'un Imperium. Ou de les assommer en leur expédiant du gaz par chouette si on voulait les maintenir en vie, sous l'effet d'un philtre de mort vivante, afin de prendre leurs cheveux pour des potions de Polynectar. Légilimancie, faux souvenirs, le sortilège de confusion… c'était ridicule, le monde magique était \emph{super-saturé} de moyens de tricher. Harry ne ferait peut-être aucune de ces choses lorsqu'il prendrait le contrôle de l'Angleterre à cause de contraintes éthiques… enfin, Harry ferait \emph{peut-être} certaines des choses moins graves, puisqu'un peu de Polynectar, une confusion temporaire et une Légilimancie en lecture seule valaient toutes mieux qu'une journée de plus à Azkaban… mais…

Si Harry n'avait pas été restreint par son éthique, il aurait potentiellement pu exterminer les pires sections du Magenmagot ce jour-là~; seul, en utilisant seulement les pouvoirs magique d'un élève en première année, juste en ayant été assez malin pour comprendre les Détraqueurs. Même s'il ne se serait peut-être pas ensuite retrouvé dans une excellente position politique~: les membres survivants du Magenmagot auraient pu trouver facile et expédient de désavouer ses actes et de le condamner pour redorer leur image, et ce même si les plus intelligents auraient compris que ça avait été pour le plus grand bien… mais \emph{quand même}.

Si on était entièrement libéré de toute éthique, armé des anciens secrets de Salazar Serpentard, suivi de dizaines d'adeptes, y compris Lucius Malfoy, et qu'on mettait plus de dix ans à \emph{échouer} à renverser le gouvernement d'Angleterre magique, cela voulait dire qu'on était stupide.

«Comment puis-je le formuler… dit Harry. Écoutez, M. le directeur, vous avez une éthique, il y a de nombreuses tactiques de combat que vous n'utilisez pas parce que vous n'êtes pas maléfique. Et vous avez combattu le Seigneur des Ténèbres, un sorcier extraordinairement puissant qui n'avait pas de telles restrictions, et vous l'avez \emph{quand même} repoussé~; Si Vous-Savez-Qui avait été super-intelligent \emph{en plus de ça}, vous seriez \emph{morts}. \emph{Tous}. Vous seriez morts \emph{instantanément…}

--- Harry», dit le professeur McGonagall. Sa voix était haletante. «Harry, nous \emph{avons} tous failli mourir. Plus de la moitié de l'Ordre du Phénix est mort. Sans Albus -- Albus Dumbledore, le plus grand sorcier en deux siècles, Harry… nous aurions sûrement péri.»

Harry se passa une main sur le front. «Je suis désolé, dit-il. Je n'essaie pas de minimiser ce que vous avez traversé. Je sais que Vous-Savez-Qui était complètement mauvais, que c'était un mage noir incroyablement puissant avec des dizaines d'adeptes, et que les choses allaient… mal, oui, certainement très mal. C'est seulement que…» \emph{Tout cela est prodigieusement moins menaçant qu'un ennemi intelligent, car un tel ennemi métamorphose de la toxine botulique et en glisse un millionième de gramme dans votre tasse de thé.} Y avait-il un moyen sans risque de faire passer ce concept sans citer d'exemple précis~? Harry ne pouvait en trouver aucun.

«S'il te plaît, Harry, dit le professeur McGonagall. S'il te plaît, Harry, je t'en supplie -- \emph{prends le Seigneur des Ténèbres au sérieux~!} Il est plus dangereux que…» la vieille sorcière semblait avoir du mal à trouver ses mots. «Il est \emph{bien} plus dangereux que la métamorphose.»

Les sourcils de Harry s'élevèrent avant qu'il puisse s'en empêcher. Un sombre gloussement s'échappa de Severus Rogue.

\emph{Euh}, dit la voix Serdaigle intérieure. \emph{Euh, honnêtement, le professeur McGonagall a raison~: nous ne prenons pas ça autant au sérieux que s'il s'agissait un problème scientifique. La difficulté consiste à réagir} tout court \emph{face à de nouvelles informations et de ne pas se contenter de les balancer par la fenêtre. Pour l'instant on dirait que nous n'avons pas} du tout \emph{modifié nos croyances après avoir rencontré un argument à la fois important et inattendu. Si nous avons initialement rejeté l'idée que Lord Voldemort est une menace sérieuse, c'est parce que la Marque des Ténèbres semblait clairement stupide. Il faudrait faire un effort et se concentrer pour revenir en arrière et remettre en cause l'intégralité du raisonnement que nous avons suivi en nous basant sur cette assomption fausse, ce que nous ne sommes} pas \emph{en train de faire.}

«Très bien, dit Harry alors que le professeur McGonagall semblait s'apprêter à parler de nouveau . Très bien, pour prendre cela au sérieux je dois réfléchir cinq minutes.

--- Je t'en prie», dit Albus Dumbledore.

Harry ferma les yeux.

Son côté Serdaigle se divisa en trois.

\emph{Estimation probabiliste}, dit Serdaigle Un qui jouait le rôle du modérateur. \emph{Le Seigneur des Ténèbres est en vie, aussi intelligent que nous, et c'est donc une vraie menace.}

\emph{Pourquoi tous ses ennemis ne sont-ils pas déjà morts~?} Dit Serdaigle Deux, qui menait la charge.

\emph{Remarque}, dit Serdaigle Un, \emph{que nous avons déjà pensé à cet argument et que nous ne pouvons donc pas l'utiliser} une fois de plus \emph{pour modifier nos croyances.}

\emph{Mais où est la faille dans cette logique~?} Dit Serdaigle Deux. \emph{Dans des mondes avec un Voldemort intelligent, tous les membres de l'Ordre du Phénix sont morts au cours des cinq premières minutes de la guerre. Le monde ne ressemble pas à ça, donc nous ne vivons pas dans un tel monde. CQFD.}

\emph{En sommes-nous vraiment certains~?} Demanda Serdaigle Trois, qui avait reçu le rôle du défenseur. \emph{Peut-être Lord Voldemort avait-il une raison de ne} pas \emph{se donner à fond à cette époque…}

\emph{Comme quoi~?} exigea Serdaigle Deux. \emph{En plus, quelle que soit ton excuse, j'exige que la probabilité de l'hypothèse soit pénalisée proportionnellement à sa complexité additionnelle…}

\emph{Laisse Trois parler}, dit Serdaigle Un.

\emph{D'accord… écoutez}, dit Serdaigle Trois. \emph{D'abord nous ne} savons \emph{pas si n'importe qui peut s'emparer du ministère juste par manipulation mentale. Peut-être l'Angleterre Magique est-elle vraiment une oligarchie et peut-être qu'il est vraiment nécessaire d'avoir assez de puissance militaire pour intimider les dirigeants des familles et les faire se soumettre…}

\emph{Lance Imperium sur eux aussi}, intervint Serdaigle Deux.

…\emph{et les Oligarques ont une Cascade du Voleur sur le pas de} leur \emph{porte…}

\emph{Pénalité de complexité~!} S'écria Serdaigle Deux. \emph{Toujours plus d'épicycles~!}

\emph{… oh, sois raisonnable}, dit Serdaigle Trois. \emph{Nous n'avons encore jamais} vu \emph{quelqu'un s'emparer du ministère à l'aide de quelques Impériums bien placés. Nous ne} savons \emph{pas si c'est vraiment aussi simple que ça.}

\emph{Mais}, dit Serdaigle Deux\emph{, même en prenant cela en compte… j'ai vraiment l'impression qu'il aurait eu d'autres moyens. Dix ans d'échec, vraiment~? En utilisant seulement des tactiques terroristes conventionnelles~? Ce n'est… ce n'est même pas faire un effort.}

\emph{Peut-être que Voldemort avait des idées plus créatives}, répondit Serdaigle Trois, \emph{mais qu'il ne voulait pas révéler sa main aux gouvernements des} autres \emph{pays, qu'il ne voulait pas qu'}eux \emph{découvrent à quel point ils étaient vulnérables et installent des Cascades du Voleur dans} leur \emph{Ministère. Pas avant d'avoir l'Angleterre comme base et assez de serviteurs pour subvertir} tous \emph{les autres gouvernements importants en même temps.}

\emph{Tu pars du principe qu'il voulait conquérir le monde}, remarqua Serdaigle Deux.

\emph{Trelawney a prophétisé qu'il serait notre égal}, entonna Serdaigle Trois d'un ton solennel. \emph{Par conséquent, il voulait conquérir le monde.}

\emph{Et s'il est ton égal, et que tu dois le combattre…}

L'espace d'un instant l'esprit de Harry tenta d'imaginer le spectre de deux sorciers \emph{créatifs} engagés dans une guerre sans merci l'un contre l'autre.

Harry avait remarqué que tous les sortilèges et toutes les potions de son livre de première année pouvaient être utilisés de façon créative dans le but de tuer des gens. Il n'avait pas pu s'en empêcher. Au sens propre. Il avait \emph{essayé} d'empêcher son cerveau de le faire à chaque fois mais ça avait été comme de regarder un poisson et d'essayer d'empêcher son cerveau de remarquer que c'était un poisson. Ce qu'il était possible de faire avec de la créativité et un niveau de septième année, ou un niveau d'Auror, ou d'anciennes magies perdues telles que celles que Lord Voldemort avait possédées… rien qu'y penser était insupportable. Un psychopathe génial, créatif et doté de pouvoirs magiques n'était pas une 'menace', c'était la fin d'une espèce.

Puis Harry secoua la tête et rejeta la lugubre trajectoire le long de laquelle son raisonnement l'avait menée. La question était de savoir s'il avait de fortes chances de faire face à la terreur que serait un Rationaliste Obscur.

\emph{La probabilité à priori qu'une personne s'essayant à un rituel d'immortalité y parvienne vraiment…}

Estimables à une chance sur mille, et c'était une estimation généreuse~; un sorcier sur mille ne survivait d'ailleurs pas à sa mort, même si Harry se devait d'admettre qu'il n'avait aucune information sur la proportion qui s'était essayée à des rituels d'immortalité.

\emph{Et si le Seigneur des Ténèbres est aussi intelligent que nous~?} Dit Serdaigle Trois. \emph{Tu sais, notre} égal \emph{comme Trelawney l'a prophétisé. Dans ce cas il se} débrouillerait \emph{pour que son rituel fonctionne. P.S. N'oublie pas la partie '… détruire l'autre, n'en laissant presque rien'.}

Nécessiter ce niveau d'intelligence constituait un détail de poids~: la probabilité à priori qu'un membre de la population choisi au hasard soit intelligent était basse…

Mais Lord Voldemort n'était pas un sorcier choisi au hasard, c'était un sorcier en particulier, que tout le monde avait remarqué. Le puzzle de la Marque des Ténèbres requérait un niveau d'intelligence minimal, même en ayant (hypothétiquement) mis longtemps à l'inventer. Mais quand même, dans le monde moldu, toutes les personnes extrêmement intelligentes que Harry connaissait par l'Histoire n'étaient \emph{pas} devenues de méchants dictateurs ou des terroristes. Ceux qui s'en approchait le plus, dans le monde moldu, étaient les gestionnaires de fonds spéculatifs, et aucun \emph{d'eux} n'avait essayé de s'emparer ne serait-ce que d'un pays du tiers monde, ce qui imposait une borne supérieure autant à leur méchanceté qu'à leur bonté maximales.

Sous certaines hypothèses, le Seigneur des Ténèbres était intelligent et l'Ordre du Phénix n'était \emph{pas} mort instantanément, mais ces hypothèses étaient plus complexes et se devaient de recevoir une pénalité de complexité. Après que ces pénalités et autres excuses eurent été intégrées, on se retrouvait avec un écart très élevés entre la vraisemblance des hypothèses 'Le Seigneur des Ténèbres est intelligent' et 'Le Seigneur des Ténèbres est stupide' vis à vis de l'observation 'Le Seigneur des Ténèbres n'a pas instantanément gagné la guerre'. Probablement un rapport de vraisemblance de 10 contre 1 en faveur de la stupidité du Seigneur des Ténèbres… mais peut-être pas 100 contre 1. On ne pouvait tout à fait dire que 'Le Seigneur des Ténèbres gagne instantanément' avait une probabilité \emph{supérieure} à 99~\% en présupposant son intelligence~; la somme de toutes les excuses possibles dépassait 0,01.

Et il y avait la prophétie… qui avait peut-être \emph{initialement} contenu un passage disant que le Seigneur des Ténèbres mourrait \emph{immédiatement} s'il se confrontait aux Potters. Qu'Albus Dumbledore aurait ensuite ôté de la mémoire de McGonagall afin de mener le Seigneur des Ténèbres à sa perte. Si un tel passage n'existait \emph{pas}, la prophétie avait \emph{plus ou moins} l'air de dire que Vous-Savez-Qui et le Survivant étaient destinés à avoir quelque confrontation ultérieure. Mais dans \emph{ce} cas, il était moins probable que Dumbledore ait inventé une excuse plausible pour ne pas avoir à emmener Harry à la salle des prophéties.

Harry se demandait s'il était possible de tirer des calculs Bayésiens de tout ceci. Bien sûr, l'objectif d'un calcul Bayésien subjectif n'était pas, après avoir inventé plein de nombres et de les avoir multipliés, d'obtenir exactement la bonne réponse. Le véritable objectif était \emph{l'acte} d'inventer les nombres, de se forcer à prendre en compte tous les faits importants et de soupeser toutes les probabilités entre elles. C'était par exemple de se rendre compte, juste après avoir vraiment \emph{pensé} à la probabilité de voir la Marque ne pas s'effacer \emph{après} la mort de Vous-Savez-Qui, que cette probabilité n'était pas assez basse pour constituer un indice important. L'une des versions du procédé consistait à analyser toutes les hypothèses, à faire la liste des indices, à inventer tous les nombres, à faire les calculs, puis à jeter la réponse par la fenêtre et à suivre son instinct \emph{maintenant} que l'on s'était forcé à vraiment tout \emph{prendre en compte}. Le problème était que les probabilités des indices n'étaient pas indépendantes entre elles et que plusieurs éléments sous-jacents interagissaient les uns avec les autres…

… enfin, au moins, \emph{une chose} était certaine.

Si les calculs étaient faisables, il lui faudrait du papier et un crayon.

Dans le feu situé sur un mur du bureau du directeur, les flammes s'embrasèrent soudain et passèrent de l'orange à un vert vif bilieux.

«Ah~! dit le professeur McGonagall au milieu de ce non-silence gêné. Je crois que c'est Maugrey Fol Œil.

--- Laissons ce problème reposer pour le moment», dit le directeur avec quelque soulagement alors qu'il se tournait pour regarder la cheminée. «Je crois aussi que nous sommes sur le point d'apprendre des nouvelles le concernant.»

\latersection{Hypothèse~: Hermione Granger\\
(8 avril 1992, 18h53)}

Pendant ce temps, dans la grande salle de Poudlard, alors que les élèves n'ayant pas de réunion secrète avec le directeur s'occupaient de dîner, assis à quatre énormes tables…

«C'est drôle, dit Dean Thomas d'un ton pensif. Je n'ai pas cru le général quand il nous a dit que ce que nous avions appris nous changerait pour toujours, que nous ne pourrions jamais revenir à une vie normale. Maintenant que nous savions. Maintenant que nous pouvions voir ce qu'\emph{il} pouvait voir.

--- Je sais~! dit Seamus Finnigan. Je pensais aussi que c'était une blague~! Un peu… comme tout ce que le général Chaos dit.

--- Mais maintenant… dit tristement Dean. On ne peut \emph{pas} revenir en arrière, hein~? Ce serait comme de retourner à l'école moldue après avoir été à Poudlard. On doit juste… on doit juste tous rester ensembles. C'est tout ce qu'on peut faire, sinon on deviendra fous.»

Seamus Finnigan, à côté de lui, se contenta de hocher silencieusement la tête et de prendre une autre bouchée de gnuvage.

Autour d'eux, les conversations de la table Gryffondor continuaient. Elles n'étaient pas aussi \emph{opiniâtres} que la veille, mais ici et là le sujet refaisait surface.

«Ben il doit y avoir eu une \emph{sorte} de triangle amoureux», dit une sorcière en deuxième année nommée Samantha Crowley (elle ne répondait jamais quand on lui demandait s'il y avait un rapport). «La question est de savoir ce qui se passait avant que tout déraille. Qui était amoureux de qui -- si cette personne aimait en retour -- je ne sais pas \emph{combien} il y a de possibilités…

--- Soixante-quatre», dit Sarah Varyabil, une beauté en pleine floraison qui aurait dû être répartie à Serdaigle ou à Poufsouffle. «Non, attends, c'est faux. Enfin si personne n'aime Malfoy et que Malfoy n'aime personne alors il ne fait pas vraiment partie du triangle amoureux… je vais avoir besoin d'Arithmancie, est-ce que vous pourriez attendre deux minutes~?

--- C'est tellement \emph{triste}, dit Sherice Ngaserin dont les yeux étaient réellement larmoyants. Ils étaient juste… ils étaient si \emph{évidemment} destinés à finir ensembles~!

--- Tu veux dire Potter et Malfoy~? dit un élève en deuxième année du nom de Colleen Johnson. Je sais -- leurs familles se haïssaient tellement, il ne pouvaient pas ne \emph{pas} tomber amoureux…

--- Non, je veux dire les trois», dit Sherice.

Ce qui provoqua un bref silence au sein du groupe. Dean Thomas s'étrangla discrètement sur sa limonade et essaya de ne faire aucun bruit alors lorsqu'elle coula de sa bouche et inonda sa chemise.

«\emph{Wow}, dit une sorcière aux cheveux noirs du nom de Nancy Hua. C'est vraiment… \emph{sophistiqué} de ta part, Sherice.

--- Écoutez-moi tous, on doit rester réalistes,» dit Eloise Rosen, une grande sorcière qui avait été générale et parlait donc avec un air d'autorité. «On \emph{sait} -- parce qu'elle l'a embrassé -- que Granger était amoureuse de Harry Potter. Donc la seule raison pour laquelle elle essaierait de tuer Malfoy serait parce qu'elle saurait qu'il gagnait les faveurs de Potter. Pas besoin de rendre tout ça compliqué -- vous vous comportez tous comme si c'était une pièce de théâtre, et pas la vraie vie~!

--- Mais même si Granger était amoureuse, c'est quand même drôle qu'elle ait juste \emph{craqué} comme ça,» dit Chloé, que ses robes noires alliées à sa peau couleur d'encre faisaient ressembler à une sombre silhouette. «Je ne sais pas… je pense qu'il y a plus qu'un roman d'amour qui a mal tourné derrière tout ça. Je pense que la plupart des gens n'ont peut-être aucune idée de ce qui se passe réellement.

--- \emph{Oui~! Merci~!} s'écria Dean Thomas. Écoutez -- vous ne comprenez pas -- comme Harry Potter nous l'a \emph{dit} à tous -- que si vous n'aviez pas \emph{prévu} que quelque chose allait se produire, et que cette chose vous a entièrement surpris, alors ce que vous croyiez être vrai au moment où vous ne l'aviez \emph{pas} prévue ne suffit pas à expliquer…» la voix de Dean resta en suspens lorsqu'il comprit que personne ne l'écoutait. «C'est \emph{sans espoir}, c'est ça~?

--- Tu ne t'en étais pas encore rendu compte~?» dit Lavande Brown, assise à la table en face de ses deux anciens partenaires chaotiques. «Comment t'as fait pour devenir Lieutenant~?

--- Oh, taisez-vous tous les deux~! éructa Sherice à leur intention. C'est évident que vous les voulez tous les trois rien que pour vous deux~!

--- Je suis sérieuse~! dit Chloé. Et si ce qui se passe est \emph{vraiment} différent des… vous savez, des choses \emph{normales} dont parlent les gens \emph{ordinaires}~? Et si quelqu'un avait… \emph{forcé} Granger à faire ce qu'elle a fait, exactement comme Potter a essayé d'en convaincre tout le monde~?

--- Je pense que Chloé a raison», dit un sorcier à l'air étranger qui se présentait toujours sous le nom d'Adrian Turnipseed, même si ses parents l'avaient en fait appelé Mad Drongo. «Je pense que pendant tout ce temps il y a eu…» Adrian baissa la voix et prit un air sinistre, «… une \emph{main dissimulée}…» Adrian éleva à nouveau la voix, «derrière tout ce qui s'est passé. Une personne derrière \emph{tout}, depuis le début. Et je ne parle pas du professeur Rogue.

--- Tu ne veux pas dire… s'étrangla Sarah.

--- Si, dit Adrian. Le véritable auteur de tout ceci est… \emph{Tracey Davis~!}

--- C'est ce que je pense aussi, dit Chloé. Après tout…» elle regarda rapidement autour d'elle. «Depuis cette affaire avec les brutes et le plafond… même les arbres dans la forêt autour de Poudlard ont l'air de \emph{trembler}, comme s'ils avaient \emph{peur…}»

Seamus Finnigan fronçait les sourcils d'un air pensif. «Je pense que je vois d'où Harry tire son… \emph{tu sais quoi}…» acheva-t-il en abaissant sa voix afin que seuls Lavande et Dean puissent entendre.

«Oh, je vois carrément ce que tu veux dire», dit Lavande. Elle ne se fatigua pas à baisser la voix. «c'est à se demander comment il a fait pour ne pas craquer et juste tuer tout le monde il y a des \emph{lustres}.

--- Personnellement, dit Dean d'une voix tout aussi basse, je trouve que ce qui est vraiment effrayant c'est que… ça aurait pu être \emph{nous}.

--- Ouais, dit Lavande. Heureusement que \emph{nous} sommes tous parfaitement sains d'esprit.»

Dean et Seamus hochèrent solennellement la tête.

\latersection{Hypothèse~: G.L.\\
(8 avril 1992, 20h08)}

Le feu de cheminette du bureau du directeur flamboya d'un vert pâle étincelant et le feu se concentra en une tornade d'émeraudes avant de s'embraser encore plus et de recracher une silhouette humaine dans les airs…

Il y eut le flou d'un mouvement lorsque la silhouette aux contours de plus en plus précis sortit sa baguette et pivota avec grâce dans l'enchaînement du mouvement précédent, à la manière d'un pas de ballet, de façon à ce que son champ de tir décrive un cercle de 360 degrés, à ce qu'il recouvre la pièce entière~; puis tout aussi abruptement, la silhouette se figea.

Dès l'instant où Harry vit cet homme, et avant d'apercevoir l'œil, il remarqua les cicatrices sur ses mains et son visage, comme s'il avait été brûlé et coupé sur toute la surface de son corps~; pourtant, ses mains et son visage étaient les seules parties à nu. Le reste du corps de l'homme était dissimulé, non pas dans des robes, mais dans un cuir qui ressemblait plus à une armure qu'à des vêtements~; du cuir gris sombre assorti à ses cheveux gris en bataille.

Le système de traitement visuel de Harry lui permit ensuite de percevoir l'œil bleu brillant situé sur le côté droit du visage de l'homme.

Une partie de l'esprit de Harry comprit que la personne que le professeur McGonagall avait appelé 'Maugrey Fol Œil' était la même que celle que Dumbledore avait appelé 'Alastor' dans le souvenir qu'il lui avait montré~; une image d'avant l'événement qui avait marqué de cicatrices chaque centimètre carré du corps de l'homme et pris un morceau de son nez…

Et une autre partie de son esprit remarqua la secousse d'adrénaline. Harry avait sorti sa baguette par pur réflexe lorsque l'homme avait pivoté hors de la cheminée de cette façon particulière, il y avait eu quelque chose qui signifiait \emph{embuscade} et la main de Harry avait déjà commencé à placer sa baguette en position de \emph{Somnium} avant qu'il ne puisse s'en empêcher. Même maintenant, l'homme en armure tenait sa baguette braquée, pas en direction de quelqu'un en particulier mais en couvrant toute la pièce, et cette baguette était toujours parfaitement alignée avec ses yeux, comme un soldat qui aurait visé avec une arme. Il y avait du danger dans sa posture et dans la position de ses bottes, du danger dans l'armure de cuir qu'il portait et du danger dans cet œil bleu étincelant.

Lorsque l'homme balafré parla à l'adresse du directeur et sa voix était coupante.

«J'imagine que tu crois que la pièce est sécurisée~?

--- Il n'y a que des amis ici, dit Dumbledore.

La tête de l'homme s'orienta rapidement vers Harry.

«Y compris \emph{lui}~?

--- Si Harry Potter n'est pas notre ami, dit Dumbledore d'un ton grave, alors nous sommes tous certainement perdus~; autant donc supposer qu'il l'est.»

La baguette de l'homme demeura braquée, pas tout à fait vers Harry.

«Le garçon a failli me tirer dessus à l'instant.

--- Euh…» dit Harry. Il remarqua que sa main était toujours serrée autour de sa baguette et la relaxa avec un effort conscient avant de la ramener contre son flanc. «Pardon, vous aviez l'air un peu… prêt au combat.»

La baguette de l'homme balafré s'écarta légèrement de sa position précédente, où elle avait été presque braquée vers Harry, mais il ne l'abaissa pas, et laissa échapper un court aboiement en guise de rire.

«Vigilance constante, hein, mon gars~? dit l'homme.

--- Ce n'est pas de la paranoïa s'ils en ont vraiment après votre peau», dit Harry en récitant le proverbe.

L'homme fit pleinement face à Harry~; et si tenté que ce dernier puisse lire quelque expression que ce soit sur le visage balafré, l'homme avait maintenant l'air \emph{intéressé}.

Les yeux de Dumbledore avaient regagné une partie du scintillement qu'ils avaient eu avant l'évasion d'Azkaban, un sourire sous sa moustache d'argent, comme si ce dernier n'était jamais parti.

«Harry, voici Alastor Maugrey, aussi appelé Fol Œil, qui dirigera l'Ordre du Phénix après moi~; si jamais quelque chose devait m'arriver, bien sûr. Alastor, voici Harry Potter. J'ai les plus grands espoirs de vous voir vous entendre \emph{à merveille}.

--- J'ai beaucoup entendu parler de toi, garçon», dit Maugrey Fol Œil. Son œil noir naturel demeura fixé sur Harry et le point de bleu étincelant pivota frénétiquement et sembla faire un tour complet dans son orbite. «Pas que du bon. Entendu dire qu'ils t'appellent l'épouve-détraqueur au département.»

Après réflexion, Harry décida de répondre d'un sourire entendu.

«Comme t'as fait ce coup-là, petit~?» dit l'homme d'une voix douce. Son œil bleu était maintenant lui aussi fixé sur Harry. «J'ai eu une petite discussion avec l'une des Aurors qui escortaient le Détraqueur depuis Azkaban. Beth Martin a dit qu'il était venu droit de la fosse et que personne ne lui avait donné d'instruction particulière en chemin. Bien sûr, elle pourrait mentir.

--- Je n'ai utilisé aucune astuce cette fois, dit Harry. J'y ai juste été à la dure. Bien sûr, je pourrais aussi mentir.»

Dumbledore était renversé dans sa chaise et gloussait à l'arrière plan, comme s'il n'était qu'un autre des appareils de son bureau et que c'était là le son qu'il faisait.

L'homme balafré se tourna vers le directeur mais sa baguette demeura pointée vers le bas et plus ou moins vers Harry. Lorsqu'il parla, sa voix était brusque, terre-à-terre. «J'ai une piste sur un hôte récent de Voldie. Tu es certain que son ombre est à Poudlard en ce moment~?

--- Pas \emph{certain}… commença Dumbledore.

--- \emph{Pardon}~?» interrompit Harry. Après avoir quasiment conclut que le Seigneur des Ténèbres n'existait pas, c'était un choc d'en entendre parler d'un ton aussi nonchalant.

«L'hôte de Voldie, dit sèchement Maugrey. Celui qu'il possédait avant de s'emparer de Granger.

--- Si les histoires sont vraies, dit Dumbledore, il existe un instrument de pouvoir qui lie l'ombre de Voldemort à ce monde~; et par ce moyen il est en mesure de négocier avec des hôtes potentiel pour obtenir possession de leur corps et leur conférer alors quelque portion de son pouvoir et de son orgueil…

--- Donc la question évidente est~: 'Qui est devenu trop puissant trop vite~?', dit soudain Maugrey. Et il s'avère qu'il y a un type qui est allé bannir la Banshee de Bandon, qui s'est farçit tout un clan de vampires renégats en Asie, qui a pisté le loup-garou de Wagga-Wagga et qui a exterminé une horde de goules avec une passoire à thé. \emph{Et} il exploite la situation au maximum~; certains parlent de l'Ordre de Merlin. L'air d'être devenu un charmeur et un politicien, pas seulement un puissant sorcier.

--- Sapristi, murmura Dumbledore. Es-tu certain qu'il ne se repose pas sur ses propres capacités~?

--- Vérifié ses notes, dit Maugrey. Bulletins montrent que Gilderoy Lockhart a reçu un Troll à son BUSE de Défense et qu'il ne s'est pas fatigué à passer l'ASPIC. Exactement le genre de couillon qui accepterait l'offre de Voldie.» L'œil tourbillonna follement dans son orbite. «À moins que tu ne te souviennes de Lockhart élève et que tu que penses qu'il avait assez de potentiel pour faire tout ça seul~?

--- Non», dit le professeur McGonagall. Elle fronça les sourcils. «Je dois dire que c'est impossible.

--- J'ai peur d'être d'accord, dit Dumbledore d'un ton empreint de peine. Ah, Gilderoy, pauvre idiot…»

Le sourire de Maugrey ressemblait plutôt à un retroussement de babines. «Trois heures du matin c'est bon pour toi, Albus~? Lockhart devrait être chez lui cette nuit.»

Harry avait écouté cela de plus en plus alarmé et se demandait si le \emph{Ministère} avait des règles imposant aux magistrats de délivrer des mandats d'arrêt -- sans même parler de l'organisation illégale de justiciers qu'il semblait maintenant avoir rejoint. «Excusez-moi, dit Harry. Qu'est-ce qui se passe \emph{exactement} à trois heures du matin~?»

Quelque chose dans le ton de sa voix dut vendre la mèche car l'homme balafré pivota vers lui~: «T'as un problème avec ça, garçon~?»

Harry marqua une pause et essaya de trouver comment formuler sa réponse pour cet étranger…

«Tu veux le descendre toi-même~? insista l'homme balafré. Avoir ta vengeance pour tes parents, hein~?

--- Non, dit Harry le plus poliment possible. Honnêtement… écoutez, si on était \emph{certain} qu'il était un hôte volontaire de Vous-Savez-Qui, ce serait une chose, mais si on en est pas \emph{certain} et que vous partez le tuer…

--- Tuer~? renifla Maugrey Fol Œil. C'est ce qui est coincé dans sa tête», ajouta Maugrey en se frappant le front «qu'on veut de lui, garçon. Si on a de la chance, Voldie ne peut pas effacer les souvenirs du couillon aussi facilement que de son vivant et Lockhart se souviendra de l'apparence du Horcruxe.»

Harry nota mentalement le mot \emph{Horcruxe} pour recherche future et dit~: «Je suis juste inquiet à l'idée qu'un innocent… que quelqu'un qui a l'air plutôt bien, s'il a \emph{tout} fait lui-même… est sur le point de souffrir.

--- Les Aurors font du mal aux gens, répondit l'homme balafré d'un ton sec. Aux gens mauvais si t'es chanceux. Parfois tu seras pas chanceux, et c'est comme ça. Rappelle toi juste~: les mages noirs font du mal à beaucoup plus de gens que nous.»

Harry prit une profonde inspiration.

«Pouvez-vous au moins \emph{essayer} de ne pas faire de mal à cette personne au cas où elle ne serait \emph{pas}…

--- Qu'est-ce qu'un première année fait dans cette pièce, Albus~?» demanda l'homme balafré, faisant maintenant face au directeur. «Et ne me dis pas que c'est à cause de ce qu'il a fait quand il était bébé.

--- Harry Potter n'est pas un élève de première année ordinaire, dit doucement le directeur. Il a déjà accompli des prouesses suffisamment impossibles pour me surprendre moi, Alastor. C'est le seul intellect de l'Ordre qui pourrait un jour rivaliser avec celui de Voldemort, ce que ni toi ni moi n'avons jamais pu faire.»

L'homme balafré se pencha sur le bureau du directeur.

«C'est un handicap. Naïf. Ne sait pas une foutue chose de la guerre. Je le veux hors d'ici et tous ses souvenirs de l'Ordre effacés avant qu'un serviteur de Voldie ne les pioche de son esprit…

--- À vrai dire, je suis Occlumens.»

Maugrey Fol Œil jeta un coup d'œil en coin au directeur, qui hocha la tête.

Puis l'homme se tourna pour faire face à Harry et leurs regards se croisèrent.

Lorsqu'une lame d'acier chauffé à blanc traversa la personne imaginaire située en façade de son esprit, la furie soudaine de l'attaque de Légilimancie faillit faire tomber Harry de sa chaise. Il n'avait pas eu l'occasion de pratiquer depuis l'entraînement de M. Bester, et Harry faillit perdre sa prise sur la personne imaginaire qu'une partie de lui prétendait être lorsque le monde de cette personne devint une lave bouillonnante, une foule furieuse de questions. Il faillit perdre prise sur sa \emph{prétention} d'halluciner, sur sa \emph{prétention} d'être la personne imaginaire qui hurlait, secouée, meurtrie par la Légilimancie déchirait sa santé mentale et la remodelait pour lui faire croire qu'elle était en feu…

Harry parvint à détacher son regard de celui de Maugrey et baissa les yeux jusqu'au menton de ce dernier.

«Tu manques de pratique, garçon», dit Maugrey. Harry ne regardait pas le visage de l'homme mais la voix de ce dernier était mortellement lugubre. «Et je ne te préviendrai qu'une fois~: Voldie n'est pas comme les autres Legilimens de l'Histoire. Il n'a pas besoin de te regarder dans les yeux et si tes boucliers sont aussi rouillés il viendra si doucement que tu ne te rendras compte de rien.

--- Bien noté», dit Harry à l'intention du menton recouvert de cicatrices. Il était plus secoué qu'il ne l'avait admis. M. Bester avait été très loin de ce niveau de puissance et n'avait jamais mis Harry à l'épreuve \emph{comme ça}. Faire semblait d'être quelqu'un qui souffrait à ce point avait été… Harry ne pouvait trouver les mots pour décrire ce que ça faisait de contenir une personne imaginaire qui souffrait autant mais ça n'avait pas été \emph{normal}. «Est-ce j'ai quand même du mérite d'être un Occlumens tout court~?

--- Alors tu penses que t'es déjà un adulte, hein~? Regarde-moi dans les yeux~!»

Harry renforça ses boucliers et regarda une fois de plus dans les yeux gris sombre et bleu étincelant.

«Déjà vu quelqu'un mourir~? demanda Maugrey Fol Œil.

--- Mes parents, dit Harry d'un ton neutre. J'ai retrouvé le souvenir en janvier quand j'ai fait face à un Détraqueur pour apprendre le Patronus. Je me souviens de la voix de Vous-Savez-Qui…» un frisson parcourut le corps de Harry, et sa baguette, dans sa main, eut un bref mouvement convulsif. «Sur le plan stratégique, mon observation principale est que Vous-Savez-Qui pouvait prononcer le sortilège de la mort en moins d'une demi-seconde, mais vous le saviez probablement déjà.»

Il y eut un hoquet venant du professeur McGonagall et le visage de Severus se contracta.

«Très bien», dit doucement Maugrey Fol Œil. Un étrange, fin sourire tordit les lèvres du visage balafré. «Je te ferai la même offre que je fais à tous mes Aurors en entraînement. Touche-moi une fois, garçon -- un coup, un sort -- et je te concéderai le droit de me parler sur le ton qui te chante.

--- Alastor~! s'exclama la voix du professeur McGonagall. Ce test n'est certainement pas raisonnable~! M. Potter, quels que soient ses autres mérites, n'a pas cent ans d'expérience au combat~!»

Les yeux de Harry parcoururent la pièce en un éclair, passèrent sur les étranges appareils, Dumbledore, Severus et le Choixpeau, s'arrêtèrent brièvement ici et là. Harry ne pouvait pas voir le professeur McGonagall de là où il était mais ça n'avait pas d'importance. Il n'y avait qu'un seul appareil qu'il avait vraiment voulu regarder, et le but de tous les autres regards avait été de masquer celui là.

«Très bien très bien», dit Harry, et il sauta de sa chaise, ignorant le hoquet du professeur McGonagall et le reniflement incrédule du maître des potions. Un sourcil de Dumbledore s'était soulevé et Maugrey souriait comme un tigre. «Assurez-vous de me réveiller dans quarante minutes s'il arrive à m'avoir.» Harry se plaça en position de duelliste, sa baguette abaissé. «Allons-y alors…»

\later

Harry ouvrit les yeux avec l'impression qu'on lui avait remplit la tête de coton.

Tout le monde était parti du bureau du directeur, le feu de cheminette était faible~; seul Dumbledore attendait derrière le bureau.

«Bonjour, Harry, dit doucement le directeur.

--- Je ne l'ai même pas vu \emph{bouger}, s'extasia-t-il tout en sentant ses muscles grincer à mesure qu'il se relevait.

--- Tu te tenais à deux pas d'Alastor Maugrey, dit Dumbledore, et tes yeux ont quitté sa baguette.»

Harry hocha la tête en prenant la Cape d'Invisibilité de sa bourse.

«je veux dire… je prenais la position de duelliste pour qu'il pense que j'étais l'idiot de base et qu'il me sous-estime mais… je dois admettre que \emph{c'était} impressionnant.

--- Alors tu avais tout prévu, Harry~? dit Dumbledore.

--- Bien sûr, dit Harry. Remarquez comme je fais ça dès que je me réveille au lieu de m'arrêter pour réfléchir.»

Harry passa la Cape au-dessus de sa tête et leva les yeux vers l'horloge murale qu'il avait subrepticement observée plus tôt.

Elle avait alors indiqué vingt heures et vingt-trois minutes, et il était maintenant vingt et une heures et cinq minutes.

\later

Minerva regarda le garçon se mettre en position de duel, baguette baissée. Pendant une seconde Minerva se demanda si Harry pourrait… non, c'était totalement ridicule, il s'agissait de \emph{Maugrey Fol Œil} et c'était plus qu'impossible. Bien sûr, c'était aussi ce qu'elle avait pensé de la métamorphose partielle…

«Allons-y alors», dit Harry, puis il tomba par terre.

Severus eut un unique gloussement. «M. Potter a ses qualités, je dois l'admettre, dit le maître des potions. Même si je ne le dirais jamais s'il était réveillé, et si vous répétez mes paroles, je les nierai, car l'ego du garçon est suffisamment grand. M. Potter a ses qualités, Maugrey, mais le duel n'en fait pas partie.»

Le gloussement de Maugrey fut plus bas et plus sombre. «Oh oui, dit Maugrey. Seuls les idiots se battent en duel. Se tenir comme ça et attendre que j'attaque, à \emph{quoi} pensait-il~? Allons, je dois lui donner une cicatrice pour qu'il n'oublie pas l'occasion…

--- Alastor~!» aboya Albus et au moment où il cria «Stop~!» Severus bondit vers l'avant et Maugrey Fol Œil abaissa ostensiblement sa baguette vers le corps de Harry Potter.

«\emph{Stupéfix~!}»

Le corps de Maugrey Fol Œil sembla presque clignoter lorsqu'il pivota sur son pied de bois aussi vite que l'éclair, plus vite que Minerva avait jamais vu quiconque bouger sans l'aide de magie, et le sortilège d'étourdissement rouge traversa les airs soudain vides, manqua Severus de peu, alla s'écraser sur le mur opposé et lorsque ses yeux revinrent à Maugrey il y avait dix-sept sphères irradiantes au rythme de \emph{Sagitta Magica}, visibles seulement un instant avant de se zébrer de lumière et de frapper \emph{quelque chose} qui tomba au sol avec un bruit sourd…

\later

«Rebonjour, Harry, dit Dumbledore.

--- Je n'arrive pas à \emph{croire} que ce type a un temps de réaction pareil», dit Harry en époussetant sa Cape et en se relevant de l'endroit où il était resté étendu et inaperçu de son lui précédent. «Je n'arrive pas non plus à croire à sa vitesse de déplacement. Je vais devoir trouver un moyen de l'avoir sans prononcer de sortilège qui me révélerait…»

\later

… Et Maugrey Fol Œil s'abaissa si vite et si violemment que ses paumes frappèrent le sol. Elle ne vit presque pas les deux petits fils blancs qui passèrent là où il s'était trouvé, mais ses yeux s'orientèrent vers l'étincelle bleue lorsque les fils heurtèrent l'un des appareils du directeur, et le temps qu'elle parvienne à ramener se yeux vers Fol Œil, ce dernier s'était élégamment remis sur pied, sa baguette dansait si vite qu'elle était indistinguable et il y eut un autre bruit sourd…

\later

«Rebonjour, Harry.

--- Excusez-moi, monsieur le directeur, mais pourriez-vous me laisser descendre vos escaliers puis revenir avant que je fasse mon dernier saut en arrière~? Ça va me prendre plus d'une heure de préparation…»

\later

Minerva demeura bouche bée devant Maugrey Fol Œil, qui n'avait pas le moins du monde abaissé sa baguette~; et Severus avait quasiment l'air secoué.

«Alors garçon~? dit Maugrey. Qu'est-ce que t'as d'autre~?»

La tête de Harry apparut, flottante dans le vide, et une main invisible rejeta la capuche de sa cape d'invisibilité.

«Cet œil», dit Harry Potter. Il y avait une étrange lueur féroce dans les yeux du garçon. «Ce n'est pas un appareil ordinaire. Il voit parfaitement à travers ma cape d'invisibilité. Vous avez évité mon taser métamorphosé dès que j'ai commencé à le brandir, alors que je n'avais prononcé aucune incantation. Et maintenant que je vous ai observé une fois de plus… vous avez remarqué tous mes moi revenus dans le passé à la seconde où vous êtes arrivé par la cheminette, n'est-ce pas~?»

Maugrey Fol Œil souriait, le même sourire carnassier qu'elle l'avait vu avoir lorsqu'ils faisaient face à Voldemort lui-même. «Passe cent ans à chasser les mages noirs et tu verras tout, dit Maugrey. Un jour j'ai arrêté un jeune japonais qui avait essayé une astuce similaire. Il a découvert d'une façon déplaisante que sa technique de clones de l'ombre n'était pas à la hauteur de cet œil.

--- Vous voyez dans toutes les directions», dit Harry Potter, cette étrange lueur féroce toujours dans son regard. «Peu importe là où il regarde, il voit tout autour de vous.»

Le sourire de tigre de Maugrey s'élargit. «Il n'y a pas d'autres toi dans cette pièce, maintenant, dit Maugrey. Tu penses que c'est parce que tu abandonneras après cette fois ou parce que tu gagneras~? Envie de parier, garçon~?

--- C'est mon dernier essai parce que j'ai décidé de mettre mes trois dernières heures sur ce coup, dit Harry Potter. Quant à savoir si je gagne…»

Un flou emplit tout le bureau du directeur. Maugrey Fol Œil bondit de côté à une vitesse aveuglante, un instant plus tard la tête de Harry se rejeta en arrière et il cria «\emph{Stuporfix~!}»

Trois miroitements dépassèrent la tête de Harry au moment où un tir rouge surgit de là où il s'était trouvé, un tir qui dépassa Maugrey lorsque ce dernier évita à nouveau…

Si elle avait cligné des yeux, elle l'aurait manqué, le tir rouge marquant un angle à mi-parcours et s'écrasant dans l'oreille de Maugrey.

Maugrey tomba.

La tête flottante de Harry descendit pour atteindre la hauteur de celle d'un élève de première année en appui sur ses mains et ses genoux, puis descendit plus bas, à terre, son visage révélant un épuisement soudain.

Minerva McGonagall dit~: «Par \emph{Merlin}, qu'est-ce qui vient de…»

\later

«Donc tu es allé voir Flitwick», dit Maugrey. L'Auror à la retraite était maintenant assis sur une chaise et buvait de longue gorgées d'une potion de restauration qu'il avait prise de sa ceinture.

Harry Potter hocha la tête, à présent assis sur sa propre chaise plutôt que perché sur un accoudoir. «J'ai d'abord essayé le professeur de Défense mais…» le garçon grimaça. «Il… n'était pas disponible. Enfin j'avais décidé que ça méritait de risquer cinq points, et quand on se dit que ça vaut le coup de prendre un risque, on ne peut pas se plaindre au moment de payer. Quoi qu'il en soit, je me suis dit que si vous aviez un œil capable de voir ce que les autres ne peuvent pas voir, alors, comme Isaac Asimov l'a fait remarqué dans \emph{Seconde Fondation}, l'arme à utiliser était une lumière forte. Vous savez ce qu'on dit, lisez assez de science-fiction et vous aurez tout lu au moins une fois. Bref, j'ai dit au professeur Flitwick que j'avais besoin d'un sortilège qui ferait un grand nombre de formes, étincelantes et clignotantes et capables de remplir tout le bureau, mais invisibles, afin que seul votre œil puisse les voir. Je n'avais aucune idée de ce que ça pouvait même vouloir \emph{dire} de lancer une illusion invisible, mais je me suis dit que si je ne mentionnais pas ça à voix haute, le professeur Flitwick s'exécuterait, et il l'a fait. Il s'avère que je n'aurais pu moi-même lancer aucun sortilège de ce genre mais Flitwick m'a enchanté un objet à usage unique -- même si j'ai dû le persuader que ce n'était pas de la triche puisque rien ne \emph{pourrait} être de la triche contre un Auror qui a vécu assez longtemps pour prendre sa retraite. Et je ne voyais toujours pas comment vous toucher puisque vous bougiez aussi vite. Donc je l'ai interrogé sur les sorts à tête chercheuse et c'est là que Flitwick m'a montré le maléfice que j'ai lancé à la fin, l'étourdisseur à embardées. C'est l'une des inventions du professeur Flitwick -- c'est un champion de duel doublé d'un maître des sortilèges…

--- Je sais ça, petit.

--- Pardon. Quoi qu'il en soit, le professeur a dit qu'il a quitté le monde du duel avant d'avoir une chance d'utiliser ce sortilège puisqu'il ne fonctionne que comme coup final contre un ennemi à découvert. Le maléfice s'approche autant que possible de la cible le long de sa trajectoire initiale, et dès qu'il détecte qu'elle s'éloigne à nouveau, il pivote en plein parcours et fonce droit vers elle. Il ne peut faire qu'une seule embardée -- mais l'incantation est très proche de 'Stupéfix' et le tir a la même couleur, donc si l'ennemi croit que c'est un sortilège d'étourdissement classique et tente d'éviter comme d'habitude, ce changement de cible à mi-parcours l'achèvera. Oh, et le professeur a demandé à ce qu'aucun de nous ne mentionne ce coup spécial juste au cas où il aurait une chance de l'utiliser un jour en compétition.

--- Mais…» dit le professeur McGonagall. Elle jeta un coup d'œil à Maugrey Fol-Œil, qui hochait la tête d'un air approbateur, et à Severus, qui gardait son visage décidément impassible. «M. Potter, vous venez d'étourdir \emph{Maugrey Fol Œil}~! Le plus célèbre chasseur de mages noirs de l'histoire du bureau des Aurors~! Ça aurait dû être impossible~!»

Maugrey laissa échapper un sombre gloussement.

«Comment tu réponds à \emph{ça}, gamin~? Je suis curieux.

--- Eh bien… dit Harry. Pour commencer, professeur McGonagall, aucun de nous deux ne se battait sérieusement.

--- \emph{Aucun de vous deux~?}

--- Bien sûr, dit Harry. Dans un combat sérieux, M. Maugrey aurait abattu toutes mes copies immédiatement sans attendre qu'elles attaquent. Et de mon côté, si ça avait été \emph{vraiment} nécessaire de descendre le plus célèbre Auror de l'histoire du bureau, j'aurais obtenu de Dumbledore qu'il le fasse pour moi. Et au-delà de ça… puisque ça \emph{n'était pas} un vrai combat…» Harry s'interrompit. «Comment formuler ça~? Les sorciers ont l'habitude de duels où les gens se battent chacun leur tour avec des sortilèges pendant un bon moment. Mais si deux Moldus avec des pistolets se font face dans une petite pièce et se tirent l'un sur l'autre… alors le premier qui touche gagne. Et si l'un d'eux rate délibérément ses tirs et n'a cesse de redonner une chance à l'autre -- comme M. Maugrey n'a eu cesse de le faire -- eh bien il faudrait être plutôt pathétique pour réussir à perdre.

--- Oh, pas \emph{si} pathétique que ça», dit Maugrey avec un sourire légèrement menaçant.

Harry ne sembla pas le remarquer.

«On pourrait dire que M. Maugrey me testait pour voir si j'allais essayer de le \emph{combattre} ou si j'allais essayer de \emph{gagner}. C'est-à-dire, si j'allais endosser le \emph{rôle} d'un combattant~: utiliser des sortilèges standards que je connaissais déjà même en ne m'attendant pas à ce que les \emph{conséquences} de ces actions me procurent la victoire~; ou si j'allais explorer des plans inhabituels jusqu'à trouver quelque chose \emph{capable} de me faire gagner. Comme la différence entre un élève assit en classe parce que c'est ce qu'un élève fait, et un autre suffisamment impliqué pour se demander ce qu'il doit faire pour \emph{réellement} apprendre un sujet, prêt à pratiquer autant que nécessaire. Vous voyez, professeur McGonagall~? Vu comme ça -- en se rendant compte que M. Maugrey me donnait des chances et que je n'aurais pas dû attaquer sans penser que je pouvais gagner -- alors ce n'est pas si flatteur que ça pour moi puisqu'il m'a en fait fallu trois essais pour l'avoir. Et puis, comme je l'ai dit, dans un vrai combat, M. Maugrey aurait pu \emph{se} rendre invisible, ou lever ses boucliers…

--- Ne te repose pas trop sur les boucliers, garçon», dit Maugrey. L'Auror vêtu de cuir prit une autre gorgée de sa fiole réparatrice. «Ce que tu apprends en première année à l'académie ne reste pas vrai pour toujours, pas contre les mages noirs les plus puissants. Pour chaque bouclier jamais conçu il y a quelque malédiction qui le traverse net si tu n'es pas assez rapide pour lancer le contre. Et il y a un sortilège qui traverse tout, et c'est celui que n'importe quel Mangemort utilisera.»

Harry Potter hocha gravement la tête.

«Oui, certains sortilèges sont impossibles à bloquer. Je me souviendrai de ça au cas où quelqu'un me lance le sortilège de la mort. Une deuxième fois.

--- Ce genre de malinerie finit par tuer, garçon, ne l'oublie pas.»

Un soupir triste du Survivant. «Je sais. Désolé.»

Harry ouvrit la bouche, puis s'interrompit. «Je ne vous dirai pas comment mener une guerre, dit enfin le Survivant. Je n'ai aucune expérience en la matière. Tout ce que je sais c'est qu'il y a des conséquences. Sachez s'il vous plaît que selon moi Lockhart est probablement innocent, donc si vous pouvez éviter de lui faire du mal sans prendre trop de risques…» le garçon haussa les épaules. «J'en ignore le coût. Juste, s'il vous plaît, dans la mesure du possible et s'il s'avère innocent, faites attention à ne pas lui faire trop de mal.

--- Si je peux, dit Maugrey.

--- Et… vous comptez observer son esprit à la recherche d'indices concernant le Seigneur des Ténèbres, n'est-ce pas~? Je ne connais pas les règles de l'Angleterre magique sur ce qui constitue une preuve recevable -- mais tout le monde est toujours coupable d'avoir violé \emph{une} loi ou une autre, il y en a trop pour qu'il en soit autrement. Donc s'il ne s'agit \emph{pas} du Seigneur des Ténèbres, ne le livrez pas au ministère, contentez-vous de l'Oublietter et de le laisser partir, d'accord~?»

Maugrey fronça les sourcils.

«Petit, personne ne devient aussi puissant aussi vite sans être en train de trafiquer \emph{quelque chose}.

--- Alors laissez ça aux Aurors normaux, si et quand ils trouveront des preuves par les méthodes habituelles. S'il vous plaît, M. Maugrey. Appelez ça un caprice issu de mon éducation moldue mais s'il ne s'agit \emph{pas} de la guerre je ne veux pas nous voir jouer le rôle de la méchante police qui entre de force chez les gens en pleine nuit, fouille leur esprit et les envoie à Azkaban.

»Je ne vois pas l'intérêt, petit, mais j'imagine que je peux te faire cette faveur.

--- Y a-t-il autre chose, Alastor~? s'enquit Albus.

--- Oui, dit Maugrey. Au sujet de ce professeur de Défense que vous avez chez vous…»

\latersection{Hypothèse~: Gilderoy Lockhart~: FIN}

\latersection{Hypothèse~: Dumbledore\\
(9 avril 1992, 17h32)}

Alors que le professeur Quirrell faisait lentement léviter sa tasse, celle-ci subit une secousse à mi-parcours et envoya le liquide noir et translucide passer presque de l'autre côté, de telle manière que trois gouttes et trois gouttes seulement franchirent le bord de la tasse. Harry l'aurait manqué s'il n'avait pas été en train de l'observer attentivement~; car la main du professeur avait été parfaitement stable avant et le demeura ensuite.

Si ce petit geste saccadé évoluait vers des tremblements permanents, ce serait la fin de toute magie, sauf sans baguette, pour le professeur de Défense. Les mouvements de baguettes ne pardonnaient aucun tremblement des doigts. Quant à savoir à quel point cela handicaperait \emph{vraiment} le professeur Quirrell, si ça devait l'handicaper tout court, Harry ne pouvait le deviner. Le professeur de Défense était certainement capable de magie en l'absence de baguette, mais il continuait à en utiliser une lorsque le sujet avait une taille suffisante -- mais pour lui ce n'était peut-être qu'une question de confort…

«La folie», dit le professeur Quirrell en sirotant précautionneusement son thé -- il regardait la tasse, pas Harry, ce qui était inhabituel chez lui - «peut constituer une signature.»

Le petit bureau du professeur de Défense était silencieux, la pièce insonorisée tranquille comme le bureau du directeur ne pourrait jamais l'être. Ils achevaient parfois d'inspirer ou d'expirer au même moment~; apparaissait alors un vide auditif qui était lui-même presque un son.

«Je suis d'accord, en un sens, dit Harry. Si quelqu'un me dit que tout le monde le \emph{regarde} et que ses sous-vêtements sont nettoyés à la poudre contrôleuse de pensées, je sais qu'il est psychotique, parce que c'est la signature standard de la psychose. Mais si vous me dites que \emph{n'importe quoi} d'incompréhensible dirige les soupçons vers Albus Dumbledore, cela me semble… aller trop loin. Ce n'est pas parce que je ne peux pas discerner de but qu'il n'y en \emph{a} pas.

--- Pas de but~? dit le professeur Quirrell. Oh, mais la folie de Dumbledore n'est pas qu'il est sans but, mais qu'il en a trop. Le directeur a peut-être prévu que cela mènerait Lucius Malfoy à jouer sa main et donc à la perdre pour se venger de vous -- ou peut-être a-t-il une dizaine d'autres intrigues en cours. Qui sait ce que le directeur pense avoir des raisons de faire quand il a déjà trouvé des raisons de faire tant de choses étranges~?»

Harry avait poliment décliné le thé, même en sachant que le professeur Quirrell saurait ce que cela signifiait. Il avait songé à apporter sa propre canette de soda -- mais il avait décidé de ne pas le faire après s'être rendu compte de la facilité avec laquelle le professeur de Défense aurait pu y téléporter un peu de potion, et ce même si aucun des deux ne pouvait directement toucher l'autre par sa magie.

«J'ai maintenant un peu observé Dumbledore, dit Harry. À moins que tout ce que j'ai vu n'ait été un mensonge, je trouve difficile de le croire ne serait-ce que capable de construire un plan dans lequel l'une de ses élèves est envoyée à Azkaban.

--- Ah», dit doucement le professeur de Défense, un petit reflet de la tasse brillant dans ses yeux pâles. «Mais peut-être est-ce une autre signature, M. Potter. Vous n'avez pas encore compris la perspective d'un homme tel que Dumbledore. S'il doit, pour quelque cause suffisamment noble, sacrifier une élève -- allons, qui choisirait-il sinon celle qui s'est déclarée être une héroïne~?»

Cela fit réfléchir Harry. Peut-être était-ce juste le biais rétrospectif mais il \emph{semblait} que cela densifiait une partie de la masse de probabilité de cette hypothèse autour de la prise au piège de Hermione plutôt que de quelqu'un d'autre. De même, le professeur \emph{avait} prédit à l'avance que Dumbledore pourrait prendre Drago pour cible…

\emph{Mais si vous êtes derrière tout ça, professeur, vous pourriez avoir arrangé vos plans pour prendre le directeur au piège et avoir pris la précaution de jeter à l'avance le doute sur lui.}

Le concept de 'preuve' avait un sens quelque peu différent lorsque vous aviez affaire à quelqu'un qui se déclarait jouer 'un niveau au-dessus de vous'.

«Je vois où vous voulez en venir, professeur», dit Harry d'un ton neutre, sans laisser rien paraître de ses autres pensées. «Vous pensez-donc que le plus probable est que le directeur a piégé Hermione~?

--- Pas nécessairement, M. Potter.» Le professeur Quirrell vida sa tasse d'une gorgée puis la déposa, et la tasse cogna sèchement la table. «Il y a aussi Severus Rogue -- mais ce qu'il pourrait espérer obtenir de tout ceci, je ne puis le deviner. Il n'est donc pas non plus mon principal suspect.

--- Alors qui est-ce~?» dit Harry, un peu perplexe. Le professeur Quirrell n'était sûrement pas sur le point de répondre 'Vous-Savez-Qui'…

«Les Aurors ont une règle, dit le professeur Quirrell. Enquêter sur la victime. De nombreux soi-disant criminels s'imaginent que s'ils semblent être les victimes d'un crime, ils ne seront pas soupçonnés. Tant de criminels se l'imaginent, à vrai dire, que tout Auror galonné a déjà fait face à cette situation plus d'une dizaine de fois.

--- Vous n'essayez pas sérieusement de me convaincre que \emph{Hermione}…»

Le professeur de Défense donnait à Harry un de ces \emph{regards} mi-clos qui voulaient dire qu'il était stupide.

\emph{Drago}~? Drago avait été interrogé sous Veritaserum… mais Lucius pourrait avoir eu assez de pouvoir pour manipuler les Aurors afin que… oh.

«Vous pensez que \emph{Lucius Malfoy} a monté le coup contre son \emph{propre fils}~? dit Harry.

--- Pourquoi pas~? dit doucement le professeur Quirrell. Ayant écouté l'enregistrement du témoignage de M. Malfoy, il m'apparaît que vous avez en partie réussi à faire changer le point de vue politique de M. Malfoy. Si Lucius Malfoy a appris cela plus tôt… il a pu décider que son \emph{ancien} héritier est devenu un fardeau.

--- Je n'y crois pas, dit catégoriquement Harry.

--- Vous voilà naïf par caprice, M. Potter. Les livres d'Histoire sont pleins de disputes familiales devenues meurtrières à cause de problèmes et des menaces bien moindres que ceux que M. Malfoy posait à son père. J'imagine que vous me direz ensuite que Lord Malfoy des Mangemorts est bien trop bon pour souhaiter autant de mal à son fils.» Un soupçon de lourd sarcasme.

«Eh bien oui, franchement, dit Harry. L'amour existe, professeur, c'est un phénomène dont les effets sont observables. Les cerveaux sont réels, les émotions sont réelles, et l'amour fait autant partie de la réalité que les pommes et les arbres. Si vous faisiez des prédictions expérimentales sans prendre l'amour parental en compte, vous auriez sacrément du mal à expliquer pourquoi mes propres parents ne m'ont pas abandonné dans un orphelinat après l'Incident du Projet Scientifique.»

Le professeur de Défense ne réagit absolument pas à cela.

Harry continua.

«À ce que Drago en dit, Lucius le faisait passer avant des votes importants au Magenmagot. C'est une observation importante, car il y existe des moyens moins chers de mimer l'amour~; si l'on souhaite le mimer. Et ce n'est pas comme si la probabilité à priori qu'un parent aime son enfant était \emph{faible}. J'imagine qu'il est possible que Lucius ait juste prit le \emph{rôle} d'un père aimant et qu'il ait renoncé à ce rôle après avoir appris que Drago fomentait avec des nés-Moldus. Mais comme on dit, professeur, il faut distinguer la possibilité de la probabilité.

--- Le crime n'en est que meilleur, dit le professeur de Défense toujours de ce ton doux, si personne n'est prêt à croire qu'il l'a commis.

--- Et pour commencer, comment Lucius modifierait-il la mémoire de Hermione sans alerter le système de sécurité~? \emph{Il} n'est pas professeur… ah, oui, vous pensez que c'est le professeur Rogue.

--- Faux, dit le professeur de Défense. Lucius Malfoy ne se reposerait sur aucun de ses serviteurs pour cette mission. Mais supposez qu'un professeur de Poudlard, assez intelligent pour bien lancer un sortilège d'amnésie mais sans grande capacité au combat, visite Pré-au-Lard. D'une sombre allée la forme noire de Malfoy s'avance -- il viendrait en personne, pour cela -- et lui un seul mot.

--- Imperium.

--- \emph{Legilimens}, plutôt, dit le professeur Quirrell. Je ne sais pas si le système de sécurité de Poudlard s'activerait au retour d'un professeur victime de l'Imperium. Et si je l'ignore, Malfoy l'ignore probablement aussi. Mais Malfoy est au moins un parfait Occlumens; il pourrait user de Légilimancie. Et quant à la cible… peut-être Aurora Sinistra~; personne ne s'interrogerait sur les mouvements nocturnes du professeur d'Astronomie.

--- Ou de façon encore plus évidente, le professeur Chourave, dit Harry. Puisqu'elle est la dernière personne que quiconque soupçonnerait.»

Le professeur de Défense hésita avec minutie.

«Peut-être.

--- En fait, dit alors Harry en fronçant pensivement les sourcils. Je suppose que vous n'auriez pas à l'esprit les noms des professeur actuels qui étaient à Poudlard en 1943, quand M. Hagrid s'est fait avoir~?

--- Dumbledore enseignait la métamorphose, Kettleburn les créatures magiques et Vector l'Arithmancie, répondit immédiatement le professeur Quirrell. Et je crois que Bathesda Babbling, maintenant aux anciennes runes, était alors préfète Serdaigle. Mais M. Potter, il n'y a aucune raison de croire que quiconque hormis Vous-Savez-Qui était impliqué dans \emph{cette} affaire.»

Harry haussa les épaules avec élégance.

«Ça avait l'air de mériter que la question soit posée, juste pour vérifier. Quoi qu'il en soit, professeur, je reconnais qu'il est possible que quelque personne extérieure ait Légilimancé un employé de Poudlard -- et qu'il lui ait lancé Oubliettes ensuite, personne n'oublierait de faire ça. Mais je ne pense \emph{pas} que Lucius Malfoy est un candidat probable au titre de cerveau de l'affaire. Il est possible, mais pas probable, que l'amour apparent de Lucius pour Drago n'ait été qu'un sens du devoir et qu'il soit depuis parti en fumée. Il est possible, mais pas probable, que tout ce que Lucius a fait face au Magenmagot n'ait été qu'un rôle. L'extérieur des gens ne ressemble pas toujours à ce qu'ils sont à l'intérieur, comme vous l'avez dit. Mais il y a un indice qui ne coïncide pas du tout avec le reste.

--- Et de quoi s'agirait-il~?» dit le professeur de Défense les yeux mi-clos.

--- Lucius a tenté de refuser une offre de cent-mille Gallions en échange de la vie de Hermione. J'ai vu à quel point le Magenmagot était surpris lorsque Lucius a dit qu'il refusait, en dépit du code d'honneur. Le Magenmagot ne \emph{s'attendait pas} à ça de sa part. Pourquoi n'a-t-il \emph{pas} juste pris l'argent en se donnant un air indigné et en faisant semblant de grincer des dents~? Vraiment envoyer Hermione à Azkaban ne pouvait pas avoir tant d'importance que cela pour lui.»

Il y eut un silence. «Peut-être s'est-il laissé emporter par le rôle qu'il jouait, dit le professeur Quirrell. Cela arrive, M. Potter, dans le feu de l'action.

--- Possible, dit Harry. Mais c'est une \emph{improbabilité} de plus à postuler -- et après s'être vue offrir autant d'excuses, une théorie ne peut plus être en tête de liste. Autre chose en particulier à laquelle vous pensez que je devrais réfléchir, parmi toutes les autres possibilités~?»

Il y eut un long silence. Les yeux du professeur de Défense s'abaissèrent vers sa tasse~; ils semblaient inhabituellement distants.

«J'imagine que je peux songer à un dernier suspect», dit enfin le professeur de Défense.

Harry hocha la tête.

Le professeur de Défense ne sembla pas le remarquer et ne fit que poursuivre.

«Le directeur vous a-t-il dit quoi que ce soit -- même un indice -- au sujet de la prophétie du professeur Trelawney~?

--- \emph{Hein~?}» dit automatiquement Harry, convertissant son choc soudain en la meilleure dissimulation qu'il puisse parvenir à afficher. Il jouait probablement au mauvais niveau pour tromper le professeur Quirrell, mais Harry ne pouvait \emph{certainement} pas prendre le temps de réfléchir avant de répondre -- \emph{attends, mais comment diable le professeur Quirrell pourrait-il être au courant de} ça \emph{-} «Le professeur Trelawney a fait une prophétie~?

--- Vous \emph{étiez} là pour en entendre le début, dit le professeur en fronçant les sourcils. Vous avez annoncé à l'école entière que la prophétie ne pouvait pas vous concerner puisque vous étiez déjà là et ne pouviez donc pas être en train d'arriver.»

\prophesy{Il vient. Celui qui déchirera l' -}

Le professeur Trelawney était allée jusque là avant que Dumbledore ne se saisisse d'elle et ne disparaisse.

«Oh, \emph{cette} prophétie, dit Harry. Désolé~! Elle m'était complètement sortie de l'esprit~!»

Harry songea qu'il avait mit trop de force dans la dernière phrase et s'attendit à 80~\% à ce que le professeur Quirrell dise

«\emph{Aha, alors M. Potter, quelle est cette mystérieuse} autre \emph{prophétie que vous vous fatiguez tant à nier…}

--- C'est idiot, dit le professeur d'un ton sec, si vous me dites effectivement la vérité. Les prophéties ne sont pas des choses triviales. Je me suis creusé la tête sur le peu que j'ai entendu, mais un fragment aussi mince ne suffit tout simplement pas.

--- Vous pensez que celui qui vient est celui qui aurait pu piéger Hermione~?» dit Harry. Au même instant, son esprit allouait une autre hypothèse, \emph{prédicat au référent incertain~: celui-qui-vient}.

«Sauf le respect de Mlle Granger, dit le professeur de Défense avec un autre froncement de sourcil, sa vie ou sa mort ne semblent pas si importantes. Mais quelqu'un \emph{devait} venir -- quelqu'un qui, selon votre interprétation, n'était pas déjà là -- et quelqu'un d'aussi important, un joueur encore inconnu… qui sait ce qu'il peut avoir fait \emph{d'autre}~?»

Harry hocha la tête et soupira mentalement car il allait avoir à refaire ses calculs de probabilité Voldemoresques avec encore un indice de plus dans le mélange.

Le professeur parla avec les yeux mi-clos, donnant l'impression qu'il regardait à travers des fentes.

«Plus que la question de savoir de qui la prophétie parlait -- qui était censé \emph{l'entendre}~? Il est dit que les destins sont dits à ceux dotés du pouvoir de les causer ou de les prévenir. Dumbledore. Moi-même. Vous. En quatrième, loin derrière, Severus Rogue. Mais de ces quatre, Dumbledore et Rogue seraient souvent en présence de Trelawney. Vous et moi êtes ceux qui n'avions pas passé beaucoup de temps près d'elle avant ce dimanche. Je pense qu'il est assez probable que la prophétie ait été dite à l'intention de l'un de \emph{nous} -- avant que Dumbledore n'enlève la prophétesse. Le directeur vous \emph{a-t-il} dit quoi que ce soit d'autre~?» La voix du professeur était maintenant exigeante. «Je pense avoir entendu trop de force dans cette dénégation, M. Potter.

--- Honnêtement, non, dit Harry. Ça m'était sincèrement sorti de la tête.

--- Alors je suis plutôt remonté contre lui, dit doucement le professeur Quirrell. En fait, je suis en colère.»

Harry ne dit rien. Il ne sua même pas. C'était peut-être une mauvaise raison d'être confiant, mais sur ce point particulier, Harry était réellement innocent.

Le professeur Quirrell hocha la tête une fois, avec force, comme pour confirmer quelque chose.

«Si nous n'avons plus rien à nous dire, M. Potter, vous pouvez disposer.

--- Je peux penser à un \emph{autre} suspect, dit Harry. Quelqu'un que vous n'avez pas du tout mis sur votre liste. Pourriez-vous l'analyser pour moi, professeur~?»

Il y eut un autre de ces moments dont le silence devenait presque un son.

«Pour ce qui est de \emph{ce} suspect, dit doucement le professeur de Défense, je pense que vous mènerez l'accusation vous-même, M. Potter, sans aide de ma part. J'ai déjà reçu de telles requêtes, et l'expérience me mène à refuser. Soit je ferai un trop bon travail en m'accusant moi-même, et je vous convaincrai que je suis coupable -- ou vous déciderez que mon accusation manquait trop de conviction et que je suis coupable. Je remarquerai seulement ceci pour ma défense~: que j'aurais effectivement eu besoin d'une très bonne raison pour mettre en danger votre fragile alliance avec l'héritier de la maison Malfoy.»

\latersection{Hypothèse~: Le professeur de Défense\\
(8 avril 1992, 20h37)}

«… et j'ai donc peur de devoir partir, dit Dumbledore d'un ton grave. J'ai promis à Quirinus… c'est-à-dire, j'ai promis au professeur de Défense… que je ne tenterai en rien de découvrir sa véritable identité, par mes propres moyens ou par ceux d'un autre.

--- Et pourquoi t'aurais fait une promesse idiote pareille, alors~? lâcha Maugrey Fol Œil.

--- Il m'a laissé entendre que c'était une condition non-négociable de son embauche.» Dumbledore jeta un coup d'œil au professeur McGonagall et un sourire entendu passa brièvement sur son visage. «Et Minerva avait été très claire sur le fait que Poudlard avait \emph{besoin} d'un professeur de Défense compétent cette année, dussé-je tirer Grindelwald de Nurmengard et faire valoir d'anciennes affections pour le persuader d'accepter le poste.

--- Je ne l'ai pas \emph{tout à fait} formulé comme ça…

--- Votre expression l'a fait pour vous, ma chère.»

Et il n'y eut bientôt plus qu'eux quatre -- Harry, le professeur McGonagall, le maître des potions et Alastor Maugrey aussi appelé 'Fol Œil' -- confortablement installés dans le bureau du directeur.

Il était étrange de constater à quel point le bureau du directeur semblait… \emph{déséquilibré}… sans le directeur à l'intérieur. Sans avoir l'ancien sorcier ridé pour donner un air \emph{solennel} à tout ceci, il ne restait plus que quatre personnes essayant d'avoir une réunion sérieuse entourées de gadgets bruyants et étranges. Clairement visible depuis l'accoudoir sur lequel s'était perché Harry, un objet conique tronqué, comme un cône dont on aurait enlevé le haut, pivotait lentement autour d'un lumière centrale pulsative à laquelle il faisait de l'ombre sans l'obscurcir~; et chaque fois que la lumière interne émettait une pulsation, l'assemblage faisait un \emph{vroom vroom vroom} bizarrement lointain, étouffé, comme s'il venait de l'autre côté d'un mur, alors que le machin pivotant et sa section section de cône n'étaient qu'à un mètre ou deux.

\emph{Vroom… vroom… vroom…}

Et il y avait les nombreux corps de Harry Potter qui respiraient toujours et qu'il avait entassés dans un coin tranquille, rangeant ainsi un désordre qui était sien à plus d'un titre (seul un corps n'était \emph{pas} à l'intérieur d'une copie de la Cape d'Invisibilité~; mais il ne fallait qu'un petit effort de concentration de la part de Harry pour qu'il perçoive ses autres lui sous la Cape dont il était le maître - un effort qu'il avait pris soin de ne \emph{pas} exercer plus tôt afin d'éviter de recevoir du futur des informations sur des faits qu'il souhaitait déterminer lui-même). Ce qui était triste, c'était qu'à ce stade, voir son propre corps allongé dans un coin ne lui semblait pas si fou que ça. C'était juste… Poudlard.

«Très bien dans ce cas», dit Maugrey, l'air mécontent de la situation. L'homme sortit un dossier noir de sous son armure de cuir. «C'est une copie de ce que les hommes d'Amélia ont rassemblé. Il est presque certain qu'elle sait qu'on l'a, mais c'est sous le manteau, compris~? Quoi qu'il en soit…»

Et Maugrey leur dit qui, selon le département de justice magique, était réellement 'Quirinus Quirrell'. Un élève de Poudlard apparemment ordinaire (bien que suffisamment talentueux pour avoir été battu de peu dans la course au poste de préfet-en-chef) parti en vacance en Albanie après avoir reçu son diplôme, disparu, revenu 25 ans plus tard, puis pris dans la guerre des sorciers…

«C'est le meurtre de la maison Monroe qui a donné son nom à Voldie, dit Maugrey. Jusqu'alors, il n'avait été qu'un autre mage noir avec des délires de grandeur et Bellatrix Black. Mais après ça…» Maugrey renifla. «Tous les idiots du pays sont accourus pour le servir. On aurait \emph{espéré} que le Magenmagot devienne sérieux après s'être rendu compte que Voldie était prêt à tuer leurs excellences. Et c'est exactement ce que ces imbéciles ont fait~: \emph{espérer} qu'un autre idiot prenne la situation au sérieux. Pas un seul de ces lâches ne voulait se tenir en première ligne. Il y avait Monroe, Crouch, Bones et Londubat. Ils étaient les seuls au ministère à oser dire un mot qui aurait pu offenser Voldie.

--- C'est comme ça que votre maison a été anoblie, M. Potter, intervint la voix solennelle du Professeur McGonagall. Il existe une ancienne loi disant que si quiconque met fin à une Très Ancienne Maison, quiconque venge son sang sera anobli. La Maison Potter était certainement déjà plus vieilles que certaines lignées dites Anciennes. Mais la vôtre ne reçut le titre de Noble Maison d'Angleterre qu'après la fin de la guerre, en signe de reconnaissance du fait qu'elle avait vengé la Très Ancienne Maison des Monroe.

--- Gratitude et tout ça, dit Maugrey avec aigreur. Ça n'a pas duré, mais au moins James et Lily ont eu un beau titre et une médaille inutile à emmener dans leur tombe. Mais c'est oublier huit ans d'horreur complète après la disparition de Monroe et l'exécution de Regulus Black - il était la source de Monroe chez les Mangemorts, on en est presque sûrs - par Voldie. Comme un barrage qui se brise et du sang qui s'écoule et inonde tout le pays. Il a fallut qu'Albus Dumbledore en personne prenne la place de Monroe, et ça a à peine suffit à nous permettre de survivre.»

Harry écoutait avec un étrange sentiment d'irréalité. Une partie \emph{semblait} coller, correspondre aux observations… en particulier au discours que le Professeur Quirrell avait fait avant Noël… et pourtant…

Ils parlaient quand même du \emph{Professeur Quirrell}.

«Donc voilà qui le département pense que le Professeur de Défense est, dit Maugrey Fol Œil concluant son rapport. Maintenant qu'est-ce que \emph{tu} en penses, petit~?

--- Eh bien…» dit lentement Harry. \emph{Il est aussi possible d'avoir un masque sous le masque.} «La pensée évidente suivante est que ce 'David Monroe' est bel et bien mort pendant la guerre et que c'est juste quelqu'un qui prétend être David Monroe prétendant être Quirinus Quirrell.

--- C'est \emph{évident}~? dit le Professeur McGonagall. Par Merlin…

--- Vraiment, petit~?» dit Maugrey Fol Œil, son œil bleu tourbillonnant rapidement. «Je dirais que c'est un peu… \emph{paranoïaque}.»

\emph{Vous ne connaissez pas le Professeur Quirrell}, ne répondit pas Harry. «C'est une théorie simple à vérifier, dit-il à voix haute. Regardez juste si le Professeur de Défense se souvient d'un fait concernant la guerre que le véritable David Monroe aurait connu. Bien que je suppose que s'il joue le rôle d'un David Monroe \emph{prétendant} être quelqu'un d'autre, il a une bonne excuse pour \emph{prétendre} qu'il prétend ne pas savoir de quoi vous parlez.

--- Un \emph{peu} paranoïaque, dit l'homme balafré d'une voix qui montait. \emph{Pas assez paranoïaque~! \shout{Vigilance constante}}~! Réfléchis, mon gars - et si le \emph{véritable} David Monroe n'était jamais revenu d'Albanie~?»

Il y eut un silence.

«Je vois… dit Harry.

--- Bien sûr que vous voyez, dit le Professeur McGonagall. Ne faites pas attention à moi, s'il vous plaît. Je vais juste rester tranquillement assise ici à devenir folle.

--- Dans ce métier, si tu survis, tu apprends qu'il y a trois genre de sorciers», dit Maugrey d'un ton lugubre~; Sa baguette n'était pointée vers personne, elle était légèrement inclinée vers le bas, mais il l'avait en main. Elle ne l'avait jamais quittée depuis qu'il était entré dans la pièce. «Il y a des mages noirs qui ont un nom. Il y en a qui ont deux noms. Et il y a ceux qui changent de nom comme toi et moi changeons de vêtements. J'ai vu 'Monroe' enchaîner trois Mangemorts comme il aurait cassé des brindilles. Il n'y a pas beaucoup de sorciers aussi bons que ça à quarante-cinq ans. Dumbledore, peut-être, mais pas beaucoup d'autres.

--- Peut-être est-ce vrai, dit le maître des potions depuis l'endroit où il se tapissait. Mais et alors, Maugrey~? Quelle que soit son identité, Monroe était certainement l'ennemi du Seigneur des Ténèbres. J'ai entendu des Mangemorts maudire son nom même après qu'ils l'ont cru mort. Ils le craignaient beaucoup.

--- Je saurais me montrer satisfaite, dit sagement le professeur McGonagall, que l'on dise cela d'un professeur de Défense.»

Maugrey fit un demi-tour pour lui lancer un regard furieux.

«Et où diable était 'Monroe' toutes ces années où il avait disparu, hein~? Peut-être qu'il pensait pouvoir se forger un nom en s'opposant à Voldie et qu'il a disparu quand il a découvert qu'il avait tort. Mais alors pourquoi revenir \emph{maintenant}, hein~? Quel est son \emph{nouveau} plan~?

--- Il, ah… s'aventura Harry. Il \emph{dit} qu'il a toujours voulu être un grand professeur de Défense parce que tous les meilleurs mages combattants ont enseigné à Poudlard. Il \emph{est} plus ou moins un professeur de Défense incroyablement génial… je veux dire que s'il voulait juste maintenir les apparences, il pourrait s'en sortir en bâclant \emph{beaucoup plus} le travail…»

Le professeur McGonagall hochait la tête avec conviction.

«Naïf, dit Maugrey d'un ton catégorique. J'imagine qu'aucun de vous ne s'est demandé si votre professeur de Défense n'avait pas organisé l'élimination de toute la maison Monroe~?

--- \emph{Quoi~?} s'écria le professeur McGonagall.

--- Notre sorcier mystère entend parler d'un gamin disparu d'une Très Ancienne Maison d'Angleterre, dit Maugrey. Il prend la place de 'David Monroe' mais reste à l'écart de la véritable famille Monroe. Mais la Maison doit bien finir par remarquer que quelque chose cloche. Donc cet imposteur pousse d'une façon ou d'une autre Voldie à tous les éliminer - peut-être en lui révélant le mot de passe qu'ils lui avaient donné pour leurs systèmes de sécurité - et il se retrouve Lord au Magenmagot~!»

Il semblait y avoir un combat dans l'esprit de Harry entre Poufsouffle Un, qui n'avait jamais fait confiance au professeur de Défense~; et Poufsouffle Deux, qui était bien trop loyal envers le professeur Quirrell pour croire à une chose pareille juste parce que Maugrey l'avait dite.

\emph{Cela dit}, c'est \emph{plutôt évident}, fit remarquer sa partie Serpentard. \emph{Je veux dire, est-ce que tu crois vraiment que des circonstances naturelles mèneraient n'importe qui à être le dernier descendant d'une Très Ancienne Maison ET à ce que Lord Voldemort ait tué toute sa famille ET qu'il doive venger son maître d'arts martiaux~? Je dirais plutôt qu'il a été trop loin dans la mise en place de sa nouvelle identité de héros fantastique idéal. Ce genre de chose n'arrive pas dans la vraie vie.}

\emph{Venu d'un orphelin ayant grandi dans l'ignorance de son héritage…} commenta le critique interne de Harry. \emph{Avec une prophétie le concernant. Tu sais, je ne pense pas avoir jamais lu d'histoire au sujet de deux héros partageant le même destin et en compétition pour savoir qui serait suffisamment cliché pour pouvoir s'occuper du méchant…}

\emph{Oui}, répondit le Harry principal par-dessus le vroom lointain en arrière-plan, \emph{nous vivons une existence très difficile et VOUS NE M'AIDEZ PAS.}

\emph{Il n'y a qu'une seule chose qu'on puisse faire, à ce stade}, dit Serdaigle. \emph{Et nous savons tous ce que c'est, alors pourquoi argumenter~?}

\emph{Mais}, répondit Harry, \emph{comment} teste\emph{-t-on expérimentalement si le professeur Quirrell est oui ou non le David Monroe original~? Je veux dire, quel genre de chose observable se comporte différemment selon qu'il soit le véritable David Monroe ou un imposteur~?}

«Qu'est-ce que tu veux que j'y fasse, Maugrey~? se plaignait le professeur McGonagall. Je ne peux pas…

--- Tu peux, dit l'homme balafré en la regardant avec férocité. Renvoie juste ce satané professeur de Défense.

--- Tu dis ça \emph{chaque} année, dit-elle.

--- Oui, et j'ai toujours raison~!

--- Vigilance constante ou pas, Alastor, les élèves doivent apprendre~!»

Maugrey renifla.

«Bah~! Je le jure, la malédiction empire chaque année à mesure que tu deviens de plus en plus réticente à te débarrasser d'eux. Ton précieux professeur Quirrell devrait \emph{être} Grindelwald sous couverture pour se faire renvoyer~!

--- C'est lui~? ne put s'empêcher de demander Harry. Je veux dire, est-ce qu'il pourrait \emph{vraiment} être…

--- Je vérifie la cellule de Grindie tous les deux mois, dit Maugrey. Il était là en mars.

--- La personne dans la cellule pourrait-elle être un sosie~?

--- J'administre un test sanguin pour vérifier son identité, petit.

--- Où gardez-vous le sang que vous utilisez pour référence~?

--- Dans un endroit sûr.» Quelque chose qui ressemblait à un sourire distendait les lèvres balafrées. «As-tu pensé au poste d'Auror après tes études~?

--- Alastor, dit le professeur McGonagall avec réticence. Le professeur de Défense \emph{a} un problème de santé. J'imagine que vous pourriez dire que c'est suspect en soi… mais il n'est en rien certain que c'est quelque méfait de sa part qui nous empêchera de renouveler son contrat.

--- Oui, ses petites siestes, dit sombrement Maugrey. Amélia pense qu'il a croisé la route d'une malédiction de haut niveau. Ça \emph{m'a} plutôt l'air un rituel noir qui a mal tourné~!

--- Tu n'as aucune preuve que c'est vrai~! dit le professeur McGonagall.

--- Cet homme pourrait aussi bien porter un signe disant 'Mage Noir' en lettres vertes lumineuses au-dessus de sa tête.

--- Ah…» dit Harry. Le moment ne semblait pas particulièrement bien choisi pour demander ce que M. Maugrey pensait du point de vue selon lequel tous les rituels sacrificiels n'étaient pas mauvais. «Excusez-moi, mais vous avez dit plus tôt que le professeur Quirrell - je veux dire l'ancien David Monroe - je veux dire celui des années soixante-dix - bref, vous avez dit que cette personne utilisait le sortilège de la mort. Qu'est-ce que ça veut dire~? Doit-on être un mage noir pour l'utiliser~?»

Maugrey secoua la tête.

«Je l'ai utilisé. Tout ce qu'il faut avoir, c'est du pouvoir et une certaine \emph{humeur}.» Les lèvres grimaçantes révélaient des dents. «La première fois que je l'ai lancé, c'était contre un sorcier nommé Gerald Grice, et tu pourras me demander ce que \emph{lui} avait fait après tes études à Poudlard.

--- Mais pourquoi est-il Impardonnable alors~? dit Harry. Je veux dire, un sortilège de coupure peut aussi tuer quelqu'un. Alors pourquoi est-ce qu'il vaut mieux utiliser Reducto plutôt qu'Avada Kedav -

--- Ferme-la~! dit brusquement Maugrey. Quelqu'un pourrait mal le prendre en t'entendant prononcer cette incantation. Tu as \emph{l'air} trop jeune pour pouvoir le lancer, mais un Polynectar est toujours possible. Et pour répondre à ta question, gamin, il y a deux raisons pour lesquelles ce sortilège est sur la liste noire des listes noires. La première, c'est que le sortilège de la mort frappe directement l'âme et qu'il ne s'arrêtera pas avant d'en avoir touché une. Il traversera les boucliers. Il traversera les \emph{murs}. Il y a une \emph{raison} pour laquelle même les Aurors n'avaient pas le droit de l'utiliser contre les Mangemorts avant la loi Monroe.

--- Ah, dit Harry. Ça semble effectivement être une excellente raison d'interdire…

--- Je n'ai pas fini, petit. La seconde raison est que le sortilège de la mort ne requiert pas \emph{seulement} un bon coup de magie. Il faut le \emph{vouloir}. Il faut \emph{vouloir} que la personne meure, et pas seulement pour le plus grand bien. Tuer Grice n'a pas ramené Blair Roche, Nathan Rehfuss ou David Capito. Ce n'était pas par souci de justice ou pour l'empêcher d'agir à nouveau. \emph{Je voulais qu'il soit mort}. Tu comprends maintenant, mon gars~? T'as pas besoin d'être un mage noir pour utiliser ce sortilège - mais tu peux pas être Albus Dumbledore non plus. Et si on t'arrête parce que tu l'as utilisé pour tuer, tu peux pas te défendre.

--- Je… vois», murmura le Survivant. \emph{On ne peut vouloir que la personne meurt pour des raisons instrumentales, dans la perspective de quelque conséquence positive future, on ne peut pas le lancer si on pense que c'est un mal nécessaire, il faut vraiment vouloir que la personne meure pour qu'elle soit morte, que ce soit une valeur terminale dans sa fonction d'utilité}. «Une préférence pour la mort et contre la vie, magiquement incarnée, frappant au niveau de la vie elle-même… ça a effectivement l'air d'être un sortilège difficile à arrêter.

--- Pas difficile, lâcha Maugrey. \emph{Impossible.}»

Harry hocha gravement la tête.

«Mais David Monroe - ou qui que ce soit d'autre - a utilisé le sortilège de la mort contre plusieurs Mangemorts \emph{avant même} qu'ils n'exterminent sa famille. Est-ce que ça veut dire qu'il devait déjà les haïr~? Que l'histoire des arts martiaux est probablement vraie~?»

Maugrey secoua légèrement la tête.

«L'une des sombres vérités du sortilège de la mort, petit, c'est qu'une fois qu'on l'a lancé une fois, on n'a pas besoin de beaucoup de haine pour le faire à nouveau.

--- Il endommage l'esprit~?»

Maugrey secoua la tête une fois de plus.

«Non. C'est tuer qui fait ça. Le meurtre déchire l'âme - mais un sortilège de coupure ferait le même effet. Ce n'est pas que le sortilège de la mort brise l'âme. C'est juste qu'il faut en avoir une brisée pour le lancer.» Si le visage balafré avait une expression triste, elle était imperceptible. «Mais cela ne nous dit pas grand chose sur Monroe. Ceux comme Dumbledore, qui n'arriveront jamais à lancer le sortilège de leur vie parce qu'ils ne se brisent pas quoi qu'il arrive - ce sont eux qui sont rares, très rares. Il suffit d'une petite fissure.»

Il y avait une étrange sensation de lourdeur dans la poitrine de Harry. Il s'était demandé exactement ce que signifiait le fait que Lily Potter avait essayé de lancer le sortilège de la mort sur Lord Voldemort dans son dernier souffle. Mais c'était sûrement pardonnable, sûrement \emph{juste} et \emph{bien} qu'une mère haïsse le mage noir venu tuer son bébé et se moquer d'elle parce qu'elle ne pouvait pas l'arrêter. Il y aurait eu quelque chose d'anormal chez le parent \emph{incapable} de lancer Avada Kedavra dans cette situation. Et aucun autre sortilège n'aurait pu traverser les boucliers du mage noir~; si c'était là le seul moyen de sauver son bébé, il fallait au moins \emph{essayer} de suffisamment haïr le Seigneur des Ténèbres pour désirer le voir mort.

\emph{Il suffit d'une petite fissure…}

«Assez, dit le professeur McGonagall. Que veux-tu que nous fassions~?»

Le sourire de Maugrey apparut. «Débarrassez-vous du professeur de Défense et voyez si tous vos problèmes se dissipent mystérieusement. Je vous parie un Gallion que oui.»

Le professeur McGonagall sembla ressentir une douleur. 

«Alastor… mais… pourras-\emph{tu} donner les cours, si…

--- Ha~! dit Maugrey. Si jamais je réponds oui, fais-moi un test de Polynectar, parce que ce ne sera pas moi.

--- Je ferai un test expérimental», dit Harry. Puis, alors que tout le monde le regardait, «je poserai au professeur Quirrell une question dont seul le véritable David Monroe connaîtrait la réponse - par exemple, qui d'autre était dans la classe de Serpentard en 1945, ou quelque chose comme ça - de préférence sans que ce soit trop évident. Ce ne sera pas une preuve concluante, il pourrait avoir étudié son rôle, mais ça pourrait toujours être un indice. Mais quand même, M. Maugrey, même si le professeur Quirrell n'est pas le Monroe original, je ne suis pas sûr que se débarrasser de lui soit gratuit. Il a sauvé ma vie deux fois…

--- \emph{Quoi~?} s'écria Maugrey. Quand~? Comment~?

--- Une fois il a assommé un groupe de sorcière qui m'attiraient vers le sol, une fois il a compris que le Détraqueur m'aspirait à travers ma baguette. Et si ce n'est \emph{pas} le professeur Quirrell qui a piégé Drago en premier lieu, alors il lui a sauvé la vie et les choses iraient beaucoup plus mal s'il ne l'avait pas fait. Si le professeur de Défense n'est pas derrière tout ça - alors il n'est \emph{pas} quelqu'un qu'on peut se permettre d'éjecter.»

Le professeur McGonagall hocha fermement la tête.

\latersection{Hypothèse~: Severus Rogue\\
(8 avril 1992, 19h03)}

Harry et le professeur McGonagall se tenaient à présent sur les escaliers à la lente rotation, une rotation qui ne descendait pas~; ou du moins \emph{un} Harry se tenait sur ces escaliers - ses trois autres lui avaient été laissés derrière, dans le bureau du directeur.

«Puis-je vous poser une question personnelle~?» dit Harry, lorsqu'il pensa qu'ils étaient assez loin pour ne pas être entendus.

--- Oui, dit le professeur McGonagall sans tout à fait soupirer. Bien que j'espère que vous comprenez que je ne peux rien \emph{faire} qui entre en conflit avec mon devoir de…

--- Oui, dit Harry, c'est exactement à ce sujet que je veux vous interroger. Face au Magenmagot, quand Lucius Malfoy disait que Hermione ne faisait pas partie de la maison Potter et qu'il n'accepterait pas l'argent, vous avez dit à Hermione de faire un serment. Je veux savoir, si quelque chose comme ça devait se produire à nouveau, si votre devoir est d'abord envers l'élève de Poudlard, Hermione Granger, ou envers le chef de l'Ordre du Phénix, Albus Dumbledore.»

Le professeur McGonagall donnait l'impression d'avoir été frappée en plein visage par une casserole en acier quelques minutes plus tôt, qu'on venait de lui dire que ça allait se produire à nouveau et qu'elle ne devait pas bouger.

Harry tressaillit lui même un peu. Un jour, il faudrait qu'il prenne l'habitude de ne \emph{pas} formuler les choses de la façon la plus abrupte possible.

Les murs tournèrent alors autour d'eux, en dessous d'eux, et, mystérieusement, ils descendirent.

«Oh, M. Potter, dit le professeur McGonagall avec une lente expiration. Je… \emph{j'aimerais} que vous ne me posiez pas de telles questions… oh, Harry, je ne réfléchissais pas alors, pas du tout. J'ai juste vu une chance d'aider Mlle Granger et… après tout, j'ai \emph{été} répartie à Gryffondor.

--- Vous avez une chance de réfléchir maintenant», dit Harry. Il le disait mal, mais il devait \emph{quand même} le dire, parce que - «Je ne vous demande pas d'être loyale envers \emph{moi}. Mais si vous savez - si vous \emph{êtes} certaine - de ce que vous ferez si un choix se dessine à nouveau entre un élève de Poudlard innocent et l'Ordre du Phénix…»

Mais le professeur McGonagall secoua la tête. «Je ne suis \emph{pas} certaine, chuchota le professeur de métamorphose. Je ne sais même pas si le choix que j'ai fait était le bon. Je suis désolé. Je ne peux pas prendre une décision aussi atroce~!

--- Mais vous ferez \emph{quelque chose} si ça se produit à nouveau, dit Harry. L'indécision est aussi un choix. Vous n'arrivez juste pas à \emph{imaginer} avoir à prendre une décision sur le moment~?

--- Non», dit le professeur McGonagall d'un ton un peu plus affirmé~; et Harry comprit qu'il lui avait accidentellement offert une échappatoire. Les mots du professeur confirmèrent ses peurs~: «Un choix aussi terrible, M. Potter… je pense que je ne devrais pas le faire avant d'y être obligée.»

Harry eut un soupir intérieur. Il songea qu'il n'avait aucun droit d'attendre du professeur McGonagall qu'elle dise quoi que ce soit d'autre. Dans un dilemme moral où l'on perdait forcément quelque chose, un choix était toujours \emph{désagréable}, et l'on pouvait donc se protéger temporairement d'une petite douleur mentale en refusant de décider. Au prix de ne rien pouvoir planifier à l'avance et au prix de subir un énorme biais d'inaction, ou d'attendre jusqu'à ce qu'il soit trop tard… mais on ne pouvait pas demander à une sorcière de savoir tout cela.

«Très bien», dit Harry.

Même si ce n'était pas bien du tout, vraiment pas. Dumbledore voudrait peut-être que cette dette soit effacée, le professeur Quirrell voudrait aussi que Harry ne soit pas endetté. Et si le professeur de Défense \emph{était} David Monroe ou pouvait \emph{sembler} l'être de façon convaincante, alors Lord Voldemort n'avait techniquement pas \emph{exterminé} la maison Monroe. Auquel cas quelqu'un pourrait faire passer un acte au Magenmagot révoquant la nobilité qui avait été accordée à la maison Potter pour avoir vengé la Très Ancienne Maison Monroe.

Auquel cas le serment d'allégeance de Hermione envers une maison noble serait nul et non avenu.

Ou peut-être pas. Harry ignorait toute des détails légaux, en particulier si la maison Potter \emph{récupérerait} l'argent si quelqu'un parvenait à envoyer Hermione à Azkaban. D'un point de vue légal, ce n'était pas parce qu'on perdait quelque chose que le paiement devait être remboursé. Harry n'en était pas certain et il n'osait pas poser la question à un juriste magique…

… il aurait été agréable de pouvoir compter sur le fait qu'au moins un adulte prendrait le parti de Hermione plutôt que celui de Dumbledore, si un problème de ce genre menaçait de se produire.

Les escaliers cessèrent de tourner, et ils furent devant les dos des grandes gargouilles de pierre, qui se déplacèrent en grondant, révélant le couloir.

Harry sortit…

Une main saisit son épaule.

«M. Potter, dit le professeur McGonagall d'une voix basse, pourquoi m'avez-vous dit de garder un œil sur le professeur Rogue~?»

Harry se tourna à nouveau.

«Vous m'avez dit de le surveiller et de voir s'il changeait, dit le professeur McGonagall avec une note d'urgence. \emph{Pourquoi} avez-vous dit ça, M. Potter~?»

Harry mit alors un moment à se souvenir, à se remémorer la raison pour laquelle il \emph{avait} dit cela. Harry et Neville avaient secouru Lesath Lestrange de brutes, puis Harry s'était alors confronté à Severus dans le couloir et, du moins selon le maître des potions lui-même, avait 'failli mourir'…

«J'ai appris quelque chose qui m'a inquiété, dit Harry au bout d'un moment. De la part de quelqu'un qui m'a fait promettre de ne le révéler à personne d'autre.» Severus avait fait promettre à Harry que leur conversation n'atteindrait les oreilles de personne et Harry était toujours engagé par cette promesse.

«\emph{Monsieur} Potter…» commença le professeur McGonagall, puis elle expira et le sursaut de sévérité disparut aussi vite qu'il était venu. «Non, rien. Si vous ne pouvez pas le dire, vous ne pouvez pas le dire.

--- Pourquoi me posez-\emph{vous} la question~?» dit Harry.

Le professeur McGonagall sembla hésiter…

«Très bien, laissez-moi être plus clair», dit Harry. Maintenant que le professeur Quirrell le \emph{lui} avait fait à lui à plusieurs fois, Harry commençait à prendre le coup de main. «Quels sont les changements que vous avez \emph{déjà} observé chez le professeur Rogue et que vous hésitez à mentionner~?

--- Harry… dit le professeur de métamorphose avant de fermer la bouche.

--- Il est clair que je sais une chose que vous ignorez, dit gentiment Harry. Vous voyez, c'est pour ça qu'on ne peut pas toujours repousser les décisions face à d'horribles dilemmes moraux.»

Le professeur McGonagall ferma les yeux, prit une profonde inspiration, pinça l'arête de son nez et serra plusieurs fois.

«Très bien, dit-elle. C'est subtil… mais inquiétant. Comment dire ça… M. Potter, avez-vous lu de nombreux livres que les jeunes enfants ne sont pas censés lire~?

--- Je les ai \emph{tous} lus.

--- Évidemment. Eh bien… je ne le comprends pas tout à fait moi-même, mais depuis que Severus a commencé à travailler pour cette école, à déambuler dans cette cape horriblement tachée, il y a eu un \emph{certain genre de fille} qui l'a observé avec des yeux pleins de convoitise…

--- Vous dites ça comme si ça n'était pas bien~! dit Harry. Enfin, s'il y a une chose que \emph{j'ai} comprise en lisant ces livres, c'est qu'on n'est pas censé remettre en question les préférences des gens.»

Le professeur McGonagall eut un regard \emph{très} étrange en direction de Harry.

«Je veux dire, continua Harry, d'après ce que j'ai lu, il y a environ 10~\% de chances que \emph{je} trouve le professeur Rogue attrayant quand je serai un peu plus vieux, et ce qui compte c'est juste que je m'accepte quel que soit…

--- \emph{Quoi qu'il en soit, M. Potter}, Severus a toujours été totalement indifférent aux regards de ces jeunes filles. Mais à présent…» le professeur McGonagall sembla se rendre compte de quelque chose et ses mains s'élevèrent comme en signe d'avertissement, «ne vous méprenez pas, je vous en prie, le professeur Rogue n'a \emph{certainement} abusé d'aucune jeune sorcière~! Absolument pas~! Il n'a même jamais ne serait-ce que souri à l'une d'elles, du moins je n'ai jamais entendu dire une chose pareille. Il a ordonné aux jeunes filles de cesser de le fixer avec des yeux ronds. Et si elles le regardent quand même, il détourne le regard. Cela, je l'ai vu de mes propres yeux.

--- Euh… dit Harry. Désolé mais ce n'est pas juste parce que j'ai \emph{lu} ces livres que je les ai compris. Qu'est-ce que tout ça veut \emph{dire}~?

--- Qu'il le \emph{remarque}, dit le professeur McGonagall d'un ton bas. C'est subtil, mais maintenant que je l'ai vu, j'en suis certaine. Et \emph{cela} signifie… j'ai bien peur… que le lien qui rapprochait Severus de la cause d'Albus… peut s'être affaibli, ou s'être même brisé.»

2 + 2 = …

«\emph{Rogue et Dumbledore}~?» Puis Harry entendit les mots qui venaient de sortir de sa bouche et ajouta avec hâte~: «Non pas que ça soit mal…

--- \emph{Non~!} dit le professeur McGonagall. Oh, par pitié… je ne peux pas vous expliquer, M. Potter~!»

L'inévitable conclusion lui apparut enfin.

\emph{Il était} encore \emph{amoureux de ma mère~?}

Cela lui sembla se situer quelque part entre le magnifiquement triste et le pathétique pendant cinq secondes avant que \emph{l'autre} inévitable conclusion ne lui apparaisse enfin.

\emph{Bien sûr, c'était avant que je lui donne mes bons conseils en matière de relations amoureuses.}

«Je vois», dit Harry avec précaution quelques instants plus tard. Parfois, dire 'Oups' était loin de faire justice à la situation. «Vous avez raison, ce n'est pas bon signe.»

Le professeur McGonagall mit ses deux mains sur son visage. «Quoi que vous pensiez maintenant, dit-elle d'une voix légèrement étouffée, je vous assure que c'est \emph{aussi} faux et je ne veux jamais l'entendre.

--- Donc… dit Harry. Si, comme vous dites, le lien qui unissait le professeur Rogue au directeur \emph{s'est} brisé… alors que fera-t-il~?»

Il y eut un long silence

\later

\emph{Alors que fera-t-il~?}

Minerva abaissa ses mains et fixa le visa du Survivant. Une simple question n'aurait pas dû lui causer tant de désarroi. Elle connaissait Severus depuis des années~; ils s'étaient rapprochés, d'une étrange façon, à travers la prophétie qu'ils avaient tous deux entendue. Bien que Minerva soupçonnait, d'après sa connaissances des règles prophétiques, qu'elle n'avait fait que la \emph{surprendre}. Ça avaient été les actes de Severus qui avaient accompli la prophétie. Et la culpabilité, le chagrin qui avait découlé de ce choix avaient tourmenté le maître des potions pendant des années. Elle ne pouvait imaginer Severus sans ces sentiments. Son esprit se vida lorsqu'elle essaya de le faire~; ses pensées devinrent un parchemin vierge.

Severus n'était \emph{certainement} plus l'homme qu'il avait jadis été, ce jeune homme en colère et terriblement idiot qui avait porté la prophétie à Voldemort en échange de son admission au sein des Mangemorts. Elle l'avait connu pendant des années, et Severus n'était certainement plus cet homme…

Le connaissait-elle vraiment~?

Est-ce que \emph{quelqu'un} avait déjà vu le véritable Severus Rogue~?

\later

«Je ne sais pas, dit enfin le professeur McGonagall. Je ne sais vraiment pas. Je ne peux même pas l'imaginer. Savez-\emph{vous} quoi que ce soit à ce sujet, M. Potter~?

--- Euh… dit Harry. Je pense pouvoir dire que les indices dont je dispose pointent dans la même direction que les vôtres. Je veux dire qu'ils augmentent la probabilité que le professeur Rogue n'est plus amoureux de ma mère.»

Le professeur McGonagall ferma les yeux.

«J'abandonne.

--- Je ne sais pas s'il a fait quoi que ce soit d'autre de mal, cela dit, ajouta Harry. Je suppose que le directeur vous a donné l'autorisation de me parler de ça~?»

Le professeur McGonagall détourna les yeux de Harry et regarda le mur.

«S'il te plaît Harry, non.

--- Très bien», dit Harry, et il pivota et s'éloigna rapidement dans les couloir, tout en entendant le pas plus lent du professeur McGonagall derrière lui et le grondement des gargouilles qui se remettaient en place.

\later

C'est le surlendemain, pendant le cours de potions, que la \emph{potion de résistance froide} de Harry entra en ébullition et déborda de son chaudron avec une mousse verdâtre et une odeur nauséabonde, et que le professeur Rogue, d'un air plus résigné que dégoûté, dit à Harry de rester après le cours. Harry avait ses propres soupçons concernant cette affaire et dès que la classe fut libérée - Hermione, comme d'habitude depuis quelques jours, fut la première à fuir la salle - la porte se referma d'un coup et se verrouilla derrière les élèves qui s'éloignaient.

«Je vous prie de m'excuser d'avoir gâché votre potion, M. Potter», dit doucement Severus Rogue. Il y avait sur son visage l'étrange regard triste que Harry avait vu chez lui quelques temps auparavant, dans un couloir. «Vos notes n'en pâtiront pas. Asseyez-vous, s'il vous plaît.»

Harry se rassit à son bureau et tua le temps en continuant de frotter la tache verte sur la surface de bois tandis que le maître des potions incantait quelques sortilèges d'intimité.

Lorsqu'il eut fini, il parla à nouveau. «Je ne… je ne sais pas comment aborder ce sujet, M. Potter, donc je me contenterai de le dire… face au Détraqueur, vous avez retrouvé le souvenir de la nuit où vos parents sont morts~?»

Harry hocha silencieusement la tête.

«Je… je sais que ce ne doit pas être un souvenir agréable mais… si vous pouviez me dire ce qui s'est passé~?

--- Pourquoi~?» demanda Harry. Sa voix était solennelle et certainement pas \emph{moqueuse} face au regard implorant que Harry ne se serait jamais attendu à voir chez cette personne. «Je ne pense pas que ce serait plaisant pour vous à entendre, professeur.»

La voix du maître des potions n'était presque plus qu'un murmure. «Je l'ai imaginé, toutes les nuits, ces dix dernières années.»

\emph{Tu sais}, dit le côté Serpentard de Harry, \emph{si sa loyauté fondé sur une culpabilité vacille déjà, ça n'est peut-être pas une si bonne idée que ça de lui offrir une chance de tourner la page…}

\emph{Tais-toi. Rejeté.}

Ce n'était pas une requête que Harry était \emph{vraiment} capable de refuser. Il accepta une suggestion de son côté Serpentard mais ce fut tout.

«Me direz-vous \emph{exactement} comment vous avez appris la prophétie~? dit Harry. Je suis désolé de transformer ça en un échange, je vous le \emph{dirai} après, c'est seulement que ça pourrait vraiment être important…

--- Il y a peu à dire. J'étais venu pour un entretien avec la directrice adjointe pour le poste de maître des potions et j'attendais donc devant la Tête de Sanglier le jour où la candidate qui me précédait, Sibylle Trelawney, venait pour le poste de professeur de Divination. Dès que Trelawney eut fini de prononcer ces mots, je fuis, abandonnant toute chance de professorat à Poudlard, et me présentais au Seigneur des Ténèbres.» Le visage du maître des potions était tiré, tendu. «Je ne me suis même pas arrêté pour me demander pourquoi cette énigme m'avait été adressée avant de la vendre à un autre.

--- Un \emph{entretien d'embauche}~? dit Harry. Vous et le professeur Trelawney aviez tous les deux candidaté et le professeur McGonagall faisait passer l'entretien~? Cela ressemble… à une assez grande coïncidence…

--- Les voyants sont les pions du temps, M. Potter. Les coïncidences sont indignes d'eux et ils les surplombent. J'étais celui censé entendre cette prophétie et devenir son pion. La présence de Minerva n'eut aucun impact sur ses conséquences. Il n'y a pas eu de sortilège de mémoire comme vous l'avez supposé, je ne sais pas pourquoi vous avez pensé à ça, mais il n'y en a pas eu, il n'y aurait pas pu en avoir. La voix d'un voyant a une certaine qualité, une énigme que même la Légilimancie ne possède pas~; comment pourrait-elle en imprégner un faux souvenir~? Pensez-vous que le Seigneur des Ténèbres aurait cru mes simples paroles~? Il s'est saisit de mon esprit, il y comprit la mystification, même s'il ne put comprendre le mystère, et il sut que la prophétie était vraie. Le Seigneur des Ténèbres aurait alors pu me tuer, ayant pris ce qu'il désirait - j'avais bel et bien été idiot d'aller le voir - mais il vit quelque chose en moi, quelque chose que j'ignore, et m'accepta parmi les Mangemorts, quoique selon ses propres conditions plutôt que selon les miennes. C'est ainsi que je l'ai provoqué, que j'ai tout provoqué, du début à la fin, moi seul.» La voix de Severus était devenue assez rauque et son visage affichait une douleur nue. «Maintenant, dites-moi, s'il vous plaît, comment Lily est-elle morte~?»

Harry déglutit deux fois et commença à raconter.

«James Potter a crié à Lily de s'enfuir avec moi, qu'il retiendrait Vous-Savez-Qui.

--- Vous-Savez-Qui a dit…» Harry s'arrêta, les frissons montaient le long de sa peau, ses muscles se contractaient comme s'ils se préparaient à une crise de nerfs. Le souvenir revenait à présent avec plus de force, accompagné de froid et de ténèbres. «Il a utilisé… le sortilège de la mort… puis il a monté les escaliers, je crois qu'il doit avoir volé, je ne me souviens d'aucun pas dans les escaliers ou quoi que ce soit du genre… et puis ma mère a dit~: “Non, pas Harry, s'il vous plaît, pas Harry~!” ou quelque chose comme ça. Et le Seigneur des Ténèbres - sa voix était tellement aiguë, comme de l'eau sifflant d'une bouilloire mais \emph{froide} - le Seigneur des Ténèbres a dit…»

\emph{Écarte-toi, femme~! Ce n'est pas pour toi que je suis venu, mais pour le garçon.}

Le mots étaient très clairs dans l'esprit de Harry.

«… il a dit à ma mère de s'écarter de son chemin, qu'il n'était là que pour \emph{moi}, et ma mère l'a supplié d'avoir pitié, et le Seigneur des Ténèbres a dit…»

\emph{Je te donne la rare chance de t'échapper.}

«… qu'il était généreux et lui donnait une chance de s'enfuir, mais qu'il ne prendrait pas la peine de se battre contre elle et que même si elle mourrait elle ne pourrait pas me sauver…» la voix de Harry était instable, «… et qu'elle devait donc s'écarter de son chemin. Et c'est alors que ma mère a supplié le Seigneur des Ténèbres de prendre sa vie plutôt que la mienne, et le Seigneur des Ténèbres… le Seigneur des Ténèbres lui a dit… et sa voix était plus grave cette fois, comme s'il abandonnait une attitude qu'il aurait endossée…»

\emph{Très bien, j'accepte le marché.}

«… il a dit qu'il acceptait son offre et qu'elle devait laisser tomber sa baguette afin qu'il puisse la tuer. Et alors il a attendu, il a juste attendu. Je… je ne sais pas ce que Lily Potter a pensé, ça n'avait eu aucun sens, ce qu'elle avait dit, ce n'était pas comme si le Seigneur des Ténèbres allait la tuer et juste \emph{partir} alors qu'il était venu ici pour moi. Lily Potter n'a rien dit et puis le Seigneur des Ténèbres a commencé à se moquer d'elle et c'était horrible et… et elle a fini par essayer la dernière chose à faire qui n'était ni m'abandonner ni laisser tomber et mourir. Je ne sais pas si elle aurait même pu le lancer, si le sortilège aurait fonctionné, mais à y réfléchir, elle se devait d'essayer. La dernière chose que ma mère a dit fut 'Avada K…' mais le Seigneur des Ténèbres a commencé son propre sortilège dès qu'elle a commencé à dire 'Av' et il l'a dit en moins d'une demi seconde et il y a eu un flash de lumière verte et alors… et alors… \emph{et alors…}

--- Cela suffit.»

Lentement, comme un corps remontant à la surface de l'eau, Harry revint de là où il avait été.

«Cela suffit, dit le maître des potions d'une voix rauque. Elle est morte… elle est morte sans souffrir, alors~? Le Seigneur des Ténèbres… ne lui a rien fait, avant qu'elle ne meure~?»

\emph{Elle est morte en pensant avoir échoué, en pensant que le Seigneur des Ténèbres allait ensuite tuer son bébé. C'est une souffrance.}

«Il… le Seigneur des Ténèbres ne l'a pas torturée… dit Harry. Si c'est ce que vous me demandez.»

Derrière Harry, la porte se déverrouilla et s'ouvrit grand.

Harry partit.

Nous étions le vendredi 10 avril 1992. 

%  LocalWords:  Shehla Borgin Burkes Karkaroff superpowered veldbeest Hua
%  LocalWords:  Varyabil Sherice Ngaserin Turnipseed Spooker Bandon Wagga
%  LocalWords:  righty Kettleburn Bathsheda vroop hah ing Pfah Ke
%  LocalWords:  Grice Kedav Rehfuss Capito Av
