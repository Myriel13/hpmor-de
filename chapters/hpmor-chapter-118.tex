\chapter{Après-coup, quelque chose à protéger~: Professeur Quirrell}

\lettrine{S}{ous} le ciel azur de ces funérailles, le soleil rayonnait sur l'herbe écossaise et faisait naître des éclats blancs dans les gouttes de rosée de certaines des feuilles lisses.

Harry avait refusé de donner un éloge funèbre. Pour la seconde fois. Le professeur Flitwick le lui avait déjà demandé plusieurs semaines auparavant, en mai, pour lui donner le temps d'écrire, avant que la parole ne devienne nécessaire~; et Harry avait aussi refusé.

La tâche incombait donc à un Gryffondor en sixième année, Oliver Habryka, qui était quatrième en points Quirrell, et qui avait aussi été général d'une armée. Le garçon de dix-sept ans était grand, pas particulièrement beau, et vêtu de robes noires. Plutôt qu'une cravate rouge, il en avait choisi une violette, comme le professeur Quirrell en avait parfois porté.

Et il improvisait. L'éloge prévu longtemps à l'avance avait été rejeté~; Oliver Habryka tenait encore le parchemin dans sa main gauche, mais il ne le regardait pas.

"Le professeur Quirrell était très malade," dit le garçon d'une voix hésitante, qui se perdait dans les murmures des autres élèves et se brisait parfois d'un sanglot. "Je pense qu'en pleine possession de ses moyens, le professeur Quirrell aurait pu se battre, que Vous-Savez-Qui ne l'aurait pas vaincu facilement, peut-être même pas du tout. Ils disent qu'en son temps, David Monroe était le seul à avoir jamais effrayé Vous-Savez-Qui. Mais…" la voix d'Oliver se brisa. "… Le professeur Quirrell n'était pas en pleine possession de ses moyens. Il était très malade. Il avait du mal à marcher sans assistance. Et il s'est dressé contre le Seigneur des Ténèbres. Seul."

Il y eut alors un silence, et certains élèves pleurèrent.

Oliver essuya ses larmes du revers de sa manche et parla à nouveau. "Nous ne savons pas ce qui s'est passé," dit Oliver. "J'imagine que le Seigneur des Ténèbres lui a rit au nez. Peut-être qu'il s'est moqué du professeur, du fait qu'il lui tienne tête sans pouvoir tenir debout. Eh bien \emph{il se marre moins, maintenant.}"

De violents hochements de tête chez les étudiants~; Harry en observa partout, de Gryffondor à Serpentard.

"Peut-être que le Seigneur des Ténèbres connaissait un moyen de soigner le professeur Quirrell~; après tout, il est revenu d'entre les morts. Peut-être qu'il a offert au professeur Quirrell de lui rendre sa vie s'il devenait son serviteur. Le professeur Quirrell a souri et a dit au Seigneur des Ténèbres que c'était l'heure de jouer à un jeu appelé Qui Est Le Plus Dangereux Sorcier Du Monde."

\emph{Quand on ne sait pas, on n'invente pas.} Mais Harry resta coi. C'était peut-être ce que Voldemort aurait essayé~; et c'était peut-être ce que le professeur Quirrell aurait répondu.

"Et ils ne nous disent pas tout," dit Oliver, "mais on peut deviner ce qui s'est passé ensuite. Nous savons tous que Hermione Granger, une des meilleures élèves du professeur, s'est fait tuer par un troll plus tôt cette l'année, et le Seigneur des Ténèbres devait être dans le coup, tout comme il l'a sûrement faite accuser avec le sortilège de refroidissement du sang. Le professeur Quirrell savait que le Seigneur des Ténèbres était derrière ça, alors il a volé le corps de Mlle Granger, il l'a préservé, il l'a gardé à l'abri…".

Une erreur pardonnable.

"Et puis le professeur Quirrell est allé se battre contre le Seigneur des Ténèbres. Il s'est fait tuer. Et Hermione Granger est revenue à la vie. Ils disent qu'elle est en vie maintenant, qu'elle va bien, peut-être mieux que bien. Le Seigneur des Ténèbres a essayé de l'attraper et tout ce qui reste de lui, ce sont ses robes brulées et ses mains autour de la gorge de Mlle Granger. Tout comme l'amour et le sacrifice de sa mère ont sauvé Harry Potter du sortilège de la Mort, le professeur Quirrell a décidé de partir, d'aller se battre… seul… contre le Seigneur des Ténèbres… ça a dû invoquer… l'esprit de Hermione Granger… d'entre les morts… ou d'ailleurs…" Oliver n'arrivait plus à parler.

"Ce n'était pas seulement ça," dit Harry depuis le premier rang, d'une voix elle aussi enrouée. Il \emph{devait} dire quelque chose avant que ça devienne n'importe quoi. Si ce n'était pas déjà le cas. "David Monroe était un sorcier puissant~; seul lui et moi savons à quel point il l'était. Je ne pense pas qu'on puisse ramener quelqu'un d'entre les morts en se sacrifiant. Personne ne devrait essayer cette méthode."

Une histoire si belle. Elle aurait dû être vraie. \emph{Elle aurait dû être vraie.}

"Je ne connais pas bien la personne qui se cachait derrière le professeur," dit Oliver Habryka après s'être repris. "Je sais que David Monroe n'était pas un homme heureux. Il n'a jamais pu créer de Patronus."

Des larmes s'amoncelaient à nouveau dans les yeux de Harry. C'était injuste, Voldemort avait tué tant de gens, il aurait dû mourir avec les siens, il ne méritait aucun traitement de faveur. La faiblesse de Harry n'était pas seule responsable~: les Horcruxes l'avaient empêché. Il ne \emph{pouvait pas} être tué directement. Alors Harry pouvait se l'admettre~: il était heureux, \emph{heureux} que le professeur Quirrell n'ait pas totalement disparu…

"Mais je… je sais," dit Oliver, des larmes scintillantes sur ses joues, "que, où qu'il soit, le professeur Quirrell est heureux, maintenant."

À la main gauche de Harry, une petite émeraude brillait sous le soleil de l'aube.

\emph{Pas au paradis, pas une étoile lointaine, pas un autre lieu~: une autre personne. Je vous montrerai, un jour je vous montrerai comment être heureux…}

Pour la première fois, le garçon jeta un regard au parchemin qu'il tenait en main. "Le professeur Quirrell," dit Oliver d'une voix devenue plus rapide, plus féroce, "était de loin le meilleur professeur de Magie de Bataille que Poudlard ait jamais eu. Peu importe les sortilèges que Salazar Serpentard connaissait, le professeur Quirrell était au moins deux fois meilleur. Le professeur Quirrell nous a dit au début de l'année que ce qu'il nous enseignait serait toujours notre socle en matière de Défense. Et c'est vrai. Ce le sera toujours. Nous le transmettrons aux nouveaux l'année prochaine, peu importe notre professeur. Les élèves plus âgés apprendront aux plus jeunes. C'est la réponse à la malédiction jetée sur le poste de professeur de Défense. Nous n'attendrons pas que l'autorité nous enseigne. Et nous nous assurerons que les enseignements du professeur Quirrell ne disparaissent jamais de Poudlard."

Harry regarda là où le directeur - non, là où la directrice McGonagall - était assise, et il la vit hocher la tête en silence, avec un air triste, sévère, et fier.

"Ils ne nous ont pas encore laissé voir Mlle Granger," dit Oliver. Sa voix trembla. "La Ressuscitée. Mais je penserai toujours au professeur de Défense en la regardant. Son sacrifice vit en elle, tout comme son savoir vit en nous." Oliver regarda Harry, puis regarda à nouveau le parchemin. "Au professeur Quirrell, le meilleur Serpentard à avoir jamais été, celui que tout Serpentard devrait être~! Pour le professeur Quirrell…

--- \emph{Hourra~! Hourra~! Hourra~!}"

Cette fois, pas un seul élève ne se tut. 

%  LocalWords:  Habryka
