\chapter{Examens}

\section{Le 4 juillet 1992.}

\lettrine{D}{aphné} Greengrass était dans la salle commune des Serpentard et écrivait une lettre à Dame sa mère (étonnamment intransigeante quant au partage de son pouvoir alors même qu'elle n'était \emph{même pas} à Poudlard) lorsqu'elle vit Drago Malfoy entrer par le portrait en chancelant, avec dans les bras au moins douze livres~; Vincent et Grégory étaient derrière lui et en portaient une autre douzaine. L'Auror qui accompagnait Malfoy glissa sa tête par le passage puis s'en fut-on ne sait où.

Drago regarda autour de lui et une idée brillante sembla soudain le frapper. Il s'avança, toujours chancelant, Vincent et Grégory derrière lui.

«~Est-ce que tu pourrais m'aider à les lire~?~» demanda Drago en s'approchant, le souffle un peu court.

«~Quoi.~» Il n'y avait plus de cours, seulement des examens, et depuis quand un \emph{Malfoy} demandait-il de l'aide à une \emph{Greengrass} pour ses devoirs~?

«~Voici, dit Drago d'un ton solennel, tous les livres empruntés à la bibliothèque par Granger entre le 1ᵉʳ et le 16 avril. Je pensais les parcourir au cas où l'un d'eux renfermerait un indice, mais je me suis dit que \emph{tu} devrais peut-être participer, vu que tu la connaissais mieux.~»

Daphné fixa les livres du regard. «~Le Général a lu tout ça en \emph{deux semaines}~?~» Une pointe de douleur traversa son cœur, mais elle la réprima.

«~Eh bien, je ne sais pas si elle les a tous finis~», dit Drago. Il éleva un index, comme pour appeler à la prudence. «~En fait, je ne sais pas si elle a lu un seul de ces livres, ni même si elle les a vraiment empruntés. En fait, tout ce qu'on a \emph{observé}, c'est que le registre de la bibliothèque dit qu'elle les a pris…~»

Daphné réprima un grognement. Malfoy parlait comme ça depuis plusieurs semaines. Il était clair que certains individus ne devraient jamais s'intéresser à des meurtres mystérieux, car l'effet que cela produisait sur leur cerveau était \emph{bizarre}.

«~M. Malfoy, je ne pourrais pas lire ces livres même si j'y passais tout l'été~;

--- Alors essayez de juste les parcourir, s'il vous plaît~? répondit Drago. Surtout si on y trouve, tu sais, d'étranges notes de sa main, ou un marque-page laissé à l'intérieur, ou…

--- Oui, j'ai vu ces pièces aussi, M. Malfoy, dit Daphné en levant les yeux au ciel. Mais on a des \emph{Aurors} pour ça maint…

--- \emph{Nous sommes perdus~!}~» glapit Millicent Bulstrode, jaillissant dans la salle commune Serpentard depuis les chambres du sous-sol.

Tout le monde s'interrompit et la regarda.

«~\emph{C'est le professeur Quirrell~!}~»

Une attention soudaine, comme si de vieilles querelles s'apprêtaient à être réglées.

«~Enfin, dit quelqu'un alors que Millicent reprenait son souffle. Il lui reste quoi, dix jours avant de mal tourner~?

--- Onze jours, dit l'élève de septième année qui prenait les paris.

--- \emph{Son état s'est soudainement amélioré et il va faire venir les élèves de première année pour leur examen final de Défense~! Un examen surprise~! Dans cinquante minutes~!}

--- Un examen final de défense~? dit Pansy choquée. Mais le professeur Quirrell ne donne pas d'examens.

--- L'examen final du \emph{ministère}~! glapit Millicent.

--- Mais le professeur Quirrell n'enseigne pas du tout le programme du ministère, fit remarquer Pansy.

Daphné fonçait déjà vers sa chambre, vers le manuel de Défense de première année qu'elle n'avait pas ouvert depuis septembre, un hurlement de jurons à l'esprit.~»

\later

Une table derrière elle, quelqu'un pleurait, les doux sanglots comme un fond sonore de désespoir qui emplissait la salle. Daphné regarda derrière elle en s'attendant à voir une Poufsouffle et en espérant que ce ne fut pas Hannah, mais elle fut initialement surprise (puis pas tant que ça, à y réfléchir) de constater que c'était un Serdaigle.

Devant eux se trouvaient les parchemins d'examen, retournés. Ils attendaient que la cloche sonne.

Cinquante minutes avaient été loin de suffire, mais c'était déjà ça, et Daphné avait maintenant honte de n'avoir pas envoyé des messagers prévenir Poufsouffle, Serdaigle et Gryffondor. Le système de points de maison avait été remis en vigueur trois jours plus tôt, début juin, mais le comité auxiliaire de protection spécialisée se devait de promouvoir l'unité des maisons.

Une autre Serdaigle, deux tables à sa gauche, se mit à son tour à pleurer. C'était Katherine Tung, de l'armée Dragon, qui, si Daphné se souvenait bien, avait un jour descendu sans ciller trois soldats Soleil.

Daphné s'était calmée après quelques minutes de lecture frénétique. Ce n'était qu'un examen, pas une \emph{mise à mort}~; et si presque tous les élèves de première année rendaient des parchemins quasi-vierges, personne n'aurait à avoir honte. Mais même sans partager leur opinion, Daphné pouvait comprendre que les Poufsouffle et les Serdaigle voient les choses autrement.

«~Il est maléfique, dit un Serdaigle d'une voix tremblante. Cent pour cent pur mage noir jusqu'à la moelle. Grindelwald ne ferait pas ça à des enfants. Il est pire que Vous-Savez-Qui.~»

Daphné regarda pensivement le professeur Quirrell, assis, penché d'un côté, mais le regard alerte, et l'espace d'un court instant elle crut voir le professeur de Défense sourire. Non, ça devait avoir été son imagination. Jamais le professeur Quirrell n'aurait pu entendre ça.

La cloche sonna.

Daphné retourna son parchemin.

Il était surmonté de tampons du ministère, du conseil d'administration de Poudlard et du département d'éducation magique~; ainsi que de runes détectrices de tricherie. En dessous, une ligne pour y inscrire son nom, et la liste des règles à suivre accompagnée d'une image de Lindsay Gagnon, directrice du département d'éducation magique, un doigt admonestant pointé vers le lecteur.

À mi-hauteur se trouvait la première question.

C'était~: \emph{Pourquoi est-il important que les enfants restent à distance des créatures étranges~?}

La salle se figea, stupéfaite.

Un élève se mit à rire, probablement Gryffondor. Le professeur Quirrell ne chercha pas à l'interrompre et le rire se propagea.

Personne ne parla, mais les élèves échangeaient des regards. Le rire se tut, et comme d'un accord muet ils regardèrent tous le professeur Quirrell, qui leur retournait un regard bienveillant.

Daphné se pencha sur son examen, avec au visage un sourire défiant dont Godric Gryffondor et Grindelwald auraient été fiers~; puis elle écrivit~: \emph{Parce que mon sortilège d'étourdissement, ma Très Ancienne Lame et mon Patronus ne marcheront pas à chaque fois}.

\later

Harry Potter retourna la dernière page de son examen de Défense.

Même Harry avait eu à réprimer un peu de nervosité, un petit vestige de son enfance, lorsqu'il avait lu la première vraie question ('Comment faire taire une anguille hurleuse~?'). Les cours du professeur Quirrell n'avaient pas abordé une seule fois les futilités certes étonnantes qu'un idiot avait cru devoir faire partie d'un “cours de Défense”. En principe, après avoir appris que l'examen aurait lieu, Harry aurait pu utiliser son Retourneur de Temps pour lire son livre de Défense~; mais cela aurait injustement décalé la courbe de notation pour les autres. Après avoir contemplé la question pendant quelques secondes, Harry avait écrit “sortilège de silence” et avait ajouté les instructions, au cas où l'employé du ministère chargé de le noter refuserait de croire que Harry connaissait ce sort.

Après qu'il eut décidé de répondre \emph{correctement} aux questions, tout était allé plus vite. La réponse la plus réaliste à plus de la moitié des questions était “sortilège d'étourdissement” et de nombreuses autres avaient des réponses optimales proches de “Se retourner et marcher dans l'autre sens” ou “Jeter le fromage à la poubelle et s'acheter de nouvelles chaussures.”
La dernière question de l'examen était~: “Que feriez-vous si vous pensiez qu'un serpent-mitaine était sous votre lit~?”. Harry se souvenait que la réponse officiellement approuvée par le ministère était de \emph{le dire à ses parents}. Le problème, c'est que c'était la première chose à laquelle Harry avait pensé, et c'était pour ça qu'il s'en souvenait.

Après un peu de réflexion, il écrivit~:

\emph{Cher examinateur du ministère~: j'ai bien peur que la bonne réponse ne soit secrète, mais soyez assuré qu'un serpent-mitaine ne me poserait pas plus de problèmes qu'un Troll des montagnes, qu'un Détraqueur ou que Vous-Savez-Qui. Veuillez dire à vos supérieurs que je trouve la réponse officielle injuste envers les nés-Moldu et que je compte bien voir cette erreur corrigée immédiatement sans que mon intervention directe ne soit nécessaire.}

\emph{Bien à vous,}

\emph{Le Survivant.}

Harry signa le dernier parchemin avait nombre de fioritures, le plaça sur la pile, posa sa plume et se redressa.

Il constata que le professeur Quirrell regardait vaguement vers lui, mais que sa tête s'était affalée d'un côté. Les autres élèves écrivaient encore. Certains d'entre eux pleuraient, mais ils écrivaient quand même. \emph{Continuer de combattre} était une autre leçon que le professeur Quirrell leur avait enseignée.

Infiniment plus tard, l'examen arriva à son terme. Une élève de septième année alla de table en table pour récupérer les parchemins à la place du professeur Quirrell.

Lorsque ce fut fait, le professeur Quirrell se redressa.

«~Mes jeunes élèves~», dit-il doucement. L'élève de septième avait sa baguette pointée vers la bouche du professeur de Défense, si bien que tous entendaient la voix du professeur aussi clairement que s'il avait été à côté d'eux. «~Je sais… que vous avez été nombreux à avoir peur… c'est une autre peur que d'être face à la baguette de l'ennemi… il vous faudra la conquérir, elle aussi. Je vous dirais donc ceci… maintenant. La coutume de Poudlard… est de donner les notes pendant la seconde semaine de juin. Je pense qu'ils peuvent faire une exception… pour moi.~» Le professeur de Défense eut son sourire sec habituel, mais il se tordit, comme si une grimace avait été réprimée. «~Je sais que vous êtes inquiets… que vous n'étiez pas préparés à l'examen… que mes cours n'en parlaient pas… et j'ai complètement oublié de mentionner… qu'il approchait… même si vous auriez dû savoir… qu'il finirait par arriver. Mais je viens de vérifier magiquement… les réponses que vous avez fournies à ce… capital examen… bien que seule la note du ministère est officielle… j'ai inscrit vos notes pour l'année en prenant compte de ce résultat… et je les ai écrites sur ces parchemins,~» le professeur Quirrell toucha une pile de parchemin située à côté de son bureau, «~ils vous seront maintenant distribués… un sortilège incroyable… non~?~»

Certains des élèves de Serdaigle semblaient indignés, mais la plupart avaient seulement l'air soulagés. Des Serpentard gloussaient. Harry aurait ri aussi, s'il n'avait autant souffert de voir le professeur Quirrell haleter ses mots.

L'élève de septième année à côté du professeur Quirrell leva sa baguette et prononça une incantation en pseudo-Latin. Les parchemins s'élevèrent et commencèrent à dériver ensemble avant de se séparer et de continuer jusqu'à chaque élève.

Harry attendit que le sien vienne à sa table et le déroula.

La note DA+ était inscrite, et elle signifiait Dépasse les Attentes. C'était la deuxième note possible, la plus haute étant Extraordinaire.

Dans un autre monde, un monde lointain et disparu, un petit garçon appelé Harry aurait hurlé d'indignation pour n'avoir reçu que la deuxième meilleure note. Ce Harry réfléchit en silence. Le professeur Quirrell cherchait à lui dire quelque chose, et ce n'était pas comme si la note exacte avait de l'importance. Voulait-il dire que Harry s'en était bien tiré mais qu'il n'avait pas totalement accompli son potentiel~? Ou la note signifiait-elle littéralement que Harry avait bien dépassé les attentes du professeur~?

«~Vous avez tous… réussi~», dit le professeur Quirrell tandis que les élèves découvraient leur note finale, et des soupirs de soulagement s'élevèrent des tables. Lavande Brown leva son parchemin tenu dans un poing serré en signe de triomphe. «~Tous les élèves du cours de magie de bataille ont réussi… sauf une.~»

De nombreuses élèves relevèrent les yeux, soudain terrifiées.

Harry demeura silencieux. Il avait immédiatement compris, et même s'il n'était pas d'accord, il savait que rien ni personne ne découragerait le professeur de dire ce qu'il avait à dire.

«~Vous tous ici… avez au moins reçu la note Acceptable. Neville Londubat… qui a passé l'examen depuis la maison Londubat… a reçu la note Extraordinaire. Mais l'autre élève qui n'est pas là… a reçu la note Affreux… car elle a échoué le seul examen d'importance… qu'elle ait passé cette année. Je lui aurais bien donné une note encore plus basse… mais ç'aurait été de mauvais goût.~»

La salle était très silencieuse, mais de nombreux élèves regardaient le professeur avec colère.

«~Vous pensez peut-être que la note Affreux… est injuste. Que Mlle Granger a fait face à un test… à laquelle ses leçons… ne l'avaient pas préparée. Qu'on ne lui avait pas dit… que l'examen viendrait ce jour-là.~»

Le professeur de Défense inspira avec difficulté.

«~Ainsi va le monde réel, dit le professeur Quirrell. Le seul véritable test… peut arriver n'importe quand… mieux vaut s'y préparer. Quant au reste d'entre vous… ceux qui ont reçu EE ou plus… vous avez reçu mes lettres de recommandation… pour certaines organisations par-delà les côtes de Grande-Bretagne… où votre entraînement pourra s'achever. Ils vous contacteront… quand vous serez assez âgés. Et souvenez-vous… à partir d'aujourd'hui… vous devrez vous entraîner… vous ne pourrez pas compter… sur les professeurs de Défense à venir. Votre première année de magie de bataille est terminée… vous pouvez disposer.~»

Le professeur Quirrell s'inclina, les yeux fermés, et sembla ne pas remarquer les babillements excités qui venaient de naître autour de lui.

La plupart des élèves finirent par partir, et un seul demeura, à une certaine distance du professeur de Défense.

Le professeur ouvrit les yeux.

Harry tenait le parchemin et son \emph{EE+}, toujours muet.

Le professeur de Défense sourit, et le sourire atteint jusqu'à ses yeux fatigués.

«~C'est la même note… que celle que j'ai reçue en première année.

--- Me… me… me…~», Harry ne parvint pas à faire sortir le mot \emph{merci} de sa gorge soudain nouée. Le professeur de défense inclina la tête, le regarda avec curiosité, et Harry s'inclina vivement avant de quitter la pièce.

Il restait neuf jours.
%  LocalWords:  aphne Tung Bogeysnake EE
