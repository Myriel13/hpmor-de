\partchapter{Travail de groupe}{II}

\section{Aftermaths:}

\lettrine{H}{arry} faisait les cents pas dans son bureau de général, qui se trouvait être un lieu idéal pour pratiquer cette activée, et Harry ne lui avait d'ailleurs découvert aucune autre utilité.

\emph{Comment~?}

\emph{Comment~?}

Hermione n'aurait pas dû remporter la bataille~! Pas du première coup, pas avec sa nature non-violente, devenir une commandante militaire géniale en plus de tout le reste, c'était trop, même pour \emph{elle}.

Avait-elle appris des notions de tactique dans un livre d'histoire militaire~? Mais ça n'avait pas été qu'une seule tactique, elle avait parfaitement disposé ses forces de façon à bloquer toute retraite, et elles avaient été mieux coordonnées que les siennes \emph{ou} que celles de Drago…

Le professeur Quirrell avait-il brisé sa promesse de ne pas l'aider~? Lui avait-il donné le journal du général Tacticus ou quelque chose du genre~?

Harry ratait quelque chose, quelque chose de très important, et son esprit tourbillonnait encore et encore et il ne pouvait toujours rien trouver.

Harry finit par soupirer. Tout cela ne le mènerait nulle part, et il lui fallait apprendre le sort de bris de bouclier auprès de Hermione ou de quelqu'un d'autre avant la prochaine bataille - le professeur Quirrell avait expliqué à Harry d'un ton amusé mais aussi lourd de mises en gardes implicites que "aucun objet magique mis à part ceux je vous donne" incluait la technologie Moldue, peu importe à quel point elle \emph{n'était pas magique}. Et puis Harry devait aussi trouver comment abattre M. Goyle la prochaine fois…

Les batailles rapportaient beaucoup de points Quirrell aux généraux et Harry devait vraiment s'y mettre s'il voulait gagner le vœu de Noël du professeur Quirrell.

\later

Dans sa chambre individuelle de Serpentard, Drago Malfoy fixait le vide comme si le mur devant son bureau avait été la chose la plus fascinante du monde.

\emph{Comment~?}

\emph{Comment~?}

Rétrospectivement, l'idée avait été évidente en termes de fourberie, mais Granger n'était pas \emph{censée} être fourbe~! Elle avait été trop Poufsouffle pour jeter un sort d'attaque simple~! Le professeur Quirrell l'avait-il conseillée en dépit de sa promesse ou…

Et Drago comprit enfin ce qu'il aurait dû faire bien plus tôt.

Ce que Harry Potter lui avait \emph{dit} de faire, ce qu'il l'avait \emph{entraîné} à faire, et Harry l'avait malgré cela prévenu que son cerveau aurait besoin de temps avant de se rendre compte que les méthodes s'appliquaient à la vie de tous les jours, et Drago ne l'avait pas \emph{compris} avant aujourd'hui. Il aurait pu éviter chacune de ses erreurs s'il avait juste \emph{appliqué} les choses que Harry lui avait déjà \emph{dites} -

Drago dit à voix haute~: "Je remarque que je suis confus."

\emph{Ta force en tant que rationaliste est d'être plus facilement rendu confus par la fiction que par la réalité.}

Drago était confus.

Donc l'une de ses croyances était une fiction.

Granger n'aurait pas dû être capable de faire ça.

Donc elle n'en avait probablement pas été capable.

\emph{Je promets de ne pas aider le général Granger à votre insu de quelque façon que ce soit.}

Dans un moment de compréhension soudaine et horrifiée, Drago fit voltiger des feuilles, fouillant le fatras qu'était son bureau jusqu'à ce qu'il trouve ce qu'il cherchait.

Et c'était là.

Au beau milieu de la liste des gens et des fournitures assignées à chacune des trois armées.

\emph{Maudit} professeur Quirrell~!

Drago l'avait \emph{lu} et malgré cela il ne l'avait pas \emph{vu} -

\later

Le soleil de l'après-midi se déversait dans le bureau du régiment du Soleil, illuminant le général Granger, assise dans sa chaise, nimbée d'une aura d'or.

"Dans combien de temps pensez-vous que Malfoy comprendra~?" dit le général Granger.

"Pas longtemps," dit le colonel Blaise Zabini. "Peut-être l'a-t-il déjà compris. Combien de temps avant que Potter ne comprenne~?"

"L'éternité," dit le général Granger, "à moins que Malfoy ne le lui dise ou que l'un de ses soldats ne comprenne. Harry Potter ne pense tout simplement pas comme ça."

"Vraiment~?" dit le capitaine Ernie McMillan, relevant les yeux d'un coin de table où il se faisait écraser aux échecs par le capitaine Ron Weasley (ils avaient rapporté toutes les autres chaises après le départ de Malfoy, bien sûr). "Je veux dire, ça m'a l'air assez évident. Qui essaierait d'avoir toutes les idées à lui seul~?"

"Harry," dit Hermione exactement au moment où Zabini dit "Malfoy."

"Malfoy pense qu'il est meilleur que tout le monde," dit Zabini.

"Et Harry… ne \emph{voit} pas vraiment les autres comme ça," dit Hermione.

C'était plutôt triste à vrai dire. Harry avait grandi seul, très seul. Non pas qu'il pense consciemment que seuls les génies avaient le droit d'exister. C'est juste qu'il ne lui \emph{viendrait jamais à l'esprit} que quiconque dans l'armée de Hermione, à part Hermione, soit capable d'avoir la moindre bonne idée.

"Quoi qu'il en soit," dit Hermione. "Capitaines Goldstein et Weasley, vous avez pour mission de trouver des idées stratégiques pour notre prochaine bataille. Capitaines Macmillan et Susan - pardon, je voulais dire Macmillan et Bones - essayez de trouver des tactiques que nous pourrions utiliser, et aussi des entraînements que nous devrions essayer. Oh, et félicitations pour notre chant de guerre, capitaine Goldstein, je pense que c'était un vrai plus pour notre \emph{esprit de corps}".

"Et qu'allez vous faire~?" dit Susan. "Et le colonel Zabini~?"

Hermione se leva en s'étirant. "Je vais essayer de découvrir à quoi Harry Potter pense et le colonel Zabini va essayer de découvrir ce que Drago Malfoy risque de faire la prochaine fois, et après avoir découvert quelque chose nous nous joindrons à vous. Je vais marcher tout en réfléchissant. Zabini, vous voulez m'accompagner~?"

"Oui, général," dit Zabini avec fraîcheur.

Elle n'avait pas voulu que ce soit un ordre. Hermione soupira un peu en pensée. Il allait falloir qu'elle s'y fasse, et bien que l'idée initiale de Zabini ait certainement fonctionné, elle n'était pas \emph{tout à fait} certaine que le citation mélange d'incitations positives et négatives fin de citation du professeur Quirrell suffirait à garder Zabini dans son camp jusqu'en décembre, lorsque les traîtres seraient autorisés pour la première fois…

Elle n'avait toujours aucune idée de ce qu'elle allait faire avec le vœu de Noël du professeur Quirrell. Peut-être que lorsque le moment viendrait, elle demanderait juste à Mandy si elle voulait quelque chose.~ 

%  LocalWords:  arry
