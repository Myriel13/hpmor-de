\chapter{Quelque chose à protéger~: Albus Dumbledore}

\lettrine{H}{arry} se tenait maintenant face aux gargouilles gardiennes du bureau du directeur… non, de la directrice. Le professeur Sinistra l'avait convoqué en lui disant que c'était une urgence, mais les portes ne s'ouvraient pas.

Des expériences avaient révélé que la Pierre rendait une Métamorphose permanente toutes les trois minutes et cinquante-quatre secondes, indépendamment de la taille de l'objet Métamorphosé. Une seule fois, au fond d'un placard obscur, avec sa lampe de poche la plus puissante braquée sur la Pierre, Harry avait cru voir une grille de petits points à l'intérieur du morceau de verre cramoisi, mais il n'arrivait plus à le voir et il commençait à se soupçonner de l'avoir imaginé. Il n'avait pas pu détecter d'autres pouvoirs dans la Pierre et elle ne répondait à aucun ordre mental.

Harry s'était laissé jusqu'à midi le lendemain pour trouver comment faire usage de la Pierre sans que quelqu'un d'autre ne s'en empare - tout en essayant de ne pas penser à ce qui se passait, ce qui continuait de se passer pendant ce temps.

Dix minutes plus tard, Minerva McGonagall s'approcha d'un pas vif. Ses bras étaient chargés de paperasse et elle portait encore le Choixpeau.

Les gargouilles s'inclinèrent devant elle dans un bref grincement pierreux.

«Le nouveau mot de passe est “Impermanence”», dit Minerva aux gargouilles, et elle s'écartèrent. «Je suis navrée, M. Potter, on m'a retardée…

--- D'accord.»

Minerva monta sur les longs escaliers en spirales qui préféraient s'élever plutôt que d'être gravis, et Harry monta derrière elle.

«Nous rencontrons Amelia Bones, directrice du département de justice magique, Alastor Maugrey, que vous avez rencontré, et Bartemius Crouch, directeur du département de coopération magique international, poursuivit-elle en montant. Ce sont les héritiers de Dumbledore, autant que vous ou moi.

--- Comment… comment va Hermione~?» Harry n'avait pas encore eu la chance de s'enquérir à son sujet.

«Filius a dit qu'elle semblait être en état de choc, ce qui ne m'étonne pas. Elle a demandé où vous étiez, on lui a dit que vous assistiez à un match de Quidditch, et elle a refusé de parler à quiconque avant d'avoir pu s'entretenir avec vous. On l'a emmenée à Sainte Mangouste où,» la directrice paraissait un peu perturbée, «un sortilège diagnostique standard a indiqué de Mlle Granger est une licorne en pleine forme, en excellente condition physique, et dont la crinière doit être peignée. Les sortilèges détecteurs de magie active indiquent à chaque fois qu'elle est en pleine métamorphose. Une Langue-de-plomb est venue mais Filius s'en est, euh, débarrassé. Il a lancé certains sortilèges qu'il ne devrait probablement pas connaître et a déclaré que l'âme de Hermione était en pleine forme mais à au moins trois kilomètres de son corps. Suite à ça, les guérisseurs les plus expérimentés ont abandonné. Elle est maintenant seule, dans sa cellule, avec les rats et les mouches…

--- Elle est quoi~?

--- Pardon M. Potter, c'est du jargon de Métamorphose. Mlle Granger est dans une chambre d'isolation avec une cage de rats apprivoisés et une boîte de mouches qui se reproduisent en vingt-quatre heures. Il semble logique que la cause mystérieuse de sa résurrection a laissé une sorte d'émanation qui embrouille les sortilèges des guérisseurs. Mais si rien de grave n'arrive aux rats ou aux enfants des mouches, on jugera sans danger le retour de Mlle Granger à Poudlard après son réveil demain matin.»

Harry n'était toujours pas sûr… pas sûr \emph{du tout} que Hermione apprécierait avoir été ressuscitée, du moins dans ces circonstances particulières. Il ne pensait pas vraiment que Hermione lui en voudrait et lui dirait qu'il avait mal agi. C'était juste un stéréotype imaginé par son cerveau. Il avait été profondément épuisé et embrouillé dans ses pensées lorsqu'il avait inventé l'histoire de sa résurrection, et Hermione comprendrait probablement. Mais il n'avait aucune idée de ce qu'elle en \emph{dirait}…

«Je me demande ce que Mlle Granger pensera de sa victoire contre Vous-Savez-Qui», dit Minerva d'un ton pensif tout en grimpant les escaliers si vite que Harry perdait son souffle à la suivre, «et de ces gens qui croient des choses fort intéressantes à son sujet.

--- Vous voulez dire parce qu'elle s'est toujours considérée n'être qu'un génie comme les autres et que maintenant ils seront nombreux à la voir comme la Ressuscitée et à vouloir lui serrer la main~?» dit Harry. \emph{Même si elle ne se rappelle pas avoir fait quoi que ce soit pour le mériter. Même si c'était le travail d'un autre et le sacrifice d'autres et qu'elle en récolte le mérite. Même si elle n'a pas l'impression d'avoir fait quoi que ce soit qui justifie la façon dont on la traite, même si elle n'est pas certaine de pouvoir être à la hauteur de ce qu'on croit savoir d'elle.} «Oh non, aucune idée. Je n'imagine même pas ce que ça doit être.»

\emph{Peut-être que je n'aurais pas dû lui imposer ça. Mais les gens auraient inventé} quelque chose \emph{, et Dieu sait ce qui leur serait passé par la tête. Ce serait bête de m'en vouloir pour ça. Je pense.}

Ils arrivèrent en haut des escaliers et entrèrent dans le bureau, rempli de dizaines d'objets étranges, tous face à un grand bureau, lui-même devant un immense trône.

La main de Minerva survola l'un de ces objets, celui avec gigoteurs dorés, et elle ferma brièvement les yeux. Puis elle ôta le Choixpeau et le posa sur un porte-chapeaux sur lequel se trouvaient trois pantoufles gauche. Elle transforma l'immense trône en un simple chaise rembourrée et le grand bureau en une table ronde autour de laquelle quatre autre chaises apparurent.

Harry observa cela avec un étrange pincement au cœur. Sans qu'un mot fut échangé, il sut qu'une cérémonie aurait dû accompagner ce changement de chaises et de table. Toute une cérémonie, pour l'installation de la directrice dans son nouveau bureau. Mais, pour une raison ou une autre, le temps leur manquait, et Minerva McGonagall ignorait tout cela, préférait l'expédience.

Un mouvement de la baguette de cette dernière alluma la cheminée alors même qu'elle s'asseyait dans la chaise qui avait appartenu à Dumbledore.

Harry s'installa sans bruit sur l'une des chaises à sa gauche.

Presque immédiatement, le feu brûla d'un vert émeraude et Alastor Maugrey surgit en tourbillonnant, baguette dressée. Il parut appréhender toute la pièce d'un seul regard, pointa sa baguette vers Harry et dit~: «Avada Kedavra.»

Cela se produisit si vite et Harry fut si surpris qu'Alastor termina son incantation avant même que Harry eut fini de lever sa propre baguette.

«Simple vérification», dit Alastor à la directrice, dont la baguette était à présent braquée sur Alastor. Elle était bouche bée, comme sur le point de prononcer des paroles qui lui échappaient. «Voldy aurait essayé d'éviter s'il avait possédé le corps du garçon. Je vais quand même devoir vérifier la petite Granger.» Alastor Maugrey alla à la droite de Minerva et s'assit.

Cette fraction de seconde avait suffit à ce que Harry pense à produire muettement une lumière argentée de Patronus, mais sa baguette avait été loin, très loin d'être positionnée à temps.

\emph{Eh bien, si jamais je m'étais senti invincible, voilà qui est réglé. Quelle précieuse leçon, M. Maugrey.}

Puis le feu de la cheminée redevint vert et cracha une vieille sorcière avec l'air le plus sévère et le plus coriace que Harry avait jamais vue~; on aurait cru que de la viande séchée avait pris forme humaine. La vieille sorcière n'avait pas sa baguette en main mais un air d'autorité plus fort, plus strict que celui de Dumbledore émanait d'elle.

«C'est la directrice Amelia Bones, M. Potter,» dit la directrice Minerva McGonagall, tout son flegme revenu. «Nous attendons toujours le directeur Crouch…

--- Le corps de Bartemius Crouch Junior a été retrouvé parmi ceux des Mangemorts décédés», dit immédiatement la vieille sorcière tout en continuant vers les chaises. «Cela nous a profondément surpris et j'ai peur que Bartemius ne soit inconsolable, pour deux raisons. Il ne nous rejoindra pas aujourd'hui.»

Harry ne montra rien de son tressaillement intérieur.

Amelia Bones s'assit, à droite de Maugrey.

«Directrice McGonagall», dit la vieille sorcière, toujours sans hésiter ni attendre, «La Lignée Ininterrompue de Merlin dont Dumbledore m'a confié la régence ne me répond plus. Le Magenmagot doit \emph{immédiatement} se munir d'un président-sorcier de confiance~; de nombreux changements grondent en Angleterre. Je dois savoir ce que Dumbledore a fait, et tout de suite~!

--- Merde,» marmonna Maugrey. Son œil fou tourbillonnait dans tous les sens. «C'est mauvais ça, c'est très mauvais.

--- Oui, eh bien,» dit Minerva McGonagall d'un ton plutôt inquiet. «Je ne suis sûre de rien. Albus… eh bien, il était clair qu'il doutait de pouvoir survivre à cette guerre. Mais je ne pense pas qu'il s'attendait à ce que Mlle Granger revienne d'entre les morts et tue Voldemort quelques heures après. Je ne pense pas qu'il se soit attendu à cela. Je ne sais pas bien qui va hériter de quoi…»

Amelia Bones se leva à moitié. «Vous suggérez que la petite \emph{Granger} pourrait avoir hérité de la Lignée Ininterrompue de Merlin~? C'est une \emph{catastrophe}~! Elle a douze ans, elle n'a jamais fait ses preuves… Albus n'aurait pas été irresponsable au point de léguer la Lignée au premier qui tuera Voldemort, sans savoir de qui il s'agirait~!

--- Eh bien, pour le dire simplement…» répondit Minerva. Ses doigts saisirent les documents qu'elle avait apporté et qui étaient posés sur le bureau. «… Albus \emph{croyait} savoir qui tuerait Voldemort. Il y avait une prophétie certifiée à ce sujet, mais elle doit être en suspens, ou alors… je ne sais pas, Madame Bones~! J'ai une lettre à remettre à Harry Potter après la mort ou le départ de Dumbledore, et une autre qui ne doit être remise à M. Potter que lorsqu'il aura vaincu le Seigneur des Ténèbres. Je ne sais pas bien ce qui va se passer maintenant. Peut-être Mlle Granger pourra-t-elle l'ouvrir, ou peut-être qu'elle restera scellée à jamais…

--- Attendez», dit Maugrey Fol-Œil. Il farfouilla dans ses robes et en sortit une longue baguette grisâtre que Harry reconnut~: c'était la baguette de Dumbledore, différente de toutes les autres baguettes de Poudlard. Maugrey la déposa sur la table. «Avant d'aller plus loin, Albus m'a donné un ordre ou deux à moi aussi. Ramasse cette baguette, petit.»

Harry hésita et réfléchit.

\emph{Albus Dumbledore s'est sacrifié pour moi. Il faisait confiance à Maugrey. Ce n'est probablement pas un piège.}

Puis il tendit la main vers la baguette.

Elle vola au-dessus de la table, vers la main de Harry. Au moment où ses doigts la saisirent, il cru entendre un chant, un hymne de gloire et de bataille, résonner dans son esprit. Une vague de feu blanc remonta le long de la baguette, grandit et explosa dans un jaillissement d'étincelles. Dans le bois sous ses doigts passait une sensation de force et de danger retenu, comme un loup tenu en laisse.

Harry reçut aussi un sentiment de scepticisme marqué, comme si la baguette était à demi consciente et se demandait comment elle avait bien pu atterrir dans la main d'un élève de Poudlard en première année.

«Ouais, dit Maugrey Fol-Œil à l'intention des regards interloqués. «Donc c'est pas Mlle Granger qui a battu Voldy. M'en doutais.

--- Quoi.» Amelia Bones avait prononcé le mot d'un ton parfaitement monocorde.

Maugrey Fol-Œil lui adressa un hochement de tête respectueux. «Albus a dit que cette baguette appartient à celui qui a vaincu son dernier maître. Je crois qu'il l'a piquée à Grindy. Ensuite Voldy a vaincu Albus, hier. Est-ce que je dois te faire un dessin, Amelia~?»

Amelia Bones regardait Harry, bouche bée.

«Il y a peut-être une erreur,» dit Harry. Il réprima un autre pincement de cette horrible culpabilité. «Parce que Voldemort m'a utilisé comme otage, parce j'ai… j'ai été stupide, et Dumbledore s'est sacrifié pour me sauver, peut-être que la baguette pense que c'est comme si j'avais vaincu Dumbledore. Euh, enfin, j'ai effectivement battu Voldemort. Je l'ai vaincu. Mais je pense que c'est mieux si personne ne sait que j'étais là.»

Bip. Tic. Ploutch. Ding. Boop.

«\emph{Voilà} qui n'a pas dû être facile», dit Maugrey. L'homme balafré inclina lentement la tête en signe de profond respect. «Ne t'en veux pas trop d'avoir perdu Albus, David et Flamel, petit, aussi bête que tu aies été. Au final, tu as gagné. Nous tous réunis n'y sommes jamais arrivés. Juste, David et toi, vous avez bien détruit le Horcruxe de Voldy~? Et tu es \emph{certain} que c'était le vrai~?»

Harry hésita, soupesa les conséquences probable d'un accord de confiance, les désastres possibles découlant d'un silence, puis il secoua la tête en guise de réponse. De toute façon, il avait prévu de dire à McGonagall ce que renfermait désormais son école. «Voldemort avait… un sacré nombre de Horcruxes, pour tout vous dire. Alors j'ai effacé la plupart de ses souvenirs et je l'ai métamorphosé en ça.» Harry leva la main et indiqua silencieusement l'émeraude à son doigt.

Ploutch. Boing. Ploutch. Ploutch.

«Ah, dit Maugrey en se renfonçant dans sa chaise. Minerva et moi placerons quelques alarmes et sortilèges sur ton petit anneau, mon garçon, si tu veux bien. Juste au cas où tu oublierais de maintenir la Métamorphose. Et ne vas jamais chasser d'autres mages noir, jamais, vis juste une vie tranquille et paisible.» L'homme balafré saisit un mouchoir et essuya les gouttes de sueur qui venaient d'apparaître sur son front. «Mais bien joué mon gars, toi et David, qu'il repose en paix. J'imagine que c'était son idée~? Bien joué en tout cas.

--- Effectivement, dit Amelia Bones qui avait retrouvé son calme. Vous avez tous les deux notre plus profonde gratitude. Mais je répète qu'il y a des affaires pressantes concernant la Lignée Ininterrompue de Merlin.

--- Je crois, dit lentement Minerva McGonagall, que je devrais donner tout de suite les lettres d'Albus à M. Potter.» Une enveloppe de parchemin se trouvait à présent sur la pile de papiers disposée devant elle, accompagnée d'un parchemin maintenu par un ruban gris.

La directrice donna d'abord l'enveloppe à Harry, et il l'ouvrit.

\later
\begin{writtenNote}
Harry Potter, si tu lis ces lignes, c'est que Voldemort m'a vaincu et et que ma quête est maintenant tienne.

Tu seras peut-être surpris de l'apprendre, mais c'est la fin que je désirais le plus. Car, alors que j'écris ces lignes, il semble possible que Voldemort tombe par mon fait. Et alors ce sera moi qui deviendrai les ténèbres contre lesquelles tu te devras te dresser avant de pouvoir atteindre tout ton potentiel. Car il a un jour été dit que tu devrais peut-être lever la main contre ton mentor, contre celui qui t'avait fait, que tu aimais~; il a été dit que tu pourrais être ma chute. Si tu lis ces lignes, c'est que cela ne se produira jamais, et j'en suis heureux.

Néanmoins Harry, je souhaite t'épargner cela, cette lutte solitaire contre Voldemort. J'écris ces lignes en faisant le vœu de te protéger le plus longtemps possible, quoi qu'il m'en coûte. Mais si j'ai échoué, sache que même si c'est égoïste, j'en suis heureux.

Après mon départ, plus personne sauf toi ne pourra rivaliser avec Voldemort. Son ombre s'étendra, longue et terrible, au-dessus de l'Angleterre magique, et nombre souffriront et mourront par sa faute. Cette ombre ne se dissipera que lorsque tu auras détruit sa source, que lorsque tu auras purifié ce cœur de ténèbres. Comment, je l'ignore. Voldemort ignore le pouvoir qui est en toi, et moi aussi. Tu dois trouver ce pouvoir en toi, apprendre à le manier, et devenir le juge de Voldemort~; je t'implore de ne pas faire l'erreur de te montrer clément envers lui.

Ma baguette, que j'ai confiée à Maugrey pour qu'il te la donne~: tu ne dois pas prendre le risque d'en faire usage contre lui. Car lorsque son maître est vaincu, elle passe alors au vainqueur. Lorsque tu auras pourfendu mon pourfendeur, alors elle te répondra vraiment, mais il est certain qu'elle se retournera contre toi le jour où tu essaieras d'en faire usage contre lui. Garde-la hors de portée de lui, à tout prix. Je devrais te conseiller de ne jamais l'utiliser, mais c'est un objet fort puissant dont tu pourras avoir l'usage aux moments désespérés. Si tu t'en sers, tu devras toujours redouter sa trahison.

En mon absence, le Magenmagot tombera inévitablement aux mains de Malfoy. Je t'ai transmis la Lignée Ininterrompue de Merlin, avec Amelia Bones comme régente jusqu'à ce que tu sois assez âgé ou que tu atteignes tout ton potentiel. Mais elle ne pourra pas longtemps tenir face à Malfoy, pas sans moi et avec Voldemort de retour pour le conseiller. Je pense que le ministère tombera bientôt et que Poudlard deviendra la dernière forteresse. J'ai laissé les clés de Poudlard à Minerva, mais toi seul es son prince, et elle t'aidera de son mieux.

Alastor dirige dorénavant l'Ordre du Phénix. Écoute-le attentivement, ses conseils comme ses confidences. L'un de mes plus grands regrets est de n'avoir pas plus écouté Alastor, et de ne pas l'avoir écouté plus tôt.

Que tu finiras par vaincre Voldemort~: cela, je n'en doute pas.

Car ce ne sera que le début de ta destinée. De cela aussi, je suis certain.

Lorsque tu auras vaincu Voldemort, lorsque tu auras sauvé ce pays, alors j'espère que tu pourras t'atteler à l'œuvre de ta vie.

Hâte-toi donc de commencer.

\letterClosing[Yours in death (or in whatever),]{Dumbledore.}

P.S. Les mots de passe son “prix du phénix”, “destin du phénix”, et “œuf du phénix”, dits dans mon bureau. Minerva peut mettre ces pièces là où elles te seront plus faciles d'accès.
\end{writtenNote}

\later

Harrry replia le parchemin et le remit dans l'enveloppe en fronçant les sourcils d'un air pensif, puis il prit le rouleau à ruban gris des mains de la directrice. Lorsque la longue baguette grise de Harry toucha le ruban, il tomba soudain à terre~; Harry déroula le parchemin et le lu.

\later
\begin{writtenNote}
\letterAddress{Cher Harry James Potter-Evans-Verres~:}

Si tu lis ces lignes, c'est que tu as vaincu Voldemort.

Félicitations.

J'espère que tu as le temps de fêter cette victoire avant d'ouvrir le parchemin, parce que ce qui s'y trouve ne te fera pas sourire.

Pendant la première guerre des sorciers, j'ai fini par me rendre compte que Voldemort gagnait, qu'il aurait bientôt tout le pays entre ses mains.

En dernière extrémité, je me suis rendu au département des mystères et j'ai prononcé un mot de passe qui n'avait pas été dit depuis le début de la Lignée Ininterrompue de Merlin~; j'ai fait une chose interdite mais pas absolument interdite.

J'ai écouté toutes les prophéties à avoir jamais été enregistrées.

Et c'est ainsi que j'ai appris que Voldemort était le cadet de mes soucis.

Un chorus s'élève chez certains voyants et devins, des prédictions d'une inévitable destruction de ce monde.

Et toi, Harry James Potter-Evans-Verres, sera l'un de ses destructeurs.

Il aurait été légitime que j'élimine la possibilité de ton existence, que je t'empêche de jamais naître, comme je me suis efforcé d'éliminer toutes les autres possibilités que j'ai découvertes ce jour terrible.

Mais dans ton cas, Harry, et dans ton cas seulement, les prophéties de ton apocalypse avaient des failles, de toutes petites failles.

C'était toujours 'il provoquera la fin du monde', jamais 'il provoquera la fin de toute vie'.

Même lorsqu'il fut dit que tu déchirerais jusqu'aux étoiles du ciel, il ne fut pas dit que tu déchirerais des gens.

Et donc, comme il est clair que ce monde n'est pas fait pour durer, j'ai littéralement tout misé sur toi, Harry James Potter-Evans-Verres. Aucune prophétie ne disait comment sauver le monde, alors j'ai trouvé celles où la destruction n'était pas assurée~; et j'ai mis en place les conditions complexes et étranges pour qu'elles aient lieu. Je me suis assuré que Voldemort découvre l'une d'elles, et j'ai donc (comme je le craignais) condamné tes parents, et fait de toi ce que tu es. J'ai écrit un étrange conseil dans le manuel de potions de ta mère sans savoir pourquoi~; et il s'est avéré montrer à Lily comment aider sa sœur, comment assurer que cette dernière t'aimerait de tout son cœur. Je suis venu, invisible, dans ta chambre à Oxford, et je t'ai administré la potion que l'on donne aux élèves munis d'un Retourneur de Temps pour allonger leur cycle de sommeil de deux heures. Lorque tu avais six ans, j'ai cassé un rocher qui était au bord de ta fenêtre~; et aujourd'hui encore, je me demande bien pourquoi.

Tout cela mû par l'espoir désespéré que tu nous guides à travers l'œil du cyclone, que tu parviennes à mettre fin à ce monde tout en sauvant ses habitants.

Maintenant que tu as réussi l'épreuve préparatoire consistant à vaincre Voldemort, je remet tout entre tes mains, tous les outils qui sont à ma disposition. La Lignée Ininterrompue de Merlin, la direction de l'Ordre du Phénix, toute ma richesse, tous mes trésors, la baguette de Sureau, l'une des Reliques de la Mort, la loyauté de mes plus chers amis. J'ai laissé Poudlard aux mains de Minerva car je ne pense pas que tu auras le temps de t'en préoccuper, mais même cela est tien si tu le souhaite.

Il y a une chose que je ne peux t'offrir, et ce sont les prophéties. Elles seront détruites à ma disparition et aucune autre ne sera jamais enregistrée, car il a été dit que tu ne devrais pas les entendre. Si tu penses que c'est frustrant, crois-moi quand je te dis que même toi, tu ne peux appréhender les frustrations qui viennent de t'être épargnées. Je mourrai - ou tu me perdras - ou je te serai arraché d'un façon ou d'une autre - les prophéties sont évidemment floues - sans jamais avoir su ce que l'avenir nous prépare vraiment, ni pourquoi j'ai dû faire ce que j'ai fait. C'était un monceau de folie indéchiffrable~; bon débarras.

Il ne peut y avoir qu'un roi sur l'échiquier.

Une seule pièce dont la valeur est inestimable.

Cette pièce n'est pas le monde, ce sont ses habitants, les sorciers comme les Moldus, les gobelins, les elfes de maisons et tout le reste.

Tant qu'il en reste un vestige, la pièce est toujours en jeu, que les étoiles meurent ou pas.

Et si cette pièce est perdue, la partie s'achève.

Apprends la valeur de toute tes autres pièces, et joue pour gagner.

\letterClosing{—Albus}
\end{writtenNote}

\later

Harry garda le parchemin en main un long moment, le regard perdu.

Donc.

Parfois, la phrase 'Alors c'était ça' ne paraissait pas suffire. Mais quand même~: Alors, c'était ça.

Harry enroula le parchemin, le regard toujours perdu.

«Que dit cette lettre~? demanda Amelia Bones.

--- C'est une lettre de confession, dit Harry. En fait, c'est Dumbledore qui a tué mon rocher de compagnie.

--- \emph{Ce n'est pas le moment de blaguer~!} s'écria la vieille sorcière. Êtes-vous le véritable titulaire de la Lignée Ininterrompue de Merlin~?

--- Oui», dit Harry d'un air absent, son esprit occupé par des choses éminemment plus importantes, et ce quel que soit le critère de quantification utilisé.

La vieille sorcière resta assise, presque figée. Elle tourna la tête et croisa la regard de Minerva McGonagall.

Pendant ce temps, le cerveau de Harry jonglait avec trop de possibilités distribuées parmi trop d'horizons temporels, et certains d'entre eux incluaient au sens propre des milliards d'années et des procédures de désassemblage stellaire… il déclara une banqueroute cognitive et recommença à partir de zéro. \emph{Bon, si je veux sauver le monde, quelle est la} première \emph{chose que je dois faire… non, encore plus local, qu'est-ce que je dois faire} aujourd'hui \emph{… en plus de trouver ce que je dois faire, bien sûr, et je devrais regarder le plus vite possible ce que Dumbledore m'a laissé dans la pièce de l'œuf de phénix…}

Harry détacha son regard du parchemin roulé et observa le professeur… la directrice McGonagall, Maugrey Fol-Œil et la sorcière au visage tanné comme pour la première fois. C'était de fait presque la première fois qu'il voyait Amelia Bones.

Amelia Bones, directrice du département de justice magique, qu'Albus Dumbledore avait jugée digne de diriger le Magenmagot au moins temporairement. Sa coopération serait inestimable, peut-être \emph{nécessaire} à… à ce à quoi Harry ferait face. Dumbledore l'avait choisie, et il avait lu des prophéties que Harry n'avait jamais vues.

Amelia Bones, qui avait cru avoir été nommée régente de la Lignée Ininterrompue de Merlin et nouvelle présidente-sorcière avant d'apprendre que ces postes étaient apparemment destinées à un garçon de onze ans.

\emph{Maintenant,} dit la voix de Poufsouffle, \emph{maintenant, tu vas être poli. Tu ne feras pas l'imbécile comme tu le fais d'habitude. Parce le destin du monde en dépend peut-être. Ou pas. On ne sait même pas ça.}

«Je suis vraiment navré», dit Harry Potter avant de s'interrompre pour voir quel effet ses paroles avaient pu produire.

«Minerva semble penser, dit la vieille sorcière, qu'un franc parler ne vous offensera pas.»

Harry hocha la tête. Sa partie Serdaigle voulait ajouter une clause mentionnant que ce n'était pas pareil que de rabaisser quelqu'un tout en se plaignant de son intolérance à la critique, mais Poufsouffle usa d'un veto. Harry entendrait ce qu'elle avait à dire.

«Je ne souhaite pas dire du mal des disparus, dit la vieille sorcière. Mais voilà des temps immémoriaux que la Lignée Ininterrompue de Merlin se transmet à ceux qui ont \emph{minutieusement} démontré n'être pas seulement des gens bien, mais aussi des gens assez sage pour distinguer des successeurs à la fois bons et sages. Un seul accroc où que ce soit dans la chaîne, et la succession pourrait s'égarer pour ne jamais se ressaisir~! Dumbledore a été fou de vous transmettre la Lignée si jeune, même en précisant que vous deviez avoir vaincu Vous-Savez-Qui. Une souillure de son héritage~: voilà comment on le verra.» La vieille sorcière hésita, ses yeux toujours sur Harry. «Je pense qu'il vaut mieux que personne à part nous ne l'apprenne jamais.

--- Euh, dit Harry. J'imagine que vous n'avez pas une très bonne opinion de Dumbledore~?

--- Je pensais… dit la vieille sorcière. Eh bien, Albus Dumbledore était meilleur sorcier que moi, une meilleure \emph{personne} que moi par bien des façons. Mais il avait ses défauts.

--- Parce que, euh. Enfin. Dumbledore \emph{savait} tout ce que vous venez de dire. Que je suis jeune, que la Lignée fonctionne comme ça. Vous vous comportez comme si Dumbledore ignorait ces faits ou qu'il n'en a pas tenu compte au moment de prendre sa décision. C'est vrai que parfois les gens bêtes, moi par exemple, prennent des décisions folles de ce genre. Mais pas Dumbledore. Il n'était \emph{pas} fou.» Harry déglutit, réprima une humidité dans ses yeux. «Je pense… je commence à me rendre compte… que depuis le début, Dumbledore était le seul à être sain d'esprit. Le \emph{seul} à faire plus ou moins les bonnes choses pour les bonnes raisons…»

Madame Bones jurait dans sa barbe, de basses et sinistre imprécations qui faisaient trembler Minerva.

«Désolé», dit Harry d'un ton d'impuissance.

Le visage balafré de Maugrey affichait un large sourire. «Toujours su qu'Albus manigançait \emph{quelque chose} sans nous le dire. Petit, tu ne sais pas à quel point j'ai du mal à ne pas utiliser mon Œil sur ce parchemin.»

Harry le rangea vivement dans sa bourse en peau de Moke.

«Alastor», dit Amelia. Sa voix monta d'un cran. «Tu es un homme sensé, tu ne peux pas penser que ce petit pourrait prendre la place de Dumbledore~! Pas \emph{aujourd'hui}~!

--- Dumbledore, dit Harry avec l'impression que le mot avait un goût étrange, a commis une erreur au moment de prendre ses décisions. Il a cru que nous combattrions Voldemort pendant des années, tous ensemble. Il ignorait que je vaincrai le Seigneur des Ténèbres immédiatement. J'ai eu raison de le faire, j'ai sauvé de nombreuses vies qui auraient été perdues si nous avions combattu pendant des années. Mais Dumbledore pensait que vous auriez tout ce temps pour apprendre à me connaître, à me faire confiance… et ça s'est fini en une soirée.» Harry inspira. «Est-ce que vous ne pourriez pas \emph{faire comme si} on avait combattu Voldemort pendant des années, comme si j'avais gagné votre confiance et tout ça~? Pour que gagner plus vite que Dumbledore ne l'avait prévu ne me \emph{pénalise} pas~?

--- Vous êtes toujours en première année à Poudlard~! dit la vieille sorcière. Vous ne \emph{pouvez pas} prendre la place de Dumbledore, quelles qu'aient été ses intentions~!

--- Ah oui, c'est vrai que j'ai l'air d'un enfant de onze ans.» Harry leva une main et frotta son nez là où se trouvaient ses lunettes. \emph{J'imagine que je pourrais toujours utiliser la Pierre et me donner l'air d'avoir quatre-vingt dix ans…}

«Je ne suis pas une imbécile, dit la vieille sorcière. Je sais que vous n'êtes pas un enfant ordinaire. Je vous ai vu parler à Lucius Malfoy, je vous ai vu effrayer un Détraqueur, et j'ai vu Fumseck répondre à votre prière. Tout sage témoin de vos actes au Magenmagot - et je parle donc de moi et au plus de deux autres personnes - a pu deviner que vous aviez absorbé une partie de l'âme broyée de Vous-Savez-Qui la nuit de sa fausse mort avant de la dominer et d'utiliser son savoir à des fins bienfaitrices.»

Il y eut un léger silence.

«Ah mais oui, bien sûr», dit Minerva McGonagall. Elle soupira et s'affaissa un peu dans sa chaise de directrice. «Comme Albus le savait clairement \emph{depuis le début} mais a eu la prévenance de ne \emph{jamais m'en parler}.

--- Ouais, dit Maugrey. Je le savais. C'est ça. Parfaitement évident. J'étais pas du tout à côté de la plaque.

--- J'imagine que c'est plus ou moins ce qui s'est passé, dit Harry. Donc, euh. Quel est le problème exactement~?

--- Le problème, dit Amelia Bones d'un ton parfaitement neutre, est que vous êtes un mélange bouillonnant et instable d'élève en première année et de Vous-Savez-Qui.» Elle se tut, comme si elle attendait quelque chose.

«Je fais des progrès», dit Harry, puisqu'elle semblait attendre une réponse. «Assez vite, en fait. Et plus important encore~: Dumbledore ne l'ignorait pas.»

La vieille sorcière continua. «Donner votre fortune et vous endetter auprès de Lucius Malfoy pour garder votre meilleure amie hors d'Azkaban a peut-être démontrer votre moralité exemplaire mais aussi votre incapacité à contrôler le Magenmagot. Je comprends maintenant que vous avez fait ce que vous aviez besoin de faire pour garder votre santé mentale, retenir les ténèbres qui vivent en vous. Mais vous avez aussi fait ce qu'un héritier de Merlin ne doit jamais faire. Un chef sentimental peut être bien pire qu'un chef égoïste. Nous avons à peine survécu à Albus, maître et serviteur d'un phénix - et ce jour là, même lui s'est opposé à vous.» Amelia fit un geste en direction de Maugrey. «Alastor sait être dur. Il sait être rusé. Il ne sait néanmoins pas gouverner. Vous, Harry Potter, n'avez ni la sévérité, ni la capacité à faire des sacrifices nécessaires à diriger ne serait-ce que l'Ordre du Phénix. Et étant donné ce que vous êtes, vous ne \emph{devez pas} essayer d'en être capable. Pas maintenant ni jamais. Alignez et fusionnez votre âme divisée, si vous le pouvez. Mais n'essayez pas d'être président-sorcier en même temps. Si Albus pensait que c'était une bonne idée, c'est qu'il inventait une belle histoire au mépris du sens pratique. Je pense qu'il avait des problèmes de ce côté là.»

Harry écarquilla un peu les yeux en écoutant ces mots. «Euh… vous pensez qu'il se passe quoi là-haut, exactement~?» Harry se tapa le crâne juste au-dessus de l'oreille.

«J'imagine qu'en vous se trouve l'âme d'un garçon toujours honnête et sincère qui use de toute sa volonté pour repousser le fragment de Voldemort qui essaie de le consumer tout en lui hurlant qu'il est sentimental et faible… est-ce que vous venez de glousser~?

--- Pardon. Mais sérieusement, ça n'a jamais été si grave que ça. C'était plus comme d'avoir plein de mauvaises habitudes de pensées à défaire.

--- Euh, dit la directrice McGonagall. M. Potter, je pense qu'au début de l'année, c'était aussi grave que ça.

--- De mauvaises habitudes qui s'enchaînaient et se déclenchaient les unes les autres. Oui, c'est assez problématique.» Harry soupira. «Et vous, Madame Bones… euh. Désolé si je fais erreur. Mais j'ai l'impression que ça vous énerve un peu de voir la Lignée atterrir dans les mains d'un enfant de onze ans.

--- Pas pour les raisons que vous croyez, dit la vieille sorcière d'un ton calme. Mais il naturel que vous me soupçonniez. La position de présidente-sorcière ne me plairait guère, même comparée aux horreurs de la justice magique. Albus a réussi à me persuader d'accepter, et même si j'aimerais dire que j'ai été dure à convaincre, j'ai préféré ne pas perdre de temps avec un débat que je m'attendais à perdre. Je savais que je détesterai ce travail, et que je le ferai quand même. Minerva dit que vous avez un minimum de sens commun, surtout quand on vous le rappelle. Vous imaginez-vous vraiment trôner au Magenmagot~? Êtes-vous sûr que ce n'est pas un reste de Vous-Savez-Qui qui s'imagine taillé pour le poste~?»

Harry enleva ses lunettes. Sa cicatrice lui faisait toujours un peu mal, conséquence du temps qu'il avait passé à la gratter pour le saignement théâtral de la veille.

«J'ai du bon sens, et oui, le travail de président-sorcier semble profondément agaçant. Ce poste me convient le moins du monde. Le problème. Euh. Je ne suis pas sûr que la Lignée de Merlin existe pour le poste de président-sorcier. Il y a, euh. Je pense… qu'il y a d'autres choses étranges qui l'accompagnent. Et que Dumbledore voulait que je sois responsable du… reste. Et le reste est… peut-être \emph{incroyablement} important.

--- Merde,» dit Maugrey. Puis il se répéta~: «Merde, gamin, est-ce que tu devrais nous dire tout ça~?

--- Je ne sais pas, dit Harry. S'il y a un manuel utilisateur, je ne l'ai pas encore lu.

--- Merde.

--- Et si ces autres choses nécessitent d'être dur, de faire des sacrifices~?» dit Amelia Bones d'un ton toujours calme. «Si vous êtes mis à l'épreuve, comme vous l'avez été devant le Magenmagot~? Je suis vieille, Harry Potter, je connais certains des mystères. Vous avez vu comme j'ai compris qui vous étiez presque d'un coup d'œil.

--- Amelia, dit Maugrey Fol-Œil. Si tu avais combattu Tu-Sais-Qui hier, il se serait passé quoi~?»

La vieille sorcière haussa les épaules.

«Je pense que je serais morte.

--- Tu aurais \emph{perdu}, dit Alastor Maugrey. Et le Survivant n'a pas seulement descendu Voldy, il s'est aussi arrangé pour que sa bonne amie Hermione Granger \emph{revienne d'entre les morts} au moment où Voldie s'est ressuscité. Jamais de la vie ni de la mort je croirai que c'est un hasard, et je pense pas non plus que c'était l'idée de David. Amelia, la vérité, c'est qu'on a pas la moindre idée de ce que les héritiers de Merlin doivent \emph{faire}. Peut-être qu'on est dingues, mais ces conneries-là, c'est pas pour nous.»

Amelia Bones fronça les sourcils.

«Alastor, tu sais que je me suis déjà occupée de choses très étranges. Je pense même que je m'en suis très bien occupée.

--- Ouais. Tu t'es \emph{occupée} de ces conneries pour pouvoir continuer ta vie. Tu n'est pas le genre de malade qui construit un château fait de ces conneries pour pouvoir s'y installer.» Maugrey soupira. «Amelia, y'a une partie de toi qui sait exactement pourquoi Albus a laissé ce job mystérieux au pauvre gamin.»

Les poings de la vieille sorcière, posés sur la table, se serrèrent.

«Est-ce que tu peux \emph{imaginer} le désastre que ce serait pour l'Angleterre~? Peut-être que je ne suis pas assez timbrée, mais je ne peux pas l'accepter~! J'ai travaillé trop longtemps dans l'espoir d'arriver à un jour comme aujourd'hui pour tout voir s'effondrer, surtout \emph{maintenant}~!

--- Excusez-moi», dit la directrice McGonagall d'un assez méticuleux et Écossais. «Pourquoi M. Potter ne pourrait-il pas simplement indiquer à la Lignée Ininterrompue que Madame Bones est présidente-sorcière régente, jusqu'à ce que M. Potter mûrisse, mais qu'elle n'a aucune autorité concernant le département des mystères~? Si Albus a pu dire à la Lignée de ne nommer une régente que jusqu'à la défaite de Voldemort, c'est qu'elle est clairement capable d'obeir à des ordres complexes.»

Ce coup de bon sens soudain fut absorbé par tous.

Harry ouvrit la bouche pour accepter de nommer Amelia Bones régente des affaires du Magenmagot, mais hésita à nouveau.

«Euh, dit Harry. Euh. Madame Bones, j'aimerais beaucoup que vous vous préoccupiez du Magenmagot à ma place.

--- Nous sommes d'accord là-dessus, dit la vieille sorcière. Pouvons-nous procéder~?

--- Mais…»

Il y eut une sorte de mouvement de recul frustré chez les autres. «Quel est le problème, M. Potter~?» dit la directrice d'un ton qui indiquait clairement qu'elle espérait bien ne pas faire face à une objection sérieuse.

«Euh. Je crois que je vais bientôt devoir faire plusieurs choses qui, politiquement… seront sujettes à controverse. Alors en échange de la gestion du pouvoir politique de la Lignée, je voudrais que Madame Bones me promette sa… euh, coopération.»

Amelia Bones partagea un long regard avec Minerva McGonagall. Puis elle regarda de nouveau Harry Potter.

«Votre requête m'indigne~! dit-elle. Votre hésitation m'indique que vous êtes faibles et peu habitué aux négociations, et que vous capitulerez probablement si je m'oppose à vous.»

Harry ferma les yeux.

Un Harry \emph{légèrement} sombre les ouvrit.

«Très bien, dit-il, laissez-moi reformuler. Je ne compte pas m'immiscer quotidiennement ni même mensuellement dans votre travail, mais je ne peux pas me débarrasser de la responsabilité que Dumbledore m'a léguée. Je ne vais pas vous envoyer des chouettes avec des parchemins bizarres, nous pourrons discuter avant, mais je devrai parfois vous donner un ordre. Si vous refusez d'obéir, je devrai peut-être reprendre le contrôle direct du Magenmagot. Pourrez-vous faire avec~?

--- Et si je dis non~? dit la vieille sorcière.

\emph{Légèrement, très légèrement sombre…} «Je ne sais pas par qui je pourrais vous remplacer. Je pourrais commencer par demander à Augusta Londubat qui pourrait correspondre au poste. Mais il peut être important que nous suivions le plan de Dumbledore d'aussi près que possible puisque je ne connais pas exactement toutes les raisons de ses choix. Et il pensait que vous devriez être présidente-sorcière pendant un moment. Je ne vais pas évoquer Merlin… en fait si, je \emph{vais} évoquer Merlin~: tout cela est peut-être incroyablement important.»

La vieille sorcière réfléchit un moment, et ses yeux allaient d'un personne à une autre. «Cela ne me satisfait pas, dit-elle au bout d'un moment. Mais le Magenmagot doit bientôt être réuni. Cela suffira pour l'instant.»

Elle passa lentement une main dans ses robes et en sortit un petit cylindre de pierre, de pierre sombre.

Elle le plaça sur la table devant Harry. «Prenez ce qui vous appartient, dit-elle. Et ensuite rendez-le moi, s'il vous plaît.»

Harry tendit la main pour s'en saisir.

Et au moment où ses doigts touchèrent la pierre sombre…

… rien ne se produisit.

Ah, peut-être que Merlin n'avait pas aimé faire dans le mélodrame. C'était peut-être la raison pour laquelle son héritage ressemblait à un banal bâton noir. Si c'était là tout le nécessaire à sa fonction, c'est tout ce qu'il y aurait.

Harry fronça les sourcils, Lignée en main. «Je souhaiterais nommer Amelia Bones au poste de régente de mes devoirs liés au Magenmagot.» Puis l'idée lui vint que la définition d'une régence se devait d'inclure une condition de terminaison. Il ajouta~: «Jusqu'à ce que je veuille la récupérer.»

Puis il fit une grimace. Il avait attendu beaucoup de la Lignée, mais elle n'était qu'une clé pour les lieux du département des mystères où on gardait les objets intéressants et pour les sceaux derrière lesquels Merlin et ses successeurs avaient rangé ce qui ne devait pas être détruit mais gardé hors de portée du public. La Lignée ne faisait pas grand-chose de plus.

Elle ne permettait pas non plus d'outrepasser l'Interdit de Merlin. Non, même si le destin de la galaxie était en jeu. Même si la personne avait l'air saine d'esprit, même si cette personne avait prononcé un Serment Inviolable et pensait vraiment que sans ça, le monde serait détruit.

Merlin avait rêvé à long terme, il avait rêvé d'un monde qui durerait des éons, et pas seulement des siècles. Le monde n'avait aucune raison de ne pas être \emph{éternel} si les forces vraiment dangereuses étaient écartées et maintenues à l'écart. À l'inverse, une seule faille dans les protections garantissait à la longue l'arrivée de la fin du monde. La Lignée serait un jour transmise à la mauvaise personne. Elle pouvait rejeter ceux qui n'étaient clairement pas dignes d'elle, mais elle finirait par tomber entre des mains aux défauts trop subtils pour être détectés. C'était inévitable avec les êtres humains, et Harry devrait garder cela à l'esprit avant de mettre sous scellé des choses que les prochains porteurs de la Lignée pourraient retrouver - le désastre de leur mauvaise utilisation \emph{éventuelle} devrait être moins lourd que les bénéfices potentiels pendant les milliers d'années à venir.

Harry poussa un petit soupir. \emph{Merlin, espèce d'idiot…}

Cette pensée ne débloqua rien de plus.

Il n'y avait pas d'incendie en cours au département des Mystères, alors Harry reposa précautionneusement la Lignée sur la table.

«Merci», dit la vieille sorcière. Elle se saisit du bâton de pierre noire. «Savez-vous comment je dois faire pour réunir le Magenmagot, ou… laissez tomber, je frapperai simplement le podium. Cela semble évident. Pour le reste du pays, je suis bien sûr présidente-sorcière. Seuls nous quatre savons qu'il en est autrement.»

Harry hésita. Puis il imagina les chouettes qu'il recevrait si on apprenait qu'il avait le pouvoir de remettre en question la présidente-sorcière et l'impact que cela aurait sur les capacités de négociation d'Amelia. «Très bien.»

Amelia remit le bâton dans ses robes. «Je ne dirai pas que ça a été un plaisir de faire affaire avec vous, Survivant, mais cela aurait pu être bien pire. Je vous en remercie.»

À voir la façon dont madame Bones agissait, Harry était déjà inquiet au sujet de l'équilibre des pouvoirs. Les autres avaient assez logiquement déduit que David Monroe avait été le principal responsable de la défaite de Voldemort, et ils le sous-estimaient donc toujours. Peut-être faudrait-il une autre crise, que Harry résoudrait bien pour une fois, avant qu'Amelia Bones ne commence à respecter son autorité. Ou même à simplement y croire… «Donc, dit Harry. Quoi que ce soit d'étrange dont vous auriez parlé si Dumbledore avait toujours été là~?»

Amelia sembla pensive. «Puisque vous me posez la question… je peux penser à trois choses. D'abord, nous ne savons absolument pas quel rituel a été utilisé pour sacrifier les Mangemorts et ressusciter Vous-Savez-Qui. Il ne correspond à aucune légende connue et les traces magiques issues du rituel ont été éradiquées. Tout ce que mes Aurors peuvent en dire, c'est que leurs têtes semblent être tombées par terre pour des raisons naturelles. Sauf Walden MacNair, tué par un feu magique après avoir lancé le sortilège de la Mort. Un rituel tout à fait mystérieux.» Elle regardait Harry Potter, et Harry Potter seulement.

Il envisagea sa réponse avec prudence. Voldemort avait dit avoir mis en place des protections contre les Retourneur de Temps, et Harry supposait donc ne pas avoir été observé par de futurs Aurors, mais au cas où… «Je ne pense pas que vous ayez besoin de trop vous en préoccuper, Madame Bones.»

La vieille sorcière sourit légèrement.

«Nous ne pouvons pas avoir l'air de traiter la mort de tant de nobles à la légère, Harry Potter. Lorsque j'ai entendu votre rapport sur les derniers instants de David, je me suis assurée d'envoyer des enquêteurs \emph{de confiance}. De fait, les Aurors Nobbs et Colon, très respectés même à l'extérieur de mon département. Leur rapport m'a pour le moins fascinée.» Elle marqua une pause. «Il est possible qu'Augustus Rookwood ait laissé un fantôme…

--- Exorcisez-le avant qu'il parle à qui que ce soit,» dit Harry, conscient des battements accélérés de son cœur.

«Oui monsieur, dit la vieille sorcière d'un ton sec. Je décrocherai un peu l'âme et personne ne se doutera de rien lorque le fantôme n'apparaîtra pas. La deuxième chose, c'est qu'un bras humain vivant a été trouvé parmi les affaires du Seigneur des Ténèbres…

--- Bellatrix», dit Harry. Son esprit avait bondit dans le passé et établi le lien que le traumatisme avait masqué. «Je pense que c'est le bras de Bellatrix.» \emph{Lesath Lestrange n'était pas dans la liste des nouveaux orphelins.} «Oh bon sang. Elle est toujours en liberté, hein~? Pouvez-vous utiliser son bras pour la retrouver~?»

Amelia Bones semblait peinée.

«Je vois. Comme je disais, un bras humain vivant a été trouvé parmi les affaires du Seigneur des Ténèbres mais son incinération s'est déroulée sans accroc.

--- Quel \emph{imbécile}…» Harry s'interrompit. «Non, \emph{pas} un imbécile. Parce que c'est la politique du département que de détruire immédiatement tout objet ténébreux. À cause de mauvais souvenirs liés à des anneaux qui auraient vraiment dû être immédiatement jetés dans des volcans. C'est ça~?»

Maugrey et Amelia hochèrent la tête. «Bien vu, mon petit», dit Maugrey.

Il semblait narrativement inévitable que les bêtises passées de Harry reviennent un jour le hanter atrocement, mais ça valait le coup d'essayer de tordre le fil de l'histoire. «Je suppose que vous y avez déjà pensé, dit Harry, mais l'étape suivante évidente est de publier l'équivalent d'un mandat d'arrêt international pour une sorcière émaciée avec un bras gauche manquant. Oh, et ajoutez vingt-cinq-mille Gallions de ma fortune personnelle - ça va, madame la directrice, faites-moi confiance - à la récompense qui sera offerte.

--- Bien dit.» La vieille sorcière se pencha un peu. «La troisième et dernière chose… vraiment troublante, et je serais curieuse d'avoir votre avis sur la question, Harry Potter. Parmi les restes humains se trouvaient la tête et le corps de Sirius Black.

--- \emph{Quoi~?} s'écria Maugrey en se levant à moitié de sa chaise. \emph{Je croyais qu'il était à Azkaban~!}

--- Il y est, dit Madame Bones. Nous avons immédiatement vérifié. Les gardes indiquent qu'il se trouve toujours dans sa cellule. Sa tête et son corps ont été transportés à la morgue de Sainte Mangouste et montrent la même cause de décès que pour les autres Mangemorts, c'est-à-dire la chute spontanée de sa tête. On m'a aussi indiqué que ce matin même, Sirius Black était assis dans un coin de sa cellule et sa balançait d'avant en arrière, la tête entre les mains. Aucun autre clone de Mangemort n'a été trouvé. Pour l'instant.»

Pendant que chacun réfléchissait, un silence comblé par des bruits d'horloges de tourbillons s'étira.

«Ah… dit Minerva. Même pour Vous-Savez-Qui, c'est impossible, non~?

--- J'aurais cru la même chose à votre âge, ma chère, dit Amelia. C'est la sixième chose la plus étrange que j'ai jamais vue.

--- Tu vois, petit~? dit Maugrey. C'est à cause de ce genre de truc que personne ne peut être assez paranoïaque, même pas moi.» L'homme balafré inclina la tête et sembla pensif~; son œil bleu clair tourbillonnait. «Frère jumeau, caché au monde~? Walpurga Black donne naissance à des jumeaux, se refuse à en tuer un, savait que le vieux Pollux l'exigerait… nan, j'y crois pas.

--- Des idées, M. Potter~? dit Amelia Bones. Ou est-ce un autre mystère dont le département ne devrait pas trop se préoccuper~?»

Harry ferma les yeux et réfléchit.

Sirius Black avait pourchassé Peter Pettigrew au lieu de fuir le pays comme le bon sens le suggérait.

Black avait été retrouvé au milieu de la rue, entouré de cadavres, hilare.

Rien ne restait de Pettigrew, hormis un doigt.

Pettigrew avait été un espion des gentils, pas un agent double mais un infiltré, quelqu'un qui découvrait des secrets.

L'une des théories de la conspiration qui le concernait disait que sa capacité à découvrir des secrets même pendant sa scolarité à Poudlard indiquait qu'il avait été un Animagus.

Les Détraqueurs absorbaient toute la magie qui les entourait.

Le professeur Quirrell avait parlé d'un certain type de magie qui réarrangeait la chair comme un forgeron Moldu réarrangeait le métal avec un marteau et des pinces…

Harry ouvrit les yeux.

«Peter Pettigrew était-il secrètement un métamorphomage~?»

Le visage d'Amelia Bones changea d'expression. Elle émit un petit croassement et s'enfonça dans sa chaise.

«Oui, effectivement… dit lentement Minerva. Pourquoi~?

--- Sirius Black a lancé un Confundus sur Peter Pettigrew, dit Harry d'un ton patient, pour le forcer à changer de forme et à prétendre être Black. Quand le Confundus s'est dissipé, Peter était à Azkaban et ne pouvait plus se retransformer. Les Aurors ont l'habitude d'entendre les prisonniers dire n'importe quoi dans l'espoir de sortir, alors ils n'ont pas écouté Peter Pettigrew quand il a hurlé encore et encore jusqu'à en perdre la voix.»

Même Maugrey Fol-Œil parut horrifié.

«Rétrospectivement», dit la voix de Harry, comme sur pilote automatique, «vous auriez dû trouver étrange de pouvoir envoyer ne serait-ce qu'\emph{un seul} Mangemort à Azkaban sans procès.

--- Nous pensions que Malfoy était distrait, dit la vieille sorcière. Qu'il essayait seulement de se sauver lui-même. Nous avons réussi à avoir d'autres Mangemorts, Bellatrix par exemple…»

Harry hocha la tête. Il avait l'impression que son cou et sa tête étaient accrochés à des fils de marionnette.

«Le serviteur le plus fanatisé, le plus dévoué du Seigneur des Ténèbres, un noyau d'opposition naturel pour quiconque opposé au contrôle des Mangemorts par Lucius. Vous avez cru qu'il était distrait.

--- Sortez-le de là», dit Minerva McGonagall. Sa voix se mua en cri. «\emph{Sortez-le de là~!}»

Amelia Bones se releva vivement, pivota vers la cheminée…

«Arrêtez.»

Tout le monde regarda Harry, abasourdi - surtout Minerva McGonagall.

Quelque chose semblait s'être emparé de Harry.

«Nous devons parler de quatre autres choses. Un homme innocent est à Azkaban depuis dix ans, huit mois et quatorze jours. Il peut y rester quelques minutes de plus. Ces quatre chose sont vraiment importantes.

--- Vous… murmura AMelia Bones. Vous ne devriez pas essayer d'être ainsi, pas à votre âge…

--- Premièrement. Je pense que nous devrions consulter les rapports de police au sujet de tous les autres Mangemort envoyés à Azkaban \emph{pendant que Lucius était distrait}. Pourrez-vous avoir fait ça ce soir~?

--- Dans l'heure,» dit Amelia Bones, le teint gris.

Harry hocha la tête.

«Deuxièmement. Azkaban, c'est fini. Vous allez devoir vous préparer à déplacer les prisonniers vers Nurmengard ou d'autres prisons sans Détraqueurs, et à les soigner de leur exposition à ceux-ci.

--- Je…» dit Amelia. La vieille sorcière semblait écrasée, diminuée. «Je… je ne pense pas que même avec ce… scandale, que le reste du Magenmagot pliera… et les Détraqueurs doivent être nourris, pas autant que ce que nous l'avons fait, mais des victimes doivent leur être fournies ou ils erreront et se nourriront d'innocents…

--- Peut importe ce que le Magenmagot dit, répondit Harry. Parce que…» il s'étrangla. «Parce que…» il inspira profondément et se reprit. Il pensait à présent pouvoir observer la forme du futur immédiat, étiré devant lui comme un chemin d'or illuminé par le soleil. \emph{Cela aussi était-il écrit, dans ce livre du Temps que je ne dois pas lire~?} «Parce que si je devine juste, alors très bientôt, Hermione Granger, la Ressuscitée, va se rendre à Azkaban et détruira tous les Détraqueurs qui s'y trouvent.

--- Impossible~! cracha Maugrey Fol-Œil.

--- Par Merlin, murmura Amelia Bones. Oh, par Merlin. C'est ce qui est arrivé au Détraqueur que Dumbledore a 'perdu'. C'est pour ça qu'ils ont peur de vous… et d'elle aussi, maintenant~?» Sa voix trembla. «Qu'est-ce que… qu'est-ce que c'est…»

\emph{Si Hermione pense que la Mort peut être vaincue.}

\emph{Qu'elle l'ait cru ou pas avant, elle y croira, maintenant.}

«J'aimerais beaucoup avoir un Portoloin vers Azkaban…» sa voix se brisa à nouveau. Des larmes coulaient le long de ses joues.

\emph{Elle ne peut pas mourir. J'ai son Horcruxe.}

\emph{Mais Hermione n'a pas besoin de le savoir. Pas avant une semaine.}

\emph{Si elle est prête à risquer sa vie pour mettre un terme à…}

«Même si je pense… qu'elle s'y rendra peut-être par ses propres moyens…

--- Harry~?» dit la directrice, Minerva McGonagall.

Harry pleurait et respirait par à-coups bruyants. Mais il ne s'arrêta pas de parler. Quelque part, pendant qu'il pleurait, Peter Pettigrew attendait.

Quelque part, pendant qu'il pleurait, tout le monde attendait.

«Troisièmement. Quelque part à l'intérieur de Poudlard. En un lieu très bien défendu. Mais où les cas d'urgences pourront être amenés par Portoloins, juste devant les remparts. Il y aura un hô… hôpital de hau… haute sécurité. Avec des gardes très puissants qui auront prononcé des Serments Inviolables, je me, je me fiche de leur coût, cela n'a vraiment plus d'importance. Et… et Alastor Maugrey va concevoir les plans de sécurité et sera aussi paranoïaque qu'il veut, sans être contraint par un budget, ni par le bon sens. Mais il devra ouvrir ses portes \emph{bientôt}.» Il ne pouvait pas s'arrêter de parler pour pleurer.

«Harry, dit la directrice, ils pensent tous les deux que vous êtes devenu fou, ils ne te connaissent pas assez. Tu dois ralentir et expliquer.»

Harry préféra plonger la main dans sa bourse et dessiner des lettres avec ses doigts pour en sortir avec peine un morceau d'or de cinq kilos, plus gros que son poing~; le fruit de ses expériences du matin même. Il fit un bruit sourd en tombant sur la table.

Maugrey s'en saisit et le toucha de sa baguette avant d'émettre un son guttural incompréhensible.

«C'est votre budget de départ, Alastor, si vous avez besoin d'argent tout de suite. Nicolas Flamel n'a pas créé la Pierre Philosophale, il l'a volée~; Dumbledore ignorait son histoire secrète, mais Monroe, non. Quand on sait s'en servir, la Pierre permet de restaurer la santé et la jeunesse d'une personne toutes les deux-cent-trente-quatre secondes. Trois-cent-soixante personnes par jour. Cent trente-quatre-mille restaurations par an. Cela devrait suffire pour empêcher tous les sorciers, et tous les gobelins et elfes de maisons et qui que se soit de mourir. De vieillesse ou d'autre chose.» Harry n'avait cesse d'essuyer ses larmes. «En comptant tous ceux qu'il aurait pu sauver, Flamel avait plus de sang sur ses mains que cent Voldemort. Pendant tout ce temps, Maugrey, la Pierre Philosophale aurait pu soigner vos cicatrices, vous rendre votre jambe, si seulement Flamel l'avait voulu. Dumbledore l'ignorait. Je suis sûr qu'il l'ignorait.» Harry eut un faible sourire. «Je n'arrive pas à vous imaginer jeune, madame Bones, mais je suis sûr que ça vous irait bien. Cela vous donnera l'énergie nécessaire pour empêcher le Magenmagot de me causer des ennuis~; parce que s'ils se mettent en tête qu'ils peuvent toucher à la Pierre, la restreindre, taxer son usage, je m'en fiche, Poudlard fera sécession et deviendra un pays. Madame la directrice, Poudlard ne dépend plus du ministère ni pour ses finances ni pour son approvisionnement. Vous pouvez modifier les programmes éducatifs comme bon vous semble. Je pense qu'on pourrait vouloir ajouter des cours de niveau supérieur, surtout en études Moldues.

--- \emph{Ralentissez~!} dit Minerva McGonagall.

--- Quatrièmement…» dit Harry, et il se tut.

\emph{Quatrièmement. Commencez à préparer une révocation en règle du Code du Secret Magique et la création massive de centres de soin magique pour le monde Moldu. Ceux qui s'opposeront à cette politique pourront se voir refuser les services de la Pierre…}

Les lèvres de Harry ne pouvait pas bouger. Elle ne refusaient pas, elle ne \emph{pouvaient pas}.

Avec six-milliards de Moldus pour appliquer leur imagination au problème de la magie…

La Métamorphose de l'antimatière n'était qu'une idée. Ce n'était même pas la pire. Il y avait aussi les trous noirs et les strangelets chargés négativement. Et si les trous noirs ne pouvaient pas être Métamorphosés parce qu'ils \emph{n'existaient pas} dans un rayon spatial défini magiquement, on pouvait toujours Métamorphoser beaucoup d'armes nucléaires, ou une Peste Noire qui se reproduirait avant que la Métamorphose ne se dissipe et Harry n'y avait pas réfléchit pendant cinq minutes mais ça n'avait pas d'importance parce qu'il y avait déjà assez réfléchi. Quelqu'un y penserait, quelqu'un parlerait, quelqu'un essaierait. La probabilité était si élevée que c'était presque une certitude.

Que se passerait-il si on métamorphosait un millimètre cube de quarks up, juste les quarks up, aucun quark down pour s'y lier~? Harry l'ignorait, et les quarks up existaient certainement déjà. Il suffirait qu'un seul né-Moldu ayant en tête le noms des quarks s'y essaie. Peut-être que c'était \emph{ça}, le compte à rebours avant la fin prophétisée du monde.

Harry aurait essayé de nier cette idée, de la rationaliser.

Il ne pouvait pas faire ça non plus.

Ce n'était pas ce que Harry Potter aurait fait.

Tout comme l'eau coulait, Harry Potter ne prenait pas de risque lorsqu'il s'agissait de ne pas détruire le monde.

«Quatrièmement~?» dit Amelia Bones, avec l'air de s'être faite frapper en plein visage et à plusieurs reprises par une planète. «\emph{Qu'est-ce qu'il y a encore~?}

--- Non, rien,» dit Harry. Sa voix ne se brisa pas. Il ne sanglota pas. Il pouvait encore sauver des vies, et c'était plus important. «Rien. Présidente-sorcière Bones, je vous ai remis la régence du Magenmagot. Utilisez ce poste pour annoncer au monde que les pouvoirs curatifs de la Pierre seront bientôt disponibles pour tous, et qu'en attendant, tous les patients mourants devront être maintenus en vie à tout prix, quelle que soit la magie requise pour y parvenir. Cette annonce est votre priorité absolue. Quand vous en aurez fini, vous pourrez sauver Peter Pettigrew et dire à votre ancien département de se préparer à fermer Azkaban. Puis arrangez-vous pour que quelqu'un prépare une liste des Mangemorts emprisonnés, ce qui a été dit à leurs procès, et si Lucius semblait étrangement peu intéressé par leur défense. C'est tout.»

Amelia Bones se retourna sans dire un mot et courut vers la cheminée comme si elle avait elle-même été en feu.

«Et que quelqu'un», dit Harry d'une voix qui se brisait à nouveau maintenant que tout avait été mis en mouvement, maintenant que pleurer ne lui faisait plus perdre de temps, même si la majorité des vies en jeu s'étaient avérées ne pas pouvoir être sauvées pour l'instant, «que quelqu'un prévienne Remus Lupin.» 

%  LocalWords:  arry wibblers Poot undeath Nobbs Rookwood literarily Pollux
%  LocalWords:  Walpurga buyin
