\chapter{Croyance en la croyance}

\lettrine[ante=<<~]{E}{t} puis Jeanne, qui était une Cracmol~>>, dit le portrait d'une petite jeune femme coiffée d'un chapeau cousu d'or.

Drago le nota. Ça n'en faisait que vingt-huit mais il était temps de retourner voir Harry.

L'anglais ayant beaucoup changé, il avait dû demander à d'autres portraits de l'aider à traduire, mais les plus âgés avaient décrit des sorts de première année qui ressemblaient fort à ceux qui existaient aujourd'hui. Drago en avait reconnu à peu près la moitié, et les autres n'avaient pas eu l'air plus puissants.

La sensation de malaise dans son estomac avait augmenté à chaque réponse jusqu'à ce qu'enfin, incapable d'en entendre plus, il ait préféré partir poser à d'autres portraits les étranges questions de Harry au sujet des mariages de Cracmols. Les cinq premiers portraits avaient prétendu ne connaître personne de ce genre, et Drago avait fini par demander à ces portraits de demander à \emph{leurs} connaissances de demander à \emph{leurs} connaissances si elles en connaissaient, et il était enfin parvenu à trouver quelques personnes vraiment prêtes à admettre qu'elles étaient amies avec des Cracmols.

(Le Serpentard en première année avait expliqué qu'il travaillait sur un projet important avec un Serdaigle et que le Serdaigle lui avait dit qu'ils avaient besoin de cette information et qu'il s'était ensuite enfui sans expliquer pourquoi. Cela lui avait valu beaucoup de regards compatissants).

Les pieds de Drago étaient lourds alors qu'il marchait dans les couloirs de Poudlard. Il aurait dû courir, mais il ne se sentait pas capable de trouver l'énergie nécessaire. Il continuait de penser qu'il ne voulait pas savoir, qu'il ne voulait rien avoir à faire avec ça, qu'il ne voulait pas de cette responsabilité, laissons juste Harry Potter faire ce qu'il y avait à faire, si la magie disparaissait, laissons juste Harry Potter s'en occuper…

Mais Drago savait que ça n'aurait pas été une sage décision.

Froids les donjons de Serpentard, gris les murs de pierre, Drago aimait généralement l'atmosphère, mais ils lui évoquaient à présent l'idée que quelque chose en train de disparaître avec bien trop de force.

Sa main sur la poignée de la porte, Harry Potter déjà à l'intérieur, attendant, vêtu de sa houppelande à capuche.

<<~Les anciens sorts de première année, dit Harry Potter. Qu'as-tu découvert~?

--- Ils ne sont pas plus puissants que les sorts qu'on utilise aujourd'hui.~>>

Le poing de Harry Potter frappa le bureau, fort. <<~Bon sang. Très bien. Drago, mon expérience à moi a été un échec. Il y a quelque chose nommé l'Interdit de Merlin -~>>

Drago se frappa sur le front, se rappelant soudain.

<<~- qui empêche quiconque d'apprendre quoi que ce soit au sujet de sorts puissants à partir de livres, même si on trouve les notes d'un sorcier puissants et qu'on les lit, elles n'auront alors aucun sens, car elles doivent passer d'un esprit vivant à un autre. Je n'ai pu trouver aucun sort puissant dont les instructions seraient disponibles mais que nous ne serions pas capables de lancer. Mais si on ne peut pas les trouver dans les vieux livres, pourquoi est-ce que quiconque s'embêterait à transmettre ces sorts par le bouche à oreille après qu'ils eurent cessé de fonctionner~? As-tu obtenu les données sur les couples de Cracmols~?~>>

Drago commença à tendre le parchemin -

Mais Harry Potter leva une main. <<~Loi de la science, Drago. D'abord, je te dis la théorie et la prédiction. Ensuite, tu me montres les données. Comme ça, tu sais que je n'invente pas juste une théorie qui convient~; tu sais que la théorie a réellement prédit les données \emph{à l'avance}. Je dois de toute façon te l'expliquer, donc autant te l'expliquer \emph{avant} que tu ne me montres les données. C'est la règle. Alors mets ta cape et asseyons-nous.~>>

Harry Potter s'assit à un bureau sur lequel des tonnes de papier brouillon étaient disposées. Drago enfila sa cape, tirée de son cartable, et il s'assit face à Harry et jeta un regard confus vers les bouts de papier. Ils étaient disposés en deux rangées et les rangées faisaient à peu près vingt feuillets de long.

<<~Le secret du sang~>>, dit Harry Potter, une expression intense dessinée sur le visage, <<~est une chose nommée acide désoxyribonucléique. Tu ne prononces pas ce nom devant quelqu'un qui n'est pas un scientifique. L'acide désoxyribonucléique est la recette qui dit à ton corps comment grandir, que tu as deux bras et deux jambes, si tu es petit ou grand, si tu as des yeux marrons ou verts. C'est un objet physique, tu peux le \emph{voir} si tu as des microscopes, qui sont comme des télescopes mais qui regardent les choses qui sont très petites au lieu des choses très lointaines. Et cette recette a deux copies de tout, tout le temps, au cas où l'une des copies serait cassée. Imagine deux longues rangées de bouts de papier. À chaque emplacement d'une rangée, il y a deux bouts de papiers, et quand tu as des enfants, ton corps choisit un bout de papier au hasard de chaque emplacement de la rangée, et le corps de la mère fera pareil, et comme ça l'enfant obtient deux bouts de papier de chaque emplacement de la rangée. Deux copies de tout, une de la mère, une du père, et quand tu as des enfants, tu leur donne un bout de papier choisi au hasard à chaque emplacement.~>>

Tandis qu'il parlait, les doigts de Harry passaient sur les paires de bouts de papier, pointant l'un des éléments de la paire quand il disait <<~de ta mère~>> et l'autre quand il disait <<~de ton père.~>> Et alors qu'il parlait du fait de prendre des bouts de papier au hasard, sa main fit surgir une Mornille de sa robe et la fit voltiger~; Harry regarda la pièce, puis il pointa le bout de papier du haut. Le tout sans interrompre son discours.

<<~Maintenant, quand il s'agit de savoir si on va être grand ou petit, il y a \emph{beaucoup} d'emplacements dans la recette qui provoquent de \emph{petits} changements. Donc si un père grand épouse une mère petite, l'enfant obtiendra quelques bouts de papiers disant 'grand' et d'autres disant 'petit', et l'enfant se retrouvera probablement avec une hauteur moyenne. Mais pas toujours. L'enfant pourrait par chance obtenir beaucoup de bouts de papier disant 'grand' et peu disant 'petit'~; il grandirait alors beaucoup. Il pourrait y avoir un père grand qui posséderait cinq bouts de papier disant 'grand' et une mère grande qui posséderait cinq bouts de papiers disant 'grand' et par une chance incroyable l'enfant obtiendrait les \emph{dix} bouts de papier disant 'grand' et il se retrouverait être plus grand qu'eux deux. Tu comprends~? Le sang n'est pas un fluide parfait, il ne se mélange pas parfaitement. L'acide désoxyribonucléique est fait de beaucoup de petits morceaux, comme un verre qui serait rempli de billes plutôt que d'eau. C'est pour ça qu'un enfant n'est pas toujours exactement dans la moyenne de ses parents.~>>

Drago écoutait, bouche bée. Nom de Merlin, comment les Moldus avaient-ils fait pour découvrir tout ça~? Ils pouvaient \emph{voir} la recette~?

<<~Maintenant, dit Harry Potter, imagine que, comme pour la taille, il y ait beaucoup de petits emplacements dans la recette où un bout de papier disant 'magique' ou 'pas magique' pourrait se trouver. Si tu as assez de bouts de papier disant 'magique', tu es un sorcier. Si tu as \emph{beaucoup} de ce genre de bouts de papier, tu es un sorcier puissant. Si tu en as trop peu, tu es un Moldu, et entre les deux, tu es un Cracmol. Alors, quand deux Cracmols se marient, la plupart du temps les enfants seront aussi des Cracmols, mais de temps en temps, un enfant sera chanceux et obtiendra la majeure partie des bouts de papier magiques de son père \emph{et} la majeure partie des bouts de papier magiques de sa mère, et il sera assez puissant pour être un sorcier. Mais probablement pas un sorcier très puissant. Si tu commençais avec beaucoup de sorciers puissants et que tu les faisais s'épouser, alors ils resteraient puissants. Mais s'ils commençaient à épouser des nés-Moldu à peine magiques, ou des Cracmols… tu comprends~? Le sang ne se mélangerait pas parfaitement, car c'est comme ça qu'il se comporte~: comme un verre de billes et pas comme un verre d'eau. Dans ce cas, on verrait des sorciers puissants de temps en temps, quand par chance, ils obtiendraient beaucoup de papiers magiques. Mais ils ne seraient pas aussi puissants que les sorciers les plus puissants qui les auraient précédés.~>>

Drago hocha lentement la tête. Il avait déjà entendu cette idée auparavant. Il y avait une surprenante beauté à voir à quel point ça correspondait à l'explication de Harry.

<<~\emph{Mais}, dit Harry, ce n'est qu'\emph{une} hypothèse. Imagine qu'au lieu de ça, il y ait \emph{un seul} endroit dans la recette qui détermine si tu es un sorcier. \emph{Un} seul endroit où un bout de papier peut dire 'magique' ou 'pas magique'. Et il y a toujours deux copies de tout. Alors il n'y a que trois possibilités. Les deux copies peuvent dire 'magique'. Une copie peut dire 'magique' et l'autre copie peut dire 'pas magique'. Ou les deux copies peuvent dire 'pas magique'. Sorciers, Cracmols et Moldus. Les nés-Moldus ne seraient pas vraiment nés de Moldus, ils seraient nés de deux Cracmols, deux parents ayant chacun une copie magique et qui auraient grandi dans le monde Moldu. Maintenant imagine qu'une sorcière épouse un Cracmol. Chaque enfant obtiendrait toujours de sa mère un bout de papier disant 'magique'. Peu importe quel bout a été choisi au hasard puisque les deux disent 'magique'. Mais comme quand on jette une pièce, la moitié du temps, l'enfant obtiendrait le papier disant 'magique' de son père, et l'autre moitié du temps, l'enfant obtiendrait le bout de papier disant 'non magique' de son père. Quand une sorcière épouserait un Cracmol, le résultat ne serait pas 'beaucoup d'enfants faibles en magie'. La moitié des enfants serait des sorciers et des sorcières tout aussi puissants que leur mère, et l'autre moitié serait des Cracmols. Parce que s'il n'y a qu'\emph{un} endroit dans la recette qui fait de toi un sorcier, alors la magie n'est pas un verre de billes qui se mélangent. C'est plutôt comme une seule bille magique, comme une pierre de sorcier.~>>

Harry disposa trois paires de papiers côte à côte. Sur une paire, il écrivit 'magique' et 'magique'. Sur une autre paire il écrivit 'magique' sur le papier du haut seulement. Et il laissa la troisième paire blanche.

<<~Auquel cas, dit Harry, soit tu as deux pierres, soit tu n'en as aucune. Soit tu es un sorcier, soit tu n'en es pas un. Les sorciers puissants le deviendraient en étudiant plus dur et en pratiquant plus. Et si le pouvoir des sorciers devient \emph{essentiellement} moins puissant, pas parce que des sorts sont perdus, mais parce que les gens ne peuvent plus les jeter… alors peut-être qu'on mange de mauvais aliments ou quelque chose comme ça. Mais si c'est devenu régulièrement pire pendant les huit-cents dernières années, alors ça pourrait vouloir dire que la magie elle-même disparaît.~>>

Harry disposa deux autres paires de papiers côte à côte et sortit une plume. Bientôt, chaque paire eut un bout de papier disant 'magique' et l'autre laissé vide.

<<~Ce qui m'amène à ma prédiction, dit Harry. Ce qui se passe quand deux Cracmols se marient. Jette une pièce deux fois. Elle peut être face puis face, face puis pile, pile puis face, ou pile puis pile. Donc un quart du temps tu obtiendras deux face, un quart du temps tu obtiendras deux piles, et la moitié du temps tu obtiendras une face et une pile. Pareil si deux Cracmols se marient. Un quart des enfants sera magique et magique, et sera des sorciers. Un quart sera pas-magique et pas-magique, et sera des Moldus. L'autre moitié sera Cracmole. C'est un motif très ancien et très classique. Il a été découvert par Gregor Mendel, louée soit sa mémoire, et c'était le premier indice jamais découvert au sujet de la façon dont la recette fonctionnait. Toute personne ayant la moindre connaissance en science du sang reconnaîtra instantanément ce motif. Il ne sera pas exact, pas plus que si tu jetais une pièce quarante fois tu n'obtiendrais pas exactement dix paires de deux faces. Mais si tu as obtenu sept ou treize sorciers parmi quarante enfants, alors c'est un fort indicateur. Voilà le test que je t'ai fait faire. Maintenant voyons tes données.~>>

Et avant que Drago ne puisse même penser, Harry Potter prit le parchemin de ses mains.

La gorge de Drago était très sèche.

Vingt-huit enfants.

Il ne se souvenait pas du nombre exact mais il était à peu près sûr qu'environ un quart avaient été sorciers.

<<~Six sorciers sur vingt-huit enfants, dit Harry Potter après un moment. Eh bien voilà. Et il y a huit siècles, les élèves de première année jetaient les mêmes sorts au même niveau de pouvoir. Ton test et mon test ont produit le même résultat.~>>

Il y eut un long silence dans la salle.

<<~Et maintenant~?~>> murmura Drago.

Il n'avait jamais été si terrifié.

<<~Ce n'est pas encore définitif, dit Harry Potter. Mon expérience a échoué, tu te souviens~? J'ai besoin que tu conçoives un autre test, Drago.

--- Je, je…~>> dit Drago. Sa voix se brisait. <<~Je ne peux pas faire ça Harry, c'est trop pour moi.~>>

Le regard de Harry était ardent. <<~Si, tu peux, parce que tu dois le faire. Moi, j'y ai déjà réfléchi après avoir découvert l'Interdit de Merlin. Drago, y a-t-il le moindre moyen permettant d'observer directement la force de la magie~? Une méthode qui n'aurait rien à voir avec le sang des sorciers ou les sorts qu'on apprend~?~>>

Le cerveau de Drago était juste vide.

<<~Tout ce qui affecte la magie affecte les sorciers, dit Harry. Mais dans ce cas on ne peut pas dire si ça vient des sorciers ou de la magie. Qu'est-ce que la magie affecte qui \emph{n'est pas} un sorcier~?

--- Les créatures magiques, évidemment~>>, dit Drago sans même y penser.

Harry Potter sourit lentement. <<~Drago, c'est \emph{génial}.~>>

\emph{C'était le genre de question stupide qu'on ne poserait que si on avait été élevé par des Moldus}.

Puis le malaise dans l'estomac de Drago devint encore pire quand il comprit ce que cela voudrait dire si les créatures magiques \emph{devenaient} plus faibles. Ils sauraient alors à coup sûr que la magie disparaissait, et une partie de Drago était déjà certaine que c'était exactement ce qu'ils allaient découvrir. Il ne voulait pas voir ça, il ne voulait pas savoir…

Harry Potter était déjà à mi-chemin de la porte. <<~\emph{Viens}, Drago~! Il y a un portrait pas loin, on lui demandera juste d'aller chercher quelqu'un de vieux, et on saura tout de suite~! On a des capes, si quelqu'un nous voit, nous pourrons simplement nous enfuir~! Allons-y~!~>>

\later

Ça ne prit pas longtemps.

C'était un large portrait, mais les trois personnes qui s'y trouvaient y avaient l'air plutôt à l'étroit. Il y avait un homme d'âge moyen du douzième siècle, habillé de pans de tissu noir~; il parlait à une jeune femme à l'air triste du quatorzième siècle dont les cheveux semblaient constamment faire des frisottis sur sa tête, comme si elle avait été chargée par un sort d'électricité statique~; et elle parlait à un vieil homme digne et desséché du dix-septième siècle doté d'un nœud papillon en or massif~; et lui ils pouvaient le comprendre.

Ils posèrent des questions au sujet des Détraqueurs.

Ils posèrent des questions au sujet des phénix.

Ils posèrent des questions au sujet des dragons et des trolls et des Elfes de maison.

Harry avait froncé les sourcils, faisant remarquer que les créatures ayant le plus besoin de magie pouvaient tout à fait être en train de disparaître, et il avait demandé quelles étaient les créatures magiques les plus puissantes que l'on ait jamais connu.

Il n'y avait rien d'inhabituel dans la liste, mis à part une espèce de créatures sombres nommées l'écorcheur d'esprit, et le traducteur avait fait remarquer qu'ils avaient finalement été exterminés par Harold Shea, et ces créatures ne semblaient pas être à moitié aussi effrayantes que les Détraqueurs.

Apparemment, les créatures magiques étaient aussi puissantes aujourd'hui qu'elles l'avaient toujours été.

Le malaise dans l'estomac de Drago se calma, et il se sentit juste dérouté.

<<~Harry~>>, dit Drago au beau milieu d'une phrase du vieil homme qui traduisait la liste des onze pouvoirs des yeux du tyrannoeil, <<~qu'est-ce que ça veut dire~?~>>

Harry leva un doigt et le vieil homme acheva sa liste.

Puis Harry remercia tous les portraits pour leur aide - Drago, presque entièrement en pilote automatique, le fit aussi, avec bien plus de grâce - et ils retournèrent dans la salle de classe.

Et Harry sortit le parchemin original avec les hypothèses, et il commença à griffonner.
\vskip 1\baselineskip plus .5\textheight minus 1\baselineskip

\savetrivseps
\setlength{\topsep}{0pt}
\setlength{\partopsep}{0pt}

\begin{centering}
\begin{samepage}
\scshape Observation~:

\itshape La sorcellerie n'est pas aussi puissante qu'elle ne l'était quand Poudlard a été fondé. \end{samepage}

\vskip 1\baselineskip plus .5\textheight minus 1\baselineskip

\begin{samepage}
\scshape Hypothèses~:

\itshape
        \begin{enumerate}[1.]
                \firmlist
                \setlength{\leftmargin}{\parindent}
                \setlength{\rightmargin}{\parindent}
        \item La magie elle-même disparaît.
        \item Les sorciers se métissent avec les Moldus et les Cracmols.
        \item Le savoir permettant de jeter des sorts puissants se perd.
        \item Les sorciers ne mangent pas ce qu'il faut quand ils sont enfants, ou quelque chose d'autre à part le sang les fait devenir plus faible.
        \item La technologie Moldue interfère avec la magie (depuis 800 ans~?).
        \item Les sorciers plus puissants ont moins d'enfants (Drago = fils unique~? Vérifier si trois sorciers puissants, Quirrell / Dumbledore / Seigneur des Ténèbres ont eu des enfants).)
        \end{enumerate}
\end{samepage}

\vskip 1\baselineskip plus .5\textheight minus 1\baselineskip

\begin{samepage}
\scshape Tests~:
\itshape
        \begin{enumerate}[A.]{
                \firmlist
                \setlength{\leftmargin}{\parindent}
                \setlength{\rightmargin}{1cm}}
        \item Y a-t-il des sorts que l'on connaît mais qu'on ne peut pas jeter (1 ou 2) ou des sorts perdus qu'on ne connaît plus (3)~? {\scshape Résultat~:} Peu concluant à cause de l'Interdit de Merlin. Pas de sort connu non jetable, mais ils pourraient simplement ne pas avoir été transmis.

        \item Les anciens élèves de première année jetaient-ils le même genre de sort qu'aujourd'hui, au même niveau de puissance~? (Faible élément de preuve pour 1 et contre 2, mais le sang pourrait aussi faire disparaître uniquement la sorcellerie puissante).{\scshape Résultat~:} Les sorts de première année sont aussi puissants qu'avant.

        \item Test supplémentaire qui distingue entre 1 et 2 en utilisant les connaissances scientifiques sur le sang, j'expliquerai plus tard. {\scshape Résultat~:} Il n'y a qu'un seul endroit dans la recette qui fait de quelqu'un un sorcier, soit on a deux papiers disant 'magique', soit on n'a rien.

        \item Les créatures magiques perdent-elles leurs pouvoirs~? Distingue 1 de (2 ou 3). {\scshape Résultat~:} Les créatures magiques semblent être aussi puissantes qu'elles l'ont toujours été.
        \end{enumerate}
\end{samepage}
\end{centering}
\vskip 1\baselineskip plus .5\textheight minus 1\baselineskip
\restoretrivseps

<<~A a échoué, dit Harry Potter. B est un faible élément de preuve pour 1 au détriment de 2. C falsifie 2. D falsifie 1. 4 était peu probable et B contredit aussi 4. 5 était peu probable et D le contredit. 6 est falsifié en même temps que 2. Ce qui nous laisse 3. Interdit de Merlin ou pas, je n'ai trouvé aucun sort connu ne pouvant être jeté. Donc quand on additionne le tout, on dirait que le savoir est perdu.~>>

Et la trappe se referma.

Dès que la panique fut partie, dès que Drago eut compris que la magie ne disparaissait \emph{pas}, il n'eut besoin que de cinq secondes pour comprendre.

Drago se repoussa loin du bureau et se leva si vite que sa chaise glissa sur le sol dans un bruit de raclement et se renversa.

<<~Donc tout ça n'était qu'un piège stupide.~>>

Harry Potter le fixa un moment, toujours assis. Quand il parla, sa voix était douce.

<<~C'était un test honnête, Drago. Si le résultat avait été différent, je l'aurais accepté. Ce n'est pas quelque chose au sujet duquel je tricherais. Jamais. Je n'ai pas regardé tes données avant d'avoir fait mes prédictions. Je t'ai dit à l'avance que l'Interdit de Merlin annulait la première expérience -

--- Oh~>>, dit Drago, la colère commençant à poindre dans sa voix, <<~tu ne savais pas comment tout ça finirait~?

--- Je ne \emph{savais} rien que tu ne saches pas toi-même~>>, dit Harry, toujours doucement. <<~J'admets l'avoir suspecté. Hermione Granger est trop puissante, elle aurait dû être à peine magique, et ce n'est pas le cas. Comment une née-Moldue pourrait-elle être la meilleure lanceuse de sorts de Poudlard~? Et elle obtient aussi les meilleures notes à ses rédactions, c'est trop de coïncidences pour qu'une seule fille soit la meilleure en magie \emph{et} en cours, à moins qu'il n'y ait une cause commune. L'existence de Hermione Granger laissait entendre qu'il n'y avait qu'une seule chose qui déterminait que l'on soit sorcier ou non, quelque chose qu'on avait ou qu'on n'avait pas, et que les différences de pouvoir venaient de la quantité de savoir et de travail. Et il n'y avait pas non plus différents sous-groupes de sang-pur et de Moldus, et ainsi de suite. Par bien trop d'aspects, le monde ne ressemblait pas à ce dont il aurait eu l'air si tu avais eu raison. Mais Drago, je n'ai rien vu que tu n'aurais pas pu toi-même remarquer. Je n'ai réalisé aucun test sans t'en parler. Je n'ai pas triché, Drago. Je voulais que nous trouvions la réponse ensemble. Et avant que tu ne le dises, je n'avais jamais pensé que la magie pouvait disparaître du monde. Pour moi aussi, c'était une idée effrayante.

--- C'est ça~>>, dit Drago. Il faisait beaucoup d'efforts pour contrôler sa voix, pour pas se contenter de crier sur Harry. <<~Tu prétends que tu ne vas pas courir le dire à qui que ce soit d'autre.

--- Pas sans t'avoir consulté avant~>>, dit Harry. Il ouvrit ses mains et fit un geste suppliant. <<~Drago, j'essaie d'être aussi gentil que possible, mais \emph{le monde s'est juste avéré ne pas être ainsi.}

--- Très bien. Alors toi et moi en avons fini. Je vais juste m'en aller et oublier que tout ça ait jamais eu lieu.~>>

Drago fit demi-tour, sentant la sensation de brûlure dans sa gorge, le sentiment de trahison, et c'est alors qu'il se rendit compte qu'il avait \emph{vraiment} bien aimé Harry Potter, et cette pensée ne le ralentit pas un instant tandis qu'il marchait vers la porte de la salle.

Et la voix de Harry Potter lui parvint, maintenant plus forte, et inquiète~:

<<~Drago… tu ne \emph{peux pas} oublier. Ne comprends-tu pas~? C'était ton sacrifice.~>>

Drago s'arrêta au milieu d'une enjambée et pivota. <<~De \emph{quoi} parles-tu~?~>>

Mais il y avait déjà un froid glacé dans la colonne vertébrale de Drago.

Il le sut avant même que Harry Potter ne le dise.

<<~Pour devenir un scientifique. Tu as remis en question une de tes croyances, pas seulement une petite croyance, mais quelque chose qui avait une grande importance pour toi. Tu as fait des expériences, amassé des données, et le résultat a montré que ta croyance était fausse. Tu as vu les résultats et tu as compris ce qu'ils signifiaient.~>> La voix de Harry Potter flanchait. <<~Souviens-toi, Drago, tu ne peux pas sacrifier une \emph{vraie} croyance de cette façon, parce que les expériences la confirmeront au lieu de la falsifier. Ton sacrifice pour devenir un scientifique a été ta \emph{fausse} croyance que le sang de sorcier se mélangeait et devenait plus faible.

--- \emph{Ce n'est pas vrai}~! dit Drago. Je n'ai pas sacrifié cette croyance. Je le crois toujours~!~>> Sa voix devenait plus forte, et le frisson empirait.

Harry Potter secoua la tête. Sa voix lui parvint comme un murmure.

<<~Drago… je suis désolé Drago, tu ne le crois \emph{pas}, plus maintenant.~>> La voix de Harry s'éleva à nouveau. <<~Je vais te le prouver. Imagine que quelqu'un te dise qu'il a un dragon dans sa maison. Tu lui dis que tu veux le voir. Il te dit que c'est un dragon invisible. Tu dis très bien, tu veux l'entendre bouger. Il te dit que c'est un dragon inaudible. Tu dis que tu vas jeter un peu de farine en l'air et voir le contour du dragon. Il dit que le dragon est perméable à la farine. Et ce qui est révélateur, c'est qu'il sait à \emph{l'avance} exactement les résultat expérimentaux pour lesquels il va devoir trouver des excuses. Il \emph{sait} que tout se passera exactement comme s'il n'y avait pas de dragon, il sait à \emph{l'avance} quelles excuses il devra inventer. Donc peut-être qu'il \emph{dit} qu'il y a un dragon. Peut-être qu'il \emph{croit} qu'il croit qu'il y a un dragon, ça s'appelle croyance en la croyance. Mais il ne le croit pas vraiment. On peut se tromper sur ce qu'on croit, la plupart des gens ne se rendent jamais comptent qu'il y a une différence entre croire quelque chose et penser qu'il est bon d'y croire.~>> Harry Potter s'était maintenant levé du bureau et avait fait quelques pas en direction de Drago. <<~Et Drago, tu ne crois plus au purisme du sang, je vais te le montrer. Si le purisme du sang est vrai, l'existence de Hermione Granger n'a aucun sens, et alors qu'est-ce qui pourrait l'expliquer~? Peut-être qu'elle est une sorcière orpheline élevée par des Moldus, comme moi~? Je pourrais aller voir Granger et lui demander de voir des photos de ses parents, pour voir si elle leur ressemble. T'attendrais-tu à ce qu'ils aient l'air différents~? Devrions-nous aller effectuer ce test~?

--- Ils l'ont sûrement mise avec des membres de sa famille éloignée, dit Drago d’une voix tremblante. Ils se ressembleront.

--- Tu vois. Tu sais déjà quel résultat expérimental tu vas devoir excuser. Si tu croyais toujours au purisme du sang, tu dirais bien sûr, allons jeter un œil, je parie qu'elle ne ressemblera pas à ses parents, elle est trop puissante pour être une vraie née-Moldue -

--- Ils l'ont \emph{sûrement} mise avec des membres de sa famille éloignée~!

--- Les scientifiques peuvent faire des tests pour vérifier si quelqu'un est vraiment l'enfant de son père. Granger le ferait probablement si je donnais assez d'argent à sa famille. \emph{Elle} n'aurait pas peur des résultats. Alors que t'attends-tu à ce que les tests montrent~? Dis-moi de les faire et nous les ferons. Mais tu sais déjà ce que les tests vont dire. Tu le sauras toujours. Tu ne pourras jamais oublier. Tu pourras \emph{souhaiter} croire au purisme du sang, mais tu \emph{t'attendras toujours} à ce que les choses se passent exactement comme s'il n'y avait qu'une seule chose qui détermine que l'on soit un sorcier ou non. C'était ton sacrifice pour devenir un scientifique.~>>

La respiration de Drago était irrégulière. <<~Te rends-tu compte de \emph{ce que tu as fait}~?~>> Drago bondit vers l'avant et saisit Harry par le col de sa robe. Sa voix devint un cri, un cri qui semblait insupportablement fort dans le silence de la salle fermée. <<~\emph{Te rends-tu compte de ce que tu as fait~?}~>>

La voix de Harry tremblait. <<~Tu avais une croyance. Cette croyance était fausse. Je t'ai aidé à le voir. Ce qui est vrai l'est déjà, l'admettre ne le rend pas pire -~>>

Les doigts de la main droite de Drago se refermèrent en un poing et cette main s'abaissa, avant de décoller, inarrêtable, et elle frappa Harry Potter à la mâchoire si fort que son corps alla s'écraser contre un bureau puis jusqu'au sol.

<<~\emph{Idiot}~! hurla Drago. \emph{Idiot~! Idiot~!}

--- Drago, murmura Harry depuis le sol, Drago, je suis désolé, je ne pensais pas que ça aurait lieu avant des mois, je ne m'attendais pas à ce que ton éveil scientifique soit aussi rapide, je pensais que j'aurais plus de temps pour te préparer, pour t'enseigner des techniques permettant d'atténuer la souffrance engendrée lorsqu'on admet qu'on a tort -

--- Et Père~?~>> dit Drago. Sa voix tremblait de rage. <<~Allais-tu le préparer, \emph{lui}, ou est-ce que tu t'en \emph{fichais} de ce qui se passerait ensuite~?

--- Tu ne peux pas \emph{lui} dire~!~>> dit Harry, sa voix montant sous l'effet de l'inquiétude. <<~Il n'est pas un scientifique~! Drago, tu as promis~!~>>

Pendant un moment, la pensée que Père ne savait pas fut un soulagement.

Puis la vraie colère commença à monter.

<<~Donc tu as prévu que je mente et que je lui dise que j'y crois toujours~>>, dit Drago, la voix tremblante. <<~Je devrai toujours lui mentir, et maintenant, je ne pourrai pas être un Mangemort quand je serai grand, et je ne pourrai même pas lui dire pourquoi.

--- Si ton père t'aime vraiment, murmura Harry depuis le sol, il t'aimera toujours, même si tu ne deviens pas un Mangemort, et on dirait bien que ton père t'aime \emph{vraiment}, Drago -

--- \emph{Ton} père adoptif est un scientifique~>>, dit Drago. Les mots sortaient, tels des couteaux acérés. <<~Si \emph{tu} n'allais pas devenir un scientifique, il t'aimerait toujours. Mais tu serais \emph{un peu moins extraordinaire} à ses yeux.~>>

Harry eut un mouvement de recul. Le garçon ouvrit la bouche, comme pour dire 'Je suis désolé', puis il la referma, semblant se raviser, ce qui était soit très intelligent, soit très chanceux, parce que s'il avait parlé, Drago aurait peut-être essayé de le tuer.

<<~Tu aurais dû me mettre en garde~>>, dit Drago. Sa voix s'éleva. <<~\emph{Tu aurais dû me mettre en garde~!}

--- Je… je l'ai fait… à chaque fois que je t'ai parlé du pouvoir, je t'ai parlé du prix. J'ai dit que tu devrais admettre que tu avais tort. J'ai dit que ce serait le chemin le plus difficile pour toi. J'ai dit que c'était le sacrifice que tout le monde devait faire pour devenir un scientifique. J'ai dit~: et si l'expérience dit une chose et que ta famille et tes amis en disent une autre -

--- \emph{Tu appelles ça une mise en garde}~?~>> Drago criait à présent. <<~\emph{Tu appelles ça une mise en garde~? Quand on fait un rituel qui exige un sacrifice permanent~?}

--- Je… je…~>> le garçon au sol avala sa salive. <<~Je suppose que je n'ai peut-être pas été clair. Je suis désolé. Mais ce qui peut être détruit par la vérité doit l'être.~>>

Le frapper n'aurait pas suffi.

<<~Tu avais tort sur un point~>>, dit Drago, la voix mortelle. <<~Granger n'est pas l'élève la plus forte de Poudlard. Elle obtient juste les meilleures notes en cours. Tu es sur le point de comprendre la différence.~>>

Un choc soudain apparut sur le visage de Harry, il essaya de faire une roulade pour se remettre sur pied -

C'était déjà trop tard pour lui.

<<~\emph{Expelliarmus}~!~>>

La baguette de Harry vola jusqu'à l'autre bout de la pièce.

<<~\emph{Gom jabbar~!}~>>

Une pulsation de noirceur encrée frappa la main gauche de Harry.

<<~C'est un sort de torture, dit Drago. C'est pour tirer des informations des gens. Je vais juste le laisser sur toi et fermer la porte derrière moi en partant. Peut-être que je réglerai le sort de loquet pour qu'il s'estompe après quelques heures. Ou peut-être qu'il ne s'estompera pas avant que tu ne meures ici. Amuse-toi bien.~>>

Drago recula d'une démarche fluide, baguette toujours pointée vers Harry. La main de Drago s'abaissa et ramassa son cartable sans que son bras ne vacille.

La douleur était déjà visible sur le visage de Harry quand il parla. <<~Dois-je comprendre que les Malfoy sont au-dessus des lois sur la magie des mineurs~? Ce n'est pas parce que ton sang est plus fort. C'est parce que tu as déjà pratiqué. Au début, tu étais aussi faible que n'importe lequel d'entre nous. Ma prédiction est-elle fausse~?~>>

La main de Drago blanchit autour de sa baguette, mais sa visée demeura stable.

<<~Juste pour que tu saches~>>, dit Harry, les dents serrées, <<~si tu m'avais dit que j'avais tort, je t'aurais écouté. \emph{Je} ne te torturerai jamais quand tu me montreras que j'ai tort. Et tu le \emph{feras}. Un jour. Tu es éveillé à la science maintenant, et même si tu n'apprends jamais à utiliser ton pouvoir, tu seras toujours,~>> Harry haleta, <<~à la recherche… de moyens… de tester… tes croyances…~>>

La démarche de Drago était maintenant moins fluide, un peu plus rapide, et il dut faire un effort pour maintenir sa baguette sur Harry tandis qu'il tendait la main en arrière pour ouvrir la porte et qu'il sortait de la salle.

Puis Drago referma la porte.

Il jeta le sort de loquet le plus puissant qu'il connaisse.

Drago attendit d'avoir entendu le premier cri de Harry avant de jeter \emph{Sourdinam}.

Puis il s'en fut.

\later

<<~\emph{Aaahhhhh~! Finite Incantatem~! Aaaahhh~!}~>>

La main gauche de Harry avait été plongée dans une marmite d'huile de cuisson bouillante et laissée plantée là. Il avait tout donné pour jeter le \emph{Finite Incantatem} et ça ne marchait toujours pas.

Certains maléfices requéraient des contre-sorts spéciaux, sans lesquels on ne pouvait pas les défaire, ou peut-être que Drago était juste bien plus fort.

<<~\emph{Aaaaahhhh~!}~>>

Sa main commençait vraiment à lui faire mal maintenant, et ça interférait avec ses tentatives d'improvisation.

Mais quelques cris plus tard, Harry comprit ce qu'il fallait qu'il fasse.

Malheureusement, sa bourse était du mauvais côté de son corps, et il lui fallut se tordre quelque peu avant de pouvoir l'atteindre, en particulier avec son autre bras qui fouettait l'air d'un mouvement réflexe incontrôlable, destiné à l'éloigner de la source de douleur.

<<~Kit \emph{ahhhhh} médical~! Kit médical~!~>>

La lumière verte sur le sol était trop faible pour permettre d'y voir.

Harry ne pouvait pas se tenir debout. Il ne pouvait pas ramper. Il se laissa rouler sur le sol jusqu'à l'endroit où il pensait que sa baguette se trouvait, elle n'était pas là, d'une main il parvint à s'élever assez haut pour la voir, il roula jusqu'à elle, la saisit, et revint en roulant jusqu'à l'endroit où le kit médical avait été laissé, ouvert. Il y eut aussi pas mal de hurlements et quelques vomissements.

Il fallut huit essais avant que Harry ne parvienne à jeter \emph{Lumos}.

Et là… le kit n'avait pas été conçu pour être ouvert d'une seule main, parce que tous les sorciers étaient des idiots, voilà pourquoi. Harry dut utiliser ses dents, et il lui fallut donc un moment avant de finalement parvenir à enrouler l'Anésthissu autour de sa main gauche.

Quand toute sensation eut enfin quitté sa main, Harry laissa son esprit se disjoindre et il resta allongé, immobile au sol, et il pleura pendant un moment.

\emph{Eh bien}, dit silencieusement l'esprit de Harry à lui-même après qu'il eut assez récupéré pour penser en mots. \emph{Ça valait le coup} \emph{?}

Lentement, la main fonctionnelle de Harry attrapa son bureau.

Harry se hissa sur ses pieds.

Prit une profonde inspiration.

Expira.

Sourit.

Ce n'était pas un grand sourire, mais c'était un sourire quand même.

\emph{Merci, professeur Quirrell, je n'aurais pas pu perdre sans vous}.

Il n'avait pas racheté Drago, loin de là. Contrairement à ce que Drago lui-même pouvait à présent croire, il était toujours l'enfant d'un Mangemort jusqu'au bout des ongles. Toujours un garçon qui avait grandi en pensant que “violer”, c'était quelque chose que les enfants cool et plus âgés faisaient. Mais c'était un sacré début.

Harry ne pouvait pas prétendre que tout s'était déroulé exactement comme prévu. Tout s'était déroulé exactement comme inventé au fur et à mesure. Le \emph{plan} avait prévu que cela n'arrive pas avant environ décembre, après que Harry eut appris à Drago les techniques permettant de ne pas nier une preuve quand on se retrouvait face à elle.

Mais il avait vu l'air sur le visage de Drago, il s'était rendu compte que Drago prenait \emph{déjà} une hypothèse alternative au sérieux, et Harry avait fait feu de tout bois. En rationalité, un cas de réelle curiosité avait le même pouvoir salvateur qu'un cas de véritable amour en avait dans les films.

Rétrospectivement, Harry s'était donné quelques heures pour faire la découverte la plus importante de l'histoire de la magie et quelques mois pour briser les barrières sous-développées d'un garçon de onze ans. Ce qui indiquait peut-être que Harry avait une déficience cognitive majeure lorsqu'il s'agissait d'estimer les temps d'exécution des tâches.

Harry allait-il aller en Enfer de la Science pour ce qu'il avait fait~? Il n'en était pas certain. Il s'était arrangé pour garder l'attention de Drago concentrée sur la possibilité que la magie disparaisse, il s'était assuré que Drago s'occupe des expériences qui semblaient au premier abord pointer dans cette direction. Il avait attendu d'avoir expliqué la génétique avant de le pousser à considérer les créatures magiques (bien que Harry ait pensé en termes d'anciens artefacts tels que le Choixpeau, que personne ne pouvait plus dupliquer mais qui continuaient de fonctionner). Mais Harry n'avait pas vraiment exagéré la moindre preuve, il n'avait pas détourné le sens d'un seul des résultats. Quand l'Interdit de Merlin avait invalidé le test qui aurait dû être décisif, il l'avait tout de suite dit à Drago.

Puis ce qui s'était passé \emph{ensuite}…

Mais il n'avait pas vraiment \emph{menti} à Drago. Drago l'avait cru, et \emph{ça le rendrait vrai}.

La fin n'avait certes pas été amusante.

Harry se retourna et chancela jusqu'à la porte.

Il était temps de tester le sort de Verrouillage de Drago.

La première étape était simplement d'essayer de tourner la poignée. Drago aurait pu bluffer.

Drago n'avait pas bluffé.

<<~\emph{Finite Incantatem}.~>> La voix de Harry était plutôt rauque, et il pouvait sentir que le sort n'avait pas pris.

Alors Harry essaya de nouveau, et cette fois il sentit le sort se lancer. Mais un autre tour de poignée montra qu'il n'avait pas fonctionné. Pas surprenant.

C'était le moment de sortir l'artillerie lourde. Harry prit une profonde inspiration. Ce sort était l'un des plus puissants qu'il ait appris jusqu'alors.

<<~\emph{Alohomora~!}~>>

Harry chancela un peu après l'avoir prononcé.

Et la porte de la salle ne s'ouvrit toujours pas.

Ce qui surprit Harry. Bien sûr, il n'avait certainement pas compté s'approcher du couloir interdit de Dumbledore. Mais un sort permettant d'ouvrir des verrous magiques semblait être de toute façon utile, alors Harry l'avait appris. Le couloir interdit de Dumbledore était-il censé attirer des gens assez stupides pour ne pas se rendre compte que la sécurité de celui-ci était pire que celle que Drago Malfoy aurait su mettre en place~?

La peur affluait à nouveau dans l'organisme de Harry. La note dans le kit médical avait dit que l'Anésthissu pouvait être utilisé de façon sûre pendant trente minutes maximum. Après quoi il se détacherait automatiquement et ne serait pas réutilisable avant 24 heures. Il était maintenant 18h51. Il avait mis l'Anésthissu environ cinq minutes auparavant.

Alors Harry fit un pas en arrière et examina la porte. C'était un solide panneau de chêne sombre, uniquement interrompu par la poignée de laiton.

Harry ne connaissait ni sort explosif, ni sort coupant, ni sort fracassant, et métamorphoser des explosifs aurait enfreint la règle contre la métamorphose de choses destinées à être brûlées. L'acide était un liquide et il aurait produit des vapeurs…

Mais ce n'était pas un obstacle pour quelqu'un de \emph{créatif}.

Harry posa sa baguette contre les charnières en laiton et se concentra sur la notion de coton en tant qu'entité abstraite, détachée de tout coton tangible, ainsi que sur le matériau du laiton en lui-même, détaché de la structure qui faisait de lui une charnière, et il mêla les deux concepts, imposant une forme sur une substance. Une heure de pratique de la Métamorphose pendant un mois avait mené Harry à un niveau où il pouvait métamorphoser un sujet de cinq centimètres cubiques en à peine une minute.

Après deux minutes, la charnière n'avait pas du tout changé.

Quiconque avait conçu le sort de verrouillage de Drago avait aussi pensé à ça. Ou la porte faisait partie de Poudlard et le château était immunisé.

Un coup d'œil révéla que les murs étaient faits de pierre solide. Tout comme le sol. Et le plafond. On ne pouvait métamorphoser une partie séparément d'un tout si le tout était solide~; Harry aurait pu essayer de métamorphoser le mur entier, ce qui aurait pris des heures ou peut-être des jours d'effort continu~; et encore, si c'était faisable, et si le mur n'était pas contigu avec le reste du château…

Le Retourneur de Temps de Harry ne s'ouvrirait pas avant 21h. Après cela il pourrait revenir à 18h, avant que la porte ne soit verrouillée.

Combien de temps durerait le sort de torture~?

Harry avala sa salive avec difficulté. Des larmes s'amoncelaient à nouveau dans ses yeux.

Son brillant esprit créatif venait d'offrir une suggestion ingénieuse, que Harry coupe sa main en utilisant la scie à métaux qui était dans la boite à outils de sa bourse, ce qui lui ferait évidemment mal, mais qui pourrait lui faire bien moins mal que le sort de douleur de Drago puisqu'il n'y aurait plus de nerfs~; et il avait toujours des garrots dans le kit de soin.

Et c'était bien sûr une idée horriblement stupide que Harry regretterait pour le restant de ses jours.

Mais Harry ne savait pas s'il pourrait tenir deux heures sous la torture.

Il voulait \emph{sortir} de cette salle, il voulait en sortir \emph{maintenant}, il ne voulait pas avoir à attendre là pendant deux heures en hurlant jusqu'à ce qu'il puisse utiliser le Retourneur de Temps, il fallait qu'il \emph{sorte} et qu'il trouve quelqu'un capable de débarrasser sa main gauche du sort de torture…

\emph{Pense~!} cria Harry à l'intention de son cerveau. \emph{Pense~! Pense~!}

\later

Le dortoir des Serpentard était majoritairement inoccupé. Les gens étaient au dîner. Sans qu'il sache pourquoi, Drago ne se sentait pas très en appétit.

Drago ferma la porte de sa chambre privée, la verrouilla, l'ensorcela, la \emph{Sourdina}, s'assit sur son lit et commença à pleurer.

Ce n'était pas juste.

Ce n'était pas juste.

C'était la première fois qu'il avait vraiment \emph{perdu}, Père l'avait prévenu que la véritable défaite lui ferait mal la première fois, mais il avait \emph{tellement} perdu, ce n'était pas juste, ce n'était pas juste qu'il perde \emph{tout} à sa première défaite.

Quelque part dans les donjons, un garçon que Drago avait vraiment bien aimé hurlait de douleur. Jamais auparavant Drago n'avait fait de mal à quelqu'un qu'il appréciait. Punir les gens qui le méritaient étaient censé être amusant, mais ça, ça le faisait se sentir malade. Père ne l'avait pas mis en garde contre cette sensation, et Drago se demanda si c'était une leçon difficile que tout le monde devait apprendre en grandissant ou s'il n'était qu'un faible.

Drago aurait aimé que ce soit Pansy qui soit en train de crier. Il se serait senti mieux.

Et le pire était de savoir que c'était peut-être une erreur d'avoir fait du mal à Harry Potter.

Qui serait là pour Drago maintenant~? Dumbledore~? Après ce qu'il avait fait~? Drago aurait préféré être brûlé vif.

Drago allait devoir revenir vers Harry Potter, parce qu'il n'avait personne d'autre vers qui se tourner. Et si Harry Potter disait qu'il ne voulait pas de lui, alors Drago ne serait rien, juste un pathétique petit garçon qui ne pourrait jamais être un Mangemort, ne pourrait jamais rejoindre le camp de Dumbledore et ne pourrait jamais apprendre la science.

Le piège avait été parfaitement mis en place, parfaitement exécuté. Père avait prévenu Drago, encore et encore, que ce que l'on sacrifiait lors d'un Rituel Noir ne pouvait pas être retrouvé. Mais Père n'avait pas su que les maudits Moldus avaient inventé des rituels qui ne nécessitaient pas de baguettes, des rituels que l'on pouvait être poussé à faire sans le savoir, et ce n'était là qu'un seul des terribles secrets que les scientifiques connaissaient et que Harry Potter avait apportés avec lui.

Drago commença alors à pleurer plus fort.

Il ne voulait pas l'être, il ne \emph{voulait pas l'être}, mais revenir en arrière était impossible. C'était trop tard. Il était déjà un scientifique.

Drago savait qu'il aurait dû retourner voir Harry Potter, le libérer et présenter ses excuses. Ça aurait été la démarche intelligente.

Au lieu de ça, Drago resta au lit et pleurnicha.

Il avait déjà fait souffrir Harry Potter. Ça serait peut-être la seule fois où il pourrait jamais lui faire du mal, et il devrait chérir ce souvenir pour le restant de ses jours.

Qu'il continue de crier.

\later

Harry Potter laissa tomber les restes de sa scie à métaux. Les charnières en laiton s'étaient révélées insensibles, à peine rayées, et Harry commençait à suspecter que même l'acte désespéré consistant à essayer de métamorphoser de l'acide ou des explosifs aurait échoué à ouvrir cette porte. Le côté positif, c'était que cette tentative avait détruit la scie à métaux.

Sa montre disait qu'il était 19h02, avec moins de quinze minutes restantes, et Harry essaya de se souvenir s'il y avait d'autres choses coupantes dans sa bourse qui avaient besoin d'être détruites, et il sentit une autre montée de larmes s'accumuler. Si seulement il avait pu, quand son Retourneur de Temps se serait ouvert, revenir dans le passé et \emph{empêcher} -

Et c'est alors que Harry se rendit compte qu'il était un \emph{idiot}.

Ce n'était pas la première fois qu'il avait été enfermé.

Le professeur McGonagall lui avait déjà donné la méthode correcte.

… elle lui avait aussi dit de ne pas utiliser le Retourneur de Temps pour ce genre de choses.

Le professeur McGonagall ne se rendrait-elle pas compte que ce cas justifiait \emph{bien} une exception~? Ou allait-elle juste lui interdire tout usage du Retourneur de Temps~?

Harry rassembla ses affaires dans sa bourse, toutes les preuves. Un \emph{Récurvite} s'occupa du vomi au sol mais pas de la sueur qui avait trempé sa robe. Il laissa les bureaux renversés car ce n'était pas assez important pour qu'il le fasse d'une seule main.

Lorsqu'il eut fini, Harry jeta un coup à sa montre. 19h04.

Et Harry attendit alors. Les secondes s'écoulèrent, lui semblant être des années.

À 19h07, la porte s'ouvrit.

Le visage à la barbe bouffante du professeur Flitwick semblait plutôt soucieux. <<~Vous allez bien, Harry~? dit la voix haut perchée du directeur de Serdaigle. J'ai eu une note disant que vous aviez été enfermé ici -~>>
%  LocalWords:  nd Gom jabbar Aaahhhhh Aaaahhh Aaaaahhhh ahhhhh
