\partchapter{L'Expérience de Prison de Stanford}{I}

\lettrinepara{S}{amedi}.

\hplettrineextrapara
Harry avait eu du mal à s'endormir vendredi soir, ce qu'il avait prévu être une éventualité possible, et il avait donc pris l'évidente précaution d'acheter une potion de sommeil~; et pour empêcher que cela constitue un signe visible de sa nervosité, il avait décidé de l'acheter à Fred et George deux mois plus tôt. (\emph{Soyez prêts, c'est la rengaine des Scouts…)}

Harry était donc en pleine forme et sa bourse contenait presque tout ce qu'il possédait et dont il était concevable qu'il puisse avoir besoin. Il avait à vrai dire atteint les limites du volume de la bourse~; et, sachant qu'il lui faudrait stocker un grand serpent, et qu'il aurait peut-être besoin de stocker qui-savait-quoi-d'autre, il avait enlevé certains des objets les plus encombrants, tels la batterie de voiture. Il était maintenant capable de métamorphoser quelque chose de la taille d'une batterie de voiture en quatre minutes chrono, ce n'était donc pas une grande perte.

Harry \emph{avait} gardé les feux de détresse, la torche d'oxycoupage et le réservoir d'essence puisqu'on ne pouvait pas métamorphoser des choses destinées à être brûlées.

(\emph{Soyez prêt, chaque jour de votre vie…})

Chez Marie.

Après que la serveuse eut pris leur commande, se soit inclinée et ait quitté la pièce, le professeur Quirrell n'avait lancé que quatre sortilèges, et ils avaient alors parlé de choses sans grande importance, de la thèse du professeur Quirrell sur la façon dont la malédiction jetée par le Seigneur des Ténèbres sur le poste de Défense avait entraîné le déclin des duels et avait modifié les mœurs d'Angleterre Magique. Harry écoutait et hochait la tête et disait des choses intelligentes tout en essayant de contrôler les coups de marteau de son cœur.

La serveuse vint alors, portant leur nourriture, et cette fois, une minute après son départ, le professeur Quirrell fit signe à la porte de se fermer à loquet puis commença à prononcer vingt-neuf sortilèges de sécurité, omettant l'un de ceux de la séquence de M. Bester, ce qui laissa Harry perplexe.

Le professeur Quirrell finit ses sortilèges…

… se leva de sa chaise…

… se fondit en un serpent vert bariolé de bleu et de blanc…

… siffla «\parsel{Faim, garçon~? mange ton content en vitesse, nouss auronss besoin de forcces et de temps.}»

Les yeux de Harry s'écarquillèrent légèrement, mais il siffla «\parsel{J'ai bien mangé cce matin},» avant de rapidement fourrer des nouilles dans sa bouche.

Le serpent le regarda un moment de ses yeux plats puis siffla~:

«\parsel{Ne ssouhaite pas expliquer icci. Préfère être ailleurs d'abord. Besoin de partir pas vu, ssans ssigne que nouss avonss jamais quitté piècce.}

--- \parsel{Pour que perssonne ne puisssse nous suivre}, siffla Harry.

--- \parsel{Oui. Me fais-tu confiancce à cce point, garççon~? Pensses avant de répondre. J'ai requête importante pour toi, qui néccesssite confiancce~; si tu ssais que tu vas dire non, dis maintenant.}»

Harry baissa le regard en évitant les yeux plats du serpent, considéra ses nouilles en sauce, et prit une autre bouchée, puis une autre, alors qu'il réfléchissait.

Le professeur de Défense… était un personnage ambitieux, et c'était un euphémisme. Harry pensait avoir démêlé certains de ses buts, mais d'autres demeuraient mystérieux.

Mais le professeur Quirrell avait assommé deux-cents filles pour arrêter celles qui attiraient Harry vers elles. Le professeur Quirrell avait déduit que le Détraqueur vidait Harry à travers sa baguette. Le professeur de Défense lui avait sauvé la vie deux fois en deux semaines.

Ce qui pouvait signifier que le professeur de Défense le gardait \emph{pour plus tard}, qu'il avait une raison cachée. De fait, il était \emph{certain} qu'il avait des buts cachés. Le professeur Quirrell ne faisait pas cela par caprice. Mais c'était aussi lui qui s'était assuré que Harry apprenne l'Occlumancie, et qui lui avait enseigné à perdre… si le professeur de Défense voulait faire usage de Harry Potter, c'était un usage qui demandait un Harry Potter renforcé, pas affaibli. C'était là la notion même d'être utilisé par un ami~: l'idée qu'il voudrait avoir besoin de vous rendre plus fort, pas plus faible.

Et s'il y avait parfois une froide atmosphère autour du professeur de Défense, de l'amertume dans sa voix ou un vide dans ses yeux, alors Harry était le seul que le professeur Quirrell autorise à le remarquer.

Harry ne savait guère quels mots il aurait pu utiliser pour décrire l'affinité qu'il ressentait pour le professeur Quirrell, seulement que le professeur de Défense était la seule personne capable de \emph{penser clairement} qu'il ait rencontrée de tout le monde magique. Tôt ou tard tous les autres se mettaient à jouer au Quidditch, à ne pas mettre de coques protectrices sur leur machine à remonter le temps, ou à penser que la Mort était leur amie. Que leurs intentions soient bonnes n'avait aucune importance. Tôt ou tard, et généralement tôt, ils démontraient que quelque chose dans leur esprit était profondément embrouillé. Tout le monde sauf le professeur Quirrell. C'était un lien qui allait au-delà des dettes, au-delà des préférences personnelles~: ils étaient seuls dans le monde magique. Et si le professeur de Défense semblait parfois un peu effrayant ou un peu ténébreux, eh bien, on disait exactement le même genre de choses au sujet de Harry.

«\parsel{Je vous fais confiancce}», siffla Harry.

Et le serpent expliqua la première étape du plan.

\later

Harry prit une dernière fourchetée de nouilles et mâcha. À côté de lui, le professeur Quirrell, de nouveau humain, mangeait sa soupe d'un air placide, comme si rien de particulièrement intéressant ne se passait.

Puis Harry déglutit et se leva de sa chaise au même moment, sentant déjà son cœur frapper contre sa poitrine. Les précautions qu'il prenait étaient, littéralement, les plus rigoureuses possibles…

«Êtes-vous prêt à faire le test, M. Potter~?» dit le professeur Quirrell avec calme.

Ce n'était \emph{pas} un test, mais le professeur Quirrell ne dirait pas cela, pas à voix haute en langue humaine, même dans cette pièce vérifiée à fond et dont le professeur Quirrell avait augmenté la sécurité avec des sortilèges supplémentaires.

«Ouaip», dit Harry, d'un ton aussi nonchalant que possible.

\emph{Première étape.}

Harry dit «cape» à sa bourse, en sortit la Cape d'Invisibilité, puis décrocha la bourse de sa ceinture et la lança à l'autre bout de la table.

Le professeur de Défense se leva de son siège, tira sa baguette, s'inclina, et de celle-ci toucha la bourse en murmurant une incantation. Les nouveaux sortilèges assureraient que le professeur Quirrell pourrait entrer dans la bourse sous sa forme de serpent, la quitter par ses propres moyens et entendre ce qui se passait dehors.

\emph{Deuxième étape.}

Alors que le professeur Quirrell se relevait de sa posture inclinée, au-dessus de la bourse, et qu'il éloignait sa baguette, celle-ci se retrouva pointée vers Harry, et ce dernier ressentit une brève sensation de raclement sur la poitrine, non loin du Retourneur de Temps, comme si quelque chose d'affreux était passé très près, sans le toucher.

\emph{Troisième étape.}

Le professeur de Défense se transforma de nouveau en serpent, et la sensation funeste diminua~; le serpent rampa jusqu'à la bourse, puis jusqu'à l'intérieur, alors que la bouche de celle-ci s'ouvrait afin d'accueillir la forme verte, puis elle se referma derrière la queue du serpent et la sensation funeste diminua à nouveau.

\emph{Quatrième étape.}

Harry sortit sa baguette en faisant attention de ne pas déplacer ses jambes, afin que le Retourneur de Temps ne dévie pas de l'endroit où le professeur Quirrell avec fixé le sablier, placé sous la coque protectrice. «\emph{Wingardium Leviosa}», murmura Harry, et la bourse commença à flotter vers lui.

Lentement, lentement, comme le professeur Quirrell le lui avait dit, la bourse commença à flotter vers Harry, qui attendait, à l'affût du moindre signe indiquant que la bourse s'ouvrait, auquel cas il devrait utiliser la lévitation pour le repousser aussi vite que possible.

Lorsque la bourse fut à un mètre de Harry, la sensation funeste revint.

Lorsque Harry ré-attacha la bourse à sa ceinture, la sensation funeste devint plus forte qu'elle ne l'avait jamais été sans pour autant devenir écrasante~; c'était tolérable.

Même avec la forme animale du professeur Quirrell située quelque part dans l'espace distendu de la bourse placée à même les hanches de Harry.

\emph{Cinquième étape.}

Harry rengaina sa baguette. Son autre main tenait toujours la Cape d'Invisibilité, et il s'en recouvrit.

\emph{Sixième étape.}

Et alors, dans cette pièce protégée contre toute intrusion possible, dont le professeur Quirrell avait personnellement amélioré la sécurité, ce ne fut qu'\emph{après} avoir revêtu la véritable Cape d'Invisibilité qu'il passa sa main sous sa chemise et opéra une rotation de la coque externe du Retourneur de Temps.

Le sablier intérieur demeura figé, immobile, l'assemblage tournant autour de lui…

La nourriture disparut de la table, les chaises reprirent leur place, la porte s'ouvrit grand.

La Chambre de Marie était déserte, comme elle devait l'être, car le professeur Quirrell avait auparavant contacté le restaurant sous un faux nom afin de demander si la pièce serait disponible à cette heure-ci -- non pour la réserver, non pour annuler une commande, ce qui aurait pu être remarqué, mais seulement pour s'informer.

\emph{Septième étape.}

Demeurant sous la Cape d'Invisibilité, Harry sortit par la porte ouverte. Il navigua à travers les couloirs de la Chambre de Marie jusqu'au bar bien fourni qui accueillait les nouveaux arrivants, tenu par le propriétaire, Jake. Il n'y avait que quelques personnes au comptoir, ce matin avant le déjeuner, et Harry dut attendre à côté de la porte, invisible, pendant quelques minutes, écoutant le murmure des conversations et le gargouillis de l'alcool, avant que la porte ne s'ouvre pour laisser place à un immense et sympathique Irlandais, et Harry se glissa silencieusement dans son sillage.

\emph{Huitième étape.}

Harry marcha quelque temps. Il était bien loin de la Chambre de Marie lorsqu'il quitta le Chemin de Traverse et passa dans une plus petite allée, au fond de laquelle se trouvait un magasin sombre, les fenêtres magiquement noircies.

\emph{Neuvième étape.}

«Épée poisson melon ami», dit Harry au loquet, et il s'ouvrit dans un cliquetis.

Le magasin était aussi sombre à l'intérieur, mais la lumière de la porte ouverte l'illumina pour révéler une grande pièce vide. Selon le professeur de Défense, le magasin de mobilier qui avait un jour exploité ces lieux avait fait banqueroute quelques mois auparavant, et le magasin avait été saisi mais n'avait pas encore été vendu. Les murs étaient peints d'un blanc simple, le sol en bois était rayé et terne, une seule porte fermée ornait le mur du fond~; le magasin avait un jour fonctionné à plein régime, mais il était à présent vide.

La porte se referma derrière Harry, et les ténèbres furent absolues.

\emph{Dixième étape.}

Harry sortit sa baguette et dit~: «\emph{Lumos}», éclairant la pièce d'un halo blanc~; il prit la bourse à sa ceinture (la sensation funeste devenant un peu plus intense à mesure que ses doigts s'approchaient) et la jeta doucement vers le côté opposé de la pièce (la sensation funeste disparaissant presque totalement). Il commença alors à enlever la Cape d'Invisibilité, alors même que sa voix sifflait~: «\parsel{Sc'est fait.}»

\emph{Onzième étape.}

De la bourse surgit une tête verte, rapidement suivie par un corps d'un mètre de long, à mesure que le serpent sortait en ondulant. Un instant plus tard, le serpent devint flou et fut remplacé par le professeur Quirrell.

\emph{Douzième étape.}

Harry attendit silencieusement, alors que le professeur de Défense récitait trente sortilèges.

«Très bien, dit-il calmement lorsqu'il eut fini. Si quelqu'un nous observe maintenant, nous sommes de toute façon perdus, je parlerai donc de façon claire et en langue humaine. J'ai bien peur que le Fourchelangue ne m'aille pas très bien, n'étant ni un descendant de Salazar ni un véritable serpent.»

Harry hocha la tête.

«Donc, M. Potter», dit le professeur Quirrell. Son regard était braqué sur Harry, ses yeux bleu pâle dans l'ombre de la lumière blanche émanant de la lumière de la baguette de Harry. «Nous sommes seuls et en privé, et j'ai une question importante à vous poser.

--- Allez-y», dit Harry, et son cœur commença à battre plus vite.

«Que pensez-vous du gouvernement d'Angleterre Magique~?»

Ce n'était pas tout à fait ce à quoi Harry s'était attendu, mais c'était assez proche, alors Harry répondit~:

«En me basant sur mes connaissances limitées, je dirais que le Ministère et le Magenmagot semblent être stupides, corrompus et maléfiques.

--- Correct, dit le professeur Quirrell. Comprenez-vous pourquoi je pose cette question~?»

Harry prit une profonde inspiration et regarda le professeur Quirrell droit dans les yeux, sans ciller. Harry avait enfin compris que la technique permettant de faire des déductions incroyables à partir de maigres preuves était de connaître la réponse à l'avance, et il avait deviné celle-ci une semaine auparavant. Elle ne nécessitait qu'un léger ajustement…

«Vous êtes sur le point de m'inviter à rejoindre une organisation secrète pleine de gens intéressants tels que vous, dit Harry, dont l'un des buts est de réformer ou de renverser le gouvernement d'Angleterre Magique, et oui, j'accepte.»

Il y eut une courte pause.

«J'ai peur que ce ne soit pas tout à fait dans cette direction que je voyais la conversation aller», dit le professeur Quirrell. Les coins de ses lèvres se contractèrent légèrement. «Je comptais simplement vous demander votre aide dans l'accomplissement d'un acte de trahison extrêmement illégal.»

\emph{Mince}, pensa Harry. Enfin, le professeur Quirrell n'avait pas \emph{nié}…

«Continuez.

--- Avant que je le fasse», dit le professeur Quirrell. Il n'y avait plus la moindre légèreté dans sa voix. «\emph{Êtes-vous} ouvert à de telles requête, M. Potter~? Je répète que s'il est probable que vous répondiez non quelle qu'elle soit, vous devez dire non maintenant. Si votre curiosité vous pousse à faire le contraire, écrasez-la.

--- La traîtrise et l'illégalité ne me dérangent pas, dit Harry. Les risques me dérangent et l'enjeu devrait être à la mesure de ceux-ci, mais je ne \emph{vous} imagine pas prendre des risques inconsidérés.»

Le professeur Quirrell acquiesça.

«En effet. C'est un abus terrible de mon amitié avec vous, et de la confiance qui m'a été accordée dans ma fonction de professeur à Poudlard…

--- Vous pouvez sauter cette étape», dit Harry.

Les lèvres se contractèrent de nouveau, puis s'aplatirent. «Je la sauterai donc. M. Potter, vous jouez parfois à mentir en disant la vérité, à jouer sur les mots afin de masquer le sens en plein jour. J'ai moi aussi la réputation de trouver cela amusant. Mais si je ne fais que vous \emph{dire} ce que j'espère nous voir faire aujourd'hui, M. Potter, vous \emph{mentirez}. Vous mentirez de façon directe, sans hésitation, sans jeu sur les mots ni indice, à toute personne qui pourrait vous interroger, qu'ils soient votre ennemi ou votre ami le plus proche. Vous mentirez à Malfoy, à Granger et à McGonagall. Vous parlerez, à chaque fois et sans hésitation, \emph{exactement} de la façon dont vous auriez parlé si vous n'aviez rien su, et ce sans vous préoccuper de votre honneur. Il doit en être ainsi.»

Il y eut un silence, pendant un moment.

C'était un prix qui se mesurait en fractions d'âme de Harry.

«Sans me le dire tout de suite… dit Harry. Pouvez-vous me dire si le besoin est désespéré~?

--- Il y a quelqu'un qui a terriblement besoin de votre aide, dit simplement Quirrell, et personne ne peut l'aider sauf vous.»

Il y eut un autre silence, mais celui-ci ne fut pas long.

«Très bien, dit doucement Harry. Parlez-moi de la mission.»

La sombre robe du professeur de Défense sembla devenir floue devant les ombres du mur projetées par sa silhouette, qui bloquait la lumière blanche de la baguette de Harry. «Le Patronus ordinaire, M. Potter, protège de la peur du Détraqueur. Mais le Détraqueur vous voit toujours au travers, il sait que vous êtes là. Mais pas à travers votre Patronus. Il les aveugle, et peut-être même plus. Ce que j'ai vu sous la cape ne regardait même pas vers nous lorsque vous l'avez tué~; comme s'il avait oublié notre existence au moment de sa mort.»

Harry hocha la tête. Ce n'était pas surprenant, pas lorsqu'on confrontait le Détraqueur à son véritable niveau d'existence, par-delà l'anthropomorphisme. La mort était peut-être le dernier ennemi, mais ce n'était pas un ennemi sentient. Lorsque l'humanité avait éradiqué la variole, elle ne s'était pas défendue.

«M. Potter, la branche centrale de Gringotts est gardée par tous les sorts connus des Gobelins, du plus faible au plus puissant. Même ainsi, ces chambres fortes ont été cambriolées avec succès~; car ce que la magie peut faire, elle peut le défaire. Et pourtant, personne ne s'est jamais échappé d'Azkaban. Personne. Pour tout sortilège il y a un contre-sortilège, pour toute barrière il y a un passe-droit. Comment se peut-il que personne n'ait jamais été sauvé d'Azkaban~?

--- Parce qu'Azkaban a une chose invincible, dit Harry. Quelque chose de si terrible que personne ne peut la vaincre.»

C'était la clé de voûte de la sécurité absolue, il fallait que ce ne soit pas humain. C'était la Mort qui gardait Azkaban.

«Les Détraqueurs n'aiment pas qu'on leur prenne leur repas», dit le professeur Quirrell. La froideur avait maintenant infiltré sa voix. «Ils le savent, si quelqu'un s'y essaie. Il a plus de cent Détraqueurs là-bas, et ils parlent aussi aux gardes. C'est aussi simple que cela, M. Potter. Si vous êtes un sorcier puissant, alors Azkaban n'est ni difficile à pénétrer ni à quitter. Tant que vous n'essayez pas d'y prendre quoi que ce soit qui appartienne aux Détraqueurs.

--- Mais les Détraqueurs ne sont \emph{pas} invincibles», dit Harry. Il aurait pu lancer le Patronus grâce à cette pensée, à l'instant même. «Ne croyez jamais qu'ils le sont.»

La voix du professeur Quirrell était très douce.

«Vous souvenez-vous de ce que vous avez ressenti face au Détraqueur la première fois, lorsque vous avez échoué~?

--- Je m'en souviens.»

Et alors, avec un soulèvement écœurant dans son estomac, Harry sut où cette conversation allait~; il aurait dû le voir plus tôt.

«Il y a une personne innocente à Azkaban», dit le professeur Quirrell.

Harry hocha la tête, sa gorge brûlait, mais il ne pleura pas.

«La personne dont je parle n'était pas victime de l'Imperium», dit le professeur de Défense, sa robe noire entourée d'une ombre plus vaste. «Il y a des moyens plus sûrs que l'Imperium lorsqu'il s'agit de briser la volonté de quelqu'un, si vous avez le temps de torturer, de pratiquer la Légilimancie, et de réaliser des rituels dont je ne parlerai pas. Je ne peux pas vous dire comment je le sais, comment je sais quoi que ce soit à ce sujet, je ne peux pas formuler d'indice, même pour vous, vous devrez me croire. Mais il y a une personne à Azkaban qui n'a jamais fait le choix de servir le Seigneur des Ténèbres, qui a passé des années à souffrir dans la solitude, dans le lieu le plus froid et le plus horrible qu'on puisse imaginer, et qui n'a pas mérité d'en subir une seule minute.»

Harry le vit, en un seul bond intuitif, sa bouche dépassant presque ses pensées.

\emph{Il n'y avait pas eu d'indice, pas d'avertissement, nous pensions tous…}

«Une personne du nom de Black», dit Harry.

Il y eut un silence. Silence, alors que les yeux bleu pâle le fixaient.

«Eh bien, dit le professeur Quirrell après un moment. Je peux faire une croix sur l'idée de ne vous révéler le nom qu'après que vous aurez accepté la mission. Je vous demanderais si vous lisez \emph{mon} esprit, mais c'est tout simplement impossible.»

Harry ne répondit rien. C'était vraiment simple si on \emph{croyait} au processus démocratique moderne. La personne incarcérée à Azkaban la plus évidemment innocente était celle qui n'avait pas eu de procès…

«Je suis \emph{certainement} impressionné», dit le professeur Quirrell. Son visage était grave. «Mais c'est une affaire sérieuse, et s'il existe un moyen pour d'autres de faire la même déduction, je \emph{dois} le savoir. Alors dites-moi, M. Potter. Par Merlin, par Atlantis, par le vide entre les étoiles, comment avez-vous deviné que je parlais de Bellatrix~?»
%  LocalWords:  aturday
