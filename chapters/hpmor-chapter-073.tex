\namedpartchapter{Accomplissement de soi}{SA}{VIII}{Le Sacré et le Profane}

\lettrine{L}{\emph{e}} \emph{trait de feu rouge atteignit Hannah en plein visage, la renversa pieds par-dessus tête et envoya sa tête percuter le mur de pierre où sa figure pâle sembla demeurer un instant, encadrée par des mèches de cheveux châtain-or voletantes avant qu'elle ne s'écroule au sol dans un désordre de robes au moment où la troisième et ultime volée de spirales vertes étincelantes abattaient le sortilège de protection de leur ennemi.}

Les jours de mars défilaient, peuplés de cours magistraux, d'étude, de devoirs, de petits déjeuners, de déjeuners et de dîners.

\emph{Le Gryffondor les regarda fixement, chaque ligne de son corps tendue, son visage révélant un effort silencieux~; et ses mains relâchèrent alors leur prise serrée sur le revers du petit Serpentard et il s'en fut sans que quiconque ne dise un mot (enfin, Lavande faillit dire un mot~: sa bouche s'ouvrit sous le coup de l'indignation, peut-être parce qu'elle n'avait pas eu l'opportunité de déclamer son discours -- mais heureusement Hermione le remarqua et fit le geste qui voulait dire LA FERME).}

Et il y avait le sommeil, bien sûr. Ce n'était pas parce que dormir semblait parfaitement normal qu'il fallait l'oublier.

«~\emph{Innerver~!~» dit la jeune voix de Susan Bones, les yeux de Hermione s'ouvrirent grand, ses lèvres aspirèrent l'oxygène dans une grande bouffée, ses poumons lui semblèrent lourds comme si un immense poids était posé sur sa poitrine. À côté d'elle, Hannah se relevait déjà, la tête entre les mains, grimaçante. Daphné les avait prévenues que ce serait un combat “difficile”, ce qui avait suscité une certaine trépidation chez Hermione, et d'ailleurs chez toutes les filles. Sauf peut-être chez Susan qui était arrivée à l'heure dite du rendez-vous, avait marché à côté d'elles sans parler et avait combattu la brute de septième année jusqu'à être la dernière debout. Peut-être le Gryffondor avait-il été réticent à se battre contre la dernière fille des Bones, ou peut-être Susan avait-elle seulement été très chanceuse~; quoi qu'il en soit, lorsque Hermione avait essayé de se redresser, elle s'était rendu compte que sa poitrine lui semblait lourde parce qu'en effet, un corps assez grand était étendu sur elle.}

Et il ne fallait pas non plus oublier la magie, même si la durée totale pendant laquelle vous lanciez des sortilèges ne représentait qu'une toute petite partie de votre journée. C'était là l'objet même de Poudlard, après tout.

«~\emph{D'accord, et si on se promenait sur des skateboards~? dit Lavande. On pourrait aller de lieu en lieu plus vite qu'en marchant. Et on aurait l'air vraiment cool sur des skateboards, les outils Moldus ne sont peut-être pas aussi rapides que les balais, mais ils ont plus de style -- on devrait voter…}~»

Et vous occupiez le temps restant selon vos penchants naturels~: ragots sur les flirts des élèves plus âgés, ou livres et séances de travail.

\emph{Hermione tendit une main tremblante pour ramasser son exemplaire de} Poudlard~: Une Histoire \emph{qui était tombé au sol, le livre rassurant à seulement quelques pas de là où elle s'était retrouvée étendue après que la fille plus âgée en robe rouge se fut “cognée” contre elle, la projetant vers un mur. Puis la sorcière Gryffondor était partie sans un regard en arrière, seulement un “Salazar et sa -” et un mot qui faisait plus mal à Hermione que tout ce que les Serpentard pouvaient dire sur les Sang-de-Bourbe~; “Sang-de-Bourbe” n'était qu'un étrange terme sorcier mais Hermione connaissait le mot que la Gryffondor avait prononcé. Cela lui faisait toujours aussi mal à chaque fois et, étrangement, encore plus quand cela venait de Gryffondor, de ceux qui étaient} censés \emph{être les gentils.}

Harry avait réparti huit de ses soldats chaotiques entre les autres armées, comme on le lui avait ordonné~; il avait volontairement abandonné \emph{deux} lieutenants chaotiques~: il avait envoyé Dean Thomas à l'armée Dragon et avait échangé Seamus Finnigan contre Blaise Zabini, dont Harry avait dit qu'il était «~sous-exploité~» à Soleil. Lavande avait fait le choix de rejoindre la majorité de la SPEHS chez Soleil et Tracey avait décidé de rester avec Chaos.

«~\emph{Alors tu peux jouer de tes charmes auprès du général Potter~?~» dit Lavande alors que Hermione tentait d'ignorer la présence des deux filles du mieux qu'elle pouvait. «~Franchement Tracette, je pense que notre général Soleil l'a bien attrapé entre ses fils maintenant -- tu aurais plus de chances en essayant de persuader Hermione d'avoir un de ces, tu sais, “arrangements à trois”…~»}

Personne n'avait encore compris ce que Malfoy trafiquait.

«~\emph{Certain~?~» dit Harry Potter d'un ton quelque peu réticent. «~Tu sais qu'un rationaliste n'est jamais certain de rien, Hermione, pas même que deux et deux font quatre. Ce n'est pas comme si je pouvais lire l'esprit de Malfoy, et même si c'était le cas, je ne pourrais pas savoir qu'il n'est pas un Occlumens parfait. Tout ce que je peux te dire, en me basant sur ce que j'ai observé chez lui, c'est qu'il est bien plus plausible, contrairement à ce que Daphné Greengrass en pense, qu'il essaie vraiment de montrer une voie meilleure à Serpentard. On devrait… on devrait vraiment essayer de suivre son mouvement, Hermione.}~»

(Bon, Harry semblait penser que Drago Malfoy était un type bien. Mais le problème, c'était que Harry avait aussi tendance à faire confiance au professeur Quirrell).

\later

«~Professeur Quirrell, dit Harry, la haine que la maison Serpentard développe à l'encontre de Hermione Granger m'inquiète.~»

Ils étaient assis dans le bureau du professeur de Défense, Harry assis aussi loin que possible du bureau du professeur (et même alors, la sensation d'un désastre imminent était présente), et la bibliothèque vide encadrait toujours le crâne dégarni du professeur Quirrell. La tasse en équilibre sur la cuisse de Harry était remplie du thé obscur, chinois et probablement cher du professeur Quirrell, et le fait que Harry ait dû consciemment décider de le boire en disant long sur la façon dont il s'était mis à réfléchir ces derniers temps.

«~Et ceci me concerne pour quelle raison~? dit le professeur Quirrell en sirotant son thé.

--- Oui, eh bien, dit Harry, je vais juste fermer les yeux sur -- oh, professeur Quirrell, arrêtez ça, \emph{vous} avez comploté pour rétablir la réputation de Serpentard depuis au moins le premier vendredi de cette année.~»

Peut-être y avait-il eu l'ombre d'un sourire aux bords de ces lèvres fines et pâles, mais peut-être pas. «~Je pense que la maison Serpentard finira par s'en tirer assez bien, M. Potter, quel que soit le sort réservé à cette fille. Mais je dois admettre que les perspectives actuelles ne sont pas favorables à votre jeune amie. Les brutes de deux maisons, nombre d'entre elles dotées de familles puissantes et bien entourées, voient Mlle Granger comme une menace contre leur réputation et comme un affront à leur fierté. Aussi puissant que soit ce motif de lui faire du mal, il ne tient pas la comparaison face à la jalousie brute des Gryffondor qui voient une étrangère obtenir les lauriers de l'héroïsme dont ils rêvent depuis l'enfance.~» Le sourire sur les lèvres du professeur Quirrell était maintenant certain, bien que léger. «~Et il y a ceux de Serpentard qui entendent dire que le fantôme de Salazar Serpentard les a abandonnés en faveur d'une Sang-de-Bourbe. Je me demande si vous pouvez même concevoir, M. Potter, comment de tels gens peuvent réagir à cela~? Ceux qui n'y croient pas tueraient Mlle Granger avec joie pour lui faire payer cette insulte. Quant à ceux qui, au fond, dans quelque endroit secret de leur esprit, se demandent si ça ne pourrait pas être \emph{vrai}… leur état de panique intérieure est à peine envisageable.~» Le professeur Quirrell sirota calmement son thé. «~Lorsque vous aurez plus d'expérience, M. Potter, vous verrez ce genre de conséquence avant l'exécution de vos plans. En l'état, vous êtes desservi par votre choix d'ignorer tous les aspects de la nature humaine que vous trouvez déplaisants.~»

Harry sirota son thé.

«~Ah… dit Harry. Professeur Quirrell… aidez-moi~?

--- J'ai déjà offert mon aide à Mlle Granger, dit le professeur Quirrell, dès que j'ai prévu ce qui allait se produire. Mon élève m'a dit en termes polis de ne pas me mêler de ses affaires. Et je m'attends à ce qu'elle vous dise la même chose. Comme je n'ai que peu à gagner ou à perdre dans cette affaire, je ne compte guère insister sur ce point.~» Le professeur de Défense haussa les épaules, sa tasse stable, tenue exactement comme il le fallait. «~Ne vous inquiétez pas trop, M. Potter. Mlle Granger est entourée d'une effervescence émotionnelle mais elle est moins en danger que vous ne l'imaginez. Lorsque vous serez plus âgé, vous apprendrez qu'avant toute autre chose, la décision la plus courante qu'un individu prend est celle de ne rien faire.~»

\later

L'enveloppe que le système Serpentard avait livrée à Daphné au déjeuner n'était pas signée, comme toujours~; le parchemin enclos indiquait une heure, un lieu et, simplement~: «~\emph{Difficile}.~»

Ce n'était pas un problème pour Daphné. Ce qui l'inquiétait, c'était que Millicent n'avait semblé regarder ni Tracey ni elle de tout le déjeuner. Elle avait juste gardé ses yeux braqués vers son assiette et s'était contenté de manger. Daphné n'avait pu voir Millicent lever les yeux qu'une seule fois, vers la table Poufsouffle, avant de les rabaisser rapidement, même si elle avait été trop loin pour voir l'expression du visage de Millicent puisque cette dernière s'était assise très à l'écart de Tracey et Daphné.

Elle avait réfléchi à cela pendant le déjeuner, saisie d'une nausée plus forte que tout ce qu'elle avait ressenti auparavant et qui l'avait poussée à arrêter de manger après avoir fini la moitié du premier plat.

\emph{Ce que je vois doit se produire… comparé à ça, se faire manger par des Moremplis ressemble à une partie de plaisir…}

Ce ne fut pas une décision consciente, ça ne ressemblait en rien à ce que les Serpentard étaient censés faire, aucun bénéfice personnel ne fut soupesé.

Au lieu de ça…

Daphné dit à Hannah, à Susan et à tout le monde que son informateur l'avait prévenu que le prochain ennemi allait viser les Poufsouffle en particulier et qu'il avait l'intention de prendre le risque de subir le courroux des professeurs, de \emph{vraiment} faire mal à Hannah ou à Susan, genre \emph{sérieusement}, et que ces deux-là devraient rester à l'écart cette fois-ci~;

Hannah avait accepté de rester à l'écart.

Susan avait…

\later

«~\emph{Qu'est-ce que tu fais là~?}~» s'écria le général Granger, bien ce fut une sorte de cri et de chuchotement mêlés.

Le visage rond de Susan ne changea pas d'expression, comme si la Poufsouffle avait soudain développé le genre d'air compassé que la mère de Daphné utilisait.

«~Je suis là, vraiment~? dit Susan d'un ton calme.

--- \emph{Tu avais dit que tu ne viendrais pas~!}

--- J'ai dit ça~?~» dit Susan. Elle faisait tourner sa baguette d'une main, nonchalamment, appuyée contre le mur de pierre du couloir où elles attendaient, ses cheveux châtain-rouge toujours parfaitement disposés autour des bordures jaunes de sa robe de sorcière. «~Je me demande pourquoi. Peut-être que je ne voulais pas que Hannah se fasse des idées. Loyauté Poufsouffle, tout ça.

--- Si tu ne pars pas, dit le général Soleil, j'ordonnerai une annulation de la mission et nous retournons \emph{toutes} à nos devoirs, Mlle Bones~!

--- \emph{Hé~!} dit Lavande. On n'a pas voté pour…

--- Ça me va~», dit Susan, qui regardait fixement l'autre bout du couloir, là où il débouchait sur la grande pièce dallée où on leur avait dit qu'elles trouveraient la brute. «~Je resterai là toute seule alors.

--- Pourquoi…~» dit Daphné. Elle avait le cœur au bord des lèvres. \emph{Si j'essaie de changer ça, si} quiconque \emph{essaie de changer ça, des choses vraiment terribles, atroces, pas bien, extrêmement mauvaises se produiront. Et alors ça aura lieu quand même.}… «~Pourquoi est-ce que tu fais ça~?

--- Ça ne me ressemble pas, dit Susan. Je sais. Mais…~» elle haussa les épaules. «~Les gens ne se ressemblent pas tout le temps, tu sais.~»

Elles plaidèrent.

Elles supplièrent.

Susan ne parlait même plus, elle ne faisait que regarder, attendre.

Daphné pleurait presque, elle continuait à se demander si elle était la \emph{cause} de tout cela, si essayer de changer le destin avait \emph{empiré} les choses…

«~Daphné, dit Hermione d'une voix plus aiguë que d'habitude, va chercher un professeur. Cours.~»

Daphné pivota sur ses talons et partit à toute vitesse vers l'autre bout du couloir pierreux, puis elle comprit, se retourna et revint voir toutes les autres filles qui l'avaient regardée partir, sauf Susan, et Daphné, avec l'impression d'être sur le point de vomir, dit~:

«~Je ne peux pas…

--- \emph{Quoi~?} dit Hermione.

--- Parfois ça devient pire quand on essaie de lutter~», dit Daphné. C'était ce qui arrivait parfois dans les pièces.

Hermione la regarda fixement puis dit~: «~Padma.~»

L'autre Serdaigle s'arracha sans discuter. Daphné la regarda partir, sachant que Padma ne courait pas aussi vite qu'elle, se demandant à présent si cela s'avérerait peut-être être la \emph{seule} raison pour laquelle l'aide allait arriver trop tard…

«~Les brutes sont là, dit Susan d'un ton laconique. Oh, elles ont un otage.~»

Elles tournoyèrent de concert, regardèrent, et virent…

\emph{Trois} brutes plus âgées qu'elles, les yeux de Daphné reconnurent Reese Belka, qui était un lieutenant de haut rang dans l'une des armées de septième année, Randolph Lee, qui était numéro deux du club de duel de Poudlard, et pire que tout, Robert Jugson III, en sixième année, dont le père était presque certainement un Mangemort.

Ils étaient tous les trois entourés de sortilèges de protection qui formaient des nuages bleus scintillants sous une surface de rubans d'autres couleurs et qui laissaient parfois voir des facettes externes, des boucliers à plusieurs couches, comme s'ils s'attendaient à se battre contre des duettistes expérimentés et qu'ils avaient fait les efforts adéquats pour s'y préparer.

Et derrière eux, attachée et soutenue par des cordes lumineuses, Hannah Abbott. Ses yeux étaient écarquillés sous l'effet de la panique, sa bouche bougeait, mais elles ne pouvaient rien entendre à cause du \emph{Quietus} qu'elles avaient lancé plus tôt.

Puis Jugson fit un geste désinvolte de la main et les cordes lumineuses firent voler Hannah vers elles, il y eut un petit “pop” lorsque le corps de Hannah franchit la barrière de silence, mais la baguette de Susan était déjà pointée vers Hannah et cette dernière marmonna~:

«~\emph{Wingardium Leviosa}…

--- \emph{Courez~!}~» s'écria Hannah alors qu'elle se faisait doucement descendre jusqu'au sol.

Mais le couloir derrière et devant elles étaient maintenant obstrué par un champ gris lumineux, un sortilège de barrière que Daphné ne reconnut pas.

«~Ai-je besoin d'expliquer ce dont il s'agit~?~» dit Lee d'un ton faussement jovial. Le duettiste en septième année avait un sourire qui n'atteignait pas ses yeux. «~Eh bien, juste au cas où, bande de petites gueuses, et cela vous inclut, Mlle Greengrass, vous avez causé suffisamment de problèmes et vous avez assez menti. Nous avons amené votre jeune amie juste pour nous assurer que tout le monde sache que nous vous avons toutes eues -- même si j'imagine que l'autre Serdaigle se cache dans un coin ou est accrochée quelque part au plafond~? Eh bien, peu importe. Ceci est votre…

--- Assez parlé, dit Robert Jugson III, c'est l'heure d'avoir mal~», et il leva sa baguette~: «~\emph{Cluthe~!}~»

Susan leva immédiatement la sienne, dit~: «~\emph{Prismatis~!}~», et une petite sphère arc-en-ciel apparut en l'air presque au même instant, une barrière si dense et lumineuse qu'elle demeura intacte même lorsque le sortilège de Jugson la frappa et rebondit vers Belka dont la baguette alla écraser le projectile noir~; puis un moment plus tard la brume multicolore fut partie.

Les yeux de Daphné s'écarquillèrent pendant quelques secondes~; elle n'avait jamais pensé à utiliser une sphère prismatique comme \emph{ça}…

«~Jugsy chéri~?~» dit Belka. Ses lèvres s'ouvrirent en un sourire vicieux. «~Je pensais qu'on en avait parlé. D'abord on les bat, \emph{ensuite} on joue.

--- S-s'il vous plaît, dit Hermione Granger d'une voix défaillante, laissez-les partir -- je, je, je promets que je…

--- Oh, vraiment, dit Lee avec agacement. Es-tu sur le point de proposer de te rendre si on les laisse partir~? On vous a déjà \emph{toutes}, tu sais.~»

Jugson sourit alors. «~Ça pourrait être amusant, dit le Mangemort de sixième année d'un ton à la fois doux et menaçant. Et si tu léchais mes chaussures, Sang-de-Bourbe, pour qu'\emph{une} de tes amies puisse partir~? Choisis celle que tu aimes le plus, les autres souffriront.

--- Nan, dit la jeune voix de Susan Bones, aucune chance que ça arrive,~» et d'un geste trop rapide pour être ne serait-ce que suivi du regard, la Poufsouffle se jeta sur le côté au moment même où un tir d'étourdissement pourpre fusait de la baguette de Belka~; Daphné put à peine \emph{voir} le mouvement lorsque Susan sembla entrer en collision avec le mur du couloir, rebondir comme s'il avait été en caoutchouc, et que ses jambes s'écrasèrent dans le \emph{visage} de Jugson, sans traverser le bouclier mais le sixième année s'effondra en arrière sous la puissance de l'impact, et Susan suivit le mouvement, son pied s'écrasa sur le bras armé du garçon et fut de nouveau repoussé par le bouclier, puis Lee hurla alors «~\emph{Elmekia~!}~» mais Parvati hurla «~\emph{Prismatis~!}~» et le mur arc-en-ciel apparut, mais le rayon bleu le traversa comme s'il n'avait pas été là et manqua Susan de quelques centimètres, puis il y eut un tourbillon de mouvement que Daphné ne put suivre durant lequel les pieds de Belka furent balayés mais la sorcière plus âgée se releva juste d'une roulade et alors…

Daphné les vit venir, ses lèvres commencèrent à dire «~\emph{Pris…}~» mais il était déjà trop tard.

Trois projectiles étincelants se fracassèrent en même temps sur Susan, qui avait levé sa baguette comme si elle aurait pu les contrer, et il y eut un éclair blanc lorsque les sortilèges touchèrent le bois enchanté mais les jambes de Susan furent alors prises de convulsions qui l'envoyèrent voler vers le mur. Sa tête heurta celui-ci avec un étrange son de craquement et Susan fut alors étendue, immobile, sa tête formant un angle visiblement étrange par-rapport au reste de son corps, sa baguette toujours serrée dans sa main tendue.

Il y eut un instant de silence glacial.

Parvati fonça vers l'endroit où Susan était étendue, pressa son pouce contre le poignet de Susan, là où l'on aurait pu sentir son pouls, puis, lentement, en tremblant, elle se leva, yeux écarquillés…

«~\emph{Vitalis Revelio},~» dit Lee alors même que Parvati ouvrait la bouche, et le corps de Susan fut entouré d'une lueur rouge et chaude. Le septième année souriait vraiment à présent. «~Probablement juste une clavicule brisée, je dirais. Bien essayé, cela dit.

--- Merlin ce qu'ils \emph{sont} rusés, dit Jugson.

--- Vous m'avez eue pendant quelques secondes, mes chéris.~» La septième année ne souriait pas du tout.

«~\emph{Tonare~!}~» s'écria Daphné en levant sa baguette au-dessus de sa tête et en se concentrant avec plus de force qu'elle ne l'avait jamais fait de sa vie. «~\emph{Rava calvaria~! Lucis…}~»

Elle ne vit même pas le sortilège qui l'abattit.

\later

Hermione sentit l'Innerver qui la réveilla et par quelque choix stratégique intuitif elle ne se releva \emph{pas} immédiatement d'une roulade~; ça avait été une bataille sans espoir et elle ne savait pas ce qu'elle pouvait faire, mais son instinct lui disait que bondir sur ses pieds ne serait pas une bonne décision.

Hermione entrouvrit à peine les yeux et les fins rayons de lumière qui entrèrent lui montrèrent Parvati qui reculait face aux trois brutes, et c'était la dernière fille debout que Hermione pouvait apercevoir.

Ses yeux lui montrèrent aussi Tracey, tombée non loin d'elle. Hermione avait encore sa baguette en main, aussi, en espérant éperdument que la Serpentard ferait montre de plus de jugeote qu'à l'habitude, Hermione mut sa baguette aussi discrètement qu'elle le pouvait et quasiment sans bouger les lèvres murmura~: «~Innerver.~»

Hermione sentit que le sortilège avait fonctionné mais Tracey ne bougea pas. Hermione espéra que c'était parce que Tracey se montrait rusée et attendait de…

Que \emph{pouvaient}-elles faire~?

Elle l'ignorait, et la panique, qui avait patienté pendant le combat, commençait à la dévorer de l'intérieur maintenant qu'elle était immobile, qu'elle essayait de penser, qu'elle pouvait voir que c'était absolument sans espoir.

C'est alors qu'elle entendit un bruit sourd, et même si cela s'était passé hors de son champ de vision, elle sut que Parvati était tombée.

Un moment de silence vint puis s'en fut.

«~Et maintenant, quoi~? dit le garçon à la voix douce-effrayante.

--- Maintenant on réveille la Sang-de-Bourbe, dit la voix précise du garçon à la voix cérémonieuse-effrayante, et on découvre qui est \emph{vraiment} derrière ça, pas le fantôme de Salazar Serpentard.

--- Non mes chéris, dit la voix angélique-effrayante de la fille, \emph{d'abord}, nous les attachons toutes \emph{très} soigneusement…~»

Et il y eut alors un son d'éclair et tonnerre, les yeux de Hermione s'écarquillèrent de surprise avant qu'elle ne puisse s'en empêcher et dans son champ de vision maintenu élargi elle vit le garçon doux-effrayant convulser sous l'impact d'arcs d'énergie jaune qui se déversaient sur lui tels d'immenses vers étincelants. Sa baguette bondit hors de sa main et il s'effondra sur le sol, agité de soubresauts, et un moment plus tard il était immobile.

«~Toutes les autres dorment, maintenant~? dit une voix. Bien.~»

Susan Bones se leva non loin de là où le garçon doux-effrayant s'était tenu, le cou toujours étrangement tordu. Puis elle fit le geste de s'assouplir la nuque, un mouvement circulaire désinvolte, et sa tête fut de nouveau droite.

La première année au visage dodu se tenait face aux deux dernières brutes, une main sur la hanche.

Elle souriait.

Et était entourée d'un nuage à facettes bleuâtre.

«~Polynectar~! cracha la fille-brute.

--- \emph{Polyfluis Reverso~!}~» rugit le dernier garçon-brute.

Sa baguette éructa quelque chose qui ressemblait au reflet d'une écharpe…

Qui passa sans rencontrer de résistance à travers le nuage qui entourait Susan…

Pendant un instant, cette dernière luit d'une étrange couleur miroir, comme un reflet d'elle-même…

Puis la lueur s'estompa.

La jeune fille se tenait toujours là, main sur la hanche.

«~Faux, dit Susan.

--- Et \emph{c'est} la vérité, continua-t-elle. Au cas où personne ne vous l'a jamais dit…~»

Dans sa main s'éleva une baguette rendue floue par le nuage bleu qui l'entourait.

«~On ne touche pas aux Pouff's~», dit Susan, et dans un éclair gris si puissant qu'il en fit mal aux yeux mi-clos de Hermione, la véritable bataille commença.

Elle continua un moment.

Une partie du plafond fondit.

La fille-brute tenta de déclarer une trêve, dit qu'ils allaient partir et emmener Jugson avec eux, et Susan rugit les syllabes d'un sortilège que Hermione sut être le Découragement Ignoble d'Abi-Dalzim, un sortilège illégal dans sept pays.

La fille-brute finit allongée au sol, inconsciente, impossible à réveiller~; le dernier garçon-brute avait fui en laissant les corps de ses compagnons derrière et Susan s'appuyait contre un mur, couverte de sueur, sa robe écorchée, trempée par endroits, haletante, et serrant son épaule droite de la main gauche.

Après s'être raidie, Susan pivota pour regarder ses camarades sorcières, endormies au sol.

Enfin, elles \emph{auraient dû} être endormies au sol.

Lavande se rasseyait déjà, les yeux gros comme des pastèques.

«~Ce… dit Lavande.

--- C'était… dit Tracey.

--- \emph{Hein~?} dit Hermione.

--- Je veux dire, \emph{quoi~?} dit Parvati.

--- \emph{Génial~!} dit Lavande.

--- Oh, bon sang~», dit Susan Bones. Son visage avait déjà pâli sous la sueur et cela ne faisait qu'empirer, lui donnant un air terriblement blanchâtre. «~Ah… pourrais-je vous convaincre que vous venez d'halluciner tout ça~?~»

Il y eut un rapide échange de regards. Hermione regarda Parvati, Parvati regarda Lavande, Lavande et Tracey échangèrent un coup d'œil.

Elles regardèrent toutes les quatre Susan et secouèrent la tête.

Et Susan s'enfuit dans le couloir à une vitesse étonnamment rapide avant que quiconque puisse dire un mot de plus.

«~Non mais sérieux, c'était \emph{quoi}~? dit Parvati.

--- \emph{Innerver}~», dit Hermione en dirigeant sa baguette vers Daphné dont elle n'avait pu apercevoir le corps plus tôt, et Lavande pointa sa baguette vers Hannah et fit de même.

Les yeux de Hannah s'ouvrirent et elle essaya désespérément de faire une roulade pour se remettre sur pied, mais elle s'effondra au sol à mi-parcours.

«~C'est bon, Hannah~! dit Lavande. On a gagné.

--- On a \emph{quoi}~?~» dit Hannah, en tas sur le sol.

Daphné n'avait pas bougé mais Hermione pouvait voir sa poitrine se soulever et s'abaisser et le rythme semblait assez normal.

«~Je pense qu'elle va bien, dit Hermione, mais…~» elle prit un moment pour déglutir car sa bouche était trop sèche. Les choses avaient vraiment, vraiment, \emph{vraiment} dégénéré. «~Je pense qu'on devrait emmener Daphné chez Mme Pomfresh…

--- Bien sûr, bien sûr, \emph{donnez-moi juste quelques minutes} et je m'en sortirai \emph{peut-être}, dit Parvati.

--- \emph{Pardon}, mais dit Daphné d'un ton poli mais ferme, comment a-t-on gagné~? Et pourquoi le plafond a-t-il l'air tout fondu~?~»

Il y eut un silence.

«~C'est Susan qui a fait ça, dit Tracey.

--- Ouais~», dit Parvati, d'une voix qui n'était que légèrement instable tout en se levant et en époussetant sa robe à bordures rouges, «~il s'avère que Susan Bones est l'héritière de Poufsouffle et qu'elle a ouvert l'entrée depuis longtemps perdue de la Chambre du Dur Labeur et de la Pratique de Helga Poufsouffle.

--- \emph{Hein~?}~» dit Hannah, qui se parcourait le corps pour vérifier que tous ses membres étaient encore là. «~Je pensais que c'était seulement quelque chose que le professeur Chourave disait pour nous enseigner une Importance Leçon de Morale -- vraiment, \emph{Susan}~?~»

Hermione reprenait lentement ses esprits. La terreur absolue n'avait pas duré plus de trente secondes, ou du moins Hermione n'avait pas été consciente plus longtemps.

«~En fait~», dit Hermione d'un ton circonspect, à mesure que son esprit se remettait à fonctionner, «~je suis assez certaine que \emph{c'est} juste quelque chose que raconte le professeur Chourave, ce n'était pas dans \emph{Poudlard~: Une Histoire} ni dans aucun autre livre que j'ai lu…

--- \emph{C'est une double sorcière~!}~» s'écria Tracey, d'une voix si aiguë qu'elle se brisa. «~Mais oui~! C'en est une~! Depuis le début~!

--- \emph{Quoi~?} hurla Parvati en se tordant pour regard Tracey. C'est le truc le plus \emph{dingue}…

--- Bien \emph{sûr}~!~» dit Lavande, à présent complètement debout, commençant à sautiller sous le coup de l'excitation. «~J'aurais dû m'en rendre compte~!

--- Susan est une \emph{quoi}~? dit Hermione.

--- Une \emph{double} sorcière~! dit Tracey.

--- Tu vois, dit Lavande en parlant très vite, il y a toujours des histoires d'enfants nés super magiciens, capables de lancer des sortilèges que personne d'autre ne peut lancer, et qu'il y a toute une école secrète à l'intérieur de Poudlard, avec des cours qu'eux seuls peuvent voir et où eux seuls peuvent se rendre…

--- Ce sont seulement des \emph{histoires}~! s'écria Parvati. Ça ne se passe pas comme ça dans la vraie vie~! Je veux dire, oui, bien sûr, j'ai lu tous ces livres…

--- Une minute, s'il vous plaît~», dit Hermione. Peut-être que son esprit \emph{était} ralenti, après tout. «~Vous voulez dire que même si vous allez \emph{déjà} à une école magique et tout ça, vous aimeriez bien aller à une \emph{double} école magique~?~»

Lavande la regarda d'un air perplexe. «~Quoi~?~» dit-elle. «~Qui ne voudrait \emph{pas} avoir des super pouvoirs magiques en plus~? Ce serait comme d'avoir un destin incroyable et tout~! Ça voudrait dire qu'on est \emph{spéciale}~!~»

Sur ces mots, Hannah hocha la tête en les observant, allongée au côté de Daphné vers laquelle elle avait rampé afin de vérifier que ses os n'étaient pas brisés. «~J'aimerais être une double sorcière~», dit alors Hannah, puis, d'un ton un peu plus triste, «~même si je ne crois pas que ça existe vraiment… vous avez vu Susan faire quoi exactement~? Je veux dire, vous êtes sûres que vous n'avez pas juste halluciné après avoir été assommées~?~»

À ce stade, Hermione ne savait vraiment, vraiment plus quoi dire.

«~Oh, non~», dit Tracey. La Serpentard pivota et regarda l'entrée du couloir, sa robe voletant autour d'elle. «~Oh non~! On doit sortir d'ici~! On doit partir avant que Susan ne revienne avec quelqu'un capable de nous lancer un sortilège de super-Oubliettes~!

--- Susan ne ferait pas une chose pareille~! dit Parvati. Même s'il y \emph{avait}…

--- \shout{Qu'est-ce qui se passe ici~?}~» rugit une voix suraiguë et couinante, et le professeur Flitwick déboula dans le couloir en partie fondu tel un petit paquet dangereusement comprimé de furie académique, suivi d'une Padma au visage de cendre qui haletait derrière lui.

\later

«~Qu'est-ce qui s'est \emph{passé~?}~» laissa échapper Susan face à la fille qui lui ressemblait comme deux gouttes d'eau mis à part la robe écorchée et imbibée de sueur.

«~Oh, excellente question~!~» dit l'autre Susan Bones tout en se débarrassant rapidement de ce qui restait des vêtements qu'elle avait empruntés. Un instant plus tard la fille commença à se métamorphoser en sa forme plus usuelle de Nymphadora Tonks. «~Désolé mais je n'ai rien su inventer alors tu as environ trois minutes pour trouver une réponse à cette…~»

\later

Comme Daphné Greengrass le fit remarquer ensuite avec quelque acerbité, l'erreur dans le plan fourbe de Hermione qui assurait que les points de maison seraient déduits de façon équitable si elles se faisaient prendre, c'était que ça ne marchait pas pour les \emph{retenues}.

Elles s'étaient toutes mises d'accord pour se taire quant aux pouvoirs mystérieux de Susan, même Tracey après que Susan eut menacé de la super-Oublietter si elle ne promettait pas. Malheureusement, elles découvrirent au dîner non seulement que quelqu'un avait oublié de parler aux \emph{brutes} de leur accord, mais aussi que Susan Bones avait sacrifié son âme à de terribles pouvoirs interdits qui habitaient maintenant sa carcasse et que c'était pour ça qu'elles étaient toutes en retenue.

«~Hermione~?~» dit Harry Potter, assis à côté d'elle au dîner, sa voix très timide. «~Ne t'offusque pas s'il te plaît, et je comprendrais si tu disais que ça n'était pas mes affaires, mais je pense de plus en plus que les choses commencent à devenir incontrôlables.~»

Hermione continua d'écraser la part de gâteau au chocolat dans son assiette pour en faire une bouillie informe de gâteau et de glaçage. «~Oui~», dit Hermione d'une voix qui était peut-être un peu acerbe, «~c'est ce que je disais au professeur Flitwick quand je m'excusais, que je savais que la situation nous avait échappée, et il a hurlé~: \emph{Vraiment, Mlle Granger, vous pensez~?} d'un couinement tellement fort que mes oreilles ont pris feu. Je veux dire que mes oreilles ont \emph{vraiment pris feu}. Le professeur Flitwick a dû les éteindre.~»

Harry s'était mis une main sur le front. «~Excuse-moi~», dit Harry. Son visage était parfaitement neutre. «~Parfois j'ai encore un peu de mal à m'habituer à ce genre de choses. Hé, Hermione, tu te souviens de l'époque où on était jeunes et naïfs et où on pensait que le monde était un endroit plus ou moins compréhensible~?~»

Hermione posa sa fourchette et le regarda un moment. «~Est-ce que ça t'arrive de souhaiter être un Moldu, Harry~?

--- \emph{Hein~?} répondit-il. Enfin, bien sûr que non~! Je veux dire, même si j'étais un Moldu, un jour où l'autre j'aurais quand même essayé de conquérir le moooooooon-~» et Hermione lui jeta un \emph{regard} et le garçon déglutit le mot avec hâte et dit «~je veux dire \emph{optimiser} bien sûr, tu \emph{sais} que c'est ce que je veux dire, Hermione~! Là où je veux en venir c'est que ce n'est pas comme si mes \emph{objectifs} allaient changer d'une manière ou d'une autre. Mais avec la magie ça va être beaucoup plus simple de faire le travail que si je n'avais eu accès qu'aux capacités moldues. Si tu y réfléchis de façon logique, c'est pour \emph{ça} que je vais à Poudlard au lieu de juste ignorer tout ça et d'étudier pour faire une carrière dans la nanotechnologie.~»

Ayant fini d'arranger sa sauce de gâteau au chocolat à sa façon, Hermione commença à y tremper ses carottes et à les manger.

«~Pourquoi cette question~? dit Harry. Est-ce que \emph{tu} aimerais revenir au monde moldu~?

--- Pas exactement~», dit Hermione, en prenant une bouchée de carotte et de chocolat. «~C'est juste que je trouvais bizarre d'avoir \emph{souhaité} être une sorcière… est-ce que tu voulais être un sorcier quand tu étais petit~?

--- Bien sûr, répondit tout de suite Harry. Je voulais aussi avoir des pouvoirs psychiques, une super-force, des os renforcés en adamantium, mon propre château volant et parfois j'étais triste de me dire que j'allais peut-être devoir me contenter d'être un scientifique célèbre et un astronaute.~»

Hermione hocha la tête.

«~Tu sais, dit-elle doucement, je pense que les sorcières et les sorciers qui \emph{grandissent} ici n'apprécient pas vraiment la magie à leur juste valeur…

--- Enfin, bien sûr que non, dit Harry, c'est ce qui nous donne notre avantage. N'est-ce pas évident~? Je veux dire, franchement, j'ai trouvé ça carrément évident cinq minutes après être arrivé au Chemin de Traverse.~» Il y avait un air perplexe sur le visage du garçon, comme s'il ne pouvait pas comprendre pourquoi elle faisait attention à quelque chose d'aussi ordinaire.
%  LocalWords:  Jugsy Elmekia Pris Vitalis Abi Dalzim’s cuse melty worrrrlllll
