\namedpartchapter[\protect\footnotemark]{L'Expérience de Prison de Stanford}{TSPE}{VIII}{Cognition sous contraintes}

\authorsnotetext{Une bande annonce du film \emph{Evil Dead 3} proche de celle que Harry a vue~: \href{https://www.youtube.com/watch?v=THV1KkPXIxQ}{THV1KkPXIxQ sur YouTube}.}

\lettrine{D}{ans}  le noir absolu, un garçon appuyait sa baguette contre le mur de métal d'Azkaban et s'essayait à une magie à laquelle seulement trois autres habitants du monde magique auraient pu croire et qu'aucun d'eux n'aurait pu manier.

Bien sûr, un sorcier puissant aurait fait un trou dans le mur en quelques secondes d'un geste silencieux.

Pour un adulte moyen, ça aurait peut-être été l'affaire de quelques minutes au prix d'une certaine fatigue.

Mais pour parvenir à cette fin quand on était un élève de Poudlard en première année, il fallait être \emph{efficace}.

Par chance - enfin, pas \emph{par chance}, la chance n'avait rien à voir là-dedans - \emdashhyp\emph{consciencieusement}, Harry avait pratiqué la métamorphose une heure de plus, tous les jours, à tel point qu'il devançait Hermione dans ce cours~; il avait tellement pratiqué la métamorphose partielle que ses pensées avaient commencé à trouver que le véritable univers était normal, si bien qu'il ne lui fallait faire qu'un léger effort supplémentaire pour garder la nature quantique et intemporelle de celui-ci à l'esprit tout en maintenant une ferme séparation mentale entre le concept de Forme et celui de substance.

Et le \emph{problème}, maintenant que cet art était devenu routine…

… c'était que Harry pouvait penser à d'autres choses tout en le pratiquant.

Ses pensées étaient mystérieusement parvenues à ne pas parvenir à ce point, à ne pas se confronter à l'évidence, jusqu'à ce qu'il se retrouve face à la perspective \emph{de le faire pour de vrai dans quelques minutes à peine.}

Ce qu'il était sur le point de faire…

… était dangereux.

Vraiment dangereux.

Dangereux comme dans "quelqu'un-pourrait-vraiment-y-passer".

Faire face à douze Détraqueurs sans Patronus avait été \emph{effrayant}, mais seulement effrayant. Il aurait pu lancer le Patronus, il \emph{aurait} lancé le Patronus dès qu'il se serait cru en danger de ne plus en être capable, dès qu'il aurait senti que sa résistance commençait à lâcher. Et même si cela n'avait pas fonctionné… même alors, à moins que les Détraqueurs n'aient reçu l'instruction d'Embrasser tous ceux qu'ils croiseraient, un échec n'aurait pas été \emph{fatal}.

Cette fois c'était différent.

L'appareil Moldu métamorphosé pourrait exploser et les tuer.

L'interface entre la technologie et la magie pourrait échouer de mille façons différentes et les tuer.

Les Aurors pourraient avoir un coup heureux.

C'était juste, eh bien…

\emph{Sérieusement dangereux.}

Harry avait pris son esprit en flagrant délit d'essayer de se faire croire que c'était sûr.

Oui le plan \emph{pourrait} marcher mais…

Mais même en omettant le fait que les rationalistes n'avaient jamais le droit de se persuader de faire quelque chose par des moyens rhétoriques, Harry savait qu'il ne pourrait jamais se convaincre d'estimer que ses chances de mourir étaient inférieures à 20~\%.

\emph{Perds}, dit Poufsouffle.

\emph{Perds}, dit la voix du professeur Quirrell dans son esprit.

\emph{Perds}, dit son modèle mental de Hermione et du professeur Quirrell et du professeur Flitwick et de Neville Londubat et, voyons voir, de toutes les personnes que Harry connaissaient mis à part Fred et George qui auraient foncé en un battement de cils.

Il aurait juste dû aller trouver Dumbledore et se rendre. Il aurait vraiment, vraiment dû faire ça, c'était la seule chose \emph{sensée} qu'il puisse faire au stade où il en était.

Et si Harry avait été seul dans cette mission, si seule sa vie avait été en jeu, c'est ce qu'il aurait fait~; c'est sûrement ce qu'il aurait fait.

Ce qui parvenait presque à le déconcentrer de la métamorphose partielle qu'il était en train d'opérer, ce qui menaçait de l'ouvrir aux Détraqueurs…

… c'était le professeur Quirrell, toujours inconscient, toujours serpent.

Si le professeur Quirrell allait à Azkaban à cause de son rôle dans l'évasion, il mourrait. Il ne tiendrait probablement pas une semaine. Il était sensible à ce point.

C'était simple.

Si Harry \emph{perdait} ici…

Il perdrait le professeur Quirrell.

\emph{Alors qu'il est probablement maléfique}, dit doucement sa partie Poufsouffle. \emph{Même alors} \emph{?}

Ce n'était pas une décision que Harry avait prise consciemment. Il en était simplement incapable. On pouvait perdre des points de Maison mais pas des \emph{gens}.

\emph{Si tu penses que ta vie a assez de valeur pour refuser 80~\% de chances de mourir en contrepartie de la vie de tous les prisonniers d'Azkaban}, nota son côté Serpentard, \emph{tu ne peux pas justifier un risque de mourir 20~\% en échange de la vie de Bellatrix et du professeur Quirrell. Mathématiquement ça ne tient pas, tu n'es pas en train d'assigner des utilités cohérentes aux conséquences.}

Son côté logique remarqua que Serpentard venait de remporter le débat.

Harry maintint la Forme dans son esprit et continua le sortilège. Il pourrait toujours annuler la mission quand il en aurait \emph{fini} avec la métamorphose mais il ne voulait pas gâcher l'effort qu'il avait déjà investi.

Et il pensa alors à quelque chose qui rendit soudain très ardue la tâche maintenir sa magie et sa résistance face aux Détraqueurs.

\emph{Et si le Portoloin ne nous amène pas là où le professeur Quirrell a dit qu'il nous amènerait~?}

Avec le recul c'était évident dès l'instant où l'on y songeait.

Même si l'évasion se déroulait exactement comme prévue, même si l'appareil Moldu fonctionnait et n'explosait \emph{pas} et que son interaction avec l'objet magique auquel il était lié ne tournait \emph{pas} mal, même si les Aurors n'avaient pas de coup chanceux, même si Harry s'éloignait suffisamment d'Azkaban pour utiliser le Portoloin…

… il n'y aurait peut-être pas de Guérisseur psychiatrique de l'autre côté.

Il y avait cru quand il avait fait confiance au professeur Quirrell et il avait oublié de réévaluer cette idée après que le professeur Quirrell se fut révélé ne pas être digne de confiance.

\emph{Tu ne peux pas faire ça}, dit Poufsouffle. \emph{À ce stade c'est de la simple bêtise.}

Le froid sembla se répandre dans la pièce, mais Harry maintint la métamorphose alors même que sa résistance contre le Détraqueurs fléchissait.

\emph{Je ne peux pas perdre le professeur Quirrell.}

\emph{Il a essayé de tuer un officier de police}, dit Poufsouffle. \emph{C'est là que tu l'as perdu. Bellatrix est probablement ce que tout le monde pense qu'elle est. Reprends juste ta Cape, vas trouver Dumbledore et dis-lui que tu as été dupé.}

\emph{Non}, pensa Harry avec désespoir, \emph{pas sans parler au professeur Quirrell, il y a peut-être une explication, je ne sais pas, peut-être qu'il se tenait assez loin de mon Patronus pour que les Détraqueurs l'atteignent… je ne comprends pas, ça n'a aucun sens, quelle que soit l'hypothèse, pourquoi ferait-il cela… je ne peux juste pas…}

Harry se détourna de ce fil de pensée avant que celui-ci ne brise entièrement sa résistance à la peur car il ne pouvait pas s'imaginer donner le professeur Quirrell à manger aux Détraqueurs tout en restant résolument opposé à la Mort~; c'était une impossibilité cognitive.

\emph{Tes facultés de raisonnement sont artificiellement réduites}, observa la partie logique de son être d'un ton calme, \emph{trouves un moyen de les rétablir.}

\emph{Très bien, générons seulement des alternatives}, pensa Harry. \emph{Sans choisir, sans soupeser, certainement sans s'engager… réfléchissons juste à ce que je pourrais faire d'autre que le plan original.}

Et Harry continua de percer le mur. Il utilisait la métamorphose partielle sur une coquille de métal cylindrique et fine de deux mètres de diamètre et d'un demi-millimètre d'épaisseur qui allait ainsi d'un bout à l'autre du mur. Il métamorphosait ce demi-millimètre d'épaisseur de métal en huile de moteur. L'huile de moteur était un liquide et l'on était pas censé métamorphoser de liquides car ils risquaient de s'évaporer mais lui, Bellatrix et le serpent avaient tous reçu un sortilège de bulle. Et Harry lancerait Finite sur l'huile juste après, dissipant ainsi sa métamorphose…

… dès que le morceau de métal ainsi séparé et lubrifié glissa hors du mur jusqu'au sol de la cellule, il l'inclina pour que la gravité l'attire une fois la métamorphose complétée.

Si Harry et Bellatrix \emph{ne sortaient pas} sur son balai volant par le trou ainsi formé…

Le cerveau de Harry suggéra qu'il essaye de métamorphoser une surface qui recouvrirait le trou dans le mur, laissant ainsi assez d'espace pour que Bellatrix et le professeur Quirrell s'y cachent sous la Cape pendant que Harry se rendrait. Le professeur Quirrell finirait par se réveiller et lui et Bellatrix pourraient trouver un moyen d'échapper d'Azkaban par leurs propres moyens.

Tout d'abord c'était une idée stupide, et ensuite il y aurait toujours un énorme morceau de métal sur le sol de la cellule, ce qui serait assez révélateur.

Puis le cerveau de Harry vit l'évidence.

\emph{Laisse Bellatrix et le professeur Quirrell utiliser l'échappatoire que tu as inventée. Tu restes derrière et tu te rends.}

C'étaient les vies de Bellatrix et du professeur Quirrell qui étaient en jeu.

Ils gagnaient plus qu'ils ne perdaient en prenant le risque.

Et il n'y avait aucun raison, aucune raison sensée pour que Harry les accompagne.

Lorsqu'il eut cette pensée, un calme recouvrit Harry, le froid et les ténèbres qui avaient ondulé à la bordure de son esprit battirent en retraite. Oui, c'était cela, c'était le chemin hors des sentiers battus, c'était la troisième alternative masquée. La fausseté du dilemme était rétrospectivement évidente. Si Harry se rendait, il \emph{n'aurait pas} à livrer Bellatrix et le professeur Quirrell. Si Bellatrix et le professeur Quirrell prenaient l'échappatoire dangereuse, Harry \emph{n'aurait pas} à les suivre.

Il n'aurait même pas à faire face à la honte d'admettre qu'il avait été dupé s'il ordonnait à Bellatrix d'effacer sa mémoire. Tout le monde partirait du principe qu'il avait été kidnappé, Harry y compris. Il fallait convenir qu'il n'existait aucun raison plausible pour laquelle le Seigneur des Ténèbres demanderait bien à Bellatrix de faire une chose pareille mais Harry pourrait se contenter de sourire, de dire à Bellatrix qu'elle n'avait pas le droit de savoir et l'affaire serait dans le sac…

\later

Son équipe d'Aurors avait descendu les trois quarts d'Azkaban, à l'instar des deux autres équipes dans les deux autres spirales. Amélia se sentait déjà plus tendue même si elle pariait sur le fait que les criminels se cachaient à l'avant-dernier étage, juste au-dessus du rez-de-chaussée~; une partie d'elle souhaitait que Dumbledore ait pensé à vérifier cet étage plus précautionneusement et une autre était heureuse qu'il ne l'ait pas fait.

Puis il y eut un bruit lointain, comme un petit 'tink' venu de très loin. Disons comme un son très puissant qui serait venu de l'avant-dernier étage juste au-dessus du rez-de-chaussée.

Amélia regarda Dumbledore avant de comprendre, avant de réussir à s'interrompre.

Le vieux sorcier haussa les épaules, lui offrit un petit sourire et dit~: "Puisque tu le demandes, Amélia," et il partit de nouveau.

\later

"\emph{Finite Incantatem}," dit Harry à l'huile qui recouvrait le gigantesque morceau de mur posé au sol. Il s'était à peine entendu car ses oreilles résonnaient encore après l'immense blam provoqué par le glissement suivi de la chute du bloc de métal (rétrospectivement, il aurait dû lancer un sortilège d'assourdissement, même si cela n'aurait pas empêché le bruit de se répandre à travers le sol de métal). Puis il répéta "\emph{Finite Incantatem}" à l'intention de l'huile qui recouvrait le trou de deux mètres de diamètre, en répartissant l'effet aussi largement que possible~; c'était sa propre magie qu'il annulait, ce qui rendait le sortilège extrêmement facile. Il se sentait maintenant un peu fatigué mais c'était la dernière fois qu'il aurait besoin de sa magie. Il n'avait même pas eu \emph{besoin} de le faire, mais il voulait pas laisser un liquide métamorphosé dans les parages et il ne voulait pas non plus trahir le secret de la métamorphose partielle.

Ce trou de deux mètres menant droit à la liberté semblait très… \emph{attirant}.

La lumière qui venait de l'extérieur… ce n'était pas exactement comme si le soleil avait luit sur son visage mais c'était plus lumineux que l'intérieur d'Azkaban.

Il \emph{était} tenté de se lancer, de juste sauter sur le balai volant avec Bellatrix et le serpent. Ils \emph{avaient} de bonnes chances de s'en sortir. Et \emph{s'ils} s'en sortaient et que Harry les accompagnait, alors lui et le professeur Quirrell pourraient remonter le temps, avoir l'air parfaitement innocents, et tout redeviendrait comme avant.

S'il restait derrière et qu'il se rendait… alors même si tout le monde partait du principe qu'il avait été un otage et qu'il avait menti au professeur McGonagall sous la menace d'une baguette… même si Harry s'en tirait à bon compte, eh bien…

Il était peu probable que le professeur Quirrell continue d'enseigner à Poudlard.

Le professeur Quirrell aurait atteint la fin prédestinée de sa carrière en février.

Et oui, le professeur McGonagall tuerait Harry, et oui, ce serait lent et douloureux.

Mais rester derrière était le choix raisonnable, le choix sûr, le choix \emph{sensé}, et cela relaxait Harry plus que cela ne l'emplissait de regrets.

Il se tourna vers Bellatrix, ouvrit la bouche pour lui donner un dernier ordre -

Et il y eut un sifflement, faible, comme lent et confus, et ce sifflement dit~:

"\parsel{Quel était… cce sson~?"}

\later

Le vieux sorcier avançait à grand pas. Il parvint à une porte de métal et l'ouvrit, sachant déjà que les cellules à l'intérieur étaient vides.

Il prononça sept incantations puissantes et révélatrices puis il continua~; avec si peu de cellules restantes à vérifier, il ne s'épuiserait pas beaucoup.

\later

"\parsel{Professseur,}" siffla Harry. Tant d'émotions remontaient à la surface d'un seul coup. Il savait, même s'il ne pouvait pas le voir, que le serpent vert autour des épaules de Bellatrix levait lentement sa tête et regardait autour de lui. "\parsel{Allez-vous… allez-vous bien, professseur~?}"

"\parsel{Professseur~?}" le sifflement parvint faible et confus. "\parsel{Où ssommes-nouss~?}"

"\parsel{En prison}," siffla Harry, "\parsel{la prison des mange-vie, nous devions ssauver une femme vous et moi. Vous avez esssayé de tuer l'homme protecteur, j'ai bloqué votre ssort de mort, il y a eut une résonansce entre nous… vous avez ssombré dans l'inconsscience, j'ai dû vaincre le protecteur moi-même… mon charme gardien a été disssipé, les manges-vies ont pu dire aux protecteurs que la femme s'était échappée. Il y a quelqu'un ici qui peut resssentir mon charme gardien, probablement le maître de l'école… j'ai dû disssiper charme gardien, trouver un autre moyen de vous cacher des manges-vies, vous et la femme, ssans charme gardien, apprendre à me protéger ssans charme gardien, à effrayer les manges-vie sans charme gardien, à inventer un nouveau plan d'évassion pour vous et la femme et enfin à percer un trou dans l'épais mur de métal de la prisson même ssi je ne ssuis qu'un élève de première année. Pas le temps d'expliquer, vous devez partir maintenant. Si nous ne nous recroissons jamais, professseur, alors je ssuis heureux de vous avoir connu pour un temps, même si vous êtes probablement maléfique. Il est bon d'avoir la chance de dire au moins cela~: Au revoir."}

Et Harry prit le balai volant, le présenta à Bellatrix, et dit simplement~: "Monte."

Il avait décidé de garder ses souvenirs. D'une part, ils étaient importants. D'autre part, lui et le professeur de Défense avaient commencé à planifier cela une semaine auparavant et Harry n'allait ni oblitérer la semaine entière \emph{ni} expliquer à Bellatrix ce qui devait précisément être oblitéré. Il pourrait probablement tromper le Veritaserum et si Dumbledore exigeait qu'il laisse tomber ses barrières Occlumantiques pour une inspection plus profonde… eh bien, il s'était comporté en héros du début à la fin.

"\parsel{Arrête~!}" dit le serpent. Sa voix était maintenant plus forte. "\parsel{Arrête, arrête, arrête~! Que veux-tu dire, au revoir~?}"

"\parsel{Plan d'évasion est rissqué,}" dit Harry. "\parsel{Ma vie n'est pas en jeu, sseulement la vôtre et la ssienne. Alors je resste, je me rends -"}

"\parsel{Non~!"} dit le serpent. Le sifflement était vigoureux. "\parsel{Ne doit pas~! Pas permis~!}"

Bellatrix monta sur le balai de feu~; Harry put sentir (mais il ne put pas voir) sa tête se tourner vers lui. Elle ne dit rien. Elle l'attendait, peut-être, ou simplement ses ordres.

"\parsel{Ne vous fais plus confiance,}" dit simplement Harry. "\parsel{Pas depuis que vous avez esssayé d'abattre homme protecteur."}

Et le serpent siffla~: "\parsel{Je n'ai pas cherché à abattre homme protecteur~! Es-tu idiot, garçon~? L'abattre n'aurait aucun sens, maléfique ou pas~!}"

La Terre cessa de tourner sur son axe, interrompit son orbite autour du soleil.

Le sifflement du serpent était le son le plus furieux que Harry avait jamais entendu sortir de la bouche du professeur Quirrell. "\parsel{L'abattre~? Ssi j'avais voulu l'abattre il sserait mort en quelques ssecondes, garçon idiot, il ne faissait pas le poids face à moi~! Je cherchais à le soumettre, le dominer, le forcer à abaissser les boucliers de sson essprit, devais lire en lui, savoir qui attendait ssa réponsse, apprendre détails pour ssortilège de mémoire -"}

"\parsel{Vous avez jeté ssortilège de mort~!"}

"\parsel{Savait qu'il éviterait~!"}

"\parsel{Sa vie valait-elle si peu~? Et s'il n'avait pas évité~?"}

"\parsel{L'aurais pousssé hors de trajectoire avec ma propre magie, garçon idiot~!"}

Encore l'arrêt giratoire de la planète. Harry n'y avait pas pensé.

"\parsel{Sstupide cancre de tous les consspirateurs,}" siffla le serpent avec tant de colère que les sifflements semblaient se chevaucher et se mordre la queue, "\parsel{imbécile dégourdi, idiot futé, ssot de Sserpentard pas entraîné, ta méfiance mal placée a ruiné -"}

"\parsel{N'est pas le bon moment pour disscuter,"} fit remarquer Harry avec douceur. La brusque montée de soulagement qui tentait de se répandre en lui était annulée par la tension montante. "\parsel{Puissque je ne peux pas me mettre correctement en colère contre vous ssans m'ouvrir aux manges-vies. Doit se dépêcher, quelqu'un a peut-être entendu bruit -"}

"\parsel{Explique plan d'évassion,"} dit le serpent d'un ton impérieux. "\parsel{Presstement~!}"

Harry expliqua. Le Fourchelangue n'avait pas de mots pour décrire la technologie Moldue mais Harry décrivit la fonction et le professeur Quirrell sembla comprendre.

Il y eut de courts sifflements, l'équivalent reptilien d'un aboiement de rire surpris, puis des ordres vifs. "\parsel{Dis à la femme de détourner le regard, lance ssortilège de ssilence, mets charme gardien devant porte. Vais me transsformer, faire rapides améliorations à ton invention, donner potion d'urgence à femme pour qu'elle puissse nous protéger, me transsformer de nouveau avant que tu ne disssipes le charme. Plan ssera alors plus sûr."}

"\parsel{Et dois-je croire,"} siffla Harry, "\parsel{que Guérissseur pour femme nous attend vraiment~?}"

"\parsel{Utilise raison, jeune garçon~! Suppose que je ssuis maléfique. Arrêter de t'utilisser maintenant n'est évidemment pas ce que j'avais prévu. Misssion est opportunisste, inventée après avoir vu ton charme gardien, affaire scenssée ne pas être remarquée, caché quand parti de lieu où on mange. Évidemment que tu verras perssonne prétendant être Guérissseur à l'arrivée~! Retourner à lieu où l'on mange ensuite, plan original continue comme avant~!"}

Harry regarda le serpent invisible.

D'un côté, l'entendre dit ainsi donnait à Harry l'impression qu'il était stupide.

De l'autre, ce n'était pas vraiment rassurant.

"\parsel{Alors,}" siffla Harry, "\parsel{quel est votre plan pour moi, exactement~?}"

"\parsel{Tu as dit pas le temps,}" lui répondit le sifflement du serpent, "\parsel{mais le plan est que tu diriges le pays, évidemment, même ton jeune ami noble a déjà compris cela, demande-lui au retour si tu ssouhaites. Ne dirai pas plus, temps de voler, pas de parler.}"

\later

Le vieux sorcier tendit la main vers une autre porte de métal, de derrière laquelle provenait un macabre marmonnement sans fin~: "Je ne suis pas sérieux, je ne suis pas sérieux, je ne suis pas sérieux…" Le phénix rouge-or criait déjà d'un ton impérieux et le vieux sorcier grimaçait déjà quand -

Un autre cri traversa le couloir, semblable à celui d'un phénix sans l'être tout à fait.

Le vieux sorcier tourna la tête et regarda la créature d'argent étincelante qui s'était trouvée sur son autre épaule au moment même où les serres éphémères et immatérielles lançaient l'entité magique dans les airs.

Le faux phénix s'envola le long du couloir.

Le vieux sorcier se précipita après lui à une allure digne d'un jeune soixantenaire fringant.

Le véritable phénix cria une fois, deux fois et trois fois tout en voletant devant la porte de métal~; puis, lorsqu'il devint clair qu'en dépit de ses appels son maître ne reviendrait pas, il le suivit avec récalcitrance.

\later

Cette fois le professeur Quirrell avait revêtu sa véritable forme - le Polynectar ne durait qu'une heure s'il n'était pas ré-administré - et malgré sa pâleur, avachi contre les barres de métal de la cellule la plus proche, sa magie fut assez forte pour attirer sa baguette à lui sans qu'un mot fut prononcé, alors qu'au même moment Bellatrix ôtait la Cape et la plaçait d'un geste servile dans les mains ouvertes de Harry. À mesure que les pouvoirs du professeur de Défense revenaient et que la périphérie de sa vaste puissance entrait en contact avec la légère aura enfantine de Harry, la sensation funeste s'intensifiait de nouveau sans néanmoins atteindre sa force maximale.

Harry décrivit l'appareil Moldu à voix haute et en donna le nom au sorcier, puis un Finite de Harry défit tout ce dur travail et le fit redevenir un cube de glace. Le professeur Quirrell ne pouvait pas lancer de sortilèges sur quelque chose que Harry avait métamorphosé car cela aurait constitué une interaction entre leurs magies, aussi légère soit-elle, mais -

Trois secondes plus tard, le professeur Quirrell tenait entre ses mains sa propre version métamorphosé de l'appareil Moldu. Un seul mot aboyé et un geste de sa baguette, et le résidu de colle avait quitté l'objet magique~; trois incantations plus tard, les objets magique et technologique étaient fusionnés l'un dans l'autre comme s'ils n'avaient été qu'un seul, et des sortilèges d'incassabilité et de fonctionnement parfait avaient été lancés sur l'appareil Moldu.

(Harry se sentait beaucoup mieux à l'idée de faire cela sous la supervision d'un adulte)

Une potion fut jetée à Bellatrix, et le professeur Quirrell et Harry ordonnèrent tous deux~: "Bois" comme s'ils avaient parlé de la même voix. La femme émaciée avait déjà commencé à lever la potion vers ses lèvres sans attendre car il aurait été évident aux yeux de quiconque que cet Animagus serpent était un serviteur de Seigneur des Ténèbres, un serviteur puissant et de confiance.

Harry acheva de placer la capuche de la Cape d'Invisibilité par-dessus sa tête.

Une magie brève et terrible se déchaîna depuis la baguette du professeur de Défense, creusant le trou dans le mur, balafrant le morceau de métal qui reposait au milieu de la pièce comme Harry l'avait demandé, expliquant que la méthode qu'il avait utilisée aurait pu permettre de l'identifier.

"Gant gauche," dit Harry à sa bourse, puis il l'extirpa et l'enfila.

Pendant que la femme finissait de boire la potion, un geste du professeur de Défense fit apparaître un harnais sur les épaules de Bellatrix ainsi qu'un autre morceau de tissu plus petit sur la main de celle-ci et quelque chose qui ressemblait à des menottes autour de ses poignets.

Une étrange et malsaine couleur sembla se répandre sur le pâle visage de Bellatrix et elle se raidit, ses yeux creusés semblèrent s'éclairer, devenir bien plus dangereux…

… de petites volutes de vapeur s'échappèrent de ses oreilles…

(Harry décida de ne pas penser à cela)

… et Bellatrix Black rit alors, un rire soudain et fou, bien trop fort dans les petites cellules d'Azkaban.

(Très bientôt, avait dit le professeur Quirrell, Bellatrix sombrerait dans l'inconscience et resterait ainsi pendant un bon moment~; c'était le prix de la potion qu'elle avait bue~; mais pour quelques moments encore elle posséderait peut-être un vingtième de la puissance dont elle avait un jour disposé).

Le professeur de Défense jeta sa baguette en direction de Bellatrix et un instant plus tard devint un serpent vert.

Un instant après \emph{ça}, la peur des Détraqueurs revint dans la pièce.

Bellatrix ne broncha que légèrement, attrapa la baguette et la bougea sans prononcer une parole~; le serpent s'éleva et fut inséré dans le harnais placé sur son dos.

Harry dit "Debout~!" à son balai volant.

Bellatrix plaça sa baguette dans l'étui lié à sa main.

Harry bondit en tête du balai volant deux places.

Bellatrix le suivit, prit les objets semblables à des menottes placés autour de ses poignets et s'enchaîna au manche du balai de feu tandis que la main droite de Harry fourrait sa baguette dans sa bourse.

Et ils s'élancèrent tous les trois à travers le trou dans le mur -

- émergeant à découvert directement au-dessus de la fosse des Détraqueurs, à l'intérieur du vaste prisme triangulaire qu'était Azkaban. Le ciel bleu était maintenant clairement visible au-dessus de leur tête et les éclairait de sa lumière.

Harry orienta le balai et commença à accélérer vers le haut, vers le centre de l'espace triangulaire. Sa main droite, gantée afin d'empêcher tout contact direct entre sa peau et quelque chose que le professeur Quirrell avait métamorphosé, tenait l'interrupteur qui contrôlait l'appareil Moldu.

Venus de haut au-dessus d'eux, des cris lointains leur parvinrent.

\emph{Très bien, bande de nazes primitifs~!}

Apparus au coin du ciel, des Aurors montés sur de rapides balais de course plongèrent droit vers eux accompagnés par de faibles étincelles de lumières, révélant ainsi que les premiers coups venaient d'être tirés.

\emph{Écoutez-moi~!}

"Protego Maximus~!" cria Bellatrix d'une voix à la fois puissante et fêlée, puis elle eut un rire caquetant alors qu'un champ de force bleu et chatoyant se formait autour d'eux.

\emph{Vous voyez ça~?}

Du fond de la fosse pourrissante située au centre d'Azkaban, plus d'une centaine de Détraqueurs s'élevèrent. Ils apparaissaient aux yeux de certains comme une immense masse de corps, un cimetière volant~; aux yeux d'un autre comme un conglomérat d'absences qui semblait former une seule et unique déchirure sur le monde à mesure qu'ils glissaient vers le haut.

\emph{Ceci…}

La voix d'un vieux et puissant sorcier mugit une terrible incantation et un grand éclat de feu blanc-or passa par le trou dans le mur d'Azkaban, informe l'espace d'un instant, avant que des ailes ne commencent à se former.

\emph{Est…}

Et les Aurors activèrent le sortilège Anti-Anti-Gravité qui avait été incorporé à Azkaban, désactivant ainsi tous les sortilèges de vol dont les incantations n'avaient pas été prononcées avec les mots de passe récemment changés.

La sustentation quitta le balai de Harry.

La gravité en revanche demeura.

La montée de leur balai ralentit, commença à décélérer, entama sa transformation en une chute.

\emph{Mon…}

Mais les sortilèges qui permettaient de faire pointer le balai dans une direction et de piloter, les sortilèges qui maintenaient les occupants attachés et relativement protégés de l'accélération, \emph{ces} sortilèges fonctionnaient encore.

\shout{Balais volant~!}

Harry alluma le contact de la fusée à combustible solide de classe N à propergol composite à perchlorate d'ammonium modèle \emph{Berserker PFRC} de chez General Technics qui avait été liée à son balai volant Nimbus \abbrev{X200} deux places.

Et il y eut du bruit. 

%  LocalWords:  TSPE unimpair tink screwheads Technics PFRC X200
