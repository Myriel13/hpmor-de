\partchapter{L'Expérience de Prison de Stanford}{II}

\lettrine{L}{'adrénaline} coulait déjà dans les veines de Harry, son cœur battait déjà contre sa poitrine, dans ce magasin obscur, en ruines. Le professeur Quirrell en avait fini avec ses explications, et Harry tenait d'une main la petite brindille de bois qui serait la clé de tout. Aujourd'hui. C'était le jour, l'instant où Harry commençait à jouer son rôle. Sa première véritable aventure, un donjon à pénétrer, un gouvernement maléfique à défier, une vierge en détresse à secourir. Harry aurait dû être plus effrayé, plus réticent, mais au lieu de cela il avait la sensation qu'il était temps, et grand temps, de devenir le genre de personne qu'il avait découvert dans ses livres~; de commencer son voyage vers ce qu'il avait toujours su être sa destination~: être un héros. De faire le premier pas sur la route qui menait au Kimball Kinnison et au capitaine Picard et à Liono et à Thundero et certainement \emph{pas} à Raistlin Majere. Selon ce que le cerveau de Harry avait appris de visionnage de dessins animés matinaux, on était censé obtenir d'incroyables pouvoirs et sauver l'univers en grandissant, c'était cela qu'il avait observé chez les adultes, cela qu'il avait adopté comme exemple de processus de maturation à suivre, et il avait très envie de commencer à grandir.

Et si le motif de l'histoire requérait que le héros perde une partie de son innocence en conséquence de sa première aventure~; alors pour le moment, en cet instant encore innocent, il semblait temps, et grand temps, de vivre cette douleur. Comme de délaisser des habits trop petits pour lui~; ou comme de progresser enfin au prochain niveau après avoir été bloqué pendant onze ans au troisième, ou même au deuxième monde de Super Mario Bros.

Harry avait lu suffisamment de romans pour soupçonner qu'il ne serait pas aussi enthousiaste après-coup, alors il en profitait tant que ça durait.

Il y eut un claquement proche lorsque quelque chose non loin de Harry disparut, et le temps des ruminations héroïques fut passé.

La main de Harry cassa la petite brindille de bois.

Un crochet immobile s'accrocha sur l'abdomen de Harry alors que le Portoloin s'activait, l'attraction cette fois beaucoup plus forte que lors des déplacements plus courts entre Poudlard et le Chemin de Traverse…

… le déposa au milieu d'un énorme grondement de tonnerre mourant, une ondée de pluie froide le fouetta au visage, l'eau enroba ses lunettes et l'aveugla instantanément, transformant le monde en une tache floue alors qu'il entamait sa chute vers les vagues d'un océan déchaîné, loin en dessous de lui.

Il était arrivé haut, haut, haut au-dessus d'un espace dégagé de la Mer du Nord.

Le choc de l'orage en furie avait failli faire perdre son balai à Harry, balai que le professeur Quirrell lui avait donné, ce qui n'aurait pas été une bonne idée. Il fallut presque une seconde entière pour qu'il reprenne ses esprits et ne redresse son balai avec aisance.

«~Je suis là~», dit une voix inconnue depuis une zone vide au-dessus de lui~; basse et râpeuse, la voix de l'échalas cireux et barbu que le professeur Quirrell était devenu en buvant du Polynectar, avant de se Désillusionner, lui, et son balai.

«~Je suis là~», dit Harry de sous la Cape d'Invisibilité. Il n'avait pas bu de Polynectar. Revêtir un autre corps entravait la magie, et Harry pourrait avoir besoin d'avoir toute sa magie à portée de main~; le plan était donc que Harry reste invisible quasiment tout le temps plutôt qu'il boive du Polynectar.

(Aucun n'avait prononcé le nom de l'autre. On n'utilisait tout simplement jamais de nom pendant une mission illégale même en survolant, invisible, une partie inconnue de la Mer du Nord. On ne le faisait tout simplement pas. Ç'aurait été stupide.)

Gardant une poigne très précautionneuse sur son balai d'une main, alors que la pluie et le vent hurlaient autour de lui, Harry leva sa baguette tenue avec tout autant de précaution et lança Impervius sur ses lunettes.

Ses verres nettoyés, Harry regarda autour de lui.

Il était entouré par de la pluie et du vent, et il faisait peut-être cinq degrés Celsius, avec de la chance~; il avait déjà fait l'objet d'un sortilège de réchauffement avant de s'aventurer dehors en février, mais le charme ne résistait pas aux gouttes d'un froid mordant. Pire que la neige, la pluie s'imprégnait dans toutes les surfaces exposées. La Cape d'Invisibilité vous rendait entièrement invisible, mais elle ne vous \emph{recouvrait} pas entièrement, et elle ne vous protégeait donc pas entièrement de la pluie. Le visage de Harry était exposé à toute la force de l'eau projetée du ciel, et celle-ci s'écrasait directement dans son cou, coulait le long de sa chemise et des manches de sa robe et des ourlets de son pantalon et dans ses chaussures. Le plus petit morceau de vêtement devenait un passage pour que l'eau s'infiltre.

«~Par là~», dit la voix polynectarée, et une étincelle de lumière verte s'alluma devant le balai de Harry et fonça dans une direction qui lui sembla impossible à suivre du regard.

À travers la pluie aveuglante, Harry suivait. Il la perdait parfois, la petite étincelle verte, mais à chaque fois qu'il appelait, l'étincelle réapparaissait devant lui quelques secondes plus tard.

Lorsque Harry eut trouvé la technique pour suivre l'étincelle, elle accéléra, et Harry fit passer le balai en vitesse supérieure afin de pouvoir rester à hauteur. La pluie le fouetta avec plus de force et lui donna la sensation de recevoir une volée de plombs de fusil à pompe, mais ses verres restaient propres et protégeaient ses yeux.

Ce n'est que quelques minutes plus tard, alors qu'il volait à pleine vitesse, que Harry entraperçut une immense ombre à travers la nuit qui s'élevait loin au-dessus des eaux.

Et il sentit un écho de vide, lointain et creux, irradiant de l'endroit où la Mort attendait, se déversant sur l'esprit de Harry, le contournant, tel une vague se brisant sur un rocher. Cette fois, Harry connaissait son ennemi, et sa volonté était d'acier et de lumière.

«~Je peux déjà sentir les Détraqueurs, dit la voix râpeuse et polynectarée du professeur Quirrell. Je ne m'attendais pas à cela, pas si tôt.

--- Pensez aux étoiles~», dit Harry par-dessus un grondement de tonnerre lointain. «~Ne laissez aucune colère entrer en vous, rien de négatif, pensez seulement aux étoiles, à ce que c'est que de s'oublier et de tomber sans corps à travers l'espace. Maintenez cette pensée comme une barrière Occlumantique à travers tout votre esprit. Le Détraqueurs auront quelque difficulté à traverser cela.~»

Il y eut un silence pendant quelques instants, puis~: «~Intéressant.~»

L'étincelle verte s'éleva et Harry inclina son balai légèrement vers le haut afin de suivre, alors même que ce mouvement les menait dans de la brume, dans un nuage bas au-dessus des eaux.

Ils survolèrent bientôt de biais l'immense bâtiment de métal à trois côtés qui les attendait bien plus bas. Le triangle d'acier était creux, c'était un bâtiment fait de trois murs épais, dépourvu de centre. Le professeur Quirrell avait dit que les Aurors de garde logeaient au niveau supérieur du côté Sud, protégés par leur Patronus. L'entrée légale d'Azkaban se trouvait sur le toit du coin Sud-Ouest du bâtiment. Qu'ils ne pouvaient évidemment voir. Au lieu de celle-ci, ils utiliseraient un couloir qui passait directement sous le coin Nord du bâtiment. Le professeur Quirrell descendrait en premier et percerait un trou dans le toit et dans ses barrières, à la pointe Nord, laissant derrière lui une illusion destinée à couvrir l'ouverture.

Les prisonniers étaient gardés dans le flanc du bâtiment, à des niveaux correspondant à leurs crimes. En bas, au point le plus profond et le plus central d'Azkaban, se trouvait un nid de plus de cent Détraqueurs. Des cargaisons de terre y étaient parfois déversées afin de garder les étages à niveau, car la matière directement exposée aux Détraqueurs se décomposait pour devenir de la boue et du néant.

«~Attendez une minute, dit la voix râpeuse, suivez-moi à pleine vitesse, et faites attention en traversant.

--- Compris~», dit lentement Harry.

L'étincelle s'éteignit et Harry commença à compter~: \emph{un un million, deux un million, trois un million…}

\emph{… soixante un million}, et Harry plongea, le vent hurlait autour de lui, vers la vaste structure de métal, vers là où il pouvait sentir les ombres de la Mort qui l'attendaient, qui puisaient la lumière et irradiaient le vide, et la structure de métal devenait de plus en plus grande. Simple et sans décoration mis à part une seule structure en forme de boîte sur le coin Sud-Ouest, la vaste forme grise les attendait. Le coin Nord était parfaitement uni, le trou du professeur Quirrell était indétectable.

Harry tira sur le manche aussi fort qu'il le pouvait en s'approchant du coin Nord, se donnant une plus grande marge de sécurité qu'il ne l'aurait fait en cours de vol sur balai, mais sans faire excès de prudence. Dès qu'il fut à l'arrêt, il commença à abaisser le balai de nouveau, vers ce qui semblait être le toit bien solide de l'extrémité du coin Nord.

Descendre à travers le toit illusoire fut une expérience étrange, et il se trouva ensuite dans un couloir de métal éclairé par une faible lueur orange -- qui, remarqua Harry d'un coup d'œil surpris, venait d'une bonne vieille lampe à gaz sous verre…

… car la magie déclinait et finissait par se faire absorber lorsqu'elle était exposée aux Détraqueurs.

Harry descendit de son balai.

L'attraction du vide était maintenant plus forte, et elle se divisait, coulait autour de Harry sans le toucher. Elles étaient distantes mais nombreuses, les blessures du monde~; Harry aurait pu indiquer leur direction les yeux fermés.

«~\parsel{Lanccez votre Patronuss}~», siffla un serpent au sol, l'air plus décoloré que vert sous la faible lumière orange.

La légère tension se communiquait même en Fourchelangue. Harry fut surpris~; le professeur Quirrell avait dit que les Animagus, une fois transformés, étaient bien moins vulnérables aux Détraqueurs (Harry présuma que c'était pour la même raison que celle qui expliquait la forme animale des Patronus). Si le professeur Quirrell avait autant de mal en forme reptilienne, qu'avait-il ressenti sous forme humaine, lorsqu'il avait eu besoin de pouvoir utiliser sa magie…~?

Dans sa main, la baguette de Harry se levait déjà.

Ce serait le commencement.

Même s'il ne s'agissait que d'une personne, seulement une personne qu'il pourrait sauver des ténèbres, même s'il n'était pas encore assez puissant pour téléporter \emph{tous} les prisonniers d'Azkaban en sécurité et brûler le triangle infernal jusqu'à ses fondations…

C'était malgré tout un début, un commencement, un paiement d'avance sur tout ce que Harry comptait accomplir dans sa vie. Plus d'attente, plus d'espoir, plus de simples promesses, tout commencerait ici. Ici et \emph{maintenant}.

La baguette de Harry fendit l'espace et pointa en direction de l'endroit où les Détraqueurs attendaient, loin en dessous.

«~\emph{Expecto Patronum~!}~»

La silhouette humanoïde prit forme dans un brasier de lumière. Elle n'était pas la chose solaire qu'elle avait été auparavant… probablement parce que Harry n'avait pas tout à fait réussi à s'empêcher de penser à tous les \emph{autres} prisonniers dans leurs cellules, ceux qu'il n'était \emph{pas} là pour sauver.

Cela valait peut-être mieux. Harry devrait garder son Patronus pendant un moment, et il était peut-être préférable qu'il ne soit pas trop lumineux.

À cette pensée, le Patronus s'assombrit un peu plus~; puis un peu plus, alors que Harry essayait d'y mettre un peu moins de sa force, jusqu'à ce que la brillante silhouette humanoïde ne luise finalement qu'un peu plus que le plus lumineux des Patronus animaux, et Harry sentit qu'il ne pouvait le ternir plus sans risquer de le perdre entièrement.

Puis~: «~\parsel{C'est stable}~», siffla Harry, et il commença à nourrir sa bourse de son balai. Sa baguette resta en main, et un léger courant durable venant de lui remplaçait les légères pertes de son Patronus.

Le serpent se fondit en un homme longiligne au teint cireux qui tenait la baguette du professeur Quirrell dans une main et le balai dans l'autre. Le grand échalas vacilla en revenant à la vie, et il alla s'appuyer contre le mur pendant un moment.

«~Bien joué, peut-être un tout petit peu trop lent~», murmura la voix râpeuse. La sécheresse du professeur Quirrell s'y trouvait, même si elle ne correspondait pas à la voix, pas plus que le regard grave ne correspondait au visage orné d'une barbe épaisse. «~Je ne peux plus les sentir du tout à présent.~»

Un moment plus tard, le balai partit vers la robe de l'homme et disparut. Puis sa baguette s'éleva et tapa son crâne, et il disparut à nouveau au son d'une coquille d'œuf brisée.

Au milieu de l'air éclot une faible étincelle verte, et Harry, toujours enveloppé de la Cape d'Invisibilité, la suivit.

Si vous aviez observé de l'extérieur, vous n'auriez rien vu d'autre qu'une petite étincelle verte qui dérivait dans les airs, et un humanoïde étincelant d'argent qui marchait derrière elle.

\later

Ils descendirent encore et encore et encore, dépassant une lampe à gaz après l'autre, et parfois une immense porte de métal, plongeant dans Azkaban au milieu de ce qui semblait être un silence absolu. Le professeur Quirrell avait mis en place une sorte de barrière à travers laquelle \emph{il} pouvait entendre ce qui se passait autour de lui, mais aucun son ne pouvait sortir, et aucun son ne pouvait atteindre Harry.

Ce dernier n'avait pas tout à fait réussi à empêcher son esprit de s'interroger sur la \emph{raison} du silence, ni à l'empêcher de la trouver. La raison déjà connue à un niveau anticipatif et infra-verbal qui l'avait poussé à essayer, futilement, de ne pas y penser.

Quelque part derrière ces immenses portes de métal, des gens hurlaient.

La silhouette humanoïde d'argent s'intensifiait et s'assombrissait à chaque fois que Harry y songeait.

Il avait reçu instruction de se lancer un sortilège de bulle. Pour qu'il ne puisse rien sentir.

Tout l'enthousiasme et l'héroïsme s'étaient déjà usés, comme Harry avait su qu'ils le feraient, ce qui n'avait pas pris longtemps même selon ses critères~; le processus s'était achevé la première fois qu'ils étaient passés devant une de ces portes de métal. Chaque porte de métal était fermée par un immense loquet, un loquet fait d'un simple métal non-magique, qui n'aurait pas arrêté un élève de Poudlard en première année -- s'il avait toujours eu sa baguette, s'il avait toujours eu sa magie, ce que les prisonniers n'avaient pas. Selon le professeur Quirrell, ces portes de métal n'étaient pas les portes de cellules individuelles, chacune donnait sur un couloir où se trouverait un groupe de cellules. Cela aidait en un sens, de ne pas avoir à penser que chaque porte correspondait directement à un prisonnier qui attendait juste derrière. Au lieu de cela il y avait peut-être \emph{plus} d'un prisonnier, ce qui diminuait l'impact émotionnel~; comme dans cette étude qui montrait que les gens donnaient plus lorsqu'on leur disait qu'une quantité d'argent donnée était nécessaire à sauver la vie d'un enfant que quand on leur disait que cette même quantité était nécessaire à sauver la vie de huit…

Harry avait de plus en plus de mal à ne pas y penser, et à chaque fois qu'il le faisait, la lumière de son Patronus fluctuait.

Ils parvinrent à un endroit où le corridor tournait à gauche, à l'angle du bâtiment triangulaire. Ils descendirent encore des marches de métal, une autre étape~; ils plongeaient à nouveau.

Les simples meurtriers n'étaient pas placés dans les cellules les plus basses. Il y avait toujours un niveau inférieur, une punition encore pire à craindre. Peu importe les abysses où vous aviez sombré, le gouvernement d'Angleterre magique avait encore une menace qu'il vous réservait au cas où vous feriez encore pire.

Mais Bellatrix Black avait été la Mangemort qui avait inspiré plus de peur que quiconque hormis le Seigneur des Ténèbres lui-même, une sorcière magnifique et mortelle absolument loyale à son maître~; elle avait été, si une telle chose était possible, encore plus sadique et maléfique que Vous-Savez-Qui lui-même, comme si elle avait essayé de surpasser son maître…

… c'était ce que le monde savait d'elle, ce que le monde croyait savoir d'elle.

Mais le professeur Quirrell avait dit à Harry qu'avant cela, avant qu'elle ne devienne le plus terrible des serviteurs du Seigneur des Ténèbres, il y avait eu une fille à Serpentard, douce, souvent solitaire, ne faisant de mal à personne. Après cela, des histoires avaient été inventées à son sujet, les souvenirs s'étaient rétrospectivement altérés (Harry connaissait bien les recherches qui avaient été faites dans ce domaine). Mais à cette époque, alors qu'elle allait encore en cours, la sorcière la plus talentueuse de Poudlard avait été considérée comme une fille gentille (avait dit le professeur Quirrell). Ses quelques amis avaient été surpris de la voir rejoindre les Mangemorts, et encore plus surpris de voir qu'elle avait caché tant de ténèbres derrière ce sourire triste et pensif.

C'était ce que Bellatrix avait un jour été, la sorcière la plus prometteuse de sa génération, avant que le Seigneur des Ténèbres ne la vole et ne la brise, qu'il ne la fasse voler en éclats et qu'il ne lui donne une nouvelle forme, qu'il ne la lie à lui par un lien plus profond et par des arts plus sombre qu'un Imperium ne pourrait jamais l'être.

Pendant dix ans, Bellatrix avait servi le Seigneur des Ténèbres, avait tué ceux qu'il lui avait ordonnés de tuer, avait torturé ceux qu'il lui avait ordonnés de torturer.

Puis le Seigneur des Ténèbres avait enfin été vaincu.

Et le cauchemar de Bellatrix avait continué.

Quelque part à l'intérieur de Bellatrix se trouvait peut-être encore quelque chose qui hurlait, qui avait hurlé pendant tout ce temps, quelque chose qu'un Guérisseur psychiatrique pourrait ramener~; ou peut-être pas, le professeur Quirrell n'avait aucun moyen de le savoir. Mais dans un cas comme dans l'autre, ils pouvaient…

… ils pouvaient au moins la faire sortir d'Azkaban…

Bellatrix Black avait été placée au plus bas niveau de la prison.

Harry avait du mal à ne pas imaginer ce qu'il verrait lorsqu'ils atteindraient sa cellule. Bellatrix n'avait dû avoir quasiment aucune peur de la mort, au début, si elle avait encore été en vie.

Ils descendirent une autre volée de marches, s'approchant un peu plus de la Mort et de Bellatrix, et le claquement de leurs chaussures invisibles était le seul son que Harry pouvait entendre. Une faible lueur orange émanait des lampes à gaz, la petite étincelle verte dérivait dans les airs, la silhouette étincelante suivait et sa lumière d'argent fluctuait parfois.

\later

Après être descendu maintes fois, ils parvinrent à un couloir qui ne se terminait pas par des escaliers mais par une dernière porte de métal, et l'étincelle verte s'arrêta devant celle-ci.

Le cœur de Harry s'était un peu calmé à mesure que leur descente dans les profondeurs d'Azkaban avait progressé sans que rien ne se produise. Mais son cœur tambourinait maintenant dans sa poitrine. Ils étaient au fond, et les ombres de la Mort étaient à portée de main.

Une sorte de clic métallique émana de la porte~; le professeur Quirrell ouvrait le chemin.

Harry prit une profonde inspiration et se souvint de tout ce que le professeur Quirrell lui avait dit. Le plus difficile ne serait pas seulement de jouer la fausse personnalité suffisamment bien pour tromper Bellatrix Black en personne, le plus difficile serait de conserver son Patronus en même temps…

L'étincelle verte disparut, et un moment plus tard un serpent d'un mètre de haut apparut dans une vibration de sortie d'invisibilité.

La porte de métal bougea avec un lent grincement lorsque Harry la poussa de sa main invisible, l'entrebâillait juste, et il jeta un coup d'œil.

Il vit un couloir droit qui s'achevait sur de la pierre. Il n'y avait aucune lumière ici hormis celle qui se faufilait depuis le Patronus de Harry. C'était suffisant pour qu'il voie les barreaux extérieurs des huit cellules encastrées dans le couloir, mais il ne pouvait voir à l'intérieur~; plus important encore, il ne pouvait voir personne dans le couloir.

«~\parsel{Je ne voiss rien}~», siffla Harry.

Le serpent darda, ondulant vivement le long du sol.

Un moment plus tard…

«~\parsel{Elle est sseule}~», siffla-t-il.

\emph{Reste}, pensa Harry à l'intention de son Patronus, qui prit position juste à côté de la porte, comme s'il la gardait~; puis Harry ouvrit la porte plus largement et entra à l'intérieur.

La première cellule contenait ce qui ressemblait à un corps desséché, peau devenue grise et marbrée, chair usée par endroits jusqu'à exposer l'os en dessous, pas d'yeux…

Harry ferma les siens. Il pouvait encore le faire, il était encore invisible, il ne trahissait rien en fermant les yeux.

Il l'avait déjà su, il l'avait lu à la page six de son livre de Métamorphose, que vous restiez à Azkaban jusqu'à ce que votre peine soit finie. Si vous mouriez avant, ils vous gardaient là jusqu'à relâcher votre corps. Si vous étiez condamnée à vie, ils gardaient juste le corps dans la cellule jusqu'à ce qu'ils en aient besoin, et alors ils jetaient votre corps dans la fosse des Détraqueurs. Mais c'était quand même un choc de l'observer, ce corps avait été une \emph{personne} et avait juste été \emph{laissée} là…

La lumière de la pièce vacilla.

\emph{Stable}, pensa Harry au plus profond de lui-même. Le professeur Quirrell souffrirait si ce Patronus cessait de l'empêcher d'avoir des pensées tristes. Si près des Détraqueurs, le professeur de Défense tomberait peut-être mort, tout simplement. \emph{Stable, Harry James Potter-Evans-Verres, stable~!}

Et avec cette pensée, Harry rouvrit les yeux~; il n'y avait pas de temps à perdre.

La seconde cellule qu'il regarda ne contenait qu'un squelette.

Et derrière les barreaux de la troisième, il vit Bellatrix Black.

À l'intérieur de Harry, quelque chose de précieux et d'irremplaçable se flétrit comme de l'herbe sèche.

On pouvait voir que la femme n'était pas un squelette et que sa tête n'était pas un crâne parce que la texture de la peau était encore différente de celle de l'os, même au stade de pâleur qu'elle avait atteint en attendant seule dans le noir. Soit ils ne la nourrissaient pas beaucoup, soit les ombres de Détraqueurs aspiraient ce qu'elle mangeait~; car ses yeux semblaient avoir réduit dans leurs orbites, ses lèvres avaient l'air trop asséchées pour recouvrir ses dents. La couleur sembla avoir déteint des vêtements noirs qu'elle avait apportés en prison, comme si les Détraqueurs avaient aussi drainé cela. Ces vêtements s'étaient voulus provocateurs, et ils recouvraient maintenant un squelette, mollement, exposant une peau ratatinée.

\emph{Je suis là pour la sauver, je suis là pour la sauver, je suis là pour la sauver}, pensa Harry en son for intérieur, avec désespoir, encore et encore, dans un effort quasi Occlumantique, imposant à son Patronus de ne pas s'éteindre, de rester et de \emph{protéger Bellatrix des Détraqueurs…}

Dans son cœur, dans son essence, Harry saisit toute sa pitié et toute sa compassion, toute sa volonté de la secourir des ténèbres~; et alors qu'il pensait à cela, la radiation d'argent qui entrait par la porte ouverte s'intensifia.

Et dans une autre partie de lui, comme s'il laissait juste une autre partie de son esprit continuer une habitude, sans y prêter trop d'attention…

Une expression froide recouvrit le visage de Harry, invisible sous la capuche.

«~Bonjour, ma chère Bella, dit un murmure glacé. T'ai-je manqué~?~»
%  LocalWords:  Kinnison Liono Thundera Raistlin Majere
