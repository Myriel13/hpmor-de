\partchapter{Compromis Tabous, Après coup}{II}

\lettrine{L}{orsque} Hermione Granger se réveilla, elle était allongée dans un lit doux et confortable de l'infirmerie de Poudlard, un rectangle de crépuscule tombant sur son ventre, chaud à travers la fine couverture. Sa mémoire lui dit qu'un voile-écran devait avoir été tendu au-dessus d'elle, soit entièrement autour de son lit soit à moitié ouvert, et que le reste du domaine de Mme Pomfresh se trouverait derrière celui-ci~: les autres lits, libres ou occupés, et les fenêtres lumineuses encastrées dans la pierre aux bas-reliefs en courbures de Poudlard.

Lorsqu'elle ouvrit les yeux, la première chose qu'elle vit furent ceux du professeur McGonagall, assise sur le côté gauche de son lit. Le professeur Flitwick n'était pas là, mais c'était compréhensible~: il était resté à côté d'elle pendant toute la matinée dans la cellule où elle avait été détenue, son corbeau d'argent déployé comme garde supplémentaire contre le Détraqueur et son sévère petit visage toujours tourné vers les Aurors. Le directeur de Serdaigle avait certainement passé trop de temps à s'occuper d'elle et avait probablement dû retourner à ses cours plutôt que de continuer à veiller sur une fille condamnée pour tentative de meurtre.

Elle se sentit atrocement, atrocement malade, et il ne lui semblait pas que c'était l'effet d'une quelconque potion. Elle se serait bien remise à pleurer mais sa gorge lui faisait mal, ses yeux la brûlaient encore et son esprit ne ressentait rien d'autre que de la fatigue. Elle aurait pu supporter de geindre un peu mais elle n'arrivait pas à trouver la force de pleurer.

"Où sont mes parents~?" chuchota Hermione à la directrice de Gryffondor. Il lui semblait que leur faire face serait la pire des choses au monde, encore pire que tout le reste, et pourtant elle voulait quand même les voir.

Le doux regard du professeur McGonagall se métamorphosa en une expression plus triste. "Je suis navrée, Mlle Granger. Bien que cela n'ait pas toujours été le cas, nous avons décidé depuis quelques années qu'il était plus sage de ne pas informer les parents de nés-Moldus des dangers que leurs enfants ont encourus. Je vous conseillerais de rester vous aussi silencieuse si vous souhaitez rester à Poudlard sans qu'ils ne vous causent d'ennuis."

"Je ne suis pas exclue~?" chuchota la fille. "Pour ce que j'ai fait~?"

"Non," dit le professeur McGonagall. "Mlle Granger… vous avez sûrement entendu… j'espère que vous avez entendu M. Potter lorsqu'il a dit que vous étiez innocente~?"

"Il disait juste ça comme ça," dit-elle d'une voix morne. "Pour me faire libérer, je veux dire."

La vieille sorcière secoua la tête avec fermeté. "Non, Mlle Granger. M. Potter croit que vous avez reçu un sortilège de faux souvenirs, que le duel n'a jamais eu lieu. Le directeur soupçonne la participation de magies encore plus sombres -- que votre main a peut-être lancé le sort, mais que ce n'était pas de votre fait. Même le professeur Rogue trouve cette affaire impossible à croire, bien qu'il ne puisse pas forcément le dire en public. Il se demandait si des drogues moldues auraient pu êtres utilisées contre vous."

Les yeux de Hermione continuèrent de regarder distraitement le professeur de Métamorphose~; elle savait qu'elle venait d'entendre quelque chose d'important mais elle n'arrivait pas à trouver l'énergie nécessaire pour propager ces changements dans son esprit.

"\emph{Vous} n'y croyez certainement pas~?" dit le professeur McGonagall. "Mlle Granger, vous ne pouvez pas vous croire capable de devenir une meurtrière~!"

"Mais j'ai…" son excellente mémoire lui repassa le souvenir avec obligeance pour la millième fois~: Drago Malfoy lui disant avec un sourire méprisant qu'elle ne le battrait jamais s'il était en pleine forme et prouvant immédiatement cette affirmation en dansant comme un duelliste entre les trophées sous bonne garde tandis qu'elle courait frénétiquement en tous sens puis en lui assénant le coup final d'un sortilège qui l'avait envoyée s'écraser contre le mur et avait fait jaillir du sang de sa joue… Et alors… Alors elle avait…"

"Mais vous vous souvenez l'avoir fait," dit la sorcière plus âgée en la regardant avec compréhension et gentillesse. "Mlle Granger, une fillette de douze ans n'a pas besoin de supporter des souvenirs aussi affreux. Demandez-le moi et je serais heureuse de les enfermer pour vous."

C'était comme si on lui avait jeté un verre d'eau en plein visage. "Quoi~?"

Le professeur McGonagall sortit sa baguette, un geste si expert et rapide qu'on aurait dit qu'elle avait simplement pointé quelque chose du doigt. "Je ne peux pas vous offrir de vous débarrasser entièrement de ces souvenirs, Mlle Granger," dit le professeur McGonagall avec sa précision habituelle. "D'importants faits pourraient s'y trouver enfouis. Mais il existe une forme d'Oubliettes qui est réversible et je serais heureuse de le lancer sur vous."

Hermione regarda la baguette et sentit l'éclosion d'un espoir pour la première fois en presque deux jours.

\emph{Faites que ça n'ait jamais eu lieu…} elle l'avait souhaité, encore et encore, souhaité que les aiguilles du temps se retournent et effacent l'horrible choix qui ne pourrait jamais, jamais être défait. Et même si l'effacement du souvenir n'allait pas jusque là, ce serait tout de même une forme de libération…

Elle revint au doux visage du professeur McGonagall.

"Vous ne pensez \emph{vraiment pas} que je l'ai fait~?" dit Hermione d'une voix tremblante.

"Je suis \emph{tout à fait} certaine que vous ne feriez pas une chose pareille de votre plein gré."

Sous les couvertures, les mains de Hermione serraient les draps. "\emph{Harry} pense que je ne l'ai pas fait~?"

"M. Potter considère que vos souvenirs ont été entièrement manufacturés. Je comprends parfaitement son point de vue."

Les mains agrippées de Hermione laissèrent filer les draps et elle s'affaissa de nouveau dans le lit, où elle s'était en partie assise.

Non.

Elle n'avait rien dit.

Elle s'était réveillée, elle s'était souvenu de ce que s'était passé la nuit précédente, et ça avait été comme… Comme… Elle ne pouvait pas trouver les mots, même en pensée, pour décrire ce qu'elle avait ressenti. Mais elle avait su que Drago Malfoy était déjà mort et elle n'avait rien dit, elle n'avait pas été voir le professeur Flitwick pour avouer. Elle s'était juste habillée et elle était allée prendre son petit déjeuner en \emph{essayant de se comporter normalement} pour que personne ne soit jamais au courant, et elle avait su que c'était mal, mal, Mal et atrocement atrocement MAL mais elle avait eu… Tellement peur…

Même si Harry Potter avait raison, même si le duel contre Drago Malfoy était un mensonge, elle avait fait \emph{ce} choix toute seule. Elle ne méritait pas d'oublier cela ni qu'on la pardonne de l'avoir fait.

Et si elle \emph{avait} bien agi, si elle était allée directement voir le professeur Flitwick, peut-être que cela aurait… Aidé, d'une façon ou d'une autre, peut-être que tout le monde aurait vu qu'elle regrettait, et Harry n'aurait pas eu à perdre tout son argent pour la sauver…

Hermione ferma les yeux, les serra très fort, car elle ne pouvait pas supporter l'idée de se remettre à pleurer. "Je suis quelqu'un d'horrible," dit-elle d'une voix chancelante. "Je suis horrible et pas héroïque du tout…"

La voix du professeur McGonagall fut très tranchante, comme si Hermione venait de faire une terrible erreur dans ses devoirs de Métamorphose. "Cessez de vous comporter comme une idiote, Mlle Granger~! La personne \emph{horrible} est celle qui vous a fait cela; Quand à l'héroïsme… Eh bien, Mlle Granger, vous avez déjà entendu mon opinion au sujet des jeunes filles qui s'essaient ce genre de choses avant même d'avoir quatorze ans, aussi je ne vous ferai pas d'autre leçon à ce sujet. Je dirai seulement que vous venez de subir une expérience absolument atroce et à laquelle vous avez survécu aussi bien qu'il aurait été possible pour une sorcière de votre année. Aujourd'hui, vous avez le droit de pleurer autant qu'il vous plaira. Demain, vous retournez en cours."

C'est alors que Hermione sut le professeur McGonagall ne pouvait pas l'aider. Elle avait besoin qu'on la gronde car elle ne pourrait pas être absoute avant d'avoir été blâmée, et le professeur McGonagall ne ferait jamais cela pour elle car elle n'en demanderait jamais autant à une petite Serdaigle.

C'était une chose pour laquelle Harry Potter ne l'aiderait pas non plus.

Hermione se retourna dans son lit et se recroquevilla contre elle-même, loin du professeur McGonagall. "S'il vous plaît," murmura-t-elle. "Je veux parler… Au directeur…"

\later

"Hermione."

Lorsque Hermione Granger ouvrit les yeux une seconde fois, elle vit le visage ridé de soucis d'Albus Dumbledore, penché au-dessus de son lit avec l'air d'avoir \emph{pleuré}, même si c'était bien sûr impossible~; et Hermione ressentit un pincement déchirant de culpabilité à l'idée de l'avoir autant dérangé.

"Minerva dit que tu souhaitais me parler," dit le vieux sorcier.

"Je…" Hermione ne savait soudain plus quoi dire. Sa gorge se referma et elle ne put que bégayer~: "Je -- je suis…"

Son ton devait être parvenu à communiquer l'autre mot, celui qu'elle ne parvenait même plus à prononcer.

"\emph{Désolée~?"} dit Dumbledore. "Pourquoi, pourquoi devrais-tu être désolée~?"

Elle dut se forcer à faire sortir les mots de sa bouche. "Vous avez dit à Harry… Qu'il ne devrait pas payer… donc \emph{je} n'aurais pas… Dû faire ce que le professeur McGonagall a dit, je n'aurais pas dû toucher sa baguette…"

"Ma chérie," dit Dumbledore, "si tu ne t'étais pas engagée auprès de la maison Potter, Harry aurait attaqué Azkaban seul et aurait peut-être gagné. Ce garçon choisit peut-être ses mots avec soin, mais je ne l'ai encore jamais vu mentir, et chez le Survivant se trouve un pouvoir que le Seigneur des Ténèbres n'a jamais connu. Il aurait bel et bien essayé de briser Azkaban, même au prix de sa vie." La voix du vieux sorcier devint plus douce et plus amicale. "Non, Hermione, tu n'as rien fait qui justifie que tu t'en veuilles à toi même."

"J'aurais pu \emph{faire} qu'il n'attaque pas."

Dans les yeux de Dumbledore apparut un léger scintillement qui fut bientôt perdu dans sa fatigue. "Vraiment, Mlle Granger~? Peut-être devriez-vous être directrice à ma place, car je n'ai moi-même pas un tel pouvoir sur les enfants bornés."

"Harry a promis…" sa voix s'interrompit. L'horrible vérité était très difficile à prononcer. "Harry Potter m'a promis… Qu'il ne m'aiderait jamais… Si je lui demandais de ne pas le faire."

Il y eut une pause. Hermione se rendit compte que les bruits lointains de l'infirmerie qui avaient accompagnés le professeur McGonagall avaient cessé lorsque Dumbledore l'avait réveillée. De là où elle se trouvait, allongée dans son lit, elle ne pouvait voir que le plafond et le haut de l'une des fenêtres, mais rien dans son champ de vision ne bougeait, et s'il y avait des sons, elle ne pouvait pas les entendre.

"Ah," dit Dumbledore. Le vieux sorcier eut un profond soupir. "J'imagine qu'il est \emph{possible} que le garçon ait alors tenu sa promesse."

""J'aurais… j'aurais dû…"

"Aller à Azkaban de ton plein gré~?" dit Dumbledore. "Mlle Granger, c'est là plus que je n'exigerai jamais d'un autre."

"Mais…" Hermione déglutit. Elle n'avait pas pu s'empêcher de remarquer la faille, car tous ceux qui voulaient franchir le portrait du dortoir Serdaigle apprenaient rapidement à faire attention aux formulations précises. "Mais ce n'est pas plus que ce que vous exigeriez de \emph{vous-même}."

"Hermione…" commença le vieux sorcier.

"Pourquoi~?" dit la voix de Hermione, et il semblait maintenant que celle-ci continuait, libérée de son esprit. "Pourquoi ne pourrais-je pas être plus courageuse~? J'allais courir devant le Détraqueur -- pour Harry -- avant, je veux dire, en janvier… Alors pourquoi… Pourquoi… Pourquoi est-ce que je n'ai pas pu…" Pourquoi l'idée d'être envoyée à Azkaban l'avait elle complètement \emph{tétanisée}, pourquoi avait-elle tout oublié du Bien…

"Ma chère petite fille," dit Dumbledore. Les yeux bleus derrière les lunettes en demi-lune exprimaient un compréhension totale de sa culpabilité. "Je n'aurais pas fait mieux moi-même lors de ma première année à Poudlard. Tout comme tu souhaiterais être plus gentille envers les autres, sois aussi plus gentille envers toi-même."

"Donc \emph{j'ai} mal agi." Il semblait qu'elle avait besoin de le dire et qu'on le lui dise, même si elle le savait déjà.

Il y eut un silence.

"Écoute, jeune Serdaigle," dit le vieux sorcier, "écoutes-moi bien, car je vais te dire la vérité. La plupart des malfaisants ne se voient pas comme tels~; de fait, la plupart d'entre eux se croient être les héros des histoires qu'ils se racontent. J'ai un jour cru que les pires choses du monde étaient faites au nom du plus grand bien. J'avais tort. Terriblement tort. Il existe un mal dans ce monde qui se sait l'être et qui hait le bien de toute ses forces. Il souhaite détruire tout ce qui est bon."

Hermione frissonna dans son lit~; tout cela semblait très réel lorsque c'était Dumbledore qui le disait.

Le vieux sorcier continua de parler. "Tu es l'une des bonnes choses de ce monde, Hermione Granger, et ce mal te hait donc aussi. Si tu étais demeurée forte pendant ce procès, il t'aurait frappé plus fort, et encore plus fort, jusqu'à ce que tu te brises. Ne penses pas que les héros ne peuvent être brisés~! Nous sommes seulement plus résistants, Hermione." Les yeux du vieux sorcier étaient devenus plus sévères qu'elle ne les avait jamais vus. "Lorsqu'on est épuisé depuis de nombreuses heures, lorsque la douleur et la mort ne sont pas une peur passagère mais une certitude, alors il est plus difficile d'être un héros. Si je devais être honnête -- alors aujourd'hui, oui, je n'hésiterais pas face à Azkaban. Mais lorsque j'étais dans ma première année à Poudlard -- j'aurais fui devant ce Détraqueur auquel tu as fait face, car mon père était mort à Azkaban et que je les craignais. Saches-le~! Le mal qui t'a frappé aurait pu briser n'importe qui, même moi. Seul Harry Potter aura ce qu'il faut pour faire face à cette horreur, le jour où il sera en pleine possession de ses pouvoirs."

Le cou de Hermione ne pouvait plus rester tendu vers le vieux sorcier~; elle laissa sa tête tomber en arrière, jusqu'à son coussin, où elle regarda le plafond, tentant d'absorber autant de ces mots qu'elle le pouvait.

"Pourquoi~?" dit sa voix à nouveau tremblante. "Pourquoi quelqu'un serait-il aussi méchant~? Je ne comprends pas."

"Je me suis moi-même posé cette question," dit la voix de Dumbledore, chargée d'une profonde tristesse. "Je m'interroge depuis trente ans et je ne comprends toujours pas. Toi et moi ne comprendrons jamais, Hermione Granger. Mais au moins je sais maintenant ce que le véritable mal répondrait, si on pouvait lui parler et lui demander pourquoi il était ainsi. Il dirait~: \emph{Pourquoi pas~?"}

Un bref éclair d'indignation la traversa. "Il doit y avoir au moins un \emph{million} de raisons de ne pas l'être~!"

"Tout à fait," dit la voix de Dumbledore. "Un million de raisons, et encore plus. Nous connaîtrons toujours ces raisons, toi et moi. Si tu insiste pour le formuler ainsi -- alors oui, Hermione, le procès d'aujourd'hui t'a brisée. Mais ce que tu fais \emph{après} avoir été brisée -- cela aussi, c'est être une héroïne. Ce que tu es, Hermione Granger, et ce que tu seras toujours."

Elle leva de nouveau la tête et le regarda.

Le vieux sorcier se leva de son lit. Sa barbe d'argent s'abaissa lorsqu'il s'inclina gravement devant elle puis il partit.

Elle continua de regarder la porte par laquelle le vieux sorcier était parti.

Cela aurait dû signifier quelque chose pour elle, cela aurait dû la toucher. Elle aurait dû se sentir mieux maintenant que Dumbledore, qui avait auparavant semblé si réticent, venait de la reconnaître en tant qu'héroïne.

Elle ne ressentait rien.

Hermione laissa sa tête retomber sur le lit, Mme Pomfresh vint et lui fit boire quelque chose qui brûla ses lèvres comme l'aurait fait une nourriture épicée, dont l'odeur était encore plus forte et qui n'avait aucun goût. Cela ne signifia rien pour elle. Elle continua de regarder les lointains blocs de pierre du plafond.

\later

Minerva attendait en faisant de son mieux pour ne pas léviter près de la double porte de l'infirmerie de Poudlard~; enfant, elle avait toujours songé à ces portes comme à un "sinistre portail" et elle ne pouvait à présent s'empêcher de s'en souvenir. Trop de mauvaises nouvelles avaient été annoncées ici…

Albus sortit de l'infirmerie. Le vieux sorcier n'arrêta pas, continua seulement de marcher vers le bureau du professeur Flitwick, et Minerva le suivit.

Le professeur McGonagall s'éclaircit la gorge. "Est-ce fait, Albus~?"

Le vieux sorcier acquiesça d'un hochement de tête. "Si une magie hostile l'atteint ou si le moindre esprit la touche, je le saurai et j'accourrai."

"J'ai parlé à M. Potter après le cours de Métamorphose," dit le professeur McGonagall. "Il était d'avis qu'à partir de maintenant, Mlle Granger devrait aller à Beauxbâtons plutôt qu'à Poudlard."

Le vieux sorcier secoua la tête. "Non. Si Voldemort désire vraiment porter atteinte à Mlle Granger -- il est tenace au-delà de toute mesure. Ses serviteurs le rejoignent, il n'aurait pas pu récupérer Bellatrix seul. Azkaban elle-même n'est pas à l'abri de sa malveillance, et quant à Beauxbâtons -- non, Minerva. Je ne pense pas que Voldemort puisse s'essayer à de telles possessions souvent ou contre des cibles plus fortes, sans quoi cette année se serait déroulée bien différemment. Et Harry Potter est ici, que Voldemort doit craindre, qu'il l'admette ou non. Maintenant que je l'ai mise sous protection, Mlle Granger sera plus en sécurité au cœur de Poudlard qu'éloignée."

"M. Potter semblait douter de cela," dit Minerva. Elle ne parvenait pas tout à fait à garder sa voix dénuée de tranchant car une partie d'elle était assez fortement d'accord. "Il semblait penser que le sens commun dicte que Mlle Granger continue sa scolarité n'importe où ailleurs qu'à Poudlard."

Le vieux sorcier soupira. "Je crains que le garçon n'ait passé trop de temps parmi les Moldus. Ils sont en permanence à la recherche de sécurité et s'imaginent toujours que celle-ci peut-être trouvée. Si Mlle Granger n'est pas en sûreté au centre de notre forteresse, elle ne le sera pas plus en la quittant."

"Tout le monde ne semble pas être d'accord," dit le professeur McGonagall. Ça avait presque été la première lettre qu'elle avait vu lorsqu'elle avait jeté un rapide coup d'œil sur son bureau~: une enveloppe du plus fin des parchemins, scellée d'une cire vert-argent à l'effigie d'un serpent qui s'était dressé et lui avait sifflé dessus. "J'ai reçu la chouette de Lord Malfoy retirant son fils de Poudlard."

Le vieux sorcier hocha la tête mais ne changea pas de rythme. "Harry est-il au courant~?"

"Oui." Sa voix flancha l'espace d'un instant lorsqu'elle se souvint de l'expression de Harry. "Après les cours, M. Potter a félicité le bon sens de Lord Malfoy et a dit qu'il écrirait à Mme Londubat pour lui conseiller de faire de même avec son petit-fils au cas où celui-ci serait la prochaine cible. Au cas où la gardienne de M. Londubat serait suffisamment négligente pour le laisser à Poudlard, M. Potter voudrait qu'il soit munit d'un Retourneur de Temps, d'une cape d'invisibilité, d'un balai et d'une bourse dans laquelle les transporter~; et aussi d'un anneau d'orteil pourvu d'un Portoloin d'urgence menant vers un lieu sûr au cas où quelqu'un kidnapperait M. Londubat et lui ferait quitter l'enceinte de Poudlard. J'ai dit à M. Potter que je ne pensais pas que le ministère consentirait à un tel usage des Retourneur de Temps et il m'a répondu que nous ne devrions pas leur demander d'autorisation. Je m'attends à ce qu'il désire voir Mlle Granger recevoir pareil équipement si jamais elle reste. Quant à lui, M. Potter veut un balai pour trois personne afin qu'il puisse le transporter dans sa bourse." Elle n'était pas éblouie par la liste de précautions. Impressionnée par leur intelligence, mais pas éblouie. Elle était maître de Métamorphose, après tout. Mais le fait que Harry trouve maintenant Poudlard aussi dangereuse que la recherche fondamentale en magie envoyait quand même des frissons d'inquiétude parcourir son corps.

"Le département des mystères n'est pas à défier à la légère," dit Albus. "Mais pour le reste -" le vieux sorcier sembla s'affaisser légèrement. "Autant donner au garçon ce qu'il souhaite. Et je protégerais aussi Neville et écrirai à Augusta pour lui dire qu'il devrait rester ici pendant les vacances."

"Et enfin," continua-t-elle, "M. Potter dit que -- et je le cite directement, Albus -- que quel que soit l'espèce d'attire-mage-noir que le directeur garde ici, il faut le faire sortir de cette école \emph{maintenant}." Cette fois elle ne parvint pas à empêcher sa voix d'être tranchante.

"J'en ai demandé autant à Flamel," dit Albus d'un ton qui laissa clairement entendre sa douleur. "Mais maître Flamel a dit -- que même \emph{lui} ne peut garder la Pierre en sûreté -- qu'il croit que Voldemort a le moyen de la trouver, où qu'elle se trouve -- et qu'il ne consent pas à ce qu'elle soit gardée ailleurs qu'à Poudlard. Minerva, je suis navré mais elle doit être gardée ici -- il le \emph{faut}~!"

"Très bien," dit le professeur McGonagall. "Quant à moi, je pense que M. Potter a raison en tous points."

Le vieux sorcier lui jeta un coup d'œil et il dit d'une voix émue~: "Minerva, vous me connaissez depuis longtemps, aussi bien que quiconque encore en vie aujourd'hui -- dites-moi si les ténèbres me tiennent déjà."

"Quoi~?" dit le professeur McGonagall, franchement surprise. Puis~: "Oh non Albus, non~!"

Les lèvres du vieux sorcier se serrèrent avec force avant qu'il ne parle. "Pour le plus grand bien. Ils sont si nombreux, ceux que j'ai sacrifiés pour le plus grand bien. Aujourd'hui, j'ai presque condamnée Hermione Granger à Azkaban pour le plus grand bien. Et je me trouve -- aujourd'hui, je me suis retrouvé -- à éprouver de la rancune envers cette innocence que je ne possède plus…" La voix du vieux sorcier resta en suspens. "Le mal fait au nom du bien. Le mal fait au nom du mal. Quel \emph{est} le pire~?"

"Vous faites l'idiot, Albus."

Le vieux sorcier lui jeta un nouveau coup d'œil avant de regarder une fois de plus devant lui. "Dis-moi, Minerva -- t'es tu arrêtée le temps de soupeser les conséquences de ton geste avant de dire à Mlle Granger comment se lier à la famille Potter~?"

Elle inspira par réflexe lorsqu'elle comprit ce qu'elle avait fait -

"Tu ne l'as donc pas fait." Les yeux d'Albus étaient attristés. "Non, Minerva, tu n'as pas à t'excuser. C'est bien ainsi. Étant donné ce que tu as vu de moi aujourd'hui -- si ta loyauté va maintenant à Harry Potter et pas à moi, alors c'est bien, c'est juste ainsi." Elle ouvrit les lèvres pour protester mais Albus continua avant qu'elle ne puisse prononcer un mot. "De fait -- de fait, cela deviendra nécessaire, et même plus que nécessaire si le Seigneur des Ténèbres que Harry doit vaincre pour accéder à sa pleine puissance se révèle ne pas être Voldemort…"

"Pas \emph{ça} encore~!" dit Minerva. "Albus, c'est Vous-Savez-Qui et, pas vous, qui a marqué Harry comme son égal. Il est absolument \emph{impossible} que la prophétie ait parlé de vous~!"

Le vieux sorcier hocha la tête, mais ses yeux semblaient perdus dans le lointain, concentrés sur le chemin qui les attendait.

\later

La cellule de détention, bien au centre du département de justice magique, était luxueusement apprêtée~; ce qui constituait plus une remarque sur ce que les sorciers adultes tenaient pour acquis que sur quelque sentiment humain envers les prisonniers. Il y avait une chaise à bascule automatique munie de coussins autochauffants moelleux et richement brodés. Il y avait une armoire qui contenait un assemblage hétéroclite de livres trouvés chez un vendeur à la sauvette ainsi qu'un étage entier de vieux magazines, dont un de 1883. Quant aux toilettes, eh bien, ce n'était pas tout à fait luxueux mais un sortilège lancé sur la pièce interrompait toutes ces petites affaires~: on n'allait nulle part où l'Auror de garde ne pouvait surveiller. Mais mis à part cela, c'était une plaisante petite cellule. Le professeur de Défense de Poudlard était détenu, pas arrêté, même pas intimidé. Il n'y avait pas de preuves pour l'accuser… mis à part qu'un crime atroce et insolite avait été commis dans l'enceinte de l'école de sorcellerie de Poudlard et qu'en s'en tenant aux observations passées les chances étaient de cinq contre une pour que l'actuel professeur de Défense y soit mêlé d'une façon ou d'une autre. Il faut ajouter à cela le fait que personne au département de justice magique ne savait même \emph{qui} était le professeur de Défense et que l'homme avait, au sens propre, écarté toute tentative de découvrir sa véritable identité d'un \emph{toussotement}. Donc non, ils \emph{n'avaient pas} encore rendu 'Quirinus Quirrell' à Poudlard.

Répétons cela, pour marquer le coup~:

Le professeur de Défense.

Était détenu.

Dans une cellule.

Le professeur de Défense regardait fixement l'Auror et fredonnait.

Le professeur de Défense n'avait pas prononcé un mot depuis qu'il était arrivé dans cette cellule. Il avait \emph{seulement} fredonné.

Le fredonnement avait commencé comme une simple berceuse pour enfants, celle qui en Angleterre moldue commençait par \emph{Bonsoir, bonne nuit…}

L'air avait été fredonné sans variation, encore et encore, pendant sept minutes, pour établir le motif sous-jacent.

Puis commencèrent les élaborations sur le thème principal. Des vers fredonnées trop lentement, entrecoupés de longues pauses, afin que l'esprit de celui qui écoutait attende désespérément la note suivante, le vers suivant. Puis, lorsque le prochain vers venait, il était tellement faux, incroyablement, atrocement faux, pas seulement faux par rapport aux vers précédents mais chanté sur une ton qui ne correspondait à \emph{aucune} note, si bien qu'on pouvait croire que cette personne avait délibérément pratiqué ce fredonnement uniquement afin d'acquérir une anti-musicalité parfaite.

La chanson était à la musique ce que l'horrible voix morte d'un Détraqueur était à la voix humaine.

Et cet horrible, horrible fredonnement est \emph{impossible} à ignorer. Il est similaire à une berceuse connue mais s'en éloigne de façon imprévisible. Il créé des attente et les trahi, mais jamais selon un motif qui lui permettrait de se fondre dans l'arrière-plan. Le cerveau de celui qui écoute ne peut s'empêcher de s'attendre à ce que les vers anti-musicaux se complètent ni à s'empêcher d'être surpris.

La seule explication possible à l'existence de ce type de fredonnement est qu'il a été délibérément inventé par quelque génie ineffablement cruel qui se serait un jour réveillé ennuyé par la torture ordinaire et qui aurait décidé de se donner un handicap et de voir s'il pouvait détruire la santé mentale de quelqu'un \emph{juste en lui fredonnant une chanson.}

L'Auror avait écouté cet épouvantable, cet inimaginable fredonnement pendant quatre heures tout en subissant le regard de cette immense présence froide et mortelle qui était tout aussi horrible qu'on la regarde directement ou du coin de l'œil…

Le fredonnement s'arrêta.

Il y eut une longue attente. Assez pour qu'un espoir monte puis soit écrasé par le souvenir des déceptions précédentes. Puis, à mesure que l'intervalle s'allongeait et s'allongeait encore, cet espoir s'éleva de nouveau, inarrêtable…

Le fredonnement recommença.

L'Auror craqua.

Il saisit un miroir à sa ceinture, le toucha une fois puis dit~: "C'est l'Auror Junior Arjun Altunay, je déclare un code RJ-L20 en cellule trois."

"Code RJ-L20~?" répondit le miroir d'un ton surpris. Il y eut le son de pages que l'on tournait, puis~: "Vous voulez être relevé parce que le prisonnier a entamé une guerre psychologique contre vous et qu'il gagne~?"

(Amélia Bones est vraiment très intelligente).

"Qu'est-ce que le prisonnier vous a dit~?" demanda le miroir.

(Cette question ne fait \emph{pas} partie de la procédure RJ-L20, mais Amélia Bones a malheureusement oublié d'inclure l'instruction explicite de ne pas la poser.)

"Il -" dit l'Auror, et il jeta un regard dans la cellule. Le professeur de Défense était maintenant appuyé dans sa chaise et avait l'air assez détendu. "Il me \emph{regardait fixement}~! Et il \emph{fredonnait} !"

Il y eut un silence.

Le miroir parla de nouveau~: "Et vous déclarez un RJ-L20 pour ça~? Vous êtes sûr de ne pas être en train d'essayer d'être soulagé de votre tour de garde~?"

(Amélia Bones est entourée d'idiots)

"Vous ne comprenez pas~!" s'écria l'Auror Altunay. "C'est vraiment un fredonnement atroce~!"

Le miroir transmit le son d'un rire étouffé dans l'arrière-plan, comme s'il était venu de plus d'une personne. Puis il parla de nouveau~: "M. Altunay, si vous ne voulez pas être rétrogradé à Auror seconde classe, je vous suggère de serrer les dents et de vous remettre au travail…"

"Ignorez ça," dit une voix sèche qui semblait assez éloignée parce qu'elle était plus loin du miroir.

(C'est pourquoi Amélia Bones s'assoit souvent au centre de coordination du département de justice magique lorsqu'elle remplit sa paperasse ministérielle.)

"Auror Altunay," dit la voix sèche tout en semblant s'approcher du miroir, "vous serez bientôt remplacé. Auror Ben Gutierrez, la procédure RJ-L20 ne dit \emph{pas} de demander pourquoi. Elle dit que l'on relève l'Auror qui l'a déclarée. \emph{Si} je découvre que les Aurors semblent en abuser, \emph{je} la modifierai afin d'empêcher les abus…" le miroir se tut brutalement.

L'Auror se retourna pour lancer un regard triomphant en direction de l'actuel professeur de Défense de Poudlard qui était confortablement assis dans son fauteuil rembourré.

Cet homme prononça alors les premiers mots à avoir quitté ses lèvres depuis qu'il était entré dans la cellule.

"Au revoir, M. Altunay," dit le professeur de Défense.

Quelques minutes plus tard, la porte de la cellule s'ouvrit et une femme aux cheveux gris entra, habillée des robes pourpres des Aurors, sans insigne indiquant son rang, sans ornement, avec sous son bras gauche un dossier en cuir noir. "Vous pouvez disposer," dit la vieille femme d'un ton abrupt.

Il y eut un bref délai pendant lequel l'Auror Altunay essaya d'expliquer ce qui s'était passé. Un hochement de tête et un doigt sévère simplement pointé vers la porte y coupèrent court.

"Bonsoir, madame la directrice," dit le professeur de Défense.

Amélia Bones ne répondit rien mais s'assit soudain dans la chaise laissée vide. La vieille sorcière ouvrit le dossier de cuir et son regard s'abaissa vers les parchemins qui s'y trouvaient. "Possibles indices quant à l'identité de l'actuel professeur de Défense de Poudlard tels qu'établis par l'Auror Robards". Le page de titre fut retournée et mise à l'écart. "Le professeur de Défense dit avoir été réparti à Serpentard. Prétend que sa famille a été tuée par Voldemort. Dit avoir étudié dans un centre d'arts martiaux situé en Asie moldue qui a été détruit par Voldemort. Une requête soumise au département de la coopération magique internationale identifie cet incident comme l'affaire Oni de 1969." Un autre parchemin fut mis de côté. "Il semble aussi que le professeur de Défense a donné un discours des plus enthousiasmants à ses élèves juste avant Yule dernier lors duquel il a blâmé la génération précédente pour leur manque d'unité face aux Mangemorts." La vieille sorcière releva les yeux du dossier de cuir. "Madame Londubat était tombée sous le charme du discours et a insisté pour que je le lise en entier. Je fus frappé par la familiarité des arguments mais je ne pus alors les reconnaître. Mais après tout, je vous avais cru mort."

Le plus haut officier de police d'Angleterre magique regardait maintenant l'actuel professeur de Défense de Poudlard avec des yeux perçants à travers le panneau de verre magiquement renforcé qui les séparait. L'homme dans la cellule lui rendit son regard calmement, sans sembler être alarmé.

"Je ne prononcerai aucun nom," dit la vieille sorcière. "Mais je raconterai une histoire et vous verrez si elle vous semble familière." Amélia Bones rabaissa les yeux et retourna un parchemin. "Né en 1927, entré à Poudlard en 1938, réparti à Serpentard, a obtenu son diplôme en 1945. Parti à l'étranger pour un voyage post-remise de diplômes et disparu alors qu'il était en Albanie. Présumé mort jusqu'en 1970 date à laquelle il est tout aussi soudainement rentré en Angleterre magique sans explication aucune pour les vingt-cinq années d'absence. Il est resté séparé de sa famille et de ses amis et a vécu dans l'isolement. En 1971, alors qu'il se trouvait au Chemin de Traverse, il a repoussé la tentative de Bellatrix Black de kidnapper la fille du ministre de la Magie et a utilisé le sortilège de la Mort pour abattre deux des trois Mangemorts qui l'accompagnaient. Toute l'Angleterre connaît le reste de l'histoire. Devrais-je la poursuivre~?" La vieille sorcière leva à nouveau les yeux de son dossier. "Très bien. Il y eut un procès au Magenmagot durant lequel ce jeune homme fut exonéré de son utilisation du sortilège de la Mort, en grande partie grâce aux efforts de sa grand mère, la Dame de sa maison. Il se réconcilia avec sa famille et ils organisèrent une grande réunion pour lui souhaiter la bienvenue. L'invité d'honneur arriva à cette réunion pour découvrir toute sa famille tuée par des Mangemorts, elfes de maison compris, et que lui, de la lignée cadette, était maintenant le dernier héritier d'une maison Noble."

Le professeur de Défense n'avait réagit à rien de tout cela, mis à part ses yeux qui s'étaient à moitié clos, comme par lassitude.

"Le jeune homme prit le siège de sa famille au Magenmagot et devint l'une des voix les plus tenaces contre Vous-Savez-Qui. Il mena plusieurs fois des forces contre les Mangemorts et les combattit au moyen d'habiles tactiques et d'un pouvoir extraordinaire. Les gens commencèrent à le dire être le prochain Dumbledore et on pensait qu'il pourrait devenir ministre de la Magie après la chute du Seigneur des Ténèbres. Le trois juillet 1973, il ne se présenta pas à un vote de premier importance du Magenmagot et plus personne n'entendit jamais parler de lui. Nous avons supposé que le Seigneur des Ténèbres l'avait tué. Ce fut un terrible coup porté contre nous tous et les choses ne firent qu'empirer." Le regard de la vieille sorcière était interrogateur. "Je vous ai pleuré. Que s'est-il passé~?"

Les épaules du professeur de Défense bougèrent légèrement, comme un petit haussement. "Vous faites beaucoup d'hypothèses," dit-il doucement. "Quant à moi, je crois que l'homme est mort il y a des années. Mais si cet homme est néanmoins en vie -- alors il est clair qu'il ne souhaite pas voir ce fait annoncé et a des raisons d'être silencieux. Il semble que cet homme vous a un jour quelque peu aidé." Les lèvres du professeur de Défense se recourbèrent en un sourire cynique. "Mais je ne suis plus surpris lorsque la gratitude est fugace. Allez-vous exiger encore plus de lui~?"

La vieille sorcière s'inclina dans son fauteuil de gardien et sembla assez surprise, peut-être même blessée. "Non…" dit-elle après un moment. Ses doigts heurtèrent le dossier de cuir, \emph{nerveusement}, vous seriez-vous dit si vous aviez cru qu'Amélia Bones pouvait être nerveuse. "Mais votre \emph{Maison} -- il ne reste pas beaucoup d'anciennes Maisons…"

"Qu'il reste huit ou sept anciennes Maisons ne changera que peu de choses pour ce pays."

La vieille sorcière soupira. "Que Dumbledore pense-t-il de cela~?"

L'homme dans la cellule secoua sa tête. "Il ignore qui je suis et a promis de ne pas se renseigner."

Les sourcils de la vieille sorcière s'élevèrent. "Alors comment vous identifie-t-il auprès du système de sécurité de Poudlard~?"

Un léger sourire. "Le directeur a dessiné un cercle et a dit à Poudlard que celui qui s'y tenait était le professeur de Défense. En parlant de cela…" le ton devint plus grave, plus plat. "Je manque mes cours, madame la directrice."

"Vous semblez -- \emph{vous reposer} parfois d'une façon particulière. On m'a aussi fait part de cela. Et vous semblez vous \emph{reposer} de plus en plus fréquemment à mesure que le temps passe." Les doigts de la vieille sorcière heurtèrent à nouveau le cuir du dossier. "Je ne puis me rappeler avoir lu quoi que ce soit au sujet d'un tel symptôme, mais lorsqu'on entend ça, on imagine… des combats contre des mages noirs, des malédictions terribles…"

Le professeur de Défense demeura inexpressif.

"Demandez-vous l'assistance d'un guérisseur~?" dit Amélia Bones. Son masque avait glissé et ses yeux révélaient clairement sa douleur. "Y a-t-il quoi que ce soit que nous puissions faire pour vous~?"

"J'ai accepté d'enseigner la Défense à Poudlard," dit l'homme dans la cellule d'un ton catégorique. "Tirez-en vos propres conclusions, madame. Et je rate le peu de cours qui me restent. Je voudrais rentrer à Poudlard maintenant."

\later

Lorsque Hermione s'éveilla la troisième fois (bien qu'il semble qu'elle n'ait fermé les yeux qu'un instant), le soleil était encore plus bas dans le ciel, presque entièrement couché. Elle se sentit un peu plus vivante et, étrangement, encore plus épuisée. Cette fois c'était le professeur Flitwick qui se tenait à côté de son lit et secouait ses épaules, un plateau d'une nourriture fumante flottant à côté de lui. Il lui semblait que Harry Potter aurait dû être en train de se pencher au-dessus de son lit, mais il n'était pas là. L'avait-elle rêvé~? Elle n'arrivait pas à se souvenir avoir rêvé.

Il apparut (selon les dires du professeur Flitwick) que Hermione avait raté le dîner dans la grande salle et qu'on la réveillait afin qu'elle mange. Elle pourrait alors rentrer au dortoir Serdaigle et finir sa nuit dans son propre lit.

Elle mangea en silence. Une partie d'elle voulait demander au professeur Flitwick s'\emph{il} pensait qu'elle avait reçut un sortilège de faux souvenirs ou s'il pensait qu'elle avait essayé de tuer Drago de son plein gré…

\emph{… comme elle se souvenait l'avoir fait…}

… mais la majeure partie de Hermione avait peur de le savoir. \emph{Peur de le savoir} était un signal d'avertissement selon Harry et ses livres~; mais son esprit semblait fatigué, \emph{blessé}, et elle n'arrivait pas à trouver la force de se surpasser.

Lorsqu'elle et le professeur Flitwick quittèrent l'infirmerie, ils trouvèrent Harry Potter assis en tailleurs devant la porte en train de tranquillement lire un livre de psychologie.

"Je continue avec elle," dit le Survivant. "Le professeur McGonagall a dit que ça irait."

Le professeur Flitwick sembla accepter cela et quitta après leur avoir jeté un regard sévère à tous deux. Elle n'arrivait pas à imaginer ce que le regard sévère était censé dire, à moins que ça n'ait été~: \emph{n'essayez pas de tuer d'autres élèves.}

Les pas du professeur Flitwick s'estompèrent et ils se retrouvèrent tous les deux devant les portes de l'infirmerie.

Elle regarda les yeux verts du Survivant, la masse de cheveux qui ne masquait pas tout à fait la cicatrice sur son front~; elle regarda le visage du garçon qui aurait donné tout son argent pour la sauver sans une seule arrière-pensée. Des sentiments s'agitaient en elle -- la culpabilité, la honte, la gêne, et d'autres encore -- mais aucun mot. Il n'y avait rien qu'elle sache dire.

"Donc," dit soudain Harry, "j'ai rapidement passé en revue mes livres de psychologie pour voir ce qu'ils avaient à dire sur les troubles de stress post-traumatique. Les vieux livres disent qu'il faut parler de son expérience immédiatement après avec un psychothérapeute. Les nouveaux résultats de recherchent disent qu'après avoir conduit de véritables expériences, il s'avère qu'en parler immédiatement après aggrave plutôt les choses. Apparemment, ce qu'il faut vraiment faire est de suivre l'impulsion naturelle de l'esprit et de réprimer le souvenir pendant un moment, de ne juste pas y penser pendant un moment."

C'était tellement \emph{normal}, tellement proche de la façon dont Harry et elle parlaient d'habitude qu'elle sentit une soudaine sensation de brûlure dans sa gorge.

\emph{Nous n'avons pas à en parler}. C'était ce que Harry venait de dire, à peu de choses près. Cela lui donnait l'impression de tricher, peut-être même de mentir. Rien \emph{n'était} normal. Tout ce qui allait mal allait encore horriblement mal, tout ce qui n'avait pas été dit devait encore l'être…

"D'accord," dit Hermione, parce qu'il n'y avait rien d'autre à dire, absolument rien d'autre.

"Je suis désolé de ne pas avoir été là quand tu t'es réveillée," dit Harry alors qu'ils commençaient à marcher. "Madame Pomfresh ne m'a pas laissé entrer alors je suis juste resté là, dehors." Il eut un petit haussement d'épaules triste. "J'imagine que je devrais être en train d'essayer de limiter les dégâts en termes de relations publiques mais… franchement je n'ai jamais été bon à ça, je finis toujours par parler durement aux gens."

"C'est très grave~?" elle songea que sa voix aurait dû être un soupir ou un croassement, mais non.

"Eh bien…" dit Harry en hésitant visiblement. "Ce que tu dois comprendre, Hermione, c'est que tu avais beaucoup de gens pour te défendre au petit déjeuner d'aujourd'hui, mais que tous les gens de ton côté… \emph{inventaient n'importe quoi}. Drago a essayé de te tuer en premier, des choses dans le genre. C'était Granger contre Malfoy, c'est comme ça que les gens l'ont vu, comme s'il y avait une scie et qu'appuyer de son côté ferait remonter le tien. Je leur ai dit que vous étiez probablement \emph{tous les deux} innocents, que vous aviez tous les deux reçu un sortilège de faux souvenirs. Ils n'ont pas écouté, les deux camps m'ont traité comme un traître à la recherche d'un juste milieu. Puis des gens ont entendu que Drago avait attesté sous Veritaserum qu'il avait essayé de t'aider avant la bataille -- arrête de faire cette tête, Hermione, tu ne lui as pas réellement fait du mal. Bref, tout le monde en a tiré que le camp pro-Malfoy avait eu raison et que le camp pro-Granger avait eut tort." Harry eut un bref soupir. "Je leur ai \emph{dit} que quand la vérité éclaterait plus tard ils seraient honteux…"

"C'est très grave~?" demanda-t-elle à nouveau. Cette fois, sa voix fut plus faible.

"Tu te souviens de l'expérience de conformité de Asch~?" dit Harry en tournant la tête pour la regarder avec sérieux.

Son esprit mit \emph{longtemps à se rappeler}, quelques secondes, ce qui l'effraya, puis la référence revint. En 1951, Solomon Asch avait pris des sujets expérimentaux et chacun avait été mis dans une ligne avec d'autres personnes qui leur ressemblaient et prétendaient être d'autres sujets expérimentaux tout en étant en réalité des complices de l'expérimentateur. Asch avait montré une ligne de référence sur un écran marquée X à côté de trois autres lignes marquées A, B et C. L'expérimentateur avait demandé quelle ligne avait la même longueur que X. La bonne réponse était C, de façon évidente. Les autres 'sujets', les complices, avaient l'un après l'autre dit que X était de la même longueur que B. Le vrai sujet avait été placé avant-dernier afin de n'éveiller aucun soupçon en le plaçant en dernier. Le test avait été de voir si le véritable sujet se 'conformerait' à la mauvaise réponse standard, B, ou dirait la réponse évidemment correcte, C.

75~\% des sujets avaient s'étaient 'conformés' au moins une fois. Un tiers des sujets s'étaient conformés plus de la moitié du temps. Certains avaient ensuite indiqué être réellement persuadés que X était de la même longueur que B. Et ça avait été dans un cas où les sujets ne connaissaient aucun des complices. Si on plaçait quelqu'un dans un groupe de gens qui lui ressemblaient, par exemple quelqu'un en fauteuil roulant au milieu de gens en fauteuil roulant, l'effet de conformité devenait encore plus puissant…

Hermione avait l'écœurante impression de savoir où cela allait. "Je me souviens," murmura-t-elle.

"Tu sais, j'ai donné un entraînement anti-conformité à la légion du Chaos. J'ai demandé à chaque Légionnaire de se placer au milieu du groupe et de dire 'Deux et deux font quatre~!' ou 'L'herbe est verte~!' pendant que tous les autres soldats de la Légion le traitaient d'idiot ou riaient de lui -- les ricanements d'Allen Flint étaient vraiment bons -- ou le regardaient d'un air atterré avant de tourner les talons et de s'en aller. Ce dont tu dois te rappeler c'est que \emph{seule} la Légion du Chaos a déjà pratiqué quelque chose comme ça. Personne d'autre à Poudlard ne sait même ce \emph{qu'est} la conformité."

"Harry~!" sa voix vacillait. "Est-ce que c'est très grave~?"

Harry eut un autre haussement d'épaules triste. "Tout le monde en deuxième année et plus, puisqu'ils ne te connaissent pas. Tout le monde dans l'armée Dragon. Tout Serpentard bien sûr. Et, eh bien, presque tout le reste de l'Angleterre magique aussi, je pense. Rappelle toi que Lucius Malfoy contrôle la \emph{Gazette du Sorcier}."

"Tout le monde~?" chuchota-t-elle. Ses membres commencèrent à devenir froid, comme si elle venait de sortir d'un piscine non chauffée.

"Ce que les gens croient vraiment ne leur semble pas être une \emph{croyance}, ça leur semble être un \emph{état du monde}. Toi et moi nous tenons dans cette petite sphère privée de l'univers où Hermione Granger a reçu un sortilège de faux souvenirs. Le reste du monde vit dans un monde où Hermione Granger a essayé de tuer Drago Malfoy. Si Ernie Macmillan…"

Hermione bloqua une inspiration à mi-parcours. \emph{Le capitaine Macmillan…}

"… pense qu'il lui est maintenant éthiquement interdit d'être ton ami, enfin, il essaie de faire ce qui est juste en fonction de ce qu'il croit, dans le monde où il pense vivre." Les yeux de Harry étaient très sérieux. "Hermione, tu m'as souvent dit que je prends trop souvent les gens de haut. Mais si j'attendais trop d'eux -- si j'attendais des gens qu'ils \emph{comprennent} -- alors je les haïrais vraiment. Idéalisme mis à part, les élèves de Poudlard ne connaissent \emph{pas} assez de sciences cognitives pour être tenus pour responsables de la façon dont leurs esprits fonctionnent. Ce n'est pas leur faute s'ils sont fous." La voix de Harry était étrangement douce, presque comme celle d'un adulte. "Je sais que ça va être plus dur pour toi que ça le serait pour moi. Mais souviens toi, le vrai coupable finit par se fait avoir. Quand la vérité éclatera, tout ceux qui étaient confiants dans leur erreur seront honteux."

"Et si le vrai coupable ne se fait pas attraper~?" dit-elle d'une voix tremblante.

… \emph{Ou s'il s'avère que c'est moi après tout~?}

"Alors tu pourras quitter Poudlard et aller à l'Institut des Sorcières de Salem, aux États-Unis."

"\emph{Quitter Poudlard~?}" Elle n'avait jamais songé à cette possibilité autrement que comme la punition ultime."

"Je… Hermione, je pense que tu pourrais vouloir faire ça de toute façon. Poudlard n'est pas un château, c'est de la folie emmurée. Tu \emph{as} d'autres possibilités."

"Je…" bégaya-t-elle. "Je devrai… y réfléchir…"

Harry hocha la tête. "Au moins personne n'essaierai de te lancer un sortilège, pas après ce que le directeur a dit au dîner ce soir. Oh, et Ron Weasley est venu me voir avec un air très sérieux et m'a dit que si je te voyais en premier, je devrais te dire qu'il est désolé d'avoir pensé du mal de toi et qu'il ne dira plus jamais rien de méchant à ton égard."

"\emph{Ron} croit que je suis innocente~?" dit Hermione.

"Eh bien… ce n'est pas qu'il croit que tu es \emph{innocente…"}

\later

Tout le dortoir de Serdaigle devint silencieux lorsqu'ils entrèrent.

Et les regarda.

La regarda.

(Elle avait eu ce genre de cauchemars)

Puis, un par un, les gens détournèrent leur regard.

Pénélope Deauclaire, la préfète de 5\textsuperscript{ème} année responsable des première année, détourna lentement et délibérément le regard.

Su Li, Lisa Turpin et Michael Corner, tous assis à la même table, qu'elle avait tous aidés pour leurs devoirs, ils détournèrent tous les yeux, leurs visages devenus soudain nerveux lorsqu'elle avait essayé de saisir leur regard.

Une troisième année appelée Latisha Randle que la S.P.E.H.S avait sauvé de brutes Serpentardes par deux fois se pencha vivement sur son bureau et se remit à ses devoirs.

Mandy Brocklehurst détourna le regard.

Si Hermione n'éclata alors pas en sanglots, ce n'est que parce qu'elle s'y était attendu, qu'elle l'avait rejoué dans son esprit encore et encore. Au moins les gens ne lui hurlaient pas dessus, ne la poussaient pas et ne lui lançaient pas de maléfices. Ils détournaient seulement le regard…

Hermione avança droit vers l'escalier qui menait directement au dortoir des filles en première année (Elle ne vit pas que Padma Patil et Anthony Goldstein la regardaient, leurs têtes seules à suivre son mouvement) De derrière elle, elle entendit Harry Potter dire d'un ton très calme~: "Vous tous, la vérité finira par éclater. Donc si vous êtes tous si sûrs qu'elle est coupable, pourrais-je tous vous demander de signer ce papier ici qui dit que si elle s'avère avoir été innocente, elle aura le droit de vous dire 'Je te l'avais dit' et de vous en vouloir jusqu'à la fin de vos jours~? Allez y, ne soyez pas lâches, si vous y croyez vraiment vous ne devriez pas avoir peur de parier…"

Elle avait parcouru la moitié des escaliers lorsqu'elle se rendit compte qu'il y aurait aussi d'autres filles dans le dortoir.

\later

Les étoiles n'étaient pas encore tout à fait visible, on ne pouvait voir qu'une ou deux des plus brillantes à travers le voile rouge et violet de l'horizon, mais le soleil s'était couché.

Les mains de Hermione pressèrent la pierre du parapet qui protégeait le petit balcon où elle s'était réfugiée, loin de la cage d'escaliers, après s'être rendue compte que…

…\emph{elle ne pouvait juste pas aller dormir…}

… les mots faisaient écho dans son esprit sur le même ton qu'auraient eu ceux-ci~: "Tu ne peux plus rentrer chez toi."

Elle regarda le terrain vide, le soleil couchant, l'herbe nouvelle, si loin, en contrebas.

Fatiguée, elle était fatiguée, elle ne pouvait plus penser, elle avait besoin de dormir. Le professeur Flitwick lui avait dit qu'elle en avait besoin et il y avait eu une potion de plus pour accompagner son dîner. Peut être était-ce ainsi que la société sorcière traitait les horribles traumatismes infligés à de jeunes filles innocentes~: en les faisant juste beaucoup dormir.

Elle aurait dû aller dans sa chambre et dormir mais elle avait peur de se rendre à un endroit où d'autre personnes pouvaient se trouver. Peur de la façon dont ils pourraient la regarder ou ne pas la regarder.

À mesure que la nuit prenait place, des fragments de pensées se couraient l'un après l'autre dans un esprit trop épuisé pour les achever ou les relier.

\emph{Pourquoi…}

\emph{Pourquoi tout cela a-t-il eu lieu…}

\emph{Tout allait bien il y a une semaine…}

\emph{Pourquoi…}

De derrière elle vint le grincement d'une porte qui s'ouvrait.

Elle tourna la tête et regarda.

Le professeur Quirrell était incliné contre le chambranle de la porte qu'elle venait de traverser, détouré comme une silhouette cartonnée par la lumière des torches de Poudlard situées derrière lui. Elle ne pouvait pas voir l'expression de son visage bien que le passage dans son dos soit éclairé~; ses yeux, son visage, tout ce qu'elle pouvait voir se trouvait dans l'ombre de la nuit.

Le professeur de Défense de Poudlard, numéro un sur la liste des coupables potentiels. Elle ne s'était même pas rendue compte qu'elle \emph{avait} une liste de suspect avant cet instant.

L'homme se tenait devant cette porte sans rien dire et elle ne pouvait pas voir ses yeux. Que \emph{faisait-}il ici, en premier lieu~?

"Êtes-vous ici pour me tuer~?" dit Hermione Granger.

Sur ces mots, la tête du professeur Quirrell s'inclina.

Puis il s'élança vers elle, la sombre silhouette levant une main lentement, délibérément, comme pour la faire tomber de la tour Serdaigle…

"\emph{Stupéfix~!"}

Le jet d'adrénaline écrasa tout le reste, elle sortit sa baguette sans avoir pensé, ses lèvres formèrent le mot d'elles-mêmes, le tir jaillit et…

…\emph{s'arrêta lentement} devant la main levée du professeur Quirrell, ondulant dans un vol suspendu comme s'il essayait encore d'avancer, émettant un sifflement.

La lueur rouge illumina le visage du professeur Quirrell pour la première fois et révéla un étrange et affectueux sourire.

"Mieux," dit le professeur Quirrell. "Mlle Granger, vous êtes toujours élève de mon cours de Défense. À ce titre, si vous me considérez comme une menace, j'attends de vous que ne fassiez pas que me regarder tristement et me demander si je suis ici pour vous tuer. Deux points Quirrell en moins."

Elle fut entièrement incapable de formuler une réponse.

Le professeur de Défense fit une chiquenaude nonchalante de l'index vers le tir suspendu et envoya le maléfice au-dessus de sa tête, loin dans la nuit, si bien qu'ils se retrouvèrent dans les ténèbres. Puis le professeur Quirrell s'éloigna de la porte qui se referma d'un coup derrière lui et une douce lumière blanche apparut autour d'eux, si bien qu'elle pu de nouveau voir son visage, toujours avec cet étrange sourire affectueux.

"Que… que \emph{faites}-vous ici~?"

Quelque pas de plus menèrent le professeur Quirrell à une partie plus élevée du balcon, où il posa ses coudes sur la pierre et s'inclina très avant, regardant la nuit.

"Je suis venu ici immédiatement après avoir été relâché par les Aurors et à l'instant où j'ai fini de faire mon rapport au directeur," dit le professeur Quirrell d'une voix tranquille, "parce que je suis votre professeur, que vous êtes mon élève et que je suis responsable de vous."

Hermione comprit alors~; elle se souvint de ce que le professeur Quirrell avait dit à Harry lors de sa seconde leçon de Défense de l'année au sujet du contrôle de sa colère. Elle sentit une vague de honte saisir sa poitrine. Il lui fallut un moment pour que le savoir surmonte la mortification, pour qu'elle force les mots à sortir…

"Je…" dit Hermione. "Harry pense… que je ne me \emph{suis pas} emportée, je veux dire…"

"C'est ce que j'ai entendu," dit le professeur Quirrell d'un ton plutôt sec. Il secoua la tête comme à l'intention des étoiles elles-mêmes. "Le garçon a la chance que j'ai dépassé le stade de l'agacement face à ses tendances autodestructrices et en sois à la pure curiosité quant à ce qu'il va faire ensuite. Mais je suis d'accord avec l'interprétation des faits de M. Potter. Ce meurtre était très bien préparé pour échapper à la détection, à la fois de Poudlard et de l'œil aux aguets du directeur. Naturellement, lors d'un meurtre si bien pensé, quelque innocent recevrait le blâme." Un bref sourire ironique passa sur les lèvres du professeur de Défense, mais il ne la regardait pas. "Quant à l'idée que vous l'ayez fait vous-même -- je me considère être un professeur de talent, mais même moi ne pourrais enseigner une telle intention meurtrière à une élève aussi obstinée et dénuée de talent que Hermione Granger."

La partie de son cerveau qui répondit \emph{Quoi~?} d'un ton indignée fut loin d'être assez forte pour atteindre ses lèvres.

"Non…" dit le professeur Quirrell. "Ce n'est pas pour cela que je suis ici. Vous n'avez fait aucun effort pour masquer votre antipathie à mon égard, Mlle Granger. Je vous remercie pour cette absence de prétentions, car je préfère de loin la véritable haine à l'amour faux. Mais vous êtes toujours mon élève et j'ai quelque chose à vous dire, si vous voulez bien l'écouter."

Hermione le regarda tout en continuant de combattre les effets de l'adrénaline de quelque instants auparavant. Le professeur de Défense semblait juste regarder le ciel noir dans lequel les étoiles apparaissaient.

"Il fut un jour où j'allais devenir un héros," dit le professeur Quirrell, les yeux toujours levés. "Pouvez-vous y croire, Mlle Granger~?"

"Non."

"Merci, une fois de plus, Mlle Granger. C'est néanmoins vrai. Il y a longtemps, longtemps avant votre époque ou celle de Harry Potter, fut un homme salué comme un sauveur. L'héritier dont chacun reconnaissait le destin grâce aux contes de fées, maniant la justice et la vengeance comme deux baguettes face à sa terrible Némésis." Le professeur Quirrell eut un rire doux et amer avant de relever les yeux vers le ciel nocturne. "Savez-vous, Mlle Granger, qu'à cette époque je me croyais déjà cynique et que pourtant… eh bien."

Le silence s'étira dans le froid et la nuit.

"En toute honnêteté," dit le professeur Quirrell en relevant la tête vers les étoiles, "je ne comprends toujours pas. Ils auraient dû savoir que leurs vies dépendaient du succès de cet homme. Et pourtant c'était comme s'ils essayaient de faire leur possible pour rendre sa vie \emph{déplaisante}. Pour placer tous les obstacles possibles sur son chemin. Je n'étais pas naïf, Mlle Granger, je ne m'attendais pas à ce que les puissants se rallient à moi aussi vite -- pas sans qu'ils aient quelque chose à y gagner. Mais leur pouvoir à eux aussi était menacé, et j'étais donc choqué de voir comme ils semblaient heureux de s'écarter et de laisser à cet hommes tous les fardeaux de la responsabilité. Ils se gaussaient de ses réussites, se faisaient remarquer l'un à l'autre à quel point ils auraient fait mieux à sa place, même s'ils ne s'abaissaient pas à le faire." Le professeur Quirrell secoua la tête comme avec perplexité. "Et c'était la chose la plus étrange… le mage noir, la terrible Némésis de cet homme -- eh bien, ceux qui \emph{le} servaient remplissaient leurs tâches avec zèle. Le mage noir devenait plus cruel envers ses adeptes et ils le suivaient d'autant plus. Les gens se battaient pour avoir la chance de \emph{le} servir tandis que ceux dont la vie dépendait de cet autre homme se sentaient libre de rendre la sienne difficile… je ne parvenais pas à comprendre, Mlle Granger." Le visage du professeur Quirrell était dans l'ombre, les yeux levés. "Peut-être en prenant sur lui le fardeau de l'action l'avait-il ôté des épaules de tous les autres~? Était-ce pour cela qu'ils se sentaient libre de nuire à sa bataille contre le mage noir qui les aurait tous mis en esclavage~? Il s'avère que ce n'était pas du cynisme mais de l'optimisme absolu que de croire que les hommes agiraient pour leur intérêt personnel. En réalité, les hommes ne s'élèvent pas jusque là. Et il finit donc par se rendre compte qu'il ferait mieux de combattre le mage noir seul qu'avec de tels adeptes sur son dos."

"Donc…" dans la nuit, la voix de Hermione était étrange. "Vous avez laissé vos amis derrière, en sécurité, et avez essayé d'attaquer le mage noir tout seul~?"

"Allons, pas du tout," dit le professeur Quirrell. "J'ai arrêté d'essayer d'être un héros et je suis parti faire quelque chose de plus plaisant."

"\emph{Quoi~?"} dit Hermione sans même réfléchir. "C'est \emph{horrible} !"

Le professeur de Défense abaissa la tête, se détourna du ciel, et elle vit dans la lumière de la porte qu'il souriait -- ou du moins que la moitié de son visage souriait. "Mlle Granger, allez-vous me dire que je suis quelqu'un d'horrible~? Eh bien peut-être. Mais alors les gens qui n'essaient même pas de devenir des héros sont-ils encore pires~? Si je n'avais jamais essayé de faire quoi que ce soit, auriez-vous eu une meilleure opinion de moi~?"

Hermione ouvrit la bouche et découvrit à nouveau qu'une fois de plus, elle n'avait rien à dire. Ce n'était pas bien se désister de la tâche de héros, on ne pouvait tout simplement \emph{pas} faire ça, mais elle ne \emph{voulait pas} dire que tous ceux qui n'étaient pas un héros ne valaient rien, ça aurait été penser comme Quirrell…

Le sourire ou demi-sourire avait disparu. "Vous étiez idiote," dit doucement le professeur de Défense, "de vous attendre à la moindre gratitude de la part de ceux que vous aviez essayé de protéger après vous être désignée comme héroïne. Tout comme attendez de cet homme qu'il continue d'être un héros et l'avez qualifié d'horrible parce qu'il s'est arrêté, alors que mille autres n'ont jamais levé un doigt. Il était \emph{attendu} que vous combattiez les brutes. C'était une taxe due et ils l'ont acceptée comme des princes, avec une moue moqueuse pour le retard de votre paiement. Et je gage que vous avez déjà été témoin de leur affection, disparaissant comme la poussière sous le vent une fois qu'il n'était plus dans leur intérêt de s'associer à vous…"

Le professeur de Défense se raidit lentement sur le balcon, se tint presque droit et se tourna pour lui faire pleinement face.

"Mais vous n'avez pas à être une héroïne, Mlle Granger, dit le professeur Quirrell. "Vous pouvez vous arrêter quand bon vous semble."

Cette idée…

… lui \emph{était} déjà venue plusieurs fois lors des deux derniers jours.

\emph{Les gens deviennent ce qu'ils sont censés devenir en faisant ce qui est juste}, lui avait dit le directeur. Le problème était qu'il y avait deux choses justes à faire. Il y avait la partie d'elle qui disait que la \emph{bonne} chose à faire était de continuer d'être une héroïne et de rester à Poudlard, car même si elle ne comprenait pas ce qui se passait, une héroïne ne se contenterait pas de fuir.

Et il y avait aussi la voix du bon sens qui lui disait que les jeunes enfants ne devraient jamais rester proche du danger, que c'était à cela que les adultes servaient~; la voix de toutes les affiches à l'école qui disaient de ne pas prendre les bonbons des inconnus. Cette voix aussi avait raison.

Hermione se tenait sur le balcon et regardait la silhouette du professeur Quirrell découpée par les étoiles qui émergeaient et elle ne comprenait pas, elle ne comprenait pas comment le professeur de Défense pouvait la regarder d'un air aussi inquiet, elle ne comprenait pas les notes de douleur qu'elle avait remarquées dans la voix du professeur, elle ne comprenait même pas \emph{pourquoi} on lui disait tout cela.

"Vous ne m'appréciez même pas, professeur," dit-elle.

Un léger sourire vacilla sur le visage du professeur Quirrell. "J'imagine que je pourrais m'étaler sur ma colère contre cette affaire qui a empiété sur mon temps précieux et a perturbé mes cours de Défense. Mais avant tout, Mlle Granger, vous êtes mon élève et quelles qu'aient été mes professions passées, je pense avoir été un bon enseignant à Poudlard, n'est-ce pas~?" Les yeux du professeur de Défense semblèrent soudain très fatigués. "Alors, en tant qu'enseignant, je vous signale que vous avez d'autres plans de carrière possibles. Je n'aimerais voir personne suivre la voie que j'ai suivi."

Hermione déglutit. C'était là un aspect du professeur Quirrell qu'elle n'avait jamais vu ni imaginé, et cet aspect dévorait ses préjugés au sujet de celui-ci.

Le professeur Quirrell la regarda pendant un moment puis détourna les yeux d'elle, les releva vers les étoiles. Lorsqu'il reparla, sa voix était plus basse. "Quelqu'un ici vous prend pour cible, Mlle Granger, et je ne peux vous protéger comme j'ai protégé M. Malfoy. Le directeur l'empêche pour des raisons qu'il juge être bonnes. Il est facile de s'attacher à Poudlard, je le sais, car j'y suis moi aussi attaché. Mais en France ils ne voient pas les Anciennes Maisons comme en Angleterre~; et je ne pense pas que Beauxbâtons vous maltraiterait. Quoi que vous pensiez de moi par ailleurs, je vous jure que si vous me demandez de vous conduire de façon sûre à Beauxbâtons, je ferai tout ce qui est en mon pouvoir pour vous y amener."

"Je ne peux pas juste…" dit Hermione.

"Mais vous le \emph{pouvez}, Mlle Granger." Les pâles yeux bleus la regardaient à présent avec intensité. "Quoi que vous souhaitiez faire de votre vie, vous ne pourrez plus l'atteindre à Poudlard, plus maintenant. Ce lieu est mort pour vous, même en ignorant les autres menaces. Demandez simplement à Harry Potter de vous ordonner d'aller à Beauxbâtons et vivez le reste de votre vie en paix. Si vous demeurez ici, il sera votre maître aux yeux de l'Angleterre et de ses lois~!"

Elle n'y avait même pas réfléchi tant c'était insignifiant face à l'idée d'être mangée par des Détraqueurs. Cela avait auparavant été important pour elle et cela semblait maintenant puéril, sans importance, vide. Mais alors pourquoi ses yeux la brûlaient-elle~?

"Et si cela échoue à vous convaincre, Mlle Granger, considérez aussi que M. Potter, ne serait-ce qu'aujourd'hui à l'heure du déjeuner, a menacé Lucius Malfoy, Albus Dumbledore et l'intégralité du Magenmagot, juste parce qu'il est incapable de réfléchir lorsque quelque chose menace de vous arracher à lui. N'avez-vous pas peur de ce qu'il fera ensuite~?"

Cela se tenait. De façon terrible. De façon atroce, épouvantable.

Cela se tenait \emph{trop bien}…

Elle n'aurait pas pu décrire avec des mots ce qui déclencha la compréhension, à moins que ce ne soit la simple \emph{pression} que le professeur de Défense exerçait sur elle.

Si le professeur de Défense \emph{était} derrière tout ça -- alors le professeur Quirrell avait fait tout ça \emph{juste pour l'écarter du chemin tracé par les plans qu'il avait concernant Harry.}

Sans décision conscience elle déplaça son centre de gravité vers son autre jambe et écarta son corps du professeur de Défense…

"Donc vous pensez que je suis le responsable~?" dit le professeur Quirrell. Sa voix fut un peu triste lorsqu'il dit cela et le cœur de Hermione s'arrêta presque lorsqu'elle l'entendit. "J'imagine que je ne peux pas vous en vouloir. Je suis le professeur de Défense de Poudlard après tout. Mais Mlle Granger, même en \emph{supposant} que je suis votre ennemi, le bon sens devrait quand même vous dire de vous écarter de moi \emph{très vite}. Vous ne pouvez pas utiliser le sortilège de la Mort, la bonne tactique est donc de Transplaner au loin. Cela ne me dérange pas d'être le méchant de votre imagination si cela rend les choses plus claires. Quittez Poudlard, et laissez-moi à ceux qui peuvent s'occuper de moi. J'arrangerai votre transport par une famille de bonne réputation et M. Potter saura me blâmer si vous n'arrivez pas entière."

"Je…" elle avait froid, l'air de la nuit rafraîchissait sa peau, à moins qu'il ne fut rafraîchi par elle. "Je dois y penser…"

Le professeur Quirrell secoua la tête. "Non, Mlle Granger. Je mettrai du temps à arranger votre départ et j'ai moins de temps que vous ne le pensez. Cette décision sera peut-être douloureuse mais elle ne doit pas être ambiguë~; il y a beaucoup de poids sur cette balance, mais il est n'est pas équitablement réparti. Je dois savoir ce soir si vous comptez partir."

\emph{Et si non…}

Le professeur de Défense la mettait-il en garde~? Que si elle ne partait pas il frapperait de nouveau~?

Pourquoi était-ce si important, que le professeur Quirrell voulait-il \emph{faire} de Harry~?

\emph{Hermione Granger, je serai moins subtil que les vieux sorciers le sont habituellement et je vous annonce catégoriquement que vous ne pouvez pas imaginer à quel point les choses pourraient mal tourner si les événements qui gravitent autour de Harry Potter prenaient un mauvais détour.}

Le plus puissant sorcier du monde lui avait dit cela en lui parlant de l'importance qu'il y avait à ce qu'elle ne cesse \emph{pas} d'être l'amie de Harry.

Hermione déglutit, elle chancela un peu, debout sur le balcon de pierre d'un château magique. La mortelle absurdité de la situation sembla soudain s'élever et la saisir à la gorge, des filles de douze ans ne \emph{devraient pas} être en danger, ne \emph{devraient pas} penser à de telles choses, et Maman lui dirait de S'ENFUIR, et son père aurait une crise cardiaque s'il savait seulement les questions auxquelles elle faisait face.

Et elle sut alors, comme Harry et Dumbledore avaient essayé de la mettre en garde, que tout ce qu'elle avait toujours cru sur la condition d'héroïne avait été erroné. Qu'il n'y avait rien de tel que des héros à l'extérieur des histoires. Seulement un horrible danger, être arrêtée par des Aurors, mise en cellule à côté de Détraqueurs, de la douleur, et de la peur et…

"Mlle Granger~?" dit le professeur de Défense.

Elle ne dit rien. Tous les mots étaient bloqués dans sa gorge.

"Il me faut une décision, Mlle Granger."

Elle garda sa mâchoire coincée et ne laissa aucun mot sortir.

Le professeur de Défense finit par soupirer. Lentement la lumière blanche disparut et lentement la porte derrière s'ouvrit, si bien qu'il fut de nouveau une silhouette noire détourée par l'ouverture. "Bonne nuit, Mlle Granger," dit-il, puis il lui tourna le dos et partit vers Poudlard.

Il lui fallut un moment pour que sa respiration ralentisse. Quoi qu'il se soit produit ici, cette nuit, cela ne ressemblait en rien à une victoire. Elle s'était battu si fort pour ne pas dire \emph{oui} face à la pression du professeur de Défense, et elle ne savait même pas si ça avait été la bonne décision.

Puis elle revint à son tour vers la lumière et commença à monter les escaliers vers son dortoir.

Les autres filles étaient probablement déjà endormies, elles ne la regardaient pas plus qu'elles ne détourneraient les yeux…

Elle sentit les larmes couler et cette fois elle n'essaya pas de les arrêter. 

%  LocalWords:  Arjun Altunay RJ L20 Oni Latisha
