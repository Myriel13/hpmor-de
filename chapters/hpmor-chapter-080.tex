\namedpartchapter{Compromis Tabous}{Compromis Tabous}{II}{Le halo d'infamie}

\lettrine{L}{a} Chambre Très Ancienne du Magenmagot est froide, sombre et dotée de demi-cercles concentriques de pierres qui s'élèvent du point le plus bas de son centre ainsi que de simples bancs de bois disposés sur ces demi-cercles surélevés. Il n'y a aucune source de lumière mais le lieu est bien éclairé, sans raison ni cause apparente~: le fait est, tout simplement, que la salle est bien éclairée. Les murs comme le sol sont de pierre, d'une pierre noire, élégant et mystérieux accord de roches des plus plaisantes à observer, d'une texture fine qui semble ondoyer et couler sous sa surface. C'est la Chambre Très Ancienne, le plus ancien lieu magique à avoir survécu jusqu'à ce jour car tous les autres lieu ont été détruit lors d'une guerre ou d'une autre. C'est la Chambre du Magenmagot, et si elle est la plus ancienne, c'est que les guerres ont pris fin avec sa construction.

C'est la Chambre du Magenmagot~; des lieux plus anciens existent, mais ils sont cachés. Des légendes soutiennent que les murs de pierre noire furent conjurés, créés, propulsés vers l'existence par la volonté de Merlin lorsqu'il rassembla les plus puissants sorciers encore de ce monde et les émerveilla tant qu'ils l'acceptèrent comme leur chef. Et lorsque (continue la légende) les Voyants persistèrent à prédire que le sacrifice n'était pas suffisant, que trop peu avait été accompli pour empêcher la fin du monde et de sa magie, alors (continue l'histoire) Merlin sacrifia sa vie, sa magie et son temps pour faire appliquer l'Interdit de Merlin. Ce ne fut pas un acte sans prix, car un lieu tel que celui-ci ne pourrait plus jamais être dressé à nouveau par aucun pouvoir connu du genre sorcier. Pas plus que détruit, car c'est sans dommages et peut-être même sans chauffer que ces murs traversaient le cœur d'une explosion nucléaire. Il est fort dommage que plus personne ne sache les construire.

Sur le plus haut des demi-cercles surélevés du Magenmagot, à l'étage pierre noire le plus élevé, se trouve un podium. Et sur ce podium se tient un vieil homme au visage ridé de souci et à la barbe d'argent qui tombe en dessous de sa ceinture~: c'est Albus Percival Wulfric Brian Dumbledore. Sa main droite porte une baguette de pouvoir et sur son épaule perche un oiseau de feu. Sa main gauche tient un bâton court, fin et sans atours, forgé de la même pierre noire que les murs, et il s'agit du Trait ininterrompu de Merlin, l'outil du président sorcier du Magenmagot. Karen Dutton transmit le Trait à Albus Dumbledore au dernier jour de sa vie, quelques heures à peine après le retour de ce dernier de sa victoire contre Grindelwald, un phénix flamboyant à ses côtés. Elle reçut elle-même le Trait du perfectionniste Nicodemus Capernaum~: chaque sorcier le passa ainsi au successeur de son choix et la chaîne peut être remontée ainsi jusqu'au jour où Merlin offrit sa vie. Cela (si vous vous posiez la question) explique comment le pays d'Angleterre magique a pu élire Cornelius Fudge comme premier ministre et pourtant se retrouver avec Albus Dumbledore en président sorcier. Pas par loi (car une loi écrite peut être réécrite) mais par la plus ancienne des traditions, le Magenmagot ne choisit pas qui présidera à ses folies. Depuis le jour du sacrifice de Merlin, le plus important devoir de tout président sorcier a été d'exercer la plus grande précaution dans son choix d'un individu à la fois bon et capable de discerner un bon successeur. On se serait attendu à ce que cette chaîne de lumière se soit ratée au moins une fois, quelque part entre les siècles~; qu'elle se serait fourvoyée au moins une fois et ne serait jamais revenue. Mais elle ne l'a pas fait. Le Trait de Merlin continue, ininterrompu.

(C'est du moins ce que disent ceux du camp de Dumbledore. Lord Malfoy vous dirait autre chose. Et en Asie ils racontent des histoires entièrement différentes qui ne contrediraient pas nécessairement celle d'Angleterre)

Sur la plus basse des plate-formes de l'Ancienne Chambre se trouve une chaise à haut dossier dépourvue de rembourrages, faite d'un métal noir plutôt que d'une pierre noire et que Merlin ne plaça jamais ici.

Le bâtiment ministériel qui a poussé autour de cet endroit est fait de bois et recouvert d'or, puissamment éclairé de feux et plein d'une bêtise affairée. Ce lieu est différent. C'est le cœur de pierre d'Angleterre magique, et il n'est ni recouvert d'or, ni fait de bois ni puissamment éclairé de flammes.

Des sorciers et sorcières aux robes couleurs de prune chacune bordée d'un M d'argent emplissent la pièce avec solennité. Ils se tiennent avec un sérieux qui montre à quel point ils se savent être extrêmement, extrêmement importants. Après tout, ils se réunissent dans l'Ancienne Chambre. Ce sont les Lords et les Dames du Magenmagot et ils se considèrent comme les meilleurs du meilleur des pays magiques. Des inférieurs sont tombés à genoux en supplique face à eux, ils sont puissants, ils sont riches et ils sont nobles~; ne sont-ils pas grands~?

Albus Dumbledore connaît le prénom de tous les occupants de cette pièce. Il a enseigné à nombre d'entre eux, quoique trop peu aient appris. Certains sont ses alliés, d'autres ses ennemis, et il fait la cour aux autres sur la prudente danse de leur neutralité. Il voit chacun d'entre eux comme une personne.

Si vous lui demandiez son opinion de ces Lords et de ces Dames, l'actuel professeur de Défense de Poudlard dirait que, bien que nombreux à être ambitions, ils sont peu à avoir la moindre ambition. Il noterait que le Magenmagot est exactement l'endroit où une personne de ce genre atterrit, que c'est exactement le genre d'opportunité dont on se saisit lorsque l'on a rien de mieux à faire. De tels individus sont rarement intéressants mais sont souvent utiles~; ce sont des pièces à manipuler, des points à marquer par les véritables joueurs de la partie.

Non pas sur l'un des demi-cercles surélevés mais mis à l'écart sur un arc monté pour les spectateurs, placé à côté d'une sorcière au chapeau pointu dont le visage est ridé d'appréhension, se trouve un garçon assis, habillé d'une robe noire la plus formelle de sa garde-robe. Ses yeux sont faits d'une glace verte et absente~; il voit à peine les Lords et les Dames affairés qui entrent. Ils ne sont pour lui qu'une collection de robes murmurantes couleur prune faites pour décorer les bancs de bois, un arrière-plan visuel à la scène de la Très Ancienne Chambre. Si un ennemi se trouve ici, s'il y a quelque chose à manipuler, c'est simplement “le Magenmagot”. Les riches élites d'Angleterre Magique ont une force collective mais pas de volonté individuelle~; leurs buts sont trop étranges et triviaux pour qu'ils puissent avoir un rôle personnel dans cet histoire. Actuellement, à cet instant, le garçon n'apprécie pas plus qu'il n'a de l'aversion pour les robes couleur prune, car son cerveau ne leur assigne pas assez de libre-arbitre pour qu'ils puissent être les sujets d'un jugement moral. Il est un PJ\footnotemark{}, et ils sont le papier peint.
\authorsnotetext{NdT~: PJ signifie Personnage Joueur.}

Ce point de vue est sur le point de changer.

\later

Harry parcourut sans la voir la chambre du Magenmagot~; le lieu semblait assez ancien et historique et Hermione aurait sans doute pu lui octroyer une leçon sur cet endroit pendant des heures. Les robes couleur prune avaient cessé d'entrer et la montre de poche de Harry qui avançait au rythme de trois minutes par demi-heure indiquait que le procès était sur le point de débuter.

Le professeur McGonagall était assise à côté de lui et ses yeux ne le quittaient jamais pendant plus de vingt secondes consécutives.

Harry avait lu la \emph{Gazette du Sorcier} ce matin-là. Le gros titre avait été~: <<~\inlineheadline{Née-moldue folle tente de mettre fin à ancienne lignée}~>> et le reste du journal avait été du même acabit. Lorsque Harry avait eu neuf ans, l'IRA avait fait sauter une caserne anglaise et il avait vu à la télévision les politiciens concourir à celui qui serait le plus vocal dans l'expression de son outrage. Et l'idée lui était venue -- même en ce temps où il n'avait que peu de connaissances en psychologie -- qu'on aurait dit que \emph{tout le monde} se battait pour savoir qui pouvait être le plus en colère et que \emph{personne} n'aurait pu se permettre de suggérer que quiconque était \emph{trop} en colère, même pour répondre à la proposition de rayer l'Irlande de la carte à coups de bombes nucléaires. Il avait été frappé, même alors, par le vide profond de l'indignation politicienne -- même s'il n'avait pas eu à cet âge les mots pour le décrire -- par le sentiment qu'ils essayaient de marquer facilement des points en visant sans risque la même cible que tous les autres.

Harry avait toujours ressenti le caractère creux de l'indignation politicienne, mais il était étrange de constater à quel point cela devenait bien plus évident en lisant une douzaine d'articles de la \emph{Gazette du Sorcier} matraquer Hermione Granger.

L'article principal, écrit par quelque nom que Harry ne reconnut pas, avait exigé que l'âge minimum pour pouvoir être envoyé à Azkaban soit diminué uniquement pour que la Sang-de-Bourbe tordue qui avait entaché l'honneur de l'Écosse par son attaque sauvage et injustifiée sur le dernier héritier d'une Très Ancienne maison au cœur du refuge sacré de Poudlard puisse être envoyé aux Détraqueurs, seule punition à la hauteur de la sévérité de son crime innommable. Cela seul suffirait à décourager toute autre brute inhumaine et étrangère qui aurait elle aussi la malsaine folie de croire qu'elle pourrait évader à la majesté de l'inévitable et impitoyable châtiment du Magenmagot à l'encontre de tous ceux qui menacent l'honorable nobilité de etc. etc…

L'article suivant avait dit la même chose en des mots moins éloquents.

Plus tôt, Albus Dumbledore lui avait dit~:
\begin{em}
<<~Je n'essaierai pas de te tenir à l'écart de ce procès.~>> La voix du vieux sorcier avait été douce et inflexible. <<~Je peux tout à fait prévoir comment cela se déroulerait. Mais je voudrais qu'en retour, tu me traite avec pareille courtoisie. La politique du Magenmagot est délicate et tu ignores tout d'elle. Ose quelque folie et il en coûtera à Hermione Granger~; et tu te souviendra de cette folie pour le restant de tes jours, Harry James Potter-Evans-Verres.

--- Je comprends, dit Harry. Je sais. Juste -- si vous comptez faire sortir un lapin de votre chapeau et sauver la mise au dernier moment alors que tout semble perdu, s'il vous plaît, dites-le moi maintenant au lieu de me laisser m'inquiéter…

--- Je ne te ferais pas ça~>>, dit le vieux sorcier comme imprégné d'une terrible lassitude alors qu'il se retournerait pour partir. <<~Encore moins à Hermione. Mais je n'ai pas de lapins dans mon chapeau, Harry. Nous pouvons seulement voir ce que Lucius Malfoy désire.~>> \end{em}

Il y eu un petit coup sec, un unique son bref qui parvint pourtant à faire taire toute la chambre et à faire vivement pivoter la tête de Harry vers le haut. Loin au-dessus, Dumbledore venait de frapper son podium du bâton sombre qu'il tenait dans sa main gauche.

<<~La quatre-vingt dixième session du deux-cent-huitième Magenmagot se réunit à la demande de Lord Lucius Malfoy~>>, dit le sorcier d'une voix sans timbre.

Loin du podium mais au milieu des cercles les plus élevés, un grand homme se dressa immédiatement, une longue crinière blanche répandue sur les épaules de sa robe couleur prune.

<<~Je présente un témoin pour interrogatoire sous Veritaserum~>>, dit Lucius Malfoy d'un ton froid et clair qui traversa la pièce, élégamment contrôlé, seulement doté d'une légère nuance de furie indignée. <<~Qu'on amène Hermione, la première Granger.

--- Je vous demande à tous de vous rappeler qu'elle est en première année à Poudlard, dit Dumbledore. Je ne tolérerai aucun abus à l'encontre de ce témoin…~>>

Quelqu'un parmi les bancs eut un “Pouah~!” très audible, il y eut une vague de reniflements dégoûtés et même une ou deux huées.

Harry regarda les robes prunes en plissant les yeux.

Et avec la colère montante vint autre chose, comme un sentiment d'inquiétude grimpant, l'idée que quelque chose était atrocement faussé, comme si la réalité elle-même avait été chamboulée. Harry le savait sans savoir comment, il ne trouvait pas ce qui clochait ni pourquoi son esprit pensait que ça empirait…

<<~\emph{De l'ordre~!}~>> mugit Dumbledore. Il donna un coup sec du bâton de pierre deux fois contre le podium et produisit deux autres petits clics qui l'emportèrent sur tous les autres bruits. <<~J'exige de l'ordre ici~!~>>

La porte par laquelle on amenait le témoin était située exactement sous les pieds de Harry, si bien que ce n'est pas avant que tout le groupe n'ait émergé dans la salle de pierre qu'il put voir…

… un trio d'Aurors…

… le dos de Hermione qu'on amenait, il ne pouvait pas voir son visage…

… suivit d'un moineau d'argent étincelant et d'un écureuil lunaire bondissant…

… et la source de l'horrible inconvenance, à moitié masquée par une cape en lambeaux.

Harry bondit sur ses pieds avant même de pouvoir penser et ce n'est que la poigne frénétique du professeur McGonagall sur son poignet qui l'empêcha d'atteindre sa baguette~; et le professeur de Métamorphose chuchota avec désespoir~: <<~\emph{Harry, tout va bien, il y a un Patronus…}~>>

Il eut besoin de quelques secondes pour se souvenir. Pour que la part de lui-même qui comprenait que Hermione n'avait pas été directement exposée à un Détraqueur parvienne à convaincre le reste de sa personne de se comporter de façon plus ou moins saine d'esprit…

\emph{Mais les Patronus animaux ne sont pas parfaits}, dit une autre voix dans son esprit. \emph{Sans quoi Dumbledore ne verrait pas la forme d'un homme nu douloureux à observer. On le sent s'approcher, Patronus animal ou pas…}

Lentement, Harry se rassit à mesure que le professeur McGonagall tirait son poignet vers le bas.

Mais il avait alors déjà déclaré la guerre contre l'Angleterre magique et l'idée que d'autres le nomment Seigneur des Ténèbres ou non ne lui semblait plus avoir d'importance.

Le visage de Hermione lui apparut lorsqu'elle s'assit dans la chaise. Elle n'était pas droite et rebelle comme elle l'avait été face à Rogue, elle ne pleurait pas comme lorsque les Aurors l'avaient arrêtée. Elle était juste assise là, un air d'horreur vacante dans le regard, alors que de noires chaînes de métal serpentaient de sous la chaise et liaient ses bras et ses jambes.

Harry ne pouvait pas le supporter. Sans même y penser, il tentait de fuir en lui-même, de fuir jusqu'à son côté obscur, de tirer la rage froide par-dessus lui comme un bouclier. Il mit bien longtemps à y parvenir, car il n'avait pas essayé d'atteindre pleinement son côté obscur depuis Azkaban. Et lorsque son sang fut redevenu à peu près froid, il releva les yeux, revit Hermione dans la chaise et découvrit que son côté obscur ignorait tout de la façon de gérer ce genre de douleur. Elle transperça la froideur comme un couteau et la douleur ne fut pas le moins du monde atténuée.

<<~Eh bien, ne serait-ce pas Harry Potter~!~>> lui parvint une voix de femme claire et haute perchée, mielleuse et complaisante jusqu'à l'écœurement.

Harry détourna lentement son visage de la chaise et vit une femme souriante tellement maquillée que sa peau semblait presque rose, assise à côté d'un homme que Harry reconnut être le ministre Cornelius Fudge grâce à des photographies qu'il avait vues.

<<~Aviez-vous quelque chose à dire, M. Potter~?~>> s'enquit la femme d'un ton si joyeux qu'on aurait cru ne pas être à un procès.

D'autres personnes le regardaient à présent.

Harry ne pouvait pas parler car il aurait été stupide de prononcer chacun des mots présents dans son esprit. Il ne trouvait rien à dire que Neville aurait lui aussi pu dire. Dumbledore avait prévenu Harry que si qui que ce soit \emph{d'autre} demandait au Survivant de parler, il devait \emph{faire semblant d'avoir son âge…}

<<~Le directeur a dit que je ne devrais pas parler~>>, répondit le garçon sans entièrement parvenir à garder sa voix dénuée de tranchant.

<<~Oh, mais tu as \emph{notre} permission de parler, dit la femme d'un ton radieux. Je suis certaine que le Magenmagot est toujours heureux d'entendre ce que le Survivant a à dire~!~>> À côté d'elle, le ministre Cornelius Fudge hochait la tête.

Le visage de la femme était boursouflé et gras, visiblement pâle sous le maquillage. Presque inévitablement, un certain mot vint à l'esprit de Harry, et ce mot était \emph{crapaud}. Ce qui, dit la partie logique de Harry, ne devrait être en rien corrélé à la moralité. Ce n'était que dans les films de Disney que les gens laids avaient plus de chances d'être méchant et inversement~; et ces films étaient probablement scénarisés par des auteurs qui n'avaient jamais été laids. Il lui donnerait une chance, tous ceux présents dans cette pièce méritaient une chance…

<<~Parce que je nous ai débarrassé du Seigneur des Ténèbres~?~>> dit le garçon, et il pointa un doigt vers le Détraqueur qui flottait derrière la chaise de Hermione. <<~Il y a quelque chose d'encore plus sombre dans cette pièce.~>>

La femme plissa les yeux et son visage devint un peu plus sévère.

<<~Je comprends qu'ils puissent effrayer un jeune garçon comme vous, M. Potter, mais les Détraqueur obéissent tout à fait au ministère de la Magie. Et ils sont bien sûr nécessaires à la garde…

--- D'une fille de douze ans~? cria le garçon. Ce sont les créatures les plus sombres du monde, j'ai pu la sentir approcher même à travers le Patronus -- le \emph{mal} qui s'approchait -- c'est atrocement maléfique et ça… ça mangerait tout le monde dans cette pièce si ça pouvait~! On ne devrait pas la laisser s'approcher d'un enfant, jamais~! Ni de moi ni d'elle ni de personne~! Vous devriez voter pour le faire partir~!

--- Nous ne voterons \emph{certainement} pas à ce sujet… cracha la femme-crapaud.

--- Cela suffit, Mme Ombrage, M. Potter~>>, dit Dumbledore d'une voix sévère, loin au-dessus d'eux. Puis après une courte pause, le sorcier continua~: <<~Même si bien sûr le garçon a parfaitement raison.~>>

Certains des membres du Magenmagot semblaient abasourdis par l'avertissement du garçon et quelques autres hochaient violemment la tête en réponse aux paroles du vieux sorcier. Mais ils étaient trop peu. Harry pouvait le voir. Ils étaient trop peu.

Le Veritaserum fut alors amené et Hermione sembla pendant un bref instant \emph{être} sur le point de sangloter en regardant Harry -- non, le professeur McGonagall -- et le professeur McGonagall articulait silencieusement des mots que Harry ne pouvait comprendre de là où il se trouvait. Puis Hermione avala trois gouttes de Veritaserum et son visage devint flasque.

<<~Gawain Robards, dit la douce voix de Lucius Malfoy. Votre probité est connue de tous ici. Si vous pouviez me faire l'honneur~?~>>

L'un des trois Aurors s'avança.

Après les premières questions, Harry détourna le regard et observa l'un des murs latéraux les doigts dans ses oreilles tandis que le cerveau de Hermione rejouait le contenu du sortilège de faux souvenirs. Il ne pouvait pas supporter l'angoisse émoussée par la drogue présente dans la voix de Hermione, son côté obscur ne pouvait pas le supporter non plus et il avait déjà entendu un résumé de son récit.

Le cerveau de Harry repartit dans un flash jusqu'à un autre jour d'horreur, et même si Harry avait été à la limite de rejeter la continuité de l'existence de Lord Voldemort comme étant l'idée sénile d'un vieux sorcier, il semblait soudain horriblement et particulièrement plausible que l'entité à avoir Oublietté Hermione soit le même esprit ayant -- \emph{fait usage} -- de Bellatrix Black. Ces deux événements avaient une signature commune. De choisir que cela ait lieu, de \emph{prévoir} que cela ait lieu -- cela demandait d'être plus que malfaisant, cela demandait d'être \emph{vide}.

Harry leva les yeux un moment et vit alors que les robes couleur prune regardaient, qu'elles ne faisaient que regarder.

Quelque temps plus tard, après que toutes les étoiles du ciel nocturne furent devenues froides et sombres, que la dernière lueur de l'univers se soit effondrée en cendres et soit devenue noire, l'interrogatoire de Hermione prit fin.

<<~S'il plaît à mes Lords, dit la voix de Lord Malfoy, je souhaiterais voir lu à voix haute le témoignage de mon fils Drago obtenu sous deux gouttes de Veritaserum.~>>

\emph{Jusqu'à ce qu'elle se mette à me chercher dans la dernière bataille, je ne complotais rien contre Granger. Après ce jour-là je me suis senti vraiment insulté, je l'avais aidée à chaque fois…}

Le son qui sortit de la gorge de Hermione donnait l'impression qu'elle venait de se faire écraser par une pierre si immense qu'elle ne pouvait plus ni pleurer ni respirer mais seulement avoir un petit hoquet triste.

<<~Excusez-moi~>>, dit une sorcière depuis le côté de la pièce qui semblait appartenir au camp Malfoy. <<~Mais, Lord Malfoy, pourquoi votre fils \emph{aiderait}-il une Sang-de-Bourbe~?

--- Mon fils, dit Lucius Malfoy d'une voix grave, semble avoir prêté l'oreille à certaines idées erronées. Il est jeune -- et il a appris, à présent, et tout le pays l'a appris, ce que de telles folies apportent en retour.~>>

Quelques marches plus bas, dans les bancs des visiteurs, un homme coiffé d'une casquette de journaliste et d'un badge de la \emph{Gazette du Sorcier} écrivait avidement avec une longue plume.

Les quelques personnes qui avaient hoché la tête plus tôt en écoutant Dumbledore semblaient maintenant assez malades. Une sorcière en robe couleur prune se leva d'un geste délibéré, quitta le côté qui avait semblé appartenir au camp de Dumbledore et alla jusqu'au camp Malfoy.

L'Auror continua de lire d'une voix monotone.

<<~\emph{Lancer tous ces sortilèges d'Emprisonnement m'avait tellement fatigué que j'étais affaibli quand j'ai lancé le dernier. Je pensais être plus fort que Granger mais je n'en étais pas certain, donc je l'ai testé empiriquement en la provoquant en duel, c'est pour ça que j-j-je l'ai fait et aussi parce que si je gagnais je comptais la battre à nouveau le lendemain pour que tout le monde puisse le voir. Saleté de Veritaserum. Mais} elle \emph{ne savait pas ça quand elle a essayé de me} tuer \emph{! Et ce qu'elle avait fait m'avait vraiment insulté, je l'avais vraiment aidée avant et je n'avais rien comploté contre elle mais} elle \emph{s'en est prise à} moi \emph{devant tout le monde~!}~>>

Lorsque le témoignage eut prit fin, les délibérations du Magenmagot commencèrent.

Si on pouvait appeler ça comme ça.

Il semblait que de nombreux membres du Magenmagot étaient très attachés à l'idée que le meurtre, c'était mal.

Les robes couleur prune du côté Dumbledore de la pièce étaient silencieuses, les soi-disant forces du bien conservant leur capital politique pour des batailles plus gagnables. Et Harry pouvait entendre le professeur Quirrell comme s'il s'était tenu à côté de lui, sa voix sèche parler dans son esprit, lui expliquer qu'ouvrir la bouche maintenant aurait été loin d'avantager ces politiciens.

Mais il y avait un sorcier dans la pièce dont le statut était assez élevé pour pouvoir, semblait-il, dépasser cette notion sans perdre la face~; un sorcier dont le statut était assez élevé pour qu'il puisse prononcer quelques saines paroles et s'en sortir indemne. Lui seul parla pour défendre Hermione, l'homme avec un phénix brillant sur l'épaule.

Seul Albus Dumbledore parla.

Le président sorcier ne mit pas en avant la possibilité que Hermione Granger soit entièrement innocente. Le directeur avait expliqué à Harry qu'on n'y croirait pas et que cela ne ferait qu'aggraver les choses.

Mais Albus Dumbledore dit, un doux rappel après l'autre, que le coupable était une fille en première année à Poudlard~; que nombre de gens avaient commis des idioties pendant leur jeunesse~; qu'une fille en première année à Poudlard était tout simplement trop jeune pour comprendre les conséquences de ses actes. Lui-même (dit doucement le président sorcier) avait fait quelques tentatives idiotes pendant son enfance, alors qu'il était bien plus âgé qu'elle.

Albus Dumbledore dit que Hermione Granger avait été aimée de tous les professeur de Poudlard, qu'elle avait aidé quatre Poufsouffle pour leurs devoirs de charmes et qu'elle avait gagné cent trois points pour Serdaigle pendant l'année scolaire.

Albus Dumbledore dit que personne n'ayant connu Hermione Granger ne manquerait d'être choqué par ces événements. Qu'ils avaient tous autant qu'il étaient entendu l'horreur dans sa voix à mesure qu'elle avait témoigné. Et que si quelque folie inhabituelle l'avait temporairement possédée alors -- sa voix s'éleva et devint un ordre sévère -- elle ne méritait rien d'autre d'eux que de la sympathie et l'attention d'un Guérisseur.

Et enfin, par-dessus des cris de protestation, Albus Dumbledore rappela au Magenmagot que le chef d'accusation était une \emph{tentative de meurtre} et non un meurtre. Il dit, par-dessus un orage d'objections naissantes, que personne n'avait souffert de dommages durables. Et il le supplia de ne pas commettre pire que tout ce que qui avait été fait jusqu'à présent…

<<~\emph{Assez~!}~>> mugit Lucius Malfoy, et un vote à main levée mit fin aux délibérations. L'homme à la crinière blanche se tenait, grand et terrible, sa canne d'argent tenue en l'air comme un marteau sur le point de s'abattre. <<~Pour ce que cette femme folle a tenté de faire à mon fils -- pour la dette de sang qu'elle s'est attirée en essayant de mettre fin à la lignée d'une maison noble et Très Ancienne -- je dis qu'elle devra…

--- Azkaban~! rugit un homme au visage balafré assis à la droite de Lord Malfoy. Envoyez la Sang-de-Bourbe folle à Azkaban~!

--- Azkaban~!~>> s'écria une autre robe couleur prune, puis une autre, puis une autre…

Un clic du bâton de Dumbledore fit taire la pièce. <<~Ceci est déplacé, dit le vieux sorcier avec sévérité. Et votre proposition est barbare, indigne de cette assemblée. Il y a certaines choses que nous ne faisons pas. Lord Malfoy~?~>>

Lucius Malfoy avait écouté d'un visage impassible. <<~Eh bien~>>, dit Lord Malfoy après quelques instants. Un éclat froid luit dans ses yeux. <<~Je n'avais pas prévu de demander cela. Mais si telle est la volonté du Magenmagot… Alors qu'elle paie ce que toute autre paierait à sa place. Que ce soit Azkaban.~>>

Une grande acclamation de rage s'éleva…

<<~Êtes-vous tous \emph{fous}~? s'écria Albus Dumbledore. Elle est trop jeune~! Son esprit ne pourrait le supporter~! En trois siècles, jamais une telle chose n'a été faite en Angleterre~!

--- Que les autres pays penseront-ils de nous~? dit la voix sèche d'une femme que Harry reconnut être la grand-mère de Neville.

--- Garderez-\emph{vous} Azkaban une fois qu'elle y sera, Lord Malfoy~? dit une vieille sorcière sévère que Harry ne connaissait pas. Car je crains que mes Aurors ne déclinent de la garder si de jeunes enfants y sont tenus prisonniers.

--- Ces délibérations ont pris fin, dit froidement Lucius Malfoy. Mais si vous êtes incapables de trouver des Aurors capables d'obéir au vote du Magenmagot, Mme Bones, vous pouvez renoncer à votre poste~; nous pouvons facilement en trouver un autre pour servir à votre place. La volonté de cette chambre est claire. Pour la monstruosité de ses crimes, la fille sera jugée comme une adulte et punie en conséquence~: dix ans à Azkaban, la sentence pour une tentative de meurtre.~>>

Lorsque le vieux sorcier parla de nouveau, sa voix était plus basse. <<~N'y a-t-il pas d'alternative à cela, Lucius~? Nous pouvons nous retirer dans mes appartements pour en discuter, si besoin.~>>

Le grand homme aux longs cheveux blancs se détourna alors pour regarder le vieux sorcier qui se tenait sur son podium, et ils s'observèrent tous les deux pendant un long moment.

Lorsque Lucius Malfoy parla de nouveau, sa voix semblait prise du plus léger des tremblements, comme si son sévère contrôle sur celle-ci commençait à flancher.

<<~Le sang demande un prix, le sang de ma famille. Je ne vendrai la dette de sang due à mon fils à aucun prix. Vous ne pouvez comprendre cela, vous qui n'avez ni amour ni enfant. Cependant plus d'une dette est due à la maison Malfoy et je pense que mon fils, s'il se tenait parmi nous, préférerait voir payé le sang de sa mère plutôt que le sien. Confessez votre crime au Magenmagot comme vous me l'avez confessé et alors je…

--- N'y pensez même pas, Albus~>>, dit la vieille sorcière sévère qui avait parlé plus tôt.

Le vieux sorcier se tenait sur le podium.

Le vieux sorcier se tenait sur le podium et son visage se tordait, se détordait…

<<~Arrêtez, dit la vieille sorcière. Vous connaissez la réponse à donner, Albus. Vous tourmenter n'y changera rien.~>>

Le vieux sorcier parla.

<<~Non, dit Albus Dumbledore.

--- Et vous, Malfoy, continua la vielle sorcière sévère, j'imagine que tout ce que vous désiriez pendant ce temps était de ruiner…

--- Loin de là~>>, dit Lucius Malfoy, et ses lèvres se tordirent alors en un sourire amer. <<~Non, je n'ai ici d'autre dessein que la vengeance de mon fils. Je souhaitais seulement montrer au Magenmagot la vérité derrière le prétendu héroïsme de ce vieil homme et son éloge de cette fille -- qu'il ne songerait jamais à se sacrifier pour la sauver.

--- Une cruauté digne d'un Mangemort, en effet, dit Augusta Londubat. Non que j'implique quoi que ce soit, bien sûr.

--- Cruauté~?~>> dit Lucius Malfoy, son sourire toujours amer. <<~Je ne pense pas. Je connaissais sa réponse à l'avance. Je vous ai toujours mis en garde~: il ne fait que jouer son rôle. Si vous croyez à son hésitation, vous n'en êtes que plus trompée. Souvenez-vous que sa réponse est demeurée la même.~>> L'homme éleva la voix. <<~Votons, mes amis. Je pense qu'un vote à main levée suffira pour cela. Je n'imagine pas qu'il y en aura beaucoup qui choisiront de se rallier à des meurtriers.~>> La voix était devenue froide sur cette dernière note et la promesse était très claire.

<<~Regardez la fille, dit Albus Dumbledore. Regardez-la, regardez l'horreur que vous commettez~! Elle a…~>> La voix du vieux sorcier se brisa. <<~Elle a peur…~>>

Le Veritaserum devait s'être dissipé car le visage de Hermione Granger commençait à se tordre sous son relâchement, ses membres tremblaient visiblement sous les chaînes comme si elle essayait de courir, de fuir cette chaise, mais qu'elle était maintenue par des poids plus lourds que ceux des liens de métal enchantés qui la liaient. Puis il y eut un effort convulsif et le cou de Hermione bougea, sa tête se tordit, assez pour que ses yeux se braquent sur…

Elle regardait Harry et même si elle ne parlait pas, son message était d'une clarté absolue.

\emph{Harry}

\emph{aide-moi}

\emph{s'il te plaît…}

Et une voix s'éleva dans la Très Ancienne Chambre du Magenmagot, des mots de la couleur de l'azote liquide, trop aiguë, car elle venait d'une gorge trop jeune, et cette voix dit~: <<~\emph{Lucius Malfoy}.~>>

\later

Dans l'ancienne chambre sanctifiée du Magenmagot on regardait, on cherchait, et on mit trop longtemps à trouver. Elle avait peut-être été aiguë, elle avait peut-être manquée de force en regard des mots qui avaient été prononcés, et pourtant on ne se serait pas attendu à entendre cette voix émaner d'un enfant.

Ce n'est que lorsque Lord Malfoy répondit que l'on comprit où l'on devait regarder.

<<~Harry Potter~>>, dit Lucius Malfoy. Il n'inclina pas la tête.

Des têtes pivotèrent, des yeux bougèrent et l'on mit au point sur le jeune garçon aux cheveux en bataille qui se tenait à côté d'une vieille sorcière en larmes. Chaussé, le garçon arrivait à peine à la hauteur de sa poitrine. Il était habillé d'une courte robe d'un noir formel. Mais à moins que vos yeux n'aient été particulièrement aiguisés, vous n'auriez pas pu voir de l'autre bout de la Chambre cette célèbre et mortelle cicatrice cachée derrière ses cheveux ébouriffés.

<<~Cette folie ne vous sied guère, Lucius, dit le garçon. Des filles de douze ans ne s'amusent pas à commettre des meurtres. Vous êtes un Serpentard, un Serpentard intelligent. Vous savez que c'est un complot. Hermione Granger a été placée de force sur l'échiquier, par la main qui se cache derrière ce complot, quelle qu'elle soit. \emph{Vous} deviez sûrement agir exactement comme vous le faites maintenant -- sauf que Drago Malfoy devait être mort et vous deviez avoir perdu la raison. Mais il est vivant et vous êtes sain d'esprit. Pourquoi vous conformez-vous au rôle prévu pour vous, dans un complot destiné à prendre la vie de votre fils~?~>>

Un orage semblait se déchaîner en Lucius, le visage sous les cheveux blancs menaçait de se craqueler, de s'ouvrir, de déverser quelque chose que nul n'aurait pu prévoir. Lord Malfoy sembla faillir parler une fois, puis deux fois, puis il ravala trois phrases que nul n'entendit avant que ses lèvres ne s'ouvrent enfin.

<<~Un complot, vous dites~?~>> articula-t-il enfin. Presque entièrement laissé à lui-même, son visage était pris de convulsions. <<~Et qui serait derrière ce complot, alors~?

--- Si je le savais, dit le garçon, je l'aurais dit bien plus tôt. Mais tous ceux qui ont un jour été camarades avec Hermione Granger vous diront que c'est une meurtrière des plus improbables. Elle aide bel et bien les Poufsouffle à faire leurs devoirs. Cet événement n'a pas eu lieu naturellement, Lord Malfoy.

--- Complot… Ou pas…~>> la voix de Lucius tremblait. <<~Cette Sang-de-Bourbe a touché mon fils, et pour cela je l'anéantirai. Vous devriez fort bien le savoir, \emph{Harry Potter}.

--- Il est douteux, dit le garçon, pour ne pas dire plus, que Hermione Granger ait réellement lancé ce sortilège de refroidissement sanguin. Je ne connais pas les circonstances exactes ni les sortilèges mis en jeux, mais une simple tromperie n'aurait pas suffit à la pousser à le faire. Elle n'a pas agit de son propre chef et n'a peut-être pas agit du tout. Votre vengeance est délibérément détournée, Lord Malfoy. Ce n'est pas une fille de douze ans qui mérite votre courroux.

--- Et qu'avez-\emph{vous} à faire de ce qui l'attend~?~>> la voix de Lucius montait. <<~Qu'avez-\emph{vous} à gagner ici~?

--- Elle est mon amie, dit le garçon, tout comme Drago l'est. Il est possible que ce coup ait été dirigé vers moi, et pas du tout vers la maison Malfoy.~>>

Les muscles du visage de Lucius tressautèrent à nouveau.

<<~Et maintenant vous me mentez -- comme vous avez menti à mon fils~!

--- Croyez-le ou non, dit doucement le garçon, je n'ai jamais désiré autre chose pour Drago sinon qu'il connaisse la vérité…

--- \emph{Assez~!} s'écria Lord Malfoy. Assez de vos mensonges~! Assez de vos \emph{jeux}~! Vous ne comprenez pas -- vous ne comprendriez jamais -- ce que cela signifie pour moi, qu'il soit mon fils~! On ne me privera pas de cette vengeance~! Pas encore~! Plus jamais~! Pour le sang qu'elle doit à la maison Malfoy, elle ira à Azkaban. Et si je trouve jamais une autre main à l'œuvre -- même si c'est la vôtre -- cette main sera elle aussi tranchée~!~>> Lucius Malfoy éleva sa cane d'argent comme pour donner un ordre, ses dents serrées, ses lèvres retroussées, comme un loup face à un dragon. <<~Et si vous n'avez rien de mieux à dire que cela… taisez-vous, Harry Potter~!~>>

\later

Le sang de Harry bouillonnait sous la glace de son côté obscur, sous sa peur pour Hermione, sous la partie de lui qui voulait se déchaîner contre Lucius, le détruire sur place pour son insolence et sa \emph{stupidité} -- mais Harry n'en avait pas le \emph{pouvoir}, il n'avait même pas un vote au Magenmagot…

Drago avait dit que pour une raison inconnue, Lucius avait peur de lui. Et Harry pouvait voir dans le rictus tiré et pincé qu'était devenu le visage de Lord Malfoy que ce dernier avait eu besoin de tout son courage pour dire à Harry de se taire.

Et Harry dit donc, d'une voix froide et mortelle, en espérant que cela veuille dire quelque chose~: <<~Vous vous attirerez mon inimité en faisant cela, Lucius…~>>

Quelqu'un parmi les rangs inférieurs de ce qui était clairement le côté Puriste du Magenmagot et qui baissait les yeux vers le jeune garçon plutôt que de les lever vers Lord Malfoy eut un rire de pure incrédulité. D'autres robes couleur prune commencèrent à rire aussi.

Lord Malfoy le contempla avec toute sa dignité pendant que le rire se répandait.

<<~Si vous désirez l'inimité de la maison Malfoy, vous l'aurez, \emph{enfant}.

--- Allons, allons, dit la femme avec trop de maquillage rose, je pense que ceci a duré bien assez longtemps, ne trouvez-vous pas, Lord Malfoy~? Le garçon risque de manquer des cours.

--- En effet~>>, dit Lucius Malfoy, et il éleva de nouveau la voix. <<~J'en appelle au vote~! À main levée, que le Magenmagot reconnaisse la dette de sang due à la noble et Très Ancienne maison Malfoy pour la tentative de meurtre sur son dernier héritier et d'anéantissement de sa lignée par Hermione, la première Granger~!~>>

Les mains s'élevèrent les unes après les autres, le secrétaire assis au cercle inférieur commença à tracer des traits sur un parchemin afin de les compter, mais le camp qu'avait choisi la majorité était évident.

Et Harry poussa un cri dans son esprit, un appel désespéré à toute partie de lui qui pourrait offrir une issue, une stratégie, une idée. Mais il n'y avait rien, rien du tout, il avait joué ses dernières cartes et il avait perdu. Et alors, dans une dernière convulsion désespérée, Harry se plongea dans son côté obscur, s'y poussa de force, étreignit sa clarté mortelle, offrir à son côté obscur tout ce qu'il voudrait si seulement il pouvait l'aider à résoudre ce problème, et le calme létal l'enveloppa enfin, la véritable glace répondit enfin à son appel. Au-delà de toute la panique et de tout le désespoir son esprit commença à passer en revue tous les faits qu'il possédait, à se souvenir de tout ce qu'il savait sur Lucius Malfoy, sur le Magenmagot, sur les lois d'Angleterre magique~; ses yeux parcoururent les rangées de chaises, chaque personne, chaque chose à portée de vue, à la recherche d'une opportunité à saisir…
%  LocalWords:  Dutton Capernaum Pfah
