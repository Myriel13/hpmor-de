\chapter{Pensée latérale}

\lettrine{M}{ercredi,} à l'instant où il posa le pied dans la salle du cours de Défense, Harry sut que \emph{ce} cours allait être très \emph{spécial}.

Mercredi, à l'instant où il posa le pied dans la salle du cours de Défense, Harry sut que \emph{ce} cours allait être très \emph{spécial}.

Pour commencer, c'était la plus grande salle qu'il ait jamais vue à Poudlard, semblable à la salle de cours d'une importante université, avec des étages de pupitres disposés face à une plate-forme immense qui semblait être faite de marbre blanc. La salle était haute dans le château~- au cinquième étage~- et Harry savait que c'était la meilleure explication qu'on lui donnerait jamais quant au fait que cette pièce pouvait tenir entre les murs de Poudlard. Il devenait de plus en plus clair que Poudlard n'avait tout simplement \emph{pas} de géométrie, Euclidienne ou autre~; elle avait des connexions, et non pas des directions.

Contrairement à une salle d'université, il n'y avait pas de rangées de sièges repliables encastrés dans des bureaux~; au lieu de cela, il y avait les très ordinaires pupitres en bois de Poudlard, agencés le long d'une courbe à chaque étage de la salle de cours. Sauf que sur chaque pupitre était installé un objet blanc, plat et rectangulaire. Harry n'avait jamais vu une chose pareille posée sur un pupitre.

Au centre de la gigantesque plate-forme, sur une petite estrade surélevée faite d'un marbre plus sombre, se trouvait un bureau d'enseignant, à l'écart. Sur lequel Quirrell s'appuyait, affalé sur sa chaise, la tête rejetée en arrière, bavant légèrement sur sa robe.

Ça \emph{me rappelle quelque chose, mais quoi~?}

Harry était arrivé en cours si tôt qu'aucun autre étudiant n'était encore présent. (La langue anglaise devenait défectueuse lorsqu'il s'agissait de décrire le voyage dans le temps~; et en particulier, il manquait à l'anglais les mots capables d'exprimer à quel point c'était pratique.) Quirrell ne semblait pas… être en état de marche… pour le moment, et Harry n'avait de toute façon pas particulièrement envie de s'approcher de lui.

Harry choisit un bureau, grimpa jusqu'à celui-ci, s'assit, et récupéra son manuel de Défense. Il en était à peu près aux sept huitièmes~- il avait prévu de finir le livre avant le début du cours, mais il était en retard sur son emploi du temps et il avait déjà utilisé le Retourneur de Temps deux fois aujourd'hui.

Après un court moment, des bruits montèrent et la salle commença à se remplir. Harry les ignora.

<<~Potter~? Qu'est-ce que \emph{tu} fais dans cette classe~?~>>

\emph{Cette} voix n'aurait pas dû être ici. Harry releva les yeux. <<~Drago~? Qu'est-ce que \emph{tu} fais dans ma oh mon dieu tu as des \emph{laquais}.~>>

L'un des garçons qui se tenaient derrière Drago semblait être vraiment très musclé pour un enfant de onze ans, et l'autre maintenait une posture suspicieusement équilibrée.

Drago sourit avec beaucoup de suffisance et fit un geste vers l'arrière. <<~Potter, je te présente M. Crabbe~>>, sa main passa de Muscles à Équilibre, <<~et M. Goyle. Vincent, Gregory, voici Harry Potter.~>>

M. Goyle pencha sa tête et jeta à Harry un regard qui était probablement censé signifier quelque chose mais qui lui donna juste l'air bigleux. M. Crabbe dit <<~Ravi d'fair vot' connaissance,~>> d'un ton qui laissait penser qu'il essayait de forcer sa voix vers des octaves aussi basses que possibles.

Une expression consternée passa de façon fugace sur le visage de Drago, mais fut rapidement remplacée par un air supérieur.

<<~Tu as des \emph{laquais}~! répéta Harry. Où est-ce que \emph{je} peux avoir des laquais~?~>>

L'air supérieur de Drago se renforça.

<<~Potter, j'ai bien peur que pour ça, la première étape ne soit d'être réparti à Serpentard~-

--- Quoi~? Ce n'est pas juste~!

--- - et l'étape deux est que vos familles aient passé un accord peu de temps après votre naissance.~>>

Harry regarda M. Crabbe et M. Goyle. Ils semblaient tous deux essayer très fort d'avoir l'air menaçants. C'est-à-dire qu'ils étaient penchés en avant, les épaules voûtées, le cou tendu, et le regard fixé sur lui.

<<~Euh… attends voir, dit Harry. Ça a été organisé il y a des \emph{années}~?

--- Exactement, Potter. Pas de chance.~>>

M. Goyle fit apparaître un cure-dent et commença à se nettoyer la bouche, l'air toujours menaçant.

<<~Et, dit Harry, Lucius a insisté pour que tu grandisses \emph{sans} jamais connaître tes gardes du corps, et que tu ne devais les rencontrer qu'au premier jour d'école.~>>

Cela fit disparaître l'air supérieur du visage de Drago.

<<~Oui, Potter, on sait tous que tu es brillant, toute l'école est au courant maintenant, tu peux arrêter de frimer~-

--- Donc on leur a répété \emph{toute leur vie} qu'ils allaient être tes laquais et ils ont passé des \emph{années} à s'imaginer à quoi des laquais sont censés ressembler~-~>>

Drago grimaça.

<<~- et le pire, c'est qu'ils \emph{se connaissent} et qu'ils se sont \emph{entraînés}~-

--- Le boss t'a dit d'la fermer~>>, gronda M. Crabbe . M. Goyle mordit son cure-dent, le tint entre ses incisives, et utilisa une main pour faire craquer les jointures de l'autre.

---\emph{Je vous ai dit de ne pas faire ça devant Harry Potter~!}~>>

Ils eurent tous deux l'air un peu penaud et M. Goyle remit le cure-dent dans sa robe avec précipitation.

Mais à l'instant où Drago se détourna d'eux pour faire à nouveau face à Harry, ils reprirent leur air menaçant.

<<~Je m'excuse, dit Drago avec raideur, pour l'insulte que ces \emph{imbéciles} t'ont fait subir.~>>

Harry jeta un regard lourd de sens à M. Crabbe et à M. Goyle. <<~Je dirais que tu es un peu dur avec eux, Drago. \emph{Je} pense que c'est exactement comme ça que je voudrais voir \emph{mes} laquais agir. Enfin, si j'avais des laquais.~>>

La mâchoire de Drago se décrocha.

<<~Eh, Gregory, tu penserais pas qu'y essaie d'nou zappâter loin du boss~?

--- Je suis sûr que M. Potter ne serait pas assez insensé pour essayer ça.

--- Oh, même pas en rêve, dit Harry d'une voix mielleuse. C'est juste un petit quelque chose à garder à l'esprit si votre employeur actuel vous semble ingrat. Et puis, ça ne peut pas faire de mal d'avoir d'autres offres pendant qu'on négocie ses conditions de travail, n'est-ce pas~?

--- Qu'est ce qu'\emph{il} fait à Serdaigle~?

--- Je n'en ai pas la moindre idée, M. Crabbe.

--- \emph{Taisez vous} tous les deux, dit Drago à travers des dents grinçantes. C'est un \emph{ordre}.~>> Il lui fallut faire un effort visible pour de nouveau transférer son attention sur Harry. <<~Quoi qu'il en soit, qu'est-ce que tu fais au cours de Défense de Serpentard~?~>>

Harry se renfrogna. <<~Attends.~>> Sa main passa dans sa bourse. <<~Emploi du temps.~>> Il regarda le parchemin. <<~Classe de Défense, 14h30, et maintenant il est…~>> Harry regarda sa montre, qui affichait 11h23. <<~14h23, à moins que j'aie perdu la notion du temps. J'ai perdu la notion du temps~?~>> Si c'était le cas, eh bien, Harry savait comment se rendre au cours où il était \emph{censé} être, quel que soit ce cours. Dieu savait s'il aimait son Retourneur de Temps, et un jour, quand il serait assez vieux, ils se marieraient.

<<~Non, c'est la bonne heure~>>, dit Drago, les sourcils froncés. Son regard parcourut le reste de l'auditorium, qui se remplissait de robes à bordures vertes et de…

--- \emph{Gryffidiots~!} cracha Drago. Qu'est ce \emph{qu'ils} font ici~?

--- Hmm, dit Harry. Le professeur Quirrell a dit… j'ai oublié ses mots exacts… qu'il allait ignorer certaines des conventions éducatives de Poudlard. Peut-être qu'il a juste combiné toutes ses classes.

--- Euh, dit Drago. Tu es le premier Serdaigle à être arrivé ici.

--- Ouep. Je suis arrivé tôt.

--- Qu'est-ce que tu fais au dernier rang alors~?~>>

Harry cligna des yeux. <<~Je sais pas, ça avait l'air d'être un bon endroit où s'asseoir~?~>>

Drago renifla. <<~Tu ne pourrais pas être plus loin du professeur même si tu essayais.~>> Puis il se pencha en avant, l'air soudain attentif.

<<~À part ça, Potter, c'est vrai ce qu'on raconte sur ce que tu as dit à Derrick et à son équipe~?

--- Qui est Derrick~?

--- Tu l'as frappé avec une tarte~?

--- Deux tartes, à vrai dire. Je suis censé lui avoir dit quoi~?

--- Que ce qu'il faisait n'était pas le moins du monde sournois ou ambitieux, et qu'il était une disgrâce à la mémoire de Salazar Serpentard.~>> Drago regardait Harry avec une grande intensité.

<<~C'était… c'était à peu près ça, dit Harry. Je pense que c'était plus proche de~: 'Cela fait-il partie d'un plan incroyablement malin vous permettant d'obtenir un avantage futur, ou est-ce autant une inutile disgrâce du nom de Salazar Serpentard que ça en a l'air,' ou quelque chose comme ça. Je ne me souviens pas des mots exacts.~>>

Drago secoua la tête.

<<~Tu nous envoies des messages contradictoires, Potter.

--- Hein~?~>> dit Harry, et il était honnêtement confus.

<<~Warrington a dit que rester longtemps sous le Choixpeau Magique est un des signes annonciateurs d'un Seigneur des Ténèbres majeur. Tout le monde en parlait, et se demandait s'il ne faudrait pas commencer à te faire de la lèche, juste au cas où. Puis tu es allé protéger une bottée de Poufsouffle, nom de Merlin~! \emph{Puis} tu as dit à Derrick qu'il était une disgrâce à la mémoire de Salazar Serpentard~! On est \emph{censé} penser quoi~?

--- Que le Choixpeau Magique a décidé de me mettre dans la maison de 'Serpentard~! Je rigole~! Serdaigle~!' et que je me suis comporté en conséquence.~>>

M. Crabbe et M. Goyle ricanèrent tous deux, ce qui poussa M. Goyle à se flanquer une main sur sa bouche.

<<~On ferait mieux de s'asseoir,~>> dit Drago. Il hésita, et sembla devenir un peu plus formel. <<~Potter, sans m'engager à quoi que ce soit, je tiens à te dire que je souhaite continuer notre conversation et que je suis prêt à accepter ta situation actuelle.~>>

Harry hocha la tête.

<<~Ça t'embêterait beaucoup si on attendait jusqu'à samedi après-midi~? Je suis en plein concours pour le moment.

--- Un concours~?

--- Pour voir si je peux lire tous mes manuels aussi vite que Hermione Granger l'a fait.

--- Granger,~>> répondit Drago comme en écho. Ses yeux se rétrécirent. <<~La sang-de-Bourbe qui croit qu'elle est Merlin~? Si tu essaies de \emph{lui} en remontrer, alors tout Serpentard te souhaite bonne chance, Potter, et je ne t'embêterai pas jusqu'à samedi.~>> Drago inclina la tête dans un geste de respect mesuré et s'en fut, suivi par ses laquais.

\emph{Oh, ça va être vraiment amusant de jongler entre les deux, je peux déjà le voir venir.}

La salle se remplissait maintenant rapidement avec les quatre couleurs d'ourlets~: vert, rouge, jaune et bleu. Drago et ses deux amis semblaient être au beau milieu de l'acquisition de trois sièges de premier rang~- déjà occupés bien sûr. M. Crabbe et M. Goyle menaçaient vigoureusement, mais ça ne semblait pas faire beaucoup d'effet.

Harry se pencha sur son manuel de Défense et continua à lire.

\later

À 14h35, tous les sièges étaient pris et personne d'autre ne semblait vouloir rentrer dans la salle. Le professeur Quirrell eut une soudaine convulsion, puis il se tint le dos droit, assis sur sa chaise, et son visage apparut sur tous les objets blancs, plats et rectangulaires qui avaient été installés sur les pupitres des élèves.

Harry fut pris par surprise, autant par l'apparition soudaine du visage du professeur Quirrell que par la ressemblance de l'objet à la télévision moldue. Il y avait là quelque chose de triste et de nostalgique, ça ressemblait tant à un élément de son foyer sans pour autant l'être réellement…

<<~Bonjour, mes jeunes apprentis~>>, dit le professeur Quirrell. Sa voix semblait venir de l'écran du pupitre et s'adresser directement à Harry. <<~Bienvenue dans votre premier cours de Magie de Combat, comme les fondateurs de Poudlard l'appelaient~; ou, comme certains le nomment en cette fin de vingtième siècle, Défense contre les forces du Mal.~>>

Il y eut une montée de griffonnages frénétiques alors que les étudiants, pris par surprise, se saisissaient de leur parchemin ou de leur carnet de notes.

<<~Non, dit le professeur Quirrell, franchement, ne vous embêtez pas à prendre note de la façon dont on appelait autrefois ce cours. Aucune question inutile n'apparaîtra sur aucun de mes contrôles. C'est une promesse.~>>

Plusieurs étudiants se redressèrent dans leur siège en entendant ça, l'air plutôt effarés.

Le professeur Quirrell eut un léger sourire. <<~Ceux d'entre vous qui ont perdu leur temps en lisant à l'avance votre inutile livre de Défense de première année~-~>>

Quelqu'un sembla s'étrangler. Harry se demanda si c'était Hermione.

<<~- ont peut-être l'impression que, bien que ce cours soit nommé Défense contre les forces du Mal, il concerne en fait la défense contre les Papillons de Cauchemar, qui provoquent des rêves vaguement mauvais, ou les Limaces acides, qui peuvent dissoudre toute l'épaisseur d'une poutre de bois de cinq centimètres si on leur laisse la journée.~>>

Le professeur Quirrell se leva, repoussant sa chaise loin du bureau. L'écran sur le pupitre de Harry suivait chacun de ses mouvements. Le professeur Quirrell s'élança jusqu'à l'avant de la salle et rugit~:

<<~Le Sirex Hongrois est plus grand que douze hommes~! Il exhale du feu si vite et si précisément qu'il peut faire fondre un Vif en plein vol~! Un sortilège de la Mort l'abattra~!~>>

Il y eut des exclamations du côté des élèves.

<<~Le Troll des montagnes est plus dangereux que le Sirex Hongrois~! Il est assez fort pour traverser de l'acier avec ses dents~! Sa peau est si résistante qu'elle dévie les sorts de Découpe~! Son odorat est si aigu qu'il peut dire de loin si sa proie fait partie d'un groupe ou si elle est seule et vulnérable~! Bien plus effroyable que tout cela~: le Troll est la seule des créatures magiques qui maintient en permanence une sorte de métamorphose de lui-même~- il se transforme en permanence en son propre corps. Si vous parveniez enfin à lui arracher un bras, un autre lui pousserait en quelques secondes~! Le feu et l'acide produiront du tissu cicatriciel qui déboussolera \emph{temporairement} les pouvoirs de régénération d'un Troll~- pour une heure ou deux~! Ils sont assez intelligents pour utiliser des outils tels que des bâtons~! Le Troll des montagnes est troisième de la liste des machines à tuer les plus parfaites de la Nature~! Un sortilège de la Mort l'abattra.~>>

Les élèves avaient l'air plutôt effarés.

Le professeur Quirrell souriait de façon plutôt sinistre.

<<~Ce qu'on ose appeler un manuel de Défense de troisième année vous suggérera d'exposer le Troll à la lumière du soleil, ce qui le gèlera sur place. Ceci, mes jeunes apprentis, est le genre de savoir inutile que vous ne trouverez à aucun de mes examens. On ne rencontre jamais de Troll en plein jour et à découvert~! La suggestion selon laquelle vous devriez utiliser la lumière du soleil pour les arrêter est le fruit d'ineptes auteurs de manuels essayant de démontrer leur maîtrise de menus détails, et ce au détriment du sens pratique. Ce n'est pas parce qu'il existe un moyen ridiculement obscur de se débarrasser des Trolls des montagnes que vous devriez l'utiliser~! Le sortilège de la Mort est imparable, inarrêtable, et fonctionne à chaque fois, sur toute chose possédant un cerveau. Si, une fois devenu un sorcier adulte, vous ne parvenez pas à utiliser le Sort de la Mort, alors vous pouvez simplement transplaner~! De même, si vous faites face à la deuxième des machines à tuer les plus parfaites, le Détraqueur. Vous vous contentez de transplaner~!

--- À moins bien sûr~>>, dit le professeur Quirrell, sa voix plus basse et plus dure, <<~que vous ne soyez sous l'influence d'une malédiction anti-transplanage. Non, il y a exactement un seul monstre qui sera capable de vous menacer une fois que vous aurez fini votre croissance. Le monstre le plus dangereux au monde, si dangereux que rien ne lui arrive à la cheville. Le sorcier adulte. C'est la seule chose qui pourra encore vous menacer.~>>

Les lèvres du professeur Quirrell formaient une ligne très fine. <<~C'est à contrecœur que je vous enseignerai assez de broutilles pour que vous ayez une note passable à la portion de vos examens de fin d'année mandatée par le Ministère. Puisque votre note exacte n'aura aucune incidence sur votre vie future, toute personne désirant une note meilleure que passable est invitée à perdre son temps en étudiant ce qu'on ose appeler un manuel. Le nom de ce cours n'est pas Défense Contre les nuisibles mineurs. Vous êtes ici pour apprendre comment vous défendre contre les forces du Mal. Ce qui signifie, soyons très clair à ce sujet, vous défendre contre les Mages Noirs. Des gens armés de baguettes désirant vous tuer et qui y parviendront probablement, à moins que vous ne les blessiez en premier~! Il n'y a pas de défense sans attaque~! Il n'y a pas de défense sans combat~! Cette réalité est jugée trop dure pour des enfants de onze ans par les politiciens gras, surpayés et gardés par des Aurors qui ont décidé de votre curriculum. Puissent ces idiots tomber dans une abysse~! Vous êtes ici pour le cours qui a été enseigné à Poudlard pendant huit cents ans~! Bienvenue dans votre premier cours de Magie de Combat~!~>>

Harry commença à applaudir. Il ne pouvait pas s'en empêcher, c'était trop exaltant.

Une fois que Harry eut commencé à applaudir, il y eut des reprises éparses venant de Gryffondor, et d'autres, plus nombreuses, venant de Serpentard, mais la plupart des élèves étaient tout simplement trop étourdis pour réagir.

Le professeur Quirrell fit un geste cassant, et les applaudissements moururent instantanément. <<~Merci beaucoup, dit le professeur Quirrell. Maintenant, passons aux questions pratiques. J'ai combiné tous mes cours en un seul, ce qui me permet de vous offrir le double de la durée d'une double session normale~-~>>

Il y eut des hoquets d'horreur.

<<~- une durée de classe augmentée pour laquelle je compenserai en ne vous donnant aucun devoir.~>>

Les hoquets d'horreur s'arrêtèrent brusquement.

<<~Oui, vous m'avez bien entendu. Je vous enseignerai comment vous battre, pas comment écrire deux rouleaux de parchemin sur la notion de combat pour lundi.~>>

Harry souhaitait désespérément être assis à côté de Hermione pour pouvoir voir l'expression qu'elle avait maintenant, mais en même temps il était assez certain d'en imaginer déjà une reproduction fidèle.

Et Harry était amoureux. Ce serait un mariage à trois~: lui, le Retourneur de Temps, et le professeur Quirrell.

<<~Pour ceux d'entre vous qui \emph{veulent} passer plus de temps à étudier la Magie de Combat, j'ai mis en place quelques activités du soir que vous trouverez, je pense, assez intéressantes en plus d'être pédagogiques. Souhaitez-vous montrer au monde vos \emph{propres} capacités au lieu de regarder quatorze personnes jouer au Quidditch~? Il peut y avoir plus de sept personnes dans une armée.~>>

Truc de \emph{fou}.

<<~Ces activités ainsi que d'autres vous permettront aussi d'obtenir des points Quirrell. Que sont des points Quirrell, vous dites~? Le système de points de Poudlard ne correspond pas à mes besoins, car il les rend trop rares. J'aime faire fréquemment savoir à mes élèves où ils en sont. Et dans les rares occasions où je vous proposerai un contrôle écrit, il se notera lui-même au fur et à mesure, et si vous vous trompez sur trop de questions de la même catégorie, votre contrôle vous montrera le nom des élèves qui auront correctement répondu à ces questions, et ces élèves pourront gagner des points en vous aidant.~>>

Waouh. Pourquoi les autres professeurs n'utilisaient-ils pas un système comme celui-ci~?

<<~À quoi servent les points Quirrell, vous demandez-vous~? Pour commencer, dix points Quirrell valent un point normal. Mais ils vous permettront aussi d'obtenir d'autres faveurs. Voudriez-vos avoir votre contrôle un jour en particulier~? Y a-t-il un cours dont vous aimeriez vraiment pouvoir être absent~? Vous découvrirez que je peux être très accommodant envers les étudiants ayant accumulé assez de points Quirrell. Les points Quirrell décideront des futurs généraux des armées. Et pour Noël~- juste avant les vacances de Noël~- j'accorderai un vœu à quelqu'un. Toute prouesse en rapport avec l'école et accessible à mon pouvoir, mon influence, et par-dessus tout, mon ingéniosité. Oui, j'étais à Serpentard, et je vous offre de mettre en place un fourbe complot à votre bénéfice, si c'est ce que la réalisation de votre souhait nécessite. Ce vœu sera accordé à celui ou celle qui, entre les élèves de chaque année, aura obtenu le plus de points Quirrell.~>>

Harry, donc.

<<~Laissez maintenant vos manuels et objets divers à vos pupitres~- ils seront en sécurité, les écrans les surveilleront pour vous~- et descendez sur cette plate-forme. Nous allons jouer à un jeu appelé \emph{Qui est l'élève le plus dangereux de la classe}.~>>

\later

Harry fit un mouvement du poignet droit et dit <<~\emph{Ma-ha-su~!}~>>

Il y eut un autre <<~bing~>> aigu venant de la sphère bleue flottante fournie comme cible par Quirrell à Harry. Ce son signifiait un coup parfait, ce que Harry avait accompli neuf de ses dix précédentes tentatives.

Le professeur Quirrell avait déniché quelque part un sort qui était incroyablement facile à prononcer, \emph{et} avait un mouvement de baguette incroyablement simple, \emph{et} avait tendance à toucher l'endroit vers lequel vos yeux étaient dirigés. Le professeur Quirrell avait proclamé avec dédain que la vraie magie de combat était bien plus difficile que cela. Que le sort était totalement inutile en vrai combat. Que c'était un éclat de magie à peine contrôlé, dont le seul vrai contenu était sa précision, et qui produirait, lorsqu'il frapperait, une douleur brièvement équivalente à celle ressentie après un grand coup porté au nez. Que le seul but de ce test était de voir qui apprenait vite, puisque le professeur Quirrell était certain que personne n'aurait jamais rencontré ce sort ni quoi que ce soit d'approchant.

Harry se fichait complètement de tout ça.

<<~\emph{Ma-ha-su~!}~>>

Un \emph{rayon d'énergie rouge} jaillit de sa baguette et frappa la cible, et la sphère bleue fit à nouveau le bing qui voulait dire que le sort \emph{avait vraiment fonctionné}.

Pour la première fois depuis son arrivée à Poudlard, Harry se sentait être un vrai sorcier. Il aurait aimé que la cible esquive, comme les petites sphères que Ben Kenobi avait utilisées pour entraîner Luke, mais pour une raison inconnue, le professeur Quirrell avait préféré aligner tous les élèves et les cibles de façon bien ordonnée et s'assurer qu'ils ne se tiraient pas les uns sur les autres.

Alors Harry abaissa sa baguette, fit un bond sur la droite, la redressa, fit un mouvement du poignet et s'écria~: <<~\emph{Ma-ha-su}~!~>>

Il y eut un <<~dong~>> plus grave, ce qui voulait dire qu'il avait presque réussi.

Harry mit sa main dans sa poche, fit un bond pour revenir à gauche, sortit sa baguette et projeta un autre rayon d'énergie rouge.

Le bing aigu qui en résulta était de loin l'un des sons les plus satisfaisants qu'il ait entendu de sa vie. Harry voulait crier de triomphe à s'en faire éclater les poumons. \emph{JE PEUX FAIRE DE LA MAGIE~! CRAIGNEZ-MOI, LOIS DE LA PHYSIQUE, CAR JE VIENS VOUS ENFREINDRE~!}

<<~\emph{Ma-ha-su}~!~>> la voix de Harry était maintenant forte, mais à peine discernable dans le brouhaha de cris similaires venus de la classe/plate-forme.

<<~Assez,~>> dit la voix amplifiée du professeur Quirrell. (Le son n'était pas fort, il semblait avoir un volume normal et venir de juste derrière votre épaule gauche où que vous vous teniez par rapport au professeur Quirrell.) <<~Je vois que toute le monde a réussi au moins une fois.~>> Les sphères-cibles devinrent rouges et commencèrent à dériver vers le plafond.

Le professeur Quirrell se tenait sur l'estrade au centre de la plate-forme, légèrement penché sur le bureau, appuyé sur une main.

<<~Je vous ai dit, dit le professeur Quirrell, que nous allions jouer à un jeu nommé \emph{Qui est l'élève le plus dangereux de la classe}. Il y a un élève dans cette classe qui a maîtrisé le Sort d'Attaque Simple Sumérien plus vite que quiconque~-~>>

Oh blah blah blah.

<<~- et a ensuite aidé sept autres étudiants. Ce pour quoi elle a gagné les sept premiers points Quirrell de votre année. Hermione Granger, merci de vous avancer. Le moment est venu de passer à l'étape suivante du jeu.~>>

Hermione Granger marcha à grands pas, avec sur son visage un air de triomphe et d'appréhension mélangés. Les Serdaigle la regardaient fièrement, les Serpentard avec mépris, et Harry avec un franc agacement. Harry avait bien réussi cette fois. Il était même probablement dans la moitié supérieure du cours, maintenant que tout le monde avait fait face à un sort uniformément peu familier et que Harry avait lu l'intégralité de \emph{Théorie Magique} de Adalbert Lasornette. Et pourtant \emph{Hermione était encore meilleure}.

Quelque part, au fond de ses pensées, se cachait la peur que Hermione soit tout simplement plus intelligente que lui.

Mais pour le moment, Harry allait ancrer ses espoirs sur les deux faits suivants~: (a) Hermione avait lu bien plus que les manuels standards, et (b) Adalbert Lasornette était un couillon peu inspiré qui avait écrit \emph{Théorie Magique} juste pour plaire à une commission scolaire qui n'avait pas une très bonne opinion des enfants de onze ans.

Hermione parvint à l'estrade centrale et grimpa sur la marche.

<<~Hermione Granger a maîtrisé un sort totalement inconnu en deux minutes, presque une minute plus vite que le deuxième plus rapide.~>> Le professeur Quirrell pivota lentement pour regarder tous les élèves qui les observaient. <<~L'intelligence de Mademoiselle Granger pourrait-elle faire d'elle l'élève la plus dangereuse de cette salle~? Eh bien~? Qu'en pensez-vous~?~>>

Pour le moment, personne n'avait l'air de penser quoi que ce soit. Même Harry ne savait pas quoi dire.

<<~Et si on le découvrait ensemble~?~>> dit le professeur Quirrell. Il se retourna vers Hermione, et fit un geste en direction de la salle. <<~Choisissez n'importe quel étudiant et jetez-lui le Sort d'Attaque Simple.~>>

Hermione se pétrifia.

<<~Allons, dit doucement le professeur Quirrell. Vous avez jeté ce sort à la perfection plus de cinquante fois. Il ne fait aucun dommage permanent et n'est pas si douloureux que ça. Il fait à peu près aussi mal qu'un bon coup de poing dans le nez et ne dure que quelques secondes.~>> La voix du professeur Quirrell devint plus dure. <<~C'est un ordre direct de votre professeur, Mademoiselle Granger. Choisissez une cible et jetez un Sort d'Attaque Simple.~>>

Le visage de Hermione était tordu d'horreur et sa baguette tremblait. Les doigts de Harry se serraient dans son poing, par pure empathie. Même s'il comprenait ce que le professeur Quirrell essayait de faire. Même s'il voyait bien ce que le professeur Quirrell essayait de démontrer.

<<~Si vous ne levez \emph{pas} votre baguette et ne tirez \emph{pas}, Mademoiselle Granger, vous perdrez un point Quirrell.~>>

Harry regarda Hermione, espérant qu'elle regarde dans sa direction. Sa main droite tapotait doucement sur sa poitrine. \emph{Choisis-moi, je n'ai pas peur…}

La baguette de Hermione pivota dans sa main, puis son visage se détendit, et elle abaissa sa baguette contre son flanc.

<<~Non,~>> dit Hermione Granger.

Sa voix était calme, et même si elle n'était pas forte, le silence était tel que tout le monde l'entendit.

<<~Alors je dois vous ôter un point, dit le professeur Quirrell. C'était un test, et vous l'avez échoué.~>>

Hermione fut touchée par ces paroles. Harry le voyait bien. Mais elle garda ses épaules droites.

La voix du professeur Quirrell était compatissante et semblait emplir la salle entière. <<~Savoir des choses ne suffit pas toujours, Mademoiselle Granger. Si vous ne pouvez donner et recevoir des coups aussi intenses qu'un choc contre le petit doigt de pied, alors vous ne pouvez pas vous défendre et vous ne réussirez pas mon cours de Défense. Rejoignez vos camarades, s'il-vous-plaît.~>>

Hermione marcha jusqu'au groupe de Serdaigle. Elle semblait être en paix avec elle-même, et, pour une étrange raison, Harry aurait bien voulu applaudir. Même si le professeur Quirrell avait eu \emph{raison}.

<<~Donc, dit le professeur Quirrell. Il devient clair que Hermione Granger n'est pas l'élève la plus dangereuse de la salle. Qui ici pense être la personne la plus dangereuse~?~- à part moi, bien sûr.~>>

Sans même y penser, les yeux de Harry se tournèrent vers le contingent de Serpentard.

<<~Drago, de la Noble et Ancienne Maison Malfoy, dit le professeur Quirrell. Il semble que nombreux sont ceux qui regardent dans votre direction. Merci de vous avancer.~>>

Drago s'exécuta, et son port comportait un certain orgueil. Il alla jusqu'à l'estrade et regarda le professeur Quirrell en souriant.

<<~M. Malfoy, dit le professeur Quirrell. Feu.~>>

Harry aurait essayé de l'en empêcher s'il en avait eu le temps, mais d'un mouvement gracieux Drago avait tournoyé vers les Serdaigle, avait levé sa baguette, avait dit <<~\emph{Mahasu~!}~>> comme si c'était un mot d'une seule syllabe, et Hermione avait dit <<~Ouh~!~>> et c'était fini.

<<~Bien envoyé, dit le professeur Quirrell. Deux points Quirrell pour vous. Mais dites-moi, pourquoi avez-vous visé Mademoiselle Granger~?~>>

Il y eut une pause.

Drago dit enfin~: <<~Parce que c'était celle qui ressortait le plus.~>>

Les lèvres du professeur Quirrell formèrent un fin sourire. <<~Et voilà la véritable raison pour laquelle Drago Malfoy est dangereux. S'il avait choisi n'importe qui d'autre, cette personne lui en aurait probablement voulu d'avoir été choisie, et M. Malfoy se serait probablement fait un ennemi. Et bien que M. Malfoy aurait pu donner une autre justification expliquant pourquoi il l'a choisie elle, cela n'aurait servi à rien d'autre qu'à énerver certains d'entre vous, alors que d'autres l'acclament déjà qu'il dise quelque chose ou pas. En bref, M. Malfoy est dangereux parce qu'il sait qui frapper et qui ne pas frapper, comment se faire des alliés, et comment éviter de se faire des ennemis. Deux points Quirrell de plus pour vous, M. Malfoy. Et comme vous venez de démontrer une vertu Serpentard, je pense que la Maison de Salazar a elle aussi gagné un point. Vous pouvez rejoindre vos amis.~>>

Drago s'inclina légèrement et retourna au contingent de Serpentard. Quelques applaudissements s'élevèrent des robes à bordures vertes, mais le professeur Quirrell fit un geste cassant et le silence retomba.

<<~Il semblerait que notre jeu soit fini, dit le professeur Quirrell. Et pourtant, il reste un étudiant dans cette salle qui est plus dangereux que le descendant de Malfoy.~>>

Et \emph{maintenant}, pour une étrange raison, il semblait y avoir vraiment beaucoup de gens regardant en direction de…

<<~Harry Potter. Veuillez vous avancer.~>>

Ça ne présageait rien de bon.

Avec réticence, Harry marcha jusqu'à l'endroit où le professeur Quirrell se trouvait, sur son estrade surélevée, toujours appuyé contre son bureau.

La nervosité d'être mis sous les projecteurs semblait acérer la sagacité de Harry à mesure qu'il s'approchait de l'estrade, et son cerveau parcourait les possibilités, essayait de deviner ce qui, selon le professeur Quirrell, pourrait démontrer la dangerosité de Harry. Lui demanderait-il de jeter un sort~? De vaincre un Seigneur des Ténèbres~?

De démontrer son immunité au sortilège de la Mort~? Le professeur Quirrell était certainement trop intelligent pour \emph{ça}…

Harry s'arrêta bien avant l'estrade, et le professeur Quirrell ne lui demanda pas de s'approcher plus.

<<~Ce qui est ironique, dit le professeur Quirrell, c'est que vous avez tous regardé la bonne personne, mais pour les mauvaises raisons. Vous vous dites~>>, les lèvres du professeur Quirrell se tordirent, <<~que Harry Potter a vaincu le Seigneur des Ténèbres, et qu'il doit donc être très dangereux. Bah. Il avait un an. Quel que soit le caprice du destin qui a tué le Seigneur des Ténèbres, cela avait bien peu à voir avec les capacités de combattant de M. Potter. Mais après avoir entendu des rumeurs parlant d'un Serdaigle faisant face à cinq Serpentard plus âgés, j'ai interrogé plusieurs témoins oculaires et en suis arrivé à la conclusion que Harry Potter serait le plus dangereux de mes étudiants.~>>

Un choc d'adrénaline se déversa dans le système sanguin de Harry. Il ne savait pas à quelle conclusion exacte le professeur Quirrell était parvenu, mais ça ne pouvait pas être bon.

<<~Ah, professeur Quirrell~-~>> commença à dire Harry.

Le professeur Quirrell semblait amusé. <<~Vous pensez que je suis parvenu à une conclusion erronée, n'est-ce pas, M. Potter~? Vous apprendrez à attendre mieux venant de \emph{moi}.~>> Le professeur Quirrell se redressa là où il avait été penché. <<~M. Potter, toutes les choses ont des usages courants. Donnez-moi dix usages inhabituels d'objets de cette pièce pouvant être faits en combat~!~>>

Pendant un moment, Harry eut le souffle coupé par le choc pur et brut d'avoir été compris.

Et les idées commencèrent à se déverser.

<<~Il y a des pupitres assez lourds pour être mortels si jetés d'une hauteur suffisante. Il y a des chaises avec des jambes en métal qui pourraient empaler quelqu'un si on les poussait assez fort. L'air dans la salle serait mortel par son absence, puisque les gens meurent dans le vide, et il peut aussi servir de porteur de gaz empoisonnés.~>>

Harry s'arrêta pour reprendre son souffle, et le professeur Quirrell dit au milieu de la pause~:

<<~En voilà trois. Il vous en faut dix. Le reste de la classe croit que vous avez épuisé tout le contenu de cette salle.

--- \emph{Ha~!} Le sol peut être enlevé pour créer une fosse de piques où tomber, le plafond peut être écroulé sur quelqu'un, les murs peuvent servir de matériel de base de métamorphose pour toutes sortes de choses mortelles~- des couteaux, par exemple.

--- En voilà six. Mais vous grattez sûrement le fond à présent~?

--- Je n'ai même pas commencé~! Regardez les gens~! Envoyer un Gryffondor attaquer l'ennemi est une utilisation \emph{ordinaire}, bien sûr~-

--- Je ne vous aurais pas laissé compter celle-là.

--- - mais leur sang peut aussi être utilisé pour noyer quelqu'un. Les Serdaigle sont connus pour leur cerveau, mais leurs organes internes pourraient être revendus au marché noir pour engager un assassin. Les Serpentard ne sont pas seulement utiles en tant qu'assassins, on peut aussi les projeter avec une vélocité suffisante pour écraser le corps d'un ennemi. Et les Poufsouffle, en plus de travailler dur, contiennent aussi certains os qui peuvent être enlevés, aiguisés, et utilisés pour poignarder quelqu'un.~>>

Le reste de la salle regardait maintenant Harry avec horreur. Même les Serpentard avaient l'air choqués.

<<~En voilà dix, même si je suis généreux en comptant celui de Serdaigle. Maintenant, en bonus, un point supplémentaire pour chaque utilisation d'objet de cette pièce que vous n'avez pas encore nommé.~>> Le professeur Quirrell gratifia Harry d'un sourire sympathique.

<<~Le reste de la classe pense que vous êtes à présent en difficulté, parce que vous avez tout nommé, mis à part les cibles, et que vous ne savez pas quoi faire d'elles.

--- Bah~! J'ai nommé les gens, mais pas ma robe, qui peut être utilisée pour faire suffoquer un ennemi une fois enroulée plusieurs fois autour de sa tête, ou la robe de Hermione Granger, qui peut être découpée en bandelettes attachées les unes aux autres pour pendre quelqu'un, ou la robe de Drago Malfoy, qui peut être utilisée pour allumer un feu~-

--- Trois points, dit le professeur Quirrell, plus de vêtements à présent.

--- Ma baguette peut être poussée dans le cerveau d'un ennemi à travers son globe oculaire~>> et quelqu'un fit un bruit d'étranglement horrifié.

<<~Quatre points, plus de baguettes.

--- Ma montre pourrait étouffer quelqu'un si je la lui fourrais dans la gorge~-

--- Cinq points, c'est assez.

--- Hpmf, dit Harry. Dix points Quirrell pour un point de Maison, c'est ça~? Vous auriez dû me laisser continuer jusqu'à ce que je gagne la coupe des Quatre Maisons, je n'ai même pas commencé à parler des utilisations inhabituelles de ce que j'ai dans mes poches~>> ni de la bourse en peau de Moke elle-même, et il ne pouvait pas parler du Retourneur de Temps ou de la Cape d'Invisibilité mais il devait y avoir \emph{quelque chose} à dire au sujet de ces sphères rouges…

<<~\emph{Assez}, M. Potter. Eh bien, pensez-vous tous bien comprendre ce qui fait de M. Potter l'élève le plus dangereux de la classe~?~>>

Il y eut un bas murmure d'assentiment.

<<~Dites-le haut et fort, s'il vous plaît. Terry Boot, qu'est-ce qui rend votre compagnon de dortoir si dangereux~?

--- Ah… euh… il est créatif~?

--- \emph{Faux}~!~>> tonna le professeur Quirrell, et son poing s'abattit avec force sur le bureau dans un bruit amplifié qui fit bondir tout le monde. <<~Toutes les idées de M. Potter étaient pires qu'inutiles~!~>>

Harry sursauta de surprise.

<<~Enlever le sol pour créer un piège à piques~? Ridicule~! En combat, vous n'avez pas le temps de préparation nécessaire à cela, et si vous l'aviez, il y aurait des façons cent fois meilleures de l'utiliser~! Métamorphoser des objets à partir des murs~? M. Potter ne sait pas effectuer une métamorphose~! M. Potter a eu exactement une idée qu'il pourrait utiliser, maintenant, sans une vaste préparation ou la coopération de l'ennemi ou une magie qu'il ne connaît pas. L'idée était de fourrer sa baguette dans l'œil de son ennemi. Ce qui briserait probablement sa baguette plutôt que de tuer son opposant~! En bref, M. Potter, j'ai bien peur que vos suggestions n'aient été uniformément nulles.

--- Quoi~? dit Harry avec indignation. Vous m'avez \emph{demandé} des idées inhabituelles, pas des idées pratiques~! Je sortais des sentiers battus~! Comment utiliseriez-\emph{vous} quelque chose dans cette salle dans le but de tuer quelqu'un~?~>>

L'expression du professeur Quirrell était désapprobatrice, mais il y avait des plis rieurs autour de ses yeux. <<~M. Potter, je n'ai jamais dit que vous deviez \emph{tuer}. Il y a un temps et un lieu pour prendre un ennemi vivant, et l'intérieur de Poudlard est généralement l'un de ces endroits. Mais pour répondre à votre question, le frapper à l'arrière du cou avec le tranchant d'une chaise.~>>

Il y eut quelques rires venant de Serpentard, mais ils riaient avec Harry, pas de lui.

Mis à part eux, tout le monde avait l'air assez horrifié.

<<~Mais M. Potter a maintenant démontré pourquoi il est l'étudiant le plus dangereux de la classe. Je lui ai demandé des utilisations inhabituelles d'objets en cas de combat. M. Potter aurait pu suggérer l'utilisation d'un bureau pour bloquer un maléfice, ou celle d'une chaise pour faire trébucher un ennemi approchant, ou d'enrouler du tissu autour de son bras pour créer un bouclier improvisé. Au lieu de ça, chaque utilisation mentionnée par M. Potter était offensive plutôt que défensive, et soit fatale, soit potentiellement fatale.~>>

Quoi~? Attendez, ça ne pouvait pas être vrai… Harry eut une sensation de vertige soudaine alors qu'il essayait de se souvenir de ce qu'il avait suggéré, il y avait sûrement un contre-exemple…

<<~Et voilà, dit le professeur Quirrell, pourquoi les idées de M. Potter étaient si étranges et si inutiles~- parce qu'il devait aller loin dans l'incommode afin d'atteindre son but~: \emph{tuer l'ennemi}. Pour lui, toute idée ne menant pas à cela ne méritait pas d'être prise en considération. Ceci reflète une qualité que nous pourrions nommer \emph{intention de tuer}. Je l'ai. Harry Potter l'a, et c'est pour cela qu'il a pu regarder cinq Serpentard plus âgés droit dans les yeux. Drago Malfoy ne l'a pas, pas encore. M. Malfoy n'éviterait certainement pas de discuter de meurtre ordinaire, mais même lui a été choqué~- oui, vous l'étiez M. Malfoy, je regardais votre visage~- lorsque M. Potter a décrit comment utiliser les corps de ses camarades comme de la matière première. Il y a des censeurs dans votre esprit qui vous font reculer face à de telles pensées. M. Potter pense \emph{uniquement} à la façon de tuer l'ennemi, il utilisera tout moyen disponible, il ne reculera pas, ses censeurs sont éteints. Bien que son jeune génie soit si indiscipliné et si incommode qu'il en devienne inutile, son \emph{intention de tuer} fait de Harry Potter \emph{le plus dangereux élève de la classe}. Un dernier point Quirrell~- non, disons un point pour Serdaigle~-- pour votre possession de cette qualité indispensable à un vrai sorcier de combat.~>>

La bouche de Harry était grande ouverte, et, dans un état de choc muet, il cherchait frénétiquement quelque chose à répondre. \emph{Ça n'a tellement aucun rapport avec qui je suis vraiment~!}

Mais il pouvait voir que les autres élèves commençaient à y croire. L'esprit de Harry passait en revue les dénis potentiels et ne trouvait rien qui pourrait tenir contre la voix autoritaire du professeur Quirrell. Le mieux que Harry avait trouvé était <<~Je ne suis pas un psychopathe, je suis juste très créatif~>>, et encore, ça semblait menaçant. Il lui fallait dire quelque chose d'inattendu, quelque chose qui pousserait les gens à s'interrompre dans leurs pensées et à reconsidérer~-

<<~Et maintenant, dit le professeur Quirrell, M. Potter. Feu.~>>

Rien ne se passa, bien sûr.

<<~Ah, bon~>>, dit le professeur Quirrell. Il soupira. <<~J'imagine que nous devons tous commencer quelque part. M. Potter, choisissez n'importe quel étudiant et jetez-lui un Sort d'Attaque Simple. Vous le \emph{ferez} avant la fin des cours. Sinon, je vais commencer à déduire des points, et je continuerai jusqu'à ce que vous vous soyez exécuté.~>>

Harry leva précautionneusement sa baguette. Il fallait qu'il le fasse, ou le professeur Quirrell risquait de commencer à déduire des points tout de suite.

Doucement, comme s'il avait été sur une plaque chauffante, Harry pivota vers les Serpentard.

Et les yeux de Harry rencontrèrent ceux de Drago.

Drago n'avait pas l'air le moins du monde effrayé. Il ne lui donnait aucun signe visible d'assentiment, tel que celui que Harry avait donné à Hermione, mais on pouvait difficilement s'attendre à ce qu'il le fasse. Les autres Serpentard auraient trouvé cela plutôt étrange.

<<~Pourquoi cette hésitation~? dit le professeur Quirrell. Il n'y a qu'un seul choix évident.

--- Oui, dit Harry. Seulement un choix \emph{évident}.~>>

Harry fit un mouvement du poignet et dit <<~\emph{Ma-ha-su}~!~>>

Un silence complet s'abattit dans la salle.

Harry secoua son bras gauche, essayait de se débarrasser de la douleur qui persistait.

Il y eut un peu plus de silence.

Et enfin le professeur Quirrell soupira. <<~Oui, oui, très ingénieux, mais il y avait là une leçon à apprendre et vous l'avez esquivée. Un point de moins à Serdaigle pour avoir démontré votre intelligence au détriment du véritable but. Le cours est terminé.~>>

Et avant que quiconque puisse dire quoi que ce soit, Harry chanta~:

<<~Je rigole~! SERDAIGLE~!~>>

Il y eut un silence pendant un bref moment, le bruit de gens réfléchissant, puis les murmures commencèrent et grimpèrent rapidement jusqu'à devenir le grondement d'une conversation.

Harry se tourna vers le professeur Quirrell, ils avaient besoin de parler~-

Quirrell s'était affalé sur lui-même et se traînait péniblement jusqu'à sa chaise.

Non. Inacceptable. Ils avaient \emph{vraiment} besoin de parler. Le numéro de zombie pouvait aller se faire voir, le professeur Quirrell se réveillerait probablement si Harry lui flanquait quelques coups de coude. Harry s'avança~-

NON

PAS BIEN

MAUVAISE IDÉE

Harry oscilla, arrêté net dans sa marche, se sentant étourdi.

Puis une nuée de Serdaigle s'abattit sur lui et les discussions commencèrent.~ 

%  LocalWords:  squinty meetcha tryna doin Gryffindorks til su Adalbert
%  LocalWords:  Waffling’s Mahasu Hmph
