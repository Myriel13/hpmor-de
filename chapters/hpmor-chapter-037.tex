\chapter{Interlude~: Franchir la limite}

\lettrinepara{I}{l} était presque minuit.

\hplettrineextrapara
StHarry n'avait eu aucune difficulté à veiller tard. Il lui avait suffit de ne pas utiliser le Retourneur de Temps, suivant ainsi une tradition qui consistait à aligner son cycle de sommeil pour être sûr d'être éveillé lorsque la veille de Noël deviendrait le jour de Noël~; parce que, même s'il n'avait jamais été assez jeune pour \emph{croire} au père Noël, il avait un jour été assez jeune pour douter.

Ç'aurait été bien si un personnage mystérieux entrait \emph{vraiment} dans votre maison, la nuit, pour vous apporter des cadeaux…

Un frisson descendit alors le long de l'épine dorsale de Harry.

Le sentiment que quelque chose d'affreux approchait.

Une terreur rampante.

Une sensation funeste.

Harry se redressa instantanément dans son lit.

Il regarda par la fenêtre.

"\emph{Professeur Quirrell~?"} glapit-il très doucement.

Le professeur Quirrell éleva légèrement la main et la fenêtre de Harry sembla se replier dans son cadre. Une bouffée d'air hivernal pénétra immédiatement à travers l'ouverture, accompagnée d'une poignée de flocons de neige venus d'un ciel tacheté de nuages nocturnes gris, entre l'obscurité et les étoiles.

"N'ayez crainte, M. Potter," dit le professeur de Défense d'une voix normale. "J'ai ensorcelé vos parents, ils ne se réveilleront pas avant mon départ."

"Personne n'est censé savoir où je suis~!" dit Harry, gardant le glapissement discret. "Même les chouettes sont censées livrer mon courrier à Poudlard, pas ici~!" Harry avait volontairement accepté cela~; il serait idiot qu'un Mangemort puisse gagner la guerre n'importe quand juste en lui envoyant par chouette une grenade armée d'un détonateur magique.

Le professeur Quirrell souriait depuis le jardin situé de l'autre côté de la fenêtre. "Oh, je ne m'inquiéterais pas à ce sujet, M. Potter. Vous \emph{êtes} bien protégé contre les sorts de localisation, et aucun Puriste du Sang ne pensera jamais à consulter l'annuaire." Son sourire s'élargit. "Et il m'a en effet fallu des efforts considérables pour franchir les barrières que le directeur a mises en place autour de cette maison - même si bien sûr toute personne connaissant votre adresse pourrait simplement attendre et vous attaquer lors votre prochaine sortie."

Harry regarda le professeur Quirrell pendant un moment. "Que \emph{faites}-vous ici~?" dit-il enfin.

Le sourire quitta le visage du professeur Quirrell. "Je suis venu vous demander pardon, M. Potter," dit doucement le professeur de Défense. "Je n'aurais pas dû vous parler aussi durement que je l'ai -"

"Évitez," dit Harry. Il baissa les yeux sur la couverture qu'il gardait serrée autour de son pyjama, "Évitez juste…"

"Vous ai-je tant offensé que cela~?" dit la voix douce du professeur Quirrell.

"Non," dit Harry. "Mais vous le \emph{ferez} si vous vous excusez."

"Je vois," dit le professeur Quirrell, et sa voix devint instantanément sévère. "Alors, si je dois vous traiter comme un égal, M. Potter, je dois dire que vous avez gravement enfreint l'étiquette qui régit l'amitié entre deux Serpentard. Si vous n'êtes pas en train de jouer contre quelqu'un, vous ne \emph{devez} pas vous mêler de ses plans comme vous l'avez fait, pas sans leur en parler \emph{avant}. Car vous ne savez pas quel peut être leur véritable dessein, ni l'enjeu qu'ils y ont placé. Cela vous marquerait comme étant un ennemi, M. Potter."

"Je suis navré," dit Harry, du même ton doux que le professeur Quirrell avait utilisé.

"Excuses acceptées," dit le professeur Quirrell.

"Mais," dit Harry, toujours doucement, "nous devrions vraiment reparler de politique à un moment ou à un autre."

Le professeur Quirrell soupira. "Je sais que vous n'aimez pas la condescendance, M. Potter -"

Ce n'était pas un petit euphémisme.

"Mais il serait encore plus condescendant," dit le professeur Quirrell, "de ne pas le dire clairement. Il vous manque une certaine expérience de vie, M. Potter."

"Et toutes les personnes avec une expérience de vie suffisante sont-elles d'accord avec vous, alors~?" dit calmement Harry.

"À quoi sert l'expérience pour quelqu'un qui joue au Quidditch~?" dit le professeur Quirrell, et il haussa les épaules. "Je pense que vous finirez par changer d'avis, quand toute votre confiance aura été trahie et que vous serez devenu cynique."

Le professeur de Défense avait dit cela comme si c'était la déclaration la plus banale du monde, entouré de l'obscurité et des étoiles et du ciel tacheté de nuages, tandis qu'un ou deux petits flocons passaient devant lui dans l'air hivernal mordant.

"Ce qui me rappelle," dit Harry. "Joyeux Noël."

"Je suppose," dit le professeur Quirrell. "Après tout, si ce ne sont \emph{pas} des excuses, alors ce doit être un cadeau de Noël. Le premier que j'ai jamais donné, à vrai dire."

Harry n'avait même pas commencé à apprendre le Latin pour pouvoir lire le journal expérimental de Roger Bacon~; et il n'osa ouvrir la bouche pour demander.

"Mettez votre manteau d'hiver," dit le professeur Quirrell, "ou prenez une potion réchauffante si vous en avez une~; et retrouvez-moi dehors, sous les étoiles. Je verrai si je peux le maintenir un peu plus longtemps cette fois-ci."

Il fallut un moment avant que Harry ne comprenne le sens des mots, et il fonça alors vers son placard à manteaux.

Le professeur Quirrell maintint le sort de lumière stellaire pendant plus d'une heure, même si son visage commença à révéler de la tension, et il fallut qu'il s'assoie au bout d'un moment. Harry protesta une seule fois et se vit intimer l'ordre de se taire.

Il franchirent la limite entre le réveillon de Noël et le jour de Noël dans ce vide intemporel où la rotation terrestre ne voulait rien dire, dans la seule véritable sainte nuit éternelle.

Et exactement comme promis, les parents de Harry dormirent à poings fermés du début à la fin, jusqu'à ce que Harry soit de retour dans sa chambre et que le professeur de Défense soit parti.
