\chapter{Mensonges contagieux}

\lettrine{H}{ermione} Granger avait lu quelque part que l'une des clés de la minceur était de prêter attention à ce que l'on mangeait et de se concentrer sur cet acte afin d'être satisfait de son repas. Ce matin elle s'était fait griller une tranche de pain, elle avait mis du beurre dessus et elle avait mis de la cannelle sur le beurre, alors ça aurait vraiment dû suffire à ce qu'elle soit \emph{consciente}, cette fois-ci, de toute bonne nourriture qui l'attendait…

Sans prêter attention ni à la cannelle ni au beurre, sans remarquer la nourriture ni se rendre compte qu'elle mangeait, Hermione avala une autre bouchée de tartine et dit~: "Est-ce que tu peux me réexpliquer ça~? Ça me sidère toujours autant."

"C'est plutôt clair quant on y pense du point de vue d'un Serpentard du côté clair," dit le garçon que tous les résidents de l'école mis à part eux deux croyaient être son véritable amour. La cuillère de Harry Potter remua distraitement ses céréales~; Hermione ne l'avait pas vu manger grand-chose ce matin. "Tout ce qui est bon en ce monde provoque l'existence de son opposé. Les phénix ne font pas exception."

Hermione prit une autre bouchée inconsciente de sa tartine beurrée et recouverte de cannelle avant de dire~: "Comment quelqu'un pourrait-il ne \emph{pas comprendre} que Fumseck pense que tu es quelqu'un d'assez bien pour le promener sur ton épaule~? Il ne ferait pas ça avec un mage noir~! Certainement pas~!"

Et elle n'avait été crier à personne que Fumseck avait touché \emph{sa} joue à elle parce qu'elle savait qu'elle aurait eu tort de le faire - que si un phénix vous touchait, vous n'étiez pas censé vous en vanter, que ce n'était pas \emph{pour ça} que les phénix existaient.

Mais elle avait vraiment \emph{espéré} que ça écraserait les rumeurs selon lesquelles Harry Potter devenait méchant et que Hermione Granger suivait sa trace.

Et ça n'avait pas marché.

Et elle ne pouvait vraiment pas comprendre pourquoi.

Harry prit une autre bouchée de céréales, ses yeux devinrent distants et s'écartèrent des siens. "Vois-le comme ça~: tu rates un jour d'école, tu mens et tu dis au professeur que tu étais malade. Le professeur te dit d'amener un mot du docteur, alors tu en fabriques un. Le professeur dit qu'elle va appeler le docteur pour vérifier, alors tu dois lui donner un faux numéro et demander à un ami de faire semblant d'être le docteur quand elle appelle -"

"Tu as fait \emph{quoi}~?"

Harry releva alors les yeux de son bol~; il souriait. "Je ne dis pas que je l'ai vraiment \emph{fait}, Hermione…" Puis ses yeux retombèrent abruptement vers ses céréales. "Non. Juste un exemple. Les mensonges se propagent, c'est ça que je veux te dire. Qu'il faut rajouter plus de mensonges pour couvrir les précédents, qu'il faut mentir sur chaque fait relié au premier mensonge. Et que si on \emph{continue} de mentir, si on \emph{continue} d'essayer de le couvrir, tôt ou tard on doit même commencer à mentir au sujet des lois générales de la pensées. Par exemple, si quelqu'un te vend une sorte de médicament alternatif qui ne fonctionne pas, n'importe quelle étude en double aveugle confirmera qu'il ne fonctionne pas. Donc si quelqu'un veut \emph{continuer} à défendre ce mensonge, ils va falloir qu'il t'amène à ne plus croire à la méthode expérimentale. En disant par exemple que la méthode expérimentale est adaptée aux médicaments qui sont seulement \emph{scientifiques}, pas aux super médicaments alternatifs comme le leur. Ou qu'une personne bonne et vertueuse devrait croire aussi fort qu'elle en est capable peu importe les preuves du contraire. Ou qu'il n'y a pas de vérité et que la réalité objective n'existe pas. Une bonne partie de la sagesse commune n'est pas seulement \emph{dans l'erreur}, elle est aussi anti-épistémologique, \emph{systématiquement} fausse. Pour chaque règle de la rationalité qui te dit comment trouver la vérité, il y a quelqu'un qui a besoin que tu croies le contraire. Si tu profères un seul mensonge, la réalité est à jamais ton ennemie~; et on est entouré par beaucoup de menteurs -" la voix de Harry se tut.

"Qu'est-ce que ça a à voir avec Fumseck~?" dit-elle.

Harry retira sa cuillère de ses céréales et la pointa vers la table d'honneur. "Le directeur a un phénix, pas vrai~? Et il est président du Magenmagot~? Donc il a de nombreux opposants politiques, Lucius par exemple. Maintenant est-ce que tu crois que l'opposition va juste s'écraser et se rendre parce que Dumbledore a un phénix et pas eux~? Est-ce que tu penses qu'ils vont admettre que Fumseck constitue même un \emph{élément de preuve} que Dumbledore est quelqu'un de bien~? Bien sûr que non. Ils doivent trouver \emph{quelque chose} à dire qui rendra Fumseck… \emph{insignifiant}. Par exemple en disant que les phénix ne suivent que les gens qui se ruent droit sur tout ce qui est maléfique, et donc qu'avoir un phénix veut seulement dire qu'on est un idiot ou un dangereux fanatique. Ou~: les phénix suivent juste les gens qui sont de purs Gryffondor, tellement Gryffondor qu'ils n'ont aucune vertu des autres maisons. Ou que ça montre seulement à quel point un animal magique pense qu'on est courageux, rien de plus, et que ça ne serait pas juste de juger les politiciens là-dessus. Ils faut qu'il disent \emph{quelque chose} pour renier le phénix. Je parie que Lucius n'a même pas eu besoin d'inventer quoi que ce soit. Je parie que tout avait déjà été dit il y a des siècles lorsque quelqu'un s'est promené avec un phénix sur son épaule pour la première fois et que quelqu'un d'autre a voulu que les gens ne considèrent pas ça comme une preuve de quoi que ce soit. Je parie que ça faisait déjà partie du savoir populaire le jour où Fumseck est né, que ça aurait juste semblé \emph{bizarre} de prendre en compte les préférences personnelles d'un phénix pour évaluer la valeur d'une personne. Ça serait comme si un journal moldu testait des candidats politiques en mesurant leur niveau de connaissance scientifique. Pour chaque force au service du Bien de cet univers, il y a quelqu'un qui profite du fait que les gens n'en tiennent pas compte ou qui étiquette cette force si bien qu'elle ne peut plus jamais l'atteindre."

"Mais -" dit Hermione. "D'accord, je vois pourquoi Lucius Malfoy ne veut pas qu'on pense que Fumseck a de l'importance, mais pourquoi des gens qui ne \emph{sont pas} méchants y \emph{croient}, eux~?"

Harry Potter eut un petit haussement d'épaules. Sa cuillère retomba dans ses céréales et continua de tourner~: "Pourquoi les cynismes de toutes sortes plaisent-ils aux gens~? Parce que ça semble être une marque de maturité et de raffinement, parce que c'est comme d'avoir tout vu et de savoir mieux que les autres. Ou parce que rabaisser quelque chose donne l'impression qu'on s'élève. Ou parce qu'ils n'ont pas de phénix, alors leur instinct politique leur dit qu'il n'y a aucun avantage politique à en dire du bien. Ou parce qu'être cynique donne l'impression de connaître une vérité secrète que le commun des mortels ignore…" Harry Potter regarda en direction de la table d'honneur et sa voix tomba presque jusqu'au chuchotement. "Je pense que c'est peut-être sur ça qu'\emph{il} se trompe - qu'il est cynique sur tout le reste mais pas sur le cynisme."

Sans y penser, Hermione regarda vers la table d'honneur, mais le siège du professeur de Défense était toujours vide, comme il l'avait été lundi et mardi~; la directrice adjointe avait déclaré plus tôt que les cours de Défense d'aujourd'hui étaient annulés.

Plus tard, lorsque Harry eut avalé quelques bouchées de tarte à la mélasse, Hermione regarda Anthony et Padma qui avaient mangé non loin, tout à fait par hasard et certainement pas pour écouter aux portes.

Ils la regardèrent en retour.

Padma dit d'un ton hésitant~: "Est-ce que c'est moi ou est-ce que Harry Potter s'est mit à parler comme un livre encore plus \emph{compliqué} ces derniers jours~? Enfin, je ne l'ai pas écouté très longtemps -"

"Ce n'est pas seulement toi," dit Anthony.

Hermione ne répondit pas mais son inquiétude allait croissante. Quoi qu'il soit arrivé à Harry le jour du phénix, cela l'avait changé~; quelque chose de nouveau l'habitait. Pas de froid, mais de \emph{dur}. Parfois elle le surprenait à regarder une fenêtre sans rien observer de particulier avec un air de sombre détermination sur le visage. Lundi, en cours de botanique, un piège de feu de Vénus était devenu incontrôlable~; Harry avait plaqué Terry au sol pour lui faire éviter la boule de feu alors même que le professeur Chourave rugissait un sortilège de Gèle-Flamme~; et lorsque Harry s'était relevé il était juste retourné à sa place comme si rien d'intéressant ne s'était produit. Et lorsque pour une fois elle avait eu une meilleure note que lui à leur contrôle de Métamorphose, plus tard le même jour, Harry avait souri comme pour la féliciter au lieu de grincer des dents et ça… ça avait \emph{beaucoup} embêté Hermione.

Elle avait l'impression que Harry…

… s'éloignait d'elle…

"Il a l'air beaucoup plus \emph{vieux} tout d'un coup," dit Anthony. "Pas comme un vrai adulte, je n'arrive pas à imaginer \emph{Harry} adulte, mais c'est comme s'il venait de devenir une version \emph{quatrième année} de… de ce qu'il \emph{est}."

"Eh bien," dit Padma. Elle grignotait délicatement un scone au chocolat recouvert d'un glaçage au scone. "Je pense que Dragon et Soleil feraient mieux de s'unir pendant la prochaine bataille, ou M. Harry Potter va nous \emph{détruire}. On était tous alliés la dernière fois et Chaos a quand même failli gagner -"

"Ouais," dit Anthony. "Vous avez raison, mademoiselle Patil. Dites au général Dragon qu'on veut vous rencontrer -"

"Non~!" dit Hermione. "On ne \emph{devrait pas} avoir à se liguer contre le général Potter juste pour avoir une chance. Ça n'a aucun sens, surtout maintenant que plus personne ne peut utiliser d'appareils moldus. On en reste à vingt-quatre soldats par armée."

Ni Padma ni Anthony ne trouvèrent à y répondre.

\later

Toc-toc, toc-toc.

"Entrez, M. Potter," dit-elle.

La porte s'ouvrit dans un craquement et Harry Potter se glissa par l'ouverture dans son bureau~; il la referma derrière lui d'une main et s'assit sans dire un mot sur la chaise rembourrée qui se trouvait maintenant devant son bureau. Elle avait métamorphosé cette chaise si souvent qu'elle changeait parfois de forme et réfléchissait son humeur sans mouvement de baguette ni incantation ni même pensée consciente. Pour l'instant, la chaise était particulièrement bien rembourrée, si bien que lorsque Harry s'y enfonça, ce fut comme si la chaise lui faisait un câlin.

Il ne sembla pas le remarquer. Une apparence de détermination tranquille se dégageait de lui~; ses yeux étaient irrémédiablement braqués sur les siens et ne les lâchaient pas. "Vous m'avez appelé~?" dit le garçon.

"Oui," dit le professeur McGonagall. "J'ai deux bonnes nouvelles pour vous, M. Potter. D'abord - avez-vous jamais rencontré M. Rubeus Hagrid~? Le gardien des Clés~? C'était un vieil ami de vos parents."

Harry hésita. Puis~: "M. Hagrid m'a un peu parlé après mon arrivée," dit-il. "Je crois que c'était un mardi pendant ma première semaine de cours. Mais il n'a pas dit qu'il connaissait mes parents. À l'époque je pensais qu'il voulait juste se présenter au Survivant… avait-il un but caché~? Il n'avait pas \emph{l'air} d'être le genre à…"

"Ah…" dit-elle. Il lui fallut un moment pour rassembler ses pensées. "C'est une longue histoire, M. Potter, mais M. Hagrid a été accusé à tort du meurtre d'une élève voilà cinq décennies. La baguette lui a été confisquée et il a été exclu. Plus tard, quand le professeur Dumbledore est devenu directeur, il a donné un travail à Hagrid en tant que gardien des Clés de Poudlard."

Les yeux de Harry étaient attentifs et fixés sur elle. "Vous avez dit qu'un élève est mort à Poudlard pour la dernière fois il y a cinquante ans et que vous étiez certaine que le message secret du Choixpeau a été entendu pour la dernière fois il y a aussi cinquante ans."

Elle eut un léger frisson - même le Directeur ou Severus n'auraient peut-être pas fait le lien si vite - et elle dit~: "Oui, M. Potter. Quelqu'un a ouvert la Chambre des Secrets, mais personne ne l'a cru, et M. Hagrid a été tenu pour responsable de la mort qui en a découlé. Le directeur a cependant localisé l'enchantement sur le Choixpeau et l'a montré à un panel choisi du Magenmagot. Ainsi, la sentence de M. Hagrid a été révoquée - ce matin même, à vrai dire - et il lui sera permis de recevoir une nouvelle baguette." Elle hésita. "Nous… ne le lui avons pas encore dit, M. Potter. Nous attendions que ce soit fait afin de ne pas lui donner de faux espoirs après tant d'années. M. Potter… nous nous demandions si vous accepteriez que nous disions à M. Hagrid que c'est vous qui l'avez aidé…?"

Elle vit le regard calculateur dans ses yeux -

"Je me souviens de M. Hagrid lorsqu'il vous portait dans ses bras alors que vous étiez un bébé," dit-elle. "Je pense qu'il serait très heureux de l'apprendre."

Mais elle put voir sur le visage de Harry l'instant où il décida que Rubeus ne lui serait d'aucune utilité.

Harry secoua la tête. "C'est déjà assez embêtant que quelqu'un puisse déduire qu'il y avait un Fourchelangue dans la cohorte d'élèves de cette année," dit-il. "Je pense qu'il serait plus prudent de garder tout cela aussi secret que possible."

Elle se souvint de James et Lily qui n'avaient jamais hésité à rendre à l'immense géant aux larges épaules l'amitié qu'il leur avait offerte, et ce en dépit du statut d'héritier de maison noble de James, du futur professorat en Sortilèges de Lily et du fait que Rubeus n'était qu'un demi géant dont la baguette avait été confisquée…

"Parce que vous ne vous attendez pas à ce qu'il vous soit utile, M. Potter~?"

Il y eut un silence. Elle n'avait probablement pas compté dire ça à voix haute.

Une vague de tristesse passa sur le visage de Harry. "Probablement," dit-il à voix basse. "Mais je pense que lui et moi ne deviendrions pas amis. Vous ne croyez pas~?"

Elle avait l'impression que quelque chose s'était coincé dans sa gorge.

"En parlant de faire usage des gens," dit Harry. "Il semble que je vais bientôt être jeté au cœur d'une guerre contre un Seigneur des Ténèbres. Alors tant que je suis dans votre bureau, je voudrais vous demander d'étendre mon cycle de sommeil pour qu'il soit de trente heures par jour. Neville Londubat veut commencer à pratiquer le duel, il y a un Poufsouffle plus âgé qui a proposé de lui apprendre et ils m'ont invité à les rejoindre. Et il y aussi d'autres choses que je veux apprendre - et si vous ou le directeur pensez que je devrais étudiez quelque chose de précis afin de devenir un puissant sorcier quand je serais grand, faites-le moi savoir. S'il vous plaît, donnez à madame Pomfresh des instructions pour qu'elle m'administre la potion adéquate, ou autre chose -"

"\emph{M. Potter~!}"

Les yeux de Harry s'engouffrèrent dans les siens. "Oui, Minerva~? Je sais que ce n'était pas votre idée, mais j'aimerais survivre à l'usage que le directeur compte faire de moi. S'il vous plaît, n'y faites pas obstacle."

Cela faillit la briser. "Harry," murmura-t-elle d'une voix aphone, "les enfants ne devraient pas avoir à \emph{penser} comme ça~!"

"Vous avez raison, ce serait préférable," dit Harry. "Mais \emph{beaucoup} d'enfants doivent grandir trop tôt, pas seulement moi~; et la plupart de ces enfants ne mettraient pas cinq secondes à décider d'échanger leur place avec moi. Je ne vais pas me prendre en pitié, professeur McGonagall, pas tant qu'il existe en ce monde, des gens qui ont de vrais ennuis et que je ne suis pas l'un d'eux."

Elle déglutit avec force et dit~: "M. Potter, à trente heures par jour, vous - \emph{vieillirez} plus vite -" \emph{Comme Albus}.

"Et en cinquième année j'aurais à peu près le même âge physiologique que Hermione," dit Harry. "Ça ne m'a pas l'air \emph{si} terrible que ça." Il arborait maintenant un sourire narquois. "Franchement, je choisirais probablement de faire ça même \emph{sans} Seigneur des Ténèbres. Les sorciers vivent un bon bout de temps et soit les sorciers soit les Moldus feront probablement reculer cette limite pendant le siècle à venir. Il n'y a aucune raison de ne \emph{pas} caser autant d'heures par jour que possible. J'ai des projets, et il vaudrait mieux qu'ils se réalisent rapidement."

Il y eut un long silence.

"Très bien," dit Minerva. C'était presque un chuchotement. Elle éleva la voix. "Très bien, M. Potter, je demanderai au directeur et s'il accepte, ce sera fait."

Les yeux de Harry se plissèrent l'espace d'un instant. "Je vois. Veuillez alors rappeler au directeur que Godric Gryffondor, dans son dernier souffle, a dit que la voie qu'il avait choisie était la bonne et qu'il ne conseillerait donc à personne de faire un choix autre, le mauvais choix, pas même aux élèves les plus jeunes de Poudlard."

Et elle sut au moment même où l'apathie s'emparait d'elle que toute chance de voir Albus empêcher cela venait de disparaître dans le néant. C'était ce que Albus lui avait dit lorsqu'elle avait objecté que Cameron Edward était trop jeune, puis lorsqu'elle avait objecté que Peter Pevensie était trop jeune, et elle avait finit par ne plus jamais faire d'objections. "Qui vous a dit ça, M. Potter~?" \emph{Pas Albus - Albus ne dirait certainement pas ça à un élève -}

"J'ai beaucoup lu dernièrement," dit Harry. Son corps commença à s'élever hors de la chaise puis s'arrêta. "Oserais-je vous interroger sur la seconde bonne nouvelle~?"

"Oh," dit-elle. "Ah - le professeur Quirrell s'est éveillé et dit que vous pouvez -"

\later

L'infirmerie de Poudlard occupait un vaste espace puissamment éclairé par le ciel depuis des fenêtres situées sur chacun de ses quatre murs, et ce en dépit de son emplacement central dans le bâtiment de Poudlard. Des lits blancs en longues rangées parcouraient la salle et seuls trois d'entre eux étaient pour le moment occupés. Un garçon plus âgé et une fille elle aussi plus âgée, chacun d'un côté de la pièce, tous deux immobiles, les yeux fermés, probablement rendus inconscients par un sortilège tandis qu'un autre, ou une potion, réorganisait leur corps sans ménagements~; et le troisième occupant avait un rideau tiré autour de son lit, ce qui était probablement préférable. Madame Pomfresh l'avait poussé avec force en lui disant de ne pas avoir l'air ahuri et Harry s'était sévèrement rappelé à lui-même que certaines personnes ne savaient toujours pas qui le Survivant était - soit ça, soit l'identité de madame Pomfresh reposait entièrement sur sa domination absolue sur son hôpital, etc, etc, bref.

Derrière la rangée de lits se trouvaient cinq portes qui menaient dans des chambres individuelles où étaient entreposés les patients qui resteraient ici quelques jours plutôt que quelques heures mais dont l'état ne justifiait pas un transfert à Sainte Mangouste.

Sans fenêtre ni ciel, éclairée uniquement d'une torche d'où aucune fumée ne s'échappait et accrochée à l'un des murs de pierre~: c'était la pièce derrière la porte centrale. Harry s'était demandé si les professeurs pouvaient demander à Poudlard de se modifier elle-même ou si l'infirmerie avait toujours une pièce de ce genre disponible pour ceux qui n'appréciaient pas la lumière.

Au centre de la pièce, entre deux tables de nuit qui semblaient avoir été extraites du même marbre gris que les murs, se trouvait un lit blanc d'hôpital qui prenait une teinte vaguement orangée sous la lumière sans fumée de la torche~; et dans ce lit, un drap remonté jusqu'aux cuisses et vêtu d'une chemise d'hôpital, le professeur Quirrell, assis, le dos légèrement appuyé contre la tête de lit.

En dépit de son apparence indemne, quelque chose dans l'image du professeur Quirrell assis dans l'un des lits de madame Pomfresh effrayait Harry. Même en sachant que le professeur Quirrell avait délibérément mis en scène sa défaite contre Severus afin de s'offrir une excuse qui lui permettrait de récupérer la force qu'il avait perdue à Azkaban. Harry n'avait jamais \emph{vraiment} vu quelqu'un mourir dans un lit d'hôpital mais il avait regardé trop de films. C'était un présage de mortalité et le professeur de Défense n'était \emph{pas} censé être mortel.

Madame Pomfresh avait dit à Harry qu'il lui était absolument interdit d'enquiquiner son patient.

Harry avait dit "je comprends," ce qui n'impliquait techniquement rien quant à son obéissance.

La vieille et sévère guérisseuse s'était alors tournée vers le professeur Quirrell et lui avait dit qu'il ne fallait absolument pas qu'il se fatigue ni qu'il… ne s'énerve…

Puis elle avait laissé sa phrase en suspens, avait fait un demi-tour empressé et s'était échappée de la pièce.

"Pas mal," remarqua Harry après que la porte se fut refermée derrière la matrone médicale en fuite. "Je devrais apprendre à faire ça un jour."

Le professeur Quirrell esquissa un sourire dénué de tout humour et dit d'une voix bien plus sèche qu'à l'habitude~: "Merci pour votre critique artistique, M. Potter."

Harry regarda les pâles yeux bleus et trouva que le professeur Quirrell avait l'air…

… plus vieux.

C'était subtil, et ça n'était peut-être qu'un effet de son imagination ou du mauvais éclairage. Mais les cheveux au-dessus du front de Quirinus Quirrell s'étaient peut-être un peu dégarnis, ceux qui demeuraient avaient peut-être minci ou grisé, comme une progression de la calvitie déjà visible à l'arrière de sa tête. Quant à son visage, il s'était peut-être un peu creusé.

Les pâles yeux bleus étaient demeurés vifs et perçants.

"Je suis heureux," dit Harry à voix basse, "de vous voir visiblement bien portant."

"Les apparences peuvent évidemment être trompeuses," dit le professeur Quirrell. Il agita les doigts et lorsque son geste fut achevé, sa baguette se trouvait entre ses mains. "Croirez-vous que cette femme pense m'avoir confisqué ceci~?"

Le professeur de Défense prononça six incantations~; six des trente qu'il avait utilisées pour protéger leur importante conversation dans la chambre de Marie.

Harry l'interrogea silencieusement d'un haussement de sourcils.

"C'est tout ce dont je suis capable pour l'instant," dit le professeur de Défense. "Je pense que cela s'avérera suffisant. Néanmoins, comme dit le proverbe~: si tu ne souhaites pas qu'on entende quelque chose, ne le dis pas. Considérez qu'il s'applique ici pleinement. J'ai cru comprendre que vous désiriez me voir~?"

"Oui," dit Harry. Il s'interrompit et rassembla ses pensées. "Le directeur ou qui que ce soit d'autre vous a-t-il dit que nous ne pourrions plus aller déjeuner~?"

"Quelque chose dans le genre," répondit le professeur de Défense. Et sans changer d'expression~: "Bien sûr, j'ai été terriblement désolé de l'apprendre."

"Cela va plus loin, pour tout vous dire," continua Harry. "Je suis confiné à Poudlard et ses terres pour une durée indéterminée. Je ne peux en sortir sans avoir un garde et une bonne raison. Je ne vais pas rentrer chez moi cet été, et peut-être jamais. J'espérais… pouvoir en parler avec vous."

Il y eut un silence.

Le professeur de Défense exhala ce qui ressembla à un bref soupir et dit~: "Nous allons devoir nous reposer sur le fait que la directrice adjointe s'occupera personnellement du meurtre de toute personne désireuse de me signaler aux autorités. M. Potter, j'ai l'intention de ne pas dévier de mon sujet afin de pouvoir rapidement achever cette conversation. Est-ce compris ?"

Harry hocha la tête et -

Éclairées d'une unique torche et décalées vers le côté rouge du spectre optique, les rayures bleues et blanches du serpent ne réfléchissaient que peu la lumière, à peine plus que ses écailles vertes. Ce dernier semblait assombri. Ses yeux, qui avaient auparavant ressemblé à des trous gris, renvoyaient la lumière de la torche et semblaient plus clairs que le reste de son corps.

"\parsel{Alors}," siffla la créature venimeuse. "\parsel{Que ssouhaitais-tu me dire~?"}

Et Harry siffla~: "\parsel{Le directeur pensse que c'est l'ancien Sseigneur de cette femme qui l'a volée à cette prison.}"

Cette fois-ci, Harry \emph{avait} réfléchi avec soin avant de décider qu'il révélerait \emph{uniquement} ce fait au professeur Quirrell, \emph{sans} rien mentionner de la prophétie qui avait fait s'abattre Voldemort sur les parents ni du fait que le directeur reconstituait l'Ordre du Phénix… c'était un risque, un risque important, mais Harry avait besoin d'un allié.

"\parsel{Il pensse que celui-là est en vie~?}" répondit enfin le serpent. La langue bifide passa rapidement d'un côté à l'autre de sa bouche, rire reptilien sardonique. "\parsel{Et pourtant je ne ssuis pas ssurpris.}"

"\parsel{Oui}," siffla Harry avec sécheresse. "\parsel{très amusant, j'en ssuis ssûr. Ssauf que maintenant je ssuis coincé à Poudlard pendant les ssix prochaines années par mesure de ssécurité~! J'ai décidé que je vais bel et bien me mettre en quête de pouvoir~; et la sséquestration ne m'y aide pas beaucoup. Je dois convaincre le directeur que le Sseigneur des Ténèbres n'est pas encore réveillé, que cette évasion était l'œuvre d'une autre force -"}

De nouveau l'oscillation rapide de la langue du serpent~; son rire était plus fort, plus sec cette fois. "\parsel{Bêtise d'amateur}."

"\parsel{Comment~?}" siffla Harry.

"\parsel{Tu vois erreur, pensses à la défaire, à annuler ce qui a été fait. Mais même avec un ssablier le temps ne peut être défait. Mieux vaut avancer. Tu pensses à convaincre les autres qu'ils ont tort. Bien plus ssimple de les convaincre qu'ils ont raison. Conssidères ceci, garçon~: quel heureux hasard pousserait le maître de l'école à décider que tu es de nouveau en ssécurité et ferait simultanément avancer tes autres projets ?"}

Harry regarda le serpent, perplexe. Son esprit essaya de comprendre, de dénouer l'énigme -

"\parsel{N'est-ce pas évident~?}" siffla le serpent. De nouveau la langue qui oscillait, le rire sardonique. "\parsel{Pour te libérer et gagner du pouvoir en Angleterre, on doit à nouveau te voir vaincre le Seigneur des Ténèbres."}

\later

Sous la lumière rouge orangée d'une torche vacillante, un serpent vert ondulait sur un lit blanc d'hôpital et un garçon scrutait l'ambre de ses yeux.

"\parsel{Donc}," finit par dire Harry. "\parsel{Ssoyons clairs quant à votre proposition. Vous ssugérez de trouver un impossteur qui jouera le rôle du Sseigneur des Ténèbres."}

"\parsel{Quelque chose comme ça. La femme ssauvée coopérera, ssera des plus convaincantes quand on la verra à sses côtés."} De nouveau l'oscillement lingual sardonique. "\parsel{Tu te fais enlever de Poudlard, emmené vers lieu public, nombreux témoins, barrières pour empêcher protecteurs d'approcher. Sseigneur des Ténèbres annonce qu'il a enfin retrouvé sson corps après avoir erré ssous forme d'essprit pendant des années~; dit que sson pouvoir est plus grand que jamais et que même toi ne pourras l'arrêter cette fois-ci. Propose duel. Tu lances sortilège gardien, Sseigneur des Ténèbres sse rit de toi, dit qu'il n'est pas un mange-vie. Lance ssort de mort vers toi, tu bloques, témoins voient le Seigneur des Ténèbres exploser -"}

"\parsel{Lancer ssort de mort~?"} siffla Harry d'un ton incrédule. "\parsel{Ssur moi~? Deuxième fois~? Personne ne croira que le Sseigneur des Ténèbres sserait asssez bête pour -"}

"\parsel{Toi et moi ssommes les deux sseules personnes du pays qui l'auraient remarqué,"} siffla le serpent. "\parsel{Fais-moi confiance sur ce point, garçon."}

"\parsel{Et ssi un jour il y en a une troisième~?"}

Le serpent ondula pensivement. "\parsel{Pourrons écrire autre sscénario à jouer si tu ssouhaites. Quel qu'il ssoit, devrons laissser ouverte posssibilité que Sseigneur des Ténèbres revienne de nouveau - pays doit pensser qu'ils dépendent toujours de toi pour les protéger."}

Harry regarda les trous rouge et vacillants des yeux du serpent.

"\parsel{Eh bien~?}" siffla la silhouette ondulante.

L'évidence était que de suivre les plans et tromperies du professeur Quirrell une \emph{seconde} fois, de conjurer un mensonge encore \emph{plus} compliqué pour couvrir l'erreur initiale et de créer une \emph{autre} vulnérabilité fatale dans l'éventualité où quelqu'un découvrirait la vérité constituerait \emph{exactement} le même genre d'erreur imbécile que celle commise par le Seigneur des Ténèbres putatif et consistant à utiliser le Sortilège de la Mort une seconde fois. Harry n'eut même pas besoin que son côté Poufsouffle le dise~: il y pensa de sa propre voix intérieure.

Mais la question était aussi de savoir si la morale à tirer de la dernière expérience était qu'il fallait toujours dire immédiatement \emph{non} au professeur de Défense ou si…

"\parsel{Y pensserai,}" siffla Harry. "\parsel{Ne répondrai pas tout de ssuite cette fois, énumérerai rissques et avantages avant -"}

"\parsel{Compris,"} siffla le serpent. "\parsel{Mais souviens-toi de ceci, garçon, autres événements se produisent sans toi. Hésitation est toujours facile, rarement utile."}

\later

Le garçon émergea de la chambre et entra dans la pièce principale de l'infirmerie en se passant des doigts nerveux le long de ses cheveux noirs en batailles~; il dépassa les lits blancs, ceux qui étaient vides et les autres.

Peu de temps après, il quitta pour de bon l'infirmerie de Poudlard, s'arrêta et s'appuya contre un mur.

Le problème…

… c'était qu'il n'avait \emph{pas} envie d'être coincé à Poudlard pendant les six prochaines années~; et maintenant qu'il y pensait…

… l'Incident issu de l'Évasion de Bellatrix Black d'Azkaban n'imposait pas \emph{seulement} des coûts à Harry. D'autres s'inquiéteraient, vivraient dans la peur du retour du Seigneur des Ténèbres, consommeraient des ressources afin de prendre des précautions inconnues. Harry pourrait demander que le scénario rende peu plausible \emph{un deuxième retour du Seigneur des Ténèbres.} Alors les gens se détendraient et tout serait fini.

Sauf bien sûr s'il y avait \emph{vraiment} un Seigneur des Ténèbres à craindre. La prophétie \emph{existait} bel et bien.

Il avait presque oublié de le faire, mais il \emph{avait} bien montré au professeur Quirrell le jeu de cartes que le 'Père Noël' lui avait donné dimanche soir et dont le roi de cœur était prétendument un Portoloin qui l'emmènerait à l'Institut des Sorcières de Salem, aux États-Unis. Sauf que bien sûr Harry n'avait pas dit au professeur Quirrell \emph{qui} lui avait envoyé la carte ni ce qu'elle était \emph{censée} faire avant de lui demander s'il était possible de savoir où le Portoloin l'emmènerait.

Le professeur de Défense était redevenu humain et avait examiné le roi de cœur en le tapotant plusieurs fois de sa baguette.

Et selon lui…

… le Portoloin emmènerait son utilisateur quelque part à Londres mais il ne pouvait localiser la destination avec plus de précision que cela.

Harry avait montré au professeur Quirrell le mot qui avait accompagné le jeu de cartes, sans mentionner les mots précédents.

Le professeur avait jeté un coup d'œil à celui-ci, avait eu un rire sec et avait remarqué que si on lisait le mot \emph{attentivement}, il n'était pas \emph{explicitement} dit que le Portoloin emmènerait son utilisateur à l'Institut des Sorcières de Salem.

Il fallait apprendre à faire attention à ce genre de subtilités, continua le professeur Quirrell, si l'on voulait être un sorcier puissant une fois devenu grand~; et même, si l'on voulait grandir tout court.

Le garçon soupira et se traîna jusqu'à son prochain cours.

Il commençait à se demander si toutes les autres écoles de sorcellerie étaient comme ça ou si c'était seulement Poudlard qui avait un problème. 

%  LocalWords:  ermione cinnamoned Pevensie
