\partchapter{Rôles}{III}

\lettrinepara{I}{l} n'y avait plus rien à faire.

\hplettrineextrapara
Il n'y avait plus rien à prévoir.

Il n'y avait plus rien à penser.

Dans ce vide s'éleva le nouveau pire souvenir…

Le Survivant-Contrairement-À-Sa-Meilleure-Amie marchait d'un pas lourd dans les longs couloirs résonnants qui menaient à la grande salle. Maintenant qu'il avait dépensé toute son énergie mentale, son esprit commençait à balancer des pensées, comme l'image d'une Hermione marchant à ses côtés, et des concepts muets comme Ça n'arrivera plus jamais jusqu'à ce qu'une autre partie de lui hurle Non et continue de hurler sur la première, déterminée à la ramener, sauf que la voix de cette partie se fatiguait tandis que l'autre semblait inépuisable. Une autre partie de son esprit insistait qu'il fallait repenser à ce qu'il avait dit au professeur McGonagall, et à Papa et Maman, bien qu'il n'ait fait qu'essayer de les faire sortir d'ici le plus vite possible avec seulement une réserve d'énergie mentale limitée. Comme s'il aurait pu faire mieux par un acte de sa volonté déficiente. Harry ignorait ce qui lui restait de sa relation avec ses parents.

Il parvint enfin à un point de jonction où l'attendait un garçon plus âgé dans des robes à bordures vertes qui lisait silencieusement un manuel, situé sur le chemin que n'importe qui désireux d'intercepter quelqu'un allant de l'infirmerie à la grande salle choisirait.

Harry portait la Cape d'Invisibilité, bien sûr, il l'avait mise après avoir quitté le bureau, se rendant ainsi insensible à presque toutes les formes de détection magiques. Il était futile de faciliter la tâche de toute personne désireuse de le trouver et de le tuer. Et Harry avait presque décidé de continuer sans se fatiguer à découvrir ce qui se passait quand il reconnu le visage du garçon Serpentard.

Il comprit alors. Bien sûr, l'un des élèves qui étaient restés à l'école pendant les vacances de Pâques serait naturellement…

<<~Tu m'attendais~>>, dit Harry à voix haute, sans enlever la Cape.

Le Serpentard eut un mouvement de recul, se cogna la tête contre le mur, et son manuel de cinquième année tomba de ses mains avant qu'il ne relève un visage aux yeux écarquillés.

<<~Vous êtes…

--- Invisible. Oui. Dis ce que tu as à dire.~>>

Lesath Lestrange se releva tant bien que mal, prit un air attentif, puis lâcha d'une traite~: <<~Seigneur, ai-je bien agi - je pensais que vous ne souhaiteriez pas me voir m'avancer devant tout le monde, car ils auraient pu soupçonner notre lien - j'ai pensé, que certainement si vous souhaitiez mon aide vous pourriez faire appel à moi directement -~>>

Incroyable, le nombre de façons qu'il y avait de tuer sa meilleure amie par pure stupidité.

<<~Je…~>> Lesath hésita, puis dit d'une petite voix, <<~j'avais tort, n'est-ce pas~?

--- Tu as agi exactement comme tu aurais dû le faire dans ces circonstances. C'est moi qui ai été un imbécile.

--- Je suis navré, maître, chuchota Lesath.

--- Si tu \emph{étais} venu avec moi, aurais-tu été capable de tuer le troll~?~>> Ce n'était même pas la bonne question, la bonne question était de savoir si Harry lui-même aurait considéré que Lesath suffisait et se serait envolé soixante secondes plus tôt, mais quand même…

<<~Je… je ne suis pas sûr, maître… je ne suis pas très bien accueilli aux entraînements de duel de Serpentard, je n'ai pas appris les gestes du sortilège de la Mort - devrais-je apprendre ces arts pour mieux vous servir, seigneur~?

--- Je continue d'insister~: je ne suis pas ton seigneur, dit Harry.

--- Oui, seigneur.

--- Quoi que, dit Harry, et ce n'est en aucun cas un ordre, juste une remarque, tout le monde devrait savoir se défendre, surtout toi. Je suis sûr que le professeur de Défense t'aiderait par principe, si tu le lui demandais.~>>

Lesath Lestrange s'inclina et dit~: <<~Oui, seigneur, je suivrais vos ordres si j'en suis capable, seigneur.~>>

Harry se serait plaint d'être incompris s'il n'avait pas été parfaitement compris.

Lesath parti.

Harry regarda le mur.

Il avait honnêtement cru avoir déjà trouvé toutes les façons qu'il avait eues d'être stupide, après une demi-journée de réflexion à ce sujet.

Apparemment, cela n'avait encore été qu'un excès de confiance.

\emph{Est-ce qu'on comprend ce qu'on a fait de mal~?} dit froidement son côté Serpentard.

\emph{Oui}, pensa Harry.

\emph{Tes scrupules moraux n'ont aucun sens. Tu ne mens pas à Lesath. Tu as fait exactement ce que Lesath pense que tu as fait. Tu n'aurais même pas à inventer d'excuse si tu devais expliquer pourquoi Lesath t'aide, tu pourrais juste dire que c'est un paiement de la dette de l'avoir sauvé des brutes, il y avait six témoins. Hermione est morte parce que tu as oublié une ressource très importante, et tu as oublié Lesath parce que… quoi~?}

\emph{Parce que avoir Lesath Lestrange comme laquais faisait trop Seigneur des Ténèbres~? Dit Poufsouffle d'une petite voix intérieure. Enfin… c'est probablement moi qui ai pris cette décision…}

Le côté Serpentard de Harry ne répondit pas par des mots, il ne fit qu'irradier du mépris et envoya une image du corps de Hermione.

\emph{Arrête~!} cria mentalement Harry.

\emph{La prochaine fois dit Serpentard d'une voix de glace, je suggère que l'on passe plus de temps à se soucier de ce qui est efficace et productif et moins de temps à se soucier de ce qui fait Seigneur des Ténèbres.}

\emph{Bien reçu}, pensa Harry, \emph{c'est ce que je ferai}.

\emph{Non, tu ne le feras pas}, dit Serpentard. \emph{Tu trouveras d'autres façons de rationaliser tes scrupules mesquins. Tu commenceras à m'écouter après la mort de ton prochain ami.}

Harry commença à se demander s'il ne devenait pas fou. Les conversations qu'il entretenait avec les voix dans sa tête ne se déroulaient habituellement pas comme ça.

Le Survivant

\emph{douleur}

Harry Verres marchait lourdement, seul

\emph{ça fait mal}

Harry continua d'avancer dans les couloirs silencieux.

\later

<<~Comment va M. Potter~?~>> s'enquit le professeur Quirrell. Il y avait une tension chez cet homme, on n'aurait pas tout à fait pu l'appeler de la préoccupation, plutôt la tension d'un homme en embuscade jaugeant le moment où s'abattre. Les Granger venaient à peine de partir, accompagnés de Madame Pomfresh, quand le professeur de Défense avait toqué à la porte de son bureau et était entré sans attendre la réponse, puis avait parlé avant qu'elle ne puisse prononcer un mot. Une partie de Minerva se demanda distraitement si Harry Potter avait pris cette habitude chez le professeur de Défense, l'habitude de ne pas être conscient de la douleur des autres quand il avait quelque chose à l'esprit, ou si ce n'était qu'un défaut d'enfant dont cet homme n'avait jamais su se défaire.

<<~M. Potter a cessé de monter la garde devant le corps de Mlle Granger~>>, dit-elle, mettant dans sa voix une partie de la froideur qu'elle ressentait. Elle était certaine que le professeur de Défense ne ressentait pas autant de peine qu'elle, l'homme n'avait pas dit un mot au sujet de Hermione Granger. Que lui présente ses exigences à elle… <<~Je crois qu'il est descendu dîner.

--- Je ne vous parle pas de son état physique~! Avez-vous… a-t-il…~>> Le professeur Quirrell fit un geste vif, comme pour communiquer un concept pour lequel il n'avait pas de mot.

<<~Pas particulièrement~>>, dit-elle. Elle était à environ trente secondes d'ordonner au professeur de Défense de quitter son bureau.

Le professeur Quirrell commença à déambuler dans l'espace réduit du bureau.

<<~Il n'y avait qu'aux inquiétudes de Mlle Granger qu'il se souciait de répondre… maintenant qu'elle est partie… il n'y a plus de limite à son inconscience. Je le comprends à présent. Qui d'autre est là~? M. Londubat~? M. Potter ne prétend pas qu'ils sont égaux. Flitwick~? Son sang de gobelin ne crierait que vengeance. M. Malfoy, s'il revenait~? Dans quel but~? Rogue~? Un désastre ambulant. Dumbledore~? Bah. Les événements sont déjà en place pour une catastrophe, ils faut les manœuvrer vers une voie qu'ils ne prendraient pas naturellement. Qui M. Potter écouterait-il, qui ne lui parle pas d'ordinaire~? Cédric Diggory lui a donné des cours, mais quel conseil M. Diggory pourrait-il donner~? Une inconnue. M. Potter a passé de longs moments avec Remus Lupin. J'ai prêté peu d'attention à celui-ci. Lupin saurait-il quels mots prononcer, comment agir, quel sacrifice faire pour modifier la trajectoire du garçon~?~>> Le professeur Quirrell pivota pour lui faire face. <<~Remus Lupin réconfortait-il les endeuillés ou raisonnait-il ceux décidés à agir inconsciemment lorsqu'il faisait partie de l'Ordre du Phénix~?

--- Ce n'est pas une mauvaise idée, dit-elle lentement. Je crois que M. Lupin a souvent été la voix de la raison de James Potter lorsqu'ils étaient à Poudlard.

--- James Potter, dit le professeur Quirrell en plissant les yeux. Les garçon ressemble peu à James Potter. Êtes-vous confiante dans le succès de ce plan~? Non, c'est la mauvaise question, nous ne sommes pas limités à un seul plan. Êtes-vous certaine que ce plan suffira, que nous n'avons pas à en tenter d'autre~? Posée ainsi, la question est sa propre réponse. Le chemin qui mène au désastre doit être évité à chaque point d'intervention possible.~>> Le professeur de Défense avait recommencé à déambuler dans les confins du bureau, atteignant un mur, pivotant sur ses talons, marchant jusqu'à l'autre.

<<~Je vous prie de m'excuser, professeur~>>, elle ne se fatigua pas à masquer le tranchant de sa voix, <<~mais j'ai largement atteint mes limites pour aujourd'hui. Vous pouvez partir.

--- Vous.~>> le professeur Quirrell pivota et elle se retrouva à regarder directement ces yeux d'un bleu de glace. <<~Vous seriez la première à laquelle j'aurais pensé, après Mlle Granger, pour faire abandonner quelque idée folle au garçon. Avez-vous déjà fait votre maximum~? Bien sûr que non.~>>

Comment osait-il suggérer cela.

<<~Si vous n'avez rien d'autre à dire, professeur, vous allez partir.

--- Votre conjuration a-t-elle deviné qui je suis vraiment~?~>> les mots furent prononcé d'un ton neutre trompeur.

<<~À vrai dire, oui. Maintenant…~>>

De la magie pure, de la puissance pure vinrent s'écraser dans la pièce comme un éclair, comme un claquement de tonnerre qui fit écho dans ses oreilles et assourdi tous ses autres sens, les papiers sur son bureau soufflés, pas par quelque vent invoqué mais par la force pure d'un obscur pouvoir.

Puis le pouvoir se retira, ne laissant que le certificat de décès de Hermione Granger flotter dans les airs et descendre jusqu'au sol.

<<~Je suis David Monroe, et j'ai combattu Voldemort, dit l'homme d'un ton neutre. Écoutez mes paroles. On ne peut laisser le garçon poursuivre dans cet état d'esprit. Il deviendra dangereux. Il est possible que vous ayez déjà fait tout ce qui est en votre pouvoir. Mais j'ai découvert que c'était là quelque chose de très rare, plus tôt dit que fait. Je pense d'ailleurs que vous n'avez fait que ce que vous faites d'habitude. Je ne puis vraiment comprendre ce qui pousse les autres à dépasser leurs limites puisque je n'en ai jamais eu. Les gens demeurent étonnamment passifs face à la perspective de mourir. La peur d'être ridiculisé en public ou d'être désargenté risque plus de pousser les hommes à leurs extrémités et à briser leurs habitudes. De l'autre côté de la guerre, le Seigneur des Ténèbres a obtenu d'excellents résultats au moyen du sortilège Doloris, judicieusement utilisé sur des serviteurs Marqués qui ne pourraient échapper à la punition sauf en réussissant, sans qu'aucun 'effort raisonnable' ne puisse être accepté. Imaginez-vous dans leur état d'esprit et demandez-vous si vous avez vraiment fait tout ce qui est en votre pouvoir pour arracher Harry Potter à la voie sur laquelle il se trouve.

--- Je suis une Gryffondor et peu encline à être motivée par la peur, rétorqua-t-elle. Vous serez plus courtois dans mon bureau~!

--- Je trouve que la peur est une excellente motivation, et c'est d'ailleurs la peur qui me fait maintenant agir. Vous-Savez-Qui, pour tout son horreur, respectait quand même certaines limites. J'estime, en tant que sorcier accompli presque au niveau de Dumbledore ou de Celui-Dont-Il-Ne-Faut-Pas-Prononcer-Le-Nom, que le garçon pourrait rejoindre les rangs de ceux dont les rituels sont inscrits sur les tombes de pays entiers. Ce n'est pas une crainte sans fondement, McGonagall~; j'ai déjà entendu des paroles susceptibles d'éveiller la plus terrible des appréhensions.

---Êtes-vous fou~? Vous pensez que M. Potter pourrait… c'est ridicule. M. Potter ne pourrait jamais…~>>

L'image muette d'un morceau de verre sur une sphère d'acier lui traversa l'esprit.

<<~… M. Potter ne ferait pas une chose pareille~!

--- Son choix conscient n'est pas nécessaire. Les sorciers se décident rarement à provoquer leur propre perte. M. Potter ne vous semble peut-être pas mal intentionné. Vous semble-t-il inconscient, une fois qu'il s'est résolu à atteindre un but~? Je répète que j'ai des raisons précises de nourrir la pire des craintes possibles~!

--- En avez-vous parlé au directeur~? dit-elle lentement.

--- Ce serait pire qu'inutile. Dumbledore ne peut atteindre le garçon. Au mieux, il est assez sage pour le savoir et ne pas aggraver les choses. Quant à moi, je ne possède pas l'état d'esprit nécessaire. Vous êtes celle qui… mais je vois que vous attendez encore d'être secourue par d'autres.~>> Le professeur de Défense se détourna d'elle et marcha jusqu'à la porte. <<~Je pense que je vais consulter Severus Rogue. L'homme est peut-être un désastre ambulant, mais il le sait, et il possédera peut-être une meilleure idée de l'humeur du garçon. Quant à vous, madame, imaginez-vous à la fin de votre vie, sachant que l'Angleterre - mais non, l'Angleterre n'est pas votre vrai pays, n'est-ce pas~? Imaginez-vous à la fin de votre vie, alors que les ténèbres dévorent ce qui reste des murs de Poudlard, sachant que vos élèves mourront avec vous, vous souvenant de ce jour, comprenant qu'il y avait autre chose que vous auriez pu faire.~>>
%  LocalWords:  Lordish Pfah
