\namedpartchapter{Accomplissement de soi}{SA}{IX}{Intensification des conflits}

\lettrine{H}{arry} avança d'un pas, puis d'un autre, jusqu'à ce qu'un malaise s'empare de lui, comme un grondement dans ses nerfs.

Il ne dit rien, ne leva pas la main~; le sentiment d'inconfort parlerait pour lui.

Depuis l'autre côté de la porte fermée vint un murmure qui traversa celle-ci comme si elle n'avait pas été là.

<<~Ce ne sont pas mes heures de bureau, dit le froid murmure, ni l'heure de notre rendez-vous. Je vous enlève dix points Quirrell, et soyez heureux que ce ne soit pas plus.~>>

Harry demeura calme. Traverser Azkaban avait recalibré son échelle des perturbations émotionnelles~; et perdre un point, qui s'était autrefois élevé à cinq sur dix, rampait maintenant aux environs de zéro virgule trois. La voix de Harry était tout aussi basse lorsqu'il répondit~: <<~Vous avez énoncé une prédiction vérifiable et elle a été falsifiée, professeur. Je souhaitais uniquement noter cela.~>>

Lorsque Harry se retourna pour partir, il entendit la porte s'ouvrir derrière lui et il pivota, quelque peu surpris.

Le professeur Quirrell était enfoncé dans sa chaise, tête renversée contre le dossier, et un parchemin flottait devant lui. Ses deux mains étaient posées sur le bureau aussi mollement que si elles n'avaient pas été innervées. Il aurait pu être un cadavre exception faite de ses yeux bleu glacé qui bougeaient encore, de gauche à droite, de gauche à droite.

Le parchemin disparut et fut remplacé par un autre si rapidement qu'on aurait dit que l'objet avait seulement clignoté.

Puis les lèvres bougèrent elles aussi. <<~Et de cela, murmurèrent-elles, que déduisez-vous, M. Potter~?~>>

Harry fut secoué par l'apparence du professeur mais sa voix demeura calme lorsqu'il répondit~: <<~Que les gens ordinaires ne font pas toujours rien et que Hermione Granger est plus menacée par la maison Serpentard que vous ne le pensiez.~>>

Les lèvres se courbèrent imperceptiblement. <<~Et vous pensez donc que ma compréhension de la nature humaine a été mise en échec. Mais c'est loin d'être la seule possibilité, enfant. Voyez-vous l'autre~?~>>

Harry fronça les sourcils tout en regardant le professeur de Défense.

<<~Cela me fatigue, murmura le professeur de Défense. Vous resterez ici jusqu'à ce que vous ayez trouvé la réponse seul ou vous partirez.~>> Comme si Harry avait cessé d'exister, les yeux du professeur revinrent au parchemin et l'examinèrent du même mouvement saccadé.

C'est six parchemins plus tard que Harry comprit et dit~: <<~Vous pensez que votre prédiction a échoué parce qu'il y a un autre facteur en jeu qui n'était pas inclus dans votre modèle. Une raison pour laquelle la maison Serpentard hait plus Hermione que vous ne l'aviez perçu. Comme quand les calculs orbitaux d'Uranus faits par les scientifiques étaient faux mais que le problème ne venait pas des lois de Newton mais de leur ignorance de l'existence de Neptune -~>>

Le parchemin disparut et ne fut pas remplacé. La tête s'extirpa alors de son prélassement, fit face à Harry, et la voix qui parla fut basse mais pas sans timbre. <<~Je pense, enfant,~>> dit le professeur avec douceur mais d'une voix qui était presque normale, <<~que si tout Serpentard la haïssait tant que cela, je m'en serais rendu compte. Et pourtant, trois formidables combattants de cette maison ont préféré l'action à l'inaction, et ce d'une façon à la fois risquée et coûteuse. Quelle force aurait pu les mouvoir ou diriger leur mouvement~?~>> La lueur bleu glacé des yeux du professeur croisèrent le regard de Harry. <<~Quelque main dotée d'influence au sein de Serpentard, peut-être. Mais dans ce cas, comment cette main bénéficierait-elle du mal causé à la fille et à ses partisans~?

--- Euh… dit Harry. Il faudrait que ce soit quelqu'un qui, d'une façon ou d'une autre, se sent menacé par Hermione, ou quelqu'un qui recevrait le mérite de tout malheur qui pourrait lui arriver. Je ne connais personne correspondant à ce profil, mais cela dit, je ne connais quasiment personne à Serpentard qui ne soit pas en première année.~>> La pensée vint aussi à Harry que déduire l'existence d'un cerveau caché à partir d'une attaque moyennement inattendue semblait se baser sur des éléments de preuves insuffisants pour compenser l'improbabilité à priori de la théorie~; mais enfin, puisqu'il s'agissait d'une déduction du \emph{professeur Quirrell}…

Le professeur de Défense se contentait de regarder Harry, paupières légèrement baissées, comme en signe d'impatience.

<<~Et oui, dit Harry, je \emph{suis} certain que Drago Malfoy n'est pas derrière ça.~>>

Un sifflement d'air expiré, comme un soupir. <<~Il est le fils de Lucius Malfoy, entraîné selon les règles les plus strictes. Quoi que vous ayez vu de lui, même dans ce qui semblait être des moments à découverts où son masque glissait et où vous croyiez avoir vu la vérité derrière celui-ci~: tout cela peut aussi bien faire partie du visage qu'il choisit de vous montrer.~>>

\emph{Seulement si Drago est capable de réussir à lancer un Patronus uniquement pour garder sa couverture}. Mais Harry ne dit bien sûr pas cela à voix haute~; il eut juste un léger sourire et dit~: <<~Donc soit vous n'avez \emph{vraiment} jamais lu dans l'esprit de Drago, soit c'est juste ce que vous voulez que je crois.~>>

Il y eut un silence. L'une des mains se retourna et, d'un doigt, fit signe d'approcher.

Harry entra dans la pièce. La porte se referma derrière.

<<~Ce n'est pas là quelque chose que vous auriez dû dire à voix haute en langue humaine, dit la douce voix du professeur Quirrell. Légilimancie sur l'héritier Malfoy~? Si Lucius Malfoy apprenait cela, il me ferait assassiner immédiatement.

--- Il \emph{essaierait}~>>, dit Harry. Cela aurait dû lui valoir un plissement d'yeux de la part du professeur mais le visage de ce dernier resta immobile. <<~Mais pardon~>>.

Lorsque le professeur parla de nouveau, sa voix était redevenue un froid murmure. <<~J'imagine que je pourrais, et j'aurais pitié de l'assassin.~>> Sa tête se renversa contre la chaise et tomba d'un côté. Ses yeux ne croisaient plus ceux de Harry. <<~Mais ces petits jeux éveillent à peine mon intérêt tels quels. Ajoutez la Légilimancie et cela cesse même d'être un jeu.~>>

Harry ne savait absolument pas quoi dire. Il avait vu le professeur Quirrell en colère une fois ou deux avant, mais l'émotion actuelle semblait plus vide et Harry ne savait pas comment y répondre. \emph{Qu'est-ce qui ne va pas, professeur Quirrell} \emph{?} cela, il ne pouvait pas le lui demander.

<<~Qu'est-ce qui \emph{éveille} votre intérêt~?~>> dit Harry quelques instants plus tard après avoir conçu cette stratégie apparemment plus sûre destinée à rediriger l'attention du professeur vers des choses plus positives. Citer des résultats expérimentaux quant aux bienfaits qu'il y avait à tenir un journal de gratitude pour augmenter son niveau de bonheur n'aurait probablement pas été bien reçu.

<<~Je vous dirai ce qui n'éveille pas mon intérêt, dit ce froid murmure. Noter des essais mandatés par le ministère. Mais j'ai pris le poste de professeur de Défense à Poudlard et je mènerai ma tâche à bien.~>> Un autre parchemin apparut devant la tête du professeur Quirrell et ses yeux commencèrent à le scanner. <<~Reese Belka avait un poste élevé dans mes armées avant qu'elle ne commette sa folie. Je lui offrirai une chance de rester plutôt que d'être expulsée si elle me dit exactement quelles sont les forces qui l'ont poussée à agir. Et je lui ferai clairement comprendre ce qui se passera si elle ment. Je m'autorise quand même à lire les visages.~>>

Le doigt du professeur de Défense pointa derrière Harry, vers la porte.

<<~Mais que vous ayez eu tort quant à la nature humaine, dit Harry, ou qu'il y ait une force supplémentaire à l'œuvre dans Serpentard - dans un cas comme dans l'autre, Hermione Granger court un danger plus grand que vous ne l'aviez prédit. La dernière fois, c'était trois puissants combattants, alors que se passera-t-il la prochaine -

--- Elle ne désire ni mon aide ni la vôtre, dit une voix basse et froide. Je ne trouve plus vos préoccupations aussi amusantes qu'autrefois, M. Potter. Partez.~>>

\later

Elles avaient beau être toutes égales, c'était toujours Hermione qui finissait par s'exprimer dans ces situations, même si elle n'était certainement pas la chef.

Ce jour là, lors du petit déjeuner, les quatre tables de Poudlard regardaient en coin là où les huit membres de la S.P.E.H.S. s'étaient rassemblées, à l'écart.

Le professeur Flitwick les regardait lui aussi d'un regard sévère depuis la table d'honneur. Hermione ne regardait pas dans cette direction mais elle pouvait sentir le regard du professeur Flitwick posé sur la base de son cou. Elle pouvait le sentir au sens \emph{propre}. C'était vraiment effrayant.

<<~M. Potter, pourquoi avez-vous dit à Tracey que vous vouliez nous parler~? dit Hermione d'un ton brusque.

--- Le professeur Quirrell a expulsé Reese Belka de son armée la nuit dernière, dit Harry Potter. Et de toutes les autres activités de Défense du soir. L'une d'entre vous comprend-t-elle le sens de cela~? Mademoiselle Greengrass~? Padma~?~>>

Les yeux de Harry les passèrent en revue, Hermione échangea un regard perplexe avec Padma et Daphné secoua la tête.

<<~Bon, dit Harry à voix basse, je m'y attendais. Ce que ça veut dire, c'est que vous êtes en danger, mais je ne sais pas à quel point.~>> Le garçon redressa les épaules et regarda droit dans les yeux de Hermione. <<~Je ne comptais pas dire ça mais… je voudrais juste vous offrir de vous mettre sous ma protection, quelle que soit l'étendue de celle-ci. Vous pourrez clairement faire comprendre que tous ceux qui s'en prennent à vous s'en prennent aussi au Survivant.

--- Harry~! dit Hermione d'un ton sec. Tu \emph{sais} que je ne veux pas -

--- Certaines ici sont aussi \emph{mes} amies, Hermione.~>> Harry ne détacha pas ses yeux des siens. <<~Et c'est leur décision, pas la mienne. Padma~? Tu m'as dit que malgré ce que j'avais fait, je n'avais aucune dette envers toi~; c'est le genre de chose qu'un ami dirait.~>>

Hermione arracha son regard à celui de Harry pour se tourner vers Padma, qui secouait la tête.

<<~Lavande~? dit Harry. Tu t'es bien battue dans mon armée, et je me battrai pour toi si tu le souhaites.

--- \emph{Merci bien}, général~! dit Lavande avec rudesse. Je veux dire M. Potter. Mais non. Je suis une héroïne ainsi qu'une Gryffondor et je peux me défendre toute seule.~>>

Il y eut un silence.

<<~Parvati~?~>> dit Harry. <<~Susan~? Hannah~? Daphné~? Je ne vous connais pas aussi bien mais je pense que j'offrirais cela à toute personne qui viendrait me le demander.~>>

L'une après l'autre, les quatre autres filles secouèrent la tête.

Hermione perçut alors ce qui allait se produire, mais elle n'eut pas l'ombre d'une idée pour l'éviter.

<<~Et mon loyal soldat, Tracey du Chaos~? dit Harry Potter.

--- \emph{Vraiment~?}~>> s'étrangla Tracey, insouciante des regards assassins que Hermione et toutes les autres filles jetaient sur elle. Les mains de Tracey volèrent avec grâce jusqu'à ses joues mais elle ne parvint pas tout à fait à rougir, du moins rien que Hermione ne put détecter~; et ses yeux marrons, s'ils n'étincelaient pas, étaient au moins très écarquillés. <<~Vous feriez cela~? Pour \emph{moi}~? Enfin - Enfin, bien sûr, absolument, général Chaos -~>>

\later

Et c'est donc ce matin là que Harry Potter se rendit à la table de Gryffondor, puis à la table Serpentard, et dit aux deux maisons que tous ceux qui feraient du mal à Tracey Davis, peu importe ce qu'elle faisait à ce moment là, allaient, je cite, apprendre le véritable sens du mot chaos, fin de citation.

C'est grâce à une maîtrise de lui-même considérable que Drago Malfoy parvint à s'empêcher de frapper sa tête plusieurs fois de suite contre son assiette remplie de pain grillé.

Les brutes de Poudlard n'étaient pas exactement des scientifiques.

Mais elles allaient vouloir tester ça.

\later

La Société Pour la Promotion de l'Égalité Héroïque des Sorcières ne l'avait pas \emph{annoncé} car il n'était pas clair qu'une \emph{annonce} ferait beaucoup de bien. Mais elles avaient toutes tranquillement décidé (ou, dans le cas de Lavande, accepté à force de subir les hurlements des autres filles) de s'absenter de la lutte contre les brutes pendant quelques temps, au moins jusqu'à ce que leur directeur de maison ne les regarde plus aussi sévèrement et que les élèves plus âgés aient cessé de pousser Hermione contre les murs.

Daphné avait \emph{dit} à Millicent qu'elles faisaient une pause.

Et c'est donc avec une certaine perplexité que, quelques jours plus tard, Daphné regarda le parchemin qui lui avait été délivré pendant le déjeuner, écrit d'une main si tremblante qu'il en était presque illisible, et sur lequel on pouvait lire~:

\emph{Deux heures cet après-midi en haut des escaliers qui montent de la bibliothèque VRAIMENT IMPORTANT tout le monde doit être là - Millicent}

Daphné regarda autour d'elle mais elle ne put voir Millicent nulle part dans la grande salle.

<<~Un message de ton informateur~? dit Hermione après que Daphné fut allée lui parler. C'est bizarre - \emph{je} n'ai pas -

--- Tu n'as pas quoi~?~>> dit Daphné lorsque la Serdaigle s'arrêta à mi-phrase.

Le général Soleil secoua la tête et dit~:

<<~Écoute Daphné, je pense qu'on doit savoir d'où viennent ces messages avant de continuer à leur faire confiance. Regarde ce qui s'est passé la dernière fois~: comment est-ce que quiconque aurait pu \emph{savoir} qu'il y aurait trois brute, à moins de faire partie du complot~?

--- Je ne peux pas te - dit Daphné. Enfin, je ne peux rien dire, mais je sais d'où viennent les messages et je sais comment cette personne aurait pu le savoir.~>>

Hermione jeta un \emph{regard} à Daphné qui, l'espace d'un instant, rendit Hermione effroyablement semblable au professeur McGonagall.

<<~Hmm, dit Hermione. Et sais-tu comment Susan s'est soudain transformée en Wonderwoman~?~>>

Daphné secoua la tête et dit~: <<~Non, mais je pense que si on reçoit un message disant qu'on devrait être quelque part, c'est peut être \emph{vraiment important} et qu'on \emph{devrait tous y aller}.~>> Daphné n'avait pas \emph{vu} ce qui était arrivé à Susan après qu'elle eut essayé d'empêcher la prophétie en maintenant cette dernière à distance. Mais on le lui avait \emph{raconté} ensuite, et maintenant Daphné avait peur d'avoir…

Peur d'avoir peut-être…

D'avoir peut-être \emph{cassé} quelque chose.

<<~Hmm~>>, dit Hermione, et ses yeux McGonagall étaient revenus.

\later

Personne ne sembla savoir qui l'avait commencé ni où cela avait commencé. Si vous aviez essayé d'en suivre ensuite la trace, mot à mot, murmure à murmure, vous auriez probablement découvert une immense boucle.

Ce matin là, Peregrine Derrick reçut une tape sur l'épaule en sortant du cours de potions.

Jaime Astorga entendit un chuchotement derrière lui pendant le déjeuner.

Robert Jugson III découvrit un petit mot plié sous son assiette.

Carl Sloper surprit une conversation entre deux Gryffondor qui lui jetèrent des regards lourds de sous-entendus lorsqu'ils passèrent près de lui.

Personne ne semblait savoir où cela avait commencé, qui en avait parlé en premier, mais on parlait d'un lieu, on parlait d'une heure, et on disait que la couleur serait le blanc.

\later

<<~Chacune d'entre vous a intérêt à ne rien trouver à redire à ça~>>, dit Susan Bones. La Poufsouffle, ou le pouvoir étrange qui avait pris possession d'elle, ne faisait même plus \emph{semblant} d'être normal. La fille au visage rondouillard traversait les couloirs d'un pas ferme et confiante. <<~Si nous arrivons là et qu'il y a juste une seule brute, pas de problème, vous pouvez vous battre contre elle normalement. Mes mystérieux superpouvoirs ne s'activeront que s'il y a des innocents en danger. Mais si cinq brutes de septième année sortent d'un placard, vous savez quoi faire~? C'est ça, vous \emph{fuyez} et vous me laissez les combattre. Il est facultatif de trouver un professeur, ce qui est important c'est que vous \emph{fuyiez} dès que je créé une ouverture. Dans un combat de ce genre, vous êtes sous ma \emph{responsabilité}, vous êtes des \emph{civils} \emph{en danger} que je dois penser à protéger. Donc vous partirez le plus vite possible et vous \emph{n'essaierez pas} de faire \emph{quoi que ce soit} d'héroïque ou je vous jure, à la seconde où vous sortirez du lit du guérisseur, je vous rendrai visite \emph{moi-même} et je vous \emph{botterai les fesses} si fort que vous y retournerez. On est toutes d'accord là-dessus~?

--- Oui,~>> couinèrent la plupart des filles, même si Hannah dit~: <<~Oui, Dame Susan~!

--- Ne m'appelez \emph{pas} comme ça, cracha Susan. Et \emph{je ne crois pas vous avoir entendu, Mlle Brown}~! Je vous préviens, j'ai des amis qui écrivent des pièces, et si vous faites quoi que ce soit de stupide, vous resterez dans l'histoire comme 'Lavande, l'incroyable otage stupide.'~>>

(Hermione commençait à se demander avec inquiétude combien d'autre élèves mis à part Harry avaient de mystérieux côtés obscur et si \emph{elle} risquait d'en développer un à force de traîner avec eux).

<<~Très bien, capitaine Bones~>>, dit Lavande d'un ton exceptionnellement respectueux alors qu'elles passaient un angle sur le chemin le plus court vers la bibliothèque et qu'elles traversaient un couloir plutôt grand muni de six ensembles de doubles portes, trois de chaque côté. <<~Pourrais-je savoir s'il y a la moindre chose que \emph{je} pourrais faire pour devenir une double sourcière~?

--- Inscris-toi au programme de préparation Auror quand tu seras en sixième année, dit Susan. C'est ce qu'il y a de mieux en second choix. Oh, et si un Auror célèbre te propose de superviser ton stage d'été, contente-toi de faire la sourde oreille à tous ceux qui te diront qu'il a une mauvaise influence et que tu vas presque certainement mourir.~>>

Lavande hochait rapidement la tête. <<~Compris, compris.~>>

(Padma, qui n'avait à vrai dire pas été là la dernière fois, lançait des regards \emph{très} sceptiques à Susan).

Puis cette dernière se figea sur place, leva sa baguette et dit~: <<~\emph{Protego Maximus~!}~>>

Une décharge d'adrénaline traversa Hermione, elle tira instantanément sa baguette, pivota -

Mais elle ne vit rien qui clochait de l'autre côté du grand nuage bleu qui les entourait à présent.

Les autres filles, qui s'étaient elles aussi mises en formation, avaient l'air tout autant perplexes.

<<~Désolé~! dit Susan. Désolé les filles. Laissez-moi un moment pour inspecter cet endroit. Repenser à quelqu'un m'a rappelé que cette pièce où nous nous trouvons, avec toutes ces portes, serait un excellent coin à embuscade.~>>

Il y eut un silence.

<<~Maintenant~>>, dit une voix dure, brouillée jusqu'à ne plus être identifiable par un grésillement.

Les six paires de portes s'ouvrirent grand.

Des robes blanches avancèrent en file, en silence, des robes intégrales sans marque d'affiliation à une maison ou à une autre et munies d'un tissu blanc qui masquait les visages surmontés de capuches. Elles se déversèrent, emplissant le grand couloir pour atteindre un nombre trop élevé pour être facilement compté. Probablement moins de cinquante. Certainement plus de trente. Toutes déjà entourées d'un nuage bleu.

Susan dit de très gros mots, si terribles qu'à n'importe quel autre moment, Hermione les aurait remarqués.

<<~Ce message~!~>> s'écria Daphné, prise d'une horreur soudaine. <<~Il ne venait \emph{pas} de -

--- Millicent Bulstrode~? dirent la voix et son grésillement. Non. Vous voyez, Mlle Greengrass, si la même fille envoie un message par le réseau Serpentard à chaque fois que vous combattez une brute, quelqu'un finira par le remarquer. Nous aurons une conversation avec elle quand nous en auront fini avec vous.

--- Mlle Susan, dit Hannah d'une voix qui commençait tout juste à trembler, pourriez-vous être assez Wonderwoman pour -~>>

Des baguettes s'élevèrent. Puis il y eut une série d'éclairs verts aveuglants, une volée massive de briseurs de boucliers, au terme de laquelle plus aucun dôme bleu n'entoura les filles, et Susan tomba à genoux, main serrée contre sa baguette.

Des barrières de ténèbres solides avaient jailli à chaque extrémité du couloir. Derrière les doubles portes que Hermione pouvait voir ne se trouvaient que des salles de cours vides, des voies sans la moindre issue.

<<~Non, dit la voix masculine surmontée d'un grésillement, elle ne peut pas. Au cas où vous ne l'auriez pas remarqué, vous avez énervé pas mal de monde et nous n'avons aucune intention de perdre cette fois-ci. Très bien, tout le monde, préparez-vous à tirer.~>>

Les baguettes qui les entouraient furent de nouveau pointées, assez bas pour que leurs ennemis ne s'atteignent pas les uns les autres en ratant leurs coups.

Puis une autre voix mâle, accompagnée d'un grésillement similaire, dit soudain~: <<~\emph{Hominum Revelio~!}~>>

Et un instant plus tard une autre volée de maléfices et de briseurs de boucliers fut lancée par réflexe sur la silhouette qui avait soudain été dévoilée, fracassant les boucliers qui avait presque immédiatement commencé à se former autour d'elle -

Et alors, tandis que cette même silhouette s'effondrait au sol, un silence estomaqué.

<<~\emph{Le professeur Rogue~?} dit la seconde voix. C'est \emph{lui} qui interférait~?~>>

C'était le professeur de potions de Poudlard qui était à présent étendu par terre, inconscient, sa robe poussiéreuse achevant son mouvement avant de se déposer au sol, sa main tendue vers sa baguette qui s'éloignait lentement en roulant.

<<~Non~>>, dit la première voix mâle, maintenant un peu moins assurée. Puis il se ressaisit~: <<~Non, c'est impossible. Il nous a entendu nous faire passer le mot, bien sûr, et il est venu pour s'assurer que personne ne ferait de nouveau tout foirer. Nous le réveillerons ensuite, nous nous excuserons et il lancera un sortilège d'Oubliettes sur les enfants afin qu'ils ne se souviennent de rien, c'est un professeur, il en est capable. Quoi qu'il en soit, nous devrions nous assurer d'être \emph{vraiment} seuls maintenant. \emph{Veritas Oculum}~!~>>

Deux douzaines de charmes furent alors prononcées, mais plus aucun invisible ne fut révélé. L'un d'entre eux en particulier étreint le cœur de Hermione~: elle reconnu le charme qui avait figuré à côté de la description de la Véritable Cape d'Invisibilité, un charme qui ne révélerait pas la Cape mais dirait si cet artefact ou d'autres semblables étaient dans les environs.

<<~Les filles~?~>> chuchota Susan. Elle se mit lentement sur pieds, même si Hermione put voir ses membres vaciller et trembler. <<~Les filles, je suis désolé d'avoir dit ça plus tôt. S'il y a quoi que ce soit d'héroïque ou de malin que vous voulez essayer, autant le faire.

--- Ah, ouais, dit alors Tracey Davis d'une voix tremblante. J'ai failli \emph{oublier}.~>> La Serpentard leva la main et parla.

<<~Hé, vous tous~! hurla Tracey d'une voix aiguë. Hé, est-ce que vous comptez me faire du mal à moi aussi~?

--- À vrai dire, oui dit la voix grésillante du chef. C'est notre intention.

--- Je suis sous la protection de Harry Potter, vous savez~! Tous ceux qui essaient de me faire du mal apprendront le véritable sens du mot chaos~! Alors vous allez me laisser partir~?~>> Ç'aurait dû être un ton de défi. Il indiquait plutôt de la terreur.

Il y eut un silence. Certaines des capuches se tournèrent pour se faire face puis revinrent aux filles.

<<~Euh… dit la voix masculine grésillante. Euh… non.~>>

Tracey Davis rangea sa baguette dans ses robes.

Lentement, d'un geste délibéré, elle leva une main haut dans les airs et appuya son pouce contre son majeur.

<<~Vas-y~>>, dit la voix.

Tracey Davis claqua des doigts.

Il y eut un long et terrible silence.

Rien ne se passa.

<<~Oui, eh bien~>>, dit la voix -

Tracey dit alors d'une voix encore plus aiguë et encore plus tremblante~: <<~Acathla, mundatus sum.~>> Sa main s'étira encore plus haut et elle claqua des doigts une deuxième fois.

Un frisson indicible parcourut l'échine de Hermione, un frisson de peur et de désorientation, comme si elle venait de sentir le sol s'incliner sous ses pieds et menacer de la faire glisser vers quelque ténèbre cachée au-dessous.

<<~Qu'est-ce qu'elle -~>> commença une voix féminine grésillante.

Le visage de Tracey était pâle, déformé par la peur, mais ses lèvres bougeaient et de celles-ci émergeait le son d'un chant aigu~: \emph{<<~Mabra, brahoring, mabra…}~>>

Un vent froid sembla apparaître aux confins du couloir, un souffle noir qui caressa leur visage et glaça leurs mains.

<<~Tirez sur elles à mon signal~! dit la voix en chef. Un, deux \emph{trois}~!~>> et peut-être quarante-deux voix rugirent des sortilèges, créant une immense nuée concentrique de tirs flamboyants qui éclairèrent le large couloir avec plus de force que le soleil -

- pendant le court instant qui précéda celui où les tirs s'évanouirent en heurtant un octogone rouge sombre qui apparut dans les airs devant les filles et disparut un moment plus tard.

Hermione le vit mais elle ne put le concevoir, elle n'arrivait pas à imaginer un sortilège de bouclier aussi puissant, un sortilège capable de résister à une armée.

La voix de Tracey continua de chanter, plus forte et plus confiante, et son visage était déformé comme si elle essayait de se souvenir de mots \emph{précis}~:

<<~\emph{Brouille, drouille, muzo, muffe.\\
Fista, ouista, mista-cuffe.}~>>

Tous ceux présents pouvaient maintenant le sentir, les héroïnes comme les brutes, la sensation d'une noirceur qui les entourait, se pressait contre eux comme un picotement dans les airs à mesure que quelque chose montait, montait et montait. Tous les nuages bleus qui entouraient les robes blanches et tous les boucliers s'étaient éteint sans qu'aucun sortilège visible ne vienne les frapper. Il y eut d'autres éclats de lumière lorsque des sortilèges désespérés furent lancés mais ils moururent en crépitant à mi-parcours comme une flamme de bougie qui aurait touché de l'eau.

Les barrières noires à chaque extrémité du couloir étaient parties en fumée comme si une pression immense les avait vaporisées mais leur évaporation révéla des sorties scellées, bloquées par des lames empilées faites d'un métal noir qui semblait avoir été taché de sang~; et lorsque Tracey chanta <<~\emph{Lermarchand, Lamente, Lemarchand}~>>, une terrible lumière bleutée commença à filtrer entre et sous les lames de métal, les six paires de portes se refermèrent d'un grand coup et des brutes en robes blanches paniquées commencèrent à tambouriner dessus et à hurler.

Puis la baguette de Tracey fit un grand geste vers la gauche et elle s'écria~: <<~\emph{Khornath~!}~>>, puis derrière elle~: <<~\emph{Slaaneth~!}~>>, au-dessus d'elle~: <<~\emph{Nurgolth~!}~>> puis, vers la droite~: <<~\shout{Tzintchi~!}~>>

Tracey s'interrompit et prit une profonde inspiration, Hermione retrouva alors sa voix et s'écria~: <<~\emph{Arrête, Tracey, arrête-toi~!}~>>

Mais il y avait un grand et étrange sourire sur le visage de Tracey. Elle leva sa baguette encore plus haut et claqua des doigts une troisième fois~; et lorsqu'elle parla de nouveau, sous sa voix aiguë et féminine on crut entendre le murmure d'un chœur plus grave qui l'accompagnait.

<<~\emph{Ténèbre au-delà des ténèbres, plus profonde que la dernière obscurité.\\
Enterrée sous le flot du temps…\\
De ténèbre en ténèbre, ta voix fait écho au cœur du néant,\\
Ignorée de la mort, inconnue de la vie.}

--- \emph{Qu'est-ce que tu fabriques~?}~>> glapit Parvati, et la Gryffondor tendit une main comme pour tirer la Serpentard vers elle, mais celle-ci commençait à s'élever en l'air~; Daphné et Susan attrapèrent toutes deux le bras de Parvati en même temps et Daphné s'écria: <<~Non, on ne sait pas ce qui se passe quand le rituel est interrompu~!

--- \shout{Ouais et qu'est-ce qui se passe s'il est \scream{accompli}~?}~>> hurla Hermione, plus proche qu'elle ne l'avait jamais été d'une défaillance mentale complète.

Le visage de Susan était blanc comme de la craie et elle murmura~: <<~Je suis désolée, Fol-Œil…~>>

Et Tracey continua, son corps flottant de plus en plus haut, ses cheveux noirs fouettant l'air autour d'elle, portés par les vents froids.

<<~\emph{Toi qui connais la porte, qui es la porte, la clé et le gardien de la porte~:\\ Je t'ordonne de lui ouvrir la voie et de manifester son pouvoir devant moi~!}~>>

Le couloir fut alors plongé dans un noir complet et un silence absolu, si bien que seule Tracey pouvait être vue et entendue, comme si plus rien dans l'univers n'existait à part elle et la lumière qui l'éclairait, venue d'une source sans nom.

La fille lumineuse leva la main une dernière fois et avec une affreuse solennité appuya son pouce contre son majeur.

Et depuis les ténèbres Hermione regarda le visage de Tracey et vit que les yeux de la Serpentard avaient maintenant, à la nuance près, le vert des yeux de Harry Potter.

<<~\emph{Harry James Potter-Evans-Verres~!\\
Harry James Potter-Evans-Verres~!\\
HARRY JAMES POTTER-EVANS-VERRES~!}~>>

Il y eut un claquement de tonnerre, et alors -

\later

Harry avait choisi une posture détendue, assit sur une petite chaise face à l'imposant bureau du directeur de Poudlard~: une jambe posée sur son genoux, ses bras nonchalamment étendus sur les accoudoirs. Harry faisait de son mieux pour ignorer le son venu des appareils alentours, même s'il avait quelque difficultés à ignorer celui situé directement derrière lui et qui faisait le bruit d'une chouette hululant désespérément alors qu'on la fourrait dans une déchiqueteuse à bois.

<<~Harry~>>, dit le vieux sorcier, assit derrière son bureau, la voix âgée et égale, ses yeux bleus perçants derrière ses scintillantes lunettes en demi-lune. Le directeur s'était habillé d'une robe d'un pourpre nocturne et non pas d'un noir formel mais toujours assez sombre pour approcher le sérieux le plus mortel selon la mode du monde des sorciers. <<~Es-tu… \emph{responsable} de ce qui a eu lieu~?

--- Je ne peux nier que mon influence ait joué un rôle~>>, dit Harry.

Le vieux sorcier ôta ses lunettes et se pencha pour regarder directement Harry, ses yeux bleus braqués vers les yeux verts. <<~Je te poserai une seule question~>>, continua-t-il d'une voix plus basse. <<~Penses-tu que ce que tu as fait aujourd'hui était - \emph{convenable}~?

--- Des brutes étaient là, et elles étaient arrivées dans ce couloir avec l'intention de faire du mal à Hermione Granger et aux sept autres enfants de première année, dit Harry d'une voix égale. Si je ne suis pas trop jeune pour être la cible de jugements moraux, alors eux non plus. Non, directeur, ils ne méritaient pas de mourir. Mais ils \emph{méritaient} d'être entièrement déshabillés et collés au plafond.~>>

Le vieux sorcier remit ses lunettes. Pour la première fois depuis que Harry le connaissait, les mots semblaient manquer au directeur. <<~Merlin lui-même m'en soit témoin, dit Dumbledore, je n'ai pas la plus vague idée de la façon dont je devrais réagir à cela.

--- C'est plus ou moins l'effet que j'espérais produire~>>, dit Harry. Il sentait qu'il aurait dû siffler un air gai, mais il n'avait malheureusement jamais appris à siffler de façon fiable.

<<~Je n'ai pas besoin de te demander qui est \emph{directement} responsable, dit le directeur. Seuls trois sorciers à Poudlard pourraient être assez puissants. Je n'y suis pour rien. Severus m'a assuré qu'il n'est pas impliqué. Et le troisième…~>> le directeur secoua la tête avec consternation. <<~Tu as prêté ta Cape au professeur de Défense, Harry. Je ne pense pas que ce soit sage. Car maintenant qu'il a échappé à la détection par de simples sortilèges, il sait certainement que c'est une Relique de la Mort - si, de fait, il ne l'a pas su à la seconde où il est entré en contact avec elle.

--- Le professeur Quirrell avait déjà déduit que je possédais une cape d'invisibilité, dit Harry. Et le connaissant, il a probablement deviné que c'est une Relique de la Mort. Mais dans \emph{ce} cas, directeur, il se trouve que le professeur Quirrell était sous une des robes blanches dotées de voiles.~>>

Il y eut un autre silence.

<<~Que c'est rusé~>>, dit le directeur. Il s'enfonça dans son fauteuil et soupira. <<~J'ai discuté avec le professeur de Défense. Juste avant toi, à vrai dire. Je ne savais pas vraiment quoi dire. Je lui ai expliqué que ce n'était pas là la politique approuvée par Poudlard en matière d'infractions aux règles en vigueur dans les couloirs et que je ne trouvais pas qu'il était convenable pour un professeur de Poudlard de faire ce qu'il avait fait.

--- Et que le professeur Quirrell a-t-il répondu à cela~?~>> dit Harry, qui n'était en rien impressionné par les politiques actuelles de maintien des règles en vigueur dans les couloirs.

Le directeur avait l'air résigné. <<~Il a dit~: \emph{Renvoyez-moi}.~>>

Harry parvint sans savoir comment à ne pas acclamer le professeur à voix haute.

Le directeur fronça les sourcils.

<<~Mais \emph{pourquoi} a-t-il fait cela, Harry~?

--- Parce que le professeur Quirrell n'aime pas les brutes de cette école et que je lui ai demandé très poliment~>>, dit Harry. \emph{Et il s'ennuyait et il s'est dit que cela pourrait lui remonter le moral.} <<~Soit ça, soit ça fait partie d'un complot incroyablement intriqué.~>>

Le directeur se leva et commença à marcher de long en large devant le porte-chapeau sur lequel se trouvaient le Choixpeau et les pantoufles rouges.

<<~Harry, n'as-tu pas l'impression que tout ceci est devenu un peu…

--- Génial~? proposa Harry.

--- \emph{Parfaitement et complètement incontrôlable} décrirait mieux la chose, dit Dumbledore. Je ne suis pas certain qu'il y ait jamais eu un temps dans l'Histoire de cette école où les choses sont devenues si, si… je n'ai pas de mot pour le décrire Harry, parce que les choses n'ont jamais été ainsi, et personne n'a donc jamais inventé un mot pour ça.~>>

Harry aurait essayé d'inventer des mots pour exprimer à quel point il se sentait flatté s'il n'en avait pas été trop fier pour pouvoir parler.

Le directeur le regardait avec une gravité qui ne faisait qu'augmenter.

<<~Harry, as-tu la \emph{moindre idée} de la raison pour laquelle je trouve ces événements inquiétants~?

--- Honnêtement~? dit Harry. Non, pas vraiment. Enfin, bien sûr, le professeur McGonagall objecterait à tout ce qui pourrait briser la morne monotonie de la vie scolaire de Poudlard. Mais enfin, le professeur McGonagall ne mettrait pas le feu à poulet, elle.~>>

Les rides du visage de Dumbledore devinrent plus profonde.

<<~Ce n'est pas cela qui m'inquiète, Harry, dit doucement le directeur. Une bataille a eu lieu dans ces couloirs~!

--- M. le directeur~>>, dit Harry, essayant avec précaution de garder un ton respectueux, <<~le professeur Quirrell et moi n'avons pas choisi que cette bataille ait lieu. Ce sont les brutes qui en ont décidé ainsi. \emph{Nous} avons juste décidé de faire gagner le côté clair. Je sais que parfois les frontières de la moralité sont incertaines, mais dans ce cas, la ligne de démarcation entre les méchants et les héroïnes faisait vingt mètres de haut et avait été tracée d'un feu blanc. Notre intervention a peut-être été \emph{bizarre} mais elle n'était certainement pas \emph{mauvaise} -~>>

Le directeur était revenu à son bureau, s'était assis sur son trône rembourré avec un bruit sourd, et avait recouvert son visage de ses mains.

<<~Est-ce que j'ai raté quelque chose~? dit Harry. Je pensais que vous seriez secrètement de notre côté, M. le directeur. C'était une réponse de Gryffondor. Les jumeaux Weasley approuveraient. \emph{Fumseck} approuverait -~>> Harry jeta un coup d'œil au perchoir doré, mais il était vide~; soit le phénix avait des choses plus importantes à faire, soit le directeur ne l'avait pas invité à la discussion d'aujourd'hui.

<<~Et ceci~>>, dit le directeur d'une voix âgée, fatiguée et passablement étouffée, <<~est précisément le problème, Harry. Si les courageux jeunes héros ne sont pas mis aux commandes d'écoles, c'est qu'il y a une raison.

--- Très bien~>>, dit Harry. Il n'arrivait pas tout à fait à empêcher sa voix de révéler son scepticisme. <<~Qu'est-ce que j'ai raté, cette fois~?~>>

Le vieux sorcier releva la tête, son visage à présent solennel et plus calme. <<~Écoutes-moi, Harry, dit Dumbledore, écoute-moi bien, car tous ceux doté de pouvoir doivent l'apprendre un jour. Certaines choses en ce monde sont en effet vraiment simples. Si tu ramasses une pierre et que tu la laisses tomber, la Terre n'en sera pas alourdie, et les étoiles ne dévieront pas de leur route. Je dis cela, Harry, pour que tu saches que je ne prétends pas être sage quand je te dis que même si certaines choses sont simples, d'autres sont complexes. Il existe des magies qui laissent une marque sur le monde et marquent aussi ceux qui les manient d'une façon qu'un simple charme ne ferait \emph{pas}. Ces magies demandent que l'on hésite, que l'on réfléchisse aux conséquences, qu'on prenne le temps de mesurer le poids de leur marque. Et pourtant, la plus complexe des magies que je connaisse est plus simple que la plus simple des âmes. Les \emph{gens}, Harry, sont toujours marqués, par ce qu'ils font et par ce qui leur est fait. Comprends-tu maintenant pourquoi dire 'Voici la ligne de démarcation entre les héros et les méchants~!' ne suffit pas à affirmer que ce que tu as fait était juste~?

--- M. le directeur, dit Harry d'une voix neutre, ce n'est pas une décision que j'ai prise au hasard. Non, je ne connais pas l'effet exact que cela aura sur chacune des brutes présentes. Mais si j'attendais toujours d'avoir une information parfaite avant d'agir, je ne ferai jamais rien. En ce qui concerne le développement psychologique de, disons Peregrine Derrick, il n'aurait probablement pas été très bon pour lui de martyriser huit filles de première année. Et les arrêter discrètement et rapidement n'aurait pas suffit puisqu'ils auraient juste réessayé plus tard. Il fallait donc qu'ils voient qu'un pouvoir protecteur digne d'effroi existait.~>> La voix de Harry demeura neutre. <<~Mais bien sûr, puisque je \emph{suis} un gentil, je ne voulais pas leur infliger des dommages permanents ni même les faire souffrir, et pourtant la peine devait être assez forte pour être prise en compte la prochaine fois qu'ils penseraient à réessayer. Donc, après avoir soupesé les résultats attendus du mieux que mon intellect rationnel à un degré limité en est capable, j'ai songé que le plus sage serait de déshabiller les brutes et de les coller au plafond.~>>

Le jeune héros regardait droit vers le vieux sorcier, ses yeux verts impassible braqués sur le bleu derrière les lunettes.

\emph{Et puisque je n'étais pas là et que je n'ai personnellement rien fait, il n'existe aucun moyen légal de me punir selon les règles de Poudlard~; le seul à avoir agi est le professeur et il est intouchable. Et briser les règles juste pour m'atteindre ne serait pas une chose très sage à faire au héros que préparez à combattre Lord Voldemort…} Cette fois, Harry \emph{avait} essayé de penser à l'avance à toutes les ramifications avant de soumettre sa suggestion au professeur Quirrell~; et pour une fois le professeur de Défense ne l'avait pas traité d'idiot, il avait juste lentement souri et avait commencé à rire.

<<~Je comprends tes intentions, Harry, dit le vieux sorcier. Tu penses avoir donné une leçon aux brutes de Poudlard. Mais si Peregrine Derrick pouvait apprendre cette leçon, il ne serait pas Peregrine Derrick. Tes actes n'auront d'autre effet que de le provoquer encore plus - ce n'est pas juste mais c'est ainsi.~>> Le vieux sorcier ferma les yeux comme sous le coup d'une brève douleur puis les rouvrit. <<~Harry, la vérité la plus douloureuse à accepter pour un héros est que le bien ne peut pas et ne doit pas gagner toutes les batailles. Tout cela a commencé quand Mlle Granger s'est battue contre trois ennemis et qu'elle a gagné. Si elle s'était contenté de cela, l'écho de ses exploits aurait fini par disparaître. Au lieu de cela, elle s'est associée à des camarades et a levé sa baguette en signe de défi contre Peregrine Derrick et tous ses semblables~; et ses semblables sont incapables de répondre autrement qu'en levant leur baguette. Jaime Astorga est donc parti en chasse, et il aurait dû la battre~; ç'aurait été fort triste, mais les choses se seraient arrêtées là. Huit sorcières de première année mises ensemble ne contiennent pas assez de magie pour vaincre un tel adversaire. Mais tu ne pouvais pas accepter ça, Harry, tu ne pouvais pas laisser Mlle Granger apprendre sa leçon par elle-même~; et tu as donc envoyé le professeur de Défense les surveiller, invisible, et percer les boucliers d'Astorga au moment où Daphné Greengrass le frappait -~>>

\emph{Quoi~?} se dit Harry.

Le vieux sorcier continua. <<~À chaque fois que tu es intervenu, Harry, cela a intensifié les choses, encore et encore. Mlle Granger s'est vite retrouvée face à Robert Jugson lui-même, le fils d'un Mangemort, avec deux puissants alliés à ses côtés. Cela aurait en effet été douloureux pour Mlle Granger si elle avait perdu cette bataille. Et à nouveau, par ta volonté et la baguette de Quirinus, cette fois plus ouvertement, elle a gagné.~>>

Harry avait encore du mal à envisager la possibilité d'un professeur de Défense invisible surveillant la S.P.E.H.S. et s'assurant que les héroïnes étaient à l'abri du danger.

<<~Et ainsi~>>, finit le vieux sorcier, <<~nous en sommes arrivés à aujourd'hui, Harry, à l'attaque de quarante-quatre élèves contre huit sorcières en première année. Une véritable bataille, dans ces murs~! Je ne sais quelle était ton intention, mais tu dois accepter une part de responsabilité. De telles choses ne se produisaient pas avant que tu n'arrives dans cette école, pas pendant les décennies que j'ai passé à Poudlard~; ni quand j'étais un élève ni quand j'étais un professeur.

--- Merci beaucoup, dit Harry d'une voix neutre. Même si je pense que le professeur Quirrell a plus de mérite que moi.~>>

Les yeux bleus s'écarquillèrent.

<<~Harry…

--- Ces brutes attaquent leurs victimes depuis bien avant le début de l'année~>>, dit Harry. Malgré ses efforts, sa voix commençait à s'élever. <<~Mais personne n'a semblé apprendre aux élèves qu'ils avaient le droit de se \emph{défendre}. Je sais qu'il est beaucoup plus simple d'\emph{ignorer} un combat à armes égales qu'une victime impuissante qui se prend des maléfices et se fait presque jeter d'une fenêtre, mais peut-on dire que c'est \emph{pire}~? J'aimerais avoir lu plus d'écrits de Godric Gryffondor pour pouvoir le citer, il doit bien y avoir quelque chose dans le tas qui parle de ce problème. Les batailles ouvertes sont peut-être plus \emph{bruyantes} que des victimes qui souffrent en silence, elles rendent peut-être plus difficile la tâche de prétendre qu'il ne s'est rien passé, mais le résultat final est \emph{meilleur} -

--- Non, dit Dumbledore. Pas du tout, Harry. \emph{Toujours} combattre les ténèbres, ne \emph{jamais} laisser le mal incontesté - ce n'est pas de l'héroïsme mais simplement de l'orgueil. Même Godric Gryffondor ne pensait pas que toutes les guerres méritaient qu'on se batte, et pourtant il a passé sa vie entière à aller d'une bataille à une autre.~>> La voix du vieux sorcier diminua de volume. <<~En vérité, Harry, tes paroles ne sont pas maléfiques. Non, pas maléfiques, et pourtant elles m'ont effrayé. Tu es quelqu'un qui pourrait un jour exercer un grand pouvoir sur les autres sorciers. Et si, ce jour là, tu penses encore qu'on ne doit jamais laisser le mal incontesté -~>> Une véritable inquiétude transparaissait maintenant dans la voix du directeur. <<~Le monde est devenu plus fragile depuis l'époque où Poudlard s'est dressée. J'ai peur qu'il ne puisse supporter la furie d'un autre Godric Gryffondor. Et Gryffondor entrait en rage moins vite que toi.~>> Le vieux sorcier secoua la tête. <<~Tu es trop prêt à te battre, Harry. Bien trop prêt, et Poudlard elle-même devient plus violente autour de toi.

--- Eh bien,~>> dit Harry avec précaution, soupesant ses mots. <<~Je ne sais pas si ça va aider, mais je pense que vous avez une fausse impression de mes préférences. Je n'aime pas non plus les vrais combats. Ça fait peur, c'est violent et on risque de se faire mal. Mais je ne me suis \emph{pas} battu aujourd'hui, M. le directeur.~>>

Ce dernier fronça les sourcils.

<<~Tu as envoyé le professeur de Défense à ta place -

--- Le professeur Quirrell ne s'est pas vraiment battu non plus, dit Harry avec calme. Il n'y avait là personne d'assez puissant pour le combattre. Il n'y a eu aucun combat aujourd'hui, seulement une victoire.~>>

Le vieux sorcier ne parla qu'un moment plus tard. <<~Peut-être est-ce bien le cas, dit-il, mais tous ces conflits doivent toucher à leur fin. Je peux entendre la tension dans l'air, et à chaque altercation, cette tension monte. Tout cela doit prendre fin, de façon décisive et rapide, et tu ne dois pas faire obstacle à cette fin.~>>

Le vieux sorcier fit un geste vers la grande porte en chêne et Harry s'en fut.

\later

C'est avec une certaine surprise que Harry passa entre les deux immenses gargouilles grises qui s'étaient écartées et vit que Quirinus Quirrell était toujours affalé contre la pierre du couloir, un épais filet de bave tombant de sa bouche molle jusqu'à sa robe de professeur, exactement dans la même position que lorsque Harry était parti pour le bureau du directeur.

Harry attendit mais l'homme ne se releva pas~; et après de longues secondes gênées, Harry commença à continuer son chemin dans le couloir.

<<~M. Potter~?~>> L'appel était doux et ne vint qu'après que Harry eut passé deux tournants~; une voix basse qui traversait les murs d'une façon anormale.

Lorsque Harry revint, il trouva le professeur toujours affalé contre le mur, mais à présent les yeux pâles le regardaient, emplis d'intelligence.

\emph{Je suis navré de vous avoir épuisé -}

C'était là quelque chose que Harry ne pouvait pas dire. Il avait remarqué la corrélation entre les efforts déployés par Quirrell et la durée pendant laquelle il devait se 'reposer'. Mais Harry avait conclu que si l'effort avait été trop douloureux ou préjudiciable, le professeur aurait certainement refusé. À présent, Harry se demandait si ce raisonnement avait vraiment été correct, et si non, comment il pouvait s'excuser…

Le professeur de Défense parla d'une voix basse et le reste de son corps demeura immobile.

<<~Comment s'est déroulé votre entretien avec le directeur, M. Potter~?

--- Je n'en suis pas certain, dit Harry. Pas comme je l'avais prédit. Il semble croire que le côté clair devrait perdre bien plus souvent que je crois être sage. Et puis je ne suis pas certain qu'il comprenne la différence entre essayer de se battre et essayer de gagner. Ce qui explique beaucoup de choses, en fait…~>> Harry n'avait pas beaucoup lu sur la guerre des sorciers mais il en savait assez pour savoir que les gentils \emph{s'étaient} probablement fait une idée assez précise de l'identité des pires Mangemorts et qu'ils ne leur \emph{avaient pas} juste envoyé des grenades par chouette en moins de cinq minutes.

Un rire doux et léger émergea des lèvres pâles.

<<~Dumbledore ne comprend pas la joie de gagner, tout comme il ne comprend pas la joie de jouer. Dites-moi, M. Potter. Avez-vous délibérément suggéré ce petit plan dans l'intention de me soulager de mon ennui~?

--- C'était un de mes nombreux motifs~>>, dit Harry, parce qu'un instinct l'avait prévenu qu'il ne pouvait pas juste dire \emph{Oui}.

<<~Savez-vous, dit le professeur de Défense d'un ton doux et pensif, que certains ont essayé d'adoucir mes humeurs les plus sombres, que d'autres ont en effet aidé à ensoleiller mes jours, mais que vous êtes la première personne à jamais réussir à le faire délibérément~?~>> Le professeur sembla se redresser et s'écarter du mur d'un mouvement très particulier qui aurait pu être autant magique que musculaire, et il commença à s'en aller sans un regard en arrière. seul un petit geste du doigt indiquant à Harry qu'il devait suivre.

<<~J'ai particulièrement apprécié le chant que vous avez composé pour Mlle Davis, dit le professeur Quirrell après qu'ils eurent parcouru une courte distance. Même si vous auriez été plus sage de venir me consulter à l'avance avant de le lui faire apprendre.~>> Une main s'agita dans sa robe et sortit une baguette qui fit un petit geste, après quoi tous les lointains sons du château disparurent. <<~Répondez honnêtement, M. Potter, êtes-vous par un moyen ou un autre devenu familier avec la théorie des rituels noirs~? Ce n'est pas la même chose que de confesser d'avoir eu l'intention de les pratiquer~; de nombreux sorciers en connaissent les principes.

--- Non…~>> dit lentement Harry. Il avait décidé il y a longtemps de ne pas essayer de s'introduire dans la zone interdite de la bibliothèque de Poudlard pour exactement la même raison qui lui avait fait décider un an plus tôt de ne \emph{pas} apprendre à fabriquer des explosifs à partir de produits domestiques. Harry s'enorgueillissait d'avoir au moins \emph{plus} de bon sens que les gens voulaient bien lui en attribuer.

<<~Oh~?~>> dit le professeur Quirrell. L'homme marchait maintenant d'un pas plus normal et les lèvres se courbèrent en un sourire étrange. <<~Eh bien, peut-être possédez-vous alors quelque talent particulier pour le domaine.

--- Oui, enfin, dit Harry d'un ton las. J'imagine que le docteur Seuss avait aussi un talent naturel pour les rituels noirs alors, parce que \emph{brouille, drouille, muzo, muffe} vient d'un livre pour enfants intitulé \emph{Barthélémiou et l'Ooberk} -

--- Non pas cette partie~>>, dit le professeur Quirrell. Sa voix devint un peu plus forte et ressembla plus à son ton professoral habituel. <<~Un charme ordinaire, M. Potter, peut-être lancé simplement en prononçant certains mots, en faisant des gestes de baguette précis et en dépensant une partie de votre force. Même les sortilèges puissants peuvent être ainsi lancés si la magie à l'œuvre est à la fois bien utilisée et efficace. Mais pour les magies les plus puissantes, la parole seule ne suffit pas à leur donner une structure. Il faut agir de façon particulière, faire des choix importants. Et une dépense temporaire de force ne suffit pas à les activer. Un rituel demande un sacrifice permanent. Le pouvoir d'un sortilège de rang supérieur peut être aussi différent de celui d'un charme ordinaires que le jour l'est de la nuit. Mais de nombreux rituel - la plupart à vrai dire - se trouvent requérir au moins un sacrifice qui pourrait inspirer le dégoût. C'est ainsi que tout le domaine de la magie rituelle, qui contient les réussites les plus avancées et les plus intéressantes du monde sorcier, est considéré par tous comme lié aux ténèbres. Avec quelques exceptions choisies par la tradition, bien sûr.~>> La voix du professeur Quirrell prit une note sardonique. <<~Le Serment Inviolable est trop utile à certaines maisons fortunées pour être rendu entièrement illégal - même si lier la volonté d'un homme jusqu'à la fin de ses jours est bel et bien un acte terrible et effrayant, bien plus que nombre de rituels inférieurs que de nombreux sorciers fuient. Un cynique pourrait en conclure que la légalité d'un rituel n'est pas tant une question de morale que d'habitude. Mais je m'égare…~>> Le professeur Quirrell toussa brièvement pour s'éclaircir la gorge. <<~Le Serment Inviolable requiert trois participants et trois sacrifices. Celui qui reçoit le Serment Inviolable doit être celui qui aurait pu faire confiance au Demandeur, mais a choisi d'exiger un serment de sa part, et il sacrifie cette possibilité d'avoir confiance. Celui qui fait le Vœu doit être quelqu'un qui aurait pu choisir de faire ce que le Serment exigera de lui, et il sacrifie cette capacité de choisir. Et le troisième sorcier, l'Enchaîneur, sacrifie de façon permanente une petite partie de sa magie afin de maintenir le Serment éternellement.

--- Ah, dit Harry. Je \emph{m'étais demandé} pourquoi ce sortilège n'était pas utilisé partout à chaque fois que deux personnes avaient du mal à se faire confiance… même si… pourquoi tous les sorciers sur leur lit de mort ne se font-ils pas payer pour lier des Serments Inviolables et pour laisser un héritage à leurs enfants -

--- Parce qu'ils sont stupides, dit le professeur Quirrell. Il y a des centaines de rituels utiles qui pourraient être pratiqués si les hommes avaient assez de bon sens pour cela~; je pourrais en nommer vingt sans m'arrêter pour reprendre mon souffle. Mais quoi qu'il en soit, M. Potter, la particularité de tels rituels - que vous choisissiez de les trouver noirs ou pas - c'est qu'ils sont conçus de façon à être magiquement efficace et non pas impressionnants au moment de leur accomplissement. J'imagine qu'il y a une certaine tendance à demander des sacrifices plus terribles pour les rituels les plus puissants. Mais quand bien même, le rituel le plus terrible que je connaisse ne requiert qu'une corde ayant pendu un homme et une épée qui a pourfendu une femme~; et cela pour un rituel qui promet d'invoquer la Mort elle-même - même si j'ignore le sens réel de cette promesse et que je ne souhaites en rien le découvrir puisqu'on dit aussi que le contre-sort permettant de renvoyer la Mort a été perdu. Le chant le plus terrible que j'ai rencontré ne provoque pas un centième de l'effroi induit par celui que vous avez composé pour Mlle Davis. Ceux parmi les brutes qui ont une familiarité minimale avec les rituels noirs - et je suis certains qu'il y en a - doivent avoir été terrifiés d'une façon indicible. S'il existait un véritable rituel dont l'accomplissement était aussi impressionnant que le vôtre, M. Potter, il ferait fondre la Terre.

--- Hum~>>, dit Harry.

Les lèvres du professeur Quirrell se déformèrent encore plus.

<<~Ah, mais la chose vraiment amusante était ceci~: voyez-vous, le chant de chaque rituel nomme ce qui va être sacrifié et ce qui va être obtenu. Le chant que vous avez donné à Mlle Davis parlait d'abord d'une ténèbre au-delà des ténèbres enterrée sous le flot du temps, qui connaît la porte et est la porte. Et le deuxième élément mentionné, M. Potter, était la manifestation de votre présence. Et toujours, pour chaque partie d'un rituel, on nomme \emph{d'abord} la chose qui doit être sacrifiée et \emph{ensuite} ce qu'elle va permettre de faire.

--- Je… vois~>>, dit Harry en marchant derrière le professeur Quirrell à travers les couloirs de Poudlard en direction du bureau de ce dernier. <<~Donc mon chant, tel que je l'ai écrit, sous-entend que le Dieu Extérieur, Yog-Sothot -

--- A été définitivement sacrifié lors d'un rituel qui ne permet de manifester votre présence que brièvement~>>, continua le professeur Quirrell. <<~J'imagine que nous découvrirons demain si quiconque a pris cela au sérieux, lorsque nous lirons les journaux et verrons si toutes les nations magiques du monde se sont alliées dans un effort désespéré pour colmater votre incursion dans notre réalité.~>>

Ils continuèrent et le professeur de Défense commença à glousser d'étranges son gutturaux.

Aucun des deux ne parla jusqu'à ce qu'ils soient arrivés au bureau du professeur de Défense, puis l'homme s'arrêta avec sa main sur la porte.

<<~C'est là une chose très étrange~>>, dit le professeur de Défense d'une voix de nouveau douce, presque inaudible. L'homme ne regardait pas Harry, et ce dernier ne voyait donc que son dos. <<~Une chose très étrange… Il fut un temps où j'aurais sacrifié un doigt de ma main armée pour manipuler les brutes de Poudlard comme nous l'avons fait aujourd'hui. Pour qu'elles me craignent comme elles vous craignent aujourd'hui, pour être l'objet de la déférence de tous les élèves et de l'adoration de beaucoup, oui, j'aurais donné mon doigt pour cela. Vous avez aujourd'hui tout ce que je désirais alors. Et tout ce que je sais de la nature humaine me dit que je devrais vous haïr. Et pourtant je ne vous hais pas. C'est une chose très étrange.~>>

Le moment aurait dû être touchant, mais Harry sentit plutôt un frisson lui traverser l'échine, comme s'il était un petit poisson dans l'eau et qu'un immense requin blanc venait de lui jeter un coup d'œil et de décider de ne pas le manger après un moment d'hésitation.

L'homme ouvrit la porte du bureau du professeur de Défense, y entra et fut partit.

\later

\emph{Après-coup:}

Ses camarades de Serpentard la regardaient comme si… comme s'ils n'avaient pas la plus petite idée de la façon dont ils devaient la regarder.

Les Gryffondor la regardaient comme s'ils n'avaient pas la plus petite idée de la façon dont ils devaient la regarder.

Daphné entra d'un pas vif dans la classe de potions sans révéler sa peur, drapée de la dignité impérieuse d'une maison Noble et Très Ancienne. En son for intérieur, elle se sentait probablement comme tous les autres.

Deux heures s'étaient écoulées depuis que le \emph{Quoi~?} avait eu lieu, et le cerveau de Daphné continuait encore~: \emph{Quoi~? Quoi~? Quoi~?}

La salle était silencieuse et attendait que le professeur Rogue arrive. Lavande et Parvati étaient assises près d'un groupe d'autres Gryffondor, tous entourés de regards silencieux. Deux d'entre eux se corrigeaient mutuellement leurs devoirs avant le début du cours et personne ne leur parlait ni ne les aidait. Même Lavande, au sujet de laquelle Daphné aurait pu jurer que rien ne pouvait l'intimider, semblait plus soumise.

Daphné s'assit à son bureau, sortit \emph{Breuvages et Potions Magiques} de son sac, commença à vérifier ses devoirs et fit de son mieux pour agir normalement. Les gens la regardaient fixement et ne disaient rien -

Un hoquet de surprise traversa la salle entière. Garçons et filles eurent un soubresauts et s'éloignèrent de la porte comme comme des tiges de blé prises dans un coup de vent.

Devant la porte se tenait Tracey Davis, entourée d'une cape noir en lambeaux qui recouvrait son uniforme de Poudlard.

Elle s'avança lentement dans la salle, oscillant légèrement à chaque pas comme si elle essayait de \emph{glisser}. Elle s'assit à son bureau habituel qui se trouvait être juste à côté de celui de Daphné.

Le visage de Tracey se tourna lentement pour faire face à celui de Daphné.

<<~Tu vois~? dit la Serpentard d'un ton bas et sépulcral. Je t'avais dit que je l'aurais avant elle.

--- Quoi~? lâcha Daphné avant de souhaiter immédiatement n'avoir rien dit.

--- J'ai eu Harry Potter avant Granger.~>> La voix de Tracey était toujours basse mais ses yeux brillaient de triomphe. <<~Tu vois Daphné, ce que le général Potter veut chez une fille ce n'est pas un joli visage ou une belle robe. Il veut un fille capable de canaliser ses pouvoirs terrifiants, voilà ce qu'il veut. Maintenant je suis à lui - et lui à moi~!~>>

L'annonce donna naissance à un silence glacé qui s'empara de toute la salle.

<<~Excusez-moi, Mlle Davis~>>, dit la voix cultivée de Drago Malfoy, qui semblait indifférent et relisait ses parchemins de potions. Cet autre héritier d'une maison Très Ancienne ne leva pas les yeux de son bureau même lorsque tout le monde se tourna pour le regarder. <<~Harry Potter vous a-t-il vraiment \emph{dit} cela~? En ces termes~?

--- Euh, non…~>> dit Tracey, puis ses yeux brillèrent de colère. <<~Mais il a \emph{intérêt} à me prendre maintenant que j'ai sacrifié mon âme pour lui et tout ça~!

--- \emph{Tu as sacrifié ton âme pour Harry Potter~?}~>> hoqueta Millicent. Il y eut un fracas à l'autre bout de la pièce, car Ron Weasley avait fait tomber son encrier.

<<~Ben, j'en suis assez certaine, dit Tracey d'un ton brièvement incertain avant de se reprendre~: enfin, je me suis regardée dans un miroir et je suis plus pâle, je peux toujours sentir une sorte de ténèbre qui m'entoure, j'ai été le conduit de ses terribles pouvoirs et tout… Daphné, tu as vu mes yeux devenir verts, non~? Je ne l'ai pas vu mais c'est ce que j'ai entendu ensuite.~>>

Il y eut un silence seulement brisé par le son de Ron Weasley qui essayait de nettoyer son bureau.

<<~Daphné~? dit Tracey.

--- Je n'y crois pas, dit une voix. Impossible que le prochain Seigneur des Ténèbres te prenne \emph{toi} comme épouse~!~>>

Lentement, avec une immense incrédulité, des têtes se tournèrent pour regarder Pansy Parkinson.

<<~Tais toi, toi, dit Tracey, ou je…~>> La Serpentard s'interrompit. Puis sa voix devint encore plus basse et elle dit~: <<~Tais toi ou je dévore ton âme.

--- Tu ne peux pas faire ça~>>, dit Pansy du ton confiant d'une poule qui avait établi une hiérarchie de basse-cour très satisfaisante avec elle tout en haut et qui n'allait quand même pas mettre à jour sa croyance à cause de simples preuves.

Lentement, comme si elle essayait de flotter, Tracey se leva de son bureau. Il y eut d'autres hoquets. Daphné eut l'impression qu'on venait de la pétrifier sur sa chaise.

<<~Tracey~? dit Lavande d'une petite voix. Ne refais pas tout ça. S'il te plaît~?~>>

Pansy était visiblement nerveuse à présent, maintenant que Tracey ondulait vers son bureau. <<~Qu'es'tu crois que tu fais~?~>> dit Pansy sans tout à fait réussir à avoir l'air indignée.

<<~Je t'ai dit, dit Tracey d'un ton menaçant. Je vais dévorer ton âme.~>>

Tracey se pencha au-dessus de Pansy qui était resté assise, figée à son bureau~; et lorsque leurs lèvres se touchèrent presque, Tracey inhala bruyamment.

<<~Voilà~! dit Tracey en se redressant. J'ai mangé ton âme.

--- Non, c'est pas vrai~! dit Pansy.

--- Oh si c'est vrai~! dit Tracey.~>>

Il y eut un très court silence -

<<~Par Merlin c'est \emph{vrai}~! s'écria Theodore Nott. Tu as l'air toute pâle maintenant, et tes yeux ont l'air vides~!~>>

Pansy devint pâle et poussa un cri strident~: <<~\emph{Quoi~?}~>>. Elle bondit de son bureau et fouilla désespérément dans son sac. Puis elle sortit un miroir, se regarda et devint encore plus pâle.

Daphné abandonna toute prétention de calme aristocratique et laissa sa tête tomber sur son bureau avec un bruit sourd tout en se demandant si aller dans la même école que toutes les autres familles importantes justifiait vraiment le fait d'aller dans la même école que la légion du Chaos.

<<~Oh, tu es mal barrée maintenant, Pansy, dit Seamus Finnigan. Je ne sais pas exactement ce qui se passe quand un Détraqueur t'embrasse, mais si Tracey Davis t'embrasse c'est probablement pire.

--- J'ai entendu parler des gens sans âmes, dit Dean Thomas d'une voix lugubre. Ils doivent tous s'habiller de noir, écrire des poèmes exécrables et rien ne les rend jamais heureux. Ils sont pleins \emph{d'angoisse existentielle}.

--- Je veux pas avoir d'angoisse existentielle~! s'écria Pansy.

--- Dommage, dit Dean Thomas. Tu es obligée maintenant que ton âme est partie.~>>

Pansy se retourna et tendit une main suppliante vers le bureau de Drago Malfoy.

<<~Drago~! dit-elle d'un ton plaidant. M. Malfoy~! S'il vous plaît, forcez Tracey à me rendre mon âme~!

--- Je ne peux pas, dit Tracey. Je l'ai \emph{mangée}.

--- Forcez-la à la vomir~!~>> s'écria Pansy.

L'héritier des Malfoys s'était effondré en avant et sa tête reposait entre ses mains pour que personne ne puisse voir son visage. <<~Pourquoi ma vie est-elle ainsi~?~>> dit Drago Malfoy.

Un babillage de chuchotements incontrôlable commença à mesure que Tracey revenait à son bureau et souriait d'un air satisfait tandis que Pansy se tenait au milieu de la salle en se tordant les mains alors que des larmes commençaient à apparaître dans ses yeux -

<<~\emph{Taisez. Vous.}~>>

La voix douce et létale sembla emplir la salle entière lorsque le professeur Rogue y entra. Il avait l'air plus en colère que Daphné ne l'avait jamais vu, ce qui envoya une décharge de véritable peur dans son échine. Elle se pencha sur ses devoirs en hâte.

<<~Asseyez-vous, Parkinson, siffla le maître des potions, et vous, Davis, ôtez-moi cette cape ridicule -

--- \emph{Professeur Rooooooogue~!} gémit une Pansy Parkinson en larmes. \emph{Tracey a mangé mon ââââââme~!}~>>
%  LocalWords:  arry Acathla mundatus Mabra brahoring mabra duffle Fista
%  LocalWords:  wista mista Lemarchand Khornath Slaaneth Nurgolth Tzintchi
%  LocalWords:  pitchest boundedly Oobleck Vower Yog Sothoth Snaaaaaape
%  LocalWords:  sooouuul
