\partchapter{Pressions temporelles}{I}

\section{16 avril 1992.}

12h07

\lettrine{H}{eure du déjeuner.}

\hplettrineextrapara
Harry marcha d'un pas lourd jusqu'à la table Gryffondor en majeure partie déserte et observa d'un coup d'œil que le déjeuner d'aujourd'hui était composé de boulettes de Roupo et de Brine. La conversation ambiante, comme Harry pouvait l'entendre, avait trait au Quidditch~; un environnement sonore relativement pire que le bruit de tronçonneuses rouillées, mais supérieure aux imbécillités que la table Serdaigle racontait encore sur Hermione. Au moins Gryffondor avait initialement été moins sensible à la cause de Drago Malfoy et avait plus de raisons politiques de souhaiter que tout le monde se contente d'oublier certains faits malheureux~; et si ce n'était pas une bonne raison de se taire, au moins ils faisaient silence. Dean, Seamus et Lavande étaient tous partis pour les vacances, mais au moins cela laissait…

«C'était quoi ce brouhaha à la grande table~? dit Harry à l'esprit de groupe Weasley tout en remplissant son assiette. On dirait que ça s'achevait juste quand je suis rentré.

--- Notre bien aimée mais maladroite professeur Trelawney…

--- Semble s'être renversé toute une soupière dessus…

--- Sans parler de M. Hagrid.»

Un regard rapide vers la grande table confirma que le professeur de Divination agitait convulsivement les mains tandis que le demi-géant essuyait ses vêtements. Personne ne semblait y prêter beaucoup d'attention, même le professeur McGonagall. Le professeur Flitwick était debout sur sa chaise, comme d'habitude, le directeur semblait encore être absent (il n'avait pas été là pendant la majeure partie des vacances), les professeurs Chourave, Sinistra et Vector mangeaient entre elles comme à leur habitude et…

«Vous savez», dit Harry en tournant la tête pour regarder l'illusion d'un ciel bleu radieux située au plafond, «ça me fout encore les jetons, parfois.

--- Quoi~?» dirent Fred et George.

Le puissant et énigmatique professeur de Défense se “reposait” ou subissait-sa-mystérieuse-affliction~: ses mains faisaient des tentatives hésitantes et tâtonnantes vers une cuisse de poulet placée sur son assiette qui semblait lui échapper.

«Euh, rien, dit Harry. Je ne suis pas encore habitué à Poudlard.»

Il continua de manger dans un silence relatif tandis que divers Weasley discutaient d'une étrange substance aux effets psychiques appelée Canons de Chudley.

«Quel genre de mystérieuses pensées profondes es-tu en train d'avoir~?» dit une sorcière à l'air jeune, aux cheveux court, assise non loin. «Je veux dire, je suis juste curieuse. Je m'appelle Brienne, au fait.» Elle avait l'un de ces regards que Harry avait fermement décidé d'ignorer tant qu'il ne serait pas plus âgé.

«Alors, dit Harry, tu vois ces programmes d'intelligence artificielle très simples, comme ELIZA, qui sont programmés pour faire des phrases syntaxiquement correctes à partir des mots de l'Anglais mais qui ne possèdent aucune compréhension du sens de ces mots~?

--- Bien sûr, dit la sorcière. J'en ai une dizaine dans ma malle.

--- Eh bien je suis à peu près certain que ma capacité à comprendre les filles est à ce niveau.»

Il y eut un silence soudain.

Harry mit quelques secondes à se rendre compte que, non, toute la grande salle ne le fixait pas lui, et il tourna alors la tête pour regarder.

La silhouette qui venait d'entrer dans la pièce en chancelant semblait être M. Rusard, le surveillant des couloirs do Poudlard, qui, accompagné de son chat prédateur Mlle Teigne, représentait une rencontre aléatoire de bas niveau que Harry, muni de sa Relique de la Mort, un objet de rang épique, avait bien souvent évité sans le moindre effort. (Harry avait un jour suggéré aux jumeaux Weasley de faire une blague à cette cible qui le méritait bien, ce sur quoi Fred ou George avait tranquillement fait remarquer que l'on n'avait jamais vu M. Rusard faire usage d'une baguette, ce qui était étrange, vraiment, étant donné le nombre de sortilèges qui seraient utiles à ce poste, et que cela poussait à se demander pourquoi Dumbledore avait donné ce poste à cet homme~; et Harry s'était tu.)

Pour l'instant, les vêtements marron de M. Rusard étaient en désordres et trempés de sueur, ses épaules tanguaient visiblement au rythme de sa respiration, et son chat omniprésent manquait à l'appel.

«Troll… s'étrangla M. Rusard. Dans les donjons…»

Minerva McGonagall se leva de la grande table si vite que sa chaise tomba derrière elle.

«Argus~! s'écria-t-elle. Qu'est-ce qui vous est arrivé~?»

Argus Rusard s'avança en chancelant, les immenses portes dans son dos. Le haut de son corps était parsemé de traits et de petits points cramoisis, comme si quelqu'un avait éclaboussé du tabasco sur son visage. «Troll… gris… deux fois ma taille… il… il…» Argus Rusard se couvrit le visage de ses mains. «Il a mangé Mlle Teigne… mangée toute crue, d'une bouchée…»

Minerva ressentit une pointe de désarroi dans son autre soi, elle n'avait pas beaucoup aimé ce chat, mais ils étaient tous les deux félins.

Un grondement se leva dans la grande salle. Severus se leva de la grande table en se débrouillant pour, sembla-t-il, ne pas attirer l'attention, et il sortit à grands pas sans prononcer un mot.

Bien sûr, songea Minerva, le couloir du troisième étage… ça pourrait être une distraction…

Elle relégua mentalement toutes ces préoccupations au bon soin de Severus, sortit sa baguette, la leva bien haut et laissa échapper cinq jets de feu violet.

Il y eut un silence stupéfait, exception faite des sanglots hachés d'Argus.

«Il semble qu'une créature dangereuse est en liberté dans Poudlard, dit-elle aux professeurs assis à la grande table. Je vais vous demander à tous d'aider à la chercher dans les couloirs.» Elle se tourna alors vers les élèves stupéfaits et attentifs et leva la voix~: «Préfets -- menez vos maisons aux dortoirs immédiatement~!»

Percy Weasley bondit de la table Gryffondor. «Suivez-moi~! dit-il d'une voix aiguë. Restez ensemble, les élèves de première année~! Non, pas vous…» mais les autres préfets élevaient déjà la voix et un babillage renouvelé prenait vie.

Puis une voix claire et froide parla sous le bruissement soudain.

«Madame la directrice adjointe»

Elle se retourna.

Le professeur de Défense essuyait calmement ses mains avec sa serviette tout en se levant de la grande table. «Avec tout mon respect, dit l'homme dont l'identité était inconnue, vous n'êtes pas experte en tactiques de combat, madame. Dans cette situation, il serait plus sage de…

--- Je vous prie de m'excuser, professeur», dit le professeur McGonagall en se tournant vers les grandes portes. Filius et Pomona s'étaient déjà levés pour la suivre, de même que le demi-géant Rubeus Hagrid, qui les surplomba en se levant. Elle en était au stade où elle n'avait déjà eu des expériences semblables que trop souvent. «Ma triste expérience m'a apprise qu'en de telles occasions, ce n'est pas le moment d'écouter quelque conseil que ce soit venant d'un professeur de Défense. Je pense d'ailleurs qu'il serait sage que nous cherchions le troll ensemble afin qu'aucun soupçon ne puisse être porté sur vous dans l'éventualité où de fâcheux événements se produiraient pendant ce temps.»

Sans aucune hésitation, le professeur de Défense bondit élégamment sur la table Gryffondor et claqua des mains, faisant un bruit qui ressemblait à celui d'un plancher qui se brisait.

«Michelle Morgan, de Gryffondor, commandante en second de l'armée de Pinnini,» dit-il calmement dans le silence qui s'était installé. «Conseillez s'il vous plaît la directrice de votre maison.»

Michelle Morgan grimpa sur son banc et parla, la petite sorcière semblait beaucoup plus sûre d'elle que dans le souvenir, datant du début de l'année, que Minerva avait d'elle.

«Des élèves marchant dans les couloirs seraient éparpillés et impossibles à défendre. Tous les élèves doivent rester dans la grande salle et former un groupe au centre de la pièce… pas entourés par des tables, un troll sauterait juste par-dessus… avec le périmètre défendu par des étudiants en septième année. Uniquement parmi ceux venus des armés, peu importe leur talent de duelliste, afin qu'ils ne traversent pas leurs lignes de mire respectives.» Michelle hésita. «Je suis désolée, M. Hagrid, mais… ça ne serait pas sûr pour vous, vous devriez rester derrière avec les élèves. Et le professeur Trelawney ne devrait pas faire face à un Troll seule non plus,» Michelle semblait beaucoup moins désolée de prodiguer ce conseil, «mais si elle fait équipe avec le professeur Quirrell, ils pourraient ensemble former une unité de combat supplémentaire digne de confiance. Ceci conclut mon analyse, professeur.

--- Acceptable, pour une réponse sur le vif, dit le professeur de Défense. Vingt points Quirrell pour vous. Mais vous négligez un argument encore plus simple~: chez soi ne veut pas dire en sécurité, et un Troll est assez puissant pour arracher une porte à portrait de ses gonds…

--- Assez, lâcha Minerva d'un ton sec. Merci, Mlle Morgan.» Elle regarda les tables, qui attendaient. «Tous les élèves, faites ce qu'elle a dit.» Elle se retourna vers la grande table. «Professeur Trelawney, vous accompagnerez le professeur de Défense…

--- Ah», dit Sibylle d'une voix faible et hésitante. Sous son maquillage excessif et son fatras de châles, la femme semblait plutôt pâle. «J'ai peur… j'ai peur de ne pas me sentir très bien aujourd'hui… en fait je me sens plutôt mal…

--- Vous n'aurez pas à vous battre contre le troll», dit Minerva d'un ton dur, sa patience à l'épreuve, comme à chacune de ses interactions avec cette femme. «Restez juste avec le professeur de Défense et ne le laissez pas un instant hors de votre champ de vision, vous devrez ensuite pouvoir témoigner que vous étiez avec lui en permanence.» Elle se tourna vers Rubeus. «Rubeus, je te laisse, tu es le responsable ici. Protège-les.» L'immense homme se redressa en entendant ceci, perdit son air abattu et hocha fièrement la tête.

Puis Minerva regarda les élèves et éleva la voix. «Il va évidemment sans dire que toute personne quittant la grande salle pour quelque raison que ce soit sera exclu. Aucune excuse ne sera acceptée. Me suis-je faite comprendre~?»

Les jumeaux Weasley, qu'elle avait regardés dans les yeux, opinèrent respectueusement du chef.

Elle se retourna sans dire un mot et marcha vers les grandes portes, les autres professeurs derrière elle.

Loin, sur un mur à l'autre bout de la pièce, inaperçue, une horloge indiquait 12h14.

… et il ne s'en rendait toujours pas compte.

Tic.

Alors que Harry regardait, les yeux étroits, les professeurs qui s'éloignaient, en se demandant ce qui se passait vraiment, ce que cela signifiait, alors que les élèves s'assemblaient en une masse plus facile à défendre, que des baguettes s'agitaient pour faire léviter des tables hors du passage, il ne s'en rendait toujours pas compte.

Tic.

«Les professeur ne devraient-ils pas tous s'être mis en équipes de deux~? dit un élève de Gryffondor dont Harry ignorait le nom. Enfin… ça serait plus lent, mais ça serait plus sûr, je pense…»

Tic.

Quelqu'un lui répondit en élevant la voix mais Harry n'en comprit pas grand-chose, l'idée était que les trolls des montagnes étaient très résistants à la magie, qu'ils étaient incroyablement forts, qu'ils pouvaient se régénérer, mais qu'ils étaient quand même bruyants, et donc qu'en l'entendant approcher un professeur de Poudlard n'aurait quand même pas de difficulté à l'enrober dans le truc machin incassable de Vadim.

Tic.

Et Harry ne s'en rendait toujours pas compte.

Tic.

Les bruits de la foule étaient assourdis, les gens se parlaient à voix basse en jetant des regards alentours, à l'écoute du bruit d'une porte qui se fracasse et d'un rugissement de colère.

Tic.

Certains élèves spéculaient par murmures en s'interrogeant sur ce que le professeur de Défense pouvait bien essayer d'accomplir en introduisant un troll, sur l'éventualité de sa colère après que le professeur McGonagall eut compris que c'était une distraction, et sur la nature de ce dont il essayait de distraire tout le monde.

Tic.

Et la pensée ne s'était toujours pas formée en lui, pas avant que, suite à ce que les élèves aient formé une masse d'environ cent corps entourés par des patrouilles d'élèves de septième année à l'air fièrement sinistres, leurs baguettes pointées vers l'extérieur, quelqu'un suggère que l'on recense les gens présents, et quelqu'un répondit avec sarcasme que ça aurait pu avoir un sens de le faire un autre jour mais que pour l'instant presque tout le monde était parti pour les vacances de printemps et que personne ne savait vraiment combien d'élèves étaient censés se trouver dans la pièce et encore moins s'il en manquait.

Tic.

C'est alors que Harry se demanda où était Hermione.

Tic.

Il regarda là où les Serdaigle s'étaient assemblés et ne vit pas Hermione, mais enfin, tout le monde était suffisamment serré l'un contre l'autre, on ne se serait pas attendu à repérer les plus petits à travers cette foule, parmi les élèves plus âgés.

Tic.

Harry regarda alors vers les Poufsouffle pour voir s'il pouvait repérer Neville, et bien que ce dernier soit debout derrière un élève beaucoup plus grand, le centre de traitement visuel de Harry parvint à le voir presque immédiatement. Hermione n'était pas non plus avec les Poufsouffle, pas qu'il puisse voir… et elle ne serait certainement pas avec les Serpentard…

Tic.

Harry joua des épaules à travers la foule serrée, passant à côté ou autour d'élèves plus âgés, et en une occasion passant entre deux jambes, jusqu'à ce qu'il se tienne au milieu des Serdaigle et puisse s'assurer que non, il n'y avait pas de Hermione par ici.

Tic.

«Hermione Granger~! dit Harry avec force. Tu es là~?»

Personne ne répondit.

Tic.

Quelque part au fond de son esprit, un sentiment d'horreur s'élevait, tandis que d'autres parties de lui essayaient de décider exactement à quel point il devait paniquer. Les premiers cours de Défense de l'année étaient assez flous dans son esprit, mais il avait un lointain souvenir de trolls capables de traquer leur proie quand elle était seule et sans défense.

Tic.

Un autre cheminement de pensée chercha frénétiquement parmi des possibilités partiellement formées~; que pouvait-il faire exactement~? Il n'était pas encore 15h, il ne pouvait donc pas encore utiliser son Retourneur de Temps. Même s'il parvenait à sortir de la pièce -- il devait y avoir un moyen de mettre sa Cape sans être remarqué, une distraction qu'il pourrait utiliser -- il n'avait aucune idée d'où se trouvait Hermione, et Poudlard était immense.

Tic.

Une autre partie de son esprit tentait de modéliser les possibilités. Selon un autre élève, les trolls n'étaient pas des prédateurs silencieux, ils étaient bruyants…

Hermione ne saura pas qu'il s'agit d'un troll, alors elle ira enquêter sur le bruit. Après tout, c'est une héroïne.

… mais la bourse de Hermione était maintenant munie d'une cape d'invisibilité et d'un balai volant. Harry avait insisté sur ce point pour elle et pour Neville, et le professeur McGonagall lui avait dit que cela avait été fait. Cela devait suffire à permettre à Hermione de s'enfuir, même si elle était mauvaise en balai volant. Tout ce qu'elle avait à faire était d'atteindre l'un des toits, le jour était limpide et le soleil était censé être néfaste aux trolls, Harry se souvenait de cette partie du cours et Hermione s'en souviendrait donc parfaitement. Et bien sûr, même si Hermione souhaitait à nouveau prouver sa valeur, elle ne pouvait pas être assez stupide pour attaquer un troll des montagnes.

Tic.

Elle ne ferait pas ça.

Tic.

Ça ne lui ressemblait juste pas.

Puis l'idée vint à Harry que quelqu'un avait déjà essayé de faire accuser Hermione Granger de meurtre en utilisant des sortilèges de faux souvenir. À l'intérieur de Poudlard, sans déclencher d'alarme. En s'arrangeant pour que Drago meure si lentement que le système de sécurité ne se déclenche qu'au moins six heures plus tard, quand personne ne pourrait utiliser un Retourneur de Temps pour voir ce qui s'était passé. Et que quiconque était assez intelligent pour faire passer un troll à travers l'antique système de sécurité de Poudlard sans que le directeur ne vienne inspecter l'étrange créature pourrait aussi être assez intelligent pour prendre l'évidente disposition de rendre les objets magiques de Hermione inutilisables…

Tic.

Il y avait une autre partie de lui qui sentait quelque chose ressemblant à de la panique lentement monter, à mesure que sa perspective sur la situation changeait, comme un cube de Necker que l'on aurait fait tourner, bon sang, mais qu'est-ce qui lui était passé par la tête, laisser Hermione et Neville dans Poudlard à cause de quelques babioles stupides qu'on leur avait données, ça n'allait arrêter personne décidé à les tuer.

Tic.

Une autre partie de son esprit résista~: cette possibilité n'était pas certaine, elle était complexe, et sa probabilité pouvait facilement être inférieure à 50~\%. Il était facile de s'imaginer paniquer complètement devant tout le monde et voir ensuite Hermione revenir des toilettes juste devant la grande salle. Ou si le troll s'avérait ne jamais l'avoir approchée… comme dans l'histoire du garçon qui criait au loup, personne ne le croirait la prochaine fois, quand elle serait vraiment dans de sales draps~; ça pourrait dépenser tout le capital réputation dont il aurait besoin pour quelqu'un d'autre, plus tard…

Tic.

Harry reconnu cette instance du schéma peur-d'être-gêné qui empêchait la plupart des gens de jamais faire quoi que ce soit lorsqu'ils n'étaient pas parfaitement sûrs d'eux et l'écrasa avec force. Même alors, il était étrange de constater quelle force de volonté il lui fallait rassembler pour se décider à crier devant tout le monde, s'il n'avait juste pas vu Hermione dans la foule, ça allait être gênant…

Tic.

Harry prit une profonde inspiration et s'écria aussi fort qu'il le pouvait~:

«Hermione Granger~! Est-ce que tu es là~?»

Les élèves se tournèrent tous vers lui. Puis certains regardèrent autour d'eux. Le bruit de la pièce diminua lorsque certaines conversations s'interrompirent.

«Est-ce que quelqu'un a vu Hermione Granger depuis… depuis environ dix heures trente ce matin~? Quelqu'un a-t-il la moindre idée de l'endroit où elle pourrait se trouver~?»

Le babillage diminua encore plus.

Personne n'éleva la voix pour crier quelque chose à son intention, et en particulier, pas ne t'inquiète pas Harry, je suis là.

«Oh, Merlin», dit quelqu'un de proche, et le babillage reprit sur un nouveau ton agité.

Harry regarda ses mains, il fit abstraction des braillements et essaya de penser, penser, PENSER…

Tic.

Tic.

Tic.

Susan Bones et un garçon roux muni d'une baguette qui semblait avoir pris des coups jouèrent en même temps des épaules à travers la foule pour se diriger vers Harry.

«Il faut qu'on trouve un moyen de le dire aux professeurs…

--- On doit la trouver…

--- La trouver~? lâcha Susan en se réorientant vers l'autre garçon. Et comment ferons-nous ça, capitaine Weasley~?

--- On doit partir et la chercher~! lâcha Ron Weasley en réponse.

--- Tu es dingue~? Il y a déjà des professeurs qui patrouillent les couloirs, qu'est-ce qui te fait penser que tu as une meilleure chance qu'eux de croiser le général Granger~? Sauf que nous on se fera manger par le troll~! Et ensuite exclure~!»

C'était étrange comme parfois entendre de mauvaises idées rendait les bonnes idées évidentes par effet de contraste.

«Attention tout le monde~! Écoutez-moi~!»

Les gens se tournèrent pour le regarder.

«\shout{Silence~! Tout le monde~! Taisez-vous}~!»

La gorge de Harry lui fit mal après cette phrase, mais il avait obtenu l'attention de tous.

«J'ai un balai volant», dit-il aussi fort que possible avec sa gorge encore douloureuse. Lorsqu'il avait demandé un balai capable de transporter trois personnes, c'était parce qu'il s'était souvenu d'Azkaban et du balai sur lequel seulement deux avaient pu s'asseoir. «Il est pour trois personnes. J'ai besoin qu'un septième armé enrôlé dans une des armées vienne avec moi. On va traverser les couloirs en volant le plus vite possible pour trouver Hermione Granger, la ramasser et revenir immédiatement. Qui est avec moi~?»

La grande salle fut alors plongée dans un silence total.

Les élèves se regardaient, mal à l'aise. Les plus jeunes regardaient les plus âgés dans l'expectative, qui eux-mêmes se tournaient vers ceux qui protégeaient le périmètre. La plupart de ceux-ci regardaient droit devant eux, leurs baguettes levées, juste au cas où le troll choisirait ce moment pour traverser le mur.

Nul ne bougea.

Nul ne parla.

Harry Potter éleva de nouveau la voix. «On ne se battra pas contre le troll. Si on le voit, on se contentera de fuir, et il ne pourra jamais rester à notre vitesse si on est sur un balai volant. Je prends la responsabilité d'arranger les choses avec l'administration. S'il vous plaît.»

Les gens continuèrent de regarder quelqu'un d'autre.

Harry regarda la foule silencieuse, la dizaine de septième année qui regardait les murs d'un air sévère, et sentit le froid s'emparer de lui. Quelque part au fond de son esprit, le professeur Quirrell riait avec mépris et se moquait de l'idée que des idiots ordinaires feraient jamais quoi que ce soit d'utile de leur propre chef sans une baguette pointée vers leur crâne…

Tic.

Le remède standard à l'effet du témoin était de se concentrer sur un individu en particulier. «Très bien», dit Harry, en essayant de maintenir la voix autoritaire du Survivant qui ne doutait pas qu'on lui obéisse. «Mlle Morgan, venez avec moi, maintenant. Nous n'avons pas de temps à perdre.»

La sorcière dont il avait prononcé le nom détacha son regard de l'extérieur du périmètre et se tourna vers lui, son visage un instant horrifié avant de se refermer.

«La directrice en chef nous a tous ordonnés de rester ici, M. Potter.»

Harry dut faire un effort pour desserrer son poing. «Le professeur Quirrell n'a pas dit ça et vous non plus. Le professeur McGonagall n'est pas tacticienne, elle n'a pas pensé à vérifier si des élèves manquaient à l'appel, et elle pensait que c'était une bonne idée de commencer à faire déambuler les élèves à travers les couloirs. Mais le professeur McGonagall comprend une fois que ses erreurs ont été portées à son attention, tu as vu comment elle vous a écoutés, toi et le professeur Quirrell, et je suis certain qu'elle ne voudrait pas que nous ignorions juste le fait que Hermione Granger est là, dehors, seule…»

Tic.

«Je pense que le professeur dirait qu'elle ne souhaite plus voir aucun autre élève se promener dans les couloirs. Elle a dit que si quelqu'un partait, quelle qu'en soit la raison, il ou elle serait exclu~! Peut-être que tu n'as pas besoin de t'en faire parce que tu es le Survivant, mais nous autres, si~!»

Tic.

Quelque part au fond de son esprit, le professeur Quirrell riait de lui à gorge déployée. S'attendre à ce qu'une personne normale agisse sans que les choses soient parfaitement limpides sur le plan stratégique, sans que la responsabilité n'ait clairement été dirigée vers elle personnellement, et alors qu'elle avait une bonne excuse pour ne rien faire… «La vie d'une élève est en jeu, dit Harry d'un ton neutre. Elle pourrait être en train de se battre contre le troll, maintenant. Simple curiosité, est-ce que cela signifie quoi que ce soit pour toi~?»

Tic.

Le visage de Mlle Morgan se tordit. «Tu… tu es le Survivant~! Vas-y juste tout seul et claque des doigts si tu veux l'aider~!»

Tic.

Harry se rendait à peine compte de ce qu'il disait~:

«Ça c'est juste de l'ingéniosité et du bluff, je n'ai pas de pouvoir comme ça dans la vraie vie, une jeune fille a besoin de ton aide, maintenant est-ce que tu es Gryffondor ou pas~?

--- Pourquoi est-ce que tu me dis tout ça à moi~? s'écria Mlle Morgan. Ce n'est pas moi qu'on a nommée responsable~! C'est M. Hagrid~!»

Un silence gêné prit place dans toute la pièce.

Harry se retourna pour regarder l'immense demi-géant qui dominait la foule et toutes les autres têtes se tournèrent vers ce dernier d'un seul mouvement.

«M. Hagrid, dit Harry en essayant de maintenir son ton imposant. J'ai besoin que vous autorisiez cette expédition et j'ai besoin que vous le fassiez maintenant.»

Rubeus Hagrid semblait hésiter, bien que ce fut difficile à juger avec sa large tête entourée qu'elle était d'une barbe et de boucles laissées à l'abandon~; seuls ses yeux semblaient vivants, encastrés dans tous ces cheveux.

«Euh… dit le demi-géant. On m'a dit d'vous protéger…

--- Génial, maintenant pourriez-vous aussi protéger Hermione Granger~? Vous savez, l'élève accusée à tort d'un crime qu'elle n'avait pas commis et qui a besoin que quelqu'un l'aide~?»

Le demi-géant tressaillit en entendant ces mots.

Harry regarda l'homme énorme en souhaitant désespérément qu'il comprendrait le sous-entendu, en espérant qu'il n'avait rien révélé à personne d'autre… il ne pouvait pas être qu'une masse de muscles, James et Lily n'avaient certainement pas été amis avec l'homme que par pitié…

«À tort~? s'écria une voix anonyme venue de la zone Serpentard. Ha, tu continues avec ça~? Ça sera bien mérité si elle se fait manger.»

Il y eut quelques rires en même temps que quelques cris d'indignations venus d'ailleurs.

Le visage du demi-géant gagna en assurance.

«Toi tu restes là, mon gars», dit M. Hagrid d'une voix retentissante qui essayait probablement d'être douce. «J'vais la trouver d'mon chef. L'truc, c'est qu'les trolls peuvt'être un brin roublards -- faut les choper par un talon et les s'couer juste c'qui faut ou y t'couperont net en deux…

--- Pouvez-vous monter un balai volant, M. Hagrid~?

--- Eh…» Rubeus Hagrid fronça les sourcils. «Non.

--- Alors vous ne pouvez pas chercher assez vite~! Sixième année~! J'appelle tous les sixième année~! Est-ce qu'il y a ici des sixième année qui ne sont pas des lâches doublés de bons à riens~?»

Silence.

«Cinquième année~? M. Hagrid, dites-leur qu'ils ont le droit de venir avec moi pour me protéger~! J'essaie d'être raisonnable, bon sang~!»

Le demi-géant se tordit les mains et son visage exprimait une hésitation atroce. «Euh… je…»

Quelque chose craqua en Harry et il commença à marcher directement vers les portes de la grande salle en repoussant tous ceux qui ne s'écartaient pas comme s'ils avaient été des statues de pâte. (Il ne courut pas, parce que courir revenait à inviter à se faire arrêter). Quelque part dans son esprit, il traversait une pièce vide emplie de marionnettes mécaniques dont les bruits dénués de sens nés de l'agitation de leurs lèvres l'avaient distrait…

Une immense silhouette se plaça sur son chemin.

Harry leva les yeux.

«J'peux point te laisser faire ça, Harry Potter, surtout pas toi. Y a des choses étranges qui rôdent dans ce château, et quelqu'un pourrait en vouloir à Mlle Granger… ou il pourrait en vouloir à toi.» La voix de Rubeus Hagrid était emplie de regrets mais ferme, et ses mains immenses reposaient contre ses flancs telles des plaques de chariots élévateurs. «J'peux point te laisser sortir, Harry Potter.

--- Stupéfix~!»

Le tir rouge s'écrasa sur la tempe de Hagrid et fit tressaillir l'immense homme. Il tourna la tête à une vitesse qui aurait dû être interdite à des objets de cette taille et mugit~: «\emph{Qu'est-ce-tu} crois faire~?» à l'intention de la jeune silhouette de Susan Bones.

«Pardon~! cria-t-elle. Incendium~! Glisseo~!»

Les immenses mains de l'homme s'abattaient maintenant sur le feu dans sa barbe et ne purent tout à fait le retenir dans sa chute, mais ça n'avait plus d'importance car Harry l'avait dépassé et…

Neville Londubat se dressa devant lui, l'air désespéré mais déterminé, la baguette du Poufsouffle déjà dans sa main.

La baguette de Harry bondit dans la sienne par pur réflexe, parvint à peine à interrompre son geste avant que Neville ne lui tire dessus et regarda son Lieutenant comme si le monde était devenu fou.

«Harry~! éclata Neville. Harry, M. Hagrid a raison, tu ne peux pas y aller, ça pourrait être un piège, ils pourraient en avoir après toi…»

Les muscles de Neville se rigidifièrent et il s'effondra au sol, raide comme une planche.

Un Ron Weasley à l'air pâle apparut derrière Neville, sa propre baguette levée, et dit~:

«Vas-y.

--- Ron, espère de malade, qu'est-ce que tu fais…» la voix distante mais reconnaissable du petit ami de Mlle Deauclaire leur parvint, mais Harry fonçait déjà vers la porte sans un regard en arrière, alors même que les voix de Ron et de Susan s'élevaient dans une nouvelle incantation. Il y eut un immense mugissement indigné et des voix inconnues commencèrent à hurler.

Puis Harry passa les portes et il plongea la main dans sa bourse, sa voix dit «balai volant» et derrière lui les grandes portes commencèrent à se refermer.

Il continua de courir dans l'entrée pendant que le long balai pour trois et ses étriers commençaient à émerger de la bourse et qu'il répétait mentalement un certain nombre de gros mots en se disant voilà ce qui se passe quand tu essaies d'être raisonnable avec la partie de son esprit qui n'essayait pas de trouver un plan de recherche qui recouvre les endroits où Hermione pourrait se trouver. La bibliothèque était au troisième étage et pratiquement à l'autre bout du château… Harry avait presque atteint le grand escalier de marbre lorsque le balai arriva dans sa main et «Debout~!» il fut dans les airs et accéléra vers le premier étage…

«Argh~!» cria-t-il, et il parvint à peine à faire pivoter son balai pour ne pas empaler l'une des silhouettes humaines qui rôdaient en haut des escaliers. Il eut un instant de peur lorsqu'il essaya de ne pas tomber du balai, et de réaliser les acrobaties qui le maintiendraient dans les étriers malgré la proximité du sol et l'absence quasi totale de place pour manœuvrer quand alors…

«Fred~? George~?

--- On n'arrivait pas à la trouver~!» laissa échapper l'un des jumeaux Weasley en agitant les mains de désarroi. «On s'est faufilés à l'extérieur parce qu'on pensait pouvoir trouver Mlle Granger -- il doit y avoir un moyen de trouver n'importe qui dans le château de Poudlard, on en est sûrs tous les deux -- mais on ne peut pas trouver ce que c'est~!»

Harry les regarda tous deux, renversé sous son balai, là où sa manœuvre désespérée l'avait emmené, et par pur réflexe sa bouche dit~: «Eh bien pourquoi étiez-vous si certain de pouvoir la trouver~?

--- On ne sait pas~! s'écria l'autre jumeau Weasley.

--- Avez-vous déjà été capable de retrouver des gens dans Poudlard~?

--- Oui~! On…» et les jumeaux Weasley se turent soudain, et ils regardèrent dans le vide, leurs deux visages roux devenus inexpressifs.

Il y eut un bruit de tonnerre et les deux immenses portes furent ouvertes par quelqu'un de très, très fort.

Harry se retourna pour montrer les deux paires d'étriers sur le balai aux jumeaux Weasley mais ne dit rien, ils n'avaient aucune raison de révéler leur position tant qu'on ne les y forçait pas. Le temps semblait avancer trop lentement alors que les jumeaux Weasley s'installaient tant bien que mal sur les étriers, et le cœur de Harry battait fort malgré son calcul mental qui lui disait qu'en courant, M. Hagrid n'arriverait même pas au pied de l'escalier à temps. Puis ils accélérèrent vite vers le couloir le plus proche, la pierre sous eux devint floue, les murs semblèrent émettre un sifflement (c'était juste le vent dans leurs oreilles) sur leur passage~; Harry se souvint qu'il conduisait un balai plus long destiné à trois personnes juste à temps pour pouvoir ralentir avant le prochain virage.

Et à présent tous les sièges étaient occupés, mais s'ils trouvaient vraiment Hermione, alors Harry pourrait mettre la Cape d'Invisibilité, cela devrait le cacher du troll et libérer un siège pour Hermione…

Harry se baissa très vite avant qu'un porche voûté ne lui arrache la tête.

«On a trouvé Jessie~!» lâchèrent les jumeaux Weasley, assis derrière Harry. «On en est sûrs~! La fois où on devait lui dire que Rusard le pourchassait~!

--- Comment~?» dit Harry, la majeure partie de son cerveau consacrée à éviter sa mort dans un horrible accident aérien. Il aurait dû ralentir par mesure de précaution mais une tension montait à l'intérieur de lui, une sensation d'effroi sans source. Il ne pouvait pas ralentir, quelque chose d'horrible se produirait s'il ralentissait…

«On… dirent les jumeaux Weasley assis un peu plus bas. On n'arrive pas à s'en souvenir~!»

Un autre virage sec à environ 0,3~\% de la vitesse de la lumière selon les estimations de Harry, et ils traversèrent un corridor courbé et tortueux que Harry empruntait toujours pour aller de la grande salle à la bibliothèque, sauf que ça n'était pas le chemin le plus court quand on était sur un balai volant, il aurait plutôt dû prendre le long couloir ouest, qui était droit…

La partie de son cerveau qui ne virait pas de bord revint à la réalité.

«Quelqu'un a altéré vos esprits~!» cria Harry tout en ondulant à travers le couloir si vite que les Weasley à l'arrière frappaient parfois légèrement le mur, car la longueur du balai n'était pas adaptée aux compétences aériennes de Harry.

«Quoi~? crièrent Fred et George.

--- Celui qui a eu Hermione a aussi trafiqué vos esprits~!» Ça pourrait être un sortilège d'Oubliettes, ou de Faux Souvenirs mal implanté, mais pour le moment Harry ne pouvait pas réfléchir…

Le balai tourna et monta à la verticale à côté d'un escalier en colimaçon, et ils s'aplatirent tous les trois contre le balai afin de pouvoir traverser l'ouverture dans le plafond qui donnait sur le troisième étage, puis ils furent face à la bibliothèque et le balai ralentit puis s'arrêta avec un couinement en dépit de l'absence totale de matière contre laquelle son freinage aurait pu faire frottement. Harry jeta aux jumeaux Weasley un rapide regard qui leur disait de rester là tout en faisant l'escalade nécessaire pour descendre du balai puis il ouvrit grand les portes de la bibliothèque en contrôlant sa respiration et passa sa tête à l'intérieur.

Hermione Granger n'était pas là.

Madame Pince, qui mangeait un sandwich à son bureau, leva les yeux d'un air furieux.

«Bibliothèque fermée~!

--- Avez-vous vu Hermione Granger~? dit Harry.

--- J'ai dit que la bibliothèque était fermée, petit~! C'est l'heure du déjeuner~!

--- C'est extrêmement important. Avez-vous vu Hermione Granger ou avez-vous la moindre idée de là où elle pourrait être~?

--- Non, dehors maintenant~!

--- Avez-vous un moyen de contacter rapidement le professeur McGonagall en cas d'urgence~?

--- Hein~?» dit la bibliothécaire, alarmée. Elle se leva de son bureau. «Qu'est-ce qui…

--- Oui on non. Répondez immédiatement s'il vous plaît.

--- Ah… il y a le feu de cheminette…

--- Elle n'est pas dans son bureau, dit Harry. Avez-vous un autre moyen d'entrer en contact avec elle. Oui ou non.

--- Jeune homme, j'insiste pour que vous…»

Le cerveau de Harry apposa l'étiquette Voilà que je reparle à des PNJ à cette discussion et il pivota sur ses talons et fonça vers le balai volant.

«Arrêtez~!» cria Madame Pince, surgissant trop tard des portes~; Harry et les jumeaux Weasley avaient déjà pris la poudre d'escampette loin des yeux de la bibliothécaire. La pression dans l'esprit de Harry montait toujours, comme une main qui aurait physiquement comprimé sa poitrine, il devait trouver Hermione et il n'avait pas d'autre idée quant à son emplacement possible, à moins qu'elle ne soit dans les dortoirs des sorcières de la tour Serdaigle auquel cas il ne pourrait pas rentrer. Parcourir l'intégralité de Poudlard avoisinait l'impossibilité mathématique~; il n'y avait probablement pas de plan de vol continu qui entrait dans toutes les pièces au moins une fois -- pourquoi n'avait-il pas pensé à exiger que Hermione et Neville et lui reçoivent ces pratiques petits miroirs que les Aurors utilisaient pour communiquer…

Le fait qu'il se comportait de façon stupide frappa Harry comme un coup à l'estomac. Il n'avait pas besoin de miroirs pour envoyer un message~; Il n'en avait pas eu besoin depuis janvier. Il ralentit le balai et l'arrêta en plein vol au milieu d'un couloir, sa baguette déjà dans sa main, la volonté de protéger Hermione Granger montant au faîte de son esprit comme un soleil de feu argenté puis coulant le long de son bras lorsqu'il s'écria~:

«EXPECTO PATRONUM~!»

et l'humanoïde d'un blanc étincelant apparut comme une nova, et les voix des jumeaux Weasley poussèrent un cri de surprise.

«Dis à Hermione Granger -- qu'il y a un troll en liberté dans Poudlard -- il pourrait la pourchasser -- elle doit se mettre au soleil, maintenant~!»

La silhouette d'argent se retourna comme pour partir puis disparut.

«Par les caleçons de Merlin», souffla Fred ou George.

La silhouette d'argent réapparut d'un coup et dit de l'étrange version externe de la voix de Harry~: «Hermione Granger dit~:», la voix de l'éclatante silhouette devint aiguë~: «\scream{Ahhhhhhhhh}~!».

Le temps sembla se briser, comme si tout se déplaçait très vite et ralentissait en même temps. Une pulsion désespérée de faire accélérer le balai, de voler à vitesse maximale, sauf que Harry ne savait pas où…

«Si tu sais où elle est», cria à Harry à la silhouette humanoïde faite de lumière, en la regardant comme on aurait regardé le soleil, «alors amène-moi à elle~!»

L'éclat d'argent se déplaça et Harry accéléra à sa suite, les jumeaux Weasley émirent des couinements aigus derrière lui lorsqu'il fonça à travers les airs comme un boulet de canon, allant à une vitesse folle, sans penser aux murs qui sifflaient sur son passage ni à sa vitesse, suivant juste la lumière d'argent à travers les couloirs, volant au-dessus d'escaliers, fusant entre des portes que Fred ou George ouvraient à coups d'incantations désespérées et tout ça prenait encore trop longtemps, quelque chose de profondément enfoui en Harry avait le sentiment qu'il s'enfonçait dans de la mélasse alors même que les fenêtres et les portraits défilaient.

Le balai décrit un ultime virage qui emboutit l'un des jumeaux Weasley contre un mur pas tout à fait avec la force qu'un Cognard y aurait mise puis ils suivirent l'étincelant Patronus jusqu'à un espace ouvert dans le plafonnage, ils foncèrent vers le haut, passèrent un étage puis un autre en moins d'une respiration.

Son Patronus ralentit, s'arrêta (Harry réagit en freinant avec force) juste alors qu'ils atteignaient un espace grand ouvert qui s'étendait jusqu'à échapper au plafonnage et se transformait en une terrasse découverte, un plan de dalles de marbres exposé à l'air et au ciel…
%  LocalWords:  breen Roopo Brienne Pomona Pinnini’s Vadim’s tol yeh meself
%  LocalWords:  yeh’ve yeh’re Gah NPCs Ahhhhhhhhh
