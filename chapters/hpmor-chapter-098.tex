\chapter{Rôles, Fin}

\section{Dimanche 19 avril, 18h34.}

\lettrine{A}{près} avoir quitté le Poudlard Express, Daphné Greengrass marcha sans bruit vers la chambre de Greengrass située sous les donjons Serpentard - privilège d'une Ancienne Maison - afin de déposer sa malle avant de rejoindre les autres élèves pour le dîner. Tous les appartements privés étaient devenus sa demeure exclusive depuis que Malfoy était parti. Sa main, dans son dos, enjoignait encore et encore son immense malle sertie d'émeraudes à la suivre. Cette dernière semblait hésiter à le faire~; peut-être les sortilèges lancés sur l'ancien et solide artefact familial devaient-ils être réappliqués~; ou peut-être sa malle était-elle peu encline à la suivre dans une école dont la sûreté n'était plus garantie.

Papa et Maman avaient longuement discuté lorsqu'il avaient entendu la nouvelle au sujet de Hermione~; Daphné Greengrass les avait écoutés cachée derrière une porte entrebâillée en ravalant ses larmes et en essayant de ne pas faire de bruit.

Mère avait dit qu'aussi triste que ce fût, il fallait admettre que même si un élève mourrait chaque année à Poudlard, l'endroit serait toujours plus sûr que Beauxbâtons, sans même parler de Durmstrang. Une jeune sorcière pouvait aisément mourir sans être la victime d'un assassin. Mère avait ajouté que le maître de métamorphose de Beauxbâtons n'était tout simplement pas aussi bon que McGonagall.

Père avait gravement fait remarquer qu'il était très important que l'héritière Greengrass reste à Poudlard, là où toutes les autres familles nobles envoyaient leurs enfants (c'était la raison derrière la vieille tradition qu'avaient les familles nobles de synchroniser la naissance de leurs héritiers afin de les envoyer, si possible, dans la même année à Poudlard). Et Père avait dit que l'héritière d'une Très Ancienne Maison ne pouvait pas toujours rester à l'écart du danger.

Elle aurait autant aimé ne pas entendre cette partie.

Après avoir tourné la poignée et ouvert la porte, Daphné déglutit soudain.

«Mlle Greengrass…» murmura une silhouette indistincte et drapée d'argent.

Daphné cria, ferma la porte, sortit sa baguette et se retourna pour fuir.

«Attends~!» s'écria la voix avec plus de force et d'un ton plus aigu.

Daphné s'interrompit. Il était \emph{impossible} que la personne à qui cette voix appartenait soit ici.

Daphné se tourna lentement et ouvrit à nouveau la porte.

«\emph{Toi~!}» dit Daphné, frappée de stupeur, et elle vit le visage sous le capuchon. «Je pensais que tu étais…

--- Je reviens vers toi, dit la silhouette à robe d'argent d'une voix forte, en ce moment dé…

--- \emph{Qu'est-ce que tu fais dans ma chambre~?} glapit Daphné.

--- Greengrass, j'ai entendu dire que tu pouvais produire la forme brumeuse du Patronus, dit la silhouette en robe d'argent. Puis-je la voir~?»

Daphné le regarda en silence, et son sang se mit soudain à bouillir. «Pourquoi~?» dit-elle en gardant sa baguette levée. «Pour que tu puisses \emph{tuer} tous ceux qui appartiennent à Serpentard mais savent lancer ce genre de sortilège~? On \emph{sait} tous qui a fait assassiner Hermione~!»

La voix de la silhouette s'éleva. «J'ai témoigné sous Veritaserum que j'ai essayé d'aider Granger~! J'essayais vraiment de l'aider quand j'ai attrapé sa main sur le toit, et quand je l'ai aidée à se relever…»

Daphné gardait sa baguette levée. «Comme si ton père ne pouvait pas manipuler le compte-rendu des Aurors~! Je ne suis pas née de la dernière pluie, \emph{Malfoy}~!»

Lentement, comme pour ne pas l'effrayer, la silhouette d'argent fit apparaître une baguette de sous ses robes. La main de Daphné se resserra sur la sienne, mais elle reconnut alors la position des doigts, la posture que la silhouette adoptait, et elle inspira soudain…

«\emph{Expecto Patronum}», dit clairement Drago Malfoy.

Une lumière d'argent bondit de l'extrémité de la baguette de Malfoy - et se condensa pour former un serpent de lumière qui parut se lover en l'air, comme s'il se reposait là.

Elle en fut bouche bée.

«J'ai \emph{vraiment} essayé d'aider Hermione Granger, dit Drago Malfoy d'une voix neutre. Parce que je connais la maladie qui ronge le cœur de la maison Serpentard, la raison pour laquelle nous sommes si nombreux à ne plus pouvoir lancer le Patronus. C'est notre haine. Les gens n'associent plus Serpentard qu'à ça, aujourd'hui, pas à notre ruse, à notre ambition, ou à notre noblesse, seulement à notre haine des Moldus. Et même moi je sais, parce que c'est évident si on se laisse le voir, que Hermione Granger n'était pas magiquement faible.»

L'esprit de Daphné était comme assommé. Ses yeux vérifièrent nerveusement l'espace sous les portes pour vérifier que du sang n'en sortait pas, comme la dernière fois que Quelque Chose s'était Brisé.

«Et j'ai aussi compris», dit doucement Drago Malfoy, alors que le serpent d'argent continuait de briller de sa lumière et de son indéniable chaleur, «que Hermione Granger n'a jamais vraiment essayé de me tuer. Peut-être qu'elle a subit un sortilège de faux souvenirs, peut-être qu'elle a été victime de Légilimancie, mais maintenant qu'elle s'est faite assassiner, il est évident que, le jour où quelqu'un a essayé de faire accuser Granger de mon assassinat, elle était la cible principale.

--- Est-est-est-ce que tu te rends compte de ce que tu \emph{dis}~?» la voix de Daphné se brisa. Si Lucius Malfoy entendait que son fils avait dit ça il \emph{l'écorcherait} et il en ferait un \emph{pantalon}~!

Drago Malfoy sourit, sa robe métallique scintillante sous la lumière de son Patronus complet~; c'était un sourire à la fois arrogant et dangereux, comme si s'inquiéter qu'on puisse le transformer en un pantalon de cuir était indigne de lui. «Oui, dit-il, mais ça n'a plus d'importance à présent. La maison Malfoy rend son argent à la maison Potter et annule sa dette.»

Daphné marcha jusqu'à son lit et s'y laissa choir, espérant qu'elle pourrait alors se réveiller de ce rêve.

«J'aimerais que tu rejoignes une conspiration, dit la silhouette en robe étincelante. Pour tous ceux à Serpentard capables de lancer le Patronus et tous ceux capables de l'apprendre. C'est grâce à cela que nous pourrons nous faire confiance quand nous nous retrouverons entre Serpentard Scintillants.» D'un geste théâtral, Drago Malfoy releva sa capuche. «Mais ça ne marchera pas sans \emph{toi}, Daphné Greengrass. Toi et ta famille. Ta mère négociera cela avec Père, mais je voudrais que les Greengrass entendent d'abord la proposition de ta bouche.» La voix de Drago Malfoy devint plus basse, plus sinistre. «Nous avons beaucoup à nous dire avant le dîner.»

\later

Harry Potter avait apparemment choisi d'être toujours invisible~; ils n'avaient que brièvement aperçu sa main lorsqu'il leur avait donné la liste, écrite sur un étrange non-parchemin. Harry avait expliqué que tout bien considéré, il n'aurait pas été malin de sa part de se laisser être facile à \emph{trouver}, mis à part lors d'occasions particulières, et qu'il allait donc se contenter d'interagir avec les gens sous la forme d'une voix sans corps, ou sous celle d'une puissante lumière argentée cachée à l'angle d'un couloir, là où personne ne pourrait la voir, et qui pourrait toujours trouver ses amis, où que ceux-ci se cachent. C'était, pour être tout à fait honnête, l'une des choses les plus effrayantes que Fred et George avaient jamais entendues, et ce bien qu'ils aient déjà remplit les chaussures de tous les élèves en deuxième année avec des mille-pattes métamorphosés. Ils étaient d'avis que ça ne serait bon pour la santé mentale de personne, mais ils ne savaient pas quoi répondre. Ils ne pouvaient pas nier, l'ayant vu de leurs quatre yeux, que Poudlard…

… n'était plus sûre…

«Je ne sais qui vous êtes allés voir pour le sortilège de faux souvenirs sur Rita Skeeter, dit la voix sans source de Harry Potter. Qui que ce soit,… il ne pourra probablement \emph{pas} satisfaire directement cette commande, mais il connaîtra peut-être quelqu'un capable de faire venir des choses du monde Moldu. Et… je sais que ça coûtera peut-être plus cher, mais un minimum de personnes doivent savoir que Harry Potter a quoi que ce soit à voir avec ça.» Une autre vision fugace de la main d'un petit garçon, et un sac tomba au sol dans un bruit de métal. «Certains de ces objets coûtent cher, même dans le monde Moldu, et votre contact devra peut-être sortir d'Angleterre~; mais ces cent Gallions devraient suffire à tout payer, j'espère. Je vous dirais bien d'où ces Gallions viennent, mais je ne voudrais pas vous gâcher la surprise de demain.

--- Qu'est-ce que c'\emph{est} que ces trucs~?» dit Fred ou George, tandis qu'ils regardaient la liste. «Notre père est expert en Moldus…

--- … et on n'en reconnaît même pas la \emph{moitié}…

--- … on n'en reconnaît même aucun…

--- … qu'est-ce que tu comptes \emph{faire} exactement~?

--- Nous sommes passés au choses sérieuses, dit doucement la voix de Harry. Je ne sais pas ce que je vais devoir faire. J'aurai peut-être besoin du pouvoir des Moldus, pas seulement de celui des sorciers, avant d'en avoir fini avec tout ceci - et j'en aurai peut-être besoin dans l'urgence, sans avoir le temps de me préparer. Je ne \emph{compte} pas utiliser tout cela. Je veux juste l'avoir dans le coin pour pouvoir parer à toute… éventualité.» La voix de Harry marqua une pause. «Je vous dois bien sûr plus que ce que je pourrai jamais vous offrir et vous ne me \emph{laissez} pas vous donner \emph{un centième} de ce que vous méritez, je ne sais même pas comment vous remercier correctement, et tout ce que j'espère c'est qu'un jour, quand vous aurez grandi, vous serez plus raisonnables à ce sujet et que vous \emph{voudrez bien} accepter une prime de dix pour cent…

--- Tais-toi, toi, dit George ou Fred.

--- Bon Dieu, vous avez combattu un troll pour moi et Fred s'est fait briser les côtes~!»

Ils secouèrent tous deux la tête. Harry était resté là quand ils lui avaient dit de fuir, puis il s'était mis en première ligne pour distraire le troll qui voulait manger George. Ils savaient que Harry était le genre de personne qui croyait que ce genre de chose n'annulait pas ce qu'il devait aux jumeaux Weasley, que son acte n'était pas à la mesure du leur. Mais ce que les jumeaux Weasley savaient, ce que Harry comprendrait quand il serait plus âgé, c'était qu'il n'y avait pas, qu'il ne pourrait même jamais y avoir de dette entre eux. C'était là un étrange égoïsme, songeaient-ils, que d'être capable, comme Harry, de comprendre sa propre bonté - de ne jamais rêver d'exiger de l'argent de ceux qu'il avait aidés plus qu'ils ne l'avaient aidé, ou d'appeler cela une dette - tout en étant apparemment incapable de concevoir que d'autres pourraient un jour vouloir agir de même envers \emph{lui}.

«Rappelez-moi de vous acheter un exemplaire du roman moldu \emph{La Grève}, dit la voix sans source. Je commence à comprendre quel genre de personne gagne à le lire.»

\latersection{Lundi 20 avril, 19h.}

Cela eut lieu sans intervention, sans signe venu de la grande table, alors que les élèves achevaient leur calme dîner~; cela eut lieu sans qu'autorisation ni pardon ne soit demandé ni auprès des professeurs ni auprès du directeur.

Peu de temps après que les desserts eurent disparus, un élève se leva de la table Serpentard et s'avança rapidement, non pas vers la grande table mais vers le mur opposé aux quatre tables de Poudlard. Quelques murmures s'élevèrent à la vue des cheveux blond-blanc coupés court lorsque Drago Malfoy se tint à ce qui avait été l'arrière des tables et regarda Poudlard en silence. Il n'avait presque rien dit depuis son retour surprise et avait été encore moins vu. Le Serpentard n'avait daigné ni confirmer ni nier qu'il était revenu parce que, une fois Hermione Granger tuée par la famille de ce dernier, il n'avait plus rien à craindre.

Puis Drago Malfoy prit une cuillère dans une main, un verre d'eau dans l'autre, et commença à taper le verre avec la cuillère, produisant ainsi un clair tintement.

Ceci produisit d'abord des bavardages agités. À la grande table, les différents professeurs regardèrent avec perplexité le directeur assit dans sa grande chaise, mais il ne leur fit aucun signe, et les professeurs ne firent donc rien.

Drago Malfoy continua de taper le verre de sa cuillère jusqu'à ce que la pièce se taise et attende.

Puis un autre élève se leva de la table Serdaigle et s'avança jusqu'à Drago Malfoy avant de se retourner pour faire face à Poudlard à son côté. Des souffles furent coupés par l'ébahissement~; il était impossible que ces deux-là soient aujourd'hui autre chose que les pires de tous les ennemis.

«Moi et mon Père, le Lord de la noble et très ancienne maison Malfoy, dit Drago Malfoy d'une voix claire, nous sommes rendus compte que des forces néfastes sont à l'œuvre à Poudlard. Que ces forces néfastes souhaitaient clairement nuire à Hermione Granger. Que Hermione Granger a peut-être été forcée, contre sa volonté propre, à lever sa baguette contre notre Maison~; ou peut-être qu'elle et moi avons subit un sortilège de faux souvenirs. Nous déclarons à présent que celui ou celle qui a osé faire ainsi usage de l'héritier de la maison Malfoy est l'ennemi de cette dernière et que nous nous vengerons de cette personne. De plus, au nom de notre honneur, nous avons rendu tout l'argent que nous avions prit à la maison Potter et avons annulé sa dette.»

Puis Harry Potter parla. «La maison Potter reconnaît qu'il s'agissait d'une erreur sincère et n'a aucun ressentiment envers la maison Malfoy. Nous croyons et déclarons publiquement que la maison Malfoy n'est pas responsable de la mort de Hermione Granger. Celui ou celle qui a tué Hermione Granger est l'ennemi de la maison Potter, et nous nous vengerons de cette personne.»

Puis Harry Potter commença à repartir vers la table Serdaigle et, face à cette réalité fracturée, un bavardage explosa, né de la plus pure, de la plus profonde des confusions.

Drago Malfoy recommença à taper sa cuillère contre son verre d'eau, produisant un clair tintement, jusqu'à ce que la pièce se taise une fois de plus.

Et d'autres élèves se levèrent, venus d'autres tables, et s'avancèrent jusqu'à Drago Malfoy, se placèrent à côté de lui, derrière lui ou devant lui.

Il y avait un terrible silence dans la grande salle, le sentiment d'un monde qui glissait, de pouvoirs qui se réagençaient de façon quasiment tangible.

«Mon père, Owen Greengrass, avec l'assentiment et le soutien absolu de ma mère, la Dame de la Noble et très Ancienne Maison de Greengrass, dit Daphné Greengrass.

--- Et mon patriarche, Charles, de la maison Nott», dit l'ancien lieutenant Nott, autrefois Théodore du Chaos, à présent derrière Drago Malfoy.

«Et ma grande-tante, Amélia, de la maison Bones, aussi directrice du département de justice magique», dit Susan Bones, qui se tenait contre le flanc de Daphné Greengrass, celle au côté de laquelle elle avait combattu.

«Et ma grand-mère, Augusta, de la Noble et très ancienne maison de Londubat, dit Neville Londubat qui était revenu juste pour cette nuit.

--- Et mon père, Lucius, le Lord des Malfoy, de la noble et très ancienne maison de Malfoy~!

--- Ensembles, avec Alanna Howe, constituent une majorité au conseil d'administration de Poudlard~! dit clairement Daphné Greengrass. Et, dans le but de garantir la sécurité de tous les élèves, y compris leurs propres enfants, ont promulgué les Décrets Éducatifs suivants au sein de l'école de sorcellerie de Poudlard~!»

\later

«Premier décret~!» dit Daphné Greengrass. Elle essayait de contenir ses tremblements, face aux quatre maisons, devant les cinq enfants. Même les cours d'écriture de discours de ses parents avaient leurs limites. Ses yeux passèrent rapidement sur sa main, sur laquelle, d'une légère encre rouge, elle avait écrit ses répliques. «Les élèves ne devront aller nulle part seuls, pas même aux toilettes~! Vous vous déplacerez par groupes d'au moins trois, et chaque groupe devra comprendre un élève de sixième ou septième année~!

--- Deuxième décret~!» dit Susan Bones derrière elle d'une voix presque ferme. «Pour augmenter encore plus la sécurité des élèves, neuf Aurors ont été dépêchés à Poudlard afin de former la Force de Protection Auxiliaire~!» Susan prit un petit objet rond en verre, l'un des communicateurs que le département de justice magique utilisait et dont ils avaient tous reçu un exemplaire. Susan le porta à sa bouche et d'une dit voix maintenant plus forte~: «Auror Brodski, c'est Susan Bones. \emph{Entrez~!}»

Les portes de la grande salle s'ouvrirent grand et neuf Aurors entrèrent vêtus du cuir renforcé qu'ils utilisaient lorsqu'ils partaient en mission. Ils se déployèrent immédiatement, deux à chacune des quatre tables et le dernier en garde à côté de la grande tables. Il y eut d'autres hoquets de stupeur.

«Troisième décret~!» dit Drago Malfoy d'une voix impérieuse. Malfoy semblait avoir mémorisé son discours puisque Daphné ne pouvait rien voir d'écrit sur sa main. «Face à un ennemi commun qui ne recule pas devant le meurtre d'élèves, quelle que soit leur maison, les quatre maisons de Poudlard doivent s'unir et agir comme une seule~! Afin de mettre ceci en évidence, le système de points est temporairement suspendu~! \emph{Tous} les professeurs encourageront la solidarité entre maisons, par décret du conseil d'administrations de Poudlard~!

--- Quatrième décret~! récita Neville Londubat. Tous les élèves qui ne se rendent pas encore aux cours du soir du professeur de Défense recevront un entraînement spécial d'auto-défense de la part d'instructeurs Aurors~!

--- Cinquième décret~! s'écria Théodore Nott d'un ton menaçant. Tout combat dans les couloirs ou ailleurs, exception faite des cours de Défense, sera traité avec une \emph{extrême} sévérité~! Battez-vous ensemble ou ne vous battez pas du tout~!

--- Sixième décret~!» dit Daphné Greengrass, et elle prit une profonde inspiration. Lorsqu'elle avait découvert le plan, elle avait formulé sa propre demande auprès de Mère, à travers la cheminette. Même si Lucius Malfoy s'associait à Amélia Bones - une notion que son esprit avait encore du mal à appréhender - le vote décisif des Greengrass était demeuré vital puisque Jugson et sa faction avaient refusé de soutenir Malfoy. Sans parler du fait que Bones n'avait pas confiance en Malfoy et que Malfoy n'avait pas confiance en Bones. Mère avait donc exigé, et les Greengrass avaient reçu~: «Puisque les sortilèges de faux souvenirs ont été utilisés sur des élèves sans qu'aucune alarme ne se déclenche, il est possible que l'un des employés de Poudlard ait été impliqué. Donc~! La Force de Protection Auxiliaire rendra directement compte à mon père, Lord Greengrass~!» Elle savait que ce qui suivait était purement symbolique, car il n'y avait aucun raison pour quiconque de ne pas contacter directement les Aurors~; mais cela pourrait un jour prendre de l'importance, et c'était pour cela qu'elle avait demandé à Mère de l'exiger - «et si quiconque veut parler de quelque chose aux Protecteurs Auxiliaires, ils peuvent parler aux Aurors ou passer par \emph{moi}…» le bras de Daphné décrit un demi cercle pour désigner les autres élèves derrière elle. «La présidente dûment nommée du Comité Spécial de Protection Auxiliaire~!»

Et Daphné laissa retomber un silence spectaculaire. Ils avaient tous répété ce moment.

«Nous ne savons pas qui est l'ennemi, dit Neville d'une voix qui ne couina pas.

--- Nous ne savons pas ce que l'ennemi veut, dit Théodore d'un air toujours menaçant.

--- Mais nous savons qui l'ennemi attaque, dit Susan aussi féroce que lorsqu'elle avait battu trois élèves de septième année.

--- L'ennemi attaque les élèves de Poudlard, dit Drago Malfoy d'une voix claire et impérieuse comme si tout ceci était pour lui une seconde nature.

--- Et Poudlard», dit Daphné de Greengrass en sentant son sang bouillir en elle comme jamais auparavant, «va se \emph{défendre}.» 

%  LocalWords:  aphne un Alanna
