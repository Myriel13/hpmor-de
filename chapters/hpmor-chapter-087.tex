\chapter{Sensibilité hédonique}

\section{Jeudi 16 avril 1992}

\lettrine{L}{'école}  était maintenant presque déserte. Les neuf dixièmes des élèves étaient rentrés chez eux pour les vacances de Pâques et presque tous ceux qu'elle connaissait n'étaient plus là. Susan était restée car sa grande-tante était assez occupée~; Ron était resté aussi, mais elle ne savait pas pourquoi - peut-être la famille Weasley était-elle assez pauvre pour ne pas souhaiter avoir plus de bouches d'enfants à nourrir pendant une semaine~? Ça l'arrangeait plutôt, car Ron et Susan étaient à peu près les seuls qui voulaient encore bien lui parler. (Ou du moins les seuls à qui elle voulait \emph{répondre}. Lavande était encore gentille avec elle, et Tracey était, euh, Tracey, mais passer une heure avec l'une des deux n'aurait pas été tout à fait \emph{reposant}~; quoi qu'il en soit, aucune des deux n'était restée pour les vacances de Pâques).

Si elle ne pouvait pas rentrer \emph{chez elle} - et elle n'en avait pas le droit, on avait menti à ses parents en leur disant qu'elle avait la lumiole - alors une école de Poudlard quasiment vide était ce qu'il y avait de mieux.

Elle pouvait même se rendre à la bibliothèque sans que les gens la regardent, puisqu'il n'y avait plus de cours et que personne n'essayait de travailler.

Ç'aurait été une erreur de penser que Hermione déambulait abattue dans les couloirs en pleurnichant à longueur de journées. Oh, elle avait beaucoup pleuré les deux premiers jours, bien sûr, mais deux jours avaient suffit. Des passages de livres empruntés à Harry parlaient de ça, de la façon dont les gens paralysés dans des accidents de voitures se retrouvaient, six mois plus tard, loin d'être aussi tristes qu'ils s'étaient attendus à l'être, tout comme les gagnants du loto étaient loin d'être aussi heureux que ce à quoi ils s'étaient attendus. Les gens s'adaptaient, leurs niveau de bonheur revenait à leur point d'ancrage et la vie continuait.

Une ombre s'abattit au-dessus du livre qu'elle lisait. Elle tournoya, sa baguette dissimulée sur ses genoux directement braquée sur le visage surpris de …

"Pardon~!" dit Harry Potter, montrant ses paumes en hâte pour révéler sa main gauche, vide, et sa main droite contenant une petite bourse de velours rouge. "Pardon. Comptais pas t'effrayer."

Il y eut un terrible silence, son rythme cardiaque accéléra, ses paumes commencèrent à suer et Harry Potter se contenta de la regarder. Elle lui avait \emph{presque} parlé, à l'aube du premier jour du reste de sa vie, mais lorsqu'elle s'était rendue au petit déjeuner Harry Potter avait eu l'air si \emph{mal en point}… alors elle s'était assise à côté de lui à la table du petit déjeuner et avait silencieusement mangé dans sa petite bulle de il-n'y-a-personne-à-coté-de-moi, et ça avait été horrible, mais Harry n'avait pas été la voir et… elle ne lui avait tout simplement pas parlé depuis (ce n'était pas difficile d'éviter tout le monde quand on évitait la salle commune de Serdaigle et qu'on déguerpissait à la fin de cours avant que quiconque ne puisse vous parler).

Et depuis, elle s'était demandé ce que Harry pouvait bien penser d'elle - s'il la détestait parce qu'il avait perdu tout son argent - s'il était \emph{vraiment} amoureux d'elle et que c'était pour ça qu'il avait agi - s'il avait abandonné tout espoir qu'elle reste à sa hauteur parce qu'\emph{elle} ne pouvait pas \emph{effrayer les Détraqueurs}. En cet instant, elle ne pouvait pas lui faire face, elle en était incapable, elle avait passé des nuits d'insomnies à se demander ce qu'il pensait d'elle, et elle avait peur, car elle avait évité le garçon qui avait dépensé tout son argent pour la sauver~; Elle n'était qu'une horrible ingrate, qu'une misérable ingrate, etc…

Puis ses yeux s'abaissèrent pour découvrir que Harry avait plongé sa main dans la bourse rouge en velours, il en tira une confiserie en forme de cœur enrobée d'un papier métallique rouge et son cerveau fondit comme du chocolat laissé au soleil.

"J'allais te donner plus d'espace," dit Harry Potter, "sauf que je lisais les théories de Critch sur l'hédonisme, sur la meilleure façon d'entraîner le pigeon qu'on a en nous, sur la façon dont les petits retours négatifs et positifs immédiats contrôlent vraiment la majorité de ce qu'on fait et je me suis dit que tu m'évitais peut-être parce que me voir déclenchait des associations négatives et je ne voulais vraiment \emph{pas} laisser ça continuer plus longtemps sans y faire quelque chose donc j'ai mis la main sur un sac de chocolat des jumeaux Weasley et je vais juste t'en donner un à chaque fois que tu me verras pour créer un renforcement positif si ça ne te pose pas de problème…"

"\emph{Respire}, Harry," dit Hermione sans même y penser.

C'était la première chose qu'elle lui avait dite depuis le jour du procès.

Ils se regardèrent l'un l'autre.

Les livres sur les étagèrent les regardèrent.

Ils se regardèrent encore un peu.

"Tu es censée manger le chocolat," dit Harry en tendant la confiserie en forme de cœur comme un Valentin. "À moins que le fait de recevoir le chocolat soit assez agréable pour constituer un renforcement positif, auquel cas tu devrais probablement le mettre dans ta poche."

Elle savait que si elle essayait de parler à nouveau elle échouerait et n'essaya donc pas.

La tête de Harry s'affaissa un peu. "\emph{Est-ce} que tu me détestes maintenant~?"

"\emph{Non~!}" dit-elle. "Non, tu ne devrais pas penser ça, Harry~! Juste… juste… juste \emph{tout~!}" Elle se rendit compte que sa baguette était toujours pointée vers Harry et elle l'abaissa un peu. Elle essayait très fort de ne pas éclater en sanglots. "\emph{Tout~!}" répéta-t-elle, sans pouvoir trouver une meilleure façon de le dire, bien qu'elle fut certaine que Harry voulait qu'elle soit plus précise.

"Je pense que je comprends," dit prudemment Harry. "Qu'est-ce que tu lis~?"

Avant qu'elle ne puisse l'arrêter, Harry se pencha sur la table de lecture pour voir le livre qu'elle lisait et avança la tête avant qu'elle ne pense à l'écarter…

Harry regarda la page à laquelle le livre était ouvert.

"Les sorciers les plus riches et comment ils y sont parvenus," lut Harry à haute voix en regardant le haut de celle-ci. "Numéro soixante-cinq, Sire Gareth, propriétaire d'une entreprise gagnante des guerres de transport du 19\textsuperscript{ème} siècle… monopole sur les oh-thé-trois… Je vois."

"J'suis sûre que tu vas me dire que je dois pas m'inquiéter et que tu vas t'occuper de tout~?" Son ton fut plus dur qu'elle ne l'avait voulu et elle sentit venir une autre vague de culpabilité à l'idée d'être une personne aussi horrible.

"Nan," dit Harry, étrangement enjoué. "Je peux suffisamment bien me mettre à ta place pour savoir que si \emph{tu} avais payé une grosse somme pour \emph{me} sauver, je \emph{serais} en train d'essayer de te rembourser. Je saurais que c'est idiot, en un sens, mais j'essaierais \emph{quand même} de tout te rembourser. \emph{Ça}, je peux parfaitement le comprendre, Hermione."

Le visage de cette dernière se tordit et elle sentit de l'humidité au coin de ses yeux.

"Je t'avertis quand même," continua Harry, "je résoudrai peut-être le problème de la dette envers Lucius Malfoy moi-même si je découvre un moyen de le faire avant toi. C'est plus important de résoudre le problème tout de suite que de savoir \emph{lequel} d'entre nous l'a résolu. Tu as déjà trouvé quelque chose d'intéressant~?"

Trois quarts de Hermione couraient en tous sens et se cognaient contre des murs à force d'essayer de déchiffrer les conséquences des paroles de Harry (est-ce qu'il la respectait \emph{toujours} en tant qu'héroïne~? Ou est-ce que ça voulait dire qu'il pensait quelle ne \emph{pouvait pas} y arriver toute seule~?). Au même moment, une partie bien plus sensée de sa personne revint à la page 37, où se trouvait la possibilité la plus prometteuse pour l'instant (même si dans son imagination elle le faisait toujours toute seule et prenait Harry complètement par surprise)…

"J'ai pensé que ça avait l'air assez intéressant," dit sa voix.

"Numéro quatorze, 'Crozier', véritable nom~: inconnu," dit Harry ."Wow, c'est… le chapeau haut de forme à carreau le plus Gaudiesque que j'ai jamais vu. Fortune~: au moins six-cent-mille Gallions… donc environ trente-millions de livres sterling, pas assez pour rendre un Moldu célèbre mais suffisant pour la petite population sorcière je suppose. Soupçonné d'être un alias moderne de Nicholas Flamel, vieux de six siècle, le seul sorcier à avoir réussi la procédure alchimique incroyablement difficile nécessaire à la création de la Pierre Philosophale qui permet la transmutation de simples métaux vers l'or et l'argent ainsi que vers… l'Élixir de longue vie qui prolonge indéfiniment la jeunesse et la santé de celui qui le boit… euh, Hermione, ça m'a l'air évidemment faux."

"J'ai trouvé d'autres références au sujet de Nicholas Flamel," dit Hermione. "\emph{Grandeur et Chute des Arts Noirs} dit qu'il a secrètement entraîné Dumbledore avant son affrontement avec Grindelwald. Il y a beaucoup de livres qui prennent cette histoire au sérieux, pas seulement celui-ci… tu penses que c'est trop beau pour être vrai~?"

"Non, bien sûr que non," dit Harry. Il tira la chaise située à côté de Hermione et s'assit à son emplacement habituel, à sa droite, comme s'il n'en était jamais parti~; elle dut refréner un hoquet. "L'idée de 'trop beau pour être vrai' ne constitue pas un raisonnement causal, l'univers ne vérifie pas si le résultat d'une équation est 'trop beau' ou 'trop mauvais' avant de le permettre. Les gens pensaient que les avions et les vaccins contre la variole étaient trop beaux pour être vrais. Les moldus ont découvert comment voyager vers d'autres étoiles sans même utiliser la magie, et toi et moi pouvons utiliser nos baguettes pour faire des choses que les physiciens moldus croient être strictement impossibles. Je ne peux même pas imaginer ce que les \emph{vraies} lois de la magie seraient incapables de faire."

"Alors quel est le problème~?" dit Hermione. Elle avait l'impression que sa voix était maintenant plus normale.

"Eh bien…" dit Harry. Le garçon passa un bras au-dessus du sien, leurs robes se frôlèrent, et il toucha l'illustration d'une sinistre pierre rouge d'où gouttait un liquide écarlate. "Le problème numéro un c'est qu'il n'y a pas de raison logique pour laquelle le \emph{même} appareil serait capable de transmuter du plomb en or \emph{et} de produire un élixir qui maintient la jeunesse. Je me demande s'il y a un nom officiel pour ça dans la littérature~? L'effet 'monte jusqu'à 11', peut-être~? Si tout le monde peut voir ce qu'est fleur, ce n'est pas crédible de dire que les fleurs sont aussi grandes que des maisons. Mais si on fait partie d'une secte adoratrice d'OVNIS, puisque personne ne peut voir le vaisseau-mère alien de toute façon, on peut dire qu'il fait la taille d'une ville, ou la taille de la Lune. Les choses observables sont contraintes par les observations, mais quand on invente une histoire, on peut pousser les choses aussi loin qu'on le désire. Donc la Pierre Philosophale donne de l'or infini \emph{et} la vie éternelle, pas parce qu'une seule découverte magique produit ces deux effets, mais parce que quelqu'un a inventé l'histoire d'un truc hyper cool."

"Harry, il y a plein de choses dans le monde de la magie qui n'ont aucun sens," dit-elle.

"Accordé," dit Harry. "Mais Hermione, le second problème c'est que même les \emph{sorciers} ne sont pas assez fous pour nonchalamment laisser passer ce que ça \emph{impliquerait}. \emph{Tout le monde} serait en train d'essayer de redécouvrir le formule de la Pierre Philosophale, des \emph{pays} entiers essaieraient de capturer ce sorcier immortel et de lui arracher son secret…"

"Ce n'est pas un \emph{secret}." Hermione tourna la page et montra les diagrammes à Harry. "Les instructions sont juste là, à la page suivante. C'est juste tellement difficile que Nicholas Flamel est le seul à l'avoir \emph{fait}."

"Alors des pays entiers essaieraient de kidnapper Flamel et de \emph{le} forcer à faire plus de Pierres. Allons, Hermione, même les sorciers ne peuvent pas entendre parler \emph{d'immortalité} et, et…" Harry s'interrompit, son éloquence sembla lui manquer, "et \emph{juste continuer comme si de rien n'était}. Les humains sont fous, mais pas fous \emph{à ce point}~!"

"Tout le monde ne pense pas comme \emph{toi}, Harry." Il avait marqué un point, mais… \emph{combien} de références à Nicholas Flamel avait-elle lues~? Hormis \emph{Les sorciers les plus riches du monde} et \emph{Grandeur et Chute des Arts Noirs}, il y avait aussi \emph{Histoires des Temps Modérément Anciens} et \emph{Biographies des justement célèbres}…

"Très bien, alors \emph{Le professeur Quirrell} aurait kidnappé ce Flamel. C'est ce que quelqu'un de mauvais \emph{ou} de bon ou \emph{d'égoïste} ferait s'il avait le moindre bon sens. Le professeur de Défense connaît de nombreux secrets et il ne voudrait pas passer à côté de \emph{celui là}." Harry soupira et leva les yeux~; elle suivit son regard, mais il semblait se contenter de regarder la bibliothèque dans son ensemble, les rangées et rangées et rangées d'étagères. "Je ne veux pas interférer avec ton projet," dit Harry, "et je ne veux certainement pas te décourager, mais… Honnêtement, Hermione, je ne suis pas certain que tu vas trouver une seule bonne idée te permettant de gagner de l'argent dans un livre comme celui là. C'est comme la vieille blague sur l'économiste qui voit un billet de vingt livres sterling par terre dans la rue et qui ne se fatigue pas à le ramasser, parce que si le billet avait été réel quelqu'un d'autre l'aurait déjà ramassé. Si un moyen de gagner beaucoup d'argent est déjà tellement \emph{célèbre} qu'on peut le trouver dans un livre comme celui-ci… tu vois ce que je veux dire~? Ça ne peut pas être possible de gagner mille Gallions par mois en trois étapes faciles à suivre, sinon tout le monde le ferait."

"Et alors~? Ça ne t'arrêterait pas \emph{toi}," dit Hermione d'une voix qui redevenait dure. "Tu fais des choses impossibles tout le temps, je parie que tu as fait quelque chose d'impossible la \emph{semaine} dernière et que tu ne t'es même pas fatigué à le \emph{dire} à quelqu'un."

(Il y eut un court silence qui, si seulement Mlle Granger l'avait su, était exactement la durée de la pause que vous auriez marquée si vous aviez combattu Maugrey Fol Œil et l'aviez battu exactement huit jours plus tôt).

"Pas dans ces sept derniers jours, non," dit Harry. "Écoute… une partie de la technique qui permet d'accomplir l'impossible est de choisir \emph{quelles} impossibilités remettre en cause et de tenter le coup seulement quand on a un avantage spécial. S'il y a une méthode permettant de gagner de l'argent dans ce livre qui a l'air difficile à un sorcier mais qui est facile si on peut utiliser le vieux Mac Plus de papa, \emph{alors} on a un plan."

"Je \emph{sais ça}, Harry," dit Hermione d'une voix qui ne vacilla que légèrement. "Je cherchais à voir s'il y avait quelque chose ici que je \emph{pouvais} réussir à faire. Je me suis dit, peut-être que la partie difficile dans la fabrication de la Pierre Philosophale, c'est que le cercle alchimique doit être super précis et que je pourrais y arriver juste en utilisant un microscope moldu…"

"C'est \emph{génial}, Hermione~!" le garçon leva rapidement sa baguette, dit "\emph{Sourdinam}," puis poursuivit lorsque le bruissement des livres les plus chahuteurs se fut tut. "Même si la Pierre Philosophale n'est qu'un mythe, cette technique pourrait marcher pour d'autres alchimies difficiles…"

"Eh bien ça ne \emph{peut pas} marcher," dit Hermione. Elle avait traversé toute la bibliothèque pour trouver le seul livre sur l'alchimie qui n'était pas dans la section interdite. Et alors… elle se souvenait de l'écrasante déception, de l'espoir soudain qui s'était dissipé comme un brouillard. "Parce que \emph{tous} les cercles alchimiques doivent être 'de la finesse d'un cheveu d'enfant', ce n'est pas plus fin pour certaines alchimies ou pour d'autres. Et les sorciers \emph{ont} des Multiplettes, et je n'ai entendu parler d'aucun sortilège permettant d'utiliser des Multiplettes pour agrandir les choses et être plus précis. J'aurais dû m'en rendre compte~!"

"Hermione," dit Harry avec sérieux en replongeant la main dans sa bourse rouge en velours, "ne te punis pas quand une idée intelligente ne fonctionne pas. Tu dois avoir \emph{beaucoup} d'idées imparfaites pour en trouver une qui peut fonctionner. Et si tu envois des retours négatifs à ton cerveau en fronçant les sourcils quand tu as une idée imparfaite au lieu de te rendre compte que la suggestion d'idée est un bon comportement de ton cerveau qui devrait être encouragé, tu n'auras bientôt plus aucune idée." Harry posa deux chocolats en forme de cœur à côté du livre. "Tiens, prends un autre chocolat. Je veux dire en plus de celui que je t'ai donné plus tôt. Celui là est pour encourager ton cerveau parce qu'il a généré une bonne stratégie potentielle."

"J'imagine que tu as raison," dit Hermione d'une petite voix, mais elle ne toucha pas au chocolat. Elle commença à revenir à la page 167 qu'elle était en train de lire avant que Harry ne vienne.

(Hermione Granger n'avait pas besoin de \emph{marque-pages}, bien sûr).

Harry se penchait légèrement, sa tête touchait presque l'épaule de Hermione, il regardait les pages à mesure qu'elle les tournait, comme s'il aurait pu obtenir quelque information de valeur en les regardant pendant un quart de secondes. Il avait petit-déjeuné peu de temps auparavant et elle pouvait clairement identifier, à la légère odeur de son haleine, qu'il avait mangé du gâteau à la banane pour le dessert.

Harry parla de nouveau. "Donc tout cela étant dit… et s'il te plaît prends ça comme un renforcement positif… est-ce que tu as vraiment essayé d'inventer un moyen de faire de la \emph{production de masse d'immortalité} pour que je puisse \emph{rembourser ma dette envers Lucius Malfoy}~?"

"Oui," dit-elle d'une voix encore plus petite. Même quand elle \emph{essayait} de penser comme Harry, il semblait qu'elle n'avait pas encore saisit le truc. "Et qu'est-ce que tu as fait pendant tout ce temps, Harry~?"

Harry prit un air dégoûté. "Essayé de récolter des preuves sur ce mystère de 'Qui a piégé Hermione Granger'".

"Je…" Hermione leva les yeux vers Harry. "Est-ce ça ne devrait pas plutôt être moi… qui essaie de résoudre mon \emph{propre} mystère~?" Ça n'avait pas été sa première idée, sa première priorité, mais maintenant que Harry le mentionnait…

"Dans ce cas, ça ne fonctionnerait pas," dit froidement Harry. "Il y a trop de gens qui acceptent de parler à moi mais pas à toi… et je suis aussi navré de te dire que certains m'ont fait promettre de ne répéter à personne ce qu'ils m'ont dit. Désolé, je ne pense pas que tu puisse beaucoup aider sur ce coup là."

"D'accord," dit Hermione d'un ton las. "Très bien. Tu fais tout. Tu récoltes tous les indices et tu parles à tous les suspects pendant que je reste assise ici, dans la bibliothèque. Préviens-moi quand tu découvres que c'est le professeur Quirrell qui était derrière tout ça."

"Hermione…" dit Harry. "Pourquoi est-ce qui c'est si important de savoir \emph{qui} fait quoi~? Est-ce que ça ne devrait pas être plus important de tout résoudre que de savoir qui a tout résolu~?"

"J'imagine que tu as raison," dit Hermione. Elle leva les mains et les pressa contre ses yeux. "J'imagine que ça n'a plus d'importance. Tout le monde va penser - je \emph{sais} que ce n'est pas ta faute Harry, tu as été - tu as été Bon, tu as été un parfait gentleman - mais peu importe ce que je fais, ils penseront tous que je suis juste - une fille bonne à sauver." Elle s'interrompit et dit d'une voix tremblante, "et peut-être qu'ils ont \emph{raison}, Harry."

"Woah, woah, attends une seconde…"

"Je ne fais pas peur aux Détraqueurs. Je peux avoir un Très Bien en sortilèges, mais je ne fais pas peur aux Détraqueurs."

"\emph{J'ai un mystérieux côté obscur~!}" siffla Harry après avoir tourné la tête pour inspecter la bibliothèque. (Il y avait un garçon dans un coin, loin, qui regardait parfois vers eux, mais il aurait été trop distant pour entendre quoi que ce soit, même sans la barrière de Sourdinam). "J'ai un côté obscur qui n'est \emph{certainement} pas un enfant~; et qui sait quel autre truc magique complètement dingue se passe dans ma tête - le professeur Quirrell a dit que je deviens qui je crois être - c'est de la \emph{triche}, tu ne comprends pas, Hermione~? L'administration et moi avons un accord dont je ne suis pas censé parler qui permet au Survivant d'étudier plus longtemps que les autres, tous les jours, je \emph{triche} et \emph{tu me bats quand même en cours de sortilèges}. Je ne… je ne suis probablement pas… le Survivant n'est probablement même pas quelque chose qu'on pourrait raisonnablement appeler un enfant - et tu \emph{rivalises quand même} avec ça. Est-ce que tu ne te rends pas compte que si les gens ne faisaient \emph{pas} attention à moi tu apparaîtrais comme la sorcière la plus puissante depuis un siècle~? Que tu peux combattre trois brutes plus âgées toute seule et gagner~?"

"Je ne sais pas," dit-elle en appuyant ses mains sur ses yeux, d'une voix vacillante "Tout ce que je sais, c'est… que même si tout ça est \emph{vrai} - personne ne me verra jamais pour qui je suis, plus jamais."

"Très bien," dit Harry au bout d'un moment. "Je vois ce que tu veux dire. Au lieu de la fameuse équipe de recherche Potter et Granger, il y aura Harry Potter et son assistante de laboratoire… Euh… voilà une idée. Et si je ne me concentrais \emph{pas} sur comment gagner de l'argent pendant un moment~? Après tout la dette ne doit pas être remboursée avant que je finisse mes études à Poudlard. Donc tu peux y arriver toi-même et montrer au monde de quoi tu es capable. Et si par hasard tu découvres en chemin le secret de l'immortalité, on dira juste que c'est un bonus."

L'idée que Harry se repose sur \emph{elle} pour trouver une solution semblait… être une responsabilité écrasante lâchée sur une pauvre fille de douze ans déjà traumatisée, et elle voulait en même temps le prendre dans ses bras pour le remercier de lui avoir offert un moyen de retrouver sa dignité d'héroïne, et c'était ce qu'elle \emph{méritait} après avoir été horrible, après lui avoir parlé sèchement pendant tout ce temps alors que depuis le début il s'était comporté comme un meilleur ami, alors qu'elle ne l'avait jamais fait pour lui, et heureusement elle se sentait encore capable d'accomplir quelque chose et…

"Est-ce que tu as un truc super rationnel que tu fais quand ton esprit se met à courir dans plein de directions à la fois~?" parvint-elle à dire.

"Mon approche est généralement d'identifier les différents désirs, de leur donner des noms, de les concevoir comme des individus différents et de les laisser débattre dans ma tête. Pour l'instant ceux qui persistent sont mes côtés Poufsouffle, Serdaigle, Gryffondor, et Serpentard, mon Critique Interne, mes copies simulées de toi, Neville, Drago, le professeur McGonagall, le professeur Flitwick, le professeur Quirrell, Papa, Maman, Richard Feynman et Douglas Hofstadter."

Hermione envisagea l'idée d'essayer avant que son Sens Commun ne la prévienne que ça pourrait être dangereux de faire ça. "Il y a une copie de \emph{moi} dans ta tête~?"

"Bien sûr~!" dit Harry. Le garçon sembla soudain un peu plus vulnérable. "Tu veux dire qu'il n'y a \emph{pas} une copie de moi qui vit dans \emph{ta} tête~?"

Elle se rendit compte qu'il y en \emph{avait une}~; et pas seulement ça~: elle parlait exactement avec la voix de Harry.

"C'est assez troublant maintenant que j'y pense," dit-elle. "J'ai effectivement une copie de toi dans ma tête. Elle me parle en ce moment même avec ta voix et elle soutient que c'est parfaitement normal."

"Bien," dit Harry avec sérieux. "Enfin je ne vois pas comment les gens pourraient être amis sans ça."

Elle continua alors de lire son livre et Harry sembla satisfait de continuer de lire les pages par-dessus son épaule.

Elle était allé jusqu'au numéro soixante-dix, Katherine Scott, qui avait apparemment inventé un moyen de transformer de petits animaux en tartes au citron, lorsqu'elle trouva enfin le courage de parler.

"Harry~?" dit-elle. (Elle s'était un peu écartée de lui, mais elle ne s'en était pas rendu compte). "S'il y a une copie de Drago Malfoy dans ta tête, est-ce que ça veut dire que tu es ami avec Drago Malfoy~?"

"Eh bien…" dit Harry. Il soupira. "Ouais, je comptais t'en parler de toute façon. J'aurais bien aimé t'en parler plus tôt. Bref, comment dire ça… je le corrompais~?"

"Qu'est-ce que tu veux dire, \emph{corrompait}~?"

"Je le tentais pour qu'il rejoigne le côté clair de la Force."

Elle resta bouche bée.

"Tu sais, comme l'Empereur et Dark Vador, mais à l'envers."

"\emph{Drago Malfoy}," dit-elle. "Harry, est-ce que tu as la \emph{moindre idée}…"

"Oui."

"… du genre de chose que Malfoy a \emph{dit} sur moi~? De ce qu'il a dit qu'il me \emph{ferait} dès qu'il en aurait l'occasion~? Je ne sais pas ce qu'il \emph{t'a} dit, mais Daphné Greengrass m'a répété ce que Malfoy dit quand il est à Serpentard. C'est \emph{indicible}, Harry~! C'est littéralement indicible, au sens que je ne peux pas le dire à voix haute~!"

"C'était quand~?" dit Harry. "Au début de l'année~? Est-ce que Daphné t'a dit \emph{quand} c'était~?"

"Non," dit Hermione. "Parce que le quand n'a pas d'importance. Quiconque dit des choses - comme celles que Malfoy a dites - ne peut pas être quelqu'un de bien. Tes tentations n'ont pas d'importance, il reste quelqu'un de pourri, parce que \emph{quoiqu'il arrive} quelqu'un de bien ne pourrait \emph{jamais}…

"Tu as tort," dit Harry en la regardant droit dans les yeux. "Je peux imaginer ce que Drago a menacé de te faire, parce que la deuxième fois que je l'ai rencontré, il parlait de le faire à une fille de dix ans. Mais est-ce que tu ne vois pas que le jour où Drago Malfoy est arrivé à Poudlard, il avait passé toute sa vie élevé par des \emph{Mangemorts}. Il aurait fallu une \emph{intervention surnaturelle} pour qu'il ait \emph{ta} moralité étant donné \emph{son} cadre éducatif…"

Hermione secoua violemment la tête. "\emph{Non}, Harry, personne n'a besoin de te \emph{dire} que c'est mal de faire souffrir les gens. Ce n'est pas quelque chose qu'on ne fait pas parce que le professeur a dit que c'était interdit, c'est quelque chose qu'on ne fait pas parce que… parce que \emph{quand les gens souffrent, ça se voit}. Tu ne sais pas ça, Harry~?" Sa voix tremblait à présent. "Ce n'est pas… ce n'est pas une \emph{règle} que les gens suivent comme les règles de l'algèbre~! Si tu ne peux pas le \emph{voir}, si tu ne peux pas le ressentir \emph{ici}," sa main frappa au centre de sa poitrine, pas tout à fait là où se trouvait son cœur, mais ça n'avait pas d'importance parce que tout se passait dans le cerveau de toute façon, "alors tu ne l'as pas~!"

L'idée vint alors à Hermione que Harry ne l'avait peut-être \emph{pas}.

"Il y a des livres d'Histoire que tu n'as pas lus," dit doucement Harry. "Il y a des livres que tu n'as pas lus et qui pourraient te faire voir les choses autrement. Il y a quelques siècles - je pense que c'était certainement encore en vogue au dix-septième siècle - c'était une forme d'amusement populaire au village que de prendre un panier en osier ou un ballot avec une dizaine de chats vivants à l'intérieur et…"

"Arrête," dit-elle.

"… de les brûler dans un feu de joie. Juste une fête ordinaire. Un plaisir sain. Et je leur accorderai ça~: c'était plus sain que te brûler des femmes qu'on pensait être des sorcières. Parce que la façon dont les gens sont faits, Hermione, la façon dont les \emph{sentiments} des gens sont faits…" Harry mit une main sur son propre cœur, à l'emplacement anatomique correct, mais il s'arrêta et déplaça sa main vers sa tête, au niveau de l'oreille, "… c'est qu'il souffrent quand ils voient leurs \emph{amis} souffrir. Quand la personne entre dans leur cercle de préoccupation, quand c'est un membre de leur tribu. Cette sensation a un interrupteur, un bouton on-off étiqueté 'ennemi', 'étranger', ou parfois juste 'inconnu'. Les gens sont comme ça si on ne leur \emph{apprend} rien d'autre. Donc non, le fait que Drago Malfoy a grandi en croyant que c'est drôle de faire souffrir ses ennemis ne veut \emph{pas} dire qu'il est inhumain, ni même inhabituellement mauvais…"

"Si tu crois ça," dit-elle d'une voix instable, "si tu \emph{peux} croire ça, alors tu es mauvais. Les gens sont toujours responsables de leurs actes. Ce qu'on te \emph{dit} de faire n'a pas d'importance~; c'est toi qui agis. Tout le monde sait ça…"

"\emph{Non, tout le monde ne le sait pas~!} Tu as grandi dans une société post-Seconde Guerre mondiale, dans laquelle \emph{tout le monde sait} que les méchants disent 'je ne faisais qu'obéir aux ordres'. Au quinzième siècle, on aurait dit que l'accusé avait agit avec un sens du devoir honorable." La voix de Harry montait. "Est-ce que tu penses que tu, que tu es juste \emph{génétiquement} meilleure que tous ceux qui vivaient à cette époque~? Que si tu avais été transportée à Londres au quinzième siècle quand tu étais bébé tu te serais rendu compte \emph{toute seule} que c'est mal de brûler des chats, que c'est mal de brûler des sorcières, que l'esclavage est mal, que tous les êtres conscients devraient entrer dans ton cercle de préoccupation~? Est-ce que tu penses que tu aurais \emph{fini} de comprendre tout ça à la fin ton premier jour à Poudlard~? Personne n'a jamais \emph{dit} à Drago que c'est sa responsabilité personnelle de devenir plus éthiquement meilleur que la société dans laquelle il a grandi. Et \emph{malgré ça} il ne lui a fallu que quatre mois pour en arriver au point où il attrape une née-Moldue par la main pour l'empêcher de tomber d'un bâtiment." Les yeux de Harry étaient plus virulents qu'elle ne les avait jamais vus. "Je n'ai pas \emph{fini} de corrompre Drago Malfoy, mais je pense que jusque là il s'en sort \emph{plutôt bien}."

Le problème d'avoir une aussi bonne mémoire, c'était de \emph{pouvoir} se souvenir.

Elle se souvenait de Drago Malfoy saisissant son poignet avec tant de force qu'elle avait eu un bleu.

Elle se souvenait de Drago l'aidant à se relever après que ce mystérieux sortilège l'eut faite trébucher dans l'assiette du capitaine de Quidditch Serpentard.

Et elle se souvenait - c'était en fait la raison pour laquelle elle avait entamé cette conversation - de ce qu'elle avait ressentit quand elle avait entendu le témoignage sous Veritaserum de Drago Malfoy.

"Pourquoi est-ce que tu ne m'as rien \emph{dit} de tout ça~?" dit Hermione, et en dépit d'elle-même, sa voix monta d'un cran. "Si j'avais \emph{su}…"

"Ce n'était pas à moi de te révéler ce secret," dit Harry. "Drago aurait été mis en danger si son père l'avait découvert."

"Je ne suis pas stupide, M. Potter. Quelle est la \emph{vraie} raison pour laquelle tu ne me l'as pas dit, et qu'est-ce que tu faisais \emph{vraiment} avec M. Malfoy~?"

"Ah. Eh bien…" Harry détourna le regard et baissa les yeux vers la table.

"Drago Malfoy a dit aux Aurors, sous Veritaserum, qu'il avait désiré savoir s'il pouvait me battre et qu'il m'avait donc provoquée en duel afin de le \emph{tester empiriquement}. Ce furent ses \emph{mots exacts}, selon la retranscription de son témoignage."

"Ouais," dit Harry, toujours sans croiser son regard. "Hermione Granger. Bien \emph{sûr} qu'elle se souviendrait des mots exacts. Peu importe qu'elle soit enchaînée à une chaise et accusée de meurtre devant tout le Magenmagot…"

"Qu'est-ce que tu faisais \emph{vraiment} avec Drago Malfoy~?"

Harry grimaça et dit~: "Probablement pas \emph{exactement} ce que tu penses, mais…"

L'horreur monta et monta en elle, puis se libéra enfin.

\emph{"Tu faisais de la SCIENCE avec lui~?"}

"Eh bien…"

\emph{"Tu faisais de la SCIENCE avec lui~? Tu étais censé faire de la science avec MOI~!"}

"Ce n'est pas ça~! Ce n'est pas comme si je faisais \emph{vraiment} de la science avec lui~! C'était juste, tu sais, pour lui \emph{apprendre} des bouts de science moldue sans danger, comme la physique élémentaire, l'algèbre, et ainsi de suite… ce n'est pas comme si j'avais fait de la recherche originale en magie avec lui comme je l'ai fait avec toi…"

"Et j'imagine que tu ne \emph{lui} a pas parlé de \emph{moi} non plus~?"

"Euh, bien sûr que non~?" dit Harry. "J'ai fait de la science avec lui depuis octobre, et à cette époque il n'était pas tout à fait prêt à entendre parler de toi…"

Le sentiment de trahison inexprimable enflait et enflait en elle, dominait tout~: sa voix de plus en plus forte, ses yeux en furie, son nez dont elle était certaine qu'il se mettait à couler, la sensation de brûlure dans sa gorge. Elle se leva avec force de la table et fit un pas en arrière pour mieux voir celui qui l'avait trahie. Sa voix fut presque un cri perçant lorsqu'elle hurla~: "\emph{Ça n'est pas tenable~! Tu ne peux pas faire de la science avec deux personnes en même temps~!}"

"Euh…"

"\emph{Je veux dire, tu ne peux pas faire de la science avec deux personnes différentes et ne pas le leur dire~!}"

"Ah…" dit précautionneusement Harry. "J'y \emph{ai} pensé, et j'ai été très prudent de ne pas mélanger tes recherches avec quelque chose que j'aurais fait avec lui…"

"Tu as été \emph{prudent}." Elle l'aurait \emph{sifflé} s'il y avait eu des sss.

Harry leva une main et frotta ses cheveux en bataille, et sans qu'elle sache pourquoi cela lui donna \emph{encore plus} envie de hurler. "Mlle Granger," dit Harry, "je pense que cette conversation est devenue \emph{métaphorique} à un niveau qui, euh…"

"\emph{Quoi~?}" dit-elle d'une voix stridente depuis l'intérieur de la barrière insonorisante.

Puis elle comprit et devint si rouge que si elle avait eu une puissance magique adulte ses cheveux auraient spontanément pris feu.

Le seul autre visiteur de la bibliothèque, le garçon Serdaigle assit dans le coin opposé, les regardait tous les deux avec de grands yeux, tout en essayant assez tristement de le masquer en tenant un livre juste en dessous de son visage.

"Oui," dit Harry avec un petit soupir. "Donc, en gardant \emph{fermement à l'esprit} que c'était juste une mauvaise métaphore et que les \emph{vrais scientifiques} collaborent \emph{tout le temps}, je ne pense pas avoir trompé qui que ce soit. Les scientifiques sont souvent discrets quant aux projets sur lesquels ils travaillent. Toi et moi faisons des recherches que nous gardons secrètes, et il y avait des raisons de ne pas en parler à Drago Malfoy en particulier - au début, il aurait entièrement arrêté de passer du temps avec moi s'il avait su que j'étais ton ami et pas ton rival. Et Drago aurait été celui à risque si j'avais parlé de \emph{lui} à quelqu'un…"

"C'est vraiment tout~?" dit-elle. "\emph{Vraiment}, Harry~? Tu ne voulais pas qu'on se sente tous les deux \emph{spéciaux}, qu'on ait tous les deux l'impression d'être les \emph{seuls} avec qui tu voulais être, les \emph{seuls} qui puissent être avec toi~?"

"Ce n'est \emph{pas} pour ça que j'ai…"

Harry s'interrompit.

Il la regarda.

Lorsqu'elle se rendit compte de ce qu'elle venait de laisser échapper, tout son sang remonta vers son visage. Il y aurait probablement dû y avoir de la fumée en train de sortir de ses oreilles, une fumée qui aurait dû faire fondre sa tête, faire couler la chair le long de son cou.

Harry la regardait avec une horreur absolue.

"Enfin…" dit-elle d'une voix plutôt aiguë, "c'est… oh, je ne sais pas Harry~! \emph{Est-ce que} c'est juste une métaphore~? Quand un garçon dépense cent-mille Gallions pour sauver une fille d'une fin certaine, elle a le droit de se poser des questions, tu ne penses pas~? C'est comme de se faire offrir des fleurs, sauf que, tu vois, c'est \emph{beaucoup plus}…

Harry se releva vivement de la table et fit un pas chancelant en arrière, tout en levant les mains et en les agitant frénétiquement. "\emph{Ce n'est pas pour ça que je l'ai fait~! Je l'ai fait parce qu'on est amis~!}"

"Juste amis~?"

La respiration de Harry commençait à grimper vers l'hyperventilation. "De très bons amis~! D'extra-super-bons amis, même~! Meilleurs amis pour toujours, peut-être~! Mais pas \emph{ce} genre d'amis~!"

"Est-ce que c'est vraiment si horrible que ça d'y penser~?" dit-elle en manquant une respiration. "Je veux dire… je ne dis pas que \emph{je} suis amoureuse de \emph{toi}, mais…"

"Ah non~? \emph{Heureusement}." Harry remonta une manche de sa robe et s'essuya le front. "Écoute Hermione, comprends-moi bien, je suis sûr que tu es quelqu'un de génial…"

Elle fit un pas chancelant en arrière.

"mais… mais avec mon côté obscur…"

"C'est de \emph{ça} qu'il s'agit~?" dit-elle. "Mais… je ne…"

"Non, non, je veux dire que j'ai un mystérieux côté obscur et probablement d'autres trucs magiques bizarres, tu \emph{sais} que je ne suis pas un enfant normal, pas vraiment…"

"Tu as le droit de ne pas être normal," dit-elle avec la sensation d'être de plus en plus désespérée et perdue. "Ça \emph{me} va…"

"Mais \emph{même avec tous ces trucs magiques bizarres} qui me permettent d'être plus un adulte que je ne devrais l'être, je ne suis pas encore pubère et il n'y a pas d'hormones dans mon sang et mon cerveau est \emph{physiquement incapable} de tomber amoureux de qui que ce soit. Donc je ne suis pas amoureux de toi~! Je ne pourrais pas être amoureux de toi~! Pour tout ce que j'en sais, dans six mois mon cerveau va se réveiller et décider de tomber amoureux du professeur Rogue~! Euh, est-ce que je peux déduire de tout ça que \emph{tu} as déjà atteint ta puberté~?"

"Hiiii," dit Hermione d'un ton aigu. Elle tangua, et un instant plus tard Harry se précipita à son côté et l'aida à s'asseoir par terre, soutenant son corps de ses mains fermes.

Le fait était qu'elle \emph{s'était} rendue, chancelante, au bureau du professeur McGonagall en décembre, pas totalement surprise, parce qu'elle s'était informée, mais quand même plutôt \emph{mal à l'aise}, et c'était avec un grand soulagement qu'elle avait appris que les sorcières avaient des sortilèges pour gérer ce genre d'ennuis et \emph{qu'est-ce qui prenait Harry de poser une question pareille à une pauvre fille innocente…"}

"Écoute, je suis \emph{désolé}," s'empressa de dire Harry. "Je ne voulais vraiment pas dire la majorité de ce que j'ai eu l'air de vouloir dire~! Je suis sûr que n'importe qui prenant une vue extérieure sur la situation et proposant des cotes de paris sur la personne que je finirai par épouser assignerait une plus grande probabilité à toi qu'à toute autre personne à laquelle je puisse penser…"

Les pensées de Hermione, qui venaient à peine de commencer à se reformer, explosèrent promptement en un jet d'étincelles et de flammes.

"Mais pas nécessairement une probabilité supérieure à cinquante pour cent, je veux dire du point de vue extérieure il y a beaucoup d'autres possibilités, et qui j'aime avant ma puberté ne \emph{diagnostique} probablement pas fortement avec qui je serai dans sept ans… je ne veux pas avoir l'air de \emph{promettre} quoi que ce soit…"

Sa gorge émit un ensemble de sons aigus, elle n'écoutait pas exactement pas de quoi il s'agissait. Tout son univers s'était restreint à la terrible, terrible voix de Harry.

"Et puis j'ai lu des livres sur la psychologie évolutionniste et, enfin, il est possible que 'un homme et une femme heureux avec beaucoup d'enfants' est peut-être plus proche de l'exception que de la règle, et dans les tribus de chasseurs-cueilleurs il s'agissait plus souvent de juste rester ensemble pendant deux ou trois ans pour élever l'enfant à ses niveaux de développement les plus vulnérables… et, enfin, étant donné combien de personnes finissent malheureuses dans les mariages traditionnels il semble que c'est le genre de choses qui pourrait bénéficier de remaniements astucieux - surtout si on résout le problème de l'immortalité…"

\later

Tano Wolfe, un Serdaigle en cinquième année, se leva lentement de son bureau de lecture d'où il venait de voir Granger fuir la bibliothèque en sanglotant. Il n'avait pas pu entendre la dispute, mais ça avait clairement été une de \emph{ces} disputes.

Lentement et les genoux tremblants, Tano s'approcha du Survivant qui regardait les portes de la bibliothèques encore tremblantes de la force avec laquelle elles avaient été rabattues.

Tano n'avait pas particulièrement envie de faire ça mais Harry Potter \emph{avait} été Réparti à Poudlard. Le Survivant était techniquement un autre Serdaigle. Et ça voulait dire qu'il y avait un Code à respecter.

Le garçon ne dit rien lorsque Tano s'approcha de lui, mais son regard n'était pas amical.

Tano déglutit, posa une main sur l'épaule de Harry Potter et récita d'une voix qui ne se brisa que légèrement~: "Ah, les sorcières~! Allez comprendre, hein~?"

"\emph{Enlève ta main avant que je ne la jette dans les ténèbres extérieures.}"

Les portes de la librairie s'ouvrirent grand une fois de plus sur le passage d'un second départ. 

%  LocalWords:  Outstandings vas followink S’s Eep Tano
