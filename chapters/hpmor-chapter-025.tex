\chapter{Se retenir de proposer des solutions}

\section{Acte 2~:\protect\footnotemark}
\authorsnotetext{Étant donné que les faits scientifiques présentés dans cette histoire sont généralement tous vérifiés, je vous préviens que dans les chapitres 22 à 25 Harry a négligé de nombreuses possibilités, la plus importante étant qu'il puisse y avoir de nombreux gènes magiques mais qu'ils soient tous sur le même chromosome (ce qui ne pourrait pas survenir naturellement, mais le chromosome magique aurait pu être fabriqué). Dans ce cas, l'hérédité aurait un caractère mendélien, mais le chromosome magique pourrait quand même être dégradé par des croisements avec son homologue non-magique (Harry a découvert Mendel et les chromosomes dans des livres d'histoire de la science, mais il n'a pas assez étudié la génétique pour connaître les croisements chromosomiques. Hé, il n'a que onze ans). Cela dit, et bien qu'une revue scientifique moderne aurait trouvé \emph{bien} des raisons d'ergoter, tout ce que Harry présente comme étant des indices de taille le sont en effet bel et bien -- les autres possibilités sont \emph{improbables}.}

(Le soleil luisait avec force dans la Grande Salle depuis la verrière enchantée au-dessus des étudiants assis sous le ciel nu, les baignant de lumière, resplendissant sur leur assiette et leur bol tandis que, frais et dispos après une nuit de sommeil, ils dévoraient leur petit déjeuner en prévision des plans qu'ils avaient faits pour dimanche)

\lettrine{D}{onc}. Il n'y avait qu'une seule chose qui faisait de vous un sorcier.

Ce n'était pas surprenant, quand on y réfléchissait. La tâche principale de l'ADN était de dire aux ribosomes comment ils devaient assembler les acides aminés pour en faire des protéines. La physique conventionnelle semblait tout à fait capable de décrire les acides aminés, et peu importe combien vous en mettiez à la file, selon la physique conventionnelle vous n'en tireriez absolument jamais la moindre magie.

Et pourtant la magie semblait être héréditaire au même titre que l'ADN.

Ce n'était probablement \emph{pas} parce que l'ADN était capable de fabriquer des protéines magiques en assemblant des acides aminés non-magiques.

La séquence ADN clé ne vous donnait pas d'elle-même de la magie.

La magie venait d'ailleurs.

(À la table des Serdaigle se trouvait un garçon qui regardait dans le vide tandis que sa main droite engouffrait dans sa bouche des cuillerées d'une nourriture sans importance depuis le récipient qui se trouvait en face de lui. Vous auriez pu y substituer un tas de terre qu'il ne s'en serait probablement pas rendu compte).

Et pour une raison ou une autre, la Source de Magie faisait attention à un marqueur ADN en particulier chez les individus qui à tout autre égard étaient des humains descendants des singes.

(Il y avait à vrai dire une bonne quantité de garçons et de filles qui regardaient dans le vide. C'était la table des \emph{Serdaigle} après tout).

Il y avait d'autres fils logiques qui menaient à cette même conclusion. Les machineries \emph{complexes} étaient toujours universelles chez les espèces à reproduction sexuée. Si le gène B dépendait d'un gène A, alors A devait avoir sa propre utilité et devenir de lui-même quasiment universel au sein d'un patrimoine génétique avant que B ne soit assez souvent utile pour pouvoir à son tour conférer une valeur sélective. Puis, une fois que B était devenu universel, vous aviez une variante A* qui dépendait de B, puis C qui dépendait de A* et B, puis B* qui dépendait de C, jusqu'à ce que la machine entière ne promette de s'effondrer le jour où vous enlèveriez une seule de ses pièces. Mais tout devait se produire de façon \emph{graduelle} -- l'évolution ne prévoyait jamais à l'avance, elle ne commencerait jamais à promouvoir B \emph{en prévision} du fait que A allait devenir universel. L'évolution était simplement la constatation historique que, quels que soient les organismes qui avaient eu le plus d'enfants, ce seraient leurs gènes qui seraient plus fréquents dans la prochaine population. Chaque pièce de la complexe machine devait donc devenir quasiment universelle avant que les autres pièces ne commencent à dépendre de sa présence.

Une machinerie \emph{complexe} et \emph{interdépendante} -- le système protéinique puissant et sophistiqué qui faisait fonctionner la vie -- était donc toujours \emph{universel} chez une espèce à reproduction sexuée -- mis à part, présente à chaque instant, une petite poignée de \emph{variantes} non-\emph{interdépendantes} engagées dans le processus de sélection tandis que des complexités supplémentaires étaient mises en place. C'était pour cela que tous les êtres humains avaient le même schéma cérébral, les mêmes émotions, les mêmes expressions faciales connectées à ces émotions~; ces adaptations étaient complexes, elle \emph{devaient} donc être universelles.

Si la magie était ainsi, c'est-à-dire une gigantesque et complexe adaptation nécessitant de nombreux gènes, alors un sorcier s'accouplant avec une Moldue aurait produit un enfant doté de la moitié des pièces, et une moitié de machine ne ferait pas grand-chose. Et il n'y aurait donc jamais eu de nés-Moldus. Même si toutes les pièces de la machine s'étaient retrouvées dans le patrimoine génétique moldu, elles ne se seraient jamais rassemblées sous la forme exacte qui permettait d'obtenir un sorcier.

Il n'y avait donc jamais eu de vallée où auraient vécu des humains génétiquement isolés qui seraient par hasard tombés sur une voie de l'évolution menant à la présence dans le cerveau de complexes structures magiques, car la machinerie génétique ne se serait dans ce cas jamais rassemblée chez des nés-Moldus lorsque des sorciers et des Moldus se seraient accouplés.

Donc quelle que soit la raison pour laquelle vos gènes faisaient de vous un sorcier, ce n'était \emph{pas} parce qu'ils contenaient les plans d'une machine compliquée.

C'était l'autre raison pour laquelle Harry avait deviné que le caractère mendélien serait présent. Si les gènes magiques n'étaient pas compliqués, alors pourquoi y en aurait-il plus d'un~?

Et pourtant la magie elle-même semblait plutôt compliquée. Un sort de verrouillage empêchait la porte de s'ouvrir \emph{et} il vous empêchait de métamorphoser les gonds \emph{et} il résistait à \emph{Finite Incantatem} et à \emph{Alohomora}. De nombreux éléments pointant tous dans la même direction que l'on aurait pu appeler “orientation vers un but” ou, plus simplement, “intentionnalité”.

Il n'y avait que deux causes connues de complexité orientée vers un but. La sélection naturelle, qui produisait des choses comme des papillons. Et l'ingénierie intelligente, qui produisait des choses comme des voitures.

La magie ne ressemblait pas à quelque chose qui se serait mise à exister par auto-réplication. Les sorts étaient intentionnellement compliqués, mais, contrairement à un papillon, pas compliqués dans le but de faire des copies d'eux-mêmes. Les sorts étaient compliqués dans le but de servir leurs utilisateurs, comme l'était une voiture.

Un ingénieur intelligent avait donc créé la Source de Magie et lui avait dit de faire attention à un marqueur ADN en particulier.

L'idée suivante de ce raisonnement était évidemment que ça avait un rapport avec “Atlantis”.

Plus tôt, Harry avait interrogé Hermione à ce sujet -- sur le train vers Poudlard, après avoir entendu Drago le prononcer -- et pour ce qu'elle en savait, on n'en savait pas plus que le mot lui-même.

Ça aurait pu être une pure légende. Mais il était aussi plausible qu'une civilisation d'utilisateurs de la magie, soit parvenue à se faire sauter, en particulier une civilisation datant d'\emph{avant} l'Interdit de Merlin.

Le raisonnement continuait~: Atlantis avait été une civilisation isolée qui était parvenue d'une façon ou d'une autre à créer la Source de Magie, et Atlantis avait ordonné à cette source de ne servir que les gens porteurs du marqueur génétique atlante, porteurs du sang d'Atlantis.

Et par une logique similaire~: les mots que les sorciers prononçaient, les mouvements de baguettes, rien de cela n'était assez compliqué pour produire les effets du sort à partir de rien -- pas de la façon dont trois-milliards de paires dans un ADN humain \emph{étaient} assez compliquées pour construire un corps humain à partir de rien, pas de la façon dont un programme informatique tenait sur des milliers d'octets de données.

Les mots et les mouvements de baguettes n'étaient donc que des déclencheurs, des leviers actionnés sur une machine cachée et plus complexe. Des boutons, pas des plans.

Et tout comme un programme informatique ne serait pas compilé si vous faisiez une seule erreur, la Source de Magie ne vous répondrait pas tant que vous n'auriez pas jeté vos sorts exactement comme il le fallait.

Le raisonnement logique était inexorable.

Et il menait à une seule conclusion finale.

Des milliers d'années plus tôt, les aïeux des sorciers avaient dit à la Source de Magie de faire léviter les choses uniquement si vous disiez…

“Wingardium Leviosa”.

Harry s'effondra sur la table des Serdaigle, reposant son front sur sa main droite avec lassitude.

Il y avait une histoire qui datait de l'aube de l'Intelligence Artificielle -- à l'époque où ils commençaient tout juste et que personne ne s'était encore rendu compte que le problème serait difficile -- au sujet d'un professeur qui avait délégué à un de ses étudiants la tâche de résoudre le problème de la vision informatique.

Harry commençait à comprendre ce que l'étudiant devait avoir ressenti.

Ça pourrait prendre un moment.

Pourquoi était-il plus difficile de jeter le sort Alohomora si c'était comme d'appuyer sur un bouton~?

Qui avait été assez idiot pour construire un sort d'\emph{Avada Kedavra} qui ne pouvait être jeté qu'avec de la haine~?

Pourquoi la métamorphose muette requérait que vous fassiez une séparation mentale complète entre le concept de forme et le concept de matériau~?

Harry n'aurait peut-être pas réglé ce problème à la fin de ses études à Poudlard. Il y travaillerait peut-être encore à \emph{trente ans}. Hermione avait raison. Harry ne s'en était pas rendu compte, pas du fond de son cœur. Il avait juste fait un discours exaltant sur le fait d'être déterminé.

L'esprit de Harry envisagea brièvement de comprendre du fond de son cœur qu'il ne résoudrait peut-être jamais le problème, puis il décida que ce serait aller beaucoup trop loin.

Et puis, du moment qu'il pouvait atteindre l'immortalité dans les premières décennies, tout irait bien.

Quelle méthode le Seigneur des Ténèbres avait-il utilisée~? Maintenant qu'il y pensait, le fait que le Seigneur des Ténèbres soit parvenu à survivre à la mort de son premier corps était presque \emph{infiniment} plus important que le fait qu'il ait essayé de conquérir l'Angleterre magique -

<<~Excusez-moi~>>, dit une voix attendue venant de dernière lui, parlant d'un ton très inattendu. <<~Au moment qui vous siéra, M. Malfoy demande la faveur d'une conversation.~>>

Harry ne s'étouffa pas sur ses céréales. Au lieu de ça, il pivota et observa M. Crabbe.

<<~Excuse-\emph{moi}, dit Harry. Ne voulais-tu pas dire “Ya l'boss qui veut t'faire un brin d'causette~?”~>>

M. Crabbe n'avait pas l'air content.

<<~M. Malfoy m'a donné instruction de parler convenablement.

--- Je n'arrive pas à t'entendre, dit Harry. Tu ne parles pas convenablement.~>> Il se retourna vers son bol de petits cristaux de neige bleus et mangea délibérément une autre cuillerée.

<<~Ya l'boss qui veut t'faire un brin d'causette, dit une voix menaçante venant de derrière. Tu f'rais mieux de venir y voir si tu sais c'qu'est bon pour toi.~>>

Voilà. \emph{Maintenant}, tout se déroulait selon le plan.

\latersection{Acte 1~:}

<<~Une \emph{raison}~?~>> dit le vieux sorcier. Il empêcha la furie de déformer ses traits. Le garçon devant lui avait été la victime, et il n'avait certainement pas besoin d'être encore plus effrayé. <<~Il n'y a \emph{rien} qui puisse excuser -

--- Ce que je lui ai fait était pire.~>>

Le vieux sorcier se raidit sous le coup de l'horreur soudaine.

<<~Harry, \emph{qu'as-tu fait}~?

--- J'ai manipulé Drago pour qu'il croie que je l'avais manipulé pour qu'il participe à un rituel qui a sacrifié sa croyance dans le purisme du sang. Ce qui veut dire qu'il ne pourra pas devenir un Mangemort quand il sera grand. Professeur, il avait tout perdu.~>>

Il y eut un long silence dans le bureau, seulement brisé par les petits soufflements et sifflements des appareils délicats qui après un moment finirent par ressembler à du silence.

<<~Doux Merlin, dit le vieux sorcier, je me \emph{sens} idiot. Et j'étais \emph{là} à croire que tu essaierais de racheter l'héritier des Malfoy, en lui, disons en lui \emph{démontrant de l'amitié et de la gentillesse véritables}.

--- \emph{Ha~!} Ouais, comme si \emph{ça} aurait marché.~>>

Le vieux sorcier soupira. Ça allait trop loin.

<<~Dis-moi, Harry, t'est-il déjà \emph{venu} à l'esprit qu'il y avait quelque chose d'\emph{incongru} dans le fait d'amener quelqu'un à la \emph{repentance} en utilisant le mensonge et la manipulation~?

--- Je l'ai fait sans dire de mensonge direct, et puisque nous parlons de Drago Malfoy, je pense que le mot que vous cherchez est \emph{congruent}.~>> Le garçon avait l'air plutôt fier.

Le vieux sorcier secoua sa tête avec désespoir. <<~Et c'est \emph{ça} le héros. Nous sommes tous foutus.~>>

\latersection{Acte 5~:}

Le long tunnel étroit fait de pierre brute semblait s'étirer sur des kilomètres. Il aurait été parfaitement obscur sans la baguette d'un enfant.

La raison à cela était simple~: il \emph{s'étirait} sur des kilomètres.

L'heure~: trois heures du matin, et Fred et George commençaient le long chemin dans le passage secret qui menait de la statue d'une sorcière à un œil, située à l'intérieur de Poudlard, jusqu'à la cave de la confiserie de Pré-au-Lard.

<<~Comment elle va~?~>> dit Fred d'une voix basse.

(Non pas que quelqu'un les ait écoutés, mais il y avait quelque chose d'étrange à parler d'une voix normale quand on traversait un passage secret).

<<~Toujours en panne, dit George.

---Les deux ou -

--- L'intermittent s'est à nouveau réparé. L'autre est comme d'habitude.~>>

La Carte était un artefact extraordinairement puissant, capable de suivre à la trace tout être sentient\protect\footnotemark présent dans l'enceinte de Poudlard, en temps réel, désigné par son nom. Elle avait presque certainement été créée lors de l'apparition de Poudlard. Il n'était \emph{pas bon signe} que des erreurs commencent à apparaître. Si elle était cassée, il était probable que personne à part Dumbledore ne puisse la réparer.
\authorsnotetext{NdT~:J'utilise le mot anglais \emph{sentient} à la façon de Guy Abadia dans l'Étoile et le Fouet.}

Et les jumeaux Weasley n'étaient pas prêts à donner la carte à Dumbledore. Cela aurait été une insulte impardonnable à l'encontre des Maraudeurs -- les quatre inconnus qui étaient parvenus à voler une partie du \emph{système de sécurité de Poudlard}, quelque chose qui avait probablement été forgé par Salazar lui-même, et à le transformer en un \emph{outil de farces estudiantines}.

Certains auraient pu considérer cela irrespectueux.

D'autres auraient pu considérer cela criminel.

Les jumeaux Weasley avaient la ferme croyance que si Godric Gryffondor avait été là pour le voir, il aurait approuvé.

Les frères marchèrent et marchèrent, principalement en silence. Les jumeaux Weasley se parlaient quand ils réfléchissaient à de nouvelles farces, ou quand l'un des deux savait quelque chose que l'autre ignorait. Sinon, ça ne servait pas à grand-chose. S'ils avaient déjà les mêmes informations, ils avaient tendance à avoir les mêmes idées et à prendre les mêmes décisions.

(Par le passé, quand des jumeaux magiques identiques naissaient, il avait été de coutume de tuer l'un des deux après leur naissance).

Les jumeaux Weasley arrivèrent enfin jusqu'à une cave poussiéreuse parsemée de tonneaux et d'étagères porteuses d'ingrédients étranges.

Fred et George attendirent. Il n'aurait pas été poli d'agir autrement.

Un vieil homme mince descendit bientôt les escaliers qui menaient à la cave en bâillant. <<~Salut les garçons, dit Ambrosius Flume. Je ne m'attendais pas à vous voir cette nuit. Déjà en rupture de stock~?~>>

Fred et George décidèrent que c'était Fred qui parlerait.

<<~Pas exactement, M. Flume, dit Fred. Nous espérions que vous pourriez nous aider pour quelque chose de considérablement plus… intéressant.

--- Allons, les garçons, dit Flume d'un air sévère, j'espère que vous ne m'avez pas réveillé juste pour que je vous répète que je ne vous vendrai rien qui puisse vraiment vous mettre dans le pétrin. Pas avant que vous n'ayez seize ans, en tout cas -~>>

George fit surgir un objet de sa robe et le passa à Flume sans mot dire. <<~Avez-vous vu ceci~?~>> dit Fred.

Flume regarda l'édition de la Gazette du sorcier de la veille et hocha la tête en faisant une mine renfrognée. Le gros titre disait LE PROCHAIN SEIGNEUR DES TÉNÈBRES~? et montrait un jeune garçon que l'appareil photo d'un étudiant était parvenu à saisir à un moment où il arborait une expression inhabituellement froide et grimaçante.

<<~Je n'arrive pas à croire que ce Malfoy fait ça, lâcha Flume. S'en prendre à un garçon quand il a à peine onze ans~! Cet homme devrait être broyé en poudre et utilisé pour faire des chocolats~!~>>

Fred et Georges clignèrent des yeux à l'unisson. \emph{Malfoy} était derrière Rita Skeeter~? Harry Potter ne leur en avait pas fait part… ce qui voulait sûrement dire que Harry n'était pas au courant. Il ne les aurait jamais impliqués s'il avait su…

Fred et George échangèrent des regards. Eh bien, Harry n'avait pas \emph{besoin} de le savoir, pas avant que le travail soit fini.

<<~M. Flume, dit doucement Fred, le Survivant a besoin de votre aide.~>>

Flume les regarda tous deux.

Puis il laissa l'air s'échapper de ses poumons et soupira.

<<~Très bien, dit Flume, qu'est-ce que vous voulez~?~>>

\latersection{Acte 6~:}

Quand Rita Skeeter était concentrée sur une proie savoureuse, elle avait tendance à ne pas remarquer les fourmis courant en tous sens qui constituaient le reste de l'univers, et c'est pour cela qu'elle faillit percuter le jeune homme à la calvitie naissance qui s'était mis sur son chemin.

<<~Mademoiselle Skeeter~>>, dit l'homme d'un ton plutôt sévère et froid pour quelqu'un qui avait l'air si jeune. <<~C'est amusant de vous rencontrer ici.

--- Hors de mon chemin, mon gars~!~>> lâcha Rita, et elle essaya de le contourner.

L'homme qui lui bloquait la route reproduit le mouvement si parfaitement qu'on aurait dit qu'aucun d'eux n'avait bougé mais qu'ils s'étaient tenus immobiles tandis que la rue se décalait autour d'eux.

Les yeux de Rita se rétrécirent.

<<~Vous vous prenez pour qui~?

--- Que c'est bête, dit sèchement l'homme. Il aurait été sage de mémoriser le visage du Mangemort déguisé qui entraînait Harry Potter à devenir le prochain Seigneur des Ténèbres. Après tout~>>, un léger sourire, <<~ce n'est \emph{vraiment} pas le genre de personne que vous voudriez croiser dans la rue, surtout après l'avoir démolie dans le journal.~>>

Rita mit un moment à comprendre la référence. C'était \emph{lui}, Quirinus Quirrell~? Il avait l'air trop jeune et trop vieux à la fois~; son visage, s'il s'était détendu et s'était défait de son air sévère et condescendant, aurait appartenu à quelqu'un de bientôt quarante ans. Et ses cheveux tombaient déjà~? Ne pouvait-il pas se payer un Médicomage~?

Non, ça n'avait pas d'importance, elle avait un horaire à respecter, un endroit où aller, et un scarabée à devenir. Elle venait de recevoir une information anonyme disant que Madame Bones faisait la cour à un de ses jeunes assistants. Ça lui vaudrait une sacrée prime si elle parvenait à le vérifier, Bones était haut placée dans la liste des cibles à abattre. L'informateur avait dit que Bones et son jeune assistant avaient rendez-vous pour déjeuner dans une pièce spéciale de \emph{Chez Marie}, une pièce très populaire pour qui avait certaines faims~; une pièce qui, comme elle l'avait découvert, était protégée contre les dispositifs d'écoute, mais pas contre un magnifique scarabée bleu lové contre un mur…

<<~Hors de mon \emph{chemin}~!~>> dit Rita, et elle essaya d'éjecter Quirrell hors de sa trajectoire. Le bras de Quirrell passa contre le sien, déviant sa force, et Rita chancela quand sa poussée atterrit dans le vide.

Quirrell releva la manche gauche de sa robe. <<~Voyez, dit Quirrell, pas de Marque des Ténèbres. Je voudrais que votre journal se rétracte publiquement.~>>

Rita laissa échapper un rire incrédule. Bien sûr que l'homme n'était pas un vrai Mangemort. Sinon l'article n'aurait pas été publié. <<~Laisse tomber mon gars. Et maintenant, va faire un tour.~>>

Quirrell la regarda un moment.

Puis il sourit.

<<~Mademoiselle Skeeter, dit Quirrell, j'avais espéré trouver quelque levier qui s'avérerait persuasif. Mais je me rends compte que je ne suis pas capable de me refuser le plaisir de simplement vous écraser.

--- Ça a déjà été essayé. Maintenant hors de mon chemin, pauvre type, ou je trouverai quelques Aurors et vous ferai arrêter pour obstruction au journalisme.~>>

Quirrell exécuta un petit salut, puis il continua son chemin. La voix de Quirrell lui parvint depuis son dos~: <<~Au revoir, Rita Skeeter.~>>

Et alors qu'elle se remettait à foncer, Rita remarqua, dans un coin de sa tête, que l'homme sifflotait une mélodie tout en s'en allant.

Comme si \emph{ça} allait lui faire peur.

\latersection{Acte 4~:}

<<~Désolé, sans moi, dit Lee Jordan. Je suis plutôt du genre araignée géante.~>>

Le Survivant avait dit qu'il avait du travail \emph{important} pour l'Ordre du Chaos, quelque chose de sérieux et de secret, de plus difficile et de plus significatif que leurs séries de blagues habituelles.

Puis Harry se lança dans un discours exaltant mais vague. Un discours qui disait que Fred et George et Lee avaient un potentiel énorme, si seulement ils apprenaient à être plus \emph{bizarres}. À rendre la vie des gens \emph{surréaliste}, au lieu de simplement les surprendre avec l'équivalent de seaux d'eau posés au-dessus de portes. (Fred et George avaient échangé des coups d'œils intéressés, ils n'y avaient jamais songé avant). Harry Potter avait invoqué l'image de la farce qu'il avait faite à Neville, au sujet de laquelle, avait-il mentionné avec quelque remords, le Choixpeau l'avait incendié, mais qui avait dû pousser Neville à \emph{douter de sa propre santé mentale}. Pour Neville, ça avait été comme d'être soudainement transporté dans un univers parallèle. Comme ce que tout le monde avait ressenti quand ils avaient vu Rogue s'excuser. C'était le \emph{véritable pouvoir de la farce}.

\emph{Êtes-vous avec moi}~? s'était écrié Harry, et Lee Jordan avait répondu non.

<<~\emph{Avec} nous~>>, dit Fred, ou peut-être George, car il ne faisait aucun doute que Godric Gryffondor aurait dit oui.

Lee Jordan fit un sourire de regret, se leva et quitta le corridor désert et sourdiné où les quatre membres de l'Ordre du Chaos s'étaient retrouvés et s'étaient assis dans un cercle conspirateur.

Les trois membres de l'Ordre du Chaos se mirent au travail.

(Ce n'était pas \emph{si} triste que ça. Fred et George travailleraient toujours avec Lee sur des farces à base d'araignées géantes, comme avant. Ils avaient seulement commencé à appeler leur groupe l'Ordre de Chaos pour pouvoir recruter Harry Potter, après que Ron leur eut dit que Harry Potter était bizarre et maléfique, et Fred et George avaient décidé de sauver Harry en lui démontrant de l'amitié et de la gentillesse véritables. Heureusement, ça ne semblait plus nécessaire -- quoique, ils n'en étaient pas \emph{tout à fait} certains…)

<<~Donc, dit l'un des jumeaux, de quoi s'agit-il~?

--- Rita Skeeter, dit Harry. Savez-vous qui c'est~?~>>

Fred et George hochèrent la tête en fronçant les sourcils.

<<~Elle s'est mise à poser des questions à mon sujet.~>>

Ce n'était pas une bonne nouvelle.

<<~Pouvez-vous deviner ce que je veux que vous fassiez~?~>>

Fred et George se regardèrent, un peu confus.

<<~Tu veux que nous lui glissions quelques-unes de nos sucreries les plus intéressantes~?

--- Non, dit Harry. Non, non, \emph{non}~! C'est une mentalité d'araignée géante, ça~! Allez, que feriez-\emph{vous} si vous aviez entendu dire que Rita Skeeter cherchait des rumeurs à \emph{votre} sujet~?~>>

Voilà qui rendait la réponse évidente.

Des sourires apparurent lentement sur les visages de Fred et George.

<<~Répandre des rumeurs à notre propre sujet, répondirent-ils.

--- \emph{Exactement}~>>, dit Harry, souriant largement. <<~Mais ça ne doit pas être \emph{n'importe quelles} rumeurs. Je veux enseigner aux gens à ne jamais croire ce que le journal dit à propos de Harry Potter, pas plus que les Moldus ne croient ce que le journal dit à propos d'Elvis. Au début, j'ai juste pensé à inonder Rita Skeeter avec tellement de rumeurs qu'elle ne saurait plus laquelle croire, mais alors elle cueillerait juste celles qui seraient à la fois plausibles et négatives. Donc ce que je veux que vous fassiez c'est de créer un mensonge me concernant et de vous débrouiller pour que Rita Skeeter y croie. Mais ça doit être quelque chose que, plus tard, tout le monde \emph{saura} avoir été faux. Nous voulons tromper Rita Skeeter et ses éditeurs, et \emph{ensuite} avoir la preuve que c'était faux. Et bien sûr -- puisque ce sont là les conditions -- le mensonge doit être aussi \emph{ridicule} que possible tout en étant quand même imprimé. Comprenez-vous ce que je veux que vous fassiez~?

--- Pas exactement… dit lentement Fred ou George. Tu veux qu'on \emph{invente} l'histoire~?

--- Je veux que vous fassiez \emph{tout}, dit Harry Potter. Je suis plutôt occupé en ce moment, et puis je veux pouvoir dire sans mentir que je n'avais aucune idée de ce qui allait se passer. Surprenez-moi.~>>

Pendant un moment, un sourire maléfique fut visible sur les visages de Fred et George.

Puis ils redevinrent sérieux.

<<~Mais Harry, nous ne savons vraiment pas comment faire une chose pareille -

--- Alors trouvez comment faire, dit Harry. Je vous fais confiance. Pas une confiance \emph{totale}, mais si vous ne \emph{pouvez pas} le faire, \emph{dites-} le moi, et j'essaierai avec quelqu'un d'autre, ou bien je le ferai moi-même. Si vous avez une très bonne idée -- pour l'histoire ridicule et sur la façon de convaincre Rita Skeeter et ses éditeurs de l'imprimer -- alors vous pouvez y aller et le faire. Mais ne faites pas quelque chose de médiocre. Si vous ne trouvez pas une idée \emph{géniale}, dites-le.~>>

Fred et George échangèrent des regards inquiets.

<<~Je n'en ai aucune, dit George.

--- Moi non plus, dit Fred. Désolé.~>>

Harry les regarda fixement.

Et il commença alors à leur expliquer comment il fallait s'y prendre pour réfléchir.

Il remarqua qu'on avait souvent vu pareille tâche prendre plus de deux secondes.

On ne déclarait \emph{jamais aucune} question impossible, dit-il, tant qu'on n'avait pas physiquement pris une montre en main et qu'on avait pas réfléchi à la question pendant cinq minutes en se basant sur l'aiguille des minutes. Pas cinq minutes métaphoriques, cinq minutes selon une montre bien réelle.

Et \emph{de plus}, dit Harry avec vigueur tandis que sa main droite frappait durement le sol, on ne commençait \emph{pas} par chercher immédiatement des solutions.

Harry se lança alors dans l'explication d'un test qui avait été réalisé par quelqu'un nommé Norman Maier, qui était ce qu'on appelait un psychologue du travail et qui avait demandé à deux séries de groupes de résoudre un problème.

Le problème, avait dit Harry, mettait en jeu trois employés et trois tâches. L'employé le plus jeune voulait faire la tâche la plus simple. L'employé le plus âgé voulait passer d'une tâche à l'autre pour ne pas s'ennuyer. Un expert en efficacité avait recommandé qu'on donne le travail le plus simple à la personne la plus jeune et le travail le plus difficile à la personne la plus âgée, qui serait 20~\% plus productive.

\emph{Une} des séries de groupes avait reçu l'instruction suivante~: <<~Ne proposez aucune solution tant qu'il est possible de débattre minutieusement du problème sans en proposer.~>>

L'autre série de groupes n'avait reçu aucune instruction. Et ces gens avaient naturellement réagi à la présence du problème en proposant des solutions. Et ils s'étaient retrouvés attachés à ces solutions, et ils avaient commencé à se disputer et à se battre au sujet de l'importance relative de la liberté et de l'efficacité et ainsi de suite.

La première série de groupes, à qui on avait donné instruction de d'abord \emph{discuter} du problème et \emph{ensuite} de le résoudre, avait trouvé la réponse bien plus souvent. Elle consistait à laisser l'employé le plus jeune garder la tâche la plus facile et à alterner les deux autres personnes entre les deux autres tâches, ce qui, selon les données de l'expert, constituerait une amélioration de 19~\%.

Commencer à chercher des solutions, c'était faire les choses \emph{complètement dans le désordre}. Comme de commencer un repas par le dessert, mais en \emph{mauvais}.

(Harry cita aussi quelqu'un nommé Robyn Dawes, qui aurait dit que plus un problème était difficile, plus il était probable que les gens essaient de le résoudre immédiatement).

Harry allait donc laisser ce problème à Fred et à George, et ils discuteraient de toutes ses facettes, et ils se remueraient les méninges à la recherche de tout ce qui pourrait avoir un rapport avec le problème. Et ils n'essaieraient pas de trouver une vraie solution avant d'en avoir fini avec cette étape de discussion, \emph{à moins} bien sûr qu'ils ne pensent à quelque chose par pure chance, auquel cas ils pourraient le noter pour plus tard et se remettre à réfléchir. Et il ne voulait pas entendre parler de leur soi-disant \emph{incapacité à avoir des idées} avant au moins une semaine. Certaines personnes passaient des \emph{décennies} à essayer d'avoir des idées.

<<~Des questions~?~>> dit Harry.

Fred et George se regardèrent.

<<~Je n'en ai aucune, dit George.

--- Moi non plus.~>>

Harry toussota gentiment. <<~Vous n'avez rien demandé concernant votre budget.~>>

\emph{Budget~?} pensèrent-ils.

<<~Je pourrais juste vous dire la somme, dit Harry, mais je pense que \emph{ceci} sera plus \emph{exaltant}.~>>

Les mains de Harry plongèrent dans sa robe, et firent apparaître…

Fred et George tombèrent presque à la renverse, bien qu'ils aient été assis.

<<~Ne le dépensez pas pour le plaisir de dépenser,~>> dit Harry. Sur le sol de pierre brillait une quantité d'argent absolument ridicule. <<~Ne le dépensez que si le génie le requiert~; et ce que le génie requiert, n'hésitez pas à le dépenser. S'il en reste, rendez-le-moi après, je vous fais confiance. Oh, et vous avez dix pour cent de ce qui est là, peu importe combien vous dépensez…

--- On ne \emph{peut pas}~! lâcha l'un des jumeaux. On n'accepte pas d'argent pour ce genre de chose~!~>>

(Les jumeaux ne prenaient jamais d'argent pour faire quelque chose d'illégal. Ils vendaient toute leur marchandise avec une marge de zéro pour cent, et ce à l'insu d'Ambrosius Flume. Fred et George voulaient pouvoir témoigner -- sous Veritaserum si nécessaire -- qu'ils n'avaient jamais profité d'activités criminelles, qu'ils assuraient seulement un service public).

Harry fronça les sourcils. <<~Mais je vous demande de faire un vrai travail. Un adulte se ferait payer pour faire quelque chose comme ça, et ça serait \emph{quand même} considéré comme une faveur à un ami. On ne peut pas juste engager des gens pour ce genre de chose.~>>

Fred et George secouèrent leur tête.

<<~Très bien, dit Harry. Je vous offrirai juste des cadeaux de Noël très chers, et si vous essayez de me les rendre, je les brûlerai. Maintenant vous ne \emph{savez} même pas combien je vais dépenser pour vous, à part bien sûr que ce sera plus que si vous aviez juste pris l'argent. Et je vais vous offrir ces cadeaux \emph{quoi qu'il arrive}, alors pensez-\emph{y} avant de me dire que vous \emph{n'arrivez pas à avoir d'idée géniale}.~>>

Harry se leva, sourit, et se tourna pour partir alors que Fred et George étaient encore bouche bée sous l'effet du choc. Il s'éloigna de quelques pas, puis se retourna.

<<~Oh, une dernière chose, dit Harry. Laissez le professeur Quirrell en dehors de ce que vous faites. Il n'aime pas la publicité. Je sais que ce serait plus facile de faire croire des choses bizarres au sujet du professeur de Défense qu'au sujet de n'importe qui d'autre, et je suis désolé de vous barrer la route comme ça, mais s'il vous plaît, laissez-le en dehors de ça.~>>

Et Harry se tourna de nouveau et fit quelques pas de plus…

Il regarda derrière lui une dernière fois, et dit, doucement~: <<~Merci.~>>

Puis il s'en fut.

Il y eut un long silence après son départ.

<<~Donc, dit l'un.

--- Donc, dit l'autre.

--- Le professeur de Défense n'aimerait donc pas la publicité.

--- Harry ne nous connaîtrait donc pas très bien.

--- Non, pas très bien.

--- Mais on n'utilisera pas son argent pour ça, bien sûr.

--- Bien sûr que non. Ça ne serait pas correct. On s'occupera du professeur de Défense séparément.

--- On dira à quelques Gryffondor d'écrire à Skeeter et de dire…

--- … sa manche relevée un jour en cours de Défense, et ils ont vu la Marque des Ténèbres…

--- … et il enseigne probablement plein de choses terribles à Harry Potter…

--- … et il est le pire professeur de Défense de mémoire d'homme à Poudlard, il n'\emph{échoue} pas seulement à nous apprendre des choses, il explique aussi tout de travers, l'inverse complet de qu'il devrait dire…

--- … comme quand il a prétendu qu'on pouvait uniquement jeter le Sort de Mort en utilisant de l'amour, ce qui le rendait plus ou moins inutile.

--- J'aime bien celle-là.

--- Merci.

--- Je parie que le professeur de Défense l'aime aussi.

--- Il a un sens de l'humour. Il n'aurait pas parlé de nous comme il l'a fait s'il n'en avait pas eu un.

--- Mais serons-nous capables de faire le travail pour Harry~?

--- Harry a dit de discuter du problème avant d'essayer de le résoudre, alors faisons ça.~>>

Les jumeaux Weasley décidèrent que George serait enthousiaste tandis que Fred douterait.

<<~Ça a l'air assez contradictoire, dit Fred. Il veut que ce soit assez ridicule pour que tout le monde se moque de Skeeter et sache que c'est faux, et il veut que Skeeter le croie. On ne peut pas faire les deux à la fois.

--- On va devoir falsifier des preuves pour convaincre Skeeter, dit George.

--- C'est une solution~?~>> dit Fred.

Ils envisagèrent cette solution.

<<~Peut-être, dit George, mais je ne pense pas qu'on devrait être \emph{aussi} strict que ça. Si~?~>>

Les jumeaux haussèrent les épaules avec impuissance.

<<~Alors la preuve falsifiée doit être assez bonne pour convaincre Skeeter, dit Fred. Peut-on vraiment faire ça nous-mêmes~?

--- On n'a pas à le faire nous-mêmes~>>, dit George, et il pointa le tas d'argent du doigt. <<~On peut engager d'autres gens pour nous aider.~>>

Les jumeaux prirent un air pensif.

<<~Ça utiliserait presque la totalité du budget de Harry, dit Fred. C'est beaucoup d'argent pour nous, mais pas pour quelqu'un comme Flume.

--- Peut-être que les gens donneront des ristournes quand ils sauront que c'est pour Harry, dit George. Mais le plus important, c'est que quoi qu'on fasse, ça doit être \emph{impossible}.~>>

Fred cligna des yeux.

<<~Qu'est-ce que tu veux dire par \emph{impossible}~?

--- Tellement impossible qu'on ne sera jamais dans le pétrin parce que personne ne croira jamais qu'on ait pu le faire. Tellement impossible que même Harry commencera à se poser la question. Ça doit être surréaliste, ça doit pousser les gens à douter de leur propre santé mentale, ça doit être… \emph{mieux que Harry}.~>>

Les yeux de Fred s'écarquillèrent sous l'effet de la stupéfaction. Ça arrivait parfois entre eux, mais pas souvent.

<<~Mais pourquoi~?

--- C'était des farces. C'était \emph{toutes} des farces. La tarte était une farce. Le Rapeltout était une farce. Le chat de Kevin Sifflebranche était une farce. \emph{Rogue} était une farce. \emph{Nous} sommes les meilleurs farceurs de Poudlard, allons-nous nous écraser et abandonner sans combattre~?

--- C'est le Survivant, dit Fred.

--- Et \emph{nous} sommes les jumeaux Weasley~! Il nous \emph{met au défi}. Il a dit qu'on pouvait faire ce qu'il fait. Mais je parie qu'il ne pense pas qu'on sera un jour aussi bon que \emph{lui}.

--- Il a raison,~>> dit Fred, se sentant plutôt nerveux. Les jumeaux Weasley étaient \emph{parfois} en désaccord, même quand ils disposaient des mêmes informations, mais à chaque fois que c'était le cas, cela ne leur semblait pas naturel, comme si au moins un d'entre eux était en train de se fourvoyer. <<~On est en train de parler de \emph{Harry Potter}. Il peut accomplir l'impossible. Pas nous.

--- Si, on peut, dit George. Et on doit accomplir \emph{plus} impossible que lui.

--- Mais -- dit Fred.

--- C'est ce que Godric Gryffondor ferait~>>, dit George.

Ce qui régla la discussion, et les jumeaux retournèrent instantanément à… leur état normal, quel qu'il fut.

<<~Très bien, alors -

--- -- réfléchissons-y.~>>
%  LocalWords:  wid youse Ya’d Honeydukes Ambrosius Entwhistle’s
