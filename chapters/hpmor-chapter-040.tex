\partchapter{Faire semblant d'être sage}{II}

\lettrine{T}{enant}sa tasse de la manière exacte que le professeur Quirrell avait dû par trois fois lui montrer, Harry prit une gorgée, petite, et méticuleuse. À l'autre bout de l'immense table située au centre de la Chambre de Marie, le professeur Quirrell but à son tour, avec bien plus de naturel et d'élégance. Le thé était fait d'une plante dont Harry était incapable de prononcer le nom. Plus précisément, il s'était tant fait rectifier par le professeur Quirrell qu'il avait abandonné l'idée d'en prononcer le nom chinois.

La fois précédente, Harry était parvenu à jeter un coup d'œil sur l'addition, et le professeur Quirrell avait fait semblant de ne rien remarquer.

Juste avant cela, il avait ressenti une forte envie de boire de l'Hilari-Thé.

\emph{Même en prenant cela en compte}, le prix l'avait sidéré.

Et pour lui, cela avait toujours le goût du, eh bien, du thé.

Mais il y avait un soupçon discret et obsédant, à la limite de sa conscience, que le professeur Quirrell le \emph{savait} et qu'il avait délibérément acheté un thé au prix exorbitant que Harry ne savait même pas apprécier \emph{juste pour jouer avec lui}. Le professeur Quirrell \emph{lui-même} ne l'aimait peut-être pas particulièrement. Peut-être que \emph{personne} n'aimait vraiment ce thé, et que sa seule fonction était d'avoir un prix exorbitant pour donner à la victime le sentiment qu'elle était ingrate. Peut-être que c'était en réalité un simple thé ordinaire, sauf qu'il fallait le demander à l'aide d'un code particulier pour qu'ils mettent un prix aussi gigantesque qu'imaginaire sur l'addition…

Le professeur Quirrell avait l'air pensif et les traits tirés. "Non", dit le professeur Quirrell, "vous n'auriez \emph{pas} dû parler au directeur de votre conversation avec Lord Malfoy. Essayez de réfléchir plus vite la prochaine fois, M. Potter.

--- Je suis désolé, professeur Quirrell," répondit humblement Harry. "Je ne vois toujours pas pourquoi." Parfois, Harry avait vraiment l'impression d'être un imposteur feignant d'être rusé lorsqu'il était en présence du professeur Quirrell.

"Lord Malfoy est l'opposant d'Albus Dumbledore," dit le professeur Quirrell. "Du moins pour le moment. Toute l'Angleterre est leur plateau d'échecs, tous les sorciers sont leurs pièces. Considérez ceci~: Lord Malfoy a menacé de tout laisser tomber, d'abandonner son jeu, pour se venger de vous si jamais il arrivait malheur à M. Malfoy. Auquel cas, M. Potter…~?"

Il lui fallut de longues secondes pour comprendre, mais il était clair que le professeur Quirrell n'allait pas donner d'autres indices, non pas que Harry les aurait acceptés.

Puis l'esprit de Harry établit enfin la connexion, et il fronça les sourcils. "Dumbledore tue Drago, fait croire que c'est ma faute, et Lucius sacrifie son jeu contre Dumbledore pour m'atteindre~? Cela… ne ressemble pas au \emph{style} du directeur, professeur Quirrell…" L'esprit de Harry eut un flash où Drago lui prodiguait une mise en garde semblable, et à laquelle Harry avait répondu la même chose.

Le professeur Quirrell haussa les épaules et sirota son thé.

Harry sirota le sien et resta assis en silence. La nappe qui recouvrait la table était faite d'un motif très paisible, qui au premier abord ressemblait à un tissu uni, mais si vous le regardiez suffisamment, ou que vous restiez silencieux assez longtemps, vous commenceriez à voir des traces à peine visibles de fleurs émettant une légère lueur~; les rideaux de la pièce avaient changé leur motif pour correspondre à celles-ci, et c'était comme s'ils brillaient d'une brise silencieuse. Le professeur Quirrell était d'une humeur contemplative ce samedi, et Harry aussi, et il semblait que la chambre de Marie n'avait pas omis de le remarquer.

"Professeur Quirrell," dit soudain Harry, "y a-t-il un au-delà~?"

Harry avait choisi sa question avec précaution. Pas \emph{croyez-vous à l'au-delà~?} mais simplement \emph{Y a-t-il un au-delà~?} Ce que les gens croyaient \emph{vraiment} ne leur semblait pas être des \emph{croyances}. Les gens ne disaient pas 'Je crois très fort que le ciel est bleu~!' Ils disaient simplement~: 'Le ciel est bleu". Votre vraie carte intérieure du monde vous semblait être exactement telle que le monde \emph{était}…

Le professeur de Défense éleva de nouveau sa tasse jusqu'à ses lèvres avant de répondre. Il avait une expression pensive. "S'il y en a un, M. Potter," dit le professeur Quirrell, "alors bon nombre de sorciers ont gâché beaucoup d'efforts dans leur quête de l'immortalité.

--- Ce n'est pas vraiment une réponse," observa Harry. Il avait maintenant appris à remarquer ce genre de chose lorsqu'il parlait au professeur Quirrell.

Le professeur Quirrell déposa sa tasse à thé, ce qui produisit un petit claquement aigu sur sa soucoupe. "Certains de ces sorciers étaient raisonnablement intelligents, M. Potter, et l'on peut donc admettre que l'existence d'un au-delà n'est pas évidente. Je me suis moi-même intéressé à ce problème. On a affirmé beaucoup de choses qui étaient similaires à ce que l'on s'attendrait à voir être produites par la peur ou par l'espoir. Parmi les témoignages dont la véracité n'est pas mise en doute, il n'existe rien qui ne pourrait être la conséquence de simple sorcellerie. Il existe certains appareils que l'on prétend être capables de communiquer avec les morts, mais je suspecte qu'ils sont seulement capables de projeter une image venue de l'esprit~; si le résultat semble indistinguable de la mémoire, c'est parce qu'il \emph{est} la mémoire. Les soi-disant esprits ne révèlent aucun secret qu'ils connaissaient de leur vivant, ni aucun qu'ils auraient pu apprendre après leur mort sans que ceux-ci soient connus de l'invocateur -

--- C'est pourquoi la Pierre de Résurrection n'est pas l'artefact magique le plus précieux au monde," dit Harry.

"Précisément," dit le professeur Quirrell, "même si je ne refuserais pas une chance de l'essayer." Il y avait un sourire fin et sec sur ses lèvres~; et quelque chose de plus froid et de plus distant dans ses yeux. "J'en conclus que vous avez aussi parlé de cela avec Dumbledore."

Harry hocha la tête.

Les rideaux adoptaient un motif légèrement bleuté, et un vague tracé représentant des flocons de neiges complexes semblait maintenant apparaître sur la nappe. La voix du professeur Quirrell était très calme. "Le directeur sait être très persuasif, M. Potter. J'espère qu'il ne vous a pas persuadé.

--- \emph{Certainement} pas," dit Harry. "Je ne l'ai pas cru une seconde.

--- Je n'espère pas," dit le professeur Quirrell, toujours de ce ton très calme. "Je serais extrêmement gêné de découvrir que le directeur vous a convaincu de laisser tomber votre vie pour un plan idiot en vous disant que la mort est la prochaine grande aventure.

--- Je ne pense pas que le directeur y croyait lui-même, à vrai dire," dit Harry. Il sirotait de nouveau son thé. "Il m'a demandé ce que je pourrais bien faire avec l'éternité, puis il m'a donné la réplique habituelle selon laquelle ce serait ennuyeux, et il ne semblait pas voir le moindre conflit entre cela et sa propre affirmation que l'âme est immortelle. En fait, il m'a fait toute une leçon pour m'expliquer à quel point il était horrible de désirer l'immortalité avant de déclarer qu'il avait une âme immortelle. Je ne peux pas tout à fait visualiser ce qui se passait dans son esprit, mais je ne pense pas qu'il ait \emph{vraiment} un modèle mental de lui-même continuant pour toujours dans l'au-delà."

La température dans la pièce semblait être en train de chuter.

"Vous percevez," dit une voix pareille à la glace depuis l'extrémité opposée de la table, "que Dumbledore ne croit pas vraiment aux paroles qu'il prononce. Ce n'est pas qu'il a compromis ses principes. C'est qu'il ne les a jamais eu en premier lieu. Commencez-vous déjà à être cynique, M. Potter~?"

Les yeux de Harry s'étaient abaissés vers sa tasse. "Un peu," dit Harry à son thé Chinois, d'une peut-être-hyper-grande-qualité, et au prix peut-être-exorbitant. "Je deviens certainement un peu \emph{frustré} par… ce qui se passe dans la tête des gens, quoi que ce soit.

--- Oui," dit la voix de glace. "Je trouve cela frustrant moi aussi.

--- Existe-t-il une façon d'obtenir des gens qu'ils ne fassent \emph{pas} cela~?" dit Harry à sa tasse à thé.

"Il existe en effet un certain sort utile qui résout ce problème."

Harry leva les yeux avec espoir et vit un sourire froid, très froid, sur le visage du professeur Quirrell.

Puis Harry comprit. "Je veux dire, \emph{à part} Avada Kedavra."

Le professeur de Défense rit. Pas Harry.

"Quoi qu'il en soit," dit Harry avec hâte, "\emph{j'ai} pensé assez vite pour ne pas suggérer une idée évidente en face de Dumbledore. Avez-vous jamais vu une pierre avec une ligne au centre d'un cercle au centre d'un triangle~?"

Le frisson mortel sembla reculer, se replier sur lui-même, et le professeur Quirrell habituel revint. "Pas que je m'en souvienne," dit le professeur Quirrell au bout d'un moment, un froncement de sourcils pensif sur le visage. "C'est la Pierre de Résurrection~?"

Harry mit sa tasse de côté puis dessina sur sa soucoupe le symbole qu'il avait vu à l'intérieur de la Cape. Et avant qu'il ne puisse sortir sa baguette pour jeter le sort de lévitation, la soucoupe s'envola en diagonale de la table vers le professeur Quirrell. Harry voulait vraiment apprendre tous ces trucs sans baguette, mais c'était apparemment loin au-dessus du programme de son année.

Le professeur Quirrell étudia la soucoupe de Harry pendant un moment, puis il secoua la tête~; et un moment plus tard, la soucoupe revint vers Harry en lévitant.

Harry remit sa tasse sur la soucoupe, et ce faisant, il remarqua distraitement que le symbole qu'il avait dessiné avait disparu. "Si vous voyez une pierre avec ce symbole," dit Harry, "et qu'elle \emph{communique} avec l'au-delà, faites-le moi savoir. J'ai quelques questions à poser à Merlin ou à quiconque était là du temps d'Atlantis.

--- Absolument," dit le professeur Quirrell. Il leva alors sa tasse de nouveau et la renversa, comme pour finir le peu qui restait. "Au fait, M. Potter, j'ai peur que nous ne devions abréger notre visite d'aujourd'hui au Chemin de Traverse. J'avais espéré qu'il soit - mais ça ne fait rien. Sachez seulement qu'il y a autre chose que je dois faire cet après-midi."

Harry hocha la tête et finit son thé, puis il se leva de sa chaise en même temps que le professeur Quirrell.

"Une dernière question," dit Harry, alors que le manteau du professeur Quirrell se soulevait au-dessus du porte-manteau et lévitait dans sa direction. "La magie existe, et je n'accorde plus à mes conjectures la confiance que je leur ai un jour accordées. Alors, selon votre meilleur jugement et sans prendre vos désirs pour des réalités, \emph{croyez-vous} à un au-delà~?

--- Si j'y croyais, M. Potter," dit le professeur Quirrell en endossant son manteau, "serais-je encore \emph{ici}~?"

%  LocalWords:  arry
