\partchapter[\protect\footnotemark]{Rôles}{II}
\authorsnotetext{Ce chapitre ne contient \emph{pas} de spoiler pour un roman spécifique d'Orson Scott Card. C'est une métaphore.}

\lettrinepara{P}{eu} de temps après, il y eut un autre coup contre la porte du débarras.

\hplettrineextrapara
«Si vous vous souciez vraiment de ma santé mentale, dit le garçon sans lever les yeux, vous partirez, vous me laisserez tranquille, et vous attendrez que je descende dîner. Vous ne m'aidez pas.»

La porte s'ouvrit et celui qui avait attendu dehors entra.

«Franchement~?» dit le garçon d'un ton catégorique.

La porte se rabattit puis se ferma derrière Severus Rogue.

Le maître des potions de Poudlard n'affichait rien de l'arrogance dont il était coutumier ni même de l'apparence d'impassibilité qu'il revêtait parfois dans le bureau du directeur~; son regard, étrange, les yeux baissés sur le garçon qui gardait la porte~; ses pensées, insondables.

«Je ne puis imaginer ce que pense la directrice adjointe, dit le maître des potions. À moins que je ne sois censé servir d'avertissement quant à ce qui vous attend si vous décidez d'endosser la responsabilité de sa mort.»

Les lèvres du garçon se serrèrent.

«Très bien. Vous avez gagné, professeur Rogue. Je concède que vous étiez plus responsable de la mort de Lily Potter que je ne le suis de celle de Hermione Granger, et que ma culpabilité n'est pas comparable à la vôtre. Et maintenant je vous demande de partir, et de leur dire qu'il serait probablement préférable de me laisser seul un moment. En avons-nous fini~?

--- Presque, dit le maître des potions. C'est moi qui mettais les mots sous l'oreiller de Mlle Granger et qui lui disais où trouver les combats lors desquels elle intervenait.»

Le garçon ne réagit pas du tout à cela. Puis il parla enfin.

«Parce que vous n'aimez pas que les gens se fassent brutaliser.

--- Pas seulement ça.» Il y avait une note de douleur dans la voix du maître des potions qui lui était étrangère~; il était difficile d'imaginer que c'était la même voix acide qui ordonnait aux enfants de ne pas remuer une fois de plus sans quoi leurs poignets seraient arrachés. «J'aurais dû m'en rendre compte… beaucoup plus tôt, j'imagine, et pourtant je ne l'ai pas remarqué, entièrement absorbé par moi-même que j'étais. Me placer à la tête de Serpentard… cela signifiait que Albus Dumbledore avait perdu tout espoir d'aider la maison Serpentard. Je suis certain que Dumbledore essaya, je ne puis l'imaginer ne pas avoir essayé, lorsqu'il prit la tête de Poudlard pour la première fois. Il dut ressentir une profonde douleur quand, après cela, tant de Serpentard répondirent à l'appel du Seigneur des Ténèbres… il ne m'aurait pas mis en position d'autorité dans cette maison, étant donné mon comportement, s'il n'avait pas perdu tout espoir.» Sous sa cape tachée et salie, les épaules du maître des potions s'affaissèrent. «Mais vous et Mlle Granger essayiez de faire quelque chose, et vous étiez même parvenus à faire changer M. Malfoy et Mlle Greengrass, et peut-être que ces deux-là auraient pu montrer un autre exemple… j'imagine qu'il fut idiot de ma part d'y croire. Le directeur ne sait pas ce que j'ai fait, et je vous demande de ne pas lui dire.

--- Pourquoi me dites-vous cela~?

--- Les choses sont devenues bien trop sérieuses pour que je n'en fasse pas part à quelqu'un.» Les lèvres de Severus Rogue se tordirent. «J'ai assez vu de plans tourner au désastre lors de ma carrière à Serpentard pour savoir comment cela fonctionne. Si, à l'avenir, tout devait être révélé -- alors au moins je vous l'aurais dit, et vous pourrez en attester.

--- Délicieux, dit le garçon. Merci d'avoir éclairci ce mystère. Est-ce tout~?

--- Comptez-vous déclarer que votre vie n'est plus qu'une ruine et qu'il ne vous reste rien d'autre que la vengeance~?

--- Non. J'ai encore… le garçon s'interrompit.

--- Alors j'ai bien peu de conseils à vous prodiguer», dit Severus Rogue.

Le garçon hocha la tête d'un air distrait. «De la part de Hermione, merci de l'avoir aidée avec les brutes. Elle vous dirait que c'était la bonne chose à faire. Et maintenant, je vous serais fort gré si vous pouviez leur dire de \emph{me laisser tranquille}.»

Le maître des potions se tourna vers la porte, et quand son visage cessa d'être visible, sa voix parvint à Harry comme un chuchotement. «Je suis véritablement navré de votre perte.»

Severus Rogue partit.

Le garçon le regarda faire, essayant de se souvenir, du mieux qu'il puisse à une telle distance, de mots prononcés quelque temps auparavant.

\emph{Vos livres vous ont trahi, Potter. Ils ne vous ont pas dit la seule chose que vous deviez savoir. Vous ne pouvez apprendre ce que c'est que de perdre la personne que vous aimez en lisant un livre. C'est une chose que vous ne pourrez connaître avant de l'avoir vécue vous-même.}

Ça avait été quelque chose d'approchant, songea le garçon, s'il s'en souvenait correctement.

\later

Des heures s'étaient maintenant écoulées dans la section de l'infirmerie dont la porte était fermée et derrière laquelle un corps allongé se trouvait.

Harry continua de regarder sa baguette, posée sur ses genoux. De regarder les petites écorchures et taches sur les vingt-huit centimètres de houx, défauts qu'il n'avait auparavant jamais observés d'assez près pour les remarquer. Un rapide calcul mental lui indiqua qu'il n'y avait aucune raison de s'inquiéter puisque, si cela représentait six ou sept mois de dommages accumulés, alors son utilisation le temps d'une vie moyenne ne l'userait pas entièrement. Sur le coup, il se serait probablement inquiété de voir son Retourneur de Temps lui être enlevé s'il avait ouvertement crié~: “Quelqu'un a-t-il un Retourneur de Temps~?” dans la grande salle, mais il aurait été assez simple de s'engager à l'avance à attendre la fin du déjeuner puis à trouver quelqu'un pour envoyer un message au professeur Flitwick deux heures dans le passé, et le professeur Flitwick aurait alors pu aller directement vers Hermione ou lui envoyer son corbeau Patronus longtemps avant que le troll ne s'approche d'elle. Ou ce Harry alternatif aurait-il déjà appris qu'il était trop tard -- aurait-il entendu l'annonce de la mort de Hermione après le déjeuner et avant de pouvoir envoyer des messages dans le passé~? Peut-être qu'un conseil d'utilisation général des voyages dans le temps était de s'assurer de ne jamais prendre le risque d'apprendre qu'on était arrivé trop tard si on n'était pas encore retourné en arrière. Il y avait maintenant une petite brûlure chimique au bout de sa baguette, probablement à cause du contact avec l'acide en lequel il avait partiellement métamorphosé le cerveau du troll, mais la baguette semblait être relativement à l'épreuve de pertes de petites quantités de bois. Vraiment, l'idée qu'une “baguette magique” soit requise devenait de plus en plus étrange à mesure qu'on y réfléchissait. Quoique si les sortilèges étaient toujours inventés de quelque mystérieuse façon, que de nouveaux rituels étaient conçus comme autant de nouveaux leviers sur la machine inconnue, peut-être que les gens continuaient juste d'inventer des rituels à base de mouvements de baguettes, tout comme ils inventaient des phrases comme “Wingardium Leviosa”. Il semblait vraiment que la magie devait, en un sens, être presque arbitrairement puissante, et il aurait certainement été pratique pour Harry de pouvoir juste ignorer la restriction conceptuelle qui empêchait les gens d'inventer des sortilèges comme “Que tout aille bien pour toujours” mais, étrangement, rien n'était si facile dans le monde de la magie. Harry regarda de nouveau sa montre, mais l'heure n'était toujours pas venue.

Il avait tenté de lancer le Patronus dans le but de lui dire d'aller voir Hermione Granger. Juste au cas où c'était un mensonge, un sortilège de Faux Souvenir ou l'un des qui-savait-combien de moyens de faire fermer les yeux d'un sorcier et de le faire rêver. Juste au cas où la vraie Hermione Granger serait en vie et détenue quelque part, malgré le fait qu'il avait senti la vie de cette dernière la quitter. Juste au cas où il y avait un au-delà et que le Vrai Patronus pouvait l'atteindre.

Mais le sortilège n'avait pas fonctionné, donc ce test en particulier n'avait apporté aucun élément de preuve et l'avait laissé avec l'a priori défavorable.

Du temps passa, puis encore plus de temps passa. De l'extérieur, vous n'auriez vu qu'un garçon, assis, regardant sa baguette avec un air contemplatif, jetant un coup d'œil à sa montre environ toutes les deux minutes.

La porte de cette partie de l'infirmerie s'ouvrit \emph{à nouveau}.

Le garçon assis là leva les yeux avec un regard mortel, glaçant.

Puis son visage s'emplit de désarroi, et il se releva tant bien que mal.

«Harry», dit l'homme en chemise boutonnée jusqu'au col, une veste noire jetée par-dessus. Sa voix était rauque. «Harry, qu'est-ce qui se passe~? Le directeur de ton école -- il est arrivé dans sa robe ridicule à mon bureau et il m'a dit que Hermione Granger était morte~!»

Un instant plus tard, une femme suivit l'homme dans la pièce~; elle semblait moins troublée que l'homme, moins déconcertée et plus effrayée.

«Papa, dit le garçon d'une voix faible. Maman. Oui, elle est morte. Ils ne vous ont rien dit d'autre~?

--- Non~! Harry, qu'est-ce qui se passe~?»

Il y eut un silence.

Le garçon appuya son dos contre le mur.

«Je ne p-peux, je ne peux, je ne peux pas faire ça.

--- Quoi~?

--- Je ne peux pas faire semblant d'être un petit garçon, j'ai j-juste pas l'énergie pour ça maintenant.

--- Harry, dit la femme d'un ton hésitant. Harry…

--- Papa, tu vois tous ces livres de fantasy où le héros doit tout cacher de ses parents parce qu'ils, qu'ils ne comprendraient pas, qu'ils réagiraient de façon stupide et se mettraient en travers de son chemin de héros~? C'est une technique narrative, bien sûr, pour que le héros doive tout résoudre lui-même au lieu d'en parler à ses parents. S-s'il vous plaît, ne sois pas cette technique narrative, Papa, et toi non plus, Maman. Juste… ne jouez juste pas ce rôle. Ne soyez pas les parents qui refusent de comprendre. N-ne me criez pas dessus, ne m'imposez pas des exigences parentales que je ne suivrai pas. Parce que je me suis égaré dans un roman de fantasy vraiment débile et maintenant Hermione est… j-je n'ai juste pas assez d'énergie pour gérer ça.»

Lentement, comme si ses membres n'étaient qu'à moitié en mouvement, l'homme dans sa veste noire s'agenouilla là où Harry se tenait afin que ses yeux soient au niveau de ceux de son fils. «Harry, dit l'homme. J'ai besoin que tu me dises tout ce qui s'est passé, tout de suite.»

Le garçon prit une profonde inspiration et déglutit. «Ils m-me disent que le Seigneur des Ténèbres que j'ai vaincu pourrait encore être en vie. Comme si ce n'était pas la t-trame de cent satanés romans, hein~? Donc il se pourrait aussi que le directeur de mon école, qui est le sorcier le plus puissant du monde, est devenu fou. Et, et Hermione a été victime d'une machination et accusée à tort de tentative de meurtre juste avant ça, non pas que quelqu'un en aurait parlé à ses parents. L'élève qu'on l'a accusée à tort de vouloir tuer est le fils de Lucius Malfoy, qui est le politicien le plus puissant d'Angleterre magique, et qui était le second du Seigneur des Ténèbres. Le poste de professeur de Défense de cette école est maudit, personne ne tient plus d'une année, ils ont un proverbe qui dit que le professeur de Défense est toujours suspect. Cette année le professeur de Défense est secrètement un mystérieux sorcier qui s'est opposé au Seigneur des Ténèbres pendant la dernière guerre et dont on ignore s'il est lui-même méchant. Aussi, le maître des potions se languit de Lily Potter depuis des années et pourrait être derrière tout ça pour quelque raison psychologique tordue.» Les lèvres du garçon se serrèrent avec amertume. «Je crois que c'est presque toute l'intrigue débile.»

L'homme, qui avait écouté tout ceci sans rien dire, se leva. Il posa doucement sa main sur l'épaule du garçon. «C'est assez, Harry, dit-il. J'en ai assez entendu. Nous quittons cette école tout de suite et nous t'emmenons avec nous.»

La femme regardait le garçon, et son visage posait une question.

Le garçon la regarda en réponse et hocha la tête.

La voix de la femme fut fluette lorsqu'elle parla.

«\emph{Ils} ne nous laisseront pas faire, Michael.

--- Légalement, ils n'ont aucun droit de nous empêcher…

--- \emph{Droit~?} Vous êtes des \emph{Moldus}», dit le garçon. Il eut un sourire tordu. «Vous avez autant de valeur que des souris aux yeux du système judiciaire magique d'Angleterre. Aucun sorcier ne se souciera de vos arguments sur vos \emph{droits}, sur la \emph{justice}, ils ne prendront même pas le temps de vous écouter. Vous voyez, comme vous n'avez aucun \emph{pouvoir}, ils n'ont pas besoin de se fatiguer. Non, Maman, je ne souris pas comme ça parce que je suis d'accord avec leur politique vis-à-vis des Moldus, je souris parce que je ne suis pas d'accord la vôtre vis-à-vis des enfants.

--- Alors, dit fermement le professeur Michael Verres-Evans, nous verrons ce que le \emph{vrai} gouvernement a à dire à ce sujet. Je connais un parlementaire ou trois…

--- Ils diront~: “Vous êtes fous, bon séjour à l'asile”. En supposant que les Oublietteurs du ministère ne vous attrapent pas avant et n'effacent pas vos souvenirs. J'ai entendu dire qu'ils font beaucoup ça aux Moldus. J'imagine que les vraiment haut placés de notre gouvernement ont formé de confortables arrangements de leur côté. Peut-être obtiennent-ils quelques sortilèges de soin de temps à autre, si quelqu'un d'important arrive à avoir le cancer.» Le garçon eut à nouveau ce sourire tordu. «Et voilà la situation, Papa, comme Maman le sait déjà. Ils ne vous auraient jamais amené ici, ils ne vous auraient jamais rien dit, s'il y avait quoi que ce soit que vous pouviez faire.»

La bouche de l'homme s'ouvrit mais aucun mot ne sortit, comme s'il avait lu le script qui décrivait ce qu'un parent inquiet se devait de faire dans ce genre de situation mais que ce script était soudain arrivé à une page vierge.

«Harry», dit la femme d'une voix hésitante.

Le garçon la regarda.

«Harry, est-ce que quelque chose t'est arrivé~? Tu sembles… différent…

--- Pétunia~!» dit l'homme, dont la langue semblait s'être remise à fonctionner. «Ne dis pas des choses pareilles~! Il est sous pression, c'est tout.

--- Eh bien, Maman, tu vois…» la voix du garçon se brisa. «Tu es sûre que tu veux entendre tout ça d'un coup, maman~?»

La femme hocha la tête mais elle ne parla pas.

«Je dois… tu sais, ce psychiatre scolaire qui pensait que j'avais des problèmes de colère~? Eh bien…» le garçon s'interrompit et déglutit. «Je ne sais pas comment t'expliquer ça, Maman. En fait, c'est quelque chose de magique. Probablement quelque chose à voir avec ce qui s'est passé la nuit où mes parents sont morts. J'ai… eh bien, j'appelais ça un mystérieux côté obscur et je sais qu'on dirait que c'est une blague et \emph{j'ai} vérifié auprès… auprès d'un ancien chapeau magique et télépathe pour m'assurer que ma cicatrice n'était pas \emph{vraiment} habitée par l'esprit du Seigneur des Ténèbres et il a dit qu'il n'y avait qu'une seule personne sous lui et je ne pense pas que les sorciers ont vraiment des âmes de toute façon puisqu'ils peuvent quand même subir des dommages cérébraux, sauf que…

--- Harry, ralentis~! dit l'homme.

--- … sauf que, quoi que ce soit, c'est quand même \emph{réel}, il y a quelque chose à l'intérieur de moi, ça renforçait ma volonté quand les choses tournaient mal, je pouvais faire face à n'importe quoi tant que j'étais en colère, Rogue, Dumbledore, tout le Magenmagot, mon côté obscur n'avait peur de rien sauf des Détraqueurs. Et je n'étais pas stupide, je savais qu'il y aurait peut-être un prix à payer pour l'utiliser et je continuais de chercher ce que ce prix pourrait être. Ça n'a pas changé ma magie, ça n'a pas semblé modifié mon alignement moral de façon permanente, ça n'a pas tenté de m'éloigner de mes amis ou quelque chose comme ça, alors j'ai continué de l'utiliser à chaque fois que nécessaire et je n'ai compris que trop tard ce qu'était vraiment le prix…» la voix du garçon était presque devenue un murmure. «Je ne l'ai compris qu'aujourd'hui… à chaque fois que j'y fais appel… ça consume mon enfance. J'ai tué la chose qui a eu Hermione. Et ce n'est pas mon côté obscur qui a fait ça, c'est moi. Oh, Maman, Papa, je suis désolé.»

Il y eut un long silence formé du son de masques qui se brisaient.

«Harry, dit l'homme en s'agenouillant à nouveau, j'ai besoin que tu recommences du début et que tu expliques ça beaucoup plus lentement.»

Le garçon parla.

Les parents écoutèrent.

Quelque temps plus tard, le père se leva.

Le garçon leva les yeux vers lui, grimaçant dans l'amère expectative.

«Harry, dit l'homme, Pétunia et moi allons te sortir d'ici aussi vite que possible…

--- Non, dit le garçon avec un ton d'avertissement. Je suis sérieux, Papa. Le ministère de la magie n'est pas une chose contre laquelle on se dresse. Dis-toi que c'est les impôts ou le recteur ou quelque chose d'autre qui n'admettra aucune remise en question de sa domination. En Angleterre Magique on a le droit de se souvenir uniquement de ce que le gouvernement pense qu'on devrait se souvenir, et se souvenir de l'existence de la magie, ou du fait que vous avez un fils nommé Harry, c'est un privilège, pas un droit. Et s'ils faisaient ça, je craquerais, je ferais du ministère un immense cratère fumant. Maman, tu connais la chanson, tu dois absolument empêcher Papa d'essayer quoi que ce soit de stupide.

--- Et fils…» l'homme se frotta les tempes. «Peut-être que je ne devrais pas dire ça maintenant… mais es-tu certain que ce dont tu parles est vraiment un côté obscur magique et pas quelque chose de normal pour un garçon de ton âge~?

--- Normal, dit le garçon avec une patience raffinée. Normal comment, exactement~? Je pourrais revérifier, mais je suis raisonnablement certain qu'il n'y a rien à ce sujet dans \emph{Enfance~: un guide pour les parents}. Mon côté obscur n'est pas seulement un état émotionnel, il me \emph{rend plus intelligent}. D'une certaine façon, en tout cas. On ne peut pas juste s'\emph{imaginer} plus intelligent.»

L'homme se frotta de nouveau la tête. «Eh bien… il y a un certain phénomène bien connu lors duquel les enfants traversent un processus biologique qui peut parfois les rendre colériques, sombres et sinistres, et ce processus peut aussi augmenter de façon significative leur intelligence et leur taille…»

Le garçon s'appuya de nouveau contre le mur.

«Non, Papa, ce n'est pas que je deviens un adolescent. J'ai vérifié auprès de mon cerveau et il pense toujours que les filles sont dégueux. Mais si c'est que tu veux croire, très bien. Peut-être que ça vaut mieux pour moi que tu ne me croies pas. Mais je…» la voix du garçon s'étrangla. «Mais je ne supporterais pas de mentir à ce sujet.

--- L'adolescence ne fonctionne pas nécessairement comme ça, Harry. Ça pourrait te prendre un moment avant de remarquer les filles. Si, de fait, tu n'en as pas déjà remarqué u…» et l'homme se tut brusquement.

«Je n'aimais pas Hermione comme ça, murmura le garçon. Pourquoi est-ce que tout le monde continue de penser que ça devait être le cas~? C'est lui manquer de respect que de penser qu'on ne pourrait que l'aimer pour ça.»

L'homme déglutit de façon visible.

«Très bien, fils, tu restes à l'abri du danger pendant qu'on trouve comment te sortir d'ici, c'est compris~? Ne va pas vraiment penser que tu es passé du côté obscur. Je sais que tu as eu tes, ah, ce que j'appelais tes moments Ender Wiggin…

--- Je pense qu'on est maintenant \emph{bien} au-delà d'Ender et plutôt à Ender après que les Formiques ont tué Valentine.

--- Gros mots~!» dit la femme, et sa main alla couvrir sa propre bouche.

Le garçon parla d'une voix usée.

«Pas des Fornique, Maman. Ce sont des aliens insectoïdes -- laisses tomber.

--- Harry, c'est exactement ce que je dis que tu ne devrais pas penser, dit le professeur Verres-Evans d'un ton ferme. Tu ne vas pas aller t'imaginer que tu deviens méchant. Tu ne vas faire de mal à personne, tu ne vas pas te mettre en danger, tu ne vas pas jouer avec quelque magie noire que ce soit pendant que ta mère et moi cherchons à te sortir de cette situation. Est-ce clair, mon fils~?»

Le garçon ferma les yeux.

«Ce seraient de merveilleux conseils, Papa, si j'étais dans un comic book.

--- \emph{Harry…} commença l'homme.

--- La police n'en est pas capable. Les soldats n'en sont pas capables. Le sorcier le plus puissant du monde n'en a pas été capable, et il a essayé. Ce n'est pas juste pour les innocents de jouer à être Batman si on ne peut pas effectivement protéger tout le monde en suivant son code. Et je viens de prouver que je n'en étais pas capable.»

Des gouttes de sueur luisaient sur le front du professeur Michael Verres-Evans. «Maintenant écoute-moi bien. Peu importe ce que tu as lu dans tes livres, tu n'es pas \emph{censé} protéger qui que ce soit~! Ou te mêler à quoi que ce soit de dangereux~! À absolument quoi que ce soit de dangereux~! Restes juste à l'abri de \emph{tout}, de chaque brin de la folie qui habite cette maison de dingues, et on te sortira d'ici dès l'instant où on pourra le faire~!»

Le garçon jeta un regard pénétrant vers son père puis vers sa mère. Puis il regarda de nouveau sa montre.

«Une excellente remarque,» dit le garçon.

Il marcha jusqu'à la porte qui menait vers l'extérieur et l'ouvrit grand.

\later

La porte s'ouvrit grand dans un craquement qui fit bondir Minerva, et avant qu'elle ait le temps de réfléchir, Harry Potter sortait de la pièce à grands pas en la regardant d'un air furieux.

«Vous avez amené mes parents \emph{ici}, dit le Survivant. À \emph{Poudlard}. Où Vous-Savez-Qui ou \emph{quelqu'un} rôde et prend mes amis pour cible. À quoi pensiez-vous, exactement~?»

Elle ne répondit pas qu'elle avait pensé à un Harry assis devant la porte du débarras qui contenait le corps de Hermione et qui refusait de bouger.

«Qui d'autre est au courant~? demanda Harry d'un ton impérieux. Quelqu'un d'autre les a-t-il vus avec vous~?

--- Le directeur les a amenés ici…

--- Je veux qu'ils sortent d'ici \emph{immédiatement} avant que quiconque ne le remarque, en particulier Vous-Savez-Qui, mais cela inclut aussi le professeur Quirrell et le professeur Rogue. Veuillez envoyer votre Patronus au directeur, et dites-lui qu'il doit les ramener tout de suite. Ne mentionnez pas le nom de mes parents, ni même leur existence, au cas où quelqu'un d'autre écouterait.

--- Tout à fait», dit le professeur Verres-Evans en hochant la tête avec sévérité, debout juste derrière le garçon, Pétunia un pas derrière lui. Sa main était fermement posée sur l'épaule de Harry. «Nous finirons de parler à notre fils chez nous.

--- Un instant, s'il vous plaît», dit Minerva avec une politesse instinctive. Sa première tentative de lancer le Patronus avait échoué, l'un des désavantages du sort lors de certaines circonstances. Ce n'était pas la première fois qu'elle l'avait fait, mais elle semblait avoir perdu une partie de l'entrain…

Minerva tut cette pensée et se concentra.

Une fois le message envoyé, elle se retourna vers le professeur Verres-Evans. «Monsieur, dit-elle, j'ai peur que M. Potter ne puisse quitter l'école de Poudlard…»

Lorsque Albus Dumbledore arriva enfin, on pouvait entendre des cris~: l'homme Moldu avait abandonné toute dignité. Au moins il y avait des cris d'un seul côté de la dispute. Minerva n'avait pas le cœur de s'y mettre. En fait, elle n'arrivait pas à croire aux mots qui sortaient de sa bouche.

Lorsque le Professeur se tourna pour se disputer avec le directeur, Harry Potter, qui était demeuré silencieux pendant tout ce temps, parla. «Pas ici, dit-il. Tu peux te disputer avec lui n'importe où mais pas à Poudlard, Papa. Maman, s'il te plaît, s'il te plaît, assure-toi que Papa n'essaiera rien qui lui amènera des ennuis de la part du ministère.»

Le visage de Michael Verres-Evans fit une grimace. Il se retourna, regarda Harry Potter. Sa voix fut rauque lorsqu'il parla et il y avait de l'eau dans ses yeux.

«Mon fils… qu'est-ce que tu fais~?

--- Tu sais parfaitement ce que je fais, dit Harry Potter. Tu as lu ces histoires de super-héros longtemps avant de me les donner. J'ai eu pas mal d'emmerdes, j'ai un peu grandi, et maintenant je protège mes proches. En fait, c'est plus simple que ça, tu sais ce que je fais parce que tu as essayé de faire la même chose. Je fais sortir ceux que j'aime de Poudlard le plus vite possible, voilà ce que je fais. Monsieur le directeur, s'il vous plaît, sortez-les d'ici avant que Vous-Savez-Qui ne découvre leur présence et ne décide de les tuer.»

Michael Verres-Evans commença un plongeon désespéré vers Harry, puis tout mouvement s'arrêta, l'homme moldu penché dans sa course.

«Je suis désolé, dit doucement le directeur. Nous reparlerons bientôt. Minerva, j'étais avec les autres quand tu m'as appelée, ils attendent dans ton bureau.»

Le directeur avança comme en glissant jusqu'à se tenir entre l'homme et de la femme figés~; et il y eut un autre éclair de flammes.

Le mouvement reprit.

Minerva regarda Harry.

Aucun mot ne lui vint.

«Très malin, de les amener ici, dit Harry Potter. Ça a probablement nui à notre relation de façon permanente. Bon sang, tout ce que je voulais c'était qu'on me laisse tranquille jusqu'à ce satané dîner. Qui,» le garçon regarda sa montre, «a de \emph{toute façon} commencé. Je vais aller dire au revoir à Hermione seul, je promets que cela prendra moins de deux minutes, et après ça je sortirai et j'irai manger quelque chose, comme je l'aurai fait de toute façon. Ne me dérangez \emph{pas} pendant ces deux satanées minutes ou je vais craquer et tuer quelqu'un, je suis sérieux, professeur.»

Le garçon se retourna, traversa la petite pièce, ouvrit la porte de derrière, où le corps de Hermione Granger était conservé, et entra avant qu'elle ne trouve quoi répondre. À travers l'ouverture elle vit le fragment d'une image qu'elle savait qu'aucun enfant n'aurait jamais dû voir…

La porte se referma.

Elle commença à avancer, sans réfléchir.

À mi-chemin vers la porte, elle s'arrêta.

Son esprit était encore ralenti et souffrait encore, et la partie d'elle que Harry Potter aurait appelée \emph{l'image d'une sévère adepte de la discipline} prononçait des mots sans vie au sujet du comportement approprié des enfants. Le reste de sa personne ne pensait pas que c'était une bonne idée de laisser n'importe quel enfant, même Harry Potter, seul dans une pièce avec le corps ensanglanté de sa meilleure amie. Mais l'acte d'ouvrir la porte, ou d'affirmer quelque autorité que ce soit, cela ne lui semblait pas sage. Il n'y avait pas de bonne chose à faire, de bonne chose à dire~; ou s'il existait une bonne voie à suivre, elle ne la connaissait pas.

Très lentement, une minute et demie s'écoulèrent.

\later

Lorsque la porte se rouvrit, Harry semblait avoir changé, comme si cette minute et demie avait duré plusieurs vies.

«Scellez la pièce, dit doucement Harry, et allons-y professeur McGonagall.»

Elle marcha jusqu'à la porte du débarras. Elle ne fut pas tout à fait capable de s'empêcher de regarder à l'intérieur et elle vit le sang séché, le drap qui recouvrait la partie inférieure du corps, la partie supérieure comme cireuse, semblable à une poupée, et une brève image des yeux fermés de Hermione Granger. Quelque chose en elle recommença à pleurer.

Elle ferma la porte.

Ses doigts bougèrent sur sa baguette, sa bouche prononça les paroles sans réfléchir, des charmes et des barrières pour sceller la pièce contre toute infraction.

«Professeur McGonagall», dit Harry d'une voix étrange, comme s'il récitait, «avez-vous le rocher~? Le rocher que le directeur m'a donné~? Je devrais le transformer à nouveau en un joyau, puisqu'il s'est avéré utile.»

Automatiquement ses yeux passèrent à l'anneau sur le petit doigt gauche de Harry et remarqua le vide sur la monture, là où le joyau aurait dû se trouver.

«J'en parlerai au directeur, répondit sa langue.

--- Est-ce une tactique courante, au fait~? dit Harry d'une voix toujours étrange. Transporter un grand objet métamorphosé en un objet petit pour l'utiliser comme une arme~? Ou est-ce un exercice de métamorphose habituel~?»

Elle secoua la tête, comme absente.

«Eh bien, allons-y, alors.

--- J'ai…» sa voix se bloqua. «J'ai peur d'avoir autre chose à faire pour l'instant. Pourrez-vous y aller tout seul, et promettrez-vous d'aller directement à la grande salle et de manger quelque chose, M. Potter~?»

Le garçon promit (sauf circonstances exceptionnelles et imprévues, une clause qu'on ne débattit pas) puis sortit de la pièce.

Ce qui l'attendait… ne serait pas plus simple, certainement, et serait peut-être même plus difficile.

\later

Minerva marcha vers son bureau d'un pas rapide -- et non pas lent -- car cela aurait été discourtois.

Elle ouvrit la porte de son bureau.

«Madame Granger, dit sa voix, M. Granger, je suis terriblement désolée de…»
