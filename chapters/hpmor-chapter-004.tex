\chapter[L'hypothèse du Marché Efficient]{L'hypothèse du Marché Efficient\protect\footnotemark}
\authorsnotetext{Comme certains l'ont fait remarquer, les livres sont en contradiction les uns avec les autres pour ce qui est du pouvoir d'achat d'un Gallion~; je vais choisir une valeur fixe et m'y tenir. Cinq livres anglaises pour un Gallion ne cadre pas avec sept Gallions pour une baguette, ni avec des enfants utilisant des baguettes de seconde main.}

\lettrinepara{D}{es} tas de Gallions d'or. Des piles de Mornilles d'argent. Des monceaux de Noises de bronze.

\hplettrineextrapara
Harry se tenait là, la bouche ouverte, et regardait la chambre forte familiale. Il avait tant de questions qu'il ne savait pas par \emph{où} commencer.

Depuis l'ouverture de la porte de la chambre forte, McGonagall le regardait avec l'air de s'appuyer négligemment contre le mur, mais ses yeux étaient attentifs. Bon, rien de surprenant. Se voir présenter un énorme tas de pièces d'or constituait un test de personnalité tellement pur que c'en était un stéréotype.

«~Ces pièces sont-elles faites de métal pur~? dit finalement Harry.

--- Quoi~? siffla le gobelin Gripsec qui attendait près de la porte. Remettez-vous en question l'intégrité de Gringotts, M. Potter-Evans-Verres~?

--- Non monsieur, dit Harry d'un air absent, pas du tout, navré si je me suis mal exprimé. C'est juste que je ne sais pas du tout comment votre système financier fonctionne. Je vous demande si les Gallions en général sont faits d'or pur.

--- Bien sûr, dit Gripsec.

--- Et n'importe peut les frapper, ou sont-ils produits par un monopole qui collecte ainsi un seigneuriage~?

--- Quoi~? dit McGonagall, les yeux vides.~»

Gripsec grimaça, révélant des dents acérées.

«~Seul un idiot ferait confiance à autre chose qu'à une pièce gobeline~!

--- En d'autres mots, dit Harry, les pièces ne sont pas supposées valoir plus que le métal dont elles sont faites~?~»

Gripsec regardait Harry. McGonagall semblait perplexe.

«~Je veux dire que, imaginez que j'arrive ici avec une tonne d'argent. Pourrais-je repartir avec une tonne de Mornilles~?

--- Contre des frais, M. Potter-Evans-Verres. Le gobelin l'observait avec des yeux scintillants. Contre des frais certains. Où trouveriez-vous une tonne d'argent, cela je me le demande.

--- Je parlais hypothétiquement, dit Harry. \emph{Pour l'instant, en tout cas.} Donc… combien feriez-vous payer, en fraction du poids total~?~»

Les yeux de Gripsec étaient fixés sur Harry.

«~Je devrais consulter mes supérieurs…

--- Donnez-moi une estimation. Je ne demanderai pas à Gringotts de s'y tenir.

--- Un vingtième du métal paierait pour la frappe des pièces.~»

Harry hocha la tête. «~Merci beaucoup, M. Gripsec.~»

\emph{Alors non seulement l'économie magique est totalement découplée de l'économie Moldue, mais personne ici n'a jamais entendu parler d'arbitrage.} L'économie Moldue, plus grande, avait un taux d'échange fluctuant entre l'or et l'argent, si bien que chaque fois que le taux or-pour-argent des Moldus se trouvait à plus de 5~\% de différence avec le rapport de poids dix-sept-Mornilles-pour-un-Gallion, l'or ou l'argent aurait dû être drainé hors de l'économie magique jusqu'à ce qu'il devienne impossible de maintenir un taux d'échange. Amenez une tonne d'argent, échangez-la contre des Mornilles (et payez 5~\%), échangez les Mornilles contre des Gallions, amenez l'or dans le monde Moldu, échangez-le contre plus d'argent que ce que vous aviez au départ, et recommencez.

Le taux d'échange or-argent des Moldus n'était-il pas aux environs de cinquante pour un~? En tout cas, pensait Harry, ce n'était certainement pas dix-sept pour un. Et il semblait que les pièces d'argent étaient en fait \emph{plus petites} que les pièces d'or.

Mais après tout, Harry se tenait dans une banque qui stockait littéralement votre argent dans des chambres fortes pleines de pièces d'or, gardées par des dragons, et où vous deviez pénétrer et récupérer vos pièces à chaque fois que vous souhaitiez dépenser de l'argent. Des détails tels que réduire l'inefficacité des marchés grâce à l'arbitrage leur passerait probablement au-dessus de la tête. Il avait été tenté de faire une remarque narquoise au sujet de la grossièreté de leur système financier…

\emph{Mais ce qui était triste, c'était que leur méthode était probablement meilleure.}

D'un autre côté, un professionnel de la gestion de portefeuille financier deviendrait probablement propriétaire du monde magique en moins d'une semaine. Harry rangea cette idée quelque part, au cas où il manquerait d'argent, ou se retrouverait avec une semaine de libre.

Pendant ce temps, les gigantesques piles de pièces d'or de la chambre forte Potter devraient répondre à ses besoins à court terme.

Harry s'avança, et commença à ramasser des pièces d'or d'une main, les déposant ensuite dans l'autre.

Lorsqu'il en fut arrivé à vingt, McGonagall toussa.

«~ Je pense que ce sera bien plus qu'assez pour payer vos fournitures scolaires, M. Potter.

--- Hmm~? dit Harry, l'esprit ailleurs. Ne bougez pas, je fais un calcul de Fermi.

--- Un \emph{quoi}~? dit McGonagall, l'air soudain alarmée.

--- C'est un truc de math. Nommé d'après Enrico Fermi. Une façon d'obtenir des résultats approximatifs de tête très rapidement…~»

Vingt Gallions d'or pesaient peut-être un dixième de kilo. Et l'or valait, quoi, dix mille livres anglaises au kilo~? Un Gallion valait donc environ cinquante livres anglaises… Les piles/tas de pièces d'or semblaient faire environ soixante pièces de haut et vingt pièces dans les autres dimensions de la base, et avait une forme pyramidale, donc ce serait environ un tiers du cube. Huit mille Gallions par tas, en gros, et il y avait cinq tas de cette tailles, soit quarante mille Gallions soit 2 millions de livres anglaises.

Pas mal. Harry sourit et fit une légère grimace de satisfaction. Il était dommage qu'il soit en plein milieu de sa découverte d'un incroyable nouveau monde magique, et qu'il ne puisse pas prendre le temps d'explorer l'incroyable nouveau monde de la richesse, qu'un rapide calcul de Fermi avait estimé être environ un milliard de fois moins intéressant.

\emph{Quand même, c'est la dernière fois que je tonds une pelouse pour une pauvre livre.}

Harry s'éloigna de l'immense tas d'or. «~Navré de poser la question, Professeur McGonagall, mais je crois comprendre que mes parents avaient entre vingt et trente ans lorsqu'ils sont morts. Est-ce une somme \emph{normale} d'argent à avoir dans sa chambre forte lorsqu'on est un jeune couple du monde magique~?~» Si c'était le cas, alors une tasse de café coûtait probablement cinq mille livres. Règle numéro un de l'économie~: vous ne pouvez pas manger l'argent.

McGonagall secoua la tête. «~Votre père était le dernier héritier d'une vieille famille, M. Potter. Il est aussi possible…~»

McGonagall hésita.

«~Une partie de cette argent pourrait provenir des primes qui avaient été mises sur la tête de Vous-Savez-Qui, payable à sa mo-~» McGonagall ravala le mot. «~À qui le vaincrait. Ou ces primes n'ont peut-être pas encore été récoltées. Je ne suis pas sûre.

--- Intéressant… dit lentement Harry. Donc une partie de ceci est mien en un sens. C'est à dire, gagné par moi. En quelque sorte. Peut-être. Même si je ne m'en souviens pas.~» Les doigts de Harry tapotaient contre les jambes de son pantalon. «~Cela me fait me sentir moins coupable à l'idée d'en dépenser \emph{une toute petite fraction~! Ne paniquez pas, Professeur McGonagall~!}

--- M. Potter~! Vous êtes mineur, et en tant que tel, vous ne serez autorisés qu'à faire des retraits \emph{raisonnables} de-

--- Je suis \emph{carrément} pour ce qui est raisonnable~! Je suis à fond pour la prudence fiscale et le contrôle de ses impulsions~! Mais j'ai \emph{bel et bien} vu quelques choses sur le chemin qui constitueraient des achats \emph{sensés et adultes…}~»

Harry accrocha ses yeux à ceux de McGonagall, s'engageant dans un duel de regard silencieux.

«~Comme quoi~? dit finalement McGonagall.

--- Des malles dont l'intérieur contient plus que l'extérieur~?~»

Le visage de McGonagall devint sévère.

«~Ces malles sont \emph{très} chères, M. Potter~!

--- Oui mais - plaida Harry. Je suis sûr que j'en voudrais une quand je serai adulte. Et je \emph{peux} m'en offrir une. Il serait plus sensé d'en acheter une maintenant que plus tard, et d'en avoir l'usage immédiatement, n'est-ce-pas~? C'est le même argent dans un cas comme dans l'autre. Je veux dire que j'en \emph{voudrais} une de bonne qualité, avec \emph{beaucoup} de place à l'intérieur, assez bonne pour que je n'ai pas à en racheter une meilleure plus tard…~» Harry laissa sa phrase en suspens, plein d'espoir.

Le regard de McGonagall ne vacilla pas.

«~Et quoi, au juste, \emph{conserveriez-vous} dans une telle malle, M. Potter -

--- Des livres.

--- Bien sûr, soupira McGonagall.

--- Vous auriez dû me dire \emph{bien plus tôt} que ce genre d'objet magique existaient~! Et que je pouvais me les offrir~! Maintenant mon père et moi allons devoir passer les deux prochains jours à parcourir \emph{frénétiquement} toutes les librairies de vieux manuels scolaires afin que je puisse avoir une bibliothèque de mathématiques et de science décente avec moi à Poudlard - et peut-être une mini-collection de SF\&F, si je peux constituer quelque chose de convenable à partir des corbeilles à 10 pennies. Ou encore mieux, je vous rend le marché un peu plus attrayant, d'accord~? Laissez-moi juste acheter -

--- \emph{M. Potter!} Vous pensez pouvoir me \emph{soudoyer}~?

--- Quoi~? \emph{Non}~! Pas comme ça~! Ce que je veux dire c'est que Poudlard pourra garder certains des livres que j'apporterai si vous pensez qu'ils constitueraient des ajouts de valeur à la bibliothèque. Je vais les acheter pour pas cher, et \emph{je} veux juste qu'ils me soient accessibles. C'est acceptable de soudoyer les gens avec des livres, non~? C'est une -

--- Tradition familiale.

--- Oui, exactement.~»

Tout le corps de McGonagall semblait s'effondrer.

«~J'ai bien peur de ne pas pouvoir contredire la logique de vos mots, bien que je souhaite ardemment en être capable. Je vais vous autoriser à retirer 100 Gallions de plus, M. Potter. Je \emph{sais} que je vais le regretter, et je le fais quand même.

--- J'aime cette façon de penser~! Et est-ce que la 'Peau de Moke' fait ce que je pense qu'elle fait~?

--- Pas autant qu'une malle, dit McGonagall avec réticence, mais une peau de Moke avec un sort de Récupération et un sort d'Extension Indétectable peut contenir un certain nombre d'objets qui pourront être rappelés par celui qui les y a rangés.

--- Oui, j'aurai absolument besoin d'une de ces peaux. C'est comme la super ceinture-gadget de génialitude ultime~! La ceinture utilitaire de Batman~! Oubliez le couteau suisse, on pourrait transporter une caisse à outils entière là-dedans~! Ou d'autres objets magiques~! Ou des \emph{livres}~! Je pourrais avoir mes trois livres du moment sur moi, tout le temps, et en piocher un n'importe où~! Je n'aurais plus jamais à gâcher une minute de ma vie~! Qu'en pensez-vous, Professeur McGonagall~? C'est la meilleure des raisons possibles.

--- Très bien. Vous pouvez ajouter dix Gallions.~»

Gripsec offrait à Harry un regard de franc respect, voir même peut-être d'admiration.

«~Et un peu d'argent de poche, comme vous l'avez mentionné plus tôt. Je crois pouvoir me rappeler d'une ou deux autres choses que je souhaite conserver dans cette peau.

--- \emph{Ne poussez pas, M. Potter.}

--- Mais, oh, Professeur McGonagall, pourquoi gâcher ce moment~? Aujourd'hui est certainement un \emph{jour heureux}, celui où je découvre toutes les choses magiques pour la première fois~! Pourquoi jouer le rôle de l'adulte grincheux alors qu'au lieu de ça vous pourriez sourire et vous remémorer votre enfance innocente, et regarder le ravissement sur mon jeune visage tandis que j'achète quelque jouets, utilisant une fraction insignifiante de la fortune que j'ai gagné en battant le plus terrible sorcier que la Grande-Bretagne aie jamais connu, non pas que je vous accuse d'être ingrate ou quoi que ce soit, mais quand même, que sont quelques jouets comparés à ça~?

--- \emph{Vous},~» grogna McGonagall. Elle avait un regard si effroyable et terrible que Harry glapit et fit un pas en arrière, reversant une pile de pièces d'or dans un grand tintement et s'étalant dans un tas d'argent. Gripsec soupira et se cacha le visage derrière sa main. «~Je rendrais un grand service à l'Angleterre magique, M. Potter, et peut-être au monde entier, si je vous enfermais dans cette chambre forte et que je vous laissais ici.~»

Et ils partirent sans autre ennui.

%  LocalWords:  eaps fundie ki
