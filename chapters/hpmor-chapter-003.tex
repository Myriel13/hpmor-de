\chapter{Comparer la réalité à ses alternatives}

\lettrine[ante=«]{M}{on} Dieu, dit le barman en dévisageant Harry, est-ce… se pourrait-il que…?»

Harry se pencha par-dessus le bar du Chaudron Baveur du mieux qu'il put, le zinc atteignant à peu près le haut de ses sourcils. Une question \emph{pareille} méritait qu'il donne le meilleur de lui-même.

«Suis-je -- pourrais-je être -- peut-être -- qui sait -- si c'\emph{est} le cas -- mais une question reste en suspend -- \emph{qui} \emph{?}

--- Bénie soit mon âme, murmura le vieux barman, Harry Potter… quel honneur.»

Harry cligna des yeux puis se reprit.

«Eh bien, oui, vous êtes perspicace~; la plupart des gens ne s'en rendent pas compte si vite…

--- Ça suffit,» dit le Professeur McGonagall. Sa main se resserra sur l'épaule de Harry. «Ne harcèle pas le garçon, Tom, tout ça est nouveau pour lui.

--- Mais c'est lui~? chevrota une vieille femme. C'est Harry Potter~?» Elle se leva de sa chaise dans un bruit de raclement.

«Doris…» dit McGonagall sur un ton d'avertissement. Le long regard avec lequel elle balaya la pièce aurait dû suffire à intimider n'importe qui.

«Je veux seulement lui serrer la main» murmura la vieille femme. Elle s'inclina profondément et brandit une main ridée. Harry, se sentant plus confus et gêné qu'il ne l'avait jamais été auparavant, la serra avec précaution. Des larmes tombèrent des yeux de la vieille femme jusqu'à leur poignée de main.

«Mon petit-fils était un Auror, lui murmura-t-elle. Mort en soixante-dix-neuf. Merci, Harry Potter. Loué soit le ciel.

--- De rien,» dit Harry, passé en pilote automatique, puis il se tourna vers McGonagall et lui jeta un regard à la fois terrifié et implorant.

McGonagall claqua son pied sur le sol juste avant que la ruée ne commence. Cela fit un bruit qui donna à Harry une nouvelle définition de l'expression «coup de tonnerre», et tous se figèrent.

«Nous sommes pressés,» dit McGonagall d'une voix parfaitement, absolument normale.

Ils quittèrent le bar sans ennui.

«McGonagall~?» dit Harry, une fois qu'ils furent dans la cour du bâtiment. Il avait voulu s'enquérir de ce qui se passait, mais s'entendit poser une toute autre question. «Qui était l'homme pâle~? L'homme au bar à l'œil convulsé~?»

«Hmm~?» dit McGonagall, l'air un peu surprise~; peut-être qu'elle non plus ne s'était pas attendue à cette question. «C'était le Professeur Quirrell. Il va enseigner le cours de Défense contre les forces du Mal à Poudlard cette année.»

«J'ai eu une sensation des plus étranges, comme si je le connaissais…» Harry se frotta le front. «Et comme si je devais éviter de lui serrer la main.» Comme de rencontrer quelqu'un qui avait autrefois été votre ami, puis que quelque chose avait très mal tourné… ce n'était pas vraiment ça, mais Harry n'arrivait pas à trouver les mots justes. «Et pour le reste~?»

McGonagall lui jeta un étrange regard. «M. Potter… savez-vous… que vous a-t-on dit \emph{au juste}… sur la façon dont vos parents sont morts~?»

Harry lui renvoya un regard ferme.

«Mes parents sont vivants et bien portants, et ils ont toujours refusé de me parler de la façon dont mes parents \emph{génétiques} sont morts. Ce dont je ne déduis rien de bon.

--- Une loyauté admirable,» dit McGonagall. Sa voix se fit plus basse~: «Mais je souffre un peu de vous l'entendre dire ainsi. Lily et James étaient des amis.»

Soudain honteux, Harry détourna le regard.

«Je suis navré, dit-il d'une petite voix.

--- Mais j'\emph{ai} un père et une mère. Et je sais que je ne ferais que me rendre malheureux si je comparais la réalité à… quelque chose de parfait que mon imagination a construite.

--- C'est étonnamment sage de votre part, dit McGonagall avec douceur. Mais vos parents \emph{génétiques} sont morts d'une belle mort~; en vous protégeant.»

\emph{En me protégeant~?}

Quelque chose d'inconnu étreint le cœur de Harry. «Que… s'\emph{est-il} passé~?»

McGonagall soupira. Sa baguette vint toucher le front de Harry, et la vision de ce dernier se brouilla un instant. «Une sorte de déguisement, dit McGonagall, pour que ceci n'ait plus lieu, pas avant que vous soyez prêt.» Puis sa baguette fut à nouveau dehors, et par trois fois toucha un mur de brique…

… où se creusa un trou qui se dilata, s'étira et trembla pour devenir une immense arcade révélant une longue rangée de magasins dotés de pancartes criant les mérites de chaudrons et de foies de dragons.

Harry ne cligna même pas des yeux. Ce n'était pas comme si quelqu'un venait de se transformer en chat.

Et ils s'avancèrent tous deux dans le monde magique.

Il y avait des marchands vantant des Bottes Rebondissantes («Faites avec du vrai Flubber~!») et «Couteaux +3~! Fourchettes +2~! Cuillères avec un bonus de +4~!» Il y avait des lunettes capables de rendre vert tout ce que vous regardiez, et une sélection de confortables fauteuils de salon dotés de sièges éjectables pour les urgences.

La tête de Harry tournait, tournait comme si elle essayait de se dévisser de son cou. C'était comme de déambuler dans la section objets magiques d'un livre de règles d'Advanced Dungeons \& Dragons (il ne jouait pas au jeu, mais il aimait lire les livres de règles). Harry souhaitait désespérément ne pas manquer un seul des objets disponibles, au cas où ce serait l'un des trois requis pour compléter un cycle de sorts de \emph{vœux} infinis.

Puis Harry remarqua quelque chose qui le fit inconsciemment dériver loin de McGonagall et il se dirigea droit vers un magasin à la devanture faite de briques bleues aux rebords bronze-acier. Il fallut que McGonagall se campe juste devant lui pour que Harry revienne à la réalité.

«M. Potter~?» dit-elle.

Harry cligna des yeux, puis se rendit compte de ce qu'il venait de faire. «Je suis navré~! J'ai oublié pendant un moment que j'étais avec vous et non avec ma famille.» Harry esquissa un geste en direction de la vitrine du magasin, qui affichait des lettres ardentes qui brillaient d'une lueur à la fois perçante et lointaine, et l'on pouvait y lire~: \emph{Bigbam's Brillant Books.}

«Lorsqu'on passe devant une librairie qu'on n'a pas encore visitée, on doit rentrer et jeter un coup d'œil. C'est la règle de la famille.

--- C'est la chose la plus Serdaigle que j'ai jamais entendue.

--- Quoi~?

--- Rien. M. Potter, notre première étape sera une visite à Gringotts, la banque du monde magique. La chambre forte de votre famille \emph{génétique} s'y trouve, ainsi que l'héritage que vos parents \emph{génétiques} vous ont laissé, et vous allez avoir besoin d'argent pour vos fournitures scolaires.» Elle soupira. «Et je suppose qu'une certaine quantité d'argent de poche destinée à l'achat de livres pourra être excusée. Cela dit vous pourriez décider d'attendre un moment. Poudlard a une bibliothèque assez conséquente consacrée à la magie. Et la tour dans laquelle je soupçonne fortement que vous allez vivre est équipée de sa propre bibliothèque plus généraliste. Tout livre que vous achèterez ici sera probablement un doublon.»

Harry hocha la tête, et ils continuèrent.

«Ne vous méprenez pas, c'est une~\emph{excellente} diversion,» dit Harry, alors que sa tête continuait de pivoter en tous sens, «probablement la meilleure diversion qu'on ait jamais essayée sur moi, mais ne croyez pas que j'ai oublié notre discussion laissée en suspens.»

McGonagall soupira.

«Vos parents -- votre mère tout du moins -- a peut-être été fort sage de ne rien vous dire.

--- Et vous souhaitez que je continue dans cette ignorance béate~? Votre plan possède une faille évidente, Professeur McGonagall.

--- J'imagine que ce serait assez futile, dit la sorcière avec fermeté, vu que n'importe quel passant pourrait vous raconter cette histoire. Très bien.»

Et elle lui parla de Celui-Dont-Il-Ne-Faut-Pas-Prononcer-Le-Nom, le Seigneur des Ténèbres, Voldemort.

«Voldemort~?» murmura Harry. Ça aurait dû être drôle, mais ça ne l'était pas. Le nom brûlait avec froideur, impitoyable, d'une clarté de diamant, tel un marteau de titane pur s'abattant sur une enclume de chair sans défense. Un frisson parcouru Harry alors même qu'il prononçait le mot, et il décida ici et maintenant d'utiliser des termes plus sûrs, comme Vous-Savez-Qui.

Le Seigneur des Ténèbres avait mis l'Angleterre magique à feu et à sang, tel un loup enragé, déchirant, déchiquetant le tissu de leur vie. D'autres pays s'étaient tordus les mains, hésitant à intervenir à cause de leur égoïsme apathique, ou par peur, car le premier d'entre eux à s'opposer au Seigneur des Ténèbres verrait sa paix devenir la cible de sa terreur.

(\emph{L'effet du témoin}, se dit Harry, songeant à l'expérience de Latane et Darley qui avait montré que vous aviez plus de chances d'être aidé si vous faisiez une crise d'épilepsie en présence d'une personne qu'en présence de trois. \emph{Diffusion de la responsabilité, chacun espérant que quelqu'un d'autre agisse en premier.})

Les Mangemorts avaient suivi le sillage du Seigneur des Ténèbres, et dans son avant-garde se trouvaient des vautours charognards qui rouvraient les blessures, ou des serpents pour mordre et affaiblir. Les Mangemorts n'étaient pas aussi épouvantables que le Seigneur des Ténèbres, mais ils étaient épouvantables~; et ils étaient nombreux. Et les Mangemorts maniaient plus que des baguettes~; et il y avait dans ces troupes masquées des fortunes, du pouvoir politique, et des secrets transformés en chantages afin de paralyser une société qui essayait de se protéger.

Un journaliste âgé et respecté, Yermy Wibble, avait réclamé une hausse des taxes et une conscription forcée. Il s'était écrié qu'il était absurde que la majorité se tapisse, effrayée par la minorité. Sa peau, seule sa peau, avait été retrouvée clouée au mur de la rédaction le matin suivant, à côté des peaux de sa femme et de ses deux filles. Chacun souhaitait que quelque chose soit fait, et personne n'osait prendre l'initiative. Le prochain à se démarquer deviendrait le prochain exemple.

Jusqu'au jour où les noms de Lily et James Potter atteignirent le haut de la liste.

Et ces deux-là auraient pu mourir la baguette à la main et n'avoir aucun regret, car c'\emph{étaient} des héros~; mais ils avaient un nouveau né, leur fils, Harry Potter.

Les larmes montaient aux yeux de Harry. Il les essuya avec colère, ou peut-être avec désespoir, \emph{Je ne connaissais pas ces gens, pas vraiment, ce ne sont pas mes parents} aujourd'hui\emph{, ce serait futile d'être triste pour eux}…

Lorsque Harry eut fini de sangloter dans la robe de McGonagall, il releva la tête, et voir que des larmes se trouvaient aussi dans ses yeux à elle le fit se sentir un peu mieux.

«Le Seigneur des Ténèbres est venu à Godric's Hollow,» dit McGonagall dans un souffle. «Vous auriez dû être à l'abri, mais on vous a trahi. Le Seigneur des Ténèbres a tué James, et il a tué Lily, et enfin il est parvenu jusqu'à vous, jusqu'à votre berceau. Il vous a jeté le sortilège de la Mort. Et tout était fini. Le sortilège de la Mort est fait de haine pure, et frappe directement l'âme en la séparant du corps. Il ne peut être arrêté. La seule défense est de ne pas être là. Mais vous avez survécu. Vous êtes la seule personne à avoir jamais survécu. Le sortilège de la Mort a été réfléchi, il a rebondi et a frappé le Seigneur des Ténèbres, ne laissant que la carcasse brûlée de son corps et une cicatrice sur votre front. C'était la fin de la terreur, et nous étions libres. Voilà, Harry Potter, pourquoi les gens veulent voir la cicatrice sur votre front, et pourquoi ils veulent vous serrer la main.»

Le torrent de pleurs qui s'était déversé en Harry avait usé toutes ses larmes. Il ne pouvait plus pleurer à nouveau~; il avait fini.

(Et quelque part, enfoui sous ses pensées, se trouvait un léger, très léger sentiment de confusion, l'idée que quelque chose dans cette histoire ne collait pas~; et cela faisait partie de son art que de remarquer ce sentiment, mais il était distrait. Car il est une triste règle disant que c'est lorsqu'on a le plus besoin de son art de rationaliste qu'on risque le plus de l'oublier.)

Harry se détacha du flanc de McGonagall. «Je vais -- avoir besoin d'y réfléchir,» dit-il, essayant de maintenir un contrôle sur sa voix. Il fixa ses chaussures. «Euh, vous pouvez les appeler mes parents, si vous le souhaitez, vous n'avez pas à dire “parents génétiques” ou quoi que ce soit. Je ne vois pas pourquoi je ne pourrais pas avoir deux mères et deux pères.»

McGonagall fut silencieuse.

Et ils marchèrent ensemble en silence, jusqu'à ce qu'ils parviennent à un grand bâtiment blanc aux vastes portes de bronze.

«\emph{Gringotts}» dit McGonagall.
%  LocalWords:  ood Bigbam’s
