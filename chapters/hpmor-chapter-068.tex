\partchapter{Accomplissement de soi}{III}

\lettrine{H}{ermione} ne se sentait pour l'instant ni très bien ni très Gentille, car une sphère de colère dont elle se demandait si cela ressemblait à l'obscurité de Harry (même si c'en était probablement très éloigné) brûlait à l'intérieur d'elle et elle n'aurait probablement pas dû ressentir cela à cause d'un petit jeu idiot mais -

Toute son armée. Deux soldats avaient battu toute son armée. C'était ce qu'on lui avait dit après son réveil.

C'en était (un peu) trop.

"Eh bien," dit le professeur Quirrell. De près, le professeur de Défense n'avait pas l'air en aussi bonne santé que la dernière fois qu'elle avait été dans son bureau~; sa peau était plus pâle et il se déplaçait un peu plus lentement. Son expression était aussi sévère que d'habitude, son regard aussi pénétrant~; ses doigts frappaient sèchement son bureau, tap-tap. "J'imagine que parmi vous trois, seul M. Malfoy a deviné pourquoi j'ai demandé votre présence."

"Quelque chose en rapport avec les maisons Nobles et Très Anciennes~?" dit Harry, situé à côté d'elle, d'un ton perplexe. "Je n'ai pas brisé une règle démente en tirant sur Daphné, ou si~?"

"Pas tout à fait," dit l'homme d'un ton lourd d'ironie. "Puisque Mlle Greengrass n'a pas respecté les formalités du duel, elle n'a pas prétention à demander à ce que votre nom vous soit retiré. Mais je n'aurais bien sûr pas autorisé un duel formel. La guerre ne respecte pas ce genre de règles." Le professeur de Défense se pencha en avant et reposa son menton sur ses mains arquées, comme si avoir le dos droit l'avait déjà fatigué. Ses yeux, perçants et dangereux, les passèrent tous les trois en revue. "Général Malfoy. Pourquoi vous ais-je appelé ici~?"

"Ce n'est plus équitable de mettre le général Potter seul contre nous deux," dit Drago Malfoy d'une voix basse.

"\emph{Quoi~?"} laissa échapper Hermione. "On les \emph{avait} presque, si Daphné ne s'était pas évanouie -"

"Mademoiselle Greengrass ne s'est pas évanouie d'épuisement magique," dit le professeur Quirrell d'un ton sec. "M. Potter lui a tiré dans le dos avec un sortilège de sommeil pendant que vos soldats étaient distraits par la vue de leur général s'écrasant contre un mur en plein vol. Mais tout de même~: félicitations, mademoiselle Granger, pour avoir \emph{presque} battu deux légionnaires chaotiques avec seulement vingt-quatre soldats Soleil."

Le sang monté à ses joues devint un peu plus chaud. "C'était -- c'était juste -- si j'avais seulement compris qu'il portait une amure -"

Le professeur Quirrell les observa au-dessus de ses doigts joints. "Bien sûr qu'il existe des moyens par lesquels vous auriez \emph{pu} gagner, Mlle Granger. Il y en a toujours, pour chaque bataille perdue. Le monde qui nous entoure regorge d'opportunités, explose d'opportunités, et presque tout le monde les ignore parce que cela demanderait qu'une habitude de pensée soit rompue~; dans chaque bataille, mille os Poufsouffle attendent d'être aiguisés pour devenir des lances. Si vous aviez pensé à tenter un \emph{Finite Incantatem} de masse par mesure de précaution, vous auriez défait les cottes de mailles de M. Potter ainsi que tout ce qu'il portait hormis ses sous-vêtements, ce qui me laisse à penser que M. Potter n'avait pas tout à fait compris à quel point il était vulnérable. Ou vous auriez pu demander à vos soldats de se ruer sur M. Potter et M. Londubat et d'arracher de force leur baguette de leur main. La réponse de M. Malfoy n'a pas été ce que je pourrais qualifier de \emph{réfléchie}, mais au moins il n'a pas complètement ignoré les milles alternatives qui lui étaient offertes." Un sourire sardonique. "Mais vous, Mlle Granger, avez eu le malheur de vous souvenir comment lancer le sortilège d'étourdissement, et vous n'avez donc pas fouillé votre excellente mémoire à la recherche de la douzaine de sortilèges plus simples que se seraient avérés efficaces. Et vous avez placé tous les espoirs de votre armée sur les épaules d'une seule personne, si bien que son moral est descendu en flèche lorsque vous êtes tombée. Ils ont ensuite continué de lancer leurs futiles sortilèges de sommeil, guidés par les habitudes de combat qui leur avaient été inculquées, incapables de briser ces motifs comme M. Malfoy l'a fait. Je ne peux pas tout à fait comprendre ce qui passe par la tête des gens lorsqu'ils répètent la même stratégie perdante encore et encore, mais il est apparemment rare de se rendre compte qu'on peut essayer autre chose. Ainsi le régiment Soleil a-t-il été balayé par deux soldats." Le professeur de Défense eut un sourire sans joie. "On peut percevoir certaines similitudes avec la façon dont cinquante Mangemorts ont dominé toute l'Angleterre magique et avec celle dont notre bien-aimé ministère continue d'exercer le pouvoir."

Le professeur de Défense soupira. "\emph{Néanmoins}, mademoiselle Granger, le fait demeure~: ce n'est pas votre première défaite de ce genre. Lors de la bataille précédente, vous et M. Malfoy avez unis vos forces et vous êtes quand même retrouvé dans une impasse, à tel point qu'il a fallu que M. Malfoy et vous poursuiviez M. Potter jusqu'au toit. La légion du Chaos a maintenant démontré deux fois de suite qu'elle possédait une force militaire égale à celle de vos deux armées réunies. Cela ne me laisse pas d'autre choix~: général Potter, vous choisirez huit soldats de votre armées, dont au moins un lieutenant chaotique, et ils seront répartis entre l'armée Dragon et le régiment Soleil -"

"\emph{Quoi~?"} éclata de nouveau Hermione, elle jeta un coup d'œil aux deux autres généraux et vit que Harry semblait aussi choqué qu'elle, tandis que Drago avait seulement l'air résigné.

"Le général Potter est plus fort que vous deux réunis," dit le professeur Quirrell d'une calme précision. "Votre concours est fini, il a gagné, et il est temps de rééquilibrer les équipes afin de le mettre face à un nouveau défi."

"\emph{Professeur Quirrell~!"} dit Harry. "Je n'ai pas -"

"C'est ma décision de professeur de magie de bataille de l'école de sorcellerie de Poudlard et elle n'est pas sujette à quelque négociation que ce soit." Les mots étaient toujours précis mais bien qu'il ait regardé Harry et pas elle, le regard du professeur de Quirrell glaça le sang de Hermione. "Et je trouve \emph{suspicieux}, M. Potter, qu'au moment où vous avez choisi d'isoler Mlle Granger et M. Malfoy et de les forcer à vous poursuivre sur le toit, vous soyez parvenu à annihiler exactement autant de leurs forces rassemblées qu'il vous a convenu. De fait, c'est le niveau de performance que \emph{j'attendais} de vous depuis le début de cette année et je suis \emph{agacé} de découvrir que vous vous êtes retenus lors de mes cours pendant tout ce temps~! J'ai vu ce dont vous êtes vraiment capable, M. Potter. Vous êtes loin au-delà du stade où M. Malfoy et Mlle Granger peuvent vous affronter sur un pied d'égalité, et vous ne saurez prétendre qu'il en est autrement. Ceci, M. Potter, je vous le dis en tant que professeur~: pour réaliser tout votre potentiel, vous devez pratiquer au maximum de vos capacités et ne vous retenir pour \emph{aucune} raison -- en particulier pas à cause d'inquiétudes puériles quant à ce que vos amis pourraient penser~!"

\later

Elle quitta le professeur de Défense avec une armée plus imposante, une dignité amoindrie, la sensation d'être un triste petit insecte qui venait de se faire écraser et en essayant très très fort de ne pas pleurer.

"Je ne me \emph{retenais pas}~!" dit Harry dès qu'ils eurent passé le premier angle du couloir qui menait hors du bureau du professeur Quirrell, au moment où la porte de bois quitta leur champ de vision derrière les murs de pierre. "Je ne faisais pas semblant, je n'ai jamais \emph{laissé} aucun de vous gagner~!"

Elle n'arrivait pas à répondre, elle en était incapable, tout craquerait si elle essayait de dire un seul mot.

"Vraiment~?" dit Drago Malfoy. Le général Dragon avait encore cet air résigné. "Parce que tu sais, Quirrell a raison, c'est \emph{suspicieux} que tu aies été capable d'abattre presque tout le monde dans nos deux armées dès que tu as voulu qu'on te pourchasse sur le toit. Et est-ce que tu n'as pas dit quelque chose à ce moment-là, Potter, sur le fait qu'on devait être capable de te vaincre quand tu te battais à fond~?"

Le sensation de brûlure remontait dans sa gorge, et elle éclaterait en sanglots lorsque cela atteindrait ses yeux, et alors elle ne serait qu'une petite pleurnicharde à leurs yeux.

"C'était -" dit la voix de Harry avec empressement, elle ne le regardait pas mais sa voix donnait l'impression que sa tête s'était tournée vers elle. "C'était -- j'ai vraiment fait de mon mieux cette fois là, il y avait une raison importante, il \emph{fallait} que je le fasse, alors j'ai utilisé plein de techniques que j'avais gardées en réserve -- et - "

Elle avait toujours fait de son mieux, à chaque fois.

"- et j'ai, j'ai laissé sortir une partie de moi que je n'utilise normalement pas pour des choses comme le cours de Défense -"

Donc si elle s'approchait jamais de la victoire contre Harry lorsque ça serait \emph{vraiment} important, il pourrait juste passer à son côté obscur et l'écraser, c'était ça~?

… bien sûr que c'était ça. Elle ne pouvait même pas \emph{regarder} Harry dans les yeux quand il était effrayant, comment avait-elle pu jamais croire qu'elle pourrait vraiment le battre~?

Le couloir atteint un embranchement, Harry Potter et Drago Malfoy prirent à gauche vers un escalier qui menait au deuxième étage et elle partit plutôt à droite~; elle ne savait pas où menait ce passage mais pour l'instant elle aurait préféré se perdre dans le château.

"Excuse-moi, Drago," dit la voix de Harry, et il y eut alors une série de pas derrière elle.

"Laisse-moi tranquille," dit-elle, le ton était sévère mais elle dut ensuite fermer sa bouche et serrer ses lèvres fort l'une une contre l'autre et retenir sa respiration pour éviter que tout ne sorte.

Ce garçon continua juste d'approcher, la dépassa et se mit face à elle, parce qu'il était stupide, voilà pourquoi, et il dit d'une voix qui était devenue un chuchotement aigu et désespéré~: "Je ne me suis pas enfui quand tu \emph{me} battais dans tous mes cours à part le vol sur balai~!"

Il ne comprenait pas et il ne comprendrait jamais. Harry Potter ne comprendrait jamais parce que peu importe les concours qu'il perdrait, il serait toujours le Survivant. Si vous étiez Harry Potter et que Hermione Granger vous battait, ça voulait dire que tout le monde s'attendait à ce que vous releviez le défi. Si vous étiez Hermione Granger et que Harry Potter vous battait, ça voulait juste dire que vous n'étiez personne.

"Ça n'est pas juste," dit-elle, sa voix tremblait mais elle ne pleurait pas, pas encore, "\emph{je} ne devrais pas avoir à combattre ton côté obscur, je suis juste -- je suis seulement -" et elle pensa~: \emph{j'ai seulement douze ans}.

"J'ai utilisé mon côté obscur \emph{une seule fois} et c'était -- parce qu'il le \emph{fallait}~!"

"Donc aujourd'hui tu as vaincu \emph{toute mon armée} en étant seulement Harry~?" Elle ne pleurait pas encore et elle se demandait à quoi son visage ressemblait à cet instant, si c'était celui d'une Hermione en colère ou celui d'une Hermione triste.

"Je -" dit Harry. Sa voix baissa un peu, "je… je ne \emph{m'attendais} pas vraiment à gagner cette fois, je sais que j'ai dit que j'étais invincible mais j'essayais juste de t'effrayer, je pensais vraiment qu'on vous ralentirait juste pendant un moment -"

Elle recommença à marcher, le dépassa, et alors le visage de Harry se contrit comme s'\emph{il} était sur le point de pleurer.

"Le professeur Quirrell a-t-il raison~?" dit un chuchotement aigu et désespéré venu de derrière elle. "Si tu es mon amie, est-ce que j'aurais toujours peur de faire mieux parce que je saurais que cela te blessera~? Ça n'est pas juste, Hermione~!"

Elle inspira, bloqua sa respiration, et courut. Ses pieds se mouvèrent aussi vite qu'elle en fut capable, elle courut aussi vite que sa vision brouillée le lui permit, elle courut pour que personne ne l'entende, et cette fois Harry ne la suivit pas.

\later

Minerva corrigeait les parchemins de métamorphoses qui avaient été à rendre pour lundi et venait de mettre une note de moins deux-cents à un parchemin de cinquième année qui comportait une erreur qui aurait pu tuer quelqu'un. Elle avait été indignée par la folie des élèves lors de sa première année en tant que professeur. Elle n'était plus que résignée. Certaines personnes n'apprenaient jamais, ne remarquaient jamais qu'ils étaient sans espoir, ils demeuraient enthousiastes et joyeux, ils essayaient encore et encore. Parfois ils la croyaient lorsqu'elle leur disait, avant leur départ de Poudlard, qu'ils ne devraient \emph{jamais} essayer quoi que ce soit d'inhabituel, d'abandonner la métamorphose libre et d'utiliser leur art uniquement au moyen de charmes déjà établis~; et parfois… ils ne l'écoutaient pas.

Elle était au beau milieu d'une réponse particulièrement alambiquée lorsqu'un coup sur la porte perturba ses pensées~; ses heures de bureau n'avaient pas commencé mais il ne lui avait fallu que très peu de temps en poste à la tête de Gryffondor pour apprendre à suspendre son jugement. On pouvait toujours déduire des points \emph{après}.

"Entrez," dit-elle d'une voix brusque.

Il était clair que la jeune fille qui venait d'entrer dans son bureau avait récemment pleuré et avait ensuite lavé son visage dans l'espoir que ce ne serait pas visible -

"Mlle Granger~!" dit le professeur McGonagall. Il lui avait fallu un moment pour reconnaître ce visage aux yeux rougis et aux joues gonflées. "Que s'est-il passé~?"

"Professeur," dit la jeune fille d'une voix vacillante, "vous avez dit que si jamais j'étais inquiète ou que quelque chose me mettait mal à l'aise, je devrais venir vous voir immédiatement -"

"Oui," dit le professeur McGonagall, "alors que s'est-il \emph{passé}~?"

La jeune fille commença à expliquer -

\later

Hermione se tenait immobile et les escaliers tournaient autour d'elle, une double hélice qui n'aurait dû la mener nulle part et la faisait pourtant \emph{s'élever} d'un mouvement continu. Hermione trouvait que cela ressemblait à l'enchantement des Escaliers Infinis qui avaient été inventés en 1733 par le sorcier Arram Sabeti qui avait vécu au sommet du mont Everest à l'époque où aucun Moldu n'était capable de l'escalader. Sauf que c'était impossible, car Poudlard était bien plus ancienne -- peut-être le sortilège avait-il été \emph{ré}inventé~?

Elle aurait dû être effrayée et nerveuse avant sa deuxième rencontre avec le directeur.

Sauf que Hermione Granger avait réfléchi~; elle avait beaucoup réfléchi après que l'épuisement l'eut fait s'arrêter de courir, après qu'elles se furent laissée glisser contre un mur, les poumons en feu, à réfléchir, roulée en boule, le dos contre la pierre fraîche, les jambes repliées, en pleurs.

Même si elle perdait contre Harry Potter elle ne perdrait jamais, \emph{jamais}, contre Drago Malfoy, c'était \emph{absolument} inacceptable, et le professeur Quirrell avait félicité le général Malfoy pour n'avoir pas ignoré les milles alternatives possibles~; et donc, après avoir épuisé toutes ses larmes, Hermione avait pensé à quatorze autre sortilèges qu'elle \emph{aurait dû} essayer contre Harry et Neville, puis elle avait commencé à se demander si elle ne faisait pas le même genre d'erreur dans d'autres domaines~; et c'est comme ça qu'elle s'était retrouvée à frapper à la porte du professeur McGonagall. Pas pour demander de l'aide, car pour l'instant Hermione n'avait aucun plan \emph{pour lequel} elle pourrait demander qu'on l'aide, mais juste pour tout dire au professeur McGonagall parce que lorsque l'idée lui était venue cela avait ressemblé à l'une des milles alternatives dont le professeur Quirrell avait parlé.

Et elle avait dit au professeur McGonagall comment Harry avait changé depuis le jour où le phénix s'était posé sur son épaule, comment les gens semblaient de plus en plus la voir comme une créature de Harry, comment Harry semblait s'éloigner de plus en plus des autres élèves de leur année, se promenait parfois avec un air triste, comme s'il était en train de perdre quelque chose, et qu'\emph{elle ne savait plus quoi faire.}

Et le professeur McGonagall avait dit qu'elles devaient en parler au directeur.

Et cela avait rendu Hermione inquiète, puis l'idée lui était venue que \emph{Harry Potter} n'aurait pas eu peur du directeur. Il aurait juste foncé et continué d'essayer d'atteindre son objectif. Peut-être (avait-elle songé) que ça valait le coup \emph{d'essayer} d'être comme ça, de ne \emph{pas} avoir peur, de juste faire ce qu'il y avait à faire et de voir ce qui lui arriverait, après tout, ça ne pouvait plus vraiment empirer.

Les Escaliers Infinis cessèrent de tourner.

La grande porte en chêne au heurtoir de laiton en forme de griffon qui leur faisait face s'ouvrit sans avoir été touchée.

Derrière un bureau de chêne noir muni de dizaines de tiroirs orientés dans toutes les directions possibles et semblant avoir des tiroirs incrustés \emph{dans} d'autres tiroirs, le directeur à barbe d'argent de Poudlard se tenait sur son trône~; Albus Percival Wulfric Brian Dumbledore, dans les doux et pétillants yeux duquel elle regarda pendant environ trois secondes avant d'être distraite par tous les autres objets présents dans la pièce.

Quelques instants plus tard -- elle n'était pas certaine de la durée mais c'était pendant qu'elle essayait de compter le nombre d'objets dans la pièce pour la troisième fois et qu'elle n'obtenait \emph{toujours pas} la même réponse même si sa mémoire insistait sur le fait que rien n'avait été ajouté ni enlevé -- le directeur s'éclaircit la gorge et dit~: "Mlle Granger~?"

La tête de Hermione se tourna brusquement et elle sentit une légère chaleur entrer dans ses joues~; mais Dumbledore ne sembla pas le moins du monde agacé, seulement serein, avec un regard curieux dans ses yeux doux surmontés de verres.

"Hermione," dit le professeur McGonagall, la voix de la vieille sorcière était douce et sa main reposait sur son épaule d'une façon rassurante, "s'il te plaît, dis au directeur ce que tu m'as dit au sujet de Harry."

Hermione commença à parler, mais en dépit de sa récente résolution sa voix trébucha quand même un peu par nervosité et elle décrivit comment Harry avait changé pendant les quelques semaines depuis que Fumseck avait été sur son épaule.

Il y eut un silence après qu'elle eut fini, puis le directeur soupira. "Je suis navré, mademoiselle Granger," dit Dumbledore. Ses yeux bleus s'étaient attristés à mesure qu'elle avait parlé. "C'est… malheureux, mais je ne peux pas dire que ce soit inattendu. C'est le fardeau du héros que vous observez."

"Du \emph{héros}~?" dit Hermione. Elle regarda le professeur McGonagall avec nervosité et vit que le visage du professeur de Métamorphose s'était pincé, même si sa main toujours rassurante serrait encore son épaule.

"Oui," dit Dumbledore. "J'étais moi-même un héros, avant d'être un vieux sorcier mystérieux, à l'époque où je combattais Grindelwald. Avez-vous lu des livres d'Histoire, Mlle Granger~?"

Hermione hocha la tête.

"Eh bien," dit Dumbledore, "c'est ce que les héros doivent faire, Mlle Granger, ils ont des missions, ils doivent devenir forts afin de les accomplir et c'est cela que vous observez chez Harry. S'il y a quoi que ce soit qui puisse être fait pour adoucir son voyage, ce sera \emph{vous} qui le ferez, pas moi. Car je ne suis, hélas, pas l'ami de Harry, seulement son vieux sorcier mystérieux."

"Je -" dit Hermione. "Je ne suis pas sûre -- de toujours vouloir être -" sa voix s'arrêta, cela semblait trop horrible pour être prononcé à voix haute.

Dumbledore ferma les yeux, et lorsqu'il les rouvrit, il semblait être devenu un peu plus vieux qu'avant. "Personne ne peut vous arrêter, Mlle Granger, si vous choisissez de ne plus être l'amie de Harry. Quant à ce que cela lui ferait, vous le savez probablement mieux que moi."

"Ça -- n'a pas l'air \emph{juste}," dit Hermione d'une voix tremblante. "Que je \emph{doive} être l'amie de Harry parce qu'il n'a personne d'autre~? Ça n'a pas l'air \emph{juste."}

"On ne peut pas vous forcer à \emph{être} une amie, Mlle Granger." Le regard bleu sembla la traverser. "Les sentiments sont là où pas. S'ils sont là, vous pouvez les accepter ou les nier. Vous \emph{êtes} l'amie de Harry -- et choisir de le nier le blesserait terriblement, peut-être au-delà de tout espoir de guérison. Mais, Mlle Granger, qu'est-ce qui pourrait vous mener à de telles extrémités~?"

Elle n'arrivait pas à trouver les mots. Elle n'avait jamais pu le faire. "Si vous vous approchez trop de Harry -- vous vous faites \emph{avaler} et plus personne ne \emph{vous} voit, vous devenez juste une de ses \emph{choses}, tout le monde pense que le monde gravite autour de lui et…" Elle n'arrivait pas à trouver les mots.

Le vieux sorcier hocha lentement la tête. "Nous vivons en effet dans un monde injuste, Mlle Granger. Tout le monde sait que c'est moi qui ai vaincu Grindelwald et peu se souviennent d'Elizabeth Beckett, qui est morte pour ouvrir le chemin qui m'a permis de passer. Et pourtant certains s'en souviennent. Harry Potter \emph{est} le héros de cette pièce, Mlle Granger~; et le monde \emph{gravite} autour de lui. Il est destiné à accomplir de grandes choses~; et je pense qu'on finira par se souvenir du nom d'Albus Dumbledore pour avoir été celui du mystérieux vieux sorcier de Harry plus que pour toutes les autres choses que j'ai accomplies. Et peut-être qu'on se souviendra du nom de Mlle Granger comme du nom de celle qui l'accompagnait, si vous vous avérez être à la hauteur, le temps venu. Car cela est vrai~: vous ne trouverez jamais plus de gloire par vous-même qu'aux côtés de Harry Potter ."

Hermione secoua sa tête rapidement. "Mais ce n'est \emph{pas} -" elle avait su qu'elle ne pourrait pas expliquer. "Ça n'a rien à voir avec la \emph{gloire}, c'est par rapport au fait d'être -- d'appartenir à quelqu'un d'autre~!"

"Vous pensez donc que vous préféreriez être une héroïne~?" Le vieux sorcier soupira. "Mlle Granger, j'ai \emph{été} un héros et un chef~; et j'aurais été mille fois plus heureux si j'avais pu appartenir à quelqu'un comme Harry Potter. Quelqu'un fait d'un matériau plus dur que le mien, capable de prendre les décisions difficiles et qui aurait pourtant mérité d'être mon supérieur. J'ai un jour pensé que je connaissais un tel homme, mais j'avais tort… Mlle Granger, vous n'avez pas la \emph{moindre} idée de la chance que vous et vos semblables avez, comparé aux héros."

La sensation brûlante montait de nouveau dans sa gorge, accompagnée d'un sentiment d'impuissance, elle ne comprenait pas pourquoi le professeur McGonagall l'avait amenée ici si le directeur ne l'aidait pas, et d'un coup d'œil vers cette dernière elle vit que cette dernière n'était probablement pas non plus certaine que cela ait été une bonne idée.

"Je ne veux pas être une héroïne," dit Hermione Granger, "je ne veux pas être le compagnon d'un héros, je veux juste être \emph{moi}."

(Quelques secondes plus tard, elle songea qu'elle \emph{voulait} être une héroïne mais elle décida de ne pas revenir sur ce qu'elle avait dit).

"Ah," répondit le vieux sorcier. "Voilà qui est fort difficile, Mlle Granger." Dumbledore se leva de son trône, s'écarta de son bureau et montra un symbole accroché au mur, un symbole si omniprésent que les yeux de Hermione étaient passés dessus dans le voir~: un bouclier usé qui portait le blason de Poudlard, le lion, le serpent, le blaireau et l'aigle, ainsi que des mots gravés en latin dont elle n'avait jamais compris la pertinence. Puis elle reprit conscience de l'emplacement du bouclier et de son apparence ancienne, et il lui vint soudain à l'esprit que cela pourrait bien être \emph{l'original} -

"Un Poufsouffle dirait," continua Dumbledore, tapotant du doigt contre le blaireau poli et faisant grimacer Hermione par ce sacrilège (si \emph{c'était} bien l'original), "que les gens échouent à devenir ceux qu'ils sont destinés à être parce qu'ils sont trop paresseux pour fournir l'effort nécessaire. Un Serdaigle," tapotant l'aigle, "répéterait ces mots que les sages savent être bien plus vieux que Socrate~: \emph{connais-toi toi-même}, et dirait que les gens échouent à devenir ceux qu'ils sont destinés à être par ignorance et manque de réflexion. Et Salazar Serpentard," Dumbledore fronça les sourcils lorsque son doigt tapota le serpent poli, "eh bien, il dirait que nous devenons ceux que nous sommes destinés à être en suivant nos désirs partout où ils nous mènent. Peut-être dirait-il que les gens échouent à devenir eux-mêmes parce qu'ils refusent de faire le nécessaire pour accomplir leurs ambitions. Mais on remarque alors que presque tous les Seigneurs des Ténèbres sortis de Poudlard étaient Serpentard. Sont-ils devenus ceux qu'ils devaient être~? Je ne pense pas." Le doigt de Gryffondor tapota le lion, puis il se tourna vers elle. "Dites-moi, Mlle Granger, que dirait un Gryffondor~? Je n'ai pas besoin de vous demander si le Choixpeau vous a proposé cette maison."

La question ne semblait pas être difficile. "Un Gryffondor dirait que les gens ne deviennent pas ceux qu'ils sont censés être parce qu'ils ont peur."

"La plupart des gens \emph{ont} peur, Mlle Granger," dit le vieux sorcier. "Ils vivent leur vie entière restreints par la peur paralysante qui fait échouer tout ce qu'ils pourraient accomplir, tout ce qu'ils pourraient devenir. La peur de dire ou de faire ce qu'il ne faut pas, la peur de perdre de simples possessions, la peur de la mort, et par-dessus tout la peur de ce que les autres vont penser d'eux. Cette peur est une chose terrible, Mlle Granger, et il est terriblement important de le savoir. Mais ce n'est pas ce que Godric Gryffondor aurait dit. Les gens deviennent ceux qu'ils sont censés être, Mlle Granger, en faisant ce qui est juste." La voix du vieux sorcier était douce. "Donc dites-moi, Mlle Granger, quel est selon vous le choix \emph{juste}~? Car c'est \emph{ça} que vous êtes vraiment, et où que cette voie vous mène, c'est en la suivant que vous deviendrez celle que vous êtes censée être."

Il y eut un long silence emplit par le bruit des choses qui ne pouvaient être comptées.

Elle y réfléchit, car elle était Serdaigle.

"Je ne \emph{pense} pas qu'il soit juste," dit lentement Hermione, "que quelqu'un doive vivre ainsi dans l'ombre de quelqu'un d'autre…"

"De nombreuses choses sont injuste en ce monde," dit le vieux sorcier, "la question est de savoir ce qu'il est juste que \emph{vous} fassiez à leur sujet. Hermione Granger, je serais moins subtil que les vieux sorciers le sont habituellement et je vous annonce catégoriquement que vous ne pouvez pas \emph{imaginer} à quel point les choses pourraient mal tourner si les événements qui gravitent autour de Harry Potter prenaient un mauvais détour. Sa quête a trait à un problème que vous ne \emph{rêveriez} même pas d'ignorer si vous le connaissiez."

"\emph{Quelle} quête~?" dit Hermione. Sa voix tremblait car la réponse que le directeur attendait était très claire et elle refusait de la lui donner. "Qu'est-ce qui est \emph{arrivé} à Harry Potter ce jour là, \emph{pourquoi} Fumseck était-il sur son épaule~?"

"Il a grandi," dit le vieux sorcier. Il cligna plusieurs fois des yeux sous les lunettes en demi-lunes et son visage eut soudain l'air très ridé. "Car voyez-vous, Mlle Granger, les gens ne grandissent pas avec le temps, ils grandissent en étant mis face à des situations d'adultes. C'est ce qui est arrivé à Harry Potter ce samedi-là. On lui a dit -- vous ne partagerez cette information avec personne, comprenez-vous~? - on lui a dit qu'il devrait combattre quelqu'un. Je ne puis vous révéler qui. Je ne puis vous dire pourquoi. Mais c'est ce qui lui est arrivé, et c'est pour cela qu'il a besoin de ses amis."

Il y eut un silence.

"\emph{Bellatrix Black}~?" dit Hermione. Le choc n'aurait pas été plus grand si quelqu'un avait branché un câble électrique dans son oreille. "Vous allez le faire combattre \emph{Bellatrix Black}~?"

"Non," dit le vieux sorcier. "Pas elle. Je ne puis vous dire qui, ni pourquoi."

Elle y réfléchit un peu plus.

"Existe-t-il un moyen me permettant de \emph{rester à la hauteur} de Harry~?" dit Hermione. "Je ne veux pas dire que c'est ce que je vais \emph{faire} mais -- s'il a besoin d'amis, alors est-ce qu'on peut être des amis \emph{égaux}~? Est-ce que je peux être une héroïne moi aussi~?"

"Ah," dit le vieux sorcier, et il sourit. "Vous seul pouvez en décider, Mlle Granger."

"Mais vous n'allez pas m'aider comme vous avez aidé Harry."

Le vieux sorcier secoua la tête. "Je ne l'ai que peu aidé, Mlle Granger. Et si vous êtes en train de me demander une quête -" le vieux sorcier eut un autre sourire, plutôt ironique. "Mlle Granger, vous êtes en première année à Poudlard. Ne soyez pas trop pressée de grandir~; vous aurez assez de temps pour ça plus tard."

"J'ai douze ans. Harry en a \emph{onze}."

"Harry Potter est spécial," dit le vieux sorcier. "Comme vous le savez, Mlle Granger." Les yeux bleus devinrent soudain perçants derrière les lunettes en croissant de lune, et elle se souvint du jour du Détraqueur, quand la voix de Dumbledore lui avait révélé dans son esprit qu'il était au courant pour le côté obscur de Harry.

Hermione leva la main, toucha celle du professeur McGonagall qui était fermement restée sur son épaule pendant tout ce temps, et dit, surprise que sa voix ne se brise pas~: "Je voudrais partir maintenant, s'il vous plaît."

"Bien sûr," dit le professeur McGonagall, et Hermione sentit la main sur son épaule la tourner gentiment pour faire face à la porte de chêne.

"Avez-vous déjà choisi votre voie, Hermione Granger~?" dit la voix d'Albus Dumbledore derrière elle, alors même que la porte s'ouvrait lentement pour révéler l'enchantement des Escaliers Infinis.

Elle hocha la tête.

"Et~?"

"Je ferai," dit-elle, sa voix bloquée, "je ferai, je ferai -"

Elle déglutit.

"Je ferai -- ce qui est juste -"

Elle ne dit rien d'autre car elle en était incapable, et les Escaliers Infinis tournèrent autour d'elle une fois de plus.

Ni elle ni le professeur McGonagall ne parlèrent pendant la descente.

Lorsque la gargouille de Pierre Fluide se fut écartée de leur chemin et qu'elles eurent toutes deux mis un pied dans les couloirs de Poudlard, le professeur McGonagall parla enfin et dit d'un murmure~: "Je suis terriblement navrée, Mlle Granger. Je ne pensais pas que le directeur vous dirait des choses pareilles. Je pense qu'il a vraiment oublié ce que c'est que d'être un enfant."

Hermione leva les yeux vers elle et vit que c'était \emph{le professeur McGonagall} qui avait l'air d'être sur le point d'éclater en sanglots… pas vraiment, mais son visage était pincé d'une façon qui poussait à le croire.

"Si je veux être une héroïne aussi," dit Hermione, "si j'ai décidé d'en être une moi aussi, y a-t-il quoi que ce soit que \emph{vous} puissiez faire pour m'aider~?"

Le professeur secoua rapidement la tête et dit~: "Mlle Granger, je ne sais pas si le directeur avait tort sur \emph{ce} point. Vous \emph{avez} douze ans."

"D'accord," dit Hermione.

Elle fit quelques pas.

"Excusez-moi," dit Hermione, "ça vous embêterait que j'aille seule jusqu'à la tour Serdaigle~? Je suis désolée, ce n'est pas votre faute, c'est juste que je voudrais être seule pour l'instant."

"Bien sûr, Mlle Granger," dit le professeur McGonagall d'une voix un peu rauque, et Hermione entendit les pas de celle-ci qui s'arrêtaient et faisaient demi-tour.

Hermione Granger s'en fut.

Elle monta une volée de marche puis une autre, se demandant s'il y avait quiconque à Poudlard qui saurait lui donner une chance d'être une héroïne. Le professeur Flitwick dirait la même chose que le professeur McGonagall et même dans le cas contraire il était probablement incapable de l'aider, et Hermione ne savait pas qui \emph{pourrait} l'aider. Enfin, le professeur Quirrell inventerait probablement quelque chose de malin si elle utilisait assez de points Quirrell mais elle avait le sentiment que ce serait une mauvaise idée de lui demander ça -- que le professeur de Défense ne pouvait aider personne à devenir le genre de héros qui valait la peine qu'on le devienne et qu'il ne comprendrait même pas la différence.

Elle avait presque atteint la tour Serdaigle lorsqu'elle aperçut l'éclat doré. 

%  LocalWords:  ermione Arram Sabeti
