\partchapter{Mesures de précaution}{II}

\lettrine{H}{arry} haletait, seul, debout au centre d'une zone plus détruite qu'elle n'aurait dû l'être par la main d'un élève en première année. Le sortilège de découpe n'avait pu abattre d'arbre, alors il avait commencé une métamorphose partielle de sections transversales des troncs. Cela n'avait pas diminué la pression à l'intérieur de lui~: abattre un petit bosquet d'arbres ne lui avait apporté aucun bien-être, toutes les émotions étaient restées, mais pendant qu'il détruisait des arbres, il n'avait pas eu à penser au fait que ses sentiments étaient bloqués à l'intérieur de lui.

Après que Harry ait vidé sa magie, il commença à arracher des branches à mains nues et à les casser en deux. Ses mains saignaient, mais ce n'était là rien que madame Pomfresh ne pourrait pas réparer le matin venu. Seule la magie noire laissait des cicatrices permanentes sur le corps des sorciers.

Puis vint le son d'un mouvement dans les bois, comme les sabots d'un cheval, et Harry pivota, baguette à nouveau levée~; une partie de sa magie était revenue pendant qu'il cassait les branches avec ses mains. L'idée lui vint pour la première fois qu'il était seul, dans la Forêt interdite, et qu'il faisait du bruit.

Ce qui émergea dans la lumière de la lune n'était pas la licorne à laquelle Harry s'était attendu mais une créature au corps plus menu, comme celui d'un cheval, aux reflets bruns et blancs, et dont la poitrine nue était celle d'un humain mâle aux longs cheveux blancs. La lumière de la lune tomba sur le visage du centaure et Harry vit des yeux presque aussi bleus que ceux de Dumbledore, à mi-chemin du saphir.

D'une main, le centaure tenait une longue lance de bois surmontée d'une lame de métal aux dimensions exagérées. Son fil ne réfléchissait pas la lumière de la lune~; et Harry avait lu un jour qu'un fil réfléchissant était le signe d'une lame émoussée.

"Alors," dit le centaure. Sa voix était basse, puissante et masculine. "Te voilà, entouré de destruction. Je peux sentir le sang de licorne flotter dans les airs, le sang d'une chose innocente, tuée en échange de la vie d'un autre."

Un sursaut de peur ramena Harry à l'instant présent et il dit rapidement~: "Ça n'est pas ce que vous croyez."

"Je sais. Ironiquement, les étoiles elles-mêmes proclament ton innocence." Le centaure fit un pas vers Harry dans la petite clairière, lance toujours verticale. "Un étrange mot, \emph{innocence}. Il signifie l'absence de savoir, comme celle d'un enfant, et signifie aussi l'absence de culpabilité. Seuls les vrais ignorants peuvent n'être en rien responsables des conséquences de leurs actes. Il ne sait pas ce qu'il fait, et peut donc n'avoir aucune mauvaise intention~: voilà ce que dit ce mot." La voix grave ne projetait aucun écho à travers les bois.

Les yeux de Harry passèrent sur la pointe de la lance et il se rendit alors compte qu'il aurait dû se saisir de son Retourneur de Temps au moment où il avait vu le centaure. Maintenant, s'il essayait de le prendre sous ses robes, et si le centaure était assez rapide, la lance pourrait le frapper avant qu'il n'y parvienne. "J'ai un jour lu," dit Harry d'une voix légèrement instable, essayant de répondre aux paroles apparemment profondes par d'autres à l'apparence toute aussi profonde, "que c'est une erreur de penser que les petits enfants sont innocents, car ne pas savoir et ne pas choisir sont deux choses différentes. Que si les enfants se causent de petites douleurs lorsqu'ils se battent dans la cour d'école, c'est parce qu'ils n'ont pas le pouvoir d'en causer de plus grandes. Et certains adultes sont la cause de grandes douleurs. Mais ceux qui ne le sont pas, ne sont-ils pas plus innocents que des enfants~?"

"La sagesse des sorciers," dit le centaure.

"À vrai dire, des Moldus."

"Des sans-magie je sais très peu. Mars a été peu visible ces derniers temps, mais elle redevient lumineuse." Le centaure fit un autre pas vers l'avant, presque assez pour pouvoir atteindre Harry de sa lance.

Harry n'osa pas lever les yeux vers le ciel. "Cela signifie que Mars se rapproche de la Terre pendant leur voyage autour du soleil. Mars renvoie la même quantité de lumière que d'habitude, c'est juste qu'elle se rapproche de nous. Qu'est-ce que vous voulez dire par 'Les étoiles proclament mon innocence'~?"

"Le ciel nocturne parle aux centaures. C'est grâce à lui que nous savons ce que nous savons. Ou n'apprend-t-on même pas cela aux sorciers, de nos jours~?" Un air de mépris passa sur le visage du centaure.

"J'ai… essayé de m'informer sur les centaures quand je faisais des recherches sur la divination. La plupart des auteurs tournaient la divination centaure en dérision sans expliquer pourquoi. Les sorciers ne comprennent pas les règles de l'argumentaire, pour eux, tourner une idée ou une personne en dérision leur semble tout aussi capable de rabaisser que d'exhiber des preuves de ce qu'on avance… je pensais que l'idée que les centaures utilisent l'astrologie était une autre technique destinée à ridiculiser les centaures."

"Pourquoi~?" entonna le centaure. Il pencha la tête d'un mouvement étrange.

"Parce que la trajectoire des planètes est prévisible des milliers d'années à l'avance. Si je parlais aux bons Moldus, je pourrais vous montrer un diagramme de l'apparence exacte des planètes de notre point de vue actuel dans dix ans. Pourriez-vous faire des prédictions à partir de ça~?"

Le centaure secoua la tête. "À partir d'un diagramme~? Non. La lumière des planètes, les comètes, les subtils changements dans les étoiles elles-mêmes, je ne verrais pas cela."

"Les orbites des comètes sont aussi fixées des milliers d'années à l'avance, elles ne devraient donc pas être très corrélées aux événements actuels. Et la lumière des étoiles met des années à voyager jusqu'à la Terre, et les étoiles ne bougent pas beaucoup, en tout cas pas de façon visible. Donc l'hypothèse évidente, c'est que les centaures naissent avec un don de divination magique et que vous, eh bien, que vous \emph{projetez} ce talent sur le ciel."

"Peut-être," dit le centaure d'un ton pensif. Il baissa la tête. "Les autres vous pourfendraient pour avoir prononcé ces paroles, mais j'ai toujours cherché à savoir ce que j'ignorais. Pourquoi le ciel nocturne peut prédire l'avenir - cela, je l'ignore très certainement. Il est déjà assez difficile de maîtriser la divination elle-même. Tout ce que je peux dire, fils de Lily, c'est que même si ce que tu dis est vrai, cela ne semble pas utile."

Harry s'autorisa à se détendre un peu~; être appelé 'fils de Lily' signifiait que le centaure le voyait comme plus qu'un intrus quelconque. Et puis, attaquer un élève de Poudlard entraînerait certainement d'immenses représailles contre la tribu centaure de la forêt, et le centaure le savait probablement… "Ce que les Moldus ont appris, c'est qu'il y a un pouvoir dans la vérité, dans tous les fragments de vérité qui interagissent les uns avec les autres, et qu'on ne peut le trouver qu'en découvrant autant de vérités que possible. Pour y parvenir, on ne peut défendre aucune fausse croyance, pas même en disant qu'une fausse croyance est utile. Savoir si vos prédictions viennent vraiment des étoiles ou si c'est un talent inné que vous projetez ne vous semble pas avoir d'importance. Mais si vous vouliez vraiment comprendre la divination, ou même comprendre les étoiles, la vérité quant aux prédictions des centaures serait d'importance pour ces autres vérités."

Le centaure hocha la tête. "Alors les sans-magie sont devenus plus sages que les sorciers. Quelle blague~! Dis-moi, fils de Lily, les Moldus dans leur sagesse disent-ils que le ciel sera bientôt vide~?"

"Vide~?" dit Harry. "Euh… non~?"

"Les autres centaures de la forêt sont restés éloignés de toi, car nous avons fait serment de ne pas nous dresser contre le devenir des cieux. Car, en nous immisçant dans ton destin, nous pourrions perdre en innocence quant à ce qui vient. Moi seul ai osé t'approcher. Comme, il y a seize ans, quand j'étais plus jeune et plus téméraire, j'ai approché une jeune sorcière qui se promenait près de ces bois."

"Je… je ne comprends pas."

"Non. Tu es innocent, comme le disent les étoiles. Et tuer une chose en échange de sa propre vie est un acte terrible. On ne vit alors qu'une vie maudite, une moitié de vie. Et un centaure, s'il devait tuer un poulain, serait certainement banni."

La lance bougea à la vitesse de l'éclair, trop vite pour que les yeux de Harry puissent la suivre, et elle fit tomber sa baguette des mains de Harry.

Un autre coup puissant frappa le plexus solaire de Harry et l'envoya, souffle coupé, saisit de hauts-le-cœur, contre le sol de la forêt.

La main de Harry bondit vers ses robes, vers son Retourneur de Temps, et l'arrière de la lance frappa sa main, assez fort pour lui briser les doigts~; il tendit son autre main et elle fut aussi repoussée…

"Je suis désolé, Harry Potter," dit le centaure, puis il leva des yeux écarquillés. La lance pivota et intercepta un sortilège rouge. Puis le centaure abaissa la lance et bondit désespérément, un éclair vert passa à côté de lui, puis un autre suivit, et enfin un troisième le frappa de plein fouet.

Le centaure tomba et ne bougea plus.

Harry mit longtemps à reprendre sa respiration, à se remettre sur pied, à ramasser sa baguette et à croasser~: "Hein~?"

En cet instant, la sensation funeste, le pouvoir tangible étaient déjà revenus.

"P-Professeur Quirrell~? Que faites-vous ici~?"

"Eh bien," dit l'homme en cape noire d'un ton pensif, "\emph{vous} aviez besoin de piquer une colère bruyante en pleine la Forêt interdite au milieu de la nuit, et \emph{moi} j'avais besoin de m'éloigner de votre capacité à me détecter et de vous surveiller. On ne laisse pas un élève seul dans la Forêt interdite. Rétrospectivement, cela devrait vous être évident."

Harry regarda le centaure tombé.

La forme chevaline ne respirait pas.

"Vous… vous l'avez \emph{tué}, c'était un Avada Kedavra…"

"Je ne comprends pas toujours comment les autres gens s'imaginent que la morale fonctionne. Mais même moi, je sais que selon les règles morales conventionnelles, il est acceptable de tuer des créatures non-humaines sur le point d'abattre un enfant sorcier. Peut-être que l'aspect 'non-humain' n'a pas d'importance pour vous, mais il était sur le point de vous \emph{tuer}. Il était loin d'être innocent…"

Le professeur Quirrell se tut et regarda Harry, qui avait levé une main tremblante jusqu'à sa bouche.

"Bien," dit alors le professeur de Défense, "j'ai fait ma remarque, libre à vous d'y songer. Les lances de centaures peuvent bloquer de nombreux sortilèges, mais personne n'essaie de bloquer quand le sort est d'un vert particulier. Dans ce but, il est utile de connaître quelques sortilèges d'étourdissement à teinte verte. Après tout ce temps, M. Potter, vous devriez vraiment comprendre comment j'opère."

Le professeur de Défense s'approcha du corps du centaure et Harry fit un pas involontaire vers l'arrière, puis un autre, sous le terrible sentiment de ARRÊTE, NON…

Le professeur de Défense s'agenouilla et appuya sa baguette contre la tête du centaure.

La baguette demeura ici un moment.

Et le centaure se leva, les yeux vides. Il respirait à nouveau.

"Oublie tout ceci," ordonna le professeur de Défense. "File, et oublie tout de cette nuit."

Le centaure s'en fut~; les quatre jambes de cheval étrangement synchronisées dans leur mouvement.

"Content maintenant~?" dit le professeur de Défense d'un ton assez sardonique.

Harry avait toujours l'impression que son cerveau était cassé. "Il essayait de me \emph{tuer}."

"Oh, par Merlin… oui, il essayait de vous tuer. Habituez vous-y. Seuls les gens ennuyeux n'ont pas cette expérience."

La voix de Harry émergea, rauque. "Pourquoi… pourquoi voulait-il…"

"Une foule de raisons possibles. Je mentirais si je disais que je n'avais moi-même jamais envisagé de vous tuer."

Harry fixa l'endroit où le centaure avait disparu entre les arbres.

Son cerveau lui semblait toujours à moitié brisé, comme un moteur qui aurait eu des ratés, mais il ne voyait vraiment pas comment ça pourrait être bon signe.

\later

L'annonce que Drago Malfoy avait failli être mangé par quelque horreur avait suffit à faire revenir Dumbledore du lieu inconnu où il était parti, à réveiller Lord Malfoy ainsi que le beau mari de Lady Greengrass et à faire venir Amelia Bones. La supposée présence de l'horreur avait éveillé du scepticisme, même chez Dumbledore, et la possibilité de faux souvenirs avait été évoquée. Harry avait dit (après un débat interne quant aux conséquences qu'il y avait à faire croire aux gens qu'un démon était en liberté) qu'il ne se souvenait pas vraiment avoir exercé le même effort qui lui avait été nécessaire à effrayer le Détraqueur, que la sombre chose était juste partie, et c'était là ce qu'on aurait attendu d'un sortilège de faux souvenirs lancé par quelqu'un qui ignorait comment Harry y était parvenu. Les noms de Bellatrix Black, Severus Rogue et Quirinus Quirrell avaient été mentionnés comme ceux de sorciers assez puissants pour maîtriser tous ceux présents et lancer des sortilèges de faux souvenirs. Harry avait su que Lucius pensait aussi à Dumbledore. Des Aurors avaient témoigné, des discussions avaient tourné en rond, on avait jeté des regards accusateurs et proféré des remarques cinglantes à deux heures du matin. Il y avait eu des propositions, des votes et des conséquences.

"Penses-tu," dit doucement le directeur à Harry, après que tout fut fini et qu'ils furent seuls, "que la Poudlard que tu as fabriquée est meilleure que l'ancienne~?"

Harry s'assit, coudes sur les genoux, visage sur les paumes, au milieu de la salle de conférence que tous les autres avaient à présent quittée. Le professeur McGonagall, qui n'utilisait pas un Retourneur de Temps de façon aussi routinière qu'eux, était prestement partie se coucher.

"Oui," répondit Harry au bout d'une hésitation trop longue. "De mon point de vue, M. le directeur, Poudlard est enfin, enfin normale. C'est ce qui devrait se produire quand quatre enfants sont envoyés nuitamment dans la Forêt interdite. Il devrait y avoir du tapage, on devrait faire venir la police, et les responsables devraient être renvoyés.

"Argus Rusard a servi cette institution pendant plusieurs décennies."

"Et quand on lui a donné du Veritaserum," dit Harry d'un ton fatigué, "il a révélé avoir envoyé un garçon de onze ans dans la Forêt interdite dans l'espoir que quelque chose d'horrible lui arriverait, parce qu'il pensait que le père de l'enfant avait tué son chat. Il n'a pas semblé avoir été alarmé par la présence des trois autres enfants en compagnie de Drago. J'aurais bien requis une peine de prison, mais dans ce pays, votre concept de prison, c'est Azkaban. Je noterais aussi que Rusard était remarquablement déplaisant envers les enfant de Poudlard et je m'attends à ce que l'indice hédonique de l'école augmente suite à son départ, non que vous vous en souciez, je suppose."

Derrière les lunettes en demi-lune, les yeux du directeur étaient impénétrables. "Argus Rusard est un Cracmol. Tout ce qu'il a, c'est cet emploi à Poudlard. Avait, plutôt."

"Le but d'une école n'est pas de fournir du travail à ses employés. Je sais que vous passez probablement plus de temps auprès d'Argus Rusard qu'auprès de n'importe quel élève, mais cela ne devrait pas proportionnellement faire grandir l'importance à vos yeux du ressenti de Rusard. Les élèves aussi ont une vie intérieure."

"Tu t'en fiches complètement, Harry~?" la voix de Dumbledore était douce. "De ceux auxquels tu fais du mal."

"Je me soucie des innocents," dit Harry. "Comme M. Hagrid, au sujet duquel vous remarquerez que j'ai soutenu qu'il devait être considéré comme inconscient, pas comme malveillant. J'aurais été heureux de voir M. Hagrid continuer à travailler ici à condition qu'il n'emmène plus personne dans la Forêt interdite."

"J'avais pensé que, innocenté, Rubeus aurait pu enseigner le soin aux créatures magiques après le départ de Silvanus. Mais une grande partie de cet enseignement a lieu dans la Forêt interdite. Cela, suite à ton passage, n'aura donc pas lieu non plus.

Harry dit lentement~: "Mais… vous nous avez dit que M. Hagrid avait un angle mort lorsqu'il s'agissait des menaces que les créatures magiques pouvaient poser aux sorciers. Que M. Hagrid a un déficit cognitif et qu'il n'a pas la capacité d'imaginer Drago ou Tracey en train de souffrir, et que c'est pour cela que M. Hagrid n'a pas trouvé problématique de les laisser seuls, dans la Forêt interdite, de nuit. N'est-ce pas~?"

"Si, c'est vrai."

"Alors M. Hagrid ne serait-il pas le pire professeur de soin aux créatures magiques possible~?"

Le vieux sorcier regarda Harry à travers ses lunettes en demi-lune. Lorsqu'il parla, sa voix était étouffée. "M. Malfoy lui-même n'avait rien à y redire. Le tour joué par Argus n'était pas si terrible, Harry Potter. Et Rubeus aurait pu s'adapter à son poste. Ça aurait été… tout ce que Rubeus aurait voulu, son plus cher désir…"

"Votre erreur," dit Harry en regardant ses genoux, et avec le sentiment d'être à au moins dix pour cent de son niveau de fatigue maximal historique, "est un biais cognitif que l'on appellerait dans le métier insensibilité à l'échelle. Incapacité à multiplier. Vous pensez au bonheur que ressentirait M. Hagrid en entendant la nouvelle. Songez aux dix prochaines années et aux mille élèves inscrits en cours de soin aux créatures magiques et au dix pour cent d'entre eux brûlés par des Serpencendres. Aucun élève ne souffrirait autant que M. Hagrid ne serait heureux, mais il y aurait cent élèves en souffrance et un seul professeur heureux."

"Peut-être," dit le vieux sorcier. "Et ton erreur, Harry, c'est de ne pas ressentir la douleur de ceux auxquels tu fais du mal, une fois tes multiplications terminées."

"Peut-être." Harry continua de regarder ses genoux. "Ou peut-être que c'est encore pire que ça. M. le directeur, si un centaure ne m'aime pas, qu'est-ce que ça veut dire~?" \emph{Si un membre d'une race de créatures magiques connues pour leurs pouvoirs de divination vous donne un cours sur les gens qui sont ignorants des conséquences, vous présente ses excuses et essaie ensuite de vous poignarder avec une lance, qu'est-ce que ça veut dire~?}

"Un centaure~?" dit le directeur. "Quand as-tu… ah, le Retourneur de Temps. C'est toi, la raison qui m'empêchait de revenir avant l'événement sans provoquer de paradoxe."

"C'est moi~? Oui, je crois que c'est moi." Harry secoua la tête d'un air distant. "Pardon."

"À quelques rares exceptions près," dit Dumbledore, "les centaures n'aiment pas du tout les sorciers."

"Il était un peu plus précis que ça."

"Que t'as-t-il dit~?"

Harry ne répondit pas.

"Ah." Le directeur hésita. "Les centaures ont eu tort nombre de fois, et s'il y a quelqu'un dans ce monde qui pourrait embrouiller les étoiles elles-mêmes, c'est toi."

Harry leva les yeux et vit que les yeux bleus derrière les lunettes en demi-lune étaient à nouveau doux.

"Ne t'en fais pas trop," dit Albus Dumbledore.  
%  LocalWords:  arry
