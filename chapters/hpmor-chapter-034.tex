\partchapter{Problèmes de coordination}{II}

\lettrine{M}{inerva} et Dumbledore avaient usé de leurs talents combinés pour invoquer la majestueuse scène vers laquelle se traînait le professeur Quirrell~; le cœur était fait d'un bois dur, mais les surfaces extérieures brillaient des reflets d'un marbre marqueté de platine et constellé de gemmes aux couleurs de chaque Maison. Ni elle ni le directeur n'étaient les fondateurs de Poudlard, mais l'invocation n'aurait besoin que de durer quelques heures. Minerva appréciait d'habitude les occasions où elle pouvait se fatiguer sur de larges métamorphoses et avec elles, chaque petite chance de déployer son talent artistique et de créer une illusion d'opulence~; mais cette fois, elle avait fait son travail avec la terrible impression de creuser sa propre tombe.

Mais Minerva se sentait à présent un peu mieux. Il y avait eu un bref moment où l'explosion aurait pu survenir~; mais alors Dumbledore était déjà debout, applaudissant avec enthousiasme, et personne ne s'était révélé être assez idiot pour se révolter devant le directeur.

Et l'humeur explosive s'était rapidement fondue en un sentiment collectif qui aurait peut-être pu être résumé par la phrase~: \emph{À d'autres~!}

Blaise Zabini s'était abattu lui-même au nom de Soleil, et le score final avait été 254 à 254 à 254.

\later

Derrière la scène, attendant de monter, trois enfants se regardaient les uns les autres dans une tempête de furie et de frustration. Le fait qu'ils soient encore humides après avoir été pêché du lac et que le charme de réchauffement ne semble pas tout à fait suffire à compenser l'air froid et mordant de décembre n'arrangeait pas les choses, ou peut-être n'était-ce que leur humeur qui était à l'œuvre.

<<~\emph{Assez}, dit Granger. J'en ai \emph{assez}~! Plus de traîtres~!

--- Je suis entièrement d'accord avec vous, Mlle Granger, dit Drago d'un ton de glace. Trop, c'est trop.

--- Et qu'est que \emph{vous} comptez y faire~? lâcha Harry Potter. Le professeur Quirrell a déjà dit qu'il ne bannirait pas les espions~!

--- Nous les bannirons \emph{pour} lui~>>, dit Drago d'un ton sinistre. Il n'avait même pas compris ce qu'il avait voulu dire lorsqu'il avait prononcé ces mots, mais l'acte de parler semblait avoir cristallisé un plan -

\later

La scène était bien réalisée, du moins pour une structure temporaire~; ses fabricants n'étaient pas tombés dans le piège habituel qui consistait à être impressionné par sa propre illusion de richesse, et ils s'y connaissaient en architecture et en style. De là où Drago se tenait, l'endroit où il fallait évidemment qu'il se tienne, les étudiants qui le regarderaient le verraient entouré du halo né du léger scintillement des émeraudes~; et Granger, se tenant là où Drago l'avait subtilement placée, serait entourée du halo des saphirs Serdaigle. Quant à Harry Potter, Drago évitait pour l'instant de le regarder.

Le professeur Quirrell s'était… éveillé, enfin il avait fait ce qu'il faisait d'habitude~; et il s'appuyait sur un podium de platine vide de toute gemme. Avec un sens du spectacle évident, le professeur de Défense empilait et alignait précautionneusement ces trois enveloppes qui contenaient les trois parchemins sur lesquels les trois généraux avaient écrit leurs vœux, tandis que tous les étudiants de Poudlard regardaient et attendaient.

Enfin, le professeur Quirrell releva les yeux. <<~Eh bien, dit le professeur de Défense. C'est assez malvenu.~>>

Un léger ricanement teinté d'une nuance acérée parcourut la foule.

<<~J'imagine que vous vous demandez tous ce que je vais faire~? dit le professeur Quirrell. Il n'y a rien d'autre à faire que ce qui est juste. Mais avant il y a un petit discours que je voudrais prononcer, et encore avant, il semble que M. Malfoy et Mlle Granger veuillent vous faire part quelque chose.~>>

Drago cligna des yeux, puis lui et Granger échangèrent des regards rapides - \emph{je peux} - \emph{oui, vas-y} - et Drago éleva la voix.

<<~Le général Granger et moi voudrions tous les deux dire~>>, dit Drago de sa voix la plus formelle, sachant qu'elle était amplifiée et entendue, <<~que nous n'accepterons plus l'aide d'aucun traître. Et si, à quelque bataille que ce soit, nous découvrons que Potter a accepté des traîtres venus de n'importe laquelle de nos armées, nous joindrons nos forces pour l'écraser.~>>

Et Drago jeta un regard de pure méchanceté au Survivant. \emph{Prends ça, général Chaos~!}

Il y eut un susurrement de surprise chez les étudiants.

<<~Très bien, dit le professeur de Défense en souriant. Cela vous a pris assez longtemps, mais il faut quand même vous féliciter pour y avoir pensé avant tous les autres généraux.~>>

Cela mit un moment à être bien compris -

<<~À l'avenir, M. Malfoy, Mlle Granger, avant de venir à mon bureau chargés d'une demande, demandez-vous s'il existe un moyen pour vous d'accomplir votre but sans mon aide. Je ne déduirai pas de points Quirrell en cette occasion, mais attendez-vous à en perdre cinquante la prochaine fois.~>> Le professeur Quirrell avait un sourire amusé. <<~Et qu'avez-vous à dire à cela, M. Potter~?~>>

Le regard de Harry Potter passa de Granger à Drago. Son visage semblait calme~; même si Drago était certain que \emph{maîtrisé} aurait été un meilleur terme.

Harry Potter parla enfin, sa voix stable. <<~La légion du Chaos est toujours ravie d'accepter les traîtres. On se verra sur le champ de bataille.~>>

Drago savait que le choc était visible sur son visage~; il y eut des murmures abasourdis chez les étudiants, et lorsque Drago jeta un coup d'œil au premier rang il vit que même les chaotiques de Harry avaient été pris par surprise.

Le visage de Granger était en colère, et cela empirait.

<<~M. Potter~>>, dit-elle sur un ton acerbe, comme si elle pensait être un professeur, <<~\emph{essayez-vous} d'être odieux~?

--- Certainement pas, dit calmement Harry Potter. Je ne vous forcerai pas à le faire à chaque fois. Battez-moi une fois, et je resterai vaincu. Mais les menaces ne suffisent pas toujours, général Soleil. Vous ne m'avez pas demandé de vous rejoindre, vous avez simplement essayé d'imposer votre volonté~; et parfois il faut vraiment vaincre l'ennemi pour pouvoir lui imposer sa volonté. Voyez-vous, je suis sceptique à l'idée que Hermione Granger, la plus grande star de Poudlard dans le domaine scolaire, et Drago, fils de Lucius, scion de l'Ancienne et Noble maison Malfoy, puissent travailler ensemble pour vaincre leur ennemi commun, Harry Potter.~>> Un sourire amusé passa sur le visage de Harry Potter. <<~Peut-être que je ferai juste ce que Drago a essayé de faire avec Zabini et que j'écrirai une lettre à Lucius Malfoy pour voir ce qu'\emph{il} pense de ça.

--- \emph{Harry} \emph{!}~>> s'étrangla Hermione, l'air absolument atterrée, et il y eut aussi des bruits d'étranglements venant de l'audience.

Drago contrôla la colère qui se déversait en lui. Dire cela en public avait été un coup \emph{stupide} de la part de Harry. Si Harry l'avait simplement \emph{fait}, cela aurait pu fonctionner, même Drago n'y avait pas pensé, mais \emph{maintenant}, si Père obtempérait, il aurait l'air de jouer le jeu de Harry -

<<~Si vous pensez que mon père Lord Malfoy peut-être manipulé par \emph{vous} si facilement, dit Drago avec froideur, attendez-vous à être surpris, Harry Potter.~>>

Puis Drago se rendit compte, alors que les mots achevaient de quitter sa bouche, qu'il venait de coincer \emph{son propre père}, plus ou moins sans le vouloir. Père n'allait probablement \emph{pas} aimer ça, pas le moins du monde, mais il lui serait maintenant impossible de le dire… Drago allait devoir s'excuser, ça \emph{avait} été un accident sincère, mais il lui était malgré tout étrange de penser qu'il l'avait fait.

<<~Alors allez-y, et vainquez le maléfique général Chaos~>>, dit Harry, l'air toujours amusé. <<~Je ne peux pas gagner contre vos deux armées - pas si vous travaillez \emph{vraiment} ensemble. Mais je me demande si je pourrais peut-être vous séparer avant cela.

--- Tu n'y arriveras pas, et nous \emph{t'écraserons}~!~>> dit Drago Malfoy.

Et, à côté de lui, Hermione hocha vigoureusement la tête.

<<~Eh bien, dit le professeur Quirrell à la suite du silence stupéfait qui s'était étiré un moment. Je n'avais \emph{pas} prévu que cette petite conversation se déroulerait ainsi.~>> Le professeur de Défense avait une expression plutôt intriguée. <<~À vrai dire, M. Potter, je m'attendais à ce que vous concédiez immédiatement et avec le sourire, puis que vous annonciez que vous aviez compris il y a bien longtemps que c'était là la leçon que je souhaitais enseigner mais que vous aviez décidé de ne pas gâcher la surprise pour les autres. De fait, j'avais prévu mon discours en fonction de cela.~>>

Harry Potter se contenta de hocher les épaules.

<<~Désolé~>>, dit-il, et il se tint coi.

<<~Oh, ne vous en faites pas, dit le professeur Quirrell. Cela aura aussi son utilité.~>>

Et le professeur Quirrell se détourna des trois enfants, et il se redressa face au podium pour s'adresser à toute la foule qui observait~; son air habituel de détachement amusé disparut comme un masque qui serait tombé, et lorsqu'il parla de nouveau sa voix était amplifiée avec plus force qu'elle ne l'avait été jusque là.

<<~Sans Harry Potter~>>, dit le professeur Quirrell, sa voix aussi froide et mordante qu'un mois de décembre, <<~Vous-Savez-Qui aurait gagné.~>>

Le silence fut instantané, total.

\later

<<~Ne vous y trompez pas, dit le professeur Quirrell. Le Seigneur des Ténèbres \emph{gagnait}. Il y avait de moins en moins d'Aurors osant lui faire face, les groupes d'autodéfenses qui s'opposaient à lui étaient pourchassés. Un Seigneur des Ténèbres et peut-être cinquante Mangemorts \emph{gagnaient} contre un pays peuplé de milliers de personnes. C'est au-delà du ridicule~! Il n'existe pas de note assez basse pour que je puisse mesurer ce degré d'incompétence~!~>>

On put voir le directeur se renfrogner, et les autres visages exprimèrent l'incompréhension, et le silence profond continua.

<<~Comprenez-vous maintenant comme cela se produit~? Vous l'avez vu aujourd'hui. J'ai autorisé les traîtres, et je n'ai donné aux généraux aucun moyen de les brider. Vous avez vu le résultat. Des plans malins et des trahisons malines, jusqu'à ce que le dernier soldat debout sur le champ de bataille se tire lui-même dessus~! Il est \emph{impossible} que vous doutiez que ces trois armées n'auraient pu être vaincues par \emph{n'importe quel} ennemi extérieur soudé.~>>

Le professeur Quirrell se pencha en avant au-dessus du podium, sa voix maintenant teintée d'une lugubre intensité. Sa main droite s'étira, doigts ouverts. <<~La division est la faiblesse,~>> dit le professeur de Défense. Sa main se referma en un poing serré. <<~L'unité est la force. Le Seigneur des Ténèbres comprenait bien cela, quelles qu'aient été ses autres erreurs~; et il a \emph{utilisé} ce savoir pour créer la seule invention qui a fait de lui le plus terrible Seigneur des Ténèbres de l'Histoire. Vos parents ont fait face à un Seigneur des Ténèbres. Et à cinquante Mangemorts qui étaient parfaitement unis, tous sachant qu'un seul manque de loyauté serait puni de mort, que toute paresse ou incompétence serait punie de douleur. Personne ne pouvait échapper à la portée du Seigneur des Ténèbres lorsqu'on prenait sa Marque. Et les Mangemorts acceptèrent de porter cette terrible Marque car ils savaient qu'une fois porteurs, ils seraient \emph{unis} face à un territoire divisé. Par le pouvoir de la Marque des Ténèbres, un Seigneur des Ténèbres et cinquante Mangemorts auraient pu vaincre un pays entier.~>>

La voix du professeur Quirrell était sinistre et dure. <<~Vos parents \emph{auraient pu} se défendre de la même façon. Ils ne l'ont pas fait. Il y avait un homme nommé Yermi Wibble, qui demanda à la nation de créer un service militaire, même s'il ne fut pas tout à fait assez visionnaire pour proposer une Marque de Grange-Bretagne. Yermi Wibble savait ce qui allait lui arriver~; il espérait que sa mort inspirerait les autres. Alors le Seigneur des Ténèbres prit sa famille pour faire bonne mesure. Leurs peaux vides n'inspirèrent rien d'autre que de la peur, et personne n'osa plus parler. Et vos parents auraient eut à faire face aux conséquences de leur abjecte lâcheté s'ils n'avaient été sauvés par un enfant de un an.~>> Le visage du professeur Quirrell révélait l'ampleur de son mépris. <<~Un dramaturge aurait appelé cela \emph{deus ex machina}, car ils n'ont rien fait pour mériter leur salut. Celui-Dont-Il-Ne-Faut-Pas-Prononcer-Le-Nom ne méritait peut-être pas de gagner, mais n'en doutez pas un instant, vos parents méritaient de perdre.~>>

La voix du professeur Quirrell sonnait comme de l'acier. <<~Et sachez ceci~: vos parents n'ont rien appris~! La pays est toujours fragmenté et faible~! Si peu de décennies se sont écoulées entre Grindelwald et Vous-Savez-Qui~! Pensez-vous que \emph{vous} ne verrez pas la prochaine menace de votre vivant~? Allez-vous \emph{répéter} les erreurs de vos parents alors que vous avez si clairement vu les résultats établis devant vos yeux aujourd'hui~? Car je peux vous dire ce que vos parents feront lorsque les ténèbres viendront~! Je peux vous dire les leçons qu'ils ont apprises~! Ils ont appris à se cacher comme des pleutres et à ne rien faire tout en attendant que Harry Potter les sauve~!~>>

Il y avait air inquisiteur dans les yeux du directeur~; et d'autres élèves levaient les yeux vers leur professeur de Défense avec stupéfaction, colère et admiration.

Les yeux du professeur Quirrell étaient maintenant aussi froids que sa voix. <<~Marquez ce jour, et marquez-le bien. Celui-Dont-Il-Ne-Faut-Pas-Prononcer-Le-Nom souhaitait diriger ce pays, le tenir pour toujours dans sa main cruelle. Mais au moins il souhait régner sur un pays \emph{vivant}, pas sur un tas de cendres~! Il y a eu des Seigneur des Ténèbres fous, qui souhaitaient ne faire du monde qu'un vaste bûcher funéraire~! Il y a eu des guerres où un pays entier marchait contre un autre~! Vos parents ont presque perdu contre une cinquantaine qui souhaitait prendre ce pays vivant~! À quelle vitesse se seraient-ils fait écraser par un ennemi plus nombreux qu'eux, par un ennemi qui ne se serait soucié de rien d'autre que de leur destruction~? Cela, je le prédis~: lorsque la prochaine menace s'élèvera, Lucius Malfoy déclarera que vous devez le suivre ou périr, que votre seul espoir est de croire en sa cruauté et en sa force. Et même si Lucius Malfoy lui-même le croira, ce sera un mensonge. Car lorsque le Seigneur des Ténèbres a péri, Lucius Malfoy n'a pas uni les Mangemorts, ils ont été détruit en un instant, ils ont fui comme des chiens battus et ils se sont trahis les uns les autres~! Lucius n'est pas assez fort pour être un vrai Seigneur, des Ténèbres ou pas.~>>

Les poings de Drago Malfoy étaient blancs, il y avait des larmes dans ses yeux, et de la furie, et une honte insupportable.

<<~Non, dit le professeur Quirrell. Je ne pense pas que Lucius Malfoy sera celui qui vous sauvera. Et si vous pensez que je parle en mon propre nom, le temps montrera assez tôt que ce n'est pas le cas. Je ne vous fais aucune recommandation, mes étudiants. Mais je dis que si tout un pays trouvait un chef aussi fort que le Seigneur des Ténèbres, mais honorable et pur, et prenait sa Marque, alors il pourrait écraser le Seigneur des Ténèbres comme un insecte, et le reste de notre monde magique divisé ne pourrait le menacer. Et si un ennemi encore plus grand devait s'élever contre nous dans une guerre d'extermination, alors seul un monde magique uni pourrait survivre.~>>

Il y eut des hoquets de stupeur, principalement venant des nés-Moldus~; les élèves en robes bordées de vert semblaient seulement interloqués. C'était à présent les poings de Harry Potter qui étaient serrés avec force et qui tremblaient~; et Hermione Granger, à côté de lui, était à la fois consternée et en colère.

Le directeur se leva, son visage maintenant grave, n'ayant pas encore ouvert la bouche~; mais l'ordre était clair.

<<~Je ne dis pas \emph{quelle} menace viendra, dit le professeur Quirrell. Mais vous ne vivrez pas en paix, pas si l'histoire du monde passé doit nous servir de guide quant à son futur. Et si vous, à l'avenir, faites ce que vous avez vu ces trois armées faire aujourd'hui, si vous ne pouvez pas écarter vos petites chamailleries et prendre la Marque d'un seul chef, alors, oui, vous pouvez souhaiter que le Seigneur des Ténèbres ait pu survivre et vous dominer, et regretter le jour où Harry Potter est né -

--- \emph{Assez~!}~>> rugit Albus Dumbledore.

Un silence s'ensuivit.

Le professeur Quirrell détourna lentement son regard vers l'endroit où se tenait Albus Dumbledore et la furie de ses pouvoirs de sorcier~; leurs yeux se rencontrèrent et une pression inaudible s'abattit sur tous les élèves, qui écoutaient mais n'osaient pas bouger.

<<~Vous aussi avez manqué à votre devoir envers ce pays, dit le professeur Quirrell. Et vous connaissez le péril aussi bien que moi.

--- De tels discours ne sont pas pour les oreilles des élèves, dit Albus Dumbledore d'une voix qui grimpait dangereusement. Ni pour les bouches des professeurs~!~>>

Le professeur Quirrell parla alors sèchement~: <<~Il y eut de nombreux discours faits pour les oreilles des adultes pendant l'ascension du Seigneur des Ténèbres. Et les adultes ont applaudit, ils ont acclamé, et ils sont rentrés chez eux après avoir apprécié une journée d'amusement. Mais je vous obéirai, directeur, et si vous ne les aimez pas, je ne ferai plus d'autre discours. Ma leçon est simple. Je continuerai à ne rien faire en ce qui concerne les traîtres, et nous verrons ce que les élèves pourront faire à ce propos lorsqu'ils ne s'attendent pas à ce qu'un professeur vienne les sauver.~>>

Et le professeur Quirrell se retourna vers ses élèves, et sa bouche fit un étrange sourire ironique qui sembla dissiper la terrible pression, comme le souffle d'un dieu qui aurait éparpillé des nuages. <<~Mais s'il vous plaît, soyez indulgents envers les traîtres jusqu'à aujourd'hui, dit le professeur Quirrell. Ils ne faisaient que s'amuser.~>>

Il y eut un rire, et s'il sembla nerveux au départ, il s'amplifia, alors que le professeur Quirrell se tenait là, un sourire ironique aux lèvres, et que la tension se relâchait.

\later

Dans l'esprit de Drago tourbillonnaient encore mille questions entourées d'une stupeur horrifiée, alors que le professeur Quirrell se préparait à ouvrir les enveloppes dans lesquelles les trois avaient inscrit leurs vœux.

Il n'était jamais venu à l'esprit de Drago que les Moldus qui savaient voyager jusqu'à la Lune étaient une plus grande menace que le lent déclin de la sorcellerie, ni que Père s'était révélé trop faible pour les arrêter.

Et plus étrange encore, l'implication évidente~: le professeur Quirrell pensait que \emph{Harry} en était capable. Le professeur de Défense clamait n'avoir fait aucune recommandation, mais il avait mentionné Harry Potter encore et encore au fil de son discours~; d'autres pensaient probablement déjà à la même chose que Drago.

C'était ridicule. Ce garçon avait recouvert un fauteuil rembourré de paillettes et avait appelé ça un trône -

\emph{Le garçon qui a fait face à Rogue et qui a gagné}, chuchota une voix traîtresse, \emph{ce garçon qui pourrait grandir et devenir un Seigneur assez fort pour régner, assez fort pour nous sauver tous -}

Harry a été \emph{élevé} par des Moldus~! Il est pratiquement sang-de-bourbe lui-même, il ne se battrait pas contre sa famille d'adoption -

\emph{Il connaît leur art, leurs secrets et leurs méthodes~; il peut prendre toute la science Moldue et l'utiliser contre eux, conjointement avec nos propres pouvoirs de sorciers.}

Mais s'il refuse~? Et s'il est trop faible~?

\emph{Alors}, dit cette voix intérieure, \emph{ce devra être toi, n'est-ce pas, Drago Malfoy~?}

Et il y eut un nouveau silence dans la foule lorsque le professeur Quirrell ouvrit la première enveloppe.

<<~M. Malfoy, dit le professeur Quirrell, votre vœu est que… Serpentard gagne la coupe des Maisons.~>>

Il y eut un moment d'arrêt interloqué venant de l'audience attentive.

<<~Oui, professeur~>>, dit Drago d'une voix claire, sachant qu'elle serait de nouveau amplifiée. <<~Si vous ne le pouvez pas, alors autre chose pour Serpentard -

--- Je n'accorderai pas de points injustement,~>> dit le professeur Quirrell. Il se tapota une joue, l'air pensif. <<~Ce qui rend votre vœu assez difficile pour être intéressant. Voudriez-vous dire quelques mots quant à vos raisons, M. Malfoy~?~>>

Drago se tourna vers le professeur de Défense, parcourut la foule du regard depuis la scène de platine et d'émeraudes. Tout Serpentard n'avait pas acclamé l'armée Dragon, il y avait même des factions anti-Malfoy qui avaient exprimées leur insatisfaction en soutenant le Survivant ou même Granger~; et ces factions seraient grandement encouragées par ce que Zabini avait fait. Il fallait qu'il leur rappelle que Serpentard voulait dire Malfoy et que Malfoy voulait dire Serpentard -

<<~Non, dit Drago. Ce sont des Serpentard, ils comprendront.~>>

Il y eut des rires venant de l'audience, en particulier chez les Serpentard, même chez certains élèves qui se seraient dit anti-Malfoy un moment plus tôt.

La flatterie était une chose bien délicieuse.

Drago se tourna afin de regarder le professeur Quirrell de nouveau, et il eut la surprise de voir un air embarrassé sur le visage de Granger.

<<~Et pour Mlle Granger…~>> dit le professeur Quirrell. Il y eut le bruit d'une enveloppe déchirée. <<~Votre vœu est que… Serdaigle gagne la coupe des Maisons~?~>>

Il y eut une hilarité considérable venant de l'audience, y compris un gloussement de Drago. Il n'avait pas pensé que Granger aurait joué à ce jeu.

<<~Eh bien, euh~>>, dit Granger, l'air de soudain trébucher sur un discours appris par cœur, <<~ce que je veux dire, c'est que…~>> elle prit une profonde inspiration. <<~Il y avait des soldats de chaque Maison dans mon armée, et je ne veux léser aucun d'entre eux. Mais les Maisons devraient aussi avoir leur importance. C'était triste quand les élèves d'une même maison se jetaient des sorts simplement parce qu'ils n'étaient pas dans la même armée. Les gens devraient pouvoir compter sur ceux de leur Maison. C'est pour cela que Godric Gryffondor, Salazar Serpentard, Rowena Serdaigle et Helga Poufsouffle ont créé les quatre Maisons de Poudlard. Je suis le général Soleil, mais bien avant cela, je suis Hermione Granger de Serdaigle, et je suis fière d'appartenir à une Maison vieille de huit-cents ans.

--- Bien dit, Mlle Granger~!~>>, tonna la voix de Dumbledore.

Harry Potter fronçait les sourcils, et quelque chose gratta à l'orée de la conscience de Drago.

<<~Un sentiment intéressant, Mlle Granger, dit le professeur Quirrell. Mais il est bon, en certaines occasions, qu'un Serpentard ait des amis à Serdaigle, ou qu'un Gryffondor ait des amis à Poufsouffle. Ne serait-il pas certainement meilleur de pouvoir à la fois compter sur les amis de sa Maison et aussi sur ceux de son armée~?~>>

Les yeux de Granger bougèrent brièvement en direction des élèves et des enseignants et elle ne répondit pas.

Le professeur Quirrell hocha la tête, comme s'il s'adressait à lui-même. Il se détourna vers le podium, puis il prit la dernière enveloppe et l'ouvrit. À côté de Drago, Harry se tendit visiblement lorsque le professeur Quirrell éleva le parchemin. <<~Et M. Potter souhaite que -~>>

Il y eut une pause tandis que le professeur Quirrell regardait le parchemin.

Puis, sans aucun changement d'expression sur le visage du professeur Quirrell, le parchemin s'enflamma et brûla dans un feu bref et intense qui ne laissa qu'une poussière noire qui dériva loin de sa main.

<<~Merci de vous en tenir au possible, M. Potter~>>, dit le professeur Quirrell, d'un ton tout à fait sec.

Il y eut une longue pause~; Harry, debout à côté de Drago, semblait assez secoué.

\emph{Par Merlin, mais qu'est-ce qu'il a demandé} \emph{?}

<<~J'espère, dit le professeur Quirrell, que vous avez préparé un autre vœu au cas où je ne pourrais vous accorder celui-ci.~>>

Il y eut une autre pause.

Harry prit une profonde inspiration. <<~Je n'ai rien fait de tel, dit-il, mais j'ai déjà trouvé un autre vœu.~>> Harry Potter pivota afin d'observer l'audience, et sa voix se raffermit au fur et à mesure qu'il parlait. <<~Les gens craignent les traîtres à cause des dommages directs que ceux-ci provoquent, des soldats qu'ils abattent et des secrets qu'ils révèlent. Mais ce n'est qu'une partie du danger. Ce que les gens font parce qu'ils ont \emph{peur} des traître leur coûte aussi. J'ai utilisé cette stratégie aujourd'hui contre Soleil et contre Dragon. Je n'ai pas dit à mes traîtres de causer autant de dommages que possible. Je leur ai dit d'agir afin de créer le maximum de méfiance et de confusion, et de pousser les généraux à agir de la plus coûteuse des manières possibles dans leurs tentatives de les empêcher de trahir à nouveau. Lorsqu'il n'y a que quelques traîtres et qu'un pays entier leur fait face, il va de soi que ce que ce petit groupe fait est moins dommageable que ce que le pays entier peut faire pour les arrêter, que le remède peut être pire que le symptôme -

--- M. Potter, dit le professeur de Défense d'une voix soudain tranchante, L'Histoire nous enseigne que vous avez tout simplement tort. La génération de vos parents a fait trop peu pour s'unir, pas trop~! L'intégralité de ce pays a failli tomber, M. Potter, bien que vous n'ayez pas été là pour le voir. Je suggère que vous demandiez à vos camarades de dortoir de Serdaigle combien d'entre eux ont perdu leur famille contre le Seigneur des Ténèbres. Ou si vous êtes plus sage, ne leur demandez \emph{pas~!} \emph{Avez}-vous un vœu à faire, M. Potter~?

--- Si vous permettez, dit la douce voix d'Albus Dumbledore, je souhaiterais entendre ce que le Survivant à a dire. Il a plus d'expérience que nous deux lorsqu'il s'agit de mettre fin à des guerres.~>>

Quelques personnes rirent, mais elles n'étaient pas nombreuses.

Le regard de Harry Potter passa à Dumbledore, et l'espace d'un instant, il sembla perdu dans des considérations.

<<~Je ne dis pas que vous avez tort, professeur Quirrell. Dans la dernière guerre, les gens n'ont pas agit ensemble, et tout un pays a failli tomber face à quelques dizaines d'attaquants, et oui, ce fut pathétique. Et si nous faisons la même erreur la prochaine fois, oui, ce sera encore plus pathétique. Mais on ne fait jamais la même guerre deux fois. Et le problème, c'est que l'ennemi a \emph{lui aussi} le droit d'être intelligent. Lorsque l'on est divisé, on est vulnérable par certains aspects~; mais lorsqu'on essaie de s'unir, alors on fait face à d'autres risques et à d'autres coûts, et l'ennemi essaiera aussi de tirer parti de ceux-ci. On ne peut pas s'arrêter de réfléchir au premier niveau de jeu.

--- La simplicité a aussi beaucoup pour elle, M. Potter,~>> dit la voix sèche du professeur de Défense. <<~J'espère que vous avez appris quelque chose aujourd'hui quant aux dangers des stratégies plus complexes que celle consistant à unir son peuple et à attaquer son ennemi. Et si tout cela n'est pas lié à votre vœu d'une façon ou d'une autre, je serai fort courroucé.

--- Oui, dit Harry Potter, ça a été assez difficile de trouver un vœu qui symbolise le coût de l'unité. Mais le problème d'agir ensemble ne concerne pas que les guerres, c'est quelque chose que nous devons résoudre tout au long de notre vie, tous les jours. Si tout le monde se coordonne en utilisant les mêmes règles et que les règles sont stupides, alors si \emph{une} personne décide de faire les choses différemment, elle brise les règles. Mais si \emph{tout le monde} décide de faire les choses différemment, alors un changement peut avoir lieu. C'est exactement le même problème quand tout le monde doit agir ensemble. Pour la \emph{première} personne qui s'exprime, elle a l'air d'aller à l'encontre du désir de la foule. Et si l'on croit que la seule chose qui importe est que les gens soient toujours unis, alors on ne peut jamais changer les règles du jeu, peu importe à quel point les règles sont stupides. Donc mon vœu, pour symboliser ce qui se passe lorsque les gens s'unissent dans la mauvaise direction, est qu'à Poudlard, on joue au Quidditch sans le Vif d'Or.

--- \scream{Quoi~?}~>> hurlèrent cent voix dans la foule, et la mâchoire de Drago s'affaissa.

<<~Le Vif gâche tout le jeu, dit Harry Potter. Tout ce que les autres joueurs accomplissent finit par n'avoir aucune importance. Il serait bien plus sensé d'acheter une horloge. C'est une de ces choses incroyablement stupides que vous ne remarquez pas parce que vous avez grandi avec, que les gens ne font que parce que tout le monde le fait aussi -~>>

Mais à ce moment, la voix de Harry Potter ne pouvait plus être entendue car l'émeute avait commencé.

\later

L'émeute prit fin environ quinze secondes plus tard, après qu'un gigantesque jet de feu eut jailli de la plus haute tour de Poudlard, au son de milles tonnerres. Drago n'avait pas su que Dumbledore pouvait \emph{faire} ça.

Les élèves se rassirent avec beaucoup de précaution et de calme.

Le professeur Quirrell riait à gorge déployée. <<~Ainsi soit-il, M. Potter. Votre volonté sera faite.~>> Le professeur Quirrell s'interrompit délibérément. <<~Bien sûr, j'ai seulement promis \emph{un} fourbe complot. Et c'est tout ce que vous trois aurez.~>>

Drago s'était presque attendu à ces mots un peu plus tôt, mais il fut malgré tout surpris~; il échangea des regards rapides avec Granger, ils auraient évidemment dû être alliés, mais leurs vœux étaient directement opposés l'un à l'autre -

<<~Vous voulez dire, dit Harry, que nous devons tous nous mettre d'accord sur un vœu~?

--- Oh, ce serait bien \emph{trop} vous demander~>>, dit le professeur Quirrell. <<~Vous trois n'avez pas d'ennemi commun, que je sache~?~>>

Et pendant un bref instant, si vite que Drago se demanda s'il aurait pu l'imaginer, les yeux du professeur de Défense clignèrent vers Dumbledore.

<<~Non, dit le professeur Quirrell, je veux dire que j'exaucerai trois vœux au moyen d'une seule intrigue.~>>

Il y eut un silence confus.

<<~Vous ne pouvez pas faire ça~>>, dit catégoriquement Harry, à côté de Drago. <<~Même \emph{moi} je ne peux pas le faire. Deux de ces vœux sont mutuellement incompatibles. C'est \emph{logiquement impossible} -~>> puis Harry s'interrompit.

<<~Il vous manque quelques années avant de pouvoir me dire ce que je ne peux pas faire, M. Potter~>>, dit le professeur Quirrell avait un bref sourire sec.

Puis le professeur Quirrell se retourna vers les élèves. <<~À vrai dire, je n'ai aucune confiance en votre capacité à apprendre la leçon d'aujourd'hui. Rentrez chez vous, appréciez ce moment avec vos familles, ou ce qu'il en reste, tant qu'ils vivent encore. La mienne est depuis longtemps morte par la main du Seigneur des Ténèbres. Je vous verrai tous lorsque les cours reprendront.~>>

Dans le silence absolu qui suivit ces mots, le professeur Quirrell marchant déjà vers la sortie de la scène, Drago entendit la voix du professeur Quirrell dire, doucement, sans amplification, <<~Mais avec vous, M. Potter, je vais m'entretenir immédiatement.~>>
%  LocalWords:  inerva
