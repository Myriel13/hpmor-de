\partchapter{Accomplissement de soi}{IV}

\lettrine{E}{lle} ne le vit que du coin de l'œil, le reflet sur le métal poli d'une statue située à l'embranchement de deux couloirs, l'éclat doré, l'éclat rouge, comme l'image d'un feu~; il fut là un instant et parti le suivant.

Elle s'arrêta, perplexe, et \emph{faillit} s'en aller, mais elle avait reconnu quelque chose de familier dans cette brève lueur…

Hermione s'avança vers l'endroit où s'était trouvée la statue et observa le couloir d'où elle pensait que le reflet flamboyant était venu.

Faiblement, comme venu d'un endroit lointain, elle entendit le cri, l'appel.

Elle commença à courir.

Elle courut un moment~; à chaque fois qu'elle atteignait un embranchement, elle s'interrompait, reprenait sa respiration autant que possible, et voyait alors un autre éclat de feu reflété d'un couloir ou de l'autre, ou bien elle entendait cet appel lointain. N'était pour son entraînement militaire, elle se serait épuisée à courir ainsi.

Elle ne vit jamais le phénix.

Et alors elle parvint à un embranchement quadruple et il n'y eut plus \emph{rien}, aucun signe, elle attendit de longues secondes, n'entendit nul cri, ne vit nul feu, et elle commençait juste à se demander, partagée entre la nausée et la tristesse, si elle avait tout imaginé, quand elle entendit \emph{quelqu'un} crier.

Lorsque la course rapide de ses pieds l'eurent menée derrière l'angle du couloir, son esprit captura toute la scène d'un coup d'œil~: trois immenses garçons dans des robes bordées de vert et qui se tournaient déjà vers elle, et un autre garçon en jaune, plus petit et plus chétif, qui pendait au milieu du vide, le pied suspendu en l'air par une main invisible.

Le général Soleil ne réfléchit même pas~; les gens qui s'arrêtaient pour réfléchir ne tendaient pas de très bonnes embuscades.

Baguette à la main, ses doigts opérèrent la torsion nécessaire, ses lèvres dirent~: <<~\emph{Somnium~!}~>>, la brute la plus grande tomba au sol, le garçon Poufsouffle chuta avec un \emph{bam}, les deux autres brutes essayèrent de la viser de leur baguette, elle répéta <<~\emph{Somnium~!}~>> et un autre des immenses garçons s'écroula -- celui qui avait ajusté sa visée le plus vite, c'est sur lui qu'elle avait tiré.

Malheureusement, lancer deux sortilèges de sommeil comme ça était difficile, même pour elle, et elle ne pourrait pas descendre le troisième avant…

La dernière brute hurla <<~\emph{Protego~!}~>> et fut entourée d'une lueur bleue iridescente.

Vingt-quatre heures plus tôt, Hermione aurait paniqué~: un \emph{vrai} sortilège de bouclier qui laisserait la brute lui jeter des sortilèges tout en restant protégé.

\emph{Maintenant} elle…

<<~\emph{Stupéfix~!}~>> hurla la brute.

L'éclair cramoisi fonça droit sur elle dans un éclat terrible, flamboyant avec plus de force que tous les sortilèges qui étaient jamais sortis de la baguette de Harry.

Hermione se décala légèlement sur la gauche et l'éclair la manqua, parce que la brute n’avait pas \emph{visé} aussi bien que Harry~; et l'idée lui vint que les brutes et les membres des armées du professeur Quirrell ne se mêlaient peut-être jamais les uns aux autres.

<<~\emph{Stupéfix~!} hurla à nouveau le garçon. \emph{Expelliarmus~! Stupéfix~!}~>>

Bref, \emph{maintenant} elle venait de passer une heure à penser aux \emph{autres} sortilèges qu'elle aurait dû lancer sur Harry et Neville…

<<~\emph{Gélifix~!}~>> hurla la brute, un sortilège à large rayon, sans éclair visible qui puisse être évité, et ses genoux lui semblèrent soudain presque trop faibles pour supporter son poids. Puis, d'un rugissement de colère qui produisit un cramoisi encore plus lumineux~: <<~\emph{Stupéfix~!}~>>

Elle évita celui-ci en tombant délibérément, et elle eut alors suffisamment récupéré pour son prochain sortilège, qui serait…

<<~\emph{Glisseo,}~>> dit Hermione, dirigeant sa remarque vers le sol.

<<~Arg~>>, dit la brute lorsque ses pieds se dérobèrent sous lui et qu'il \emph{laissa carrément tomber sa baguette.}

Le \emph{Protego} disparut instantanément.

<<~\emph{Somnium,}~>> dit Hermione.

Elle était encore haletante lorsqu'elle rampa jusqu'au Poufsouffle qui se redressait et grondait en massant son crâne là où il avait chuté tête la première contre le sol~; Hermione se rendit compte qu'il était heureux qu'il ne soit pas Moldu sans quoi le Poufsouffle aurait pu se briser le cou. Elle n'y avait à vrai dire pas pensé.

<<~Euh~>>, dit le garçon aux cheveux bruns, aux yeux d'un marron commun qui parvenait à sembler correspondre exactement à la notion de Poufsouffle et dont le visage, bien que dénué de larmes, était plutôt pâle. Elle l'aurait situé en quatrième ou peut-être troisième année.

Puis les yeux marron s'écarquillèrent à mesure qu'ils mirent au point sur elle.

<<~\emph{Général Soleil~?}

--- Ouais, dit-elle. C'est (\emph{halètement}) moi.~>> Elle décida alors que si le Poufsouffle disait quoi que ce soit sur son statut d'objet des émois de Harry Potter, il mourrait.

<<~Waouh, dit le Poufsouffle. C'était -- tu viens juste de -- je veux dire, je t'ai vu sur les écrans avant Noël mais -- waoh~! J'arrive pas à croire que tu viens de faire ça~!~>>

Il y eut un silence.

\emph{J'arrive pas à croire que tu viens de faire ça}, pensa Hermione Granger qui se sentit soudain un peu défaillante~; ça devait être à cause de sa course récente. <<~Excuse (\emph{halètement}) moi, dit-elle, est-ce que tu pourrais juste (\emph{halètement}) me dégélifier les jambes~?~>>

Le garçon hocha la tête, se remit sur pied et chercha sa baguette dans sa robe~; puis Hermione dut corriger son geste pour que le contre-sort fonctionne.

<<~Je m'appelle Michael Hopkins~>>, dit le garçon après que Hermione se fut relevée d'une roulade. Il tendit la main. <<~Ou juste Mike pour ceux de Poufsouffle, il n'y a aucun autre Mike à Poufsouffle cette année, incroyable non~?~>>

Ils se serrèrent la main et Mike dit~: <<~Enfin bref, \emph{merci}.~>>

Hermione n'était pas préparée à la bouffée d'euphorie qui la saisit alors~; sauver ainsi quelqu'un l'avait faite se sentir mieux qu'elle ne s'était jamais sentie \emph{de toute sa vie}, littéralement.

Elle se tourna pour regarder les brutes.

Ils étaient très grands et, songea-t-elle, semblaient avoir environ quinze ans, et elle se rendit soudain compte de \emph{l'immensité} de la distance qui s'était établie entre les élèves de Poudlard qui s'étaient inscrits à toutes les activités du soir du professeur Quirrell et ceux qui avaient plusieurs années durant reçu l'enseignement des pires professeur à avoir jamais professé. Pouvoir \emph{atteindre} les cibles qu'on visait, par exemple~; ou pouvoir suffisamment réfléchir au milieu d'un combat pour se rendre compte qu'on avait intérêt à \emph{Innerver} ses alliés tombés. Et d'autres choses que le professeur avait dites, par exemple que dans le monde réel presque tous les combats seraient résolus par une attaque surprise, voilà qui lui semblait soudain très sensé.

Alors qu'elle essayait encore de reprendre son souffle, elle se tourna de nouveau vers Mike.

<<~Est-ce que tu (\emph{halètement}) me croirais, dit Hermione Granger, si je te disais qu'il y a cinq minutes j'avais (\emph{halètement}) du mal à trouver comment devenir une (\emph{halètement}) une héroïne~?~>>

Avait-elle vraiment cru qu'elle avait besoin de la \emph{permission} de quelqu'un, ou que les héros restaient juste là assis à attendre qu'on leur donne des quêtes~? C'était en réalité très simple~: il suffisait d'aller là où le Mal se trouvait, et c'était tout ce qu'il y avait à faire pour être une héroïne. Elle aurait dû s'en souvenir, elle n'aurait pas dû avoir besoin qu'un phénix lui dise que des méfaits étaient parfois commis ici, à Poudlard.

Puis elle jeta un regard nerveux en arrière, là où les trois garçons plus âgés étaient étendus, inconscients, et elle réalisa soudain qu'ils l'avaient \emph{vue}, qu'ils \emph{savaient} peut-être qui elle était, qu'ils pourraient se faufiler jusqu'à elle et \emph{la} prendre par surprise et -- et ils pourraient vraiment lui faire mal…

Elle s'arrêta.

Elle se souvint que Harry Potter s'était mis devant \emph{cinq} brutes de Serpentard lors de son premier jour, alors qu'il ne savait même pas comment utiliser une baguette.

Elle se souvint du directeur lui disant qu'on grandissait en étant mis face à des situations d'adultes et que la plupart des gens passaient leur vie enfermés dans un cycle contraignant de peur.

Et elle se souvint de la voix du professeur McGonagall qui lui disait~: “Vous \emph{avez} douze ans”.

Elle prit une profonde inspiration, puis une deuxième, puis une troisième.

Elle demanda à Mike s'il avait besoin d'aller au bureau de Mme Pomfresh, et la réponse était non~; puis elle obtint de lui qu'il nomme les garçons Serpentard, juste au cas où.

Puis Hermione Granger s'éloigna d'un pas tranquille du tas de brutes inconscientes en s'assurant d'avoir un sourire sur le visage.

Elle savait qu'elle allait probablement souffrir, tôt ou tard. Mais si on avait trop peur d'avoir mal pour faire ce qui était juste alors on ne pouvait pas devenir un héros, c'était aussi simple que cela~; et auriez-vous mis le Choixpeau sur sa tête à cet instant il n'aurait pas attendu \emph{une seule seconde} avant de s'écrier~: “GRYFFONDOR~!”

\later

Elle y réfléchissait encore lorsque vint l'heure du dîner~; l'euphorie d'avoir sauvé quelqu'un ne s'était pas encore estompée et elle commençait à s'inquiéter de la possibilité que cela ait cassé quelque chose dans son cerveau.

Une épidémie de chuchotements se déclencha dès qu'elle s'approcha de la table Serdaigle, et elle se demanda si le garçon Poufsouffle avait dit quelque chose avant de se rendre compte que les chuchotements ne parlaient probablement pas de \emph{ça}.

Elle s'assit face à Harry Potter qui avait l'air \emph{extrêmement} nerveux, probablement parce qu'elle souriait encore.

<<~Euh…~>> dit Harry alors qu'elle se servait une tranche de pain de mie grillé, de beurre, de cannelle, d'aucun fruit, d'aucun légume et de trois parts de brownie au chocolat. <<~Euh…~>>

Elle le laissa continuer comme ça jusqu'à ce qu'elle eût fini de se verser un verre de jus de pamplemousse puis elle dit~:

<<~J'ai une question pour vous, M. Potter. Comment pensez-vous que les gens échouent à devenir eux-mêmes~?

--- \emph{Quoi~?}~>> dit Harry.

Elle le regarda. <<~Fais comme si tout était comme d'habitude, dit-elle, et dis juste ce que tu aurais dit hier.

--- Euh…~>> dit Harry, et il avait l'air très perplexe et inquiet. <<~Je pense que nous sommes \emph{déjà} nous-mêmes… ce n'est pas comme si j'étais la copie imparfaite de quelqu'un d'autre. Mais je suppose que si j'essayais de m'en tenir au sens de la question, je dirais que les gens ne deviennent pas eux-mêmes parce qu'ils absorbent toutes ces idioties de leur environnement et qu'ils les régurgitent ensuite. Je veux dire, combien de joueurs de Quidditch joueraient à un jeu de ce genre s'ils l'avaient inventé eux-mêmes~? Ou en Angleterre moldue, combien de gens qui se voient comme des travaillistes, des conservateurs ou des démocrates libéraux auraient inventé exactement le même paquet de croyances politiques s'ils avaient dû tout inventer eux-mêmes~?~>>

Hermione y songea. Elle s'était demandée si Harry aurait une réponse de Serpentard, ou peut-être même de Gryffondor, mais cela ne semblait pas correspondre à la liste du directeur~; et l'idée lui vint qu'il existait peut-être bien plus de quatre perspectives sur le sujet.

<<~D'accord, dit Hermione, autre question. Qu'est-ce qui fait de quelqu'un un héros~?

--- Un \emph{héros~?} dit Harry.

--- Ouais, dit Hermione.

--- Ah…~>> dit Harry. Sa fourchette et son couteau tranchèrent un morceau de steak avec nervosité, le coupèrent en des morceaux de plus en plus petits. <<~Je pense que beaucoup de gens sont capables d'accomplir des choses lorsque le monde les canalise dans cette direction… si d'autres s'attendent à ce que tu fasses quelque chose, par exemple, ou si ça ne demande que des compétences que tu as déjà, ou s'il y a une autorité présente pour remarquer tes erreurs et t'assurer que tu remplis ton rôle. Mais les problèmes de ce genre sont probablement déjà résolus, tu sais, et il n'y a donc pas besoin de héros. Donc je pense que les gens qu'on appelle des “héros” sont rares parce qu'ils doivent tout inventer au fur et à mesure, et la plupart des gens trouvent ça pénible. Pourquoi cette question~?~>> La fourchette de Harry poignarda trois morceaux de steak minutieusement découpés et les fit monter vers sa bouche.

<<~Oh, je viens juste d'assommer trois brutes de Serpentard et de sauver un Poufsouffle, dit Hermione. Je vais devenir une héroïne.~>>

Lorsque Harry eut fini de s'étrangler sur sa bouchée (certains des autres Serdaigle à portée d'oreille toussaient encore), il répondit~: <<~\emph{Quoi~?}~>>

Hermione raconta son histoire et celle-ci commença à se propager de chuchotements en chuchotements à mesure qu'elle parlait (mais elle omit le phénix car cela lui semblait être une affaire privée entre eux deux. Hermione avait été surprise, lorsqu'elle y avait ensuite repensé, par le fait qu'un phénix apparaisse pour quelqu'un qui \emph{voulait} être un héros~; ça semblait un peu égoïste vu sous cet angle~; mais peut-être que les phénix s'en fichaient du moment qu'ils voyaient bien que vous aviez envie d'aider les gens).

Lorsqu'elle eut fini de parler, Harry la fixa du regard et ne dit pas un mot.

<<~Je suis désolé d'avoir agi comme ça plus tôt~>>, dit Hermione. Elle sirota un peu de son jus de pamplemousse. <<~J'aurais dû me rappeler que si je te bats toujours à plates couture en cours de Charmes ça n'est pas grave si tu es meilleur en Défense.

--- \emph{S'il te plaît}, ne le prends pas mal~>>, dit Harry. Il avait maintenant l'air sombre et trop-adulte. <<~Mais es-tu certaine que c'est \emph{toi}, ça, et pas, pour parler franchement, moi~?

--- J'en suis assez certaine, dit Hermione. Allons, mon nom est presque une anagramme de “héroïne” mis à part ce “m” en trop, je ne m'en étais jamais rendu compte avant aujourd'hui.

--- Ce n'est pas toujours la fête d'être un héros, dit Harry. Pas le vrai boulot de héros du genre que les adultes ont à faire~; ça n'est pas comme ça, ça ne va pas être aussi simple.

--- Je sais, dit Hermione.

--- C'est difficile et c'est douloureux et il faut prendre des décisions alors qu'aucune bonne réponse n'existe…

--- Oui Harry, j'ai lu ces livres moi aussi.

--- Non, dit Harry, tu ne comprends pas, même si les livres te mettent en garde il est impossible que tu \emph{puisses} comprendre jusqu'à ce que…

--- Ça ne t'arrête pas, dit Hermione. Ça ne t'arrête pas le moins du monde. Je parie que tu n'as même jamais \emph{envisagé} de ne pas être un héros à cause de ce problème. Alors pourquoi est-ce que tu penses que ça va m'arrêter~?~>>

Il y eut un silence.

Un immense sourire éclaira soudain le visage de Harry, un sourire aussi enfantin et radieux que sa mine avait été sombre et adulte, et tout alla de nouveau bien entre Hermione et Harry.

<<~Ça va tourner atrocement, extraordinairement mal, d'une façon ou d'une autre,~>> dit Harry, toujours avec son immense sourire. <<~Tu es au courant, n'est-ce pas~?

--- Oh, je sais~>>, dit Hermione. Elle prit une autre bouchée de pain grillé. <<~D'ailleurs, Dumbledore a refusé d'être mon vieux sorcier mystérieux. Est-ce qu'il y a un endroit auquel je pourrais écrire pour en demander un autre~?~>>

\latersection{Après-coup~:}

<<~… et le professeur Flitwick dit que sa détermination est inébranlable~>>, dit Minerva d'une voix pincée en regardant le vieux sorcier à barbe d'argent qui était responsable de tout cela. Albus Dumbledore était assis, silencieux, et il l'écoutait, le regard lointain. <<~Mlle Granger n'a même pas cillé quand le professeur Flitwick a menacé de la transférer à Gryffondor, elle a juste dit que si elle partait elle emmènerait tous les livres avec elle. Hermione Granger a décidé qu'elle serait une héroïne et ne tolérera aucun refus. Je doute que vous auriez pu la pousser vers cette voie avec plus de force si vous aviez \emph{essayé} de…~>>

Le cerveau de Minerva mit cinq bonnes secondes à traiter l'éclair de compréhension.

<<~\scream{Albus~!} glapit-elle.

--- Ma chère, dit le vieux sorcier, lorsque vous aurez eu affaire à votre trentième héros environ, vous comprendrez qu'ils réagissent de façon assez prévisible à certaines choses~; par exemple d'entendre qu'ils sont trop jeunes, qu'ils ne sont pas destinés à devenir des héros ou qu'il est désagréable d'en être un~; et si vous souhaitez vraiment être sûr, vous devriez leur dire les trois. Bien que~>>, avec un bref soupir, <<~il vaille mieux éviter d'être \emph{trop} flagrant dans votre tentative, sans quoi votre directrice adjointe pourrait vous surprendre.

--- Albus, dit Minerva d'une voix encore plus pincée, s'il lui arrive du mal, je jure que cette fois je…

--- Elle aurait fini par en arriver au même point~>>, dit Albus, son regard toujours lointain. <<~Si quelqu'un est censé devenir un héros, alors il n'écoutera pas nos mises en gardes, Minerva, peu importe nos efforts. Cela étant acquis, il vaut mieux pour Harry que Mlle Granger ne se laisse pas trop distancer par lui.~>> Albus fit apparaître, comme jaillie de nulle part, une boîte en étain qui s'ouvrit pour révéler de petites formes jaunes, elle n'avait jamais réussi à découvrir où il les cachait et elle n'avait jamais pu détecter quelle magie était à l'œuvre. <<~Bonbons au citron~?

--- \emph{C'est une petite fille de douze ans, Albus~!}~>>

\latersection{Après-coup~:}

À peine visibles dans l'obscurité du soir, des poissons nageaient dans les eaux noires derrière les fenêtres, illuminés par le vif éclat de la chambre commune de Serpentard lorsqu'ils s'approchaient, se fondant dans les ténèbres lorsqu'ils s'éloignaient.

Daphné Greengrass était confortablement assise dans un divan de cuir noir, sa tête entre ses mains, entourée d'un halo jaune-or tandis que de vives étincelles de lumière blanche apparaissaient et disparaissaient autour d'elle.

Elle avait été prête à ce qu'on la taquine au sujet de Neville Londubat. Elle s'était attendue à entendre plein de remarques sournoises au sujet de Poufsouffle. Elle avait inventé des \emph{royaumes} entiers de promptes réparties sur le chemin du retour vers les donjons Serpentard.

Elle s'était \emph{réjouie} à l'idée qu'on la taquine parce qu'elle aimait bien Neville Londubat. Se faire taquiner sur ce genre de choses voulait dire qu'on était devenue une vraie fille.

Il s'avéra que personne n'avait deviné que défier Neville au Très Ancien Duel voulait dire qu'elle l'aimait bien. Elle avait pensé que ce serait \emph{évident} mais non, personne d'autre n'y avait apparemment songé.

C'était toujours le sort que vous n'aviez pas vu qui réussissait à vous atteindre.

Elle aurait juste dû s'appeler Daphné du Soleil, comme Neville du Chaos. Ou Daphné Soleil comme Ron Soleil. Ou \emph{n'importe quoi} à part Greengrass du Soleil.

Greengrass du Soleil. Herbe Verte du Soleil.

On était allé de là à Herbe Verte du Soleil et du Ciel Bleu.

Puis quelqu'un avait ajouté Montagnes aux Cimes Enneigées et aux Batifolantes Créatures des Bois.

On la mentionnait maintenant sous le nom de Princesse Licorne Scintillante de la Noble et Très Ancienne Maison de Cacatillant.

Et une maudite élève en sixième année l'avait frappée d'un sortilège d'étincelles, elle ne savait même pas que ça \emph{existait}, et \emph{Finite Incantatem} n'avait pas marché, et elle avait demandé à d'autres filles plus âgées dont elle avait \emph{cru} qu'elles étaient ses amies (elle avait apparemment eu tort sur ce point) puis elle avait menacé la maudite élève de sérieux désordres politiques déclenchés par son père, et malgré cela, Daphné Greengrass était toujours assise dans la salle commune de Serpentard, sa tête entre ses mains, étincelante, se demandant comment elle avait fini par se retrouver être la seule personne saine d'esprit de Poudlard.

On était \emph{après le dîner} et ils \emph{continuaient} et s'ils ne s'étaient pas arrêtés demain matin elle allait se faire transférer à Durmstrang et devenir la prochaine Dame des Ténèbres.

<<~Hé, tout le monde~!~>> dirent les jumeaux Carrow d'un ton théâtral en agitant un exemplaire de la \emph{Gazette du Sorcier}. <<~Vous avez entendu la nouvelle~? Le Magenmagot vient juste de déclarer qu'“on va voir ce que t'as dans le ventre” constitue un défi légal qui doit se solder par un duel qui continuera jusqu'à ce que l'initiateur du combat s'allonge et fasse la sieste~!

--- Comment oses-tu insulter l'honneur de la Princesse Licorne Scintillante~! s'écria Tracey. On va voir ce que t'as dans le ventre~!~>> Puis Tracey s'allongea sur le sofa et commença à ronfler puissamment.

La tête scintillante de Daphné s'enfonça plus profondément entre ses mains luisantes. <<~Lorsque ma famille aura pris le pouvoir je vous lancerai tous des sortilèges anti-Transplanage et je vous jetterai au fond de la mer par Portoloin, dit-elle à l'intention de personne en particulier. Ça ne vous dérange pas j'espère~?~>>

\emph{Toc-toc, toc-toc-toc, toc.}

<<~\emph{J'ouïs quelqu'un frapper~!} mugit M. Goyle. \emph{Frapper à la porte~!}

--- \emph{On va voir ce que t'as dans le ventre, porte~!}~>> s'écria un garçon plus âgé situé non loin de celle-ci, et il l'ouvrit grand.

Il y eut un moment de surprise générale.

<<~Je suis venue parler avec Mlle Greengrass~>>, dit le général Soleil avec l'air de vouloir donner l'impression qu'elle avait confiance en elle. <<~S'il vous plaît, quelqu'un pourrait-il…~>>

À en voir l'expression de Hermione, elle venait de se rendre compte que Daphné scintillait.

Et c'est \emph{là} que Millicent Bulstrode arriva à toute allure depuis les dortoirs inférieurs et hurla~: <<~Hé tout le monde, devinez quoi, maintenant \emph{Granger} est allée mettre une raclée à Derrick et ce qui reste de son équipe, et son père lui a envoyé une chouette pour lui dire que s'il ne…~>>

Millicent aperçut Hermione qui se tenait dans l'embrasure de la porte.

Il y eut un silence très prononcé.

<<~Euh~>>, dit Daphné. \emph{Quoi~?} dit son cerveau. <<~Euh, qu'est-ce que vous faites ici, général~?

--- Eh bien~>>, dit Hermione Granger, un étrange sourire sur visage, <<~j'ai décidé que ça n'est pas juste si de mystérieux vieux sorciers donnent seulement à certaines personnes une chance d'être un héros, et j'ai aussi lu des livres d'Histoire et il n'y a vraiment pas assez d'héroïnes dedans. Alors je me suis dit que je passerais juste pour voir si tu voulais être une héroïne, et pourquoi est-ce que tu brilles comme ça~?~>>

Il y eut un autre silence.

<<~Ce n'est, dit Daphné, probablement \emph{pas} le meilleur moment pour me demander ça…

--- \emph{J'accepte~!}~>> s'écria Tracey Davis en bondissant de son sofa.

\later

C'est ainsi que naquit la Société pour la Promotion de l'Égalité Héroïque pour les Sorcières.
%  LocalWords:  Jellify Oof Unjellify Afteraftermath Sparklypoo
