\chapter{Priorités utilitaristes}

\lettrine{O}{n} était samedi, le premier matin de février, et à la table Serdaigle, un garçon faisait face à une assiette contenant un haut monticule de légumes. Il les inspectait nerveusement, à la recherche de la moindre trace de viande.

La réaction avait \emph{peut-être} été excessive. Après s'être remis du choc initial, le bon sens de Harry s'était réveillé et avait émis l'hypothèse que “Fourchelangue” n'était probablement qu'une interface linguistique permettant de contrôler les serpents…

… après tout, les serpents ne pouvaient pas \emph{vraiment} être aussi intelligents que les humains, \emph{quelqu'un} aurait fini par s'en rendre compte. Le plus petit cerveau permettant une quelconque capacité linguistique dont Harry avait jamais entendu parler était celui du perroquet gris d'Afrique, entraîné par Irene Pepperberg. Et c'était un protolangage sans structure, chez une espèce qui jouait à de complexes jeux adultérins et qui devait donc être capable de se représenter d'autres perroquets. En revanche, selon les souvenirs de Drago, les serpents parlaient Fourchelangue dans ce qui semblait être un langage humain normal, c'est-à-dire avec une grammaire complète et syntaxiquement récursive. Il avait fallu \emph{longtemps} pour que, par le biais de l'évolution, les hominidés acquièrent cette capacité, et ils avaient des cerveaux immenses, et de fortes pressions de sélection sociale. Pour ce que Harry en savait, les serpents n'avaient pour ainsi dire pas de société. Et avec des milliers et des milliers d'espèces différentes dans le monde, comment auraient-ils tous pu utiliser la \emph{même} version de leur langage mythique~?

Bien sûr, tout cela n'était que du bon sens, auquel Harry commençait à complètement cesser de croire.

Mais il était certain d'avoir déjà entendu des serpents siffler à la télévision - après tout, il devait bien avoir entendu ce son \emph{quelque part} - et \emph{ça} ne lui avait pas semblé être un langage, ce qui avait été fort rassurant…

… au début. Le problème, c'était que Drago avait aussi affirmé que les Fourchelangue pouvaient aussi envoyer des serpents exécuter des missions complexes. Et si c'était vrai, alors il fallait que les Fourchelangue rendent les serpents \emph{intelligents de façon permanente} en leur parlant. Dans le pire des cas, cela rendrai le serpent conscient de sa propre existence, comme ce que Harry avait accidentellement fait au Choixpeau.

Et quand Harry avait offert \emph{cette} hypothèse, Drago avait prétendu se souvenir d'une histoire - Harry priait Cthulhu pour que \emph{celle-ci} ne soit qu'un compte de fée, elle y ressemblait, mais l'histoire \emph{existait} quand même - disant que Salazar Serpentard avait envoyé une jeune et brave vipère \emph{collecter des informations auprès d'autres serpents.}

Si un serpent auquel un Fourchelangue avait parlé pouvait rendre d'\emph{autres} serpents conscients de leur propre existence en \emph{leur} parlant, alors…

Alors…

Harry ne savait même pas pourquoi son cerveau continuait de dire <<~alors… alors…~>> puisqu'il savait parfaitement bien comment la progression exponentielle fonctionnerait, c'était juste que l'horreur morale profonde de la situation bloquait son esprit.

Et si quelqu'un avait inventé un sort similaire permettant de parler aux vaches~?

Et s'il y avait des Volaillelangue~?

Ou tant qu'on y était…

Harry se figea sous l'effet de la compréhension soudaine, alors même qu'une pleine fourchetée de carottes était sur le point d'entrer dans sa bouche.

\emph{C'était impossible, ça ne pouvait pas être vrai, certainement qu'aucun sorcier ne serait assez stupide pour faire ÇA…}

Et il le sut, alors que naissait une sensation morbide au fond de son estomac. \emph{Bien sûr} que quelqu'un serait assez stupide. Salazar Serpentard n'avait probablement jamais considéré les implications morales de l'intelligence reptilienne, pas même pendant une seconde, tout comme ça ne lui était même pas venu à l'esprit que les \emph{nés-Moldus} puissent être assez intelligents pour mériter d'avoir des droits. La plupart des gens ne remarquaient même pas qu'un problème moral existait tant qu'on ne le leur avait pas directement montré…

<<~Harry~?~>> dit Terry, à côté de lui, comme s'il avait peur d'être sur le point de regretter d'avoir posé la question.

<<~Pourquoi est-ce que tu regardes ta fourchette comme ça~?

--- Je commence à penser que la magie devrait être illégale, dit Harry. Au fait, est-ce que tu as déjà entendu des histoires au sujet de sorciers capables de parler aux plantes~?~>>

\later

Terry n'avait jamais rien entendu de cet acabit.

Pas plus qu'aucun des autres Serdaigle de septième année auxquels Harry avait posé la question.

Il revint donc à sa place mais ne s'assit pas immédiatement, regardant son assiette de légumes avec un air désespéré. Il avait de plus en plus faim et allait, plus tard dans la journée, rendre visite à la Chambre de Marie pour y déguster l'un de leurs plats incroyablement délicieux… la tentation de revenir à ses habitudes alimentaires de la veille et à ne plus y penser commençait à être douloureuse.

\emph{Tu dois manger quelque chose}, dit son Serpentard intérieur. \emph{Et la probabilité qu'un sorcier ait saupoudré un peu de conscience de soi sur des volailles n'est pas si élevée que} ça \emph{comparée à la probabilité qu'il l'ait fait sur des plantes, alors, puisque de toute façon tu en es à manger de la nourriture dont la sentience est soupçonnable, pourquoi ne pas manger ces délicieuses tranches de Dirico grillées dans l'huile~?}

\emph{Je ne suis pas sûr que ce soit de la bonne logique utilitariste, là -}

\emph{Oh, tu veux de la logique utilitariste~? Eh bien en voilà une bonne part~: même au cas peu probable où un abruti} aurait \emph{réussi à conférer la sentience aux poulets, c'est} tes \emph{recherches qui ont le plus de chances de le découvrir et de faire quelque chose à ce sujet. Si tu peux finir ton travail, même un peu plus vite, en ne chamboulant} pas \emph{ton régime alimentaire, alors, aussi contre-intuitif que ce soit, la} meilleure \emph{chose que tu puisses faire pour sauver le plus grand nombre possible de je-ne-sais-quoi possiblement sentients, c'est de ne} pas \emph{perdre ton temps à conjecturer follement sur la bonne marche à suivre. Ce n'est pas comme si les elfes de maison n'avaient pas déjà préparé la nourriture, que tu la mettes dans ton assiette ou pas.}

Harry considéra cela pendant un moment. C'était un raisonnement séduisant -

\emph{Bien~!} dit Serpentard. \emph{Je suis heureux que tu voies maintenant que le choix moral est de sacrifier la vie d'êtres sentients pour ton confort personnel, pour nourrir tes appétits morbides, pour le plaisir ancestral de les déchirer de tes dents -}

\emph{Quoi~?} pensa Harry avec indignation. \emph{Tu es dans quel camp, là~?}

La voix de son Serpentard mental était sinistre. \emph{Toi aussi tu ouvriras un jour les bras à la doctrine selon laquelle… la faim justifie les moyens.} Ceci fut suivit par une sorte de rire sarcastique mental.

Depuis que Harry avait commencé à s'inquiéter de la possibilité que les plantes soient sentientes, ses composants non-Serdaigle avaient eu du mal à prendre ses doutes moraux au sérieux. Poufsouffle criait \emph{Cannibalisme~!} à chaque fois que Harry essayait de penser à quelque chose de comestible, et Gryffondor visualisait cette nourriture en train de crier, même s'il s'agissait d'un sandwich -

\emph{Cannibalisme~!}

\emph{AAAIIEEEEEE NE ME MANGE PAS -}

\emph{Ignore les cris, mange-le quand même~! C'est le genre de situation où il faut compromettre son éthique au service de ses idéaux,} tout le monde \emph{pense que c'est normal de manger des sandwiches alors tu ne peux pas utiliser ta rationalisation habituelle au sujet de la faible probabilité d'un grand désavantage dans le cas où tu serais pris -}

Harry laissa échapper un soupir mental et pensa~: \emph{Tant ça ne te dérange pas que} nous \emph{soyons dévorés par des monstres géants qui n'ont pas assez faits de recherches pour découvrir si on était sentients ou pas.}

\emph{Ça ne me dérange pas}, dit Serpentard. \emph{Ça ne dérange personne~?} Hochements de tête mentaux. \emph{Génial, on peut en revenir aux tranches de Dirico plongées dans l'huile bouillante maintenant~?}

\emph{Pas avant que j'ai fait quelques recherches pour découvrir ce qui est sentient et ce qui ne l'est pas. Maintenant la ferme.} Et Harry se détourna d'un geste volontaire de son assiette pleine de légumes si tentants avant de se diriger vers la bibliothèque -

\emph{Mange juste les élèves}, dit Poufsouffle. \emph{Il n'y a aucun doute quant à} leur \emph{sentience.}

\emph{Tu sais que tu en as envie,} dit Gryffondor. \emph{Je parie que les plus jeunes sont les plus goûtus.}

Harry commençait à se demander si le Détraqueur n'était pas parvenu à endommager ses personnalités imaginaires.

\later

<<~\emph{Franchement}~>>, dit Hermione d'une voix acerbe tandis que son regard scannait les rayons de Botanique de la bibliothèque de Poudlard. Harry lui avait laissé un message lui demandant si elle pouvait se rendre à la bibliothèque après le petit déjeuner que lui-même avait sauté~; mais lorsque Harry avait introduit le sujet du jour, elle avait semblé un peu décontenancée. <<~Tu sais ce que c'est ton problème, Harry~? Tu n'a aucun sens des priorités. Une idée te vient et tu dois tout de suite la poursuivre.

--- J'ai un \emph{excellent} sens des priorités~>>, dit Harry. Sa main s'avança et il attrapa \emph{Rouerie végétale}, de Casey McNama, puis il commença à faire défiler les premières pages, à la recherche de la table des matières. <<~C'est pourquoi je veux découvrir si les plantes peuvent parler \emph{avant} de manger mes carottes.

--- Tu ne penses pas qu'on a peut-être des choses plus \emph{importantes} à faire~?~>>

\emph{On dirait Drago}, pensa Harry, mais il ne le dit évidemment pas. Ce qu'il dit fut~: <<~Qu'est qui \emph{pourrait} être plus importante que la découverte que les plantes sont sentientes~?~>>

Il y eut un silence éloquent, et il baissa les yeux sur la table des matières. Il y avait bien un chapitre sur le langage des plantes, ce qui fit manquer un battement à son cœur~; et ses mains commencèrent alors à rapidement tourner les pages, en direction du bon numéro.

<<~Parfois, dit Hermione Granger, je n'ai vraiment, mais alors vraiment aucune idée de ce qui se passe dans ta tête.

--- Écoute, c'est une question de multiplication. Il y a \emph{beaucoup} de plantes dans le monde, et si elles ne sont \emph{pas} sentientes alors elles n'ont pas d'importance, mais si les plantes \emph{sont} des gens, alors cela leur donne plus de poids moral que toute l'humanité réunie. Bien sûr, ton cerveau ne s'en rend pas intuitivement compte, mais c'est parce que le cerveau ne sait pas multiplier. Par exemple si tu demandes à trois groupes de maisonnées canadiennes ce qu'elles seraient prêtes à payer pour sauver deux-mille, vingt-mille ou deux-cents-mille oiseaux d'une mort par marée noire, les trois groupes répondront qu'ils sont respectivement prêts à payer soixante dix huit, quatre-vingt huit et quatre-vingt dollars. En d'autres termes, aucune différence. Ça s'appelle l'insensibilité à l'échelle. Ton cerveau imagine un seul oiseau luttant dans une mare de pétrole, et cette image créée une quantité donnée d'émotion qui détermine ce que tu es prête à payer. Mais personne ne peut visualiser ne serait-ce que deux-mille “quelque chose”, alors la \emph{quantité} est immédiatement évacuée. Essaie maintenant de \emph{corriger} ce biais pour \emph{cent billions} de brins d'herbe sentients, et tu comprendras qu'ils pourraient avoir des milliers de fois plus d'importance que celle que nous accordions à l'espèce humaine… oh, Azathot merci, ça dit qu'il ne s'agit que de quelques plantes magiques capables de parler et qu'elles parlent le langage humain à voix haute, pas qu'il existe un sort permettant de parler à \emph{n'importe quelle} plante -

--- Ron est venu me voir au petit déjeuner hier matin~>>, dit Hermione. Sa voix était à présent un peu étouffée, un peu triste, et peut-être même un peu effrayée. <<~Il a dit que te voir m'embrasser l'avait mortifié. Que ce que tu avais dit quand tu étais Détraqué aurait dû me montrer tout le mal que tu cachais à l'intérieur de toi. Et que si je comptais être la disciple d'un Mage Noir, alors il ne comptait certainement plus faire partie de mon armée.~>>

Les mains de Harry cessèrent de tourner les pages. Il semblait que le cerveau de Harry, en dépit de tout son savoir abstrait, était toujours incapable d'apprécier des différences d'échelle à un niveau véritablement émotionnel, car il venait de vigoureusement rediriger son attention, loin de billions de brins d'herbes peut-être sentients qui pouvaient être en train de souffrir, voir de mourir à l'instant même, vers la vie d'un seul être humain qui se trouvait être plus proche et lui être plus cher.

<<~Ron est le plus grand des crétins, dit Harry. Et ça ne risque pas de paraître dans les journaux, parce que ce n'est pas une nouvelle. Alors après que tu l'eus viré, combien de ses bras et de ses jambes as-tu brisés~?

--- J'ai essayé de lui dire que les choses n'étaient pas ce qu'elles semblaient être, dit Hermione de la même petite voix. j'ai essayé de lui dire que \emph{tu} n'es pas comme ça, et que ce n'était pas comme ça entre nous, mais ça n'a semblé faire que le… le renforcer dans ses convictions.

--- Eh bien oui~>>, dit Harry. Il était surpris de n'être pas plus en colère contre le capitaine Weasley, mais sa préoccupation pour Hermione semblait prendre le dessus pour le moment. <<~Plus tu essaies de te justifier auprès de ce genre de personne, plus cela entérine le fait qu'ils ont le \emph{droit} de te remettre en question. Ça montre que tu penses qu'ils peuvent être tes inquisiteurs, et une fois que tu donnes ce genre de pouvoir à quelqu'un, il ne va faire qu'insister encore et encore.~>> C'était l'une des leçons de Drago Malfoy que Harry avait trouvées assez intelligentes~: les gens qui \emph{essayaient} de se défendre se faisaient interroger sur chaque petit détail et n'arrivaient jamais à satisfaire leurs interrogateurs~; mais si vous faisiez clairement sentir dès le début que vous étiez une célébrité au-dessus des conventions sociales, l'esprit des gens ne se fatiguerait pas à prendre note de toutes les entorses aux règles. <<~C'est pourquoi quand Ron est venu \emph{me} voir, quand j'étais assis à la table Serdaigle, et qu'il m'a dit de rester loin de toi, j'ai mis ma main au-dessus du sol et j'ai dit 'Tu vois la hauteur où j'ai mis ma main~? C'est le niveau d'intelligence minimum pour avoir le droit de me parler.' Et alors il m'a accusé de, je cite, t'aspirer dans les ténèbres, fin de citation, alors j'ai resserré mes lèvres et j'ai fait \emph{shluuuuurp} et après ça sa bouche faisait encore ces bruits en forme de parole alors j'ai lancé un Silencio. Je ne pense pas qu'il essaiera à nouveau de me faire la morale.

--- Je comprends pourquoi tu as fait ça, dit Hermione d'une voix pincée, je \emph{voulais} le renvoyer, moi aussi, mais j'aurais vraiment aimé que tu ne le fasses pas, ça va rendre les choses plus difficiles pour \emph{moi}, Harry~!~>>

Ce dernier releva de nouveau les yeux de \emph{Roueries Végétales}, de toute façon il n'avançait pas dans sa lecture~; et il vit que Hermione lisait toujours son livre, qu'elle ne le regardait pas. Alors même qu'il regardait, les mains de cette dernière tournèrent une autre page.

<<~Je pense que tu as choisi une mauvaise approche en essayant de te défendre, dit Harry. Vraiment. Tu es qui tu es. Tu es l'amie de qui tu veux. Dis à tous ceux qui questionnent cela de mettre leurs questions où je pense.~>>

Hermione se contenta de secouer la tête, et elle tourna une autre page.

<<~Option deux, dit Harry. Vas voir Fred et George et dis leur d'avoir une petite conversation avec leur frère difficile, \emph{ces deux-là} sont vraiment des mecs bien -

--- Il ne s'agit pas que de Ron, dit Hermione presque dans un murmure. Plein de gens disent ça, Harry. Même Mandy me regarde d'un air inquiet quand elle pense que je ne regarde pas. C'est drôle, non~? Je n'arrête pas de m'inquiéter à l'idée que le professeur Quirrell puisse t'aspirer dans les ténèbres, et maintenant les gens me mettent en garde exactement comme j'essaie de le faire pour toi.

--- Ben \emph{ouais}, dit Harry. Est-ce que ça ne te rassure pas un peu sur le professeur Quirrell et moi~?

--- En un mot, dit Hermione, non.~>>

Il y eut un silence qui dura assez pour que Hermione tourne une autre page, puis sa voix, un vrai chuchotement cette fois-ci,

<<~et, et Padma court partout en disant que, que puisque je n'ai pas pu lancer le P-Patronus, ça veut dire que je fais seulement s-semblant d'être g-gentille…

--- Padma n'a même pas \emph{essayé}~! dit Harry d'un ton indigné. Si tu étais une Mage Noire qui faisait seulement semblant, tu n'aurais pas \emph{essayé} devant tout le monde, est-ce qu'ils pensent que tu es \emph{stupide}~?~>>

Hermione sourit un peu et cligna plusieurs fois des yeux.

<<~Hé, \emph{moi} je dois m'inquiéter de la possibilité que je devienne \emph{vraiment} maléfique. \emph{Là}, le scénario catastrophe, c'est que les gens pensent que tu es plus maléfique que tu ne l'es en réalité. Est-ce que ça va te tuer~? Je veux dire, est-ce que c'est si \emph{grave} que ça~?~>>

La jeune fille hocha la tête, le visage renfrogné.

<<~Écoute Hermione… si tu t'inquiètes tant de ce que les gens pensent, si tu es malheureuse à chaque fois que les gens ne te voient pas exactement de la même façon que tu te vois toi-même, alors tu t'es \emph{déjà} condamnée au malheur. Personne ne nous voit jamais comme nous nous voyons nous-même.

--- Je ne sais pas comment t'expliquer, dit Hermione d'une voix douce. Je ne sais pas si c'est quelque chose que tu pourras un jour comprendre. Tout ce que je peux te dire c'est~: comment te sentirais-tu si \emph{je} pensais que tu étais maléfique~?

--- Euh…~>> Harry le visualisa. <<~Oui, ça me \emph{ferait} mal. Beaucoup. Mais tu es quelqu'un de bien qui pense intelligemment à ce genre de choses, tu as \emph{mérité} ce pouvoir sur moi, ça \emph{voudrait dire} quelque chose si tu pensais que je m'étais égaré. Je n'arrive pas à imaginer un seul élève, à part toi, dont l'opinion m'importerait autant -

--- Tu peux vivre comme ça, chuchota Hermione Granger. Moi pas.~>>

La fille avait lu trois autres pages en silence, et Harry était revenu à son livre, essayant de garder sa concentration, quand Hermione dit enfin, d'une petite voix~:

<<~Es-tu vraiment certain que je ne dois pas savoir comment lancer le Patronus~?

--- Je…~>> Harry dut avaler une boule dans sa gorge. Il se vit soudain, lui, ne sachant \emph{pas} pourquoi il n'arrivait pas à lancer le Patronus, ne pouvant \emph{pas} le montrer à Drago, se faisant dire qu'il y avait une raison, ne sachant rien de plus. <<~Hermione, ton Patronus brillerait de la même lumière mais il ne serait pas \emph{normal}, il ne ressemblerait pas à ce que les gens attendent d'un Patronus, tous ceux qui le verraient sauraient que ce n'est pas normal. Même si je te donnais le secret tu ne pourrais pas le \emph{démontrer} à qui que ce soit, à moins de les faire regarder ailleurs, pour qu'ils puissent seulement voir la lumière, et… et la partie la plus importante d'un secret, c'est de savoir que le secret existe, tu pourrais seulement le montrer à un ou deux amis, si tu les faisais promettre…~>> La voix de Harry resta en suspens.

<<~J'accepte.~>> Sa voix était encore étouffée.

Harry eut beaucoup de mal à ne pas lui dire le secret, ici, dans la bibliothèque.

<<~Je, je ne devrais pas, je ne devrais \emph{vraiment} pas, c'est \emph{dangereux}, Hermione, ça pourrait causer beaucoup de dommages si le secret était éventé~! Tu n'as pas entendu le dicton, trois personnes peuvent garder un secret si deux d'entre elles sont mortes~? Qu'essayer de le dire seulement à tes amis les plus proches, c'est comme de le dire à tout le monde, parce que tu ne fais pas confiance qu'à eux, tu fais aussi confiance à tous ceux en qui ils ont confiance~? C'est trop important, le risque est trop grand, ce n'est pas le genre de décision qui devrait être prise dans le but d'arranger la réputation de quelqu'un à l'école~!

--- D'accord~>>, dit Hermione. Elle ferma le livre et le remit sur l'étagère. <<~Je ne peux pas me concentrer Harry, je suis désolée.

--- S'il y a \emph{quoi que ce soit} d'autre que je puisse faire -

--- Sois plus gentil avec tout le monde.~>>

Elle ne se retourna pas en quittant la bibliothèque, ce qui fut peut-être une bonne chose, parce que le garçon était figé sur place, immobile.

Après un moment, le garçon recommença à tourner les pages. 

%  LocalWords:  Pepperberg Poultrymouths whats Aiiieeee McNamara
%  LocalWords:  schluuuuurp
