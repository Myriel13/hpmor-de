\chapter{Remarquer que l'on est confus}

\lettrine{L}{es} heures de bureau du professeur Quirrell étaient le jeudi de 11h40 à 11h55. Pour tous ses élèves de toutes les années. Se contenter de frapper à la porte coûtait un point Quirrell, et s'il pensait que votre raison de venir le voir ne justifiait pas que vous l'ayez dérangé, vous en perdiez cinquante de plus.

Harry frappa à la porte.

Il y eut une pause. Puis une voix mordante dit~: <<~J'imagine qu'au point où vous en êtes, M. Potter, vous pouvez aussi bien entrer.~>>

Et avant que Harry ne puisse toucher la poignée, la porte s'ouvrit avec fracas, frappant le mur dans un dur bruit de craquement, comme si quelque chose avait cassé le bois, ou la pierre, ou les deux.

Le professeur Quirrell était enfoncé dans sa chaise et lisait un livre à l'air étrangement ancien, doté d'une reliure d'un cuir bleu nuit avec des runes d'argent sur la tranche. Ses yeux n'avaient pas quitté les pages. <<~Je ne suis pas de bonne humeur, M. Potter. Et quand je ne suis pas de bonne humeur, je ne suis pas quelqu'un d'agréable à côtoyer. Pour votre bien, faites rapidement ce que vous êtes venu faire, puis partez.~>>

Un courant froid filtra au travers de la pièce, comme si elle avait contenu quelque chose qui projetait de la noirceur de la même façon que les lampes projettent de la lumière, et que ce quelque chose n'avait pas été totalement occulté.

Harry était un peu pris de court. \emph{Pas de bonne humeur} était un sacré euphémisme. Qu'est-ce qui pouvait autant embêter le professeur Quirrell~?

Eh bien, on ne laissait pas tomber ses amis quand ils ne se sentaient pas bien. Harry s'avança précautionneusement à l'intérieur la pièce.

<<~Y a-t-il quelque chose que je puisse faire pour aider -

--- Non~>>, dit le professeur Quirrell, ne levant toujours pas les yeux.

<<~Je veux dire, si vous avez eu affaire à des idiots et que vous voulez avoir quelqu'un à qui parler…~>>

Il y eut un silence étonnamment long.

Le professeur Quirrell referma le livre avec force et celui-ci disparut dans un petit chuintement. Il leva alors les yeux, et Harry tressaillit.

<<~J'imagine qu'une discussion intelligente \emph{me} ferait du bien en ce moment~>>, dit le professeur Quirrell du même ton mordant qu'avec lequel il avait invité Harry à entrer. <<~Ce ne sera probablement pas le cas pour \emph{vous}, soyez-en prévenu.~>>

Harry prit une profonde inspiration. <<~Je promets de ne pas vous en vouloir si vous vous énervez. Qu'est-ce qui s'est passé~?~>>

Le froid dans la pièce sembla s'intensifier. <<~Un Gryffondor en sixième année a jeté un sort à l'un de mes élèves les plus prometteurs, un Serpentard en sixième année.~>>

Harry avala sa salive. <<~Quelle… sorte de sort~?~>>

Et la furie qui habitait le visage du professeur Quirrell ne fut plus contenue.

<<~Pourquoi prendre la peine de poser une question sans importance~? Notre ami le Gryffondor de sixième année ne pensait pas que c'était important~!

--- Vous êtes \emph{sérieux}~? dit Harry avant de pouvoir s'en empêcher.

--- Non, je suis d'une humeur particulièrement mauvaise sans raison particulière. \emph{Oui je suis sérieux, imbécile}~! Il ne savait pas. Il ne \emph{savait vraiment pas}. J'ai refusé d'y croire jusqu'à ce que les Aurors le confirment avec du Veritaserum. Il est dans sa \emph{sixième année à Poudlard} et il a jeté un sort noir de haut niveau \emph{sans connaître son effet}.

--- Vous ne voulez pas dire, dit Harry, qu'il se \emph{trompait} au sujet de son effet, qu'il avait malencontreusement lu la mauvaise description -

--- Tout ce qu'il savait, c'était que le sort était censé être dirigé vers un ennemi. Il \emph{savait} que c'était tout ce qu'il savait.~>>

Et ça lui avait suffi pour jeter le sort.

<<~Je ne comprends pas comment quelque chose doté d'un si petit cerveau peut marcher droit.

--- En effet, M. Potter~>>, dit le professeur Quirrell.

Il y eut un silence. Le professeur Quirrell se pencha en avant et prit l'encrier argenté posé sur son bureau, le tournant entre ses mains, le regardant comme s'il se demandait comment il pourrait s'y prendre pour torturer un encrier à mort.

<<~Le Serpentard en sixième année a-t-il été sévèrement blessé~? dit Harry.

---Oui.

--- Le Gryffondor en sixième année a-t-il été élevé par des Moldus~?

--- \emph{Oui}.

--- Dumbledore refuse-t-il de le renvoyer parce que le pauvre garçon ne savait pas~?~>>

Les mains du professeur Quirrell blanchirent autour de l'encrier.

<<~\emph{Voulez-vous en venir à quelque chose, M. Potter, ou énoncez-vous simplement l'évidence~?}

--- Professeur Quirrell, dit gravement Harry, tous les étudiants élevés-Moldu de Poudlard ont besoin d'un cours sur la sécurité lors duquel on leur dira des choses tellement incroyablement évidentes qu'aucun né-sorcier ne songerait à les mentionner. Ne jetez pas de sorts si vous ne connaissez pas leur effet, ne préparez aucune potion de haut niveau sans supervision hors d'un laboratoire, la raison pour laquelle il y a des lois contre la magie chez les mineurs, tous les fondamentaux.

--- Pourquoi~? dit le professeur Quirrell. Laissons les imbéciles mourir avant qu'ils ne se reproduisent.

--- Si ça ne vous dérange pas de perdre quelques Serpentard de sixième année avec eux.~>>

L'encrier prit feu dans les mains du professeur Quirrell et brûla avec une terrible lenteur, de hideuses flammes noir-orange déchirant le métal et semblant en arracher de petites parties, l'argent se tordant alors qu'il fondait, comme s'il essayait de s'échapper sans toutefois y parvenir. Il y eut un petit bruit aigu, comme si le métal avait hurlé.

<<~J'imagine que vous avez raison, dit le professeur Quirrell avec un sourire résigné. Je mettrai au point un cours pour m'assurer que les nés-Moldus qui sont trop bêtes pour vivre n'emporteront jamais personne de valeur avec eux.~>>

L'encrier continua de brûler et de hurler dans les mains du professeur Quirrell, de petites gouttelettes de métal, toujours en feu, dégoulinant sur le bureau, comme si l'encrier pleurait.

<<~Vous ne vous enfuyez pas~>>, observa le professeur Quirrell.

Harry ouvrit la bouche -

<<~Si vous êtes sur le point de dire que vous n'avez pas peur de moi, dit le professeur Quirrell, \emph{évitez}.

--- Vous êtes la personne la plus effrayante que je connaisse, dit Harry, et l'une des raisons principale à cela est le contrôle que vous avez sur vous-même. Je ne peux tout simplement pas vous imaginer faisant du mal à quelqu'un que vous n'auriez pas délibérément décidé de faire souffrir.~>>

Le feu dans les mains du professeur Quirrell s'éteignit, et il reposa l'encrier détruit sur le bureau avec précaution. <<~Vous avez des paroles des plus agréables, M. Potter. Avez-vous pris des leçons de flatterie~? Auprès de M. Malfoy peut-être~?~>>

Harry garda une expression neutre et se rendit compte une seconde trop tard qu'il aurait aussi bien pu signer sa confession. Le professeur Quirrell se fichait de votre expression, c'était l'état d'esprit la rendant probable qui lui importait.

<<~Je vois, dit le professeur Quirrell. M. Malfoy est un ami utile à avoir, M. Potter, et il a beaucoup à vous apprendre, mais j'espère que vous n'avez pas fait l'erreur de lui témoigner votre confiance par de trop grandes confidences.

--- Il ne sait rien que je ne souhaite être découvert, dit Harry.

--- Bon travail~>>, dit le professeur Quirrell, souriant légèrement. <<~Et pourquoi êtes-vous initialement venu ici~?

--- Je crois en avoir fini avec les exercices d'Occlumancie préliminaires et être prêt pour le précepteur.~>>

Le professeur Quirrell hocha la tête. <<~Je vous conduirai à Gringotts ce dimanche.~>> Il s'interrompit, regarda Harry, et sourit. <<~Et nous pourrions même en faire une petite sortie, si vous le souhaitez. Je viens d'avoir une plaisante idée.~>>

Harry hocha la tête, souriant en retour.

En sortant du bureau, il entendit le professeur Quirrell fredonner une petite mélodie.

Harry était heureux d'avoir pu lui remonter le moral.

\later

Il semblait y avoir un grand nombre de personne chuchotant dans les couloirs ce dimanche, du moins quand Harry Potter les croisait.

Et beaucoup de doigts pointés.

Et beaucoup de gloussements féminins.

Ça avait commencé au petit déjeuner, quand quelqu'un avait demandé à Harry s'il avait entendu la nouvelle, et Harry l'avait rapidement interrompu et avait dit que si l'information avait été écrite par Rita Skeeter, alors il ne voulait pas en \emph{entendre} parler, il voulait la lire dans le journal lui-même.

Il s'était ensuite avéré que peu d'élèves de Poudlard achetaient des exemplaires de la Gazette du sorcier et que les exemplaires qui n'avaient pas déjà été rachetés à leur propriétaire passaient de main en main selon une espèce d'ordre compliqué et que personne ne savait vraiment qui en avait un pour l'instant…

Harry avait alors utilisé un sort de \emph{Sourdinam} et était allé prendre son petit déjeuner, comptant sur ses voisins pour éloigner les nombreux, nombreux curieux, et faisant de son mieux pour ignorer l'incrédulité, les rires, les sourires de félicitation, les regards emplis de pitié, les coups d'œils apeurés et les assiettes laissées tombées au sol quand de nouvelles personnes descendaient pour le petit déjeuner et entendaient la nouvelle.

Harry se sentait \emph{carrément} curieux, mais ça ne l'aurait \emph{vraiment} pas fait de gâcher cette œuvre d'art en l'entendant de seconde main.

Il avait fini ses devoirs à l'abri dans sa malle pendant les deux heures suivantes, après avoir dit à ses camarades de dortoir de venir le chercher si quiconque trouvait un exemplaire du journal.

À 10h, Harry ne savait toujours pas ce qui s'était passé, et il quittait Poudlard en attelage avec le professeur Quirrell, qui était à la place avant-droite pour l'instant effondré en mode zombie. Harry était assis en diagonale de lui, au fond à gauche, aussi loin du professeur que l'attelage le permettait. Même ainsi, Harry avait en permanence la sensation d'une catastrophe imminente tandis que l'attelage avançait dans un bruit d'entrechocs sur un petit chemin à travers une partie de la forêt non-interdite. Ce qui rendait la lecture un peu difficile, surtout que le sujet était ardu, et Harry souhaita soudain pouvoir lire les livres de science-fiction de son enfance -

<<~Nous sommes hors du domaine de Poudlard, M. Potter,~>> dit la voix du professeur Quirrell, en face de lui. <<~Temps d'y aller.~>>

Le professeur Quirrell débarqua avec précaution de l'attelage, s'agrippant au moment de descendre la marche. Harry, de son côté, sauta hors du véhicule.

Harry se demandait comment ils allaient se rendre à leur destination quand le professeur Quirrell dit <<~Attrapez~!~>> et lui jeta une Noise de bronze, et Harry l'attrapa sans réfléchir.

Un crochet géant intangible saisit Harry par l'abdomen et le projeta durement en arrière, mais sans créer la moindre sensation d'accélération, et un instant plus tard Harry se tenait au milieu du Chemin de Traverse.

(\emph{Euh pardon, quoi~?} dit son cerveau.)

(\emph{On vient de se téléporter}, expliqua Harry.)

(\emph{Ça n'arrivait tout simplement jamais dans l'environnement ancestral}, se plaignit son cerveau, et il le désorienta.)

Harry chancela tandis que ses pieds s'adaptaient au sol de brique qui avait remplacé la terre du chemin forestier qu'ils avaient été en train de traverser. Il se redressa, toujours étourdi, les sorciers et sorcières affairés semblant tanguer légèrement, et les cris des marchands semblant se déplacer alors que son cerveau essayait de mettre en place le monde où ils seraient positionnés.

Quelques instants plus tard, il y eut une sorte de bruit de succion-éclatement venu de quelques pas derrière Harry, et quand il se retourna, le professeur Quirrell était là.

<<~Ça vous dérangerait si…~>> dit Harry en même temps que le professeur Quirrell, <<~j'ai bien peur de devoir…~>>

Harry s'interrompit, au contraire du professeur Quirrell.

<<~ …aller mettre quelque chose en mouvement, M. Potter. Comme il m'a été longuement expliqué que je suis responsable de tout ce qui vous arrive, quoi que ce fût, je vous laisserai avec…

--- Kiosque, dit Harry.

--- Pardon~?

--- Ou n'importe quel endroit où je pourrai acheter un exemplaire de la Gazette du sorcier. Laissez-moi là et je serai heureux.~>>

Peu de temps après, Harry avait été livré à une librairie, accompagné de plusieurs menaces ambiguës doucement murmurées. Et le marchand avait reçu des menaces \emph{moins} ambiguës, vu la façon dont il avait fait la grimace, et ses yeux dardaient maintenant de Harry à la porte d'entrée.

Si la librairie se mettait à brûler, Harry allait rester ici au milieu du feu jusqu'à ce que le professeur Quirrell revienne.

En attendant -

Harry jeta un rapide coup d'œil autour de lui.

La librairie semblait plutôt petite et délabrée, avec seulement quatre rangées d'étagères visibles, et l'étagère vers laquelle avaient bondi les yeux de Harry semblait porter uniquement de petits livres de mauvaise qualité, avec des titres sinistres tel que \emph{Le massacre d'Albanie au quinzième siècle}.

Chaque chose en son temps. Harry marcha jusqu'au comptoir.

<<~Excusez-moi, dit Harry, un exemplaire de la Gazette du sorcier, s'il vous plaît.

--- Cinq Mornilles, dit le marchand. Désolé petit, je n'en ai plus que trois.~>>

Cinq Mornilles tombèrent sur le comptoir. Harry avait l'impression qu'il aurait pu négocier deux ou trois Mornilles de moins, mais au point où il en était, il s'en fichait un peu.

Les yeux du marchand s'écarquillèrent et il sembla qu'il venait de vraiment remarquer Harry~:

<<~\emph{Vous~!}

--- \emph{Moi~!}

--- Est-ce \emph{vrai}~? Êtes-vous \emph{vraiment} -

--- \emph{La ferme}~! Désolé, j'ai attendu \emph{toute la journée} pour pouvoir lire la nouvelle dans le journal au lieu de l'entendre de seconde main, alors s'il vous plaît, \emph{donnez-le-moi}, d'accord~?~>>

Le marchand regarda Harry un moment, puis il se pencha silencieusement sous son comptoir et lui passa un exemplaire plié du Gazette du sorcier.

Le gros titre disait~:
\headline{Harry Potter\\
secrètement fiancé\\
à Ginevra Weasley}

Harry regarda les mots.

Il leva le journal au-dessus du comptoir, doucement, avec révérence, comme s'il tenait une œuvre originale d'Escher, et le déplia pour lire…

… la preuve qui avait convaincu Rita Skeeter.

… et d'autres détails intéressants.

… et encore plus de preuves.

Fred et George devaient sûrement avoir obtenu l'accord de leur sœur avant, non~? Oui, bien sûr qu'ils avaient obtenu son accord. Il y avait une image de Ginevra Weasley soupirant avec nostalgie en regardant ce que Harry pu constater, en regardant de près, être un photo de lui. Ça devait avoir été mis en scène.

Mais \emph{comment…}~?

Harry était assis dans une chaise pliante de mauvaise qualité, relisant le journal pour la quatrième fois quand la porte fit un doux murmure et le professeur Quirrell revint dans le magasin.

<<~Mes excuses pour -- nom de Merlin, \emph{qu'est-ce} que vous lisez~?

--- Il semblerait~>>, dit Harry, la voix pleine d'estime, <<~qu'un certain M. Arthur Weasley a été victime d'un sort d'Imperium jeté par un Mangemort que mon père a tué, créant ainsi une dette envers la Noble Maison Potter, dont mon père a exigé le paiement par la main de la récemment née Ginevra Weasley. Les gens font-ils vraiment ce genre de chose par ici~?

--- Comment mademoiselle Skeeter aurait-elle \emph{pu} être assez idiote pour croire…~>>

Et la voix du professeur Quirrell s'interrompit.

Harry avait lu le journal en le tenant déplié et vertical, ce qui voulait dire que le professeur Quirrell, de là où il se tenait, pouvait lire le texte sous le gros titre.

L'air stupéfait sur le visage du professeur Quirrell était une œuvre d'art qui valait presque le journal lui-même.

<<~Ne vous en faites pas, dit joyeusement Harry, tout est truqué.~>>

Ailleurs dans le magasin, il entendit le marchand manquer d'air. Il y eut le bruit d'une pile de livres qui s'effondrait.

<<~M. Potter, dit lentement le professeur Quirrell, en êtes-vous \emph{certain}~?

--- Assez certain. Pouvons-nous y aller~?~>>

Le professeur Quirrell hocha la tête d'un air assez absent et Harry replia le journal puis le suivit à l'extérieur.

Pour une raison ou une autre, Harry avait l'impression de ne plus entendre le moindre bruit venant de la rue environnante.

Ils marchèrent en silence pendant trente seconde avant que le professeur Quirrell ne parle.

<<~Mademoiselle Skeeter a vu le compte-rendu original de la session confidentielle du Magenmagot.

--- Oui.

--- Les \emph{comptes-rendus originaux du Magenmagot}.

--- Oui.

--- \emph{J'}aurais du mal à faire ça.

--- Vraiment~? dit Harry. Parce que si mes soupçons sont fondés, ça a été fait par un élève de Poudlard.

--- C'est plus qu'impossible, dit catégoriquement le professeur Quirrell. M. Potter… j'ai le regret de vous informer que cette jeune demoiselle s'attend à ce que vous l'épousiez.

--- Mais \emph{ceci} est improbable, dit Harry. Pour citer Douglas Adams, l'impossible a souvent une forme d'intégrité qui manque au simple improbable.

--- Je vois ce que vous voulez dire, dit lentement le professeur Quirrell. Mais… non, M. Potter. C'est peut-être impossible, mais je peux néanmoins \emph{imaginer} une falsification des comptes-rendus du Magenmagot. Il est en revanche \emph{inimaginable} que le Grand Manager de Gringotts appose le sceau de sa fonction en témoignage d'un faux contrat de fiançailles, et mademoiselle Skeeter a personnellement vérifié ce sceau.

--- En effet, dit Harry, il faudrait s'attendre à ce que le Grand Manager de Gringotts soit impliqué, avec autant d'argent passant d'une main à l'autre. Il semblerait que M. Weasley était grandement endetté, et qu'il a donc demandé un paiement supplémentaire de dix-mille Gallions -

--- \emph{Dix-mille} Gallions pour un \emph{Weasley}~? Vous pourriez acheter la fille d'une Maison Noble avec ça~!

--- Excusez-moi, dit Harry. Maintenant je dois vraiment vous poser la question~: les gens font-ils réellement ce genre de choses par ici-

--- Rarement, dit le professeur Quirrell en fronçant les sourcils. Et plus du tout, je pense, depuis que le Seigneur des Ténèbres s'en est allé. J'imagine que le journal prétend que votre père a juste payé la somme~?

--- Il n'avait pas le choix, dit Harry. Pas s'il voulait remplir les conditions de la prophétie.

--- \emph{Donnez-moi ça}~>>, dit le professeur Quirrell, et le journal bondit hors des mains de Harry si vite que la tranche du papier le coupa.

Harry plaça automatiquement son doigt dans sa bouche pour le sucer, plutôt choqué, et il se tourna pour faire des remontrances au professeur Quirrell -

Le professeur Quirrell s'était arrêté net au milieu de la rue, et ses yeux bondissaient rapidement de gauche à droite tandis qu'une force invisible tenait le journal suspendu devant lui.

Harry regarda, bouche bée, ressentant un mélange de crainte et de vénération tandis que le journal s'ouvrait pour révéler les pages deux et trois. Et peu après, quatre et cinq. C'était comme si l'homme avait abandonné toute prétention au statut de simple mortel.

Et après un temps si court que c'en était troublant, le journal se replia proprement de lui-même. Le professeur Quirrell l'attrapa dans l'air et le jeta à Harry, qui le saisit par pur réflexe~; et le professeur Quirrell commença de nouveau à marcher, et Harry le suivit automatiquement en traînant les pieds.

<<~Non, dit le professeur Quirrell, cette prophétie ne me semble pas très vraisemblable à moi non plus.~>>

Harry hocha la tête, toujours stupéfait.

<<~Les centaures auraient pu être placés sous un sort \emph{d'Imperium}, dit le professeur Quirrell en fronçant les sourcils, \emph{cela} semble compréhensible. Ce que la magie peut créer, la magie peut corrompre, et il n'est pas impensable que le Grand Sceau de Gringotts soit apposé par la main d'un autre. Le Polynectar aurait pu permettre de jouer le rôle des Langues-de-plomb, idem pour le prophète bavarois. Et avec \emph{assez} d'efforts, il pourrait être possible de falsifier les comptes-rendus du Magenmagot. Avez-vous la moindre idée de la façon dont cela a été fait~?

--- Je n'ai pas une seule hypothèse plausible, dit Harry. Je sais que cela a été accompli avec un budget total de quarante Gallions.~>>

Le professeur Quirrell s'arrêta net et fit volte-face vers Harry. Son visage arborait maintenant une incrédulité totale. <<~Quarante Gallions paieront un briseur de barrière compétent pour percer un chemin dans une maison que vous souhaitez cambrioler~! Quarante \emph{mille} Gallions \emph{pourraient} payer une équipe composée des plus grands criminels professionnels du monde pour falsifier les comptes-rendus du Magenmagot~!~>>

Harry haussa les épaules avec impuissance.

<<~Je m'en souviendrai la prochaine fois que je veux économiser trente-neuf-mille-neuf-cent-soixante Gallions en trouvant le le bon fournisseur.

--- Je ne dis pas cela souvent, dit le professeur Quirrell. Je suis impressionné.

--- De même, dit Harry.

--- Et qui est cet incroyable étudiant de Poudlard~?

--- J'ai peur de ne pouvoir le dire.~>>

Harry fut assez surpris quand le professeur Quirrell ne formula aucune objection à cela.

Ils marchèrent, pensifs, en direction du bâtiment de Gringotts, car aucun d'eux n'était le genre de personne qui aurait abandonné un problème sans l'avoir considéré pendant au moins cinq minutes.

<<~J'ai l'impression, finit par dire Harry, que nous n'attaquons pas le problème du bon angle. J'ai un jour entendu une histoire au sujet d'étudiants qui entrèrent dans leur cours de physique, et le professeur leur montra une large plaque de métal située près d'un feu. Elle leur ordonna de toucher la plaque, et ils sentirent que le métal proche du feu était plus froid tandis que le métal plus éloigné était plus chaud. Et elle leur dit d'essayer de deviner l'explication. Alors certains étudiants écrivirent~: “à cause de la façon dont le métal conduit la chaleur”, et d'autres~: “à cause de la façon dont l'air se déplace”, et personne ne dit~: “ça a juste l'air impossible”, et la véritable réponse était qu'avant que les étudiants n'entrent dans la pièce, le professeur avait juste retourné la plaque.

--- Intéressant, dit le professeur Quirrell. Ça me semble en effet similaire. Y a-t-il une morale~?

--- Que la force d'un rationaliste est d'être plus facilement confus par la fiction que par la réalité, dit Harry. Si on est capable d'expliquer n'importe quel phénomène, alors on ne sait rien. Les étudiants pensèrent qu'ils pouvaient utiliser des mots, comme “à cause de la conduction de la chaleur” pour expliquer n'importe quoi, même une plaque de métal plus froide du côté proche du feu. Ils ne se sont donc pas rendu compte à quel point ils étaient confus, et cela voulait dire qu'ils ne pouvaient pas être rendus plus confus par un mensonge que par une vérité. Si vous me dites que les centaures étaient sous l'emprise d'un \emph{Imperium}, j'ai toujours l'impression qu'il y a quelque chose qui ne colle pas. Je remarque que je suis toujours confus, même après avoir entendu votre explication.

--- Hmm~>>, dit le professeur Quirrell.

Ils continuèrent de marcher.

<<~Je suppose qu'il n'est pas possible, dit Harry, de vraiment \emph{permuter} des gens entre des univers parallèles~? Par exemple, ce ne serait pas notre Rita Skeeter, ou bien ils pourraient l'avoir temporairement envoyée ailleurs.

--- Si \emph{c'était} possible, dit le professeur Quirrell d'une voix plutôt sèche, serais-je toujours \emph{ici}~?~>>

Et, juste avant qu'ils n'atteignent l'immense façade blanche du bâtiment de Gringotts, le professeur Quirrell dit~:

<<~Ah. Bien \emph{sûr}. Laissez-moi deviner, les jumeaux Weasley~?

--- \emph{Quoi}~?~>> dit Harry, sa voix montant d'une octave. <<~\emph{Comment}~?

--- J'ai peur de ne pouvoir le dire.

--- … ce n'est \emph{pas} juste.

--- Je pense que c'est extrêmement juste~>>, dit le professeur Quirrell, et ils entrèrent, passant les portes de bronze.

\later

Il était presque midi, et Harry et le professeur Quirrell étaient assis aux extrémités d'une table longue, large, et plate, dans une pièce privée somptueusement apprêtée, dotée de canapés et de chaises minutieusement rembourrés répartis le long des murs et de rideaux épais qui pendaient un peu partout.

Ils étaient sur le point de déjeuner chez \emph{Marie}, dont le professeur Quirrell avait dit que c'était selon lui l'un des meilleurs restaurants du Chemin de Traverse, en particulier pour -- sa voix baissa d'un ton, chargée de sous-entendus -- qui avait certains \emph{desseins}.

C'était le meilleur restaurant où Harry avait été, et le fait que le professeur Quirrell était celui qui \emph{invitait} le rongeait vraiment.

La première partie de la mission consistait à trouver un précepteur d'Occlumancie et cela s'était soldé par un succès. Le professeur Quirrell, en souriant d'un air démoniaque, avait dit à Gripsec de recommander le meilleur qu'il connaisse et de ne pas se soucier des dépenses puisque c'était Dumbledore qui payait~; et le Gobelin avait souri en retour. Il y avait peut-être aussi eu quelques sourires du côté de Harry.

La deuxième partie du plan avait été un échec complet.

Harry n'était pas autorisé à retirer de l'argent de son coffre-fort si le directeur de Poudlard ou un autre officiel de l'école n'était pas présent, et le professeur Quirrell ne s'était pas vu remettre la clé du coffre. Les parents Moldus de Harry ne pouvaient pas l'y autoriser parce qu'ils étaient des Moldus, et les Moldus avaient à peu près le même statut juridique que des enfants ou des bébés chats~: ils étaient mignons, donc si vous les torturiez en public, vous pouviez vous faire arrêter, mais ce n'étaient pas des \emph{gens}. Une clause avait été ajoutée avec réticence afin de reconnaître les parents de nés-Moldus comme des êtres humains à titre limité, mais les parents adoptifs de Harry n'entraient pas dans cette catégorie juridique.

Il semblait qu'aux yeux du monde magique, Harry était un orphelin de fait. En tant que tel, le directeur de Poudlard ou ceux désignés par lui \emph{au sein} du système scolaire étaient les gardiens de Harry jusqu'à ce qu'il obtienne ses diplômes. Harry \emph{pouvait} respirer sans autorisation de Dumbledore, mais seulement si le directeur ne l'interdisait pas spécifiquement.

Harry avait alors demandé à Gripsec s'il pouvait simplement lui \emph{dire} comment diversifier ses investissements à plus que des piles de pièces d'or posées dans un coffre-fort.

Gripsec l'avait regardé d'un œil vide et lui avait demandé ce que “diversifier” voulait dire.

Il semblait que les banques ne faisaient pas d'investissements. Les banques stockaient vos pièces d'or dans des coffres-forts protégés en échange de frais annuels.

Le monde magique n'avait pas de concept d'action. Ni de capitaux propres. Ni d'entreprise. Les affaires étaient menées par des familles à partir de leurs coffres-forts personnels.

Les prêts étaient accordés par les gens riches, pas par les banques. Même si Gringotts se serait porté témoin du contrat en échange d'honoraires et aurait fait respecter sa collecte pour des honoraires bien plus élevés.

Les gentils riches laissaient leurs amis emprunter de l'argent et rembourser quand ils pouvaient. Les \emph{méchants} riches vous faisaient payer un \emph{intérêt}.

Il n'y avait pas de marché secondaire des emprunts.

Les méchants riches vous faisaient payer un intérêt annuel d'au moins 20~\%.

Harry s'était levé, puis il s'était détourné, et il avait laissé sa tête reposer contre un mur.

Harry avait demandé s'il avait besoin de l'autorisation du directeur pour pouvoir ouvrir une banque.

Le professeur Quirrell avait alors interrompu la conversation, disant qu'il était l'heure d'aller déjeuner, et il avait prestement reconduit un Harry fulminant à travers les portes de bronze de Gringotts, à travers le Chemin de Traverse, et jusqu'à un excellent restaurant nommé \emph{Chez Marie}, où une pièce leur avait été réservée. Le propriétaire avait eu l'air choqué de voir le professeur Quirrell accompagné par Harry Potter, mais il les avait conduits jusqu'à la pièce sans se plaindre.

Et le professeur Quirrell avait assez délibérément annoncé qu'il paierait la note, et il avait eu l'air de plutôt apprécier la tête que faisait Harry.

<<~Non, dit le professeur Quirrell à la serveuse, nous n'aurons pas besoin de menus. J'aurai le menu spécial du jour, accompagné d'une bouteille de Chianti, et M. Potter aura la soupe de Dirico pour commencer, puis une assiette de boulettes de Roopo, et un pudding de mélasse en dessert.~>>

La serveuse, vêtue d’une robe qui, bien que plus courte que la normale, avait quand même l'air stricte et formelle, s'inclina respectueusement et partit, fermant la porte derrière elle.

Le professeur Quirrell agita une main en direction de la porte, et un verrou glissa.

<<~Remarquez le verrou à l'intérieur. Cette pièce, M. Potter, est connue sous le nom de la \emph{chambre de Marie}. Elle se trouve être protégée contre toute observation, et je dis bien \emph{toute}~; Dumbledore lui-même ne pourrait rien détecter de ce qui se passe ici. La chambre de Marie est utilisée par deux sortes de personnes. Les premiers se livrent à d'illicites badinages. Et les seconds ont des vies intéressantes.

--- \emph{Vraiment}~>>, dit Harry.

Le professeur Quirrell hocha la tête.

Les lèvres de Harry étaient entrouvertes sous l'effet de l'anticipation. <<~Ce serait alors du gâchis que de rester assis ici à déjeuner quand nous pourrions faire quelque chose de spécial.~>>

Le professeur Quirrell sourit, puis il sortit sa baguette et l'agita en direction de la porte. <<~Bien sûr, dit-il, les gens qui ont une vie intéressante prennent des précautions plus \emph{minutieuses} que les chalands. Je viens de sceller la porte. Rien n'entrera ou ne sortira -- par une craquelure sous la porte par exemple. Et…~>>

Le professeur Quirrell prononça pas moins de quatre enchantements différents, et Harry n'en reconnut aucun.

<<~Même cela ne suffit pas \emph{vraiment}, dit le professeur Quirrell. Si nous faisions quoi que ce soit de véritable importance, il serait nécessaire d'opérer vingt-trois vérifications supplémentaires en plus de celle-ci. Par exemple, si Rita Skeeter sait ou a deviné que nous sommes ici, il est possible qu'elle soit dans cette pièce, vêtue de la vraie Cape d'Invisibilité. Ou elle pourrait peut-être être un Animagus de petite taille. Il existe des vérifications qui éliminent des possibilités improbables telles que celles-ci, mais il serait laborieux de toutes les réaliser. Mais tout de même, je me demande si je ne devrais pas le faire, juste pour ne pas vous enseigner de mauvaises habitudes~?~>> Et le professeur Quirrell se tapota la joue du doigt, l'air distrait.

<<~C'est bon, dit Harry. Je comprends, et je m'en souviendrai.~>> Même s'il était un peu déçu qu'ils ne soient pas en train de faire quelque chose de véritablement important.

<<~Très bien~>>, dit le professeur Quirrell. Il s'enfonça dans sa chaise et eut un grand sourire. <<~Vous avez bien œuvré aujourd'hui, M. Potter. L'idée de base était vôtre, j'en suis sûr, même si vous en avez délégué l'exécution. Je ne pense pas que nous entendrons beaucoup parler de Rita Skeeter après cela. Lucius Malfoy ne sera pas content de son échec. Si elle est intelligente, elle fuira le pays à l'instant où elle se rendra compte qu'elle a été dupée.~>>

Un nœud se forma dans l'estomac de Harry.

<<~Lucius était derrière Rita Skeeter…~?

--- Oh, vous ne vous en étiez pas rendu compte~?~>> dit le professeur Quirrell.

Harry n'avait pas réfléchi à ce qui arriverait ensuite à Rita Skeeter.

Du tout.

Pas le moins du monde.

Mais si elle se faisait renvoyer, \emph{bien sûr} qu'elle se ferait renvoyer, elle avait peut-être des enfants à Poudlard pour ce que Harry en savait, et maintenant c'était pire, bien pire -

<<~Lucius va-t-il la faire tuer~?~>> dit Harry d'une voix à peine audible. Quelque part dans sa tête, le Choixpeau lui hurlait dessus.

Le professeur Quirrell sourit sèchement. <<~Si vous n'avez pas encore eu affaire aux journalistes, croyez-moi sur parole quand je vous dis que le monde s'éclaire un peu chaque fois que l'un d'entre eux s'éteint.~>>

Harry bondit de sa chaise dans un mouvement convulsif, il fallait qu'il trouve Rita Skeeter et qu'il la prévienne avant qu'il ne soit trop tard -

<<~\emph{Asseyez-vous}, dit le professeur Quirrell d'une voix cassante. \emph{Non}, Lucius ne la tuera pas. Mais Lucius rend la vie \emph{extrêmement} déplaisante à ceux qui le servent mal. Mademoiselle Skeeter va fuir et recommencer sa vie sous un nouveau nom. \emph{Asseyez-vous}, M. Potter~; il n'y a rien que vous puissiez faire à présent, et vous avez une leçon à apprendre.~>>

Harry s'assit lentement. Il y avait un air déçu et agacé sur le visage du professeur Quirrell qui contribua plus à l'arrêter que les mots ne l'avaient fait.

<<~Il arrive~>>, dit le professeur Quirrell, la voix coupante, <<~que je sois inquiet à l'idée que votre possession d'un brillant esprit Serpentard ne soit du gâchis complet. Répétez après moi. Rita Skeeter était une femme vile et dégoûtante.

--- Rita Skeeter était une femme vile et dégoûtante~>>, dit Harry. Il ne se sentit pas à l'aise en le disant, mais après tout il ne semblait pas y avoir d'autre choix possible.

<<~Rita Skeeter a essayé de détruire ma réputation, mais j'ai exécuté un plan ingénieux et j'ai détruit \emph{sa} réputation en premier.

--- Rita Skeeter m'a défié. Elle a perdu le jeu, et j'ai gagné.

--- Rita Skeeter était un obstacle à mes plans futurs. Je n'avais d'autre choix que de m'occuper d'elle si je voulais que ces plans réussissent.

--- Rita Skeeter était mon ennemi.

--- Je ne pourrai jamais rien accomplir dans ma vie si je n'apprends pas à vaincre mes ennemis.

--- Aujourd'hui, j'ai vaincu l'un de mes ennemis.

--- Je suis un bon garçon.

--- Je mérite une récompense spéciale.

--- Ah~>>, dit le professeur Quirrell, qui avait souri avec gentillesse pendant les quelques dernières phrases, <<~je vois que j'ai réussi à attirer votre attention.~>>

C'était vrai. Et bien que Harry ait l'impression d'avoir été dirigé quelque part -- non, ce n'était pas une impression, il \emph{avait} été dirigé -- il ne pouvait pas nier que de dire ces choses et de voir le professeur Quirrell sourire le \emph{faisait} se sentir mieux.

Le professeur Quirrell fouilla dans sa robe, le geste lent, délibérément étudié, et il fit surgir…

… un \emph{livre}.

Il était différent de tous les livres que Harry avait jamais lus, les coins et les bords visiblement déformés~; \emph{dégrossi} était le mot qui venait à l'esprit, comme s'il avait été extrait à la pioche dans une mine de livres.

<<~Qu'est-ce~? respira Harry.

--- Un journal, dit le professeur Quirrell.

--- De qui~?

--- Celui d'une personne célèbre.~>> Le professeur Quirrell arborait un large sourire.

<<~D'accord…~>>

Le visage du professeur Quirrell devint extrêmement sérieux.

<<~M. Potter, l'un des prérequis pour devenir un grand sorcier est d'avoir une excellente mémoire. La clé d'un puzzle est parfois quelque chose que vous avez lu il y a vingt ans dans un vieux rouleau de parchemin, ou un anneau particulier que vous avez vu au doigt d'un homme que vous avez rencontré une fois. Si je mentionne cela, c'est pour expliquer comment je suis parvenu à me souvenir de cet objet et de l'affiche qui y était apposée lorsque, bien des années plus tard, je vous ai rencontré. Voyez-vous M. Potter, j'ai au cours de ma vie pu voir un certain nombre de collections privées détenues par des individus qui ne méritent peut-être pas vraiment tout ce qu'ils possèdent…

--- Vous l'avez \emph{volé}~? dit Harry d'un ton incrédule.

--- C'est juste, dit le professeur Quirrell. Très récemment à vrai dire. Je pense que vous apprécierez cet objet précis bien plus que le vil petit homme qui ne le détenait pour aucun autre but que celui d'impressionner ses amis également vils grâce à sa rareté.~>>

Harry était maintenant bouche bée.

<<~Mais si vous pensez que mes actions n'ont pas été correctes, M. Potter, j'imagine que vous n'avez pas besoin d'accepter votre cadeau spécial. Même si je n'irai certainement \emph{pas} le rendre. Alors, qu'allez-vous choisir~?~>>

Le professeur Quirrell jetait le livre d'une main à l'autre, ce qui poussait Harry à tendre involontairement les bras, un air consterné sur le visage.

<<~Oh, dit le professeur Quirrell, ne vous souciez pas de cette manipulation brutale. Vous pourriez jeter ce journal dans un feu et il en sortirait indemne. Quoi qu'il en soit, j'attends votre décision.~>>

Le professeur Quirrell jeta nonchalamment le livre dans les airs et le rattrapa en souriant.

\emph{Non}, dirent Gryffondor et Poufsouffle.

\emph{Oui}, dit Serdaigle. \emph{Quelle partie du mot “livre” n'as-tu pas comprise~?}

\emph{La partie vol}, dit Poufsouffle.

\emph{Oh, allez}, dit Serdaigle, \emph{tu ne peux pas sérieusement nous demander de dire non et de passer le reste de notre vie à nous demander ce que c'était}.

\emph{On dirait un net positif d'un point de vue utilitariste}, dit Serpentard. \emph{Vois cela comme une transaction économique qui génère des gains grâce à l'échange, mais sans l'échange. En plus, on ne l'a pas volé, et ça n'aiderait personne que le professeur Quirrell le garde pour lui.}

\emph{Il essaie de te rendre Obscur~!} glapit Gryffondor, et Poufsouffle hocha vigoureusement la tête.

\emph{Ne sois pas un petit garçon naïf}, dit Serpentard, \emph{il essaie de t'enseigner le Serpentard}.

\emph{Ouais}, dit Serdaigle. \emph{Celui qui possédait initialement le livre était probablement un Mangemort ou quelque chose du genre. Il nous appartient.}

La bouche de Harry commença à s'ouvrir, puis s'arrêta en chemin~; il avait l'air d'agoniser.

Le professeur Quirrell, lui, avait l'air de beaucoup s'amuser. Il avait placé le livre en équilibre sur un coin, posé sur un doigt, et il le gardait à la verticale tout en fredonnant une petite mélodie.

Il y eut un coup contre la porte.

Le livre disparut dans la robe du professeur Quirrell, et il se leva de sa chaise. Il commença à marcher vers la porte…

… et chancela, faisant une embardée soudaine en direction du mur.

<<~Tout va bien~>>, dit la voix du professeur Quirrell, qui semblait soudain beaucoup plus faible que d'habitude. <<~Restez assis, M. Potter, c'est juste un sort d'étourdissement. Restez assis.~>>

Les doigts de Harry s'agrippèrent au rebord de sa chaise, il n'était pas certain de ce qu'il était censé faire, de ce qu'il \emph{pouvait} faire. Harry ne pouvait même pas trop s'approcher du professeur Quirrell, pas à moins de vouloir défier cette sensation funeste-

Le professeur Quirrell se redressa alors, et sa respiration semblait un peu lourde. Il ouvrit la porte.

La serveuse entra, portant un plateau de nourriture~; et le professeur Quirrell revint lentement à la table tandis qu'elle distribuait les assiettes.

Mais lorsqu'elle s'inclina et sortit, le professeur Quirrell était de nouveau assis bien droit et souriant.

Tout de même, le bref épisode de quoi-que-ça-ait-été avait décidé Harry. Il ne pouvait pas dire non, pas après que le professeur Quirrell eut fait tant d'efforts.

<<~Oui~>>, dit Harry.

Le professeur Quirrell leva un doigt en signe d'avertissement, puis il sortit de nouveau sa baguette, verrouilla de nouveau la porte, et répéta trois des enchantements qu'il avait prononcés plus tôt.

Le professeur Quirrell récupéra alors le livre depuis sa robe et le jeta à Harry, qui le fit presque tomber dans sa soupe.

Harry jeta un regard d'indignation impuissante au professeur Quirrell. On n'était pas censé \emph{faire} ça aux livres, enchantés ou pas.

Harry ouvrit le livre avec une précaution instinctive ancrée au plus profond de lui. Les pages semblaient trop épaisses, avec une texture qui n'était ni celle du papier moldu ni celle du parchemin. Et il était…

… vide~?

<<~Suis-je censé voir…

--- Regardez plus près du début~>>, dit le professeur Quirrell, et Harry (de nouveau avec la précaution enracinée dont il ne pouvait se défaire) feuilleta le livre de plusieurs pages vers l'arrière.

L'écriture était évidemment manuelle et très dure à lire, mais Harry pensa que les mots étaient peut-être en latin.

<<~De \emph{quoi} s'agit-il~?

--- Ceci, dit le professeur Quirrell, est le compte-rendu des recherches en magie d'un né-Moldu qui ne se rendit jamais à Poudlard. Il refusa sa lettre d'acceptation et conduisit ses propres petites recherches, qui n'allèrent pas bien loin sans baguette. D'après la description de l'affiche, je m'attends à ce que son nom vous importe plus qu'à moi. Harry Potter, ceci est le journal de Roger Bacon.~>>

Harry s'évanouit presque.

Nichés contre le mur, là où le professeur Quirrell avait trébuché, brillaient les restes écrasés d'un magnifique scarabée bleu.
%  LocalWords:  rofessor Roopo
