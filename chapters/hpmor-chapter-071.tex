\partchapter{Accomplissement de soi}{VI}

\lettrine[ante=<<~]{E}{h} bien~>>, chuchota Daphné aussi bas qu'elle le pouvait, <<~au moins maintenant je n'ai plus l'impression d'être la seule personne saine d'esprit de Poudlard.

--- Parce que maintenant tu nous as~? chuchota Lavande Brown qui marchait sur la pointe des pieds à côté d'elle.

--- Je ne pense pas que ce soit ça qu'elle veut dire~>>, murmura le général Granger, située à gauche de Lavande.

Elles se faufilaient lentement et prudemment entre les couloirs de Poudlard, leur huit paires d'oreilles grandes ouvertes au moindre signe de problème, comme si c'était une bataille et qu'elles cherchaient des soldats ennemis embusqués~; sauf que là elles cherchaient des brutes à Vaincre et des victimes à Sauver entre la fin du petit déjeuner et le moment où Lavande et Parvati devraient se rendre à leur cours de Botanique.

Lavande avait soutenu que si une fille de première année pouvait faire leur affaire à trois brutes plus âgées, alors huit filles de première années devraient pouvoir battre vingt-quatre brutes par la grâce de la Multiplication.

À en juger par le bafouillage frénétique et les gesticulations hasardeuses de cette dernière, le général Granger n'avait pas trouvé cela convainquant.

Padma était resté silencieuse un moment pendant le débat puis avait pensivement fait remarquer que même à Poudlard, taper sur des filles de première année ne donnerait probablement pas une bonne réputation à une brute.

Parvati s'était alors raidi et s'était exclamée que cela voulait dire qu'elles étaient \emph{les seules} à pouvoir faire quelque chose contre le problème des brutes à Poudlard ce qui les rendait \emph{vraiment carrément} héroïniques. En plus, la \emph{seule raison} pour laquelle ses parents avaient déménagé en Angleterre était qu'elle puisse aller à la seule école magique au monde à avoir un taux de mortalité de 0~\%, alors quel intérêt si on ne pouvait pas en profiter et faire quelques expériences~?

Ce à quoi le général Granger avait répondu que Parvati ne comprenait pas \emph{du tout} le principe d'un bilan de sécurité parfait.

Lavande avait dit que si elles étaient \emph{vraiment} ses amies et pas ses partisanes comme le pensait le professeur Quirrell, elles devraient voter sur ce genre de questions.

Daphné s'était attendue à détenir le vote décisif après les non de Hermione, Susan et Hannah. Et elle avait donc étudié la proposition avec prudence après que la première montée d'enthousiasme fut passée. Elle \emph{était} Serpentard, après tout, et cela voulait dire que c'était \emph{sa} responsabilité de garder ses propres intérêts à l'œil pendant qu'elles couraient partout à essayer d'aider les autres - sa tâche serait de découvrir à quel point c'était vraiment risqué et de savoir si ça valait le coup \emph{qu'elles} le fassent exactement comme Mère aurait fait pour elle. Toujours faire attention à soi et à ses amis, c'était ça la vraie Serpentardise…

Hannah Abbott, la nerveuse petite Poufsouffle, eut une petite voix tremblante lorsqu'elle dit~: <<~Oui~>>.

Et maintenant Daphné, Susan et Hermione \emph{devaient} rester avec les cinq autres, car il aurait été \emph{inconcevable} qu'elles les laissent se débrouiller seules. Parce que tout Gryffondor ne pardonnerait jamais qu'on s'en prenne au dernier enfant en vie de la famille Bones et qu'aucune Serpentard n'oserait porter atteinte à la fille de la Noble et Très Ancienne maison Greengrass (c'est du moins ce que Daphné \emph{espérait}). Quant au général Granger, qui avait tout commencé… inutile de poser la question.

Les couloirs de Poudlard les virent défiler l'un après l'autre, leur main tendue jamais loin de leur baguette, la pierre, le bois et les torches infinies entrant et sortant de leur champ de vision. À un moment elles entendirent des bruits de pas et retinrent leur respiration, main presque à la baguette, mais ce n'était qu'un autre Serdaigle esseulé qui les regarda avec curiosité avant de renifler et de replonger la tête dans son livre et de continuer son chemin.

Les héroïnes avancèrent furtivement devant de solennels panneaux de chêne gravés de fresques dorées, arrivèrent à une impasse qui menait à des toilettes pour hommes, firent demi-tour, \emph{revinrent} au solennels panneaux de chêne gravés de fresques dorées puis obliquèrent vers de vieux couloirs de brique poussiéreux coulés dans du ciment qui sembla en fait décrire un cercle si bien qu'elles consultèrent un portrait et partirent alors vers un \emph{autre} vieux couloir de brique poussiéreux, ce qui les mena à une brève volée de marches en marbre, ce qui aurait dû les placer au troisième étage et demi si elles avaient été n'importe ailleurs qu'à Poudlard et elles se retrouvèrent alors sur un pavage de pierre avec des lucarnes qui creusaient des puits de lumière qui tombait jusqu'à elles, alors qu'elles n'étaient certainement pas proches du toit, et après qu'elles eurent suivi ce passage à travers quelques détours il les mena à d'autres toilettes pour homme clairement indiquées par une plaque montrant une silhouette en robes qui pissait dans une cuvette.

Elles se tinrent toutes les huit devant la porte et l'observèrent avec une certaine lassitude.

<<~Je m'ennuie~>>, dit Lavande.

Padma sortit ostentatoirement une montre de poche de ses robes et la regarda. <<~Seize minutes et trente secondes, dit-elle. Un nouveau record de durée de concentration pour Gryffondor.

--- \emph{Je} ne pense pas que ça va marcher non plus, dit Susan. Et je suis une \emph{Poufsouffle}.

--- Vous savez, dit Lavande d'un ton pensif, je me demande si ce qui fait \emph{vraiment} que quelqu'un est un héros c'est que lorsqu'ils essaient un truc de ce genre, quelque chose d'intéressant \emph{se passe}.

--- Je parie que t'as raison, dit Tracey. Je parie que si on avait \emph{Harry Potter} avec nous, on tomberait sur trois brutes et une chambre remplie de trésors en moins de cinq minutes. Je parie que tout ce que le général Chaos aurait à faire ça serait d'aller aux toilettes et il, tiens, il trouverait la chambre des secrets de Serpentard ou quelque chose comme ça -~>>

Daphné ne pouvait vraiment pas laisser passer ça.

<<~Tu penses que Salazar Serpentard aurait mis l'entrée de la chambre des secrets dans des \emph{toilettes} -

--- Ce que je \emph{dis}, continua Susan alors que Tracey ouvrait la bouche pour répondre, c'est qu'on n'a aucun moyen de \emph{trouver} des brutes. Enfin tout ce \emph{qu'ils} ont à faire c'est de trouver un Poufsouffle quelque part, mais on doit les croiser exactement au bon \emph{moment}, vous comprenez~? Ce qui se trouve être un \emph{excellent problème}, parce que si on les \emph{trouvait} on se ferait écraser comme des insectes. Vous ne voulez pas qu'on se fasse juste le couloir interdit du troisième étage, comme on est \emph{censé} le faire~?~>>

Lavande eut un rire de dédain. <<~Tu ne deviens pas une \emph{vraie} héroïne en faisant juste les choses interdites que le directeur te \emph{dit} de ne pas faire~!~>>

(L'esprit de Daphné essaya de comprendre cette phrase tout en remerciant silencieusement le Choixpeau de ne lui avoir pas fait ne serait-ce qu'approcher Gryffondor.)

<<~À bien y réfléchir… dit lentement Parvati, je veux dire, quelles sont les chances pour que Harry Potter croise ces cinq brutes lors de \emph{son premier jour de cours}~? Il doit avoir eu \emph{un moyen} de les trouver.~>>

Daphné se trouvait être à un endroit qui lui permettait de voir Hermione tout en regardant Parvati, si bien qu'elle remarqua le changement d'expression du visage de la fille - et elle se rendit alors compte que le général Soleil avait \emph{aussi} trouvé quelques brutes récemment -

<<~Oh~! dit Padma du ton de quelqu'un qui venait soudain de tout comprendre. Bien sûr~! C'est le fantôme de Salazar Serpentard qui le lui a dit~!

--- \emph{Quoi~?} dit Daphné en même temps que plusieurs autres personnes.

--- C'est lui le fantôme qui m'a fait peur, j'en suis presque sûre, expliqua Padma. Enfin, je l'ai seulement compris ensuite mais… ouais. Le fantôme de Salazar Serpentard n'aime pas ça quand les Serpentard martyrisent les gens, il pense que ça fait honte à son nom, et je parie que son fantôme a toujours un haut niveau d'accès à Poudlard, si bien qu'il sait tout ce qui s'y passe,.~>>

La mâchoire de Daphné s'était décrochée et elle vit que Hannah s'était mise une main sur le front et s'appuyait contre le mur de pierre tandis que les yeux de Tracey brûlaient comme de petites étoiles brunes.

\emph{Le fantôme de Salazar Serpentard~?}

S'était allié à \emph{Harry Potter~?}

Et avait envoyé \emph{Hermione Granger} arrêter l'équipe de Derrick~?

Elle aurait payé cent Gallions pour être là quand Drago Malfoy entendrait ça.

Quoique, à en juger par la vitesse à laquelle les rumeurs se répandaient à Poudlard, maintenant que Padma avait craché le morceau Millicent le lui avait probablement dit il y a une demi-heure…

En fait… maintenant que Daphné y \emph{pensait}…

<<~Donc, dit Parvati. On doit demander au Survivant où est le fantôme de Salazar Serpentard~? Waouh, j'imagine que si je me mets à dire ce genre de trucs à voix haute ça veut peut-être dire que je deviens vraiment une héroïne -

--- Oui~! dit Lavande. On doit demander au Survivant où est le fantôme de Salazar Serpentard~!

--- On doit demander… au Survivant… où est le fantôme de Salazar Serpentard…~>> répéta Hannah d'une voix nerveuse, comme si elle se forçait à le dire.

<<~Et si \emph{ça} ne marche pas, hurla Tracey, on va assommer Harry Potter, l'attacher et l'emmener \emph{avec} nous~!~>>

\later

Hermione Granger trouvait que cela en disait long et que c'était assez triste - alors que les huit filles parcouraient en sens inverse le labyrinthe de passages tordus qu'était Poudlard, n'ayant trouvé aucune brute avant l'heure de leur prochain cours - de constater qu'elle ne savait vraiment pas si Harry Potter avait été guidé par le fantôme de Salazar Serpentard, par un phénix, ou par… \emph{quoi~?} Quelle que soit la réponse, elle espérait que cela ne \emph{fonctionnerait pas} pour elles. Et plus que tout, elle espérait que les autres ne voteraient pas pour l'idée stupide de Tracey consistant à assommer Harry et à traîner son corps inconscient pour attirer des Péripéties. Les choses ne pouvaient certainement pas se passer comme ça dans la vraie vie, ou si c'était le cas autant laisser tomber.

Hermione regarda les sorcières l'une après l'autre, Tracey qui discutait avec Lavande et les autres qui participaient occasionnellement~; et son regard tomba sur une fille réservée et silencieuse, la seule dont elle était incapable de deviner les pensées.

<<~Hannah~!~>> dit-elle à la fille qui marchait à côté d'elle. Hermione essaya de rendre sa voix aussi douce que possible. <<~Tu n'es pas obligée de répondre, mais est-ce que je peux te demander pourquoi tu as voté pour qu'on se batte contre des brutes~?~>>

Hermione pensait avoir parlé à voix basse mais tout le monde s'arrêta de marcher, Lavande et Tracey interrompirent leur conversation et les regardèrent.

Les joues de Hannah rougissaient déjà et alors qu'elle ouvrait sa bouche -

<<~C'est pac'qu'elle a plus de courage que \emph{tu} le penses, évidemment~>>, dit Lavande.

Hannah s'arrêta, bouche ouverte.

Elle la referma.

Déglutit avec force alors que ses joues rougissaient encore plus.

Puis elle prit une profonde inspiration et dit d'une petite voix~: <<~Il y a un garçon que j'aime bien.~>>

La Poufsouffle broncha en prononçant ces mots et sa tête tourna nerveusement à gauche et à droite pour regarder tous ceux qui l'observaient alors que le silence s'étirait.

<<~Euh, d'accord~? dit enfin Susan.

--- Il y a \emph{cinq} garçons que j'aime bien, dit Lavande.

--- Padma et moi savions qu'on aimerait les mêmes garçons, dit Parvati, alors on a fait une liste et on a lancé une Noise en l'air pour voir qui aurait droit au premier choix.

--- Je sais à qui est promis \emph{ma} main, dit Tracey. Je me fiche de ce que le reste du monde a à dire, il sera à moi~!~>>

Sur ces paroles, toutes les filles se tournèrent vers Hermione, dans l'expectative, mais le cerveau de celle-ci était déjà passé à la suite et avait totalement évacué la dernière phrase de Tracey pour pouvoir se concentrer sur la première chose que Hannah avait dite.

<<~Euh~>>, dit Hermione. Elle fit attention de garder une voix douce. <<~Hannah, la raison pour laquelle tu as rejoint la Société pour la Promotion de l'Égalité Héroïque pour les Sorcières, c'est qu'il y a un garçon dont tu penses qu'il t'aimerait plus si tu devenais une héroïne~?~>>

La Poufsouffle hocha de nouveau la tête, ses joues rougirent encore plus et elle baissa les yeux pour regarder son reflet dans ses chaussures noires polies.

<<~En fait elle aime Neville Londubat~>>, dit Daphné. La Serpentard eut un soupir affligé. <<~Et malheureusement pour elle, il va épouser quelqu'un d'autre. Tout à fait tragique.~>>

Ceci déclencha un geignement aigu chez Hannah qui continua de regarder ses pieds.

<<~Attends, quoi~? dit Lavande. Neville va épouser quelqu'un d'autre~? Comment tu le sais~? \emph{Qui}~?~>>

Daphné se contenta de secouer la tête avec un air abattu.

<<~\emph{Pardon, mais}~>>, dit Hermione, et lorsque les autres la regardèrent à nouveau~: <<~Ah…~>> alors qu'elle essayait d'organiser ses pensées. <<~Je veux dire, euh… Hannah… essayer de devenir une héroïne pour qu'un garçon t'aime bien, ça ne te rend pas très \emph{féministe}.

--- À vrai dire ça se prononce \emph{féminine}, dit Padma.

--- Et pourquoi est-ce que tu dis que Padma n'est pas féminine~? dit Susan. C'est très féminin de vouloir impressionner un garçon.

--- Et puis, dit Parvati d'un ton perplexe, est-ce que le but n'était pas justement de devenir des héroïnes même si ça n'est pas féminin~?~>>

Hermione ne se souvint pas de la discussion qui suivit comme l'une de ses excursions les plus réussies dans le royaume de l'éducation politique. Elle essaya de leur expliquer, et après le débat qui en découla, essaya de leur expliquer à nouveau tandis que les sept autres filles la regardaient avec un scepticisme croissant. Puis Daphné déclara du ton impérieux d'une future Dame Greengrass que si cette histoire de féminisme voulait dire que les filles n'avaient plus le droit de courir après les garçons comme ça leur chantait, alors le féminisme pouvait bien rester chez lui en terre Moldue. Lavande suggéra que le sorciérisme pourrait peut-être dire que les sorcières pouvaient faire exactement ce qu'elles voulaient, ce qui avait l'air bien plus amusant que le féminisme. Et Padma mit enfin un terme à la discussion en faisant remarquer d'un ton las qu'elle ne voyait pas vraiment le but de ce débat puisque SPEHS n'avait en premier lieu \emph{rien à voir} avec le féminisme et que c'était juste pour que plus de filles deviennent des héroïnes.

Hermione avait alors déjà abandonné.

\later

À la fin de leur cours de Charmes, la première année Serdaigle commençait à quitter la salle en grimaçant déjà en son for intérieur. Elles étaient arrivées en cours à peine avant la sonnerie et avaient dû courir jusqu'à leurs pupitres et s'asseoir si bien que l'horreur ne s'était pas \emph{encore} produite~; mais cela voulait seulement dire que Hermione pouvait maintenant s'attendre à ce que le désastre à venir occupe le cours \emph{entier}.

Et de fait, après que le professeur Flitwick eut couiné que le cours était fini et que tout le monde se soit levé de ses chaises, Harry commença à marcher vers Hermione juste quand cette dernière fourra son livre dans sa bourse en peau de Moke, marcha très vite vers la porte, l'ouvrit grand, entra dans le couloir, et Harry la suivit bien sûr avec un air surpris parce qu'ils avaient une séance de bibliothèque de prévue -

<<~Hermione~? dit Harry en fermant la porte derrière lui. Qu'est-ce qui ne va pas~?~>>

La porte s'ouvrit en grand à peine un instant après que Harry ne l'eut refermé, le frappant presque alors qu'il s'écartait du chemin, et Padma Patil sortit de la salle avec un air de détermination terrible sur le visage.

<<~Excusez-moi, M. Potter~>>, dit-elle à la plus grande horreur de Hermione, d'une voix perchée qui résonna à travers le corridor tel les lugubres cloches de la perdition, <<~pourrais-je vous demander votre aide~?~>>

Les sourcils de Harry s'arquèrent et il répondit~:

<<~Tu peux me \emph{demander}, bien sûr.

--- Peux-tu me dire comment tu parles au fantôme de Salazar Serpentard~? On veut qu'il nous dise où trouver des brutes comme il te le dit à toi.~>>

Il y eut un léger silence dans le couloir qui jouxtait la salle.

La porte s'ouvrit de nouveau et Su sortit sa tête avec un regard interrogateur -

<<~Eh bien nous devons aller à la bibliothèque~>>, dit Harry d'un ton assez nonchalant, le visage détendu, <<~ça ne t'embête pas de nous suivre~?~>> et il commença à partir vers l'endroit où se trouvait la bibliothèque les jours impairs du mois, et Su commença à donner l'impression qu'elle allait les suivre mais Harry pivota et la regarda pendant un moment.

Ce n'est qu'après qu'il eut passé un angle qu'il sortit sa baguette, dit d'une voix précise et basse~: <<~\emph{Quietus}~>>, se tourna vers Padma et dit~: <<~Une conjecture intéressante, Mlle Patil.~>>

Padma eut alors l'air plutôt contente d'elle et répondit~: <<~\emph{J'aurais} dû le comprendre plus tôt, en fait. Il y avait un \emph{sifflement} dans la voix du Serpent, j'aurais dû penser au Fourchelangue tout de suite, avant même qu'il ne commence à parler de Godric Gryffondor.~>>

Le visage de Harry ne changea pas d'expression.

<<~Pourrais-je vous demander, Mlle Patil, si vous avez partagé cette idée avec -

--- Elle l'a dit devant tout SPEHS~>>, dit Hermione.

Les yeux de Harry prirent cet aspect qu'ils avaient lorsque Harry calculait rapidement quelque chose, puis il dit~:

<<~Hermione, quelles sont les chances que -

--- Elle l'a dit devant Lavande \emph{et} Tracey.

--- Euh, dit Padma. Je n'aurais pas dû~?~>>

\later

<<~Attends ici~>>, gronda M. Goyle, et il passa un angle~; et on put l'entendre frapper à la porte de la chambre privée de M. Malfoy.

Elle avait légèrement la nausée et se rappela à nouveau que, puisque Padma avait craché le morceau, \emph{quelqu'un} allait forcément le dire à Drago Malfoy, alors autant que ça soit \emph{elle}, et puis ce n'était pas comme si elle \emph{devait} quoi que ce soit à Harry Potter, et une Serpentard devait faire le nécessaire pour accomplir ses Ambitions.

Elle avait collectionné des Ambitions depuis que le professeur Quirrell l'avait réprimandée et pour l'instant elle avait décidé qu'elle voulait son propre Nimbus 2000, devenir super connue, épouser Harry Potter, manger des grenouilles en chocolat à chaque petit déjeuner et vaincre au moins \emph{trois} Seigneurs des Ténèbres juste pour faire bien voir au professeur Quirrell qui était ordinaire.

<<~M. Malfoy va te recevoir~>>, dit la voix grave et menaçante de M. Goyle lorsque celui-ci revint. <<~Et tu ferais mieux d'espérer qu'il ne pense pas que tu lui fais perdre son temps.~>> Il la regarda brièvement d'un air toujours menaçant puis s'écarta.

Tracey ajouta à sa liste d'Ambitions qu'elle voulait avoir ses propres serviteurs puis entra.

Le chambre privée de M. Malfoy ressemblait exactement à celle de Daphné. Elle avait secrètement espérer voir des chandeliers de diamant ou des fresques en or sur les murs - elle ne l'avait jamais dit devant Daphné mais la maison Malfoy \emph{était} un cran au-dessus de Greengrass. Mais c'était juste une petite chambre comme celle de Daphné et la seule différence était que le mobilier de Malfoy était décoré par des serpents d'argent à la place des plantes en émeraude.

Alors qu'elle passait l'embrasure de la porte, Drago Malfoy - qui était d'une apparence impeccable même dans sa propre chambre - se leva de sa chaise pour l'accueillir d'un petit salut amical, affublé d'un sourire charmant exactement comme si elle était quelqu'un \emph{d'importance}, ce qui la rendit si nerveuse qu'elle oublia tout ce qu'elle avait répété dans sa tête et qu'elle dit juste~:

<<~J'ai un truc à te dire~!

--- Oui, Gregory me l'a fait savoir, dit doucement Drago Malfoy. S'il vous plaît, Mlle Davis, asseyez-vous.~>> Il indiqua \emph{sa propre chaise} tout en s'asseyant sur son lit.

Elle eut quelque peu le tournis en s'asseyant précautionneusement sur la chaise de Malfoy, ses doigts jouant inconsciemment avec la chute de ses robes formelles au-dessus de ses genoux, essayant de la rendre aussi élégante et lisse que celle de Drago Malfoy -

<<~Donc, Mlle Davis, dit Drago Malfoy. Que vouliez-vous me dire~?~>>

Tracey hésita puis, lorsque le visage de Drago Malfoy commença à avoir l'air impatient, elle lâcha tout, tout ce que Padma avait dit sur le fantôme de Salazar Serpentard qui envoyait Harry Potter arrêter les brutes et aussi ce que Daphné lui avait dit sur le fait que Hermione Granger était dans le coup -

Le visage de Drago Malfoy ne changea pas à mesure qu'elle parlait, pas le moins du monde, et Tracey se sentit malade lorsqu'elle s'en rendit compte.

<<~Tu ne me \emph{crois} pas~!~>> dit-elle.

Il y eut un court silence.

<<~Eh bien~>>, dit Drago Malfoy avec un sourire qui n'était pas aussi charmant que le précédent, <<~je \emph{crois} que c'est ce que Padma et Daphné ont dit, donc merci quand même, Mlle Davis.~>> Le garçon se leva de son lit et Tracey, sans réfléchir, se leva de sa chaise.

Alors qu'il l'escortait à la porte, juste avant qu'il ne tourne la poignée, l'idée vint à Tracey que - <<~Tu ne m'as pas demandé ce que je voulais en échange de cette information~>>, dit-elle.

Drago Malfoy lui jeta un étrange regard, elle ne savait pas bien ce qu'il était censé signifier et il ne dit rien.

<<~Eh bien en tout cas~>>, dit Tracey, modifiant ses plans sur-le-champ, <<~je ne veux \emph{rien} en échange de cette information, je me montrais juste amicale.~>>

Un bref air de surprise traversa le visage de Drago, uniquement l'espace d'un instant, avant que son expression ne devienne à nouveau neutre et qu'il ne dise~: <<~Ce n'est pas aussi facile de devenir ami avec un Malfoy, Mlle Davis.~>>

Tracey sourit avec sincérité. <<~Alors je vais juste continuer de me montrer amicale~>>, dit-elle, et elle sortit de la pièce en gambadant presque, se sentant être une vraie Serpentard peut-être pour la première fois de sa vie et ayant décidé à l'instant que Drago Malfoy serait lui aussi un de ses maris.

\later

Après le départ de la fille, Gregory entra, referma la porte et dit~: <<~Vous allez bien, M. Malfoy~?~>>

Drago ne dit rien à son serviteur et ami. Ses yeux étaient dirigés vers le vide, comme s'il essayait de regarder à travers le mur de sa chambre, à travers le lac de Poudlard qui entourait les donjons Serpentard, à travers l'écorce terrestre, l'atmosphère, la poussière interstellaire et la Voie Lactée, dans le néant absolument vide et noir entre les galaxies, là où aucun scientifique et aucun sorcier n'étaient jamais allés.

<<~M. Malfoy~? dit Gregory d'un ton qui commençait à être légèrement inquiet.

--- Je n'arrive pas à croire que je crois à tout ça~>>, dit Drago.

\later

Daphné finit son dernier paragraphe de métamorphose et leva la tête~; son regard traversa la salle commune de Serpentard vers l'endroit où Millicent Bulstrode travaillait encore sur ses devoirs. Il était temps de prendre une Décision.

Si SPEHS continuait à se balader en essayant d'assommer des brutes, les brutes n'aimeraient pas ça, c'était certain. Et elles essaierait d'y répondre de façon déplaisante, c'était aussi certain. En revanche, si les brutes devenaient vraiment méchantes, alors Hermione Granger pourrait demander de l'aide à Harry Potter, ou bien ils pourraient combiner leurs points Quirrell et demander une faveur au professeur de Défense… Non, ce qui inquiétait \emph{vraiment} Daphné c'était que toute cette histoire les fasse mal voir du professeur Rogue. Il ne fallait \emph{jamais} être mal vu du professeur Rogue.

Mais depuis le jour où elle avait défié Neville à un Très Ancien Duel, elle avait remarqué que les gens la regardaient différemment. Même les Serpentard qui s'étaient moqués d'elle la regardaient différemment. Daphné commençait à comprendre qu'être la fille de la Noble et Très Ancienne maison Greengrass octroyait beaucoup plus de respect si on était une belle \emph{héroïne} que si on n'était qu'une mignonne petite noble. C'était la différence entre voir son rôle joué par une star ou le voir joué par une figurante à deux Gallions au rire éraillé.

Combattre des brutes n'était peut-être pas le \emph{meilleur} moyen de devenir une héroïne. Mais Père lui avait un jour dit que le problème quand on laissait passer des opportunités, c'était que ça devenait une habitude. Si on se disait qu'on attendait une meilleure opportunité, alors on se dirait probablement la même chose quand la prochaine opportunité arriverait. Père avait dit que la plupart des gens passaient leurs \emph{vies} entières à attendre qu'une opportunité assez bonne se présente et ensuite mouraient. Père avait dit que, même si se saisir d'opportunités impliquait que plein de choses tourneraient mal, ça n'était pas aussi grave que d'être un raté sans espoir. Père avait dit qu'\emph{une fois} qu'elle aurait pris l'habitude de se saisir des opportunités qui se présentaient à elle, \emph{alors} il serait temps d'être exigeant.

D'un autre côté, Mère l'avait prévenue de ne pas suivre tous les conseils de son Père et avait dit que Daphné n'aurait pas le droit de poser de questions sur la sixième année à Poudlard de son père avant qu'elle n'ait au moins trente ans.

Mais Père avait bien \emph{fini} par obtenir de Mère qu'elle l'épouse et avait ainsi réussi son plan destiné à le faire entrer dans une Très Ancienne maison, ce qui n'était \emph{pas} négligeable.

Millicent Bulstrode finit ses devoirs et commença à ranger ses affaires.

Daphné se leva de son bureau et marcha vers elle.

Millicent descendit ses jambes de la table et se leva, balança son sac sur une épaule puis regarda Daphné qui s'approchait avec un air perplexe.

<<~Hé, Millicent~>>, dit Daphné en s'approchant d'une voix basse et exaltée, <<~devine ce que j'ai compris aujourd'hui~?

--- Le truc avec le fantôme de Salazar Serpentard qui aide Hermione Granger~? dit Millicent. J'en ai déjà entendu parler -

--- Non, dit Daphné dans un souffle, c'est encore \emph{mieux}.

--- Vraiment~?~>> dit Millicent d'une voix tout aussi basse et tout autant exaltée. <<~Qu'est-ce que c'est~?~>>

Daphné regarda autour d'elle avec un air de conspiratrice. <<~Viens dans ma chambre et je te le dirai.~>>

Elles descendirent les escaliers qui menaient aux chambres privées situées encore plus bas dans le lac que ne l'étaient les dortoirs de septième année…

Bientôt Daphné était assise dans sa confortable chaise de bureau et Millicent était posée au bord de son lit.

<<~\emph{Quietus}~>>, dit Daphné une fois qu'elles furent toutes deux assises~; puis au lieu de ranger sa baguette dans sa robe, Daphné laissa juste sa main glisser sur son flanc, baguette toujours serrée, juste au cas où.

<<~Très \emph{bien}~! dit Millicent. Qu'est-ce qu'il y \emph{a}~?

--- Tu sais ce que j'ai découvert~? dit Daphné. J'ai compris que tu connais les rumeurs \emph{tellement} vite qu'en fait tu les connais \emph{avant qu'elles aient lieu}.~>>

Daphné s'était à moitié attendue à ce que Millicent devienne blanche et tombe par terre, ce qu'elle ne fit pas tout à fait, mais elle tressaillit assez fort avant de commencer à bégayer des dénégations.

<<~Ne t'en fais pas, dit Daphné avec son sourire le plus doux, je ne dirai à personne que tu es une voyante. Après tout, on est amies, pas vrai~?~>>

\later

Rianne Felthorne, une Serpentard de septième année, travaillait avec zèle sur une autre dissertation de trente pages (elle suivait tous les cours sauf Divination et Étude Moldues et son année d'A.S.P. semblait être \emph{entièrement} composée de devoirs maison) quand son directeur de Maison marcha jusqu'à la table où elle travaillait, aboya <<~Vous allez me suivre, Mlle Felthorne~!~>> et s'éloigna alors qu'elle rangeait frénétiquement son parchemin, son livre et sa plume.

Lorsqu'elle eut rattrapé le professeur Rogue, il attendait juste devant la pièce et la regardait avec des yeux mis-clos qui semblaient bien trop intenses~; et avant qu'elle ne puisse demander ce qu'il y avait il pivota sans dire un mot et s'élança à travers les couloirs, si bien qu'elle dut courir pour rester à sa hauteur.

Leur marche les mena une volée de marches plus bas, puis une autre, en dessous de ce qu'elle avait cru être le niveau le plus bas des donjons Serpentard. Et les couloirs commençaient à avoir l'air plus vieux, l'architecture remontait les siècles, devenait faite de pierres brutes maintenues entre elles par un mortier à l'air primitif. Elle commença à se demander si le professeur Rogue l'emmenait aux \emph{vrais} donjons au sujet desquels elle avait entendu des rumeurs, les véritables donjons de Poudlard qui avaient été condamnés pour tous sauf pour les membres du personnel~; et peut-être le professeur Rogue faisait-il là des choses terrible à d'innocentes jeunes filles sans défense, mais Rianne était probablement juste en train de prendre ses désirs pour des réalités.

Ils descendirent une autre volée de marches et entrèrent dans une pièce qui n'en était pas vraiment une, plutôt une caverne de pierre dotée d'une seule porte percée de nombreuses ouvertures sombres et éclairée d'une unique torche au style ancien qui s'alluma lorsqu'ils entrèrent.

Le professeur Rogue prit alors sa baguette et commença à lancer sortilège après sortilège, elle en perdit le compte, et lorsque le maître des potions en eut fini, il se retourna vers elle, braqua ses intenses yeux sur les siens et dit d'une voix neutre très différente de son ton traînant habituel~: <<~Vous ne direz rien de cette affaire, Mlle Felthorne, ni aujourd'hui ni jamais. Si vous pouvez l'accepter, hochez la tête. Sinon, nous ferons demi-tour.~>>

Elle hocha la tête, effrayée, avec un étrange espoir qui naissait dans son cœur (enfin, pas exactement dans son cœur).

<<~La tâche que j'ai pour vous est extrêmement simple, Mlle Felthorne, dit le professeur Rogue d'une voix sans timbre, et votre salaire extrêmement généreux de cinquante Gallions n'est là que pour compenser le fait que vous recevrez ensuite un sortilège d'Oubliettes.~>>

Elle inspira soudainement. Sa famille était peut-être riche, mais ils avaient d'autres filles et étaient très stricts avec elle. Pour \emph{elle}, cela représentait beaucoup d'argent.

Puis ses oreilles en arrivèrent au mot \emph{Oubliettes} et elle se sentit outrée l'espace d'un instant, quel intérêt si elle ne pouvait garder les souvenirs, pour quel genre de fille le professeur Rogue la \emph{prenait-il}~?

<<~Vous connaissez sûrement, dit Severus Rogue, Mlle Hermione Granger, le général Soleil~?

--- \emph{Quoi?} dit Rianne Felthorne soudain horrifiée et dégoûtée. Elle est en \emph{première année~! Berk~!}~>>
%  LocalWords:  heroinic Slytherining Everburning halfth Y’know eeping Ew
%  LocalWords:  witchism
