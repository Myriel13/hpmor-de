\partchapter{Humanisme}{III}

\lettrinepara{L}{a} chanson de Fumseck s'éteignit doucement.

\hplettrineextrapara
Harry se redressa de l'endroit où il avait été allongé, sur l'herbe constellée d'hiver, Fumseck toujours perché sur son épaule.

Il y eut de grandes inspirations autour de lui.

«~Harry, dit Seamus d'une voix vacillante, tu vas bien~?~»

La paix du phénix était toujours en lui, et une chaleur émanait de là où Fumseck était perché. Une chaleur qui se répandait à travers lui, et le souvenir de la chanson était toujours là, vivant par la présence du phénix. Des choses terribles venaient de lui arriver, des pensées terribles l'avaient traversé. Malgré la profanation que le Détraqueur l'avait forcé à commettre envers un souvenir impossible, il avait néanmoins recouvré celui-ci. Un mot étrange continuait de résonner dans son esprit. Et tout cela pourrait être mis de côté, à plus tard, tant que le phénix brillerait d'or et de rouge sous le soleil couchant.

Fumseck croassa en sa direction.

«~Quelque chose que je dois faire~? dit Harry à Fumseck. Quoi~?~»

Fumseck inclina sa tête en direction du Détraqueur.

Harry regarda l'horreur impossible à regarder qui était toujours dans sa cage, puis de nouveau le phénix, confus.

«~M. Potter~? dit la voix de Minerva McGonagall dans son dos. Vous \emph{allez} bien~?~»

Harry se mit sur ses pieds et pivota.

Minerva McGonagall le regardait, l'air très inquiète~; à côté d'elle, Albus Dumbledore l'observait avec attention~; Filius Flitwick semblait extrêmement soulagé~; et tous les élèves le regardaient bouche bée.

«~Je pense que oui, professeur McGonagall~», dit Harry d'une voix calme. Il avait presque dit \emph{Minerva}, avant de s'en empêcher. Tout irait bien, du moins tant que Fumseck serait sur son épaule~; il était possible qu'il s'écroule un instant après le départ de ce dernier, mais ce genre de pensée ne lui semblait étrangement pas avoir d'importance. «~Je pense que je vais bien.~»

Il y aurait dû y avoir des acclamations ou des soupirs de soulagement ou quelque chose, mais personne ne semblait savoir quoi dire, personne.

La paix du phénix persista.

Harry se retourna. «~Hermione~?~» dit-il.

Tous ceux qui avaient la moindre sensibilité romantique retinrent leur respiration.

«~Je ne sais pas vraiment comment remercier avec grâce, dit doucement Harry, pas plus que je ne sais m'excuser. Tout ce que je peux dire, c'est qui si tu te demandes si tu as bien agi, la réponse est oui.~»

Le garçon et la fille se regardèrent dans les yeux l'un de l'autre.

«~Désolé, dit Harry. Pour ce qui va se passer ensuite. S'il y a quoi que ce soit que je puisse faire…

--- Non, répondit Hermione. Tu ne peux rien faire. Mais ce n'est pas grave.~» Puis elle se détourna de Harry et partit vers le chemin qui menait aux portes de Poudlard.

Un certain nombre de filles jetèrent des regards perplexes en direction de Hermione puis la suivirent. Sur leur chemin on pouvait entendre les questions animées qui commençaient.

Harry les regarda partir puis il se tourna de nouveau, vers les autres élèves. Ils l'avaient vu au sol, alors qu'il criait, et…

Fumseck frotta son nez contre une joue de Harry, brièvement.

… et cela les aiderait, un jour, à comprendre que le Survivant pouvait aussi souffrir, qu'il pouvait aussi être malheureux. Afin que lorsqu'ils souffriraient et qu'ils seraient eux-mêmes malheureux, ils se souviendraient avoir vu Harry se tortiller au sol, et ils sauraient que leur souffrance et leurs problèmes n'avaient en fin de compte aucune importance. Le directeur avait-il pris cela en compte quand il avait laissé les autres élèves rester et regarder~?

Les yeux de Harry revinrent sur la cape en lambeaux, distraitement, et sans vraiment savoir à qui il s'adressait, il dit~:

«~Ça ne devrait pas exister.

--- Ah, dit une voix sèche et précise. Je pensais que vous pourriez dire cela. M. Potter, je suis navré de vous apprendre que les Détraqueurs ne peuvent être tués. Nombreux sont ceux qui ont essayé.

--- Vraiment, dit Harry toujours l'air absent. Qu'ont-ils essayé~?

--- Il existe un sort extrêmement dangereux et destructeur, dit le professeur Quirrell, que je ne nommerai pas ici~; un sortilège de feu maudit. C'est ce que vous utiliseriez pour détruire un ancien artefact tel que le Choixpeau. Il n'a aucun effet sur les Détraqueurs. Ils sont éternels.

--- Il n'y a pas d'éternité~», dit le directeur. Les mots étaient doux~; le regard acéré. «~Ils ne possèdent pas la vie éternelle. Ils sont des plaies à la surface du monde, et attaquer une plaie ne fait que l'élargir.

--- Hmm, dit Harry. Imaginez qu'on le jette dans le soleil~? L'astre serait-il détruit~?

--- Le \emph{jeter} dans le \emph{soleil}~?~» piailla le professeur Flitwick, et il avait l'air de vouloir s'évanouir.

«~Cela semble peu probable, M. Potter, dit le professeur d'un ton sec. Le soleil est très grand, après tout~; je doute que le Détraqueur lui fasse beaucoup d'effet. Mais ce n'est pas un test que je souhaiterais effectuer, juste au cas où.

--- Je vois~», dit Harry.

Fumseck coassa une dernière fois, puis il recouvrit le visage de Harry de ses ailes et se propulsa alors loin de lui. Droit vers le Détraqueur, jetant un grand cri perçant de défi qui fit écho à travers le champ. Et avant que quiconque ait pu réagir, il y eut une explosion de feu, et Fumseck était parti.

La paix se dissipa, un peu.

La chaleur se dissipa, un peu.

Harry prit une profonde inspiration puis la laissa ressortir.

«~Ouais, dit Harry. Toujours en vie.~»

Encore ce silence, encore l'absence d'acclamations~; personne ne semblait savoir comment répondre…

«~Il est bon d'apprendre que vous êtes totalement rétabli~», dit le professeur Quirrell d'une voix ferme, comme s'il niait la possibilité d'une alternative. «~Maintenant, je crois que mademoiselle Ransom vient ensuite~?~»

Cela fut le point de départ d'une dispute animée où le professeur Quirrell avait raison et tous les autres avaient tort. Le professeur Quirrell faisait remarquer qu'en dépit des émotions fort compréhensibles ressenties par toutes les personnes concernées, la chance qu'une mésaventure similaire arrive à un autre élève tendait à l'infinitésimal~; d'autant plus qu'ils savaient maintenant éviter celles causées par les baguettes. Et de plus, il y avait d'autres élèves qui devaient à leur tour faire de leur mieux pour lancer un Patronus corporel, ou du moins pour apprendre la sensation provoquée par le Détraqueur afin de pouvoir le fuir, et découvrir leur propre degré de vulnérabilité…

Il s'avéra que Dean Thomas et Ron Weasley de Gryffondor étaient les seuls à vouloir encore s'approcher du Détraqueur, ce qui simplifia la dispute.

Harry jeta un coup d'œil en direction de celui-ci. Le mot résonna de nouveau dans son esprit.

\emph{Très bien}, pensa Harry, \emph{si le Détraqueur est une énigme, quelle est la réponse~?}

Et juste comme ça, ce fut évident.

Harry regarda la cage ternie et légèrement rouillée.

Il vit ce qui se trouvait derrière la grande cape en lambeaux.

Très bien, dans ce cas.

Le professeur McGonagall s'approcha de Harry et lui parla. Elle n'avait pas vu le pire, et il n'y avait donc qu'un léger scintillement aqueux dans ses yeux. Harry lui dit qu'il faudrait qu'il lui parle plus tard et qu'il lui pose une question qu'il avait évitée de poser pendant longtemps, mais ça n'avait pas à se faire maintenant si elle était occupée. Il y avait quelque chose sur son visage qui suggérait qu'on l'avait éloignée de quelque chose d'important~; et Harry le lui fit remarquer, et il lui dit qu'elle ne devrait franchement pas se sentir coupable de partir. Cela lui valut un regard dur, mais elle partit bel et bien, avec la promesse qu'ils parleraient plus tard.

Dean Thomas lança de nouveau son ours blanc, même en présence du Détraqueur~; et Ron Weasley leva un bouclier convenable fait d'une brume étincelante. Ce qui concluait la journée en ce qui concernait les autres, et le professeur Flitwick commença à mener les élèves vers Poudlard. Lorsqu'il fut devenu clair que Harry comptait rester en arrière, le professeur Flitwick lui jeta un regard interrogateur~; quant à Harry, il jeta un regard lourd de sens à Dumbledore. Harry ne savait pas ce que Flitwick en comprit, mais après un dur regard d'avertissement, son directeur de Maison s'en fut.

Et il ne resta que Harry, le professeur Quirrell, le directeur Dumbledore et un trio d'Aurors.

Il aurait mieux valu commencer par se débarrasser du trio, mais Harry n'arrivait pas à trouver un bon moyen d'y parvenir.

«~Très bien, dit l'Auror Komodo, ramenons-le.

--- Excusez-moi, dit Harry. J'aimerais réessayer le Détraqueur.~»

\later

La requête de Harry rencontra une certaine dose d'opposition, de celles du genre \emph{mais vous êtes complètement cinglé}, même si l'Auror Butnaru fut le seul à le dire à voix haute.

«~Fumseck m'a dit de le faire~», dit Harry.

Cela ne surmonta pas toute l'opposition, malgré l'expression choquée qui apparut alors sur le visage de Dumbledore. Le débat continua, et cela commença à ronger une partie de ce qui restait de la paix du phénix, ce qui agaça Harry, mais un peu seulement.

«~Écoutez, dit Harry, je suis presque sûr de savoir ce que j'ai mal fait la dernière fois. Il existe des gens qui doivent utiliser un autre genre de pensée heureuse. Laissez-moi juste essayer, d'accord~?~»

Cela ne fut pas plus persuasif.

«~Je pense, dit enfin le professeur Quirrell en regardant Harry avec des yeux étroits, que si nous ne le laissons pas le faire sous supervision, il pourrait bien, à un moment ou un autre, nous fausser compagnie et chercher un Détraqueur seul. Vous accuserais-je faussement, M. Potter~?~»

Il y eut une pause horrifiée. Cela semblait un bon moment pour jouer l'atout qu'il avait gardé dans sa manche.

«~Ça ne me dérange pas que le directeur garde son Patronus~», dit Harry. \emph{Car je serai tout autant en présence du Détraqueur, Patronus ou pas.}

Cela fut accueilli par une grande confusion~; même le professeur Quirrell semblait perplexe~; mais le directeur finit par accepter, puisqu'il semblait peu probable que Harry puisse être atteint à travers quatre Patronus.

\emph{Si le Détraqueur ne pouvait traverser votre Patronus d'une façon ou d'une autre, Albus Dumbledore, vous ne verriez pas un homme nu et douloureux à regarder…}

Harry ne le dit pas à voix haute, pour des raisons évidentes.

Et ils commencèrent à marcher vers le Détraqueur.

«~Professeur, dit Harry, imaginons que la porte de Serdaigle vous pose cette énigme~: Qu'est-ce qui se trouve au centre d'un Détraqueur~? Que diriez-vous~?

--- La peur~», dit le directeur.

C'était une erreur facile à commettre. Le Détraqueur approchait, et la peur vous englobait. La peur faisait mal, vous la sentiez vous affaiblir, vous vouliez qu'elle s'en aille.

Il était naturel de penser que la peur était le problème.

Alors vous décidiez que le Détraqueur était une pure créature de peur, qu'il n'y avait rien à craindre sinon la peur elle-même, que le Détraqueur ne pouvait pas vous faire de mal si vous n'aviez pas peur…

Mais…

\emph{Qu'est-ce qui se trouve au centre d'un Détraqueur~?}

\emph{La peur.}

\emph{Qu'est-ce qui est tellement horrible que l'esprit refuse de le voir~?}

\emph{La peur.}

\emph{Qu'est-ce qu'on ne peut pas tuer~?}

\emph{La peur.}

… ça ne collait pas vraiment quand y réfléchissait.

Même si la raison pour laquelle les gens seraient réticents à regarder au-delà de la peur était assez claire.

Les gens \emph{comprenaient} la peur.

Les gens savaient ce qu'ils étaient censés \emph{faire} face à la peur.

Alors, face à un Détraqueur, il ne serait pas tout à fait réconfortant de se demander~: “Et si la peur était juste un effet collatéral plutôt que le problème principal~?”
Ils étaient maintenant très près de la cage du Détraqueur gardée par quatre Patronus, et il y eut soudain quatre inspirations courtes venant des trois Aurors et du professeur Quirrell. Tous les visages se tournèrent vers le Détraqueur et semblèrent écouter~; il y avait de l'horreur sur le visage de l'Auror Goryanof.

Puis le professeur Quirrell leva la tête, le visage dur, et il cracha en direction du Détraqueur.

«~Je suppose qu'il n'a pas aimé voir sa proie lui être enlevée, dit Dumbledore d'une voix douce. Eh bien, si cela devient nécessaire, Quirinus, Poudlard vous offrira toujours refuge.

--- Qu'a-t-il dit~?~» dit Harry.

Toutes les têtes se tournèrent vers lui.

«~Tu ne l'as pas entendu…~?~» dit Dumbledore.

Harry secoua sa tête.

«~Il m'a dit, dit le professeur Quirrell, qu'il me connaissait, et qu'il me pourchasserait un jour, où que je me cache.~» Son visage était rigide et n'exprimait aucune peur.

«~Ah, dit Harry, je ne m'inquiéterais pas pour ça, professeur Quirrell.~» \emph{Ce n'est pas comme si les Détraqueurs pouvaient vraiment parler, ou penser~; leur structure est empruntée à votre esprit et à ce à quoi vous vous attendez…}

Maintenant tout le monde lui jetait des regards \emph{très} bizarres. Les Aurors se jetaient des coups d'œil nerveux les uns aux autres, et au Détraqueur, et à Harry.

Et ils se tinrent directement en face de la cage du Détraqueur.

«~Il existe des blessures dans le monde, dit Harry. C'est une pure conjecture, mais j'imagine que c'est Godric Gryffondor qui a dit ça.

--- Oui… dit Dumbledore. Comment l'as-tu su~?~»

\emph{On croit souvent à tort}, pensa Harry, \emph{que tous les meilleurs rationalistes sont répartis à Serdaigle, n'en laissant aucun pour les autres Maisons. Ce n'est pas le cas~; être réparti à Serdaigle indique que votre vertu la plus forte est la curiosité, la capacité à s'interroger et à trouver la vraie réponse. Et ce n'est pas le} seule \emph{vertu dont un rationaliste a besoin. Parfois il faut travailler dur sur un problème, et s'y atteler pendant longtemps. Parfois il faut trouver un plan malin qui permet d'obtenir la réponse. Et parfois ce dont vous avez le plus besoin pour voir une réponse, c'est le courage d'y faire face…}

Le regard de Harry alla vers ce qui se trouvait sous la cape, vers l'horreur bien pire que n'importe quelle momie pourrissante. Rowena Serdaigle aurait pu le savoir, elle aussi, car c'était une énigme évidente une fois qu'on le voyait comme une énigme.

Et la raison pour laquelle les Patronus étaient des animaux était elle aussi évidente. Les animaux ne savaient pas, et ils étaient donc à l'abri de la peur.

Mais Harry savait, et il saurait toujours, et il ne pourrait jamais oublier. Il avait essayé de s'enseigner à faire face à la réalité sans faillir, et même s'il n'était pas encore devenu maître de cet art, ces sillons avaient quand même été creusés dans son esprit, ce réflexe acquis de regarder \emph{vers} le douloureux plutôt que de s'en détourner. Harry ne serait jamais capable d'oublier en ayant des pensées heureuses à propos d'autre chose, et c'est pour cela que le sort n'avait pas fonctionné pour lui.

Alors Harry aurait une pensée heureuse qui n'était \emph{pas} à propos d'autre chose.

Il leva sa baguette, que le professeur Flitwick lui avait rendue, et mit ses pieds dans la pose de départ du Patronus.

En son esprit, Harry abandonna les derniers restes de la paix du phénix, il mit le calme et la sensation de sommeil éveillé de côté, il préféra se souvenir du cri perçant de Fumseck et il se prépara à la bataille. Il demanda à toutes les parties de son être de s'éveiller. Il assembla en lui-même toute la force que le charme du Patronus pourrait jamais puiser, il se mit dans l'état d'esprit qui conviendrait à la pensée heureuse finale~; il se souvint de toutes les choses radieuses.

Les livres que son père lui avait achetés.

Le sourire de Maman quand Harry lui avait fabriqué une carte pour la fête des mères, un objet complexe qui avait utilisé près d'un quart de kilo de pièces électroniques détachées trouvées dans le garage et qui allumait des lumières et qui jouait une petite chanson et qu'il avait mis trois jours à construire.

Le professeur McGonagall qui lui disait que ses parents étaient morts dignement, en le protégeant. Et ils l'avaient fait.

Se rendre compte que Hermione tenait sa cadence, et courait même plus vite que lui, qu'ils pourraient être de vrais rivaux et des amis.

Ruser Drago pour qu'il sorte des ténèbres, le regarder se déplacer lentement vers la lumière.

Neville et Seamus et Lavande et Dean et tous ceux qui l'admiraient, tous ceux pour qui il se serait battu, pour les protéger si quoi que ce soit menaçait Poudlard.

Tout ce qui rendait la vie digne d'être vécue.

Sa baguette se mit en position pour le Patronus.

Harry pensa aux étoiles, à l'image qui avait presque retenu le Détraqueur, même sans Patronus. Sauf que cette fois, Harry ajouta l'ingrédient manquant, il ne l'avait jamais vraiment observé, mais il avait vu les images et la vidéo. La terre, d'un bleu éclatant, avec du blanc venu de la lumière solaire qu'elle réfléchissait, suspendue dans l'espace, au milieu du vide noir et des points de lumières brillants. Elle appartenait à cet endroit, à cette image, parce que c'était ça qui donnait du sens à tout le reste. La Terre était ce qui donnait du sens aux étoiles, qui faisait d'elles plus que des réactions de fusions incontrôlées, parce que c'était la Terre qui coloniserait un jour la galaxie et répondrait à la promesse du ciel nocturne.

Seraient-ils encore infestés de Détraqueurs, les enfants des enfants des enfants, les lointains descendants de l'humanité, pendant leur voyage d'étoile en étoile~? Non. Bien sûr que non. Les Détraqueurs n'étaient que de petites nuisances, qui à la lumière de cette promesse devenaient dérisoires, jusqu'à ne plus exister~; pas intuables, pas invincibles, loin de là. Il fallait supporter les petites nuisances si on était l'un des rares, chanceux et malchanceux, à être nés sur terre~; sur l'Ancienne Terre, comme on se souviendrait un jour d'elle. Cela aussi faisait partie du sens qu'il y avait à être en vie si l'on faisait partie de la petite poignée d'êtres sentients nés au début de toute chose, avant que la vie intelligente n'ait encore acquis tout son pouvoir. Le futur, bien plus vaste, dépendait de ce que vous faisiez ici, maintenant, aux plus précoces des jours de l'aube, alors qu'il y avait encore tant de ténèbres à combattre, et des nuisances temporaires telles que les Détraqueurs.

Maman et Papa, l'amitié de Hermione et le voyage de Drago, Neville et Seamus et Lavande et Dean, le ciel bleu et le soleil brillant et toutes les choses radieuses, la Terre, les étoiles, la promesse, tout ce que l'humanité était et tout ce qu'elle deviendrait…

Sur la baguette, les doigts de Harry prirent la position initiale~; il était prêt maintenant, prêt à penser le bon genre de pensée heureuse.

Et ses yeux regardèrent droit vers ce qui était sous la cape en lambeaux, droit vers ce qui avait été nommé un Détraqueur. Le néant, le vide, le trou dans l'univers, l'absence de couleur et d'espace, le drain ouvert à travers lequel la chaleur se déversait hors du monde.

La peur qu'il exsudait volait toutes les pensées heureuses, sa proximité drainait votre pouvoir et vos forces, son baiser détruisait tout ce que vous étiez.

\emph{Je te connais, à présent}, pensa Harry, tandis que sa main s'inclinait une fois, deux fois, trois fois et quatre fois, tandis que ses doigts glissaient des distances exactes, \emph{je comprends ta nature, tu symbolises la Mort, par une loi de la magie tu es une ombre que la Mort projette sur le monde.}

\emph{Et la Mort est une chose que je n'étreindrai jamais.}

\emph{Ce n'est qu'une chose infantile que l'espèce humaine n'est pas encore assez mûre pour avoir pu quitter.}

\emph{Et un jour…}

\emph{Nous nous en passerons…}

\emph{Et les gens n'auront plus à se dire adieu…}

La baguette s'éleva et pointa droit vers le Détraqueur.

«~\emph{EXPECTO PATRONUM~!}~»

La pensée explosa hors de lui comme une digue brisée, déferla le long de son bras et dans sa baguette, en jaillit sous la forme d'une lumière blanche éclatante. Une lumière qui devint corporelle, qui prit une forme et une substance.

Une silhouette avec deux bras, deux jambes et une tête, qui se tenait debout~; l'animal \emph{Homo sapiens}, la forme d'un être humain.

Brillant de plus en plus alors que Harry déversait toute sa force dans le sortilège, éclatant d'une lumière incandescente, plus brillante qu'un soleil bas, les Aurors et le professeur Quirrell se protégèrent les yeux sous le choc…

\emph{Et un jour, quand les descendants de l'humanité se seront répandus d'étoile en étoile, ils ne raconteront pas l'histoire de l'Ancienne Terre à leurs enfants avant qu'ils ne soient assez vieux pour le supporter~; et quand ils l'entendront, ils pleureront d'apprendre qu'une chose telle que la Mort avait un jour existé~!}

La silhouette d'un humain brilla plus fort que le soleil de midi, flamboyant d'une telle force que Harry put en sentir la chaleur sur sa peau~; et il envoya toute sa défiance vers l'ombre de la Mort, ouvrant toutes les vannes, pour rendre la forme brillante plus forte, toujours plus forte.

\emph{Tu n'es pas invincible, et un jour l'espèce humaine t'achèvera.}

\emph{Je t'achèverai si je le peux, par le pouvoir de mon esprit et de la magie et de la science.}

\emph{Je ne me recroquevillerai pas par peur de la Mort, pas tant que j'ai une chance de gagner.}

\emph{Je ne laisserai pas la Mort me toucher, je ne laisserai pas la Mort toucher ceux que j'aime.}

\emph{Et même si tu m'achèves avant que je ne t'achève,}

\emph{Un autre prendra ma place, et un autre,}

\emph{Jusqu'à ce que la blessure du monde soit enfin guérie…}

Harry abaissa sa baguette, et la brillante silhouette humaine se dissipa.

Il exhala lentement.

Comme s'il s'éveillait d'un rêve, comme s'il ouvrait les yeux après s'être endormi, son regard s'éloigna de la cage, il observa autour de lui et vit que tout le monde le regardait.

Albus Dumbledore le fixait.

Le professeur Quirrell le fixait.

Le trio d'Aurors le fixait.

Ils le regardaient tous comme s'ils venaient de le voir détruire un Détraqueur.

Dans la cage, la cape en lambeaux était vide.

%  LocalWords:  awkes’s
