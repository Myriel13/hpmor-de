\chapter{Compromis Tabous, Prélude~: Tricherie}

\lettrinepara{N}{ous} étions le samedi 4 avril, de l'année 1992.

\hplettrineextrapara
M. et Mme Davis semblaient plutôt nerveux, assis dans un partie assez spéciale des gradins du stade de Quidditch de Poudlard - bien qu'aujourd'hui les bancs rembourrés ne donnaient pas sur des balais volants mais sur un immense carré fait d'une matière semblable à du parchemin~: une immensité blanche et vide qui allait bientôt projeter l'image vacillante d'herbe et de soldats. Pour l'instant, elle ne montrait que le reflet de la couleur gris terne du ciel nuageux (qui semblait assez orageux, mais les sorciers-météo avaient promis que la pluie ne tomberait pas avant la nuit).

Il était d'ordinaire traditionnel à Poudlard que les simples parents se doivent de Rester à L'écart pour exactement les mêmes raisons qui poussent à interdire aux enfants d'entrer dans la cuisine et de se mêler des affaires du cuisinier. La seule cause possible à une rencontre parent-professeur aurait été qu'un professeur trouve qu'un parent ne rentrait pas assez dans le moule. Des circonstances exceptionnelles étaient requises pour que ce soit l'administration de Poudlard qui aie le sentiment d'avoir à \emph{se} justifier auprès de \emph{vous}. La plupart du temps et dans le cas général, l'administration de Poudlard était soutenue par huit-cents ans d'une éminente Histoire et vous ne l'étiez pas.

C'est donc avec une certaine trépidation que M. et Mme Davis avaient insisté pour avoir une audience avec la directrice adjointe, le professeur McGonagall. Il leur était difficile de se sentir suffisamment indigné alors qu'ils se confrontaient à la même digne sorcière qui, douze ans et quatre mois plus tôt, leur avaient donné à chacun deux semaine de retenue après les avoir surpris dans l'acte de concevoir Tracey.

Cela dit, M. et Mme Davis avaient retrouvé beaucoup de courage en agitant furieusement un exemplaire du \emph{Chicaneur} dont le gros titre disait, d'un texte gras et de couleur vive afin que tout le monde puisse bien le voir~:

\headline{Pactes avec Potter~?\\
Bones, Davis, Granger\\
Le rectangle amoureux de la peur}

Ainsi M. et Mme Davis avaient tempêté jusqu'à se retrouver dans la loge professorale des gradins de Poudlard, bien installés avec une excellente vue sur les écrans enchantés du professeur Quirrell afin de pouvoir tous deux se faire leur propre idée des "colleries qui pouvaient bien se passer ici, si vous me pardonnez l'expression, madame la directrice adjointe~!"

Assis à gauche de M. Davis se trouvait un autre parent soucieux, un homme aux cheveux blancs, vêtu de robes noires élégantes d'une qualité inégalable, un certain Lucius Malfoy, dirigeant politique de la plus forte faction du Magenmagot.

À gauche de Lord Malfoy, un aristocrate ricanant au visage balafré qu'on leur avait présenté sous le nom de Lord Jugson.

Puis un brave homme vieux mais à l'œil perçant, Charles Nott, dont on disait qu'il était presque aussi riche que Lord Malfoy, assis à gauche de Lord Jugson.

À droite de Mme Davis, on trouvait l'avenante Dame et l'encore plus séduisant Lord de la Très Ancienne maison Greengrass. Jeunes en âge de sorcier, costumés de robes soyeuses et grises dotées de petites émeraudes sombres brodées dans des contours de feuilles d'herbe. Dame Greengrass était considérée un vote décisif du Magenmagot, sa propre mère s'étant rétractée de l'assemblée à une vitesse surprenante. Quant à son charmant mari, il avait pris un siège au conseil d'administration de Poudlard bien que sa famille à lui ne soit ni noble ni riche.

À droite de ceux-ci, un vieille sorcière à la mâchoire carrée, l'air incroyablement coriace, qui avait serré les mains de M. et Mme Davis sans la moindre trace de condescendance. C'était Amelia Bones, directrice du département de justice magique.

À droite d'Amelia se trouvait une femme d'un âge assez mur qui avait défrayé les chroniques de mode d'Angleterre magique en intégrant un vautour vivant à son chapeau, une certaine Augusta Londubat. Même si on ne s'adressait pas à elle par le titre de Dame, Madame Londubat exercerait les pleins droits de la famille Londubat aussi longtemps que le dernier héritier de celle-ci n'aurait pas atteint sa majorité, et on la considérait comme une figure importante d'une faction mineure du Magenmagot.

À côté de madame Londubat ne se trouvait nul autre que l'Enchanteur en Chef, le Manitou suprême, Albus Percival Wulfric Brian Dumbledore, légendaire vainqueur de Grindelwald, protecteur d'Angleterre, redécouvreur des mythiques douze usages du sang de dragon, le plus puissant sorcier du monde, etc.

Et enfin, à l'extrême droite, on trouvait l'énigmatique professeur de Défense de Poudlard, Quirinus Quirrell, qui appuyait son dos sur les bancs rembourrés comme s'il se reposait et semblait parfaitement, naturellement à son aise en l'étouffante compagnie du quorum votant du conseil d'administration de Poudlard qui était passé par là en ce doux samedi pour apprendre quelles colleries pouvaient bien se dérouler à Poudlard en général et avec Drago Malfoy, Théodore Nott, Daphné Greengrass, Susan Bones et Neville Londubat en particulier. Le nom de Harry Potter avait lui aussi été beaucoup mentionné.

Oh, et il n'aurait bien sûr pas fallu oublier Tracey Davis. Les sourcils de la directrice Bones s'élevèrent d'intérêt en entendant que les parents de celle-ci n'étaient autre que ce jeune couple. Lord Jugson leur avait jeté un regard bref et incrédule avant de les congédier d'un grognement. Lucius Malfoy les avait accueillis poliment avec un sourire qui avait contenu un soupçon d'amusement morbide mêlé à de la pitié.

M. et Mme Davis, dont le dernier vote sur quoi que ce soit d'importance avait été de toucher le nom de Cornelius Fudge de leur baguette, qui n'avaient rien d'autre que trois-cents Gallions dans leur chambre forte à Gringotts, dont l'un vendait des chaudrons dans un magasin de potions et dont l'une enchantait des multiplettes, M. et Mme Davis était pressés l'un contre l'autre, assis très droit dans leurs bancs rembourrés, et souhaitaient désespérément avoir mis de meilleures robes avant de venir.

Le ciel au-dessus d'eux était un immense bloc de nuages composé de gris du plus sombre au plus clair, sinistres de promesses d'orages bien qu'aucun éclair n'ait encore éclaté ni qu'on entende l'écho lointain des roulements de tonnerre et que seules quelques gouttes menaçantes soient tombées.

\later

C'est vers le point de départ qui leur avait été assigné que le régiment Soleil avançait en rang, quoique d'un pas assez lent. Il valait mieux ne pas se fatiguer avant même que la bataille n'ait commencé et les brises d'avril étaient agaçantes d'humidité en dépit de leur fraîcheur. Devant eux, une flamme jaune errait lentement en l'air, les guidant en suivant leur rythme.

Alors qu'il marchaient dans la forêt baignée d'une lueur grisâtre, Susan Bones n'avait cesse de jeter des regards inquiets vers le général Soleil. L'attaque du professeur Rogue sur Hermione semblait avoir vraiment secoué cette dernière. Elle avait même manqué la réunion de préparation officielle du régiment Soleil, ce qui était assez compréhensible, mais lorsque Susan le lui avait ensuite dit, Hermione avait bégayé avoir perdu conscience du temps, ce qui ne lui ressemblait pas du tout, et elle avait eu l'air aussi épuisée et effrayée que si elle venait de passer trois jours enfermée dans les toilettes avec un Détraqueur. Même maintenant, alors que toute l'attention du général Soleil aurait dû être dirigée vers la bataille qui approchait, le regard de la Serdaigle passait constamment d'un point à l'autre, comme si elle s'attendait à voir des mages noirs surgir de buissons et la sacrifier.

"L'interdiction d'utiliser des objets moldus réduit beaucoup nos possibilités," dit Anthony Goldstein du ton sombre qu'il utilisait pour indiquer un pessimisme voulu. "J'ai pensé à métamorphoser des filets pour les jeter sur des gens mais -

--- Mauvaise idée," dit Ernie Macmillan. Le Poufsouffle secoua la tête d'un air encore plus sérieux que celui d'Anthony. "Enfin c'est exactement comme de lancer un sortilège~: ils l'\emph{éviteront."}

Anthony hocha la tête. "C'est ce que je me disais aussi. Tu aurais une idée, Seamus~?"

L'ancien lieutenant chaotique semblait encore nerveux et hors de son élément à marcher avec ses nouveaux camarades du régiment Soleil. "Désolé," dit le récemment nommé capitaine Finnigan. "Je suis plutôt un grand maître stratège.

--- \emph{Je} suis plutôt un grand maître stratège," dit Ron Weasley comme s'il était dégoûté.

"Il y a \emph{trois} armées," dit le général Soleil d'un ton acerbe, "donc on combat \emph{deux} armées à la fois, donc on a besoin de plus d'un seul stratège, donc la ferme, Ron~!"

Ron jeta un regard surpris et inquiet au général. "Hé," dit le Gryffondor d'un ton apaisant, "tu ne devrais pas laisser Rogue te stresser autant -

--- Et que pensez-\emph{vous} qu'on devrait faire, général~?" dit Susan très fort et très vite. "Franchement, je n'ai pas vraiment de plan pour l'instant." La réunion de planification officielle avait \emph{incroyablement} foiré avec une Hermione absente et un Ron et un Anthony tous les deux persuadés d'être les chefs.

"A-t-on vraiment besoin d'un plan~?" dit le général Soleil d'un ton légèrement distrait. "Il y a toi, moi, Lavande, Parvati, Hannah, Daphné, Ron, Ernie, Anthony \emph{et} le capitaine Finnigan.

--- Ça -" commença Anthony.

"M'a l'air d'être une assez bonne stratégie," dit Ron avec un hochement de tête approbateur. "Nous avons autant de bon soldats que dans les deux autres armées réunies. Chaos n'a plus que Potter, Londubat et Nott - enfin, et Zabini aussi j'imagine -

--- Et Tracey," dit Hermione.

Plusieurs personnes déglutirent nerveusement.

"Oh, \emph{arrêtez}," dit Susan d'un ton sec. "C'est juste une membre de la S.P.E.H.S. endurcie au combat, c'est tout ce que le général Soleil voulait dire.

--- Quand même," dit Ernie en se retournant pour regarder Susan avec sérieux, "je pense que vous feriez mieux d'aller avec le groupe qui combattra Chaos, capitaine Bones. Je sais que vous ne pouvez pas utiliser vos pouvoirs magiques sauf lorsque des innocents sont en danger, mais je pense que - juste au cas où Mlle Davis, vous savez, deviendrait incontrôlable et essaierait de dévorer une âme -

--- Je peux m'occuper d'elle," répondit Susan en gardant une voix rassurante. Il fallait bien admettre qu'elle n'avait pas pour le moment été remplacée par un Métamorphomage, mais après tout Tracey n'était probablement pas un Dumbledore polynectaré non plus.

Le capitaine Finnigan entonna d'une voix grave et presque rocailleuse~: "Je suis troublé par votre manque de scepticisme." Il leva une main, pouce et index presque collés, et la pointa vers Ernie.

Sans raison apparente, Anthony Goldstein sembla être pris d'une quinte de toux. "Qu'est-ce que c'est censé vouloir dire~?" demanda Ernie.

"C'est juste quelque chose que le général Potter dit parfois," continua le capitaine Finnigan. "C'est drôle, quand on rejoint la légion du Chaos, ça a d'abord l'air complètement fou, et puis au bout de deux mois on se rend compte qu'en fait que ce sont tous ceux qui ne sont \emph{pas} dans la légion du Chaos qui sont fous -

--- J'ai \emph{dit}," continua Ron d'une voix forte, "que ça m'a l'air d'être une bonne stratégie. On ne métamorphose rien, on ne se fatigue pas, on gère tout ce qu'ils nous balancent, et on finit par les surmonter.

--- D'accord," dit Hermione. "Faisons ça.

--- Mais -" dit Anthony en jetant un regard furieux à Ron. "Mais général, Harry Potter n'a plus que \emph{seize} personnes dans son armée. Dragon et nous en avons vingt-huit chacun. Harry le \emph{sait}, il sait qu'il \emph{doit} trouver quelque chose d'\emph{incroyable} -

--- Comme \emph{quoi}~?" demanda Hermione d'un ton stressé. "Si on ne \emph{sait pas} ce qu'ils préparent, autant économiser notre magie pour lancer des \emph{Finite} en masse. Comme on \emph{aurait} dû le faire la \emph{dernière} fois~!"

Susan toucha doucement l'épaule de Hermione. "Général Granger~?" demanda-t-elle. "Je pense que vous devriez vous reposer un moment avant la bataille."

Elle s'était attendue à ce que Hermione proteste mais celle-ci hocha juste la tête et marcha un peu plus vite, s'écartant du groupe officiel des officiers du régiment Soleil en continuant à surveiller la forêt et parfois le ciel du regard.

Susan la suivit. Il n'aurait pas été convenable que le général Soleil ait l'air d'avoir été exclus de son propre cercle d'officiers.

"Hermione~?" dit Susan avec douceur après qu'elles se furent un peu éloignées. "Tu dois te concentrer. C'est le professeur Quirrell qui contrôle tout ici, pas Rogue, et il ne laissera rien de mal t'arriver, ni à toi ni à personne d'autre.

--- Vous ne m'aidez pas," dit Hermione d'une voix tremblante. "Vous ne m'aidez pas du tout, capitaine Bones."

Elles marchèrent plus vite, firent un cercle autour de certains soldats, inspectèrent le périmètre, observèrent les arbres alentours.

"Susan~?" dit Hermione d'une petite voix après qu'elles se furent retrouvées plus loin des autres. "Est-ce que tu penses que Daphné a raison quand elle dit que Drago Malfoy manigance quelque chose~?

--- Oui," répondit immédiatement Susan sans même y réfléchir. "C'est facile à savoir~: il y a les lettres M, A, L, F, O et Y dans son nom de famille."

Hermione regarda autour d'elle comme pour s'assurer que personne ne les regardait, même si c'était bien sûr un merveilleuse façon d'attirer l'attention. "Malfoy aurait-il pu être derrière ce que Rogue a fait~?

--- Rogue pourrait être derrière Malfoy," dit Susan, pensive, en se souvenant les conversations qu'elle avait entendues lors de dîners chez sa tante, "ou Lucius Malfoy pourrait être derrière les deux." Un léger frisson descendit le long de son échine lorsqu'elle eut cette pensée. Soudain, dire à Hermione de juste se concentrer sur la bataille à venir lui sembla être beaucoup moins raisonnable. "Pourquoi, est-ce que tu as trouvé un genre d'indice qui te ferait penser ça~?"

Hermione secoua la tête. "Non," dit la Serdaigle d'une voix qui laissait presque penser qu'elle était sur le point de pleurer. "Je… j'y songeais juste… c'est tout."

\later

Dans la zone de la forêt proche de Poudlard qui leur avait été assignée, le général Dragon et les guerriers de l'armée Dragon attendaient là où une flamme rouge les avait guidés sous le ciel gris.

À la droite de Drago se tenait Padma Patil, sa commandante en second, qui avait un jour mené toute l'armée Dragon un jour où Drago s'était fait assommer. Derrière lui se trouvait Vincent, le fils des Crabbe, une famille qui avait servi les Malfoys depuis des temps oubliés. Le garçon musclé était vigilant, comme il l'était toujours, qu'une bataille ait été déclarée ou non. Plus loin en arrière, Grégory des Goyles se tenait à côté de l'un des deux balais qui avaient été fournis à l'armée Dragon~; et si les Goyles n'avaient pas servi les Malfoys aussi longtemps que les Crabbes, ils ne l'avaient pas moins bien fait.

À la gauche de Drago se tenait maintenant Dean Thomas de Gryffondor, un Sang-de-Bourbe ou peut-être un Sang-Mêlé qui ignorait tout de son père.

Envoyer Dean Thomas à l'armée Dragon avait été un geste délibéré de la part Harry, Drago en était certain. Trois autres anciens chaotiques avaient été transférés vers l'armée Dragon et ils scrutaient tous Drago pour voir s'il insultait l'ancien lieutenant de quelque façon que ce soit.

Certains auraient appelé cela du sabotage, mais Drago n'était pas dupe. Harry avait aussi envoyé le lieutenant Finnigan au régiment Soleil alors que le mandat du professeur Quirrell n'avait requis l'abandon que \emph{d'un seul} lieutenant. Cela aussi avait été fait délibérément, dans le but de rendre parfaitement clair à tous que Harry ne se débarrassait \emph{pas} des soldats qu'il aimait le moins.

En un sens, il aurait pu être plus simple pour Drago de gagner la loyauté de ses nouveaux soldats si ceux-ci avaient cru que Harry ne voulait pas d'eux. Mais vu sous un autre angle… c'était difficile à exprimer en mots. Harry lui avait donné de bons soldats à l'amour-propre intact, mais ce n'était pas tout. Il avait fait preuve de gentillesse envers ses soldats, mais ce n'était toujours pas tout. Ce n'était pas seulement que Harry jouait franc jeu mais aussi un comportement qu'on… qu'on ne pouvait s'empêcher de mettre en opposition avec la façon dont le jeu se jouait au sein de Serpentard.

Drago n'avait donc pas fait la plus petite insulte à M. Thomas mais l'avait plutôt positionné à ses côtés, le plaçant en dessous de lui et de Padma mais de personne d'autre. C'était un test, avait dit Drago à M. Thomas et aux autres, pas une promotion. M. Thomas devrait se montrer digne de son rang dans l'armée Dragon - mais il aurait sa chance, offerte avec bonne foi. M. Thomas avait semblé surpris par l'aspect cérémonieux de la chose (Drago avait entendu dire que la légion du Chaos n'appréciait pas les formalités) mais le Gryffondor s'était légèrement raidi et avait hoché la tête.

Puis, après que M. Thomas se fut bien sorti de l'une des sessions d'entraînement de l'armée Dragon, il avait été emmené au conseil stratégique qui avait lieu dans l'énorme bureau militaire de l'armée Dragon. Et au bout de quelques minutes, Padma lui avait demandé - comme si c'était une question parfaitement normale - s'il n'aurait pas quelque idée sur la façon de vaincre la légion du Chaos.

Le Gryffondor avait joyeusement répondu que Harry avait prédit que le général Malfoy s'arrangerait pour que l'un des soldats de ce dernier lui pose la question et qu'il avait reçu du général Chaos un message disant que Drago devrait se demander quel était son avantage relatif, ce qu'il pouvait faire ou ce que l'armée Dragon pouvait faire qui soit impossible à la légion du Chaos, puis de s'essayer à exploiter cet avantage au maximum. Dean Thomas n'avait aucune idée de ce que cet avantage pourrait être mais \emph{si} une idée intéressante quant à la façon de vaincre Chaos lui venait, il leur en ferait part. Après tout, Harry lui avait ordonné de le faire.

Puisqu'il était exclu qu'il le fasse haut et fort, Drago avait soupiré intérieurement. Mais le conseil était bon et Drago l'avait suivi, assis au bureau de sa chambre muni d'une plume et d'un parchemin, listant tout ce qui pourrait s'avérer constituer un avantage relatif.

Et, presque à sa propre surprise, il avait eu une \emph{idée}, une vraie. En fait, il en avait eu \emph{deux}.

Le son creux de la cloche sonna à travers la forêt, parvenant à être encore plus menaçante que jamais. À cet instant, deux pilotes s'écrièrent "\emph{Debout~!}" et bondirent sur leur balai en direction du ciel gris.

\later

M. et Mme Davis étaient maintenant affalés l'un sur l'autre, plus par fatigue musculaire qu'à cause d'une diminution de leur état de tension. Devant eux, l'immense parchemin blanc et vierge affichait trois immenses fenêtres comme si des trous menant à la forêt y avaient été découpés. Elles montraient trois armées en marche. De plus petites fenêtres montraient six élèves juchés sur leur balai et le coin du parchemin révélait une vue globale de la forêt, avec des points lumineux pour indiquer les armées et les éclaireurs.

La fenêtre qui donnait sur Soleil montrait le général Granger et ses capitaines, marchant au centre du régiment Soleil et protégés par des écrans de \emph{Contego} ainsi que par nombre de jeunes sorcières. Le professeur de Défense avait fait remarquer que le régiment Soleil savait fort bien qu'il venait d'acquérir un grand avantage sous la forme de soldats expérimentés et comptait bien protéger ces derniers d'une attaque surprise. Mis à part ça, les soldats Soleil continuaient d'avancer et conservaient leurs forces.

Les soldats de l'armée du général Malfoy, ou du moins ceux qui avaient les meilleures notes de métamorphose, ramassaient des feuilles et les métamorphosaient en… eh bien, si on regardait Padma Patil, qui en avait presque fini avec la sienne, il semblait que sa feuille devenait un gant gauche doté d'une sangle encore pendante (la fenêtre venait de faire un zoom pour le montrer).

Lord Jugson regardait l'écran sans montrer le moindre intérêt~; sa voix, lorsqu'il parla, sembla suinter, ruisseler de dédain~: "Que fait \emph{donc} votre fils, Lucius~?"

La sorcière venue de l'étranger qui se tenait à droite de Malfoy venait de finir de métamorphoser son gant et le portait à présent au général Dragon comme elle lui aurait fait offrande d'un sacrifice.

"Je l'ignore," dit Lucius Malfoy d'un ton aussi calme qu'aristocratique, "mais il me faut croire qu'il a de bonnes raisons de le faire."

Toute l'armée Dragon s'arrêta un instant lorsque Padma fit glisser le gant sur sa main gauche, attacha la sangle et le présenta à Drago Malfoy qui s'arrêta lui aussi, prit quelques profondes inspirations, leva sa baguette, exécuta huit mouvements précis et mugit~: "\emph{Collaporta~!}"

La guerrière Dragon leva alors sa main, fit jouer ses articulations et offrit un petit salut à Drago Malfoy qui lui en rendit un encore plus léger tout en chancelant un peu. Padma retourna alors au côté de Drago et les dragons se remirent en marche.

"Alors," dit Augusta Londubat, "j'imagine que personne ne voudrait expliquer…~?" Amelia Bones fronçait légèrement les sourcils en regardant l'écran.

"Pour une raison ou une autre," dit la voix amusée du professeur Quirrell, "il semble que l'héritier des Malfoys est capable de manier une magie étonnamment puissante pour un élève de première année. À cause de la pureté de son sang, bien sûr. Lord Malfoy n'aurait certainement pas bafoué les lois sur la magie des mineurs en s'arrangeant pour que son fils reçoive une baguette avant d'être admis à Poudlard.

--- Je vous suggère de faire attention à ce que vous sous-entendez, Quirrell," dit Lucius Malfoy avec froideur.

"Oh, mais c'est ce que je fais," dit le professeur Quirrell. "Un \emph{Collaporta} ne peut être défait par un \emph{Finite Incantatem}~; il nécessite un \emph{Alohomora} de force égale. Sans cela, un gant ainsi ensorcelé résistera à des forces physique faibles et défléchira le sortilège de sommeil ainsi que celui d'étourdissement. Et comme ni M. Potter ni Mlle Granger ne sont capables de lancer un contre-sort assez puissant, ce charme est donc invincible sur ce champ de bataille. Ce n'est pas le but originel ni l'intention qu'avait la personne qui a enseigné à M. Malfoy un sortilège d'urgence destiné à lui permettre d'échapper à ses ennemis. Mais il semblerait que M. Malfoy ait reçu des leçons de créativité."

Lucius Malfoy s'était raidit dans son siège à mesure que le professeur de Défense avait parlé, et il se tenait maintenant droit sur son banc rembourré, sa tête tenue sensiblement plus haut qu'avant, et lorsqu'il parla, ce fut avec une fierté contenue~: "Il sera le plus grand Lord Malfoy à avoir jamais vécu.

--- Bien médiocre éloge," marmonna Augusta Londubat~; Amelia Bones gloussa, tout comme le fit M. Davis pendant une fraction de seconde fatale avant de s'arrêter dans un gargouillement étranglé.

"Je suis tout à fait d'accord," dit le professeur Quirrell, bien qu'on ne sache pas bien à qui il s'adressait. "Malheureusement pour M. Malfoy, il est encore débutant en matière de créativité et a donc commis une classique erreur de Serdaigle.

--- Et quelle est-elle~?" demanda Lucius Malfoy d'une voix redevenue fraîche.

Le professeur Quirrell s'enfonça dans son siège et les pâles yeux bleus perdirent brièvement leur mise au point lorsque l'une des fenêtres changea de point de vue et zooma pour montrer la sueur qui se trouvait à présent sur le front de Drago. "C'est une idée si belle que M. Malfoy a négligé sa difficulté pratique.

--- Quelqu'un pourrait-il m'expliquer cela~?" dit Lady Greengrass. "Nous ne sommes pas tous experts en… ce domaine."

Amelia Bones répondit, et la voix de la vieille sorcière était plutôt sèche. "Ils seront tentés d'attraper des sortilèges qu'il aurait été plus simple d'éviter. Encore plus s'ils ont un peu pratiquer l'attrapage auparavant. Et lancer tant de sortilèges fatiguerait leur guerrier le plus fort."

Le professeur Quirrell donna un demi hochement de tête de reconnaissance à la directrice du département de justice magique. "Comme vous dites, Mme Bones. M. Malfoy est débutant en nouvelles idées, et il devient donc fier de lui lorsqu'il en a une. Il n'a pas encore eu assez d'idées pour se débarrasser sans hésitation de celles qui sont en partie belles et en partie irréalisables~; il n'a pas encore acquis assez de confiance dans sa capacité à trouver de nouvelles idées à mesure que le besoin s'en fait sentir. Ce que nous voyons la n'est pas la meilleure idée de M. Malfoy, j'en ai peur, mais plutôt sa seule idée."

Lord Malfoy se détourna tout simplement et regarda les écrans comme si le professeur de Défense venait d'épuiser son droit à exister.

"Mais -" dit Lord Greengrass. "Mais par Merlin, \emph{que} \emph{fait} Harry Potter~?"

\later

Les seize soldats restants de l'armée du Chaos - ou plutôt les quinze soldats restants accompagnés de Blaise Zabini - avançaient avec confiance dans la forêt, leurs chaussures battant le sol encore sec. Leurs uniformes de camouflage se fondaient encore plus dans la forêt que d'habitude car leurs couleurs étaient délavées par la lumière d'un ciel nuageux.

Seize légionnaires du Chaos contre vingt-huit guerriers Dragon et vingt-huit soldats Soleil.

Le consensus avait été qu'avec des chances \emph{aussi mauvaises}, il était quasiment impossible qu'ils perdent. Après tout, le général Chaos allait \emph{forcément} inventer quelque chose de vraiment \emph{spectaculaire}, face à des chance pareilles.

Il y avait quelque chose de presque cauchemardesque dans la façon dont tout le monde semblait maintenant \emph{attendre} de Harry qu'il fasse sortir des miracles de son chapeau sur demande à chaque fois que le besoin s'en faisait sentir. Ça voulait dire que s'il ne pouvait pas accomplir l'impossible, alors il \emph{décevait ses amis} et il \emph{n'était pas à la hauteur de son potentiel}…

Harry n'avait pas pris la peine de se plaindre après de professeur Quirrell que la 'pression était trop forte'. Le modèle mental du professeur Quirrell qu'avait Harry avait prédit qu'il aurait l'air particulièrement agacé, dirait des choses comme \emph{Vous êtes parfaitement capable de résoudre ce problème, M. Potter, avez-vous au moins essayé~?} et lui enlèverait ensuite plusieurs centaines de points Quirrell.

Venue d'au-dessus, où deux balais surveillaient l'avancée de l'armée, la jeune voix haute perchée de Tess Walsh cria "Ami~!" et un instant plus tard~: "Croquignolle~!"

Une poignée de secondes plus tard, la soldate qui s'était elle-même affublé du nom de code Croquignolle revint les mains chargées de glands, suant légèrement d'avoir trotté par ce temps frais mais humide jusqu'au chêne que Neville avait repéré. Croquignolle s'approcha de Shannon, qui portait un T-shirt d'uniforme dont le cou avait été noué afin de ne pas avoir à métamorphoser un sac. Lorsque Croquignolle avança ses mains et essaya de laisser tomber les glands dans le T-shirt-sac, Shannon du Chaos décala le vêtement à gauche en gloussant, puis à droite lorsque Croquignolle fit une autre tentative pour déposer les glands, jusqu'à ce qu'un "Mlle Friedman~!" tranchant venu du lieutenant Nott ne fasse soupirer Shannon avant qu'elle ne stabilise le T-shirt. Croquignolle versa les glands par-dessus ceux qui avaient déjà été accumulés puis repartit en chercher d'autres.

Quelque part à l'arrière-plan, Ellie Knight chantait sa version personnelle de la marche de la légion du Chaos et près de la moitié des autres soldats essayaient de la suivre sans pour autant en connaître l'air. Non loin, Nita Berdine, qui avait de bonnes notes de métamorphose, acheva de créer une autre paire de lunettes de soleil vertes et les tendit à Adam Beringer qui les plia avant de les ranger dans sa poche d'uniforme. D'autres soldats portaient déjà leurs propres lunettes vertes malgré le ciel nuageux.

Vous auriez pu soupçonner qu'une explication incroyablement compliquée et fascinante se cachait derrière tout cela, et vous auriez eu raison.

Deux jours plus tôt, assis dans sa bibliothèque, dans la chaise à bascule qu'il avait dégotée pour le niveau caverne de son coffre, lors du moment calme entre les cours et le dîner, Harry avait silencieusement médité sur la notion de pouvoir.

Pour que seize chaotiques battent vingt-huit soleils et vingt-huit dragons, ils auraient besoin d'un amplificateur de force. Il y avait des limites à ce que les manœuvres permettaient d'accomplir. Il \emph{fallait} avoir une arme secrète, et elle devait être invincible, ou du moins modérément inarrêtable.

Les objets moldus étaient maintenant interdits lors des fausses batailles de Poudlard, bannis par un décret du ministère. Et le problème qu'il y avait à chercher un sortilège malin et peu courant c'était qu'une armée deux fois plus puissante que la vôtre pouvait \emph{Finite} de force à peu près tout ce que vous auriez essayé. Le régiment Soleil avait peut-être raté cette tactique face à la cotte de maille métamorphosée, mais plus personne ne la manquerait maintenant que le professeur Quirrell l'avait clairement mise en évidence. Et \emph{Finite Incantatem} était un contre-sort qui marchait à la force pure et demandait au moins autant de magie que le sortilège à annuler… ce qui, si vous étiez sérieusement surpassés en nombre, devenait un défi militaire d'une toute autre dimension. L'ennemi pouvait \emph{Finite} tout ce que vous essayiez et avoir encore assez de magie en réserve pour des boucliers et des volées de sortilèges de sommeil.

À moins de parvenir à faire appel à des forces supérieures à celles des élèves de Poudlard en première année, à quelque chose de trop fort pour qu'un ennemi puisse le \emph{Finite}.

Harry avait donc demandé à Neville s'il avait jamais entendu parler de rituels sacrificiels \emph{mineurs} et \emph{sûrs} -

Puis lorsque les cris et les hurlements s'étaient apaisés, que Harry avait arrêté d'essayer de discuter des Serments Inviolables et qu'il avait entièrement abandonné le projet, clairement impossible en termes de relations publiques, il s'était rendu compte qu'il n'avait même pas eu besoin d'aller jusque là. On apprenait à faire appel à des forces bien supérieures à la sienne à Poudlard même, en cours.

Parfois, même quand on regardait directement quelque chose, il fallait attendre d'avoir posé exactement la bonne question pour savoir ce qu'on avait sous les yeux.

\emph{Défense. Charmes. Métamorphose. Potions. Histoire de la magie. Astronomie. Vol sur balai. Botanique…}

"\emph{Ennemi~!}" s'écria une voix au-dessus d'eux.

\later

Il était heureux que Neville Londubat ignore entièrement que sa grand-mère le regardait, sans quoi il aurait été plus gêné à l'idée de hurler des cris de guerre de toutes ses forces tout en lançant \emph{Luminos} toutes les trois secondes et en fonçant comme une fusée à travers une dense forêt d'arbres à la poursuite de Gregory Goyle.

("Mais -" dit Augusta Londubat, avec une expression qui montrait presque autant de stupéfaction que d'inquiétude. "Mais Neville a peur du vide~!")

("Toutes les peurs ne durent pas," dit Amelia Bones. La vieille sorcière scrutait le plus grand écran d'un regard calculateur. "Ou peut-être a-t-il trouvé du courage. C'est en fin de compte la même chose").

Une lueur rouge -

Neville évita et faillit percuter un arbre mais il \emph{évita}, et il parvint alors à éviter aussi \emph{presque} toutes les branches avant qu'elles ne le fouettent en plein visage.

Le balai de M. Goyle s'éloignait à présent de plus en plus - alors qu'ils étaient tous les deux juchés sur le même modèle de balai et que M. Goyle pesait plus lourd, Neville se faisait quand même distancer. Il ralentit donc, vira de bord, monta au-dessus de la forêt et commença à accélérer vers la légion du Chaos qui progressait toujours.

Vingt secondes plus tard -- la course poursuite n'avait pas été \emph{longue}, seulement \emph{exaltante} -- Neville était de retour parmi ses compagnons chaotiques et descendait de son balai pour faire quelques pas.

"Neville -" dit le général Potter. La voix de Harry était un peu distante~; il traversait la forêt précautionneusement mais d'un pas ferme et sa baguette était toujours plaquée contre la forme presque achevée de l'objet qu'il métamorphosait lentement. À côté de lui, Blaise Zabini travaillait sur une version plus petite de la même métamorphose et trébucha soudain, ce qui le fit ressembler à un Inferi rampant. "Je te l'ai dit, Neville, tu n'as pas à…

--- Si," répondit Neville. Il baissa ses yeux vers ses doigts fraîchement arrachés au balai et vit que ce n'étaient pas seulement ses mains mais aussi ses bras qui tremblaient. Mais à moins qu'un autre chaotique n'ait pratiqué le duel une heure par jour avec M. Diggory avant de s'exercer à viser en privé pendant une heure de plus, Neville était probablement le meilleur tireur sur balai de leur armée, et ce même s'il ne volait pas très bien.

"Tu t'en es bien tiré, Neville," dit Théodore, en tête et à l'écart du groupe, menant la légion à travers la forêt uniquement vêtu de son maillot de corps.

(Augusta Londubat et Charles Nott échangèrent un bref coup d'œil stupéfait puis arrachèrent leur regard l'un à l'autre comme si quelque chose venait de les piquer).

Neville prit quelques profondes inspirations, essaya de stabiliser ses mains, de réfléchir~; Harry n'était peut-être pas au mieux de sa forme stratégique au milieu d'une longue métamorphose. "Lieutenant Nott, sauriez-vous pourquoi l'armée Dragon vient de faire cela~? Ils ont \emph{perdu} un balai -" Les Dragons avaient commencé le combat par une feinte afin de les distraire de l'approche de M. Goyle~; Neville n'avait remarqué qu'il y avait \emph{deux} balais attaquants que peu de temps avant qu'il ne soit trop tard. Mais la légion du Chaos avait \emph{eu} l'autre pilote. C'était pour cette raison que les balais n'attaquaient généralement pas avant que les armées ne se rencontrent~: toute une armée pouvait alors concentrer son tir sur un seul balai. "Et les Dragons n'ont eu personne, n'est-ce pas~?

--- Nan~!" dit Tracey Davis avec fierté. Elle marchait maintenant à côté du général Potter, sa baguette maintenue sous sa taille, le regard vigilant, passant en revue la forêt qui les entourait. "J'ai lancé une sphère prismatique pas loin d'une fraction de seconde avant que le sort de M. Goyle ne touche Zabini, et vu comment M. Goyle avait étendu son autre bras je pense qu'il comptait aussi faire tomber le général." La sorcière de Serpentard sourit avec un air d'assurance malveillante. "M. Goyle a essayé un sortilège de bris de bouclier mais a appris à son grand dam que sa faible magie ne pouvait pas rivaliser avec mes nouveaux pouvoirs des ténèbres, hahahaha~!"

Quelques chaotiques rirent avec elle mais Neville sentit son estomac commencer à se retourner lorsqu'il se rendit compte à quel point la légion du Chaos avait été proche du désastre complet. Si M. Goyle était parvenu à perturber les deux métamorphoses…

\later

"Au rapport~!" lâcha le général Dragon tout en faisant de son mieux pour dissimuler la fatigue qu'il ressentait après avoir lancé les dix-sept premiers sortilèges d'emprisonnement.

Des gouttes de sueur perlaient maintenant au front de Gregory. "L'ennemi a eu raison de Dylan Vaughan," dit-il d'un ton cérémonieux. "Harry Potter et Blaise Zabini métamorphosaient chacun quelque chose de gris sombre et de plus ou moins circulaire, je ne pense pas que c'était terminé mais on aurait dit que ça allait être grand et creux, peut-être en forme de chaudron. Celui de Zabini était plus petit que celui de Potter. Je n'ai pas pu les avoir ni perturber leur métamorphose, Tracey Davis m'a bloquée. Neville Londubat est sur un balai et il est toujours exécrable en vol mais il vise vraiment bien."

Drago écouta, fronça les sourcils puis regarda vers Padma et Dean Thomas qui secouaient tous les deux la tête, indiquant par là qu'ils n'avaient pas idée de ce qui pourrait être grand, gris et en forme de chaudron.

"Autre chose~?" dit Drago. Si c'était tout, ils auraient perdu un balai pour rien -

"L'autre chose étrange que j'ai vue," dit Gregory d'un ton perplexe, "c'était que certain chaotiques portaient… des sortes de lunettes~?"

Drago y réfléchit sans remarquer qu'il avait cessé de marcher et que toute l'armée Dragon s'était automatiquement arrêtée avec lui.

"Ces lunettes avaient-elles quoi que ce soit de spécial~?" demanda Drago.

"Euh…" dit Gregory. "Elles étaient… verdâtres, peut-être~?

--- D'accord," dit Drago. De nouveau sans s'en rendre compte, il recommença à marcher et ses dragons le suivirent. "Voilà notre nouvelle stratégie. Nous allons envoyer onze dragons contre la légion du Chaos, pas quatorze. Cela devrait suffire à les battre maintenant qu'on peut neutraliser leur avantage." C'était un coup de poker, mais il fallait être prêt à en faire quand on voulait sortir gagnant d'un combat à trois.

"Vous avez compris le plan de Chaos, général Malfoy~?" dit M. Thomas, considérablement surpris.

"Que font-ils~?" demanda Padma.

"Je n'en ai pas la moindre idée," dit Drago avec un sourire d'une suffisance des plus raffinées. "Nous ferons juste ce qu'il y a de plus évident à faire."

\later

Harry en avait maintenant fini avec son chaudron et versait précautionneusement des glands dans le récipient pendant que les éclaireurs cherchaient une source d'eau qui pourrait servir de base liquide. Ils avaient déjà fréquemment croisé des dolines et de petites criques dans la forêt, cela ne prendrait donc certainement pas longtemps. Un autre éclaireur avait apporté un bâton droit qui permettrait de remuer sans que Harry n'aie à en métamorphoser un.

Parfois, même quand on regardait directement quelque chose, il fallait attendre d'avoir posé exactement la bonne question pour savoir ce qu'on avait sous les yeux.

\emph{Comment puis-je faire appel à des forces qui devraient être hors de portée d'élèves en première année~?}

Il y avait ce récit édifiant raconté par le maître des potions (avec nombre de railleries et ricanements afin que la bêtise dont il était question semble plus dégradante qu'audacieuse ou romantique) au sujet d'une sorcière en deuxième année à Beauxbâtons qui avait volé des ingrédients extrêmement réglementés et chers afin d'essayer de préparer du \emph{Polynectar} dans le but d'emprunter l'apparence d'une autre fille à des fins qu'il vaut mieux ne pas mentionner. Sauf qu'elle était parvenue à contaminer la potion avec des \emph{poils de chat}~; et ensuite, au lieu d'aller immédiatement chercher un guérisseur, la sorcière s'était cachée dans les toilettes en espérant que les effets s'estomperaient, et lorsqu'elle fut enfin trouvé il était trop tard pour entièrement annuler la métamorphose ce qui la condamna à vivre la vie tragique d'une espèce d'hybride femme-chat.

Harry n'avait pas compris ce que cela \emph{impliquait} avant d'avoir pensé à la bonne question -- ce que cela impliquait, c'était qu'un jeune sorcier ou une jeune sorcière pouvait faire des choses en préparant des potions qu'ils n'auraient \emph{aucune chance} de pouvoir faire avec des sortilèges. Le Polynectar était l'une des potions les plus puissantes connues… mais ce qui faisait du Polynectar une potion de niveau ASPIC n'était apparemment pas l'âge avant lequel on aurait assez de pouvoir magique pour pouvoir la préparer mais la difficulté qu'il y avait à \emph{précisément} préparer la potion et ce qu'il advenait quand on se ratait.

Aucun membre des trois armées n'avait encore essayé de préparer des potions. Mais le professeur Quirrell laissait tout faire tant que la chose aurait été possible dans une véritable guerre. La triche est une technique, leur avait-il un jour professé. \emph{Ou plutôt, la technique est ce que les perdants appellent la triche et vous rapportera des points Quirrell supplémentaires si vous la pratiquez avec succès.} Et en principe, il n'y avait rien \emph{d'irréaliste} dans le fait de métamorphoser deux chaudrons et d'y préparer des potions à partir de ce qu'on avait sous la main si tant est que l'on avait assez de temps avant que les armées ne se rencontrent.

Harry avait donc récupéré un exemplaire de \emph{Breuvages et Potions Magiques} et avait commencé à chercher une potion sûre mais utile qu'il pourrait préparer pendant les minutes précédant le début de la bataille -- une potion qui ferait gagner trop vite pour qu'on ait le temps de lancer des contre-sorts ou qui produirait des effets magiques trop puissants pour être \emph{Finite} par des première année.

Parfois, même quand on regardait directement quelque chose, il fallait attendre d'avoir posé exactement la bonne question pour savoir ce qu'on avait sous les yeux…

\emph{Quelle potion puis-je préparer en utilisant uniquement des ingrédients trouvés dans une forêt ordinaire~?}

Toutes les recettes de \emph{Breuvages et Potions Magiques} utilisaient au moins un ingrédient venu d'une plante ou d'un animal magique. Ce qui était dommage car toutes les plantes et les animaux \emph{magiques} se trouvaient dans la Forêt Interdite, pas dans les bois plus petits et plus sûrs où se déroulaient les batailles.

Arrivé là, un autre aurait pu abandonner.

Harry avait tourné les pages des recettes l'une après l'autre, les survolant de plus en plus vite, se rendant progressivement compte de quelque chose, confirmant ce qu'il avait déjà lu mais qu'il \emph{voyait} pour la première fois.

Chaque potion incluait au moins un ingrédient magique, \emph{mais pourquoi aurait-il dû en être ainsi~?}

Les sortilèges ne requéraient aucun composant matériel~; il suffisait de prononcer la formule et d'agiter sa baguette. Harry avait jusque là considéré la préparation de potions comme essentiellement similaire~: plutôt que de voir des syllabes prononcées déclencher un effet magique sans raison apparente, on réunissait un tas d'ingrédients dégoûtants, on remuait quatre fois dans le sens des aiguilles d'une montre, et \emph{cela} déclenchait un effet magique arbitraire.

Auquel cas, étant donné que la plupart des potions utilisaient des composants ordinaires comme des aiguilles de porcs-épics ou du ragoût de limace, on aurait pu s'attendre à voir quelques potions n'utilisant \emph{que} des ingrédients ordinaires.

Mais au lieu de cela, \emph{toutes} les recettes de \emph{Breuvages et Potions Magiques} nécessitaient au moins \emph{un} ingrédient issu d'une plante ou d'un animal magique -- comme de la soie d'Acromantule ou des pétales de piège de feu de vénus.

Parfois, même quand on regardait directement quelque chose, il fallait attendre d'avoir posé exactement la bonne question pour savoir ce qu'on avait sous les yeux…

\emph{Si préparer une potion, c'est comme de lancer un sortilège, pourquoi est-ce que je ne m'effondre pas de fatigue après avoir préparé un breuvage aussi puissant que le remède contre les furoncles~?}

Deux vendredis plus tôt, Harry et sa classe avaient préparé un \emph{remède contre les furoncles} en double cours de potions… alors que le plus trivial des sortilèges de soin à la baguette et à la formule étaient au moins des sorts de quatrième année. Et ils s'étaient ensuite sentis exactement comme d'habitude après un cours de potions, à savoir \emph{pas du tout} épuisés magiquement~; du moins pas qu'ils puissent le percevoir.

Harry avait brutalement refermé son exemplaire de \emph{Breuvages et Potions Magiques} et avait foncé vers la salle commune de Serdaigle. Il y avait trouvé un Serdaigle en septième année qui préparait ses devoirs de potions pour ses ASPICs et l'avait payé une Mornille pour emprunter \emph{Maxima Potente Potions} pendant cinq minutes parce qu'il n'avait pas eu envie de courir jusqu'à la bibliothèque pour avoir sa confirmation.

Après avoir survolé cinq recettes du livre de septième année, Harry avait lu la sixième recette, une \emph{potion de souffle de feu}, qui nécessitait des œufs de Serpencendre… et le livre prévenait que le feu qui résulterait de la préparation ne pourrait pas être plus chaud que le feu magique à l'origine du Serpencendre qui avait pondu les œufs.

Harry s'était écrié "\emph{Eurêka}~!" au milieu de la salle commune de Serdaigle et avait été sévèrement réprimandé par un préfet qui se trouvait dans les parages et qui avait cru que M. Potter essayait de lancer un sortilège. Personne chez les sorciers ne connaissait ni ne se souciait d'un ancien Moldu nommé Archimède, ni ne la prise de conscience soudaine chez le protoscientifique que l'eau déplacée dans une baignoire serait égale au volume de l'objet qui était entré dans celle-ci.

Les lois de conservations. Elles avait été l'élément clé de plus de découvertes moldues que Harry n'aurait pu en dénombrer facilement. Dans le domaine de la technologie moldue, on ne pouvait pas soulever une plume un mètre au-dessus du sol sans que l'énergie ne vienne de \emph{quelque part}. Si, face à de la lave en fusion jaillissant d'un volcan, vous demandiez d'où venait la chaleur, un physicien vous parlerait des métaux lourds radioactifs au centre du cœur en fusion de la terre. Si vous lui demandiez d'où venait l'énergie qui alimentait cette radioactivité, le physicien vous parlerait d'une époque avant la formation de la terre et d'une supernova primordiale aux débuts de la galaxie qui avait mijoté des noyaux atomique plus lourds que la limite naturelle en compressant des protons et des neutrons jusqu'à former un paquet serré et instable qui libérait une partie de l'énergie de la supernova à chaque fois qu'il se scindait. Une ampoule était alimentée par l'électricité, qui était alimentée par une centrale nucléaire, qui était alimentée par une supernova… on pouvait remonter jusqu'au Big Bang en jouant à ce jeu.

La magie ne semblait \emph{pas} fonctionner ainsi, et c'était un euphémisme. L'attitude de la magie face à des lois telles que la conservation de l'énergie était quelque part entre un majeur géant dressé et un haussement d'épaule d'indifférence absolue. \emph{Aguamenti} créait de l'eau à partir de rien pour autant qu'on le savait~; il n'était fait mention d'aucun lac dont le niveau descendait à chaque fois que le sortilège était lancé. C'était un simple sortilège de cinquième année et qu'aucun sorcier ne trouvait impressionnait, parce que la création d'un simple verre d'eau ne leur semblait pas digne d'émerveillement. Ils n'avaient pas l'idée folle que la masse devait être conservée ou que créer un gramme de masse correspondait en définitive à la création de 90~000~000~000~000 joules d'énergie. Il existait un sortilège de section supérieure sur lequel Harry était tombé dont la \emph{formule} était exactement \emph{Arresto Momentum~!}, et lorsque Harry avait demandé si le mouvement allait \emph{ailleurs}, il n'avait eu droit qu'à des regards perplexes. Il était donc resté aux aguets pour le \emph{moindre} signe d'un principe de conservation en magie, \emph{n'importe quel signe}…

… et pendant tout ce temps, la réponse avait été devant lui, en cours de potions. La fabrication de potions ne \emph{créait} pas de magie, elle \emph{préservait} la magie, et c'était pour cela que chaque potion avait besoin d'au moins un ingrédient magique. Et en suivant des instructions comme 'remuez quatre fois dans le sens inverse des aiguilles d'une montre et une fois dans le sens des aiguilles d'une montre', on jetait un genre de sortilège qui -- Harry en avait fait l'hypothèse -- remodelait la magie des ingrédients (et déliait leur forme physique afin que des ingrédients tels que les aiguilles de porc-épic se dissolvent en un liquide buvable. Harry soupçonnait fort qu'un Moldu n'obtiendrait rien d'autre qu'un fatras épineux en suivant exactement la même recette). Voilà ce qu'était \emph{vraiment} la préparation de potions~: l'art de transformer des essences magiques préexistantes. C'était pour ça qu'on était \emph{un peu} fatigué après le cours de potions, mais pas trop~: parce qu'on ne donnait pas soi-même de pouvoir aux potions, on ne faisait que remodeler la magie déjà présente. Et c'était pour cela qu'une sorcière de deuxième année pouvait préparer du Polynectar, ou presque.

Harry avait continué de survoler \emph{Maxima Potente Potions} à la recherche de quelque chose qui falsifierait sa belle théorie. Au bout de cinq minutes il avait jeté une autre Mornille au garçon plus âgé pour faire taire ses protestations et avait continué.

La potion de force gigantesque requérait qu'un Re'em ait piétiné les Fangieux incorporés à la potion. Harry se rendit compte au bout d'un moment que c'était étrange car les Fangieux écrasés n'étaient pas très forts eux-mêmes, ils étaient seulement… très, très écrasés après que le Re'em en eut fini avec eux.

Une autre recette disait de toucher la potion avec 'du bronze forgé', c'est-à-dire de saisir une Noise entre des pinces afin d'effleurer la surface de la potion, et le livre mettait en garde~: si l'on faisait tomber tout la Noise dans la potion, celle-ci surchaufferait instantanément et déborderait du chaudron.

Harry avait regardé les recettes et leurs mises en gardes fixement et avait alors formulé une seconde hypothèse, plus étrange. Bien sûr que ce n'était pas aussi simple que de préparer des potions en utilisant le potentiel magique imprégné dans chaque ingrédient comme une voiture moldue était propulsée par le potentiel de combustion de l'essence. Jamais la magie ne serait aussi \emph{sensée}…

Harry était donc allé voir le professeur Flitwick -- puisqu'il ne souhaitait pas s'approcher du professeur Rogue en dehors des cours -- lui avait dit qu'il souhaitait inventer une nouvelle potion, qu'il savait de quels ingrédients il avait besoin et de ce que la potion devait faire mais qu'il ignorait comment en déduire la façon de remuer le breuvage -

Après que le professeur Flitwick eut cessé de hurler d'horreur et de courir en petit cercles et que le professeur McGonagall ait été appelée pour intervenir dans l'interrogatoire qui s'était ensuivi afin de promettre à Harry que, dans ce cas, il était à la fois acceptable et important qu'il révèle la théorie à l'origine de son projet, il s'était avéré que Harry n'avait \emph{pas} fait une découverte magique fondamentale mais redécouvert une loi si ancienne qu'on ne savait pas qui l'avait formulée le premier~:

\emph{Une potion fait usage de ce qui a été investi dans la création de ses ingrédients.}

La chaleur des forges gobelines qui avaient fondu la Noise de bronze, la force du Re'em qui avait piétiné les Fangieux, le feu magique d'où le Serpencendre était né~: toutes ces forces pouvaient être rappelées, libérées et restructurées par l'acte, similaire à l'invocation d'un sortilège, de remuer les ingrédients selon une procédure précise.

(D'un point de vu moldu c'était juste \emph{bizarre}, comme une version malade de la thermodynamique inventée par quelqu'un qui aurait trouvé que la vie devait être \emph{juste}. D'un point de vu Moldu, la chaleur utilisée pour forger la Noise n'était pas allée \emph{dans} le bronze car la chaleur était \emph{partie}~: elle s'était dissipée dans l'environnement et il était dorénavant plus difficile d'en faire usage. L'énergie était conservée, elle ne pouvait être ni créé ni détruite, c'était \emph{l'entropie} qui augmentait toujours. Mais les sorciers ne pensaient pas ainsi~: selon eux, si vous faisiez un certain effort pour fabriquer une Noise, il était raisonnable de penser que ce même effort pouvait être extrait de la Noise plus tard. Harry avait essayé d'expliquer pourquoi c'était un peu \emph{bizarre} quand on avait été élevé par des Moldus et le professeur McGonagall avait demandé avec perplexité pourquoi le point de vu Moldu aurait mieux valu que le point de vue sorcier).

Le principe fondamental de la fabrication de potions n'avait ni nom ni formulation classique, car on aurait alors pu être tenté de l'écrire.

Et une personne n'ayant pas la sagesse nécessaire pour découvrir le principe elle-même aurait pu le lire.

Et cette personne aurait alors eu toutes sortes d'idées de nouvelles potions géniales.

Et cette personne se serait alors transformée en fille-chat.

On avait clairement fait comprendre à Harry qu'il ne partagerait \emph{pas} cette découverte avec Neville, ni avec Hermione après leur prochaine bataille. Harry avait essayé de dire que Hermione avait vraiment eu l'air absente ces derniers temps et que c'était exactement le genre de chose qui aurait pu remonter le moral de cette dernière. Le professeur McGonagall avait répondu d'un ton catégorique qu'il ne devrait même pas y songer et le professeur Flitwick avait levé ses petites mains et fait le geste d'une baguette qu'on brise en deux.

Les deux professeurs avaient cependant été assez bons pour suggérer que, si M. Potter pensait connaître les ingrédients nécessaires, il pourrait peut-être trouver une recette \emph{déjà existante} et pourvue du même effet. Le professeur Flitwick avait même mentionné plusieurs volumes disponibles à la bibliothèque de Poudlard qui pourraient s'avérer utiles…

\later

Le vaste écran-parchemin n'offrait maintenant qu'une vue aérienne de la forêt depuis laquelle on pouvait à peine voir les silhouettes camouflées des soldats des trois armées, chacune divisée en deux groupes qui convergeaient vers la triple bataille.

Les bancs du stade de Quidditch se remplissaient maintenant rapidement, peuplés par le genre de spectateurs facilement ennuyés qui ne souhaitait être là que pour la bataille finale et sauter les parties ennuyeuses (il était couramment convenu que si les batailles du professeur Quirrell avaient un défaut, c'était que ses spectacles étaient loin de durer aussi \emph{longtemps} que le matchs de Quidditch une fois qu'ils avaient vraiment commencé. Ce à quoi le professeur Quirrell avait répondu \emph{ainsi va la vraie vie} et le débat avait été clos).

Au centre de l'immense fenêtre -- ce n'était maintenant plus qu'une seule fenêtre très haut dans le ciel -- la vague collection de petites silhouettes camouflées se rapprochait.

Se rapprochait.

Se touchait presque…

\later

L'immense fenêtre de parchemin révélait une première esquisse de bataille entre Soleil et Chaos, une masse hurlante d'enfants, smileys sur la poitrine, qui chargeaient protégés par des \emph{Contego} tandis que d'autres hurlaient \emph{Somnium~!}…

Jusqu'à ce que l'un d'eux s'écrie \emph{Prismatis~!} d'une voix terrifiée et que la charge ne s'arrête soudain face au mur de force scintillant qui venait d'apparaître devant eux.

Tracey Davis était sortie du bois.

"C'est ça," dit-elle d'une voix basse et sinistre, baguette pointée vers la barrière. "Vous \emph{devriez} me craindre. Car je suis Tracey Davis, la Dame des Ténèèèbres~! C'est Dame des Ténèèèbres T-É-N-È-È-È-B-R-E avec trois È~!"

(Amelia Bones, directrice du département de justice magique, jeta un regard inquisiteur vers M. et Mme Davis qui semblaient tous deux avoir particulièrement envie de mourir sur le champ).

Derrière la barrière prismatique, il y eut une sorte de débat chuchoté entre les soldats Soleil, dont un en particulier qui semblait se faire réprimander par certains autres.

Puis, un moment plus tard, \emph{Tracey} recula d'un pas.

Susan Bones s'était placée à l'avant du contingent Soleil.

("Bonté Merline," dit Augusta Londubat. "Et que pensez-vous que votre nièce ait appris à Poudlard~?")

("Je l'ignore", répondit Amelia Bones avec calme, "mais il faudra que je lui envoie une grenouille en chocolat par chouette avec ordre d'en apprendre plus").

La barrière prismatique se volatilisa.

Les soldats Soleil continuèrent leur charge.

Tracey hurla, sa voix tendue à l'extrême~: "\emph{Inflammare~!"} et la charge Soleil s'arrêta de nouveau alors qu'une ligne de feu s'élevait brusquement devant eux dans l'herbe mi-sèche le long de la direction vers laquelle Tracey avait pointé sa baguette, et un instant plus tard Susan Bones s'écria \emph{"Finite Incantatem}~!" et les flammes faiblirent, rejaillirent, faiblirent au gré de leurs volontés en opposition tandis que d'autres soldats levaient leurs baguettes pour viser Tracey, et c'est \emph{là} que Neville Londubat fondit du ciel en hurlant.

\later

L'un des soldats Dragons, Raymond Arnold, fit un signe de la main vers l'avant puis la diagonale gauche et un chuintement réprimé parcourut soudain le contingent de l'armée Dragon alors qu'ils se réorientaient tous discrètement vers l'ennemi. Les Soleil savaient qu'ils étaient là, bien sûr que les deux armées le savaient, mais pourtant, à cet instant, ils étaient tous devenus instinctivement silencieux.

Les Dragons glissèrent plus avant, puis encore plus, les silhouettes camouflées et ternes des Soleil commençaient à apparaître entre les arbres lointains et pourtant personne ne parlait, personne ne mugissait de charger.

Drago était maintenant en tête de ses soldats, Vincent derrière lui et Padma à peine plus loin en arrière~; si eux trois pouvaient soutenir l'impact des meilleurs de Soleil, le reste de l'armée Dragon avait peut-être une chance.

Puis Drago vit une Soleil le regarder au loin, depuis l'avant-garde de sa propre armée, le regarder avec un air de furie -

Leurs regards se croisèrent au-dessus du champ de bataille.

Drago n'eut qu'une fraction de seconde pour s'interroger à demi consciemment sur ce qui pouvait avoir mis Hermione autant en colère avant que le cri ne s'élève dans leurs armées respectives~; et ils chargèrent alors tous, Hermione et lui placés sur une trajectoire qui les mènerait directement à une collision mutuelle.

\later

Les autres chaotiques apparaissaient maintenant entre les arbres, d'autres avaient \emph{sauté d'arbres} et la bataille battait son plein, chacun tirant en tous sens sur tout ce qui ressemblait à un ennemi. Plus un certain nombre de Soleil qui criaient "\emph{Luminos~!"} et Neville Londubat, le Poufsouffle chaotique, qui zigzaguait et fusait à travers les airs le long de trajectoires qui n'auraient pu en effet être qualifiées autrement que 'chaotiques'…

Et une chose se produisit, une chose qui n'arrivait qu'une fois sur vingt pendant les répétitions de combat aérien~: le \emph{balai} de Neville luit d'un rouge vif entre ses mains serrées.

Cela aurait dû signifier que Londubat était hors-jeu.

Mais alors, dans les gradins de Poudlard, au milieu de la foule d'élèves qui assistaient au spectacle, un cri s'éleva…

\emph{Le combat doit être réaliste.} C'était la règle maîtresse du professeur Quirrell. On pouvait tout se permettre tant que c'était \emph{réaliste}, et dans la vraie vie, un soldat ne se volatilisait pas quand son \emph{balai} était victime d'un sortilège.

Neville tomba vers le sol en hurlant~: "\emph{Atterrissage chaotique~!",} les chaotiques arrachèrent leur attention au combat pour lancer un sortilège de lévitation (tout en courant afin de ne pas devenir des cibles faciles), presque tout le monde s'arrêta, bouche bée -

Et Neville Londubat se fracassa contre le sol jonché de feuilles de la forêt, atterrit sur un genoux, un pied et les deux mains comme s'il s'agenouillait pour être adoubé.

Tout se figea. Même Tracey et Susan interrompirent leur duel.

Dans le stade, tous les bruits de la foule s'évanouirent.

Il y eut un silence général fait de stupéfaction, d'inquiétude et d'un ébahissement profond tandis que tout le monde attendait de voir ce qui allait se passer.

Et Neville Londubat se leva lentement et pointa sa baguette qu'il n'avait toujours pas lâchée vers les soldats Soleil.

Même si personne sur le champ de bataille ne l'entendit, une grande partie du public avait commencé à chanter avec une intensité croissante à chaque nouvelle répétition du mot~: "DOOM DOOM DOOM DOOM DOOM", parce qu'il était tout simplement impossible d'assister à ça et de ne \emph{pas} se dire qu'un accompagnement musical était de rigueur.

"La foule acclame votre petit-fils," dit Amelia Bones. La vieille sorcière scrutait l'écran d'un regard calculateur.

"En effet," dit Augusta Londubat. "Certains d'entre eux, si j'entends bien, entonnent \emph{Notre sang pour Neville~! Nos âmes pour Neville~!}

--- En effet," dit Amelia en prenant une gorgée d'une tasse qui, un instant plus tôt, n'avait pas été là. "On dirait que le gamin est taillé pour être un meneur.

--- Ces acclamations," continua Augusta d'une voix qui s'alourdissait encore de stupéfaction, "semblent venir des bancs Poufsouffle.

--- Après tout, c'est la maison de la loyauté, ma chère" dit Amelia.

"Albus Percival Wulfric Brian Dumbledore~! \emph{Nom de Merlin, qu'est-ce qui se passe dans cette école~!}"

Lucius Malfoy regardait les écrans avec un sourire ironique tandis que ses doigts frappaient son dossier de chaise selon un rythme indiscernable. "Je ne sais pas ce qui est le plus effrayant, l'idée qu'il ait un plan secret derrière tout cela ou l'idée qu'il n'en ait aucun.

--- Regardez~!" s'écria Lord Greengrass. Le jeune homme soigné s'était à moitié levé de sa chaise et pointait l'écran du doigt. "Là voilà~!"

\later

"On va l'attaquer toutes les deux en même temps," murmura Daphné. Elle savait que quelques minutes d'effroi en situation de combat réel un poignée de fois par semaine ne suffirait peut-être pas à se mesurer à la pratique régulière du duel que Neville avait eu avec Harry et Cédric Diggory pendant le même laps de temps. "Il est trop fort contre chacune de nous, mais ensemble -- j'utiliserai mon charme, toi, essaie juste de l'étourdir…"

À côté d'elle, Hannah hocha la tête, puis elles hurlèrent à pleins poumons et chargèrent, rendues plus rapides et légères par les sortilèges de lévitation de deux soldats Soleil. Daphné criait déjà "\emph{Tonare~!"} alors que Hannah maintenait un immense \emph{Contego} mouvant devant elles, et grâce à une brève poussée supplémentaire elles bondirent par-dessus la tête de la première ligne de soldats et atterrirent face à Neville, leurs cheveux tourbillonnant au-dessus de leurs têtes…

(Les photographes étaient strictement interdites à tous les jeux organisés par Poudlard mais pourtant cet instant finit \emph{quand même} sur la couverture du \emph{Chicaneur}).

… et au même instant, parce que le combat contre les brutes avait fait partir en fumée la moindre once d'hésitation, Hannah tira son premier sortilège de sommeil contre Neville (elle avait commencé l'incantation avant d'avoir touché le sol) au moment où Daphné, privilégiant la vitesse à la force, fendait l'air de sa Lame Très Ancienne là où elle pensait que les cuisses de Neville se retrouveraient \emph{après} qu'il ait esquivé -

Mais Neville sauta \emph{en l'air}, non pas sur le côté mais plus haut qu'il n'aurait dû être capable de le faire, si bien que l'épée lumineuse ne trancha que l'air sous ses pieds. Daphné parvint à comprendre à temps que cela signifiait que des chaotiques faisaient encore léviter Neville et put lever sa Lame au-dessus de sa tête, mais Neville \emph{tomba trop vite} et lorsque la Lame de celui-ci se fracassa contre la sienne, ce fut comme d'avoir reçu un Cognard. Le coup fit valdinguer Daphné, l'envoya s'effondrer en arrière sur l'herbe et son dos frappa durement le sol. Tout aurait été alors fini pour elle si Neville n'avait pas lui aussi atterrit trop brusquement et ne s'était pas retrouvé à genoux avec un hoquet de douleur. Alors, avant que celui-ci ne puisse abattre sa Lame lumineuse, Hannah cria "\emph{Somnium~!"} et Neville bondit désespérément en arrière -- même si bien sûr aucun sortilège n'était vraiment sorti de la baguette de Hannah car il était impossible pour la Poufsouffle de tirer une deuxième fois si vite -- ce qui donna à Daphné une seconde pour se relever tant bien que mal, remettre ses deux mains sur sa baguette…

\later

"Par Merlin," dit Dame Greengrass. Sa voix semblait instable et son port aristocratique était tout à fait démoli. "Ma fille se \emph{bat} avec le charme de la Très Ancienne Lame alors qu'elle est en première année. Je n'avais jamais su… qu'elle possédait un talent aussi extraordinaire…

--- Excellent sang," dit Charles Nott d'un ton approbateur, ce qui fit grogner Augusta.

"Ma bonne Dame," dit le professeur Quirrell d'un ton grave. "Ne faites pas ainsi tort à votre fille. Ce n'est pas que du \emph{talent} que vous voyez là." Sa voix devint un peu plus sèche. "\emph{Ça}, c'est ce qui se passe quand de jeunes sorciers et sorcières investissent leurs efforts dans une compétition qui, au contraire du Quidditch ou de la bataille explosive, met en jeu -- pour le dire franchement -- une véritable pratique de la magie."

\later

"\emph{Expelliarmus~!"} s'écria Drago en essayant d'empêcher sa voix de se briser et en esquivant en même temps le jet rouge que Hermione Granger avait tiré vers lui. Ses muscles se tordirent pour bondir dans la mauvaise direction~: elle avait pointé vers la gauche puis d'une mystérieuse convulsion avait tiré vers la droite…

Hermione évita le rapide sortilège de duelliste et s'écria sans s'être le moins du monde interrompue~: "\emph{Steleus~!"}, un maléfice à effet de zone que Drago ne pouvait pas éviter, mais il réussit à diriger sa baguette vers son propre visage et à s'écrier "\emph{Quiescus~!"} avant que le besoin soudain d'inspirer de l'air ne se transforme en une quinte de tout qui aurait mis fin à la bataille.

Drago Malfoy était déjà à moitié épuisé suite à tous les sortilèges d'emprisonnement et à toutes les métamorphoses précédentes, mais son état de confusion commençait à laisser place à un coup de sang bouillonnant. Il ne savait pas \emph{pourquoi} Granger l'attaquait soudain avec autant de colère, mais \emph{si elle cherchait la bagarre, elle l'aurait…}

(Les Dragons et les Soleils ne s'arrêtaient pas pour regarder le duel de leurs généraux. Les Dragons étaient trop disciplinés pour s'arrêter et regarder, si bien que les Soleil devaient eux aussi continuer de se battre, mais le public bouche bée des gradins de Quidditch de Poudlard étaient même distraits du spectacle offert par Neville et Daphné, réorientaient leur regard vers le duel des deux généraux, alors que Malfoy et Granger se tiraient sort après sort, maléfice après maléfice, tiraient plus vite qu'aucun autre élève de leur année n'en aurait été capable, la danse de duel étudiée du général Dragon égalée par l'énergie frénétique du général Soleil, le combat ressemblant de plus en plus à un duel d'adulte à mesure que les deux première année aux magies les plus fortes recouraient à des sortilèges plus exotiques que le sortilège de sommeil habituel).

… même si Drago commençait à se rendre compte que lorsque lui, Harry et le professeur Quirrell avaient dédaigné Mlle Granger, lui attribuant autant d'intention de tuer que n'en aurait un bol de raisins mûrs, \emph{ils ne l'avaient pas encore vue en colère.}

\later

Daphné frappa de son Ancienne Lame, toujours sans essayer de frapper fort mais en la déplaçant aussi vite que possible tandis qu'au même instant Hannah criait "\emph{Somnium~!"} et que Neville bondissait de nouveau vers l'arrière, mais ça avait été un autre bluff, et Hannah s'avança pour tirer un véritable sort, presque à bout portant…

… et Neville Londubat fit exactement, il l'expliquerait ensuite, ce que Cédric Diggory l'avait entraîné à faire si jamais il se battait contre Bellatrix Black, à savoir de tourner sur lui-même et de lui donner un \emph{énorme coup de pied} dans l'estomac.

La Poufsouffle fit un petit bruit triste, eut un hoquet de douleur et fut soulevée de terre par la chaussure qui s'enfonça dans son abdomen propulsée par la force de tout le corps de Neville.

L'espace d'un instant le champ de bataille fut immobile, tout s'arrêta sauf la silhouette de Hannah qui tombait.

Puis le visage de Neville révéla un désarroi complet, il abaissa sa baguette, et le lieutenant chaotique s'élança instinctivement vers sa camarade, tendit son autre main vers elle…

Au même instant, Hannah transforma sa chute en roulade, se releva baguette dressée et l'abattit.

Une fraction de seconde plus tard, Daphné, qui n'avait pas non plus hésité, fit sombrer sa Très Ancienne Lame droit dans le dos de Neville et les muscles du lieutenant chaotique s'agitèrent de convulsions causées par l'action conjointe de la magie étourdissante qui se déchargeait dans son corps et du sortilège de sommeil de Hannah qui prenait effet. Le dernier descendant des Londubat se retrouva alors étendu au sol, un air de surprise total cristallisé sur le visage.

\later

"Aujourd'hui, M. Londubat a appris une précieuse leçon quant à son sentiment de pitié et de remord," dit le professeur Quirrell.

"Et de chevalerie," dit Amelia, sirotant de nouveau son thé.

\later

"Tu vas bien~?" chuchota Daphné, surplombant Hannah afin de la protéger tandis que cette dernière, étendue au sol, se tenait le ventre. Cette dernière ne répondit rien hormis d'autres sons de gorges qui laissaient penser que Hannah essayait de ne pas vomir tout en s'efforçant de ne pas pleurer.

Sans que l'on sache comment, et même si ce n'était peut-être pas un bon choix tactique -- il aurait mieux valu que Hannah soit directement touchée par un sortilège plutôt que d'autres soldats soient retenus à essayer de la \emph{protéger} -- un certain nombre de soleils semblèrent se retrouver debout devant Hannah, leur baguette fermement tenue, un regard de colère braqué sur les chaotiques. Quelqu'un avait levé une barrière prismatique entre les deux groupes mais Daphné ne pouvait pas voir qui.

Et, sans que l'on sache pourquoi, les chaotiques ne semblaient pas continuer leur attaque. Même Tracey avait entièrement laissé tomber son air sinistre et elle se balançait nerveusement d'un pied sur l'autre, comme si elle avait du mal à se souvenir dans quel camp elle était…

"\emph{Arrêtez~!"} hurla une voix. "\emph{Arrêtez la bataille~!"}

Il n'y avait de toute façon pas de bataille en cours à proprement parler, mais on s'arrêta.

Le général Potter, plus Survivant que jamais, émergea des bois d'un pas vif avec quelque chose de grand et de recouvert par du tissu de camouflage maintenu sous le bras.

"Mlle Abbott respire-t-elle bien~?" cria le général Potter.

Daphné ne regarda pas en arrière. Elle ne rejetait pas la possibilité qu'il s'agisse d'un piège -- il était absolument certain que si les chaotiques se saisissaient de cette opportunité et attaquaient, le professeur Quirrell ne déclarerait pas seulement ce coup légal, il leur donnerait ensuite des points supplémentaires. Mais Daphné pouvait fort bien entendre la réponse~; ce n'était pas comme si Hannah essayait de respirer \emph{discrètement}, et elle répondit donc~: "Plus ou moins.

--- Elle devrait sortir de là et aller voir quelqu'un qui peut utiliser des sortilèges de soin," dit Harry. "Juste au cas où le coup aurait cassé quelque chose."

De derrière Daphné, une petite voix haletante dit~: "Je… peux… encore… me battre…

--- Mlle Abbott, ne…" dit Harry au moment même ou le son de quelqu'un qui s'effondrait après avoir essayé de se relever survint derrière Daphné. Tout le monde grimaça mais Daphné ne fit pas dos à Harry.

"Pourquoi les professeur n'ont-ils pas interrompu la bataille~?" demanda Susan avec colère.

"Je suppose que c'est parce que Mlle Abbott ne risque pas de dommages permanents et que le professeur Quirrell pense que nous sommes en train d'apprendre de précieuses leçons," dit Harry d'une voix dure. "Écoutez, Mlle Abbott, si vous partez, Tracey se retirera de la bataille. Vous êtes déjà en surnombre, alors cette offre vous avantage beaucoup. Acceptez-la, s'il vous plaît.

--- Hannah, \emph{pars}~!" dit Daphné. "Enfin, dis juste que tu es hors jeu~!"

Lorsque Daphné jeta un regard en arrière elle vit que Hannah secouait la tête, toujours roulée en boule sur l'herbe.

"Bon, ça suffit," dit Harry. "\emph{Chaotiques~! Plus vite on les étourdi plus vite elle sortira d'ici~! On va faire ça très vite même si on doit subir des pertes~! Fin de la trêve~! \shout{Poissonthon}~!"}

La partie politique du cerveau de Daphné n'eut qu'un instant pour admirer comment les quelques mots de Harry venaient de mettre les chaotiques dans le camp des \emph{gentils}, puis, dans une union presque parfaite, les chaotiques plongèrent leur main dans une poche de leur uniforme et en sortirent des lunettes de soleil vertes à l'apparence peu familière. Pas comme celles qu'on porterait à la plage, plutôt comme des lunettes pour cours de potions avancé…

Puis elle comprit ce qui allait se produire et leva son autre main pour protéger ses yeux au moment où Harry arrachait le tissu placé au-dessus du chaudron.

La fluide qui s'écoula lorsque Harry jeta le contenu du chaudron dans l'air était trop lumineux pour être vu, trop brillant pour être imaginé, incandescent comme un soleil par dix fois agrandi…

(et c'était exactement ce dont il s'agissait)

(la lumière du soleil qui avait été investie dans la création des glands, la lumineuse énergie qui avait alimenté l'élévation d'un arbre au-dessus d'une terre nue)

(sans quasiment aucune des longueurs d'onde vertes que la chlorophylle réfléchissait pour créer la couleur verte des feuilles)

(ce qui se trouvait être la couleur des lunettes de soleil de Chaos, faites pour laisser passer les longueurs d'onde vertes et bloquer les rouges et les bleues, réduisant l'éclat violet le plus incandescent à quelque chose de supportable)

… la lumière violette flambait continuellement. Daphné essayait d'ôter son bras de ses yeux mais découvrit qu'elle ne pouvait \emph{rien} regarder car même l'éclat violet réfléchi était tellement intense qu'elle devait plisser les yeux et elle n'eut le temps de crier \emph{Finite Incantatem} qu'une fois, ce qui n'eut aucun effet, avant qu'un sortilège de sommeil ne l'emporte.

Ce qui restait de la bataille n'en eut plus pour longtemps.

\later

"\shout{Maintenant}~!" mugit Blaise Zabini, anciennement du Soleil, à présent commandeur d'un détachement de légionnaires du Chaos. "Je veux dire, \shout{Poissonthon}~!" La main du Serpentard se saisit du tissu qui protégeait le chaudron de la caresse du jour, qui était l'élément déclencheur, et commença déjà à l'écarter.

"\shout{Maintenant}~!" mugit Dean Thomas, anciennement du Chaos, commandeur d'un lot de guerriers dragons. "\shout{Quoi qu'ils fassent, faites pareil}~!"

Les chaotiques du détachement de Zabini plongèrent leur main dans une poche de leur uniforme et s'avancèrent, des lunettes de soleil vertes en main…

… un acte quasiment parfaitement reflété par Dean et les guerriers dragons qui sortirent des lunettes de potions vertes et en passèrent rapidement les sangles par-dessus leur tête alors que les chaotiques mettaient leurs lunettes de soleil et que l'incandescence violette explosait.

(Comme le général Malfoy l'avait expliqué, si M. Goyle rapportait que la légion du Chaos portait des lunettes de potions vertes, il n'y avait pas besoin de savoir \emph{pourquoi} pour pouvoir en métamorphoser quelques exemplaires).

"\scream{C'est de la triche}~!" s'écria Blaise Zabini.

"\scream{C'est de la technique}~!" hurla Dean en retour. "\scream{Dragons, chargez}~!"

("Excusez-moi," dit Dame Greengrass. "Pourriez-vous arrêter de \emph{rire} comme ça, M. Quirrell~? C'est assez déconcertant.")

"\shout{Lancez Finite sur leurs lunettes}~!" hurla Blaise Zabini alors que les deux armées se ruaient droit l'une sur l'autre au sein de l'éclat violet omniprésent et aveuglant. "\shout{On peut encore gagner}~!

--- \shout{Vous l'avez entendu}~!" mugit Dean. "\shout{À leurs lunettes}~!"

La réponse de Blaise Zabini fut loin d'être construite.

Cette bataille continua bien plus longtemps.

\later

"\emph{Stupéfix~!"} s'écria le général Soleil.

Drago n'évita ni ne contra car il n'avait plus assez d'énergie pour faire l'un ou l'autre~; il ne put que vite mettre sa main en position et espérer…

Le tir rouge se dissipa de nouveau sur le gant \emph{Collaporté} de Drago qu'il avait métamorphosé et magiquement attaché à sa main, comme il l'avait fait pour le reste de l'armée Dragon. Ce bouclier était maintenant la seule chose qui le sauvait.

Ça aurait dû être le moment de contre-attaquer mais Drago ne put rien faire d'autre que reprendre son souffle alors qu'ils dansaient tous deux d'avant en arrière au gré des mouvements infinis de leur duel à l'ombre des feuillages. Face à lui, le général Granger avait le souffle très court. Le visage de la jeune fille luisait tant de sueur qu'il semblait couvert de rosée et ses cheveux châtains formaient des tresses humides plus sombres. Son uniforme de camouflage était taché de cercles moites, ses épaules tremblaient visiblement sous le coup de l'épuisement mais sa baguette était d'une stabilité de pierre, toujours pointée vers Drago, quels que soient les mouvements de ce dernier. Ses yeux étincelaient, ses joues étaient empourprées de rage.

\emph{Alors petite, pourquoi est-ce que tu fais semblant de te battre comme une grande aujourd'hui~?}

La raillerie lui vint à l'esprit mais il ne pensait pas avoir besoin d'une Granger encore plus en colère~; alors Drago dit juste -- bien qu'il puisse entendre sa propre voix se briser - "Tu as une raison particulière de m'en vouloir, Granger~?"

La fille haletait à la recherche d'oxygène, sa voix chancelait~: "Je sais ce que tu prépares," dit Hermione Granger, montant d'une octave. "Je sais ce que Rogue et toi préparez, Malfoy, et je sais qui est derrière ça~!

--- Hein~?" dit Drago sans même réfléchir.

Cela ne sembla qu'augmenter la furie de Granger, ses doigts se blanchirent sur la baguette qu'elle tenait braquée sur lui.

Et Drago comprit alors, et son sang en bouillit. Même \emph{elle} pensait qu'il manigançait secrètement quelque chose contre elle…

"\shout{Toi aussi~?"} cria Drago. "\shout{Je t'ai aidée, petite morveuse aux dents de lapin~! Tu, tu, tu,"} - il bégaya entre tous les sortilèges noirs qui lui vinrent à l'esprit jusqu'à trouver quelque chose qu'il puisse vraiment lui lancer - "\shout{Densaugeo~!"}

Mais Granger bondit et tourbillonna autour du sortilège d'allongement dentaire puis sa propre baguette acheva de tourner et se retrouva pointée presque à bout portant sur Drago au moment où celui-ci levait sa main gauche en guise de bouclier, plaçant ainsi le gant magiquement fermé entre lui et ce qu'elle était sur le point de tirer, quand le général Soleil poussa un cri audible à travers tout le champ de bataille…

"\scream{Alohomora~!"}

Le temps aurait dû s'arrêter.

Il ne s'arrêta pas.

Mais le gant cadenassé sur la main de Drago luit d'un gris bref puis le cadenas émit un clic et tomba.

Juste comme ça.

Juste comme ça.

Les écrans le montrèrent tous très clairement à tout le stade de Poudlard.

Et le silence de mort qui s'abattit alors sur chaque banc de chaque gradin révéla que tout le monde comprenait très clairement ce que cela voulait dire~: que le descendant de la maison Malfoy venait de voir sa magie surmontée par celle d'une née-Moldue.

Hermione Granger ne s'arrêta pas de combattre, elle n'émit aucun signe indiquant qu'elle savait même ce qu'elle venait de faire~; mais son pied jaillit et donna un coup à la manière moldue qui fit sauter la baguette de Drago hors de sa main. Le corps et l'esprit encore choqués de Drago avaient réagit un peu trop lentement. Drago plongea sur sa baguette, fouilla le sol avec frénésie, mais de derrière lui la voix brisée d'une fille dit "\emph{Somnium~!"}, et Drago Malfoy tomba pour ne pas se relever.

Il y eut un autre moment de silence figé. Le général Soleil vacillait et donnait l'impression de risquer de s'évanouir.

Puis les guerriers dragons hurlèrent à pleins poumons et chargèrent pour venger leur commandant tombé.

\later

M. et Mme Davis tremblaient en se levant des confortables chaises de la loge professorale du stade de Quidditch~; ils ne pouvaient pas vraiment s'agripper l'un à l'autre en marchant mais ils se tenaient la main fermement et prétendait être invisibles du plus fort qu'ils le pouvaient. S'ils avaient été des enfants assez jeunes pour générer de la magie accidentelle ils se seraient probablement désillusionnés eux-mêmes.

Le vieux Charles Nott ne dit rien en se levant de sa chaise. Le balafré Lord Jugson ne dit rien en se levant de sa chaise.

Lucius Malfoy ne dit rien en se levant.

Ils se détournèrent tous les trois sans s'arrêter et s'avancèrent rapidement vers l'escalier qui menait aux gradins surélevés, se déplaçant d'un concert inquiétant, comme un trio d'Aurors…

"Lord Malfoy," dit le professeur de Défense avec douceur. L'homme était toujours assis sur sa chaise et regardait ses écrans-parchemins, ses bras flasques le long de ses flancs, comme s'il n'avait pas particulièrement envie de bouger.

L'homme aux cheveux blancs s'arrêta juste avant d'atteindre l'arcade qui constituait la sortie, l'homme âgé et l'homme balafré s'arrêtèrent aussi, l'encadrant. La tête de Lord Malfoy se tourna, trop légèrement pour que le geste soit pris comme le moindre signe de réponse, mais toutefois vers le professeur de Défense.

"Votre fils a accompli une performance exceptionnelle aujourd'hui," dit le professeur Quirrell. "Je dois avouer que je l'ai sous-estimé. Et comme vous l'avez vu, il a gagné la loyauté de son armée." Toujours très douce, la voix du professeur de Défense. "En tant que professeur de votre fils, je suis de l'avis que votre fils ne bénéficiera pas d'une interférence de votre part dans cette -"

Lord Malfoy et ses comparses disparurent le long des escaliers.

"Un belle tentative Quirinus," dit doucement Dumbledore. Le visage du vieux sorcier révélait de petites rides d'inquiétude. Lui non plus ne s'était pas levé de son siège et regardait les écrans-parchemins comme s'ils étaient encore allumés. "Pensez-vous qu'il écoutera~?"

Les épaules du professeur de Défense eurent un bref haussement saccadé, seul mouvement du professeur depuis la fin de la bataille.

"\emph{Eh bien}," dit Dame Greengrass en se levant, en se faisant craquer les doigts et en s'étirant, son mari silencieux à côté d'elle. "Je dois dire que c'était assez… intéressant…"

Amelia Bones s'était levée de son siège rembourré sans faire de manières. "Intéressant, en effet," dit-elle. "Je confesse être perturbée par l'habileté avec laquelle ces enfants se battaient.

--- L'habileté~?" dit Lord Greengrass. "Leurs sortilèges ne me semblaient pas si impressionnants que ça. Sauf celui de Daphné, bien sûr."

La vieille sorcière ne détourna pas ses yeux du crâne dégarni du professeur Quirrell. "Le sortilège d'étourdissement n'est pas enseigné en première année, Lord Greengrass, mais ce n'est pas cette sorte d'habileté que j'avais à l'esprit. Ils se soutenaient les uns les autres grâce à des sorts simples, ils réagissaient rapidement aux surprises…" La directrice du département de justice magique s'interrompit comme si elle cherchait des mots qu'un simple civil pourrait comprendre. "Au cœur de la bataille," dit-elle enfin, "alors que des sortilèges volaient en tous sens… ces enfants se sentaient comme chez eux.

--- Tout à fait, Mme la directrice," dit le professeur de Défense. "Certains arts gagnent à être pratiqués jeune."

Les yeux de la vieille sorcière se plissèrent. "Vous les préparez à devenir une force militaire, professeur. À quelle fin~?

--- Attendez~!" s'interposa Lord Greengrass. "Il y a plein d'écoles où l'on enseigne l'art du duel en première année~!

--- Le duel~?" dit le professeur de Défense. De derrière, on ne pouvait pas dire si le pâle visage souriait. "Cela n'est \emph{rien}, Lord Greengrass, comparé à ce que mes élèves ont appris. Ils ont appris à ne pas hésiter face aux embuscades et à des ennemis plus puissants qu'eux. Ils ont appris à s'adapter lorsque les conditions de combat changent encore et encore. Ils ont appris à protéger leurs alliés, à protéger ceux qui ont le plus de valeur, à abandonner les pièces qui ne peuvent être sauvées. Ils ont appris que pour survivre, il leur faut suivre des ordres. Certains ont même appris un peu de créativité. Oh non, Lord Greengrass, \emph{ces} sorciers ne se cacheront pas dans leur manoir en attendant qu'on les protège lorsque la prochaine menace viendra. Ils sauront qu'ils sont capables de se battre."

Par trois fois, Augusta Londubat applaudit bruyamment.

\later

\emph{Nous avons gagné.}

C'était la première chose que Drago avait entendue quand il s'était réveillé sur le champ de bataille~: Padma qui lui racontait comment ses soldats s'étaient rassemblés après qu'il fut tombé. Comment, grâce à la prévoyance du général Dragon, M. Thomas avait mené son détachement jusqu'à une victoire contre Chaos. Comment le général Potter avait vaincu la partie du régiment Soleil qui l'avait affronté. Comment les guerriers dragons de M. Thomas avaient rejoint le corps principal des soldats, munis de leurs propres lunettes et de celles des chaotiques défaits. Comment, à peine quelques instants plus tard, ce qui restait du contingent du général Potter avait attaqué les deux autres armées avec une potion qui émettait une fulgurante lumière pourpre. Mais Dragon avait maintenu son avantage numérique à la fois contre Soleil et Chaos tout en ayant assez de lunettes pour ses guerriers, et ainsi Padma était parvenue à mener l'armée dont elle avait héritée jusqu'à la victoire.

À en voir la lumière dans les yeux de cette dernière et le sourire arrogant qui aurait rendu un Malfoy fier, elle s'attendait à des félicitations. Drago parvint à siffler quelque chose qui ressemblait à une éloge entre ses dents serrées, mais il n'aurait su dire ce qu'elle avait été un instant plus tard. Il semblait que la sorcière, née à l'étranger, ignorait entièrement ce qui venait de se produire ou ce que cela signifiait.

\emph{J'ai perdu.}

Les Dragons traînaient les pieds jusqu'à Poudlard sous un ciel gris, de lourdes gouttes froides tombaient une à une sur la peau de Malfoy. Elle avait commencé pendant qu'il était étourdi, la pluie longtemps promise commençant enfin à tomber. Drago n'avait maintenant plus qu'une seule option. Un coup forcé, comme l'aurait appelé M. MacNair, qui avait enseigné les échecs à Drago. Harry Potter n'aimerait probablement pas ça s'il était vraiment amoureux de Hermione, comme tout le monde le disait. Mais le coup forcé, tel que M. MacNair l'avait définit, était un coup qu'on devait faire si l'on voulait que la partie puisse seulement continuer.

Il continua à se jouer en boucle dans l'esprit de Drago, encore et encore, alors que celui-ci franchissait en automate l'immense portail de Poudlard, renvoyait Vincent et Gregory de deux mots acerbes et s'isolait dans sa chambre privée, assis sur son lit, face au mur derrière son bureau. Il remplissait son esprit comme si un Détraqueur l'avait attaché à ce souvenir.

Le flash gris venu de son gant, le loquet qui émettait un cliquetis et tombait…

Drago savait, il \emph{savait} où il s'était trompé. Il avait été tellement fatigué après avoir lancé vingt-sept sortilèges d'Emprisonnement pour tous les autres guerriers Dragon. Moins d'une minute ne suffisait pas à récupérer après chaque sortilège. Et il avait donc \emph{seulement} lancé Collaporta \emph{sur son cadenas, il avait seulement lancé le sortilège mais il n'y avait pas mis toute sa force afin de le fermer assez solidement pour que ni Harry Potter ni Hermione Granger ne puissent le défaire.}

Mais même si c'était vrai, personne n'allait \emph{croire} ça. Même à Serpentard, personne n'allait croire ça. Ça \emph{ressemblait} à une excuse et tout le monde n'y verrait qu'une excuse.

\emph{Granger bondit, tourbillonna et s'écria '\scream{Alohomora}~!'…}

L'esprit de Drago rejouait le souvenir encore et encore, à mesure que la rancœur s'accumulait. Il avait \emph{aidé} Granger… il avait coopéré avec elle afin de bannir les traîtres… il lui avait tenu la main lorsqu'elle pendait du toit… il avait empêché une émeute de se déclencher autour d'elle dans la grande salle… avait-elle la moindre idée de ce qu'il avait \emph{risqué}, de ce qu'il avait probablement déjà \emph{perdu}, de ce que cela impliquait pour le descendant de la \emph{maison Malfoy} de faire ça pour une S\emph{ang-de-Bourbe}…

Et il n'avait maintenant plus qu'un coup jouable, et la particularité d'un coup forcé était qu'il \emph{fallait} le faire, même si cela voulait dire qu'on perdrait des points et qu'on irait en retenue. Le professeur Rogue \emph{comprendrait} certainement mais il y avait des limites (Père l'avait mis en garde) à ce que Rogue laisserait passer.

Provoquer Granger en duel magique en se défiant ouvertement des règles de Poudlard. L'attaquer immédiatement si elle essayait de refuser. La vaincre en un contre un, en public, pas avec des techniques de duel astucieuses mais en \emph{l'écrasant} de sa magie. La battre à plates coutures, complètement, la \emph{broyer} aussi parfaitement que le Seigneur des Ténèbres lui-même avait broyé ses ennemis. Le rendre parfaitement clair à tous, empêcher que quiconque en doute~: Drago Malfoy avait simplement été épuisé à force de lancer le sortilège autant de fois d'affilée. Prouver que le sang Malfoy était plus fort que celui de n'importe quel Sang-de-Bourbe…

\emph{Sauf qu'il ne l'est pas}, murmura la voix de Harry Potter dans l'esprit de Drago. \emph{Il est facile d'oublier ce qui est réellement vrai, Drago, quand on commence à essayer de gagner en politique. Mais en réalité, il n'y a qu'une seule chose qui fait de toi un sorcier, tu t'en souviens~?}

Drago le sut alors, il sut la raison derrière l'anxiété de ses arrières-pensées, face au mur vide derrière son bureau, contemplant son coup forcé. Cela aurait dû être simple -- lorsqu'on avait qu'un seul coup à jouer, il suffisait de le jouer -- mais…

\emph{Granger pivota, tourbillonna, ses cheveux trempés de sueur volants autour d'elle, des tirs jaillissants de sa baguette aussi vite que les siens, maléfice et contre-maléfice, des chauves-souris incandescentes volant vers son visage, et de bout en bout, la furie dans les yeux de Granger…}

Une partie de Drago avait \emph{admiré} cela, avant que tout tourne mal, admiré la furie de Granger et son pouvoir~; une partie de lui avait exulté de vivre son premier \emph{vrai} combat, contre…

… le premier adversaire à sa hauteur.

S'il défiait Granger et \emph{perdait}…

Cela n'aurait pas dû être possible, Drago avait obtenu sa baguette deux années entières avant que qui que ce soit d'autre ne l'obtienne à Poudlard.

Sauf qu'il y avait une \emph{raison} pour laquelle on ne se fatiguait généralement pas à donner des baguettes aux enfants de neuf ans. L'âge comptait aussi, il ne s'agissait pas seulement de savoir combien de temps on avait eu une une baguette en main. L'anniversaire de Granger avait eu lieu seulement quelques jours après le début de l'année, lorsque Harry lui avait offert sa bourse. Ce qui voulait dire qu'elle avait maintenant douze ans, qu'elle avait eu douze ans presque depuis le début de Poudlard. Et pour tout dire, Drago ne s'était pas beaucoup entraîné en dehors des cours, probablement beaucoup moins que Hermione Granger de Serdaigle. Drago n'avait pas pensé qu'il avait \emph{besoin} de plus de pratique pour rester en tête…

\emph{Et Granger était épuisée elle aussi}, chuchota la Voix des Preuves du Contraire dans sa tête. Granger devait être épuisée après tous ces sortilèges d'étourdissement, et même dans cet état, elle était parvenue à défaire son sortilège d'Emprisonnement.

Et Drago ne \emph{pouvait pas} se permettre de défier Granger en public, un contre un, pas d'excuses, puis de \emph{perdre}.

Il savait ce qu'on était censé faire dans ce genre de situation. On était censé tricher. Mais si quiconque découvrait qu'il avait triché ce serait \emph{désastreux}, une manne à chantage parfaite même sans jamais être découvert, et tous les Serpentard le \emph{sauraient}, ils \emph{chercheraient}…

Si vous l'aviez observé, vous auriez alors vu Drago Malfoy se lever de son lit, se rendre à son bureau, tirer une feuille de son parchemin en peau de mouton le plus fin ainsi qu'un encrier creusé dans une perle remplit d'une encre d'un argent verdâtre qui avait été faite à partir de véritable argent et d'émeraudes broyées. De l'immense coffre au pied de son lit, le Serpentard en tira un livre lui aussi relié d'argent et d'émeraudes intitulé \emph{Étiquette des maisons d'Angleterre}. Et avec une nouvelle plume propre, Drago Malfoy commença à écrire en regardant fréquemment le livre ouvert en référence à côté de lui. Le visage du garçon portait un sourire sinistre, ce qui faisait grandement ressembler le jeune Malfoy à son père tandis qu'il traçait précautionneusement chaque lettre comme si chacune avait été une œuvre d'art à part entière.

\begin{writtenNote}
De Drago, fils de Lucius, fils d'Abraxis, Lords de la Noble et Très Ancienne maison Malfoy, aussi fils de Narcissa, fille de Druella Dame de la Noble et Très Ancienne maison Black, descendant et héritier de la Noble et Très Ancienne maison Malfoy~:

À Hermione, la première Granger~:
\end{writtenNote}

(On avait peut-être utilisé cette formule par politesse il y a bien longtemps quand elle avait été inventée~; de nos jours, après des siècles à être utilisée pour s'adresser à des Sang-de-Bourbe, elle portait avec elle un délicieux soupçon de venin raffiné).

\begin{writtenNote}
Moi, Drago, d'une Très Ancienne maison, demande réparation pour
\end{writtenNote}

Drago s'interrompit, déplaçant précautionneusement la plume afin qu'elle ne goutte pas. Il avait besoin d'un prétexte pour cela, du moins s'il voulait pouvoir imposer ses conditions de duel. La personne provoquée avait le choix des termes \emph{à moins} d'avoir insulté une maison Noble. Il lui fallait donner l'impression que Granger l'avait insulté…

Mais à quoi songeait-il donc~? Granger \emph{l'avait} insulté.

Drago fit défiler les pages jusqu'à atteindre celle des formules standard et en trouva une qui lui sembla appropriée.

\begin{writtenNote}
Moi, Drago, d'une Très Ancienne maison, demande réparation, pour vous avoir par trois fois aidé et offert seulement ma bienveillance et qu'en retour vous m'accusiez {faussement} de comploter contre vous, \end{writtenNote}

Drago dut s'arrêter, reprendre son souffle et forcer la colère bouillonnante à retomber. Il commençait à vraiment \emph{ressentir} l'insulte à présent, et il venait d'écrire la dernière phrase et de la souligner sans y penser, comme si c'était une lettre ordinaire. Après y avoir réfléchi quelques instants, il décida de la laisser telle quelle~; ce n'était peut-être pas la formulation la plus officielle mais elle avait un ton brut et colérique qui semblait approprié.

\begin{writtenNote}
Insulte que vous avez commise face aux yeux d'Angleterre.

Ainsi, moi, Drago, vous assujettis, Hermione, par tradition, par droit, par
\end{writtenNote}

"La dix-septième décision du trente et unième Magenmagot," dit Drago à voix haute et sans vérifier, une réplique donnée dans de nombreuses pièces~; il se raidit en la disant et sentit chaque once du pouls de sang noble qui coulait dans ses veines.

\begin{writtenNote}
Ainsi, moi, Drago, vous assujettis, Hermione, par tradition, par droit, par la 17\textsuperscript{ème} décision du 31\textsuperscript{ème} Magenmagot, à me faire face lors d'un duel magique selon les termes suivants~: que nous nous présentions tous deux seuls et en silence, que nous n'en parlions à personne ni avant ni après,\end{writtenNote}

Si le duel se déroulait mal, Drago pourrait juste se taire et en rester là. Et s'il battait Granger, il aurait appris de façon expérimentale qu'il pouvait la battre \emph{de nouveau} lors d'un défi public. Ce n'était pas tricher mais c'était de la Science, et c'était presque aussi bien.

\begin{writtenNote}
que nous combattions par magie uniquement, sans mort ni blessure durable,
\end{writtenNote}

… où~? Drago avait entendu parler d'une pièce de Poudlard qui était propice aux duels, où tous les objets de valeur étaient déjà protégés par les lieux et où il n'y avait aucun portrait pour cafarder… quelle était cette pièce déjà…

\begin{writtenNote}
dans la salle des trophées du château de l'École de Sorcellerie de Poudlard. \end{writtenNote}

Et leur \emph{second} duel public avait intérêt à avoir lieu \emph{bientôt}, genre demain, car il ne faudrait pas longtemps pour que sa réputation à Serpentard ne soit irrémédiablement traînée dans la boue. Il lui fallait combattre Granger pour la première fois \emph{cette nuit}.

\begin{writtenNote}
Au dernier coup de minuit qui mettra fin à ce jour même.

Drago, de la Noble et Très Ancienne maison Malfoy.
\end{writtenNote}

Drago signa le parchemin officiel puis tira un parchemin ordinaire et inférieur ainsi que son encre habituelle, pour son post-scriptum~:

\begin{writtenNote}Si tu ne connais pas les règles, Granger, voilà comment ça fonctionne. Tu as insulté une maison Très Ancienne et j'ai le droit légal de te défier. Et si tu bafoues les termes du duel, par exemple en faisant venir Flitwick dans la salle des trophées, ou même en en parlant à qui que ce soit d'autre, mon père t'amènera toi et ton honneur de pacotille droit au Magenmagot.

Drago Malfoy
\end{writtenNote}

Sa plume s'appuya sur la dernière lettre avec tant de hargne sur son bec se brisa, créant une traînée d'encre et une légère éraflure à la surface du parchemin que Drago jugea elles aussi être de de circonstance.

\later

Ce soir là au dîner, Susan Bones alla voir Harry Potter et lui dit qu'elle soupçonnait que Drago Malfoy allait mettre son plan contre Hermione à exécution très bientôt. Elle prévenait tous les membres de la S.P.E.H.S., elle avait prévenu le professeur Chourave et le professeur Flitwick, elle allait envoyer une lettre à sa tante ce soir là, et maintenant elle prévenait aussi Harry Potter. Sauf qu'ils ne pouvaient pas vraiment en parler avec Padma -- dit Susan avec un air très sérieux -- parce que celle-ci se sentait déchirée entre sa loyauté envers Hermione et sa loyauté envers son général.

Harry James Potter-Evans-Verres, qui à ce stade était tellement frustré par la situation que ça en cessait d'être productif, répondit vivement que \emph{oui}, il savait qu'il fallait faire quelque chose.

Après que Susan Bones fut partie, Harry leva les yeux vers la table Serdaigle, où Hermione s'était assise loin de lui, de Padma, d'Anthony et de tous ses amis.

Plus tard, en y repensant, Harry songerait comment, dans ses romans de science-fiction et de fantasy, les gens faisaient toujours leurs choix importants pour des raisons importantes, pour des raisons de taille. Hari Seldon avait bâti sa Fondation pour reconstruire les cendres de l'Empire Galactique plutôt que pour se donner un air important en étant à la tête de son propre groupe de recherche. Raistlin Majere avait coupé les ponts avec son frère parce qu'il souhaitait devenir un dieu, pas parce qu'il était inapte en relations humaines et réticent à demander conseil pour s'améliorer. Frodo Baggins avait pris l'Anneau parce qu'il était un héros désireux de sauver la Terre du Milieu, pas parce que ça aurait été trop gênant de refuser de le faire. Si quelqu'un écrivait un jour une véritable histoire du monde -- non que ce soit possible ni même désirable -- il était probable que plus de 97~\% des moments clés du destin du monde s'avèrent être le fruit d'un tissu de mensonges et de petite pensées sans importances qui auraient aussi bien pu aboutir à d'autres conclusions.

Harry James Potter-Evans-Verres regarda Hermione Granger, assise à l'autre bout de la table, et il sentit en lui une réticence à l'embêter alors qu'elle était déjà de mauvaise humeur.

Il se dit donc qu'il était probablement plus raisonnable de parler d'abord à Drago Malfoy juste pour pouvoir définitivement et irrévocablement assurer Hermione que Drago ne complotait vraiment rien contre elle.

Et plus tard au dîner, quand Harry descendit à la cave Serpentard et s'entendit dire par Vincent que \emph{le boss voulait pas êt'dérangé…} alors Harry \emph{songea} qu'il devrait peut-être aller voir si Hermione accepterait de lui parler immédiatement. Qu'il devrait juste commencer à défaire ce paquet de nœuds avant qu'il ne s'emmêle encore plus. Il se demanda s'il ne faisait pas que procrastiner, si son esprit n'avait pas trouvé une excuse habile pour remettre à plus tard quelque chose de déplaisant mais de nécessaire.

Cette pensée lui traversa vraiment l'esprit.

Puis Harry James Potter-Evans-Verres décida qu'il parlerait juste à Drago Malfoy le lendemain matin après le petit déjeuner du dimanche et qu'il parlerait \emph{ensuite} à Hermione.

Les humains faisaient ce genre de chose en permanence.

\later

Nous étions dimanche matin, le 5 avril 1992, et le ciel simulé au-dessus de la grande salle de Poudlard révélait des torrents de pluie qui tombaient avec une densité telle que les éclairs lumineux en étaient atténués et brisés en de petites pulsations de lumière blanche qui transformaient parfois les tables de chaque maison, faisant pâlir les visages et donnant brièvement aux élèves l'allure de fantômes.

Harry était à la table Serdaigle et mangeait une gaufre avec lassitude, attendant que Drago fasse son apparition afin de pouvoir commencer à débrouiller toute l'affaire. Un exemplaire du \emph{Chicaneur} qu'on passait de mains en mains avait réussir à placer Hannah et Daphné sur sa couverture mais le journal n'avait pas encore atteint la place de Harry.

Quelques minutes plus tard Harry finit de manger sa gaufre et regarda \emph{encore} autour de lui pour voir si Drago était arrivé à la table Serpentard pour prendre son petit déjeuner.

C'était étrange.

Drago Malfoy n'était presque jamais en retard.

Puisque Harry regardait vers la table Serpentard, il ne vit pas que Hermione Granger était entrée, franchissant les immenses portes de la grande salle. Il fut donc assez surpris lorsqu'il se retourna et vit que Hermione était assise juste à côté de lui à la table Serdaigle, exactement comme elle n'avait pas arrêté de faire ça depuis plus d'un semaine.

"Salut, Harry," dit Hermione d'une voix qui était presque parfaitement normale. Elle commença à mettre des tranches de pain grillé et un assortiment de fruits et de légumes sur son assiette. "Comment ça va~?

--- À un écart type de l'étrange moyenne qui m'est propre," répondit Harry automatiquement. "Comment vas-tu, \emph{toi}~? Est-ce que tu as dormi correctement~?"

Il y avait des cernes sombres sous les yeux de Hermione.

"Oui pourquoi, je vais très bien," répondit Hermione Granger.

"Hmm," dit Harry. Il prit une part de tarte de son assiette (comme son cerveau avait d'autres préoccupations, sa main avait simplement pris l'objet le plus savoureux à portée sans évaluer des concepts complexes tels que le positionnement du dessert dans un repas normal). "Hmm, Hermione, je vais devoir discuter avec toi un peu plus tard aujourd'hui, est-ce que ça te va~?

--- Bien sûr," répondit-elle. "Pourquoi est-ce que ça n'irait pas~?

--- Parce que…" dit Harry. "Je veux dire… toi et moi \emph{n'avons pas}… pendant les derniers jours… "

\emph{La ferme}, suggéra un composant interne de Harry qui semblait avoir été récemment alloué à la résolution des problèmes liés à Hermione.

Cette dernière ne semblait de toute façon pas lui prêter une attention particulière. Elle regarda juste son assiette, puis, au bout environ dix secondes d'un silence gêné, elle commença à manger ses tranches de tomate, l'une après l'autre, sans s'interrompre.

Harry détourna les yeux de Hermione et commença à manger une part de tarte qui, comme il le découvrait à l'instant, venait de se matérialiser dans son assiette.

"Donc~!" dit soudain Hermione Granger après avoir silencieusement expédié quasiment tout le contenu de son assiette. "Il se passe quelque chose aujourd'hui~?

--- Euh…" dit Harry. Il regarda autour de lui avec désespoir comme à la recherche d'une chose-se-passant qu'il aurait pu utiliser comme chair à conversation.

Et il fut donc le premier à les voir et à pointer du doigt vers eux sans dire un mot, même si le crescendo soudain de murmures qui traversa la grande salle révéla que de nombreux autres les avaient aussi vus.

La teinte distinctement cramoisi des robes aurait été reconnaissable n'importe où, mais le cerveau de Harry mit quand même quelques instants à remettre les visages. Un homme au visage pseudo-asiatique, solennel, et aujourd'hui plutôt sinistre. Un homme dont le regard perçant balayait la salle, ses longs cheveux noirs attachés dans son dos en queue de cheval. Un homme fin, pâle et mal rasé, avec un visage si inexpressif qu'on l'aurait cru de pierre. Harry mit un moment à remettre les visages et à se souvenir des noms, de ce lointain jour de janvier où le Détraqueur était venu à Poudlard~: \emph{Komodo, Butnaru, Goryanof}.

"Un trio d'Aurors~?" dit Hermione d'une voix étrangement joyeuse. "Eh bien, je me demande ce qu'ils font ici."

Dumbledore était lui aussi avec eux, l'air plus inquiet que Harry ne l'avait jamais vu~; et après quelques instants d'immobilité pendant lesquels les yeux du sorcier parcoururent la grande salle et que les élèves murmurèrent par-dessus leurs assiettes, il pointa le doigt…

… droit vers Harry.

"Oh, \emph{quoi} maintenant," marmonna Harry. Ses pensées étaient bien plus paniquées et il se demandait avec désespoir si quiconque était parvenu à le relier à l'intrusion à Azkaban. Il regarda la table d'honneur en essayant de rendre son coup d'œil nonchalant et se rendit compte que le professeur Quirrell était introuvable ce matin…

Les Aurors s'approchèrent de lui à grands pas vifs, Auror Goryanof depuis l'autre côté de la table Serdaigle comme pour bloquer toute fuite dans cette direction, Aurors Komodo et Butnaru depuis le flanc de Harry, et le directeur sur les talons de Komodo.

Toutes les conversations s'était figées en un silence absolu.

Les Aurors s'approchèrent de la place de Harry et l'encerclèrent sur trois côtés.

"Oui~?" dit Harry aussi normalement que possible. "Qu'y a-t-il~?

--- Hermione Granger," dit Auror Komodo d'une voix sans timbre, "vous êtes en état d'arrestation pour avoir tenté d'assassiner Drago Malfoy." 

%  LocalWords:  Snocks Tess Gingersnap Berdine Beringer hahahaha Moste ur
%  LocalWords:  Potente Arresto Re’em Dugbogs Re’em’s Darke Hari
%  LocalWords:  Sunnie Steleus Quiescus Tunafish Colloportused
%  LocalWords:  Abraxis Druella Seldon Raistlin Majere Asianish
