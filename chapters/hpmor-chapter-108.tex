\namedpartchapter{La Vérité}{La Vérité}{V}{Réponses et Jeux du sort}

\lettrine{D}{'un} ~: leurs pétales indigo semblaient presque lumineux sous la lumière blanche des murs et leur courbure évoquait un besoin d'intimité. La première fleur avait immédiatement été jetée dans le chaudron, puis la cuillère avait continué de remuer.

Le professeur de Défense s'était arrangé pour pouvoir regarder Harry en tournant très légèrement la tête, et ce dernier savait qu'il était encore dans la vision périphérique du professeur.

Dans un coin, un phénix de Feudeymon attendait. Les pierres les plus proches avaient commencé à mollir. Les ailes de feu projetaient une lumière cramoisie qui donnait à la pièce la teinte du sang et scintillait d'écarlate dans l'argenterie.

"Nous perdons du temps," dit le professeur Quirrell. "Pose-moi tes questions, si tu en as."

\emph{Pourquoi, professeur Quirrell, pourquoi êtes-vous ainsi, pourquoi vous êtes-vous fait monstre, pourquoi Lord Voldemort, je sais que nous n'avons pas les mêmes buts mais je n'arrive pas imaginer que c'est} ça, \emph{le meilleur moyen d'atteindre les vôtres…}

Voilà ce que le cerveau de Harry désirait savoir.

Ce qu'il avait \emph{besoin} de savoir… un moyen d'empêcher ce qui allait suivre. Mais le professeur Quirrell avait dit qu'il ne parlerait pas de ses plans futurs. Il était déjà assez étrange de le voir prêt à parler de \emph{quoi que ce soit}. Cela devait bien contredire une de ses Règles…

"Je réfléchis," dit Harry.

Le professeur Quirrell eut un léger sourire. Il utilisait un pilon pour moudre le premier ingrédient magique de la potion, un hexagone rouge et lumineux. "Je comprends \emph{tout à fait}," dit le professeur Quirrell. "Mais ne prends pas trop longtemps, petit."

\emph{Buts~: Empêcher Lord Voldemort de faire du mal à d'autres, trouver comment le tuer ou le neutraliser, mais d'abord obtenir la Pierre et ressusciter Hermione…}

\emph{… Convaincre le professeur Quirrell d'ARRÊTER ÇA…}

Harry déglutit et repoussa ses émotions, tenta d'empêcher les larmes de naître. Cela ne plairait probablement pas à Lord Voldemort. Le professeur Quirrell fronçait déjà les sourcils, même s'il semblait examiner une feuille vivement teintée de blanc, de vert et de violet.

Aucun moyen évident d'atteindre ses buts. Pas encore. Tout ce qu'il pouvait faire, c'était de poser les questions les plus à même de l'aider, même sans plan préalable.

\emph{Alors on pose juste des questions sur ce qui nous intéresse le plus~?}, dit la partie Serdaigle de Harry. \emph{Ça me va.}

\emph{Tais-toi}, dit Harry à la voix~; puis, après réflexion, il décida de ne plus s'imaginer que la voix existait.

Quatre sujets vinrent à l'esprit de Harry, prioritaires en termes de curiosité suscitée et d'importance. Quatre questions, donc quatre sujets majeurs, à explorer le temps que la potion se prépare.

Quatre questions…

"Je pose ma première question," dit Harry. "Qu'est-ce qui s'est vraiment passé le 31 octobre 1981~?" \emph{Pourquoi cette nuit fut-elle différente des autres nuits}… "Je veux tout savoir, s'il vous plaît."

La façon dont Lord Voldemort avait survécu à sa mort apparente risquait fort d'être important plus tard.

"Je m'attendais à ce que tu me demandes ça," dit le professeur Quirrell, et il fit tomber une campanule et une pierre blanche scintillante dans la potion. "Pour commencer, tout ce que je t'ai dit sur le sortilège de Horcruxe est vrai~; ce que tu devrais savoir, puisque je parlais en Fourchelangue."

Harry acquiesça.

"Quelques secondes après avoir découvert ce sortilège, tu en as compris la faille centrale et tu as commencé à te demander comment l'améliorer. Penses-tu que le jeune Tom Jedusor était différent~?"

Harry secoua la tête.

"Eh bien si," dit le professeur Quirrell. "Lorsque l'envie de me désespérer de toi me prenait, je me rappelais que j'étais deux fois plus idiot que toi lorsque j'avais le double de ton âge. À quinze ans, je me suis fabriqué un Horcruxe, comme un certain livre me l'indiquait, en utilisant la mort d'Abigail Myrtle par les yeux du basilic de Serpentard. Je comptais en faire un par an après mon départ de Poudlard et voir ça comme mon plan de secours au cas où mes autres espoirs d'immortalité seraient déçus. Rétrospectivement, le jeune Tom Jedusor refusait d'accepter les faits. L'idée de faire un \emph{meilleur} Horcruxe, de ne pas me satisfaire du sortilège que j'avais déjà appris… Cette pensée ne me vint que lorsque je compris l'idiotie des gens ordinaires et découvris lesquelles de leurs folies j'avais imitées. Mais je finis par apprendre l'habitude que tu as héritée de moi~: celle de toujours demander comment on pourrait faire mieux. Me contenter du sortilège que j'avais appris dans un livre, alors qu'il n'était qu'un pâle reflet de mon vrai désir~? Absurde~! Alors je me suis engagé dans la création d'un meilleur sortilège."

"Vous êtes vraiment immortel, maintenant~?" Harry se rendait compte que, en dépit de tout, cette question surpassait toute considération tactique.

"Oui," dit le professeur Quirrell. Il interrompit son travail et se retourna, pour faire pleinement face à Harry. Il y avait dans ses yeux une exultation que Harry n'avait jamais vue avant. "Au cœur des arts les plus noirs que j'ai pu trouver, dans tous les secrets interdits dont le monstre de Serpentard m'a donné les clés, dans tous les savoir des sorciers, je n'ai pu trouver qu'une poignée d'indices sur ce dont j'avais besoin. Alors j'ai reconstruit, j'ai inventé, devisé de nouveaux rituels fondés sur de nouveaux principes. J'ai laissé ce rituel brûler en pensée pendant des années, je l'ai perfectionné, j'ai songé à son sens, effectué de subtils ajustements, attendu que l'intention se stabilise. Et, enfin, j'ai osé invoquer mon rituel sacrificiel, inventé, fondé sur un principe jamais testé par la magie. Et j'ai survécu. Et je vis toujours." Du professeur de Défense émanait un silencieux triomphe, comme si aucune parole ne pourrait jamais faire justice à un acte d'une telle grandeur. "J'utilise toujours le mot 'Horcruxe', mais seulement par sentimentalité. C'est quelque chose d'autre. C'est ma plus grande création."

"J'aimerais savoir, et vous avez dit que vous répondriez à mes questions, comment lancer ce sortilège," dit Harry.

"Non." dit le professeur Quirrell. Il se retourna vers ses potions et fit tomber un plume blanche et une campanule dans la potion. "J'avais envisagé de te l'apprendre quand tu serais plus âgé, car rien d'autre ne saurait satisfaire un Tom Jedusor, mais j'ai changé d'avis."

Les souvenirs sont parfois difficiles à faire revenir, et Harry essaya de se rappeler si le professeur Quirrell avait déjà laissé des indices à ce sujet. Quelque chose dans la tournure éveilla un souvenir~: \emph{Peut-être l'apprendrez-vous quand vous serez plus âgé…}

"Votre immortalité a toujours besoin d'une ancre physique," dit Harry. "En cela au moins, il ressemble au Horcruxe original, et c'est aussi pour cela que vous les appelez toujours Horcruxes." Le dire à voix haute était dangereux, mais il avait besoin de \emph{savoir}. "Si j'ai tort, vous pouvez toujours nier en Fourchelangue."

Le professeur Quirrell avait un mauvais sourire. "\parsel{Tu as deviné jusste, petit, pour tout le bien que çcela te fait.}"

Pour un ennemi intelligent, cette brèche n'était malheureusement pas difficile à colmater. Harry n'aurait pas proféré la suggestion en temps normal, juste cas où l'ennemi n'y aurait \emph{pas} pensé, mais dans ce cas précis, il l'avait déjà fait~: "Un Horcruxe au fond d'un volcan actif, lesté pour s'enfoncer dans le manteau terrestre," dit Harry d'une voix grave. "Le même endroit où j'ai pensé mettre le Détraqueur, si jamais je n'arrivais pas à le détruire. Et vous m'avez alors demandé où je cacherais une chose que je souhaite ne jamais voir découverte. Un Horcruxe enterré plusieurs kilomètres sous la croûte terrestre, dans un mètre cube anonyme. Un autre dans la fosse Marianne. Un dans la stratosphère, transparent. Même vous ne savez pas où ils sont, parce que vous avez fait disparaître les détails de vos souvenirs. Et le dernier Horcruxe est la plaque de Pioneer 11 que vous avez modifiée avant le décollage. C'est là où vous allez chercher votre image des étoiles, quand vous lancez ce sortilège de nuit céleste. Le feu, la terre, l'eau et l'air, le vide." \emph{Les fruits de jeux du sort}, avait dit le professeur Quirrell, et Harry s'en était souvenu. Les fruits de Jedusor.

"Tout à fait," dit le professeur de Défense. "J'ai été pour le moins surpris quand tu t'en soit souvenu si vite, mais je suppose que cela n'a pas d'importance~: ils sont tous hors de ma portée et de la tienne."

Ce n'était peut-être pas vrai, surtout s'il existait un moyen de suivre le lien magique et de déterminer l'emplacement… mais Voldemort aurait probablement fait de son mieux pour le masquer… et ce que la magie pouvait faire, peut-être pouvait-elle aussi le défaire. Pioneer 11 était peut-être loin pour un sorcier, mais la NASA connaissait son emplacement exact et elle serait probablement beaucoup plus simple à atteindre quand on aurait utilisé la magie pour dire à l'équation spatiale de Tsiolkovski d'aller se faire voir…

Une inquiétude naquit soudain en lui. Rien n'interdisait au professeur de Défense de \emph{mentir} quant à la sonde spatiale qu'il avait horcruxée~; et s'il se souvenait bien, la communication avec Pioneer 10 avait été perdue peu après son survol de Jupiter.

Pourquoi ne pas avoir mis un Horcruxe sur les deux~?

L'idée évidente lui vint alors. Ce n'était pas le genre de suggestion à faire à un ennemi ignorant. Mais il était extrêmement probable que l'ennemi y avait déjà pensé.

"\parsel{Dites-moi, professseur,}" siffla Harry, "\parsel{détruire les cinq ancres vous tuerait-il~?}"

"\parsel{Pourquoi cette quesstion~?}" siffla le professeur de Défense en retour, avec une cadence qui, en Fourchelangue, exprimait l'amusement du serpent. "\parsel{Pensses-tu que la réponsse est non~?}"

Harry ignorait comment répondre, mais il était presque certain que sa réponse n'aurait pas d'importance.

"\parsel{Tu as raisson, petit. Détruire ces cinq-là ne me rendrait pas mortel.}"

Harry eut de nouveau la gorge sèche. Si le sortilège ne coûtait pas atrocement cher… "\parsel{Combien d'ancres avez-vous faites~?}"

"\parsel{Je ne répondrais pas en temps normal, mais il est évident que tu as deviné.}" Le sourire du professeur de Défense s'élargit. "\parsel{La réponsse, c'est que je l'ignore. J'ai cesssé de compter après les çcent ssept premiers. C'est devenu une habitude à chaque fois que je tuais quelqu'un ssans témoin.}"

Plus de \emph{cent} meurtres sans témoin avant que Lord Voldemort ait arrêté de compter. Et même pire - "Votre immortalité nécessite toujours une mort humaine~? \emph{Pourquoi~?}"

"\parsel{Grande création maintient vie et magie dans appareil créé par ssacrifice de vie et magie d'autres.}" Encore le rire sifflé de serpent. "\parsel{J'ai tellement aimé la faussse desscription du Horcruxe préçcédent, été tellement déççu quand j'ai compris la vérité, çcela a influençcé mes idées d'amélioration.}"

Harry ne savait pas pourquoi le professeur de Défense lui donnait ces informations vitales mais il \emph{fallait qu'il y ait une raison}, et cela le rendait nerveux. "Donc vous êtes vraiment un esprit sans corps qui possède Quirinus Quirrell."

"\parsel{Oui. Je reviendrais vite, si ce corps est détruit. Je serais très énervé et je me vengerai}. Je te dit cela, petit, pour que tu ne tentes rien de stupide."

"Je comprends," dit Harry. Il fit de son mieux pour organiser ses idées et se souvenir de sa prochaine question tandis que le professeur de Défense revenait à ses potions. La main gauche versait des coquillages en miettes dans le chaudron, et la droite ajoutait une autre campanule. "Alors, que s'est-il passé le 31 octobre~? Vous… avez essayé de transformer le bébé Harry Potter en un Horcruxe, soit le nouveau soit l'ancien genre. Je sais que c'était voulu, parce que vous avez dit à Lily Potter…" - il inspira. Maintenant qu'il savait \emph{pourquoi} il frissonnait, il pouvait l'endurer. "Très bien, j'accepte le marché. Tu meurs et l'enfant vit. Maintenant lâche ta baguette, que je puisse te tuer." Avec le recul, il était clair que Harry se souvenait de cet événement surtout du point de vu de Lord Voldemort, et qu'il n'avait observé par les yeux du bébé Harry Potter qu'à la fin. "Qu'est-ce que vous avez fait~? \emph{Pourquoi} est-ce que vous l'avez fait~?"

"La prophétie de Trelawney," dit le professeur Quirrell. Sa main frappa une campanule d'un morceau de cuivre avant de le faire tomber. "J'y ai réfléchi pendant longtemps après avoir entendu la prophétie rapportée par Rogue. Les prophéties ne sont pas des choses triviales. Et comment dire ça sans que ta réponse soit stupide… Bon, je vais le dire, et si tu fais l'idiot je serai exaspéré. J'étais fasciné par l'affirmation, contenue dans la prophétie, selon laquelle quelqu'un serait mon égal, car alors, peut-être que cette personne pourrait avoir une conversation intelligente avec moi. Après cinquante ans entouré de stupidité profonde, l'idée que ma réaction soit un cliché littéraire m'importait peu. Je n'allais pas laisser passer cette opportunité sans y réfléchir. Et c'est là, vois-tu, que j'ai eu une \emph{idée astucieuse}." Le professeur Quirrell soupira. "Je pensai à accomplir la prophétie à ma façon, à mon avantage. Je marquerai le bébé comme mon égal en lui lançant l'ancien sortilège de Horcruxe de façon à imprimer mon esprit dans celui, vierge, du bébé. Ce serait une pure copie, puisqu'il n'y aurait personne à mélanger à ma personnalité. Je prévoyais que des années plus tard, ennuyé d'avoir dirigé l'Angleterre, passé à autre chose, je m'arrangerais pour que l'autre Tom Jedusor semble me vaincre et règne sur l'Angleterre ainsi sauvée. Nous jouerions à ce jeu l'un contre l'autre pour toujours, et nos vies, dans un monde d'imbéciles, resteraient intéressantes. Je savais qu'un dramaturge prédirait que nous finirions par nous détruire l'un-l'autre, mais j'y réfléchis pendant longtemps et je décidai que nous refuserions tous les deux de mener ce drame à son terme. C'était ma décision et j'avais confiance en sa pérennité~: je songeais que les deux Tom Jedusor seraient trop intelligents pour s'aventurer sur cette voie. La prophétie semblait laisser entendre que si je ne laissais qu'un vestige de Harry Potter, alors nos esprit ne seraient plus si différents que ça et que nous pourrions exister dans le même monde."

"Quelque chose a raté," dit Harry. "Quelque chose qui a fait sauter le toit des Potter à Godric's Hollow, qui m'a donné ma cicatrice et qui a laissé les restes calcinés de votre corps."

Le professeur Quirrell hocha la tête. Ses mains s'étaient ralenties dans leur travail. "La résonance de nos magies," dit-il doucement. "Quand j'ai donné à l'esprit de ce bébé la forme du mien…"

Harry se souvint du moment, à Azkaban, où le sortilège de la mort du professeur Quirrell était entré en collision avec son Patronus. La douleur brûlante, déchirante dans son front, comme si on allait lui fendre la tête en deux.

"Je ne sais combien de fois j'ai repensé à cette nuit, rejoué mon erreur, pensé à des réactions plus sages," continua le professeur Quirrell. "J'ai plus tard décidé que j'aurais dû jeter ma baguette et me changer en Animagus. Mais cette nuit… cette nuit, j'ai instinctivement essayé de contrôler les fluctuations chaotiques de ma magie alors même que je me sentais brûler de l'intérieur. C'était la mauvaise décision, et j'ai échoué. Mon corps a été détruit alors même que je réécrivais l'esprit du jeune Harry Potter~; nous n'avons \emph{tous les deux} laissé qu'un vestige de l'autre. Et alors…" le professeur Quirrell contrôlait l'expression de son visage. "Et alors, lorsque je me suis réveillé dans mes Horcruxes, il s'est avéré que ma grande création ne fonctionnait pas comme je l'avais espéré. J'aurais dû être capable de flotter, libéré de mes Horcruxes, capable de posséder celui qui y consentirait ou qui serait trop faible pour me rejeter. \emph{Voilà} la partie de ma grande création qui a dévié de mon intention première. Comme avec le sortilège originel, je ne pouvais entrer dans une victime qu'après son contact physique avec un Horcruxe… et j'avais cachés ces innombrables Horcruxes là où personne ne les trouverait jamais. Écoutes ton instinct, petit~: \emph{ce n'est pas le moment de rire}."

Harry demeura très silencieux.

La fabrication de potion s'était temporairement interrompue, un temps sans ajout d'ingrédient, où le chaudron devait frémir un moment. "J'ai passé presque tout mon temps à regarder les étoiles," dit-il d'une voix maintenant plus basse. Il s'était détourné des potions et regardait les murs de la salle, éclairés de blanc. "Mon dernier espoir était les Horcruxes que j'avais cachés pendant mon imbécile jeunesse. Je les avais placés dans de vieux couvre-chefs plutôt que dans des cailloux anonymes, fait garder par des puits de poisons au centre de lacs d'Inferi au lieu de les téléporter au fond de la mer. Si quelqu'un trouvait l'un d'eux, traversait ces protections ridicules… mais cet espoir semblait lointain. Je n'étais pas sûr de jamais retrouver un corps. Mais, au moins, j'étais immortel. Le pire avait été évité~; cela au moins avait été accompli. Je n'avais plus beaucoup ni à espérer ni à craindre. Je décidais de ne pas devenir fou, car cela ne m'apporterait rien. Je préférais regarder les étoiles et réfléchir, le soleil toujours plus petit derrière moi. Je songeais aux erreurs de ma vie passée~; j'en voyais beaucoup. J'imaginais de nouveaux et puissants rituels à essayer si un jour je retrouvais ma magie et restais immortel. Aussi patient que j'ai pu me croire jusque-là, je contemplais d'anciens mystère plus longuement que jamais auparavant. Je savais que si je me libérais, je serais bien plus puissant que dans ma vie passée~; mais je ne m'attendais vraiment pas à ce que cela arrive." Le professeur Quirrell se retourna vers la potion. "Neuf ans et quatre mois après cette nuit, un aventurier nommé Quirinus Quirrell franchit les protections qui gardaient l'un de mes premiers Horcruxes. Tu connais le reste. Et maintenant, petit, tu peux dire tout haut ce que nous pensons tous les deux."

"Euh," dit Harry, "je ne sais pas si ce serait malin de ma part…"

"Effectivement. Il ne serait pas sage de me le dire. Pas du tout. Pas le moins du monde. Mais \emph{je sais que tu le penses}, et tu \emph{continuera de le penser} et je \emph{continuerai de le savoir} que tu le dises ou pas. Alors, parle."

"Donc. Euh. Je sais que c'est plus évident avec du recul que sur le moment, et je ne vous encourage certainement pas à corriger l'erreur aujourd'hui, mais si vous êtes un mage noir et que vous entendez parler d'un enfant qui, selon une prophétie, va vous détruire, il existe un sortilège, inarrêtable, qui fonctionne parfaitement sur tout ce qui est doté d'un cerveau…"

"\emph{Oui, merci M. Potter, j'y ai pensé plusieurs fois pendant les neufs années qui ont suivies.}" Le professeur Quirrell prit une autre campanule et commença à la réduire en miette dans son poing serré. "Si j'ai fait de ce sortilège le cœur de mon cours de magie de bataille, c'est parce que j'ai appris son importance à la dure. Ce n'était \emph{pas} la première règle de la liste du jeune Tom Jedusor. Ce n'est qu'à travers des expériences difficiles que l'on apprend quels principes sont les plus importants~; à les entendre, aucun n'est plus persuasif que l'autre. Avec le recul, il aurait été préférable que j'envoie Bellatrix chez les Potter à ma place, mais j'avais une règle me disant, dans de tels cas, d'y aller moi-même plutôt que d'envoyer un lieutenant digne de confiance. \emph{Oui}, j'ai pensé au sortilège de la mort, mais je me suis demandé si, lancé sur un nourrisson, il ne rebondirait pas sur moi et n'accomplirait pas la prophétie. Comment pouvais-je savoir~?"

"Alors utilisez une hache, ça ne doit pas être facile de faire jaillir un sortilège d'accomplissement prophétique d'une hache," dit Harry juste avant de décider de se la fermer.

"J'ai décidé que le plus sûr était d'accomplir la prophétie à ma façon," dit le professeur Quirrell. "Inutile de dire que la prochaine fois que j'entends parler d'une prophétie qui me déplaît, je la réduirai en pièce \emph{à chaque point d'intervention possible} plutôt que de vouloir m'en accommoder." Toujours de son poing, le professeur Quirrell écrasait une rose comme s'il voulait en extraire le jus. "Et maintenant, tout le monde pense que le Survivant est immunisé contre le sortilège de la mort, même si le sortilège de la mort ne détruit pas de maisons et ne laisse pas de corps calcinés, \emph{parce qu'ils n'ont jamais pensé que Lord Voldemort pourrait utiliser un autre sortilège}."

Harry demeura silencieux, une fois de plus. Il venait de penser à une autre façon simple d'éviter l'erreur que Lord Voldemort avait commise. Un moyen peut-être plus simple à envisager lorsque l'on avait été élevé par des Moldus, et pas par des sorciers.

Il n'avait pas encore décidé s'il allait le lui dire~; il y avait des avantages et des inconvénients à le faire.

Au bout d'un moment, le professeur Quirrell prit le prochain ingrédient, une mèche de ce qui ressemblait à des poils de licorne. "Je te dis cela en guise d'avertissement," dit le professeur Quirrell. "Si jamais tu réussissais à détruire ce corps, ne compte pas sur un répit de neuf ans. Mes Horcruxes sont mieux positionnés, maintenant, et même cela est inutile. Grâce à toi, j'ai trouvé la pierre de résurrection. Elle ne fait pas revenir les morts, évidemment~; mais elle contient une magie plus ancienne que la mienne, capable de projeter un esprit. Et puisque j'ai vaincu la mort, la Relique de Cadmus m'a reconnu comme son maître et m'a obéi. Elle est maintenant incorporée à ma grande création." Le professeur Quirrell eut un léger sourire. "Il y a bien longtemps, j'avais pensé à transformer cet anneau en Horcruxe, mais j'avais abandonné l'idée, car l'anneau contenait une magie inconnue… ah, l'ironie du sort. Mais je digresse. \emph{Toi}, petit, c'est grâce à \emph{toi}, tu as libéré mon esprit, il peut maintenant voler où bon lui semble, séduire les victimes les plus opportunes, tout ça parce que tu as mal gardé tes secrets. C'est une catastrophe pour tous mes ennemis, et tu as révélé ce secret tout en faisant des ronds sur une tasse à thé. Ce monde sera bien plus sûr si tu apprends la rectitude absorbée par tous les nés-sorciers pendant leur enfance. \parsel{Tout ce que je viens de dire est vrai.}"

Harry ferma les yeux et se massa le front~; vu de l'extérieur, on aurait dit le professeur Quirrell en pleine réflexion.

Vaincre le professeur Quirrell semblait de plus en plus difficile, même mesuré à l'aune des problèmes impossible que Harry avait déjà résolus. Si le professeur Quirrell essayait de faire ressentir cette difficulté, c'était \emph{réussi}. Harry commençait sérieusement à envisager d'offrir toute l'Angleterre au représentant non-meurtrier de Lord Voldemort, si le professeur Quirrell acceptait d'arrêter de \emph{passer son temps à tuer}. Ou même de \emph{moins le faire}.

Mais ça ne risquait pas d'arriver.

Harry regardait ses mains, assis par terre, et il sentit sa tristesse glisser vers le désespoir. Le Lord Voldemort qui avait donné son côté obscur à Harry avait passé \emph{très longtemps} à songer, à réfléchir à lui-même, à ses pensées… et de cela était sorti le professeur Quirrell, calme, lucide, et toujours meurtrier.

Le professeur Quirrell ajouta une pincée de cheveux d'or à la potion de splendeur, ce qui rappela à Harry que le temps ne s'était pas arrêté~; les mèches de cheveux se faisaient plus rares que les campanules.

"Je pose ma deuxième question," dit Harry. "Parlez-moi de la Pierre Philosophale. Que fait-elle, à part rendre les métamorphoses permanentes~? Est-il possible de fabriquer d'autres pierres, et pourquoi est-ce que c'est difficile~?"

Le professeur Quirrell était penché sur la potion, et Harry ne pouvait pas voir son visage. "Très bien, je te dirai l'histoire de la pierre, telle que je l'ai devinée. L'unique pouvoir de la pierre est l'imposition de permanence~; celui de faire d'une forme temporaire une substance durable - un pouvoir qui dépasse celui de tous les sortilèges. Les conjurations telles que le château de Poudlard sont maintenus par un apport constant de magie. Même les Métamorphomages ne peuvent se faire pousser des ongles d'or et les vendre ensuite. La théorie est qu'ils ne font que réarranger leur substance, comme un forgeron Moldu manipule le fer avec son marteau et ses pinces~; leur corps ne contient pas d'or. L'Histoire ne dit pas que Merlin pouvait faire surgir de l'or du néant. Donc on peut deviner, avant même de commencer à chercher, que la Pierre doit être très ancienne. En revanche, Nicholas Flamel n'est connu que depuis six-cents ans. Dis-moi, petit, quelle est la question évidente à poser, pour retracer l'histoire de la Pierre~?"

"Euh," dit Harry. Concentré, il se frotta le front. Si la Pierre était vieille mais que le monde ne connaissait Nicholas Flamel que depuis six siècles… "Y a-t-il un autre sorcier très âgé qui aurait disparu peu de temps avant que Nicholas Flamel apparaisse~?"

"Presque," dit le professeur Quirrell. "Tu te souviens qu'il y a environ six-cents ans vivait une mage noire, dite l'immortelle, la sorcière Baba Yaga~? On disait qu'elle pouvait guérir toutes ses blessures, se métamorphoser à volonté… elle possédait la Pierre de Permanence, évidemment. Puis, un jour, Baba Yaga accepta d'enseigner la magie de bataille à Poudlard, selon une ancienne tradition de trêve, depuis toujours respectée." Le professeur Quirrell semblait… \emph{en colère}, ce que Harry avait rarement vu chez lui. "Mais on ne lui faisait pas confiance, alors on jeta une malédiction. Elles sont plus simples lorsque le mage et la cible sont autant liés l'un que l'autre, comme pour le Fourchelangue de Serpentard. Dans ce cas, on plaça la signature de Baba Yaga, de tous les enseignants et de tous les élèves de Poudlard dans un ancien artefact appelé la Coupe de Feu. Elle jura de ne verser le sang d'aucun élève, ni de leur prendre ce qui leur appartenait. En retour, le élèves jurèrent de ne pas verser le sang de Baba Yaga, ni de prendre ce qui lui appartenait. Ils signèrent tous, avec la Coupe de Feu pour témoin et bourreau en cas de transgression."

Le professeur Quirrell prit un autre ingrédient, un fil d'or enroulé autour d'une boulette d'une immonde substance. "Il y avait alors une sorcière appelée Perenelle, en sixième année à Poudlard. Et même si sa jeune beauté venait d'éclore, son cœur était déjà plus noir que celui de Baba Yaga…."

"\emph{Vous} dites que c'est une méchante~?" dit Harry, et il se rendit compte qu'il venait d'utiliser l'argument \emph{ad hominem} \emph{tu quoque}.

"Chut, petit, je raconte l'histoire. Où en étais-je~? Ah oui, la belle et cupide Perenelle. Pendant des mois, Perenelle séduit la mage noire, par des effleurements, des paroles, et une timide prétention d'innocence. Le cœur de la mage noire fut capturé et elles devinrent amantes. Puis une nuit, Perenelle murmura à Baba Yaga qu'elle avait entendu parler de son pouvoir de métamorphe, et que cela avait attisé son désir~; elle obtint de Baba Yaga qu'elle vienne à elle, Pierre en main, et que pour leur plaisir elle recouvre plusieurs formes en une seule nuit. Entre autres, Perenelle enjoint Baba Yaga à prendre la forme d'un homme, et elles s'unirent comme le font les hommes et les femmes. Mais jusqu'à cette nuit, Perenelle avait été vierge. Et puisqu'ils étaient assez traditionnels, la Coupe de Feu vit que le sang de Perenelle avait été versé, et que ce qui était sien avait été pris~; Baba Yaga avait été rendue parjure malgré elle, et la Coupe la rendit vulnérable. Perenelle tua alors une Baba Yaga sans méfiance, endormie dans le lit de Perenelle~; elle tua la mage noire qui l'avait aimée et était venue en paix à Poudlard~; et ce fut la fin du pacte grâce auquel les mages noirs enseignaient la magie de bataille à Poudlard. Pendant les siècles suivants, on utilisa la Coupe de Feu pour superviser de vains tournois inter-écoles avant de le placer dans une pièce peu usitée de Beauxbâtons, et ce jusqu'à ce que je finisse par la voler." Le professeur Quirrell fit tomber une brindille beige et rose dans le chaudron qui devint blanche à l'instant où elle toucha la surface du liquide. "Mais je digresse. Perenelle prit la Pierre de Baba Yaga avant de prendre l'apparence et le nom de Nicholas Flamel. Elle maintint aussi son identité de Perenelle et se dit la femme de Flamel. Ils sont déjà apparus ensemble en public, mais cela peut avoir été accompli par un grand nombre de méthodes évidentes."

"Et la fabrication de la pierre~?" dit Harry, tout en essayant d'absorber cette histoire. "J'ai vu une recette alchimique, dans un livre…"

"Un autre mensonge. Perenelle voulait laisser croire que 'Nicholas Flamel' avait mérité de vivre pour toujours en accomplissant un rituel accessible à tous. Et elle donnait une fausse piste à suivre, pour éloigner de la quête de la véritable Pierre, celle qui avait conduit Perenelle à l'arracher à Baba Yaga." Le professeur Quirrell avait l'air plutôt aigri. "Tu ne seras pas surpris d'apprendre que j'ai passé des années à maîtriser la fausse recette. Ensuite, tu me demanderas pourquoi je n'ai pas kidnappé, torturé et tué Perenelle lorsque j'ai découvert la vérité."

Cette question n'était pas du tout venue à Harry.

Le professeur Quirrell poursuivit. "La réponse, c'est que Perenelle avait prévu et devancé les ambitions de mages noirs tels que moi. 'Nicholas Flamel' a fait en public le Serment Inviolable de ne jamais être contraint à donner la Pierre - de garder les plus cupides loin de l'immortalité, dit-il, comme si c'était là un service public. J'avais peur que la Pierre soit perdue pour toujours si Perenelle mourrait sans révéler son emplacement, et son Serment rendait la torture inutile. J'espérais aussi m'emparer du savoir de Perenelle, armé de la bonne stratégie d'extraction. Même si elle a débuté avec bien peu de secrets, elle a pris en otage la vie de sorciers bien plus puissants qu'elle, donnant des gouttes de santé contre des secrets~; de jeunesse contre du pouvoir. Elle n'offre jamais aux autres de réelle jeunesse - mais si tu entends parler d'un sorcier qui a survécu, apparemment âgé, jusqu'à deux-cent-cinquante ans, tu peux être certain qu'elle aura joué un rôle. À mon époque, les siècles avait donné un tel avantage à Perenelle qu'elle put envoyer Albus Dumbledore contre le mage noir Grindelwald. Quand je suis apparu sous les traits de Lord Voldemort, Perenelle a de nouveau joué Dumbledore, offrant des gouttes de son savoir accumulé à chaque fois que Lord Voldemort semblait prendre l'avantage. Je pense que j'aurais dû être capable d'exploiter cette situation à mon avantage, mais je n'ai jamais trouvé comment. Je ne l'ai jamais attaquée directement, car je n'étais pas certain de ma grande création~; il n'était pas impossible que je doive un jour venir la supplier pour une tranche de jeunesse." Le professeur Quirrell fit tomber deux campanules dans la potion et ils parurent fusionner en touchant le liquide bouillonnant. "Mais je suis à présent certain de ma création, et j'ai donc décidé que le moment de prendre la Pierre de force était venu."

Harry hésita. "Je voudrais que vous répondiez en Fourchelangue. Est-ce que tout cela est vrai~?"

"\parsel{J'y crois}," dit le professeur Quirrell. "On comble certaines zones d'ombre quand on raconte une histoire~; je n'étais pas présent quand Perenelle a séduit Baba Yaga. \parsel{Cela devrait être en grand partie correct}."

Harry remarqua qu'il était légèrement confus. "Alors je ne comprends pas pourquoi la Pierre est ici, à Poudlard. La meilleure défense ne serait-elle pas de placer la Pierre sous un caillou, au Groenland~?"

"Peut-être qu'elle respectait mes capacités de pisteur," dit le professeur de Défense. Il semblait concentré sur le chaudron, et il trempa une campanule dans une carafe d'un liquide étiqueté par une goutte de pluie.

\emph{Nous nous ressemblons beaucoup, le professeur de Défense et moi, par certains aspects. Si j'imagine ce que je ferais, si j'avais ce problème à résoudre…}

"Est-ce que vous avez fait \emph{croire} à tout le monde que vous pouviez retrouver la Pierre~?" dit Harry. "Pour que Perenelle la mette dans Poudlard, là où Dumbledore pourrait la garder~?"

Le professeur de Défense soupira mais ne releva pas les yeux. "Je suppose qu'il serait futile de tenter de te dissimuler ce stratagème. Oui, après avoir possédé Quirrell et être revenu, j'ai mis en place une stratégie qui m'était venue en contemplant les étoiles. Je me suis d'abord assuré d'être engagé comme professeur de Défense à Poudlard, car il aurait été contre-productif d'agiter la fourmilière avant d'y avoir obtenu un emploi. Une fois fait, je m'arrangeais pour qu'une expédition anti-malédictions de Perenelle découvre une inscription, fausse mais crédible, décrivant comment la Couronne de Serpent permet de trouver la Pierre, où qu'elle soit. Juste après, mais avant que Perenelle ne puisse acheter la Couronne, elle fut volée~; j'ai laissé des indices montrant clairement que le voleur pouvait parler aux serpents. Donc Perenelle a cru que je pouvais obtenir l'emplacement de la Pierre de façon infaillible et qu'il lui fallait un gardien assez puissant pour me vaincre. C'est ainsi que la Pierre s'est retrouvée à Poudlard, dans le domaine de Dumbledore. Comme je l'espérais, naturellement, puisque j'avais déjà obtenu un an d'accès à Poudlard. Je pense que c'est tout ce qui peut t'intéresser, mis à part mes plans futurs."

Harry fronça les sourcils. Le professeur Quirrell n'aurait pas dû lui dire cela. À moins que la stratégie ne dépende d'aucune tromperie future de Perenelle…~? Ou à moins que, en répondant si vite, le professeur de Défense ait espéré amener tout le monde à conclure que c'était un double bluff et que la couronne du Serpent pouvait vraiment trouver la Pierre…

Harry décida de ne pas creuser en Fourchelangue.

Une autre mèche de cheveux clairs, comme blancs mais sans âge, fut lentement versée dans le chaudron, ce qui rappela une fois de plus à Harry qu'ils n'avaient pas tout leur temps. Il envisagea de poursuivre avec ces questions, mais il ne voyait pas comment avancer~; il n'y avait aucune méthode connue de manufacture de la Pierre Philosophale, et aucune façon évidente d'inventer une telle méthode~; ce qui, \emph{objectivement}, était probablement la pire nouvelle de la journée.

Harry prit une profonde inspiration. "Je pose ma troisième question," dit Harry. "Quelle est la vérité derrière cette année~? Tous les complots dont vous êtes l'auteur, tous ceux que vous connaissiez."

"Hmm," dit le professeur Quirrell, en faisant tomber une autre campanule dans la potion, accompagné par une plante en forme de petite croix. "Voyons voir… la révélation la plus incroyable, c'est que le professeur de Défense est en réalité Voldemort."

"Oui, évidemment," dit Harry, très amer envers lui-même.

"Alors où veux-tu que je commence~?"

"Pourquoi avez-vous tué Hermione~?" la question était sortie toute seule.

Les yeux pâles du professeur Quirrell remontèrent vers lui, et le regardèrent intensément. "Cela devrait être évident… Mais je ne peux pas t'en vouloir de ne pas croire à l'évidence. Pour comprendre le but d'un plan complexe, il faut regarder ses conséquences, et se demander qui pourrait les avoir désirées. J'ai tué Mlle Granger pour améliorer ta position face à Lucius Malfoy, puisque mes plans n'admettaient pas qu'il ait tant de pouvoir sur toi. J'avoue avoir été impressionné par la façon dont tu as exploité cette opportunité."

Harry fit l'effort de desserrer les dents. "C'est après avoir échoué à \emph{faire accuser} Hermione d'avoir tenté de tuer Drago et, échoué à \emph{l'envoyer à Azkaban} pour… quoi~? Parce que vous n'aimiez pas l'influence qu'elle avait sur moi~?"

"Ne sois pas ridicule," dit le professeur Quirrell. "Si j'avais uniquement souhaité supprimer Mlle Granger, je n'aurais pas fait entrer les Malfoy en jeu. J'ai observé ton jeu avec Drago Malfoy, et je l'ai trouvé amusant, mais j'ai su qu'il ne continuerai pas bien longtemps avant que Lucius n'en ai vent et n'intervienne. Et alors tes emportements t'auraient coûté cher, car Lucius n'aurait pas pris cette histoire à la légère. Si tu avais seulement pu \emph{perdre} pendant le procès au Magenmagot, \emph{perdre} comme je te l'avais appris, alors moins de deux semaines plus tard, des preuves irréfutables auraient montré que Lucius Malfoy, découvrant l'apparente perfidie de son fils, avait lancé un Imperius sur le professeur Chourave pour qu'elle utilise le sortilège de refroidissement du sang sur M. Malfoy et un sortilège de faux souvenirs sur Mlle Granger. Lucius aurait été éliminé de l'échiquier politique, envoyé en exil, voir à Azkaban~; Drago aurait hérité de la fortune de la maison Malfoy et ton influence sur lui aurait été sans conteste. Mais j'ai dû interrompre ce plan en pleine course. Tu es parvenu à complètement défaire le véritable plan, et ce faisant à sacrifier le double de ta fortune tout en offrant à Lucius Malfoy une opportunité parfaite de prouver l'authenticité de sa préoccupation pour son fils. Je dois dire que tu as un incroyable anti-talent lorsqu'il s'agit de te mêler des choses."

"Et vous avez aussi pensé," dit Harry - même plongé dans ses motifs de pensée obscurs, il avait du mal à garder une voix neutre, "que deux semaines à Azkaban amélioreraient Mlle Granger et qu'elle cesserait d'avoir une mauvaise influence sur moi. Donc vous vous êtes arrangé pour que des articles de journaux réclament son enfermement à Azkaban."

Les lèvres du professeur dessinèrent un léger sourire. "Bien vu, petit. Oui, je la voyais bien devenir ta Bellatrix. Cette conclusion t'aurait aussi fourni un rappel permanent du respect dû à la loi et t'aurait aidé à apprendre comment se comporter face au Ministère."

"Votre plan était d'une complexité imbécile et n'avait aucune chance de fonctionner." Harry savait qu'il aurait dû faire preuve de plus de tact, qu'il faisait preuve de ce que le professeur Quirrell aurait appelé de l'\emph{insouciance}, mais il ne put, alors, se forcer à y accorder la moindre importance.

"C'était moins complexe que le plan de Dumbledore destiné à obtenir une égalité entre les 3 armées lors de la bataille de Noël, et pas beaucoup plus que mon plan destiné à te faire croire que Dumbledore avait fait chanter M. Zabini. Ce que tu ne comprends pas, c'est que ces plans n'avaient pas \emph{besoin} de réussir." Le professeur Quirrell continua d'agiter la potion avec nonchalance, sourire aux lèvres. "Certains plans \emph{doivent} marcher~; leur idée centrale est simplifiée au possible, toutes les précautions sont prises. Il y en a d'autres où l'échec est envisageable, et ceux-ci permettent de se faire plaisir, de tester les limites de sa capacité à gérer des situations complexes. Ce n'était pas comme si un déboire m'aurait tué." Le professeur Quirrell ne souriait plus. "Notre voyage à Azkaban appartenait à la première catégorie, et tes extravagances d'alors m'ont moins amusé."

"Qu'est-ce que vous avez fait à Hermione, \emph{exactement}~?" Une partie de Harry se surprit du ton neutre de sa voix.

"Oubliettes et sortilèges de faux souvenirs. Rien d'autre n'aurait échappé avec certitude au système de sécurité de Poudlard et à l'examen approfondi auquel je savais que son esprit serait soumis." Un éclair de frustration passa sur le visage du professeur Quirrell. "Une partie de cette complexité dont tu parles vient de l'échec de ma première version du plan, que j'ai dû ensuite modifier. Je suis apparu face à Mlle Granger dans les couloirs sous l'apparence du professeur Chourave et je lui ai offert de participer à une conspiration. Cette première tentative de pression morale a échoué. Je lui ai effacé la mémoire et j'ai essayé à nouveau, avec un nouveau discours. Le deuxième appât a échoué. Le \emph{dixième} appât a échoué. J'étais tellement frustré que j'ai commencé à parcourir toute ma panoplie de déguisements, y compris ceux plus appropriés à M. Zabini. Et \emph{rien} n'a marché. L'enfant \emph{refusait} de violer son code puéril."

"\emph{Vous} ne pouvez pas la qualifier de puérile, professeur." La voix de Harry lui semblait étrange. "Son code a \emph{marché}. Il vous a empêché de la tromper. Tout l'intérêt d'avoir des injonctions déontologiques, c'est que les arguments en faveur de leur violation sont souvent beaucoup moins dignes de confiance qu'ils n'en ont l'air. Vous ne pouvez pas critiquer ses règles si elles ont marché exactement comme prévu." Après avoir ressuscité Hermione, Harry lui dirait que Lord Voldemort en personne avait échoué à l'amener à mal agir et qu'il l'avait tuée à cause de ça.

"C'est juste," dit le professeur Quirrell. "Un dicton dit que même une horloge arrêtée donne l'heure deux fois par jour, et je ne pense pas que Mlle Granger s'est vraiment comporté de façon raisonnable. Cela dit, règle 10~: on ne nie pas les mérites de ses opposants après avoir perdu face à eux. Quoi qu'il en soit. Après deux heures d'échecs répétés, je me suis rendu compte que j'étais trop têtu et que je n'avais pas besoin de Mlle Granger pour que le rôle que je lui avais réservé soit joué. J'ai abandonné mon projet initial et j'ai préféré imprégner l'esprit de Mlle Granger de faux souvenirs, ceux d'avoir regardé M. Malfoy conspirer contre elle, dans des circonstances qui l'empêchaient d'en parler à toi ou aux autorités. C'est M. Malfoy qui m'a donné l'opportunité dont j'avais besoin, par pure chance." Le professeur Quirrell laissa tomber une campanule et un bout de parchemin dans le chaudron.

"Pourquoi le système de sécurité a-t-il dit que le professeur de Défense avait tué Hermione~?"

"Je portais le troll des montagnes dans une fausse dent le jour où Dumbledore m'a introduit auprès du système de sécurité de Poudlard comme étant le professeur de Défense." Un léger sourire. "D'autres armes vivantes ne peuvent être métamorphosées~; elles ne survivraient pas sous leur véritable forme pendant les six heures nécessaires à éviter qu'elles puissent être retracées par Retourneur de Temps. Le fait qu'un troll des montagnes ait été utilisé pour cet assassinat révélait clairement que l'assassin avait eu besoin d'un intermédiaire métamorphosable. Allié au rapport du système de sécurité et à la façon dont Dumbledore m'avait introduit à ce système, tu aurais pu découvrir le responsable - en théorie. Mais l'expérience m'a apprise que ce genre de puzzle est bien plus difficile à résoudre quand on en ignore la solution, si bien que le risque m'a paru minime. Ah, ce qui me rappelle que j'ai une question pour toi." Le professeur de Défense regarda Harry droit dans les yeux. "Qu'est-ce qui m'a trahi, à la fin, dans le couloir devant ces salles~?"

Harry écarta ses autres émotions afin d'estimer les coûts et avantages d'une réponse honnête, parvint à la conclusion que le professeur de Défense lui fournissait bien plus d'information qu'il n'en recevait (\emph{pourquoi~?}) et qu'il valait mieux ne pas sembler réticent à partager. "L'élément principal", dit Harry, "c'est qu'il était trop peu probable que tout le monde arrive dans le couloir de Dumbledore en même temps. J'ai essayé de suivre l'hypothèse selon laquelle toutes les arrivées, la vôtre comprise, avaient été coordonnées."

"Mais j'ai dit que je suivais Rogue," dit le professeur de Défense. "Ça n'était pas plausible~?"

"Si, mais…" dit Harry. "Hmm. Les lois qui gouvernent la nature d'une bonne explication ne mentionnent pas les excuses plausibles entendues après-coup. Elles parlent de probabilités assignées à l'avance. C'est pour ça que la science dit de faire des prédictions plutôt que de faire confiance aux explications que les gens inventent après-coup. Et je n'aurais pas prévu que vous suivriez Rogue et arriveriez comme ça. Même si j'avais su à l'avance que vous étiez capable de mettre un traceur sur la baguette de Rogue, je ne me serais \emph{pas attendu} à ce que vous le suiviez. Puisque votre explication ne ressemblait pas à un résultat que j'aurais prédit, elle est restée peu probable. J'ai commencé à me demander si celui qui contrôlait Chourave s'était arrangé pour vous faire venir ici aussi. Et là je me suis rendu compte que le message ne venait en fait pas de mon moi futur, ce qui a rendu tout le reste clair."

"Ah," dit le professeur de Défense, et il soupira. "Eh bien, je pense que c'est pour le mieux. Tu as compris, mais trop tard, et ton ignorance aurait amené sa part d'avantages et d'inconvénients."

"Mais qu'est-ce que pouviez bien essayer de faire~? La seule raison pour laquelle je me suis donné tant de mal pour comprendre, c'est que la situation était extrêmement étrange."

"Cela aurait dû te diriger vers Dumbledore, pas vers moi," dit le professeur Quirrell, et il fronça les sourcils. "Le fait est que Mlle Greengrass n'aurait dû arriver dans ce couloir que plusieurs heures plus tard… mais je suppose que, puisque je me suis arrangé pour que M. Malfoy lui donne l'indice qui lui était destiné, il n'est pas étonnant qu'ils aient formé un groupe. Si M. Nott était arrivé seul en apparence, les choses auraient été moins grotesques. Mais je me considère comme un spécialiste du contrôle de champs de bataille magiques, et j'ai pu m'assurer que le combat se déroule comme je le souhaitais. Cela a néanmoins dû sembler un peu ridicule." Le professeur de Défense laissa tomber un quartier de pêche et une campanule dans le chaudron. "Mais nous parlerons du Miroir quand nous serons face à lui. As-tu d'autres questions quant au décès regrettable - et espérons-le, temporaire - de Mlle Granger~?"

"Oui," dit Harry d'un ton neutre. "Qu'avez-vous fait aux jumeaux Weasley~? Dumbledore pensait… enfin, l'école a vu le directeur aller voir les jumeaux après l'arrestation de Hermione. Dumbledore pensait que vous, ou plutôt que Voldemort se demandait pourquoi, et que vous aviez été voir les jumeaux, aviez trouvé leur carte, et leur aviez ensuite effacé la mémoire."

"Dumbledore avait tout à fait raison," dit le professeur Quirrell, en secouant la tête, comme saisit d'incrédulité. "Laisser la carte de Poudlard dans les mains de ces deux idiots était d'une profonde sottise. J'ai eu un choc déplaisant en récupérant la carte~: elle indiquait correctement mon nom et le tien~! Ces idiots de Weasley avaient cru à une défaillance, surtout depuis que tu avais obtenu ta Cape et ton Retourneur de Temps. Si Dumbledore avait gardé la carte… si les Weasley en avaient jamais parlé à Dumbledore… mais ils n'en firent rien, heureusement."

\emph{Indiquait correctement mon nom et le tien…}

"Je voudrais voir ça," dit Harry.

Sans lever les yeux de son chaudron, le professeur Quirrell sortit un parchemin de ses robes et lui siffla~: "\parsel{Montre ce qui t'entoure}," avant de le jeter à Harry. Il vola droit vers Harry, accompagné d'un courant de sensation funeste, puis rebondit doucement à ses pieds.

Harry ramassa le parchemin et le déroula.

Il parut d'abord blanc. Puis, comme si un stylo invisible l'avait parcouru, les contours des murs et des portes apparurent, tous apparemment manuscrits. Les traces révélaient une série de salles, la plupart marquées vides~; la dernière avait une trace confuse au centre, comme si la carte tentait d'indiquer sa propre perplexité~; et l'avant-dernière salle portait deux noms, écrits aux emplacements correspondant à ceux de Harry et du professeur Quirrell.

\emph{Tom E. Jedusor.}

\emph{Tom E. Jedusor.}

Harry regarda le parchemin et un désagréable frisson le parcourut. C'était une chose d'entendre Lord Voldemort prétendre qu'on s'appelait Tom Jedusor~; c'en était une autre de voir la magie de Poudlard le soutenir. "\parsel{Avez-vous modifié cette carte pour obtenir ce réssultat ou avez-vous été ssurpris en le découvrant~?}"

"\parsel{Ssurpris}," répondit le professeur Quirrell, avec un ton sardonique. "\parsel{Pas un tour}."

Harry enroula la carte et la relança au professeur Quirrell. Une force l'arrêta en plein vol et ramena la carte dans les robes du professeur Quirrell.

Le professeur de Défense parla~: "Je voudrais aussi ajouter que Rogue amenait Mlle Granger et ses laquais vers les brutes et intervenait parfois pour les protéger."

"Je le savais."

"Intéressant," dit le professeur Quirrell. "Dumbledore l'a-t-il aussi appris~? Réponds en Fourchelangue."

"\parsel{Pas que je le ssache,}" siffla Harry.

"Fascinant," dit le professeur Quirrell. "Ceci pourrait t'intéresser, alors~: \parsel{Le faisseur de potions devait œuvrer en ssecret parce que sson plan ss'oppossait au plan du maître de l'école.}"

Harry y songea, pendant que le professeur Quirrell soufflait sur la potion, comme pour la refroidir, alors que le feu brûlait toujours sous le chaudron~; puis il ajouta une pincée de terre, une goutte d'eau, et une campanule. "Expliquez-moi," dit Harry.

"T'es-tu jamais demandé pourquoi Dumbledore avait placé Severus Rogue à la tête de Serpentard~? Dire que c'était une couverture pour son rôle d'espion de Dumbledore n'explique rien. Rogue aurait pu simplement être maître des potions. Il aurait pu être garde-chasse, si le but était de le garder à Poudlard~! Pourquoi \emph{directeur de Serpentard~?} Il t'est sûrement venu à l'esprit que l'effet sur les Serpentard ne pouvait pas être bon, du point de vue moral officiel qu'adopte Dumbledore."

Non, l'idée n'était pas venue à Harry sous \emph{cette} formulation… "Je me suis posé une question similaire. Je n'ai pas formulé le dilemme précisément ainsi."

"Et maintenant que c'est fait, la réponse est-elle évidente~?"

"Non," dit Harry.

"Décevant. Tu n'as pas absorbé suffisamment de cynisme moral, tu n'as pas saisi la \emph{flexibilité} de ce que les moralistes appellent moralité. Pour découvrir l'existence d'un complot, observe les conséquences et demande-toi si quelqu'un aurait pu les manufacturer. Dumbledore sabotait délibérément Serpentard - ne me regarde pas comme ça, petit, \parsel{je ssuis ssincère}. Pendant la dernière guerre des sorciers, nombre de mes laquais me sont venus de Serpentard, et d'autres Serpentard du Magenmagot m'ont soutenu. Maintenant, regarde les choses du point de vue de Dumbledore, et souviens-toi que Serpentard n'est pas sa première langue. Imagine Dumbledore devenir de plus en plus attristé par cette maison qui semble être la source de tant de mauvaises actions. Et là, surprise, Dumbledore place Rogue à la tête de Serpentard. Rogue~! Severus Rogue~! Un homme qui n'enseignera à sa Maison ni la ruse ni l'ambition, un homme qui impose une discipline molle et rend ses enfants faibles~! Un homme qui s'en prend aux élèves d'autres maisons, qui ruine la réputation de Serpentard auprès d'eux~! Un homme dont le nom de famille est inconnu, et certainement pas noble, qui se promène en haillons~! Penses-tu que Dumbledore ignore les conséquences de ce choix~? Alors que c'est lui qui l'a fait, et avait toutes les raisons de le faire~? Je pense que Dumbledore s'est dit que plus de vies seraient sauvées pendant la prochaine guerre des sorciers si les futurs Mangemorts de Voldemort étaient affaiblis." Le professeur Quirrell laissa tomber un morceau de glace dans le chaudron qui fondit lentement en touchant l'écume de la surface. "En continuant ce processus, plus aucun enfant ne voudrait aller à Serpentard. La Maison serait mise au rebut, et les enfants sur lesquels le Choixpeau crierait ce nom seraient frappés d'ignominie avant d'être triés entre les trois Maisons restantes. À partir de ce jour, Poudlard aurait trois Maisons honorables, courageuses, studieuse et entreprenantes, sans Maison des Sales Gamins~; comme si les fondateurs de Poudlard avaient été assez sage dès le début pour refuser d'inclure Salazar Serpentard dans leur groupe. Je pense que c'était le but final de Dumbledore~; un sacrifice à court terme, pour le plus grand bien." Le professeur Quirrell eut un sourire sardonique. "Et Lucius l'a laissé faire sans protester, ni même, je pense, \emph{remarquer} que quelqu'un chose clochait. J'ai bien peur qu'en mon absence, mes anciens serviteurs n'aient été dépassés par l'intelligence de Dumbledore."

Harry avait du mal à juger ces paroles, mais il décida après quelques instants de réflexions que ce n'était pas le moment de s'y attarder. L'important, c'était que Lord Voldemort y croyait. Harry évaluerait l'accusation plus tard.

Le professeur Quirrell avait aussi mentionné ses \emph{serviteurs}, ce qui rappela à Harry une question qu'il était… probablement obligé de poser. La mauvaise nouvelle était à prévoir. En d'autres circonstances, elle aurait été atroce. Mais aujourd'hui, elle était noyée dans le reste. "Bellatrix Black", dit-il. "Que lui est-il arrivé~?"

"Elle était brisée avant même que je ne la rencontre," dit le professeur Quirrell. Il ramassa ce qui ressemblait à un bracelet en plastique blanc-gris et le tint au-dessus du chaudron. Dans la vapeur, le bracelet devint noir. "J'ai fait l'erreur d'utiliser ma Légilimancie sur elle. Mais comme ce rapide coup d'œil m'a montré à quel point il serait simple de la rendre amoureuse de moi, je l'ai fait. Par la suite, elle est restée ma servante la plus loyale, la seule à qui je pouvais presque faire confiance. Je n'avais aucune intention de lui donner ce qu'elle attendait de moi, alors je l'ai donnée au bon plaisir des frères Lestrange. À leur façon, ils en furent heureux."

"J'en doute," dit Harry, quasiment sans réfléchir. "Si c'était le cas, elle ne se serait pas souvenu d'eux quand nous l'avons récupérée à Azkaban."

Le professeur Quirrell haussa les épaules. "Peut-être bien."

"Et qu'est-ce qu'on faisait là-bas~?"

"Je voulais savoir où Bellatrix avait mis ma baguette. J'avais révélé mon immortalité aux Mangemorts dans l'espoir - qui s'est avéré vain - qu'ils resteraient ensemble au moins quelques \emph{jours} si jamais je passais pour mort. Bellatrix avait reçu l'ordre de récupérer ma baguette puis de l'apporter dans un cimetière précis où mon esprit lui apparaîtrait."

Harry déglutit. L'image lui vint~: Bellatrix Black qui attendait, attendait, attendait dans le cimetière, de plus en plus désespérée… pas étonnant qu'elle n'ait pas eu toute sa tête lorsqu'elle avait attaqué la maison Londubat. "Qu'avez-vous fait de Bellatrix après l'avoir libérée~?"

"\parsel{Envoyée dans endroit paissible pour retrouver forces}," dit le professeur Quirrell. Un sourire froid. "J'avais encore l'intention de faire usage de sa personne, ou plutôt d'une partie de sa personne, mais je ne répondrai à aucune question quant à ces futurs plans."

Harry inspira profondément et tenta de garder le contrôle de lui-même. "Y a-t-il eu d'autres plans ou complots pendant l'année~?"

"Oh, un certain nombre, mais j'en vois peu d'autre te concernant. La véritable raison pour laquelle j'ai demandé à ce qu'on enseigne le Patronus à des élèves de première année, c'était pour amener un Détraqueur devant toi. Je me suis ensuite arrangé pour que ta baguette tombe près de lui afin qu'il continue de te vider à travers elle. \parsel{Mes intentions étaient bonnes, j'espérais seulement que tu retrouverais une partie de tes souvenirs}. C'est aussi pour cela que j'ai fait en sorte que certaines sorcières te fassent tomber pendant ton escapade sur le toit~: je voulais donner l'impression de t'avoir sauvé juste au cas où l'on me soupçonnerait pendant l'incident du Détraqueur, qui était imminent. \parsel{Là ausssi, mes intentions étaient bonnes.} J'ai organisé quelques attaques à l'encontre du groupe de Mlle Granger juste pour lui donner quelques victoires. Moi non plus, je n'aime pas les brutes. \parsel{Je crois que ce sont tous les plans et complots te concernant qui ont eu lieu cette année, à moins que j'en ai oublié.}"

\emph{Leçon de vie~:}, dit sa partie Poufsouffle, \emph{il faut résister à la tentation de s'immiscer dans la vie des gens. Comme celle de Padma Patil, par exemple. Enfin, si on ne veut pas finir comme lui.}

Une pincée de poussière rouge sombre fut doucement versée dans le chaudron et Harry posa sa quatrième et dernière question, celle qui, bien qu'importante, avait semblé la moins vitale.

"Quel était votre but pendant la guerre des sorciers~?" demanda-t-il. "Enfin, qu'est-ce…" sa voix vacilla. "\emph{Pourquoi} faire \emph{tout ça}~?" Son cerveau répétait inlassablement~: \emph{Pourquoi, pourquoi, pourquoi Lord Voldemort…}

Le professeur Quirrell leva un sourcil. "Ils t'ont parlé de David Monroe~?"

"Oui, vous étiez à la fois David Monroe et Lord Voldemort pendant la guerre des sorciers, j'ai compris ça. Vous l'avez tué, vous vous êtes fait passer pour lui et vous avez anéanti sa famille pour que personne ne remarque la différence…"

"En effet."

"Vous comptiez contrôler le camp victorieux, quel qu'il soit. Mais pourquoi avoir \emph{Voldemort} dans un des camps~? Enfin… est-ce que ça n'aurait pas été plus simple de trouver du soutien en étant un peu… un peu moins Voldemort~?"

En écrasant des ailes de papillon blanc mêlées à une autre campanule, le maillet du professeur Quirrell frappa plus fort que d'habitude. "Le \emph{plan}," dit le professeur Quirrell d'une voix dure, "était de faire \emph{perdre} Lord Voldemort face à David Monroe. La faille dans ce plan était l'ignominie absolue de…" - il s'interrompit. "Non, je raconte dans le désordre. Écoute, petit, après avoir accompli ma grande œuvre et atteint le firmament de ma puissance, j'ai songé que le temps était venu de m'emparer du pouvoir politique. Cela aurait certainement des inconvénients et me forcerai à passer du temps sur des activités peu amusantes. Mais je savais que les Moldus finiraient par détruire le monde, par faire la guerre des sorciers, ou par faire les deux~; et, si je ne voulais pas finir éternellement seul dans un monde mort ou morne, il fallait que quelque chose soit fait. Ayant atteint l'immortalité, j'avais besoin d'une ambition nouvelle, capable de m'occuper pendant des décennies, et empêcher les Moldus de tout gâcher m'est apparu comme un but suffisamment ambitieux et difficile. Le fait d'avoir jamais été le seul à réellement agir en ce sens n'a jamais cessé de m'amuser. Même si j'imagine qu'il est logique que ces éphémères insectes se soucient peu de la fin de leur monde~: à quoi bon, puisqu'ils vont de toute façon mourir et que rien ne les force donc à se fatiguer~? Mais je digresse. J'ai vu l'ascension au pouvoir de Dumbledore après sa victoire face à Grindelwald, alors j'ai voulu faire de même. Je m'étais vengé de David Monroe il y a bien longtemps - il m'avait agacé pendant que j'étais à Serpentard - et l'idée m'est venue de lui prendre son identité, de balayer sa famille et de me faire l'héritier de sa Maison. Puis j'ai inventé un ennemi à combattre, le plus terrifiant des Seigneurs des Ténèbres, à l'intelligence défiant toute imagination, bien plus dangereux que Grindelwald, car parfait là où Grindelwald était faible et autodestructeur. Un Seigneur des Ténèbres qui ferait tout pour démembrer les alliances se formant face à lui, un Seigneur des Ténèbres qui ferait naître une profonde loyauté chez ses partisans grâce à son talent oratoire. Le pire des Seigneur des Ténèbres à avoir jamais menacé l'Angleterre magique, ou le monde~: voilà qui David Monroe vaincrait."

Le maillet du professeur Quirrell frappa une campanule puis, de deux autres coups, une autre fleur pâle. "Mais, tout en ayant parfois joué le rôle d'un mage noir au gré de mes voyages, je n'avais jamais adopté l'identité d'un véritable Seigneur des Ténèbres, avec des laquais, et un programme politique. Je n'avais pas d'expérience, et j'avais à l'esprit l'histoire de Dark Evangel et le désastre de sa première apparition publique. Elle a dit ensuite avoir voulu s'appeler la Catastrophe Ambulante et l'Apôtre des Ténèbres, mais dans l'excitation elle a annoncé être l'Apostrophe des Ténèbres. Après ça, elle a dû massacrer deux villages pour qu'on commence à la prendre au sérieux."

"Alors vous avez décidé de commencer par une petite expérience," dit Harry. Il se sentit un peu nauséeux, car il avait parfaitement \emph{compris}, il s'était vu comme dans un miroir~: l'étape suivante qu'il aurait lui-même franchie s'il n'avait pas eu la moindre trace d'éthique, si son cœur avait été assez vide. "Vous avez créé une identité jetable pour apprendre, pour vous familiariser avec les rouages et ne pas être appesanti par vos erreurs."

"Tout à fait. Avant de devenir le terrible Seigneur des Ténèbres qui ferait face à David Monroe, j'ai créé une identité d'entraînement, celle d'un Seigneur des Ténèbres aux yeux rouges et luisants, inutilement cruel envers ses serviteurs, engagé dans un programme politique sans profondeur, mélange d'ambition personnelle et de purisme de sang tel que défendu par les ivrognes de l'allée des Embrumes. Mes premiers serviteurs furent engagés dans une taverne, reçurent des capes et des masques en forme de crâne, et je leur dis de se présenter sous le nom de Mangemorts."

Harry comprenait de mieux en mieux et ne s'en sentait que plus malade. "Et vous vous êtes appelé Voldemort."

"Tout à fait, Général Chaos." Face au chaudron, le professeur Quirrell souriait. "J'aurais voulu que ce soit un anagramme de mon nom, mais cela n'aurait marché que si, par un commode hasard, mon deuxième prénom avait été Elvis. Notre véritable deuxième prénom est Erfin, si ça t'intéresse. Mais je digresse. Je pensais que la carrière de Lord Voldemort ne durerait que quelques mois, une année tout au plus, avant que les Aurors n'abattent ses serviteurs et que le Seigneur des Ténèbres jetable ne disparaisse. Comme tu l'as compris, j'avais largement surestimé mes adversaires. Et je ne parvenais pas à aller jusqu'à torturer mes serviteurs quand ils me faisaient part de mauvaises nouvelles, même si c'était ce que les Seigneurs des Ténèbres des livres de contes faisaient. Je ne parvenais pas à soutenir le purisme du sang avec les arguments incohérents d'un ivrogne de l'allée des Embrumes. Je n'essayais pas de jouer au plus fin en envoyant mes serviteurs accomplir des missions, mais je ne leur donnais pas non plus des ordres absurdes…" Le professeur Quirrell eut un sourire qui, en d'autres circonstances, aurait pu être charmeur. "Un mois plus tard, Bellatrix Black se prostrait devant moi, et trois mois après, Lucius Malfoy négociait autour de verres d'un vieux Whiskey Pur Feu. J'ai soupiré, perdu tout espoir quant à l'humanité et aux sorciers, et j'ai placé David Monroe face à ce terrible Voldemort."

"Et alors qu'est-ce qui s'est…"

Un grognement déforma le visage du professeur Quirrell. "L'ineptitude absolue de toutes les institutions créées par la civilisation d'Angleterre Magique, voilà ce qui s'est passé~! Tu ne peux pas comprendre, petit~! Je ne peux pas le comprendre~! Il faut le voir, et même alors, on n'arrive pas à y croire~! Tu auras peut-être remarqué que, parmi ceux de tes camarades qui parlent du métier de leurs parents, les trois quarts font mention d'un travail dans un département ou un autre du ministère. Tu te demanderas comment un pays arrive à employer les trois quarts de ses employés dans une bureaucratie. La réponse est que s'ils ne s'empêchaient pas les uns les autres de faire leur travail, ils n'auraient plus rien à faire~! Les Aurors sont des guerriers compétents qui se sont battus contre des mages noirs, et seuls les meilleurs sont restés pour passer le flambeau aux nouvelles recrues, mais le désordre dans leurs rangs était total. Le ministère était tellement occupé à brasser de la paperasse que le pays ne disposait d'\emph{aucune} défense face à Voldemort, hormis moi-même, Dumbledore, et une poignée de francs-tireurs sans entraînement. Mondingus Fletcher, un fainéant apathique, incompétent et pleutre, était un élément clé de l'Ordre du Phénix - parce que, comme il était au chômage, il n'avait pas à gérer deux emplois à la fois~! J'ai essayé de mollir les attaques de Voldemort, pour voir s'il \emph{pouvait} perdre~; le ministère a immédiatement diminué l'effectif d'Aurors missionnés contre moi~! J'avais lu le petit livre rouge de Mao, j'avais entraîné les Mangemorts aux tactiques de guérilla - pour rien~! pour rien~! J'attaquais toute l'Angleterre Magique et à chaque bataille mes forces étaient \emph{plus nombreuses} que celles de mes opposants~! Désespéré, j'ai ordonné à mes Mangemorts d'assassiner de façon systématique tous les incompétents en poste au département de justice magique. Un gratte-papier après l'autre accepta de monter en grade malgré la fin tragique de son prédécesseur, en se frottant joyeusement les mains à l'idée d'être promu. Ils pensaient tous pouvoir faire affaire avec Voldemort en secret. J'ai mis \emph{sept mois} à tous les éliminer, et pas un seul Mangemort ne m'a demandé pourquoi nous faisions cela. Et alors, même avec Bartemius Crouch au poste de directeur et Amelia Bones en chef des Aurors, ça n'a pas suffit. J'aurais mieux combattu \emph{seul}. Les chaînes morales que s'imposait Dumbledore rendaient son soutien presque inutile, et le respect de la loi de Crouch avait le même effet." Le professeur Quirrell attisa le feu sous la potion.

"Et vous avez fini," dit Harry, toujours nauséeux, "par comprendre que vous vous amusiez plus sous les traits de Voldemort."

"C'est le rôle le moins agaçant que j'ai jamais joué. Si Lord Voldemort veut qu'une chose soit faite, on lui \emph{obéit} et on ne \emph{discute pas}. Je n'avais pas à réprimer mes envies de torturer les gens lorsqu'ils se comportaient comme des imbéciles, cela faisait partie du personnage. Si quelqu'un rendait le jeu moins amusant, je n'avais qu'à dire \emph{Avadakedavra}, que ce soit stratégiquement censé ou pas, et ils ne m'embêtaient plus jamais." Le professeur Quirrell découpa nonchalament un ver en petits morceaux. "Mais je n'ai vraiment compris que le jour où David Monroe a essayé d'obtenir un permis de séjour pour un professeur de combat asiatique, et qu'un employé du ministère a refusé la demande avec un sourire suffisant. J'ai demandé à l'employé s'il comprenait que ce séjour était destinée à \emph{sauver sa vie}, et l'employé a sourit encore plus. Pris de furie, j'ai fait fit de toute prudence et j'ai utilisé la Légilimancie, j'ai plongé mes doigts dans la fosse sceptique de sa stupidité et j'ai \emph{arraché} la vérité de son esprit. Je ne comprenais pas, et je \emph{voulais comprendre}. Grâce à ma maîtrise de la Légilimancie, j'ai forcé son petit cerveau d'employé à vivre d'autres possibilités, et j'ai vu ce qu'il aurait pensé si Lucius Malfoy, Lord Voldemort ou Dumbledore, avaient été devant lui." Les mains du professeur Quirrell s'étaient ralenties, il pelait délicatement de petits bouts de cire de bougie. "Ce que j'ai enfin compris est complexe, petit, et c'est pour cela que je ne l'ai pas compris plus tôt. J'essaierai néanmoins de te le décrire. Aujourd'hui, je sais que Dumbledore n'est pas le roi du monde, bien qu'il soit le Manitou Suprême de la confédération internationale. Les gens disent ouvertement du mal de lui, ils le critiquent avec suffisance, face à lui, comme ils n'oseraient jamais le faire face à Lucius Malfoy. \emph{Tu} as manqué de respect à Dumbledore, petit, mais sais-tu pourquoi~?"

"Je… je n'en suis pas sûr," dit Harry. L'hypothèse évidente était~: j'ai des restes de motifs neuronaux de Tom Jedusor.

"Les loups, les chiens, même les poulets se battent pour être dominants. Ce que j'ai enfin compris, dans l'esprit de cet employé, c'est que pour lui, Lucius Malfoy était un dominant, tout comme Lord Voldemort, mais que David Monroe et Albus Dumbledore ne l'étaient pas. En choisissant le camp du bien, en déclarant son soutien à la lumière, nous nous étions rendus \emph{inoffensifs}. En Angleterre, Lucius Malfoy est un dominant car il est créancier, car il peut envoyer des employés faire fermer de petits commerces, car si jamais on s'oppose à lui, il peut crucifier qui bon lui chante dans la \emph{Gazette du Sorcier}. Mais le sorcier le plus puissant du monde n'est pas un dominant, parce que tout le monde sait qu'il est", les lèvres du professeur Quirrell se relevèrent, "\emph{un héros de contes}, toujours effacé, trop humble pour se venger. Dis-moi, petit, as-tu déjà vu une pièce où le héros, avant d'accepter de sauver son pays, exige autant d'or que ce qu'un avocat à la cour pourrait recevoir~?"

"À vrai dire oui, il y a \emph{beaucoup} de héros Moldus de ce genre, Han Solo pour commencer…"

"Eh bien dans les histoires du monde magique, ce n'est pas le cas. Ce ne sont que des héros humbles, comme Dumbledore. C'est le fantasme du puissant \emph{esclave} qui ne s'élèvera jamais vraiment au-dessus de soi, qui n'exigera jamais le respect, ne demandera même jamais de paiement. Est-ce que tu comprends, maintenant~?"

"Je… je crois que oui," dit Harry. Frodon et Samsagace, du \emph{Seigneur des Anneaux}, semblaient correspondre à l'archétype du héros inoffensif. "Vous dites que les gens voient Dumbledore comme ça~? Je ne pense pas que les élèves de Poudlard le prennent pour un hobbit."

"À Poudlard, Dumbledore punit certaines transgressions de sa volonté, et il est donc légèrement craint - mais les élèves se sentent toujours libres de le moquer à voix haute. Hors du château, on le méprise~; on le dit fou, et il accepte bêtement ce rôle. Jouer le sauveur des contes revient à être vu comme un esclave dont les services sont dûs et qu'il est doux de critiquer~; car c'est le privilège du maître que de rester assis à rectifier l'esclave au travail. Ce n'est que dans les histoires des grecs anciens, où les illusions des hommes étaient plus simples, que le héros a encore sa grandeur. Hector, Énée~: ces héros avaient encore droit de vengeance sur ceux qui les avaient insultés. Ils pouvaient exiger or et joyaux en contrepartie de leur aide sans susciter l'indignation. Et si Lord Voldemort avait conquis l'Angleterre, il aurait pu lui faire l'honneur d'être beau vainqueur. Personne n'aurait tenu sa bonté pour acquise, personne n'aurait critiqué son travail. Victorieux, il aurait eu un \emph{vrai} respect. Ce jour-là, au ministère, j'ai compris que dans ma jalousie de Dumbledore, je m'étais fait autant d'illusions que lui. J'ai compris que ce n'était pas sa place qu'il fallait convoiter. Tu devrais savoir que je dis vrai, petit, car tu t'es senti plus libre de dire du mal de Dumbledore que tu n'en as jamais dit de moi. Même en pensée, je parie, car ces instincts sont profonds. Tu savais qu'il pourrait t'en coûter de te moquer du puissant et vengeur professeur Quirrell, mais pas de manquer de respect au faible, à l'inoffensif Dumbledore.

"Merci," dit Harry, empli de douleur, "merci pour cette leçon importante, professeur Quirrell, je me rends compte que ce que vous dites sur ma façon de penser est vrai." Même si les souvenirs de Tom Jedusor avaient probablement un rapport avec ses emportements soudains contre Dumbledore~; il n'avait pas été ainsi envers le professeur McGonagall… quoi qu'il faille admettre qu'elle pouvait prendre des points de Maison, et qu'elle n'avait pas l'air tolérant de Dumbledore… non, cela demeurait vrai~: Harry aurait été plus respectueux, même en pensé, s'il avait semblé \emph{imprudent} de manquer de respect envers Dumbledore.

Voilà pour David Monroe, voilà pour Lord Voldemort.

Cela ne répondait pas à la question la plus intriguante. Peut-être n'était-il pas sage de la poser. Si par un curieux hasard Lord Voldemort n'y avait \emph{pas} pensé et que le professeur Quirrell n'y avait pas non plus pensé pendant ses neuf années de réflexion, peut-être valait-il mieux ne pas le dire… ou peut-être que si~: l'agonie de la guerre des sorciers n'avait pas fait de bien à l'Angleterre.

Il se décida, et prit la parole~: "Ce qui m'étonne, c'est que la guerre ait duré si longtemps," se hasarda-t-il, "enfin, peut-être que je sous-estime les difficultés auxquelles Lord Voldemort faisait face…"

"Tu veux savoir pourquoi je n'ai pas lancé un Imperius sur les plus puissants sorciers capables de lancer un Imperius sur d'autres, tué ceux des puissants qui pouvaient résister à mon Imperius, et fait main basse sur le ministère en 3 jours au plus~?"

Harry hocha la tête.

Le professeur Quirrell prit un air contemplatif. Sa main versait de l'herbe hachée dans le chaudron. Si Harry se souvenait bien, nous en étions aux quatre cinquièmes de la recette.

"Je me suis posé la même question," dit-il enfin, "quand j'ai entendu Rogue prononcer la prophétie de Trelawney, et que j'ai contemplé le passé autant que l'avenir. Si tu avais demandé à mon moi passé pourquoi il n'avait pas utilisé l'Imperius, il aurait parlé du besoin d'être \emph{vu}, de diriger ouvertement la bureaucratie du ministère avant de se tourner vers d'autres pays. Il aurait ajouté qu'une victoire rapide et silencieuse aurait été elle-même source de complications futures. Il aurait mentionné l'obstacle de Dumbledore et de ses incroyables prouesses défensives. Et il aurait eu des excuses similaires pour presque toutes les méthodes de victoire rapide envisagées. Il y avait à chaque fois une raison de repousser la phase finale, à chaque fois une dernière chose à faire. Puis j'ai entendu la prophétie et j'ai \emph{su} que le moment était venu, car le Temps lui-même m'avait remarqué. Que le temps de l'hésitation avait pris fin. Et j'ai contemplé mes actes, et je me suis rendu compte que je jouais à ce jeu depuis des années. Je pense…" Un peu d'herbe tombait encore parfois de sa main, mais le professeur Quirrell ne semblait pas y prêter la moindre attention. "J'ai pensé, sous les étoiles, contemplant mon passé, que je m'étais habitué au jeu contre Dumbledore. Il était intelligent, il s'appliquait à inventer des ruses, il n'attendait pas mes attaques~: il me surprenait. Il jouait des coups étranges aux conséquences aussi imprévisibles que fascinantes. Avec le recul, il y avait de nombreux moyens évidents de détruire Dumbledore, mais je pense qu'une partie de moi préférait continuer à jouer aux échecs plutôt que de reprendre le solitaire. C'est en ayant l'idée d'avoir un autre Tom Jedusor pour adversaire, d'avoir un ennemi encore plus digne que Dumbledore, que j'ai pour la première accepté d'envisager la fin de ma guerre. Oui, avec le recul, ça semble stupide, mais nos émotions sont parfois plus bêtes que notre raison ne veut l'admettre. Je n'aurais jamais délibérément suivi une telle politique. Ça aurait été enfreindre les règles neuf, seize, vingt et vingt-deux. C'est beaucoup trop, même si on s'amuse beaucoup. Mais décider encore et encore qu'on a une dernière chose à faire, un dernier avantage à saisir, une pièce qu'il \emph{faut} mettre en position, avant d'abandonner une époque amusante et de passer au fatiguant contrôle de l'Angleterre… Eh bien même moi, je ne suis pas à l'abri d'une erreur de ce genre si je ne me rends pas compte que je suis en train de la commettre."

Et c'est alors que Harry comprit ce qui allait se passer ensuite, après que la Pierre Philosophale ait été récupérée.

Après ça, le professeur Quirrell allait le tuer.

Le professeur Quirrell ne voulait pas le tuer. Il était possible que Harry soit la seule personne au monde contre laquelle le professeur Quirrell serait \emph{incapable} d'utiliser le sortilège de la mort. Mais le professeur Quirrell, quelles que soient ses raisons, était persuadé de devoir le faire.

C'est pour cela qu'il avait décidé qu'il fallait préparer la \emph{potion de splendeur}. Pour cela qu'il avait été si simple de le pousser à répondre aux questions, à enfin parler de sa vie à quelqu'un qui pourrait comprendre. Tout comme Lord Voldemort avait fait durer la guerre des sorciers pour continuer de jouer contre Dumbledore.

Il ne se souvenait pas bien de ce que le professeur Quirrell avait dit sur le fait de ne pas tuer Harry. Ce n'avait pas été parfaitement clair, comme "Je ne n'ai absolument aucune intention de tuer Harry de quelque façon que ce soit à moins qu'il se comporte vraiment comme le dernier des imbéciles." Harry lui-même n'avait pas voulu trop promettre et insister sur une absence d'ambiguïté, car il pensait déjà au besoin de neutraliser Lord Voldemort et avait craint qu'une formulation plus précise ne révèle ce fait. Donc, quoi qu'ils aient dit, il y avait probablement eu des failles.

Il n'y eut pas de choc particulier, seulement un sentiment d'urgence accrue. Une partie de Harry avait déjà été au courant et avait simplement attendu une occasion de s'insérer dans la délibération mentale. Le professeur Quirrell en avait révélé bien plus qu'il ne dirait jamais à quelqu'un dont l'espérance de vie dépassait quelques heures. L'isolation, la solitude profonde de la vie qu'il avait décrite pouvait expliquer qu'il ait été prêt à violer ses Règles et à parler à Harry, \emph{sachant} qu'il allait bientôt mourir et que le monde n'était pas vraiment celui des pièces de théâtres, où le méchant qui révèle ses plans échoue toujours à achever le héros ensuite. Mais la mort de Harry faisait certainement partie de ces plans.

Il déglutit, contrôla sa respiration. Le professeur Quirrell avait juste ajouté une touffe de crinière à la \emph{potion de splendeur}, et c'était presque la phase finale, si sa mémoire était bonne. Il ne restait pas non plus beaucoup de campanules.

Il était probablement temps de faire moins preuve de prudence et de jouer cette conversation de façon plus risquée.

"Si je fais remarquer une erreur de Lord Voldemort," dit Harry, "serais-je puni~?"

Le professeur Quirrell leva un sourcil. "Pas si l'erreur est réelle. Je te suggère d'éviter de me faire la morale. Mais je ne tuerai pas le porteur de mauvaise nouvelle, ni le subordonné qui tente honnêtement de signaler un problème. Même sous les traits de Lord Voldemort, je n'ai jamais pu m'abaisser à de telles sottises. Bien sûr, certains idiots ont prit cette politique comme un signe de faiblesse et ont essayé de se mettre en avant en me rabaissant en public, me croyant obligé de tolérer leurs paroles déguisées en simples critiques." Le professeur Quirrell sourit, perdu dans ses souvenirs. "Les Mangemorts se passèrent bien d'eux, et je te conseille d'éviter cette erreur."

Harry hocha la tête, et un léger frisson le traversa. "Euh, quand vous m'avez parlé de ce qui s'est passé à Godric's Hollow, la nuit de Halloween, en 1981, j'ai, euh… j'ai cru voir une autre erreur de raisonnement. Une façon dont vous auriez pu éviter le désastre. Mais, euh, je pense que vous avez un angle mort, un ensemble de stratégies que vous n'envisagez pas, donc vous ne l'avez même pas vue par la suite…"

"J'espère que tu n'es pas sur le point de dire quelque chose de bête, comme 'ne tuez pas les gens'," dit le professeur Quirrell. "Sinon, je serai mécontent."

"\parsel{Pas différence de valeurs. Véritable erreur, étant donné vos buts. Me ferez-vous du mal, ssi je joue rôle du professseur, et vous ensseigne leçon~? Ou ssi erreur est ssimple et évidente, vous fait vous ssentir sstupide~?}"

"\parsel{Non,}" siffla le professeur Quirrell. "\parsel{Pas ssi véritable leçon.}"

Harry déglutit. "Hmm. Pourquoi ne pas avoir testé le système de Horcruxe avant de vraiment devoir l'utiliser~?"

"Le tester~?" dit le professeur Quirrell. Il leva les yeux et prit un ton indigné~: "Qu'est-ce que tu veux dire, le \emph{tester}~?"

"Pourquoi ne pas avoir vérifié si votre système de Horcruxe fonctionnait bien avant d'en avoir besoin le soir d'Halloween~?"

Le professeur Quirrell avait l'air dégoûté. "Espèce de ridicule… je ne voulais pas \emph{mourir}, petit, et c'était la seule façon que j'avais de tester ma grande création~! Quel bien cela m'aurait-il fait de risquer ma vie avant d'y être forcé~? En quoi cela m'aurait-il avantagé~?"

Harry avait une boule dans la gorge. "\parsel{Il y avait un moyen de le tester sans mourir.} L'idée au sens plus général est importante. Est-ce que vous comprenez, maintenant~?"

"Non," dit le professeur Quirrell au bout d'un moment. Il broya doucement l'une des dernières campanules avec un fil de longs cheveux blonds et les fit tomber dans la potion, qui bouillonnait et s'était éclaircie. Seuls deux campanules restaient sur la table. "Et j'espère, pour ton bien, que c'est une bonne leçon."

"Supposez, professeur, que j'ai appris à lancer le Horcruxe amélioré et que je sois prêt à m'en servir. Qu'est-ce que j'en ferais~?"

Le professeur Quirrell répondit immédiatement. "Tu trouverais quelqu'un qui te répugne moralement et dont tu pourrais te convaincre que la mort sauverait beaucoup d'autres, puis tu l'assassinerais pour créer un Horcruxe."

"Et ensuite~?"

"Toujours plus de Horcruxes," dit le professeur Quirrell. Il prit un bocal qui semblait emplit d'écailles de dragon.

"Avant ça," dit Harry.

Au bout d'un moment, le professeur de Défense secoua la tête. "Je ne comprends toujours pas, et tu vas arrêter ce petit jeu et me dire la réponse."

"Je ferais des Horcruxes pour mes amis. Si vous vous étiez soucié d'une seule personne à part vous, une seule qui aurait donné du \emph{sens} à votre immortalité, une seule avec laquelle vous auriez voulu vivre pour toujours…" la gorge de Harry se serra. "Alors l'idée de faire un Horcruxe pour quelqu'un d'autre n'aurait pas été si contre-intuitive." Il pleurait presque. "Vous avez un angle mort dirigé vers les stratégies qui font du bien aux autres, à un point tel que cela vous empêche d'atteindre vos propres buts égoïstes. Vous pensez… J'imagine que de votre point de vue, ce n'est tout simplement pas votre style. C'est cet… aspect particulier de l'image que vous avez de vous-même. C'est ce qui vous a coûté ces neuf ans."

Le professeur Quirrell de Défense tenait une bouteille. Des gouttes d'huile de mente tombaient dans le chaudron.

"Je vois…" dit-il lentement. "Je vois. J'aurais dû apprendre la nouvelle version du rituel de Horcruxe à Rabastan et le forcer à tester l'invention. Oui, avec le recul, c'est profondément évident. J'aurais même pu lui ordonner de s'imprimer dans l'esprit d'un nourrisson jetable afin de pouvoir observer le résultat avant de me rendre à Godric's Hollow et de te créer." Le professeur Quirrell secoua la tête, amusé. "Eh bien. Je suis content de comprendre ça maintenant et pas il y a dix ans. J'avais assez de choses à me reprocher à l'époque."

"Vous ne voyez pas les moyens gentils de faire \emph{ce que vous voulez faire}," dit Harry. Il sentait le désespoir dans sa propre voix. "Même quand une stratégie gentille serait \emph{plus efficace}, vous ne la voyez pas parce que vous avez cette image de \emph{personne méchante}."

"C'est juste," dit le professeur Quirrell. "De fait, maintenant que tu me l'as fait remarquer, je viens de penser à des gentillesse que je pourrais faire aujourd'hui même et qui me seront très utiles."

Harry le regarda fixement.

Le professeur Quirrell souriait. "C'est une bonne leçon, petit. Dorénavant, et jusqu'à ce que j'y sois habitué, je m'assurerai d'envisager des ruses qui impliquent d'être bon envers les autres. Peut-être irais-je pratiquer un peu de charité jusqu'à ce que cela me soit naturel."

Des frissons traversèrent l'échine de Harry.

Le professeur Quirrell avait dit cela sans la moindre trace d'hésitation.

Lord Voldemort était absolument certain d'être à l'abri de toute rédemption. Il n'était en rien inquiet que cela puisse lui arriver.

L'avant-dernière campanule fut doucement placée dans la potion.

"D'autres leçons utiles que tu souhaiterais m'enseigner aujourd'hui, petit~?" dit le professeur Quirrell. Il avait relevé les yeux et souriait comme s'il savait exactement ce que pensait Harry.

"Oui," dit-il, d'une voix presque brisée. "Si votre but est d'être heureux~: être bon envers les autres est plus agréable que de l'être envers soi-même…"

"Est-ce que tu crois \emph{vraiment} que je n'y ai pas pensé, petit~?" le sourire avait disparu. "Crois-tu que je sois stupide~? Après avoir été diplômé de Poudlard, j'ai parcouru le monde pendant des années avant de revenir en Angleterre sous les traits de Lord Voldemort. J'ai arrêté de compter le nombre de masques que j'ai porté. Crois-tu que je n'ai jamais essayé de jouer au héros, pour voir ce que ça faisait~? As-tu entendu parler d'un certain Alexander Chernyshov~? Sous ce déguisement, je me suis rendu dans un trou à rats oublié dirigé par un mage noir et j'ai libéré ses misérables habitants de sa domination. Ils en pleurèrent de gratitude. Je n'ai rien ressenti de particulier. Je suis même resté dans les environs et j'ai tué les cinq mages noirs qui essayèrent de s'emparer du pouvoir. J'ai dépensé mon or - enfin, pas le mien, mais le principe est le même - pour améliorer leur petit pays et instaurer un semblant d'ordre. Ils ont continué de ramper, ils ont donné à un enfant sur trois le nom d'Alexander. Je n'ai toujours rien senti, alors j'ai hoché la tête, je me suis dit 'ça valait le coup d'essayer', et j'ai continué mon chemin."

"Et alors vous étiez heureux, quand vous étiez Lord Voldemort~?" La voix de Harry était montée d'un cran, comme s'il en avait perdu la maîtrise.

Le professeur Quirrell hésita puis haussa les épaules. "Il semble que tu connaisses déjà la réponse à cette question."

"Alors \emph{pourquoi}~? Pourquoi être Voldemort si \emph{ça ne vous rend même pas heureux}~?" sa voix se brisa. "Je suis \emph{vous}, je suis basé sur vous, alors \emph{je sais} que le professeur Quirrell n'est pas qu'un masque~! Je \emph{sais} que vous auriez vraiment pu l'être~! Pourquoi ne pas simplement rester comme ça~? Levez la malédiction du poste de Défense et \emph{restez ici}, utilisez la Pierre Philosophale pour prendre l'apparence de David Monroe, laissez Quirinus Quirrell partir~; si vous dites que vous ne tuerez plus personne, je promets de ne dire à personne qui vous êtes vraiment. Soyez juste le \emph{professeur Quirrell}, pour toujours~! Vos élèves vous apprécieraient, ceux de \emph{mon père} l'apprécient bien…"

Le professeur Quirrell gloussait tout en remuant le contenu du chaudron. "Il reste peut-être quinze-mille sorciers en Angleterre magique, petit. Il y en avait plus, avant. Ce n'est pas pour rien qu'on a peur de prononcer mon nom. Tu me pardonnerais parce que tu aimais mes leçons de magie de bataille~?"

\emph{Il a raison}, dit le Poufsouffle interne de Harry. \emph{Non mais franchement, c'est n'importe quoi~!}

Harry garda la tête droite, mais elle tremblait. "Ce n'est pas à moi de pardonner vos actes. Mais ça vaut mieux qu'une autre guerre."

"Ha," dit le professeur de Défense. "Si tu trouves un jour un Retourneur de Temps qui peut revenir quarante ans dans le passé et altérer l'histoire, assure-toi de dire ça à Dumbledore avant qu'il ne refuse la candidature de Tom Jedusor au poste de professeur de Défense. Mais hélas, j'ai peur que Tom Jedusor n'ait jamais trouvé son bonheur à Poudlard."

"\emph{Pourquoi pas~?}"

"Parce que j'aurais toujours été entouré d'idiots et que je n'aurais pas pu les tuer," dit-il d'un ton calme. "Ma plus grande joie est de tuer des idiots, et je te serais reconnaissant de ne pas me critiquer avant d'avoir toi-même essayé."

"Il doit bien y avoir \emph{quelque chose} qui vous rendrait plus heureux que ça," dit Harry, et sa voix se brisa à nouveau. "Il le faut."

"Pourquoi~?" dit le professeur Quirrell. "Est-ce qu'il y a une loi scientifique que j'ignore~? Dis-m'en plus."

Harry ouvrit la bouche, mais il ne put trouver les mots, il fallait bien, il \emph{fallait qu'il y ait quelque chose}, si seulement il trouvait quoi répondre…

"Et \emph{toi}," dit le professeur Quirrell, "tu n'as aucun droit de parler de bonheur. Ce n'est pas ce qui t'est le plus précieux. Tu l'as décidé dès le départ, au début de l'année, quand le Choixpeau t'a proposé d'aller à Poufsouffle. Ce que je sais, car j'ai reçu une offre similaire, accompagnée des mêmes avertissements, il y a de nombreuses années, et je l'ai refusée, tout comme toi. Passé ça, je pense que nous n'avons plus grand-chose à nous dire, d'un Tom Jedusor à l'autre." Le professeur de Défense se retourna vers le chaudron.

Il lâcha la dernière campanule et des bulles lumineuses s'élevèrent du chaudron avant que Harry ne trouve quoi répondre.

"Je crois que nous en avons fini," dit le professeur Quirrell. "Si tu as d'autres questions, elles devront attendre."

Harry se leva, les jambes flageolantes; le professeur Quirrell prit le chaudron et versa un volume démesurément immense du liquide splendide, bien plus que le chaudron n'aurait dû pouvoir contenir, sur le feu violet qui gardait l'entrée.

Le feu s'éteint soudainement.

"Maintenant, le Miroir," dit le professeur Quirrell. Il sortit la Cape d'Invisibilité de ses Robes et l'envoya léviter jusqu'aux pieds de Harry. 

%  LocalWords:  Tsiolkovsky quoque Hufflepart Evangel Knockturn Marvolo
%  LocalWords:  Morfin Chernyshov
