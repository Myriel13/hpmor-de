\chapter{Différences de statut}

\lettrine{U}{ne} désorientation déchirante, voilà ce que Harry ressentit en descendant de la plate-forme neuf trois-quart et en mettant pied sur le reste du monde, un monde dont Harry avait longtemps cru qu'il était le seul qui existât vraiment. Les gens s'habillaient de t-shirts et de pantalons au lieu des robes bien plus dignes des sorciers et des sorcières. Des morceaux de détritus éparpillés ici et là autour des bancs. Une odeur oubliée, les fumées d'essence brûlée, brutes et intenses. L'ambiance de la gare de King's Cross, moins joviale et radieuse qu'à Poudlard ou qu'au chemin de Traverse~; les gens semblaient plus petits, plus peureux, et ils se seraient probablement empressés d'échanger leurs problèmes contre un Seigneur des Ténèbres à combattre. Harry aurait voulu jeter \emph{Récurvite} sur la saleté et \emph{Everto} sur les déchets, et, s'il l'avait connu, un sortilège de bulle pour ne pas avoir à respirer l'air environnant. Mais il ne pouvait utiliser sa baguette en un lieu pareil…

Ceci, comprit Harry, était ce que l'on devait ressentir en passant d'un pays du premier monde à un pays du tiers monde.

Sauf que c'était le monde zéro que Harry venait de quitter, le monde magique, le monde des sortilèges de nettoyage et des elfes de maison~; où, entre l'art du soigneur et sa propre magie, on pouvait aller jusqu'à cent-soixante-dix ans avant que le grand âge ne commence à vraiment laisser sa marque.

C'était Londres non-magique, la terre moldue, où Harry était temporairement de retour. C'était là que maman et papa vivraient jusqu'à la fin de leurs jours, à moins que la technologie ne fasse un bond en avant et permette d'atteindre la qualité de vie des sorciers, ou que quelque chose de plus profond ne change dans le monde.

Sans même y penser, la tête de Harry pivota et ses yeux dardèrent par-dessus son épaule pour apercevoir la malle de bois qui se précipitait après lui sans que les Moldus ne la remarquent, les tentacules griffus offrant une rapide confirmation du fait que, oui, il n'avait pas simplement tout imaginé.

Et il y avait l'autre raison à l'origine du poids dans sa poitrine.

Ses parents ne savaient pas.

Ils ne savaient \emph{rien du tout}.

Ils ne savaient pas…

<<~Harry~>>, dit une femme fine et blonde dont la peau parfaite et sans marques lui donnait l'air d'avoir bien moins de trente-trois ans~; et Harry se rendit compte avec surprise que \emph{c'était} de la magie, il n'avait pas reconnu les signes auparavant, mais il pouvait maintenant les voir. Et quel que soit le genre de potion capable de durer aussi longtemps, ç'avait dû être horriblement dangereux, car la plupart des sorcières ne se faisaient pas ça à elles-mêmes, elles n'étaient pas désespérées à ce point…

De l'eau s'amoncela dans les yeux de Harry.

<<~\emph{Harry~?}~>> héla un homme à l'air plus âgé avec une brioche naissante autour de l'estomac, habillé d'une veste noire jetée sur une chemise gris-vert sombre avec la négligence ostentatoire des universitaires, quelqu'un qui serait toujours un professeur où qu'il aille, qui aurait certainement été l'un des sorciers les plus brillants de sa génération s'il était né avec deux copies de ce gène au lieu de zéro…

Harry leva sa main et leur fit un signe. Il ne pouvait pas parler. Il en était totalement incapable.

Ils avancèrent jusqu'à lui, sans courir, mais d'une démarche sûre et digne~; c'était à cette vitesse que le professeur Michael Verres-Evans marchait, et madame Petunia Evans-Verres n'allait pas marcher plus vite.

Le sourire sur le visage de son père n'était pas très large, mais cela dit, son père n'avait pas l'habitude de prodiguer d'immenses sourires~; il était au moins aussi grand que le plus large de ceux que Harry avait vu chez lui, plus grand que lorsqu'une nouvelle subvention de recherche arrivait ou que lorsque l'un de ses étudiants obtenait un poste, et on ne pouvait pas demander un sourire plus grand que celui-là.

Maman clignait des yeux, elle essayait de sourire mais ne se débrouillait pas très bien.

<<~Alors~! dit son père en approchant d'un pas leste. Déjà fait des découvertes révolutionnaires~?~>>

Bien sûr, papa pensait qu'il blaguait.

Ça ne lui avait pas fait aussi mal que ses parents ne croient pas en lui à l'époque où \emph{personne} ne croyait en lui, à l'époque où Harry ne \emph{savait pas} ce que c'était que d'être pris au sérieux par des gens comme Dumbledore et le professeur Quirrell.

Et c'est alors que Harry comprit que le Survivant n'existait qu'en Angleterre magique, qu'une telle personne n'existait pas dans la Londres moldue, qu'il y avait juste un mignon petit garçon de onze qui rentrait chez lui pour Noël.

<<~Excusez-moi~>>, dit Harry, sa voix tremblante, <<~~>>je vais maintenant m'effondrer et fondre en larmes, ça ne veut pas dire que quelque chose ne va pas à l'école.<<~

Harry commença à avancer, puis il s'arrêta, déchiré entre la possibilité d'étreindre son père ou sa mère, il ne souhaitait qu'aucun ne se sente lésé ou n'ait l'impression que Harry l'aime plus que l'autre -

<<~Toi, dit son père, tu es un garçon très bête, M. Verres~>>, et il prit gentiment Harry par les épaules et le poussa dans les bras de sa mère, qui était agenouillée, des larmes coulant déjà le long de ses joues.

<<~Bonjour maman, dit Harry d'une voix vacillante, je suis revenu.~>> Et il la prit dans ses bras, au milieu des bruyants sons mécaniques et de l'odeur d'essence brûlée~; et Harry se mit à pleurer, car il savait que rien ne \emph{pourrait} revenir comme avant, lui encore moins que tout le reste.

\later

Le ciel était totalement noir et les étoiles apparaissaient lorsqu'ils eurent fini de négocier avec la circulation hivernale de la ville universitaire qu'était Oxford et qu'ils se garèrent dans l'allée de la petite et vieille maison à l'air minable que leur famille utilisait pour garder les livres à l'abri de la pluie.

En marchant sur le petit bout de chaussée qui menait à la porte principale, ils dépassèrent une série de pots de fleur qui contenaient de petites lumières électriques tamisées (tamisée, car elles devaient se recharger à l'énergie solaire pendant la journée), et elles s'intensifièrent à leur approche. Le plus difficile avait été de trouver des détecteurs de mouvement imperméables qui se déclenchaient exactement à la bonne distance…

À Poudlard, il y avait de véritables torches qui faisaient ça.

Et la porte d'entrée s'ouvrit alors et Harry entra dans leur salon, clignant des yeux.

\emph{Chaque centimètre d'espace mural est couvert par des bibliothèques. Chacune a six étages et atteint presque le plafond. Certains étages sont remplis à ras bord de livres grand format~: science, mathématiques, histoire et tout le reste. D'autres étages ont deux rangées de livres de science-fiction brochés, avec la rangée arrière surélevée grâce à de vieilles boîtes à mouchoirs ou des planches de bois, afin que l'on puisse la voir au-dessus de la rangée avant. Et ça ne suffit pas. Les livres débordent des tables et des sofas et forment de petits monticules sous les fenêtres.}

La maisonnée Verres était telle qu'il l'avait laissée, mais avec plus de livres, et c'était aussi comme ça qu'il l'avait laissée.

Et un arbre de Noël, nu, sans décoration, juste deux jours avant le réveillon, ce qui déboussola brièvement Harry avant qu'il ne comprenne, avec une chaleur naissant dans sa poitrine, que ses parents avaient bien sûr \emph{attendu}.

<<~Nous avons sorti le lit de ta chambre pour faire de la place pour d'autres bibliothèques, dit son père. Tu peux dormir dans ta malle, non~?

--- \emph{Tu} peux dormir dans ma malle, dit Harry.

--- Ce qui me rappelle, dit son père. Qu'\emph{est}-ce qu'ils ont fini par faire au sujet de ton cycle de sommeil~?

--- De la magie~>>, dit Harry, se dirigeant droit vers la porte qui donnait sur sa chambre, juste au cas où papa n'aurait \emph{pas} plaisanté…

<<~Ce n'est pas une explication~!~>> dit le professeur Verres-Evans juste quand Harry hurla~: <<~\emph{Vous avez utilisé tout l'espace libre dans mes bibliothèques~?}~>>

\later

Harry avait passé son 23 décembre à faire des emplettes pour les choses moldues qu'il ne pouvait pas juste métamorphoser~; son père avait été occupé et avait dit à Harry qu'il devrait marcher ou prendre le bus, ce qui avait très bien convenu à Harry. Certains des employés de la quincaillerie lui avaient jeté des regards interrogateurs, mais il avait dit d'une voix innocente que son père faisait des achats non loin et qu'il était très occupé et qu'il l'avait envoyé chercher certaines choses (en tenant une liste faite de caractères d'une écriture qui ressemblaient précisément à celle quasi-illisible des adultes)… et en fin de compte, de l'argent, c'était de l'argent.

Ils avaient décoré le sapin de Noël tous ensemble, et Harry avait mis une petite fée dansante au sommet (deux Mornilles et cinq Noises chez Gambol et Japes).

Gringotts avaient été prêts à échanger des Gallions contre des billets, mais ils ne semblaient pas connaître de moyen simple de transformer de larges quantités d'or en argent Moldu exempté d'impôts et placé dans un compte bancaire Suisse. Cela avait nettement contrarié le plan de Harry consistant à transformer la majeure partie de l'argent qu'il s'était volé à lui-même en un judicieux mélange composé à 60~\% d'indices de fonds internationaux et à 40~\% de Berkshire Hathaway. Pour l'instant, Harry avait diversifié ses biens un peu plus en sortant subrepticement la nuit, invisible et remonté dans le temps, et en enterrant cent Gallions d'or dans le jardin. De toute façon, il avait toujours, toujours, \emph{toujours} voulu faire ça.

Une partie du 24 décembre avait été passée aux côtés du professeur tandis que celui-ci lisait les livres de Harry et posait des questions. La plupart des expériences suggérées par son père n'étaient pas réalisables, du moins pour le moment~; de celles qui restaient, Harry en avait déjà fait la plupart (<<~Oui papa, j'ai vérifié ce qui se passait si je donnais à Hermione une prononciation légèrement différente et qu'elle ne savait pas si elle avait changé, c'est la première expérience que j'ai faite, papa~!~>>)

La dernière question que le père de Harry avait posée, après avoir relevé les yeux de \emph{Potions et breuvages magiques} avec un air de dégoût abasourdi, était si tout ça avait un sens quand on était un sorcier~; et Harry avait répondu non.

Sur quoi son père avait déclaré que la magie n'était pas scientifique.

Harry était encore un peu choqué à l'idée qu'on puisse dire d'une section de la \emph{réalité} qu'elle n'était pas scientifique. Papa semblait penser que le conflit entre ses intuitions et l'univers signifiait que l'univers avait un problème.

(Mais après tout, il y avait beaucoup de physiciens qui pensaient que la mécanique quantique était bizarre au lieu de penser que la mécanique était normale et que c'était eux qui étaient bizarres).

Harry avait montré à sa mère le kit de soin qu'il avait apporté pour qu'ils le gardent à la maison, même si la plupart des potions ne marcheraient pas sur Papa. Maman avait fixé le kit du regard avec un air qui avait poussé Harry à lui demander si sa sœur avait jamais apporté quoi que ce soit du genre pour papy Edwin et mamie Elaine. Et quand maman avait continué à ne pas répondre, Harry avait dit hâtivement qu'elle n'y avait probablement juste jamais pensé. Et puis il avait fini par fuir la pièce.

Lily Evans n'y \emph{avait} probablement \emph{pas} pensé, c'était ça qui était triste. Harry savait que les autres avaient une tendance à ne pas penser aux sujets douloureux, tout comme ils avaient une tendance à ne pas délibérément poser leurs mains sur des cuisinières allumées~; et Harry commençait à soupçonner que la plupart des nés-Moldus acquéraient rapidement une tendance à ne pas penser à leur famille, qui allait de toute façon mourir avant d'avoir atteint son premier siècle.

Non que Harry ait la moindre intention de laisser \emph{ça} se produire, bien sûr.

Et la journée du 24 décembre fut alors plus avancée et ils partirent vers leur dîner de réveillon.

\later

La maison était immense, non pas d'après les standards de Poudlard mais d'après ceux dictant ce qu'on pouvait avoir si son père était un distingué professeur essayant de vivre à Oxford. Deux étages de briques étincelantes sous le soleil couchant, avec des fenêtres au-dessus d'autres fenêtres et une, plus grande que les autres, qui montait à des hauteurs que du verre n'aurait jamais dû pouvoir atteindre. Ça allait être un immense salon…

Harry prit une profonde inspiration et sonna à la porte d'entrée.

Il y eut un appel lointain, <<~Chéri, tu peux y aller~?~>>

Suivi par le lent piétinement de quelqu'un qui s'approchait.

Et la porte s'ouvrit alors pour révéler un homme cordial aux joues grasses et roses et aux cheveux clairsemés, dans une chemise bleue boutonnée qui tirait légèrement aux coutures.

<<~Dr. Granger,~>> dit brusquement le père de Harry avant que celui-ci ne puisse ouvrir la bouche. <<~Je suis Michael, et voici Pétunia et notre fils Harry. La nourriture est dans la malle magique~>>, et papa fit un vague geste vers l'arrière - à vrai dire pas tout à fait vers la malle.

<<~Oui, entrez, je vous en prie~>>, dit Léo Granger. Il fit un pas vers l'avant et prit la bouteille de vins des mains tendues du professeur avec un <<~Merci~>> marmonné, puis il recula et indiqua le salon d'un geste de la main. <<~Asseyez-vous. Et~>>, sa tête se baissant pour s'adresser à Harry, <<~tous les jouets sont en bas à la cave, je suis sûr que Herm descendra bientôt, c'est la première porte à ta droite~>>, et il pointa un doigt vers un couloir.

Harry se contenta de l'observer un moment, conscient du fait qu'il bloquait ses parents et les empêchait d'entrer.

<<~Des jouets~?~>> dit Harry d'une voix radieuse et aiguë, les yeux écarquillés. <<~J'adore les jouets~!~>>

Il y eut une inspiration bruyante venant de sa mère, située derrière lui, et Harry se précipita dans la maison, parvenant tout juste à ne pas taper des pieds sur le sol exagérément.

Le salon était aussi large qu'il avait semblé être de l'extérieur, avec un immense plafond voûté d'où pendait un gigantesque lustre et un arbre de Noël dont le passage par la porte d'entrée avait probablement fait des victimes. Les niveaux inférieurs de l'arbre étaient décorés de fond en comble de beaux motifs rouge et vert et or, avec une touche novatrice de bleu et de bronze~; les hauteurs, que seul un adulte pouvait atteindre, étaient aléatoirement recouvertes de chaînes de lampions et de guirlandes sans que le moindre soin ait été apportée à leur disposition. Un couloir se prolongeait jusqu'à s'achever dans l'antichambre d'une cuisine, et des escaliers de bois à la rambarde de métal poli continuaient jusqu'à un deuxième étage.

<<~Waouh~! dit Harry. Quelle grande maison~! J'espère que je ne vais pas m'y perdre~!~>>

\later

Le Dr. Roberta Granger se sentait de plus en plus nerveuse à l'approche du dîner. La dinde et le rôti, leurs contributions au projet commun, cuisaient tranquillement dans le four~; les autres plats devaient être apportés par les invités, la famille Verres, qui avait adopté un garçon prénommé Harry. Qui était connu sous le nom de Survivant dans le monde magique. Et qui était aussi le seul garçon que Hermione ait jamais qualifié de <<~mignon~>>, et même le seul qu'elle ait jamais remarqué.

Les Verres avaient dit que Hermione était le seul enfant de l'âge de Harry dont il avait jamais reconnu l'existence de quelque façon que ce soit.

Et peut-être brûlaient-ils là quelques étapes, mais les deux couples avaient comme un vague soupçon que des cloches nuptiales se profilaient à quelques années d'ici.

Alors même si le jour de Noël se déroulerait, comme toujours, avec la famille de son mari, ils avaient décidé de passer le réveillon à rencontrer les possibles futurs beaux-parents de leur fille.

On sonna à la porte alors qu'elle était en train d'arroser la dinde, et elle éleva la voix et cria~: <<~\emph{Chéri, tu peux y aller~?}~>>

Il y eut le bref grognement d'une chaise et de son occupant, puis le son des lourds pas de son mari et d'une porte qui s'ouvrait.

<<~Dr. Granger~? dit la voix brusque d'un homme plus âgé. Je suis Michael, et voici Pétunia et notre fils Harry. La nourriture est dans la malle magique.

--- Oui, entrez, je vous en prie~>>, dit son mari, suivi d'un <<~Merci~>> marmonné qui indiquait qu'une espèce de cadeau avait été acceptée, puis <<~Asseyez-vous.~>> La voix de Léo passa alors à un ton enthousiaste artificiel et dit~: <<~Et tous les jouets sont en bas à la cave, je suis sûr que Herm descendra bientôt, c'est la première porte à ta droite.~>>

Il y eut une brève pause.

Puis la voix radieuse d'un jeune garçon dit~: <<~Des jouets~? J'adore les jouets~!~>>

Il y eut le son de bruits de pas entrant dans la maison, puis la même voix radieuse dit~: <<~Waouh~! Quelle grande maison~! J'espère que je ne vais pas m'y perdre~!~>>

Roberta ferma le four en souriant. Elle avait été un peu inquiétée par la façon dont les lettres de Hermione avaient décrites le Survivant - même si sa fille n'avait certainement rien dit qui puisse indiquer que Harry Potter était \emph{dangereux}~; rien de semblable aux sombres sous-entendus écrits dans les livres qu'elle avec achetés, prétendument pour Hermione, pendant leur excursion au Chemin de Traverse. Sa fille n'avait pas dit grand-chose, seulement que Harry parlait comme s'il sortait d'un livre et que Hermione étudiait plus dur qu'elle ne l'avait jamais fait de sa vie juste pour rester meilleure que lui en cours. Mais à l'entendre, Harry était un enfant de onze ans ordinaire.

Elle se rendit à la porte principale juste quand, dans un grand fracas, sa fille descendit frénétiquement les escaliers à une vitesse qui ne semblait pas sûre du tout, Hermione avait prétendu que les sorcières étaient plus résistantes aux chutes mais Roberta n'était tout à fait sûre d'y croire -

Roberta eut un premier aperçu du professeur et de Mme Verres, qui semblaient tous deux plutôt nerveux, juste au moment où le garçon à la cicatrice légendaire se tournait vers sa fille et disait, d'une voix maintenant plus grave, <<~Heureux de vous revoir en cette exceptionnelle soirée, Mlle Granger.~>> Sa main s'étendit, comme s'il offrait ses parents sur un plateau d'argent. <<~Je vous présente mon père, le professeur Michael Verres-Evans, et ma mère, Mme Pétunia Evans-Verres.~>>

Et, alors que la bouche de Roberta s'ouvrait grand, le garçon se retourna vers ses parents et dit, reprenant sa voix radieuse~:

<<~Maman, Papa, c'est Hermione~! Elle est vraiment maline~!

--- \emph{Harry}~! siffla sa fille, Arrête ça~!~>>

Le garçon pivota de nouveau pour regarder Hermione. <<~J'ai peur, mademoiselle Granger, dit-il avec le plus grand sérieux, que vous et moins ayons été exilés aux recoins labyrinthiques de la cave. Laissons-les à leurs conversations adultes, qui s'élèvent sans nul doute loin au-dessus de nos intellects puérils, et reprenons notre conversation interrompue sur ce que le projectivisme Humien implique pour la Métamorphose.

--- Excusez-nous, s'il vous plaît~>>, dit sa fille d'un ton très ferme, et elle attrapa le garçon par sa manche gauche et le traîna dans le couloir - Roberta pivota, impuissante, essayant de les suivre alors qu'ils la dépassaient, le garçon lui fit un jovial salut de la main - puis Hermione tira le garçon dans le passage menant à la cave et claqua la porte derrière elle.

<<~Je, ah, je vous demande pardon pour…~>> dit Mme Verres d'une voix défaillante.

<<~Je suis désolé,~>> dit le professeur, un sourire plein d'affection sur le visage, <<~Harry peut être assez susceptible sur ce genre de chose. Mais je pense qu'il a raison en disant que nous ne serions pas intéressés par leur conversation.~>>

\emph{Est-il dangereux~?} voulait-elle demander, mais elle resta silencieuse et essaya de trouver des questions plus subtiles. À côté d'elle, son mari gloussait, comme si ce qu'il venait de voir lui avait semblé plus amusant qu'effrayant.

Le plus terrible Seigneur des Ténèbres de l'Histoire avait essayé de tuer ce garçon, et l'enveloppe calcinée de son corps avait été retrouvée à côté du berceau.

Son possible futur beau-fils.

Roberta avait été de plus en plus appréhensive quant au fait d'avoir donné sa fille à la sorcellerie - en particulier après avoir lu les livres, fait correspondre les dates, et s'être rendu compte que sa mère sorcière avait probablement été tuée au sommet de la terreur de Grindelwald et \emph{pas} en lui donnant naissance comme son père l'avait toujours prétendu. Mais le professeur McGonagall avait fait d'autres visites après son premier voyage, pour <<~voir comment mademoiselle Granger allait~>>~; et Roberta ne pouvait s'empêcher de penser que si Hermione disait un jour que ses parents posaient problème à sa carrière de sorcière, alors quelqu'un ferait quelque chose pour les corriger \emph{eux}…

Roberta afficha son plus beau sourire et fit ce qu'elle put pour répandre un peu de fausse joie de Noël.

\later

La salle à manger était bien trop grande pour six personnes - euh, pour quatre personnes et deux enfants, mais elle était recouverte d'une nappe de fin lin blanc et les plats avaient été inutilement transférés dans des assiettes de luxes, mais au moins elles étaient faites d'acier inoxydable et pas de vrai argent.

Harry avait un peu de mal à se concentrer sur la dinde.

La conversation avait naturellement dérivé vers Poudlard~; et il avait été évident aux yeux de Harry que ses parents espéraient que Hermione ferait une gaffe et en dirait plus sur sa vie scolaire que ce qu'il leur en avait révélé. Et soit Hermione s'en était rendu compte, soit elle restait automatiquement à l'écart de tout sujet qui pourrait poser problème.

Donc tout allait bien pour \emph{Harry}.

Mais il avait malheureusement fait l'erreur de bassiner ses parents avec toutes sortes de faits concernant Hermione qu'elle n'avait pas mentionné aux \emph{siens}.

Comme le fait qu'elle était générale d'une armée dans leurs activités du soir.

La mère de Hermione avait eut l'air très alarmée, et Harry avait rapidement interrompu la discussion et avait fait de son mieux pour expliquer que tous les sorts étaient étourdissants, que le professeur Quirrell surveillait en permanence, et que l'existence du soin magique signifiait que beaucoup de choses était bien moins dangereuses qu'elles n'en avaient l'air, et c'est à ce moment que Hermione lui avait donné un coup de pied sous la table. Heureusement le père de Harry, au sujet duquel ce dernier devait admettre qu'il était meilleur que lui dans certains domaines, avait annoncé avec une ferme autorité professorale qu'il n'était pas inquiet du tout puisqu'il ne pouvait pas imaginer qu'on aurait laissé des enfants le faire si ça avait été dangereux.

Cela dit, ce n'était pas la raison pour laquelle Harry avait du mal à apprécier le dîner.

… le problème, quand on s'apitoyait sur son sort, c'était que ça ne prenait toujours qu'un instant pour trouver quelqu'un pour qui les choses étaient pires.

À un moment, le Dr. Léo Granger avait demandé si le professeur McGonagall, ce gentil professeur qui semblait aimer Hermione, lui donnait beaucoup de points à l'école.

Hermione avait dit oui, avec un sourire apparemment authentique.

Harry était parvenu, avec quelques efforts, à s'empêcher de faire remarquer d'un ton glacial que le professeur McGonagall ne montrerait jamais aucun favoritisme envers un élève de Poudlard et que Hermione gagnait beaucoup de points parce qu'elle avait mérité \emph{chacun. d'entre. eux.}

À un autre moment, Léo Granger avait offert à la table son opinion selon laquelle Hermione était très intelligente et aurait pu faire médecine et devenir dentiste, s'il n'y avait eu toute cette histoire de sorcellerie.

Hermione avait de nouveau sourit, et un regard rapide avait empêché Harry de suggérer que Hermione aurait aussi pu être une \emph{scientifique internationalement reconnue}, et de demander si cette pensée serait venue aux Granger s'ils avaient eu un \emph{fils} au lieu d'une \emph{fille}, ou s'il était de toute façon inacceptable que leur descendance réussisse un jour mieux qu'eux.

Mais Harry arrivait rapidement à ébullition.

Et il appréciait de plus en plus le fait que son propre père avait \emph{toujours} fait tout ce qu'il pouvait pour soutenir son développement à la hauteur de ses capacités prodigieuses et qu'il l'avait \emph{toujours} encouragé à viser plus haut et qu'il n'avait \emph{jamais} minimisé le moindre de ses accomplissements, même si un enfant prodige demeurait malgré tout un enfant. Était-ce le genre de foyer où il aurait pu finir si maman avait épousé Vernon Dursley~?

Cela dit, Harry faisait ce qu'il pouvait.

<<~Et elle te bat vraiment dans \emph{tous} tes cours à part vol sur balai et Métamorphose~? dit le professeur Michal Verres-Evans.

--- Oui~>>, dit Harry avec un calme forcé en se découpant une autre bouchée de dinde de Noël. <<~Avec une bonne marge dans la plupart.~>> Il y avait d'autres circonstances lors desquelles Harry aurait été plus réticent à l'admettre, et c'était la raison pour laquelle il n'avait pas encore trouvé l'occasion de le dire à son père.

<<~Hermione s'est toujours pas mal débrouillé à l'école, dit le Dr. Léo Granger d'un ton satisfait.

--- Harry fait des compétitions au niveau national~! dit le professeur Michael Verres-Evans.

--- Chéri~!~>> dit Pétunia.

Hermione gloussait, et ça n'aidait pas Harry passer un bon moment. Ça ne semblait pas embêter Hermione et \emph{ça embêtait Harry.}

<<~Ça ne me gêne pas de perdre contre elle, papa~>>, dit Harry. À ce moment précis, c'était le cas. <<~Ai-je mentionné qu'elle a mémorisé tous ses manuels avant le premier jour de cours~? Et oui, j'ai vérifié.

--- Est-ce, euh, \emph{habituel} pour elle~? dit le professeur Verres-Evans aux Granger.

--- Oh, oui, Hermione a toujours mémorisé des choses,~>> dit le Dr. Roberta Granger avec un sourire jovial. <<~Elle connaît toutes les recettes de mes livres de cuisine par cœur. Elle me manque à chaque fois que je fais à dîner.~>>

À en juger l'expression sur le visage de son père, papa ressentait au moins une partie de ce que Harry ressentait.

<<~Ne t'en fais pas, papa, dit Harry, elle accède au matériel pédagogique le plus avancé qu'elle soit capable de comprendre maintenant. Ses professeurs savent qu'elle est intelligente, \emph{contrairement à ses parents~!}~>>

Sa voix avait monté d'un cran sur ces trois derniers mots, alors même que tous les visages s'étaient tournés pour le fixer et que Hermione lui avait donné un nouveau coup de pied, Harry savait qu'il s'était planté, mais ç'avait été trop, beaucoup trop.

<<~Bien sûr que nous savons qu'elle est intelligente~>>, dit Léo Granger, en commençant à avoir l'air offensé face à l'enfant qui avait eu la témérité d'élever la voix à leur table.

<<~Vous n'en avez pas la moindre idée~>>, dit Harry, la glace s'insinuant maintenant dans sa voix. <<~Vous pensez qu'elle lit beaucoup de livres et que c'est mignon, c'est ça~? Vous voyez un bulletin avec des notes parfaites et vous pensez que c'est bien qu'elle soit forte en cours. Votre fille est la sorcière la plus talentueuse de sa génération et la plus grande célébrité de Poudlard, et un jour, Dr. et Dr. Granger, le fait que vous étiez ses parents sera la seule raison pour laquelle l'Histoire s'est souvenue de vous~!~>>

Hermione, qui s'était calmement levée de son siège et avait fait le tour de la table, choisit ce moment pour attraper la chemise de Harry au niveau de son épaule et pour le tirer hors de sa chaise. Harry se laissa faire, mais alors que Hermione le traînait loin de la table, il dit, élevant encore plus sa voix~: <<~Il est entièrement possible que dans mille ans, le fait que les parents de Hermione Granger étaient dentistes soit la seule raison pour laquelle on se souvienne du métier de dentiste~!~>>

\later

Roberta fixa l'espace par lequel sa fille venait de traîner le Survivant hors de la pièce avec un air patient gravé sur son jeune visage.

<<~Je suis terriblement désolé~>>, dit le professeur Verres avec un sourire amusé. <<~Mais ne vous en faites pas s'il vous plaît, Harry parle toujours comme ça. N'ont-ils pas déjà l'air d'un couple marié~?~>>

Ce qui était effrayant, c'était qu'ils en \emph{avaient} l'air.

\later

Harry s'était attendu à une leçon assez sévère de la part de Hermione.

Mais après qu'elle l'eut traîné dans le passage menant à la cave et qu'elle ait refermé la porte derrière eux, elle s'était retournée -

- en souriant, d'un sourire authentique pour ce que Harry pouvait en discerner.

<<~Harry, ne fais pas ça, s'il te plaît, dit-elle d'une voix douce. Même si c'est très gentil de ta part. Tout va bien.~>>

Harry se contenta de la regarder. <<~Comment peux-tu le supporter~?~>> dit-il. Il devait garder une voix basse, il ne voulait pas que les parents entendent, mais si elle ne monta pas en volume, elle monta en timbre. <<~\emph{Comment peux-tu le supporter~?}~>>

Hermione haussa les épaules et dit~:

<<~Parce que c'est comme ça que les parents \emph{devraient} être~?

--- Non~>>, dit Harry, sa voix basse et intense, <<~pas du tout, mon père ne me rabaisse \emph{jamais} - enfin, \emph{si}, mais jamais comme ça -~>>

Hermione leva un seul doigt, et Harry attendit, la regardant chercher ses mots. Un bon moment s'écoula avant qu'elle ne dise~: <<~Harry… Le professeur McGonagall et le professeur Flitwick m'aiment parce que je suis la sorcière la plus talentueuse de ma génération et la plus grande célébrité de Poudlard. Et maman et papa ne le savent pas, et tu n'arriveras jamais à le leur expliquer, mais ils m'aiment quand même. Ce qui veut dire que les choses sont comme elles devraient être, à Poudlard et à la maison. Et puisque ce sont \emph{mes} parents, M. Potter, \emph{vous} n'avez pas le droit d'arguer.~>> Elle souriait à nouveau du sourire mystérieux qu'elle avait eu au dîner, et elle regardait Harry avec beaucoup d'affection. <<~\emph{Est-}ce clair, M. Potter~?~>>

Harry hocha la tête d'un air guindé.

<<~Bien~>>, dit Hermione, et elle se pencha et l'embrassa sur la joue.

\later

La conversation venait de reprendre quand un glapissement aigu et lointain flotta jusqu'à eux.

<<~\emph{Hé~! Pas de bisou~!}~>>

Les deux pères éclatèrent de rire au moment même où les deux mères se levèrent de leur chaise avec une expression d'horreur identique et qu'elles s'élancèrent vers la cave.

Lorsque les enfants eurent été rapportés, Hermione dit d'un ton de glace qu'elle n'embrasserait plus jamais Harry, et celui-ci dit d'une voix outragée que le soleil serait devenu un tas de cendres froides avant qu'il ne la laisse s'approcher assez près pour qu'elle puisse réessayer.

Ce qui voulait dire que les choses étaient exactement comme elles devraient être, et ils se rassirent tous pour finir le dîner de Noël.

%  LocalWords:  renching Herm Verreses Léo’s
